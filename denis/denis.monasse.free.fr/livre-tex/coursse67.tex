\documentclass[]{article}
\usepackage[T1]{fontenc}
\usepackage{lmodern}
\usepackage{amssymb,amsmath}
\usepackage{ifxetex,ifluatex}
\usepackage{fixltx2e} % provides \textsubscript
% use upquote if available, for straight quotes in verbatim environments
\IfFileExists{upquote.sty}{\usepackage{upquote}}{}
\ifnum 0\ifxetex 1\fi\ifluatex 1\fi=0 % if pdftex
  \usepackage[utf8]{inputenc}
\else % if luatex or xelatex
  \ifxetex
    \usepackage{mathspec}
    \usepackage{xltxtra,xunicode}
  \else
    \usepackage{fontspec}
  \fi
  \defaultfontfeatures{Mapping=tex-text,Scale=MatchLowercase}
  \newcommand{\euro}{€}
\fi
% use microtype if available
\IfFileExists{microtype.sty}{\usepackage{microtype}}{}
\ifxetex
  \usepackage[setpagesize=false, % page size defined by xetex
              unicode=false, % unicode breaks when used with xetex
              xetex]{hyperref}
\else
  \usepackage[unicode=true]{hyperref}
\fi
\hypersetup{breaklinks=true,
            bookmarks=true,
            pdfauthor={},
            pdftitle={Formes bilineaires},
            colorlinks=true,
            citecolor=blue,
            urlcolor=blue,
            linkcolor=magenta,
            pdfborder={0 0 0}}
\urlstyle{same}  % don't use monospace font for urls
\setlength{\parindent}{0pt}
\setlength{\parskip}{6pt plus 2pt minus 1pt}
\setlength{\emergencystretch}{3em}  % prevent overfull lines
\setcounter{secnumdepth}{0}
 
/* start css.sty */
.cmr-5{font-size:50%;}
.cmr-7{font-size:70%;}
.cmmi-5{font-size:50%;font-style: italic;}
.cmmi-7{font-size:70%;font-style: italic;}
.cmmi-10{font-style: italic;}
.cmsy-5{font-size:50%;}
.cmsy-7{font-size:70%;}
.cmex-7{font-size:70%;}
.cmex-7x-x-71{font-size:49%;}
.msbm-7{font-size:70%;}
.cmtt-10{font-family: monospace;}
.cmti-10{ font-style: italic;}
.cmbx-10{ font-weight: bold;}
.cmr-17x-x-120{font-size:204%;}
.cmsl-10{font-style: oblique;}
.cmti-7x-x-71{font-size:49%; font-style: italic;}
.cmbxti-10{ font-weight: bold; font-style: italic;}
p.noindent { text-indent: 0em }
td p.noindent { text-indent: 0em; margin-top:0em; }
p.nopar { text-indent: 0em; }
p.indent{ text-indent: 1.5em }
@media print {div.crosslinks {visibility:hidden;}}
a img { border-top: 0; border-left: 0; border-right: 0; }
center { margin-top:1em; margin-bottom:1em; }
td center { margin-top:0em; margin-bottom:0em; }
.Canvas { position:relative; }
li p.indent { text-indent: 0em }
.enumerate1 {list-style-type:decimal;}
.enumerate2 {list-style-type:lower-alpha;}
.enumerate3 {list-style-type:lower-roman;}
.enumerate4 {list-style-type:upper-alpha;}
div.newtheorem { margin-bottom: 2em; margin-top: 2em;}
.obeylines-h,.obeylines-v {white-space: nowrap; }
div.obeylines-v p { margin-top:0; margin-bottom:0; }
.overline{ text-decoration:overline; }
.overline img{ border-top: 1px solid black; }
td.displaylines {text-align:center; white-space:nowrap;}
.centerline {text-align:center;}
.rightline {text-align:right;}
div.verbatim {font-family: monospace; white-space: nowrap; text-align:left; clear:both; }
.fbox {padding-left:3.0pt; padding-right:3.0pt; text-indent:0pt; border:solid black 0.4pt; }
div.fbox {display:table}
div.center div.fbox {text-align:center; clear:both; padding-left:3.0pt; padding-right:3.0pt; text-indent:0pt; border:solid black 0.4pt; }
div.minipage{width:100%;}
div.center, div.center div.center {text-align: center; margin-left:1em; margin-right:1em;}
div.center div {text-align: left;}
div.flushright, div.flushright div.flushright {text-align: right;}
div.flushright div {text-align: left;}
div.flushleft {text-align: left;}
.underline{ text-decoration:underline; }
.underline img{ border-bottom: 1px solid black; margin-bottom:1pt; }
.framebox-c, .framebox-l, .framebox-r { padding-left:3.0pt; padding-right:3.0pt; text-indent:0pt; border:solid black 0.4pt; }
.framebox-c {text-align:center;}
.framebox-l {text-align:left;}
.framebox-r {text-align:right;}
span.thank-mark{ vertical-align: super }
span.footnote-mark sup.textsuperscript, span.footnote-mark a sup.textsuperscript{ font-size:80%; }
div.tabular, div.center div.tabular {text-align: center; margin-top:0.5em; margin-bottom:0.5em; }
table.tabular td p{margin-top:0em;}
table.tabular {margin-left: auto; margin-right: auto;}
div.td00{ margin-left:0pt; margin-right:0pt; }
div.td01{ margin-left:0pt; margin-right:5pt; }
div.td10{ margin-left:5pt; margin-right:0pt; }
div.td11{ margin-left:5pt; margin-right:5pt; }
table[rules] {border-left:solid black 0.4pt; border-right:solid black 0.4pt; }
td.td00{ padding-left:0pt; padding-right:0pt; }
td.td01{ padding-left:0pt; padding-right:5pt; }
td.td10{ padding-left:5pt; padding-right:0pt; }
td.td11{ padding-left:5pt; padding-right:5pt; }
table[rules] {border-left:solid black 0.4pt; border-right:solid black 0.4pt; }
.hline hr, .cline hr{ height : 1px; margin:0px; }
.tabbing-right {text-align:right;}
span.TEX {letter-spacing: -0.125em; }
span.TEX span.E{ position:relative;top:0.5ex;left:-0.0417em;}
a span.TEX span.E {text-decoration: none; }
span.LATEX span.A{ position:relative; top:-0.5ex; left:-0.4em; font-size:85%;}
span.LATEX span.TEX{ position:relative; left: -0.4em; }
div.float img, div.float .caption {text-align:center;}
div.figure img, div.figure .caption {text-align:center;}
.marginpar {width:20%; float:right; text-align:left; margin-left:auto; margin-top:0.5em; font-size:85%; text-decoration:underline;}
.marginpar p{margin-top:0.4em; margin-bottom:0.4em;}
.equation td{text-align:center; vertical-align:middle; }
td.eq-no{ width:5%; }
table.equation { width:100%; } 
div.math-display, div.par-math-display{text-align:center;}
math .texttt { font-family: monospace; }
math .textit { font-style: italic; }
math .textsl { font-style: oblique; }
math .textsf { font-family: sans-serif; }
math .textbf { font-weight: bold; }
.partToc a, .partToc, .likepartToc a, .likepartToc {line-height: 200%; font-weight:bold; font-size:110%;}
.chapterToc a, .chapterToc, .likechapterToc a, .likechapterToc, .appendixToc a, .appendixToc {line-height: 200%; font-weight:bold;}
.index-item, .index-subitem, .index-subsubitem {display:block}
.caption td.id{font-weight: bold; white-space: nowrap; }
table.caption {text-align:center;}
h1.partHead{text-align: center}
p.bibitem { text-indent: -2em; margin-left: 2em; margin-top:0.6em; margin-bottom:0.6em; }
p.bibitem-p { text-indent: 0em; margin-left: 2em; margin-top:0.6em; margin-bottom:0.6em; }
.paragraphHead, .likeparagraphHead { margin-top:2em; font-weight: bold;}
.subparagraphHead, .likesubparagraphHead { font-weight: bold;}
.quote {margin-bottom:0.25em; margin-top:0.25em; margin-left:1em; margin-right:1em; text-align:justify;}
.verse{white-space:nowrap; margin-left:2em}
div.maketitle {text-align:center;}
h2.titleHead{text-align:center;}
div.maketitle{ margin-bottom: 2em; }
div.author, div.date {text-align:center;}
div.thanks{text-align:left; margin-left:10%; font-size:85%; font-style:italic; }
div.author{white-space: nowrap;}
.quotation {margin-bottom:0.25em; margin-top:0.25em; margin-left:1em; }
h1.partHead{text-align: center}
.sectionToc, .likesectionToc {margin-left:2em;}
.subsectionToc, .likesubsectionToc {margin-left:4em;}
.subsubsectionToc, .likesubsubsectionToc {margin-left:6em;}
.frenchb-nbsp{font-size:75%;}
.frenchb-thinspace{font-size:75%;}
.figure img.graphics {margin-left:10%;}
/* end css.sty */

\title{Formes bilineaires}
\author{}
\date{}

\begin{document}
\maketitle

\textbf{Warning: \href{http://www.math.union.edu/locate/jsMath}{jsMath}
requires JavaScript to process the mathematics on this page.\\ If your
browser supports JavaScript, be sure it is enabled.}

\begin{center}\rule{3in}{0.4pt}\end{center}

{[}\href{coursse68.html}{next}{]}
{[}\hyperref[tailcoursse67.html]{tail}{]}
{[}\href{coursch13.html\#coursse67.html}{up}{]}

\subsubsection{12.1 Formes bilinéaires}

\paragraph{12.1.1 Généralités}

Définition~12.1.1 Soit E un K-espace vectoriel . On appelle forme
bilinéaire sur E toute application φ : E × E → K telle que

\begin{itemize}
\itemsep1pt\parskip0pt\parsep0pt
\item
  (i) \textbackslash{}mathop\{∀\}x ∈ E,
  y\textbackslash{}mathrel\{↦\}φ(x,y) est linéaire
\item
  (ii) \textbackslash{}mathop\{∀\}y ∈ E,
  x\textbackslash{}mathrel\{↦\}φ(x,y) est linéaire
\end{itemize}

Remarque~12.1.1 On a en particulier \textbackslash{}mathop\{∀\}x,y ∈ E,
φ(x,0) = φ(0,y) = 0.

Remarque~12.1.2 Il est clair que si φ et ψ sont deux formes bilinéaires
sur E, il en est de même de αφ + βψ, d'où la proposition

Proposition~12.1.1 L'ensemble \{L\}\_\{2\}(E) des formes bilinéaires sur
E est un sous-espace vectoriel de l'espace \{K\}\^{}\{E×E\} des
applications de E × E dans K.

Remarque~12.1.3 Soit φ une forme bilinéaire sur E. Pour chaque x ∈ E,
l'application y\textbackslash{}mathrel\{↦\}φ(x,y) est une forme linéaire
sur E donc un élément, noté \{g\}\_\{φ\}(x), du dual \{E\}\^{}\{∗\} de
E. De même, pour chaque y ∈ E, l'application
x\textbackslash{}mathrel\{↦\}φ(x,y) est une forme linéaire sur E, donc
un élément, noté \{d\}\_\{φ\}(y), de \{E\}\^{}\{∗\}. La relation

\textbackslash{}begin\{eqnarray*\} \textbackslash{}left
{[}\{g\}\_\{φ\}(αx + βx')\textbackslash{}right {]}(y)\& =\& φ(αx +
βx',y) = αφ(x,y) + βφ(x',y)\%\& \textbackslash{}\textbackslash{} \& =\&
\textbackslash{}left {[}α\{g\}\_\{φ\}(x) +
β\{g\}\_\{φ\}(x')\textbackslash{}right {]}(y) \%\&
\textbackslash{}\textbackslash{} \textbackslash{}end\{eqnarray*\}

montre clairement que \{g\}\_\{φ\} :
x\textbackslash{}mathrel\{↦\}\{g\}\_\{φ\}(x) est une application
linéaire de E dans \{E\}\^{}\{∗\}. Il en est évidemment de même de
\{d\}\_\{φ\} : y\textbackslash{}mathrel\{↦\}\{d\}\_\{φ\}(y).

Définition~12.1.2 L'application \{g\}\_\{φ\} : E → \{E\}\^{}\{∗\} (resp.
\{d\}\_\{φ\}) est appelée l'application linéaire gauche (resp. droite)
associée à la forme bilinéaire φ.

\paragraph{12.1.2 Formes bilinéaires symétriques, antisymétriques}

Définition~12.1.3 Soit φ ∈ \{L\}\_\{2\}(E). On dit que φ est symétrique
(resp. antisymétrique) si \textbackslash{}mathop\{∀\}x,y ∈ E, φ(y,x) =
φ(x,y) (resp. = −φ(x,y)).

Proposition~12.1.2 Soit φ ∈ \{L\}\_\{2\}(E). Alors φ est symétrique
(resp. antisymétrique) si et seulement si~\{d\}\_\{φ\} = \{g\}\_\{φ\}
(resp. \{d\}\_\{φ\} = −\{g\}\_\{φ\}).

Démonstration En effet φ(x,y) =\textbackslash{}big
{[}\{g\}\_\{φ\}(x)\textbackslash{}big {]}(y) et φ(y,x)
=\textbackslash{}big {[}\{d\}\_\{φ\}(x)\textbackslash{}big {]}(y). Donc

\textbackslash{}begin\{eqnarray*\} \textbackslash{}mathop\{∀\}x,y ∈ E,
φ(y,x) = εφ(x,y)\&\& \%\& \textbackslash{}\textbackslash{} \&
\textbackslash{}mathrel\{⇔\} \& \textbackslash{}mathop\{∀\}x,y ∈ E,
\textbackslash{}big {[}\{g\}\_\{φ\}(x)\textbackslash{}big {]}(y) =
ε\textbackslash{}big {[}\{d\}\_\{φ\}(x)\textbackslash{}big {]}(y) \%\&
\textbackslash{}\textbackslash{} \& \textbackslash{}mathrel\{⇔\} \&
\textbackslash{}mathop\{∀\}x ∈ E, \{g\}\_\{φ\}(x) = ε\{d\}\_\{φ\}(x)
\textbackslash{}mathrel\{⇔\} \{g\}\_\{φ\} = ε\{d\}\_\{φ\}\%\&
\textbackslash{}\textbackslash{} \textbackslash{}end\{eqnarray*\}

Proposition~12.1.3 L'ensemble \{S\}\_\{2\}(E) (resp. \{A\}\_\{2\}(E))
des formes bilinéaires symétriques (resp. antisymétriques) est un
sous-espace vectoriel de \{L\}\_\{2\}(E). Si la caractéristique de K est
différente de 2, alors \{L\}\_\{2\}(E) = \{S\}\_\{2\}(E) ⊕
\{A\}\_\{2\}(E).

Démonstration La première affirmation est laissée aux soins du lecteur.
Si la caractéristique de K est différente de 2, on a clairement
\{S\}\_\{2\}(E) ∩ \{A\}\_\{2\}(E) =
\textbackslash{}\{0\textbackslash{}\} et la relation φ = ψ + θ avec
ψ(x,y) =\{ 1 \textbackslash{}over 2\} (φ(x,y) + φ(y,x)), θ(x,y) =\{ 1
\textbackslash{}over 2\} (φ(x,y) − φ(y,x)), qui sont respectivement
symétrique et antisymétrique, montre que \{L\}\_\{2\}(E) =
\{S\}\_\{2\}(E) + \{A\}\_\{2\}(E).

\paragraph{12.1.3 Matrice d'une forme bilinéaire}

Supposons que E est de dimension finie et soit ℰ =
(\{e\}\_\{1\},\textbackslash{}mathop\{\textbackslash{}mathop\{\ldots{}\}\},\{e\}\_\{n\})
une base de E.

Définition~12.1.4 Soit φ ∈ \{L\}\_\{2\}(E). On appelle matrice de φ dans
la base ℰ la matrice

\textbackslash{}mathop\{\textbackslash{}mathrm\{Mat\}\} (φ,ℰ) =
\{(φ(\{e\}\_\{i\},\{e\}\_\{j\}))\}\_\{1≤i,j≤n\} ∈ \{M\}\_\{K\}(n)

Proposition~12.1.4
\textbackslash{}mathop\{\textbackslash{}mathrm\{Mat\}\} (φ,ℰ) est
l'unique matrice Ω ∈ \{M\}\_\{K\}(n) vérifiant

\textbackslash{}mathop\{∀\}(x,y) ∈ E × E, φ(x,y) \{= \}\^{}\{t\}XΩY

où X (resp. Y ) désigne le vecteur colonne des coordonnées de x (resp.
y) dans la base ℰ.

Démonstration Si Ω = (\{ω\}\_\{i,j\}), on a

\{ \}\^{}\{t\}XΩY =\{ \textbackslash{}mathop\{∑
\}\}\_\{i=1\}\^{}\{n\}\{x\}\_\{ i\}\{(ΩY )\}\_\{i\} =\{
\textbackslash{}mathop\{∑ \}\}\_\{i=1\}\^{}\{n\}\{x\}\_\{ i\}\{
\textbackslash{}mathop\{∑ \}\}\_\{j=1\}\^{}\{n\}\{ω\}\_\{
i,j\}\{y\}\_\{j\} =\{ \textbackslash{}mathop\{∑
\}\}\_\{i,j\}\{ω\}\_\{i,j\}\{x\}\_\{i\}\{y\}\_\{j\}

Mais d'autre part φ(x,y) =
φ(\{\textbackslash{}mathop\{\textbackslash{}mathop\{∑ \}\}
\}\_\{i=1\}\^{}\{n\}\{x\}\_\{i\}\{e\}\_\{i\},\{\textbackslash{}mathop\{\textbackslash{}mathop\{∑
\}\} \}\_\{j=1\}\^{}\{n\}\{y\}\_\{j\}\{e\}\_\{j\})
=\{\textbackslash{}mathop\{ \textbackslash{}mathop\{∑ \}\}
\}\_\{i,j\}φ(\{e\}\_\{i\},\{e\}\_\{j\})\{x\}\_\{i\}\{y\}\_\{j\} en
utilisant la bilinéarité de φ. Ceci montre que
\textbackslash{}mathop\{\textbackslash{}mathrm\{Mat\}\} (φ,ℰ) vérifie
bien la relation voulue. Inversement, si Ω vérifie cette formule, on a
φ(\{e\}\_\{k\},\{e\}\_\{l\}) \{= \}\^{}\{t\}\{E\}\_\{k\}Ω\{E\}\_\{l\}
=\{\textbackslash{}mathop\{ \textbackslash{}mathop\{∑ \}\}
\}\_\{i,j\}\{ω\}\_\{i,j\}\{δ\}\_\{i\}\^{}\{k\}\{δ\}\_\{j\}\^{}\{l\} =
\{ω\}\_\{k,l\} ce qui montre que Ω =\textbackslash{}mathop\{
\textbackslash{}mathrm\{Mat\}\} (φ,ℰ).

Théorème~12.1.5 L'application
φ\textbackslash{}mathrel\{↦\}\textbackslash{}mathop\{\textbackslash{}mathrm\{Mat\}\}
(φ,ℰ) est un isomorphisme d'espaces vectoriels de \{L\}\_\{2\}(E) sur
\{M\}\_\{K\}(n).

Démonstration Les détails sont laissés aux soins du lecteur.
L'application réciproque est bien entendu l'application qui à Ω ∈
\{M\}\_\{K\}(n) associe φ : E × E → K définie par φ(x,y) \{=
\}\^{}\{t\}XΩY qui est clairement bilinéaire.

Corollaire~12.1.6 Si E est de dimension finie,
\textbackslash{}mathop\{dim\} \{L\}\_\{2\}(E) =
\{(\textbackslash{}mathop\{dim\} E)\}\^{}\{2\}.

Théorème~12.1.7 Soit E de dimension finie, ℰ =
(\{e\}\_\{1\},\textbackslash{}mathop\{\textbackslash{}mathop\{\ldots{}\}\},\{e\}\_\{n\})
une base de E, \{ℰ\}\^{}\{∗\} =
(\{e\}\_\{1\}\^{}\{∗\},\textbackslash{}mathop\{\textbackslash{}mathop\{\ldots{}\}\},\{e\}\_\{n\}\^{}\{∗\})
la base duale. Soit φ ∈ \{L\}\_\{2\}(E). Alors

\textbackslash{}mathop\{\textbackslash{}mathrm\{Mat\}\} (φ,ℰ)
=\textbackslash{}mathop\{ \textbackslash{}mathrm\{Mat\}\}
(\{d\}\_\{φ\},ℰ,\{ℰ\}\^{}\{∗\}) \{= \}\^{}\{t\}\textbackslash{}mathop\{
\textbackslash{}mathrm\{Mat\}\} (\{g\}\_\{ φ\},ℰ,\{ℰ\}\^{}\{∗\})

Démonstration Notons Ω =\textbackslash{}mathop\{
\textbackslash{}mathrm\{Mat\}\} (φ,ℰ), A =\textbackslash{}mathop\{
\textbackslash{}mathrm\{Mat\}\} (\{d\}\_\{φ\},ℰ,\{ℰ\}\^{}\{∗\}) et B
=\textbackslash{}mathop\{ \textbackslash{}mathrm\{Mat\}\}
(\{g\}\_\{φ\},ℰ,\{ℰ\}\^{}\{∗\}). On a

\textbackslash{}begin\{eqnarray*\}\{ ω\}\_\{i,j\}\& =\&
φ(\{e\}\_\{i\},\{e\}\_\{j\}) = \textbackslash{}left
(\{d\}\_\{φ\}(\{e\}\_\{j\})\textbackslash{}right )(\{e\}\_\{i\}) \%\&
\textbackslash{}\textbackslash{} \& =\& \textbackslash{}left
(\{\textbackslash{}mathop\{∑ \}\}\_\{k=1\}\^{}\{n\}\{a\}\_\{
k,j\}\{e\}\_\{k\}\^{}\{∗\}\textbackslash{}right )(\{e\}\_\{ i\}) =
\{a\}\_\{i,j\}\%\& \textbackslash{}\textbackslash{}
\textbackslash{}end\{eqnarray*\}

compte tenu de \{e\}\_\{k\}\^{}\{∗\}(\{e\}\_\{i\}) =
\{δ\}\_\{k\}\^{}\{i\}~; de même

\textbackslash{}begin\{eqnarray*\}\{ ω\}\_\{i,j\}\& =\&
φ(\{e\}\_\{i\},\{e\}\_\{j\}) = \textbackslash{}left
(\{g\}\_\{φ\}(\{e\}\_\{i\})\textbackslash{}right )(\{e\}\_\{j\}) \%\&
\textbackslash{}\textbackslash{} \& =\& \textbackslash{}left
(\{\textbackslash{}mathop\{∑ \}\}\_\{k=1\}\^{}\{n\}\{b\}\_\{
k,i\}\{e\}\_\{k\}\^{}\{∗\}\textbackslash{}right )(\{e\}\_\{ j\}) =
\{b\}\_\{j,i\}\%\& \textbackslash{}\textbackslash{}
\textbackslash{}end\{eqnarray*\}

ce qui démontre le résultat.

Corollaire~12.1.8 La forme bilinéaire φ est symétrique (resp.
antisymétrique) si et seulement si~sa matrice dans la base ℰ est
symétrique (resp. antisymétrique).

Le rang de \textbackslash{}mathop\{\textbackslash{}mathrm\{Mat\}\}
(\{d\}\_\{φ\},ℰ,\{ℰ\}\^{}\{∗\}) est indépendant du choix de la base ℰ~;
il en est donc de même du rang de
\textbackslash{}mathop\{\textbackslash{}mathrm\{Mat\}\} (φ,ℰ). Ceci
conduit à la définition suivante

Définition~12.1.5 Soit E de dimension finie et φ ∈ \{L\}\_\{2\}(E). On
appelle rang de E le rang de sa matrice dans n'importe quelle base de E.
On a

\textbackslash{}mathop\{\textbackslash{}mathrm\{rg\}\}φ
=\textbackslash{}mathop\{ \textbackslash{}mathrm\{rg\}\}\{d\}\_\{φ\}
=\textbackslash{}mathop\{ \textbackslash{}mathrm\{rg\}\}\{g\}\_\{φ\}
=\textbackslash{}mathop\{
\textbackslash{}mathrm\{rg\}\}\textbackslash{}mathop\{\textbackslash{}mathrm\{Mat\}\}
(φ,ℰ)

\paragraph{12.1.4 Changements de bases, discriminant}

Théorème~12.1.9 Soit E un espace vectoriel de dimension finie, ℰ et ℰ'
deux bases de E, P = \{P\}\_\{ℰ\}\^{}\{ℰ'\} la matrice de passage de ℰ à
ℰ'. Soit φ ∈ \{L\}\_\{2\}(E), Ω =\textbackslash{}mathop\{
\textbackslash{}mathrm\{Mat\}\} (φ,ℰ) et Ω' =\textbackslash{}mathop\{
\textbackslash{}mathrm\{Mat\}\} (φ,ℰ'). Alors

Ω' \{= \}\^{}\{t\}PΩP

Démonstration Si X (resp. Y ) désigne le vecteur colonne des coordonnées
de x (resp. y) dans la base ℰ et X' (resp. Y ') désigne le vecteur
colonne des coordonnées de x (resp. y) dans la base ℰ', on a X = PX', Y
= PY ', d'où

φ(x,y) \{= \}\^{}\{t\}(PX')Ω(PY ) \{= \}\^{}\{t\}X'\{(\}\^{}\{t\}PΩP)Y '

Comme Ω' est l'unique matrice vérifiant \textbackslash{}mathop\{∀\}(x,y)
∈ E × E, φ(x,y) \{= \}\^{}\{t\}X'Ω'Y ', on a Ω' \{= \}\^{}\{t\}PΩP.

Définition~12.1.6 Soit E un espace vectoriel de dimension finie, ℰ une
base de E et φ ∈ \{L\}\_\{2\}(E). On appelle discriminant de φ dans la
base ℰ le déterminant de la matrice
\textbackslash{}mathop\{\textbackslash{}mathrm\{Mat\}\} (φ,ℰ).

Remarque~12.1.4 La formule ci dessus montre que lors d'un changement de
base, le discriminant est multiplié par
\{(\textbackslash{}mathop\{\textbackslash{}mathrm\{det\}\}
P)\}\^{}\{2\}.

On introduit ainsi une nouvelle relation d'équivalence sur les matrices
carrées d'ordre n~: représenter une même forme bilinéaire dans des bases
différentes.

Définition~12.1.7 Soit Ω,Ω' ∈ \{M\}\_\{K\}(n). On dit que ces deux
matrices sont congruentes s'il existe P ∈ G\{L\}\_\{K\}(n) telle que Ω'
\{= \}\^{}\{t\}PΩP. Il s'agit d'une relation d'équivalence sur
\{M\}\_\{K\}(n).

Remarque~12.1.5 Bien entendu cette relation de congruence laisse stables
les sous-espaces vectoriels des matrices symétriques ou antisymétriques.

\paragraph{12.1.5 Orthogonalité}

Soit E un K-espace vectoriel ~et φ une forme bilinéaire sur E.

Définition~12.1.8 On dit que x est orthogonal à y (relativement à φ), et
on pose x ⊥ y, si φ(x,y) = 0.

Définition~12.1.9 Soit A une partie de E. On pose

\begin{itemize}
\itemsep1pt\parskip0pt\parsep0pt
\item
  (i) \{A\}\^{}\{⊥\} = \textbackslash{}\{x ∈
  E\textbackslash{}mathrel\{∣\}\textbackslash{}mathop\{∀\}a ∈ A, φ(a,x)
  = 0\textbackslash{}\}
\item
  (ii) \{\}\^{}\{⊥\} A = \textbackslash{}\{x ∈
  E\textbackslash{}mathrel\{∣\}\textbackslash{}mathop\{∀\}a ∈ A, φ(x,a)
  = 0\textbackslash{}\}
\end{itemize}

Remarque~12.1.6 Notons \{A\}\^{}\{\{⊥\}\^{}\{∗\} \} l'orthogonal de A
dans le dual \{E\}\^{}\{∗\} de E, c'est-à-dire l'espace vectoriel des
formes linéaires sur E qui sont nulles sur A. On a

\textbackslash{}begin\{eqnarray*\} x ∈ \{A\}\^{}\{⊥\}\&
\textbackslash{}mathrel\{⇔\} \& \textbackslash{}mathop\{∀\}a ∈ A, φ(a,x)
= 0 \%\& \textbackslash{}\textbackslash{} \&
\textbackslash{}mathrel\{⇔\} \& \textbackslash{}mathop\{∀\}a ∈ A,
\textbackslash{}big {[}\{d\}\_\{φ\}(x)\textbackslash{}big {]}(a) = 0
\%\& \textbackslash{}\textbackslash{} \& \textbackslash{}mathrel\{⇔\} \&
\{d\}\_\{φ\}(x) ∈ \{A\}\^{}\{\{⊥\}\^{}\{∗\} \}
\textbackslash{}mathrel\{⇔\} x ∈
\{d\}\_\{φ\}\^{}\{−1\}(\{A\}\^{}\{\{⊥\}\^{}\{∗\} \})\%\&
\textbackslash{}\textbackslash{} \textbackslash{}end\{eqnarray*\}

On en déduit que \{A\}\^{}\{⊥\} =
\{d\}\_\{φ\}\^{}\{−1\}(\{A\}\^{}\{\{⊥\}\^{}\{∗\} \}) et \{\}\^{}\{⊥\} A
= \{g\}\_\{φ\}\^{}\{−1\}(\{A\}\^{}\{\{⊥\}\^{}\{∗\} \}).

Proposition~12.1.10 Soit A une partie de E~; alors

\begin{itemize}
\itemsep1pt\parskip0pt\parsep0pt
\item
  (i)\{A\}\^{}\{⊥\} et \{\}\^{}\{⊥\} A sont des sous espaces vectoriels
  de E
\item
  (ii)\{A\}\^{}\{⊥\} =\textbackslash{}mathop\{
  \textbackslash{}mathrm\{Vect\}\}\{(A)\}\^{}\{⊥\} et \{\}\^{}\{⊥\} A
  \{= \}\^{}\{⊥\}
  \textbackslash{}mathop\{\textbackslash{}mathrm\{Vect\}\}(A)
\item
  (iii) A \{⊂\}\^{}\{⊥\} (\{A\}\^{}\{⊥\}) et A ⊂ \{\{(\}\^{}\{⊥\}
  A)\}\^{}\{⊥\}
\item
  (iv) A ⊂ B ⇒ \{B\}\^{}\{⊥\}⊂ \{A\}\^{}\{⊥\} et \{\}\^{}\{⊥\} B
  \{⊂\}\^{}\{⊥\} A.
\end{itemize}

Démonstration (i) découle immédiatement de la bilinéarité de φ ou de la
remarque précédente. Il en est de même pour (ii) puisqu'un vecteur x est
orthogonal (aussi bien à gauche qu'à droite) à tout vecteur de A si et
seulement si il est orthogonal à toute combinaison linéaire de vecteurs
de A, c'est à dire à
\textbackslash{}mathop\{\textbackslash{}mathrm\{Vect\}\}(A). En ce qui
concerne (iii), il suffit de remarquer que tout vecteur a de A est
orthogonal à tout vecteur qui est orthogonal à tout vecteur de A. Pour
(iv), un vecteur x qui est orthogonal à tout vecteur de B est évidemment
orthogonal à tout vecteur de A.

Remarque~12.1.7 Dans le cas où φ est symétrique ou antisymétrique, on a
φ(x,y) = 0 \textbackslash{}mathrel\{⇔\} φ(y,x) = 0, si bien que la
relation d'orthogonalité est symétrique. Dans ce cas, il n'y a pas lieu
de distinguer \{\}\^{}\{⊥\} A de \{A\}\^{}\{⊥\}. Dans toute la suite
nous ferons l'hypothèse que φ est soit symétrique, soit antisymétrique.

\paragraph{12.1.6 Formes non dégénérées}

En règle générale on posera

Définition~12.1.10 Soit E un K-espace vectoriel , φ une forme bilinéaire
symétrique (resp. antisymétrique) sur E. On appelle noyau de φ le
sous-espace

\textbackslash{}mathop\{\textbackslash{}mathrm\{Ker\}\}φ =
\textbackslash{}\{x ∈
E\textbackslash{}mathrel\{∣\}\textbackslash{}mathop\{∀\}y ∈ E, φ(x,y) =
0\textbackslash{}\} = \{E\}\^{}\{⊥\} =\textbackslash{}mathop\{
\textbackslash{}mathrm\{Ker\}\}\{d\}\_\{ φ\}

Définition~12.1.11 Soit E un K-espace vectoriel , φ une forme bilinéaire
symétrique (resp. antisymétrique) sur E. On dit que φ est non dégénérée
si elle vérifie les conditions équivalentes

\begin{itemize}
\itemsep1pt\parskip0pt\parsep0pt
\item
  (i) \textbackslash{}mathop\{\textbackslash{}mathrm\{Ker\}\}φ =
  \{E\}\^{}\{⊥\} = \textbackslash{}\{0\textbackslash{}\}
\item
  (ii) pour x ∈ E on a \textbackslash{}left
  (\textbackslash{}mathop\{∀\}y ∈ E, φ(x,y) = 0\textbackslash{}right ) ⇒
  x = 0
\item
  (iii) \{d\}\_\{φ\} (resp. \{g\}\_\{φ\}) est une application linéaire
  injective de E dans \{E\}\^{}\{∗\}.
\end{itemize}

L'équivalence entre ces trois propriétés est évidente.

Si E est un espace vectoriel de dimension finie, on sait que
\textbackslash{}mathop\{dim\} \{E\}\^{}\{∗\} =\textbackslash{}mathop\{
dim\} E. Si \{d\}\_\{φ\} est injective, elle est nécessairement
bijective et on obtient

Théorème~12.1.11 Soit E un K-espace vectoriel ~de dimension finie, φ une
forme bilinéaire symétrique (resp. antisymétrique) non dégénérée sur E.
Alors l'application linéaire droite \{d\}\_\{φ\} est un isomorphisme
d'espace vectoriel de E sur \{E\}\^{}\{∗\}~; autrement dit, pour toute
forme linéaire f sur E, il existe un unique vecteur \{v\}\_\{f\} ∈ E tel
que \textbackslash{}mathop\{∀\}x ∈ E, f(x) = φ(x,\{v\}\_\{f\}).

Corollaire~12.1.12 Soit E un K-espace vectoriel ~de dimension finie, φ
une forme bilinéaire symétrique (resp. antisymétrique) non dégénérée sur
E. Soit A un sous-espace vectoriel de E. Alors
\textbackslash{}mathop\{dim\} A +\textbackslash{}mathop\{ dim\}
\{A\}\^{}\{⊥\} =\textbackslash{}mathop\{ dim\} E et A = \{A\}\^{}\{⊥⊥\}.

Démonstration On a en effet

\textbackslash{}mathop\{dim\} \{A\}\^{}\{⊥\} =\textbackslash{}mathop\{
dim\} \{d\}\_\{ φ\}\^{}\{−1\}(\{A\}\^{}\{\{⊥\}\^{}\{∗\} \})
=\textbackslash{}mathop\{ dim\} \{A\}\^{}\{\{⊥\}\^{}\{∗\} \}
=\textbackslash{}mathop\{ dim\} E −\textbackslash{}mathop\{ dim\} A

puisque \{d\}\_\{φ\} est un isomorphisme d'espaces vectoriels. On sait
d'autre part que A ⊂ \{A\}\^{}\{⊥⊥\} et que
\textbackslash{}mathop\{dim\} \{A\}\^{}\{⊥⊥\} =\textbackslash{}mathop\{
dim\} E −\textbackslash{}mathop\{ dim\} \{A\}\^{}\{⊥\}
=\textbackslash{}mathop\{ dim\} A, d'où l'égalité.

Remarque~12.1.8 Il ne faudrait pas en déduire abusivement que A et
\{A\}\^{}\{⊥\} sont supplémentaires~; en effet, en général A ∩
\{A\}\^{}\{⊥\}\textbackslash{}mathrel\{≠\}\textbackslash{}\{0\textbackslash{}\}.
Nous nous intéresserons plus particulièrement à ce point dans le
paragraphe suivant.

Si ℰ est une base de E, alors Ω =\textbackslash{}mathop\{
\textbackslash{}mathrm\{Mat\}\} (φ,ℰ) =\textbackslash{}mathop\{
\textbackslash{}mathrm\{Mat\}\} (\{d\}\_\{φ\},ℰ,\{ℰ\}\^{}\{∗\}) et
\textbackslash{}mathop\{\textbackslash{}mathrm\{rg\}\}φ
=\textbackslash{}mathop\{ \textbackslash{}mathrm\{rg\}\}Ω. On en déduit

Théorème~12.1.13 Soit E un K-espace vectoriel ~de dimension finie n, φ
une forme bilinéaire symétrique (resp. antisymétrique) sur E, ℰ une base
de E et Ω =\textbackslash{}mathop\{ \textbackslash{}mathrm\{Mat\}\}
(φ,ℰ). Alors les propositions suivantes sont équivalentes

\begin{itemize}
\itemsep1pt\parskip0pt\parsep0pt
\item
  (i) φ est non dégénérée
\item
  (ii) Ω est une matrice inversible
\item
  (iii) \textbackslash{}mathop\{\textbackslash{}mathrm\{rg\}\}φ = n.
\end{itemize}

Remarque~12.1.9 En général,
\textbackslash{}mathop\{\textbackslash{}mathrm\{Ker\}\}φ
=\textbackslash{}mathop\{ \textbackslash{}mathrm\{Ker\}\}\{d\}\_\{φ\},
\textbackslash{}mathop\{\textbackslash{}mathrm\{rg\}\}φ
=\textbackslash{}mathop\{ \textbackslash{}mathrm\{rg\}\}\{d\}\_\{φ\}, si
bien que le théorème du rang devient

Proposition~12.1.14 Soit E un K-espace vectoriel ~de dimension finie n,
φ une forme bilinéaire symétrique (resp. antisymétrique) sur E, ℰ une
base de E. Alors \textbackslash{}mathop\{dim\} E
=\textbackslash{}mathop\{ \textbackslash{}mathrm\{rg\}\}φ
+\textbackslash{}mathop\{ dim\}
\textbackslash{}mathop\{\textbackslash{}mathrm\{Ker\}\}φ.

\paragraph{12.1.7 Isotropie}

Définition~12.1.12 Soit E un K-espace vectoriel , φ une forme bilinéaire
symétrique (resp. antisymétrique) sur E. On dit qu'un sous-espace
vectoriel A de E est non isotrope s'il vérifie les conditions
équivalentes

\begin{itemize}
\itemsep1pt\parskip0pt\parsep0pt
\item
  (i) A ∩ \{A\}\^{}\{⊥\} = \textbackslash{}\{0\textbackslash{}\}
\item
  (ii) la restriction de φ à A × A est non dégénérée.
\end{itemize}

Démonstration On a en effet
\textbackslash{}mathop\{\textbackslash{}mathrm\{Ker\}\}\{φ\}\_\{\{\textbar{}\}\_\{A×A\}\}
= \textbackslash{}\{x ∈
A\textbackslash{}mathrel\{∣\}\textbackslash{}mathop\{∀\}y ∈ A, φ(x,y) =
0\textbackslash{}\} = \textbackslash{}\{x ∈
A\textbackslash{}mathrel\{∣\}x ∈ \{A\}\^{}\{⊥\}\textbackslash{}\} = A ∩
\{A\}\^{}\{⊥\}.

Définition~12.1.13 Soit E un K-espace vectoriel , φ une forme bilinéaire
symétrique (resp. antisymétrique) sur E. On dit que x ∈ E est un vecteur
isotrope s'il vérifie les conditions équivalentes suivantes

\begin{itemize}
\itemsep1pt\parskip0pt\parsep0pt
\item
  (i) la droite Kx est un sous-espace isotrope ou x = 0
\item
  (ii) φ(x,x) = 0
\end{itemize}

Démonstration (i) ⇒(ii)~: soit y ∈ Kx ∩
\{(Kx)\}\^{}\{⊥\}∖\textbackslash{}\{0\textbackslash{}\}~; on a y = λx
avec λ\textbackslash{}mathrel\{≠\}0, d'où 0 = φ(y,y) =
\{λ\}\^{}\{2\}φ(x,x), soit φ(x,x) = 0.

(ii) ⇒(i) si φ(x,x) = 0, on a clairement Kx ⊂ \{(Kx)\}\^{}\{⊥\}.

Remarque~12.1.10 Il est clair que si φ est antisymétrique et si
\textbackslash{}mathop\{\textbackslash{}mathrm\{car\}\}K\textbackslash{}mathrel\{≠\}2,
alors tout vecteur est isotrope. La notion n'est donc réellement
intéressante que pour les formes symétriques.

Exemple~12.1.1 Pour la forme bilinéaire symétrique sur \{ℝ\}\^{}\{4\},
φ(x,y) = \{x\}\_\{1\}\{y\}\_\{1\} + \{x\}\_\{2\}\{y\}\_\{2\} +
\{x\}\_\{3\}\{y\}\_\{3\} − \{x\}\_\{4\}\{y\}\_\{4\} (forme de Lorentz,
celle de la relativité), le vecteur (1,0,0,1) est isotrope~; cette forme
est bien entendu non dégénérée puisque sa matrice dans la base canonique
est la matrice
\textbackslash{}mathop\{\textbackslash{}mathrm\{diag\}\}(1,1,1,−1) qui
est inversible~; on voit donc que φ peut être non dégénérée, alors que
sa restriction à un sous-espace est dégénérée (et même nulle).

Définition~12.1.14 On dit que la forme bilinéaire symétrique φ sur E est
définie s'il n'existe pas de vecteur isotrope autre que 0.

Remarque~12.1.11 Si A est un sous-espace isotrope, alors tout vecteur de
A ∩ \{A\}\^{}\{⊥\}∖\textbackslash{}\{0\textbackslash{}\} est clairement
isotrope (étant orthogonal à tout vecteur de A, il est orthogonal à lui
même). On en déduit que si φ est une forme bilinéaire symétrique
définie, alors tout sous-espace de E est non isotrope. En particulier, E
lui même est non isotrope et donc

Proposition~12.1.15 Soit φ une forme bilinéaire symétrique définie~;
alors φ est non dégénérée et tout sous-espace est non isotrope pour φ.

Théorème~12.1.16 Soit E un K-espace vectoriel ~de dimension finie, φ une
forme bilinéaire symétrique (resp. antisymétrique) non dégénérée sur E.
Soit A un sous-espace vectoriel de E. Alors on a l'équivalence de

\begin{itemize}
\itemsep1pt\parskip0pt\parsep0pt
\item
  (i) A est non isotrope
\item
  (ii) E = A ⊕ \{A\}\^{}\{⊥\}
\end{itemize}

Démonstration En effet on sait que \textbackslash{}mathop\{dim\} A
+\textbackslash{}mathop\{ dim\} \{A\}\^{}\{⊥\} =\textbackslash{}mathop\{
dim\} E. On a donc

A ∩ \{A\}\^{}\{⊥\} = \textbackslash{}\{0\textbackslash{}\}
\textbackslash{}mathrel\{⇔\} E = A ⊕ \{A\}\^{}\{⊥\}

Corollaire~12.1.17 Soit E un K-espace vectoriel ~de dimension finie, φ
une forme bilinéaire symétrique définie sur E. Soit A un sous-espace
vectoriel de E. Alors E = A ⊕ \{A\}\^{}\{⊥\}.

{[}\href{coursse68.html}{next}{]} {[}\href{coursse67.html}{front}{]}
{[}\href{coursch13.html\#coursse67.html}{up}{]}

\end{document}
