\documentclass[]{article}
\usepackage[T1]{fontenc}
\usepackage{lmodern}
\usepackage{amssymb,amsmath}
\usepackage{ifxetex,ifluatex}
\usepackage{fixltx2e} % provides \textsubscript
% use upquote if available, for straight quotes in verbatim environments
\IfFileExists{upquote.sty}{\usepackage{upquote}}{}
\ifnum 0\ifxetex 1\fi\ifluatex 1\fi=0 % if pdftex
  \usepackage[utf8]{inputenc}
\else % if luatex or xelatex
  \ifxetex
    \usepackage{mathspec}
    \usepackage{xltxtra,xunicode}
  \else
    \usepackage{fontspec}
  \fi
  \defaultfontfeatures{Mapping=tex-text,Scale=MatchLowercase}
  \newcommand{\euro}{€}
\fi
% use microtype if available
\IfFileExists{microtype.sty}{\usepackage{microtype}}{}
\ifxetex
  \usepackage[setpagesize=false, % page size defined by xetex
              unicode=false, % unicode breaks when used with xetex
              xetex]{hyperref}
\else
  \usepackage[unicode=true]{hyperref}
\fi
\hypersetup{breaklinks=true,
            bookmarks=true,
            pdfauthor={},
            pdftitle={Formes bilineaires},
            colorlinks=true,
            citecolor=blue,
            urlcolor=blue,
            linkcolor=magenta,
            pdfborder={0 0 0}}
\urlstyle{same}  % don't use monospace font for urls
\setlength{\parindent}{0pt}
\setlength{\parskip}{6pt plus 2pt minus 1pt}
\setlength{\emergencystretch}{3em}  % prevent overfull lines
\setcounter{secnumdepth}{0}
 
/* start css.sty */
.cmr-5{font-size:50%;}
.cmr-7{font-size:70%;}
.cmmi-5{font-size:50%;font-style: italic;}
.cmmi-7{font-size:70%;font-style: italic;}
.cmmi-10{font-style: italic;}
.cmsy-5{font-size:50%;}
.cmsy-7{font-size:70%;}
.cmex-7{font-size:70%;}
.cmex-7x-x-71{font-size:49%;}
.msbm-7{font-size:70%;}
.cmtt-10{font-family: monospace;}
.cmti-10{ font-style: italic;}
.cmbx-10{ font-weight: bold;}
.cmr-17x-x-120{font-size:204%;}
.cmsl-10{font-style: oblique;}
.cmti-7x-x-71{font-size:49%; font-style: italic;}
.cmbxti-10{ font-weight: bold; font-style: italic;}
p.noindent { text-indent: 0em }
td p.noindent { text-indent: 0em; margin-top:0em; }
p.nopar { text-indent: 0em; }
p.indent{ text-indent: 1.5em }
@media print {div.crosslinks {visibility:hidden;}}
a img { border-top: 0; border-left: 0; border-right: 0; }
center { margin-top:1em; margin-bottom:1em; }
td center { margin-top:0em; margin-bottom:0em; }
.Canvas { position:relative; }
li p.indent { text-indent: 0em }
.enumerate1 {list-style-type:decimal;}
.enumerate2 {list-style-type:lower-alpha;}
.enumerate3 {list-style-type:lower-roman;}
.enumerate4 {list-style-type:upper-alpha;}
div.newtheorem { margin-bottom: 2em; margin-top: 2em;}
.obeylines-h,.obeylines-v {white-space: nowrap; }
div.obeylines-v p { margin-top:0; margin-bottom:0; }
.overline{ text-decoration:overline; }
.overline img{ border-top: 1px solid black; }
td.displaylines {text-align:center; white-space:nowrap;}
.centerline {text-align:center;}
.rightline {text-align:right;}
div.verbatim {font-family: monospace; white-space: nowrap; text-align:left; clear:both; }
.fbox {padding-left:3.0pt; padding-right:3.0pt; text-indent:0pt; border:solid black 0.4pt; }
div.fbox {display:table}
div.center div.fbox {text-align:center; clear:both; padding-left:3.0pt; padding-right:3.0pt; text-indent:0pt; border:solid black 0.4pt; }
div.minipage{width:100%;}
div.center, div.center div.center {text-align: center; margin-left:1em; margin-right:1em;}
div.center div {text-align: left;}
div.flushright, div.flushright div.flushright {text-align: right;}
div.flushright div {text-align: left;}
div.flushleft {text-align: left;}
.underline{ text-decoration:underline; }
.underline img{ border-bottom: 1px solid black; margin-bottom:1pt; }
.framebox-c, .framebox-l, .framebox-r { padding-left:3.0pt; padding-right:3.0pt; text-indent:0pt; border:solid black 0.4pt; }
.framebox-c {text-align:center;}
.framebox-l {text-align:left;}
.framebox-r {text-align:right;}
span.thank-mark{ vertical-align: super }
span.footnote-mark sup.textsuperscript, span.footnote-mark a sup.textsuperscript{ font-size:80%; }
div.tabular, div.center div.tabular {text-align: center; margin-top:0.5em; margin-bottom:0.5em; }
table.tabular td p{margin-top:0em;}
table.tabular {margin-left: auto; margin-right: auto;}
div.td00{ margin-left:0pt; margin-right:0pt; }
div.td01{ margin-left:0pt; margin-right:5pt; }
div.td10{ margin-left:5pt; margin-right:0pt; }
div.td11{ margin-left:5pt; margin-right:5pt; }
table[rules] {border-left:solid black 0.4pt; border-right:solid black 0.4pt; }
td.td00{ padding-left:0pt; padding-right:0pt; }
td.td01{ padding-left:0pt; padding-right:5pt; }
td.td10{ padding-left:5pt; padding-right:0pt; }
td.td11{ padding-left:5pt; padding-right:5pt; }
table[rules] {border-left:solid black 0.4pt; border-right:solid black 0.4pt; }
.hline hr, .cline hr{ height : 1px; margin:0px; }
.tabbing-right {text-align:right;}
span.TEX {letter-spacing: -0.125em; }
span.TEX span.E{ position:relative;top:0.5ex;left:-0.0417em;}
a span.TEX span.E {text-decoration: none; }
span.LATEX span.A{ position:relative; top:-0.5ex; left:-0.4em; font-size:85%;}
span.LATEX span.TEX{ position:relative; left: -0.4em; }
div.float img, div.float .caption {text-align:center;}
div.figure img, div.figure .caption {text-align:center;}
.marginpar {width:20%; float:right; text-align:left; margin-left:auto; margin-top:0.5em; font-size:85%; text-decoration:underline;}
.marginpar p{margin-top:0.4em; margin-bottom:0.4em;}
.equation td{text-align:center; vertical-align:middle; }
td.eq-no{ width:5%; }
table.equation { width:100%; } 
div.math-display, div.par-math-display{text-align:center;}
math .texttt { font-family: monospace; }
math .textit { font-style: italic; }
math .textsl { font-style: oblique; }
math .textsf { font-family: sans-serif; }
math .textbf { font-weight: bold; }
.partToc a, .partToc, .likepartToc a, .likepartToc {line-height: 200%; font-weight:bold; font-size:110%;}
.chapterToc a, .chapterToc, .likechapterToc a, .likechapterToc, .appendixToc a, .appendixToc {line-height: 200%; font-weight:bold;}
.index-item, .index-subitem, .index-subsubitem {display:block}
.caption td.id{font-weight: bold; white-space: nowrap; }
table.caption {text-align:center;}
h1.partHead{text-align: center}
p.bibitem { text-indent: -2em; margin-left: 2em; margin-top:0.6em; margin-bottom:0.6em; }
p.bibitem-p { text-indent: 0em; margin-left: 2em; margin-top:0.6em; margin-bottom:0.6em; }
.paragraphHead, .likeparagraphHead { margin-top:2em; font-weight: bold;}
.subparagraphHead, .likesubparagraphHead { font-weight: bold;}
.quote {margin-bottom:0.25em; margin-top:0.25em; margin-left:1em; margin-right:1em; text-align:\jmathustify;}
.verse{white-space:nowrap; margin-left:2em}
div.maketitle {text-align:center;}
h2.titleHead{text-align:center;}
div.maketitle{ margin-bottom: 2em; }
div.author, div.date {text-align:center;}
div.thanks{text-align:left; margin-left:10%; font-size:85%; font-style:italic; }
div.author{white-space: nowrap;}
.quotation {margin-bottom:0.25em; margin-top:0.25em; margin-left:1em; }
h1.partHead{text-align: center}
.sectionToc, .likesectionToc {margin-left:2em;}
.subsectionToc, .likesubsectionToc {margin-left:4em;}
.subsubsectionToc, .likesubsubsectionToc {margin-left:6em;}
.frenchb-nbsp{font-size:75%;}
.frenchb-thinspace{font-size:75%;}
.figure img.graphics {margin-left:10%;}
/* end css.sty */

\title{Formes bilineaires}
\author{}
\date{}

\begin{document}
\maketitle

\textbf{Warning: 
requires JavaScript to process the mathematics on this page.\\ If your
browser supports JavaScript, be sure it is enabled.}

\begin{center}\rule{3in}{0.4pt}\end{center}

{[}
{[}{]}
{[}

\subsubsection{12.1 Formes bilinéaires}

\paragraph{12.1.1 Généralités}

Définition~12.1.1 Soit E un K-espace vectoriel . On appelle forme
bilinéaire sur E toute application \phi : E \times E \rightarrow~ K telle que

\begin{itemize}
\itemsep1pt\parskip0pt\parsep0pt
\item
  (i) \forall~~x \in E,
  y\mapsto~\phi(x,y) est linéaire
\item
  (ii) \forall~~y \in E,
  x\mapsto~\phi(x,y) est linéaire
\end{itemize}

Remarque~12.1.1 On a en particulier \forall~~x,y \in E,
\phi(x,0) = \phi(0,y) = 0.

Remarque~12.1.2 Il est clair que si \phi et \psi sont deux formes bilinéaires
sur E, il en est de même de \alpha~\phi + \beta~\psi, d'où la proposition

Proposition~12.1.1 L'ensemble L\_2(E) des formes bilinéaires sur
E est un sous-espace vectoriel de l'espace K^E\timesE des
applications de E \times E dans K.

Remarque~12.1.3 Soit \phi une forme bilinéaire sur E. Pour chaque x \in E,
l'application y\mapsto~\phi(x,y) est une forme linéaire
sur E donc un élément, noté g\_\phi(x), du dual E^∗ de
E. De même, pour chaque y \in E, l'application
x\mapsto~\phi(x,y) est une forme linéaire sur E, donc
un élément, noté d\_\phi(y), de E^∗. La relation

\begin{align*} \left
{[}g\_\phi(\alpha~x + \beta~x')\right {]}(y)& =& \phi(\alpha~x +
\beta~x',y) = \alpha~\phi(x,y) + \beta~\phi(x',y)\%& \\ & =&
\left {[}\alpha~g\_\phi(x) +
\beta~g\_\phi(x')\right {]}(y) \%&
\\ \end{align*}

montre clairement que g\_\phi :
x\mapsto~g\_\phi(x) est une application
linéaire de E dans E^∗. Il en est évidemment de même de
d\_\phi : y\mapsto~d\_\phi(y).

Définition~12.1.2 L'application g\_\phi : E \rightarrow~ E^∗ (resp.
d\_\phi) est appelée l'application linéaire gauche (resp. droite)
associée à la forme bilinéaire \phi.

\paragraph{12.1.2 Formes bilinéaires symétriques, antisymétriques}

Définition~12.1.3 Soit \phi \in L\_2(E). On dit que \phi est symétrique
(resp. antisymétrique) si \forall~~x,y \in E, \phi(y,x) =
\phi(x,y) (resp. = -\phi(x,y)).

Proposition~12.1.2 Soit \phi \in L\_2(E). Alors \phi est symétrique
(resp. antisymétrique) si et seulement si~d\_\phi = g\_\phi
(resp. d\_\phi = -g\_\phi).

Démonstration En effet \phi(x,y) =\big
{[}g\_\phi(x)\big {]}(y) et \phi(y,x)
=\big {[}d\_\phi(x)\big {]}(y). Donc

\begin{align*} \forall~~x,y \in E,
\phi(y,x) = \epsilon\phi(x,y)&& \%& \\ &
\Leftrightarrow & \forall~~x,y \in E,
\big {[}g\_\phi(x)\big {]}(y) =
\epsilon\big {[}d\_\phi(x)\big {]}(y) \%&
\\ & \Leftrightarrow &
\forall~x \in E, g\_\phi(x) = \epsilond\_\phi~(x)
\Leftrightarrow g\_\phi = \epsilond\_\phi\%&
\\ \end{align*}

Proposition~12.1.3 L'ensemble S\_2(E) (resp. A\_2(E))
des formes bilinéaires symétriques (resp. antisymétriques) est un
sous-espace vectoriel de L\_2(E). Si la caractéristique de K est
différente de 2, alors L\_2(E) = S\_2(E) \oplus~
A\_2(E).

Démonstration La première affirmation est laissée aux soins du lecteur.
Si la caractéristique de K est différente de 2, on a clairement
S\_2(E) \bigcap A\_2(E) =
\0\ et la relation \phi = \psi + \theta avec
\psi(x,y) = 1 \over 2 (\phi(x,y) + \phi(y,x)), \theta(x,y) = 1
\over 2 (\phi(x,y) - \phi(y,x)), qui sont respectivement
symétrique et antisymétrique, montre que L\_2(E) =
S\_2(E) + A\_2(E).

\paragraph{12.1.3 Matrice d'une forme bilinéaire}

Supposons que E est de dimension finie et soit \mathcal{E} =
(e\_1,\\ldots,e\_n~)
une base de E.

Définition~12.1.4 Soit \phi \in L\_2(E). On appelle matrice de \phi dans
la base \mathcal{E} la matrice

\mathrmMat~ (\phi,\mathcal{E}) =
(\phi(e\_i,e\_\jmath))\_1\leqi,\jmath\leqn \in M\_K(n)

Proposition~12.1.4
\mathrmMat~ (\phi,\mathcal{E}) est
l'unique matrice \Omega \in M\_K(n) vérifiant

\forall~(x,y) \in E \times E, \phi(x,y) = ^t~X\OmegaY

où X (resp. Y ) désigne le vecteur colonne des coordonnées de x (resp.
y) dans la base \mathcal{E}.

Démonstration Si \Omega = (\omega\_i,\jmath), on a

 ^tX\OmegaY = \\sum
\_i=1^nx\_ i(\OmegaY )\_i =
\sum \_i=1^nx\_ i~
\sum \_\jmath=1^n\omega~\_
i,\jmathy\_\jmath = \\sum
\_i,\jmath\omega\_i,\jmathx\_iy\_\jmath

Mais d'autre part \phi(x,y) =
\phi(\\sum ~
\_i=1^nx\_ie\_i,\\\sum
 \_\jmath=1^ny\_\jmathe\_\jmath)
= \\sum ~
\_i,\jmath\phi(e\_i,e\_\jmath)x\_iy\_\jmath en
utilisant la bilinéarité de \phi. Ceci montre que
\mathrmMat~ (\phi,\mathcal{E}) vérifie
bien la relation voulue. Inversement, si \Omega vérifie cette formule, on a
\phi(e\_k,e\_l) = ^tE\_k\OmegaE\_l
= \\sum ~
\_i,\jmath\omega\_i,\jmath\delta\_i^k\delta\_\jmath^l =
\omega\_k,l ce qui montre que \Omega =\
\mathrmMat (\phi,\mathcal{E}).

Théorème~12.1.5 L'application
\phi\mapsto~\mathrmMat~
(\phi,\mathcal{E}) est un isomorphisme d'espaces vectoriels de L\_2(E) sur
M\_K(n).

Démonstration Les détails sont laissés aux soins du lecteur.
L'application réciproque est bien entendu l'application qui à \Omega \in
M\_K(n) associe \phi : E \times E \rightarrow~ K définie par \phi(x,y) =
^tX\OmegaY qui est clairement bilinéaire.

Corollaire~12.1.6 Si E est de dimension finie,
dim L\_2~(E) =
(dim E)^2~.

Théorème~12.1.7 Soit E de dimension finie, \mathcal{E} =
(e\_1,\\ldots,e\_n~)
une base de E, \mathcal{E}^∗ =
(e\_1^∗,\\ldots,e\_n^∗~)
la base duale. Soit \phi \in L\_2(E). Alors

\mathrmMat~ (\phi,\mathcal{E})
= \mathrmMat~
(d\_\phi,\mathcal{E},\mathcal{E}^∗) = ^t\
\mathrmMat (g\_ \phi,\mathcal{E},\mathcal{E}^∗)

Démonstration Notons \Omega =\
\mathrmMat (\phi,\mathcal{E}), A =\
\mathrmMat (d\_\phi,\mathcal{E},\mathcal{E}^∗) et B
= \mathrmMat~
(g\_\phi,\mathcal{E},\mathcal{E}^∗). On a

\begin{align*} \omega\_i,\jmath& =&
\phi(e\_i,e\_\jmath) = \left
(d\_\phi(e\_\jmath)\right )(e\_i) \%&
\\ & =& \left
(\sum \_k=1^na~\_
k,\jmathe\_k^∗\right )(e\_ i) =
a\_i,\jmath\%& \\
\end{align*}

compte tenu de e\_k^∗(e\_i) =
\delta\_k^i~; de même

\begin{align*} \omega\_i,\jmath& =&
\phi(e\_i,e\_\jmath) = \left
(g\_\phi(e\_i)\right )(e\_\jmath) \%&
\\ & =& \left
(\sum \_k=1^nb~\_
k,ie\_k^∗\right )(e\_ \jmath) =
b\_\jmath,i\%& \\
\end{align*}

ce qui démontre le résultat.

Corollaire~12.1.8 La forme bilinéaire \phi est symétrique (resp.
antisymétrique) si et seulement si~sa matrice dans la base \mathcal{E} est
symétrique (resp. antisymétrique).

Le rang de \mathrmMat~
(d\_\phi,\mathcal{E},\mathcal{E}^∗) est indépendant du choix de la base \mathcal{E}~;
il en est donc de même du rang de
\mathrmMat~ (\phi,\mathcal{E}). Ceci
conduit à la définition suivante

Définition~12.1.5 Soit E de dimension finie et \phi \in L\_2(E). On
appelle rang de E le rang de sa matrice dans n'importe quelle base de E.
On a

\mathrmrg~\phi
= \mathrmrgd\_\phi~
= \mathrmrgg\_\phi~
=\
\mathrmrg\mathrmMat~
(\phi,\mathcal{E})

\paragraph{12.1.4 Changements de bases, discriminant}

Théorème~12.1.9 Soit E un espace vectoriel de dimension finie, \mathcal{E} et \mathcal{E}'
deux bases de E, P = P\_\mathcal{E}^\mathcal{E}' la matrice de passage de \mathcal{E} à
\mathcal{E}'. Soit \phi \in L\_2(E), \Omega =\
\mathrmMat (\phi,\mathcal{E}) et \Omega' =\
\mathrmMat (\phi,\mathcal{E}'). Alors

\Omega' = ^tP\OmegaP

Démonstration Si X (resp. Y ) désigne le vecteur colonne des coordonnées
de x (resp. y) dans la base \mathcal{E} et X' (resp. Y ') désigne le vecteur
colonne des coordonnées de x (resp. y) dans la base \mathcal{E}', on a X = PX', Y
= PY ', d'où

\phi(x,y) = ^t(PX')\Omega(PY ) = ^tX'(^tP\OmegaP)Y '

Comme \Omega' est l'unique matrice vérifiant \forall~~(x,y)
\in E \times E, \phi(x,y) = ^tX'\Omega'Y ', on a \Omega' = ^tP\OmegaP.

Définition~12.1.6 Soit E un espace vectoriel de dimension finie, \mathcal{E} une
base de E et \phi \in L\_2(E). On appelle discriminant de \phi dans la
base \mathcal{E} le déterminant de la matrice
\mathrmMat~ (\phi,\mathcal{E}).

Remarque~12.1.4 La formule ci dessus montre que lors d'un changement de
base, le discriminant est multiplié par
(\mathrm{det}~
P)^2.

On introduit ainsi une nouvelle relation d'équivalence sur les matrices
carrées d'ordre n~: représenter une même forme bilinéaire dans des bases
différentes.

Définition~12.1.7 Soit \Omega,\Omega' \in M\_K(n). On dit que ces deux
matrices sont congruentes s'il existe P \in GL\_K(n) telle que \Omega'
= ^tP\OmegaP. Il s'agit d'une relation d'équivalence sur
M\_K(n).

Remarque~12.1.5 Bien entendu cette relation de congruence laisse stables
les sous-espaces vectoriels des matrices symétriques ou antisymétriques.

\paragraph{12.1.5 Orthogonalité}

Soit E un K-espace vectoriel ~et \phi une forme bilinéaire sur E.

Définition~12.1.8 On dit que x est orthogonal à y (relativement à \phi), et
on pose x \bot y, si \phi(x,y) = 0.

Définition~12.1.9 Soit A une partie de E. On pose

\begin{itemize}
\itemsep1pt\parskip0pt\parsep0pt
\item
  (i) A^\bot = \x \in
  E∣\forall~~a \in A, \phi(a,x)
  = 0\
\item
  (ii) ^\bot A = \x \in
  E∣\forall~~a \in A, \phi(x,a)
  = 0\
\end{itemize}

Remarque~12.1.6 Notons A^\bot^∗  l'orthogonal de A
dans le dual E^∗ de E, c'est-à-dire l'espace vectoriel des
formes linéaires sur E qui sont nulles sur A. On a

\begin{align*} x \in A^\bot&
\Leftrightarrow & \forall~~a \in A, \phi(a,x)
= 0 \%& \\ &
\Leftrightarrow & \forall~~a \in A,
\big {[}d\_\phi(x)\big {]}(a) = 0
\%& \\ & \Leftrightarrow &
d\_\phi(x) \in A^\bot^∗ 
\Leftrightarrow x \in
d\_\phi^-1(A^\bot^∗ )\%&
\\ \end{align*}

On en déduit que A^\bot =
d\_\phi^-1(A^\bot^∗ ) et ^\bot A
= g\_\phi^-1(A^\bot^∗ ).

Proposition~12.1.10 Soit A une partie de E~; alors

\begin{itemize}
\itemsep1pt\parskip0pt\parsep0pt
\item
  (i)A^\bot et ^\bot A sont des sous espaces vectoriels
  de E
\item
  (ii)A^\bot =\
  \mathrmVect(A)^\bot et ^\bot A
  = ^\bot
  \mathrmVect~(A)
\item
  (iii) A \subset~^\bot (A^\bot) et A \subset~ (^\bot
  A)^\bot
\item
  (iv) A \subset~ B \rigtharrow~ B^\bot\subset~ A^\bot et ^\bot B
  \subset~^\bot A.
\end{itemize}

Démonstration (i) découle immédiatement de la bilinéarité de \phi ou de la
remarque précédente. Il en est de même pour (ii) puisqu'un vecteur x est
orthogonal (aussi bien à gauche qu'à droite) à tout vecteur de A si et
seulement si il est orthogonal à toute combinaison linéaire de vecteurs
de A, c'est à dire à
\mathrmVect~(A). En ce qui
concerne (iii), il suffit de remarquer que tout vecteur a de A est
orthogonal à tout vecteur qui est orthogonal à tout vecteur de A. Pour
(iv), un vecteur x qui est orthogonal à tout vecteur de B est évidemment
orthogonal à tout vecteur de A.

Remarque~12.1.7 Dans le cas où \phi est symétrique ou antisymétrique, on a
\phi(x,y) = 0 \Leftrightarrow \phi(y,x) = 0, si bien que la
relation d'orthogonalité est symétrique. Dans ce cas, il n'y a pas lieu
de distinguer ^\bot A de A^\bot. Dans toute la suite
nous ferons l'hypothèse que \phi est soit symétrique, soit antisymétrique.

\paragraph{12.1.6 Formes non dégénérées}

En règle générale on posera

Définition~12.1.10 Soit E un K-espace vectoriel , \phi une forme bilinéaire
symétrique (resp. antisymétrique) sur E. On appelle noyau de \phi le
sous-espace

\mathrmKer~\phi =
\x \in
E∣\forall~~y \in E, \phi(x,y) =
0\ = E^\bot =\
\mathrmKerd\_ \phi

Définition~12.1.11 Soit E un K-espace vectoriel , \phi une forme bilinéaire
symétrique (resp. antisymétrique) sur E. On dit que \phi est non dégénérée
si elle vérifie les conditions équivalentes

\begin{itemize}
\itemsep1pt\parskip0pt\parsep0pt
\item
  (i) \mathrmKer~\phi =
  E^\bot = \0\
\item
  (ii) pour x \in E on a \left
  (\forall~~y \in E, \phi(x,y) = 0\right ) \rigtharrow~
  x = 0
\item
  (iii) d\_\phi (resp. g\_\phi) est une application linéaire
  in\jmathective de E dans E^∗.
\end{itemize}

L'équivalence entre ces trois propriétés est évidente.

Si E est un espace vectoriel de dimension finie, on sait que
dim E^∗~ =\
dim E. Si d\_\phi est in\jmathective, elle est nécessairement
bi\jmathective et on obtient

Théorème~12.1.11 Soit E un K-espace vectoriel ~de dimension finie, \phi une
forme bilinéaire symétrique (resp. antisymétrique) non dégénérée sur E.
Alors l'application linéaire droite d\_\phi est un isomorphisme
d'espace vectoriel de E sur E^∗~; autrement dit, pour toute
forme linéaire f sur E, il existe un unique vecteur v\_f \in E tel
que \forall~x \in E, f(x) = \phi(x,v\_f~).

Corollaire~12.1.12 Soit E un K-espace vectoriel ~de dimension finie, \phi
une forme bilinéaire symétrique (resp. antisymétrique) non dégénérée sur
E. Soit A un sous-espace vectoriel de E. Alors
dim A +\ dim~
A^\bot = dim E et A = A^\bot\bot~.

Démonstration On a en effet

dim A^\bot~ =\
dim d\_ \phi^-1(A^\bot^∗ )
= dim A^\bot^∗ ~
= dim E -\ dim~ A

puisque d\_\phi est un isomorphisme d'espaces vectoriels. On sait
d'autre part que A \subset~ A^\bot\bot et que
dim A^\bot\bot~ =\
dim E - dim A^\bot~
= dim~ A, d'où l'égalité.

Remarque~12.1.8 Il ne faudrait pas en déduire abusivement que A et
A^\bot sont supplémentaires~; en effet, en général A \bigcap
A^\bot\neq~\0\.
Nous nous intéresserons plus particulièrement à ce point dans le
paragraphe suivant.

Si \mathcal{E} est une base de E, alors \Omega =\
\mathrmMat (\phi,\mathcal{E}) =\
\mathrmMat (d\_\phi,\mathcal{E},\mathcal{E}^∗) et
\mathrmrg~\phi
= \mathrmrg~\Omega. On en déduit

Théorème~12.1.13 Soit E un K-espace vectoriel ~de dimension finie n, \phi
une forme bilinéaire symétrique (resp. antisymétrique) sur E, \mathcal{E} une base
de E et \Omega = \mathrmMat~
(\phi,\mathcal{E}). Alors les propositions suivantes sont équivalentes

\begin{itemize}
\itemsep1pt\parskip0pt\parsep0pt
\item
  (i) \phi est non dégénérée
\item
  (ii) \Omega est une matrice inversible
\item
  (iii) \mathrmrg~\phi = n.
\end{itemize}

Remarque~12.1.9 En général,
\mathrmKer~\phi
= \mathrmKerd\_\phi~,
\mathrmrg~\phi
= \mathrmrgd\_\phi~, si
bien que le théorème du rang devient

Proposition~12.1.14 Soit E un K-espace vectoriel ~de dimension finie n,
\phi une forme bilinéaire symétrique (resp. antisymétrique) sur E, \mathcal{E} une
base de E. Alors dim~ E
= \mathrmrg~\phi
+ dim~
\mathrmKer~\phi.

\paragraph{12.1.7 Isotropie}

Définition~12.1.12 Soit E un K-espace vectoriel , \phi une forme bilinéaire
symétrique (resp. antisymétrique) sur E. On dit qu'un sous-espace
vectoriel A de E est non isotrope s'il vérifie les conditions
équivalentes

\begin{itemize}
\itemsep1pt\parskip0pt\parsep0pt
\item
  (i) A \bigcap A^\bot = \0\
\item
  (ii) la restriction de \phi à A \times A est non dégénérée.
\end{itemize}

Démonstration On a en effet
\mathrmKer\phi\_\textbar{}\_A\timesA~
= \x \in
A∣\forall~~y \in A, \phi(x,y) =
0\ = \x \in
A∣x \in A^\bot\ = A \bigcap
A^\bot.

Définition~12.1.13 Soit E un K-espace vectoriel , \phi une forme bilinéaire
symétrique (resp. antisymétrique) sur E. On dit que x \in E est un vecteur
isotrope s'il vérifie les conditions équivalentes suivantes

\begin{itemize}
\itemsep1pt\parskip0pt\parsep0pt
\item
  (i) la droite Kx est un sous-espace isotrope ou x = 0
\item
  (ii) \phi(x,x) = 0
\end{itemize}

Démonstration (i) \rigtharrow~(ii)~: soit y \in Kx \bigcap
(Kx)^\bot\diagdown\0\~; on a y = \lambda~x
avec \lambda~\neq~0, d'où 0 = \phi(y,y) =
\lambda~^2\phi(x,x), soit \phi(x,x) = 0.

(ii) \rigtharrow~(i) si \phi(x,x) = 0, on a clairement Kx \subset~ (Kx)^\bot.

Remarque~12.1.10 Il est clair que si \phi est antisymétrique et si
\mathrmcarK\mathrel\neq~~2,
alors tout vecteur est isotrope. La notion n'est donc réellement
intéressante que pour les formes symétriques.

Exemple~12.1.1 Pour la forme bilinéaire symétrique sur \mathbb{R}~^4,
\phi(x,y) = x\_1y\_1 + x\_2y\_2 +
x\_3y\_3 - x\_4y\_4 (forme de Lorentz,
celle de la relativité), le vecteur (1,0,0,1) est isotrope~; cette forme
est bien entendu non dégénérée puisque sa matrice dans la base canonique
est la matrice
\mathrmdiag~(1,1,1,-1) qui
est inversible~; on voit donc que \phi peut être non dégénérée, alors que
sa restriction à un sous-espace est dégénérée (et même nulle).

Définition~12.1.14 On dit que la forme bilinéaire symétrique \phi sur E est
définie s'il n'existe pas de vecteur isotrope autre que 0.

Remarque~12.1.11 Si A est un sous-espace isotrope, alors tout vecteur de
A \bigcap A^\bot\diagdown\0\ est clairement
isotrope (étant orthogonal à tout vecteur de A, il est orthogonal à lui
même). On en déduit que si \phi est une forme bilinéaire symétrique
définie, alors tout sous-espace de E est non isotrope. En particulier, E
lui même est non isotrope et donc

Proposition~12.1.15 Soit \phi une forme bilinéaire symétrique définie~;
alors \phi est non dégénérée et tout sous-espace est non isotrope pour \phi.

Théorème~12.1.16 Soit E un K-espace vectoriel ~de dimension finie, \phi une
forme bilinéaire symétrique (resp. antisymétrique) non dégénérée sur E.
Soit A un sous-espace vectoriel de E. Alors on a l'équivalence de

\begin{itemize}
\itemsep1pt\parskip0pt\parsep0pt
\item
  (i) A est non isotrope
\item
  (ii) E = A \oplus~ A^\bot
\end{itemize}

Démonstration En effet on sait que dim~ A
+ dim A^\bot~ =\
dim E. On a donc

A \bigcap A^\bot = \0\
\Leftrightarrow E = A \oplus~ A^\bot

Corollaire~12.1.17 Soit E un K-espace vectoriel ~de dimension finie, \phi
une forme bilinéaire symétrique définie sur E. Soit A un sous-espace
vectoriel de E. Alors E = A \oplus~ A^\bot.

{[}
{[}

\end{document}
