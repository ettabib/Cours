\section{Formes bilinéaires}

\subsection{Généralités}

\begin{de}
\index{forme bilinéaire}
Soit $E$ un $K$-espace vectoriel. On appelle forme bilinéaire sur $E$ toute application $\phi : E \times E \rightarrow K$ telle que
\begin{enumerate}
\item $\forall x \in E$, $y \mapsto \phi(x,y)$ est linéaire
\item $\forall y \in E$, $x \mapsto \phi(x,y)$ est linéaire
\end{enumerate}
\end{de}

\begin{rem}
On a en particulier $\forall x,y \in E$, $\phi(x,0) = \phi(0,y) = 0$.
\end{rem}

\begin{rem}
Il est clair que si $\phi$ et $\psi$ sont deux formes bilinéaires sur $E$, il en est de même de $\alpha \phi + \beta \psi$, d'où la proposition
\end{rem}

\begin{prop}
L'ensemble $L_2(E)$ des formes bilinéaires sur $E$ est un sous-espace vectoriel de l'espace $K^{E \times E}$ des applications de $E \times E$ dans $K$.
\end{prop}

\begin{rem}
Soit $\phi$ une forme bilinéaire sur $E$. Pour chaque $x \in E$, l'application $y \mapsto \phi(x,y)$ est une forme linéaire sur $E$ donc un élément, noté $g_\phi(x)$, du dual $E^*$ de $E$. De même, pour chaque $y \in E$, l'application $x \mapsto \phi(x,y)$ est une forme linéaire sur $E$, donc un élément, noté $d_\phi(y)$, de $E^*$. La relation

\begin{align*}
[g_\phi(\alpha x + \beta x')](y) &= \phi(\alpha x + \beta x',y) = \alpha \phi(x,y) + \beta \phi(x',y) \\
&= [\alpha g_\phi(x) + \beta g_\phi(x')](y)
\end{align*}

montre clairement que $g_\phi : x \mapsto g_\phi(x)$ est une application linéaire de $E$ dans $E^*$. Il en est évidemment de même de $d_\phi : y \mapsto d_\phi(y)$.
\end{rem}

\begin{de}
\index{application linéaire gauche}
\index{application linéaire droite}
L'application $g_\phi : E \rightarrow E^*$ (resp. $d_\phi$) est appelée l'application linéaire gauche (resp. droite) associée à la forme bilinéaire $\phi$.
\end{de}

\subsection{Formes bilinéaires symétriques, antisymétriques}

\begin{de}
\index{forme bilinéaire symétrique}
\index{forme bilinéaire antisymétrique}
Soit $\phi \in L_2(E)$. On dit que $\phi$ est symétrique (resp. antisymétrique) si $\forall x,y \in E$, $\phi(y,x) = \phi(x,y)$ (resp. $= -\phi(x,y)$).
\end{de}

\begin{prop}
Soit $\phi \in L_2(E)$. Alors $\phi$ est symétrique (resp. antisymétrique) si et seulement si $d_\phi = g_\phi$ (resp. $d_\phi = -g_\phi$).
\end{prop}

\begin{proof}
En effet $\phi(x,y) = [g_\phi(x)](y)$ et $\phi(y,x) = [d_\phi(x)](y)$. Donc

\begin{align*}
\forall x,y \in E, \phi(y,x) = \epsilon \phi(x,y) &\Leftrightarrow \forall x,y \in E, [g_\phi(x)](y) = \epsilon [d_\phi(x)](y) \\
&\Leftrightarrow \forall x \in E, g_\phi(x) = \epsilon d_\phi(x) \Leftrightarrow g_\phi = \epsilon d_\phi
\end{align*}
\end{proof}

\begin{prop}
L'ensemble $S_2(E)$ (resp. $A_2(E)$) des formes bilinéaires symétriques (resp. antisymétriques) est un sous-espace vectoriel de $L_2(E)$. Si la caractéristique de $K$ est différente de 2, alors $L_2(E) = S_2(E) \oplus A_2(E)$.
\end{prop}

\begin{proof}
La première affirmation est laissée aux soins du lecteur. Si la caractéristique de $K$ est différente de 2, on a clairement $S_2(E) \cap A_2(E) = \{0\}$ et la relation $\phi = \psi + \theta$ avec $\psi(x,y) = \frac{1}{2}(\phi(x,y) + \phi(y,x))$, $\theta(x,y) = \frac{1}{2}(\phi(x,y) - \phi(y,x))$, qui sont respectivement symétrique et antisymétrique, montre que $L_2(E) = S_2(E) + A_2(E)$.
\end{proof}

\subsection{Matrice d'une forme bilinéaire}

Supposons que $E$ est de dimension finie et soit $\mathcal{E} = (e_1,\ldots,e_n)$ une base de $E$.

\begin{de}
\index{matrice d'une forme bilinéaire}
Soit $\phi \in L_2(E)$. On appelle matrice de $\phi$ dans la base $\mathcal{E}$ la matrice

$\mathrm{Mat}(\phi,\mathcal{E}) = (\phi(e_i,e_j))_{1 \leq i,j \leq n} \in M_K(n)$
\end{de}

\begin{prop}
$\mathrm{Mat}(\phi,\mathcal{E})$ est l'unique matrice $\Omega \in M_K(n)$ vérifiant

$\forall (x,y) \in E \times E, \phi(x,y) = {}^t X \Omega Y$

où $X$ (resp. $Y$) désigne le vecteur colonne des coordonnées de $x$ (resp. $y$) dans la base $\mathcal{E}$.
\end{prop}

\begin{proof}
Si $\Omega = (\omega_{i,j})$, on a

${}^t X \Omega Y = \sum_{i=1}^n x_i (\Omega Y)_i = \sum_{i=1}^n x_i \sum_{j=1}^n \omega_{i,j} y_j = \sum_{i,j} \omega_{i,j} x_i y_j$

Mais d'autre part $\phi(x,y) = \phi(\sum_{i=1}^n x_i e_i, \sum_{j=1}^n y_j e_j) = \sum_{i,j} \phi(e_i,e_j) x_i y_j$ en utilisant la bilinéarité de $\phi$. Ceci montre que $\mathrm{Mat}(\phi,\mathcal{E})$ vérifie bien la relation voulue. Inversement, si $\Omega$ vérifie cette formule, on a $\phi(e_k,e_l) = {}^t E_k \Omega E_l = \sum_{i,j} \omega_{i,j} \delta_i^k \delta_j^l = \omega_{k,l}$ ce qui montre que $\Omega = \mathrm{Mat}(\phi,\mathcal{E})$.
\end{proof}

\begin{thm}
L'application $\phi \mapsto \mathrm{Mat}(\phi,\mathcal{E})$ est un isomorphisme d'espaces vectoriels de $L_2(E)$ sur $M_K(n)$.
\end{thm}

\begin{proof}
Les détails sont laissés aux soins du lecteur. L'application réciproque est bien entendu l'application qui à $\Omega \in M_K(n)$ associe $\phi : E \times E \rightarrow K$ définie par $\phi(x,y) = {}^t X \Omega Y$ qui est clairement bilinéaire.
\end{proof}

\begin{thm}
Si $E$ est de dimension finie, $\dim L_2(E) = (\dim E)^2$.
\end{thm}

\begin{thm}
Soit $E$ de dimension finie, $\mathcal{E} = (e_1,\ldots,e_n)$ une base de $E$, $\mathcal{E}^* = (e_1^*,\ldots,e_n^*)$ la base duale. Soit $\phi \in L_2(E)$. Alors

$\mathrm{Mat}(\phi,\mathcal{E}) = \mathrm{Mat}(d_\phi,\mathcal{E},\mathcal{E}^*) = {}^t \mathrm{Mat}(g_\phi,\mathcal{E},\mathcal{E}^*)$
\end{thm}

\begin{proof}
Notons $\Omega = \mathrm{Mat}(\phi,\mathcal{E})$, $A = \mathrm{Mat}(d_\phi,\mathcal{E},\mathcal{E}^*)$ et $B = \mathrm{Mat}(g_\phi,\mathcal{E},\mathcal{E}^*)$. On a

\begin{align*}
\omega_{i,j} &= \phi(e_i,e_j) = (d_\phi(e_j))(e_i) \\
&= (\sum_{k=1}^n a_{k,j} e_k^*)(e_i) = a_{i,j}
\end{align*}

compte tenu de $e_k^*(e_i) = \delta_k^i$; de même

\begin{align*}
\omega_{i,j} &= \phi(e_i,e_j) = (g_\phi(e_i))(e_j) \\
&= (\sum_{k=1}^n b_{k,i} e_k^*)(e_j) = b_{j,i}
\end{align*}

ce qui démontre le résultat.
\end{proof}

\begin{thm}
La forme bilinéaire $\phi$ est symétrique (resp. antisymétrique) si et seulement si sa matrice dans la base $\mathcal{E}$ est symétrique (resp. antisymétrique).
\end{thm}

Le rang de $\mathrm{Mat}(d_\phi,\mathcal{E},\mathcal{E}^*)$ est indépendant du choix de la base $\mathcal{E}$; il en est donc de même du rang de $\mathrm{Mat}(\phi,\mathcal{E})$. Ceci conduit à la définition suivante

\begin{de}
\index{rang d'une forme bilinéaire}
Soit $E$ de dimension finie et $\phi \in L_2(E)$. On appelle rang de $E$ le rang de sa matrice dans n'importe quelle base de $E$. On a

$\mathrm{rg} \phi = \mathrm{rg} d_\phi = \mathrm{rg} g_\phi = \mathrm{rg} \mathrm{Mat}(\phi,\mathcal{E})$
\end{de}

\subsection{Changements de bases, discriminant}

\begin{thm}
Soit $E$ un espace vectoriel de dimension finie, $\mathcal{E}$ et $\mathcal{E}'$ deux bases de $E$, $P = P_\mathcal{E}^{\mathcal{E}'}$ la matrice de passage de $\mathcal{E}$ à $\mathcal{E}'$. Soit $\phi \in L_2(E)$, $\Omega = \mathrm{Mat}(\phi,\mathcal{E})$ et $\Omega' = \mathrm{Mat}(\phi,\mathcal{E}')$. Alors

$\Omega' = {}^t P \Omega P$
\end{thm}

\begin{proof}
Si $X$ (resp. $Y$) désigne le vecteur colonne des coordonnées de $x$ (resp. $y$) dans la base $\mathcal{E}$ et $X'$ (resp. $Y'$) désigne le vecteur colonne des coordonnées de $x$ (resp. $y$) dans la base $\mathcal{E}'$, on a $X = PX'$, $Y = PY'$, d'où

$\phi(x,y) = {}^t (PX') \Omega (PY') = {}^t X' ({}^t P \Omega P) Y'$

Comme $\Omega'$ est l'unique matrice vérifiant $\forall (x,y) \in E \times E$, $\phi(x,y) = {}^t X' \Omega' Y'$, on a $\Omega' = {}^t P \Omega P$.
\end{proof}

\begin{de}
\index{discriminant d'une forme bilinéaire}
Soit $E$ un espace vectoriel de dimension finie, $\mathcal{E}$ une base de $E$ et $\phi \in L_2(E)$. On appelle discriminant de $\phi$ dans la base $\mathcal{E}$ le déterminant de la matrice $\mathrm{Mat}(\phi,\mathcal{E})$.
\end{de}

\begin{rem}
La formule ci-dessus montre que lors d'un changement de base, le discriminant est multiplié par $(\det P)^2$.
\end{rem}

On introduit ainsi une nouvelle relation d'équivalence sur les matrices carrées d'ordre $n$ : représenter une même forme bilinéaire dans des bases différentes.

\begin{de}
\index{matrices congruentes}
Soit $\Omega,\Omega' \in M_K(n)$. On dit que ces deux matrices sont congruentes s'il existe $P \in GL_K(n)$ telle que $\Omega' = {}^t P \Omega P$. Il s'agit d'une relation d'équivalence sur $M_K(n)$.
\end{de}

\begin{rem}
Bien entendu cette relation de congruence laisse stables les sous-espaces vectoriels des matrices symétriques ou antisymétriques.
\end{rem}

\subsection{Orthogonalité}

Soit $E$ un $K$-espace vectoriel et $\phi$ une forme bilinéaire sur $E$.

\begin{de}
\index{orthogonalité}
On dit que $x$ est orthogonal à $y$ (relativement à $\phi$), et on pose $x \bot y$, si $\phi(x,y) = 0$.
\end{de}

\begin{de}
Soit $A$ une partie de $E$. On pose
\begin{enumerate}
\item $A^\bot = \{x \in E | \forall a \in A, \phi(a,x) = 0\}$
\item ${}^\bot A = \{x \in E | \forall a \in A, \phi(x,a) = 0\}$
\end{enumerate}
\end{de}

\begin{rem}
Notons $A^{\bot *}$ l'orthogonal de $A$ dans le dual $E^*$ de $E$, c'est-à-dire l'espace vectoriel des formes linéaires sur $E$ qui sont nulles sur $A$. On a

\begin{align*}
x \in A^\bot &\Leftrightarrow \forall a \in A, \phi(a,x) = 0 \\
&\Leftrightarrow \forall a \in A, [d_\phi(x)](a) = 0 \\
&\Leftrightarrow d_\phi(x) \in A^{\bot *} \Leftrightarrow x \in d_\phi^{-1}(A^{\bot *})
\end{align*}

On en déduit que $A^\bot = d_\phi^{-1}(A^{\bot *})$ et ${}^\bot A = g_\phi^{-1}(A^{\bot *})$.
\end{rem}

\begin{prop}
Soit $A$ une partie de $E$; alors
\begin{enumerate}
\item $A^\bot$ et ${}^\bot A$ sont des sous espaces vectoriels de $E$
\item $A^\bot = \mathrm{Vect}(A)^\bot$ et ${}^\bot A = {}^\bot \mathrm{Vect}(A)$
\item $A \subset {}^\bot(A^\bot)$ et $A \subset ({}^\bot A)^\bot$
\item $A \subset B \Rightarrow B^\bot \subset A^\bot$ et ${}^\bot B \subset {}^\bot A$.
\end{enumerate}
\end{prop}

\begin{proof}
(i) découle immédiatement de la bilinéarité de $\phi$ ou de la remarque précédente. Il en est de même pour (ii) puisqu'un vecteur $x$ est orthogonal (aussi bien à gauche qu'à droite) à tout vecteur de $A$ si et seulement si il est orthogonal à toute combinaison linéaire de vecteurs de $A$, c'est à dire à $\mathrm{Vect}(A)$. En ce qui concerne (iii), il suffit de remarquer que tout vecteur $a$ de $A$ est orthogonal à tout vecteur qui est orthogonal à tout vecteur de $A$. Pour (iv), un vecteur $x$ qui est orthogonal à tout vecteur de $B$ est évidemment orthogonal à tout vecteur de $A$.
\end{proof}

\begin{rem}
Dans le cas où $\phi$ est symétrique ou antisymétrique, on a $\phi(x,y) = 0 \Leftrightarrow \phi(y,x) = 0$, si bien que la relation d'orthogonalité est symétrique. Dans ce cas, il n'y a pas lieu de distinguer ${}^\bot A$ de $A^\bot$. Dans toute la suite nous ferons l'hypothèse que $\phi$ est soit symétrique, soit antisymétrique.
\end{rem}

\subsection{Formes non dégénérées}

En règle générale on posera

\begin{de}
\index{noyau d'une forme bilinéaire}
Soit $E$ un $K$-espace vectoriel, $\phi$ une forme bilinéaire symétrique (resp. antisymétrique) sur $E$. On appelle noyau de $\phi$ le sous-espace

$\mathrm{Ker} \phi = \{x \in E | \forall y \in E, \phi(x,y) = 0\} = E^\bot = \mathrm{Ker} d_\phi$
\end{de}

\begin{de}
\index{forme bilinéaire non dégénérée}
Soit $E$ un $K$-espace vectoriel, $\phi$ une forme bilinéaire symétrique (resp. antisymétrique) sur $E$. On dit que $\phi$ est non dégénérée si elle vérifie les conditions équivalentes
\begin{enumerate}
\item $\mathrm{Ker} \phi = E^\bot = \{0\}$
\item pour $x \in E$ on a $(\forall y \in E, \phi(x,y) = 0) \Rightarrow x = 0$
\item $d_\phi$ (resp. $g_\phi$) est une application linéaire injective de $E$ dans $E^*$.
\end{enumerate}
\end{de}

L'équivalence entre ces trois propriétés est évidente.

Si $E$ est un espace vectoriel de dimension finie, on sait que $\dim E^* = \dim E$. Si $d_\phi$ est injective, elle est nécessairement bijective et on obtient

\begin{thm}
Soit $E$ un $K$-espace vectoriel de dimension finie, $\phi$ une forme bilinéaire symétrique (resp. antisymétrique) non dégénérée sur $E$. Alors l'application linéaire droite $d_\phi$ est un isomorphisme d'espace vectoriel de $E$ sur $E^*$; autrement dit, pour toute forme linéaire $f$ sur $E$, il existe un unique vecteur $v_f \in E$ tel que $\forall x \in E, f(x) = \phi(x,v_f)$.
\end{thm}

\begin{thm}
Soit $E$ un $K$-espace vectoriel de dimension finie, $\phi$ une forme bilinéaire symétrique (resp. antisymétrique) non dégénérée sur $E$. Soit $A$ un sous-espace vectoriel de $E$. Alors $\dim A + \dim A^\bot = \dim E$ et $A = A^{\bot\bot}$.
\end{thm}

\begin{proof}
On a en effet

$\dim A^\bot = \dim d_\phi^{-1}(A^{\bot *}) = \dim A^{\bot *} = \dim E - \dim A$

puisque $d_\phi$ est un isomorphisme d'espaces vectoriels. On sait d'autre part que $A \subset A^{\bot\bot}$ et que $\dim A^{\bot\bot} = \dim E - \dim A^\bot = \dim A$, d'où l'égalité.
\end{proof}

\begin{rem}
Il ne faudrait pas en déduire abusivement que $A$ et $A^\bot$ sont supplémentaires; en effet, en général $A \cap A^\bot \neq \{0\}$. Nous nous intéresserons plus particulièrement à ce point dans la section suivante.
\end{rem}

Si $\mathcal{E}$ est une base de $E$, alors $\Omega = \mathrm{Mat}(\phi,\mathcal{E}) = \mathrm{Mat}(d_\phi,\mathcal{E},\mathcal{E}^*)$ et $\mathrm{rg} \phi = \mathrm{rg} \Omega$. On en déduit

\begin{thm}
Soit $E$ un $K$-espace vectoriel de dimension finie $n$, $\phi$ une forme bilinéaire symétrique (resp. antisymétrique) sur $E$, $\mathcal{E}$ une base de $E$ et $\Omega = \mathrm{Mat}(\phi,\mathcal{E})$. Alors les propositions suivantes sont équivalentes
\begin{enumerate}
\item $\phi$ est non dégénérée
\item $\Omega$ est une matrice inversible
\item $\mathrm{rg} \phi = n$.
\end{enumerate}
\end{thm}

\begin{rem}
En général, $\mathrm{Ker} \phi = \mathrm{Ker} d_\phi$, $\mathrm{rg} \phi = \mathrm{rg} d_\phi$, si bien que le théorème du rang devient
\end{rem}

\begin{prop}
Soit $E$ un $K$-espace vectoriel de dimension finie $n$, $\phi$ une forme bilinéaire symétrique (resp. antisymétrique) sur $E$, $\mathcal{E}$ une base de $E$. Alors $\dim E = \mathrm{rg} \phi + \dim \mathrm{Ker} \phi$.
\end{prop}

\subsection{Isotropie}

\begin{de}
\index{sous-espace non isotrope}
Soit $E$ un $K$-espace vectoriel, $\phi$ une forme bilinéaire symétrique (resp. antisymétrique) sur $E$. On dit qu'un sous-espace vectoriel $A$ de $E$ est non isotrope s'il vérifie les conditions équivalentes
\begin{enumerate}
\item $A \cap A^\bot = \{0\}$
\item la restriction de $\phi$ à $A \times A$ est non dégénérée.
\end{enumerate}
\end{de}

\begin{proof}
On a en effet $\mathrm{Ker} \phi|_{A \times A} = \{x \in A | \forall y \in A, \phi(x,y) = 0\} = \{x \in A | x \in A^\bot\} = A \cap A^\bot$.
\end{proof}

\begin{de}
\index{vecteur isotrope}
Soit $E$ un $K$-espace vectoriel, $\phi$ une forme bilinéaire symétrique (resp. antisymétrique) sur $E$. On dit que $x \in E$ est un vecteur isotrope s'il vérifie les conditions équivalentes suivantes
\begin{enumerate}
\item la droite $Kx$ est un sous-espace isotrope ou $x = 0$
\item $\phi(x,x) = 0$
\end{enumerate}
\end{de}

\begin{proof}
(i) $\Rightarrow$ (ii) : soit $y \in Kx \cap (Kx)^\bot \setminus \{0\}$; on a $y = \lambda x$ avec $\lambda \neq 0$, d'où $0 = \phi(y,y) = \lambda^2 \phi(x,x)$, soit $\phi(x,x) = 0$.

(ii) $\Rightarrow$ (i) si $\phi(x,x) = 0$, on a clairement $Kx \subset (Kx)^\bot$.
\end{proof}

\begin{rem}
Il est clair que si $\phi$ est antisymétrique et si $\mathrm{car} K \neq 2$, alors tout vecteur est isotrope. La notion n'est donc réellement intéressante que pour les formes symétriques.
\end{rem}

\begin{ex}
Pour la forme bilinéaire symétrique sur $\mathbb{R}^4$, $\phi(x,y) = x_1 y_1 + x_2 y_2 + x_3 y_3 - x_4 y_4$ (forme de Lorentz, celle de la relativité), le vecteur $(1,0,0,1)$ est isotrope; cette forme est bien entendu non dégénérée puisque sa matrice dans la base canonique est la matrice $\mathrm{diag}(1,1,1,-1)$ qui est inversible; on voit donc que $\phi$ peut être non dégénérée, alors que sa restriction à un sous-espace est dégénérée (et même nulle).
\end{ex}

\begin{de}
\index{forme bilinéaire définie}
On dit que la forme bilinéaire symétrique $\phi$ sur $E$ est définie s'il n'existe pas de vecteur isotrope autre que 0.
\end{de}

\begin{rem}
Si $A$ est un sous-espace isotrope, alors tout vecteur de $A \cap A^\bot \setminus \{0\}$ est clairement isotrope (étant orthogonal à tout vecteur de $A$, il est orthogonal à lui même). On en déduit que si $\phi$ est une forme bilinéaire symétrique définie, alors tout sous-espace de $E$ est non isotrope. En particulier, $E$ lui même est non isotrope et donc
\end{rem}

\begin{prop}
Soit $\phi$ une forme bilinéaire symétrique définie; alors $\phi$ est non dégénérée et tout sous-espace est non isotrope pour $\phi$.
\end{prop}

\begin{thm}
Soit $E$ un $K$-espace vectoriel de dimension finie, $\phi$ une forme bilinéaire symétrique (resp. antisymétrique) non dégénérée sur $E$. Soit $A$ un sous-espace vectoriel de $E$. Alors on a l'équivalence de
\begin{enumerate}
\item $A$ est non isotrope
\item $E = A \oplus A^\bot$
\end{enumerate}
\end{thm}

\begin{proof}
En effet on sait que $\dim A + \dim A^\bot = \dim E$. On a donc

$A \cap A^\bot = \{0\} \Leftrightarrow E = A \oplus A^\bot$
\end{proof}

\begin{thm}
Soit $E$ un $K$-espace vectoriel de dimension finie, $\phi$ une forme bilinéaire symétrique définie sur $E$. Soit $A$ un sous-espace vectoriel de $E$. Alors $E = A \oplus A^\bot$.
\end{thm}