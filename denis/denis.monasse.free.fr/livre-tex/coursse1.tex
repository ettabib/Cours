\documentclass[]{article}
\usepackage[T1]{fontenc}
\usepackage{lmodern}
\usepackage{amssymb,amsmath}
\usepackage{ifxetex,ifluatex}
\usepackage{fixltx2e} % provides \textsubscript
% use upquote if available, for straight quotes in verbatim environments
\IfFileExists{upquote.sty}{\usepackage{upquote}}{}
\ifnum 0\ifxetex 1\fi\ifluatex 1\fi=0 % if pdftex
  \usepackage[utf8]{inputenc}
\else % if luatex or xelatex
  \ifxetex
    \usepackage{mathspec}
    \usepackage{xltxtra,xunicode}
  \else
    \usepackage{fontspec}
  \fi
  \defaultfontfeatures{Mapping=tex-text,Scale=MatchLowercase}
  \newcommand{\euro}{€}
\fi
% use microtype if available
\IfFileExists{microtype.sty}{\usepackage{microtype}}{}
\ifxetex
  \usepackage[setpagesize=false, % page size defined by xetex
              unicode=false, % unicode breaks when used with xetex
              xetex]{hyperref}
\else
  \usepackage[unicode=true]{hyperref}
\fi
\hypersetup{breaklinks=true,
            bookmarks=true,
            pdfauthor={},
            pdftitle={Ensembles et relations},
            colorlinks=true,
            citecolor=blue,
            urlcolor=blue,
            linkcolor=magenta,
            pdfborder={0 0 0}}
\urlstyle{same}  % don't use monospace font for urls
\setlength{\parindent}{0pt}
\setlength{\parskip}{6pt plus 2pt minus 1pt}
\setlength{\emergencystretch}{3em}  % prevent overfull lines
\setcounter{secnumdepth}{0}
 
/* start css.sty */
.cmr-5{font-size:50%;}
.cmr-7{font-size:70%;}
.cmmi-5{font-size:50%;font-style: italic;}
.cmmi-7{font-size:70%;font-style: italic;}
.cmmi-10{font-style: italic;}
.cmsy-5{font-size:50%;}
.cmsy-7{font-size:70%;}
.cmex-7{font-size:70%;}
.cmex-7x-x-71{font-size:49%;}
.msbm-7{font-size:70%;}
.cmtt-10{font-family: monospace;}
.cmti-10{ font-style: italic;}
.cmbx-10{ font-weight: bold;}
.cmr-17x-x-120{font-size:204%;}
.cmsl-10{font-style: oblique;}
.cmti-7x-x-71{font-size:49%; font-style: italic;}
.cmbxti-10{ font-weight: bold; font-style: italic;}
p.noindent { text-indent: 0em }
td p.noindent { text-indent: 0em; margin-top:0em; }
p.nopar { text-indent: 0em; }
p.indent{ text-indent: 1.5em }
@media print {div.crosslinks {visibility:hidden;}}
a img { border-top: 0; border-left: 0; border-right: 0; }
center { margin-top:1em; margin-bottom:1em; }
td center { margin-top:0em; margin-bottom:0em; }
.Canvas { position:relative; }
li p.indent { text-indent: 0em }
.enumerate1 {list-style-type:decimal;}
.enumerate2 {list-style-type:lower-alpha;}
.enumerate3 {list-style-type:lower-roman;}
.enumerate4 {list-style-type:upper-alpha;}
div.newtheorem { margin-bottom: 2em; margin-top: 2em;}
.obeylines-h,.obeylines-v {white-space: nowrap; }
div.obeylines-v p { margin-top:0; margin-bottom:0; }
.overline{ text-decoration:overline; }
.overline img{ border-top: 1px solid black; }
td.displaylines {text-align:center; white-space:nowrap;}
.centerline {text-align:center;}
.rightline {text-align:right;}
div.verbatim {font-family: monospace; white-space: nowrap; text-align:left; clear:both; }
.fbox {padding-left:3.0pt; padding-right:3.0pt; text-indent:0pt; border:solid black 0.4pt; }
div.fbox {display:table}
div.center div.fbox {text-align:center; clear:both; padding-left:3.0pt; padding-right:3.0pt; text-indent:0pt; border:solid black 0.4pt; }
div.minipage{width:100%;}
div.center, div.center div.center {text-align: center; margin-left:1em; margin-right:1em;}
div.center div {text-align: left;}
div.flushright, div.flushright div.flushright {text-align: right;}
div.flushright div {text-align: left;}
div.flushleft {text-align: left;}
.underline{ text-decoration:underline; }
.underline img{ border-bottom: 1px solid black; margin-bottom:1pt; }
.framebox-c, .framebox-l, .framebox-r { padding-left:3.0pt; padding-right:3.0pt; text-indent:0pt; border:solid black 0.4pt; }
.framebox-c {text-align:center;}
.framebox-l {text-align:left;}
.framebox-r {text-align:right;}
span.thank-mark{ vertical-align: super }
span.footnote-mark sup.textsuperscript, span.footnote-mark a sup.textsuperscript{ font-size:80%; }
div.tabular, div.center div.tabular {text-align: center; margin-top:0.5em; margin-bottom:0.5em; }
table.tabular td p{margin-top:0em;}
table.tabular {margin-left: auto; margin-right: auto;}
div.td00{ margin-left:0pt; margin-right:0pt; }
div.td01{ margin-left:0pt; margin-right:5pt; }
div.td10{ margin-left:5pt; margin-right:0pt; }
div.td11{ margin-left:5pt; margin-right:5pt; }
table[rules] {border-left:solid black 0.4pt; border-right:solid black 0.4pt; }
td.td00{ padding-left:0pt; padding-right:0pt; }
td.td01{ padding-left:0pt; padding-right:5pt; }
td.td10{ padding-left:5pt; padding-right:0pt; }
td.td11{ padding-left:5pt; padding-right:5pt; }
table[rules] {border-left:solid black 0.4pt; border-right:solid black 0.4pt; }
.hline hr, .cline hr{ height : 1px; margin:0px; }
.tabbing-right {text-align:right;}
span.TEX {letter-spacing: -0.125em; }
span.TEX span.E{ position:relative;top:0.5ex;left:-0.0417em;}
a span.TEX span.E {text-decoration: none; }
span.LATEX span.A{ position:relative; top:-0.5ex; left:-0.4em; font-size:85%;}
span.LATEX span.TEX{ position:relative; left: -0.4em; }
div.float img, div.float .caption {text-align:center;}
div.figure img, div.figure .caption {text-align:center;}
.marginpar {width:20%; float:right; text-align:left; margin-left:auto; margin-top:0.5em; font-size:85%; text-decoration:underline;}
.marginpar p{margin-top:0.4em; margin-bottom:0.4em;}
.equation td{text-align:center; vertical-align:middle; }
td.eq-no{ width:5%; }
table.equation { width:100%; } 
div.math-display, div.par-math-display{text-align:center;}
math .texttt { font-family: monospace; }
math .textit { font-style: italic; }
math .textsl { font-style: oblique; }
math .textsf { font-family: sans-serif; }
math .textbf { font-weight: bold; }
.partToc a, .partToc, .likepartToc a, .likepartToc {line-height: 200%; font-weight:bold; font-size:110%;}
.chapterToc a, .chapterToc, .likechapterToc a, .likechapterToc, .appendixToc a, .appendixToc {line-height: 200%; font-weight:bold;}
.index-item, .index-subitem, .index-subsubitem {display:block}
.caption td.id{font-weight: bold; white-space: nowrap; }
table.caption {text-align:center;}
h1.partHead{text-align: center}
p.bibitem { text-indent: -2em; margin-left: 2em; margin-top:0.6em; margin-bottom:0.6em; }
p.bibitem-p { text-indent: 0em; margin-left: 2em; margin-top:0.6em; margin-bottom:0.6em; }
.paragraphHead, .likeparagraphHead { margin-top:2em; font-weight: bold;}
.subparagraphHead, .likesubparagraphHead { font-weight: bold;}
.quote {margin-bottom:0.25em; margin-top:0.25em; margin-left:1em; margin-right:1em; text-align:justify;}
.verse{white-space:nowrap; margin-left:2em}
div.maketitle {text-align:center;}
h2.titleHead{text-align:center;}
div.maketitle{ margin-bottom: 2em; }
div.author, div.date {text-align:center;}
div.thanks{text-align:left; margin-left:10%; font-size:85%; font-style:italic; }
div.author{white-space: nowrap;}
.quotation {margin-bottom:0.25em; margin-top:0.25em; margin-left:1em; }
h1.partHead{text-align: center}
.sectionToc, .likesectionToc {margin-left:2em;}
.subsectionToc, .likesubsectionToc {margin-left:4em;}
.subsubsectionToc, .likesubsubsectionToc {margin-left:6em;}
.frenchb-nbsp{font-size:75%;}
.frenchb-thinspace{font-size:75%;}
.figure img.graphics {margin-left:10%;}
/* end css.sty */

\title{Ensembles et relations}
\author{}
\date{}

\begin{document}
\maketitle

\textbf{Warning: \href{http://www.math.union.edu/locate/jsMath}{jsMath}
requires JavaScript to process the mathematics on this page.\\ If your
browser supports JavaScript, be sure it is enabled.}

\begin{center}\rule{3in}{0.4pt}\end{center}

{[}\href{coursse2.html}{next}{]}
{[}\hyperref[tailcoursse1.html]{tail}{]}
{[}\href{coursch2.html\#coursse1.html}{up}{]}

\subsubsection{1.1 Ensembles et relations}

\paragraph{1.1.1 Relations d'équivalences}

Définition~1.1.1 Soit E un ensemble. On appelle relation sur E toute
partie de E × E. Si ℛ est une relation, on note habituellement xℛy à la
place de (x,y) ∈ℛ.

On dit que la relation est

\begin{itemize}
\itemsep1pt\parskip0pt\parsep0pt
\item
  réflexive si \textbackslash{}mathop\{∀\}x ∈ E, xℛx,
\item
  symétrique si \textbackslash{}mathop\{∀\}x,y ∈ E,\textbackslash{}quad
  xℛy ⇒ yℛx,
\item
  antisymétrique si \textbackslash{}mathop\{∀\}x,y ∈
  E,\textbackslash{}quad (xℛy\textbackslash{}text\{ et \}yℛx) ⇒ x = y,
\item
  transitive si \textbackslash{}mathop\{∀\}x,y,z ∈
  E,\textbackslash{}quad (xℛy\textbackslash{}text\{ et \}yℛz) ⇒ xℛz.
\end{itemize}

Définition~1.1.2 On appelle relation d'équivalence sur un ensemble E
toute relation réflexive, symétrique et transitive.

Définition~1.1.3 Pour un élément x de E, on appelle classe de x par
rapport à la relation d'équivalence ℛ l'ensemble des éléments y de E
tels que yℛx, notée \{C\}\_\{ℛ\}(x).

Proposition~1.1.1 Deux classes d'équivalences sont soit confondues, soit
disjointes.

Démonstration Soit x,x' ∈ E, et supposons que y ∈C(x) ∩C(x'). Soit z
∈C(x). On a alors zℛx, xℛy (car on a yℛx et la relation est symétrique)
et yℛx'. Par transitivité de la relation, on a zℛx', soit z ∈C(x'). On a
donc C(x) ⊂C(x') et comme x et x' jouent un rôle symétrique on a aussi
C(x') ⊂C(x), et donc C(x') = C(x).

Remarque~1.1.1 On a les équivalences suivantes

xℛy \textbackslash{}mathrel\{⇔\} x ∈C(y) \textbackslash{}mathrel\{⇔\}
C(x) = C(y)

Définition~1.1.4 On appelle système de représentants des classes
d'équivalences une partie A de E telle que, pour tout y ∈ E, il existe
un unique x ∈ A tel que yℛx. Dans ce cas la famille
\{\textbackslash{}left (C(x)\textbackslash{}right )\}\_\{x∈A\} est une
partition de E.

Remarque~1.1.2 Inversement, à toute partition
\{(\{A\}\_\{i\})\}\_\{i∈I\} d'un ensemble E, on peut associer une unique
relation d'équivalence en posant

xℛy \textbackslash{}mathrel\{⇔\} \textbackslash{}mathop\{∃\}i ∈ I, x,y ∈
\{A\}\_\{i\}

Les classes d'équivalences sont alors exactement les \{A\}\_\{i\}.

Définition~1.1.5 Soit E un ensemble et ℛ une relation d'équivalence sur
E. L'ensemble des classes d'équivalences des éléments de E est appelé
ensemble quotient de E par ℛ et noté E∕ℛ. L'application surjective
x\textbackslash{}mathrel\{↦\}C(x) de E dans E∕ℛ est appelée la
projection (ou la surjection) canonique.

Remarque~1.1.3 A est un système de représentants des classes
d'équivalence modulo ℛ, si et seulement si la restriction de la
projection canonique à A est bijective de A sur E∕ℛ.

\paragraph{1.1.2 Relations d'ordre}

Définition~1.1.6 Soit E un ensemble. On appelle relation d'ordre sur E
toute relation binaire ≼ sur E qui est à la fois réflexive, transitive
et antisymétrique. On appelle relation d'ordre strict sur E toute
relation binaire ≺ sur E qui est transitive et qui vérifie x ≺ y ⇒
x\textbackslash{}mathrel\{≠\}y.

Remarque~1.1.4 On vérifie immédiatement que les applications

≼ \textbackslash{}mathrel\{↦\} ≺\textbackslash{}text\{ définie par \}x ≺
y \textbackslash{}mathrel\{⇔\} (x ≼ y\textbackslash{}text\{ et
\}x\textbackslash{}mathrel\{≠\}y)

et

≺ \textbackslash{}mathrel\{↦\} ≼\textbackslash{}text\{ définie par \}x ≼
y \textbackslash{}mathrel\{⇔\} (x ≺ y\textbackslash{}text\{ ou \}x = y)

réalisent des bijections réciproques l'une de l'autre entre les
relations d'ordre sur E et les relations d'ordre strict sur E. On
conviendra donc dans la suite d'associer ainsi canoniquement une
relation d'ordre strict à toute relation d'ordre et réciproquement.

Définition~1.1.7 On dit que la relation d'ordre ≼ sur E est totale si

\textbackslash{}mathop\{∀\}a,b ∈ E,\textbackslash{}text\{ on a \}a ≼
b\textbackslash{}text\{ ou \}b ≼ a

Dans le cas contraire, on dit que la relation d'ordre est partielle.

Remarque~1.1.5 On prendra garde aux relations d'ordre partielles qui ont
des propriétés un peu déroutantes. L'exemple typique d'une telle
relation est la relation d'inclusion entre deux sous-ensembles d'un même
ensemble E.

\paragraph{1.1.3 Eléments extrémaux}

Définition~1.1.8 Soit (E,≼) un ensemble ordonné et A une partie de E.
Soit a ∈ E. On dit que

\begin{itemize}
\itemsep1pt\parskip0pt\parsep0pt
\item
  a est un majorant de A si \textbackslash{}mathop\{∀\}x ∈ A, x ≼ a
\item
  a est un plus grand élément de A s'il appartient à A et est un
  majorant de A
\end{itemize}

On définit de même minorant et plus petit élément.

Définition~1.1.9 Soit (E,≼) un ensemble ordonné et A une partie de E.
Soit a ∈ E. On dit que a est borne supérieure de A si l'ensemble des
majorants de A est non vide et admet a comme plus petit élément. On
définit de même une borne inférieure.

Remarque~1.1.6 L'antisymétrie de la relation d'ordre assure clairement
l'unicité d'un plus grand ou plus petit élément, et donc l'unicité d'une
borne supérieure ou inférieure. On prendra garde que les uns comme les
autres peuvent très bien ne pas exister, même lorsque la relation
d'ordre est totale comme le montre l'exemple de E = ℝ,A = ℝ.

Définition~1.1.10 Soit (E,≼) un ensemble ordonné et A une partie de E.
Soit a ∈ E. On dit que a est un un élément maximal de A si

a ∈ A\textbackslash{}text\{ et \}\textbackslash{}quad
\textbackslash{}mathop\{∀\}x ∈ A, a ≼ x ⇒ a = x

autrement dit si A n'admet aucun élément strictement plus grand que a.
On définit de même la notion d'élément minimal de A.

Remarque~1.1.7 Lorsque ≼ est une relation d'ordre total, il est clair
que la notion d'élément maximal coïncide avec la notion de plus grand
élément. Mais il n'en est pas de même pour une relation d'ordre partiel.
Plus grand élément signifie ''plus grand que tous les autres'' alors que
élément maximal signifie ''il n'y en a pas de strictement plus grand''.

\paragraph{1.1.4 L'axiome de Zorn}

L'existence d'éléments maximaux dans certains ensembles partiellement
ordonnés est souvent une propriété essentielle comme on le verra par la
suite. Cette existence est claire dans les ensembles finis. Dans les
ensembles infinis, elle résulte la plupart du temps d'un axiome appelé
l'axiome de Zorn. Pour cela on introduira la définition suivante

Définition~1.1.11 On dit qu'un ensemble ordonné (E,≼) est inductif si
toute partie non vide totalement ordonnée de E admet un majorant.

Axiome~1.1.1 (Zorn) Tout ensemble inductif admet un élément maximal.

Remarque~1.1.8 On montre que cet axiome est équivalent à l'axiome
beaucoup plus naturel suivant

Axiome~1.1.2 (Axiome du choix) Soit E un ensemble. Il existe une
application f de P(E) ∖\textbackslash{}\{∅\textbackslash{}\} dans E
telle que \textbackslash{}mathop\{∀\}A ⊂ E,
A\textbackslash{}mathrel\{≠\}∅, f(A) ∈ A.

Remarque~1.1.9 Ce dernier axiome signifie simplement que l'on peut
choisir ''simultanément'' un élément a = f(A) dans chaque partie non
vide A de E.

{[}\href{coursse2.html}{next}{]} {[}\href{coursse1.html}{front}{]}
{[}\href{coursch2.html\#coursse1.html}{up}{]}

\end{document}
