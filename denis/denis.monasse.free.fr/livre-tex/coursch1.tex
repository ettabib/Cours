\documentclass[]{article}
\usepackage[T1]{fontenc}
\usepackage{lmodern}
\usepackage{amssymb,amsmath}
\usepackage{ifxetex,ifluatex}
\usepackage{fixltx2e} % provides \textsubscript
% use upquote if available, for straight quotes in verbatim environments
\IfFileExists{upquote.sty}{\usepackage{upquote}}{}
\ifnum 0\ifxetex 1\fi\ifluatex 1\fi=0 % if pdftex
  \usepackage[utf8]{inputenc}
\else % if luatex or xelatex
  \ifxetex
    \usepackage{mathspec}
    \usepackage{xltxtra,xunicode}
  \else
    \usepackage{fontspec}
  \fi
  \defaultfontfeatures{Mapping=tex-text,Scale=MatchLowercase}
  \newcommand{\euro}{€}
\fi
% use microtype if available
\IfFileExists{microtype.sty}{\usepackage{microtype}}{}
\ifxetex
  \usepackage[setpagesize=false, % page size defined by xetex
              unicode=false, % unicode breaks when used with xetex
              xetex]{hyperref}
\else
  \usepackage[unicode=true]{hyperref}
\fi
\hypersetup{breaklinks=true,
            bookmarks=true,
            pdfauthor={},
            pdftitle={Avant-propos},
            colorlinks=true,
            citecolor=blue,
            urlcolor=blue,
            linkcolor=magenta,
            pdfborder={0 0 0}}
\urlstyle{same}  % don't use monospace font for urls
\setlength{\parindent}{0pt}
\setlength{\parskip}{6pt plus 2pt minus 1pt}
\setlength{\emergencystretch}{3em}  % prevent overfull lines
\setcounter{secnumdepth}{0}
 
/* start css.sty */
.cmr-5{font-size:50%;}
.cmr-7{font-size:70%;}
.cmmi-5{font-size:50%;font-style: italic;}
.cmmi-7{font-size:70%;font-style: italic;}
.cmmi-10{font-style: italic;}
.cmsy-5{font-size:50%;}
.cmsy-7{font-size:70%;}
.cmex-7{font-size:70%;}
.cmex-7x-x-71{font-size:49%;}
.msbm-7{font-size:70%;}
.cmtt-10{font-family: monospace;}
.cmti-10{ font-style: italic;}
.cmbx-10{ font-weight: bold;}
.cmr-17x-x-120{font-size:204%;}
.cmsl-10{font-style: oblique;}
.cmti-7x-x-71{font-size:49%; font-style: italic;}
.cmbxti-10{ font-weight: bold; font-style: italic;}
p.noindent { text-indent: 0em }
td p.noindent { text-indent: 0em; margin-top:0em; }
p.nopar { text-indent: 0em; }
p.indent{ text-indent: 1.5em }
@media print {div.crosslinks {visibility:hidden;}}
a img { border-top: 0; border-left: 0; border-right: 0; }
center { margin-top:1em; margin-bottom:1em; }
td center { margin-top:0em; margin-bottom:0em; }
.Canvas { position:relative; }
li p.indent { text-indent: 0em }
.enumerate1 {list-style-type:decimal;}
.enumerate2 {list-style-type:lower-alpha;}
.enumerate3 {list-style-type:lower-roman;}
.enumerate4 {list-style-type:upper-alpha;}
div.newtheorem { margin-bottom: 2em; margin-top: 2em;}
.obeylines-h,.obeylines-v {white-space: nowrap; }
div.obeylines-v p { margin-top:0; margin-bottom:0; }
.overline{ text-decoration:overline; }
.overline img{ border-top: 1px solid black; }
td.displaylines {text-align:center; white-space:nowrap;}
.centerline {text-align:center;}
.rightline {text-align:right;}
div.verbatim {font-family: monospace; white-space: nowrap; text-align:left; clear:both; }
.fbox {padding-left:3.0pt; padding-right:3.0pt; text-indent:0pt; border:solid black 0.4pt; }
div.fbox {display:table}
div.center div.fbox {text-align:center; clear:both; padding-left:3.0pt; padding-right:3.0pt; text-indent:0pt; border:solid black 0.4pt; }
div.minipage{width:100%;}
div.center, div.center div.center {text-align: center; margin-left:1em; margin-right:1em;}
div.center div {text-align: left;}
div.flushright, div.flushright div.flushright {text-align: right;}
div.flushright div {text-align: left;}
div.flushleft {text-align: left;}
.underline{ text-decoration:underline; }
.underline img{ border-bottom: 1px solid black; margin-bottom:1pt; }
.framebox-c, .framebox-l, .framebox-r { padding-left:3.0pt; padding-right:3.0pt; text-indent:0pt; border:solid black 0.4pt; }
.framebox-c {text-align:center;}
.framebox-l {text-align:left;}
.framebox-r {text-align:right;}
span.thank-mark{ vertical-align: super }
span.footnote-mark sup.textsuperscript, span.footnote-mark a sup.textsuperscript{ font-size:80%; }
div.tabular, div.center div.tabular {text-align: center; margin-top:0.5em; margin-bottom:0.5em; }
table.tabular td p{margin-top:0em;}
table.tabular {margin-left: auto; margin-right: auto;}
div.td00{ margin-left:0pt; margin-right:0pt; }
div.td01{ margin-left:0pt; margin-right:5pt; }
div.td10{ margin-left:5pt; margin-right:0pt; }
div.td11{ margin-left:5pt; margin-right:5pt; }
table[rules] {border-left:solid black 0.4pt; border-right:solid black 0.4pt; }
td.td00{ padding-left:0pt; padding-right:0pt; }
td.td01{ padding-left:0pt; padding-right:5pt; }
td.td10{ padding-left:5pt; padding-right:0pt; }
td.td11{ padding-left:5pt; padding-right:5pt; }
table[rules] {border-left:solid black 0.4pt; border-right:solid black 0.4pt; }
.hline hr, .cline hr{ height : 1px; margin:0px; }
.tabbing-right {text-align:right;}
span.TEX {letter-spacing: -0.125em; }
span.TEX span.E{ position:relative;top:0.5ex;left:-0.0417em;}
a span.TEX span.E {text-decoration: none; }
span.LATEX span.A{ position:relative; top:-0.5ex; left:-0.4em; font-size:85%;}
span.LATEX span.TEX{ position:relative; left: -0.4em; }
div.float img, div.float .caption {text-align:center;}
div.figure img, div.figure .caption {text-align:center;}
.marginpar {width:20%; float:right; text-align:left; margin-left:auto; margin-top:0.5em; font-size:85%; text-decoration:underline;}
.marginpar p{margin-top:0.4em; margin-bottom:0.4em;}
.equation td{text-align:center; vertical-align:middle; }
td.eq-no{ width:5%; }
table.equation { width:100%; } 
div.math-display, div.par-math-display{text-align:center;}
math .texttt { font-family: monospace; }
math .textit { font-style: italic; }
math .textsl { font-style: oblique; }
math .textsf { font-family: sans-serif; }
math .textbf { font-weight: bold; }
.partToc a, .partToc, .likepartToc a, .likepartToc {line-height: 200%; font-weight:bold; font-size:110%;}
.chapterToc a, .chapterToc, .likechapterToc a, .likechapterToc, .appendixToc a, .appendixToc {line-height: 200%; font-weight:bold;}
.index-item, .index-subitem, .index-subsubitem {display:block}
.caption td.id{font-weight: bold; white-space: nowrap; }
table.caption {text-align:center;}
h1.partHead{text-align: center}
p.bibitem { text-indent: -2em; margin-left: 2em; margin-top:0.6em; margin-bottom:0.6em; }
p.bibitem-p { text-indent: 0em; margin-left: 2em; margin-top:0.6em; margin-bottom:0.6em; }
.paragraphHead, .likeparagraphHead { margin-top:2em; font-weight: bold;}
.subparagraphHead, .likesubparagraphHead { font-weight: bold;}
.quote {margin-bottom:0.25em; margin-top:0.25em; margin-left:1em; margin-right:1em; text-align:\\jmathmathustify;}
.verse{white-space:nowrap; margin-left:2em}
div.maketitle {text-align:center;}
h2.titleHead{text-align:center;}
div.maketitle{ margin-bottom: 2em; }
div.author, div.date {text-align:center;}
div.thanks{text-align:left; margin-left:10%; font-size:85%; font-style:italic; }
div.author{white-space: nowrap;}
.quotation {margin-bottom:0.25em; margin-top:0.25em; margin-left:1em; }
h1.partHead{text-align: center}
.sectionToc, .likesectionToc {margin-left:2em;}
.subsectionToc, .likesubsectionToc {margin-left:4em;}
.subsubsectionToc, .likesubsubsectionToc {margin-left:6em;}
.frenchb-nbsp{font-size:75%;}
.frenchb-thinspace{font-size:75%;}
.figure img.graphics {margin-left:10%;}
/* end css.sty */

\title{Avant-propos}
\author{}
\date{}

\begin{document}
\maketitle

\textbf{Warning: 
requires JavaScript to process the mathematics on this page.\\ If your
browser supports JavaScript, be sure it is enabled.}

\begin{center}\rule{3in}{0.4pt}\end{center}

{[}
{[}{]}
{[}

\subsection{Avant-propos}

Pour Roger Godement et Edmond Ramis, mes maîtres

Publier un cours de mathématiques au\\jmathmathourd'hui peut sembler une gageure.
Après avoir longtemps reculé devant ce défi, \\jmathmath'ai profité des récents
changements de programme en classes préparatoires pour remettre mes
notes au propre. Je n'ai pu résister alors à l'envie de mettre ce
travail à la disposition de la communauté enseignante et étudiante.

Le contenu Ce cours est volontairement concis. Mon ob\\jmathmathectif est d'offrir
au lecteur, en un seul volume, la totalité (et même plus) de ce que l'on
peut enseigner en un an à un bon étudiant de classes préparatoires ou de
premier cycle universitaire. Il ne s'agit aucunement d'un ouvrage
destiné à l'auto-apprentissage des mathématiques et le lecteur n'y
trouvera aucun exercice et très peu d'exemples développés. Par contre
toutes les définitions et tous les résultats du cours y figurent, avec
presque tou\\jmathmathours des démonstrations complètes. Il s'agit donc d'un
ouvrage qui peut servir d'instrument de révision et de référence tout au
long de l'année. L'étudiant s'y reportera pour retrouver sous une autre
forme l'enseignement magistral qu'il a dé\\jmathmathà suivi, pour approfondir cet
enseignement par des compléments qui font souvent l'ob\\jmathmathet de questions à
l'écrit ou à l'oral, ou pour effectuer des révisions rapides avant les
contrôles, examens et concours. Il ne manque pas, chez tous les
éditeurs, de livres d'exercices et de problèmes pour permettre au
lecteur de mettre en oeuvre toute sa sagacité et ses capacités de
recherche et pour compléter, par des exemples et des applications, les
résultats exposés ici.

Pour quel enseignement~? Nous avons voulu faire un cours de
mathématiques, pas un cours de Mathématiques Spéciales. Pour cette
raison, nous n'avons pas suivi à la lettre le programme des classes MP.
Nous voulions d'une part que cet ouvrage soit facilement utilisable par
les étudiants de premier cycle universitaire ou par les candidats au
CAPES et à l'Agrégation~; d'autre part il est un proverbe bien connu
dans le système éducatif~: les programmes passent, les mathématiques
(resp. les sciences physiques, les sciences industrielles, le
fran\ccais, la philosophie, etc.) restent. C'est
pourquoi, sur des su\\jmathmathets traités de manière un peu originale à ce niveau
dans le programme actuel de taupe, comme l'intégration, nous avons
traité deux points de vue, aussi bien celui des fonctions intégrables
que celui des intégrales impropres, au prix bien entendu de certaines
redites. De plus, nous avons volontairement inclus un certain nombre de
compléments de cours qui sont donc hors de tout programme, soit pour
leur intérêt mathématique propre et leur beauté intrinsèque (réduction
de Jordan, topologie générale, convexité dans les espaces vectoriels
normés, applications du théorème de Baire), soit parce qu'ils exposent
des techniques un peu vieillottes mais qui font tou\\jmathmathours la \\jmathmathoie des
examinateurs de concours (enveloppes, tra\\jmathmathectoires orthogonales,
podaires, etc.). Nous espérons que les puristes des programmes nous
pardonneront.

Remerciements Je voudrais remercier ici tous mes collègues du lycée
Louis le Grand pour l'atmosphère d'amitié qui règne dans la salle des
professeurs et pour les longues discussions qui m'ont permis d'améliorer
peu à peu le contenu de mon cours~; de ce point de vue, ma
reconnaissance va tout naturellement à Jacques Chevallet, Claude
Deschamps, Daniel Mollier et Alain Pommellet. Merci également à tous mes
élèves qui n'ont \\jmathmathamais manqué de souligner avec gentillesse tout
passage un peu faible ou un peu obscur dans mon cours. Mais \\jmathmathe voudrais
remercier tout particulièrement Jacques Chevallet pour l'aide qu'il m'a
apportée en gâchant toute une partie de ses vacances à relire et à
corriger mon manuscrit~; ses corrections \\jmathmathudicieuses ont largement
contribué à améliorer la rédaction et les démonstrations sur bien des
points. Je ne voudrais pas terminer sans remercier mon épouse et mes
enfants pour le soutien moral et l'affection dont ils m'ont entouré tout
au long de ces années.

Bibliographie

Marcel Berger, Géométrie, tomes 1 à 5, CEDIC-Nathan, 1977

Jean Dieudonné, Calcul infinitésimal, Hermann, 1968

Jean Dieudonné, Eléments d'analyse, tomes 1 et 2, Gauthier-Villars,1969

Roger Godement, Cours d'algèbre, Hermann, Paris 1966

Walter Rudin, Real and Complex Analysis, McGraw-Hill, 1970

Walter Rudin, Functional Analysis, McGraw-Hill, 1974

{[}
{[}

\end{document}
