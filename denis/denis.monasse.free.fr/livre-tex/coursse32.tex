\documentclass[]{article}
\usepackage[T1]{fontenc}
\usepackage{lmodern}
\usepackage{amssymb,amsmath}
\usepackage{ifxetex,ifluatex}
\usepackage{fixltx2e} % provides \textsubscript
% use upquote if available, for straight quotes in verbatim environments
\IfFileExists{upquote.sty}{\usepackage{upquote}}{}
\ifnum 0\ifxetex 1\fi\ifluatex 1\fi=0 % if pdftex
  \usepackage[utf8]{inputenc}
\else % if luatex or xelatex
  \ifxetex
    \usepackage{mathspec}
    \usepackage{xltxtra,xunicode}
  \else
    \usepackage{fontspec}
  \fi
  \defaultfontfeatures{Mapping=tex-text,Scale=MatchLowercase}
  \newcommand{\euro}{€}
\fi
% use microtype if available
\IfFileExists{microtype.sty}{\usepackage{microtype}}{}
\ifxetex
  \usepackage[setpagesize=false, % page size defined by xetex
              unicode=false, % unicode breaks when used with xetex
              xetex]{hyperref}
\else
  \usepackage[unicode=true]{hyperref}
\fi
\hypersetup{breaklinks=true,
            bookmarks=true,
            pdfauthor={},
            pdftitle={Relations de comparaison},
            colorlinks=true,
            citecolor=blue,
            urlcolor=blue,
            linkcolor=magenta,
            pdfborder={0 0 0}}
\urlstyle{same}  % don't use monospace font for urls
\setlength{\parindent}{0pt}
\setlength{\parskip}{6pt plus 2pt minus 1pt}
\setlength{\emergencystretch}{3em}  % prevent overfull lines
\setcounter{secnumdepth}{0}
 
/* start css.sty */
.cmr-5{font-size:50%;}
.cmr-7{font-size:70%;}
.cmmi-5{font-size:50%;font-style: italic;}
.cmmi-7{font-size:70%;font-style: italic;}
.cmmi-10{font-style: italic;}
.cmsy-5{font-size:50%;}
.cmsy-7{font-size:70%;}
.cmex-7{font-size:70%;}
.cmex-7x-x-71{font-size:49%;}
.msbm-7{font-size:70%;}
.cmtt-10{font-family: monospace;}
.cmti-10{ font-style: italic;}
.cmbx-10{ font-weight: bold;}
.cmr-17x-x-120{font-size:204%;}
.cmsl-10{font-style: oblique;}
.cmti-7x-x-71{font-size:49%; font-style: italic;}
.cmbxti-10{ font-weight: bold; font-style: italic;}
p.noindent { text-indent: 0em }
td p.noindent { text-indent: 0em; margin-top:0em; }
p.nopar { text-indent: 0em; }
p.indent{ text-indent: 1.5em }
@media print {div.crosslinks {visibility:hidden;}}
a img { border-top: 0; border-left: 0; border-right: 0; }
center { margin-top:1em; margin-bottom:1em; }
td center { margin-top:0em; margin-bottom:0em; }
.Canvas { position:relative; }
li p.indent { text-indent: 0em }
.enumerate1 {list-style-type:decimal;}
.enumerate2 {list-style-type:lower-alpha;}
.enumerate3 {list-style-type:lower-roman;}
.enumerate4 {list-style-type:upper-alpha;}
div.newtheorem { margin-bottom: 2em; margin-top: 2em;}
.obeylines-h,.obeylines-v {white-space: nowrap; }
div.obeylines-v p { margin-top:0; margin-bottom:0; }
.overline{ text-decoration:overline; }
.overline img{ border-top: 1px solid black; }
td.displaylines {text-align:center; white-space:nowrap;}
.centerline {text-align:center;}
.rightline {text-align:right;}
div.verbatim {font-family: monospace; white-space: nowrap; text-align:left; clear:both; }
.fbox {padding-left:3.0pt; padding-right:3.0pt; text-indent:0pt; border:solid black 0.4pt; }
div.fbox {display:table}
div.center div.fbox {text-align:center; clear:both; padding-left:3.0pt; padding-right:3.0pt; text-indent:0pt; border:solid black 0.4pt; }
div.minipage{width:100%;}
div.center, div.center div.center {text-align: center; margin-left:1em; margin-right:1em;}
div.center div {text-align: left;}
div.flushright, div.flushright div.flushright {text-align: right;}
div.flushright div {text-align: left;}
div.flushleft {text-align: left;}
.underline{ text-decoration:underline; }
.underline img{ border-bottom: 1px solid black; margin-bottom:1pt; }
.framebox-c, .framebox-l, .framebox-r { padding-left:3.0pt; padding-right:3.0pt; text-indent:0pt; border:solid black 0.4pt; }
.framebox-c {text-align:center;}
.framebox-l {text-align:left;}
.framebox-r {text-align:right;}
span.thank-mark{ vertical-align: super }
span.footnote-mark sup.textsuperscript, span.footnote-mark a sup.textsuperscript{ font-size:80%; }
div.tabular, div.center div.tabular {text-align: center; margin-top:0.5em; margin-bottom:0.5em; }
table.tabular td p{margin-top:0em;}
table.tabular {margin-left: auto; margin-right: auto;}
div.td00{ margin-left:0pt; margin-right:0pt; }
div.td01{ margin-left:0pt; margin-right:5pt; }
div.td10{ margin-left:5pt; margin-right:0pt; }
div.td11{ margin-left:5pt; margin-right:5pt; }
table[rules] {border-left:solid black 0.4pt; border-right:solid black 0.4pt; }
td.td00{ padding-left:0pt; padding-right:0pt; }
td.td01{ padding-left:0pt; padding-right:5pt; }
td.td10{ padding-left:5pt; padding-right:0pt; }
td.td11{ padding-left:5pt; padding-right:5pt; }
table[rules] {border-left:solid black 0.4pt; border-right:solid black 0.4pt; }
.hline hr, .cline hr{ height : 1px; margin:0px; }
.tabbing-right {text-align:right;}
span.TEX {letter-spacing: -0.125em; }
span.TEX span.E{ position:relative;top:0.5ex;left:-0.0417em;}
a span.TEX span.E {text-decoration: none; }
span.LATEX span.A{ position:relative; top:-0.5ex; left:-0.4em; font-size:85%;}
span.LATEX span.TEX{ position:relative; left: -0.4em; }
div.float img, div.float .caption {text-align:center;}
div.figure img, div.figure .caption {text-align:center;}
.marginpar {width:20%; float:right; text-align:left; margin-left:auto; margin-top:0.5em; font-size:85%; text-decoration:underline;}
.marginpar p{margin-top:0.4em; margin-bottom:0.4em;}
.equation td{text-align:center; vertical-align:middle; }
td.eq-no{ width:5%; }
table.equation { width:100%; } 
div.math-display, div.par-math-display{text-align:center;}
math .texttt { font-family: monospace; }
math .textit { font-style: italic; }
math .textsl { font-style: oblique; }
math .textsf { font-family: sans-serif; }
math .textbf { font-weight: bold; }
.partToc a, .partToc, .likepartToc a, .likepartToc {line-height: 200%; font-weight:bold; font-size:110%;}
.chapterToc a, .chapterToc, .likechapterToc a, .likechapterToc, .appendixToc a, .appendixToc {line-height: 200%; font-weight:bold;}
.index-item, .index-subitem, .index-subsubitem {display:block}
.caption td.id{font-weight: bold; white-space: nowrap; }
table.caption {text-align:center;}
h1.partHead{text-align: center}
p.bibitem { text-indent: -2em; margin-left: 2em; margin-top:0.6em; margin-bottom:0.6em; }
p.bibitem-p { text-indent: 0em; margin-left: 2em; margin-top:0.6em; margin-bottom:0.6em; }
.paragraphHead, .likeparagraphHead { margin-top:2em; font-weight: bold;}
.subparagraphHead, .likesubparagraphHead { font-weight: bold;}
.quote {margin-bottom:0.25em; margin-top:0.25em; margin-left:1em; margin-right:1em; text-align:justify;}
.verse{white-space:nowrap; margin-left:2em}
div.maketitle {text-align:center;}
h2.titleHead{text-align:center;}
div.maketitle{ margin-bottom: 2em; }
div.author, div.date {text-align:center;}
div.thanks{text-align:left; margin-left:10%; font-size:85%; font-style:italic; }
div.author{white-space: nowrap;}
.quotation {margin-bottom:0.25em; margin-top:0.25em; margin-left:1em; }
h1.partHead{text-align: center}
.sectionToc, .likesectionToc {margin-left:2em;}
.subsectionToc, .likesubsectionToc {margin-left:4em;}
.subsubsectionToc, .likesubsubsectionToc {margin-left:6em;}
.frenchb-nbsp{font-size:75%;}
.frenchb-thinspace{font-size:75%;}
.figure img.graphics {margin-left:10%;}
/* end css.sty */

\title{Relations de comparaison}
\author{}
\date{}

\begin{document}
\maketitle

\textbf{Warning: \href{http://www.math.union.edu/locate/jsMath}{jsMath}
requires JavaScript to process the mathematics on this page.\\ If your
browser supports JavaScript, be sure it is enabled.}

\begin{center}\rule{3in}{0.4pt}\end{center}

{[}\href{coursse33.html}{next}{]}
{[}\hyperref[tailcoursse32.html]{tail}{]}
{[}\href{coursch7.html\#coursse32.html}{up}{]}

\subsubsection{6.1 Relations de comparaison}

\paragraph{6.1.1 Notations}

Soit A ⊂ ℝ et a ∈\textbackslash{}overline\{ℝ\} tel que a
∈\textbackslash{}overline\{A\} (adhérence dans
\textbackslash{}overline\{ℝ\}). Si E est un espace vectoriel normé, on
notera \{ℱ\}\_\{a,A\}(E) l'ensemble des fonctions de ℝ vers E telles
qu'il existe V ∈ V (a) tel que f soit définie sur V ∩ A (autrement dit f
est définie sur A au voisinage de a).

Exemple~6.1.1 Si A = ℕ et a = +∞, \{ℱ\}\_\{a,A\}(E) est l'ensemble des
suites \{(\{x\}\_\{n\})\}\_\{n≥\{n\}\_\{0\}\} d'éléments de E.

\paragraph{6.1.2 Domination, prépondérance}

Définition~6.1.1 Soit f ∈\{ℱ\}\_\{a,A\}(E) et g ∈\{ℱ\}\_\{a,A\}(F).

\begin{itemize}
\item
  (i) On dit que f est dominée par g au voisinage de a suivant A et on
  note f = O(g) si

  \textbackslash{}mathop\{∃\}K ≥ 0, \textbackslash{}mathop\{∃\}V ∈ V
  (a), \textbackslash{}mathop\{∀\}t ∈ V ∩ A,
  \textbackslash{}\textbar{}f(t)\textbackslash{}\textbar{} ≤
  K\textbackslash{}\textbar{}g(t)\textbackslash{}\textbar{}
\item
  (ii) On dit que g est prépondérante devant f (ou que f est négligeable
  devant g) au voisinage de a suivant A et on note f = o(g) si

  \textbackslash{}mathop\{∀\}ε \textgreater{} 0,
  \textbackslash{}mathop\{∃\}V ∈ V (a), \textbackslash{}mathop\{∀\}t ∈ V
  ∩ A, \textbackslash{}\textbar{}f(t)\textbackslash{}\textbar{} ≤
  ε\textbackslash{}\textbar{}g(t)\textbackslash{}\textbar{}
\end{itemize}

Remarque~6.1.1 Il est clair que f = o(g) ⇒ f = O(g). De plus, si f =
O(g) et si X est choisi comme ci dessus, on voit que g(t) = 0 ⇒ f(t) =
0~; on constate donc que

\begin{itemize}
\itemsep1pt\parskip0pt\parsep0pt
\item
  (i) aux indéterminations près de type \{ 0 \textbackslash{}over 0\} ,
  f = O(g) \textbackslash{}mathrel\{⇔\}\{
  \textbackslash{}\textbar{}f\textbackslash{}\textbar{}
  \textbackslash{}over
  \textbackslash{}\textbar{}g\textbackslash{}\textbar{}\} est bornée (au
  voisinage de a suivant A)
\item
  (ii) aux indéterminations près de type \{ 0 \textbackslash{}over 0\} ,
  f = o(g) \textbackslash{}mathrel\{⇔\}
  \{\textbackslash{}mathop\{lim\}\}\_\{t→a,t∈A\}\{
  \textbackslash{}\textbar{}f(t)\textbackslash{}\textbar{}
  \textbackslash{}over
  \textbackslash{}\textbar{}g(t)\textbackslash{}\textbar{}\} = 0
\end{itemize}

Exemple~6.1.2

\begin{itemize}
\itemsep1pt\parskip0pt\parsep0pt
\item
  f = O(1) \textbackslash{}mathrel\{⇔\} f est bornée au voisinage de a
  suivant A~;
\item
  f = o(1) \textbackslash{}mathrel\{⇔\}
  \{\textbackslash{}mathop\{lim\}\}\_\{t→a,t∈A\}f(t) = 0.
\end{itemize}

Proposition~6.1.1

\begin{itemize}
\item
  (i) \{f\}\_\{1\} = O(g)\textbackslash{}text\{ et \}\{f\}\_\{2\} = O(g)
  ⇒ α\{f\}\_\{1\} + μ\{f\}\_\{2\} = O(g)
\item
  (ii) \{f\}\_\{1\} = o(g)\textbackslash{}text\{ et \}\{f\}\_\{2\} =
  o(g) ⇒ α\{f\}\_\{1\} + μ\{f\}\_\{2\} = o(g)
\item
  (iii) soit φ,ψ ∈\{ℱ\}\_\{a,A\}(K) et f,g ∈\{ℱ\}\_\{a,A\}(E), alors

  \textbackslash{}begin\{eqnarray*\} φ = O(ψ)\textbackslash{}text\{ et
  \}f = O(g)\& ⇒\& φf = O(ψg)\%\& \textbackslash{}\textbackslash{} φ =
  o(ψ)\textbackslash{}text\{ et \}f = O(g)\& ⇒\& φf = o(ψg) \%\&
  \textbackslash{}\textbackslash{} φ = O(ψ)\textbackslash{}text\{ et \}f
  = o(g)\& ⇒\& φf = o(ψg) \%\& \textbackslash{}\textbackslash{}
  \textbackslash{}end\{eqnarray*\}
\item
  (iv) f ∈\{ℱ\}\_\{a,A\}(E),g ∈\{ℱ\}\_\{a,A\}(F),h ∈\{ℱ\}\_\{a,A\}(G)~;
  alors

  \textbackslash{}begin\{eqnarray*\} f = O(g)\textbackslash{}text\{ et
  \}g = O(h)\& ⇒\& f = O(h)\%\& \textbackslash{}\textbackslash{} f =
  O(g)\textbackslash{}text\{ et \}g = o(h)\& ⇒\& f = o(h) \%\&
  \textbackslash{}\textbackslash{} f = o(g)\textbackslash{}text\{ et \}g
  = O(h)\& ⇒\& f = o(h) \%\& \textbackslash{}\textbackslash{}
  \textbackslash{}end\{eqnarray*\}
\end{itemize}

Démonstration Facile

\paragraph{6.1.3 Equivalence}

Lemme~6.1.2 Soit f,g ∈\{ℱ\}\_\{a,A\}(E). Alors f − g = o(g) ⇒ g = O(f).

Démonstration Il existe V ∈ V (a) tel que \textbackslash{}mathop\{∀\}t ∈
V ∩ A, \textbackslash{}\textbar{}f(t) − g(t)\textbackslash{}\textbar{}
≤\{ 1 \textbackslash{}over 2\}
\textbackslash{}\textbar{}g(t)\textbackslash{}\textbar{}. Pour t ∈ V ∩
A, on a donc \textbackslash{}\textbar{}g(t)\textbackslash{}\textbar{}
=\textbackslash{}\textbar{} g(t) − f(t) + f(t)\textbackslash{}\textbar{}
≤\textbackslash{}\textbar{} g(t) − f(t)\textbackslash{}\textbar{}
+\textbackslash{}\textbar{} f(t)\textbackslash{}\textbar{} ≤\{ 1
\textbackslash{}over 2\}
\textbackslash{}\textbar{}g(t)\textbackslash{}\textbar{}
+\textbackslash{}\textbar{} f(t)\textbackslash{}\textbar{} soit encore
\textbackslash{}\textbar{}g(t)\textbackslash{}\textbar{} ≤
2\textbackslash{}\textbar{}f(t)\textbackslash{}\textbar{}, et donc g =
O(f).

Théorème~6.1.3 Pour f,g ∈\{ℱ\}\_\{a,A\}(E), on pose f ∼ g si f − g =
o(g). Il s'agit d'une relation d'équivalence appelée l'équivalence des
fonctions (au voisinage de a suivant A).

Démonstration La réflexivité est claire puisque f − f = 0 = o(f). Si f ∼
g, on a f − g = o(g) et aussi d'après le lemme, g = O(f), d'où f − g =
o(f) et donc aussi g − f = o(f), soit g ∼ f. La relation est donc
symétrique. Si f ∼ g et g ∼ h, on a f − g = o(g) et g − h = o(h). Mais
on a h ∼ g, soit h − g = o(g) soit g = O(h). Alors f − g = o(g) et g =
O(h) donne f − g = o(h) et donc f − h = (f − g) + (g − h) = o(h), d'où f
∼ h, ce qui démontre la transitivité.

Proposition~6.1.4

\begin{itemize}
\itemsep1pt\parskip0pt\parsep0pt
\item
  (i) f ∼ g ⇒ f = O(g)\textbackslash{}text\{ et \}g = O(f)
\item
  (ii) φ,ψ ∈\{ℱ\}\_\{a,A\}(K), f,g ∈\{ℱ\}\_\{a,A\}(E), alors φ ∼
  ψ\textbackslash{}text\{ et \}f ∼ g ⇒ φf ∼ ψg
\end{itemize}

Démonstration (i) est évident d'après le lemme ci dessus et la symétrie
de la relation. Pour (ii), on écrit φf − ψg = (φ − ψ)f + ψ(f − g). On a
φ − ψ = o(ψ)\textbackslash{}text\{ et \}f = O(g) ⇒ (φ − ψ)f = o(ψg) et f
− g = o(g) ⇒ ψ(f − g) = o(ψg), d'où φf − ψg = o(ψg) et φf ∼ ψg.

Remarque~6.1.2 La relation d'équivalence est donc compatible avec la
multiplication~; par contre, elle n'est pas compatible avec l'addition~:
\{f\}\_\{1\} ∼ \{g\}\_\{1\}\textbackslash{}text\{ et \}\{f\}\_\{2\} ∼
\{g\}\_\{2\}⇏\{f\}\_\{1\} + \{f\}\_\{2\} ∼ \{g\}\_\{1\} + \{g\}\_\{2\}
comme le montre l'exemple a = 0, \{f\}\_\{1\}(t) = 1 + t,\{g\}\_\{1\}(t)
= 1 + \{t\}\^{}\{2\},\{f\}\_\{2\}(t) = \{g\}\_\{2\}(t) = −1~; on a
\{f\}\_\{1\} ∼ \{g\}\_\{1\}\textbackslash{}text\{ et \}\{f\}\_\{2\} =
\{g\}\_\{2\}, pourtant \{f\}\_\{1\}(t) + \{f\}\_\{2\}(t) = t et
\{g\}\_\{1\}(t) + \{g\}\_\{2\}(t) = \{t\}\^{}\{2\} ne sont pas
équivalentes au voisinage de 0.

Lemme~6.1.5 Soit f,g ∈\{ℱ\}\_\{a,A\}(K). Alors on a équivalence de

\begin{itemize}
\itemsep1pt\parskip0pt\parsep0pt
\item
  (i) f ∼ g
\item
  (ii) il existe φ ∈\{ℱ\}\_\{a,A\}(K) telle que f = gφ et
  \{\textbackslash{}mathop\{lim\}\}\_\{t→a,t∈A\}φ(t) = 1
\end{itemize}

Démonstration (ii) ⇒(i). On écrit f − g = g(φ − 1) avec
\{\textbackslash{}mathop\{lim\}\}\_\{t→a,t∈A\}(φ(t) − 1) = 0, d'où f − g
= o(g) et f ∼ g.

(i) ⇒(ii). On a f = O(g). D'après une remarque précédente, il existe V ∈
V (a) tel que \textbackslash{}mathop\{∀\}t ∈ V ∩ A, g(t) = 0 ⇒ f(t) = 0.
Définissons φ sur V ∩ A de la manière suivante~: φ(t) =
\textbackslash{}left \textbackslash{}\{ \textbackslash{}cases\{ \{ f(t)
\textbackslash{}over g(t)\} \&si g(t)\textbackslash{}mathrel\{≠\}0
\textbackslash{}cr 1 \&si g(t) = 0 \} \textbackslash{}right .~; si
g(t)\textbackslash{}mathrel\{≠\}0, on a f(t) = φ(t)g(t) de manière
évidente et cela reste vrai si g(t) = 0 puisque alors on a aussi f(t) =
0. Montrons que \{\textbackslash{}mathop\{lim\}\}\_\{t→a,t∈A\}φ(t) = 1.
Soit ε \textgreater{} 0~; il existe \{V \}\_\{0\} ∈ V (a) tel que
\textbackslash{}mathop\{∀\}t ∈ \{V \}\_\{0\} ∩ A, \textbar{}f(t) −
g(t)\textbar{}≤ ε\textbar{}g(t)\textbar{} soit encore pour t ∈ \{V
\}\_\{0\} ∩ V ∩ A, \textbar{}φ(t) −
1\textbar{}\textbackslash{},\textbar{}g(t)\textbar{}≤
ε\textbar{}g(t)\textbar{}. Si g(t)\textbackslash{}mathrel\{≠\}0 on a
donc \textbar{}φ(t) − 1\textbar{}≤ ε mais cela reste vrai si g(t) = 0
puisqu'alors φ(t) = 1. On a donc bien
\{\textbackslash{}mathop\{lim\}\}\_\{t→a,t∈A\}φ(t) = 1.

Théorème~6.1.6

\begin{itemize}
\item
  (i) si f,g ∈\{ℱ\}\_\{a,A\}(K) et n ∈ ℕ, alors f ∼ g ⇒ \{f\}\^{}\{n\} ∼
  \{g\}\^{}\{n\}
\item
  (ii) si f,g ∈\{ℱ\}\_\{a,A\}(ℝ) et s'il existe V ∈ V (a) tel que
  \textbackslash{}mathop\{∀\}t ∈ V ∩ A, g(t) ≥ 0 (resp. \textgreater{}
  0) alors

  \textbackslash{}mathop\{∀\}α ∈ \{ℝ\}\^{}\{+\}\textbackslash{}text\{
  (resp. \$\textbackslash{}mathop\{∀\}α ∈ ℝ\$) \}f ∼ g ⇒ \{f\}\^{}\{α\}
  ∼ \{g\}\^{}\{α\}
\end{itemize}

Démonstration Résulte immédiatement du lemme précédent en remarquant
pour (ii) que si φ tend vers 1, elle est strictement positive au
voisinage de a et que \textbackslash{}mathop\{lim\}\{φ\}\^{}\{α\} = 1.

Remarque~6.1.3 La relation d'équivalence est donc compatible avec les
puissances entières ou réelles~; par contre elle n'est pas compatible
avec l'exponentielle~: en fait on a \{e\}\^{}\{f\} ∼ \{e\}\^{}\{g\}
\textbackslash{}mathrel\{⇔\} \textbackslash{}mathop\{lim\}(f − g) = 0.

Le théorème suivant justifie l'intérêt de l'utilisation des équivalents
pour les recherches de limites

Théorème~6.1.7 Soit f,g ∈\{ℱ\}\_\{a,A\}(E) telles que f ∼ g. Si g admet
une limite ℓ en a suivant A, f admet la même limite en a suivant A.

Démonstration Puisque \{\textbackslash{}mathop\{lim\}\}\_\{t→a,t∈A\}g(t)
= ℓ, il existe \{V \}\_\{0\} ∈ V (a) tel que t ∈ \{V \}\_\{0\} ∩ A
⇒\textbackslash{}\textbar{} g(t) − ℓ\textbackslash{}\textbar{}
\textless{} 1 soit
\textbackslash{}\textbar{}g(t)\textbackslash{}\textbar{} ≤ 1
+\textbackslash{}\textbar{} ℓ\textbackslash{}\textbar{}. Soit alors ε
\textgreater{} 0. Il existe V ∈ V (a) tel que
\textbackslash{}mathop\{∀\}t ∈ V ∩ A, \textbackslash{}\textbar{}g(t) −
ℓ\textbackslash{}\textbar{} \textless{}\{ ε \textbackslash{}over 2\} et
il existe V ' ∈ V (a) tel que \textbackslash{}mathop\{∀\}t ∈ V ' ∩ A,
\textbackslash{}\textbar{}f(t) − g(t)\textbackslash{}\textbar{} ≤\{ ε
\textbackslash{}over
2(1+\textbackslash{}\textbar{}ℓ\textbackslash{}\textbar{})\}
\textbackslash{}\textbar{}g(t)\textbackslash{}\textbar{}. Pour t ∈ \{V
\}\_\{0\} ∩ V ∩ V ' ∩ A on a

\textbackslash{}begin\{eqnarray*\} \textbackslash{}\textbar{}f(t) −
ℓ\textbackslash{}\textbar{}\& ≤\& \textbackslash{}\textbar{}f(t) −
g(t)\textbackslash{}\textbar{} +\textbackslash{}\textbar{} g(t) −
ℓ\textbackslash{}\textbar{}\%\& \textbackslash{}\textbackslash{} \&
≤\&\{ ε \textbackslash{}over 2(1 +\textbackslash{}\textbar{}
ℓ\textbackslash{}\textbar{})\} (1 +\textbackslash{}\textbar{}
ℓ\textbackslash{}\textbar{}) +\{ ε \textbackslash{}over 2\} \%\&
\textbackslash{}\textbackslash{} \& =\& ε \%\&
\textbackslash{}\textbackslash{} \textbackslash{}end\{eqnarray*\}

et donc f admet ℓ pour limite en a suivant A.

\paragraph{6.1.4 Changement de variables}

Soit A,B ⊂ ℝ et a,b ∈\textbackslash{}overline\{ℝ\} tel que a
∈\textbackslash{}overline\{A\} et b ∈\textbackslash{}overline\{B\}.

Soit φ une fonction de ℝ vers ℝ telle que φ(A) ⊂ B et
\{\textbackslash{}mathop\{lim\}\}\_\{t→a,t∈A\}φ(t) = b. Par définition
de la notion de limite on a aussitôt

Lemme~6.1.8 \textbackslash{}mathop\{∀\}V `∈ V (b),
\textbackslash{}mathop\{∃\}V ∈ V (a), φ(V ∩ A) ⊂ V' ∩ B.

Il en découle immédiatement le théorème suivant

Théorème~6.1.9 Soit f,g ∈\{ℱ\}\_\{b,B\}(E). Alors

\begin{itemize}
\itemsep1pt\parskip0pt\parsep0pt
\item
  (i) f = \{O\}\_\{b,B\}(g) ⇒ f ∘ φ = \{O\}\_\{a,A\}(g ∘ φ)
\item
  (i) f = \{o\}\_\{b,B\}(g) ⇒ f ∘ φ = \{o\}\_\{a,A\}(g ∘ φ)
\item
  (i) f \{∼\}\_\{b,B\}g ⇒ f ∘ φ \{∼\}\_\{a,A\}g ∘ φ
\end{itemize}

autrement dit on peut faire tout changement de variable raisonnable dans
des relations de comparaison.

{[}\href{coursse33.html}{next}{]} {[}\href{coursse32.html}{front}{]}
{[}\href{coursch7.html\#coursse32.html}{up}{]}

\end{document}
