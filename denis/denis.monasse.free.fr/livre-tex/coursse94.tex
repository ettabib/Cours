\documentclass[]{article}
\usepackage[T1]{fontenc}
\usepackage{lmodern}
\usepackage{amssymb,amsmath}
\usepackage{ifxetex,ifluatex}
\usepackage{fixltx2e} % provides \textsubscript
% use upquote if available, for straight quotes in verbatim environments
\IfFileExists{upquote.sty}{\usepackage{upquote}}{}
\ifnum 0\ifxetex 1\fi\ifluatex 1\fi=0 % if pdftex
  \usepackage[utf8]{inputenc}
\else % if luatex or xelatex
  \ifxetex
    \usepackage{mathspec}
    \usepackage{xltxtra,xunicode}
  \else
    \usepackage{fontspec}
  \fi
  \defaultfontfeatures{Mapping=tex-text,Scale=MatchLowercase}
  \newcommand{\euro}{€}
\fi
% use microtype if available
\IfFileExists{microtype.sty}{\usepackage{microtype}}{}
\ifxetex
  \usepackage[setpagesize=false, % page size defined by xetex
              unicode=false, % unicode breaks when used with xetex
              xetex]{hyperref}
\else
  \usepackage[unicode=true]{hyperref}
\fi
\hypersetup{breaklinks=true,
            bookmarks=true,
            pdfauthor={},
            pdftitle={Espaces affines euclidiens},
            colorlinks=true,
            citecolor=blue,
            urlcolor=blue,
            linkcolor=magenta,
            pdfborder={0 0 0}}
\urlstyle{same}  % don't use monospace font for urls
\setlength{\parindent}{0pt}
\setlength{\parskip}{6pt plus 2pt minus 1pt}
\setlength{\emergencystretch}{3em}  % prevent overfull lines
\setcounter{secnumdepth}{0}
 
/* start css.sty */
.cmr-5{font-size:50%;}
.cmr-7{font-size:70%;}
.cmmi-5{font-size:50%;font-style: italic;}
.cmmi-7{font-size:70%;font-style: italic;}
.cmmi-10{font-style: italic;}
.cmsy-5{font-size:50%;}
.cmsy-7{font-size:70%;}
.cmex-7{font-size:70%;}
.cmex-7x-x-71{font-size:49%;}
.msbm-7{font-size:70%;}
.cmtt-10{font-family: monospace;}
.cmti-10{ font-style: italic;}
.cmbx-10{ font-weight: bold;}
.cmr-17x-x-120{font-size:204%;}
.cmsl-10{font-style: oblique;}
.cmti-7x-x-71{font-size:49%; font-style: italic;}
.cmbxti-10{ font-weight: bold; font-style: italic;}
p.noindent { text-indent: 0em }
td p.noindent { text-indent: 0em; margin-top:0em; }
p.nopar { text-indent: 0em; }
p.indent{ text-indent: 1.5em }
@media print {div.crosslinks {visibility:hidden;}}
a img { border-top: 0; border-left: 0; border-right: 0; }
center { margin-top:1em; margin-bottom:1em; }
td center { margin-top:0em; margin-bottom:0em; }
.Canvas { position:relative; }
li p.indent { text-indent: 0em }
.enumerate1 {list-style-type:decimal;}
.enumerate2 {list-style-type:lower-alpha;}
.enumerate3 {list-style-type:lower-roman;}
.enumerate4 {list-style-type:upper-alpha;}
div.newtheorem { margin-bottom: 2em; margin-top: 2em;}
.obeylines-h,.obeylines-v {white-space: nowrap; }
div.obeylines-v p { margin-top:0; margin-bottom:0; }
.overline{ text-decoration:overline; }
.overline img{ border-top: 1px solid black; }
td.displaylines {text-align:center; white-space:nowrap;}
.centerline {text-align:center;}
.rightline {text-align:right;}
div.verbatim {font-family: monospace; white-space: nowrap; text-align:left; clear:both; }
.fbox {padding-left:3.0pt; padding-right:3.0pt; text-indent:0pt; border:solid black 0.4pt; }
div.fbox {display:table}
div.center div.fbox {text-align:center; clear:both; padding-left:3.0pt; padding-right:3.0pt; text-indent:0pt; border:solid black 0.4pt; }
div.minipage{width:100%;}
div.center, div.center div.center {text-align: center; margin-left:1em; margin-right:1em;}
div.center div {text-align: left;}
div.flushright, div.flushright div.flushright {text-align: right;}
div.flushright div {text-align: left;}
div.flushleft {text-align: left;}
.underline{ text-decoration:underline; }
.underline img{ border-bottom: 1px solid black; margin-bottom:1pt; }
.framebox-c, .framebox-l, .framebox-r { padding-left:3.0pt; padding-right:3.0pt; text-indent:0pt; border:solid black 0.4pt; }
.framebox-c {text-align:center;}
.framebox-l {text-align:left;}
.framebox-r {text-align:right;}
span.thank-mark{ vertical-align: super }
span.footnote-mark sup.textsuperscript, span.footnote-mark a sup.textsuperscript{ font-size:80%; }
div.tabular, div.center div.tabular {text-align: center; margin-top:0.5em; margin-bottom:0.5em; }
table.tabular td p{margin-top:0em;}
table.tabular {margin-left: auto; margin-right: auto;}
div.td00{ margin-left:0pt; margin-right:0pt; }
div.td01{ margin-left:0pt; margin-right:5pt; }
div.td10{ margin-left:5pt; margin-right:0pt; }
div.td11{ margin-left:5pt; margin-right:5pt; }
table[rules] {border-left:solid black 0.4pt; border-right:solid black 0.4pt; }
td.td00{ padding-left:0pt; padding-right:0pt; }
td.td01{ padding-left:0pt; padding-right:5pt; }
td.td10{ padding-left:5pt; padding-right:0pt; }
td.td11{ padding-left:5pt; padding-right:5pt; }
table[rules] {border-left:solid black 0.4pt; border-right:solid black 0.4pt; }
.hline hr, .cline hr{ height : 1px; margin:0px; }
.tabbing-right {text-align:right;}
span.TEX {letter-spacing: -0.125em; }
span.TEX span.E{ position:relative;top:0.5ex;left:-0.0417em;}
a span.TEX span.E {text-decoration: none; }
span.LATEX span.A{ position:relative; top:-0.5ex; left:-0.4em; font-size:85%;}
span.LATEX span.TEX{ position:relative; left: -0.4em; }
div.float img, div.float .caption {text-align:center;}
div.figure img, div.figure .caption {text-align:center;}
.marginpar {width:20%; float:right; text-align:left; margin-left:auto; margin-top:0.5em; font-size:85%; text-decoration:underline;}
.marginpar p{margin-top:0.4em; margin-bottom:0.4em;}
.equation td{text-align:center; vertical-align:middle; }
td.eq-no{ width:5%; }
table.equation { width:100%; } 
div.math-display, div.par-math-display{text-align:center;}
math .texttt { font-family: monospace; }
math .textit { font-style: italic; }
math .textsl { font-style: oblique; }
math .textsf { font-family: sans-serif; }
math .textbf { font-weight: bold; }
.partToc a, .partToc, .likepartToc a, .likepartToc {line-height: 200%; font-weight:bold; font-size:110%;}
.chapterToc a, .chapterToc, .likechapterToc a, .likechapterToc, .appendixToc a, .appendixToc {line-height: 200%; font-weight:bold;}
.index-item, .index-subitem, .index-subsubitem {display:block}
.caption td.id{font-weight: bold; white-space: nowrap; }
table.caption {text-align:center;}
h1.partHead{text-align: center}
p.bibitem { text-indent: -2em; margin-left: 2em; margin-top:0.6em; margin-bottom:0.6em; }
p.bibitem-p { text-indent: 0em; margin-left: 2em; margin-top:0.6em; margin-bottom:0.6em; }
.paragraphHead, .likeparagraphHead { margin-top:2em; font-weight: bold;}
.subparagraphHead, .likesubparagraphHead { font-weight: bold;}
.quote {margin-bottom:0.25em; margin-top:0.25em; margin-left:1em; margin-right:1em; text-align:\\jmathmathustify;}
.verse{white-space:nowrap; margin-left:2em}
div.maketitle {text-align:center;}
h2.titleHead{text-align:center;}
div.maketitle{ margin-bottom: 2em; }
div.author, div.date {text-align:center;}
div.thanks{text-align:left; margin-left:10%; font-size:85%; font-style:italic; }
div.author{white-space: nowrap;}
.quotation {margin-bottom:0.25em; margin-top:0.25em; margin-left:1em; }
h1.partHead{text-align: center}
.sectionToc, .likesectionToc {margin-left:2em;}
.subsectionToc, .likesubsectionToc {margin-left:4em;}
.subsubsectionToc, .likesubsubsectionToc {margin-left:6em;}
.frenchb-nbsp{font-size:75%;}
.frenchb-thinspace{font-size:75%;}
.figure img.graphics {margin-left:10%;}
/* end css.sty */

\title{Espaces affines euclidiens}
\author{}
\date{}

\begin{document}
\maketitle

\textbf{Warning: 
requires JavaScript to process the mathematics on this page.\\ If your
browser supports JavaScript, be sure it is enabled.}

\begin{center}\rule{3in}{0.4pt}\end{center}

{[}
{[}
{[}{]}
{[}

\subsubsection{17.3 Espaces affines euclidiens}

\paragraph{17.3.1 Notion d'espace affine euclidien}

Définition~17.3.1 On appelle espace affine euclidien un couple (E,\Phi)
d'un espace affine de dimension finie sur le corps \mathbb{R}~ des nombres réels
et d'une forme quadratique définie positive \Phi sur
\overrightarrowE.

Remarque~17.3.1 Comme d'habitude on notera
\Phi(\overrightarrow\xi)
=\\overrightarrow
\xi\^2 et on notera la forme polaire
de \Phi sous la forme
(\overrightarrow\xi,\overrightarrow\eta)\mapsto~(\overrightarrow\xi\mathrel∣\overrightarrow\eta)
(le produit scalaire associé).

Proposition~17.3.1 Soit E un espace affine euclidien. L'application d :
E \times E \rightarrow~ \mathbb{R}~,
(x,y)\mapsto~\\overrightarrowxy\
est une distance sur E (appelée la distance euclidienne).

Démonstration Vérification laissée au lecteur.

\paragraph{17.3.2 Formule de Leibnitz et applications}

Théorème~17.3.2 Soit E un espace affine euclidien,
(a_i)_i\inI une famille finie de points de E,
(\lambda_i)_i\inI une famille de scalaires telle que
\\sum ~
_i\inI\lambda_i\neq~0. Soit g le
barycentre de la famille de points massiques \left
((a_i,\lambda_i)\right )_i\inI. Alors

\forall~~m \in E, \\sum
_i\inI\lambda_i\\overrightarrowma_i\^2
= \left (\\sum
_i\inI\lambda_i\right )
\\overrightarrowmg\^2
+ \\sum
_i\inI\lambda_i\\overrightarrowga_i\^2

Démonstration On a

\begin{align*} \\sum
_i\inI\lambda_i\\overrightarrowma_i\^2
= \\sum
_i\inI\lambda_i\\overrightarrowmg
+\overrightarrow
ga_i\^2&& \%&
\\ & =& \left
(\\sum
_i\inI\lambda_i\right )
\\overrightarrowmg\^2
+ 2\\sum
_i\inI\lambda_i(\overrightarrowmg∣\overrightarrowga_i)
+ \\sum
_i\inI\lambda_i\\overrightarrowga_i\^2\%&
\\ & =& \left
(\\sum
_i\inI\lambda_i\right )
\\overrightarrowmg\^2
+ 2\left
(\overrightarrowmg∣\\sum
_i\inI\lambda_i\overrightarrowga_i\right
) + \\sum
_i\inI\lambda_i\\overrightarrowga_i\^2\%&
\\ & =& \left
(\\sum
_i\inI\lambda_i\right )
\\overrightarrowmg\^2
+ \\sum
_i\inI\lambda_i\\overrightarrowga_i\^2
\%& \\ \end{align*}

en utilisant l'identité de polarisation et la formule
\\sum ~
_i\inI\lambda_i\overrightarrowga_i
=\overrightarrow 0.

Corollaire~17.3.3 Soit E un espace affine euclidien,
(a_i)_i\inI une famille finie de points de E,
(\lambda_i)_i\inI une famille de scalaires telle que
\\sum ~
_i\inI\lambda_i\neq~0. Soit g le
barycentre de la famille de points massiques \left
((a_i,\lambda_i)\right )_i\inI. Soit k \in
\mathbb{R}~. Alors \m \in
E∣\\\sum

_i\inI\lambda_i\\overrightarrowma_i\^2
= k\ est soit l'ensemble vide, soit le singleton
\g\ soit une sphère de centre g.

Démonstration D'après la formule de Leibnitz, on a

\begin{align*} \\sum
_i\inI\lambda_i\\overrightarrowma_i\^2
= k& \Leftrightarrow & \left
(\\sum
_i\inI\lambda_i\right )
\\overrightarrowmg\^2
= k -\\sum
_i\inI\lambda_i\\overrightarrowga_i\^2\%&
\\ & \Leftrightarrow &
\\overrightarrowgm\^2
= 1 \over \\sum
_i\inI\lambda_i \left (k
-\\sum
_i\inI\lambda_i\\overrightarrowga_i\^2\right
) \%& \\ \end{align*}

Donc, suivant que  1 \over
\\sum ~
_i\inI\lambda_i \left (k
-\\sum ~
_i\inI\lambda_i\\overrightarrowga_i\^2\right
) est strictement négatif, nul ou strictement positif, on trouve \varnothing~,
\g\ ou la sphère de centre g de rayon
\sqrt 1 \over
\\sum ~
_i\inI\lambda_i \left (k
-\\sum ~
_i\inI\lambda_i\\overrightarrowga_i\^2\right
).

Remarque~17.3.2 Soit E un espace affine euclidien,
(a_i)_i\inI une famille finie de points de E,
(\lambda_i)_i\inI une famille de scalaires telle que
\\sum ~
_i\inI\lambda_i = 0. On sait qu'il existe un vecteur
\overrightarrowu tel que \forall~~m
\in E, \\sum ~
\lambda_i\overrightarrowma_i
=\overrightarrow u. Soit a \in E. On a alors

\begin{align*} \\sum
_i\inI\lambda_i\\overrightarrowma_i\^2
= \\sum
_i\inI\lambda_i\\overrightarrowma
+\overrightarrow
aa_i\^2&& \%&
\\ & =& \left
(\\sum
_i\inI\lambda_i\right )
\\overrightarrowma\^2
+ 2\\sum
_i\inI\lambda_i(\overrightarrowma∣\overrightarrowaa_i)
+ \\sum
_i\inI\lambda_i\\overrightarrowaa_i\^2\%&
\\ & =& 2\left
(\overrightarrowma∣\\sum
_i\inI\lambda_i\overrightarrowaa_i\right
) + \\sum
_i\inI\lambda_i\\overrightarrowaa_i\^2
\%& \\ & =&
2(\overrightarrowma∣\overrightarrowu)
+ \\sum
_i\inI\lambda_i\\overrightarrowaa_i\^2
\%& \\ \end{align*}

On en déduit que

\\sum
_i\inI\lambda_i\\overrightarrowma_i\^2
= k \Leftrightarrow
(\overrightarrowma∣\overrightarrowu)
= 1 \over 2 \left (k
-\\sum
_i\inI\lambda_i\\overrightarrowaa_i\^2\right
)

qui est soit l'ensemble vide (si \overrightarrowu
=\overrightarrow 0 et k
-\\sum ~
_i\inI\lambda_i\\overrightarrowaa_i\^2\neq~0),
soit un hyperplan orthogonal à \overrightarrowu
(prendre par exemple un repère orthonormé), soit encore l'espace tout
entier.

Corollaire~17.3.4 Soit k \textgreater{} 0, a,b \in E distincts. Alors
\m \in E∣d(m,a) =
kd(m,b)\ est

\begin{itemize}
\itemsep1pt\parskip0pt\parsep0pt
\item
  si k\neq~1, une sphère de centre g barycentre
  de (a,1) et de (b,-k^2)
\item
  si k = 1, l'hyperplan médiateur de a et b.
\end{itemize}

Démonstration On a en effet d(m,a) = kd(m,b)
\Leftrightarrow
\\overrightarrowma\^2
-
k^2\\overrightarrowmb\^2
= 0. Si k\neq~1, alors 1 -
k^2\neq~0 et l'ensemble qui ne peut
être ni l'ensemble vide, ni un singleton (car il y a deux solutions
évidentes sur la droite ab à savoir le barycentre de (a,1) et (b,k) et
le barycentre de (a,1) et (b,-k)) est une sphère de centre g. Si par
contre k = 1, on a (avec les notations de la remarque)
\overrightarrowu =\overrightarrow
ma -\overrightarrow mb
=\overrightarrow ba, si bien que l'ensemble qui n'est
pas l'ensemble vide (car le milieu de a et b convient) est un hyperplan
orthogonal à \overrightarrowba et contenant le milieu
de a et b, donc c'est l'hyperplan médiateur de a et b.

\paragraph{17.3.3 Isométries affines}

Définition~17.3.2 Soit E un espace affine euclidien et f : E \rightarrow~ E une
application affine. On dit que f est une isométrie affine si elle
vérifie les conditions équivalentes

\begin{itemize}
\itemsep1pt\parskip0pt\parsep0pt
\item
  (i) \forall~~x,y \in E, d(f(x),f(y)) = d(x,y)
\item
  (ii) \vecf est un endomorphisme orthogonal de
  \overrightarrowE.
\end{itemize}

Démonstration On a en effet d(f(x),f(y))
=\\overrightarrow
f(x)f(y)\
=\\vec
f(\overrightarrowxy)\ si
bien que la condition (i) est équivalente à
\forall~\overrightarrow\xi~
\in\overrightarrow E,
\\vecf(\overrightarrow\xi)\
=\\overrightarrow
\xi\ ce qui caractérise les endomorphismes
orthogonaux.

On peut définir une isométrie affine à l'aide de repères orthonormés par
le théorème suivant

Théorème~17.3.5 Soit
(a,\vece_1,\\ldots,\vece_n~)
et
(a',\vece_1',\\ldots,\vece_n~')
deux repères orthonormés de E~; alors il existe une unique isométrie
affine f de E vérifiant f(a) = a' et \forall~~i \in
{[}1,n{]}, \vecf(\vece_i)
=\vec e_i'.

Démonstration On sait qu'il existe une unique application affine
vérifiant f(a) = a' et \forall~~i \in {[}1,n{]},
\vecf(\vece_i)
=\vec e_i'~; comme \vecf
envoie une base orthonormée sur une base orthonormée, c'est un
endomorphisme orthogonal.

\paragraph{17.3.4 Forme réduite d'une isométrie affine}

Théorème~17.3.6 Soit f : E \rightarrow~ E une isométrie affine. Alors il existe un
unique couple (g,\overrightarrow\xi) d'une isométrie
affine ayant un point fixe et d'un vecteur
\overrightarrow\xi \in\overrightarrow
E vérifiant les propriétés équivalentes suivantes

\begin{itemize}
\itemsep1pt\parskip0pt\parsep0pt
\item
  (i) f = g \cdot t_\overrightarrow\xi =
  t_\overrightarrow\xi \cdot g
\item
  (ii) f = t_\overrightarrow\xi \cdot g et
  \overrightarrow\xi est parallèle à l'ensemble des
  points fixes de g
\item
  (iii) f = t_\overrightarrow\xi \cdot g et
  \vecf(\overrightarrow\xi)
  =\overrightarrow \xi
\end{itemize}

Démonstration Vérifions tout d'abord l'équivalence de (i), (ii) et
(iii). Tout d'abord soit a un point fixe de g. On a g(a) = a et donc
g(x) = x \Leftrightarrow
\vecg(\overrightarrowax)
=\overrightarrow ax ce qui montre que la direction
\overrightarrowF de l'ensemble F des points fixes de
g n'est autre que l'ensemble des vecteurs
\overrightarrow\xi tels que
\vecg(\overrightarrow\xi)
=\overrightarrow \xi. Mais d'autre part f =
t_\overrightarrow\xi \cdot g
\rigtharrow~\vec f =\vec g~; on a donc
immédiatement l'équivalence de (ii) et (iii). Il nous reste donc à
montrer l'équivalence de (i) et (iii). Mais g \cdot
t_\overrightarrow\xi(x) = g(x
+\overrightarrow \xi) = g(x) +\vec
g(\overrightarrow\xi) = g(x) +\vec
f(\overrightarrow\xi) et
t_\overrightarrow\xi \cdot g(x) = g(x)
+\overrightarrow \xi ce qui montre bien l'équivalence
de (i) et (iii).

Montrons donc l'existence et l'unicité d'un couple
(g,\overrightarrow\xi) d'une isométrie affine ayant un
point fixe et d'un vecteur \overrightarrow\xi
\in\overrightarrow E vérifiant (iii). On doit donc
rechercher un vecteur \overrightarrow\xi
\in\overrightarrow E tel que
\vecf(\overrightarrow\xi)
=\overrightarrow \xi et tel que g =
t_-\overrightarrow\xi \cdot f ait un point fixe.
Soit a \in E et cherchons à résoudre l'équation
t_-\overrightarrow\xi \cdot f(x) = x, soit encore
f(x) = x +\overrightarrow \xi, c'est-à-dire f(a)
+\vec f(\overrightarrowax) = a
+\overrightarrow ax +\overrightarrow
\xi, soit \vecf(\overrightarrowax)
-\overrightarrow ax =\overrightarrow
f(a)a +\overrightarrow \xi.

Soit \overrightarrowF =
\\overrightarrowu∣\vecf(\overrightarrowu)
=\overrightarrow u\. On a
\overrightarrowE =\overrightarrow
F \oplus~\overrightarrow F^\bot et comme
\overrightarrowF est stable par l'endomorphisme
orthogonal \vecf, il en est de même de
\overrightarrowF^\bot. Si on pose
\overrightarrowax =\overrightarrow
x_1 +\overrightarrow x_2,
\overrightarrowf(a)a
=\overrightarrow a_1
+\overrightarrow a_2,
\overrightarrow\xi =\overrightarrow
\xi +\overrightarrow 0 les décompositions des
différents vecteurs dans la somme directe
\overrightarrowE =\overrightarrow
F \oplus~\overrightarrow F^\bot, on a

\begin{align*}
\vecf(\overrightarrowax)
-\overrightarrow ax =\overrightarrow
f(a)a +\overrightarrow \xi&
\Leftrightarrow & \left
\\array
\vecf(\overrightarrowx_1)
-\overrightarrow x_1&
=\overrightarrow a_1
+\overrightarrow \xi \cr
\vecf(\overrightarrowx_2)
-\overrightarrow x_2&
=\overrightarrow a_2
+\overrightarrow 0  \right .\%&
\\ & \Leftrightarrow &
\left \\array
\overrightarrow0 & =\overrightarrow
a_1 +\overrightarrow \xi \cr
\vecf(\overrightarrowx_2)
-\overrightarrow x_2&
=\overrightarrow a_2 
\right .\%&\\
\end{align*}

puisque
\vecf(\overrightarrowx_1)
-\overrightarrow x_1
=\overrightarrow 0. Ceci montre dé\\jmathmathà que
\overrightarrow\xi =
-\overrightarrowa_1 d'où l'unicité de
\overrightarrow\xi et donc de g.

Il nous suffit donc de montrer que la deuxième équation a une solution.
Mais \vecf -\mathrmId laisse
stable \overrightarrowF^\bot et définit un
endomorphisme in\\jmathmathectif de cet espace car si
\overrightarrowu \in\overrightarrow
F^\bot, on a

f(\overrightarrowu) -\overrightarrow
u =\overrightarrow 0
\rigtharrow~\overrightarrow u \in\overrightarrow
F \bigcap\overrightarrow F^\bot =
\\overrightarrow0\

Comme cet espace est de dimension finie, cet endomorphisme de
\overrightarrowF^\bot est aussi sur\\jmathmathectif, et
donc la deuxième équation a bien une solution.

Remarque~17.3.3 Puisque g a un point fixe, en vectorialisant l'espace en
un tel point, l'isométrie affine g s'identifie à l'endomorphisme
orthogonal \vecg =\vec f. Une
isométrie affine s'identifie au produit commutatif d'une isométrie
vectorielle (que l'on connait dé\\jmathmathà) et d'une translation parallèlement à
l'ensemble des points fixes de cette isométrie vectorielle. Ceci va nous
permettre de décrire complètement les isométries affines en dimension 2
ou 3 en distinguant suivant que f est un déplacement (c'est-à-dire que
\vecf est une rotation) ou un antidéplacement
(c'est-à-dire que \vecf \in
O^-(\overrightarrowE)).

Théorème~17.3.7 Soit E un espace affine euclidien de dimension 2. Alors

\begin{itemize}
\itemsep1pt\parskip0pt\parsep0pt
\item
  (i) les déplacements de E sont d'une part les translations et d'autre
  part les rotations ayant pour centre un point de E et pour angle un
  élément non nul de \mathbb{R}~\diagup2\pi~\mathbb{Z}.
\item
  (ii) les antidéplacements de E sont les produits (commutatifs) d'une
  symétrie orthogonale par rapport à une droite et d'une translation
  parallèlement à cette droite.
\end{itemize}

Théorème~17.3.8 Soit E un espace affine euclidien de dimension 3. Alors

\begin{itemize}
\itemsep1pt\parskip0pt\parsep0pt
\item
  (i) les déplacements de E sont d'une part les translations et d'autre
  part les vissages~: produits commutatifs d'une rotation autour d'un
  axe D et d'une translation parallèlement à D
\item
  (ii) les antidéplacements de E sont d'une part les produits
  (commutatifs) d'une symétrie orthogonale par rapport à un plan et
  d'une translation parallèlement à ce plan et d'autre part les produits
  commutatifs d'une rotation autour d'un axe D et d'une symétrie
  orthogonale par rapport à un plan orthogonal à D.
\end{itemize}

\paragraph{17.3.5 Distance à un sous-espace affine}

Théorème~17.3.9 Soit F un sous-espace affine de direction
\overrightarrowF et x \in E. Il existe un unique point
p \in F tel que d(x,F) = d(x,p). Le point p est l'unique point
d'intersection de F et de x +\overrightarrow
F^\bot~; on l'appelle la pro\\jmathmathection orthogonale de x sur F.
Soit a \in F et
(\vece_1,\\ldots,\vece_k~)
une base de \overrightarrowF. On a

d(x,F)^2 =
\mathrm{det}~
Gram(\overrightarrowax,\vece_1,\\\ldots,\vece_k~)
\over
\mathrm{det}~
Gram(\vece_1,\\\ldots,\vece_k)~

Démonstration Comme \overrightarrowF et
\overrightarrowF^\bot sont supplémentaires,
un résultat précédent montre que F \bigcap\left (x
+\overrightarrow F^\bot\right
) est un singleton \p\. On a donc p \in
F et \overrightarrowxp
\in\overrightarrow F^\bot Soit alors y \in F. On
a

d(x,y)^2
=\\overrightarrow
xy\^2
=\\overrightarrow xp
+\overrightarrow
py\^2
=\\overrightarrow
xp\^2
+\\overrightarrow
py\^2

car \overrightarrowxp
\in\overrightarrow F^\bot et
\overrightarrowpy \in F. On en déduit que d(x,y) ≥
d(x,p) avec égalité si et seulement
si~\\overrightarrowpy\
= 0 soit p = y, ce qui démontre que p est l'unique point de F tel que
d(x,F) = d(x,p).

De plus, si a \in F, soit y \in F. On a d(x,y)
=\\overrightarrow
xy\
=\\overrightarrow ax
-\overrightarrow ay\. Mais
quand y décrit F, \overrightarroway décrit
\overrightarrowF, si bien que d(x,F) =
d(\overrightarrowax,\overrightarrowF).
Mais on a vu dans le chapitre sur les formes quadratiques que
d(\overrightarrow\xi,\overrightarrowF)^2
= \mathrm{det}~
\
Gram(\overrightarrow\xi,\vece_1,\\ldots,\vece_k~)
\over
\mathrm{det}~
\
Gram(\vece_1,\\ldots,\vece_k)~
, ce qui donne la formule voulue.

En dimension 3, en interprétant le déterminant de Gram de trois vecteurs
comme le carré du produit mixte et le déterminant de Gram de deux
vecteurs comme le carré de la norme de leur produit vectoriel, on
obtient les formules suivantes pour la distance à une droite ou à un
plan

Corollaire~17.3.10 Soit E un espace euclidien de dimension 3, a,x \in E,
\overrightarrowu et
\overrightarrowv deux vecteurs non colinéaires de
\overrightarrowE. Alors

d(x,a + \mathbb{R}~\overrightarrowu) =
\\overrightarrowax
∧\overrightarrow u\
\over
\\overrightarrowu\

d(x,a + \mathbb{R}~\overrightarrowu +
\mathbb{R}~\overrightarrowv) = \Big
{[}\overrightarrowax,\overrightarrowu,\overrightarrowv{]}\Big
 \over
\\overrightarrowu
∧\overrightarrow v\

En ce qui concerne les hyperplans, on a le résultat suivant

Théorème~17.3.11 Soit
(a,\vece_1,\\ldots,\vece_n~)
un repère orthonormé de E, H un hyperplan affine de E d'équation
u_1x_1 +
\\ldots~ +
u_nx_n + h = 0. Alors le vecteur
\overrightarrown =
u_1\vece_1 +
\\ldots~ +
u_n\vece_n est un vecteur normal au
plan et pour tout x \in E de coordonnées
x_1,\\ldots,x_n~,
on a

d(x,H) = u_1x_1 +
\\ldots~ +
u_nx_n + h \over
\sqrtu_1 ^2  +
\\ldots~ +
u_n ^2

Démonstration La direction \overrightarrowH de H
admet pour équation u_1x_1 +
\\ldots~ +
u_nx_n = 0, c'est donc visiblement
\overrightarrown^\bot ce qui montre que
\overrightarrown =
u_1\vece_1 +
\\ldots~ +
u_n\vece_n est un vecteur normal au
plan. Soit f la forme affine sur E définie par f(x) =
u_1x_1 +
\\ldots~ +
u_nx_n + h. On a donc
\vecf(\overrightarrow\xi) =
u_1x_1 +
\\ldots~ +
u_nx_n =
(\overrightarrow\xi∣\overrightarrown).
Soit x \in E et p sa pro\\jmathmathection orthogonale sur H. On a donc f(p) = 0,
soit f(x) = f(x) - f(p) =\vec
f(\overrightarrowpx) =
(\overrightarrowpx∣\overrightarrown).
Mais, les deux vecteurs \overrightarrowpx et
\overrightarrown qui sont tous deux orthogonaux à
\overrightarrowH sont colinéaires, si bien que

f(x) =
(\overrightarrowpx∣\overrightarrown)
=\\overrightarrow
px\
\\overrightarrown\
=
d(x,H)\\overrightarrown\

On en déduit que d(x,H) = f(x)
\over
\\overrightarrown\
, ce qui n'est autre que la formule cherchée.

Corollaire~17.3.12 Soit H_1 et H_2 deux hyperplans non
parallèles de E. Alors l'ensemble des points x de E tels que
d(x,H_1) = d(x,H_2) est la réunion de deux hyperplans
orthogonaux, appelés les deux hyperplans bissecteurs de H_1 et
H_2.

Démonstration Sans nuire à la généralité on peut supposer que
H_1 est d'équation u_1x_1 +
\\ldots~ +
u_nx_n + h = 0 et H_2 d'équation
v_1x_1 +
\\ldots~ +
v_nx_n + k = 0 avec u_1^2 +
\\ldots~ +
u_n^2 = v_1^2 +
\\ldots~ +
v_n^2 = 1. Alors les vecteurs normaux
\overrightarrown_1 et
\overrightarrown_2 à ces deux hyperplans
sont unitaires et

\begin{align*} d(x,H_1) =
d(x,H_2)&& \%& \\ &
\Leftrightarrow & u_1x_1 +
\\ldots~ +
u_nx_n + h =
v_1x_1 +
\\ldots~ +
v_nx_n + k\%&
\\ & \Leftrightarrow &
(u_1 + \epsilonv_1)x_1 +
\\ldots~ +
(u_n + \epsilonv_n)x_n + h + \epsilonk = 0 \%&
\\ \end{align*}

avec \epsilon = ±1. Il s'agit visiblement de deux équations d'hyperplans de
vecteurs normaux \overrightarrown_1
+\overrightarrow n_2 et
\overrightarrown_1
-\overrightarrow n_2. Or ces deux vecteurs
normaux sont orthogonaux puisque
(\overrightarrown_1
+\overrightarrow
n_2∣\overrightarrown_1
-\overrightarrow n_2)
=\\overrightarrow
n_1\^2
-\\overrightarrow
n_2\^2 = 0. Ceci achève la
démonstration.

\paragraph{17.3.6 Distance de deux sous-espaces affines}

Théorème~17.3.13 Soit E un espace affine euclidien, F et G deux
sous-espaces affines de E. Alors il existe a \in F et b \in G tel que le
vecteur \overrightarrowab soit orthogonal à la fois à
F et à G. Ces points sont uniques si \overrightarrowF
\bigcap\overrightarrow G =
\\overrightarrow0\.
On a d(F,G) = d(a,b).

Démonstration Soit H = G +\overrightarrow F le
sous-espace affine contenant G et auquel F est faiblement parallèle . On
doit avoir b \in H et \overrightarrowab \bot H ce qui
montre que b est nécessairement la pro\\jmathmathection orthogonale de a sur H.
Soit donc F' la pro\\jmathmathection orthogonale de F sur H. Comme F est
faiblement parallèle à H, F' est parallèle à F. Donc
\overrightarrowF' +\overrightarrow
G =\overrightarrow F
+\overrightarrow G =\overrightarrow
H~; le théorème d'intersection des sous-espaces affines appliqués aux
sous-espaces affines F' et G de H, montre que F' \bigcap
G\neq~\varnothing~. Soit donc b \in F' \bigcap G et a \in F dont la
pro\\jmathmathection orthogonale sur H est b. On a a \in F, b \in G et
\overrightarrowab \bot H, donc
\overrightarrowab est orthogonal à la fois à F et à
G.

Si x \in F et y \in G, on a

\\overrightarrowxy\^2
=\\overrightarrow ab +
(\overrightarrowxa
+\overrightarrow
by)\^2
=\\overrightarrow
ab\^2
+\\overrightarrow xa
+\overrightarrow
by\^2

d'après le théorème de Pythagore, puisque
\overrightarrowab \bot\overrightarrow
H et \overrightarrowxa
+\overrightarrow by \in\overrightarrow
F +\overrightarrow G
=\overrightarrow H. On en déduit que
\\overrightarrowxy\^2
≥\\overrightarrow
ab\^2, soit encore d(x,y) ≥ d(a,b).
Ceci montre que d(F,G) = d(a,b). De plus l'égalité nécessite que
\overrightarrowxa +\overrightarrow
by =\overrightarrow 0. Si
\overrightarrowF et
\overrightarrowG sont en somme directe, ceci ne peut
se produire que si x = a et y = b, ce qui assure dans ce cas l'unicité
de a et b.

Définition~17.3.3 Soit E un espace affine euclidien, F et G deux
sous-espaces affines de E. On appelle perpendiculaire commune à F et à G
toute droite \\jmathmathoignant un point de F à un point de G et orthogonale à ces
deux sous-espaces affines.

Remarque~17.3.4 D'après le théorème ci dessus, une telle droite est
unique si \overrightarrowF
\bigcap\overrightarrow G =
\\overrightarrow0\
et si a et b sont distincts, soit F \bigcap G = \varnothing~. En particulier, en
dimension 3, deux droites non parallèles et non sécantes ont une unique
perpendiculaire commune (le résultat subsistant d'ailleurs évidemment
pour deux droites sécantes non confondues)~: si ces deux droites
D_1 et D_2 ont pour vecteurs directeurs
\overrightarrowu_1 et
\overrightarrowu_2, cette perpendiculaire
commune est l'intersection du plan contenant D_1 et parallèle à
\overrightarrowu_1
∧\overrightarrow u_2 et du plan contenant
D_2 et parallèle à
\overrightarrowu_1
∧\overrightarrow u_2.

Pour calculer la distance de F à G à savoir d(a,b), on peut également
remarquer que c'est la distance de a à H = G
+\overrightarrow F (puisque b est la pro\\jmathmathection
orthogonale de a sur H). Mais comme F est faiblement parallèle à H, on a
\forall~~x \in F, d(x,H) = d(a,H). On en déduit que la
distance de F à G est la distance de n'importe quel point x de F à H = G
+\overrightarrow F, que l'on sait calculer à l'aide
de déterminants de Gram moyennant la connaissance d'une base de
\overrightarrowF +\overrightarrow
G. En particulier, si E est de dimension 3, on a

Proposition~17.3.14 Soit E un espace euclidien de dimension 3,
D_1 = a_1 +
\mathbb{R}~\overrightarrowu_1 et D_2 =
a_2 + \mathbb{R}~\overrightarrowu_2 deux
droites non parallèles. Alors

d(D_1,D_2) = \Big
{[}\overrightarrowa_1a_2,\overrightarrowu_1,\overrightarrowu_2{]}\Big
 \over
\\overrightarrowu_1
∧\overrightarrow
u_2\

Démonstration On a d(D_1,D_2) =
d(a_1,a_2 +
\mathbb{R}~\overrightarrowu_1 +
\mathbb{R}~\overrightarrowu_2) et il suffit
d'appliquer la formule donnant la distance d'un point à un plan.

{[}
{[}
{[}
{[}

\end{document}
