\documentclass[]{article}
\usepackage[T1]{fontenc}
\usepackage{lmodern}
\usepackage{amssymb,amsmath}
\usepackage{ifxetex,ifluatex}
\usepackage{fixltx2e} % provides \textsubscript
% use upquote if available, for straight quotes in verbatim environments
\IfFileExists{upquote.sty}{\usepackage{upquote}}{}
\ifnum 0\ifxetex 1\fi\ifluatex 1\fi=0 % if pdftex
  \usepackage[utf8]{inputenc}
\else % if luatex or xelatex
  \ifxetex
    \usepackage{mathspec}
    \usepackage{xltxtra,xunicode}
  \else
    \usepackage{fontspec}
  \fi
  \defaultfontfeatures{Mapping=tex-text,Scale=MatchLowercase}
  \newcommand{\euro}{€}
\fi
% use microtype if available
\IfFileExists{microtype.sty}{\usepackage{microtype}}{}
\ifxetex
  \usepackage[setpagesize=false, % page size defined by xetex
              unicode=false, % unicode breaks when used with xetex
              xetex]{hyperref}
\else
  \usepackage[unicode=true]{hyperref}
\fi
\hypersetup{breaklinks=true,
            bookmarks=true,
            pdfauthor={},
            pdftitle={Integrale des fonctions reglees sur un segment},
            colorlinks=true,
            citecolor=blue,
            urlcolor=blue,
            linkcolor=magenta,
            pdfborder={0 0 0}}
\urlstyle{same}  % don't use monospace font for urls
\setlength{\parindent}{0pt}
\setlength{\parskip}{6pt plus 2pt minus 1pt}
\setlength{\emergencystretch}{3em}  % prevent overfull lines
\setcounter{secnumdepth}{0}
 
/* start css.sty */
.cmr-5{font-size:50%;}
.cmr-7{font-size:70%;}
.cmmi-5{font-size:50%;font-style: italic;}
.cmmi-7{font-size:70%;font-style: italic;}
.cmmi-10{font-style: italic;}
.cmsy-5{font-size:50%;}
.cmsy-7{font-size:70%;}
.cmex-7{font-size:70%;}
.cmex-7x-x-71{font-size:49%;}
.msbm-7{font-size:70%;}
.cmtt-10{font-family: monospace;}
.cmti-10{ font-style: italic;}
.cmbx-10{ font-weight: bold;}
.cmr-17x-x-120{font-size:204%;}
.cmsl-10{font-style: oblique;}
.cmti-7x-x-71{font-size:49%; font-style: italic;}
.cmbxti-10{ font-weight: bold; font-style: italic;}
p.noindent { text-indent: 0em }
td p.noindent { text-indent: 0em; margin-top:0em; }
p.nopar { text-indent: 0em; }
p.indent{ text-indent: 1.5em }
@media print {div.crosslinks {visibility:hidden;}}
a img { border-top: 0; border-left: 0; border-right: 0; }
center { margin-top:1em; margin-bottom:1em; }
td center { margin-top:0em; margin-bottom:0em; }
.Canvas { position:relative; }
li p.indent { text-indent: 0em }
.enumerate1 {list-style-type:decimal;}
.enumerate2 {list-style-type:lower-alpha;}
.enumerate3 {list-style-type:lower-roman;}
.enumerate4 {list-style-type:upper-alpha;}
div.newtheorem { margin-bottom: 2em; margin-top: 2em;}
.obeylines-h,.obeylines-v {white-space: nowrap; }
div.obeylines-v p { margin-top:0; margin-bottom:0; }
.overline{ text-decoration:overline; }
.overline img{ border-top: 1px solid black; }
td.displaylines {text-align:center; white-space:nowrap;}
.centerline {text-align:center;}
.rightline {text-align:right;}
div.verbatim {font-family: monospace; white-space: nowrap; text-align:left; clear:both; }
.fbox {padding-left:3.0pt; padding-right:3.0pt; text-indent:0pt; border:solid black 0.4pt; }
div.fbox {display:table}
div.center div.fbox {text-align:center; clear:both; padding-left:3.0pt; padding-right:3.0pt; text-indent:0pt; border:solid black 0.4pt; }
div.minipage{width:100%;}
div.center, div.center div.center {text-align: center; margin-left:1em; margin-right:1em;}
div.center div {text-align: left;}
div.flushright, div.flushright div.flushright {text-align: right;}
div.flushright div {text-align: left;}
div.flushleft {text-align: left;}
.underline{ text-decoration:underline; }
.underline img{ border-bottom: 1px solid black; margin-bottom:1pt; }
.framebox-c, .framebox-l, .framebox-r { padding-left:3.0pt; padding-right:3.0pt; text-indent:0pt; border:solid black 0.4pt; }
.framebox-c {text-align:center;}
.framebox-l {text-align:left;}
.framebox-r {text-align:right;}
span.thank-mark{ vertical-align: super }
span.footnote-mark sup.textsuperscript, span.footnote-mark a sup.textsuperscript{ font-size:80%; }
div.tabular, div.center div.tabular {text-align: center; margin-top:0.5em; margin-bottom:0.5em; }
table.tabular td p{margin-top:0em;}
table.tabular {margin-left: auto; margin-right: auto;}
div.td00{ margin-left:0pt; margin-right:0pt; }
div.td01{ margin-left:0pt; margin-right:5pt; }
div.td10{ margin-left:5pt; margin-right:0pt; }
div.td11{ margin-left:5pt; margin-right:5pt; }
table[rules] {border-left:solid black 0.4pt; border-right:solid black 0.4pt; }
td.td00{ padding-left:0pt; padding-right:0pt; }
td.td01{ padding-left:0pt; padding-right:5pt; }
td.td10{ padding-left:5pt; padding-right:0pt; }
td.td11{ padding-left:5pt; padding-right:5pt; }
table[rules] {border-left:solid black 0.4pt; border-right:solid black 0.4pt; }
.hline hr, .cline hr{ height : 1px; margin:0px; }
.tabbing-right {text-align:right;}
span.TEX {letter-spacing: -0.125em; }
span.TEX span.E{ position:relative;top:0.5ex;left:-0.0417em;}
a span.TEX span.E {text-decoration: none; }
span.LATEX span.A{ position:relative; top:-0.5ex; left:-0.4em; font-size:85%;}
span.LATEX span.TEX{ position:relative; left: -0.4em; }
div.float img, div.float .caption {text-align:center;}
div.figure img, div.figure .caption {text-align:center;}
.marginpar {width:20%; float:right; text-align:left; margin-left:auto; margin-top:0.5em; font-size:85%; text-decoration:underline;}
.marginpar p{margin-top:0.4em; margin-bottom:0.4em;}
.equation td{text-align:center; vertical-align:middle; }
td.eq-no{ width:5%; }
table.equation { width:100%; } 
div.math-display, div.par-math-display{text-align:center;}
math .texttt { font-family: monospace; }
math .textit { font-style: italic; }
math .textsl { font-style: oblique; }
math .textsf { font-family: sans-serif; }
math .textbf { font-weight: bold; }
.partToc a, .partToc, .likepartToc a, .likepartToc {line-height: 200%; font-weight:bold; font-size:110%;}
.chapterToc a, .chapterToc, .likechapterToc a, .likechapterToc, .appendixToc a, .appendixToc {line-height: 200%; font-weight:bold;}
.index-item, .index-subitem, .index-subsubitem {display:block}
.caption td.id{font-weight: bold; white-space: nowrap; }
table.caption {text-align:center;}
h1.partHead{text-align: center}
p.bibitem { text-indent: -2em; margin-left: 2em; margin-top:0.6em; margin-bottom:0.6em; }
p.bibitem-p { text-indent: 0em; margin-left: 2em; margin-top:0.6em; margin-bottom:0.6em; }
.paragraphHead, .likeparagraphHead { margin-top:2em; font-weight: bold;}
.subparagraphHead, .likesubparagraphHead { font-weight: bold;}
.quote {margin-bottom:0.25em; margin-top:0.25em; margin-left:1em; margin-right:1em; text-align:justify;}
.verse{white-space:nowrap; margin-left:2em}
div.maketitle {text-align:center;}
h2.titleHead{text-align:center;}
div.maketitle{ margin-bottom: 2em; }
div.author, div.date {text-align:center;}
div.thanks{text-align:left; margin-left:10%; font-size:85%; font-style:italic; }
div.author{white-space: nowrap;}
.quotation {margin-bottom:0.25em; margin-top:0.25em; margin-left:1em; }
h1.partHead{text-align: center}
.sectionToc, .likesectionToc {margin-left:2em;}
.subsectionToc, .likesubsectionToc {margin-left:4em;}
.subsubsectionToc, .likesubsubsectionToc {margin-left:6em;}
.frenchb-nbsp{font-size:75%;}
.frenchb-thinspace{font-size:75%;}
.figure img.graphics {margin-left:10%;}
/* end css.sty */

\title{Integrale des fonctions reglees sur un segment}
\author{}
\date{}

\begin{document}
\maketitle

\textbf{Warning: \href{http://www.math.union.edu/locate/jsMath}{jsMath}
requires JavaScript to process the mathematics on this page.\\ If your
browser supports JavaScript, be sure it is enabled.}

\begin{center}\rule{3in}{0.4pt}\end{center}

{[}\href{coursse52.html}{next}{]} {[}\href{coursse50.html}{prev}{]}
{[}\href{coursse50.html\#tailcoursse50.html}{prev-tail}{]}
{[}\hyperref[tailcoursse51.html]{tail}{]}
{[}\href{coursch10.html\#coursse51.html}{up}{]}

\subsubsection{9.2 Intégrale des fonctions réglées sur un segment}

\paragraph{9.2.1 Intégrale des applications en escalier}

Théorème~9.2.1 Soit f : {[}a,b{]} → E en escalier et σ =
\{(\{a\}\_\{i\})\}\_\{0≤i≤n\} une subdivision adaptée à f~; alors la
somme \{\textbackslash{}mathop\{\textbackslash{}mathop\{∑ \}\}
\}\_\{i=1\}\^{}\{n\}(\{a\}\_\{i\} − \{a\}\_\{i−1\})\{f\}\_\{i\} (où l'on
désigne par \{f\}\_\{i\} la constante telle que
\textbackslash{}mathop\{∀\}t ∈{]}\{a\}\_\{i−1\},\{a\}\_\{i\}{[}, f(t) =
\{f\}\_\{i\}) est indépendante du choix de σ~; on la note
\{\textbackslash{}mathop\{∫ \} \}\_\{a\}\^{}\{b\}f et on l'appelle
l'intégrale sur {[}a,b{]} de la fonction en escalier f.

Démonstration Soit σ' une subdivision adaptée à f telle que
\textbackslash{}mathop\{\textbackslash{}mathrm\{Pt\}\}(σ')
=\textbackslash{}mathop\{ \textbackslash{}mathrm\{Pt\}\}(σ)
∪\textbackslash{}\{c\textbackslash{}\}. Alors, si c
∈{]}\{a\}\_\{k−1\},\{a\}\_\{k\}{]}, la somme relative à σ' est

\textbackslash{}begin\{eqnarray*\} \{\textbackslash{}mathop\{∑
\}\}\_\{i=1\}\^{}\{k−1\}(\{a\}\_\{ i\} − \{a\}\_\{i−1\})\{f\}\_\{i\} +
(c − \{a\}\_\{k−1\})\{f\}\_\{k\} + (\{a\}\_\{k\} − c)\{f\}\_\{k\} +\{
\textbackslash{}mathop\{∑ \}\}\_\{i=k+1\}\^{}\{n\}(\{a\}\_\{ i\} −
\{a\}\_\{i−1\})\&\&\%\& \textbackslash{}\textbackslash{} \& =\&
\{\textbackslash{}mathop\{∑ \}\}\_\{i=1\}\^{}\{k−1\}(\{a\}\_\{ i\} −
\{a\}\_\{i−1\})\{f\}\_\{i\} + (\{a\}\_\{k\} −
\{a\}\_\{k−1\})\{f\}\_\{i\} +\{ \textbackslash{}mathop\{∑
\}\}\_\{i=k+1\}\^{}\{n\}(\{a\}\_\{ i\} − \{a\}\_\{i−1\})\%\&
\textbackslash{}\textbackslash{} \& =\& \{\textbackslash{}mathop\{∑
\}\}\_\{i=1\}\^{}\{n\}(\{a\}\_\{ i\} − \{a\}\_\{i−1\})\{f\}\_\{i\} \%\&
\textbackslash{}\textbackslash{} \textbackslash{}end\{eqnarray*\}

ce qui est encore la somme relative à σ~; une récurrence évidente montre
que si σ' est plus fine que σ (autrement dit si on a ajouté un nombre
fini de points à σ), la somme relative à σ' est égale à celle relative à
σ. Maintenant si σ et σ' sont deux subdivisions adaptées à f, la
subdivision σ ∪ σ' est encore adaptée à f et elle est plus fine que σ et
que σ'~; la somme relative à σ' est donc égale à celle relative à σ ∪ σ'
et donc à celle relative à σ.

Les propriétés de l'intégrale des fonctions en escalier sont tout à fait
élémentaires à partir de la définition. On obtient

Théorème~9.2.2 (i) L'application
f\textbackslash{}mathrel\{↦\}\{\textbackslash{}mathop\{∫ \}
\}\_\{a\}\^{}\{b\}f est linéaire de l'espace vectoriel ~des applications
en escalier de {[}a,b{]} dans E dans l'espace vectoriel ~E. (ii) Soit u
: E → F linéaire et f : {[}a,b{]} → E en escalier~; alors u ∘ f :
{[}a,b{]} → F est en escalier et \{\textbackslash{}mathop\{∫ \}
\}\_\{a\}\^{}\{b\}u ∘ f = u\textbackslash{}left
(\{\textbackslash{}mathop\{∫ \} \}\_\{a\}\^{}\{b\}f\textbackslash{}right
) (iii) Soit f : {[}a,b{]} → E en escalier~; alors
\textbackslash{}\textbar{}f\textbackslash{}\textbar{} : {[}a,b{]} → ℝ,
t\textbackslash{}mathrel\{↦\}\textbackslash{}\textbar{}f(t)\textbackslash{}\textbar{}
est en escalier et \textbackslash{}\textbar{}\{\textbackslash{}mathop\{∫
\} \}\_\{a\}\^{}\{b\}f\textbackslash{}\textbar{}
≤\{\textbackslash{}mathop\{∫ \}
\}\_\{a\}\^{}\{b\}\textbackslash{}\textbar{}f\textbackslash{}\textbar{}
(iv) Soit f : {[}a,b{]} → E en escalier, alors
\textbackslash{}\textbar{}\{\textbackslash{}mathop\{∫ \}
\}\_\{a\}\^{}\{b\}f\textbackslash{}\textbar{} ≤ (b −
a)\textbackslash{}\textbar{}f\textbackslash{}\textbar{}∞. (v) Si c
∈{]}a,b{[} et si f : {[}a,b{]} → E est en escalier, alors
\{f\}\_\{\{\textbar{}\}\_\{{[}a,c{]}\}\} et
\{f\}\_\{\{\textbar{}\}\_\{{[}c,b{]}\}\} sont en escalier et
\{\textbackslash{}mathop\{∫ \} \}\_\{a\}\^{}\{b\}f
=\{\textbackslash{}mathop\{∫ \} \}\_\{a\}\^{}\{c\}f
+\{\textbackslash{}mathop\{∫ \} \}\_\{c\}\^{}\{b\}f.

Démonstration En prenant une subdivision adaptée à la fois à f et à g,
on a \{\textbackslash{}mathop\{∫ \} \}\_\{a\}\^{}\{b\}(αf + βg)
=\{\textbackslash{}mathop\{ \textbackslash{}mathop\{∑ \}\}
\}\_\{i=1\}\^{}\{n\}(\{a\}\_\{i\} − \{a\}\_\{i−1\})(α\{f\}\_\{i\} +
β\{g\}\_\{i\}) = α\{\textbackslash{}mathop\{∫ \} \}\_\{a\}\^{}\{b\}f +
β\{\textbackslash{}mathop\{∫ \} \}\_\{a\}\^{}\{b\}g. Si σ est adaptée à
f elle est aussi adaptée à u ∘ f et à
\textbackslash{}\textbar{}f\textbackslash{}\textbar{} et on a
\{\textbackslash{}mathop\{∫ \} \}\_\{a\}\^{}\{b\}u ∘ f
=\{\textbackslash{}mathop\{ \textbackslash{}mathop\{∑ \}\}
\}\_\{i=1\}\^{}\{n\}(\{a\}\_\{i\} − \{a\}\_\{i−1\})u(\{f\}\_\{i\}) =
u(\{\textbackslash{}mathop\{∫ \} \}\_\{a\}\^{}\{b\}f) et
\textbackslash{}\textbar{}\{\textbackslash{}mathop\{∫ \}
\}\_\{a\}\^{}\{b\}f\textbackslash{}\textbar{}
=\textbackslash{}\textbar{}\{\textbackslash{}mathop\{
\textbackslash{}mathop\{∑ \}\} \}\_\{i=1\}\^{}\{n\}(\{a\}\_\{i\} −
\{a\}\_\{i−1\})\{f\}\_\{i\}\textbackslash{}\textbar{}
≤\{\textbackslash{}mathop\{\textbackslash{}mathop\{∑ \}\}
\}\_\{i=1\}\^{}\{n\}(\{a\}\_\{i\} −
\{a\}\_\{i−1\})\textbackslash{}\textbar{}\{f\}\_\{i\}\textbackslash{}\textbar{}
=\{\textbackslash{}mathop\{∫ \}
\}\_\{a\}\^{}\{b\}\textbackslash{}\textbar{}f\textbackslash{}\textbar{}.
Pour montrer (iv) on écrit

\textbackslash{}begin\{eqnarray*\}
\textbackslash{}\textbar{}\{\textbackslash{}mathop\{∫ \}
\}\_\{a\}\^{}\{b\}f\textbackslash{}\textbar{}\& =\&
\textbackslash{}\textbar{}\{\textbackslash{}mathop\{∑
\}\}\_\{i=1\}\^{}\{n\}(\{a\}\_\{ i\} −
\{a\}\_\{i−1\})\{f\}\_\{i\}\textbackslash{}\textbar{}
≤\{\textbackslash{}mathop\{∑ \}\}\_\{i=1\}\^{}\{n\}(\{a\}\_\{ i\} −
\{a\}\_\{i−1\})\textbackslash{}\textbar{}\{f\}\_\{i\}\textbackslash{}\textbar{}\%\&
\textbackslash{}\textbackslash{} \& ≤\&
\textbackslash{}\textbar{}f\textbackslash{}\textbar{}∞\{\textbackslash{}mathop\{∑
\}\}\_\{i=1\}\^{}\{n\}(\{a\}\_\{ i\} − \{a\}\_\{i−1\}) = (b −
a)\textbackslash{}\textbar{}f\textbackslash{}\textbar{}∞ \%\&
\textbackslash{}\textbackslash{} \textbackslash{}end\{eqnarray*\}

En ce qui concerne (v), il suffit d'introduire une subdivision σ adaptée
à f, de lui ajouter éventuellement le point c pour obtenir encore une
subdivision adaptée à f~; on coupe alors la somme en deux au point c.

\paragraph{9.2.2 Intégrale des fonctions réglées}

On suppose désormais que E est un espace vectoriel normé~complet

Théorème~9.2.3 Soit f : {[}a,b{]} → ℝ une fonction réglée et
\{(\{φ\}\_\{n\})\}\_\{n∈ℕ\} une suite de fonctions en escalier telles
que \textbackslash{}mathop\{lim\}\textbackslash{}\textbar{}f −
\{φ\}\_\{n\}\textbackslash{}\textbar{}∞ = 0. Alors la suite
\{\textbackslash{}left (\{\textbackslash{}mathop\{∫ \}
\}\_\{a\}\^{}\{b\}\{φ\}\_\{n\}\textbackslash{}right )\}\_\{n∈ℕ\}
converge~; sa limite est indépendante du choix de la suite
\{(\{φ\}\_\{n\})\}\_\{n∈ℕ\}~; on l'appelle l'intégrale de f sur le
segment {[}a,b{]} et on la note \{\textbackslash{}mathop\{∫ \}
\}\_\{a\}\^{}\{b\}f.

Démonstration Soit ε \textgreater{} 0 et N ∈ ℕ tel que n ≥ N
⇒\textbackslash{}\textbar{} f − \{φ\}\_\{n\}\textbackslash{}\textbar{}∞
\textless{}\{ ε \textbackslash{}over 2(b−a)\} . Pour p,q ≥ N on a
\textbackslash{}\textbar{}\{φ\}\_\{p\} −
\{φ\}\_\{q\}\textbackslash{}\textbar{}∞ ≤\textbackslash{}\textbar{}
\{φ\}\_\{p\} − f\textbackslash{}\textbar{}∞ +\textbackslash{}\textbar{}
f − \{φ\}\_\{q\}\textbackslash{}\textbar{}∞ \textless{}\{ ε
\textbackslash{}over b−a\} et donc
\textbackslash{}\textbar{}\{\textbackslash{}mathop\{∫ \}
\}\_\{a\}\^{}\{b\}\{φ\}\_\{p\} −\{\textbackslash{}mathop\{∫ \}
\}\_\{a\}\^{}\{b\}\{φ\}\_\{q\}\textbackslash{}\textbar{}
=\textbackslash{}\textbar{}\{\textbackslash{}mathop\{∫ \}
\}\_\{a\}\^{}\{b\}(\{φ\}\_\{p\} −
\{φ\}\_\{q\})\textbackslash{}\textbar{} ≤ (b −
a)\textbackslash{}\textbar{}\{φ\}\_\{p\} −
\{φ\}\_\{q\}\textbackslash{}\textbar{}∞ \textless{} ε. La suite
(\{\textbackslash{}mathop\{∫ \} \}\_\{a\}\^{}\{b\}\{φ\}\_\{n\}) vérifie
donc le critère de Cauchy, donc elle converge (E étant complet). Soit
(\{ψ\}\_\{n\}) une autre suite en escalier telle que
\textbackslash{}mathop\{lim\}\textbackslash{}\textbar{}f −
\{ψ\}\_\{n\}\textbackslash{}\textbar{}∞ = 0. Comme
\textbackslash{}\textbar{}\{φ\}\_\{n\} −
\{ψ\}\_\{n\}\textbackslash{}\textbar{}∞ ≤\textbackslash{}\textbar{}
\{φ\}\_\{n\} − f\textbackslash{}\textbar{}∞ +\textbackslash{}\textbar{}
f − \{ψ\}\_\{n\}\textbackslash{}\textbar{}∞, on a
\textbackslash{}mathop\{lim\}\textbackslash{}\textbar{}\{φ\}\_\{n\} −
\{ψ\}\_\{n\}\textbackslash{}\textbar{}∞ = 0 et la majoration
\textbackslash{}\textbar{}\{\textbackslash{}mathop\{∫ \}
\}\_\{a\}\^{}\{b\}\{φ\}\_\{n\} −\{\textbackslash{}mathop\{∫ \}
\}\_\{a\}\^{}\{b\}\{ψ\}\_\{n\}\textbackslash{}\textbar{}
=\textbackslash{}\textbar{}\{\textbackslash{}mathop\{∫ \}
\}\_\{a\}\^{}\{b\}(\{φ\}\_\{n\} −
\{ψ\}\_\{n\})\textbackslash{}\textbar{} ≤ (b −
a)\textbackslash{}\textbar{}\{φ\}\_\{n\} −
\{ψ\}\_\{n\}\textbackslash{}\textbar{}∞ montre que les deux suites
(\{\textbackslash{}mathop\{∫ \} \}\_\{a\}\^{}\{b\}\{φ\}\_\{n\}) et
(\{\textbackslash{}mathop\{∫ \} \}\_\{a\}\^{}\{b\}\{ψ\}\_\{n\}) (dont on
sait déjà qu'elles convergent) ont la même limite.

Remarque~9.2.1 La méthode ci dessus est la méthode classique de
prolongement d'une application uniformément continue
(f\textbackslash{}mathrel\{↦\}\{\textbackslash{}mathop\{∫ \}
\}\_\{a\}\^{}\{b\}f) d'un sous-ensemble (celui des fonctions en
escalier) à son adhérence (les fonctions réglées). Remarquons également
que si f est une fonction en escalier, on peut prendre pour tout n ,
\{φ\}\_\{n\} = f et que donc son intégrale en tant que fonction en
escalier est la même que son intégrale en tant que fonction réglée. En
particulier l'intégrale d'une constante m est m(b − a).

Théorème~9.2.4 (i) L'application
f\textbackslash{}mathrel\{↦\}\{\textbackslash{}mathop\{∫ \}
\}\_\{a\}\^{}\{b\}f est linéaire de l'espace vectoriel ~des applications
réglées de {[}a,b{]} dans E dans l'espace vectoriel ~E. (ii) Soit u : E
→ F linéaire continue et f : {[}a,b{]} → E réglée~; alors u ∘ f :
{[}a,b{]} → F est réglée et \{\textbackslash{}mathop\{∫ \}
\}\_\{a\}\^{}\{b\}u ∘ f = u\textbackslash{}left
(\{\textbackslash{}mathop\{∫ \} \}\_\{a\}\^{}\{b\}f\textbackslash{}right
) (iii) Soit f : {[}a,b{]} → E réglée~; alors
\textbackslash{}\textbar{}f\textbackslash{}\textbar{} : {[}a,b{]} → ℝ,
t\textbackslash{}mathrel\{↦\}\textbackslash{}\textbar{}f(t)\textbackslash{}\textbar{}
est réglée et \textbackslash{}\textbar{}\{\textbackslash{}mathop\{∫ \}
\}\_\{a\}\^{}\{b\}f\textbackslash{}\textbar{}
≤\{\textbackslash{}mathop\{∫ \}
\}\_\{a\}\^{}\{b\}\textbackslash{}\textbar{}f\textbackslash{}\textbar{}
(iv) Si c ∈{]}a,b{[} et si f : {[}a,b{]} → E est réglée, alors
\{f\}\_\{\{\textbar{}\}\_\{{[}a,c{]}\}\} et
\{f\}\_\{\{\textbar{}\}\_\{{[}c,b{]}\}\} sont réglées et
\{\textbackslash{}mathop\{∫ \} \}\_\{a\}\^{}\{b\}f
=\{\textbackslash{}mathop\{∫ \} \}\_\{a\}\^{}\{c\}f
+\{\textbackslash{}mathop\{∫ \} \}\_\{c\}\^{}\{b\}f.

Démonstration (i) Soit f et g de {[}a,b{]} dans E réglées, soit
(\{φ\}\_\{n\}) et (\{ψ\}\_\{n\}) deux suites de fonctions en escalier
telles que \textbackslash{}mathop\{lim\}\textbackslash{}\textbar{}f −
\{φ\}\_\{n\}\textbackslash{}\textbar{}∞ = 0 et
\textbackslash{}mathop\{lim\}\textbackslash{}\textbar{}g −
\{ψ\}\_\{n\}\textbackslash{}\textbar{}∞ = 0. Si α,β ∈ K, on a
\textbackslash{}\textbar{}(αf + βg) − (α\{φ\}\_\{n\} +
β\{ψ\}\_\{n\})\textbackslash{}\textbar{}∞
≤\textbar{}α\textbar{}\textbackslash{}\textbar{}f −
\{φ\}\_\{n\}\textbackslash{}\textbar{}∞ +
\textbar{}β\textbar{}\textbackslash{}\textbar{}g −
\{ψ\}\_\{n\}\textbackslash{}\textbar{}∞ et donc
\textbackslash{}mathop\{lim\}\textbackslash{}\textbar{}(αf + βg) −
(α\{φ\}\_\{n\} + β\{ψ\}\_\{n\})\textbackslash{}\textbar{}∞ = 0. On en
déduit que αf + βg est encore réglée et que

\textbackslash{}begin\{eqnarray*\} \{\textbackslash{}mathop\{∫ \}
\}\_\{a\}\^{}\{b\}(αf + βg)\& =\&
\textbackslash{}mathop\{lim\}\{\textbackslash{}mathop\{∫ \}
\}\_\{a\}\^{}\{b\}(α\{φ\}\_\{ n\} + β\{ψ\}\_\{n\}) \%\&
\textbackslash{}\textbackslash{} \& =\&
α\textbackslash{}mathop\{lim\}\{\textbackslash{}mathop\{∫ \}
\}\_\{a\}\^{}\{b\}\{φ\}\_\{ n\} +
β\textbackslash{}mathop\{lim\}\{\textbackslash{}mathop\{∫ \}
\}\_\{a\}\^{}\{b\}\{ψ\}\_\{ n\}\%\& \textbackslash{}\textbackslash{} \&
=\& α\{\textbackslash{}mathop\{∫ \} \}\_\{a\}\^{}\{b\}f +
β\{\textbackslash{}mathop\{∫ \} \}\_\{a\}\^{}\{b\}g \%\&
\textbackslash{}\textbackslash{} \textbackslash{}end\{eqnarray*\}

(ii) Soit u : E → F linéaire continue et f : {[}a,b{]} → E réglée, soit
(\{φ\}\_\{n\}) une suite de fonctions en escalier telle que
\textbackslash{}mathop\{lim\}\textbackslash{}\textbar{}f −
\{φ\}\_\{n\}\textbackslash{}\textbar{}∞ = 0. Alors u ∘ \{φ\}\_\{n\} est
en escalier (toute subdivision adaptée à \{φ\}\_\{n\} l'est encore à u ∘
\{φ\}\_\{n\}) et \textbackslash{}\textbar{}u ∘ f(t) − u ∘
\{φ\}\_\{n\}(t)\textbackslash{}\textbar{} =\textbackslash{}\textbar{}
u(f(t) − \{φ\}\_\{n\}(t))\textbackslash{}\textbar{}
≤\textbackslash{}\textbar{}
u\textbackslash{}\textbar{}\textbackslash{},\textbackslash{}\textbar{}f(t)
− \{φ\}\_\{n\}(t)\textbackslash{}\textbar{} d'où
\textbackslash{}\textbar{}u ∘ f − u ∘
\{φ\}\_\{n\}\textbackslash{}\textbar{}∞ ≤\textbackslash{}\textbar{}
u\textbackslash{}\textbar{}\textbackslash{},\textbackslash{}\textbar{}f
− \{φ\}\_\{n\}\textbackslash{}\textbar{}∞. Ceci montre que u ∘ f est
encore réglée et on a

\textbackslash{}begin\{eqnarray*\} \{\textbackslash{}mathop\{∫ \}
\}\_\{a\}\^{}\{b\}u ∘ f\& =\&
\textbackslash{}mathop\{lim\}\{\textbackslash{}mathop\{∫ \}
\}\_\{a\}\^{}\{b\}u ∘ \{φ\}\_\{ n\} =\textbackslash{}mathop\{
lim\}u(\{\textbackslash{}mathop\{∫ \} \}\_\{a\}\^{}\{b\}\{φ\}\_\{
n\})\%\& \textbackslash{}\textbackslash{} \& =\&
u(\textbackslash{}mathop\{lim\}\{\textbackslash{}mathop\{∫ \}
\}\_\{a\}\^{}\{b\}\{φ\}\_\{ n\}) = u(\{\textbackslash{}mathop\{∫ \}
\}\_\{a\}\^{}\{b\}f) \%\& \textbackslash{}\textbackslash{}
\textbackslash{}end\{eqnarray*\}

puisque u est continue et \{\textbackslash{}mathop\{∫ \}
\}\_\{a\}\^{}\{b\}u ∘ \{φ\}\_\{n\} = u(\{\textbackslash{}mathop\{∫ \}
\}\_\{a\}\^{}\{b\}\{φ\}\_\{n\}) (intégrale des fonctions en escalier).

(iii) La démonstration est similaire en remarquant que
\textbackslash{}\textbar{}\{φ\}\_\{n\}\textbackslash{}\textbar{} est
encore en escalier et que
\textbar{}\textbackslash{}\textbar{}f(t)\textbackslash{}\textbar{}
−\textbackslash{}\textbar{}
\{φ\}\_\{n\}(t)\textbackslash{}\textbar{}\textbar{}≤\textbackslash{}\textbar{}
f(t) − \{φ\}\_\{n\}(t)\textbackslash{}\textbar{}, soit
\textbackslash{}\textbar{}\textbackslash{}\textbar{}f\textbackslash{}\textbar{}
−\textbackslash{}\textbar{}
\{φ\}\_\{n\}\textbackslash{}\textbar{}\textbackslash{}\textbar{}∞
≤\textbackslash{}\textbar{} f − \{φ\}\_\{n\}\textbackslash{}\textbar{}∞.
On a donc \textbackslash{}\textbar{}f\textbackslash{}\textbar{} réglée
et

\textbackslash{}begin\{eqnarray*\} \{\textbackslash{}mathop\{∫ \}
\}\_\{a\}\^{}\{b\}\textbackslash{}\textbar{}f\textbackslash{}\textbar{}\&
=\& \textbackslash{}mathop\{lim\}\{\textbackslash{}mathop\{∫ \}
\}\_\{a\}\^{}\{b\}\textbackslash{}\textbar{}\{φ\}\_\{
n\}\textbackslash{}\textbar{} ≥\textbackslash{}mathop\{
lim\}\textbackslash{}\textbar{}\{\textbackslash{}mathop\{∫ \}
\}\_\{a\}\^{}\{b\}\{φ\}\_\{ n\}\textbackslash{}\textbar{}\%\&
\textbackslash{}\textbackslash{} \& =\&
\textbackslash{}\textbar{}\textbackslash{}mathop\{lim\}\{\textbackslash{}mathop\{∫
\} \}\_\{a\}\^{}\{b\}\{φ\}\_\{ n\}\textbackslash{}\textbar{}
=\textbackslash{}\textbar{}\{\textbackslash{}mathop\{∫ \}
\}\_\{a\}\^{}\{b\}f\textbackslash{}\textbar{} \%\&
\textbackslash{}\textbackslash{} \textbackslash{}end\{eqnarray*\}

puisque
x\textbackslash{}mathrel\{↦\}\textbackslash{}\textbar{}x\textbackslash{}\textbar{}
est continue et \{\textbackslash{}mathop\{∫ \}
\}\_\{a\}\^{}\{b\}\textbackslash{}\textbar{}\{φ\}\_\{n\}\textbackslash{}\textbar{}
≥\textbackslash{}\textbar{}\{\textbackslash{}mathop\{∫ \}
\}\_\{a\}\^{}\{b\}\{φ\}\_\{n\}\textbackslash{}\textbar{} (intégrale des
fonctions en escalier).

(iv) On remarque ici que
\{(\{φ\}\_\{n\})\}\_\{\{\textbar{}\}\_\{{[}a,c{]}\}\} et
\{(\{φ\}\_\{n\})\}\_\{\{\textbar{}\}\_\{{[}c,b{]}\}\} sont encore en
escalier et que
\textbackslash{}\textbar{}\{f\}\_\{\{\textbar{}\}\_\{{[}a,c{]}\}\} −
\{(\{φ\}\_\{n\})\}\_\{\{\textbar{}\}\_\{{[}a,c{]}\}\}\textbackslash{}\textbar{}∞
≤\textbackslash{}\textbar{} f − \{φ\}\_\{n\}\textbackslash{}\textbar{}∞
et de même sur {[}c,b{]}. On a donc

\textbackslash{}begin\{eqnarray*\} \{\textbackslash{}mathop\{∫ \}
\}\_\{a\}\^{}\{b\}f\& =\&
\textbackslash{}mathop\{lim\}(\{\textbackslash{}mathop\{∫ \}
\}\_\{a\}\^{}\{c\}\{φ\}\_\{ n\} +\{\textbackslash{}mathop\{∫ \}
\}\_\{c\}\^{}\{b\}\{φ\}\_\{ n\}) =\textbackslash{}mathop\{
lim\}\{\textbackslash{}mathop\{∫ \} \}\_\{a\}\^{}\{c\}\{φ\}\_\{ n\}
+\textbackslash{}mathop\{ lim\}\{\textbackslash{}mathop\{∫ \}
\}\_\{c\}\^{}\{b\}\{φ\}\_\{ n\}\%\& \textbackslash{}\textbackslash{} \&
=\& \{\textbackslash{}mathop\{∫ \} \}\_\{a\}\^{}\{c\}f
+\{\textbackslash{}mathop\{∫ \} \}\_\{c\}\^{}\{b\}f \%\&
\textbackslash{}\textbackslash{} \textbackslash{}end\{eqnarray*\}

puisque l'existence de toutes les limites est garantie.

La propriété (iii) a un certain nombre de conséquences extrêmement
importantes

Théorème~9.2.5 (i) Soit f : {[}a,b{]} → ℝ réglée positive~; alors
\{\textbackslash{}mathop\{∫ \} \}\_\{a\}\^{}\{b\}f ≥ 0. (ii) Soit f et g
des applications réglées de {[}a,b{]} dans ℝ telles que f ≤ g. Alors
\{\textbackslash{}mathop\{∫ \} \}\_\{a\}\^{}\{b\}f
≤\{\textbackslash{}mathop\{∫ \} \}\_\{a\}\^{}\{b\}g (iii) Soit f :
{[}a,b{]} → E réglée~; alors
\textbackslash{}\textbar{}\{\textbackslash{}mathop\{∫ \}
\}\_\{a\}\^{}\{b\}f\textbackslash{}\textbar{} ≤ (b −
a)\textbackslash{}\textbar{}f\textbackslash{}\textbar{}∞.

Démonstration (i) On a d'après l'assertion (iii) du théorème précédent
\textbackslash{}left \textbar{}\{\textbackslash{}mathop\{∫ \}
\}\_\{a\}\^{}\{b\}f\textbackslash{}right \textbar{}
≤\{\textbackslash{}mathop\{∫ \} \}\_\{a\}\^{}\{b\}\textbar{}f\textbar{}
=\{\textbackslash{}mathop\{∫ \} \}\_\{a\}\^{}\{b\}f~; ceci exige
\{\textbackslash{}mathop\{∫ \} \}\_\{a\}\^{}\{b\}f ≥ 0.

(ii) La fonction g − f est réglée positive, donc
\{\textbackslash{}mathop\{∫ \} \}\_\{a\}\^{}\{b\}g
−\{\textbackslash{}mathop\{∫ \} \}\_\{a\}\^{}\{b\}f
=\{\textbackslash{}mathop\{∫ \} \}\_\{a\}\^{}\{b\}(g − f) ≥ 0

(iii) On a \textbackslash{}\textbar{}\{\textbackslash{}mathop\{∫ \}
\}\_\{a\}\^{}\{b\}f\textbackslash{}\textbar{}
≤\{\textbackslash{}mathop\{∫ \}
\}\_\{a\}\^{}\{b\}\textbackslash{}\textbar{}f\textbackslash{}\textbar{}
≤\{\textbackslash{}mathop\{∫ \}
\}\_\{a\}\^{}\{b\}\textbackslash{}\textbar{}f\textbackslash{}\textbar{}∞
= (b − a)\textbackslash{}\textbar{}f\textbackslash{}\textbar{}∞ puisque
\textbackslash{}mathop\{∀\}t,
\textbackslash{}\textbar{}f(t)\textbackslash{}\textbar{}
≤\textbackslash{}\textbar{} f\textbackslash{}\textbar{}∞.

En fait le résultat (i) peut être précisé de la manière suivante

Théorème~9.2.6 Soit f : {[}a,b{]} → ℝ réglée positive~; on suppose qu'il
existe \{t\}\_\{0\} ∈ {[}a,b{]} tel que
f(\{t\}\_\{0\})\textbackslash{}mathrel\{≠\}0 et f continue au point
\{t\}\_\{0\}. Alors \{\textbackslash{}mathop\{∫ \} \}\_\{a\}\^{}\{b\}f
\textgreater{} 0.

Démonstration Supposons par exemple
\{t\}\_\{0\}\textbackslash{}mathrel\{≠\}b. On peut alors trouver η
\textgreater{} 0 tel que \{t\}\_\{0\} + η \textless{} b et tel que
\textbackslash{}mathop\{∀\}t ∈ {[}\{t\}\_\{0\},\{t\}\_\{0\} + η{]},
\textbar{}f(t) − f(\{t\}\_\{0\})\textbar{} \textless{}\{ f(\{t\}\_\{0\})
\textbackslash{}over 2\} , soit encore f(t) \textgreater{}\{
f(\{t\}\_\{0\}) \textbackslash{}over 2\} . Comme
\{\textbackslash{}mathop\{∫ \} \}\_\{a\}\^{}\{\{t\}\_\{0\}\}f ≥ 0 et
\{\textbackslash{}mathop\{∫ \} \}\_\{\{t\}\_\{0\}+η\}\^{}\{b\}f ≥ 0, on
a

\{\textbackslash{}mathop\{∫ \} \}\_\{a\}\^{}\{b\}f
≥\{\textbackslash{}mathop\{∫ \}
\}\_\{\{t\}\_\{0\}\}\^{}\{\{t\}\_\{0\}+η\}f ≥\{\textbackslash{}mathop\{∫
\} \}\_\{\{t\}\_\{0\}\}\^{}\{\{t\}\_\{0\}+η\}\{ f(\{t\}\_\{0\})
\textbackslash{}over 2\} =\{ ηf(\{t\}\_\{0\}) \textbackslash{}over 2\}
\textgreater{} 0

Corollaire~9.2.7 Soit f : {[}a,b{]} → ℝ continue positive~; si
\{\textbackslash{}mathop\{∫ \} \}\_\{a\}\^{}\{b\}f = 0, alors f = 0.

Théorème~9.2.8 (première formule de la moyenne). Soit f,g : {[}a,b{]} →
ℝ. On suppose que f est continue et que g est réglée positive. Alors, il
existe c ∈ {[}a,b{]} tel que \{\textbackslash{}mathop\{∫ \}
\}\_\{a\}\^{}\{b\}fg = f(c)\{\textbackslash{}mathop\{∫ \}
\}\_\{a\}\^{}\{b\}g.

Démonstration L'image de {[}a,b{]} par f est à la fois connexe et
compact dans ℝ, c'est donc un segment de ℝ. Soit f({[}a,b{]}) =
{[}m,M{]}. On a \textbackslash{}mathop\{∀\}t ∈ {[}a,b{]},m ≤ f(t) ≤ M et
donc, puisque g(t) ≥ 0, on a \textbackslash{}mathop\{∀\}t ∈
{[}a,b{]},mg(t) ≤ f(t)g(t) ≤ Mg(t). En intégrant, on a alors
m\{\textbackslash{}mathop\{∫ \} \}\_\{a\}\^{}\{b\}g
≤\{\textbackslash{}mathop\{∫ \} \}\_\{a\}\^{}\{b\}fg ≤
M\{\textbackslash{}mathop\{∫ \} \}\_\{a\}\^{}\{b\}g. Si
\{\textbackslash{}mathop\{∫ \} \}\_\{a\}\^{}\{b\}g = 0, on en déduit que
\{\textbackslash{}mathop\{∫ \} \}\_\{a\}\^{}\{b\}fg = 0 et n'importe
quel c ∈ {[}a,b{]} convient. Sinon, on a m ≤\{
\{\textbackslash{}mathop\{∫ \} \}\_\{a\}\^{}\{b\}fg \textbackslash{}over
\{\textbackslash{}mathop\{∫ \} \}\_\{a\}\^{}\{b\}g\} ≤ M et donc

\textbackslash{}mathop\{∃\}c ∈ {[}a,b{]},\{ \{\textbackslash{}mathop\{∫
\} \}\_\{a\}\^{}\{b\}fg \textbackslash{}over \{\textbackslash{}mathop\{∫
\} \}\_\{a\}\^{}\{b\}g\} = f(c)

ce que nous voulions démontrer.

\paragraph{9.2.3 Convention de Chasles}

Définition~9.2.1 Soit I un intervalle de ℝ~; on dit que f est réglée sur
I si sa restriction à tout segment inclus dans I est réglée.

Dans ce cas, si a et b sont dans I et a \textless{} b, on peut définir
\{\textbackslash{}mathop\{∫ \} \}\_\{a\}\^{}\{b\}f. On étendra la
définition en posant \{\textbackslash{}mathop\{∫ \} \}\_\{a\}\^{}\{b\}f
= 0 si a = b et \{\textbackslash{}mathop\{∫ \} \}\_\{a\}\^{}\{b\}f =
−\{\textbackslash{}mathop\{∫ \} \}\_\{b\}\^{}\{a\}f si a \textgreater{}
b. On a alors

Proposition~9.2.9 (relation de Chasles). Soit f : I → E réglée. Alors

\textbackslash{}mathop\{∀\}a,b,c ∈ I, \{\textbackslash{}mathop\{∫ \}
\}\_\{a\}\^{}\{c\}f =\{\textbackslash{}mathop\{∫ \} \}\_\{a\}\^{}\{b\}f
+\{\textbackslash{}mathop\{∫ \} \}\_\{b\}\^{}\{c\}f

Démonstration Examiner toutes les configurations possibles de a,b,c.

Remarque~9.2.2 Le lecteur prendra garde à ne pas utiliser les diverses
majorations ou minorations rencontrées auparavant dans les cas où a
\textgreater{} b~; dans ce cas, toutes les inégalités précédentes sont
changées de sens.

\paragraph{9.2.4 Sommes de Riemann}

Soit σ = \{(\{a\}\_\{i\})\}\_\{0≤i≤n\} une subdivision de {[}a,b{]}, ξ =
\{(\{ξ\}\_\{i\})\}\_\{1≤i≤n\} une famille de points de {[}a,b{]} tels
que \textbackslash{}mathop\{∀\}i ∈ {[}1,n{]}, \{ξ\}\_\{i\} ∈
{[}\{a\}\_\{i−1\},\{a\}\_\{i\}{]}. Si f est une application de {[}a,b{]}
dans E, on posera

S(f,σ,ξ) =\{ \textbackslash{}mathop\{∑ \}\}\_\{i=1\}\^{}\{n\}(\{a\}\_\{
i\} − \{a\}\_\{i−1\})f(\{ξ\}\_\{i\})

Définition~9.2.2 On dira que S(f,σ,ξ) est une somme de Riemann associée
à la subdivision σ et à la famille de points ξ.

Théorème~9.2.10 Soit f : {[}a,b{]} → E réglée~; alors les sommes de
Riemann de f tendent vers l'intégrale de f quand le pas de la
subdivision tend vers 0, plus précisément

\textbackslash{}mathop\{∀\}ε \textgreater{}
0,\textbackslash{}mathop\{∃\}η \textgreater{} 0,
\textbackslash{}mathop\{∀\}(σ,ξ),\textbackslash{}quad δ(σ) \textless{} η
⇒\textbackslash{}\textbar{}\{\textbackslash{}mathop\{∫ \}
\}\_\{a\}\^{}\{b\}f − S(f,σ,ξ)\textbackslash{}\textbar{} \textless{} ε

Démonstration Nous allons tout d'abord démontrer ce résultat pour une
fonction φ : {[}a,b{]} → E en escalier. Soit
\{(\{x\}\_\{k\})\}\_\{0≤k≤K\} une subdivision adaptée à φ. Soit σ =
\{(\{a\}\_\{i\})\}\_\{0≤i≤n\} une subdivision de {[}a,b{]}, ξ =
\{(\{ξ\}\_\{i\})\}\_\{1≤i≤n\} une famille de points de {[}a,b{]} tels
que \textbackslash{}mathop\{∀\}i ∈ {[}1,n{]}, \{ξ\}\_\{i\} ∈
{[}\{a\}\_\{i−1\},\{a\}\_\{i\}{]}. On écrit alors

\textbackslash{}begin\{eqnarray*\} \{\textbackslash{}mathop\{∫ \}
\}\_\{a\}\^{}\{b\}φ − S(φ,σ,ξ)\& =\& \{\textbackslash{}mathop\{∑
\}\}\_\{i=1\}\^{}\{n\}\textbackslash{}left
(\{\textbackslash{}mathop\{\textbackslash{}mathop\{∫ \} \}
\}\_\{\{a\}\_\{i−1\}\}\^{}\{\{a\}\_\{i\} \}φ − (\{a\}\_\{i\} −
\{a\}\_\{i−1\})φ(\{ξ\}\_\{i\})\textbackslash{}right )\%\&
\textbackslash{}\textbackslash{} \& =\& \{\textbackslash{}mathop\{∑
\}\}\_\{i=1\}\^{}\{n\}\{
\textbackslash{}mathop\{\textbackslash{}mathop\{∫ \} \}
\}\_\{\{a\}\_\{i−1\}\}\^{}\{\{a\}\_\{i\} \}(φ − φ(\{ξ\}\_\{i\})) \%\&
\textbackslash{}\textbackslash{} \textbackslash{}end\{eqnarray*\}

Notons H = \textbackslash{}\{i ∈
{[}1,n{]}\textbackslash{}mathrel\{∣\}\textbackslash{}mathop\{∃\}k ∈
{[}0,K{]}, \{x\}\_\{k\} ∈
{[}\{a\}\_\{i−1\},\{a\}\_\{i\}{]}\textbackslash{}\}. On remarque tout
d'abord que chaque \{x\}\_\{k\} ne peut appartenir qu'à au plus 2
intervalles {[}\{a\}\_\{i−1\},\{a\}\_\{i\}{]} et que donc le cardinal de
H est plus petit que 2K + 2. Soit i ∈ {[}1,n{]}. Deux cas sont
possibles~; si i\textbackslash{}mathrel\{∉\}H, la fonction φ est
constante sur {[}\{a\}\_\{i−1\},\{a\}\_\{i\}{]} et donc
\{\textbackslash{}mathop\{∫ \}
\}\_\{\{a\}\_\{i−1\}\}\^{}\{\{a\}\_\{i\}\}(φ − φ(\{ξ\}\_\{i\})) = 0. Si
par contre, i ∈ H, on a

\textbackslash{}\textbar{}\{\textbackslash{}mathop\{∫ \}
\}\_\{\{a\}\_\{i−1\}\}\^{}\{\{a\}\_\{i\} \}(φ −
φ(\{ξ\}\_\{i\}))\textbackslash{}\textbar{} ≤\{\textbackslash{}mathop\{∫
\} \}\_\{\{a\}\_\{i−1\}\}\^{}\{\{a\}\_\{i\}
\}2\textbackslash{}\textbar{}φ\textbackslash{}\textbar{}∞ ≤
2δ(σ)\textbackslash{}\textbar{}φ\textbackslash{}\textbar{}∞

On en déduit que

\textbackslash{}\textbar{}\{\textbackslash{}mathop\{∫ \}
\}\_\{a\}\^{}\{b\}φ − S(φ,σ,ξ)\textbackslash{}\textbar{} ≤
2δ(σ)\textbackslash{}\textbar{}φ\textbackslash{}\textbar{}
∞\textbackslash{}mathop\{Card\}H ≤ 4(K +
1)δ(σ)\textbackslash{}\textbar{}φ\textbackslash{}\textbar{}∞

On voit donc que δ(σ) \textless{}\{ ε \textbackslash{}over
4(K+1)\textbackslash{}\textbar{}φ\textbackslash{}\textbar{}∞\}
⇒\textbackslash{}\textbar{}\{\textbackslash{}mathop\{∫ \}
\}\_\{a\}\^{}\{b\}φ − S(φ,σ,ξ)\textbackslash{}\textbar{} \textless{} ε.

Supposons maintenant que f est réglée, et soit ε \textgreater{} 0. Il
existe une fonction φ en escalier telle que \textbackslash{}\textbar{}f
− φ\textbackslash{}\textbar{}∞ \textless{}\{ ε \textbackslash{}over
4(b−a)\} . On a alors
\textbackslash{}\textbar{}\{\textbackslash{}mathop\{∫ \}
\}\_\{a\}\^{}\{b\}f −\{\textbackslash{}mathop\{∫ \}
\}\_\{a\}\^{}\{b\}φ\textbackslash{}\textbar{} ≤\{ ε \textbackslash{}over
4\} et

\textbackslash{}begin\{eqnarray*\} \textbackslash{}\textbar{}S(f,σ,ξ) −
S(φ,σ,ξ)\textbackslash{}\textbar{}\& ≤\& \{\textbackslash{}mathop\{∑
\}\}\_\{i=1\}\^{}\{n\}(\{a\}\_\{ i\} −
\{a\}\_\{i−1\})\textbackslash{}\textbar{}f(\{ξ\}\_\{i\}) −
φ(\{ξ\}\_\{i\})\textbackslash{}\textbar{}\%\&
\textbackslash{}\textbackslash{} \& ≤\& (b −
a)\textbackslash{}\textbar{}f − φ\textbackslash{}\textbar{}∞
\textless{}\{ ε \textbackslash{}over 4\} \%\&
\textbackslash{}\textbackslash{} \textbackslash{}end\{eqnarray*\}

Mais pour la fonction en escalier φ il existe η \textgreater{} 0 tel que
\textbackslash{}mathop\{∀\}(σ,ξ), δ(σ) \textless{} η
⇒\textbackslash{}\textbar{}\{\textbackslash{}mathop\{∫ \}
\}\_\{a\}\^{}\{b\}φ − S(φ,σ,ξ)\textbackslash{}\textbar{} ≤\{ ε
\textbackslash{}over 2\} . Alors, δ(σ) \textless{} η implique

\textbackslash{}begin\{eqnarray*\}
\textbackslash{}\textbar{}\{\textbackslash{}mathop\{∫ \}
\}\_\{a\}\^{}\{b\}f − S(f,σ,ξ)\textbackslash{}\textbar{}\&\& \%\&
\textbackslash{}\textbackslash{} \& ≤\&
\textbackslash{}\textbar{}\{\textbackslash{}mathop\{∫ \}
\}\_\{a\}\^{}\{b\}f −\{\textbackslash{}mathop\{∫ \}
\}\_\{a\}\^{}\{b\}φ\textbackslash{}\textbar{}
+\textbackslash{}\textbar{}\{\textbackslash{}mathop\{∫ \}
\}\_\{a\}\^{}\{b\}φ − S(φ,σ,ξ)\textbackslash{}\textbar{}\%\&
\textbackslash{}\textbackslash{} \& \textbackslash{}text\{\} \&
+\textbackslash{}\textbar{}S(φ,σ,ξ) − S(f,σ,ξ)\textbackslash{}\textbar{}
\%\& \textbackslash{}\textbackslash{} \& ≤\& ε \%\&
\textbackslash{}\textbackslash{} \textbackslash{}end\{eqnarray*\}

ce qu'on voulait démontrer.

Remarque~9.2.3 On a vu ici une des techniques essentielles pour l'étude
des fonctions réglées (ou même continues)~: on commence par démontrer le
résultat cherché pour des fonctions en escalier et on en déduit le
résultat général par approximation.

Remarque~9.2.4 L'intérêt essentiel de ce résultat est non pas de
calculer des intégrales (il est bien rare que l'on y arrive par cette
méthode) ni même de calculer des valeurs approchées d'intégrales (la
convergence étant beaucoup trop lente), mais plutôt de trouver les
limites de certaines suites en les identifiant comme sommes de Riemann
d'une certaine fonction réglée. De ce point de vue, les subdivisions
courantes sont les subdivisions régulières \{σ\}\_\{n\} = \{(a + i\{ b−a
\textbackslash{}over n\} )\}\_\{0≤i≤n\} avec divers choix possibles de
\{ξ\}\_\{i\}, \{ξ\}\_\{i\} = \{a\}\_\{i−1\} ou \{ξ\}\_\{i\} =
\{a\}\_\{i\} ou plus rarement \{ξ\}\_\{i\} =\{
\{a\}\_\{i−1\}+\{a\}\_\{i\} \textbackslash{}over 2\} .

\paragraph{9.2.5 Sommes de Darboux}

Pour une fonction réglée f : {[}a,b{]} → ℝ et σ =
\{(\{a\}\_\{i\})\}\_\{0≤i≤n\} une subdivision de {[}a,b{]}, d'autres
sommes se présentent naturellement~; puisque f est bornée, on peut poser
\{m\}\_\{i\} =\{\textbackslash{}mathop\{ inf\}
\}\_\{t∈{[}\{a\}\_\{i−1\},\{a\}\_\{i\}{]}\}f(t) et \{M\}\_\{i\}
=\{\textbackslash{}mathop\{
sup\}\}\_\{t∈{[}\{a\}\_\{i−1\},\{a\}\_\{i\}{]}\}f(t). On introduit alors
les sommes de Darboux inférieures et supérieures d(f,σ)
=\{\textbackslash{}mathop\{ \textbackslash{}mathop\{∑ \}\}
\}\_\{i=1\}\^{}\{n\}(\{a\}\_\{i\} − \{a\}\_\{i−1\})\{m\}\_\{i\} et
D(f,σ) =\{\textbackslash{}mathop\{ \textbackslash{}mathop\{∑ \}\}
\}\_\{i=1\}\^{}\{n\}(\{a\}\_\{i\} − \{a\}\_\{i−1\})\{M\}\_\{i\}~;
remarquons que si f est continue, la fonction f atteint ses bornes et on
retombe sur de classiques sommes de Riemann. Dans le cas général, on a

Proposition~9.2.11 (i) d(f,σ) ≤\{\textbackslash{}mathop\{∫ \}
\}\_\{a\}\^{}\{b\}f ≤ D(f,σ) (ii) les sommes de Darboux de f tendent
vers l'intégrale de f quand le pas de la subdivision tend vers 0.

Démonstration (i) On écrit \{\textbackslash{}mathop\{∫ \}
\}\_\{a\}\^{}\{b\}f − d(f,σ) =\{\textbackslash{}mathop\{
\textbackslash{}mathop\{∑ \}\} \}\_\{i=1\}\^{}\{n\}\textbackslash{}left
(\{\textbackslash{}mathop\{∫ \}
\}\_\{\{a\}\_\{i−1\}\}\^{}\{\{a\}\_\{i\}\}f − (\{a\}\_\{i\} −
\{a\}\_\{i−1\})\{m\}\_\{i\}\textbackslash{}right )
=\{\textbackslash{}mathop\{ \textbackslash{}mathop\{∑\}\}
\}\_\{i=1\}\^{}\{n\}\{\textbackslash{}mathop\{∫ \}
\}\_\{\{a\}\_\{i−1\}\}\^{}\{\{a\}\_\{i\}\}(f − \{m\}\_\{i\}) ≥ 0 et de
même pour la somme de Darboux supérieure.

(ii) Soit ε \textgreater{} 0~; soit η \textgreater{} 0 tel que
\textbackslash{}mathop\{∀\}(σ,ξ),\textbackslash{}quad δ(σ) \textless{} η
⇒\textbackslash{}left \textbar{}\{\textbackslash{}mathop\{∫ \}
\}\_\{a\}\^{}\{b\}f − S(f,σ,ξ)\textbackslash{}right \textbar{}
\textless{}\{ ε \textbackslash{}over 2\} , et σ =
\{(\{a\}\_\{i\})\}\_\{0≤i≤n\} une subdivision de {[}a,b{]} de pas plus
petit que η~; pour chaque i ∈ {[}1,n{]}, soit \{ξ\}\_\{i\} ∈
{[}\{a\}\_\{i−1\},\{a\}\_\{i\}{]} tel que \{m\}\_\{i\} ≤ f(\{ξ\}\_\{i\})
\textless{} \{m\}\_\{i\} +\{ ε \textbackslash{}over 2(b−a)\} . En
multipliant par (\{a\}\_\{i\} − \{a\}\_\{i−1\}) et en sommant les
inégalités obtenues, on a d(f,σ) ≤ S(f,σ,ξ) ≤ d(f,σ) +\{ ε
\textbackslash{}over 2\} et donc \textbackslash{}left
\textbar{}\{\textbackslash{}mathop\{∫ \} \}\_\{a\}\^{}\{b\}f −
d(f,σ)\textbackslash{}right \textbar{}≤\textbackslash{}left
\textbar{}\{\textbackslash{}mathop\{∫ \} \}\_\{a\}\^{}\{b\}f −
S(f,σ,ξ)\textbackslash{}right \textbar{} + \textbar{}S(f,σ,ξ) −
d(f,σ)\textbar{} \textless{} ε.

{[}\href{coursse52.html}{next}{]} {[}\href{coursse50.html}{prev}{]}
{[}\href{coursse50.html\#tailcoursse50.html}{prev-tail}{]}
{[}\href{coursse51.html}{front}{]}
{[}\href{coursch10.html\#coursse51.html}{up}{]}

\end{document}
