\section{Polynômes à plusieurs variables}

\subsection{Généralités}

\begin{de}
\index{polynôme à plusieurs variables}
Soit $A$ un anneau commutatif.

$A[X_1,\ldots,X_n] = \left\{\sum_{(k_1,\ldots,k_n)\in\mathbb{N}^n} a_{k_1,\ldots,k_n} X_1^{k_1}\ldots X_n^{k_n} \mid \text{nombre fini de } a_{k_1,\ldots,k_n} \text{ non nuls}\right\}$
\end{de}

\begin{rem}
On a bien entendu un isomorphisme naturel entre $A[X_1,\ldots,X_n]$ et $A[X_1,\ldots,X_{n-1}][X_n]$ qui montre que si $A$ est intègre, il en est de même de $A[X_1,\ldots,X_n]$.
\end{rem}

\begin{prop}[règle de substitution]
Soit $A$ et $B$ deux anneaux commutatifs et $\phi:A \rightarrow B$ un morphisme d'anneaux. Soit $\beta_1,\ldots,\beta_n \in B$. Alors l'application $T_{\phi,\beta_1,\ldots,\beta_n}:A[X_1,\ldots,X_n] \rightarrow B$,

$\sum_{(k_1,\ldots,k_n)\in\mathbb{N}^n} a_{k_1,\ldots,k_n} X_1^{k_1}\ldots X_n^{k_n} \mapsto \sum_{(k_1,\ldots,k_n)\in\mathbb{N}^n} \phi(a_{k_1,\ldots,k_n}) \beta_1^{k_1}\ldots \beta_n^{k_n}$

est un morphisme d'anneaux.
\end{prop}

\subsection{Dérivées partielles, formule de Taylor}

\begin{de}
\index{dérivée partielle}
Soit $P \in A[X_1,\ldots,X_n]$. On note $\frac{\partial P}{\partial X_i}$ la dérivée de $P$ dans $(A[X_1,\ldots,X_{i-1},X_{i+1},\ldots,X_n])[X_i]$.
\end{de}

Par simple calcul sur les monômes on montre alors

\begin{lem}[Schwarz]
Soit $P \in A[X_1,\ldots,X_n]$, $i,j \in [1,n]$. Alors $\frac{\partial}{\partial X_i} (\frac{\partial P}{\partial X_j}) = \frac{\partial}{\partial X_j} (\frac{\partial P}{\partial X_i})$.
\end{lem}

Ceci permet de définir des dérivées partielles itérées $\frac{\partial^k P}{\partial X_1^{k_1}\ldots \partial X_n^{k_n}}$ si $k = k_1 + \ldots + k_n$. On a alors

\begin{thm}[Formule de Taylor]
\index{formule de Taylor}
Soit $K$ un corps de caractéristique 0, $P \in K[X_1,\ldots,X_n]$ et $(a_1,\ldots,a_n) \in K^n$. Alors

$P(X_1 + a_1,\ldots,X_n + a_n) = \sum_{k_1,\ldots,k_n} \frac{1}{k_1!\ldots k_n!} \frac{\partial^{k_1+\ldots+k_n} P}{\partial X_1^{k_1}\ldots \partial X_n^{k_n}} (a_1,\ldots,a_n) X_1^{k_1}\ldots X_n^{k_n}$
\end{thm}

\subsection{Degré total, polynômes homogènes}

\begin{de}
\index{degré d'un polynôme à plusieurs variables}
\index{polynôme homogène}
On définit le degré d'un monôme non nul $a X_1^{k_1}\ldots X_n^{k_n}$ comme étant l'entier $k_1 + \ldots + k_n$. On appelle degré d'un polynôme $P$ non nul, le plus grand des degrés de ses monômes non nuls. On dit qu'un polynôme $P \in K[X_1,\ldots,X_n]$ est homogène de degré $p$ si tous ses monômes non nuls ont le même degré $p$ ou s'il est nul. On notera $H_p(X_1,\ldots,X_n)$ l'espace vectoriel des polynômes homogènes de degré $p$.
\end{de}

\begin{thm}
On a $H_p \cdot H_q \subset H_{p+q}$ et $K[X_1,\ldots,X_n] = \bigoplus_{p\in\mathbb{N}} H_p(X_1,\ldots,X_n)$ (c'est-à-dire que tout polynôme s'écrit de manière unique comme somme finie de polynômes homogènes de degrés distincts).
\end{thm}

\begin{proof}
La décomposition correspond tout simplement au regroupement des termes de même degré au sein d'un polynôme homogène. Cela montre à la fois l'existence et l'unicité de la décomposition.
\end{proof}

\begin{thm}
Soit $K$ un corps. On a $\deg PQ = \deg P + \deg Q$.
\end{thm}

\begin{proof}
On décompose $P$ et $Q$ en polynômes homogènes $P = P_m + \ldots$ et $Q = Q_n + \ldots$ où $P_m$ et $Q_n$ sont les parties homogènes de plus haut degré. Alors $P_m Q_n \neq 0$ et c'est la partie homogène de plus haut degré de $PQ$, d'où le résultat.
\end{proof}

\begin{thm}[Euler]
\index{théorème d'Euler}
Soit $K$ un corps commutatif de caractéristique 0 et $P \in K[X_1,\ldots,X_n]$. On a équivalence de 
\begin{enumerate}
\item $P$ est homogène de degré $p$ 
\item $\sum_{i=1}^n X_i \frac{\partial P}{\partial X_i} = pP$.
\end{enumerate}
\end{thm}

\begin{proof}
On pose $D = \sum_{i=1}^n X_i \frac{\partial}{\partial X_i}$. On démontre (i) $\Rightarrow$ (ii) en calculant sur les monômes et en utilisant la linéarité de $D$. On démontre (ii) $\Rightarrow$ (i) en décomposant $P$ en somme de polynômes homogènes : si $P = P_m + \ldots + P_0$, on a $pP_m + \ldots + pP_0 = pP = DP = DP_m + \ldots + DP_0 = mP_m + \ldots + 0P_0$. Par unicité de la décomposition en polynômes homogènes, on a pour tout $k$, $pP_k = kP_k$ ce qui exige $P_k = 0$ si $k \neq p$. Finalement $P = P_p$ est homogène de degré $p$.
\end{proof}

\subsection{Polynômes symétriques}

\begin{de}
\index{polynôme symétrique}
On dit que $P \in K[X_1,\ldots,X_n]$ est symétrique si pour toute permutation $\sigma$ on a

$P(X_{\sigma(1)},\ldots,X_{\sigma(n)}) = P(X_1,\ldots,X_n)$
\end{de}

\begin{ex}
\index{polynôme symétrique élémentaire}
Pour $1 \leq k \leq n$, on définit les polynômes symétriques élémentaires à $n$ variables $\sigma_k(X_1,\ldots,X_n) = \sum_{1\leq i_1<\cdots<i_k\leq n} X_{i_1}\ldots X_{i_n}$ (homogène de degré $k$, $C_n^k$ monômes). Ces polynômes symétriques vérifient la formule de récurrence (séparer les termes ne contenant pas $X_n$ de ceux contenant $X_n$) :

$\sigma_k(X_1,\ldots,X_n) = \sigma_k(X_1,\ldots,X_{n-1}) + \sigma_{k-1}(X_1,\ldots,X_{n-1})X_n$
\end{ex}

\begin{thm}
$\prod_{i=1}^n (T - X_i) = T^n - \sum_{k=1}^n (-1)^k \sigma_k(X_1,\ldots,X_n) T^{n-k} = T^n - \sigma_1(X_1,\ldots,X_n) T^{n-1} + \ldots + (-1)^n \sigma_n(X_1,\ldots,X_n)$
\end{thm}

\begin{proof}
Par récurrence sur $n$ en utilisant la formule de récurrence vérifiée par les $\sigma_k$.
\end{proof}

\begin{thm}
Soit $P \in K[X]$ scindé sur $K$. On peut donc écrire

$P(X) = a_n X^n + \ldots + a_0 = a_n \prod_{i=1}^n (X - \alpha_i)$

Alors on a, $\forall k \in [1,n]$, $\sigma_k(\alpha_1,\ldots,\alpha_n) = (-1)^k \frac{a_{n-k}}{a_n}$.
\end{thm}

On admettra le résultat suivant

\begin{thm}
\index{théorème fondamental sur les polynômes symétriques}
Soit $P \in K[X_1,\ldots,X_n]$ un polynôme symétrique. Il existe un unique polynôme $Q \in K[X_1,\ldots,X_n]$ tel que $P = Q(\sigma_1,\ldots,\sigma_n)$.
\end{thm}