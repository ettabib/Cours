\documentclass[]{article}
\usepackage[T1]{fontenc}
\usepackage{lmodern}
\usepackage{amssymb,amsmath}
\usepackage{ifxetex,ifluatex}
\usepackage{fixltx2e} % provides \textsubscript
% use upquote if available, for straight quotes in verbatim environments
\IfFileExists{upquote.sty}{\usepackage{upquote}}{}
\ifnum 0\ifxetex 1\fi\ifluatex 1\fi=0 % if pdftex
  \usepackage[utf8]{inputenc}
\else % if luatex or xelatex
  \ifxetex
    \usepackage{mathspec}
    \usepackage{xltxtra,xunicode}
  \else
    \usepackage{fontspec}
  \fi
  \defaultfontfeatures{Mapping=tex-text,Scale=MatchLowercase}
  \newcommand{\euro}{€}
\fi
% use microtype if available
\IfFileExists{microtype.sty}{\usepackage{microtype}}{}
\ifxetex
  \usepackage[setpagesize=false, % page size defined by xetex
              unicode=false, % unicode breaks when used with xetex
              xetex]{hyperref}
\else
  \usepackage[unicode=true]{hyperref}
\fi
\hypersetup{breaklinks=true,
            bookmarks=true,
            pdfauthor={},
            pdftitle={Polynomes `a plusieurs variables},
            colorlinks=true,
            citecolor=blue,
            urlcolor=blue,
            linkcolor=magenta,
            pdfborder={0 0 0}}
\urlstyle{same}  % don't use monospace font for urls
\setlength{\parindent}{0pt}
\setlength{\parskip}{6pt plus 2pt minus 1pt}
\setlength{\emergencystretch}{3em}  % prevent overfull lines
\setcounter{secnumdepth}{0}
 
/* start css.sty */
.cmr-5{font-size:50%;}
.cmr-7{font-size:70%;}
.cmmi-5{font-size:50%;font-style: italic;}
.cmmi-7{font-size:70%;font-style: italic;}
.cmmi-10{font-style: italic;}
.cmsy-5{font-size:50%;}
.cmsy-7{font-size:70%;}
.cmex-7{font-size:70%;}
.cmex-7x-x-71{font-size:49%;}
.msbm-7{font-size:70%;}
.cmtt-10{font-family: monospace;}
.cmti-10{ font-style: italic;}
.cmbx-10{ font-weight: bold;}
.cmr-17x-x-120{font-size:204%;}
.cmsl-10{font-style: oblique;}
.cmti-7x-x-71{font-size:49%; font-style: italic;}
.cmbxti-10{ font-weight: bold; font-style: italic;}
p.noindent { text-indent: 0em }
td p.noindent { text-indent: 0em; margin-top:0em; }
p.nopar { text-indent: 0em; }
p.indent{ text-indent: 1.5em }
@media print {div.crosslinks {visibility:hidden;}}
a img { border-top: 0; border-left: 0; border-right: 0; }
center { margin-top:1em; margin-bottom:1em; }
td center { margin-top:0em; margin-bottom:0em; }
.Canvas { position:relative; }
li p.indent { text-indent: 0em }
.enumerate1 {list-style-type:decimal;}
.enumerate2 {list-style-type:lower-alpha;}
.enumerate3 {list-style-type:lower-roman;}
.enumerate4 {list-style-type:upper-alpha;}
div.newtheorem { margin-bottom: 2em; margin-top: 2em;}
.obeylines-h,.obeylines-v {white-space: nowrap; }
div.obeylines-v p { margin-top:0; margin-bottom:0; }
.overline{ text-decoration:overline; }
.overline img{ border-top: 1px solid black; }
td.displaylines {text-align:center; white-space:nowrap;}
.centerline {text-align:center;}
.rightline {text-align:right;}
div.verbatim {font-family: monospace; white-space: nowrap; text-align:left; clear:both; }
.fbox {padding-left:3.0pt; padding-right:3.0pt; text-indent:0pt; border:solid black 0.4pt; }
div.fbox {display:table}
div.center div.fbox {text-align:center; clear:both; padding-left:3.0pt; padding-right:3.0pt; text-indent:0pt; border:solid black 0.4pt; }
div.minipage{width:100%;}
div.center, div.center div.center {text-align: center; margin-left:1em; margin-right:1em;}
div.center div {text-align: left;}
div.flushright, div.flushright div.flushright {text-align: right;}
div.flushright div {text-align: left;}
div.flushleft {text-align: left;}
.underline{ text-decoration:underline; }
.underline img{ border-bottom: 1px solid black; margin-bottom:1pt; }
.framebox-c, .framebox-l, .framebox-r { padding-left:3.0pt; padding-right:3.0pt; text-indent:0pt; border:solid black 0.4pt; }
.framebox-c {text-align:center;}
.framebox-l {text-align:left;}
.framebox-r {text-align:right;}
span.thank-mark{ vertical-align: super }
span.footnote-mark sup.textsuperscript, span.footnote-mark a sup.textsuperscript{ font-size:80%; }
div.tabular, div.center div.tabular {text-align: center; margin-top:0.5em; margin-bottom:0.5em; }
table.tabular td p{margin-top:0em;}
table.tabular {margin-left: auto; margin-right: auto;}
div.td00{ margin-left:0pt; margin-right:0pt; }
div.td01{ margin-left:0pt; margin-right:5pt; }
div.td10{ margin-left:5pt; margin-right:0pt; }
div.td11{ margin-left:5pt; margin-right:5pt; }
table[rules] {border-left:solid black 0.4pt; border-right:solid black 0.4pt; }
td.td00{ padding-left:0pt; padding-right:0pt; }
td.td01{ padding-left:0pt; padding-right:5pt; }
td.td10{ padding-left:5pt; padding-right:0pt; }
td.td11{ padding-left:5pt; padding-right:5pt; }
table[rules] {border-left:solid black 0.4pt; border-right:solid black 0.4pt; }
.hline hr, .cline hr{ height : 1px; margin:0px; }
.tabbing-right {text-align:right;}
span.TEX {letter-spacing: -0.125em; }
span.TEX span.E{ position:relative;top:0.5ex;left:-0.0417em;}
a span.TEX span.E {text-decoration: none; }
span.LATEX span.A{ position:relative; top:-0.5ex; left:-0.4em; font-size:85%;}
span.LATEX span.TEX{ position:relative; left: -0.4em; }
div.float img, div.float .caption {text-align:center;}
div.figure img, div.figure .caption {text-align:center;}
.marginpar {width:20%; float:right; text-align:left; margin-left:auto; margin-top:0.5em; font-size:85%; text-decoration:underline;}
.marginpar p{margin-top:0.4em; margin-bottom:0.4em;}
.equation td{text-align:center; vertical-align:middle; }
td.eq-no{ width:5%; }
table.equation { width:100%; } 
div.math-display, div.par-math-display{text-align:center;}
math .texttt { font-family: monospace; }
math .textit { font-style: italic; }
math .textsl { font-style: oblique; }
math .textsf { font-family: sans-serif; }
math .textbf { font-weight: bold; }
.partToc a, .partToc, .likepartToc a, .likepartToc {line-height: 200%; font-weight:bold; font-size:110%;}
.chapterToc a, .chapterToc, .likechapterToc a, .likechapterToc, .appendixToc a, .appendixToc {line-height: 200%; font-weight:bold;}
.index-item, .index-subitem, .index-subsubitem {display:block}
.caption td.id{font-weight: bold; white-space: nowrap; }
table.caption {text-align:center;}
h1.partHead{text-align: center}
p.bibitem { text-indent: -2em; margin-left: 2em; margin-top:0.6em; margin-bottom:0.6em; }
p.bibitem-p { text-indent: 0em; margin-left: 2em; margin-top:0.6em; margin-bottom:0.6em; }
.paragraphHead, .likeparagraphHead { margin-top:2em; font-weight: bold;}
.subparagraphHead, .likesubparagraphHead { font-weight: bold;}
.quote {margin-bottom:0.25em; margin-top:0.25em; margin-left:1em; margin-right:1em; text-align:justify;}
.verse{white-space:nowrap; margin-left:2em}
div.maketitle {text-align:center;}
h2.titleHead{text-align:center;}
div.maketitle{ margin-bottom: 2em; }
div.author, div.date {text-align:center;}
div.thanks{text-align:left; margin-left:10%; font-size:85%; font-style:italic; }
div.author{white-space: nowrap;}
.quotation {margin-bottom:0.25em; margin-top:0.25em; margin-left:1em; }
h1.partHead{text-align: center}
.sectionToc, .likesectionToc {margin-left:2em;}
.subsectionToc, .likesubsectionToc {margin-left:4em;}
.subsubsectionToc, .likesubsubsectionToc {margin-left:6em;}
.frenchb-nbsp{font-size:75%;}
.frenchb-thinspace{font-size:75%;}
.figure img.graphics {margin-left:10%;}
/* end css.sty */

\title{Polynomes `a plusieurs variables}
\author{}
\date{}

\begin{document}
\maketitle

\textbf{Warning: \href{http://www.math.union.edu/locate/jsMath}{jsMath}
requires JavaScript to process the mathematics on this page.\\ If your
browser supports JavaScript, be sure it is enabled.}

\begin{center}\rule{3in}{0.4pt}\end{center}

{[}\href{coursse5.html}{prev}{]}
{[}\href{coursse5.html\#tailcoursse5.html}{prev-tail}{]}
{[}\hyperref[tailcoursse6.html]{tail}{]}
{[}\href{coursch2.html\#coursse6.html}{up}{]}

\subsubsection{1.6 Polynômes à plusieurs variables}

\paragraph{1.6.1 Généralités}

Définition~1.6.1 Soit A un anneau commutatif.

\textbackslash{}begin\{eqnarray*\}
A{[}\{X\}\_\{1\},\textbackslash{}mathop\{\textbackslash{}mathop\{\ldots{}\}\},\{X\}\_\{n\}{]}
=\& \textbackslash{}\{\{\textbackslash{}mathop\{∑
\}\}\_\{(\{k\}\_\{1\},\textbackslash{}mathop\{\ldots{}\},\{k\}\_\{n\})∈\{ℕ\}\^{}\{n\}\}\{a\}\_\{\{k\}\_\{1\},\textbackslash{}mathop\{\ldots{}\},\{k\}\_\{n\}\}\{X\}\_\{1\}\^{}\{\{k\}\_\{1\}\}\textbackslash{}mathop\{\ldots{}\}\{X\}\_\{n\}\^{}\{\{k\}\_\{n\}\}\textbackslash{}mathrel\{∣\}\&
\%\& \textbackslash{}\textbackslash{} \& \textbackslash{}text\{nombre
fini de
\}\{a\}\_\{\{k\}\_\{1\},\textbackslash{}mathop\{\textbackslash{}mathop\{\ldots{}\}\},\{k\}\_\{n\}\}\textbackslash{}text\{
non nuls\}\textbackslash{}\} \& \%\& \textbackslash{}\textbackslash{}
\textbackslash{}end\{eqnarray*\}

Remarque~1.6.1 On a bien entendu un isomorphisme naturel entre
A{[}\{X\}\_\{1\},\textbackslash{}mathop\{\textbackslash{}mathop\{\ldots{}\}\},\{X\}\_\{n\}{]}
et
A{[}\{X\}\_\{1\},\textbackslash{}mathop\{\textbackslash{}mathop\{\ldots{}\}\},\{X\}\_\{n−1\}{]}{[}\{X\}\_\{n\}{]}
qui montre que si A est intègre, il en est de même de
A{[}\{X\}\_\{1\},\textbackslash{}mathop\{\textbackslash{}mathop\{\ldots{}\}\},\{X\}\_\{n\}{]}.

Proposition~1.6.1 (règle de substitution). Soit A et B deux anneaux
commutatifs et φ:A → B un morphisme d'anneaux. Soit
\{β\}\_\{1\},\textbackslash{}mathop\{\textbackslash{}mathop\{\ldots{}\}\},\{β\}\_\{n\}
∈ B. Alors l'application
\{T\}\_\{φ,\{β\}\_\{1\},\textbackslash{}mathop\{\textbackslash{}mathop\{\ldots{}\}\},\{β\}\_\{n\}\}:A{[}\{X\}\_\{1\},\textbackslash{}mathop\{\textbackslash{}mathop\{\ldots{}\}\},\{X\}\_\{n\}{]}
→ B,

\textbackslash{}begin\{eqnarray*\} \{\textbackslash{}mathop\{∑
\}\}\_\{(\{k\}\_\{1\},\textbackslash{}mathop\{\ldots{}\},\{k\}\_\{n\})∈\{ℕ\}\^{}\{n\}\}\{a\}\_\{\{k\}\_\{1\},\textbackslash{}mathop\{\ldots{}\},\{k\}\_\{n\}\}\{X\}\_\{1\}\^{}\{\{k\}\_\{1\}
\}\textbackslash{}mathop\{\ldots{}\}\{X\}\_\{n\}\^{}\{\{k\}\_\{n\}
\}\&\& \%\& \textbackslash{}\textbackslash{} \& \&
\textbackslash{}mathrel\{↦\}\{\textbackslash{}mathop\{∑
\}\}\_\{(\{k\}\_\{1\},\textbackslash{}mathop\{\ldots{}\},\{k\}\_\{n\})∈\{ℕ\}\^{}\{n\}\}φ(\{a\}\_\{\{k\}\_\{1\},\textbackslash{}mathop\{\ldots{}\},\{k\}\_\{n\}\})\{β\}\_\{1\}\^{}\{\{k\}\_\{1\}
\}\textbackslash{}mathop\{\ldots{}\}\{β\}\_\{n\}\^{}\{\{k\}\_\{n\}
\}\%\& \textbackslash{}\textbackslash{} \textbackslash{}end\{eqnarray*\}

est un morphisme d'anneaux.

\paragraph{1.6.2 Dérivées partielles, formule de Taylor}

Définition~1.6.2 Soit P ∈
A{[}\{X\}\_\{1\},\textbackslash{}mathop\{\textbackslash{}mathop\{\ldots{}\}\},\{X\}\_\{n\}{]}.
On note \{ ∂P \textbackslash{}over ∂\{X\}\_\{i\}\} la dérivée de P dans
(A{[}\{X\}\_\{1\},\textbackslash{}mathop\{\textbackslash{}mathop\{\ldots{}\}\},\{X\}\_\{i−1\},\{X\}\_\{i+1\},\textbackslash{}mathop\{\textbackslash{}mathop\{\ldots{}\}\},\{X\}\_\{n\}{]}){[}\{X\}\_\{i\}{]}.

Par simple calcul sur les monômes on montre alors

Lemme~1.6.2 (Schwarz) Soit P ∈
A{[}\{X\}\_\{1\},\textbackslash{}mathop\{\textbackslash{}mathop\{\ldots{}\}\},\{X\}\_\{n\}{]},
i,j ∈ {[}1,n{]}. Alors \{ ∂ \textbackslash{}over ∂\{X\}\_\{i\}\} (\{ ∂P
\textbackslash{}over ∂\{X\}\_\{j\}\} ) =\{ ∂ \textbackslash{}over
∂\{X\}\_\{j\}\} (\{ ∂P \textbackslash{}over ∂\{X\}\_\{i\}\} ).

Ceci permet de définir des dérivées partielles itérées \{
\{∂\}\^{}\{k\}P \textbackslash{}over
∂\{X\}\_\{1\}\^{}\{\{k\}\_\{1\}\}\textbackslash{}mathop\{\textbackslash{}mathop\{\ldots{}\}\}∂\{X\}\_\{n\}\^{}\{\{k\}\_\{n\}\}\}
si k = \{k\}\_\{1\} +
\textbackslash{}mathop\{\textbackslash{}mathop\{\ldots{}\}\} +
\{k\}\_\{n\}. On a alors

Théorème~1.6.3 (Formule de Taylor) Soit K un corps de caractéristique 0,
P ∈
K{[}\{X\}\_\{1\},\textbackslash{}mathop\{\textbackslash{}mathop\{\ldots{}\}\},\{X\}\_\{n\}{]}
et
(\{a\}\_\{1\},\textbackslash{}mathop\{\textbackslash{}mathop\{\ldots{}\}\},\{a\}\_\{n\})
∈ \{K\}\^{}\{n\}. Alors

\textbackslash{}begin\{eqnarray*\} P(\{X\}\_\{1\} +
\{a\}\_\{1\},\textbackslash{}mathop\{\textbackslash{}mathop\{\ldots{}\}\},\{X\}\_\{n\}
+ \{a\}\_\{n\})\&\& \%\& \textbackslash{}\textbackslash{} \& \& =\{
\textbackslash{}mathop\{∑
\}\}\_\{\{k\}\_\{1\},\textbackslash{}mathop\{\ldots{}\},\{k\}\_\{n\}\}\{
1 \textbackslash{}over
\{k\}\_\{1\}!\textbackslash{}mathop\{\ldots{}\}\{k\}\_\{n\}!\} \{
\{∂\}\^{}\{\{k\}\_\{1\}+\textbackslash{}mathop\{\ldots{}\}+\{k\}\_\{n\}\}P
\textbackslash{}over
∂\{X\}\_\{1\}\^{}\{\{k\}\_\{1\}\}\textbackslash{}mathop\{\ldots{}\}∂\{X\}\_\{n\}\^{}\{\{k\}\_\{n\}\}\}
(\{a\}\_\{1\},\textbackslash{}mathop\{\ldots{}\},\{a\}\_\{n\})\{X\}\_\{1\}\^{}\{\{k\}\_\{1\}
\}\textbackslash{}mathop\{\ldots{}\}\{X\}\_\{n\}\^{}\{\{k\}\_\{n\}
\}\%\& \textbackslash{}\textbackslash{} \textbackslash{}end\{eqnarray*\}

\paragraph{1.6.3 Degré total, polynômes homogènes}

Définition~1.6.3 On définit le degré d'un monôme non nul
a\{X\}\_\{1\}\^{}\{\{k\}\_\{1\}\}\textbackslash{}mathop\{\textbackslash{}mathop\{\ldots{}\}\}\{X\}\_\{n\}\^{}\{\{k\}\_\{n\}\}
comme étant l'entier \{k\}\_\{1\} +
\textbackslash{}mathop\{\textbackslash{}mathop\{\ldots{}\}\} +
\{k\}\_\{n\}. On appelle degré d'un polynôme P non nul, le plus grand
des degrés de ses monômes non nuls. On dit qu'un polynôme P ∈
K{[}\{X\}\_\{1\},\textbackslash{}mathop\{\textbackslash{}mathop\{\ldots{}\}\},\{X\}\_\{n\}{]}
est homogène de degré p si tous ses monômes non nuls ont le même degré p
ou s'il est nul. On notera
\{H\}\_\{p\}(\{X\}\_\{1\},\textbackslash{}mathop\{\textbackslash{}mathop\{\ldots{}\}\},\{X\}\_\{n\})
l'espace vectoriel des polynômes homogènes de degré p.

Théorème~1.6.4 On a \{H\}\_\{p\}.\{H\}\_\{q\} ⊂ \{H\}\_\{p+q\} et
K{[}\{X\}\_\{1\},\textbackslash{}mathop\{\textbackslash{}mathop\{\ldots{}\}\},\{X\}\_\{n\}{]}
=
\{⊕\}\_\{p∈ℕ\}\{H\}\_\{p\}(\{X\}\_\{1\},\textbackslash{}mathop\{\textbackslash{}mathop\{\ldots{}\}\},\{X\}\_\{n\})
(c'est-à-dire que tout polynôme s'écrit de manière unique comme somme
finie de polynômes homogènes de degrés distincts).

Démonstration La décomposition correspond tout simplement au
regroupement des termes de même degré au sein d'un polynôme homogène.
Cela montre à la fois l'existence et l'unicité de la décomposition.

Corollaire~1.6.5 Soit K un corps. On a \textbackslash{}mathop\{deg\} PQ
=\textbackslash{}mathop\{ deg\} P +\textbackslash{}mathop\{ deg\} Q.

Démonstration On décompose P et Q en polynômes homogènes P =
\{P\}\_\{m\} +
\textbackslash{}mathop\{\textbackslash{}mathop\{\ldots{}\}\} et Q =
\{Q\}\_\{n\} +
\textbackslash{}mathop\{\textbackslash{}mathop\{\ldots{}\}\} où
\{P\}\_\{m\} et \{Q\}\_\{n\} sont les parties homogènes de plus haut
degré. Alors \{P\}\_\{m\}\{Q\}\_\{n\}\textbackslash{}mathrel\{≠\}0 et
c'est la partie homogène de plus haut degré de PQ, d'où le résultat.

Théorème~1.6.6 (Euler). Soit K un corps commutatif de caractéristique 0
et P ∈
K{[}\{X\}\_\{1\},\textbackslash{}mathop\{\textbackslash{}mathop\{\ldots{}\}\},\{X\}\_\{n\}{]}.
On a équivalence de (i) P est homogène de degré p (ii)
\{\textbackslash{}mathop\{\textbackslash{}mathop\{∑ \}\}
\}\_\{i=1\}\^{}\{n\}\{X\}\_\{i\}\{ ∂P \textbackslash{}over
∂\{X\}\_\{i\}\} = pP.

Démonstration On pose D =\{\textbackslash{}mathop\{
\textbackslash{}mathop\{∑ \}\} \}\_\{i=1\}\^{}\{n\}\{X\}\_\{i\}\{ ∂
\textbackslash{}over ∂\{X\}\_\{i\}\} . On démontre (i) ⇒(ii) en
calculant sur les monômes et en utilisant la linéarité de D. On démontre
(ii) ⇒(i) en décomposant P en somme de polynômes homogènes~: si P =
\{P\}\_\{m\} +
\textbackslash{}mathop\{\textbackslash{}mathop\{\ldots{}\}\} +
\{P\}\_\{0\}, on a p\{P\}\_\{m\} +
\textbackslash{}mathop\{\textbackslash{}mathop\{\ldots{}\}\} +
p\{P\}\_\{0\} = pP = DP = D\{P\}\_\{m\} +
\textbackslash{}mathop\{\textbackslash{}mathop\{\ldots{}\}\} +
D\{P\}\_\{0\} = m\{P\}\_\{m\} +
\textbackslash{}mathop\{\textbackslash{}mathop\{\ldots{}\}\} +
0\{P\}\_\{0\}. Par unicité de la décomposition en polynômes homogènes,
on a pour tout k, p\{P\}\_\{k\} = k\{P\}\_\{k\} ce qui exige
\{P\}\_\{k\} = 0 si k\textbackslash{}mathrel\{≠\}p. Finalement P =
\{P\}\_\{p\} est homogène de degré p.

\paragraph{1.6.4 Polynômes symétriques}

Définition~1.6.4 On dit que P ∈
K{[}\{X\}\_\{1\},\textbackslash{}mathop\{\textbackslash{}mathop\{\ldots{}\}\},\{X\}\_\{n\}{]}
est symétrique si pour toute permutation σ on a

P(\{X\}\_\{σ(1)\},\textbackslash{}mathop\{\textbackslash{}mathop\{\ldots{}\}\},\{X\}\_\{σ(n)\})
=
P(\{X\}\_\{1\},\textbackslash{}mathop\{\textbackslash{}mathop\{\ldots{}\}\},\{X\}\_\{n\})

Exemple~1.6.1 Pour 1 ≤ k ≤ n, on définit les polynômes symétriques
élémentaires à n variables
\{σ\}\_\{k\}(\{X\}\_\{1\},\textbackslash{}mathop\{\textbackslash{}mathop\{\ldots{}\}\},\{X\}\_\{n\})
=\{\textbackslash{}mathop\{ \textbackslash{}mathop\{∑ \}\}
\}\_\{1≤\{i\}\_\{1\}\textless{}\textbackslash{}mathrel\{⋯\}\textless{}\{i\}\_\{k\}≤n\}\{X\}\_\{\{i\}\_\{1\}\}\textbackslash{}mathop\{\textbackslash{}mathop\{\ldots{}\}\}\{X\}\_\{\{i\}\_\{n\}\}
(homogène de degré k, \{C\}\_\{n\}\^{}\{k\} monômes). Ces polynômes
symétriques vérifient la formule de récurrence (séparer les termes ne
contenant pas \{X\}\_\{n\} de ceux contenant \{X\}\_\{n\})~:

\{σ\}\_\{k\}(\{X\}\_\{1\},\textbackslash{}mathop\{\textbackslash{}mathop\{\ldots{}\}\},\{X\}\_\{n\})
=
\{σ\}\_\{k\}(\{X\}\_\{1\},\textbackslash{}mathop\{\textbackslash{}mathop\{\ldots{}\}\},\{X\}\_\{n−1\})
+
\{σ\}\_\{k−1\}(\{X\}\_\{1\},\textbackslash{}mathop\{\textbackslash{}mathop\{\ldots{}\}\},\{X\}\_\{n−1\})\{X\}\_\{n\}

Théorème~1.6.7

\textbackslash{}begin\{eqnarray*\} \{\textbackslash{}mathop\{∏
\}\}\_\{i=1\}\^{}\{n\}(T − \{X\}\_\{ i\})\&\& \%\&
\textbackslash{}\textbackslash{} \& =\& \{T\}\^{}\{n\}
−\{\textbackslash{}mathop\{∑
\}\}\_\{k=1\}\^{}\{n\}\{(−1)\}\^{}\{k\}\{σ\}\_\{
k\}(\{X\}\_\{1\},\textbackslash{}mathop\{\ldots{}\},\{X\}\_\{n\})\{T\}\^{}\{n−k\}
\%\& \textbackslash{}\textbackslash{} \& =\& \{T\}\^{}\{n\} − \{σ\}\_\{
1\}(\{X\}\_\{1\},\textbackslash{}mathop\{\textbackslash{}mathop\{\ldots{}\}\},\{X\}\_\{n\})\{T\}\^{}\{n−1\}
+ \textbackslash{}mathop\{\textbackslash{}mathop\{\ldots{}\}\} +
\{(−1)\}\^{}\{n\}\{σ\}\_\{
n\}(\{X\}\_\{1\},\textbackslash{}mathop\{\textbackslash{}mathop\{\ldots{}\}\},\{X\}\_\{n\})\%\&
\textbackslash{}\textbackslash{} \textbackslash{}end\{eqnarray*\}

Démonstration Par récurrence sur n en utilisant la formule de récurrence
vérifiée par les \{σ\}\_\{k\}.

Corollaire~1.6.8 Soit P ∈ K{[}X{]} scindé sur K. On peut donc écrire

P(X) = \{a\}\_\{n\}\{X\}\^{}\{n\} +
\textbackslash{}mathop\{\textbackslash{}mathop\{\ldots{}\}\} + \{a\}\_\{
0\} = \{a\}\_\{n\}\{ \textbackslash{}mathop\{∏ \}\}\_\{i=1\}\^{}\{n\}(X
− \{α\}\_\{ i\})

Alors on a, \textbackslash{}mathop\{∀\}k ∈ {[}1,n{]},
\{σ\}\_\{k\}(\{α\}\_\{1\},\textbackslash{}mathop\{\textbackslash{}mathop\{\ldots{}\}\},\{α\}\_\{n\})
= \{(−1)\}\^{}\{k\}\{ \{a\}\_\{n−k\} \textbackslash{}over \{a\}\_\{n\}\}
.

On admettra le résultat suivant

Théorème~1.6.9 Soit P ∈
K{[}\{X\}\_\{1\},\textbackslash{}mathop\{\textbackslash{}mathop\{\ldots{}\}\},\{X\}\_\{n\}{]}
un polynôme symétrique. Il existe un unique polynôme Q ∈
K{[}\{X\}\_\{1\},\textbackslash{}mathop\{\textbackslash{}mathop\{\ldots{}\}\},\{X\}\_\{n\}{]}
tel que P =
Q(\{σ\}\_\{1\},\textbackslash{}mathop\{\textbackslash{}mathop\{\ldots{}\}\},\{σ\}\_\{n\}).

{[}\href{coursse5.html}{prev}{]}
{[}\href{coursse5.html\#tailcoursse5.html}{prev-tail}{]}
{[}\href{coursse6.html}{front}{]}
{[}\href{coursch2.html\#coursse6.html}{up}{]}

\end{document}
