\documentclass[]{article}
\usepackage[T1]{fontenc}
\usepackage{lmodern}
\usepackage{amssymb,amsmath}
\usepackage{ifxetex,ifluatex}
\usepackage{fixltx2e} % provides \textsubscript
% use upquote if available, for straight quotes in verbatim environments
\IfFileExists{upquote.sty}{\usepackage{upquote}}{}
\ifnum 0\ifxetex 1\fi\ifluatex 1\fi=0 % if pdftex
  \usepackage[utf8]{inputenc}
\else % if luatex or xelatex
  \ifxetex
    \usepackage{mathspec}
    \usepackage{xltxtra,xunicode}
  \else
    \usepackage{fontspec}
  \fi
  \defaultfontfeatures{Mapping=tex-text,Scale=MatchLowercase}
  \newcommand{\euro}{€}
\fi
% use microtype if available
\IfFileExists{microtype.sty}{\usepackage{microtype}}{}
\ifxetex
  \usepackage[setpagesize=false, % page size defined by xetex
              unicode=false, % unicode breaks when used with xetex
              xetex]{hyperref}
\else
  \usepackage[unicode=true]{hyperref}
\fi
\hypersetup{breaklinks=true,
            bookmarks=true,
            pdfauthor={},
            pdftitle={Integrales multiples},
            colorlinks=true,
            citecolor=blue,
            urlcolor=blue,
            linkcolor=magenta,
            pdfborder={0 0 0}}
\urlstyle{same}  % don't use monospace font for urls
\setlength{\parindent}{0pt}
\setlength{\parskip}{6pt plus 2pt minus 1pt}
\setlength{\emergencystretch}{3em}  % prevent overfull lines
\setcounter{secnumdepth}{0}
 
/* start css.sty */
.cmr-5{font-size:50%;}
.cmr-7{font-size:70%;}
.cmmi-5{font-size:50%;font-style: italic;}
.cmmi-7{font-size:70%;font-style: italic;}
.cmmi-10{font-style: italic;}
.cmsy-5{font-size:50%;}
.cmsy-7{font-size:70%;}
.cmex-7{font-size:70%;}
.cmex-7x-x-71{font-size:49%;}
.msbm-7{font-size:70%;}
.cmtt-10{font-family: monospace;}
.cmti-10{ font-style: italic;}
.cmbx-10{ font-weight: bold;}
.cmr-17x-x-120{font-size:204%;}
.cmsl-10{font-style: oblique;}
.cmti-7x-x-71{font-size:49%; font-style: italic;}
.cmbxti-10{ font-weight: bold; font-style: italic;}
p.noindent { text-indent: 0em }
td p.noindent { text-indent: 0em; margin-top:0em; }
p.nopar { text-indent: 0em; }
p.indent{ text-indent: 1.5em }
@media print {div.crosslinks {visibility:hidden;}}
a img { border-top: 0; border-left: 0; border-right: 0; }
center { margin-top:1em; margin-bottom:1em; }
td center { margin-top:0em; margin-bottom:0em; }
.Canvas { position:relative; }
li p.indent { text-indent: 0em }
.enumerate1 {list-style-type:decimal;}
.enumerate2 {list-style-type:lower-alpha;}
.enumerate3 {list-style-type:lower-roman;}
.enumerate4 {list-style-type:upper-alpha;}
div.newtheorem { margin-bottom: 2em; margin-top: 2em;}
.obeylines-h,.obeylines-v {white-space: nowrap; }
div.obeylines-v p { margin-top:0; margin-bottom:0; }
.overline{ text-decoration:overline; }
.overline img{ border-top: 1px solid black; }
td.displaylines {text-align:center; white-space:nowrap;}
.centerline {text-align:center;}
.rightline {text-align:right;}
div.verbatim {font-family: monospace; white-space: nowrap; text-align:left; clear:both; }
.fbox {padding-left:3.0pt; padding-right:3.0pt; text-indent:0pt; border:solid black 0.4pt; }
div.fbox {display:table}
div.center div.fbox {text-align:center; clear:both; padding-left:3.0pt; padding-right:3.0pt; text-indent:0pt; border:solid black 0.4pt; }
div.minipage{width:100%;}
div.center, div.center div.center {text-align: center; margin-left:1em; margin-right:1em;}
div.center div {text-align: left;}
div.flushright, div.flushright div.flushright {text-align: right;}
div.flushright div {text-align: left;}
div.flushleft {text-align: left;}
.underline{ text-decoration:underline; }
.underline img{ border-bottom: 1px solid black; margin-bottom:1pt; }
.framebox-c, .framebox-l, .framebox-r { padding-left:3.0pt; padding-right:3.0pt; text-indent:0pt; border:solid black 0.4pt; }
.framebox-c {text-align:center;}
.framebox-l {text-align:left;}
.framebox-r {text-align:right;}
span.thank-mark{ vertical-align: super }
span.footnote-mark sup.textsuperscript, span.footnote-mark a sup.textsuperscript{ font-size:80%; }
div.tabular, div.center div.tabular {text-align: center; margin-top:0.5em; margin-bottom:0.5em; }
table.tabular td p{margin-top:0em;}
table.tabular {margin-left: auto; margin-right: auto;}
div.td00{ margin-left:0pt; margin-right:0pt; }
div.td01{ margin-left:0pt; margin-right:5pt; }
div.td10{ margin-left:5pt; margin-right:0pt; }
div.td11{ margin-left:5pt; margin-right:5pt; }
table[rules] {border-left:solid black 0.4pt; border-right:solid black 0.4pt; }
td.td00{ padding-left:0pt; padding-right:0pt; }
td.td01{ padding-left:0pt; padding-right:5pt; }
td.td10{ padding-left:5pt; padding-right:0pt; }
td.td11{ padding-left:5pt; padding-right:5pt; }
table[rules] {border-left:solid black 0.4pt; border-right:solid black 0.4pt; }
.hline hr, .cline hr{ height : 1px; margin:0px; }
.tabbing-right {text-align:right;}
span.TEX {letter-spacing: -0.125em; }
span.TEX span.E{ position:relative;top:0.5ex;left:-0.0417em;}
a span.TEX span.E {text-decoration: none; }
span.LATEX span.A{ position:relative; top:-0.5ex; left:-0.4em; font-size:85%;}
span.LATEX span.TEX{ position:relative; left: -0.4em; }
div.float img, div.float .caption {text-align:center;}
div.figure img, div.figure .caption {text-align:center;}
.marginpar {width:20%; float:right; text-align:left; margin-left:auto; margin-top:0.5em; font-size:85%; text-decoration:underline;}
.marginpar p{margin-top:0.4em; margin-bottom:0.4em;}
.equation td{text-align:center; vertical-align:middle; }
td.eq-no{ width:5%; }
table.equation { width:100%; } 
div.math-display, div.par-math-display{text-align:center;}
math .texttt { font-family: monospace; }
math .textit { font-style: italic; }
math .textsl { font-style: oblique; }
math .textsf { font-family: sans-serif; }
math .textbf { font-weight: bold; }
.partToc a, .partToc, .likepartToc a, .likepartToc {line-height: 200%; font-weight:bold; font-size:110%;}
.chapterToc a, .chapterToc, .likechapterToc a, .likechapterToc, .appendixToc a, .appendixToc {line-height: 200%; font-weight:bold;}
.index-item, .index-subitem, .index-subsubitem {display:block}
.caption td.id{font-weight: bold; white-space: nowrap; }
table.caption {text-align:center;}
h1.partHead{text-align: center}
p.bibitem { text-indent: -2em; margin-left: 2em; margin-top:0.6em; margin-bottom:0.6em; }
p.bibitem-p { text-indent: 0em; margin-left: 2em; margin-top:0.6em; margin-bottom:0.6em; }
.paragraphHead, .likeparagraphHead { margin-top:2em; font-weight: bold;}
.subparagraphHead, .likesubparagraphHead { font-weight: bold;}
.quote {margin-bottom:0.25em; margin-top:0.25em; margin-left:1em; margin-right:1em; text-align:justify;}
.verse{white-space:nowrap; margin-left:2em}
div.maketitle {text-align:center;}
h2.titleHead{text-align:center;}
div.maketitle{ margin-bottom: 2em; }
div.author, div.date {text-align:center;}
div.thanks{text-align:left; margin-left:10%; font-size:85%; font-style:italic; }
div.author{white-space: nowrap;}
.quotation {margin-bottom:0.25em; margin-top:0.25em; margin-left:1em; }
h1.partHead{text-align: center}
.sectionToc, .likesectionToc {margin-left:2em;}
.subsectionToc, .likesubsectionToc {margin-left:4em;}
.subsubsectionToc, .likesubsubsectionToc {margin-left:6em;}
.frenchb-nbsp{font-size:75%;}
.frenchb-thinspace{font-size:75%;}
.figure img.graphics {margin-left:10%;}
/* end css.sty */

\title{Integrales multiples}
\author{}
\date{}

\begin{document}
\maketitle

\textbf{Warning: \href{http://www.math.union.edu/locate/jsMath}{jsMath}
requires JavaScript to process the mathematics on this page.\\ If your
browser supports JavaScript, be sure it is enabled.}

\begin{center}\rule{3in}{0.4pt}\end{center}

{[}\href{coursse106.html}{next}{]} {[}\href{coursse104.html}{prev}{]}
{[}\href{coursse104.html\#tailcoursse104.html}{prev-tail}{]}
{[}\hyperref[tailcoursse105.html]{tail}{]}
{[}\href{coursch21.html\#coursse105.html}{up}{]}

\subsubsection{20.2 Intégrales multiples}

\paragraph{20.2.1 Pavés et subdivisions. Ensembles négligeables}

Définition~20.2.1 On appelle pavé de \{ℝ\}\^{}\{n\} tout ensemble P de
la forme P = {[}\{a\}\_\{1\},\{b\}\_\{1\}{]}
×\textbackslash{}mathrel\{⋯\} × {[}\{a\}\_\{n\},\{b\}\_\{n\}{]}. On
notera mesure du pavé P le nombre réel positif m(P)
=\{\textbackslash{}mathop\{ \textbackslash{}mathop\{∏ \}\}
\}\_\{i=1\}\^{}\{n\}(\{b\}\_\{i\} − \{a\}\_\{i\}).

Remarque~20.2.1 Un pavé est clairement compact.

Définition~20.2.2 Soit P = {[}\{a\}\_\{1\},\{b\}\_\{1\}{]}
×\textbackslash{}mathrel\{⋯\} × {[}\{a\}\_\{n\},\{b\}\_\{n\}{]} un pavé
de \{ℝ\}\^{}\{n\}. On appelle subdivision de P toute famille σ =
(\{σ\}\_\{1\},\textbackslash{}mathop\{\textbackslash{}mathop\{\ldots{}\}\},\{σ\}\_\{n\})
où chaque \{σ\}\_\{i\} est une subdivision de
{[}\{a\}\_\{i\},\{b\}\_\{i\}{]}. Si \{σ\}\_\{i\} =
\{(\{a\}\_\{i,j\})\}\_\{1≤j≤\{n\}\_\{i\}\}, les sous pavés
\{P\}\_\{\{j\}\_\{1\},\textbackslash{}mathop\{\textbackslash{}mathop\{\ldots{}\}\},\{j\}\_\{n\}\}
= {[}\{a\}\_\{1,\{j\}\_\{1\}−1\},\{a\}\_\{1,\{j\}\_\{1\}\}{]}
×\textbackslash{}mathrel\{⋯\} ×
{[}\{a\}\_\{n,\{j\}\_\{n\}−1\},\{a\}\_\{n,\{j\}\_\{n\}\}{]} sont appelés
les sous pavés de la subdivision. On appelle pas de la subdivision σ =
(\{σ\}\_\{1\},\textbackslash{}mathop\{\textbackslash{}mathop\{\ldots{}\}\},\{σ\}\_\{n\})
le plus grand des diamètres des sous pavés de σ.

Définition~20.2.3 Un sous-ensemble A de \{ℝ\}\^{}\{n\} est dit
négligeable (au sens de Riemann) si quelque soit ε \textgreater{} 0, il
existe une famille finie de pavés \{(\{P\}\_\{i\})\}\_\{1≤i≤N\}
vérifiant

\begin{itemize}
\itemsep1pt\parskip0pt\parsep0pt
\item
  (i) A ⊂\{\textbackslash{}mathop\{\textbackslash{}mathop\{⋃ \}\}
  \}\_\{i=1\}\^{}\{N\}\{P\}\_\{i\}
\item
  (ii) \{\textbackslash{}mathop\{\textbackslash{}mathop\{∑ \}\}
  \}\_\{i=1\}\^{}\{N\}m(\{P\}\_\{i\}) ≤ ε.
\end{itemize}

Proposition~20.2.1

\begin{itemize}
\itemsep1pt\parskip0pt\parsep0pt
\item
  (i) tout ensemble négligeable est borné
\item
  (ii) si A est négligeable et B ⊂ A, alors B est aussi négligeable
\item
  (iii) une partie A est négligeable si et seulement
  si~\textbackslash{}overline\{A\} est négligeable
\item
  (iv) toute réunion finie de parties négligeables est négligeable
\end{itemize}

Démonstration Tout est à peu près évident. L'affirmation (iii) résulte
de ce que, si A ⊂\{\textbackslash{}mathop\{\textbackslash{}mathop\{⋃
\}\} \}\_\{i=1\}\^{}\{N\}\{P\}\_\{i\}, alors on a aussi
\textbackslash{}overline\{A\}
⊂\{\textbackslash{}mathop\{\textbackslash{}mathop\{⋃ \}\}
\}\_\{i=1\}\^{}\{N\}\{P\}\_\{i\} puisque
\{\textbackslash{}mathop\{\textbackslash{}mathop\{⋃ \}\}
\}\_\{i=1\}\^{}\{N\}\{P\}\_\{i\} est fermé.

Théorème~20.2.2 Soit Q un pavé de \{ℝ\}\^{}\{n−1\} et f une application
continue de Q dans ℝ. Alors le graphe de f est une partie négligeable de
\{ℝ\}\^{}\{n\}.

Démonstration Puisque f est continue sur le compact Q, elle est
uniformément continue. Soit donc ε \textgreater{} 0. Il existe η
\textgreater{} 0 tel que \textbackslash{}mathop\{∀\}x,x' ∈ Q,
\textbackslash{}\textbar{}x − x'\textbackslash{}\textbar{} \textless{} η
⇒\textbar{}f(x) − f(x')\textbar{} \textless{}\{ ε \textbackslash{}over
2m(Q)\} . Soit alors σ une subdivision de Q de pas inférieur strictement
à η et \{(\{Q\}\_\{i\})\}\_\{1≤i≤N\} les sous pavés de la subdivision.
Choisissons un point \{x\}\_\{i\} dans chaque \{Q\}\_\{i\} et posons
\{P\}\_\{i\} = \{Q\}\_\{i\} × {[}f(\{x\}\_\{i\}) −\{ ε
\textbackslash{}over 2m(Q)\} ,f(\{x\}\_\{i\}) +\{ ε \textbackslash{}over
2m(Q)\} {]}. Chaque \{P\}\_\{i\} est un pavé de \{ℝ\}\^{}\{n\} et
m(\{P\}\_\{i\}) =\{ ε \textbackslash{}over m(Q)\} m(\{Q\}\_\{i\}) si
bien que \textbackslash{}mathop\{\textbackslash{}mathop\{∑ \}\}
m(\{P\}\_\{i\}) =\{ ε \textbackslash{}over m(Q)\}
\textbackslash{}mathop\{ \textbackslash{}mathop\{∑ \}\} m(\{Q\}\_\{i\})
=\{ ε \textbackslash{}over m(Q)\} m(Q) = ε. Mais soit (x,f(x)) un point
du graphe de f avec x ∈ Q~; soit i tel que x ∈ \{Q\}\_\{i\}~; on a alors
\textbackslash{}\textbar{}x − \{x\}\_\{i\}\textbackslash{}\textbar{} ≤
δ(σ) \textless{} η et donc \textbar{}f(x) − f(\{x\}\_\{i\})\textbar{}≤\{
ε \textbackslash{}over 2m(Q)\} , soit encore f(x) ∈ {[}f(\{x\}\_\{i\})
−\{ ε \textbackslash{}over 2m(Q)\} ,f(\{x\}\_\{i\}) +\{ ε
\textbackslash{}over 2m(Q)\} {]}, si bien que (x,f(x)) ∈ \{P\}\_\{i\}.
On en déduit que le graphe de f est contenu dans
\{\textbackslash{}mathop\{\textbackslash{}mathop\{⋃ \}\}
\}\_\{i=1\}\^{}\{N\}\{P\}\_\{i\} avec
\textbackslash{}mathop\{\textbackslash{}mathop\{∑ \}\} m(\{P\}\_\{i\}) =
ε. Donc le graphe de f est négligeable.

Corollaire~20.2.3 Toute partie de \{ℝ\}\^{}\{n\} qui est une réunion
finie de graphes d'applications continues sur des pavés

\textbackslash{}begin\{eqnarray*\}
(\{x\}\_\{1\},\textbackslash{}mathop\{\textbackslash{}mathop\{\ldots{}\}\},\{x\}\_\{i−1\},\{x\}\_\{i+1\},\textbackslash{}mathop\{\textbackslash{}mathop\{\ldots{}\}\},\{x\}\_\{n\})\&\&
\%\& \textbackslash{}\textbackslash{} \& \textbackslash{}mathrel\{↦\}\&
\{x\}\_\{i\} =
f(\{x\}\_\{1\},\textbackslash{}mathop\{\textbackslash{}mathop\{\ldots{}\}\},\{x\}\_\{i−1\},\{x\}\_\{i+1\},\textbackslash{}mathop\{\textbackslash{}mathop\{\ldots{}\}\},\{x\}\_\{n\})\%\&
\textbackslash{}\textbackslash{} \textbackslash{}end\{eqnarray*\}

est négligeable.

\paragraph{20.2.2 Intégrales multiples sur un pavé de \{ℝ\}\^{}\{n\}}

Remarque~20.2.2 On appelle point de discontinuité de f tout point où f
n'est pas continue. Si f est une application de l'espace métrique X dans
l'espace métrique E, on notera
\textbackslash{}mathop\{\textbackslash{}mathrm\{Disc\}\} (f) l'ensemble
des points de discontinuité de f.

Proposition~20.2.4 Soit E un espace vectoriel normé de dimension finie
et P un pavé de \{ℝ\}\^{}\{n\}. L'ensemble ℰ des fonctions f : P → E
bornées et dont l'ensemble des points de discontinuité est négligeable
est un sous-espace vectoriel de l'espace des applications de P dans E.

Démonstration Cet ensemble est évidemment non vide (il contient par
exemple toutes les fonctions continues sur P)~; si f et g sont dans ℰ,
si α et β sont des scalaires, on a évidemment αf + βg qui est bornée et
de plus \textbackslash{}mathop\{\textbackslash{}mathrm\{Disc\}\} (αf +
βg) ⊂\textbackslash{}mathop\{\textbackslash{}mathrm\{Disc\}\} (f)
∪\textbackslash{}mathop\{\textbackslash{}mathrm\{Disc\}\} (g) (puisque
là où f et g sont toutes deux continues, αf + βg l'est également), donc
\textbackslash{}mathop\{\textbackslash{}mathrm\{Disc\}\} (αf + βg) est
négligeable.

On admettra le théorème suivant

Théorème~20.2.5 Il existe une application qui à toute fonction f bornée
de P dans E dont l'ensemble des points de discontinuité est négligeable
associe un élément de E noté \{\textbackslash{}mathop\{∫ \} \}\_\{P\}f
vérifiant les propriétés suivantes

\begin{itemize}
\itemsep1pt\parskip0pt\parsep0pt
\item
  (i) l'application
  f\textbackslash{}mathrel\{↦\}\{\textbackslash{}mathop\{∫ \} \}\_\{P\}f
  est linéaire (\{\textbackslash{}mathop\{∫ \} \}\_\{P\}(αf + βg) =
  α\{\textbackslash{}mathop\{∫ \} \}\_\{P\}f +
  β\{\textbackslash{}mathop\{∫ \} \}\_\{P\}g)
\item
  (ii) \textbackslash{}\textbar{}\{\textbackslash{}mathop\{∫ \}
  \}\_\{P\}f\textbackslash{}\textbar{} ≤\{\textbackslash{}mathop\{∫ \}
  \}\_\{P\}\textbackslash{}\textbar{}f\textbackslash{}\textbar{}
\item
  (iii) \{\textbackslash{}mathop\{∫ \} \}\_\{P\}1 = m(P)
\item
  (iv) si P est la réunion de deux pavés \{P\}\_\{1\} et \{P\}\_\{2\}
  dont l'intersection est contenue dans l'intersection des frontières,
  alors \{\textbackslash{}mathop\{∫ \} \}\_\{P\}f
  =\{\textbackslash{}mathop\{∫ \} \}\_\{\{P\}\_\{1\}\}f
  +\{\textbackslash{}mathop\{∫ \} \}\_\{\{P\}\_\{2\}\}f
\item
  (v) si \textbackslash{}\{x ∈
  P\textbackslash{}mathrel\{∣\}f(x)\textbackslash{}mathrel\{≠\}0\textbackslash{}\}
  est négligeable, alors \{\textbackslash{}mathop\{∫ \} \}\_\{P\}f = 0.
\end{itemize}

Proposition~20.2.6

\begin{itemize}
\itemsep1pt\parskip0pt\parsep0pt
\item
  (i) Si f : P → ℝ est une fonction bornée dont l'ensemble des points de
  discontinuité est négligeable et si f est positive, alors
  \{\textbackslash{}mathop\{∫ \} \}\_\{p\}f ≥ 0
\item
  (ii) Si f,g : P → ℝ sont deux fonctions bornées dont l'ensemble des
  points de discontinuité est négligeable et si f ≤ g alors
  \{\textbackslash{}mathop\{∫ \} \}\_\{P\}f ≤\{\textbackslash{}mathop\{∫
  \} \}\_\{P\}g
\item
  (iii) Si f : P → ℝ est une fonction bornée dont l'ensemble des points
  de discontinuité est négligeable, alors
  \textbackslash{}\textbar{}\{\textbackslash{}mathop\{∫ \}
  \}\_\{P\}f\textbackslash{}\textbar{} ≤
  m(P)\{\textbackslash{}mathop\{sup\}\}\_\{x∈P\}\textbackslash{}\textbar{}f(x)\textbackslash{}\textbar{}.
\end{itemize}

Démonstration (i) On a \{\textbackslash{}mathop\{∫ \} \}\_\{P\}f
=\{\textbackslash{}mathop\{∫ \}
\}\_\{P\}\textbar{}f\textbar{}≥\textbackslash{}left
\textbar{}\{\textbackslash{}mathop\{∫ \} \}\_\{P\}f\textbackslash{}right
\textbar{} d'où \{\textbackslash{}mathop\{∫ \} \}\_\{P\}f ≥ 0

(ii) On a \{\textbackslash{}mathop\{∫ \} \}\_\{P\}g
−\{\textbackslash{}mathop\{∫ \} \}\_\{P\}f =\{\textbackslash{}mathop\{∫
\} \}\_\{P\}(g − f) ≥ 0 puisque g − f ≥ 0

(iii) Si M =\{\textbackslash{}mathop\{
sup\}\}\_\{x∈P\}\textbackslash{}\textbar{}f(x)\textbackslash{}\textbar{},
on a \textbackslash{}\textbar{}\{\textbackslash{}mathop\{∫ \}
\}\_\{P\}f\textbackslash{}\textbar{} ≤\{\textbackslash{}mathop\{∫ \}
\}\_\{P\}\textbackslash{}\textbar{}f\textbackslash{}\textbar{}
≤\{\textbackslash{}mathop\{∫ \} \}\_\{P\}M =
M\{\textbackslash{}mathop\{∫ \} \}\_\{P\}1 = Mm(P)

Définition~20.2.4 Soit f : P → E une application, soit σ une subdivision
de P, \{(\{P\}\_\{i\})\}\_\{1≤i≤N\} les sous pavés de la subdivision et
pour chaque i ∈ {[}1,N{]}, \{x\}\_\{i\} un point de \{P\}\_\{i\}~; la
somme S(f,σ,x) =\{\textbackslash{}mathop\{ \textbackslash{}mathop\{∑
\}\} \}\_\{i=1\}\^{}\{N\}m(\{P\}\_\{i\})f(\{x\}\_\{i\}) sera appelée une
somme de Riemann associée à la subdivision σ et à la famille x =
\{(\{x\}\_\{i\})\}\_\{1≤i≤N\}.

On admettra également le résultat suivant

Théorème~20.2.7 Soit f une fonction bornée de P dans E dont l'ensemble
des points de discontinuité est négligeable. Alors, pour tout ε
\textgreater{} 0, il existe un réel η \textgreater{} 0 tel que pour
toute subdivision σ de P de pas plus petit que η et pour toute famille x
= (\{x\}\_\{i\}) associée, on a
\textbackslash{}\textbar{}\{\textbackslash{}mathop\{∫ \} \}\_\{P\}f −
S(f,σ,x)\textbackslash{}\textbar{} \textless{} ε.

Remarque~20.2.3 Comme dans le cas des fonctions d'une variable, on peut
aussi définir, lorsque f est à valeurs réelles des sommes de Darboux
supérieure et inférieure D(f,σ) =\{\textbackslash{}mathop\{
\textbackslash{}mathop\{∑ \}\}
\}\_\{i=1\}\^{}\{N\}m(\{P\}\_\{i\})\{M\}\_\{i\} et
\{\textbackslash{}mathop\{\textbackslash{}mathop\{∑ \}\}
\}\_\{i=1\}\^{}\{N\}m(\{P\}\_\{i\})\{m\}\_\{i\} où \{M\}\_\{i\}
=\{\textbackslash{}mathop\{ sup\}\}\_\{x∈\{P\}\_\{i\}\}f(t) et
\{m\}\_\{i\} =\{\textbackslash{}mathop\{ inf\}
\}\_\{x∈\{P\}\_\{i\}\}f(t). La même démonstration que pour les fonctions
d'une variable montre alors que ces sommes de Darboux inférieure et
supérieure tendent vers l'intégrale de f sur P lorsque le pas de la
subdivision tend vers 0.

\paragraph{20.2.3 Intégrales multiples sur une partie de \{ℝ\}\^{}\{n\}}

Soit A une partie de \{ℝ\}\^{}\{n\} bornée de frontière négligeable et f
: A → E continue et bornée. Soit P un pavé de \{ℝ\}\^{}\{n\} contenant A
et \{f\}\^{}\{∗\} l'application de P dans E définie par

\{ f\}\^{}\{∗\}(x) = \textbackslash{}left \textbackslash{}\{
\textbackslash{}cases\{ f(x)\&si x ∈ A \textbackslash{}cr 0 \&si x ∈ A
\} \textbackslash{}right .

L'ensemble des points de discontinuité de \{f\}\^{}\{∗\} est contenu
dans la frontière de A car si x est dans l'intérieur de A, la fonction
\{f\}\^{}\{∗\} coïncide avec la fonction continue f sur tout un
voisinage de x, donc est continue au point x et si x appartient à
l'intérieur de P ∖ A, alors \{f\}\^{}\{∗\} coïncide avec la fonction
nulle sur tout un voisinage de x, donc est continue au point x. On peut
donc définir \{\textbackslash{}mathop\{∫ \} \}\_\{P\}\{f\}\^{}\{∗\}. De
plus, si \{P\}\_\{1\} et \{P\}\_\{2\} sont deux pavés contenant A, ''la
fonction \{f\}\^{}\{∗\}'' est nulle sur \{P\}\_\{2\} ∖ \{P\}\_\{1\} et
sur \{P\}\_\{1\} ∖ \{P\}\_\{2\}, si bien que \{\textbackslash{}mathop\{∫
\} \}\_\{\{P\}\_\{1\}\}\{f\}\^{}\{∗\} =\{\textbackslash{}mathop\{∫ \}
\}\_\{\{P\}\_\{2\}\}\{f\}\^{}\{∗\}, ce qui montre que
\{\textbackslash{}mathop\{∫ \} \}\_\{P\}\{f\}\^{}\{∗\} ne dépend pas du
choix du pavé P contenant A.

Définition~20.2.5 Soit A une partie de \{ℝ\}\^{}\{n\} bornée de
frontière négligeable et f : A → E continue et bornée~; on posera
\{\textbackslash{}mathop\{∫ \} \}\_\{A\}f =\{\textbackslash{}mathop\{∫
\} \}\_\{P\}\{f\}\^{}\{∗\}.

Théorème~20.2.8

\begin{itemize}
\itemsep1pt\parskip0pt\parsep0pt
\item
  (i) L'application
  f\textbackslash{}mathrel\{↦\}\{\textbackslash{}mathop\{∫ \} \}\_\{A\}f
  est linéaire (\{\textbackslash{}mathop\{∫ \} \}\_\{A\}(αf + βg) =
  α\{\textbackslash{}mathop\{∫ \} \}\_\{A\}f +
  β\{\textbackslash{}mathop\{∫ \} \}\_\{A\}g)
\item
  (ii) \textbackslash{}\textbar{}\{\textbackslash{}mathop\{∫ \}
  \}\_\{A\}f\textbackslash{}\textbar{} ≤\{\textbackslash{}mathop\{∫ \}
  \}\_\{A\}\textbackslash{}\textbar{}f\textbackslash{}\textbar{}
\item
  (iii) si \{A\}\_\{1\} ∩ \{A\}\_\{2\} est négligeable, alors
  \{\textbackslash{}mathop\{∫ \} \}\_\{\{A\}\_\{1\}∪\{A\}\_\{2\}\}f
  =\{\textbackslash{}mathop\{∫ \} \}\_\{\{A\}\_\{1\}\}f
  +\{\textbackslash{}mathop\{∫ \} \}\_\{\{A\}\_\{2\}\}f
\item
  (iv) si A est négligeable, alors \{\textbackslash{}mathop\{∫ \}
  \}\_\{A\}f = 0.
\end{itemize}

Démonstration (i) et (ii) résultent de \{(αf + βg)\}\^{}\{∗\} =
α\{f\}\^{}\{∗\} + β\{g\}\^{}\{∗\} et de
\textbackslash{}\textbar{}\{f\}\^{}\{∗\}\textbackslash{}\textbar{}
=\textbackslash{}\textbar{} \{f\textbackslash{}\textbar{}\}\^{}\{∗\} qui
sont évidents. Pour (iii) si on considère P un pavé contenant
\{A\}\_\{1\} ∪ \{A\}\_\{2\}, \{f\}\^{}\{∗\} l'extension de f de
\{A\}\_\{1\} ∪ \{A\}\_\{2\} à P, \{f\}\_\{1\}\^{}\{∗\} et
\{f\}\_\{2\}\^{}\{∗\} les extensions de f depuis respectivement
\{A\}\_\{1\} et \{A\}\_\{2\} à P, on a \{f\}\^{}\{∗\} =
\{f\}\_\{1\}\^{}\{∗\} + \{f\}\_\{2\}\^{}\{∗\} sauf sur \{A\}\_\{1\} ∩
\{A\}\_\{2\}. On a donc \{f\}\^{}\{∗\} = \{f\}\_\{1\}\^{}\{∗\} +
\{f\}\_\{2\}\^{}\{∗\} + g où g est une fonction nulle sauf sur
l'ensemble négligeable \{A\}\_\{1\} ∩ \{A\}\_\{2\}~; on en déduit que

\textbackslash{}begin\{eqnarray*\} \{\textbackslash{}mathop\{∫ \}
\}\_\{\{A\}\_\{1\}∪\{A\}\_\{2\}\}f\& =\& \{\textbackslash{}mathop\{∫ \}
\}\_\{P\}\{f\}\^{}\{∗\} =\{\textbackslash{}mathop\{∫ \}
\}\_\{P\}(\{f\}\_\{1\}\^{}\{∗\} + \{f\}\_\{ 2\}\^{}\{∗\} + g) \%\&
\textbackslash{}\textbackslash{} \& =\& \{\textbackslash{}mathop\{∫ \}
\}\_\{P\}\{f\}\_\{1\}\^{}\{∗\} +\{\textbackslash{}mathop\{∫ \}
\}\_\{P\}\{f\}\_\{2\}\^{}\{∗\} +\{\textbackslash{}mathop\{∫ \}
\}\_\{P\}g =\{\textbackslash{}mathop\{∫ \}
\}\_\{P\}\{f\}\_\{1\}\^{}\{∗\} +\{\textbackslash{}mathop\{∫ \}
\}\_\{P\}\{f\}\_\{2\}\^{}\{∗\}\%\& \textbackslash{}\textbackslash{} \&
=\& \{\textbackslash{}mathop\{∫ \} \}\_\{\{A\}\_\{1\}\}f
+\{\textbackslash{}mathop\{∫ \} \}\_\{\{A\}\_\{2\}\}f \%\&
\textbackslash{}\textbackslash{} \textbackslash{}end\{eqnarray*\}

Quand à (iv), il est évident puisque \textbackslash{}\{x ∈
P\textbackslash{}mathrel\{∣\}\{f\}\^{}\{∗\}(x)\textbackslash{}mathrel\{≠\}0\textbackslash{}\}
⊂ A

\paragraph{20.2.4 Mesure d'un sous-ensemble borné de \{ℝ\}\^{}\{n\}}

Définition~20.2.6 On dit qu'une partie A de \{ℝ\}\^{}\{n\} est quarrable
si elle est bornée et de frontière négligeable.

Définition~20.2.7 Soit A une partie quarrable de \{ℝ\}\^{}\{n\}~; on
appelle mesure de A le nombre réel positif m(A)
=\{\textbackslash{}mathop\{∫ \} \}\_\{A\}1.

Proposition~20.2.9 Soit A une partie quarrable de \{ℝ\}\^{}\{n\}~; alors
l'intérieur et l'adhérence de A sont aussi quarrables et ont la même
mesure.

Démonstration En effet, la frontière de l'intérieur et de l'adhérence de
A sont contenues dans la frontière de A qui est négligeable. De plus
\{\textbackslash{}mathop\{∫ \} \}\_\{\textbackslash{}overline\{A\}\}1
=\{\textbackslash{}mathop\{∫ \} \}\_\{\textbackslash{}overline\{A\}∖A\}1
+\{\textbackslash{}mathop\{∫ \} \}\_\{A\}1 =\{\textbackslash{}mathop\{∫
\} \}\_\{A\}1 puisque \textbackslash{}overline\{A\} ∖ A est contenu dans
la frontière de A et donc est négligeable~; la démonstration est
similaire pour l'intérieur.

Proposition~20.2.10 Si f : A → E est une fonction continue bornée sur
l'ensemble quarrable A, alors
\textbackslash{}\textbar{}\{\textbackslash{}mathop\{∫ \}
\}\_\{A\}f\textbackslash{}\textbar{} ≤
m(A)\{\textbackslash{}mathop\{sup\}\}\_\{x∈A\}\textbackslash{}\textbar{}f(x)\textbackslash{}\textbar{}.

Démonstration Si M =\{\textbackslash{}mathop\{
sup\}\}\_\{x∈A\}\textbackslash{}\textbar{}f(x)\textbackslash{}\textbar{},
on a \textbackslash{}\textbar{}\{\textbackslash{}mathop\{∫ \}
\}\_\{A\}f\textbackslash{}\textbar{} ≤\{\textbackslash{}mathop\{∫ \}
\}\_\{A\}\textbackslash{}\textbar{}f\textbackslash{}\textbar{}
≤\{\textbackslash{}mathop\{∫ \} \}\_\{A\}M =
M\{\textbackslash{}mathop\{∫ \} \}\_\{A\}1 = Mm(A)

{[}\href{coursse106.html}{next}{]} {[}\href{coursse104.html}{prev}{]}
{[}\href{coursse104.html\#tailcoursse104.html}{prev-tail}{]}
{[}\href{coursse105.html}{front}{]}
{[}\href{coursch21.html\#coursse105.html}{up}{]}

\end{document}
