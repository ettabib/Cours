\documentclass[]{article}
\usepackage[T1]{fontenc}
\usepackage{lmodern}
\usepackage{amssymb,amsmath}
\usepackage{ifxetex,ifluatex}
\usepackage{fixltx2e} % provides \textsubscript
% use upquote if available, for straight quotes in verbatim environments
\IfFileExists{upquote.sty}{\usepackage{upquote}}{}
\ifnum 0\ifxetex 1\fi\ifluatex 1\fi=0 % if pdftex
  \usepackage[utf8]{inputenc}
\else % if luatex or xelatex
  \ifxetex
    \usepackage{mathspec}
    \usepackage{xltxtra,xunicode}
  \else
    \usepackage{fontspec}
  \fi
  \defaultfontfeatures{Mapping=tex-text,Scale=MatchLowercase}
  \newcommand{\euro}{€}
\fi
% use microtype if available
\IfFileExists{microtype.sty}{\usepackage{microtype}}{}
\ifxetex
  \usepackage[setpagesize=false, % page size defined by xetex
              unicode=false, % unicode breaks when used with xetex
              xetex]{hyperref}
\else
  \usepackage[unicode=true]{hyperref}
\fi
\hypersetup{breaklinks=true,
            bookmarks=true,
            pdfauthor={},
            pdftitle={Integrales multiples},
            colorlinks=true,
            citecolor=blue,
            urlcolor=blue,
            linkcolor=magenta,
            pdfborder={0 0 0}}
\urlstyle{same}  % don't use monospace font for urls
\setlength{\parindent}{0pt}
\setlength{\parskip}{6pt plus 2pt minus 1pt}
\setlength{\emergencystretch}{3em}  % prevent overfull lines
\setcounter{secnumdepth}{0}
 
/* start css.sty */
.cmr-5{font-size:50%;}
.cmr-7{font-size:70%;}
.cmmi-5{font-size:50%;font-style: italic;}
.cmmi-7{font-size:70%;font-style: italic;}
.cmmi-10{font-style: italic;}
.cmsy-5{font-size:50%;}
.cmsy-7{font-size:70%;}
.cmex-7{font-size:70%;}
.cmex-7x-x-71{font-size:49%;}
.msbm-7{font-size:70%;}
.cmtt-10{font-family: monospace;}
.cmti-10{ font-style: italic;}
.cmbx-10{ font-weight: bold;}
.cmr-17x-x-120{font-size:204%;}
.cmsl-10{font-style: oblique;}
.cmti-7x-x-71{font-size:49%; font-style: italic;}
.cmbxti-10{ font-weight: bold; font-style: italic;}
p.noindent { text-indent: 0em }
td p.noindent { text-indent: 0em; margin-top:0em; }
p.nopar { text-indent: 0em; }
p.indent{ text-indent: 1.5em }
@media print {div.crosslinks {visibility:hidden;}}
a img { border-top: 0; border-left: 0; border-right: 0; }
center { margin-top:1em; margin-bottom:1em; }
td center { margin-top:0em; margin-bottom:0em; }
.Canvas { position:relative; }
li p.indent { text-indent: 0em }
.enumerate1 {list-style-type:decimal;}
.enumerate2 {list-style-type:lower-alpha;}
.enumerate3 {list-style-type:lower-roman;}
.enumerate4 {list-style-type:upper-alpha;}
div.newtheorem { margin-bottom: 2em; margin-top: 2em;}
.obeylines-h,.obeylines-v {white-space: nowrap; }
div.obeylines-v p { margin-top:0; margin-bottom:0; }
.overline{ text-decoration:overline; }
.overline img{ border-top: 1px solid black; }
td.displaylines {text-align:center; white-space:nowrap;}
.centerline {text-align:center;}
.rightline {text-align:right;}
div.verbatim {font-family: monospace; white-space: nowrap; text-align:left; clear:both; }
.fbox {padding-left:3.0pt; padding-right:3.0pt; text-indent:0pt; border:solid black 0.4pt; }
div.fbox {display:table}
div.center div.fbox {text-align:center; clear:both; padding-left:3.0pt; padding-right:3.0pt; text-indent:0pt; border:solid black 0.4pt; }
div.minipage{width:100%;}
div.center, div.center div.center {text-align: center; margin-left:1em; margin-right:1em;}
div.center div {text-align: left;}
div.flushright, div.flushright div.flushright {text-align: right;}
div.flushright div {text-align: left;}
div.flushleft {text-align: left;}
.underline{ text-decoration:underline; }
.underline img{ border-bottom: 1px solid black; margin-bottom:1pt; }
.framebox-c, .framebox-l, .framebox-r { padding-left:3.0pt; padding-right:3.0pt; text-indent:0pt; border:solid black 0.4pt; }
.framebox-c {text-align:center;}
.framebox-l {text-align:left;}
.framebox-r {text-align:right;}
span.thank-mark{ vertical-align: super }
span.footnote-mark sup.textsuperscript, span.footnote-mark a sup.textsuperscript{ font-size:80%; }
div.tabular, div.center div.tabular {text-align: center; margin-top:0.5em; margin-bottom:0.5em; }
table.tabular td p{margin-top:0em;}
table.tabular {margin-left: auto; margin-right: auto;}
div.td00{ margin-left:0pt; margin-right:0pt; }
div.td01{ margin-left:0pt; margin-right:5pt; }
div.td10{ margin-left:5pt; margin-right:0pt; }
div.td11{ margin-left:5pt; margin-right:5pt; }
table[rules] {border-left:solid black 0.4pt; border-right:solid black 0.4pt; }
td.td00{ padding-left:0pt; padding-right:0pt; }
td.td01{ padding-left:0pt; padding-right:5pt; }
td.td10{ padding-left:5pt; padding-right:0pt; }
td.td11{ padding-left:5pt; padding-right:5pt; }
table[rules] {border-left:solid black 0.4pt; border-right:solid black 0.4pt; }
.hline hr, .cline hr{ height : 1px; margin:0px; }
.tabbing-right {text-align:right;}
span.TEX {letter-spacing: -0.125em; }
span.TEX span.E{ position:relative;top:0.5ex;left:-0.0417em;}
a span.TEX span.E {text-decoration: none; }
span.LATEX span.A{ position:relative; top:-0.5ex; left:-0.4em; font-size:85%;}
span.LATEX span.TEX{ position:relative; left: -0.4em; }
div.float img, div.float .caption {text-align:center;}
div.figure img, div.figure .caption {text-align:center;}
.marginpar {width:20%; float:right; text-align:left; margin-left:auto; margin-top:0.5em; font-size:85%; text-decoration:underline;}
.marginpar p{margin-top:0.4em; margin-bottom:0.4em;}
.equation td{text-align:center; vertical-align:middle; }
td.eq-no{ width:5%; }
table.equation { width:100%; } 
div.math-display, div.par-math-display{text-align:center;}
math .texttt { font-family: monospace; }
math .textit { font-style: italic; }
math .textsl { font-style: oblique; }
math .textsf { font-family: sans-serif; }
math .textbf { font-weight: bold; }
.partToc a, .partToc, .likepartToc a, .likepartToc {line-height: 200%; font-weight:bold; font-size:110%;}
.chapterToc a, .chapterToc, .likechapterToc a, .likechapterToc, .appendixToc a, .appendixToc {line-height: 200%; font-weight:bold;}
.index-item, .index-subitem, .index-subsubitem {display:block}
.caption td.id{font-weight: bold; white-space: nowrap; }
table.caption {text-align:center;}
h1.partHead{text-align: center}
p.bibitem { text-indent: -2em; margin-left: 2em; margin-top:0.6em; margin-bottom:0.6em; }
p.bibitem-p { text-indent: 0em; margin-left: 2em; margin-top:0.6em; margin-bottom:0.6em; }
.paragraphHead, .likeparagraphHead { margin-top:2em; font-weight: bold;}
.subparagraphHead, .likesubparagraphHead { font-weight: bold;}
.quote {margin-bottom:0.25em; margin-top:0.25em; margin-left:1em; margin-right:1em; text-align:\jmathustify;}
.verse{white-space:nowrap; margin-left:2em}
div.maketitle {text-align:center;}
h2.titleHead{text-align:center;}
div.maketitle{ margin-bottom: 2em; }
div.author, div.date {text-align:center;}
div.thanks{text-align:left; margin-left:10%; font-size:85%; font-style:italic; }
div.author{white-space: nowrap;}
.quotation {margin-bottom:0.25em; margin-top:0.25em; margin-left:1em; }
h1.partHead{text-align: center}
.sectionToc, .likesectionToc {margin-left:2em;}
.subsectionToc, .likesubsectionToc {margin-left:4em;}
.subsubsectionToc, .likesubsubsectionToc {margin-left:6em;}
.frenchb-nbsp{font-size:75%;}
.frenchb-thinspace{font-size:75%;}
.figure img.graphics {margin-left:10%;}
/* end css.sty */

\title{Integrales multiples}
\author{}
\date{}

\begin{document}
\maketitle

\textbf{Warning: 
requires JavaScript to process the mathematics on this page.\\ If your
browser supports JavaScript, be sure it is enabled.}

\begin{center}\rule{3in}{0.4pt}\end{center}

{[}
{[}
{[}{]}
{[}

\subsubsection{20.2 Intégrales multiples}

\paragraph{20.2.1 Pavés et subdivisions. Ensembles négligeables}

Définition~20.2.1 On appelle pavé de \mathbb{R}~^n tout ensemble P de
la forme P = {[}a\_1,b\_1{]}
\times⋯ \times {[}a\_n,b\_n{]}. On
notera mesure du pavé P le nombre réel positif m(P)
= \∏ ~
\_i=1^n(b\_i - a\_i).

Remarque~20.2.1 Un pavé est clairement compact.

Définition~20.2.2 Soit P = {[}a\_1,b\_1{]}
\times⋯ \times {[}a\_n,b\_n{]} un pavé
de \mathbb{R}~^n. On appelle subdivision de P toute famille \sigma =
(\sigma\_1,\\ldots,\sigma\_n~)
où chaque \sigma\_i est une subdivision de
{[}a\_i,b\_i{]}. Si \sigma\_i =
(a\_i,\jmath)\_1\leq\jmath\leqn\_i, les sous pavés
P\_\jmath\_1,\\ldots,\jmath\_n~
= {[}a\_1,\jmath\_1-1,a\_1,\jmath\_1{]}
\times⋯ \times
{[}a\_n,\jmath\_n-1,a\_n,\jmath\_n{]} sont appelés
les sous pavés de la subdivision. On appelle pas de la subdivision \sigma =
(\sigma\_1,\\ldots,\sigma\_n~)
le plus grand des diamètres des sous pavés de \sigma.

Définition~20.2.3 Un sous-ensemble A de \mathbb{R}~^n est dit
négligeable (au sens de Riemann) si quelque soit \epsilon \textgreater{} 0, il
existe une famille finie de pavés (P\_i)\_1\leqi\leqN
vérifiant

\begin{itemize}
\itemsep1pt\parskip0pt\parsep0pt
\item
  (i) A \subset~\⋃ ~
  \_i=1^NP\_i
\item
  (ii) \\sum ~
  \_i=1^Nm(P\_i) \leq \epsilon.
\end{itemize}

Proposition~20.2.1

\begin{itemize}
\itemsep1pt\parskip0pt\parsep0pt
\item
  (i) tout ensemble négligeable est borné
\item
  (ii) si A est négligeable et B \subset~ A, alors B est aussi négligeable
\item
  (iii) une partie A est négligeable si et seulement
  si~\overlineA est négligeable
\item
  (iv) toute réunion finie de parties négligeables est négligeable
\end{itemize}

Démonstration Tout est à peu près évident. L'affirmation (iii) résulte
de ce que, si A \subset~\\⋃
 \_i=1^NP\_i, alors on a aussi
\overlineA
\subset~\⋃ ~
\_i=1^NP\_i puisque
\⋃ ~
\_i=1^NP\_i est fermé.

Théorème~20.2.2 Soit Q un pavé de \mathbb{R}~^n-1 et f une application
continue de Q dans \mathbb{R}~. Alors le graphe de f est une partie négligeable de
\mathbb{R}~^n.

Démonstration Puisque f est continue sur le compact Q, elle est
uniformément continue. Soit donc \epsilon \textgreater{} 0. Il existe \eta
\textgreater{} 0 tel que \forall~~x,x' \in Q,
\\textbar{}x - x'\\textbar{} \textless{} \eta
\rigtharrow~\textbar{}f(x) - f(x')\textbar{} \textless{} \epsilon \over
2m(Q) . Soit alors \sigma une subdivision de Q de pas inférieur strictement
à \eta et (Q\_i)\_1\leqi\leqN les sous pavés de la subdivision.
Choisissons un point x\_i dans chaque Q\_i et posons
P\_i = Q\_i \times {[}f(x\_i) - \epsilon
\over 2m(Q) ,f(x\_i) + \epsilon \over
2m(Q) {]}. Chaque P\_i est un pavé de \mathbb{R}~^n et
m(P\_i) = \epsilon \over m(Q) m(Q\_i) si
bien que \\sum ~
m(P\_i) = \epsilon \over m(Q)
 \\sum  m(Q\_i~)
= \epsilon \over m(Q) m(Q) = \epsilon. Mais soit (x,f(x)) un point
du graphe de f avec x \in Q~; soit i tel que x \in Q\_i~; on a alors
\\textbar{}x - x\_i\\textbar{} \leq
\delta(\sigma) \textless{} \eta et donc \textbar{}f(x) - f(x\_i)\textbar{}\leq
\epsilon \over 2m(Q) , soit encore f(x) \in {[}f(x\_i)
- \epsilon \over 2m(Q) ,f(x\_i) + \epsilon
\over 2m(Q) {]}, si bien que (x,f(x)) \in P\_i.
On en déduit que le graphe de f est contenu dans
\⋃ ~
\_i=1^NP\_i avec
\\sum  m(P\_i~) =
\epsilon. Donc le graphe de f est négligeable.

Corollaire~20.2.3 Toute partie de \mathbb{R}~^n qui est une réunion
finie de graphes d'applications continues sur des pavés

\begin{align*}
(x\_1,\\ldots,x\_i-1,x\_i+1,\\\ldots,x\_n~)&&
\%& \\ & \mapsto~&
x\_i =
f(x\_1,\\ldots,x\_i-1,x\_i+1,\\\ldots,x\_n~)\%&
\\ \end{align*}

est négligeable.

\paragraph{20.2.2 Intégrales multiples sur un pavé de \mathbb{R}~^n}

Remarque~20.2.2 On appelle point de discontinuité de f tout point où f
n'est pas continue. Si f est une application de l'espace métrique X dans
l'espace métrique E, on notera
\mathrmDisc~ (f) l'ensemble
des points de discontinuité de f.

Proposition~20.2.4 Soit E un espace vectoriel normé de dimension finie
et P un pavé de \mathbb{R}~^n. L'ensemble \mathcal{E} des fonctions f : P \rightarrow~ E
bornées et dont l'ensemble des points de discontinuité est négligeable
est un sous-espace vectoriel de l'espace des applications de P dans E.

Démonstration Cet ensemble est évidemment non vide (il contient par
exemple toutes les fonctions continues sur P)~; si f et g sont dans \mathcal{E},
si \alpha~ et \beta~ sont des scalaires, on a évidemment \alpha~f + \beta~g qui est bornée et
de plus \mathrmDisc~ (\alpha~f +
\beta~g) \subset~\mathrmDisc~ (f)
\cup\mathrmDisc~ (g) (puisque
là où f et g sont toutes deux continues, \alpha~f + \beta~g l'est également), donc
\mathrmDisc~ (\alpha~f + \beta~g) est
négligeable.

On admettra le théorème suivant

Théorème~20.2.5 Il existe une application qui à toute fonction f bornée
de P dans E dont l'ensemble des points de discontinuité est négligeable
associe un élément de E noté \int  \_P~f
vérifiant les propriétés suivantes

\begin{itemize}
\itemsep1pt\parskip0pt\parsep0pt
\item
  (i) l'application
  f\mapsto~\int  \_P~f
  est linéaire (\int  \_P~(\alpha~f + \beta~g) =
  \alpha~\int  \_P~f +
  \beta~\int  \_P~g)
\item
  (ii) \\textbar{}\int ~
  \_Pf\\textbar{} \leq\int ~
  \_P\\textbar{}f\\textbar{}
\item
  (iii) \int  \_P~1 = m(P)
\item
  (iv) si P est la réunion de deux pavés P\_1 et P\_2
  dont l'intersection est contenue dans l'intersection des frontières,
  alors \int  \_P~f
  =\int  \_P\_1~f
  +\int  \_P\_2~f
\item
  (v) si \x \in
  P∣f(x)\mathrel\neq~0\
  est négligeable, alors \int  \_P~f = 0.
\end{itemize}

Proposition~20.2.6

\begin{itemize}
\itemsep1pt\parskip0pt\parsep0pt
\item
  (i) Si f : P \rightarrow~ \mathbb{R}~ est une fonction bornée dont l'ensemble des points de
  discontinuité est négligeable et si f est positive, alors
  \int  \_p~f ≥ 0
\item
  (ii) Si f,g : P \rightarrow~ \mathbb{R}~ sont deux fonctions bornées dont l'ensemble des
  points de discontinuité est négligeable et si f \leq g alors
  \int  \_P~f \leq\\int
   \_Pg
\item
  (iii) Si f : P \rightarrow~ \mathbb{R}~ est une fonction bornée dont l'ensemble des points
  de discontinuité est négligeable, alors
  \\textbar{}\int ~
  \_Pf\\textbar{} \leq
  m(P)sup\_x\inP~\\textbar{}f(x)\\textbar{}.
\end{itemize}

Démonstration (i) On a \int  \_P~f
=\int ~
\_P\textbar{}f\textbar{}≥\left
\textbar{}\int  \_P~f\right
\textbar{} d'où \int  \_P~f ≥ 0

(ii) On a \int  \_P~g
-\int  \_P~f =\\int
 \_P(g - f) ≥ 0 puisque g - f ≥ 0

(iii) Si M =\
sup\_x\inP\\textbar{}f(x)\\textbar{},
on a \\textbar{}\int ~
\_Pf\\textbar{} \leq\int ~
\_P\\textbar{}f\\textbar{}
\leq\int  \_P~M =
M\int  \_P~1 = Mm(P)

Définition~20.2.4 Soit f : P \rightarrow~ E une application, soit \sigma une subdivision
de P, (P\_i)\_1\leqi\leqN les sous pavés de la subdivision et
pour chaque i \in {[}1,N{]}, x\_i un point de P\_i~; la
somme S(f,\sigma,x) =\ \\sum
 \_i=1^Nm(P\_i)f(x\_i) sera appelée une
somme de Riemann associée à la subdivision \sigma et à la famille x =
(x\_i)\_1\leqi\leqN.

On admettra également le résultat suivant

Théorème~20.2.7 Soit f une fonction bornée de P dans E dont l'ensemble
des points de discontinuité est négligeable. Alors, pour tout \epsilon
\textgreater{} 0, il existe un réel \eta \textgreater{} 0 tel que pour
toute subdivision \sigma de P de pas plus petit que \eta et pour toute famille x
= (x\_i) associée, on a
\\textbar{}\int  \_P~f -
S(f,\sigma,x)\\textbar{} \textless{} \epsilon.

Remarque~20.2.3 Comme dans le cas des fonctions d'une variable, on peut
aussi définir, lorsque f est à valeurs réelles des sommes de Darboux
supérieure et inférieure D(f,\sigma) =\
\sum ~
\_i=1^Nm(P\_i)M\_i et
\\sum ~
\_i=1^Nm(P\_i)m\_i où M\_i
= sup\_x\inP\_i~f(t) et
m\_i = inf~
\_x\inP\_if(t). La même démonstration que pour les fonctions
d'une variable montre alors que ces sommes de Darboux inférieure et
supérieure tendent vers l'intégrale de f sur P lorsque le pas de la
subdivision tend vers 0.

\paragraph{20.2.3 Intégrales multiples sur une partie de \mathbb{R}~^n}

Soit A une partie de \mathbb{R}~^n bornée de frontière négligeable et f
: A \rightarrow~ E continue et bornée. Soit P un pavé de \mathbb{R}~^n contenant A
et f^∗ l'application de P dans E définie par

 f^∗(x) = \left \
\cases f(x)&si x \in A \cr 0 &si x \in A
 \right .

L'ensemble des points de discontinuité de f^∗ est contenu
dans la frontière de A car si x est dans l'intérieur de A, la fonction
f^∗ coïncide avec la fonction continue f sur tout un
voisinage de x, donc est continue au point x et si x appartient à
l'intérieur de P \diagdown A, alors f^∗ coïncide avec la fonction
nulle sur tout un voisinage de x, donc est continue au point x. On peut
donc définir \int  \_Pf^∗~. De
plus, si P\_1 et P\_2 sont deux pavés contenant A, ''la
fonction f^∗'' est nulle sur P\_2 \diagdown P\_1 et
sur P\_1 \diagdown P\_2, si bien que \\int
 \_P\_1f^∗ =\int ~
\_P\_2f^∗, ce qui montre que
\int  \_Pf^∗~ ne dépend pas du
choix du pavé P contenant A.

Définition~20.2.5 Soit A une partie de \mathbb{R}~^n bornée de
frontière négligeable et f : A \rightarrow~ E continue et bornée~; on posera
\int  \_A~f =\\int
 \_Pf^∗.

Théorème~20.2.8

\begin{itemize}
\itemsep1pt\parskip0pt\parsep0pt
\item
  (i) L'application
  f\mapsto~\int  \_A~f
  est linéaire (\int  \_A~(\alpha~f + \beta~g) =
  \alpha~\int  \_A~f +
  \beta~\int  \_A~g)
\item
  (ii) \\textbar{}\int ~
  \_Af\\textbar{} \leq\int ~
  \_A\\textbar{}f\\textbar{}
\item
  (iii) si A\_1 \bigcap A\_2 est négligeable, alors
  \int  \_A\_1\cupA\_2~f
  =\int  \_A\_1~f
  +\int  \_A\_2~f
\item
  (iv) si A est négligeable, alors \int ~
  \_Af = 0.
\end{itemize}

Démonstration (i) et (ii) résultent de (\alpha~f + \beta~g)^∗ =
\alpha~f^∗ + \beta~g^∗ et de
\\textbar{}f^∗\\textbar{}
=\\textbar{} f\\textbar{}^∗ qui
sont évidents. Pour (iii) si on considère P un pavé contenant
A\_1 \cup A\_2, f^∗ l'extension de f de
A\_1 \cup A\_2 à P, f\_1^∗ et
f\_2^∗ les extensions de f depuis respectivement
A\_1 et A\_2 à P, on a f^∗ =
f\_1^∗ + f\_2^∗ sauf sur A\_1 \bigcap
A\_2. On a donc f^∗ = f\_1^∗ +
f\_2^∗ + g où g est une fonction nulle sauf sur
l'ensemble négligeable A\_1 \bigcap A\_2~; on en déduit que

\begin{align*} \int ~
\_A\_1\cupA\_2f& =& \int ~
\_Pf^∗ =\int ~
\_P(f\_1^∗ + f\_ 2^∗ + g) \%&
\\ & =& \int ~
\_Pf\_1^∗ +\int ~
\_Pf\_2^∗ +\int ~
\_Pg =\int ~
\_Pf\_1^∗ +\int ~
\_Pf\_2^∗\%& \\ &
=& \int  \_A\_1~f
+\int  \_A\_2~f \%&
\\ \end{align*}

Quand à (iv), il est évident puisque \x \in
P∣f^∗(x)\mathrel\neq~0\
\subset~ A

\paragraph{20.2.4 Mesure d'un sous-ensemble borné de \mathbb{R}~^n}

Définition~20.2.6 On dit qu'une partie A de \mathbb{R}~^n est quarrable
si elle est bornée et de frontière négligeable.

Définition~20.2.7 Soit A une partie quarrable de \mathbb{R}~^n~; on
appelle mesure de A le nombre réel positif m(A)
=\int  \_A~1.

Proposition~20.2.9 Soit A une partie quarrable de \mathbb{R}~^n~; alors
l'intérieur et l'adhérence de A sont aussi quarrables et ont la même
mesure.

Démonstration En effet, la frontière de l'intérieur et de l'adhérence de
A sont contenues dans la frontière de A qui est négligeable. De plus
\int  \_\overlineA~1
=\int  \_\overlineA\diagdownA~1
+\int  \_A~1 =\\int
 \_A1 puisque \overlineA \diagdown A est contenu dans
la frontière de A et donc est négligeable~; la démonstration est
similaire pour l'intérieur.

Proposition~20.2.10 Si f : A \rightarrow~ E est une fonction continue bornée sur
l'ensemble quarrable A, alors
\\textbar{}\int ~
\_Af\\textbar{} \leq
m(A)sup\_x\inA~\\textbar{}f(x)\\textbar{}.

Démonstration Si M =\
sup\_x\inA\\textbar{}f(x)\\textbar{},
on a \\textbar{}\int ~
\_Af\\textbar{} \leq\int ~
\_A\\textbar{}f\\textbar{}
\leq\int  \_A~M =
M\int  \_A~1 = Mm(A)

{[}
{[}
{[}
{[}

\end{document}
