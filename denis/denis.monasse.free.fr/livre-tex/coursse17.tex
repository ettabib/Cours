\documentclass[]{article}
\usepackage[T1]{fontenc}
\usepackage{lmodern}
\usepackage{amssymb,amsmath}
\usepackage{ifxetex,ifluatex}
\usepackage{fixltx2e} % provides \textsubscript
% use upquote if available, for straight quotes in verbatim environments
\IfFileExists{upquote.sty}{\usepackage{upquote}}{}
\ifnum 0\ifxetex 1\fi\ifluatex 1\fi=0 % if pdftex
  \usepackage[utf8]{inputenc}
\else % if luatex or xelatex
  \ifxetex
    \usepackage{mathspec}
    \usepackage{xltxtra,xunicode}
  \else
    \usepackage{fontspec}
  \fi
  \defaultfontfeatures{Mapping=tex-text,Scale=MatchLowercase}
  \newcommand{\euro}{€}
\fi
% use microtype if available
\IfFileExists{microtype.sty}{\usepackage{microtype}}{}
\ifxetex
  \usepackage[setpagesize=false, % page size defined by xetex
              unicode=false, % unicode breaks when used with xetex
              xetex]{hyperref}
\else
  \usepackage[unicode=true]{hyperref}
\fi
\hypersetup{breaklinks=true,
            bookmarks=true,
            pdfauthor={},
            pdftitle={A propos de Jordan},
            colorlinks=true,
            citecolor=blue,
            urlcolor=blue,
            linkcolor=magenta,
            pdfborder={0 0 0}}
\urlstyle{same}  % don't use monospace font for urls
\setlength{\parindent}{0pt}
\setlength{\parskip}{6pt plus 2pt minus 1pt}
\setlength{\emergencystretch}{3em}  % prevent overfull lines
\setcounter{secnumdepth}{0}
 
/* start css.sty */
.cmr-5{font-size:50%;}
.cmr-7{font-size:70%;}
.cmmi-5{font-size:50%;font-style: italic;}
.cmmi-7{font-size:70%;font-style: italic;}
.cmmi-10{font-style: italic;}
.cmsy-5{font-size:50%;}
.cmsy-7{font-size:70%;}
.cmex-7{font-size:70%;}
.cmex-7x-x-71{font-size:49%;}
.msbm-7{font-size:70%;}
.cmtt-10{font-family: monospace;}
.cmti-10{ font-style: italic;}
.cmbx-10{ font-weight: bold;}
.cmr-17x-x-120{font-size:204%;}
.cmsl-10{font-style: oblique;}
.cmti-7x-x-71{font-size:49%; font-style: italic;}
.cmbxti-10{ font-weight: bold; font-style: italic;}
p.noindent { text-indent: 0em }
td p.noindent { text-indent: 0em; margin-top:0em; }
p.nopar { text-indent: 0em; }
p.indent{ text-indent: 1.5em }
@media print {div.crosslinks {visibility:hidden;}}
a img { border-top: 0; border-left: 0; border-right: 0; }
center { margin-top:1em; margin-bottom:1em; }
td center { margin-top:0em; margin-bottom:0em; }
.Canvas { position:relative; }
li p.indent { text-indent: 0em }
.enumerate1 {list-style-type:decimal;}
.enumerate2 {list-style-type:lower-alpha;}
.enumerate3 {list-style-type:lower-roman;}
.enumerate4 {list-style-type:upper-alpha;}
div.newtheorem { margin-bottom: 2em; margin-top: 2em;}
.obeylines-h,.obeylines-v {white-space: nowrap; }
div.obeylines-v p { margin-top:0; margin-bottom:0; }
.overline{ text-decoration:overline; }
.overline img{ border-top: 1px solid black; }
td.displaylines {text-align:center; white-space:nowrap;}
.centerline {text-align:center;}
.rightline {text-align:right;}
div.verbatim {font-family: monospace; white-space: nowrap; text-align:left; clear:both; }
.fbox {padding-left:3.0pt; padding-right:3.0pt; text-indent:0pt; border:solid black 0.4pt; }
div.fbox {display:table}
div.center div.fbox {text-align:center; clear:both; padding-left:3.0pt; padding-right:3.0pt; text-indent:0pt; border:solid black 0.4pt; }
div.minipage{width:100%;}
div.center, div.center div.center {text-align: center; margin-left:1em; margin-right:1em;}
div.center div {text-align: left;}
div.flushright, div.flushright div.flushright {text-align: right;}
div.flushright div {text-align: left;}
div.flushleft {text-align: left;}
.underline{ text-decoration:underline; }
.underline img{ border-bottom: 1px solid black; margin-bottom:1pt; }
.framebox-c, .framebox-l, .framebox-r { padding-left:3.0pt; padding-right:3.0pt; text-indent:0pt; border:solid black 0.4pt; }
.framebox-c {text-align:center;}
.framebox-l {text-align:left;}
.framebox-r {text-align:right;}
span.thank-mark{ vertical-align: super }
span.footnote-mark sup.textsuperscript, span.footnote-mark a sup.textsuperscript{ font-size:80%; }
div.tabular, div.center div.tabular {text-align: center; margin-top:0.5em; margin-bottom:0.5em; }
table.tabular td p{margin-top:0em;}
table.tabular {margin-left: auto; margin-right: auto;}
div.td00{ margin-left:0pt; margin-right:0pt; }
div.td01{ margin-left:0pt; margin-right:5pt; }
div.td10{ margin-left:5pt; margin-right:0pt; }
div.td11{ margin-left:5pt; margin-right:5pt; }
table[rules] {border-left:solid black 0.4pt; border-right:solid black 0.4pt; }
td.td00{ padding-left:0pt; padding-right:0pt; }
td.td01{ padding-left:0pt; padding-right:5pt; }
td.td10{ padding-left:5pt; padding-right:0pt; }
td.td11{ padding-left:5pt; padding-right:5pt; }
table[rules] {border-left:solid black 0.4pt; border-right:solid black 0.4pt; }
.hline hr, .cline hr{ height : 1px; margin:0px; }
.tabbing-right {text-align:right;}
span.TEX {letter-spacing: -0.125em; }
span.TEX span.E{ position:relative;top:0.5ex;left:-0.0417em;}
a span.TEX span.E {text-decoration: none; }
span.LATEX span.A{ position:relative; top:-0.5ex; left:-0.4em; font-size:85%;}
span.LATEX span.TEX{ position:relative; left: -0.4em; }
div.float img, div.float .caption {text-align:center;}
div.figure img, div.figure .caption {text-align:center;}
.marginpar {width:20%; float:right; text-align:left; margin-left:auto; margin-top:0.5em; font-size:85%; text-decoration:underline;}
.marginpar p{margin-top:0.4em; margin-bottom:0.4em;}
.equation td{text-align:center; vertical-align:middle; }
td.eq-no{ width:5%; }
table.equation { width:100%; } 
div.math-display, div.par-math-display{text-align:center;}
math .texttt { font-family: monospace; }
math .textit { font-style: italic; }
math .textsl { font-style: oblique; }
math .textsf { font-family: sans-serif; }
math .textbf { font-weight: bold; }
.partToc a, .partToc, .likepartToc a, .likepartToc {line-height: 200%; font-weight:bold; font-size:110%;}
.chapterToc a, .chapterToc, .likechapterToc a, .likechapterToc, .appendixToc a, .appendixToc {line-height: 200%; font-weight:bold;}
.index-item, .index-subitem, .index-subsubitem {display:block}
.caption td.id{font-weight: bold; white-space: nowrap; }
table.caption {text-align:center;}
h1.partHead{text-align: center}
p.bibitem { text-indent: -2em; margin-left: 2em; margin-top:0.6em; margin-bottom:0.6em; }
p.bibitem-p { text-indent: 0em; margin-left: 2em; margin-top:0.6em; margin-bottom:0.6em; }
.paragraphHead, .likeparagraphHead { margin-top:2em; font-weight: bold;}
.subparagraphHead, .likesubparagraphHead { font-weight: bold;}
.quote {margin-bottom:0.25em; margin-top:0.25em; margin-left:1em; margin-right:1em; text-align:justify;}
.verse{white-space:nowrap; margin-left:2em}
div.maketitle {text-align:center;}
h2.titleHead{text-align:center;}
div.maketitle{ margin-bottom: 2em; }
div.author, div.date {text-align:center;}
div.thanks{text-align:left; margin-left:10%; font-size:85%; font-style:italic; }
div.author{white-space: nowrap;}
.quotation {margin-bottom:0.25em; margin-top:0.25em; margin-left:1em; }
h1.partHead{text-align: center}
.sectionToc, .likesectionToc {margin-left:2em;}
.subsectionToc, .likesubsectionToc {margin-left:4em;}
.subsubsectionToc, .likesubsubsectionToc {margin-left:6em;}
.frenchb-nbsp{font-size:75%;}
.frenchb-thinspace{font-size:75%;}
.figure img.graphics {margin-left:10%;}
/* end css.sty */

\title{A propos de Jordan}
\author{}
\date{}

\begin{document}
\maketitle

\textbf{Warning: \href{http://www.math.union.edu/locate/jsMath}{jsMath}
requires JavaScript to process the mathematics on this page.\\ If your
browser supports JavaScript, be sure it is enabled.}

\begin{center}\rule{3in}{0.4pt}\end{center}

{[}\href{coursse16.html}{prev}{]}
{[}\href{coursse16.html\#tailcoursse16.html}{prev-tail}{]}
{[}\hyperref[tailcoursse17.html]{tail}{]}
{[}\href{coursch4.html\#coursse17.html}{up}{]}

\subsubsection{3.3 A propos de Jordan}

\paragraph{3.3.1 Décomposition de Jordan}

Soit E un K-espace vectoriel de dimension finie et u ∈ L(E) dont le
polynôme caractéristique est scindé sur K,
\{E\}\_\{1\},\textbackslash{}mathop\{\textbackslash{}mathop\{\ldots{}\}\},\{E\}\_\{k\}
les sous-espaces caractéristiques de u associés aux valeurs propres
\{λ\}\_\{1\},\textbackslash{}mathop\{\textbackslash{}mathop\{\ldots{}\}\},\{λ\}\_\{k\}.
Soit \{u\}\_\{i\} la restriction de u à \{E\}\_\{i\} et \{n\}\_\{i\} =
\{u\}\_\{i\} −
\{λ\}\_\{i\}\{\textbackslash{}mathrm\{Id\}\}\_\{\{E\}\_\{i\}\}. Avec les
notations précédentes, on a \{n\}\_\{i\}\^{}\{\{r\}\_\{i\}\} = 0, donc
\{n\}\_\{i\} est nilpotent. Soit d : E → E définie par d(\{x\}\_\{1\} +
\textbackslash{}mathop\{\textbackslash{}mathop\{\ldots{}\}\} +
\{x\}\_\{k\}) = \{λ\}\_\{1\}\{x\}\_\{1\} +
\textbackslash{}mathop\{\textbackslash{}mathop\{\ldots{}\}\} +
\{λ\}\_\{k\}\{x\}\_\{k\} et n : E → E défini par n(\{x\}\_\{1\} +
\textbackslash{}mathop\{\textbackslash{}mathop\{\ldots{}\}\} +
\{x\}\_\{k\}) =\{\textbackslash{}mathop\{ \textbackslash{}mathop\{∑ \}\}
\}\_\{i\}\{n\}\_\{i\}(\{x\}\_\{i\}). L'endomorphisme d est
diagonalisable (ses sous-espaces propres sont les \{E\}\_\{i\}), n est
nilpotent (\{n\}\^{}\{\textbackslash{}mathop\{max\}(\{r\}\_\{i\})\} = 0)
et on a u = d + n. De plus, si \{d\}\_\{i\} =
\{λ\}\_\{i\}\{\textbackslash{}mathrm\{Id\}\}\_\{\{E\}\_\{i\}\} désigne
la restriction de d à \{E\}\_\{i\}, on a \{d\}\_\{i\} ∘ \{n\}\_\{i\} =
\{n\}\_\{i\} ∘ \{d\}\_\{i\} et on en déduit donc que d ∘ n = d ∘ n.

Théorème~3.3.1 (décomposition de Jordan). Soit E un K-espace vectoriel
de dimension finie et u ∈ L(E) dont le polynôme caractéristique est
scindé sur K. Alors u s'écrit de manière unique sous la forme u = d + n
avec d diagonalisable, n nilpotent et d ∘ n = n ∘ d.

Démonstration L'existence de la décomposition vient d'être démontrée.
Soit u = d' + n' une autre décomposition vérifiant les conditions
imposées. Alors d' et n' commutent à u, donc à tous les P(u) et donc
laissent stables leurs noyaux. En particulier ils laissent stables les
sous-espaces caractéristiques de u. Soit \{d\}\_\{i\}' et \{n\}\_\{i\}'
les restrictions de d' et n' à \{E\}\_\{i\}. \{n\}\_\{i\}' est bien
entendu nilpotent. De plus, puisque d' est diagonalisable, il existe P
scindé à racines simples tel que P(d') = 0 et on a encore
P(\{d\}\_\{i\}') = 0 ce qui montre que \{d\}\_\{i\}' est diagonalisable.
On a \{d\}\_\{i\} + \{n\}\_\{i\} = \{d\}\_\{i\}' + \{n\}\_\{i\}' soit
encore \{λ\}\_\{i\}\textbackslash{}mathrm\{Id\} + \{n\}\_\{i\} =
\{d\}\_\{i\}' + \{n\}\_\{i\}' ou encore
\{λ\}\_\{i\}\{\textbackslash{}mathrm\{Id\}\}\_\{\{E\}\_\{i\}\} −
\{d\}\_\{i\}' = \{n\}\_\{i\}' − \{n\}\_\{i\}. Comme \{n\}\_\{i\}'
commute à
\{λ\}\_\{i\}\{\textbackslash{}mathrm\{Id\}\}\_\{\{E\}\_\{i\}\},\{d\}\_\{i\}'
et \{n\}\_\{i\}', il commute à \{n\}\_\{i\} et donc \{n\}\_\{i\} −
\{n\}\_\{i\}' est encore nilpotent. De plus
\{λ\}\_\{i\}\textbackslash{}mathrm\{Id\} − \{d\}\_\{i\}' est clairement
diagonalisable. Un endomorphisme à la fois diagonalisable et nilpotent
est nul puisqu'il doit avoir une matrice nulle dans une certaine base,
donc \{d\}\_\{i\} = \{d\}\_\{i\}' et \{n\}\_\{i\} = \{n\}\_\{i\}'.
Alors, d et d' coïncident sur des espaces dont la somme est E, donc ils
sont égaux. Ceci implique alors que n = n'. D'où l'unicité de la
décomposition.

\paragraph{3.3.2 Applications}

Puissances d'un endomorphisme

Soit E un K-espace vectoriel de dimension finie et u ∈ L(E) dont le
polynôme caractéristique est scindé sur K,
\{E\}\_\{1\},\textbackslash{}mathop\{\textbackslash{}mathop\{\ldots{}\}\},\{E\}\_\{k\}
les sous-espaces caractéristiques de u associés aux valeurs propres
\{λ\}\_\{1\},\textbackslash{}mathop\{\textbackslash{}mathop\{\ldots{}\}\},\{λ\}\_\{k\}.
Soit \{u\}\_\{i\} la restriction de u à \{E\}\_\{i\} et \{n\}\_\{i\} =
\{u\}\_\{i\} −
\{λ\}\_\{i\}\{\textbackslash{}mathrm\{Id\}\}\_\{\{E\}\_\{i\}\}.
L'endomorphisme \{n\}\_\{i\} est nilpotent d'indice de nilpotence
\{r\}\_\{i\}. On a \{u\}\_\{i\} =
\{λ\}\_\{i\}\{\textbackslash{}mathrm\{Id\}\}\_\{\{E\}\_\{i\}\} +
\{n\}\_\{i\} et donc si q ∈ ℕ

\{u\}\_\{i\}\^{}\{q\} =\{ \textbackslash{}mathop\{∑
\}\}\_\{p=0\}\^{}\{q\}\{C\}\_\{ q\}\^{}\{p\}\{λ\}\_\{
i\}\^{}\{q−p\}\{n\}\_\{ i\}\^{}\{p\} =\{ \textbackslash{}mathop\{∑
\}\}\_\{p=0\}\^{}\{min(q,\{r\}\_\{i\}−1)\}\{C\}\_\{
q\}\^{}\{p\}\{λ\}\_\{ i\}\^{}\{q−p\}\{n\}\_\{ i\}\^{}\{p\}

Supposons désormais que \{λ\}\_\{i\}\textbackslash{}mathrel\{≠\}0. On
obtient

\textbackslash{}begin\{eqnarray*\}\{ u\}\_\{i\}\^{}\{q\}\& =\& \{λ\}\_\{
i\}\^{}\{q\}\{ \textbackslash{}mathop\{∑
\}\}\_\{p=0\}\^{}\{min(q,\{r\}\_\{i\}−1)\}\{ q(q −
1)\textbackslash{}mathop\{\ldots{}\}(q − p + 1) \textbackslash{}over
p!\} \{λ\}\_\{i\}\^{}\{−p\}\{n\}\_\{ i\}\^{}\{p\}\%\&
\textbackslash{}\textbackslash{} \& =\& \{λ\}\_\{i\}\^{}\{q\}\{
\textbackslash{}mathop\{∑ \}\}\_\{p=0\}\^{}\{\{r\}\_\{i\}−1\}\{ q(q −
1)\textbackslash{}mathop\{\ldots{}\}(q − p + 1) \textbackslash{}over
p!\} \{λ\}\_\{i\}\^{}\{−p\}\{n\}\_\{ i\}\^{}\{p\} \%\&
\textbackslash{}\textbackslash{} \textbackslash{}end\{eqnarray*\}

(puisque \{
q(q−1)\textbackslash{}mathop\{\textbackslash{}mathop\{\ldots{}\}\}(q−p+1)
\textbackslash{}over p!\} = 0 si p \textgreater{} q). Posons alors, pour
t ∈ K

\{\textbackslash{}mathop\{∑ \}\}\_\{p=0\}\^{}\{\{r\}\_\{i\}−1\}\{ t(t −
1)\textbackslash{}mathop\{\ldots{}\}(t − p + 1) \textbackslash{}over
p!\} \{λ\}\_\{i\}\^{}\{−p\}\{n\}\_\{ i\}\^{}\{p\} =\{
\textbackslash{}mathop\{∑ \}\}\_\{p=0\}\^{}\{\{r\}\_\{i\}−1\}\{w\}\_\{
i,p\}\{t\}\^{}\{p\}

avec \{w\}\_\{i,p\} ∈ L(\{E\}\_\{i\})~; c'est une fonction polynomiale
en t à valeurs dans L(\{E\}\_\{i\}) de degré inférieur ou égal à
\{r\}\_\{i\} − 1. On obtient

\textbackslash{}mathop\{∀\}q ∈ ℕ,\textbackslash{}quad
\{u\}\_\{i\}\^{}\{q\} = \{λ\}\_\{ i\}\^{}\{q\}\{
\textbackslash{}mathop\{∑ \}\}\_\{p=0\}\^{}\{\{r\}\_\{i\}−1\}\{w\}\_\{
i,p\}\{q\}\^{}\{p\}

Supposons maintenant que u est inversible, si bien que
\textbackslash{}mathop\{∀\}i ∈ {[}1,k{]},
\{λ\}\_\{i\}\textbackslash{}mathrel\{≠\}0. Soit \{π\}\_\{i\}(x) la
projection sur \{E\}\_\{i\} parallèlement à
\{\textbackslash{}mathop\{\textbackslash{}mathop\{⊕ \}\}
\}\_\{j\textbackslash{}mathrel\{≠\}i\}\{E\}\_\{j\}. On a
\{\textbackslash{}mathop\{\textbackslash{}mathop\{∑ \}\}
\}\_\{i\}\{π\}\_\{i\} =\{ \textbackslash{}mathrm\{Id\}\}\_\{E\} si bien
que u = u ∘ (\{\textbackslash{}mathop\{\textbackslash{}mathop\{∑ \}\}
\}\_\{i\}\{π\}\_\{i\}) =\{\textbackslash{}mathop\{
\textbackslash{}mathop\{∑ \}\} \}\_\{i\}\{u\}\_\{i\} ∘ \{π\}\_\{i\}.

Lemme~3.3.2 \textbackslash{}mathop\{∀\}q ∈ ℕ, \{u\}\^{}\{q\}
=\textbackslash{}mathop\{ \textbackslash{}mathop\{∑ \}\}
\{u\}\_\{i\}\^{}\{q\} ∘ \{π\}\_\{i\}.

Démonstration Evident par récurrence sur q en remarquant que les
\{E\}\_\{i\} sont stables par u et les \{u\}\_\{i\}.

On en déduit donc que \textbackslash{}mathop\{∀\}q ∈
ℕ,\textbackslash{}quad \{u\}\^{}\{q\} =\{\textbackslash{}mathop\{
\textbackslash{}mathop\{∑ \}\}
\}\_\{i=1\}\^{}\{k\}\{λ\}\_\{i\}\^{}\{q\}\{\textbackslash{}mathop\{
\textbackslash{}mathop\{∑ \}\}
\}\_\{p=0\}\^{}\{\{r\}\_\{i\}−1\}\{q\}\^{}\{p\}\{w\}\_\{i,p\} ∘
\{π\}\_\{i\}, d'où le théorème (en remarquant que \{r\}\_\{i\} ≤
\{m\}\_\{i\})

Théorème~3.3.3 Soit u ∈ L(E) inversible dont le polynôme caractéristique
est scindé sur K,
\{λ\}\_\{1\},\textbackslash{}mathop\{\textbackslash{}mathop\{\ldots{}\}\},\{λ\}\_\{k\}
ses valeurs propres de multiplicités respectives
\{m\}\_\{1\},\textbackslash{}mathop\{\textbackslash{}mathop\{\ldots{}\}\},\{m\}\_\{k\}.
Alors il existe une famille
\{(\{v\}\_\{i,p\})\}\_\{1≤i≤k,0≤p≤\{m\}\_\{i\}−1\} d'endomorphismes de E
tels que

\textbackslash{}mathop\{∀\}q ∈ ℕ,\textbackslash{}quad \{u\}\^{}\{q\} =\{
\textbackslash{}mathop\{∑ \}\}\_\{i=1\}\^{}\{k\}\{λ\}\_\{ i\}\^{}\{q\}\{
\textbackslash{}mathop\{∑
\}\}\_\{p=0\}\^{}\{\{m\}\_\{i\}−1\}\{q\}\^{}\{p\}\{v\}\_\{ i,p\}

Remarque~3.3.1 Bien entendu, on a un résultat similaire pour les
matrices inversibles

Théorème~3.3.4 Soit A ∈ \{M\}\_\{K\}(n) inversible dont le polynôme
caractéristique est scindé sur K,
\{λ\}\_\{1\},\textbackslash{}mathop\{\textbackslash{}mathop\{\ldots{}\}\},\{λ\}\_\{k\}
ses valeurs propres de multiplicités respectives
\{m\}\_\{1\},\textbackslash{}mathop\{\textbackslash{}mathop\{\ldots{}\}\},\{m\}\_\{k\}.
Alors il existe une famille
\{(\{B\}\_\{i,p\})\}\_\{1≤i≤k,0≤p≤\{m\}\_\{i\}−1\} de matrices carrées
d'ordre n telles que

\textbackslash{}mathop\{∀\}q ∈ ℕ,\textbackslash{}quad \{A\}\^{}\{q\} =\{
\textbackslash{}mathop\{∑ \}\}\_\{i=1\}\^{}\{k\}\{λ\}\_\{ i\}\^{}\{q\}\{
\textbackslash{}mathop\{∑
\}\}\_\{p=0\}\^{}\{\{m\}\_\{i\}−1\}\{q\}\^{}\{p\}\{B\}\_\{ i,p\}

Suites à récurrence linéaire

Remarque~3.3.2 Soit p ∈ ℕ,
\{a\}\_\{0\},\textbackslash{}mathop\{\textbackslash{}mathop\{\ldots{}\}\},\{a\}\_\{p−1\}
une famille d'éléments de K et

V = \textbackslash{}\{\{(\{u\}\_\{n\})\}\_\{n∈ℕ\} ∈
\{K\}\^{}\{ℕ\}\textbackslash{}mathrel\{∣\}\textbackslash{}mathop\{∀\}n ∈
ℕ, \{u\}\_\{ n+p\} = \{a\}\_\{p−1\}\{u\}\_\{n+p−1\} +
\textbackslash{}mathop\{\textbackslash{}mathop\{\ldots{}\}\} +
\{a\}\_\{0\}\{u\}\_\{n\}\textbackslash{}\}

V est un sous-espace vectoriel de \{K\}\^{}\{ℕ\}. Il est clair que la
donnée de
\{u\}\_\{0\},\textbackslash{}mathop\{\textbackslash{}mathop\{\ldots{}\}\},\{u\}\_\{p−1\}
détermine parfaitement un élément de V et on a donc

Théorème~3.3.5 L'application V → \{K\}\^{}\{p\},
\{(\{u\}\_\{n\})\}\_\{n∈ℕ\}\textbackslash{}mathrel\{↦\}(\{u\}\_\{0\},\textbackslash{}mathop\{\textbackslash{}mathop\{\ldots{}\}\},\{u\}\_\{p−1\})
est un isomorphisme d'espaces vectoriels. On a en particulier
\textbackslash{}mathop\{dim\} V = p.

Remarque~3.3.3 Il est clair que l'on peut se limiter à étudier le cas où
\{a\}\_\{0\}\textbackslash{}mathrel\{≠\}0 sinon notre récurrence
linéaire d'ordre p se réduit à une récurrence linéaire d'ordre k ≤ p
valable pour n ≥ \{n\}\_\{0\}.

Soit \{(\{u\}\_\{n\})\}\_\{n∈ℕ\} ∈ V et considérons la suite (\{V
\}\_\{n\}) définie par \{V \}\_\{n\} = \textbackslash{}left
(\textbackslash{}matrix\{\textbackslash{},\{u\}\_\{n\}
\textbackslash{}cr \{u\}\_\{n+1\} \textbackslash{}cr
\textbackslash{}mathop\{\textbackslash{}mathop\{⋮\}\} \textbackslash{}cr
\{u\}\_\{n+p−1\}\}\textbackslash{}right ) ∈ \{K\}\^{}\{p\}. On a
clairement \{V \}\_\{n+1\} = A\{V \}\_\{n\} avec

A = \textbackslash{}left (\textbackslash{}matrix\{\textbackslash{},0 \&1
\&0\&\textbackslash{}mathop\{\textbackslash{}mathop\{\ldots{}\}\}\&0
\textbackslash{}cr \&\textbackslash{}mathrel\{⋱\}
\&\textbackslash{}mathrel\{⋱\}\&
\&\textbackslash{}mathop\{\textbackslash{}mathop\{⋮\}\}
\textbackslash{}cr \&
\&\textbackslash{}mathrel\{⋱\}\&\textbackslash{}mathrel\{⋱\}\&\textbackslash{}mathop\{\textbackslash{}mathop\{⋮\}\}
\textbackslash{}cr 0
\&\textbackslash{}mathop\{\textbackslash{}mathop\{\ldots{}\}\}
\&\textbackslash{}mathop\{\textbackslash{}mathop\{\ldots{}\}\}\&0\&1
\textbackslash{}cr
\{a\}\_\{0\}\&\{a\}\_\{1\}\&\textbackslash{}mathop\{\textbackslash{}mathop\{\ldots{}\}\}\&\textbackslash{}mathop\{\textbackslash{}mathop\{\ldots{}\}\}\&\{a\}\_\{p−1\}\}\textbackslash{}right
) ∈ \{M\}\_\{K\}(p)

et donc \{V \}\_\{n\} = \{A\}\^{}\{n\}\{V \}\_\{0\}. On a
\textbackslash{}mathop\{\textbackslash{}mathrm\{det\}\} A =
\{(−1)\}\^{}\{n−1\}\{a\}\_\{0\}\textbackslash{}mathrel\{≠\}0 et donc la
matrice est inversible. On peut donc appliquer le résultat précédent.
Soit χ le polynôme caractéristique de la matrice A (encore appelé
polynôme caractéristique de la récurrence linéaire). Un calcul simple
donne

Lemme~3.3.6 On a χ(X) = \{X\}\^{}\{p\} − \{a\}\_\{p−1\}\{X\}\^{}\{p−1\}
−\textbackslash{}mathop\{\textbackslash{}mathop\{\ldots{}\}\} −
\{a\}\_\{0\}. Pour λ ∈ \{K\}\^{}\{∗\}, on a

χ(λ) = 0 \textbackslash{}mathrel\{⇔\} \{(\{λ\}\^{}\{n\})\}\_\{ n∈ℕ\} ∈ V

Soit
\{λ\}\_\{1\},\textbackslash{}mathop\{\textbackslash{}mathop\{\ldots{}\}\},\{λ\}\_\{k\}
les racines de χ de multiplicités respectives
\{m\}\_\{1\},\textbackslash{}mathop\{\textbackslash{}mathop\{\ldots{}\}\},\{m\}\_\{k\}.
On sait qu'il existe une famille
\{(\{B\}\_\{i,q\})\}\_\{1≤i≤k,0≤q≤\{m\}\_\{i\}−1\} de matrices carrées
d'ordre p telles que

\textbackslash{}mathop\{∀\}n ∈ ℕ,\textbackslash{}quad \{A\}\^{}\{n\} =\{
\textbackslash{}mathop\{∑ \}\}\_\{i=1\}\^{}\{k\}\{λ\}\_\{ i\}\^{}\{n\}\{
\textbackslash{}mathop\{∑
\}\}\_\{q=0\}\^{}\{\{m\}\_\{i\}−1\}\{n\}\^{}\{q\}\{B\}\_\{ i,q\}

On a donc en particulier \{A\}\^{}\{n\}\{V \}\_\{0\}
=\{\textbackslash{}mathop\{ \textbackslash{}mathop\{∑ \}\}
\}\_\{i=1\}\^{}\{k\}\{λ\}\_\{i\}\^{}\{n\}\{\textbackslash{}mathop\{
\textbackslash{}mathop\{∑ \}\}
\}\_\{q=0\}\^{}\{\{m\}\_\{i\}−1\}\{n\}\^{}\{q\}\{B\}\_\{i,q\}\{V
\}\_\{0\} et en prenant la première coordonnée,

\{u\}\_\{n\} =\{ \textbackslash{}mathop\{∑
\}\}\_\{i=1\}\^{}\{k\}\{λ\}\_\{ i\}\^{}\{n\}\{ \textbackslash{}mathop\{∑
\}\}\_\{q=0\}\^{}\{\{m\}\_\{i\}−1\}\{α\}\_\{ i,q\}\{n\}\^{}\{q\}

Soit alors W le sous-espace de \{K\}\^{}\{ℕ\} engendré par les suites
\{(\{λ\}\_\{i\}\^{}\{n\}\{n\}\^{}\{q\})\}\_\{1≤i≤k,0≤q≤\{m\}\_\{i\}−1\}.
On a \textbackslash{}mathop\{dim\} W
≤\textbackslash{}mathop\{\textbackslash{}mathop\{∑ \}\} \{m\}\_\{i\} = p
et V ⊂ W avec \textbackslash{}mathop\{dim\} V = p. On en déduit que V =
W et que la famille
\{(\{λ\}\_\{i\}\^{}\{n\}\{n\}\^{}\{q\})\}\_\{1≤i≤k,0≤q≤\{m\}\_\{i\}−1\}
est une base de V . On a donc le théorème suivant

Théorème~3.3.7 Soit p ∈ ℕ,
\{a\}\_\{0\},\textbackslash{}mathop\{\textbackslash{}mathop\{\ldots{}\}\},\{a\}\_\{p−1\}
une famille d'éléments de K avec
\{a\}\_\{0\}\textbackslash{}mathrel\{≠\}0, et V l'espace des suites
vérifiant la récurrence linéaire \textbackslash{}mathop\{∀\}n ∈ ℕ,
\{u\}\_\{n+p\} = \{a\}\_\{p−1\}\{u\}\_\{n+p−1\} +
\textbackslash{}mathop\{\textbackslash{}mathop\{\ldots{}\}\} +
\{a\}\_\{0\}\{u\}\_\{n\}\textbackslash{}\}. Soit χ(X) = \{X\}\^{}\{p\} −
\{a\}\_\{p−1\}\{X\}\^{}\{p−1\}
−\textbackslash{}mathop\{\textbackslash{}mathop\{\ldots{}\}\} −
\{a\}\_\{0\} le polynôme caractéristique de la récurrence linéaire
(obtenu en recherchant des solutions particulières de la forme
\{u\}\_\{n\} = \{λ\}\^{}\{n\}),
\{λ\}\_\{1\},\textbackslash{}mathop\{\textbackslash{}mathop\{\ldots{}\}\},\{λ\}\_\{k\}
les racines de χ de multiplicités respectives
\{m\}\_\{1\},\textbackslash{}mathop\{\textbackslash{}mathop\{\ldots{}\}\},\{m\}\_\{k\}.
Alors la famille
\{(\{λ\}\_\{i\}\^{}\{n\}\{n\}\^{}\{q\})\}\_\{1≤i≤k,0≤q≤\{m\}\_\{i\}−1\}
est une base de V . Les solutions de la récurrence linéaire sont
exactement les suites qui s'écrivent sous la forme

\{u\}\_\{n\} =\{ \textbackslash{}mathop\{∑
\}\}\_\{i=1\}\^{}\{k\}\{λ\}\_\{ i\}\^{}\{n\}\{P\}\_\{
i\}(n),\textbackslash{}quad \{P\}\_\{i\} ∈ K{[}X{]}, deg \{P\}\_\{i\} ≤
\{m\}\_\{i\} − 1

Retour aux puissances d'un endomorphisme

Soit u ∈ L(E) et soit P(X) = \{X\}\^{}\{p\} −
\{a\}\_\{p−1\}\{X\}\^{}\{p−1\}
−\textbackslash{}mathop\{\textbackslash{}mathop\{\ldots{}\}\} −
\{a\}\_\{0\} un polynôme qui annule u (par exemple le polynôme
caractéristique). On a immédiatement

Lemme~3.3.8 \{(\{u\}\^{}\{n\})\}\_\{0≤n≤p−1\} est une famille
génératrice de
\textbackslash{}mathop\{\textbackslash{}mathrm\{Vect\}\}(\{u\}\^{}\{n\},n
∈ ℕ).

On peut donc chercher à exprimer \{u\}\^{}\{n\} sous la forme
\{u\}\^{}\{n\} = \{α\}\_\{n\}\^{}\{(p−1)\}\{u\}\^{}\{p−1\} +
\textbackslash{}mathop\{\textbackslash{}mathop\{\ldots{}\}\} +
\{α\}\_\{n\}\^{}\{(0)\}\textbackslash{}mathrm\{Id\}.

Théorème~3.3.9 Soit
\{(\{α\}\_\{n\}\^{}\{(p−1)\})\}\_\{n∈ℕ\},\textbackslash{}mathop\{\textbackslash{}mathop\{\ldots{}\}\},\{(\{α\}\_\{n\}\^{}\{0\})\}\_\{n∈ℕ\}
les suites solutions de la récurrence linéaire \{α\}\_\{n+p\} =
\{a\}\_\{p−1\}\{α\}\_\{n+p−1\} +
\textbackslash{}mathop\{\textbackslash{}mathop\{\ldots{}\}\} +
\{a\}\_\{0\}\{α\}\_\{n\} vérifiant

\textbackslash{}mathop\{∀\}i ∈ {[}0,p − 1{]},
\textbackslash{}mathop\{∀\}j ∈ {[}0,p − 1{]},\textbackslash{}quad
\{α\}\_\{i\}\^{}\{(j)\} = \{δ\}\_\{ i\}\^{}\{j\}

Alors

\textbackslash{}mathop\{∀\}n ∈ ℕ,\textbackslash{}quad \{u\}\^{}\{n\} =
\{α\}\_\{ n\}\^{}\{(p−1)\}\{u\}\^{}\{p−1\} +
\textbackslash{}mathop\{\textbackslash{}mathop\{\ldots{}\}\} + \{α\}\_\{
n\}\^{}\{(0)\}\textbackslash{}mathrm\{Id\}

Démonstration Par récurrence sur n. C'est manifestement vérifié si n ≤ p
− 1. De plus, si n ≥ p la relation \{u\}\^{}\{p\} =
\{a\}\_\{p−1\}\{u\}\^{}\{p−1\} +
\textbackslash{}mathop\{\textbackslash{}mathop\{\ldots{}\}\} +
\{a\}\_\{0\}\textbackslash{}mathrm\{Id\} donne \{u\}\^{}\{n\} =
\{a\}\_\{p−1\}\{u\}\^{}\{n−1\} +
\textbackslash{}mathop\{\textbackslash{}mathop\{\ldots{}\}\} +
\{a\}\_\{0\}\{u\}\^{}\{n−p\} soit par l'hypothèse de récurrence

\textbackslash{}begin\{eqnarray*\}\{ u\}\^{}\{n\}\& =\&
\{\textbackslash{}mathop\{∑ \}\}\_\{i=1\}\^{}\{p\}\{a\}\_\{
p−i\}\{u\}\^{}\{n−i\} =\{ \textbackslash{}mathop\{∑
\}\}\_\{i=1\}\^{}\{p\}\{a\}\_\{ p−i\}\{ \textbackslash{}mathop\{∑
\}\}\_\{j=0\}\^{}\{p−1\}\{α\}\_\{ n−i\}\^{}\{(j)\}\{u\}\^{}\{j\} \%\&
\textbackslash{}\textbackslash{} \& =\& \{\textbackslash{}mathop\{∑
\}\}\_\{j=0\}\^{}\{p−1\}(\{\textbackslash{}mathop\{∑
\}\}\_\{i=1\}\^{}\{p\}\{a\}\_\{
p−i\}\{α\}\_\{n−i\}\^{}\{(j)\})\{u\}\^{}\{j\} =\{
\textbackslash{}mathop\{∑ \}\}\_\{j=0\}\^{}\{p−1\}\{α\}\_\{
n\}\^{}\{(j)\}\{u\}\^{}\{j\}\%\& \textbackslash{}\textbackslash{}
\textbackslash{}end\{eqnarray*\}

d'après la relation vérifiée par les (\{α\}\_\{n\}\^{}\{(j)\}). Ceci
achève la démonstration.

\paragraph{3.3.3 Réduction des endomorphismes nilpotents}

Définition~3.3.1 Soit E un K-espace vectoriel et u ∈ L(E). On dit que u
est nilpotent d'indice de nilpotence r si \{u\}\^{}\{r\} = 0 et
\{u\}\^{}\{r−1\}\textbackslash{}mathrel\{≠\}0.

Remarque~3.3.4 Remarquons que la seule valeur propre d'un endomorphisme
nilpotent est 0, car si u(x) = λx, on a 0 = \{u\}\^{}\{r\}(x) =
\{λ\}\^{}\{r\}x.

Proposition~3.3.10 Soit E un K-espace vectoriel de dimension n et u ∈
L(E). Alors u est nilpotent si et seulement si \{χ\}\_\{u\}(X) =
\{X\}\^{}\{n\}.

Démonstration ( ⇒) Supposons que u est nilpotent. Comme u est annulé par
le polynôme scindé \{X\}\^{}\{r\} (si \{u\}\^{}\{r\} = 0), u est
trigonalisable. Mais u admet comme seule valeur propre 0. On en déduit
que \{χ\}\_\{u\}(X) = \{X\}\^{}\{n\}. Pour la réciproque, on peut par
exemple utiliser le théorème de Cayley Hamilton, ou trigonaliser u.

Remarque~3.3.5 On en déduit que l'indice de nilpotence r est inférieur
ou égal à n. On a bien entendu \{μ\}\_\{u\}(X) = \{X\}\^{}\{r\}.

Définition~3.3.2 Soit p ≥ 1. On appelle matrice élémentaire de Jordan
d'ordre p la matrice

\{J\}\_\{p\} = \textbackslash{}left
(\textbackslash{}matrix\{\textbackslash{},0\&1\&0\&\textbackslash{}mathop\{\textbackslash{}mathop\{\ldots{}\}\}\&0
\textbackslash{}cr
\textbackslash{}mathop\{\textbackslash{}mathop\{⋮\}\}\&\textbackslash{}mathrel\{⋱\}\&\textbackslash{}mathrel\{⋱\}\&\textbackslash{}mathrel\{⋱\}\&\textbackslash{}mathop\{\textbackslash{}mathop\{⋮\}\}
\textbackslash{}cr
\textbackslash{}mathop\{\textbackslash{}mathop\{⋮\}\}\&
\&\textbackslash{}mathrel\{⋱\}\&\textbackslash{}mathrel\{⋱\}\&\textbackslash{}mathop\{\textbackslash{}mathop\{⋮\}\}
\textbackslash{}cr
0\&\textbackslash{}mathop\{\textbackslash{}mathop\{\ldots{}\}\}\&\textbackslash{}mathop\{\textbackslash{}mathop\{\ldots{}\}\}\&0\&1
\textbackslash{}cr
0\&\textbackslash{}mathop\{\textbackslash{}mathop\{\ldots{}\}\}\&\textbackslash{}mathop\{\textbackslash{}mathop\{\ldots{}\}\}\&\textbackslash{}mathop\{\textbackslash{}mathop\{\ldots{}\}\}\&0\}\textbackslash{}right
)

Soit E un K-espace vectoriel de dimension n et u ∈ L(E) nilpotent
d'indice de nilpotence r. Supposons par exemple que r = n. On a donc
\{u\}\^{}\{n−1\}\textbackslash{}mathrel\{≠\}0 avec \{u\}\^{}\{n\} = 0.
Soit a ∈ E tel que \{u\}\^{}\{n−1\}(a)\textbackslash{}mathrel\{≠\}0 et
posons \{e\}\_\{i\} = \{u\}\^{}\{n−i\}(a) pour 1 ≤ i ≤ n. Montrons que
(\{e\}\_\{1\},\textbackslash{}mathop\{\textbackslash{}mathop\{\ldots{}\}\},\{e\}\_\{n\})
est une base de E. Il suffit de montrer que c'est une famille libre.
Pour cela supposons que \{λ\}\_\{1\}\{e\}\_\{1\} +
\textbackslash{}mathop\{\textbackslash{}mathop\{\ldots{}\}\} +
\{λ\}\_\{n\}\{e\}\_\{n\} = 0, soit encore

\{λ\}\_\{1\}\{u\}\^{}\{n−1\}(a) +
\textbackslash{}mathop\{\textbackslash{}mathop\{\ldots{}\}\} + \{λ\}\_\{
n−1\}u(a) + \{λ\}\_\{n\}a = 0

Appliquons aux deux membres \{u\}\^{}\{n−1\} en tenant compte de
\{u\}\^{}\{n\}(a) =
\textbackslash{}mathop\{\textbackslash{}mathop\{\ldots{}\}\} =
\{u\}\^{}\{2n−2\}(a) = 0~; on obtient \{λ\}\_\{n\}\{u\}\^{}\{n−1\}(a) =
0 soit \{λ\}\_\{n\} = 0. Supposons montré que \{λ\}\_\{n\} =
\{λ\}\_\{n−1\} =
\textbackslash{}mathop\{\textbackslash{}mathop\{\ldots{}\}\} =
\{λ\}\_\{n−k+1\} = 0 si bien que l'on a

\{λ\}\_\{1\}\{u\}\^{}\{n−1\}(a) +
\textbackslash{}mathop\{\textbackslash{}mathop\{\ldots{}\}\} + \{λ\}\_\{
n−k−1\}\{u\}\^{}\{k+1\}(a) + \{λ\}\_\{ n−k\}\{u\}\^{}\{k\}(a) = 0

Appliquons aux deux membres \{u\}\^{}\{n−k−1\} en tenant compte de
\{u\}\^{}\{n\}(a) =
\textbackslash{}mathop\{\textbackslash{}mathop\{\ldots{}\}\} =
\{u\}\^{}\{2n−k−2\}(a) = 0~; on obtient
\{λ\}\_\{n−k\}\{u\}\^{}\{n−1\}(a) = 0 soit \{λ\}\_\{n−k\} = 0. Par
récurrence, on a bien \textbackslash{}mathop\{∀\}i, \{λ\}\_\{i\} = 0.
Donc
(\{e\}\_\{1\},\textbackslash{}mathop\{\textbackslash{}mathop\{\ldots{}\}\},\{e\}\_\{n\})
est une base de E. Dans cette base, la matrice de u est clairement
\{J\}\_\{n\}~: on a u(\{e\}\_\{i\}) = \{e\}\_\{i−1\} si i ≥ 2 et
u(\{e\}\_\{1\}) = 0. Ce cas particulier est à la base du résultat
suivant

Théorème~3.3.11 Soit E un K-espace vectoriel de dimension n et u ∈ L(E)
nilpotent. Alors il existe une base ℰ de E telle que la matrice de u
dans la base ℰ soit un tableau diagonal de matrices élémentaires de
Jordan

\textbackslash{}mathop\{\textbackslash{}mathrm\{Mat\}\} (u,ℰ)
=\textbackslash{}mathop\{
\textbackslash{}mathrm\{diag\}\}(\{J\}\_\{\{p\}\_\{1\}\},\textbackslash{}mathop\{\textbackslash{}mathop\{\ldots{}\}\},\{J\}\_\{\{p\}\_\{k\}\})

Démonstration Elle va faire l'objet des deux sections suivantes

\paragraph{3.3.4 Première démonstration}

Par récurrence sur n =\textbackslash{}mathop\{ dim\} E. Le résultat est
évident pour n = 1. Supposons le vrai pour tous les endomorphismes
nilpotents d'espaces de dimensions inférieures ou égales à n − 1. Soit r
l'indice de nilpotence de u. Si r = n, on a déjà vu que le résultat
était vrai (avec une seule matrice élémentaire de Jordan). On peut donc
supposer que r \textless{} n. Puisque
\{u\}\^{}\{r−1\}\textbackslash{}mathrel\{≠\}0, soit a ∈ E tel que
\{u\}\^{}\{r−1\}(a)\textbackslash{}mathrel\{≠\}0. Comme précédemment la
famille \{ℰ\}\_\{1\} =
(\{u\}\^{}\{r−1\}(a),\textbackslash{}mathop\{\textbackslash{}mathop\{\ldots{}\}\},u(a),a)
est libre et il est clair que le sous-espace F =\textbackslash{}mathop\{
\textbackslash{}mathrm\{Vect\}\}(\{u\}\^{}\{r−1\}(a),\textbackslash{}mathop\{\textbackslash{}mathop\{\ldots{}\}\},u(a),a)
est stable par u (chaque vecteur est décalé d'un cran vers la gauche,
sauf le premier qui est annulé par u). On a
\textbackslash{}mathop\{\textbackslash{}mathrm\{Mat\}\}
(u\{\textbar{}\}\_\{F\},\{ℰ\}\_\{1\}) = \{J\}\_\{r\}.

Puisque \{u\}\^{}\{r−1\}(a)\textbackslash{}mathrel\{≠\}0 on peut trouver
une forme linéaire f telle que
f(\{u\}\^{}\{r−1\}(a))\textbackslash{}mathrel\{≠\}0. Soit

\textbackslash{}begin\{eqnarray*\} G\& =\& \{\textbackslash{}mathop\{⋂
\}\}\_\{k=0\}\^{}\{r−1\} \textbackslash{}mathrm\{Ker\}f ∘ \{u\}\^{}\{k\}
\%\& \textbackslash{}\textbackslash{} \& =\& \textbackslash{}\{x ∈
E\textbackslash{}mathrel\{∣\}f(x) = f(u(x)) =
\textbackslash{}mathop\{\textbackslash{}mathop\{\ldots{}\}\} =
f(\{u\}\^{}\{r−1\}(x) = 0\textbackslash{}\}\%\&
\textbackslash{}\textbackslash{} \textbackslash{}end\{eqnarray*\}

Lemme~3.3.12 G est un supplémentaire de F stable par u.

Démonstration La stabilité par u est claire, car si x ∈ G on a

\textbackslash{}begin\{eqnarray*\} f(u(x)) = 0,f(u(u(x))) =
0,f(\{u\}\^{}\{r−2\}(u(x)) = f(\{u\}\^{}\{r−1\}(x) = 0,\& \& \%\&
\textbackslash{}\textbackslash{} f(\{u\}\^{}\{r−1\}(u(x))) =
f(\{u\}\^{}\{r\}(x)) = f(0) = 0\& \& \%\&
\textbackslash{}\textbackslash{} \textbackslash{}end\{eqnarray*\}

Montrons que F ∩ G = \textbackslash{}\{0\textbackslash{}\}. Pour cela
soit x = \{λ\}\_\{1\}\{u\}\^{}\{r−1\}(a) +
\textbackslash{}mathop\{\textbackslash{}mathop\{\ldots{}\}\} +
\{λ\}\_\{r−1\}u(a) + \{λ\}\_\{r\}a ∈ F et supposons que x appartienne à
G. On a 0 = f(\{u\}\^{}\{r−1\}(x)) = \{λ\}\_\{1\}f(\{u\}\^{}\{2r−2\}(a))
+ \textbackslash{}mathop\{\textbackslash{}mathop\{\ldots{}\}\} +
\{λ\}\_\{r−1\}f(\{u\}\^{}\{r\}(a)) + \{λ\}\_\{r\}f(\{u\}\^{}\{r−1\}(a))
et tenant compte de \{u\}\^{}\{r\}(a) =
\textbackslash{}mathop\{\textbackslash{}mathop\{\ldots{}\}\} =
\{u\}\^{}\{2r−2\}(a) = 0 on obtient \{λ\}\_\{r\}f(\{u\}\^{}\{r−1\}(a)) =
0 soit \{λ\}\_\{r\} = 0. Comme précédemment une récurrence descendante
montre que \{λ\}\_\{r\} = \{λ\}\_\{r−1\} =
\textbackslash{}mathop\{\textbackslash{}mathop\{\ldots{}\}\} =
\{λ\}\_\{1\} = 0 soit x = 0. Donc F et G sont en somme directe. Mais G =
\{∩\}\_\{k=0\}\^{}\{r−1\}\textbackslash{}mathop\{
\textbackslash{}mathrm\{Ker\}\}f ∘ \{u\}\^{}\{k\}, et donc

\textbackslash{}mathop\{dim\} G = n
−\textbackslash{}mathop\{\textbackslash{}mathrm\{rg\}\}(f ∘
\{u\}\^{}\{k\}, 0 ≤ k ≤ r − 1) ≥ n − r =\textbackslash{}mathop\{ dim\} E
−\textbackslash{}mathop\{ dim\} F

On a donc E = F ⊕ G.

(Fin de la démonstration) On peut maintenant terminer la démonstration
du théorème. En appliquant notre hypothèse de récurrence à
l'endomorphisme nilpotent u\{\textbar{}\}\_\{G\} de G, on peut trouver
une base de G telle que
\textbackslash{}mathop\{\textbackslash{}mathrm\{Mat\}\}
(u\{\textbar{}\}\_\{G\},\{ℰ\}\_\{2\}) =\textbackslash{}mathop\{
\textbackslash{}mathrm\{diag\}\}(\{J\}\_\{\{p\}\_\{2\}\},\textbackslash{}mathop\{\textbackslash{}mathop\{\ldots{}\}\},\{J\}\_\{\{p\}\_\{k\}\}).
Alors ℰ = \{ℰ\}\_\{1\} ∪\{ℰ\}\_\{2\} est une base de E dans laquelle
\textbackslash{}mathop\{\textbackslash{}mathrm\{Mat\}\} (u,ℰ)
=\textbackslash{}mathop\{
\textbackslash{}mathrm\{diag\}\}(\{J\}\_\{r\},\{J\}\_\{\{p\}\_\{2\}\},\textbackslash{}mathop\{\textbackslash{}mathop\{\ldots{}\}\},\{J\}\_\{\{p\}\_\{k\}\}),
ce qui achève la démonstration.

\paragraph{3.3.5 Deuxième démonstration}

Posons \{V \}\_\{i\} =\textbackslash{}mathop\{
\textbackslash{}mathrm\{Ker\}\}\{u\}\^{}\{i\}.

Lemme~3.3.13 On a \textbackslash{}\{0\textbackslash{}\} = \{V \}\_\{0\}
⊂ \{V \}\_\{1\}
⊂\textbackslash{}mathop\{\textbackslash{}mathop\{\ldots{}\}\} ⊂ \{V
\}\_\{r\} = E avec une suite strictement croissante.

Démonstration Les inclusions sont claires. Supposons que \{V \}\_\{i\} =
\{V \}\_\{i+1\} pour i ≤ r − 1. Soit x ∈ E. On a 0 = \{u\}\^{}\{r\}(x) =
\{u\}\^{}\{i+1\}(\{u\}\^{}\{r−i−1\}(x)) donc \{u\}\^{}\{r−i−1\}(x) ∈ \{V
\}\_\{i+1\} = \{V \}\_\{i\} et donc \{u\}\^{}\{r−1\}(x) =
\{u\}\^{}\{i\}(\{u\}\^{}\{r−i−1\}(x)) = 0. On aurait donc
\{u\}\^{}\{r−1\} = 0 ce qui est exclu.

Soit \{W\}\_\{1\} un supplémentaire de \{V \}\_\{r−1\} dans E = \{V
\}\_\{r\}.

Lemme~3.3.14 On peut construire une suite de sous-espaces
\{W\}\_\{2\},\textbackslash{}mathop\{\textbackslash{}mathop\{\ldots{}\}\},\{W\}\_\{r\}
de E vérifiant

\begin{itemize}
\itemsep1pt\parskip0pt\parsep0pt
\item
  (i) \textbackslash{}mathop\{∀\}k ∈ {[}1,r{]},\textbackslash{}quad \{V
  \}\_\{r−k+1\} = \{V \}\_\{r−k\} ⊕ \{W\}\_\{k\}
\item
  (ii) \textbackslash{}mathop\{∀\}k ∈ {[}2,r{]},\textbackslash{}quad
  u(\{W\}\_\{k−1\}) ⊂ \{W\}\_\{k\}
\item
  (iii) \textbackslash{}mathop\{∀\}k ∈ {[}1,r −
  1{]},\textbackslash{}quad u\{\textbar{}\}\_\{\{W\}\_\{k\}\} est
  injective
\item
  On a alors E = \{W\}\_\{1\} ⊕\textbackslash{}mathrel\{⋯\} ⊕
  \{W\}\_\{r\}.
\end{itemize}

Démonstration On va construire \{W\}\_\{k\} par récurrence sur k. Pour
ce qui concerne k = 1, il suffit de montrer que
u\{\textbar{}\}\_\{\{W\}\_\{1\}\} est injective. Mais si x ∈
\{W\}\_\{1\} ∖\textbackslash{}\{0\textbackslash{}\}, on a
x\textbackslash{}mathrel\{∉\}\{V \}\_\{r−1\}, donc
\{u\}\^{}\{r−1\}(x)\textbackslash{}mathrel\{≠\}0 et donc
u(x)\textbackslash{}mathrel\{≠\}0. Supposons donc
\{W\}\_\{1\},\textbackslash{}mathop\{\textbackslash{}mathop\{\ldots{}\}\},\{W\}\_\{k−1\}
construits. Soit x ∈ \{W\}\_\{k−1\}
∖\textbackslash{}\{0\textbackslash{}\}. On a
x\textbackslash{}mathrel\{∉\}\{V \}\_\{r−k+1\}, donc
\{u\}\^{}\{r−k+1\}(x)\textbackslash{}mathrel\{≠\}0, soit
\{u\}\^{}\{r−k\}(u(x))\textbackslash{}mathrel\{≠\}0 et donc
u(x)\textbackslash{}mathrel\{∉\}\{V \}\_\{r−k\}. On a ainsi
u(\{W\}\_\{k−1\}) ∩ \{V \}\_\{r−k\} =
\textbackslash{}\{0\textbackslash{}\}. Mais d'autre part x ∈
\{W\}\_\{k−1\} ⊂ \{V \}\_\{r−k+2\} et donc u(x) ∈ \{V \}\_\{r−k+1\}. On
a donc u(\{W\}\_\{k−1\}) ⊂ \{V \}\_\{r−k+1\}, \{V \}\_\{r−k\} ⊂ \{V
\}\_\{r−k+1\} avec u(\{W\}\_\{k−1\}) ∩ \{V \}\_\{r−k\} =
\textbackslash{}\{0\textbackslash{}\}. On peut donc trouver un
supplémentaire \{W\}\_\{k\} de \{V \}\_\{r−k\} dans \{V \}\_\{r−k+1\}
tel que u(\{W\}\_\{k−1\}) ⊂ \{W\}\_\{k\}. Alors, si x ∈ \{W\}\_\{k\}
∖\textbackslash{}\{0\textbackslash{}\}, x\textbackslash{}mathrel\{∉\}\{V
\}\_\{r−k\}, soit \{u\}\^{}\{r−k\}(x)\textbackslash{}mathrel\{≠\}0 et
donc si k \textless{} r, u(x)\textbackslash{}mathrel\{≠\}0. Ceci montre
bien que u\{\textbar{}\}\_\{\{W\}\_\{k\}\} est injective. On a donc bien
construit notre suite \{W\}\_\{k\}. Il est clair par récurrence que \{V
\}\_\{k\} = \{W\}\_\{r−k+1\}
⊕\textbackslash{}mathop\{\textbackslash{}mathop\{\ldots{}\}\} ⊕
\{W\}\_\{r\} et donc E = \{V \}\_\{r\} = \{W\}\_\{1\}
⊕\textbackslash{}mathrel\{⋯\} ⊕ \{W\}\_\{r\}.

Soit alors maintenant \{(\{e\}\_\{i,1\})\}\_\{1≤i≤\{s\}\_\{1\}\} une
base de \{W\}\_\{1\}. Comme u\{\textbar{}\}\_\{\{W\}\_\{1\}\} est
injective, \{(\{e\}\_\{i,2\} =
u(\{e\}\_\{i,1\}))\}\_\{1≤i≤\{s\}\_\{1\}\} est une base de
u(\{W\}\_\{1\}) que l'on peut compléter en une base
\{(\{e\}\_\{i,2\})\}\_\{1≤i≤\{m\}\_\{2\}\} de \{W\}\_\{2\}. Une
récurrence immédiate nous permet de construire des bases
\{(\{e\}\_\{i,k\})\}\_\{1≤i≤\{m\}\_\{k\}\} des \{W\}\_\{k\} telles que
pour k ≤ r − 1, et 1 ≤ i ≤ \{m\}\_\{k\}, u(\{e\}\_\{i,k\}) =
\{e\}\_\{i,k+1\}. On a \{e\}\_\{i,r\} ∈ \{W\}\_\{r\} ⊂ \{V \}\_\{1\}
=\textbackslash{}mathop\{ \textbackslash{}mathrm\{Ker\}\}u, donc
u(\{e\}\_\{i,r\}) = 0. On obtient ainsi une base (\{e\}\_\{i,j\}) de E.
Si on ordonne cette base en posant que (i,j) \textless{} (i',j')
\textbackslash{}mathrel\{⇔\} j \textgreater{} j'\textbackslash{}text\{
ou \}(j = j'\textbackslash{}text\{ et \}i \textless{} i'), la matrice de
u est un tableau diagonal de matrices de Jordan.

\paragraph{3.3.6 Réduction de Jordan}

Soit E un K-espace vectoriel de dimension finie et u ∈ L(E) dont le
polynôme caractéristique est scindé sur K,
\{E\}\_\{1\},\textbackslash{}mathop\{\textbackslash{}mathop\{\ldots{}\}\},\{E\}\_\{k\}
les sous-espaces caractéristiques de u associés aux valeurs propres
\{λ\}\_\{1\},\textbackslash{}mathop\{\textbackslash{}mathop\{\ldots{}\}\},\{λ\}\_\{k\}.
Soit \{u\}\_\{i\} la restriction de u à \{E\}\_\{i\} et \{n\}\_\{i\} =
\{u\}\_\{i\} −
\{λ\}\_\{i\}\{\textbackslash{}mathrm\{Id\}\}\_\{\{E\}\_\{i\}\}. Avec les
notations précédentes, on a \{n\}\_\{i\}\^{}\{\{r\}\_\{i\}\} = 0, donc
\{n\}\_\{i\} est nilpotent. On peut donc trouver une base \{ℰ\}\_\{i\}
de \{E\}\_\{i\} dans laquelle la matrice de \{n\}\_\{i\} est
\textbackslash{}mathop\{\textbackslash{}mathrm\{diag\}\}(\{J\}\_\{\{p\}\_\{1\}\},\textbackslash{}mathop\{\textbackslash{}mathop\{\ldots{}\}\},\{J\}\_\{\{p\}\_\{k\}\})
et alors la matrice de \{u\}\_\{i\} dans cette base est
\textbackslash{}mathop\{\textbackslash{}mathrm\{diag\}\}(\{J\}\_\{\{p\}\_\{1\}\}(\{λ\}\_\{i\}),\textbackslash{}mathop\{\textbackslash{}mathop\{\ldots{}\}\},\{J\}\_\{\{p\}\_\{k\}\}(\{λ\}\_\{i\}))
avec

\{J\}\_\{p\}(λ) = \textbackslash{}left
(\textbackslash{}matrix\{\textbackslash{},λ\&1\&0\&\textbackslash{}mathop\{\textbackslash{}mathop\{\ldots{}\}\}\&0
\textbackslash{}cr
0\&λ\&1\&\textbackslash{}mathop\{\textbackslash{}mathop\{\ldots{}\}\}\&0
\textbackslash{}cr
\textbackslash{}mathop\{\textbackslash{}mathop\{⋮\}\}\&
\&\textbackslash{}mathrel\{⋱\}\&\textbackslash{}mathrel\{⋱\}\&\textbackslash{}mathop\{\textbackslash{}mathop\{⋮\}\}
\textbackslash{}cr
0\&\textbackslash{}mathop\{\textbackslash{}mathop\{\ldots{}\}\}\&\textbackslash{}mathop\{\textbackslash{}mathop\{\ldots{}\}\}\&λ\&1
\textbackslash{}cr
0\&\textbackslash{}mathop\{\textbackslash{}mathop\{\ldots{}\}\}\&\textbackslash{}mathop\{\textbackslash{}mathop\{\ldots{}\}\}\&0\&λ\}\textbackslash{}right
) = λ\{I\}\_\{p\} + \{J\}\_\{p\} ∈ \{M\}\_\{K\}(p)

En réunissant ces bases on obtient

Théorème~3.3.15 Soit E un K-espace vectoriel de dimension finie et u ∈
L(E) dont le polynôme caractéristique est scindé sur K. Alors il existe
une base ℰ de E, des scalaires
\{μ\}\_\{1\},\textbackslash{}mathop\{\textbackslash{}mathop\{\ldots{}\}\},\{μ\}\_\{l\}
(non nécessairement distincts) et des entiers
\{n\}\_\{1\},\textbackslash{}mathop\{\textbackslash{}mathop\{\ldots{}\}\},\{n\}\_\{l\}
tels que \textbackslash{}mathop\{\textbackslash{}mathrm\{Mat\}\} (u,ℰ)
=\textbackslash{}mathop\{
\textbackslash{}mathrm\{diag\}\}(\{J\}\_\{\{n\}\_\{1\}\}(\{μ\}\_\{1\}),\textbackslash{}mathop\{\textbackslash{}mathop\{\ldots{}\}\},\{J\}\_\{\{n\}\_\{l\}\}(\{μ\}\_\{l\})).

{[}\href{coursse16.html}{prev}{]}
{[}\href{coursse16.html\#tailcoursse16.html}{prev-tail}{]}
{[}\href{coursse17.html}{front}{]}
{[}\href{coursch4.html\#coursse17.html}{up}{]}

\end{document}
