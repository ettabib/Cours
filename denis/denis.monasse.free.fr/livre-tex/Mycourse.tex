\documentclass{Math}
\usepackage[T1]{fontenc}
\usepackage[utf8]{inputenc}
\usepackage{amsmath, amssymb}
\usepackage{mathrsfs}
\usepackage{ntheorem}
\usepackage{tcolorbox}
\usepackage{xcolor}
\usepackage{makeidx}
\usepackage{hyperref}

% Configuration des couleurs
\definecolor{theoremcolor}{rgb}{0.0, 0.0, 0.5}
\definecolor{definitioncolor}{rgb}{0.0, 0.5, 0.0}
\definecolor{propositioncolor}{rgb}{0.5, 0.0, 0.0}
\definecolor{lemmacolor}{rgb}{0.5, 0.5, 0.0}
\definecolor{remarkcolor}{rgb}{0.0, 0.5, 0.5}

% Styles des environnements avec tcolorbox
\tcolorboxenvironment{theorem}{
  colback=theoremcolor!10!white,
  colframe=theoremcolor,
  fonttitle=\bfseries,
  coltitle=white,
  before skip=10pt,
  after skip=10pt,
  title={\thetcbtitle}
}
\tcolorboxenvironment{definition}{
  colback=definitioncolor!10!white,
  colframe=definitioncolor,
  fonttitle=\bfseries,
  coltitle=white,
  before skip=10pt,
  after skip=10pt,
  title={\thetcbtitle}
}
\tcolorboxenvironment{proposition}{
  colback=propositioncolor!10!white,
  colframe=propositioncolor,
  fonttitle=\bfseries,
  coltitle=white,
  before skip=10pt,
  after skip=10pt,
  title={\thetcbtitle}
}
\tcolorboxenvironment{lemma}{
  colback=lemmacolor!10!white,
  colframe=lemmacolor,
  fonttitle=\bfseries,
  coltitle=white,
  before skip=10pt,
  after skip=10pt,
  title={\thetcbtitle}
}
\tcolorboxenvironment{remark}{
  colback=remarkcolor!10!white,
  colframe=remarkcolor,
  fonttitle=\bfseries,
  coltitle=white,
  before skip=10pt,
  after skip=10pt,
  title={\thetcbtitle}
}

% Définition des nouveaux environnements
\newtheorem{theorem}{Theorem}[section]
\newenvironment{thm}
  {\begin{theorem}}
  {\end{theorem}}

\newtheorem{definition}{Definition}[section]
\newenvironment{de}
  {\begin{definition}}
  {\end{definition}}

\newtheorem{proposition}{Proposition}[section]
\newenvironment{prop}
  {\begin{proposition}}
  {\end{proposition}}

\newtheorem{lemma}{Lemma}[section]
\newenvironment{lem}
  {\vspace{1em}\begin{lemma}}
  {\end{lemma}}

\newtheorem{remark}{Remark}[section]
\newenvironment{rem}
  {\begin{remark}}
  {\end{remark}}

% Redéfinir l'environnement proof pour avoir une police plus petite
\theoremstyle{plain}
\newtheorem{proof}{Proof}
\AtBeginEnvironment{proof}{\small}

\makeindex

\begin{document}

\title{Cours de Mathématiques de Denis Monasse}
\author{Denis Monasse}
\date{\today}
\maketitle

% Table des matières
\tableofcontents

% Inclusion des chapitres
% \chapter{Ensembles et Structures}
\newpage
\part{Ensembles et Structures}


\subsubsection{1.1 Ensembles et relations}

\paragraph{1.1.1 Relations d'équivalences}

Définition~1.1.1 Soit E un ensemble. On appelle relation sur E toute
partie de E \times E. Si \mathcal{R} est une relation, on note habituellement x\mathcal{R}y à la
place de (x,y) \in\mathcal{R}.

On dit que la relation est

\begin{itemize}
\itemsep1pt\parskip0pt\parsep0pt
\item
  réflexive si \forall~~x \in E, x\mathcal{R}x,
\item
  symétrique si \forall~~x,y \in E,\quad
  x\mathcal{R}y \rigtharrow~ y\mathcal{R}x,
\item
  antisymétrique si \forall~~x,y \in
  E,\quad (x\mathcal{R}y\text et y\mathcal{R}x) \rigtharrow~ x = y,
\item
  transitive si \forall~~x,y,z \in
  E,\quad (x\mathcal{R}y\text et y\mathcal{R}z) \rigtharrow~ x\mathcal{R}z.
\end{itemize}

Définition~1.1.2 On appelle relation d'équivalence sur un ensemble E
toute relation réflexive, symétrique et transitive.

Définition~1.1.3 Pour un élément x de E, on appelle classe de x par
rapport à la relation d'équivalence \mathcal{R} l'ensemble des éléments y de E
tels que y\mathcal{R}x, notée C\_\mathcal{R}(x).

Proposition~1.1.1 Deux classes d'équivalences sont soit confondues, soit
dis\jmathointes.

Démonstration Soit x,x' \in E, et supposons que y \inC(x) \bigcapC(x'). Soit z
\inC(x). On a alors z\mathcal{R}x, x\mathcal{R}y (car on a y\mathcal{R}x et la relation est symétrique)
et y\mathcal{R}x'. Par transitivité de la relation, on a z\mathcal{R}x', soit z \inC(x'). On a
donc C(x) \subset~C(x') et comme x et x' \jmathouent un rôle symétrique on a aussi
C(x') \subset~C(x), et donc C(x') = C(x).

Remarque~1.1.1 On a les équivalences suivantes

x\mathcal{R}y \Leftrightarrow x \inC(y) \mathrel\Leftrightarrow
C(x) = C(y)

Définition~1.1.4 On appelle système de représentants des classes
d'équivalences une partie A de E telle que, pour tout y \in E, il existe
un unique x \in A tel que y\mathcal{R}x. Dans ce cas la famille
\left (C(x)\right )\_x\inA est une
partition de E.

Remarque~1.1.2 Inversement, à toute partition
(A\_i)\_i\inI d'un ensemble E, on peut associer une unique
relation d'équivalence en posant

x\mathcal{R}y \Leftrightarrow \exists~i \in I, x,y \in
A\_i

Les classes d'équivalences sont alors exactement les A\_i.

Définition~1.1.5 Soit E un ensemble et \mathcal{R} une relation d'équivalence sur
E. L'ensemble des classes d'équivalences des éléments de E est appelé
ensemble quotient de E par \mathcal{R} et noté E\diagup\mathcal{R}. L'application sur\jmathective
x\mapsto~C(x) de E dans E\diagup\mathcal{R} est appelée la
pro\jmathection (ou la sur\jmathection) canonique.

Remarque~1.1.3 A est un système de représentants des classes
d'équivalence modulo \mathcal{R}, si et seulement si la restriction de la
pro\jmathection canonique à A est bi\jmathective de A sur E\diagup\mathcal{R}.

\paragraph{1.1.2 Relations d'ordre}

Définition~1.1.6 Soit E un ensemble. On appelle relation d'ordre sur E
toute relation binaire \leqslant sur E qui est à la fois réflexive, transitive
et antisymétrique. On appelle relation d'ordre strict sur E toute
relation binaire \prec~ sur E qui est transitive et qui vérifie x \prec~ y \rigtharrow~
x\neq~y.

Remarque~1.1.4 On vérifie immédiatement que les applications

\leqslant \mapsto~ \prec~\text définie par x \prec~
y \Leftrightarrow (x \leqslant y\text et
x\neq~y)

et

\prec~ \mapsto~ \leqslant\text définie par x \leqslant
y \Leftrightarrow (x \prec~ y\text ou x = y)

réalisent des bi\jmathections réciproques l'une de l'autre entre les
relations d'ordre sur E et les relations d'ordre strict sur E. On
conviendra donc dans la suite d'associer ainsi canoniquement une
relation d'ordre strict à toute relation d'ordre et réciproquement.

Définition~1.1.7 On dit que la relation d'ordre \leqslant sur E est totale si

\forall~a,b \in E,\text on a ~a \leqslant
b\text ou b \leqslant a

Dans le cas contraire, on dit que la relation d'ordre est partielle.

Remarque~1.1.5 On prendra garde aux relations d'ordre partielles qui ont
des propriétés un peu déroutantes. L'exemple typique d'une telle
relation est la relation d'inclusion entre deux sous-ensembles d'un même
ensemble E.

\paragraph{1.1.3 Eléments extrémaux}

Définition~1.1.8 Soit (E,\leqslant) un ensemble ordonné et A une partie de E.
Soit a \in E. On dit que

\begin{itemize}
\itemsep1pt\parskip0pt\parsep0pt
\item
  a est un ma\jmathorant de A si \forall~~x \in A, x \leqslant a
\item
  a est un plus grand élément de A s'il appartient à A et est un
  ma\jmathorant de A
\end{itemize}

On définit de même minorant et plus petit élément.

Définition~1.1.9 Soit (E,\leqslant) un ensemble ordonné et A une partie de E.
Soit a \in E. On dit que a est borne supérieure de A si l'ensemble des
ma\jmathorants de A est non vide et admet a comme plus petit élément. On
définit de même une borne inférieure.

Remarque~1.1.6 L'antisymétrie de la relation d'ordre assure clairement
l'unicité d'un plus grand ou plus petit élément, et donc l'unicité d'une
borne supérieure ou inférieure. On prendra garde que les uns comme les
autres peuvent très bien ne pas exister, même lorsque la relation
d'ordre est totale comme le montre l'exemple de E = \mathbb{R}~,A = \mathbb{R}~.

Définition~1.1.10 Soit (E,\leqslant) un ensemble ordonné et A une partie de E.
Soit a \in E. On dit que a est un un élément maximal de A si

a \in A\text et \quad
\forall~~x \in A, a \leqslant x \rigtharrow~ a = x

autrement dit si A n'admet aucun élément strictement plus grand que a.
On définit de même la notion d'élément minimal de A.

Remarque~1.1.7 Lorsque \leqslant est une relation d'ordre total, il est clair
que la notion d'élément maximal coïncide avec la notion de plus grand
élément. Mais il n'en est pas de même pour une relation d'ordre partiel.
Plus grand élément signifie ''plus grand que tous les autres'' alors que
élément maximal signifie ''il n'y en a pas de strictement plus grand''.

\paragraph{1.1.4 L'axiome de Zorn}

L'existence d'éléments maximaux dans certains ensembles partiellement
ordonnés est souvent une propriété essentielle comme on le verra par la
suite. Cette existence est claire dans les ensembles finis. Dans les
ensembles infinis, elle résulte la plupart du temps d'un axiome appelé
l'axiome de Zorn. Pour cela on introduira la définition suivante

Définition~1.1.11 On dit qu'un ensemble ordonné (E,\leqslant) est inductif si
toute partie non vide totalement ordonnée de E admet un ma\jmathorant.

Axiome~1.1.1 (Zorn) Tout ensemble inductif admet un élément maximal.

Remarque~1.1.8 On montre que cet axiome est équivalent à l'axiome
beaucoup plus naturel suivant

Axiome~1.1.2 (Axiome du choix) Soit E un ensemble. Il existe une
application f de P(E) \diagdown\\varnothing~\ dans E
telle que \forall~~A \subset~ E,
A\neq~\varnothing~, f(A) \in A.

Remarque~1.1.9 Ce dernier axiome signifie simplement que l'on peut
choisir ''simultanément'' un élément a = f(A) dans chaque partie non
vide A de E.


\documentclass[]{article}
\usepackage[T1]{fontenc}
\usepackage{lmodern}
\usepackage{amssymb,amsmath}
\usepackage{ifxetex,ifluatex}
\usepackage{fixltx2e} % provides \textsubscript
% use upquote if available, for straight quotes in verbatim environments
\IfFileExists{upquote.sty}{\usepackage{upquote}}{}
\ifnum 0\ifxetex 1\fi\ifluatex 1\fi=0 % if pdftex
  \usepackage[utf8]{inputenc}
\else % if luatex or xelatex
  \ifxetex
    \usepackage{mathspec}
    \usepackage{xltxtra,xunicode}
  \else
    \usepackage{fontspec}
  \fi
  \defaultfontfeatures{Mapping=tex-text,Scale=MatchLowercase}
  \newcommand{\euro}{€}
\fi
% use microtype if available
\IfFileExists{microtype.sty}{\usepackage{microtype}}{}
\ifxetex
  \usepackage[setpagesize=false, % page size defined by xetex
              unicode=false, % unicode breaks when used with xetex
              xetex]{hyperref}
\else
  \usepackage[unicode=true]{hyperref}
\fi
\hypersetup{breaklinks=true,
            bookmarks=true,
            pdfauthor={},
            pdftitle={Cardinaux et entiers naturels},
            colorlinks=true,
            citecolor=blue,
            urlcolor=blue,
            linkcolor=magenta,
            pdfborder={0 0 0}}
\urlstyle{same}  % don't use monospace font for urls
\setlength{\parindent}{0pt}
\setlength{\parskip}{6pt plus 2pt minus 1pt}
\setlength{\emergencystretch}{3em}  % prevent overfull lines
\setcounter{secnumdepth}{0}
 
/* start css.sty */
.cmr-5{font-size:50%;}
.cmr-7{font-size:70%;}
.cmmi-5{font-size:50%;font-style: italic;}
.cmmi-7{font-size:70%;font-style: italic;}
.cmmi-10{font-style: italic;}
.cmsy-5{font-size:50%;}
.cmsy-7{font-size:70%;}
.cmex-7{font-size:70%;}
.cmex-7x-x-71{font-size:49%;}
.msbm-7{font-size:70%;}
.cmtt-10{font-family: monospace;}
.cmti-10{ font-style: italic;}
.cmbx-10{ font-weight: bold;}
.cmr-17x-x-120{font-size:204%;}
.cmsl-10{font-style: oblique;}
.cmti-7x-x-71{font-size:49%; font-style: italic;}
.cmbxti-10{ font-weight: bold; font-style: italic;}
p.noindent { text-indent: 0em }
td p.noindent { text-indent: 0em; margin-top:0em; }
p.nopar { text-indent: 0em; }
p.indent{ text-indent: 1.5em }
@media print {div.crosslinks {visibility:hidden;}}
a img { border-top: 0; border-left: 0; border-right: 0; }
center { margin-top:1em; margin-bottom:1em; }
td center { margin-top:0em; margin-bottom:0em; }
.Canvas { position:relative; }
li p.indent { text-indent: 0em }
.enumerate1 {list-style-type:decimal;}
.enumerate2 {list-style-type:lower-alpha;}
.enumerate3 {list-style-type:lower-roman;}
.enumerate4 {list-style-type:upper-alpha;}
div.newtheorem { margin-bottom: 2em; margin-top: 2em;}
.obeylines-h,.obeylines-v {white-space: nowrap; }
div.obeylines-v p { margin-top:0; margin-bottom:0; }
.overline{ text-decoration:overline; }
.overline img{ border-top: 1px solid black; }
td.displaylines {text-align:center; white-space:nowrap;}
.centerline {text-align:center;}
.rightline {text-align:right;}
div.verbatim {font-family: monospace; white-space: nowrap; text-align:left; clear:both; }
.fbox {padding-left:3.0pt; padding-right:3.0pt; text-indent:0pt; border:solid black 0.4pt; }
div.fbox {display:table}
div.center div.fbox {text-align:center; clear:both; padding-left:3.0pt; padding-right:3.0pt; text-indent:0pt; border:solid black 0.4pt; }
div.minipage{width:100%;}
div.center, div.center div.center {text-align: center; margin-left:1em; margin-right:1em;}
div.center div {text-align: left;}
div.flushright, div.flushright div.flushright {text-align: right;}
div.flushright div {text-align: left;}
div.flushleft {text-align: left;}
.underline{ text-decoration:underline; }
.underline img{ border-bottom: 1px solid black; margin-bottom:1pt; }
.framebox-c, .framebox-l, .framebox-r { padding-left:3.0pt; padding-right:3.0pt; text-indent:0pt; border:solid black 0.4pt; }
.framebox-c {text-align:center;}
.framebox-l {text-align:left;}
.framebox-r {text-align:right;}
span.thank-mark{ vertical-align: super }
span.footnote-mark sup.textsuperscript, span.footnote-mark a sup.textsuperscript{ font-size:80%; }
div.tabular, div.center div.tabular {text-align: center; margin-top:0.5em; margin-bottom:0.5em; }
table.tabular td p{margin-top:0em;}
table.tabular {margin-left: auto; margin-right: auto;}
div.td00{ margin-left:0pt; margin-right:0pt; }
div.td01{ margin-left:0pt; margin-right:5pt; }
div.td10{ margin-left:5pt; margin-right:0pt; }
div.td11{ margin-left:5pt; margin-right:5pt; }
table[rules] {border-left:solid black 0.4pt; border-right:solid black 0.4pt; }
td.td00{ padding-left:0pt; padding-right:0pt; }
td.td01{ padding-left:0pt; padding-right:5pt; }
td.td10{ padding-left:5pt; padding-right:0pt; }
td.td11{ padding-left:5pt; padding-right:5pt; }
table[rules] {border-left:solid black 0.4pt; border-right:solid black 0.4pt; }
.hline hr, .cline hr{ height : 1px; margin:0px; }
.tabbing-right {text-align:right;}
span.TEX {letter-spacing: -0.125em; }
span.TEX span.E{ position:relative;top:0.5ex;left:-0.0417em;}
a span.TEX span.E {text-decoration: none; }
span.LATEX span.A{ position:relative; top:-0.5ex; left:-0.4em; font-size:85%;}
span.LATEX span.TEX{ position:relative; left: -0.4em; }
div.float img, div.float .caption {text-align:center;}
div.figure img, div.figure .caption {text-align:center;}
.marginpar {width:20%; float:right; text-align:left; margin-left:auto; margin-top:0.5em; font-size:85%; text-decoration:underline;}
.marginpar p{margin-top:0.4em; margin-bottom:0.4em;}
.equation td{text-align:center; vertical-align:middle; }
td.eq-no{ width:5%; }
table.equation { width:100%; } 
div.math-display, div.par-math-display{text-align:center;}
math .texttt { font-family: monospace; }
math .textit { font-style: italic; }
math .textsl { font-style: oblique; }
math .textsf { font-family: sans-serif; }
math .textbf { font-weight: bold; }
.partToc a, .partToc, .likepartToc a, .likepartToc {line-height: 200%; font-weight:bold; font-size:110%;}
.chapterToc a, .chapterToc, .likechapterToc a, .likechapterToc, .appendixToc a, .appendixToc {line-height: 200%; font-weight:bold;}
.index-item, .index-subitem, .index-subsubitem {display:block}
.caption td.id{font-weight: bold; white-space: nowrap; }
table.caption {text-align:center;}
h1.partHead{text-align: center}
p.bibitem { text-indent: -2em; margin-left: 2em; margin-top:0.6em; margin-bottom:0.6em; }
p.bibitem-p { text-indent: 0em; margin-left: 2em; margin-top:0.6em; margin-bottom:0.6em; }
.paragraphHead, .likeparagraphHead { margin-top:2em; font-weight: bold;}
.subparagraphHead, .likesubparagraphHead { font-weight: bold;}
.quote {margin-bottom:0.25em; margin-top:0.25em; margin-left:1em; margin-right:1em; text-align:justify;}
.verse{white-space:nowrap; margin-left:2em}
div.maketitle {text-align:center;}
h2.titleHead{text-align:center;}
div.maketitle{ margin-bottom: 2em; }
div.author, div.date {text-align:center;}
div.thanks{text-align:left; margin-left:10%; font-size:85%; font-style:italic; }
div.author{white-space: nowrap;}
.quotation {margin-bottom:0.25em; margin-top:0.25em; margin-left:1em; }
h1.partHead{text-align: center}
.sectionToc, .likesectionToc {margin-left:2em;}
.subsectionToc, .likesubsectionToc {margin-left:4em;}
.subsubsectionToc, .likesubsubsectionToc {margin-left:6em;}
.frenchb-nbsp{font-size:75%;}
.frenchb-thinspace{font-size:75%;}
.figure img.graphics {margin-left:10%;}
/* end css.sty */

\title{Cardinaux et entiers naturels}
\author{}
\date{}

\begin{document}
\maketitle

\textbf{Warning: \href{http://www.math.union.edu/locate/jsMath}{jsMath}
requires JavaScript to process the mathematics on this page.\\ If your
browser supports JavaScript, be sure it is enabled.}

\begin{center}\rule{3in}{0.4pt}\end{center}

{[}\href{coursse3.html}{next}{]} {[}\href{coursse1.html}{prev}{]}
{[}\href{coursse1.html\#tailcoursse1.html}{prev-tail}{]}
{[}\hyperref[tailcoursse2.html]{tail}{]}
{[}\href{coursch2.html\#coursse2.html}{up}{]}

\subsubsection{1.2 Cardinaux et entiers naturels}

\paragraph{1.2.1 Notion de cardinal}

Définition~1.2.1 On dit que deux ensembles E et F ont même cardinal s'il
existe une bijection de E sur F. On notera
\textbackslash{}mathop\{Card\}E ou encore \textbar{}E\textbar{} le
cardinal d'un ensemble E.

Remarque~1.2.1 La relation il existe une bijection de E sur F est bien
entendu une relation d'équivalence~; les cardinaux sont en quelque sorte
les classes d'équivalence pour cette relation (pas tout à fait puisque
l'ensemble de tous les ensembles n'existe pas).

Définition~1.2.2 On définit alors des opérations sur les cardinaux en
posant

\textbackslash{}begin\{eqnarray*\} \textbackslash{}mathop\{Card\}A
+\textbackslash{}mathop\{ Card\}B\& =\& \textbackslash{}mathop\{Card\}A
×\textbackslash{}\{0\textbackslash{}\} ∪ B
×\textbackslash{}\{1\textbackslash{}\}\%\&
\textbackslash{}\textbackslash{}
\textbackslash{}mathop\{Card\}A.\textbackslash{}mathop\{Card\}B\& =\&
\textbackslash{}mathop\{Card\}A × B \%\&
\textbackslash{}\textbackslash{} \textbackslash{}end\{eqnarray*\}

Remarque~1.2.2 Cette définition est justifiée par le fait que si on a
\textbackslash{}mathop\{Card\}A =\textbackslash{}mathop\{ Card\}A' et
\textbackslash{}mathop\{Card\}B =\textbackslash{}mathop\{ Card\}B', on a
aussi

\textbackslash{}mathop\{Card\}A ×\textbackslash{}\{0\textbackslash{}\} ∪
B ×\textbackslash{}\{1\textbackslash{}\} =\textbackslash{}mathop\{
Card\}A' ×\textbackslash{}\{0\textbackslash{}\} ∪ B'
×\textbackslash{}\{1\textbackslash{}\}

et \textbackslash{}mathop\{Card\}A × B =\textbackslash{}mathop\{
Card\}A' × B', comme on le vérifie facilement en construisant les
bijections appropriées.

Définition~1.2.3 On pose \textbackslash{}mathop\{Card\}A
≤\textbackslash{}mathop\{ Card\}B s'il existe une injection de A dans B.

On admettra que c'est une relation d'ordre total sur les cardinaux~; les
seuls points non évidents sont l'antisymétrie et la totalité~:
l'antisymétrie constitue le théorème de Cantor-Bernstein qui dit que
s'il existe une injection de A dans B et une injection de B dans A,
alors il existe une bijection de A sur B~; la totalité résulte assez
facilement de l'axiome de Zorn.

\paragraph{1.2.2 Les entiers naturels}

On dira qu'un ensemble A est fini si \textbackslash{}mathop\{Card\}A
\textless{}\textbackslash{}mathop\{ Card\}A + 1 (c'est équivalent à~: il
n'existe pas de bijection de A sur une partie stricte de A). L'ensemble
des cardinaux finis forme alors un ensemble totalement ordonné appelé
ensemble des entiers naturels et noté ℕ. Il vérifie les propriétés
suivantes qui le caractérisent à un isomorphime près d'ensembles
ordonnés

Axiome~1.2.1 (de Peano) ℕ est un ensemble infini où toute partie non
vide a un plus petit élément et où toute partie non vide majorée a un
plus grand élément

On en déduit immédiatement l'existence d'un successeur de tout élément a
de ℕ et on montre en théorie des cardinaux que ce n'est autre que a + 1.

L'existence d'un plus petit élément pour toute partie non vide conduit
immédiatement aux deux résultats suivants~:

Théorème~1.2.1 (Principe de récurrence forte) Soit P(n) une propriété
qui peut être vraie ou fausse pour tout entier naturel n. On suppose que
P(\{n\}\_\{0\}) est vraie et que si P(n) est vraie, alors P(n + 1) est
vraie. Alors P(n) est vraie pour tout n ≥ \{n\}\_\{0\}.

Démonstration Soit en effet X l'ensemble des n ≥ \{n\}\_\{0\} tels que
P(n) soit fausse et supposons que X est non vide~; alors il admet un
plus petit élément \{n\}\_\{1\} ∈X. Comme
\{n\}\_\{0\}\textbackslash{}mathrel\{∉\}X, on a \{n\}\_\{1\}
\textgreater{} \{n\}\_\{0\}~; mais alors \{n\}\_\{1\} −
1\textbackslash{}mathrel\{∉\}X et \{n\}\_\{1\} − 1 ≥ \{n\}\_\{0\}~; donc
P(\{n\}\_\{1\} − 1) est vraie, et il en est de même de P((\{n\}\_\{1\} −
1) + 1) = P(\{n\}\_\{1\}), soit
\{n\}\_\{1\}\textbackslash{}mathrel\{∉\}X. C'est absurde. Donc X = ∅, et
par conséquent, P(n) est vraie pour tout n ≥ \{n\}\_\{0\}.

Théorème~1.2.2 (Principe de récurrence faible) On suppose que
P(\{n\}\_\{0\}) est vraie et que si P(\{n\}\_\{0\} + 1),P(\{n\}\_\{0\} +
2),\textbackslash{}mathop\{\textbackslash{}mathop\{\ldots{}\}\},P(n)
sont vraies, alors P(n + 1) est vraie. Alors P(n) est vraie pour tout n
≥ \{n\}\_\{0\}.

Démonstration Soit en effet X l'ensemble des n ≥ \{n\}\_\{0\} tels que
P(n) soit fausse et supposons que X est non vide~; alors il admet un
plus petit élément \{n\}\_\{1\} ∈X. Comme
\{n\}\_\{0\}\textbackslash{}mathrel\{∉\}X, on a \{n\}\_\{1\}
\textgreater{} \{n\}\_\{0\}~; mais alors
\{n\}\_\{0\},\textbackslash{}mathop\{\textbackslash{}mathop\{\ldots{}\}\},\{n\}\_\{1\}
− 1\textbackslash{}mathrel\{∉\}X et donc
P(\{n\}\_\{0\}),\textbackslash{}mathop\{\textbackslash{}mathop\{\ldots{}\}\},P(\{n\}\_\{1\}
− 1) sont vraies~; il en est donc de même de P(\{n\}\_\{1\}), soit
\{n\}\_\{1\}\textbackslash{}mathrel\{∉\}X. C'est absurde. Donc X = ∅, et
par conséquent, P(n) est vraie pour tout n ≥ \{n\}\_\{0\}.

{[}\href{coursse3.html}{next}{]} {[}\href{coursse1.html}{prev}{]}
{[}\href{coursse1.html\#tailcoursse1.html}{prev-tail}{]}
{[}\href{coursse2.html}{front}{]}
{[}\href{coursch2.html\#coursse2.html}{up}{]}

\end{document}

 \section{1.3 Groupes}

\subsection{1.3.1 Définitions et premières propriétés}

Définition~1.3.1 On dit qu'un couple $(G,∗)$ d'un ensemble $G$ et d'une loi
interne $∗$ sur $G$ est un groupe si la loi est associative, possède un
élément neutre et si tout élément a un inverse, autrement dit

\begin{itemize}
\itemsep1pt\parskip0pt\parsep0pt
\item
  $\forall x,y,z \in G, x ∗ (y ∗ z) = (x ∗ y) ∗ z$
\item
  il existe un élément neutre $e_G \in G$ tel que
  $\forall x \in G, e_G ∗ x = x ∗ e_G = x$
\item
  pour tout $x \in G$, il existe un inverse $y \in G$ tel que $x ∗ y = y ∗ x = e_G$
\end{itemize}

Remarque~1.3.1 On montre alors que l'élément neutre est unique, de même
que l'inverse d'un élément. On dit que le groupe est abélien ou
commutatif si la loi est commutative ($x ∗ y = y ∗ x$). Enfin un groupe
possède la propriété essentielle suivante

Proposition~1.3.1 Pour tout $a \in G$, les applications
$x \mapsto a ∗ x$ et $x \mapsto x ∗ a$
sont des bijections de $G$ dans $G$.

Démonstration En effet, si $a'$ désigne l'inverse de $a$, $a ∗ x = a ∗ y \Rightarrow a' ∗ a ∗ x = a' ∗ a ∗ y \Rightarrow x = y$.

Notations habituelles Les groupes sont en général notés
multiplicativement ($xy$) ou additivement ($x + y$). Les conventions
suivantes sont alors utilisées

\begin{center}
\begin{tabular}{|c|c|c|}
\hline
Notation & multiplicative & additive \\
\hline
$x ∗ y$ & $xy$ & $x + y$ \\
élément neutre & $e_G$ ou $e$ & $0_G$ ou $0$ \\
inverse & $x^{-1}$ & $-x$ \\
puissance & $x^n$ & $nx$ \\
\hline
\end{tabular}
\end{center}

Définition~1.3.2 Soit $G$ un groupe. On dit que deux éléments $x$ et $y$ de $G$
sont conjugués s'il existe $g \in G$ tel que $y = gxg^{-1}$.

Remarque~1.3.2 On montre facilement qu'il s'agit d'une relation
d'équivalence sur $G$, dont les classes d'équivalence sont appelées les
classes de conjugaison.

\subsection{1.3.2 Sous-groupes}

On dit qu'une partie $H$ de $G$ en est un sous-groupe si elle est stable
pour la loi interne et est munie d'une structure de groupe pour la loi
induite. On montre alors facilement que l'élément neutre de $H$ doit être
celui de $G$, de même que l'inverse d'un élément dans $G$ doit être le même
que l'inverse dans $H$. On aboutit alors aux deux caractérisations
suivantes :

Proposition~1.3.2 (Caractérisation 1) Une partie $H$ de $G$ en est un
sous-groupe si et seulement si elle vérifie (i)
$H \neq \varnothing$, (ii) $\forall x,y \in H,
xy \in H$, (iii) $\forall x \in H, x^{-1} \in H$.

Démonstration Les conditions sont bien entendu nécessaires. Elles sont
également suffisantes car si $H$ vérifie ces conditions, il contient un
élément $x$ donc aussi $x^{-1}$, donc aussi $e_G =
xx^{-1}$. Comme de plus $H$ est stable pour la loi de groupe,
c'est un sous-groupe de $G$.

Proposition~1.3.3 (Caractérisation 2) Une partie $H$ de $G$ en est un
sous-groupe si et seulement si elle vérifie (i)
$H \neq \varnothing$, (ii) $\forall x,y \in H,
xy^{-1} \in H$.

Démonstration Les conditions sont bien entendu nécessaires. Elles sont
également suffisantes car si $H$ vérifie ces conditions, il contient un
élément $x$, donc aussi $e_G = xx^{-1}$ (prendre $y = x$).
Mais alors $e_G, x \in G \Rightarrow x^{-1} =
e_G x^{-1} \in G$ et donc $x,y \in G \Rightarrow x,y^{-1} \in G
\Rightarrow xy = x(y^{-1})^{-1} \in G$ ce qui ramène à la première
caractérisation.

Exemple~1.3.1 $\{e\}$ et $G$ sont bien
évidemment des sous-groupes de $G$ (dits triviaux). De même $Z(G) =
\{x \in G | \forall y \in G, xy = yx\}$ est un sous-groupe de $G$ appelé le centre de $G$.

Remarque~1.3.3 On vérifie immédiatement que toute intersection de
sous-groupes est encore un sous-groupe à l'aide de la première
caractérisation et du fait que tout sous-groupe contient $e_G$
(ce qui garantit que l'intersection est non vide). On obtient donc la
proposition suivante

Proposition~1.3.4 Soit $A$ une partie de $G$. Alors l'ensemble des
sous-groupes de $G$ qui contiennent $A$ admet un plus petit élément (pour
l'inclusion). On l'appelle le groupe engendré par la partie $A$ et on le
note $\text{Groupe}(A)$ ou $\langle A \rangle$. On a les deux caractérisations suivantes

\begin{align*} 
\text{Groupe}(A) &= \bigcap_{H \text{ sous-groupe de } G \atop A \subset H} H \\
&= \{x_1 \ldots x_k | k \geq 0 \text{ et } x_i \in A \cup A^{-1}\}
\end{align*}

Démonstration La première caractérisation est évidente, puisque
$\bigcap_{H \text{ sous-groupe de } G \atop A \subset H} H$ est un sous-groupe de $G$ (comme intersection de sous-groupes de $G$),
contenant $A$ et inclus dans tout sous-groupe de $G$ contenant $A$.

En ce qui concerne la seconde (plus constructive), posons $H =
\{x_1 \ldots x_k | k \geq 0 \text{ et } x_i \in A \cup A^{-1}\}$. L'une quelconque des caractérisations
des sous-groupes montre que $H$ est un sous-groupe de $G$ ; il contient bien
évidemment $A$. De plus, si $H'$ est un sous-groupe de $G$ contenant $A$, il
doit contenir tous les éléments de $A$, tous leurs inverses, et tous les
produits d'éléments de $A$ et de leurs inverses, donc il doit contenir $H$.
Donc $H$ est bien le plus petit sous-groupe de $G$ contenant $A$.

\subsection{1.3.3 Quotient par un sous-groupe}

Considérons $H$ un sous-groupe de $G$. Un calcul élémentaire montre le
résultat suivant

Théorème~1.3.5 La relation $\mathcal{R}$ définie par $x \mathcal{R} y
\Leftrightarrow x^{-1}y \in H$ est une relation
d'équivalence sur $G$. Si $x$ appartient à $G$, la classe d'équivalence de $x$
est $xH = \{xh | h \in H\}$ (en particulier la classe de $e$ est $H$). L'ensemble
quotient $G/\mathcal{R}$ est noté $G/H$.

Démonstration On a

\begin{itemize}
\itemsep1pt\parskip0pt\parsep0pt
\item
  $e_G = x^{-1}x \in H \Rightarrow x \mathcal{R} x$ (réflexivité)
\item
  $x \mathcal{R} y \Rightarrow x^{-1}y \in H \Rightarrow (x^{-1}y)^{-1} \in H \Rightarrow y^{-1}x \in H \Rightarrow y \mathcal{R} x$ (symétrie)
\item
  $x \mathcal{R} y$ et $y \mathcal{R} z \Rightarrow x^{-1}y, y^{-1}z \in H \Rightarrow x^{-1}z = x^{-1}y y^{-1}z \in H \Rightarrow x \mathcal{R} z$ (transitivité)
\end{itemize}

On a de même

Théorème~1.3.6 La relation $\mathcal{R}'$ définie par $x \mathcal{R}' y
\Leftrightarrow yx^{-1} \in H$ est une relation
d'équivalence sur $G$. Si $x$ appartient à $G$, la classe d'équivalence de $x$
est $Hx = \{hx | h \in H\}$ (en particulier la classe de $e$ est $H$). L'ensemble
quotient $G/\mathcal{R}'$ est noté $H \setminus G$.

Remarque~1.3.4 Bien évidemment, si le groupe $G$ est commutatif, les deux
relations sont confondues ainsi que les deux ensembles $G/H$ et $H \setminus G$.

Lorsque le groupe est noté additivement, on a $x \mathcal{R} y
\Leftrightarrow x - y \in H$ et $C_\mathcal{R}(x) = x + H$.

Théorème~1.3.7 Soit $G$ un groupe commutatif (noté additivement) et $H$ un
sous-groupe de $G$. On définit alors une loi de groupe sur $G/H$ en posant

$(x + H) + (y + H) = (x + y) + H$

Le groupe ainsi obtenu est appelé groupe quotient du groupe commutatif $G$
par le sous-groupe $H$.

Démonstration Le principal point est de vérifier que l'on définit bien
une application, c'est à dire que si $x + H = x' + H$ et $y + H = y' + H$,
alors $x + y + H = x' + y' + H$. Mais dans ce cas, il existe $h$ et $h'$ dans
$H$ tels que $x' = x + h$ et $y' = y + h'$, d'où

\begin{align*} 
(x' + y') + H &= (x + h + y + h') + H \\
&= (x + y) + (h + h' + H) = (x + y) + H
\end{align*}

puisque $h + h' \in H$ et que $\forall h \in H, h + H = H$.
(On remarquera le rôle essentiel joué par la commutativité du groupe $G$
lors de ce calcul.)

A partir de là, dans $G/H$, l'associativité est évidente, l'élément neutre
est $0 + H = H$ et l'opposé de $x + H$ est $(-x) + H$.

Lorsque le groupe n'est pas commutatif, on est amené à introduire les
notions suivantes :

Définition~1.3.3 On dit qu'un sous-groupe $H$ de $G$ est distingué dans $G$ si

$\forall x \in G, \forall h \in H,
xhx^{-1} \in H$

(autrement dit $H$ est stable par conjugaison par tous les éléments de $G$).

Exemple~1.3.2 $\{e\}$, $G$, $Z(G)$ sont des
sous-groupes distingués de $G$.

Dans un groupe abélien, tout sous-groupe est distingué.

On a alors

Théorème~1.3.8 Soit $H$ un sous-groupe de $G$. Alors les relations
d'équivalence $\mathcal{R}$ et $\mathcal{R}'$ définies ci-dessus coïncident si et seulement si
$H$ est distingué dans $G$. Dans ce cas on a $\forall x \in G, xH = Hx$. L'ensemble $G/H$ est muni d'une structure de groupe en posant
$xH \cdot yH = xyH$.

Démonstration Les deux relations d'équivalence sont égales si et
seulement si leurs classes d'équivalence sont égales. On a donc

\begin{align*} 
\mathcal{R} = \mathcal{R}' &\Leftrightarrow \forall x \in G, xH = Hx \\
&\Leftrightarrow \forall x \in G, xHx^{-1} = H \\
&\Leftrightarrow \forall x \in G, xHx^{-1} \subset H \text{ et } H \subset xHx^{-1} \\
&\Leftrightarrow \forall x \in G, xHx^{-1} \subset H \text{ et } x^{-1}Hx \subset H \\
&\Leftrightarrow \forall x \in G, xHx^{-1} \subset H
\end{align*}

(car $x \in G \Rightarrow x^{-1} \in G$). Or ceci équivaut au fait que $H$ soit
distingué dans $G$.

En ce qui concerne la structure de groupe sur $G/H$, le seul point non
évident est le fait que l'on définit bien une application en posant
$xH \cdot yH = xyH$, c'est-à-dire que si $xH = x'H$ et $yH = y'H$, alors $x'y'H =
xyH$ ; mais dans ce cas il existe $h,k \in H$ tels que $x' = xh$, $y' = yk$, d'où
$x'y' = xhyk = xy(y^{-1}hy)k$. Mais, comme $H$ est distingué dans
$G$, $y \in G, h \in H \Rightarrow y^{-1}hy \in H$ et donc $k' = (y^{-1}hy)k
\in H$, soit $k'H = H$ (car $H$
 \documentclass[]{article}
\usepackage[T1]{fontenc}
\usepackage{lmodern}
\usepackage{amssymb,amsmath}
\usepackage{ifxetex,ifluatex}
\usepackage{fixltx2e} % provides \textsubscript
% use upquote if available, for straight quotes in verbatim environments
\IfFileExists{upquote.sty}{\usepackage{upquote}}{}
\ifnum 0\ifxetex 1\fi\ifluatex 1\fi=0 % if pdftex
  \usepackage[utf8]{inputenc}
\else % if luatex or xelatex
  \ifxetex
    \usepackage{mathspec}
    \usepackage{xltxtra,xunicode}
  \else
    \usepackage{fontspec}
  \fi
  \defaultfontfeatures{Mapping=tex-text,Scale=MatchLowercase}
  \newcommand{\euro}{€}
\fi
% use microtype if available
\IfFileExists{microtype.sty}{\usepackage{microtype}}{}
\ifxetex
  \usepackage[setpagesize=false, % page size defined by xetex
              unicode=false, % unicode breaks when used with xetex
              xetex]{hyperref}
\else
  \usepackage[unicode=true]{hyperref}
\fi
\hypersetup{breaklinks=true,
            bookmarks=true,
            pdfauthor={},
            pdftitle={Anneaux et corps},
            colorlinks=true,
            citecolor=blue,
            urlcolor=blue,
            linkcolor=magenta,
            pdfborder={0 0 0}}
\urlstyle{same}  % don't use monospace font for urls
\setlength{\parindent}{0pt}
\setlength{\parskip}{6pt plus 2pt minus 1pt}
\setlength{\emergencystretch}{3em}  % prevent overfull lines
\setcounter{secnumdepth}{0}
 
/* start css.sty */
.cmr-5{font-size:50%;}
.cmr-7{font-size:70%;}
.cmmi-5{font-size:50%;font-style: italic;}
.cmmi-7{font-size:70%;font-style: italic;}
.cmmi-10{font-style: italic;}
.cmsy-5{font-size:50%;}
.cmsy-7{font-size:70%;}
.cmex-7{font-size:70%;}
.cmex-7x-x-71{font-size:49%;}
.msbm-7{font-size:70%;}
.cmtt-10{font-family: monospace;}
.cmti-10{ font-style: italic;}
.cmbx-10{ font-weight: bold;}
.cmr-17x-x-120{font-size:204%;}
.cmsl-10{font-style: oblique;}
.cmti-7x-x-71{font-size:49%; font-style: italic;}
.cmbxti-10{ font-weight: bold; font-style: italic;}
p.noindent { text-indent: 0em }
td p.noindent { text-indent: 0em; margin-top:0em; }
p.nopar { text-indent: 0em; }
p.indent{ text-indent: 1.5em }
@media print {div.crosslinks {visibility:hidden;}}
a img { border-top: 0; border-left: 0; border-right: 0; }
center { margin-top:1em; margin-bottom:1em; }
td center { margin-top:0em; margin-bottom:0em; }
.Canvas { position:relative; }
li p.indent { text-indent: 0em }
.enumerate1 {list-style-type:decimal;}
.enumerate2 {list-style-type:lower-alpha;}
.enumerate3 {list-style-type:lower-roman;}
.enumerate4 {list-style-type:upper-alpha;}
div.newtheorem { margin-bottom: 2em; margin-top: 2em;}
.obeylines-h,.obeylines-v {white-space: nowrap; }
div.obeylines-v p { margin-top:0; margin-bottom:0; }
.overline{ text-decoration:overline; }
.overline img{ border-top: 1px solid black; }
td.displaylines {text-align:center; white-space:nowrap;}
.centerline {text-align:center;}
.rightline {text-align:right;}
div.verbatim {font-family: monospace; white-space: nowrap; text-align:left; clear:both; }
.fbox {padding-left:3.0pt; padding-right:3.0pt; text-indent:0pt; border:solid black 0.4pt; }
div.fbox {display:table}
div.center div.fbox {text-align:center; clear:both; padding-left:3.0pt; padding-right:3.0pt; text-indent:0pt; border:solid black 0.4pt; }
div.minipage{width:100%;}
div.center, div.center div.center {text-align: center; margin-left:1em; margin-right:1em;}
div.center div {text-align: left;}
div.flushright, div.flushright div.flushright {text-align: right;}
div.flushright div {text-align: left;}
div.flushleft {text-align: left;}
.underline{ text-decoration:underline; }
.underline img{ border-bottom: 1px solid black; margin-bottom:1pt; }
.framebox-c, .framebox-l, .framebox-r { padding-left:3.0pt; padding-right:3.0pt; text-indent:0pt; border:solid black 0.4pt; }
.framebox-c {text-align:center;}
.framebox-l {text-align:left;}
.framebox-r {text-align:right;}
span.thank-mark{ vertical-align: super }
span.footnote-mark sup.textsuperscript, span.footnote-mark a sup.textsuperscript{ font-size:80%; }
div.tabular, div.center div.tabular {text-align: center; margin-top:0.5em; margin-bottom:0.5em; }
table.tabular td p{margin-top:0em;}
table.tabular {margin-left: auto; margin-right: auto;}
div.td00{ margin-left:0pt; margin-right:0pt; }
div.td01{ margin-left:0pt; margin-right:5pt; }
div.td10{ margin-left:5pt; margin-right:0pt; }
div.td11{ margin-left:5pt; margin-right:5pt; }
table[rules] {border-left:solid black 0.4pt; border-right:solid black 0.4pt; }
td.td00{ padding-left:0pt; padding-right:0pt; }
td.td01{ padding-left:0pt; padding-right:5pt; }
td.td10{ padding-left:5pt; padding-right:0pt; }
td.td11{ padding-left:5pt; padding-right:5pt; }
table[rules] {border-left:solid black 0.4pt; border-right:solid black 0.4pt; }
.hline hr, .cline hr{ height : 1px; margin:0px; }
.tabbing-right {text-align:right;}
span.TEX {letter-spacing: -0.125em; }
span.TEX span.E{ position:relative;top:0.5ex;left:-0.0417em;}
a span.TEX span.E {text-decoration: none; }
span.LATEX span.A{ position:relative; top:-0.5ex; left:-0.4em; font-size:85%;}
span.LATEX span.TEX{ position:relative; left: -0.4em; }
div.float img, div.float .caption {text-align:center;}
div.figure img, div.figure .caption {text-align:center;}
.marginpar {width:20%; float:right; text-align:left; margin-left:auto; margin-top:0.5em; font-size:85%; text-decoration:underline;}
.marginpar p{margin-top:0.4em; margin-bottom:0.4em;}
.equation td{text-align:center; vertical-align:middle; }
td.eq-no{ width:5%; }
table.equation { width:100%; } 
div.math-display, div.par-math-display{text-align:center;}
math .texttt { font-family: monospace; }
math .textit { font-style: italic; }
math .textsl { font-style: oblique; }
math .textsf { font-family: sans-serif; }
math .textbf { font-weight: bold; }
.partToc a, .partToc, .likepartToc a, .likepartToc {line-height: 200%; font-weight:bold; font-size:110%;}
.chapterToc a, .chapterToc, .likechapterToc a, .likechapterToc, .appendixToc a, .appendixToc {line-height: 200%; font-weight:bold;}
.index-item, .index-subitem, .index-subsubitem {display:block}
.caption td.id{font-weight: bold; white-space: nowrap; }
table.caption {text-align:center;}
h1.partHead{text-align: center}
p.bibitem { text-indent: -2em; margin-left: 2em; margin-top:0.6em; margin-bottom:0.6em; }
p.bibitem-p { text-indent: 0em; margin-left: 2em; margin-top:0.6em; margin-bottom:0.6em; }
.paragraphHead, .likeparagraphHead { margin-top:2em; font-weight: bold;}
.subparagraphHead, .likesubparagraphHead { font-weight: bold;}
.quote {margin-bottom:0.25em; margin-top:0.25em; margin-left:1em; margin-right:1em; text-align:\jmathustify;}
.verse{white-space:nowrap; margin-left:2em}
div.maketitle {text-align:center;}
h2.titleHead{text-align:center;}
div.maketitle{ margin-bottom: 2em; }
div.author, div.date {text-align:center;}
div.thanks{text-align:left; margin-left:10%; font-size:85%; font-style:italic; }
div.author{white-space: nowrap;}
.quotation {margin-bottom:0.25em; margin-top:0.25em; margin-left:1em; }
h1.partHead{text-align: center}
.sectionToc, .likesectionToc {margin-left:2em;}
.subsectionToc, .likesubsectionToc {margin-left:4em;}
.subsubsectionToc, .likesubsubsectionToc {margin-left:6em;}
.frenchb-nbsp{font-size:75%;}
.frenchb-thinspace{font-size:75%;}
.figure img.graphics {margin-left:10%;}
/* end css.sty */

\title{Anneaux et corps}
\author{}
\date{}

\begin{document}
\maketitle

\textbf{Warning: 
requires JavaScript to process the mathematics on this page.\\ If your
browser supports JavaScript, be sure it is enabled.}

\begin{center}\rule{3in}{0.4pt}\end{center}

{[}
{[}
{[}{]}
{[}

\subsubsection{1.4 Anneaux et corps}

\paragraph{1.4.1 Généralités sur les anneaux}

Définition~1.4.1 Un anneau est un triplet (A,+,.) tel que (A,+) est un
groupe abélien et dans lequel la loi de multiplication est associative,
possède un élément neutre 1\_A et est distributive par rapport à
l'addition.

Remarque~1.4.1 Nous admettrons dans la suite que l'on puisse avoir
1\_A = 0\_A, auquel cas on obtient immédiatement que A =
\0\_A\.

Définition~1.4.2 On dit qu'un élément x d'un anneau A (non réduit à
\0\) est inversible s'il possède un
inverse (nécessairement unique) pour la multiplication. L'ensemble
A^\times des éléments inversibles d'un anneau A forme un groupe
multiplicatif.

Définition~1.4.3 On dit qu'un anneau A est intègre s'il est commutatif
et si xy = 0 \rigtharrow~ x = 0\text ou y = 0.

Définition~1.4.4 On dit qu'une partie B de A en est un sous-anneau si
elle est stable pour les deux lois et si pour les lois induites, B est
un anneau de même élément unité que A.

Proposition~1.4.1 Une partie B de A en est un sous-anneau si et
seulement si elle vérifie

\begin{itemize}
\itemsep1pt\parskip0pt\parsep0pt
\item
  (i) 1\_A \in B
\item
  (ii) \forall~~x,y \in B, x - y \in B
\item
  (iii) \forall~~x,y \in B, xy \in B
\end{itemize}

Démonstration Elémentaire.

\paragraph{1.4.2 Idéaux et quotients}

Définition~1.4.5 On dit qu'une partie I de A est un idéal à gauche
(resp. à droite) de A si c'est un sous-groupe additif de A et si
\forall~a \in A,\\forall~~x \in I, ax \in
I (resp. xa \in I).

Définition~1.4.6 Un idéal bilatère est une partie qui est à la fois un
idéal à gauche et à droite (exemples triviaux
\0\ et A).

Proposition~1.4.2 Une partie I de A en est un idéal à gauche si et
seulement si elle vérifie

\begin{itemize}
\itemsep1pt\parskip0pt\parsep0pt
\item
  (i) I\neq~\varnothing~
\item
  (ii) \forall~~x,y \in I, x - y \in I
\item
  (iii) \forall~a \in A,\\forall~~x \in
  I, ax \in I
\end{itemize}

Proposition~1.4.3 Soit I un idéal à gauche de l'anneau A. Alors les
propriétés suivantes sont équivalentes

\begin{itemize}
\itemsep1pt\parskip0pt\parsep0pt
\item
  (i) I = A
\item
  (ii) 1\_A \in I
\item
  (iii) I \bigcap A^\times\neq~\varnothing~
\end{itemize}

Démonstration On a clairement (i) \rigtharrow~(iii). De plus (iii) \rigtharrow~(ii) puisque si
x \in I \bigcap A^\times, alors x^-1 \in A,x \in I \rigtharrow~ 1\_A =
x^-1x \in I. Enfin (ii) \rigtharrow~(i) puisque, si (ii) est vérifié, a \in
A,1\_A \in I \rigtharrow~ a = a1\_A \in I, soit A \subset~ I et A = I.

Remarque~1.4.2 Soit I un idéal à gauche de l'anneau A. Puisque c'est un
sous-groupe additif, on peut parler de l'ensemble quotient A\diagupI qui est
un groupe additif. Mais si l'on veut munir A\diagupI d'une structure d'anneau
raisonnable, il faut supposer que I est un idéal bilatère.

Théorème~1.4.4 Soit I un idéal bilatère. On munit l'ensemble quotient
A\diagupI d'une structure d'anneau en posant

\pi~(a) + \pi~(b) = \pi~(a + b),\quad \pi~(a)\pi~(b) = \pi~(ab)

autrement dit (a + I) + (b + I) = (a + b) + I et (a + I)(b + I) = ab +
I.

Démonstration Le seul point qui ne résulte pas d'un calcul évident est
qu'on obtient bien une application en posant (a + I)(b + I) = ab + I,
c'est-à-dire que si a + I = a' + I et b + I = b' + I, alors ab + I =
a'b' + I. Mais dans ce cas, on a a' = a + i,b' = b + \jmath (i,\jmath \in I), et
donc a'b' = ab + ib + a\jmath + i\jmath. Puisque I est un idéal bilatère, on a ib
+ a\jmath + i\jmath \in I et donc ib + a\jmath + i\jmath + I = I (un sous-groupe est stable
par translation par ses éléments), soit a'b' + I = ab + (ib + a\jmath + i\jmath +
I) = ab + I.

\paragraph{1.4.3 Morphisme d'anneaux}

Définition~1.4.7 On dit que f : A \rightarrow~ B est un morphisme d'anneaux si on a

\begin{itemize}
\itemsep1pt\parskip0pt\parsep0pt
\item
  (i)f(1\_A) = 1\_B
\item
  (ii) \forall~~x,y \in A, f(x + y) = f(x) + f(y)
\item
  (iii) \forall~~x,y \in A, f(xy) = f(x)f(y)
\end{itemize}

Remarque~1.4.3 On a alors f(0) = 0, f(-x) = -f(x) (comme pour tout
morphisme de groupes) et de plus, si x est inversible, f(x) l'est
également et f(x^-1) = f(x)^-1.

Théorème~1.4.5 (factorisation canonique). Soit f : A \rightarrow~ B un morphisme
d'anneaux. Alors
\mathrmKer~f =
\x \in A∣f(x) =
0\ est un idéal bilatère de A et
\mathrmIm~f = f(A) est un
sous-anneau de B. Il existe une unique application
\overlinef :
A\diagup\mathrmKer~f
\rightarrow~\mathrmIm~f vérifiant
\forall~x \in A, \overlinef~(x
+ \mathrmKer~f) = f(x).
L'application \overlinef est un isomorphisme
d'anneaux.

Démonstration L'existence, l'unicité et la bi\jmathectivité de
\overlinef résultent du théorème analogue sur les
groupes. Le fait que ce soit un morphisme d'anneaux résulte d'un calcul
élémentaire.

\paragraph{1.4.4 Corps}

Définition~1.4.8 Un corps est un anneau non réduit à
\0\ où tout élément non nul est
inversible.

Proposition~1.4.6 Soit K un corps. Alors les seuls idéaux de K sont
\0\ et K. En particulier tout
morphisme d'un corps dans un anneau non réduit à
\0\ est in\jmathectif.

Démonstration En effet un idéal non nul doit contenir un élément non
nul, donc un élément inversible, et donc doit être égal au corps tout
entier. Le noyau d'un morphisme d'un corps dans un anneau étant un idéal
du corps, ce doit être soit \0\
(auquel cas le morphisme est in\jmathectif), soit le corps tout entier. Mais
ceci est exclu puisqu'on doit avoir f(1\_K) = 1\_A.

Remarque~1.4.4 Tout corps commutatif est un anneau intègre. Inversement,
tout anneau intègre fini est un corps~: en effet si a \in A, l'application
x\mapsto~ax doit être in\jmathective à cause de
l'intégrité de A, donc bi\jmathective puisque A est fini~; il doit donc
exister b\_1 tel que ab\_1 = 1\_A~; de même il
doit exister b\_2 tel que b\_2a = 1\_A et alors
b\_2 = b\_21\_A = b\_2ab\_1 =
1\_Ab\_1 = b\_1 ce qui montre que b\_1 =
b\_2 est inverse de a.

\paragraph{1.4.5 Idéaux maximaux}

Définition~1.4.9 Soit A un anneau commutatif. On dit qu'un idéal I de A,
distinct de A, est maximal si les seuls idéaux contenant I sont I et A.

Proposition~1.4.7 Soit A un anneau commutatif et I un idéal de A. Alors
A est maximal si et seulement si A\diagupI est un corps.

Démonstration Supposons tout d'abord que A\diagupI est un corps. On a donc
A\diagupI\neq~\0\ et
donc I\neq~A. Soit \pi~ : A \rightarrow~ A\diagupI la pro\jmathection
canonique de A sur A\diagupI définie par \pi~(x) = x + I. Soit J un idéal de A
contenant strictement I et soit a \in J \diagdown I. Alors \pi~(a) \in A\diagupI
\diagdown\0\ et donc a + I est un élément
inversible de A\diagupI. Il existe donc b \in A tel que (a + I)(b + I) =
1\_A + I ce qui se traduit encore par ab = 1\_A + i avec
i \in I. Mais alors i \in I \subset~ J et ab \in J (puisque a \in J), et donc
1\_A = ab - i \in J. L'idéal J qui contient l'élément unité est
donc nécessairement égal à A, et donc I est un idéal maximal de A.

Inversement, supposons que I est un idéal maximal de A et soit a + I un
élément non nul de A\diagupI, si bien que a +
I\neq~0\_A + I, soit
a∉I~; alors J = I + aA est un idéal de A qui
contient strictement I, c'est donc A. On a alors 1\_A \in A = I +
aA et donc il existe b \in A et i \in I tel que 1\_A = i + ab. Mais
alors (a + I)(b + I) = ab + I = 1\_A + I ce qui montre que a + I
est un élément inversible de A\diagupI, qui est donc un corps.

Théorème~1.4.8 Soit A un anneau commutatif. Alors tout idéal de A
distinct de A est contenu dans un idéal maximal (non unique).

Démonstration Soit I un idéal distinct de A et soit X l'ensemble des
idéaux distincts de A et qui contiennent I, ordonné par l'inclusion.
Nous allons utiliser le théorème de Zorn pour montrer que X admet un
élément maximal (qui sera bien évidemment un idéal maximal de A
contenant I). Il nous suffit de montrer que toute partie Y totalement
ordonnée de X admet un ma\jmathorant~; pour cela considérons J\_Y
= \⋃  \_J\inY~J
et montrons que J\_Y est dans X (auquel cas il sera bien un
ma\jmathorant tel qu'on le cherche).

Tout d'abord, J\_Y est bien un idéal de A~:

\begin{itemize}
\itemsep1pt\parskip0pt\parsep0pt
\item
  si a\_1,a\_2 \in J\_Y, il existe
  J\_1,J\_2 \inY tels que a\_1 \in
  J\_1,a\_2 \in J\_2~; mais comme Y est une partie
  totalement ordonnée, on a soit J\_1 \subset~ J\_2, soit
  J\_2 \subset~ J\_1~; supposons par exemple que J\_1 \subset~
  J\_2, alors a\_1 et a\_2 sont dans
  J\_2 et donc a\_1 - a\_2 \in J\_2 \subset~
  J\_Y~; donc J\_Y est un sous-groupe additif de A
\item
  si a \in J\_Y et b \in A, il existe J \inY tel que a \in J et alors ab
  \in J \subset~ J\_Y, ce qui montre que J\_Y est bien un idéal
  de A.
\end{itemize}

Cet idéal contient bien évidemment I. Supposons qu'il est égal à A, on
aurait alors 1\_A \in J\_Y et donc il existerait J \inY tel
que 1\_A \in J, soit J = A~; mais ceci contredit la définition de
X. On a donc bien J\_Y\inX.

\paragraph{1.4.6 Idéaux et anneaux principaux}

Remarque~1.4.5 Soit A un anneau commutatif. On vérifie facilement que si
a \in A, l'idéal engendré par a est aA =
\ax∣x \in A\
c'est-à-dire encore l'ensemble des multiples de a.

Proposition~1.4.9 Soit A un anneau intègre. Alors aA = bA
\Leftrightarrow \exists~x \in
A^\times,b = ax.

Démonstration Supposons que aA = bA~; on a b \in bA et donc il existe x \in
A tel que b = ax et de même il existe y \in A tel que a = by~; on a donc a
= axy. Si a\neq~0, alors on a 1\_A = xy
et donc x est un élément inversible qui vérifie b = ax~; si par contre a
= 0, on a aussi b = 0 et alors b = a1\_A avec 1\_A
inversible.

Inversement, si b = ax avec x inversible, on a xA = A (car x inversible)
et donc bA = axA = aA.

Remarque~1.4.6 Autrement dit, l'élément qui engendre l'idéal est unique
à la multiplication près par un élément inversible de l'anneau.

Définition~1.4.10 On dit que a et b sont associés, et on note a ∼ b,
s'il existe x \in A^\times tel que b = ax (c'est-à-dire lorsqu'ils
engendrent le même idéal). C'est bien évidemment une relation
d'équivalence.

Définition~1.4.11 Un idéal I de A est dit principal s'il est engendré
par un élément. L'anneau A est dit principal s'il est intègre et si tout
idéal de A est principal.

Théorème~1.4.10 (propriété de Noether). Soit A un anneau principal, et
soit (I\_n)\_n\in\mathbb{N}~ une suite croissante d'idéaux. Alors
cette suite est stationnaire, c'est à dire qu'il existe N \in \mathbb{N}~ tel que
\forall~n ≥ N, I\_n = I\_N~.

Démonstration Considérons en effet I =\
⋃  \_n\in\mathbb{N}~I\_n~. Montrons que
c'est un idéal de A.

\begin{itemize}
\itemsep1pt\parskip0pt\parsep0pt
\item
  il est évidemment non vide
\item
  si a\_1,a\_2 \in I, il existe n\_1,n\_2
  \in \mathbb{N}~ tels que a\_1 \in I\_n\_1,a\_2 \in
  I\_n\_2~; mais alors, si n =\
  max(n\_1,n\_2), on a a\_1,a\_2 \in
  I\_n, soit a\_1 - a\_2 \in I\_n \subset~ I~;
  donc I est un sous groupe additif de A
\item
  si a \in I et b \in A, il existe n \in \mathbb{N}~ tel que a \in I\_n, alors ab
  \in I\_n \subset~ I~; donc I est bien un idéal de A
\end{itemize}

Puisque A est principal, il existe a \in I tel que I = aA~; mais alors, il
existe N \in \mathbb{N}~ tel que a \in I\_N et donc I = aA \subset~ I\_N.
Alors, pour tout n ≥ N, on a I\_N \subset~ I\_n \subset~ I \subset~
I\_N, et donc I\_n = I\_N.

On dit que a divise b et on note a∣b s'il
existe x \in A tel que b = ax (soit bA \subset~ aA)

Définition~1.4.12 Soit A un anneau intègre, a et b deux éléments de A.
Soit d \in A. On dit que d est un PGCD de a et b si on a

\forall~~x \in A,\quad
x∣a\text et
x∣b \mathrel\Leftrightarrow
x∣d

Soit m \in A. On dit que m est un PPCM de a et b si

\forall~~x \in A,\quad
a∣x\text et
b∣x \mathrel\Leftrightarrow
m∣x

Remarque~1.4.7 On vérifie immédiatement qu'un PGCD est défini à la
multiplication par un élément inversible près~; il en est de même d'un
PPCM.

Théorème~1.4.11 Soit A un anneau principal, a et b deux éléments de A.
Soit d \in A vérifiant aA + bA = dA. Alors d est un PGCD de a et b. Il est
caractérisé par la propriété de Bézout~:
d∣a,d\mathrel∣b et
\exists~u,v \in A, d = ua + vb.

Démonstration En effet d∣a
\Leftrightarrow aA \subset~ dA, d\mathrel∣b
\Leftrightarrow bA \subset~ dA et donc
d∣a et d\mathrel∣b est
équivalent à aA + bA \subset~ dA. De même \exists~u,v \in A, d
= ua + vb \Leftrightarrow dA \subset~ aA + bA. D'où la
caractérisation par Bézout. Le fait que d est un PGCD en découle
immédiatement.

De même

Théorème~1.4.12 Soit A un anneau principal, a et b deux éléments de A.
Soit m \in A vérifiant aA \bigcap bA = mA. Alors m est un PPCM de a et b.

Définition~1.4.13 Soit A un anneau, a et b deux éléments de A. On dit
que a et b sont premiers entre eux si 1 est un PGCD de a et b (autrement
dit si les seuls diviseurs communs à a et b sont les éléments
inversibles de l'anneau).

Théorème~1.4.13 (Bézout). Soit A un anneau principal, a et b dans A. On
a équivalence de (i) a et b sont premiers entre eux (ii) aA + bA = A
(iii) \exists~u,v \in A, ua + vb = 1.

Théorème~1.4.14 (Gauss). Soit A un anneau principal, a,b et c dans A. On
suppose que (i)~a∣bc (ii)~a et b sont
premiers entre eux. Alors a divise c.

Démonstration On écrit ua + vb = 1 d'où c = uac + vbc les deux termes de
la somme étant divisibles par a.

Corollaire~1.4.15 Soit A un anneau principal,
a\_1,\\ldots,a\_n~
des éléments deux à deux premiers entre eux. Soit b \in A. On suppose que
\forall~i, a\_i\mathrel∣~b.
Alors
a\_1\\ldotsa\_n\mathrel∣~b.

Démonstration Récurrence facile. On suppose que
a\_1\\ldotsa\_k\mathrel∣~b,
on écrit b =
a\_1\\ldotsa\_k~c
et le théorème de Gauss nécessite que
a\_k+1∣c, soit
a\_1\\ldotsa\_ka\_k+1\mathrel∣~b.

Définition~1.4.14 Soit A un anneau. On dit qu'un élément non nul a de A
est irréductible s'il est non inversible et vérifie les trois propriétés
équivalentes suivantes (i) x∣a \rigtharrow~ x ∼ a ou x \in
A^\times (ii) a = bc \rigtharrow~ a ∼ b ou a ∼ c (iii) a = bc \rigtharrow~ b \in
A^\times ou c \in A^\times.

Théorème~1.4.16 Soit A un anneau principal et a \in A
\diagdown\0\ non inversible. Alors les
propriétés suivantes sont équivalentes (i) a est irréductible dans A
(ii) a∣bc \rigtharrow~ a\mathrel∣b ou
a∣c (iii) A\diagupaA est un anneau intègre (iv)
A\diagupaA est un corps.

Démonstration On a clairement (ii) \Leftrightarrow (iii),
(ii) \rigtharrow~(i) et (iv) \rigtharrow~(iii). Si a est irréductible, il est premier avec
tout élément qu'il ne divise pas~; soit X un élément non nul de A\diagupaA, X
= \pi~(x)~; puisque X\neq~0, a ne divise pas x, donc
est premier avec x~; d'après Bézout, il existe u,v tels que ua + vx = 1
et on a alors \pi~(u)\pi~(x) = 1, donc X = \pi~(x) est inversible dans A\diagupaA qui
est donc un corps, ce qui montre que (i) \rigtharrow~(iv) et achève la
démonstration.

Définition~1.4.15 On notera P un système de représentants des éléments
irréductibles de A à la multiplication par un élément inversible près.

On a alors le théorème suivant

Théorème~1.4.17 (décomposition en éléments irréductibles). Soit A un
anneau principal et a \in A, a\neq~0. Alors a
s'écrit de manière unique sous la forme a =
up\_1^n\_1\\ldotsp\_k^n\_k~
avec u inversible, k ≥ 0,
p\_1,\\ldots,p\_k~
\inP,
n\_1,\\ldots,n\_k~
\textgreater{} 0.

Démonstration L'existence découle immédiatement d'une double application
du lemme de Noether~: d'une part a doit être divisible par un élément
irréductible sinon on pourrait trouver une chaîne infinie de diviseurs
de a et donc une chaîne infinie strictement croissante d'idéaux~; enfin
le processus de division par des éléments irréductibles doit s'arrêter
au bout d'un nombre fini d'opérations pour la même raison~; quand il
s'achève, c'est que l'on est tombé sur un élément inversible et donc sur
une décomposition adéquate.

L'unicité provient bien évidemment du théorème de Gauss et d'une petite
récurrence sur m\_1 +
\\ldots~ +
m\_k~: si on a

up\_1^m\_1
\\ldotsp\_k^m\_k~
 = vq\_1^n\_1
\\ldotsq\_l^n\_l~


alors q\_l doit diviser le terme de droite, donc l'un des
p\_i (puisqu'ils sont irréductibles), donc être égal à ce
p\_i que l'on peut supposer être p\_k. Mais alors en
simplifiant par p\_k (car A est intègre) on obtient

up\_1^m\_1
\\ldotsp\_k^m\_k-1~
= vq\_ 1^n\_1
\\ldotsq\_l^n\_l-1~

et d'après l'hypothèse de récurrence u = v, les p\_i sont les
mêmes que les q\_i et les exposants sont les mêmes. Le démarrage
de la récurrence avec m\_1 +
\\ldots~ +
m\_k = 0 est laissé au lecteur.

Définition~1.4.16 Pour p \inP on notera v\_p(a) la puissance de p
qui intervient dans la décomposition en facteurs irréductibles de a.

Remarque~1.4.8 On a alors a∣b
\Leftrightarrow \forall~~p \inP,
v\_p(a) \leq v\_p(b) d'où

Proposition~1.4.18 Soit a,b \in A. Alors un PGCD de a et b est
\∏ ~
\_p\inPp^min(v\_p(a),v\_p(b))~
et un PPCM est \∏ ~
\_p\inPp^max(v\_p(a),v\_p(b))~.

\paragraph{1.4.7 Anneaux euclidiens}

Définition~1.4.17 Soit A un anneau. On appelle stathme euclidien sur A
toute application v:A \diagdown\0\ \rightarrow~ \mathbb{N}~
vérifiant a∣b \rigtharrow~ v(a) \leq v(b). On dit que A
intègre est un anneau euclidien s'il existe sur A un stathme euclidien v
sur A vérifiant

\forall~~(a,b) \in
A^2,b\neq~0,
\exists~q,r \in A,\quad a = bq + r

avec r = 0 ou v(r) \textless{} v(b) (division euclidienne)

Théorème~1.4.19 Tout anneau euclidien est principal.

Démonstration Soit I un idéal non réduit à
\0\ et b un élément de I tel que v(b)
soit minimal. Si a \in I, \exists~q,r \in
A,\quad a = bq + r avec r = 0 ou v(r) \textless{} v(b)~;
la dernière possibilité est exclue car r = a - bq \in I. Donc I \subset~ bA,
l'inclusion réciproque étant évidente.

Remarque~1.4.9 Algorithme d'Euclide. Dans un anneau euclidien on dispose
d'un algorithme pour déterminer un PGCD de a et b en construisant des
suites q\_n et r\_n de la manière suivante~:
r\_0 = a, r\_1 = b et pour n ≥ 1, r\_n-1 =
q\_nr\_n + r\_n+1 avec r\_n+1 = 0
(auquel cas on stoppe l'algorithme) ou v(r\_n+1) \textless{}
v(r\_n). La construction s'arrête au bout d'un nombre fini
d'opérations par décroissance stricte de la suite d'entiers naturels
v(r\_n)~; comme l'ensemble des diviseurs communs à r\_n
et r\_n+1 reste constant (invariant de boucle), le dernier reste
non nul est un PGCD de a et b.

Remarque~1.4.10 On verra en particulier que ℤ est un anneau euclidien
pour v(a) = \textbar{}a\textbar{} et que K{[}X{]} est un anneau
euclidien pour v(P) = deg~ P.

\paragraph{1.4.8 L'anneau ℤ. Caractéristique d'un anneau}

Théorème~1.4.20 Soit a \in ℤ et b \in \mathbb{N}~,b\neq~0. Alors il existe un unique
couple (q,r) \in ℤ \times \mathbb{N}~ tel que a = bq + r et 0 \leq r \textless{} b.

Démonstration Il suffit de considérer q =\
max\x∣a - bx ≥
0\ en remarquant que cet ensemble est ma\jmathoré par a.

Corollaire~1.4.21 L'anneau ℤ est un anneau euclidien, donc principal.

Remarque~1.4.11 On choisit pour représentants des éléments irréductibles
les nombres premiers.

Soit alors A un anneau d'élément unité 1\_A et considérons
l'application f : ℤ \rightarrow~ A définie par f(n) = n1\_A =
\left \ \cases
1\_A +
\\ldots~ +
1\_A (n\text fois)&si n ≥ 0
\cr -1\_A
-\\ldots~ -
1\_A (-n\text fois)&si n \textless{} 0 
\right .. On vérifie immédiatement que f est un morphisme
d'anneaux dont le noyau est un idéal de ℤ, donc de la forme mℤ pour un
unique m \in \mathbb{N}~.

Définition~1.4.18 L'entier m est appelée la caractéristique de A.
L'image de f est un sous-anneau de A isomorphe à ℤ\diagupmℤ.

Proposition~1.4.22 Soit A un anneau intègre. Alors sa caractéristique
est soit 0 soit un nombre premier. S'il est de caractéristique 0, il
contient un sous-anneau isomorphe à ℤ. S'il est de caractéristique p
premier, il contient un sous-corps isomorphe à ℤ\diaguppℤ.

\paragraph{1.4.9 Théorème chinois, indicateur d'Euler}

Théorème~1.4.23 Soit A un anneau principal,
a\_1,\\ldots,a\_n~
des éléments de A deux à deux premiers entre eux. Alors, pour tous
x\_1,\\ldots,x\_n~
\in A, il existe un élément x \in A, unique à un multiple près de a =
a\_1\\ldotsa\_n~,
tel que

x ≡ x\_1
(mod\,\,a\_1),\quad
\\ldots~\quad
, x ≡ x\_n (mod\,\,a\_n)

(en notant x ≡ y (mod\,\,a) pour
a∣x - y)

Démonstration En ce qui concerne l'unicité, remarquons que si x et y
conviennent, alors x - y doit être divisible à la fois par
a\_1,\\ldots,a\_n~
et donc par leur produit puisqu'ils sont deux à deux premiers entre eux.

Montrons maintenant l'existence par récurrence sur n. Pour n = 1, il n'y
a rien à démontrer. Lorsque n = 2, prenons u\_1 et u\_2
tels que u\_1a\_1 + u\_2a\_2 = 1 et
posons x = x\_2u\_1a\_1 +
x\_1u\_2a\_2. On a alors

x ≡ x\_2u\_1a\_1 ≡
x\_2(u\_1a\_1 + u\_2a\_2) ≡
x\_2 (mod\,\,a\_2)

et de même x ≡ x\_1
(mod\,\,a\_1). Donc x convient.

Supposons maintenant le résultat démontré pour n - 1 et choisissons y
tel que

y ≡ x\_1
(mod\,\,a\_1),\quad
\\ldots~\quad
, y ≡ x\_n-1
(mod\,\,a\_n-1)

Comme
a\_1\\ldotsa\_n-1~
et a\_n sont premiers entre eux, on peut trouver u et v tels que
ua\_1\\ldotsa\_n-1~
+ va\_n = 1. Posons x =
x\_nua\_1\\ldotsa\_n-1~
+ yva\_n, on a alors pour i \leq n - 1

x ≡ yva\_n ≡ y(va\_n +
ua\_1\\ldotsa\_n-1~)
= y (mod\,\,a\_i)

et

x ≡
x\_nua\_1\\ldotsa\_n-1~
≡
x\_n(ua\_1\\ldotsa\_n-1~
+ va\_n) = x\_n
(mod\,\,a\_n)

ce qui montre que x convient.

Corollaire~1.4.24 Soit A un anneau principal,
a\_1,\\ldots,a\_n~
des éléments de A deux à deux premiers entre eux. Alors l'anneau
A\diagup(a\_1\\ldotsa\_n~)A
est isomorphe à l'anneau produit A\diagupa\_1A
\times⋯ \times A\diagupa\_nA.

Démonstration En effet le théorème ci dessus ne fait que traduire la
bi\jmathectivité de l'application de
A\diagup(a\_1\\ldotsa\_n~)A
dans A\diagupa\_1A \times⋯ \times A\diagupa\_nA
définie par

x +
a\_1\\ldotsa\_nA\mathrel\mapsto~~(x
+
a\_1A,\\ldots~,x
+ a\_nA)

Or on vérifie immédiatement que cette application est un morphisme
d'anneaux.

Définition~1.4.19 A tout nombre entier naturel n ≥ 2, on associe son
indicateur d'Euler \phi(n) défini par

\begin{align*} \phi(n)& =&
\mathrmCard~\x
\in {[}1,n - 1{]}∣x ∧ n = 1\
\%& \\ & =&
Card\\overlinex~
\in
ℤ\diagupnℤ∣\overlinex\text
engendre (ℤ\diagupnℤ,+)\\%&
\\ & =&
Card~\left
(ℤ\diagupnℤ\right )^\times \%&
\\ \end{align*}

Un isomorphisme d'anneaux réalisant évidemment une bi\jmathection entre les
éléments inversibles des deux anneaux et les éléments inversibles de
ℤ\diagupa\_1ℤ \times⋯ \times ℤ\diagupa\_nℤ étant
exactement les n-uples
(x\_1,\\ldots,x\_n~)
dont toutes les composantes x\_i sont inversibles, on en déduit
que

Théorème~1.4.25 Soit
a\_1,\\ldots,a\_n~
des éléments de ℤ deux à deux premiers entre eux. Alors
\phi(a\_1\\ldotsa\_n~)
=
\phi(a\_1)\\ldots\phi(a\_n~).

Remarque~1.4.12 Ceci va nous permettre de calculer \phi(n) à l'aide d'une
décomposition en nombres premiers de n, soit n =
p\_1^\alpha~\_1\\ldotsp\_k^\alpha~\_k~.
En effet, si n est de la forme n = p^k où p est premier, les
éléments inférieurs à n non premiers avec n sont exactement
p,2p,3p,\\ldots,p^k-1~p,
et donc \phi(p^k) = p^k - p^k-1 =
p^k-1(p - 1). D'après le résultat précédent on a donc, si n =
p\_1^\alpha~\_1\\ldotsp\_k^\alpha~\_k~,

\phi(n) =
p\_1^\alpha~\_1-1\\ldotsp~\_
k^\alpha~\_k-1(p\_ 1 -
1)\\ldots(p\_k~
- 1)

{[}
{[}
{[}
{[}

\end{document}

% \documentclass[]{article}
\usepackage[T1]{fontenc}
\usepackage{lmodern}
\usepackage{amssymb,amsmath}
\usepackage{ifxetex,ifluatex}
\usepackage{fixltx2e} % provides \textsubscript
% use upquote if available, for straight quotes in verbatim environments
\IfFileExists{upquote.sty}{\usepackage{upquote}}{}
\ifnum 0\ifxetex 1\fi\ifluatex 1\fi=0 % if pdftex
  \usepackage[utf8]{inputenc}
\else % if luatex or xelatex
  \ifxetex
    \usepackage{mathspec}
    \usepackage{xltxtra,xunicode}
  \else
    \usepackage{fontspec}
  \fi
  \defaultfontfeatures{Mapping=tex-text,Scale=MatchLowercase}
  \newcommand{\euro}{€}
\fi
% use microtype if available
\IfFileExists{microtype.sty}{\usepackage{microtype}}{}
\ifxetex
  \usepackage[setpagesize=false, % page size defined by xetex
              unicode=false, % unicode breaks when used with xetex
              xetex]{hyperref}
\else
  \usepackage[unicode=true]{hyperref}
\fi
\hypersetup{breaklinks=true,
            bookmarks=true,
            pdfauthor={},
            pdftitle={Polynomes `a une variable},
            colorlinks=true,
            citecolor=blue,
            urlcolor=blue,
            linkcolor=magenta,
            pdfborder={0 0 0}}
\urlstyle{same}  % don't use monospace font for urls
\setlength{\parindent}{0pt}
\setlength{\parskip}{6pt plus 2pt minus 1pt}
\setlength{\emergencystretch}{3em}  % prevent overfull lines
\setcounter{secnumdepth}{0}
 
/* start css.sty */
.cmr-5{font-size:50%;}
.cmr-7{font-size:70%;}
.cmmi-5{font-size:50%;font-style: italic;}
.cmmi-7{font-size:70%;font-style: italic;}
.cmmi-10{font-style: italic;}
.cmsy-5{font-size:50%;}
.cmsy-7{font-size:70%;}
.cmex-7{font-size:70%;}
.cmex-7x-x-71{font-size:49%;}
.msbm-7{font-size:70%;}
.cmtt-10{font-family: monospace;}
.cmti-10{ font-style: italic;}
.cmbx-10{ font-weight: bold;}
.cmr-17x-x-120{font-size:204%;}
.cmsl-10{font-style: oblique;}
.cmti-7x-x-71{font-size:49%; font-style: italic;}
.cmbxti-10{ font-weight: bold; font-style: italic;}
p.noindent { text-indent: 0em }
td p.noindent { text-indent: 0em; margin-top:0em; }
p.nopar { text-indent: 0em; }
p.indent{ text-indent: 1.5em }
@media print {div.crosslinks {visibility:hidden;}}
a img { border-top: 0; border-left: 0; border-right: 0; }
center { margin-top:1em; margin-bottom:1em; }
td center { margin-top:0em; margin-bottom:0em; }
.Canvas { position:relative; }
li p.indent { text-indent: 0em }
.enumerate1 {list-style-type:decimal;}
.enumerate2 {list-style-type:lower-alpha;}
.enumerate3 {list-style-type:lower-roman;}
.enumerate4 {list-style-type:upper-alpha;}
div.newtheorem { margin-bottom: 2em; margin-top: 2em;}
.obeylines-h,.obeylines-v {white-space: nowrap; }
div.obeylines-v p { margin-top:0; margin-bottom:0; }
.overline{ text-decoration:overline; }
.overline img{ border-top: 1px solid black; }
td.displaylines {text-align:center; white-space:nowrap;}
.centerline {text-align:center;}
.rightline {text-align:right;}
div.verbatim {font-family: monospace; white-space: nowrap; text-align:left; clear:both; }
.fbox {padding-left:3.0pt; padding-right:3.0pt; text-indent:0pt; border:solid black 0.4pt; }
div.fbox {display:table}
div.center div.fbox {text-align:center; clear:both; padding-left:3.0pt; padding-right:3.0pt; text-indent:0pt; border:solid black 0.4pt; }
div.minipage{width:100%;}
div.center, div.center div.center {text-align: center; margin-left:1em; margin-right:1em;}
div.center div {text-align: left;}
div.flushright, div.flushright div.flushright {text-align: right;}
div.flushright div {text-align: left;}
div.flushleft {text-align: left;}
.underline{ text-decoration:underline; }
.underline img{ border-bottom: 1px solid black; margin-bottom:1pt; }
.framebox-c, .framebox-l, .framebox-r { padding-left:3.0pt; padding-right:3.0pt; text-indent:0pt; border:solid black 0.4pt; }
.framebox-c {text-align:center;}
.framebox-l {text-align:left;}
.framebox-r {text-align:right;}
span.thank-mark{ vertical-align: super }
span.footnote-mark sup.textsuperscript, span.footnote-mark a sup.textsuperscript{ font-size:80%; }
div.tabular, div.center div.tabular {text-align: center; margin-top:0.5em; margin-bottom:0.5em; }
table.tabular td p{margin-top:0em;}
table.tabular {margin-left: auto; margin-right: auto;}
div.td00{ margin-left:0pt; margin-right:0pt; }
div.td01{ margin-left:0pt; margin-right:5pt; }
div.td10{ margin-left:5pt; margin-right:0pt; }
div.td11{ margin-left:5pt; margin-right:5pt; }
table[rules] {border-left:solid black 0.4pt; border-right:solid black 0.4pt; }
td.td00{ padding-left:0pt; padding-right:0pt; }
td.td01{ padding-left:0pt; padding-right:5pt; }
td.td10{ padding-left:5pt; padding-right:0pt; }
td.td11{ padding-left:5pt; padding-right:5pt; }
table[rules] {border-left:solid black 0.4pt; border-right:solid black 0.4pt; }
.hline hr, .cline hr{ height : 1px; margin:0px; }
.tabbing-right {text-align:right;}
span.TEX {letter-spacing: -0.125em; }
span.TEX span.E{ position:relative;top:0.5ex;left:-0.0417em;}
a span.TEX span.E {text-decoration: none; }
span.LATEX span.A{ position:relative; top:-0.5ex; left:-0.4em; font-size:85%;}
span.LATEX span.TEX{ position:relative; left: -0.4em; }
div.float img, div.float .caption {text-align:center;}
div.figure img, div.figure .caption {text-align:center;}
.marginpar {width:20%; float:right; text-align:left; margin-left:auto; margin-top:0.5em; font-size:85%; text-decoration:underline;}
.marginpar p{margin-top:0.4em; margin-bottom:0.4em;}
.equation td{text-align:center; vertical-align:middle; }
td.eq-no{ width:5%; }
table.equation { width:100%; } 
div.math-display, div.par-math-display{text-align:center;}
math .texttt { font-family: monospace; }
math .textit { font-style: italic; }
math .textsl { font-style: oblique; }
math .textsf { font-family: sans-serif; }
math .textbf { font-weight: bold; }
.partToc a, .partToc, .likepartToc a, .likepartToc {line-height: 200%; font-weight:bold; font-size:110%;}
.chapterToc a, .chapterToc, .likechapterToc a, .likechapterToc, .appendixToc a, .appendixToc {line-height: 200%; font-weight:bold;}
.index-item, .index-subitem, .index-subsubitem {display:block}
.caption td.id{font-weight: bold; white-space: nowrap; }
table.caption {text-align:center;}
h1.partHead{text-align: center}
p.bibitem { text-indent: -2em; margin-left: 2em; margin-top:0.6em; margin-bottom:0.6em; }
p.bibitem-p { text-indent: 0em; margin-left: 2em; margin-top:0.6em; margin-bottom:0.6em; }
.paragraphHead, .likeparagraphHead { margin-top:2em; font-weight: bold;}
.subparagraphHead, .likesubparagraphHead { font-weight: bold;}
.quote {margin-bottom:0.25em; margin-top:0.25em; margin-left:1em; margin-right:1em; text-align:justify;}
.verse{white-space:nowrap; margin-left:2em}
div.maketitle {text-align:center;}
h2.titleHead{text-align:center;}
div.maketitle{ margin-bottom: 2em; }
div.author, div.date {text-align:center;}
div.thanks{text-align:left; margin-left:10%; font-size:85%; font-style:italic; }
div.author{white-space: nowrap;}
.quotation {margin-bottom:0.25em; margin-top:0.25em; margin-left:1em; }
h1.partHead{text-align: center}
.sectionToc, .likesectionToc {margin-left:2em;}
.subsectionToc, .likesubsectionToc {margin-left:4em;}
.subsubsectionToc, .likesubsubsectionToc {margin-left:6em;}
.frenchb-nbsp{font-size:75%;}
.frenchb-thinspace{font-size:75%;}
.figure img.graphics {margin-left:10%;}
/* end css.sty */

\title{Polynomes `a une variable}
\author{}
\date{}

\begin{document}
\maketitle

\textbf{Warning: \href{http://www.math.union.edu/locate/jsMath}{jsMath}
requires JavaScript to process the mathematics on this page.\\ If your
browser supports JavaScript, be sure it is enabled.}

\begin{center}\rule{3in}{0.4pt}\end{center}

{[}\href{coursse6.html}{next}{]} {[}\href{coursse4.html}{prev}{]}
{[}\href{coursse4.html\#tailcoursse4.html}{prev-tail}{]}
{[}\hyperref[tailcoursse5.html]{tail}{]}
{[}\href{coursch2.html\#coursse5.html}{up}{]}

\subsubsection{1.5 Polynômes à une variable}

On désignera par A un anneau commutatif.

\paragraph{1.5.1 L'anneau des séries formelles à coefficients dans A}

Définition~1.5.1 On appelle anneau des séries formelles à coefficients
dans A, l'anneau ainsi défini~: son ensemble de base est l'ensemble
\{A\}\^{}\{ℕ\} des suites d'éléments de A muni des lois suivantes

(\{a\}\_\{n\}) + (\{b\}\_\{n\}) = (\{a\}\_\{n\} +
\{b\}\_\{n\})\textbackslash{}quad (\{a\}\_\{n\})(\{b\}\_\{n\}) =
(\{c\}\_\{n\})

avec

\{c\}\_\{n\} =\{ \textbackslash{}mathop\{∑
\}\}\_\{p+q=n\}\{a\}\_\{p\}\{b\}\_\{q\} =\{ \textbackslash{}mathop\{∑
\}\}\_\{p=0\}\^{}\{n\}\{a\}\_\{ p\}\{b\}\_\{n−p\}

L'élément unité est la suite
(1,0,\textbackslash{}mathop\{\textbackslash{}mathop\{\ldots{}\}\},0,\textbackslash{}mathop\{\textbackslash{}mathop\{\ldots{}\}\})
et on note X =
(0,1,0,\textbackslash{}mathop\{\textbackslash{}mathop\{\ldots{}\}\},0,\textbackslash{}mathop\{\textbackslash{}mathop\{\ldots{}\}\}).

Démonstration Facile, sauf peut-être en ce qui concerne l'associativité
de la multiplication. Mais on a~:
((\{a\}\_\{n\})(\{b\}\_\{n\}))(\{c\}\_\{n\}) = (\{d\}\_\{n\}) avec
\{d\}\_\{n\} =\{\textbackslash{}mathop\{ \textbackslash{}mathop\{∑ \}\}
\}\_\{p+q+r=n\}\{a\}\_\{p\}\{b\}\_\{q\}\{c\}\_\{r\}, ce qui conduit à
une vérification facile de cette associativité.

Remarque~1.5.1 On vérifie immédiatement que \{X\}\^{}\{n\} est la suite
qui a un 1 à la (n + 1)-ième place et des 0 partout ailleurs si bien que
(\{a\}\_\{n\}) =\{\textbackslash{}mathop\{ \textbackslash{}mathop\{∑
\}\} \}\_\{n≥0\}\{a\}\_\{n\}\{X\}\^{}\{n\}. Pour cette raison, cet
anneau est noté A{[}{[}X{]}{]}. On utilisera systématiquement la
notation \{\textbackslash{}mathop\{\textbackslash{}mathop\{∑ \}\}
\}\_\{n≥0\}\{a\}\_\{n\}\{X\}\^{}\{n\} pour désigner un de ses éléments.

Exercice Montrer que les éléments inversibles de A{[}{[}X{]}{]} sont les
séries formelles \{\textbackslash{}mathop\{\textbackslash{}mathop\{∑
\}\} \}\_\{n≥0\}\{a\}\_\{n\}\{X\}\^{}\{n\} telles que \{a\}\_\{0\} soit
un élément inversible de A. Montrer que si A est un corps, l'anneau
A{[}{[}X{]}{]} est un anneau principal n'ayant qu'un seul élément
irréductible (à multiplication près par un élément inversible).

\paragraph{1.5.2 L'anneau des polynômes à coefficients dans A}

Définition~1.5.2 On appelle polynôme à coefficients dans A une série
formelle à support fini, c'est-à-dire
\{\textbackslash{}mathop\{\textbackslash{}mathop\{∑ \}\}
\}\_\{n≥0\}\{a\}\_\{n\}\{X\}\^{}\{n\} telle que
\textbackslash{}\{n\textbackslash{}mathrel\{∣\}\{a\}\_\{n\}\textbackslash{}mathrel\{≠\}0\textbackslash{}\}
est fini.

Proposition~1.5.1 L'ensemble A{[}X{]} des polynômes à coefficients dans
A est un sous anneau de A{[}{[}X{]}{]}.

Définition~1.5.3 Soit P ∈ A{[}X{]}. On appelle degré et valuation de P~:

\textbackslash{}mathop\{deg\} P = \textbackslash{}left
\textbackslash{}\{ \textbackslash{}cases\{ −∞ \&si P = 0
\textbackslash{}cr
\textbackslash{}mathop\{max\}\textbackslash{}\{k\textbackslash{}mathrel\{∣\}\{a\}\_\{k\}\textbackslash{}mathrel\{≠\}0\textbackslash{}\}\&si
P = \textbackslash{}mathop\{∑
\}\{a\}\_\{k\}\{X\}\^{}\{k\}\textbackslash{}mathrel\{≠\}0 \}
\textbackslash{}right .

v(P) = \textbackslash{}left \textbackslash{}\{ \textbackslash{}cases\{
+∞ \&si P = 0 \textbackslash{}cr
\textbackslash{}mathop\{min\}\textbackslash{}\{k\textbackslash{}mathrel\{∣\}\{a\}\_\{k\}\textbackslash{}mathrel\{≠\}0\textbackslash{}\}\&si
P = \textbackslash{}mathop\{∑
\}\{a\}\_\{k\}\{X\}\^{}\{k\}\textbackslash{}mathrel\{≠\}0 \}
\textbackslash{}right .

Proposition~1.5.2 On a (i) \textbackslash{}mathop\{deg\} (P + Q)
≤\textbackslash{}mathop\{ max\}(\textbackslash{}mathop\{deg\}
P,\textbackslash{}mathop\{deg\} Q) avec égalité si
\textbackslash{}mathop\{deg\}
P\textbackslash{}mathrel\{≠\}\textbackslash{}mathop\{deg\} Q (ii)
\textbackslash{}mathop\{deg\} (PQ) ≤\textbackslash{}mathop\{ deg\} P
+\textbackslash{}mathop\{ deg\} Q avec égalité sauf si le produit des
termes de plus haut degré de P et Q est nul. En particulier on a égalité
si A est intègre.

Remarque~1.5.2 On a des résultats similaires pour la valuation.

Proposition~1.5.3 (règle de substitution). Soit A et B deux anneaux
commutatifs et φ:A → B un morphisme d'anneaux. Soit β ∈ B. Alors
l'application \{T\}\_\{φ,β\}:A{[}X{]} → B,
\{\textbackslash{}mathop\{\textbackslash{}mathop\{∑ \}\}
\}\_\{k=0\}\^{}\{n\}\{a\}\_\{k\}\{X\}\^{}\{k\}\textbackslash{}mathrel\{↦\}\{\textbackslash{}mathop\{\textbackslash{}mathop\{∑
\}\} \}\_\{k=0\}\^{}\{n\}φ(\{a\}\_\{k\})\{β\}\^{}\{k\} est un morphisme
d'anneaux.

Démonstration Faire le calcul.

Remarque~1.5.3 Lorsque A ⊂ B et φ(a) = a, on note P(β) =
\{T\}\_\{φ,β\}(P).

\paragraph{1.5.3 Division euclidienne et racines}

Théorème~1.5.4 (division euclidienne). Soit \{P\}\_\{1\},\{P\}\_\{2\} ∈
A{[}X{]} tels que \{P\}\_\{2\}\textbackslash{}mathrel\{≠\}0 et le terme
de plus haut degré de \{P\}\_\{2\} est inversible dans A. Alors il
existe un unique couple (Q,R) ∈ A\{{[}X{]}\}\^{}\{2\} tel que
\{P\}\_\{1\} = \{P\}\_\{2\}Q + R avec \textbackslash{}mathop\{deg\} R
\textless{}\textbackslash{}mathop\{ deg\} \{P\}\_\{2\}.

Démonstration On démontre l'existence par récurrence sur n
=\textbackslash{}mathop\{ deg\} \{P\}\_\{1\}. Si n
\textless{}\textbackslash{}mathop\{ deg\} \{P\}\_\{2\}, alors Q = 0 et R
= \{P\}\_\{1\} conviennent. Supposons le résultat montré pour tout
polynôme de degré strictement inférieur à n ≥\textbackslash{}mathop\{
deg\} \{P\}\_\{2\} et soit \textbackslash{}mathop\{deg\} \{P\}\_\{1\} =
n. On écrit \{P\}\_\{1\}(X) = \{a\}\_\{n\}\{X\}\^{}\{n\} +
\textbackslash{}mathop\{\textbackslash{}mathop\{\ldots{}\}\} et
\{P\}\_\{2\}(X) = \{b\}\_\{m\}\{X\}\^{}\{m\} +
\textbackslash{}mathop\{\textbackslash{}mathop\{\ldots{}\}\} avec
\{b\}\_\{m\} inversible. Posons \{S\}\_\{1\}(X) = \{P\}\_\{1\}(X) −
\{a\}\_\{n\}\{b\}\_\{m\}\^{}\{−1\}\{X\}\^{}\{n−m\}\{P\}\_\{2\}(X)~;
alors \textbackslash{}mathop\{deg\} \{S\}\_\{1\}
≤\textbackslash{}mathop\{ max\}(\textbackslash{}mathop\{deg\}
P,\textbackslash{}mathop\{deg\} (\{X\}\^{}\{n−m\}\{P\}\_\{2\})) = n et
\{S\}\_\{1\} n'a plus de terme de degré n. D'après l'hypothèse de
récurrence, on peut donc écrire \{S\}\_\{1\}(X) =
\{Q\}\_\{1\}(X)\{P\}\_\{2\}(X) + R(X) avec \textbackslash{}mathop\{deg\}
R \textless{}\textbackslash{}mathop\{ deg\} \{P\}\_\{2\}. Mais alors
P(X) = \{S\}\_\{1\}(X) +
\{a\}\_\{n\}\{b\}\_\{m\}\^{}\{−1\}\{X\}\^{}\{n−m\}\{P\}\_\{2\}(X) =
(\{Q\}\_\{1\}(X) +
\{a\}\_\{n\}\{b\}\_\{m\}\^{}\{−1\}\{X\}\^{}\{n−m\})\{P\}\_\{2\}(X) +
R(X) et donc Q(X) = \{Q\}\_\{1\}(X) +
\{a\}\_\{n\}\{b\}\_\{m\}\^{}\{−1\}\{X\}\^{}\{n−m\} et R(X) répondent aux
exigences voulues.

Pour l'unicité, on remarque que P = \{P\}\_\{2\}Q + R = \{P\}\_\{2\}Q' +
R' exige \{P\}\_\{2\}(Q − Q') = R' − R. Or \textbackslash{}mathop\{deg\}
(R' − R) \textless{}\textbackslash{}mathop\{ deg\} \{P\}\_\{2\}, et, si
Q − Q'\textbackslash{}mathrel\{≠\}0, \textbackslash{}mathop\{deg\}
(\{P\}\_\{2\}(Q − Q')) =\textbackslash{}mathop\{ deg\} \{P\}\_\{2\}
+\textbackslash{}mathop\{ deg\} (Q − Q') ≥\textbackslash{}mathop\{ deg\}
\{P\}\_\{2\} (car le coefficient de plus haut degré de \{Q\}\_\{2\} est
inversible, et donc le produit des coefficients de plus haut degré de
\{P\}\_\{2\} et Q − Q' ne peut pas être nul). C'est absurde. Donc Q =
Q', ce qui implique également R = R', et montre l'unicité.

Corollaire~1.5.5 Soit P ∈ A{[}X{]} et a ∈ A. Alors X −
a\textbackslash{}mathrel\{∣\}P \textbackslash{}mathrel\{⇔\} P(a) = 0.

Démonstration Si X − a divise P, on a P(X) = (X − a)Q(X) et donc P(a) =
0. Inversement, supposons que P(a) = 0 et effectuons la division
euclidienne de P(X) par X − a (dont le coefficient de plus haut degré
est 1, donc inversible)~; on peut écrire P(X) = (X − a)Q(X) + R(X) avec
\textbackslash{}mathop\{deg\} R \textless{} 1. Donc R est une constante
b~; mais alors 0 = P(a) = (a − a)Q(a) + b = b et donc P(X) = (X − a)Q(X)
et X − a divise P.

Remarque~1.5.4 On dit que a est racine de P si P(a) = 0.

Corollaire~1.5.6 Supposons A intègre. Soit P ∈ A{[}X{]} et
\{a\}\_\{1\},\textbackslash{}mathop\{\textbackslash{}mathop\{\ldots{}\}\},\{a\}\_\{k\}
des racines de P. Alors (X −
\{a\}\_\{1\})\textbackslash{}mathop\{\textbackslash{}mathop\{\ldots{}\}\}(X
− \{a\}\_\{k\})\textbackslash{}mathrel\{∣\}P. Si
P\textbackslash{}mathrel\{≠\}0, on a k ≤\textbackslash{}mathop\{ deg\}
P.

Démonstration Par récurrence sur k. On a déjà vu le résultat pour k = 1.
Supposons le vérifié pour k − 1~; on a donc déjà le fait que (X −
\{a\}\_\{1\})\textbackslash{}mathop\{\textbackslash{}mathop\{\ldots{}\}\}(X
− \{a\}\_\{k−1\}) divise P~; donc P(X) = (X −
\{a\}\_\{1\})\textbackslash{}mathop\{\textbackslash{}mathop\{\ldots{}\}\}(X
− \{a\}\_\{k−1\})Q(X). Mais alors 0 = P(\{a\}\_\{k\}) = (\{a\}\_\{k\} −
\{a\}\_\{1\})\textbackslash{}mathop\{\textbackslash{}mathop\{\ldots{}\}\}(\{a\}\_\{k\}
− \{a\}\_\{k−1\})Q(\{a\}\_\{k\}), et donc Q(\{a\}\_\{k\}) = 0 (les
autres termes sont non nuls et A est intègre)~; en particulier X −
\{a\}\_\{k\} divise Q et donc (X −
\{a\}\_\{1\})\textbackslash{}mathop\{\textbackslash{}mathop\{\ldots{}\}\}(X
− \{a\}\_\{k\})\textbackslash{}mathrel\{∣\}P.

Remarque~1.5.5 Sur un anneau intègre un polynôme non nul n'a donc qu'un
nombre fini de racines. En particulier on en déduit

Corollaire~1.5.7 Soit A un anneau intègre infini. Alors l'application
A{[}X{]} → \{A\}\^{}\{A\},
P\textbackslash{}mathrel\{↦\}\textbackslash{}tilde\{P\} avec
\textbackslash{}tilde\{P\}(x) = P(x), qui à un polynôme associe sa
fonction polynomiale, est un morphisme d'anneaux injectif.

\paragraph{1.5.4 Dérivation}

Définition~1.5.4 Soit P =\{\textbackslash{}mathop\{
\textbackslash{}mathop\{∑ \}\} \}\_\{k≥0\}\{a\}\_\{k\}\{X\}\^{}\{k\} ∈
A{[}X{]}. On appelle dérivée de P le polynôme

P' =\{ \textbackslash{}mathop\{∑
\}\}\_\{k≥1\}k\{a\}\_\{k\}\{X\}\^{}\{k−1\}

Proposition~1.5.8 On a les formules (αP + βQ)' = αP' + βQ', (PQ)' = P'Q
+ PQ', (\{P\}\^{}\{n\})' = nP'\{P\}\^{}\{n−1\}.

Démonstration On montre les résultats sur les monômes et on les étend
aux polynômes par linéarité.

\paragraph{1.5.5 L'anneau principal K{[}X{]}}

Soit K un corps commutatif. Le coefficient de plus haut degré d'un
polynôme non nul de K{[}X{]} étant par essence même différent de 0 donc
inversible, la division euclidienne est toujours possible. L'anneau
K{[}X{]} est donc un anneau euclidien, et par conséquent un anneau
principal. Tous les résultats sur les anneaux principaux s'appliquent
donc à K{[}X{]}~: existence du PGCD et du PPCM, identité de Bézout,
théorème de Gauss, existence et unicité de la décomposition en polynômes
irréductibles normalisés (ceux-ci étant des représentants naturels des
classes d'éléments irréductibles). Il en est de même des résultats sur
les anneaux euclidiens, et en particulier de l'algorithme de calcul du
PGCD.

Dans toute la suite du chapitre, les corps seront toujours supposés
commutatifs.

\paragraph{1.5.6 Formule de Taylor. Multiplicité d'une racine}

Définition~1.5.5 Soit K un corps, P ∈ K{[}X{]} et a ∈ K. On dit que a
est racine de multiplicité k de P si \{(X −
a)\}\^{}\{k\}\textbackslash{}mathrel\{∣\}P et \{(X − a)\}\^{}\{k+1\} ne
divise pas P.

Proposition~1.5.9 Soit K un corps, P ∈ K{[}X{]},
P\textbackslash{}mathrel\{≠\}0. Soit
\{a\}\_\{1\},\textbackslash{}mathop\{\textbackslash{}mathop\{\ldots{}\}\},\{a\}\_\{k\}
les racines de P de multiplicités respectives
\{m\}\_\{1\},\textbackslash{}mathop\{\textbackslash{}mathop\{\ldots{}\}\},\{m\}\_\{k\}.
Alors \{(X −
\{a\}\_\{1\})\}\^{}\{\{m\}\_\{1\}\}\textbackslash{}mathop\{\textbackslash{}mathop\{\ldots{}\}\}\{(X
− \{a\}\_\{k\})\}\^{}\{\{m\}\_\{k\}\}\textbackslash{}mathrel\{∣\}P et
donc \{m\}\_\{1\} +
\textbackslash{}mathop\{\textbackslash{}mathop\{\ldots{}\}\} +
\{m\}\_\{k\} ≤\textbackslash{}mathop\{ deg\} P. Si on a égalité, on a P
= λ\{(X −
\{a\}\_\{1\})\}\^{}\{\{m\}\_\{1\}\}\textbackslash{}mathop\{\textbackslash{}mathop\{\ldots{}\}\}\{(X
− \{a\}\_\{k\})\}\^{}\{\{m\}\_\{k\}\}, avec λ ∈ K. On dit alors que P
est scindé sur A.

Démonstration En effet les polynômes \{(X −
\{a\}\_\{i\})\}\^{}\{\{m\}\_\{i\}\} sont deux à deux premiers entre eux.

Théorème~1.5.10 (formule de Taylor pour les polynômes). Soit K un corps
de caractéristique 0, P ∈ K{[}X{]} et a ∈ K. Alors P(X + a)
=\{\textbackslash{}mathop\{ \textbackslash{}mathop\{∑ \}\}
\}\_\{k=0\}\^{}\{\textbackslash{}mathop\{deg\} P\}\{ \{P\}\^{}\{(k)\}(a)
\textbackslash{}over k!\} \{X\}\^{}\{k\}. Si A est un sous-anneau de K
qui contient à la fois a et les coefficients de P, alors
\textbackslash{}mathop\{∀\}k,\{ \{P\}\^{}\{(k)\}(a) \textbackslash{}over
k!\} ∈ A.

Démonstration Le polynôme P(X + a) s'écrit sous la forme Q(X)
=\{\textbackslash{}mathop\{ \textbackslash{}mathop\{∑ \}\}
\}\_\{k=0\}\^{}\{\textbackslash{}mathop\{deg\}
P\}\{b\}\_\{k\}\{X\}\^{}\{k\} (en développant chacun des \{(X +
a)\}\^{}\{k\} par la formule du binôme) et un calcul trivial montre que
\{P\}\^{}\{(k)\}(a) = \{Q\}\^{}\{(k)\}(0) = k!\{b\}\_\{k\}, soit
\{b\}\_\{k\} =\{ \{P\}\^{}\{(k)\}(a) \textbackslash{}over k!\} ~; de
plus, si les \{a\}\_\{k\} et a sont dans A, il en est de même des
\{b\}\_\{k\} (toujours par la formule du binôme).

Corollaire~1.5.11 Soit K un corps de caractéristique 0, P ∈ K{[}X{]} et
a ∈ K. Alors a est racine de multiplicité m de P si et seulement si P(a)
= P'(a) = \textbackslash{}mathop\{\textbackslash{}mathop\{\ldots{}\}\} =
\{P\}\^{}\{(m−1)\}(a) = 0 et
\{P\}\^{}\{(m)\}(a)\textbackslash{}mathrel\{≠\}0.

Démonstration Si P(X) = \{(X − a)\}\^{}\{m\}Q(X), une récurrence facile
montre que \{P\}\^{}\{(k)\}(X) = m(m −
1)\textbackslash{}mathop\{\textbackslash{}mathop\{\ldots{}\}\}(m − k +
1)\{(X − a)\}\^{}\{m−k\}Q(X) + \{(X − a)\}\^{}\{m−k+1\}\{R\}\_\{k\}(X),
pour k ≤ m~; on en déduit que \{P\}\^{}\{(k)\}(a) = 0 pour k ≤ m − 1 et
que \{P\}\^{}\{(m)\}(a) = m!Q(a)\textbackslash{}mathrel\{≠\}0 si X − a
ne divise pas Q, soit \{(X − a)\}\^{}\{m+1\} ne divise pas P.

Inversement, si P(a) = P'(a) =
\textbackslash{}mathop\{\textbackslash{}mathop\{\ldots{}\}\} =
\{P\}\^{}\{(m−1)\}(a) = 0, la formule de Taylor (dont les m premiers
termes sont nuls) montre que \{(X − a)\}\^{}\{m\} divise P. Mais, le
fait que \{P\}\^{}\{(m)\}(a)\textbackslash{}mathrel\{≠\}0, montre
d'après le sens direct, que \{(X − a)\}\^{}\{m+1\} ne divise pas P.

Remarque~1.5.6 En particulier P a une racine multiple si et seulement si
P et P' ont une racine commune.

\paragraph{1.5.7 Racines et extensions de corps}

Théorème~1.5.12 Soit K un corps et P ∈ K{[}X{]}. Alors il existe un
corps L contenant K dans lequel P a une racine.

Démonstration Il suffit bien entendu de démontrer ce résultat lorsque P
est un polynôme irréductible. On prend alors L = K{[}X{]}∕PK{[}X{]} (qui
contient K{[}X{]} en identifiant a ∈ K à π(a) ∈ L) et alors, si on pose
x = π(X) on a, puisque π est un morphisme d'anneaux, P(x) = P(π(X)) =
π(P(X)) = 0. Donc P a bien une racine dans L.

On montre alors par récurrence sur \textbackslash{}mathop\{deg\} P le
corollaire suivant

Corollaire~1.5.13 Soit K un corps et P ∈ K{[}X{]}. Alors il existe un
corps L contenant K sur lequel P est scindé.

Définition~1.5.6 On dit qu'un corps K est algébriquement clos s'il
vérifie les conditions équivalentes suivantes (i) Tout polynôme de
K{[}X{]} non constant a une racine dans K (ii) Tout polynôme de K{[}X{]}
est scindé sur K (iii) Les seuls polynômes irréductibles de K{[}X{]}
sont les polynômes de degré 1

Théorème~1.5.14 On montre, et on admettra que le corps des nombres
complexes est algébriquement clos (théorème de d'Alembert-Gauss)

\paragraph{1.5.8 Polynômes sur ℂ et ℝ}

On a vu que ℂ est algébriquement clos. On en déduit le résultat suivant

Théorème~1.5.15 Tout polynôme non constant de ℂ{[}X{]} a une racine dans
ℂ. Les seuls polynômes irréductibles de ℂ{[}X{]} sont les polynômes de
degré 1. Si
\{α\}\_\{1\},\textbackslash{}mathop\{\textbackslash{}mathop\{\ldots{}\}\},\{α\}\_\{k\}
sont les racines dans ℂ du polynôme P ∈ ℂ{[}X{]}, de multiplicités
respectives
\{m\}\_\{1\},\textbackslash{}mathop\{\textbackslash{}mathop\{\ldots{}\}\},\{m\}\_\{k\},
on a \{m\}\_\{1\} +
\textbackslash{}mathop\{\textbackslash{}mathop\{\ldots{}\}\} +
\{m\}\_\{k\} =\textbackslash{}mathop\{ deg\} P et P(X) =
\{a\}\_\{n\}\{\textbackslash{}mathop\{ \textbackslash{}mathop\{∏ \}\}
\}\_\{i=1\}\^{}\{k\}\{(X − \{α\}\_\{i\})\}\^{}\{\{m\}\_\{i\}\}.

Théorème~1.5.16 Soit P ∈ ℝ{[}X{]}.

\begin{itemize}
\item
  Si α ∈ ℂ est racine de P de multiplicité m, il en est de même de
  \textbackslash{}overline\{α\}.
\item
  Le nombre de racines non réelles de P est pair.
\item
  Si P est de degré impair, il a au moins une racine réelle.
\item
  Les polynômes irréductibles sur ℝ sont d'une part les polynômes de
  degré 1, d'autre part les polynômes de degré 2 sans racine réelle (Δ
  \textless{} 0).
\item
  Soit P ∈ ℝ{[}X{]}, soit
  \{α\}\_\{1\},\textbackslash{}mathop\{\textbackslash{}mathop\{\ldots{}\}\},\{α\}\_\{k\}
  ses racines réelles de multiplicités
  \{m\}\_\{1\},\textbackslash{}mathop\{\textbackslash{}mathop\{\ldots{}\}\},\{m\}\_\{k\},
  \{β\}\_\{1\},\textbackslash{}mathop\{\textbackslash{}mathop\{\ldots{}\}\},\{β\}\_\{l\}
  ses racines complexes de parties imaginaires strictement positives, de
  multiplicités
  \{n\}\_\{1\},\textbackslash{}mathop\{\textbackslash{}mathop\{\ldots{}\}\},\{n\}\_\{l\}.
  Alors la décomposition de P en produit de polynômes irréductibles dans
  ℝ{[}X{]} est

  P(X) = \{a\}\_\{n\}\{ \textbackslash{}mathop\{∏
  \}\}\_\{i=1\}\^{}\{k\}\{(X − \{α\}\_\{ i\})\}\^{}\{\{m\}\_\{i\} \}\{
  \textbackslash{}mathop\{∏ \}\}\_\{j=1\}\^{}\{l\}\{(\{X\}\^{}\{2\} −
  2\textbackslash{}mathrm\{Re\}(\{β\}\_\{ j\})X +
  \textbar{}\{β\}\_\{j\}\{\textbar{}\}\^{}\{2\})\}\^{}\{\{n\}\_\{j\} \}
\end{itemize}

Démonstration Pour les racines, on remarque que
\{P\}\^{}\{(k)\}(\textbackslash{}overline\{α\}) =
\textbackslash{}overline\{\{P\}\^{}\{(k)\}(α)\} si P est à coefficients
réels. Il suffit pour obtenir la décomposition de regrouper les racines
non réelles deux à deux conjuguées en remarquant que (X − β)(X
−\textbackslash{}overline\{β\}) = \{X\}\^{}\{2\} −
2\textbackslash{}mathop\{\textbackslash{}mathrm\{Re\}\}(β)X +
\textbar{}β\{\textbar{}\}\^{}\{2\} La caractérisation des polynômes
irréductibles en découle immédiatement.

\paragraph{1.5.9 Division suivant les puissances croissantes}

Théorème~1.5.17 Soit A ∈ K{[}X{]} et P ∈ K{[}X{]} tel que
P(0)\textbackslash{}mathrel\{≠\}0. Pour tout n ∈ ℕ, il existe un unique
couple (Q,R) de polynômes vérifiant A(X) = P(X)Q(X) +
\{X\}\^{}\{n+1\}R(X) avec \textbackslash{}mathop\{deg\} Q ≤ n.

Démonstration Similaire à celle de la division euclidienne, la valuation
prenant la place du degré.

{[}\href{coursse6.html}{next}{]} {[}\href{coursse4.html}{prev}{]}
{[}\href{coursse4.html\#tailcoursse4.html}{prev-tail}{]}
{[}\href{coursse5.html}{front}{]}
{[}\href{coursch2.html\#coursse5.html}{up}{]}

\end{document}

% \section{Polynômes à plusieurs variables}

\subsection{Généralités}

\begin{de}
\index{polynôme à plusieurs variables}
Soit $A$ un anneau commutatif.

$A[X_1,\ldots,X_n] = \left\{\sum_{(k_1,\ldots,k_n)\in\mathbb{N}^n} a_{k_1,\ldots,k_n} X_1^{k_1}\ldots X_n^{k_n} \mid \text{nombre fini de } a_{k_1,\ldots,k_n} \text{ non nuls}\right\}$
\end{de}

\begin{rem}
On a bien entendu un isomorphisme naturel entre $A[X_1,\ldots,X_n]$ et $A[X_1,\ldots,X_{n-1}][X_n]$ qui montre que si $A$ est intègre, il en est de même de $A[X_1,\ldots,X_n]$.
\end{rem}

\begin{prop}[règle de substitution]
Soit $A$ et $B$ deux anneaux commutatifs et $\phi:A \rightarrow B$ un morphisme d'anneaux. Soit $\beta_1,\ldots,\beta_n \in B$. Alors l'application $T_{\phi,\beta_1,\ldots,\beta_n}:A[X_1,\ldots,X_n] \rightarrow B$,

$\sum_{(k_1,\ldots,k_n)\in\mathbb{N}^n} a_{k_1,\ldots,k_n} X_1^{k_1}\ldots X_n^{k_n} \mapsto \sum_{(k_1,\ldots,k_n)\in\mathbb{N}^n} \phi(a_{k_1,\ldots,k_n}) \beta_1^{k_1}\ldots \beta_n^{k_n}$

est un morphisme d'anneaux.
\end{prop}

\subsection{Dérivées partielles, formule de Taylor}

\begin{de}
\index{dérivée partielle}
Soit $P \in A[X_1,\ldots,X_n]$. On note $\frac{\partial P}{\partial X_i}$ la dérivée de $P$ dans $(A[X_1,\ldots,X_{i-1},X_{i+1},\ldots,X_n])[X_i]$.
\end{de}

Par simple calcul sur les monômes on montre alors

\begin{lem}[Schwarz]
Soit $P \in A[X_1,\ldots,X_n]$, $i,j \in [1,n]$. Alors $\frac{\partial}{\partial X_i} (\frac{\partial P}{\partial X_j}) = \frac{\partial}{\partial X_j} (\frac{\partial P}{\partial X_i})$.
\end{lem}

Ceci permet de définir des dérivées partielles itérées $\frac{\partial^k P}{\partial X_1^{k_1}\ldots \partial X_n^{k_n}}$ si $k = k_1 + \ldots + k_n$. On a alors

\begin{thm}[Formule de Taylor]
\index{formule de Taylor}
Soit $K$ un corps de caractéristique 0, $P \in K[X_1,\ldots,X_n]$ et $(a_1,\ldots,a_n) \in K^n$. Alors

$P(X_1 + a_1,\ldots,X_n + a_n) = \sum_{k_1,\ldots,k_n} \frac{1}{k_1!\ldots k_n!} \frac{\partial^{k_1+\ldots+k_n} P}{\partial X_1^{k_1}\ldots \partial X_n^{k_n}} (a_1,\ldots,a_n) X_1^{k_1}\ldots X_n^{k_n}$
\end{thm}

\subsection{Degré total, polynômes homogènes}

\begin{de}
\index{degré d'un polynôme à plusieurs variables}
\index{polynôme homogène}
On définit le degré d'un monôme non nul $a X_1^{k_1}\ldots X_n^{k_n}$ comme étant l'entier $k_1 + \ldots + k_n$. On appelle degré d'un polynôme $P$ non nul, le plus grand des degrés de ses monômes non nuls. On dit qu'un polynôme $P \in K[X_1,\ldots,X_n]$ est homogène de degré $p$ si tous ses monômes non nuls ont le même degré $p$ ou s'il est nul. On notera $H_p(X_1,\ldots,X_n)$ l'espace vectoriel des polynômes homogènes de degré $p$.
\end{de}

\begin{thm}
On a $H_p \cdot H_q \subset H_{p+q}$ et $K[X_1,\ldots,X_n] = \bigoplus_{p\in\mathbb{N}} H_p(X_1,\ldots,X_n)$ (c'est-à-dire que tout polynôme s'écrit de manière unique comme somme finie de polynômes homogènes de degrés distincts).
\end{thm}

\begin{proof}
La décomposition correspond tout simplement au regroupement des termes de même degré au sein d'un polynôme homogène. Cela montre à la fois l'existence et l'unicité de la décomposition.
\end{proof}

\begin{thm}
Soit $K$ un corps. On a $\deg PQ = \deg P + \deg Q$.
\end{thm}

\begin{proof}
On décompose $P$ et $Q$ en polynômes homogènes $P = P_m + \ldots$ et $Q = Q_n + \ldots$ où $P_m$ et $Q_n$ sont les parties homogènes de plus haut degré. Alors $P_m Q_n \neq 0$ et c'est la partie homogène de plus haut degré de $PQ$, d'où le résultat.
\end{proof}

\begin{thm}[Euler]
\index{théorème d'Euler}
Soit $K$ un corps commutatif de caractéristique 0 et $P \in K[X_1,\ldots,X_n]$. On a équivalence de 
\begin{enumerate}
\item $P$ est homogène de degré $p$ 
\item $\sum_{i=1}^n X_i \frac{\partial P}{\partial X_i} = pP$.
\end{enumerate}
\end{thm}

\begin{proof}
On pose $D = \sum_{i=1}^n X_i \frac{\partial}{\partial X_i}$. On démontre (i) $\Rightarrow$ (ii) en calculant sur les monômes et en utilisant la linéarité de $D$. On démontre (ii) $\Rightarrow$ (i) en décomposant $P$ en somme de polynômes homogènes : si $P = P_m + \ldots + P_0$, on a $pP_m + \ldots + pP_0 = pP = DP = DP_m + \ldots + DP_0 = mP_m + \ldots + 0P_0$. Par unicité de la décomposition en polynômes homogènes, on a pour tout $k$, $pP_k = kP_k$ ce qui exige $P_k = 0$ si $k \neq p$. Finalement $P = P_p$ est homogène de degré $p$.
\end{proof}

\subsection{Polynômes symétriques}

\begin{de}
\index{polynôme symétrique}
On dit que $P \in K[X_1,\ldots,X_n]$ est symétrique si pour toute permutation $\sigma$ on a

$P(X_{\sigma(1)},\ldots,X_{\sigma(n)}) = P(X_1,\ldots,X_n)$
\end{de}

\begin{ex}
\index{polynôme symétrique élémentaire}
Pour $1 \leq k \leq n$, on définit les polynômes symétriques élémentaires à $n$ variables $\sigma_k(X_1,\ldots,X_n) = \sum_{1\leq i_1<\cdots<i_k\leq n} X_{i_1}\ldots X_{i_n}$ (homogène de degré $k$, $C_n^k$ monômes). Ces polynômes symétriques vérifient la formule de récurrence (séparer les termes ne contenant pas $X_n$ de ceux contenant $X_n$) :

$\sigma_k(X_1,\ldots,X_n) = \sigma_k(X_1,\ldots,X_{n-1}) + \sigma_{k-1}(X_1,\ldots,X_{n-1})X_n$
\end{ex}

\begin{thm}
$\prod_{i=1}^n (T - X_i) = T^n - \sum_{k=1}^n (-1)^k \sigma_k(X_1,\ldots,X_n) T^{n-k} = T^n - \sigma_1(X_1,\ldots,X_n) T^{n-1} + \ldots + (-1)^n \sigma_n(X_1,\ldots,X_n)$
\end{thm}

\begin{proof}
Par récurrence sur $n$ en utilisant la formule de récurrence vérifiée par les $\sigma_k$.
\end{proof}

\begin{thm}
Soit $P \in K[X]$ scindé sur $K$. On peut donc écrire

$P(X) = a_n X^n + \ldots + a_0 = a_n \prod_{i=1}^n (X - \alpha_i)$

Alors on a, $\forall k \in [1,n]$, $\sigma_k(\alpha_1,\ldots,\alpha_n) = (-1)^k \frac{a_{n-k}}{a_n}$.
\end{thm}

On admettra le résultat suivant

\begin{thm}
\index{théorème fondamental sur les polynômes symétriques}
Soit $P \in K[X_1,\ldots,X_n]$ un polynôme symétrique. Il existe un unique polynôme $Q \in K[X_1,\ldots,X_n]$ tel que $P = Q(\sigma_1,\ldots,\sigma_n)$.
\end{thm}
% 
\subsection{2.1 Généralités sur les espaces vectoriels}


\subsubsection{Notion de K-espace vectoriel}
\label{sec:notion-de-k}


\begin{de}
   On appelle K-espace vectoriel un triplet $(E,+,.)$ où
$(E,+)$ est un groupe abélien, . une loi externe à domaine d'opérateurs K,
doublement distributive par rapport à l'addition dans K et dans E,
vérifiant $\forall x \in E, 1_K~x = x$ et
$\forall \alpha,\beta \in K, \forall x \in E,
\alpha (\beta x) = (\alpha \beta) x$ .

\end{de}
Exemples fondamentaux~: si L est un sur-corps de K, L est naturellement
un K-espace vectoriel .

Soit n \in \mathbb{N}~~; K^n muni des lois
(x_1,\\ldots,x_n~)
+
(y_1,\\ldots,y_n~)
= (x_1 +
y_1,\\ldots,x_n~
+ y_n) et
\lambda~(x_1,\\ldots,x_n~)
=
(\lambda~x_1,\\ldots,\lambda~x_n~)est
un K-espace vectoriel. Plus généralement, si I est un ensemble,
\[
K^(I) = (a_i)_i\in I ∣ \forall i,
a_i \in K\text{ et nombre fini de}
a_i \text{non nuls}
\]
est un K-espace vectoriel pour les lois (a_i) + (b_i)
= (a_i + b_i) et \lambda~(a_i) = (\lambda~a_i).

\paragraph{2.1.2 Notion de sous-espace vectoriel}

Remarque~2.1.1 Un sous-espace vectoriel est une partie stable aussi bien
par la loi interne que par la loi externe et qui est muni par les lois
induites d'une structure d'espace vectoriel. On vérifie immédiatement
que c'est équivalent à la définition suivante que l'on retiendra~:

Définition~2.1.2 On appelle sous-espace vectoriel de l'espace vectoriel
E une partie F de E telle que

\begin{itemize}
\itemsep1pt\parskip0pt\parsep0pt
\item
  (i) F\neq~\varnothing~
\item
  (ii) \forall~~\alpha~,\beta~ \in K,
  \forall~~x,y \in F, \alpha~x + \beta~y \in F.
\end{itemize}

Remarque~2.1.2 L'intersection d'une famille quelconque de sous-espaces
vectoriels en est encore un ce qui conduit à la définition suivante

Définition~2.1.3 L'ensemble des sous-espaces vectoriels contenant une
partie A de E admet un plus petit élément appelé le sous-espace
vectoriel engendré par A et noté
\mathrmVect~(A). On a
\mathrmVect~(A)
= \⋂  _ A\subset~F
\atop F\textsev  F

Définition~2.1.4 On appelle sous-espace vectoriel engendré par la
famille (x_i)_i\inI le plus petit sous-espace vectoriel
contenant tous les vecteurs de la famille, et on le note
\mathrmVect(x_i~,i
\in I). On a

\begin{align*}
\mathrmVect(x_i~,i
\in I)& =&
\mathrmVect~(\⋃
_i\inI\x_i\) \%&
\\ & =&
\\\sum
_i\inI\alpha_ix_i∣(\alpha_i)
\in K^(I)\\%&
\\ \end{align*}

(ensemble des combinaisons linéaires de la famille (x_i)).

\paragraph{2.1.3 Produits, quotients}

Définition~2.1.5 (espace produit) Si E et F sont deux espaces vectoriels
, l'espace E \times F est muni d'une structure d'espace vectoriel en posant
(x_1,y_1) + (x_2,y_2) =
(x_1 + x_2,y_1 + y_2), \lambda~(x,y) =
(\lambda~x,\lambda~y).

Définition~2.1.6 (espace quotient) Soit E un espace vectoriel et F un
sous-espace vectoriel de E. La relation ''x\mathcal{R}y
\Leftrightarrow x - y \in F'' est une relation d'équivalence
sur E. La classe d'un élément x de E est x + F. Il existe sur E\diagupF une
unique structure d'espace vectoriel telle que la pro\\jmathmathection \pi~ : E \rightarrow~ E\diagupF
vérifie \forall~~\alpha~,\beta~ \in K,
\forall~~x,y \in E, \pi~(\alpha~x + \beta~y) = \alpha~\pi~(x) + \beta~\pi~(y).

Démonstration La relation d'équivalence et la caractérisation de la
classe d'équivalence proviennent du même résultat sur les groupes
additifs. La loi d'espace vectoriel sur E\diagupF doit être définie de telle
sorte que \alpha~(x + F) + \beta~(y + F) = (\alpha~x + \beta~y) + F~; il suffit donc de
vérifier que si x + F = x' + F et y + F = y' + F, alors (\alpha~x + \beta~y) + F =
(\alpha~x' + \beta~y') + F. Or les deux premières relations signifient que x - x' \in
F et y - y' \in F. On a donc (\alpha~x + \beta~y) - (\alpha~x' + \beta~y') = \alpha~(x - x') + \beta~(y -
y') \in F, soit encore (\alpha~x + \beta~y) + F = (\alpha~x' + \beta~y') + F. Ceci définit
parfaitement une structure d'espace vectoriel sur E\diagupF (vérification
facile) et on a bien \pi~(\alpha~x + \beta~y) = \alpha~\pi~(x) + \beta~\pi~(y).

\paragraph{2.1.4 Applications linéaires}

Proposition~2.1.1 Soit E et F deux espaces vectoriels . On appelle
application linéaire une application f : E \rightarrow~ F telle que
\forall~\alpha~,\beta~ \in K, \\forall~~x,y \in E,
f(\alpha~x + \beta~y) = \alpha~f(x) + \beta~f(y).

Notation~: L(E,F) l'ensemble des applications linéaires de E dans F.

Proposition~2.1.2 L'ensemble L(E,F) est muni d'une structure de K espace
vectoriel en posant (f + g)(x) = f(x) + g(x) et (\lambda~f)(x) = \lambda~f(x).

Proposition~2.1.3 Soit f \in L(E,F). L'image par f de tout sous-espace
vectoriel de E est un sous-espace vectoriel de F. L'image réciproque de
tout sous-espace vectoriel de F est un sous-espace vectoriel de E.

Remarque~2.1.3 En particulier
\mathrmKer~f =
f^-1(\0\) et
\mathrmIm~f = f(E) sont des
sous-espaces vectoriels respectivement de E et F.

Théorème~2.1.4 Soit f \in L(E,F). L'application f est in\\jmathmathective si et
seulement si \mathrmKer~f =
\0\.

Démonstration C'est une traduction du résultat sur les groupes.

Théorème~2.1.5 Soit f \in L(E,F). Il existe une unique application
\overlinef :
E\diagup\mathrmKer~f
\rightarrow~\mathrmIm~f vérifiant
\forall~x \in E, \overlinef~(\pi~(x)) =
f(x) (où \pi~ désigne la pro\\jmathmathection canonique de E sur
E\diagup\mathrmKer~f).
L'application \overlinef est un isomorphisme
d'espaces vectoriels .

Démonstration Analogue au résultat similaire sur les groupes.

\paragraph{2.1.5 Somme de sous-espaces}

Soit E un K-espace vectoriel et
F_1,\\ldots,F_k~
des sous-espaces vectoriels de E. Soit f : F_1
\times⋯ \times F_k \rightarrow~ E définie par
f(x_1,\\ldots,x_k~)
= x_1 +
\\ldots~ +
x_k. On vérifie facilement que f est linéaire.

Définition~2.1.7 On appelle somme des sous-espaces vectoriels
F_1,\\ldots,F_k~
le sous-espace vectoriel F_1 + ⋯ +
F_k = \mathrmIm~f =
\x_1 +
\\ldots~ +
x_k∣\forall~~i,
x_i \in F_i\. On dit que
F_1,\\ldots,F_k~
sont en somme directe si f est in\\jmathmathective (c'est-à-dire si l'écriture
d'un x sous la forme x = x_1 +
\\ldots~ +
x_k lorsqu'elle existe, est unique). Dans ce cas on écrit la
somme sous la forme F_1 \oplus~⋯ \oplus~
F_k.

Théorème~2.1.6 Les sous-espaces
F_1,\\ldots,F_k~
sont en somme directe si et seulement si

x_1 +
\\ldots~ +
x_k = 0 \rigtharrow~ x_1 =
0,\\ldots,x_k~
= 0

Démonstration Ceci traduit simplement que f est in\\jmathmathective si et
seulement si son noyau est réduit à
\0\.

Remarque~2.1.4 Il n'existe pas d'autre caractérisation correcte et
réellement utile de la somme directe dans le cas où k ≥ 3. Par contre,
si k = 2 on a

Théorème~2.1.7 Les sous-espaces F_1 et F_2 sont en
somme directe si et seulement si F_1 \bigcap F_2 =
\0\.

Démonstration On a en effet

x_1 + x_2 = 0 \Leftrightarrow
x_1 = -x_2 \in F_1 \bigcap F_2

Définition~2.1.8 On dit que deux sous-espaces vectoriels F et G de
l'espace vectoriel E sont supplémentaires s'ils vérifient les trois
propriétés équivalentes

\begin{itemize}
\itemsep1pt\parskip0pt\parsep0pt
\item
  (i) E = F \oplus~ G
\item
  (ii) E = F + G et F \bigcap G = \0\
\item
  (iii) Tout élément x de E s'écrit de manière unique sous la forme x =
  y + z avec y \in F et z \in G.
\end{itemize}

On dit que y est la pro\\jmathmathection de x sur F parallèlement à G~: y =
\pi_F\parallelG(x).

Proposition~2.1.8 Si F et G sont supplémentaires, \pi_F\parallelG est une
application linéaire de E dans E et on a \pi_F\parallelG + \pi_G\parallelF
= \mathrmId_E.

Le théorème suivant peut rendre de grands services

Théorème~2.1.9 Soit f : E \rightarrow~ F une application linéaire et V un
supplémentaire de
\mathrmKer~f dans E. Alors
la restriction de f à V , f__V , induit un
isomorphisme de V sur
\mathrmIm~f.

Démonstration Soit f' la restriction de f à V ~; on a
\mathrmKer~f'
= \mathrmKer~f \bigcap V =
\0\ ce qui montre que f' est
in\\jmathmathective. De plus, si y
\in\mathrmIm~f, il existe x \in
E tel que y = f(x). Cet élément x peut s'écrire x = x_1 +
x_2 avec x_1 \in V,x_2
\in\mathrmKer~f, d'où y = f(x)
= f(x_1) + f(x_2) = f(x_1) = f'(x_1)
ce qui montre que f' est sur\\jmathmathective de V sur
\mathrmIm~f.

Remarque~2.1.5 Appelons g l'isomorphisme réciproque~; on a alors f \cdot g
=
\mathrmId_\mathrmIm~
f et g \cdot f = \pi_V
\parallel\mathrmKer f~. Le
résultat suivant n'est qu'un cas particulier utile du théorème énoncé~:

Proposition~2.1.10 Si F et G sont supplémentaires, soit \pi~ la pro\\jmathmathection
canonique de E sur E\diagupF. Alors la restriction de \pi~ à G est un
isomorphisme de G sur E\diagupF

Remarque~2.1.6 Contrairement au quotient E\diagupF qui est unique, un
supplémentaire G ne l'est pas~; par contre deux supplémentaires d'un
même sous-espace vectoriel sont isomorphes (puisqu'ils sont tous deux
isomorphes à E\diagupF).

\paragraph{2.1.6 Algèbres}

Définition~2.1.9 On appelle K-algèbre un quadruplet (A,+,∗,.) tel que
(A,+,∗) est un anneau, (A,+,.) est un K-espace vectoriel avec

\forall~\lambda~ \in K, \\forall~~x,y \in A,
(\lambda~x) ∗ y = x ∗ (\lambda~y) = \lambda~(x ∗ y)

(avec la distributivité, ces propriétés traduisent la bilinéarité du
produit ∗).

Remarque~2.1.7 Notions évidentes~: sous-algèbres, morphisme d'algèbres.

Exemple~2.1.1 L'ensemble L(E) = L(E,E) des endomorphismes de E est une
K-algèbre, la multiplication étant la composition. Le groupe des
éléments inversibles (automorphismes de E) est noté GL(E).

\paragraph{2.1.7 Familles libres, génératrices. Bases}

Soit E un espace vectoriel, X = (x_i)_i\inI une famille
de vecteurs de E. A cette famille on peut associer une application
linéaire f_X : K^(I) \rightarrow~ E par
f((\alpha_i)_i\inI) =\
\sum  _i\inI\alpha_ix_i~.

Définition~2.1.10 On dit que la famille X est (i) libre si f_X
est in\\jmathmathective (ii) génératrice si f_X est sur\\jmathmathective (iii) une
base de E si f_X est bi\\jmathmathective.

Proposition~2.1.11 La famille X est (i) libre si et seulement si
\forall~(\alpha_i) \in K^(I)~,
\\sum ~
\alpha_ix_i = 0 \rigtharrow~\forall~~i \in I,
\alpha_i = 0 (ii) génératrice si et seulement si tout élément x de E
s'écrit sous la forme x =\
\sum  \alpha_ix_i~ (iii) une base
si et seulement si tout élément x de E s'écrit de manière unique sous la
forme x = \\sum ~
\alpha_ix_i (on dit alors que les \alpha_i sont les
coordonnées de x dans la base X).

Démonstration Seul (i) n'est pas totalement évident. Il traduit que
f_X est in\\jmathmathective si et seulement si son noyau est réduit à la
famille nulle. On remarque alors facilement qu'une famille est libre si
et seulement si toute sous-famille finie est libre.

Définition~2.1.11 On dit qu'une famille est liée lorsqu'elle n'est pas
libre.

Proposition~2.1.12 Toute sous-famille d'une famille libre est libre,
toute surfamille d'une famille génératrice est génératrice. L'image par
une application linéaire in\\jmathmathective d'une famille libre est libre.
L'image par une application linéaire sur\\jmathmathective d'une famille
génératrice est génératrice. L'image par une application linéaire d'une
famille liée est liée. L'image par un isomorphisme d'une base est une
base.

Démonstration Elémentaire

\paragraph{2.1.8 Théorèmes fondamentaux}

Théorème~2.1.13 Une famille (x_i)_i\inI est liée si et
seulement si il existe i_0 \in I tel que x_i_0
soit combinaison linéaire de la famille
(x_i)_i\inI\diagdown\i_0\

Démonstration Si x_i_0 =\
\sum ~
_i\inI\diagdown\i_0\\alpha_ix_i,
on a \\sum ~
_i\inI\alpha_ix_i = 0 en posant
\alpha_i_0 = -1 et la famille est donc liée. En ce qui
concerne la réciproque, on écrit
\\sum ~
_i\inI\alpha_ix_i = 0 avec par exemple
\alpha_i_0\neq~0. Alors
x_i_0 =\
\sum ~
_i\inI\diagdown\i_0\(-
\alpha_i \over \alpha_i_0
)x_i. En adaptant de fa\ccon évidente la
démonstration on a

Théorème~2.1.14 Soit (x_i)_i\inI une famille liée. On
suppose que la famille
(x_i)_i\inI\diagdown\i_0\
est libre. Alors x_i_0 est combinaison linéaire de la
famille
(x_i)_i\inI\diagdown\i_0\

Démonstration En effet le fait que la famille
(x_i)_i\inI\diagdown\i_0\
soit libre implique que nécessairement
\alpha_i_0\neq~0.

Théorème~2.1.15 Soit E et F deux K-espaces vectoriels et \mathcal{E} =
(e_i)_i\inI une base de E. Pour toute famille
(b_i)_i\inI d'éléments de F indexée par I, il existe une
unique application linéaire f : E \rightarrow~ F vérifiant

\forall~i \in I, f(e_i) = b_i~

Démonstration L'application f est bien évidemment définie par
f(\\sum ~
x_ie_i) =\
\sum  x_ib_i~. On vérifie
facilement qu'elle est linéaire.

Remarque~2.1.8 Les deux théorèmes suivants découlent simplement de la
relation

\sum _i\inI\alpha_ie_i~ =
\\sum
_\\jmathmath=1^p\underbrace
\\sum
_i\inI_\\jmathmath\alpha_ie_i _\inE_\\jmathmath

et des caractérisations d'une base et d'une somme directe~:

Théorème~2.1.16 Soit E un K-espace vectoriel , \mathcal{E} =
(e_i)_i\inI une base de E, I = I_1
\cup\\ldots~ \cup
I_p une partition de I, E_\\jmathmath =\
\mathrmVect(e_i, i \in I_\\jmathmath). Alors
E = E_1 \oplus~⋯ \oplus~ E_p.

Théorème~2.1.17 Soit E un K-espace vectoriel , E = E_1
\oplus~⋯ \oplus~ E_p une décomposition en somme
directe. Pour \\jmathmath \in {[}1,p{]}, soit \mathcal{E}_\\jmathmath =
(e_i)_i\inI_\\jmathmath une base de E_\\jmathmath (les
ensembles I_\\jmathmath sont dis\\jmathmathoints). Alors la famille \mathcal{E}_1
\cup\\ldots~
\cup\mathcal{E}_p est une base de E (dite adaptée à la décomposition en
somme directe).

% \documentclass[]{article}
\usepackage[T1]{fontenc}
\usepackage{lmodern}
\usepackage{amssymb,amsmath}
\usepackage{ifxetex,ifluatex}
\usepackage{fixltx2e} % provides \textsubscript
% use upquote if available, for straight quotes in verbatim environments
\IfFileExists{upquote.sty}{\usepackage{upquote}}{}
\ifnum 0\ifxetex 1\fi\ifluatex 1\fi=0 % if pdftex
  \usepackage[utf8]{inputenc}
\else % if luatex or xelatex
  \ifxetex
    \usepackage{mathspec}
    \usepackage{xltxtra,xunicode}
  \else
    \usepackage{fontspec}
  \fi
  \defaultfontfeatures{Mapping=tex-text,Scale=MatchLowercase}
  \newcommand{\euro}{€}
\fi
% use microtype if available
\IfFileExists{microtype.sty}{\usepackage{microtype}}{}
\ifxetex
  \usepackage[setpagesize=false, % page size defined by xetex
              unicode=false, % unicode breaks when used with xetex
              xetex]{hyperref}
\else
  \usepackage[unicode=true]{hyperref}
\fi
\hypersetup{breaklinks=true,
            bookmarks=true,
            pdfauthor={},
            pdftitle={Bases et dimension},
            colorlinks=true,
            citecolor=blue,
            urlcolor=blue,
            linkcolor=magenta,
            pdfborder={0 0 0}}
\urlstyle{same}  % don't use monospace font for urls
\setlength{\parindent}{0pt}
\setlength{\parskip}{6pt plus 2pt minus 1pt}
\setlength{\emergencystretch}{3em}  % prevent overfull lines
\setcounter{secnumdepth}{0}
 
/* start css.sty */
.cmr-5{font-size:50%;}
.cmr-7{font-size:70%;}
.cmmi-5{font-size:50%;font-style: italic;}
.cmmi-7{font-size:70%;font-style: italic;}
.cmmi-10{font-style: italic;}
.cmsy-5{font-size:50%;}
.cmsy-7{font-size:70%;}
.cmex-7{font-size:70%;}
.cmex-7x-x-71{font-size:49%;}
.msbm-7{font-size:70%;}
.cmtt-10{font-family: monospace;}
.cmti-10{ font-style: italic;}
.cmbx-10{ font-weight: bold;}
.cmr-17x-x-120{font-size:204%;}
.cmsl-10{font-style: oblique;}
.cmti-7x-x-71{font-size:49%; font-style: italic;}
.cmbxti-10{ font-weight: bold; font-style: italic;}
p.noindent { text-indent: 0em }
td p.noindent { text-indent: 0em; margin-top:0em; }
p.nopar { text-indent: 0em; }
p.indent{ text-indent: 1.5em }
@media print {div.crosslinks {visibility:hidden;}}
a img { border-top: 0; border-left: 0; border-right: 0; }
center { margin-top:1em; margin-bottom:1em; }
td center { margin-top:0em; margin-bottom:0em; }
.Canvas { position:relative; }
li p.indent { text-indent: 0em }
.enumerate1 {list-style-type:decimal;}
.enumerate2 {list-style-type:lower-alpha;}
.enumerate3 {list-style-type:lower-roman;}
.enumerate4 {list-style-type:upper-alpha;}
div.newtheorem { margin-bottom: 2em; margin-top: 2em;}
.obeylines-h,.obeylines-v {white-space: nowrap; }
div.obeylines-v p { margin-top:0; margin-bottom:0; }
.overline{ text-decoration:overline; }
.overline img{ border-top: 1px solid black; }
td.displaylines {text-align:center; white-space:nowrap;}
.centerline {text-align:center;}
.rightline {text-align:right;}
div.verbatim {font-family: monospace; white-space: nowrap; text-align:left; clear:both; }
.fbox {padding-left:3.0pt; padding-right:3.0pt; text-indent:0pt; border:solid black 0.4pt; }
div.fbox {display:table}
div.center div.fbox {text-align:center; clear:both; padding-left:3.0pt; padding-right:3.0pt; text-indent:0pt; border:solid black 0.4pt; }
div.minipage{width:100%;}
div.center, div.center div.center {text-align: center; margin-left:1em; margin-right:1em;}
div.center div {text-align: left;}
div.flushright, div.flushright div.flushright {text-align: right;}
div.flushright div {text-align: left;}
div.flushleft {text-align: left;}
.underline{ text-decoration:underline; }
.underline img{ border-bottom: 1px solid black; margin-bottom:1pt; }
.framebox-c, .framebox-l, .framebox-r { padding-left:3.0pt; padding-right:3.0pt; text-indent:0pt; border:solid black 0.4pt; }
.framebox-c {text-align:center;}
.framebox-l {text-align:left;}
.framebox-r {text-align:right;}
span.thank-mark{ vertical-align: super }
span.footnote-mark sup.textsuperscript, span.footnote-mark a sup.textsuperscript{ font-size:80%; }
div.tabular, div.center div.tabular {text-align: center; margin-top:0.5em; margin-bottom:0.5em; }
table.tabular td p{margin-top:0em;}
table.tabular {margin-left: auto; margin-right: auto;}
div.td00{ margin-left:0pt; margin-right:0pt; }
div.td01{ margin-left:0pt; margin-right:5pt; }
div.td10{ margin-left:5pt; margin-right:0pt; }
div.td11{ margin-left:5pt; margin-right:5pt; }
table[rules] {border-left:solid black 0.4pt; border-right:solid black 0.4pt; }
td.td00{ padding-left:0pt; padding-right:0pt; }
td.td01{ padding-left:0pt; padding-right:5pt; }
td.td10{ padding-left:5pt; padding-right:0pt; }
td.td11{ padding-left:5pt; padding-right:5pt; }
table[rules] {border-left:solid black 0.4pt; border-right:solid black 0.4pt; }
.hline hr, .cline hr{ height : 1px; margin:0px; }
.tabbing-right {text-align:right;}
span.TEX {letter-spacing: -0.125em; }
span.TEX span.E{ position:relative;top:0.5ex;left:-0.0417em;}
a span.TEX span.E {text-decoration: none; }
span.LATEX span.A{ position:relative; top:-0.5ex; left:-0.4em; font-size:85%;}
span.LATEX span.TEX{ position:relative; left: -0.4em; }
div.float img, div.float .caption {text-align:center;}
div.figure img, div.figure .caption {text-align:center;}
.marginpar {width:20%; float:right; text-align:left; margin-left:auto; margin-top:0.5em; font-size:85%; text-decoration:underline;}
.marginpar p{margin-top:0.4em; margin-bottom:0.4em;}
.equation td{text-align:center; vertical-align:middle; }
td.eq-no{ width:5%; }
table.equation { width:100%; } 
div.math-display, div.par-math-display{text-align:center;}
math .texttt { font-family: monospace; }
math .textit { font-style: italic; }
math .textsl { font-style: oblique; }
math .textsf { font-family: sans-serif; }
math .textbf { font-weight: bold; }
.partToc a, .partToc, .likepartToc a, .likepartToc {line-height: 200%; font-weight:bold; font-size:110%;}
.chapterToc a, .chapterToc, .likechapterToc a, .likechapterToc, .appendixToc a, .appendixToc {line-height: 200%; font-weight:bold;}
.index-item, .index-subitem, .index-subsubitem {display:block}
.caption td.id{font-weight: bold; white-space: nowrap; }
table.caption {text-align:center;}
h1.partHead{text-align: center}
p.bibitem { text-indent: -2em; margin-left: 2em; margin-top:0.6em; margin-bottom:0.6em; }
p.bibitem-p { text-indent: 0em; margin-left: 2em; margin-top:0.6em; margin-bottom:0.6em; }
.subsectionHead, .likesubsectionHead { margin-top:2em; font-weight: bold;}
.sectionHead, .likesectionHead { font-weight: bold;}
.quote {margin-bottom:0.25em; margin-top:0.25em; margin-left:1em; margin-right:1em; text-align:justify;}
.verse{white-space:nowrap; margin-left:2em}
div.maketitle {text-align:center;}
h2.titleHead{text-align:center;}
div.maketitle{ margin-bottom: 2em; }
div.author, div.date {text-align:center;}
div.thanks{text-align:left; margin-left:10%; font-size:85%; font-style:italic; }
div.author{white-space: nowrap;}
.quotation {margin-bottom:0.25em; margin-top:0.25em; margin-left:1em; }
h1.partHead{text-align: center}
.sectionToc, .likesectionToc {margin-left:2em;}
.subsectionToc, .likesubsectionToc {margin-left:4em;}
.sectionToc, .likesectionToc {margin-left:6em;}
.frenchb-nbsp{font-size:75%;}
.frenchb-thinspace{font-size:75%;}
.figure img.graphics {margin-left:10%;}
/* end css.sty */

\title{Bases et dimension}
\author{}
\date{}

\begin{document}
\maketitle

\textbf{Warning: 
requires JavaScript to process the mathematics on this page.\\ If your
browser supports JavaScript, be sure it is enabled.}

\begin{center}\rule{3in}{0.4pt}\end{center}

[
[
[]
[

\section{2.2 Bases et dimension}

\subsection{2.2.1 Existence de bases}

Lemme~2.2.1 fondamental. Soit E un K-espace vectoriel et \mathcal{E} une famille
de E. On a équivalence de (i) \mathcal{E} est une base de E (ii) \mathcal{E} est une famille
libre maximale (toute surfamille stricte est liée) (iii) \mathcal{E} est une
famille génératrice minimale (toute sous-famille stricte est non
génératrice).

Démonstration Si \mathcal{E} est une base, on ne peut lui adjoindre aucun vecteur
x sans obtenir une famille liée puisque x est combinaison linéaire de
\mathcal{E}~; on ne peut lui retirer aucun vecteur e_i_0 sans
obtenir une famille non génératrice puisque e_i_0
n'est pas combinaison linéaire des autres vecteurs. Donc une base est
une famille libre maximale et une famille génératrice minimale.

Inversement, soit \mathcal{E} une famille libre maximale~; si on lui adjoint un
vecteur x, la famille devient liée, c'est donc que x est combinaison
linéaire de \mathcal{E} et donc \mathcal{E} est également génératrice, donc c'est une base.

De même soit \mathcal{E} une famille génératrice minimale~; si elle était liée,
l'un des vecteurs serait combinaison linéaire des autres vecteurs et la
famille obtenue en retirant ce vecteur serait encore génératrice, ce qui
n'est pas~; la famille est donc libre, donc c'est une base.

Théorème~2.2.2 (de la base incomplète). Soit E un K-espace vectoriel ,
(e_i)_i\inI une famille génératrice de E. Soit L une
partie de I telle que la famille (e_i)_i\inL soit libre.
Alors il existe une partie J avec L \subset~ J \subset~ I telle que
(e_i)_i\inJ est une base de E.

Démonstration On considère X =
\J∣L \subset~ J \subset~
I\text et
(e_i)_i\inJ\text
libre\ ordonné par l'inclusion. Si
(J_s)_s\inS est une famille totalement ordonnée de X,
alors J = \cup_s\inSJ_s est encore dans X (facile car si on
a une sous-famille finie de la famille (e_j)_j\inJ, ils
sont tous dans un même J_s, donc forment une famille libre) et
c'est un majorant de la famille. Le théorème de Zorn garantit que
l'ensemble X admet un élément maximal. Soit J un tel élément maximal. La
famille (e_i)_i\inJ est libre. D'autre part, pour tout j
\in I \diagdown J, la famille
(e_i)_i\inI\cup\j\ est
liée, donc e_j est combinaison linéaire de la famille
(e_i)_i\inJ. On en déduit que la famille est aussi
génératrice.

Corollaire~2.2.3 Tout K-espace vectoriel admet des bases.

Démonstration Prendre la famille (x)_x\inE comme famille
génératrice et la famille vide comme sous-famille libre.

Corollaire~2.2.4 Soit E un K-espace vectoriel et F un sous-espace
vectoriel de E. Alors F admet des supplémentaires dans E.

Démonstration Il suffit de prendre une base de F, de la compléter en une
base de E et de prendre pour supplémentaire le sous-espace vectoriel
engendré par les vecteurs de la base introduits en supplément.

\subsection{2.2.2 Espaces vectoriels de dimension finie. Dimension}

Définition~2.2.1 On dit qu'un espace vectoriel E est de dimension finie,
s'il admet une famille génératrice finie.

Lemme~2.2.5 Soit E un espace vectoriel de dimension finie,
(x_1,\\ldots,x_n~)
une famille génératrice finie. Alors toute famille libre a un cardinal
inférieur ou égal à n.

Démonstration Il suffit de démontrer par récurrence sur n que si n + 1
vecteurs
y_1,\\ldots,y_n+1~
sont combinaisons linéaires de n vecteurs
x_1,\\ldots,x_n~,
alors la famille
(y_1,\\ldots,y_n+1~)
est liée (c'est évident pour n = 1). Pour cela on écrit y_j
= \\sum ~
_i=1^na_i,jx_j. Si, pour tout i,
a_i,n = 0 alors
y_1,\\ldots,y_n~
sont combinaisons linéaires de
x_1,\\ldots,x_n-1~,
donc forment une famille liée~; il en est de même a fortiori de la
famille complète. Sinon par exemple
a_n+1,n\neq~0. On pose alors
\forall~i \in [1,n], z_i = y_i~
- a_i,n \over a_n+1,n
y_n+1 =\ \\sum
 _j=1^n-1b_i,jx_j. Par récurrence,
la famille
z_1,\\ldots,z_n~
est liée (combinaisons linéaires de
x_1,\\ldots,x_n-1~),
donc il existe
\alpha_1,\\ldots,\alpha_n~
non tous nuls tels que \alpha_1z_1 +
\\ldots~ +
\alpha_nz_n = 0 ce qui donne après remplacement
\alpha_1y_1 +
\\ldots~ +
\alpha_ny_n + \beta~y_n+1 = 0 et montre que la famille
est liée. Ceci achève la démonstration.

Remarque~2.2.1 Ceci permet de redémontrer dans ce cas le théorème de la
base incomplète sans faire appel au théorème de Zorn, l'existence d'un
sous-ensemble J de X maximal étant garanti par la limitation sur son
cardinal induite par le lemme précédent. Le lemme précédent montre aussi
que le cardinal d'une famille libre est nécessairement fini et que le
cardinal d'une famille libre est inférieur au cardinal d'une famille
génératrice. Comme les bases sont à la fois libres et génératrices, on a
le théorème suivant~:

Théorème~2.2.6 Soit E un K-espace vectoriel de dimension finie. Alors
toutes les bases de E ont un même cardinal fini, appelé la dimension de
E. Toute famille libre est finie et a un cardinal inférieur ou égal à
dim~ E avec égalité si et seulement si c'est
une base de E. Toute famille génératrice a un cardinal supérieur ou égal
à dim~ E avec égalité si et seulement si c'est
une base de E.

\subsection{2.2.3 Résultats sur la dimension}

Théorème~2.2.7 Deux espaces sont isomorphes si et seulement si ils ont
la même dimension.

Démonstration Deux espaces de même dimension sont isomorphes~: prendre
deux bases et envoyer l'une sur l'autre. Inversement, l'image d'une base
par un isomorphisme étant une base, deux espaces isomorphes ont même
dimension.

Théorème~2.2.8

\begin{itemize}
\itemsep1pt\parskip0pt\parsep0pt
\item
  dim (E \times F) =\ dim~
  E + dim~ F
\item
  dim (F_1~
  \oplus~⋯ \oplus~ F_k)
  = \\sum ~
  dim F_i~
\item
   dim (E\diagupF) =\ dim~ E
  - dim~ F
\item
  f : E \rightarrow~ F linéaire, dim~ E
  = dim~
  \mathrmKer~f
  + dim~
  \mathrmIm~f (théorème du
  rang)
\item
  dim (F + G) =\ dim~
  F + dim G -\ dim~ F
  \bigcap G
\item
  F \subset~ E,\quad dim~ F
  \leq dim~ E avec égalité si et seulement si F =
  E
\item
  dim L(E,F) =\ dim~
  E.dim~ F
\end{itemize}

Démonstration

\begin{itemize}
\item
  Si (e_i)_i\inI et (f_j)_j\inJ sont des
  bases respectives de E et F, on vérifie immédiatement que la famille
  \left ((e_i,0)\right
  )_i\inI \cup\left
  ((0,f_j)\right )_j\inJ) est une base de E
  \times F en écrivant (x,y) = (x,0) + (0,y).
\item
  F_1 \oplus~⋯ \oplus~ F_k est
  isomorphe à F_1 \times⋯ \times F_k
  (prendre
  (x_1,\\ldots,x_k)\mapsto~x_1~
  + \\ldots~ +
  x_k) qui est de dimension
  \\sum ~
  dim F_i~ par récurrence à partir du
  résultat précédent.
\item
  L'espace E\diagupF est isomorphe à tout supplémentaire de F dans E qui est
  de dimension dim~ E -\
  dim F d'après le résultat précédent.
\item
  Il existe un isomorphisme \overlinef de
  E\diagup\mathrmKer~f sur
  \mathrmIm~f, d'où
  dim~
  \mathrmIm~f
  = dim~
  E\diagup\mathrmKer~f
  = dim E -\ dim~
  \mathrmKer~f.
\item
  L'application linéaire f : F \times G \rightarrow~ E,
  (x,y)\mapsto~x + y a pour image F + G et pour
  noyau \(x,-x)∣x \in F \bigcap
  G\ qui est naturellement isomorphe à F \bigcap G. Il suffit
  donc d'appliquer le résultat précédent à f.
\item
  D'après le théorème de la base incomplète, toute base de F est une
  famille libre de E, donc peut être complétée en une base de E, d'où
  dim F \leq\ dim~ E avec
  égalité si et seulement si F = E
\item
  On peut soit utiliser le résultat analogue sur les matrices que l'on
  verra plus loin, soit montrer que si (e_i)_i\inI et
  (f_j)_j\inJ sont des bases respectives de E et F,
  alors les applications linéaires \phi_j,i \in L(E,F) définies par

   \phi_j,i(e_k) = \left
  \\cases f_j&si k = i
  \cr 0 &si k\neq~i 
  \right .

  forment une base de L(E,F), ce qui est laissé en exercice.
\end{itemize}

Théorème~2.2.9 Soit F et G deux sous-espaces vectoriels de E de
dimension finie. On a équivalence de (i) F et G sont supplémentaires
dans E (ii) E = F + G et dim~ E
= dim F +\ dim~ G
(iii) F \bigcap G = \0\ et
dim E =\ dim~ F
+ dim~ G

Démonstration Elémentaire.

Remarque~2.2.2 On sait que si F est un sous-espace vectoriel de E, la
restriction à tout supplémentaire G de F dans E de la projection
canonique de E sur E\diagupF est un isomorphisme d'espace vectoriel. Ceci
justifie la définition suivante~:

Définition~2.2.2 Soit E un K espace vectoriel et F un sous-espace
vectoriel de E. On appelle codimension de F la dimension (éventuellement
infinie) de l'espace vectoriel quotient E\diagupF, qui est encore la dimension
de tout supplémentaire de F dans E.

[
[
[
[

\end{document}

% \documentclass[]{article}
\usepackage[T1]{fontenc}
\usepackage{lmodern}
\usepackage{amssymb,amsmath}
\usepackage{ifxetex,ifluatex}
\usepackage{fixltx2e} % provides \textsubscript
% use upquote if available, for straight quotes in verbatim environments
\IfFileExists{upquote.sty}{\usepackage{upquote}}{}
\ifnum 0\ifxetex 1\fi\ifluatex 1\fi=0 % if pdftex
  \usepackage[utf8]{inputenc}
\else % if luatex or xelatex
  \ifxetex
    \usepackage{mathspec}
    \usepackage{xltxtra,xunicode}
  \else
    \usepackage{fontspec}
  \fi
  \defaultfontfeatures{Mapping=tex-text,Scale=MatchLowercase}
  \newcommand{\euro}{€}
\fi
% use microtype if available
\IfFileExists{microtype.sty}{\usepackage{microtype}}{}
\ifxetex
  \usepackage[setpagesize=false, % page size defined by xetex
              unicode=false, % unicode breaks when used with xetex
              xetex]{hyperref}
\else
  \usepackage[unicode=true]{hyperref}
\fi
\hypersetup{breaklinks=true,
            bookmarks=true,
            pdfauthor={},
            pdftitle={Rang},
            colorlinks=true,
            citecolor=blue,
            urlcolor=blue,
            linkcolor=magenta,
            pdfborder={0 0 0}}
\urlstyle{same}  % don't use monospace font for urls
\setlength{\parindent}{0pt}
\setlength{\parskip}{6pt plus 2pt minus 1pt}
\setlength{\emergencystretch}{3em}  % prevent overfull lines
\setcounter{secnumdepth}{0}
 
/* start css.sty */
.cmr-5{font-size:50%;}
.cmr-7{font-size:70%;}
.cmmi-5{font-size:50%;font-style: italic;}
.cmmi-7{font-size:70%;font-style: italic;}
.cmmi-10{font-style: italic;}
.cmsy-5{font-size:50%;}
.cmsy-7{font-size:70%;}
.cmex-7{font-size:70%;}
.cmex-7x-x-71{font-size:49%;}
.msbm-7{font-size:70%;}
.cmtt-10{font-family: monospace;}
.cmti-10{ font-style: italic;}
.cmbx-10{ font-weight: bold;}
.cmr-17x-x-120{font-size:204%;}
.cmsl-10{font-style: oblique;}
.cmti-7x-x-71{font-size:49%; font-style: italic;}
.cmbxti-10{ font-weight: bold; font-style: italic;}
p.noindent { text-indent: 0em }
td p.noindent { text-indent: 0em; margin-top:0em; }
p.nopar { text-indent: 0em; }
p.indent{ text-indent: 1.5em }
@media print {div.crosslinks {visibility:hidden;}}
a img { border-top: 0; border-left: 0; border-right: 0; }
center { margin-top:1em; margin-bottom:1em; }
td center { margin-top:0em; margin-bottom:0em; }
.Canvas { position:relative; }
li p.indent { text-indent: 0em }
.enumerate1 {list-style-type:decimal;}
.enumerate2 {list-style-type:lower-alpha;}
.enumerate3 {list-style-type:lower-roman;}
.enumerate4 {list-style-type:upper-alpha;}
div.newtheorem { margin-bottom: 2em; margin-top: 2em;}
.obeylines-h,.obeylines-v {white-space: nowrap; }
div.obeylines-v p { margin-top:0; margin-bottom:0; }
.overline{ text-decoration:overline; }
.overline img{ border-top: 1px solid black; }
td.displaylines {text-align:center; white-space:nowrap;}
.centerline {text-align:center;}
.rightline {text-align:right;}
div.verbatim {font-family: monospace; white-space: nowrap; text-align:left; clear:both; }
.fbox {padding-left:3.0pt; padding-right:3.0pt; text-indent:0pt; border:solid black 0.4pt; }
div.fbox {display:table}
div.center div.fbox {text-align:center; clear:both; padding-left:3.0pt; padding-right:3.0pt; text-indent:0pt; border:solid black 0.4pt; }
div.minipage{width:100%;}
div.center, div.center div.center {text-align: center; margin-left:1em; margin-right:1em;}
div.center div {text-align: left;}
div.flushright, div.flushright div.flushright {text-align: right;}
div.flushright div {text-align: left;}
div.flushleft {text-align: left;}
.underline{ text-decoration:underline; }
.underline img{ border-bottom: 1px solid black; margin-bottom:1pt; }
.framebox-c, .framebox-l, .framebox-r { padding-left:3.0pt; padding-right:3.0pt; text-indent:0pt; border:solid black 0.4pt; }
.framebox-c {text-align:center;}
.framebox-l {text-align:left;}
.framebox-r {text-align:right;}
span.thank-mark{ vertical-align: super }
span.footnote-mark sup.textsuperscript, span.footnote-mark a sup.textsuperscript{ font-size:80%; }
div.tabular, div.center div.tabular {text-align: center; margin-top:0.5em; margin-bottom:0.5em; }
table.tabular td p{margin-top:0em;}
table.tabular {margin-left: auto; margin-right: auto;}
div.td00{ margin-left:0pt; margin-right:0pt; }
div.td01{ margin-left:0pt; margin-right:5pt; }
div.td10{ margin-left:5pt; margin-right:0pt; }
div.td11{ margin-left:5pt; margin-right:5pt; }
table[rules] {border-left:solid black 0.4pt; border-right:solid black 0.4pt; }
td.td00{ padding-left:0pt; padding-right:0pt; }
td.td01{ padding-left:0pt; padding-right:5pt; }
td.td10{ padding-left:5pt; padding-right:0pt; }
td.td11{ padding-left:5pt; padding-right:5pt; }
table[rules] {border-left:solid black 0.4pt; border-right:solid black 0.4pt; }
.hline hr, .cline hr{ height : 1px; margin:0px; }
.tabbing-right {text-align:right;}
span.TEX {letter-spacing: -0.125em; }
span.TEX span.E{ position:relative;top:0.5ex;left:-0.0417em;}
a span.TEX span.E {text-decoration: none; }
span.LATEX span.A{ position:relative; top:-0.5ex; left:-0.4em; font-size:85%;}
span.LATEX span.TEX{ position:relative; left: -0.4em; }
div.float img, div.float .caption {text-align:center;}
div.figure img, div.figure .caption {text-align:center;}
.marginpar {width:20%; float:right; text-align:left; margin-left:auto; margin-top:0.5em; font-size:85%; text-decoration:underline;}
.marginpar p{margin-top:0.4em; margin-bottom:0.4em;}
.equation td{text-align:center; vertical-align:middle; }
td.eq-no{ width:5%; }
table.equation { width:100%; } 
div.math-display, div.par-math-display{text-align:center;}
math .texttt { font-family: monospace; }
math .textit { font-style: italic; }
math .textsl { font-style: oblique; }
math .textsf { font-family: sans-serif; }
math .textbf { font-weight: bold; }
.partToc a, .partToc, .likepartToc a, .likepartToc {line-height: 200%; font-weight:bold; font-size:110%;}
.chapterToc a, .chapterToc, .likechapterToc a, .likechapterToc, .appendixToc a, .appendixToc {line-height: 200%; font-weight:bold;}
.index-item, .index-subitem, .index-subsubitem {display:block}
.caption td.id{font-weight: bold; white-space: nowrap; }
table.caption {text-align:center;}
h1.partHead{text-align: center}
p.bibitem { text-indent: -2em; margin-left: 2em; margin-top:0.6em; margin-bottom:0.6em; }
p.bibitem-p { text-indent: 0em; margin-left: 2em; margin-top:0.6em; margin-bottom:0.6em; }
.subsectionHead, .likesubsectionHead { margin-top:2em; font-weight: bold;}
.sectionHead, .likesectionHead { font-weight: bold;}
.quote {margin-bottom:0.25em; margin-top:0.25em; margin-left:1em; margin-right:1em; text-align:justify;}
.verse{white-space:nowrap; margin-left:2em}
div.maketitle {text-align:center;}
h2.titleHead{text-align:center;}
div.maketitle{ margin-bottom: 2em; }
div.author, div.date {text-align:center;}
div.thanks{text-align:left; margin-left:10%; font-size:85%; font-style:italic; }
div.author{white-space: nowrap;}
.quotation {margin-bottom:0.25em; margin-top:0.25em; margin-left:1em; }
h1.partHead{text-align: center}
.sectionToc, .likesectionToc {margin-left:2em;}
.subsectionToc, .likesubsectionToc {margin-left:4em;}
.sectionToc, .likesectionToc {margin-left:6em;}
.frenchb-nbsp{font-size:75%;}
.frenchb-thinspace{font-size:75%;}
.figure img.graphics {margin-left:10%;}
/* end css.sty */

\title{Rang}
\author{}
\date{}

\begin{document}
\maketitle

\textbf{Warning: 
requires JavaScript to process the mathematics on this page.\\ If your
browser supports JavaScript, be sure it is enabled.}

\begin{center}\rule{3in}{0.4pt}\end{center}

[
[
[]
[

\section{2.3 Rang}

\subsection{2.3.1 Rang d'une famille de vecteurs}

Définition~2.3.1
\mathrmrg(x_i)_i\inI~
= dim~
\mathrmVect(x_i~,i
\in I) =\
sup\J∣(x_i)_i\inJ\text
libre \

Démonstration L'égalité provient évidemment du théorème de la base
incomplète qui garantit que l'on peut extraire de la famille
(x_i) une base du sous-espace
\mathrmVect(x_i~).

Une application linéaire transformant une famille liée en une famille
liée, on a~:

Théorème~2.3.1 Soit u \in L(E,F) et (x_i)_i\inI une
famille de E. Alors
\mathrmrg(u(x_i))_i\inI~
\leq\mathrmrg(x_i)_i\inI~.

Recherche pratique~: voir le rang d'une matrice.

\subsection{2.3.2 Rang d'une application linéaire}

Définition~2.3.2 Soit u \in L(E,F). Alors
\mathrmrg~u
= dim~
\mathrmIm~u \in \mathbb{N}~
\cup\ + \infty~\.

Remarque~2.3.1 Recherche pratique~: Si (e_i)_i\inI est
une base de E, (u(e_i))_i\inI est une famille
génératrice de \mathrmIm~u
et donc \mathrmrg~u
=\
\mathrmrg(u(e_i))_i\inI.

Théorème~2.3.2 Soit u \in L(E,F). (i) Si dim~ E
< +\infty~, alors u est de rang fini,
\mathrmrg~u
= dim E -\ dim~
\mathrmKer~u~; on a
\mathrmrg~u
= dim~ E si et seulement si u est injectif (ii)
Si dim~ F < +\infty~, alors u est de rang
fini, \mathrmrg~u
\leq dim~ F~; on a
\mathrmrg~u
= dim~ F si et seulement si u est surjectif

Démonstration Découle immédiatement des résultats sur la dimension.

Remarque~2.3.2 On a donc dans tous les cas
\mathrmrg~u
\leq min(\dim~
E,dim~ F)

Proposition~2.3.3 Soit u \in L(E,F),v \in L(F,G). Alors
\mathrmrg~v \cdot u
\leq\
min(\mathrmrgv,\\mathrmrg~u).
Si u est bijectif,
\mathrmrg~v \cdot u
= \mathrmrg~v. Si v est
bijectif, \mathrmrg~v \cdot u
= \mathrmrg~u.

Démonstration Il suffit de remarquer que
\mathrmIm~v \cdot u = v(u(E)).

Théorème~2.3.4 On suppose ici dim~ E
= dim~ F < +\infty~, u \in L(E,F). On a
équivalence de

\begin{itemize}
\itemsep1pt\parskip0pt\parsep0pt
\item
  (i) u injective
\item
  (ii) u surjective
\item
  (iii) u bijective
\item
  (iv) \exists~v \in L(F,E), v \cdot u =
  \mathrmId_E
\item
  (v) \exists~v \in L(F,E), u \cdot v =
  \mathrmId_F
\end{itemize}

Démonstration L'équivalence entre (i), (ii) et (iii) est une compilation
des résultats précédents. De plus (iii) \rigtharrow~ (iv) et (v), (iv) \rigtharrow~ (i) et (v)
\rigtharrow~(ii), ce qui boucle la boucle.

[
[
[
[

\end{document}

% \documentclass[]{article}
\usepackage[T1]{fontenc}
\usepackage{lmodern}
\usepackage{amssymb,amsmath}
\usepackage{ifxetex,ifluatex}
\usepackage{fixltx2e} % provides \textsubscript
% use upquote if available, for straight quotes in verbatim environments
\IfFileExists{upquote.sty}{\usepackage{upquote}}{}
\ifnum 0\ifxetex 1\fi\ifluatex 1\fi=0 % if pdftex
  \usepackage[utf8]{inputenc}
\else % if luatex or xelatex
  \ifxetex
    \usepackage{mathspec}
    \usepackage{xltxtra,xunicode}
  \else
    \usepackage{fontspec}
  \fi
  \defaultfontfeatures{Mapping=tex-text,Scale=MatchLowercase}
  \newcommand{\euro}{€}
\fi
% use microtype if available
\IfFileExists{microtype.sty}{\usepackage{microtype}}{}
\ifxetex
  \usepackage[setpagesize=false, % page size defined by xetex
              unicode=false, % unicode breaks when used with xetex
              xetex]{hyperref}
\else
  \usepackage[unicode=true]{hyperref}
\fi
\hypersetup{breaklinks=true,
            bookmarks=true,
            pdfauthor={},
            pdftitle={Dualite : approche restreinte},
            colorlinks=true,
            citecolor=blue,
            urlcolor=blue,
            linkcolor=magenta,
            pdfborder={0 0 0}}
\urlstyle{same}  % don't use monospace font for urls
\setlength{\parindent}{0pt}
\setlength{\parskip}{6pt plus 2pt minus 1pt}
\setlength{\emergencystretch}{3em}  % prevent overfull lines
\setcounter{secnumdepth}{0}
 
/* start css.sty */
.cmr-5{font-size:50%;}
.cmr-7{font-size:70%;}
.cmmi-5{font-size:50%;font-style: italic;}
.cmmi-7{font-size:70%;font-style: italic;}
.cmmi-10{font-style: italic;}
.cmsy-5{font-size:50%;}
.cmsy-7{font-size:70%;}
.cmex-7{font-size:70%;}
.cmex-7x-x-71{font-size:49%;}
.msbm-7{font-size:70%;}
.cmtt-10{font-family: monospace;}
.cmti-10{ font-style: italic;}
.cmbx-10{ font-weight: bold;}
.cmr-17x-x-120{font-size:204%;}
.cmsl-10{font-style: oblique;}
.cmti-7x-x-71{font-size:49%; font-style: italic;}
.cmbxti-10{ font-weight: bold; font-style: italic;}
p.noindent { text-indent: 0em }
td p.noindent { text-indent: 0em; margin-top:0em; }
p.nopar { text-indent: 0em; }
p.indent{ text-indent: 1.5em }
@media print {div.crosslinks {visibility:hidden;}}
a img { border-top: 0; border-left: 0; border-right: 0; }
center { margin-top:1em; margin-bottom:1em; }
td center { margin-top:0em; margin-bottom:0em; }
.Canvas { position:relative; }
li p.indent { text-indent: 0em }
.enumerate1 {list-style-type:decimal;}
.enumerate2 {list-style-type:lower-alpha;}
.enumerate3 {list-style-type:lower-roman;}
.enumerate4 {list-style-type:upper-alpha;}
div.newtheorem { margin-bottom: 2em; margin-top: 2em;}
.obeylines-h,.obeylines-v {white-space: nowrap; }
div.obeylines-v p { margin-top:0; margin-bottom:0; }
.overline{ text-decoration:overline; }
.overline img{ border-top: 1px solid black; }
td.displaylines {text-align:center; white-space:nowrap;}
.centerline {text-align:center;}
.rightline {text-align:right;}
div.verbatim {font-family: monospace; white-space: nowrap; text-align:left; clear:both; }
.fbox {padding-left:3.0pt; padding-right:3.0pt; text-indent:0pt; border:solid black 0.4pt; }
div.fbox {display:table}
div.center div.fbox {text-align:center; clear:both; padding-left:3.0pt; padding-right:3.0pt; text-indent:0pt; border:solid black 0.4pt; }
div.minipage{width:100%;}
div.center, div.center div.center {text-align: center; margin-left:1em; margin-right:1em;}
div.center div {text-align: left;}
div.flushright, div.flushright div.flushright {text-align: right;}
div.flushright div {text-align: left;}
div.flushleft {text-align: left;}
.underline{ text-decoration:underline; }
.underline img{ border-bottom: 1px solid black; margin-bottom:1pt; }
.framebox-c, .framebox-l, .framebox-r { padding-left:3.0pt; padding-right:3.0pt; text-indent:0pt; border:solid black 0.4pt; }
.framebox-c {text-align:center;}
.framebox-l {text-align:left;}
.framebox-r {text-align:right;}
span.thank-mark{ vertical-align: super }
span.footnote-mark sup.textsuperscript, span.footnote-mark a sup.textsuperscript{ font-size:80%; }
div.tabular, div.center div.tabular {text-align: center; margin-top:0.5em; margin-bottom:0.5em; }
table.tabular td p{margin-top:0em;}
table.tabular {margin-left: auto; margin-right: auto;}
div.td00{ margin-left:0pt; margin-right:0pt; }
div.td01{ margin-left:0pt; margin-right:5pt; }
div.td10{ margin-left:5pt; margin-right:0pt; }
div.td11{ margin-left:5pt; margin-right:5pt; }
table[rules] {border-left:solid black 0.4pt; border-right:solid black 0.4pt; }
td.td00{ padding-left:0pt; padding-right:0pt; }
td.td01{ padding-left:0pt; padding-right:5pt; }
td.td10{ padding-left:5pt; padding-right:0pt; }
td.td11{ padding-left:5pt; padding-right:5pt; }
table[rules] {border-left:solid black 0.4pt; border-right:solid black 0.4pt; }
.hline hr, .cline hr{ height : 1px; margin:0px; }
.tabbing-right {text-align:right;}
span.TEX {letter-spacing: -0.125em; }
span.TEX span.E{ position:relative;top:0.5ex;left:-0.0417em;}
a span.TEX span.E {text-decoration: none; }
span.LATEX span.A{ position:relative; top:-0.5ex; left:-0.4em; font-size:85%;}
span.LATEX span.TEX{ position:relative; left: -0.4em; }
div.float img, div.float .caption {text-align:center;}
div.figure img, div.figure .caption {text-align:center;}
.marginpar {width:20%; float:right; text-align:left; margin-left:auto; margin-top:0.5em; font-size:85%; text-decoration:underline;}
.marginpar p{margin-top:0.4em; margin-bottom:0.4em;}
.equation td{text-align:center; vertical-align:middle; }
td.eq-no{ width:5%; }
table.equation { width:100%; } 
div.math-display, div.par-math-display{text-align:center;}
math .texttt { font-family: monospace; }
math .textit { font-style: italic; }
math .textsl { font-style: oblique; }
math .textsf { font-family: sans-serif; }
math .textbf { font-weight: bold; }
.partToc a, .partToc, .likepartToc a, .likepartToc {line-height: 200%; font-weight:bold; font-size:110%;}
.chapterToc a, .chapterToc, .likechapterToc a, .likechapterToc, .appendixToc a, .appendixToc {line-height: 200%; font-weight:bold;}
.index-item, .index-subitem, .index-subsubitem {display:block}
.caption td.id{font-weight: bold; white-space: nowrap; }
table.caption {text-align:center;}
h1.partHead{text-align: center}
p.bibitem { text-indent: -2em; margin-left: 2em; margin-top:0.6em; margin-bottom:0.6em; }
p.bibitem-p { text-indent: 0em; margin-left: 2em; margin-top:0.6em; margin-bottom:0.6em; }
.subsectionHead, .likesubsectionHead { margin-top:2em; font-weight: bold;}
.sectionHead, .likesectionHead { font-weight: bold;}
.quote {margin-bottom:0.25em; margin-top:0.25em; margin-left:1em; margin-right:1em; text-align:justify;}
.verse{white-space:nowrap; margin-left:2em}
div.maketitle {text-align:center;}
h2.titleHead{text-align:center;}
div.maketitle{ margin-bottom: 2em; }
div.author, div.date {text-align:center;}
div.thanks{text-align:left; margin-left:10%; font-size:85%; font-style:italic; }
div.author{white-space: nowrap;}
.quotation {margin-bottom:0.25em; margin-top:0.25em; margin-left:1em; }
h1.partHead{text-align: center}
.sectionToc, .likesectionToc {margin-left:2em;}
.subsectionToc, .likesubsectionToc {margin-left:4em;}
.sectionToc, .likesectionToc {margin-left:6em;}
.frenchb-nbsp{font-size:75%;}
.frenchb-thinspace{font-size:75%;}
.figure img.graphics {margin-left:10%;}
/* end css.sty */

\title{Dualite : approche restreinte}
\author{}
\date{}

\begin{document}
\maketitle

\textbf{Warning: 
requires JavaScript to process the mathematics on this page.\\ If your
browser supports JavaScript, be sure it is enabled.}

\begin{center}\rule{3in}{0.4pt}\end{center}

[
[
[]
[

\section{2.4 Dualité~: approche restreinte}

\subsection{2.4.1 Formes linéaires, dual, formes coordonnées}

Définition~2.4.1 Soit E un K-espace vectoriel . On appelle forme
linéaire sur E toute application linéaire de E dans K. On appelle dual
de E le K-espace vectoriel E^∗ = L(E,K).

Remarque~2.4.1 Soit (e_i)_i\inI une base de E et
i_0 \in I. Tout vecteur x de E s'écrit de manière unique sous la
forme x = \\sum ~
_i\inIx_ie_i. L'application
\phi_i_0 :
x\mapsto~x_i_0 est clairement une
forme linéaire sur E, appelée forme linéaire coordonnée d'indice
i_0 dans la base (e_i)_i\inI. Elle est définie
par \phi_i_0(e_i_0) = 1 et
\phi_i_0(e_i) = 0 si
i\neq~i_0, soit encore par
\phi_i_0(e_i) =
\delta_i_0^i.

Proposition~2.4.1 Soit E un K-espace vectoriel et x \in E,
x\neq~0. Alors il existe une forme linéaire \phi sur
E telle que \phi(x) = 1.

Démonstration Le vecteur x forme à lui tout seul une famille libre que
l'on peut compléter en une base (e_i)_i\inI de E avec x
= e_i_0. Soit \phi la forme linéaire qui associe à tout
vecteur de E sa i_0-ième coordonnée dans cette base. On a bien
entendu \phi(x) = 1.

Remarque~2.4.2 Le résultat précédent peut encore s'interpréter sous la
forme~: si x \in E,

\forall~\phi \in E^∗~, \phi(x) =
0\quad \Leftrightarrow x = 0

\subsection{2.4.2 Base duale d'un espace vectoriel de dimension finie}

Définition~2.4.2 Soit E un espace vectoriel de dimension finie, \mathcal{E} =
(e_1,\\ldots,e_n~)
une base de E. Pour i \in [1,n], soit \phi_i la forme linéaire
coordonnée d'indice i dans la base \mathcal{E}. Alors \mathcal{E}^∗ =
(\phi_1,\\ldots,\phi_n~)
est une base du dual E^∗, appelée la base duale de la base \mathcal{E}.
Elle est caractérisée par les relations \forall~~i,j \in
[1,n], \phi_i(e_j) = \delta_i^j.

Démonstration Tout d'abord, montrons que \mathcal{E}^∗ est une famille
libre en utilisant les relations de définition des formes coordonnées

\forall~i,j \in [1,n], \phi_i(e_j~) =
\delta_i^j

Soit
\lambda_1,\\ldots,\lambda_n~
\in K tels que \lambda_1\phi_1 +
\\ldots~ +
\lambda_n\phi_n = 0~; on a alors, pout tout i \in [1,n]

0 = 0(e_i) = \lambda_1\phi_1(e_i) +
\\ldots~ +
\lambda_n\phi_n(e_i) = \lambda_i

ce qui montre bien que la famille est libre. Pour montrer qu'elle est
génératrice, soit \phi \in E et considérons \psi =\
\sum ~
_i=1^n\phi(e_i)\phi_i~; on a alors pour tout j
\in [1,n],

\psi(e_j) = \\sum
_i=1^n\phi(e_ i)\phi_i(e_j) =
\sum _i=1^n\phi(e_
i)\delta_i^j = \phi(e_ j)

Les deux applications linéaires \phi et \psi coïncidant sur une base, sont
égales, ce qui montre que la famille est génératrice.

Remarque~2.4.3 Attention~: la dimension finie est essentielle~; elle
garantit qu'il n'y a qu'un nombre fini de \phi(e_i) non nuls et
permet de considérer la somme
\\sum ~
_i\inI\phi(e_i)\phi_i~; en dimension infinie,
\mathcal{E}^∗ n'est pas une base de E^∗, car elle n'est pas
génératrice (considérer la forme linéaire \phi qui à tout e_i
associe 1).

Corollaire~2.4.2 La dimension de l'espace dual d'un espace vectoriel de
dimension finie est égale à la dimension de l'espace.

\subsection{2.4.3 Orthogonalité 1}

Soit E un K-espace vectoriel de dimension finie,
(e_1,\\ldots,e_p~)
une famille d'éléments de E. Nous pouvons associer à cette famille
l'application u : E^∗\rightarrow~ K^p, définie par u(\phi) =
(\phi(e_1),\\ldots,\phi(e_p~)).
On vérifie immédiatement que u est linéaire. Son noyau est constitué des
\phi \in E^∗ vérifiant \forall~~i \in [1,p],
\phi(e_i) = 0.

Proposition~2.4.3 Si
(e_1,\\ldots,e_p~)
est une base de E, alors u est un isomorphisme d'espace vectoriel de
E^∗ sur K^p.

Démonstration En effet dans ce cas, u envoie la base duale
\mathcal{E}^∗ sur la base canonique de K^p~; c'est donc un
isomorphisme.

Proposition~2.4.4 La famille
(e_1,\\ldots,e_p~)
est libre si et seulement si u est surjective. Sous ces conditions,
\mathrmKer~u est de
codimension p et

\forall~~x \in E,\quad (x
\in\mathrmVect(e_1,\\\ldots,e_p~)
\Leftrightarrow \forall~~\phi
\in\mathrmKer~u, \phi(x) = 0)

Démonstration Si u est surjective, notons
(\epsilon_1,\\ldots,\epsilon_p~)
la base canonique de K^p et soit \phi_i \in
E^∗ tel que u(\phi_i) = \epsilon_i~; on a donc
\forall~i,j \in [1,p], \phi_i(e_j~) =
\delta_i^j ce qui implique évidemment que la famille est
libre~: si \\sum ~
_j=1^p\lambda_je_j = 0, on a pour tout i \in
[1,p]

0 = \phi_i(0) = \phi_i(\\sum
_j=1^p\lambda_ je_j) =
\sum _j=1^p\lambda~_
j\phi_i(e_j) = \lambda_i

Inversement, si la famille est libre, on peut compléter la famille
(e_1,\\ldots,e_p~)
en une base
(e_1,\\ldots,e_n~)
de E et soit
(\phi_1,\\ldots,\phi_n~)
la base duale. On a alors \forall~~i,j \in [1,p],
\phi_i(e_j) = \delta_i^j, soit
\forall~i \in [1,p], u(\phi_i~) =
\epsilon_i. L'image de u contient une base de K^p, c'est
donc K^p et u est surjective. Dans ces conditions, on peut
appliquer le théorème du rang, et donc dim~
\mathrmKer~u
= dim E^∗~- p
= dim~ E - p.

Soit \phi \in\mathrmKer~u~; alors
\forall~i \in [1,p], \phi(e_i~) = 0 et donc
\forall~~x
\in\mathrmVect(e_1,\\\ldots,e_p~),
\phi(x) = 0. Inversement, supposons que
x∉\mathrmVect(e_1,\\\ldots,e_p~)~;
alors la famille
(e_1,\\ldots,e_p~,x)
est libre, on peut la compléter en une base de E et la forme coordonnée
suivant x dans cette base, soit \phi, appartient à
\mathrmKer~u alors que \phi(x)
= 1. On a donc bien l'équivalence

\forall~~x \in E,\quad (x
\in\mathrmVect(e_1,\\\ldots,e_p~)
\Leftrightarrow \forall~~\phi
\in\mathrmKer~u, \phi(x) = 0)

Remarque~2.4.4 Application~: Soit F un sous-espace vectoriel de E,
(e_1,\\ldots,e_p~)
une base de F, u : E^∗\rightarrow~ K^p l'application linéaire
associée,
(\phi_1,\\ldots,\phi_n-p~)
une base de \mathrmKer~u~;
alors x \in F \Leftrightarrow \forall~~i \in
[1,n - p], \phi_i(x) = 0. On dit encore que F est défini par
le système d'équations linéaires \phi_1(x) =
0,\\ldots,\phi_n-p~(x)
= 0.

\subsection{2.4.4 Hyperplans}

Définition~2.4.3 On appelle hyperplan de E tout sous-espace vectoriel H
de E vérifiant les conditions équivalentes

\begin{itemize}
\itemsep1pt\parskip0pt\parsep0pt
\item
  (i) dim~ E\diagupH = 1
\item
  (ii) \exists~f \in E
  \diagdown\0\, H =\
  \mathrmKerf
\item
  (iii) H admet une droite comme supplémentaire.
\end{itemize}

Démonstration

\begin{itemize}
\itemsep1pt\parskip0pt\parsep0pt
\item
  (i) \rigtharrow~(ii)~: prendre \overlinee une base de E\diagupH et
  écrire \pi~(x) = f(x)\overlinee.
\item
  (ii) \rigtharrow~ (iii)~: on prend a \in E tel que
  f(a)\neq~0. Tout élément x s'écrit de manière
  unique sous la forme x = (x - f(x) \over f(a) a)
  + f(x) \over f(a) a avec x - f(x)
  \over f(a) a
  \in\mathrmKer~f, soit E
  = \mathrmKer~f \oplus~ Ka.
\item
  (iii) \rigtharrow~(i)~: tout supplémentaire de H est isomorphe à E\diagupH.
\end{itemize}

Théorème~2.4.5 Soit H un hyperplan de E. Alors deux formes linéaires
nulles sur H sont proportionnelles.

Démonstration Si E = H \oplus~ Ka et H =\
\mathrmKerf, soit g \in E^∗ nulle sur H.
Alors g et  g(a) \over f(a) f coïncident sur H et sur
Ka, donc sont égales.

\subsection{2.4.5 Orthogonalité 2}

Remarque~2.4.5 Soit E un K-espace vectoriel de dimension finie,
(\phi_1,\\ldots,\phi_p~)
une famille d'éléments de E^∗. Nous pouvons associer à cette
famille l'application v : E \rightarrow~ K^p, définie par v(x) =
(\phi_1(x),\\ldots,\phi_p~(x)).
On vérifie immédiatement que v est linéaire. Son noyau est constitué de
l'intersection des
\mathrmKer\phi_i~ (en
général des hyperplans, sauf si la forme linéaire est nulle).

Proposition~2.4.6 Si
(\phi_1,\\ldots,\phi_p~)
est une base de E^∗, alors v est un isomorphisme d'espace
vectoriel de E^∗ sur K^p.

Démonstration En effet dans ce cas, v est injective car

\begin{align*} v(x) = 0&
\Leftrightarrow & \forall~~i \in
[1,p], \phi_i(x) = 0\%& \\ &
\Leftrightarrow & \forall~~\phi \in
E^∗, \phi(x) = 0 \%& \\ &
\Leftrightarrow & x = 0 \%&
\\ \end{align*}

Comme dim E =\ dim~
E^∗ = p = dim K^p~, il
s'agit d'un isomorphisme.

Théorème~2.4.7 Soit
(\phi_1,\\ldots,\phi_p~)
une base de E^∗~; alors il existe une unique base
(e_1,\\ldots,e_p~)
de E dont
(\phi_1,\\ldots,\phi_p~)
soit la base duale.

Démonstration On a en effet \forall~~i \in
[1,p],\phi_i(e_j) = \delta_i^j
\Leftrightarrow v(e_j) = \epsilon_j (j-ième
vecteur de la base canonique). La famille
(e_1,\\ldots,e_p~)
est donc l'image de la base canonique de K^p par
l'isomorphisme v^-1.

Proposition~2.4.8 La famille
(\phi_1,\\ldots,\phi_p~)
est libre si et seulement si v est surjective. Sous ces conditions,
\mathrmKer~v est de
codimension p et \forall~\phi \in E^∗~,

(\phi
\in\mathrmVect(\phi_1,\\\ldots,\phi_p~)
\Leftrightarrow \forall~~x
\in\mathrmKer~v, \phi(x) = 0)

Démonstration Si v est surjective, notons
(\epsilon_1,\\ldots,\epsilon_p~)
la base canonique de K^p et soit e_i \in E tel que
v(e_i) = \epsilon_i~; on a donc
\forall~i,j \in [1,p], \phi_i(e_j~) =
\delta_i^j ce qui implique évidemment que la famille est
libre~: si \\sum ~
_j=1^p\lambda_j\phi_j = 0, on a pour tout i \in
[1,p]

0 = 0(e_i) = \\sum
_j=1^p\lambda_ j\phi_j(e_i) =
\sum _j=1^p\lambda~_
j\phi_j(e_i) = \lambda_i

Inversement, si la famille est libre, on peut compléter la famille
(\phi_1,\\ldots,\phi_p~)
en une base
(\phi_1,\\ldots,\phi_n~)
de E^∗ qui est la base duale de la base
(e_1,\\ldots,e_n~)
de E. On a alors \forall~~i,j \in [1,p],
\phi_i(e_j) = \delta_i^j, soit
\forall~i \in [1,p], v(e_i~) =
\epsilon_i. L'image de v contient une base de K^p, c'est
donc K^p et v est surjective. Dans ces conditions, on peut
appliquer le théorème du rang, et donc dim~
\mathrmKer~v
= dim~ E - p.

Soit x \in\mathrmKer~v~; alors
\forall~i \in [1,p], \phi_i~(x) = 0 et donc
\forall~~\phi
\in\mathrmVect(\phi_1,\\\ldots,\phi_p~),
\phi(x) = 0. Inversement, supposons que
\phi∉\mathrmVect(\phi_1,\\\ldots,\phi_p~)~;
alors la famille
(\phi_1,\\ldots,\phi_p~,\phi)
est libre, on peut la compléter en une base de E^∗ qui est la
base duale d'une base
(e_1,\\ldots,e_n~)
de E~; on a alors e_p+1
\in\mathrmKer~v et
\phi(e_p+1) = 1. On a donc bien l'équivalence

\begin{align*} \forall~~\phi \in
E^∗,\quad (\phi
\in\mathrmVect(\phi_
1,\\ldots,\phi_p~)&&\%&
\\ & \Leftrightarrow &
\forall~~x
\in\mathrmKer~v, \phi(x) = 0)\%&
\\ \end{align*}

Remarque~2.4.6 Application~: soit
H_1,\\ldots,H_p~
des hyperplans de E d'équations respectives \phi_1(x) =
0,\\ldots,\phi_p~(x)
= 0~; soit r =\
\mathrmrg(\phi_1,\\ldots,\phi_p~).
Quitte à renuméroter les H_i, on peut supposer que
(\phi_1,\\ldots,\phi_r~)
est une base de
\mathrmVect(\phi_1,\\\ldots,\phi_p~).
On a alors, si v : E \rightarrow~ K^r est l'application linéaire
associée à cette famille,

\begin{align*} x \in\⋂
_i=1^pH_ i& \Leftrightarrow &
\forall~i \in [1,p], \phi_i~(x) = 0\%&
\\ & \Leftrightarrow &
\forall~i \in [1,r], \phi_i~(x) = 0\%&
\\ & \Leftrightarrow & x
\in\mathrmKer~v \%&
\\ \end{align*}

ce qui montre que \\⋂
 _i=1^pH_i est un sous-espace vectoriel de
dimension dim~ E - r. Soit alors H un hyperplan
de E d'équation \phi(x) = 0. On a alors

\begin{align*} \⋂
_i=1^pH_ i \subset~ H& \Leftrightarrow
& \mathrmKer~v
\subset~\mathrmKer~\phi \%&
\\ & \Leftrightarrow & \phi
\in\mathrmVect(\phi_1,\\\ldots,\phi_r~)
=\
\mathrmVect(\phi_1,\\ldots,\phi_p~)\%&
\\ \end{align*}

\subsection{2.4.6 Application~: polynômes d'interpolation de Lagrange}

Théorème~2.4.9 Soit K un corps commutatif,
x_1,\\ldots,x_n~
\in K distincts. Soit
a_1,\\ldots,a_n~
\in K. Alors il existe un unique polynôme P \in K[X] tel que
deg P \leq n - 1 et \\forall~~i
\in [1,n], P(x_i) = a_i.

Démonstration Soit \phi_i : K_n-1[X] \rightarrow~ K,
P\mapsto~P(x_i) (où K_n-1[X] =
\P \in
K[X]∣deg~ P \leq n
- 1\). Les \phi_i sont des formes linéaires sur
l'espace vectoriel K_n-1[X] de dimension n~; soit v :
K_n-1[X] \rightarrow~ K^n,
P\mapsto~(\phi_1(P),\\ldots,\phi_n~(P))
=
(P(x_1),\\ldots,P(x_n~)).
Alors v est une application linéaire injective car

\begin{align*} v(P) = 0&
\Leftrightarrow & \forall~~i \in
[1,n], P(x_i) = 0 \%& \\ &
\Leftrightarrow & \∏
_i=1^n(X - x_
i)∣P(X) \mathrel\Leftrightarrow P =
0\%& \\ \end{align*}

pour des raisons de degré évidentes. On en déduit que v est un
isomorphisme d'espaces vectoriels, ce qui démontre le résultat.

Remarque~2.4.7 Comme v est surjective, la famille
(\phi_1,\\ldots,\phi_n~)
est libre~; comme son cardinal est n, c'est une base du dual
K_n-1[X]^∗. Cherchons la base dont c'est la
duale, c'est-à-dire des polynômes P_i vérifiant
P_i(x_j) = \delta_i^j~; un tel polynôme
doit être divisible par
\∏ ~
_j\neq~i(X - x_j). Pour des
raisons de degrés, il doit lui être proportionnel et le fait que
P_i(x_i) = 1 nécessite

P_i(X) = \∏
_j\neq~i(X - x_j)
\over \∏
_j\neq~i(x_i - x_j)

On a alors

\forall~P \in K_n-1~[X], P =
\sum _i=1^n\phi_
i(P)P_i = \\sum
_i=1^nP(x_ i) \∏
_j\neq~i(X - x_j)
\over \∏
_j\neq~i(x_i - x_j)

[
[
[
[

\end{document}

% \section{Dualité : approche générale}

\begin{rem}
Cette section ne figure pas au programme des classes préparatoires. Elle reprend les définitions et les résultats de la section précédente en les généralisant.
\end{rem}

\subsection{Notion de dual. Orthogonalité}

\begin{de}
\index{dualité!forme linéaire}
\index{dual}
Soit $E$ un $K$-espace vectoriel. On appelle forme linéaire sur $E$ toute application linéaire de $E$ dans $K$. On appelle dual de $E$ le $K$-espace vectoriel $E^* = L(E,K)$.
\end{de}

\begin{rem}
\index{forme bilinéaire!canonique}
\index{orthogonalité}
On dispose d'une application bilinéaire de $E^*\times E$ dans $K$ donnée par $\langle f|x\rangle = f(x)$ appelée la forme bilinéaire canonique. À cette forme bilinéaire est associée une notion d'orthogonalité. On notera donc :
\begin{enumerate}
\item si $A \subset E$, $A^\perp = \{f \in E^* | \forall x \in A, f(x) = 0\}$
\item si $B \subset E^*$, $B^o = \{x \in E | \forall f \in B, f(x) = 0\}$
\end{enumerate}
\end{rem}

\begin{prop}
\index{orthogonalité!propriétés}
Les notations $A,A_1,A_2$ désignant des parties de $E$ et $B,B_1,B_2$ désignant des parties de $E^*$, on a :
\begin{enumerate}
\item $A^\perp$ et $B^o$ sont des sous-espaces vectoriels de $E^*$ et $E$; $A^\perp = \operatorname{Vect}(A)^\perp$ et $B^o = \operatorname{Vect}(B)^o$
\item $A_1 \subset A_2 \Rightarrow A_1^\perp \supset A_2^\perp$ et $B_1 \subset B_2 \Rightarrow B_1^o \supset B_2^o$
\item $A \subset (A^\perp)^o$ et $B \subset (B^o)^\perp$
\item Soit $A$ un sous-espace vectoriel de $E$, alors $A^\perp = \{0\} \Leftrightarrow A = E$ et $A^\perp = E^* \Leftrightarrow A = \{0\}$
\item Soit $B$ un sous-espace vectoriel de $E^*$, alors $B^o = E \Leftrightarrow B = \{0\}$
\end{enumerate}
\end{prop}

\subsection{Hyperplans}

\begin{de}
\index{hyperplan!définition}
On appelle hyperplan de $E$ tout sous-espace vectoriel $H$ de $E$ vérifiant les conditions équivalentes :
\begin{enumerate}
\item $\dim E/H = 1$
\item $\exists f \in E^*\setminus\{0\}, H = \operatorname{Ker}(f)$
\item $H$ admet une droite comme supplémentaire
\end{enumerate}
\end{de}

\begin{thm}
\index{hyperplan!orthogonal}
Soit $H$ un hyperplan de $E$. Alors $H^\perp$ est de dimension 1 (droite vectorielle) : deux formes linéaires nulles sur $H$ sont proportionnelles.
\end{thm}

\subsection{Bidual}

\begin{de}
\index{bidual}
On désigne par $E^{**}$ le dual de $E^*$.
\end{de}

\begin{rem}
Si $E$ est de dimension finie, $E^*$ aussi et $\dim E^* = \dim E$. On en déduit que $E^{**}$ est aussi de dimension finie encore égale à $\dim E$.
\end{rem}

\begin{thm}
\index{bidual!isomorphisme canonique}
L'application $u : E \to E^{**}, x \mapsto u_x$ définie par $u_x(f) = f(x)$ est une application linéaire injective. Si $E$ est un espace vectoriel de dimension finie, c'est un isomorphisme d'espaces vectoriels.
\end{thm}

\subsection{Transposée}

\begin{de}
\index{transposée!application linéaire}
Soit $u \in L(E,F)$. On note ${}^tu : F^* \to E^*$ définie par ${}^tu(g) = g \circ u$ (c'est une application linéaire).
\end{de}

\begin{rem}
Cela revient à poser, pour $x \in E$ et $g \in F^*$, $\langle {}^tu(g)|x\rangle_E = \langle g|u(x)\rangle_F$.
\end{rem}

\begin{thm}
\index{transposée!propriétés}
On a les propriétés suivantes :
\begin{enumerate}
\item $u \mapsto {}^tu$ est linéaire de $L(E,F)$ dans $L(F^*,E^*)$
\item $u \in L(E,F), v \in L(F,G)$; alors ${}^t(v \circ u) = {}^tu \circ {}^tv$
\item Si $u$ est bijective, ${}^tu$ aussi et $({}^tu)^{-1} = {}^t(u^{-1})$
\item $\operatorname{Ker}({}^tu) = (\operatorname{Im}(u))^\perp$
\item $\operatorname{Im}({}^tu) = (\operatorname{Ker}(u))^\perp$
\end{enumerate}
\end{thm}

\subsection{Dualité en dimension finie}

\begin{prop}
\index{base!duale}
Soit $E$ un espace vectoriel de dimension finie, $\mathcal{E} = (e_1,\ldots,e_n)$ une base de $E$. La famille $\mathcal{E}' = (e_1^*,\ldots,e_n^*)$ de $E^*$ définie par $e_i^*(e_j) = \delta_i^j$ est une base de $E^*$ appelée la base duale de la base $\mathcal{E}$.
\end{prop}

\begin{thm}
\index{base!duale!bijection}
Soit $E$ un espace vectoriel de dimension finie. L'application $\mathcal{E} \mapsto \mathcal{E}'$ est une bijection de l'ensemble des bases de $E$ sur l'ensemble des bases de $E^*$.
\end{thm}

\begin{cor}
\index{dualité!dimension finie}
Soit $E$ un espace vectoriel de dimension finie.
\begin{enumerate}
\item Soit $A$ un sous-espace vectoriel de $E$. On a :
\[ \dim E = \dim A + \dim A^\perp \text{ et } (A^\perp)^o = A \]
\item Soit $B$ un sous-espace vectoriel de $E^*$. On a :
\[ \dim E = \dim B + \dim B^o \text{ et } (B^o)^\perp = B \]
\end{enumerate}
\end{cor}

\begin{cor}
\index{dualité!système d'équations}
Soit $E$ un espace vectoriel de dimension finie, $f_1,\ldots,f_k \in E^*$, $V = \{x \in E | f_1(x) = \cdots = f_k(x) = 0\}$. Alors
\[ \dim V = \dim E - \operatorname{rg}(f_1,\ldots,f_k) \]
\end{cor}

\begin{thm}
\index{transposée!rang}
Soit $E$ et $F$ deux espaces vectoriels de dimensions finies, $u \in L(E,F)$. Alors 
\[ \operatorname{rg}(u) = \operatorname{rg}({}^tu) \]
\end{thm}
% \documentclass[]{article}
\usepackage[T1]{fontenc}
\usepackage{lmodern}
\usepackage{amssymb,amsmath}
\usepackage{ifxetex,ifluatex}
\usepackage{fixltx2e} % provides \textsubscript
% use upquote if available, for straight quotes in verbatim environments
\IfFileExists{upquote.sty}{\usepackage{upquote}}{}
\ifnum 0\ifxetex 1\fi\ifluatex 1\fi=0 % if pdftex
  \usepackage[utf8]{inputenc}
\else % if luatex or xelatex
  \ifxetex
    \usepackage{mathspec}
    \usepackage{xltxtra,xunicode}
  \else
    \usepackage{fontspec}
  \fi
  \defaultfontfeatures{Mapping=tex-text,Scale=MatchLowercase}
  \newcommand{\euro}{€}
\fi
% use microtype if available
\IfFileExists{microtype.sty}{\usepackage{microtype}}{}
\usepackage{graphicx}
% Redefine \includegraphics so that, unless explicit options are
% given, the image width will not exceed the width of the page.
% Images get their normal width if they fit onto the page, but
% are scaled down if they would overflow the margins.
\makeatletter
\def\ScaleIfNeeded{%
  \ifdim\Gin@nat@width>\linewidth
    \linewidth
  \else
    \Gin@nat@width
  \fi
}
\makeatother
\let\Oldincludegraphics\includegraphics
{%
 \catcode`\@=11\relax%
 \gdef\includegraphics{\@ifnextchar[{\Oldincludegraphics}{\Oldincludegraphics[width=\ScaleIfNeeded]}}%
}%
\ifxetex
  \usepackage[setpagesize=false, % page size defined by xetex
              unicode=false, % unicode breaks when used with xetex
              xetex]{hyperref}
\else
  \usepackage[unicode=true]{hyperref}
\fi
\hypersetup{breaklinks=true,
            bookmarks=true,
            pdfauthor={},
            pdftitle={Matrices},
            colorlinks=true,
            citecolor=blue,
            urlcolor=blue,
            linkcolor=magenta,
            pdfborder={0 0 0}}
\urlstyle{same}  % don't use monospace font for urls
\setlength{\parindent}{0pt}
\setlength{\parskip}{6pt plus 2pt minus 1pt}
\setlength{\emergencystretch}{3em}  % prevent overfull lines
\setcounter{secnumdepth}{0}
 
/* start css.sty */
.cmr-5{font-size:50%;}
.cmr-7{font-size:70%;}
.cmmi-5{font-size:50%;font-style: italic;}
.cmmi-7{font-size:70%;font-style: italic;}
.cmmi-10{font-style: italic;}
.cmsy-5{font-size:50%;}
.cmsy-7{font-size:70%;}
.cmex-7{font-size:70%;}
.cmex-7x-x-71{font-size:49%;}
.msbm-7{font-size:70%;}
.cmtt-10{font-family: monospace;}
.cmti-10{ font-style: italic;}
.cmbx-10{ font-weight: bold;}
.cmr-17x-x-120{font-size:204%;}
.cmsl-10{font-style: oblique;}
.cmti-7x-x-71{font-size:49%; font-style: italic;}
.cmbxti-10{ font-weight: bold; font-style: italic;}
p.noindent { text-indent: 0em }
td p.noindent { text-indent: 0em; margin-top:0em; }
p.nopar { text-indent: 0em; }
p.indent{ text-indent: 1.5em }
@media print {div.crosslinks {visibility:hidden;}}
a img { border-top: 0; border-left: 0; border-right: 0; }
center { margin-top:1em; margin-bottom:1em; }
td center { margin-top:0em; margin-bottom:0em; }
.Canvas { position:relative; }
li p.indent { text-indent: 0em }
.enumerate1 {list-style-type:decimal;}
.enumerate2 {list-style-type:lower-alpha;}
.enumerate3 {list-style-type:lower-roman;}
.enumerate4 {list-style-type:upper-alpha;}
div.newtheorem { margin-bottom: 2em; margin-top: 2em;}
.obeylines-h,.obeylines-v {white-space: nowrap; }
div.obeylines-v p { margin-top:0; margin-bottom:0; }
.overline{ text-decoration:overline; }
.overline img{ border-top: 1px solid black; }
td.displaylines {text-align:center; white-space:nowrap;}
.centerline {text-align:center;}
.rightline {text-align:right;}
div.verbatim {font-family: monospace; white-space: nowrap; text-align:left; clear:both; }
.fbox {padding-left:3.0pt; padding-right:3.0pt; text-indent:0pt; border:solid black 0.4pt; }
div.fbox {display:table}
div.center div.fbox {text-align:center; clear:both; padding-left:3.0pt; padding-right:3.0pt; text-indent:0pt; border:solid black 0.4pt; }
div.minipage{width:100%;}
div.center, div.center div.center {text-align: center; margin-left:1em; margin-right:1em;}
div.center div {text-align: left;}
div.flushright, div.flushright div.flushright {text-align: right;}
div.flushright div {text-align: left;}
div.flushleft {text-align: left;}
.underline{ text-decoration:underline; }
.underline img{ border-bottom: 1px solid black; margin-bottom:1pt; }
.framebox-c, .framebox-l, .framebox-r { padding-left:3.0pt; padding-right:3.0pt; text-indent:0pt; border:solid black 0.4pt; }
.framebox-c {text-align:center;}
.framebox-l {text-align:left;}
.framebox-r {text-align:right;}
span.thank-mark{ vertical-align: super }
span.footnote-mark sup.textsuperscript, span.footnote-mark a sup.textsuperscript{ font-size:80%; }
div.tabular, div.center div.tabular {text-align: center; margin-top:0.5em; margin-bottom:0.5em; }
table.tabular td p{margin-top:0em;}
table.tabular {margin-left: auto; margin-right: auto;}
div.td00{ margin-left:0pt; margin-right:0pt; }
div.td01{ margin-left:0pt; margin-right:5pt; }
div.td10{ margin-left:5pt; margin-right:0pt; }
div.td11{ margin-left:5pt; margin-right:5pt; }
table[rules] {border-left:solid black 0.4pt; border-right:solid black 0.4pt; }
td.td00{ padding-left:0pt; padding-right:0pt; }
td.td01{ padding-left:0pt; padding-right:5pt; }
td.td10{ padding-left:5pt; padding-right:0pt; }
td.td11{ padding-left:5pt; padding-right:5pt; }
table[rules] {border-left:solid black 0.4pt; border-right:solid black 0.4pt; }
.hline hr, .cline hr{ height : 1px; margin:0px; }
.tabbing-right {text-align:right;}
span.TEX {letter-spacing: -0.125em; }
span.TEX span.E{ position:relative;top:0.5ex;left:-0.0417em;}
a span.TEX span.E {text-decoration: none; }
span.LATEX span.A{ position:relative; top:-0.5ex; left:-0.4em; font-size:85%;}
span.LATEX span.TEX{ position:relative; left: -0.4em; }
div.float img, div.float .caption {text-align:center;}
div.figure img, div.figure .caption {text-align:center;}
.marginpar {width:20%; float:right; text-align:left; margin-left:auto; margin-top:0.5em; font-size:85%; text-decoration:underline;}
.marginpar p{margin-top:0.4em; margin-bottom:0.4em;}
.equation td{text-align:center; vertical-align:middle; }
td.eq-no{ width:5%; }
table.equation { width:100%; } 
div.math-display, div.par-math-display{text-align:center;}
math .texttt { font-family: monospace; }
math .textit { font-style: italic; }
math .textsl { font-style: oblique; }
math .textsf { font-family: sans-serif; }
math .textbf { font-weight: bold; }
.partToc a, .partToc, .likepartToc a, .likepartToc {line-height: 200%; font-weight:bold; font-size:110%;}
.chapterToc a, .chapterToc, .likechapterToc a, .likechapterToc, .appendixToc a, .appendixToc {line-height: 200%; font-weight:bold;}
.index-item, .index-subitem, .index-subsubitem {display:block}
.caption td.id{font-weight: bold; white-space: nowrap; }
table.caption {text-align:center;}
h1.partHead{text-align: center}
p.bibitem { text-indent: -2em; margin-left: 2em; margin-top:0.6em; margin-bottom:0.6em; }
p.bibitem-p { text-indent: 0em; margin-left: 2em; margin-top:0.6em; margin-bottom:0.6em; }
.paragraphHead, .likeparagraphHead { margin-top:2em; font-weight: bold;}
.subparagraphHead, .likesubparagraphHead { font-weight: bold;}
.quote {margin-bottom:0.25em; margin-top:0.25em; margin-left:1em; margin-right:1em; text-align:\jmathustify;}
.verse{white-space:nowrap; margin-left:2em}
div.maketitle {text-align:center;}
h2.titleHead{text-align:center;}
div.maketitle{ margin-bottom: 2em; }
div.author, div.date {text-align:center;}
div.thanks{text-align:left; margin-left:10%; font-size:85%; font-style:italic; }
div.author{white-space: nowrap;}
.quotation {margin-bottom:0.25em; margin-top:0.25em; margin-left:1em; }
h1.partHead{text-align: center}
.sectionToc, .likesectionToc {margin-left:2em;}
.subsectionToc, .likesubsectionToc {margin-left:4em;}
.subsubsectionToc, .likesubsubsectionToc {margin-left:6em;}
.frenchb-nbsp{font-size:75%;}
.frenchb-thinspace{font-size:75%;}
.figure img.graphics {margin-left:10%;}
/* end css.sty */

\title{Matrices}
\author{}
\date{}

\begin{document}
\maketitle

\textbf{Warning: 
requires JavaScript to process the mathematics on this page.\\ If your
browser supports JavaScript, be sure it is enabled.}

\begin{center}\rule{3in}{0.4pt}\end{center}

{[}
{[}
{[}{]}
{[}

\subsubsection{2.6 Matrices}

\paragraph{2.6.1 Généralités}

Définition~2.6.1 M\_K(m,n) =
\(a\_i,\jmath)\_ 1\leqi\leqm \atop
1\leq\jmath\leqn \ est un K-espace vectoriel de dimension mn.
Il admet pour base la famille (E\_k,l)\_ 1\leqk\leqm
\atop 1\leql\leqn  avec

 E\_k,l = (\delta\_i,\jmath^k,l)\_ 1\leqi\leqm
\atop 1\leq\jmath\leqn  =\left (
\includegraphics{cours0x.png} \,\right )

Définition~2.6.2 Soit E et F deux espaces vectoriels de dimensions
finies n et m respectivement et u \in L(E,F). Soit \mathcal{E} =
(e\_1,\\ldots,e\_n~)
une base de E et ℱ =
(f\_1,\\ldots,f\_m~)
une base de F de base duale ℱ^∗ =
(f\_1^∗,\\ldots,f\_m^∗~).
On définit la matrice de u dans les bases \mathcal{E} et ℱ comme étant la matrice
\mathrmMat~ (u,\mathcal{E},ℱ) =
(a\_i,\jmath)\_ 1\leqi\leqm \atop 1\leq\jmath\leqn 
construite de fa\ccon équivalente par (i)
\forall~\jmath \in {[}1,n{]}, u(e\_\jmath~)
= \\sum ~
\_i=1^ma\_i,\jmathf\_i (ii)
\forall~i \in {[}1,m{]}, \\forall~~\jmath \in
{[}1,n{]}, a\_i,\jmath = f\_i^∗(u(e\_\jmath))
=\langle
f\_i^∗∣u(e\_\jmath)\rangle

Proposition~2.6.1 L'application L(E,F) \rightarrow~ M\_K(m,n),
u\mapsto~\mathrmMat~
(u,\mathcal{E},ℱ) est un isomorphisme de K-espaces vectoriels .

Produit de matrices A = (a\_i,\jmath) \in M\_K(m,n) et B =
(b\_i,\jmath) \in M\_K(n,p). On définit AB = (c\_i,\jmath) \in
M\_K(m,p) par

\forall~(i,\jmath) \in {[}1,m{]} \times {[}1,p{]}, c\_i,\jmath~
= \sum \_k=1^na~\_
i,kb\_k,\jmath

Théorème~2.6.2 Soit u \in L(E,F), v \in L(F,G) où E, F et G sont trois
espaces vectoriels de dimensions finies admettant des bases \mathcal{E}, ℱ et G.
Alors on a

\mathrmMat~ (v \cdot u,\mathcal{E},G)
= \mathrmMat~
(v,ℱ,G)\mathrmMat~ (u,\mathcal{E},ℱ)

Démonstration En effet, si les dimensions des espaces sont
respectivement p, n et m, et si l'on note A =\
\mathrmMat (u,\mathcal{E},ℱ) et B =\
\mathrmMat (v,ℱ,G), on a

\begin{align*} v \cdot u(e\_\jmath)& =&
v(u(e\_\jmath) = v(\\sum
\_k=1^na\_ k,\jmathf\_k)\%&
\\ & =& \\sum
\_k=1^na\_ k,\jmathv(f\_k) \%&
\\ & =& \\sum
\_k=1^na\_ k,\jmath \\sum
\_i=1^mb\_ i,kg\_i \%&
\\ & =& \\sum
\_i=1^m\left (\\sum
\_k=1^nb\_ i,ka\_k,\jmath\right
)g\_i\%& \\
\end{align*}

ce qui montre que la i-ième coordonnée de v \cdot u(e\_\jmath) est bien
le terme d'indice i,\jmath de la matrice BA.

Remarque~2.6.1 Ceci lié à l'isomorphisme avec les applications linéaires
permet d'obtenir immédiatement les propriétés essentielles du produit
des matrices~: associativité et bilinéarité.

Définition~2.6.3 Soit x \in E et y \in F. Si x =\
\sum ~
\_i=1^nx\_ie\_i, on définit X =
\left
(\matrix\,x\_1
\cr \⋮~
\cr x\_n\right ) (vecteur
colonne des coordonnées de x dans la base \mathcal{E}). De même, on définit Y
vecteur colonne des coordonnées de y dans la base ℱ. On a alors

Théorème~2.6.3 \mathrmMat~
(u,\mathcal{E},ℱ) est l'unique matrice A \in M\_K(m,n) vérifiant

\forall~~(x,y) \in E \times F,\quad u(x) = y
\Leftrightarrow Y = AX

Démonstration En effet

\begin{align*} u(x) = y&
\Leftrightarrow & \\sum
\_\jmath=1^nx\_ \jmathu(e\_\jmath) =
\sum \_i=1^my~\_
if\_i \%& \\ &
\Leftrightarrow & \\sum
\_\jmath=1^nx\_ \jmath \\sum
\_i=1^ma\_ i,\jmathf\_i =
\sum \_i=1^my~\_
if\_i\%& \\ &
\Leftrightarrow & \forall~~i,
y\_i = \\sum
\_\jmath=1^na\_ i,\jmathx\_\jmath
\Leftrightarrow Y = AX \%&
\\ \end{align*}

Inversement, si une matrice A vérifie cette condition, en prenant x =
e\_\jmath, on voit que a\_i,\jmath est la i-ième coordonnée de
u(e\_\jmath), et donc A =\
\mathrmMat (u,\mathcal{E},ℱ).

\paragraph{2.6.2 Matrices carrées}

Lemme~2.6.4 Soit (E\_i,\jmath) la base canonique de M\_K(n).
On a E\_i,\jmathE\_k,l = \delta\_\jmath^kE\_i,l

Démonstration Calcul élémentaire

Proposition~2.6.5 M\_K(n) (= M\_K(n,n)) est une
K-algèbre de dimension n^2 dont le centre est constitué des
matrices scalaires. Le groupe de ses éléments inversibles est noté
GL\_K(n) (groupe linéaire d'indice n).

Démonstration Tout est élémentaire, sauf la recherche du centre. Soit A
\in M\_k(n) une matrice qui commute à toutes les autres matrices
carrées. On a A =\ \\sum
 \_i,\jmatha\_i,\jmathE\_i,\jmath. On écrit pour
k\neq~l, E\_k,lA = AE\_k,l soit
encore \\sum ~
\_\jmatha\_l,\jmathE\_k,\jmath =\
\sum  \_ia\_i,kE\_i,l~. En
prenant la coordonnée suivant E\_k,k, on a a\_l,k = 0.
En prenant la coordonnée suivant E\_k,l on a a\_l,l =
a\_k,k soit A = a\_1,1I\_n.

Remarque~2.6.2 Bien entendu si E est un K-espace vectoriel de dimension
n admettant une base \mathcal{E}, les résultats précédents impliquent que
l'application L(E) \rightarrow~ M\_K(n),
u\mapsto~\mathrmMat~
(u,\mathcal{E}) est un isomorphisme de K-algèbres.

Définition~2.6.4 Si A = (a\_i,\jmath) \in M\_K(n), on définit
la trace de A comme
\mathrm{tr}~A
= \\sum ~
\_i=1^na\_i,i.

Théorème~2.6.6 Soit A \in M\_K(m,n) et B \in M\_K(n,m).
Alors \mathrm{tr}~(AB)
= \mathrm{tr}~(BA). En
particulier, si A \in M\_K(n) et P \in GL\_K(n), alors
\mathrm{tr}(P^-1~AP)
= \mathrm{tr}~A.

Démonstration On a en effet
\mathrm{tr}~(AB)
= \\sum ~
\_i,\jmatha\_i,\jmathb\_\jmath,i qui est une expression
visiblement symétrique en a et b.

\paragraph{2.6.3 Transposée}

Définition~2.6.5 Si A = (a\_i,\jmath) \in M\_K(m,n) on pose
^tA = (b\_i,\jmath) \in M\_K(n,m) définie par
\forall~~(i,\jmath) \in {[}1,n{]} \times
{[}1,m{]},\quad b\_i,\jmath = a\_\jmath,i.

Proposition~2.6.7 L'application
M\mapsto~^tM est linéaire bi\jmathective de
M\_K(m,n) sur M\_K(n,m) et on a
^t(^tM) = M. Si M \in M\_K(m,n),N \in
M\_K(n,p), alors ^t(MN) =
^tN^tM. Si M \in M\_K(n) est inversible,
^tM aussi et (^tM)^-1 =
^t(M^-1).

Démonstration Calcul élémentaire pour montrer que ^t(\alpha~M + \beta~N)
= \alpha~^tM + \beta~^tN et que ^t(MN) =
^tN^tM. Si M est inversible, on a MM^-1 =
M^-1M = I\_n et prenant la transposée,
^t(M^-1)^tM =
^tM^t(M^-1) = ^tI\_n =
I\_n. On en déduit que ^tM est inversible et que
(^tM)^-1 = ^t(M^-1).

Théorème~2.6.8 Soit E et F deux K-espaces vectoriels de dimensions
finies admettant des bases \mathcal{E} et ℱ. Alors

\mathrmMat~
(^tu,ℱ^∗,\mathcal{E}^∗) =
^t \mathrmMat~
(u,\mathcal{E},ℱ)

Démonstration Soit A et B les deux matrices. On a
^tu(f\_\jmath^∗) =\
\sum ~
\_ib\_i,\jmathe\_i^∗ soit b\_i,\jmath =
^tu(f\_\jmath^∗)(e\_i) =
f\_\jmath^∗\cdot u(e\_i) =
f\_\jmath^∗(u(e\_i)) = a\_\jmath,i.

Définition~2.6.6 Soit M \in M\_K(n). On dit que M est symétrique
si ^tM = M et antisymétrique si ^tM = -M.

Proposition~2.6.9 L'ensemble S\_K(n) (resp. A\_K(n)) des
matrices symétriques (resp. antisymétriques) est un sous-espace
vectoriel de M\_K(n). On a dim~
S\_K(n) = n(n+1) \over 2 . Si
carK\mathrel\neq~~2, on a
M\_K(n) = S\_K(n) \oplus~ A\_K(n) et
dim A\_K~(n) = n(n-1)
\over 2 .

Démonstration Une base de S\_K(n) est clairement constituée des
E\_i,i 1 \leq i \leq n et des E\_i,\jmath + E\_\jmath,i, 1 \leq i
\textless{} \jmath \leq n, donc dim S\_K~(n)
= n(n+1) \over 2 . Si
carK\mathrel\neq~~2, on peut
écrire A = 1\over 2(A + ^tA) +
1\over 2(A -^tA), ce qui montre que
M\_K(n) = S\_K(n) + A\_K(n) et on a clairement
S\_K(n) \bigcap A\_K(n) =
\0\.

Remarque~2.6.3 Par contre, si car~K = 2, on a 1
= -1 et donc S\_K(n) = A\_K(n).

\paragraph{2.6.4 Rang d'une matrice}

Définition~2.6.7 Soit A \in M\_K(m,n). On appelle rang de A le
rang dans K^m de la famille
(c\_1,\\ldots,c\_n~)
de ses vecteurs colonnes.

Théorème~2.6.10 Si u \in L(E,F), \mathcal{E} une base de E, ℱ une base de F, A
= \mathrmMat~ (u,\mathcal{E},ℱ). Alors
\mathrmrg~A
= \mathrmrg~u.

Démonstration On a
\mathrmrg~u
=\
\mathrmrg(u(e\_1),\\ldots,u(e\_n~)).
Mais le rang de la famille des u(e\_\jmath) est aussi le rang de son
image par l'isomorphisme de F dans K^m qui à un vecteur
associe la famille de ses coordonnées dans la base ℱ, c'est-à-dire le
rang de la famille des vecteurs colonnes de la matrice A.

Corollaire~2.6.11
\mathrmrg~(AB)
\leq\
min(\mathrmrgA,\\mathrmrg~B).

Théorème~2.6.12 Soit A \in M\_K(n). On a équivalence de

\begin{itemize}
\itemsep1pt\parskip0pt\parsep0pt
\item
  (i) A est inversible
\item
  (ii) \mathrmrg~A = n
\item
  (iii) \existsB \in M\_K~(n), AB =
  I\_n
\item
  (iv) \existsB \in M\_K(n), BA = I\_n~
\end{itemize}

Démonstration Se déduit immédiatement du théorème analogue sur les
endomorphismes d'un espace vectoriel de dimension finie.

\paragraph{2.6.5 La méthode du pivot}

La recherche pratique du rang d'une matrice peut se faire par la méthode
du pivot~; dans les algorithmes qui suivent, la flèche vers la gauche
décrira un remplacement~: x ← y signifiera remplacer x par y. Il s'agit
là de l'analogue d'une affectation dans un langage de programmation.

Définition~2.6.8 On définit les opérations élémentaires sur les vecteurs
colonnes
c\_1,\\ldots,c\_n~
d'une matrice A \in M\_K(m,n)~:

\begin{itemize}
\item
  (i) a\jmathouter à une colonne c\_\jmath une combinaison linéaire des
  autres vecteurs colonnes

  c\_i ← c\_i + \\sum
  \_\jmath\neq~i\lambda~\_\jmathc\_\jmath

  ou encore

  A ← A\left
  (\matrix\,1& &\lambda~\_1& &
  \cr
  &⋱&\⋮~
  & & \cr & &1 & & \cr &
  &\⋮~
  &⋱& \cr &
  &\lambda~\_n& &1\right )
\item
  (ii) multiplier la colonne c\_i par un scalaire non nul

  c\_i ← \lambda~c\_i

  ou encore

  A ← A\left
  (\matrix\,1& & & &
  \cr &⋱& & &
  \cr & &\lambda~& & \cr & &
  &⋱& \cr & & &
  &1\right )
\item
  (iii) effectuer une permutation \sigma sur les vecteurs colonnes

  (c\_1,\\ldots,c\_n~)
  ←
  (c\_\sigma(1),\\ldots,c\_\sigma(n)~)

  ou encore

  A ← AP\_\sigma

  où P\_\sigma = (\delta\_i^\sigma(\jmath)).
\end{itemize}

Proposition~2.6.13 Les opérations élémentaires ne changent pas le rang
d'une matrice.

Démonstration En effet, il est clair que les opérations élémentaires ne
changent pas le sous-espace de K^n,
\mathrmVect(c\_1,\\\ldots,c\_n~)~;
on peut remarquer aussi que les opérations élémentaires s'effectuent par
multiplications à droite par des matrices inversibles, donc ne changent
pas le rang.

Théorème~2.6.14 Soit M \in M\_K(n,p). Alors il existe une matrice
M' qui se déduit de M par une suite d'opérations élémentaires, de la
forme

M' = \left (\matrix\,0
&0 &\\ldots~&0
&0&\\ldots~&0
\cr \⋮~
&\⋮~ &
&\⋮~
&\⋮~&
&\⋮~
\cr 0
&\⋮~ &
&\⋮~
&\⋮~&
&\⋮~
\cr 1 &0
&\\ldots~&0
&0&\\ldots~&0
\cr a\_\sigma(1)+1,1& &
&\⋮~
&\⋮~&
&\⋮~
\cr \⋮~
&\⋮~ &
&\⋮~
&\⋮~&
&\⋮~
\cr \⋮~
&1 & & & & & \cr
\⋮~
&a\_\sigma(2)+1,2 \cr
\⋮~
&\⋮~ &
&\⋮~
\cr & & &1 \cr & &
&a\_\sigma(r)+1,r&\⋮~&
&\⋮~
\cr \⋮~
&\⋮~ &
&\⋮~ &0&
&0\right )

où \sigma est une application strictement croissante de {[}1,r{]} dans
{[}1,n{]}. Dans toute telle écriture, on a
\mathrmrg~M = r.

Démonstration Par récurrence sur le nombre de colonnes de M. C'est
évident s'il existe une seule colonne ou si M = 0. Sinon, soit \sigma(1)
l'indice de la première ligne non nulle et \jmath tel que
a\_\sigma(1),\jmath\neq~0. On effectue une
permutation des colonnes pour amener la colonne c\_\jmath en première
colonne, puis on effectue les opérations c\_1 ← 1
\over a\_\sigma(1),1c\_1 puis pour \jmath allant
de 2 à n, c\_\jmath ← c\_\jmath - a\_\sigma(1),\jmathc\_1.
On aboutit alors à une matrice

\left (\matrix\,0
&0&\\ldots&0&0&\\\ldots~&0
\cr \⋮~
&\⋮~&
&\⋮&\\⋮~&
&\⋮~
\cr 0
&\⋮~&
&\⋮&\\⋮~&
&\⋮~
\cr 1
&0&\\ldots&0&0&\\\ldots~&0
\cr a\_\sigma(1)+1,1& &
&\⋮&\\⋮~&
&\⋮~
\cr \⋮~
&\⋮~&
&\⋮&\\⋮~&
&\⋮~\right
)

Il suffit alors d'utiliser l'hypothèse de récurrence sur les n - 1
dernières colonnes de la matrice. Il est clair que
\mathrmrg~M' = r, mais on a
\mathrmrg~M
= \mathrmrg~M', d'où
\mathrmrg~M = r.

Remarque~2.6.4 On peut ensuite utiliser les 1 qui sont dans les r
premières colonnes pour éliminer au fur et à mesure en partant du bas
tous les a\_\sigma(i),\jmath. Si M est inversible, nécessairement \sigma =
\mathrmId et alors la matrice obtenue est
I\_n. Mais on a M' =
MP\_1\\ldotsP\_k~
où les P\_i sont les matrices des différentes opérations
élémentaires effectuées. On en déduit que M^-1 =
P\_1\\ldotsP\_k~.
On peut calculer ce produit en partant de B ← I\_n et en
effectuant sur la matrice B les mêmes opérations élémentaires que sur la
matrice A. On a donc à la fin B ←
P\_1\\ldotsP\_k~,
soit B ← M^-1.

\paragraph{2.6.6 Changement de bases}

Proposition~2.6.15 Soit \mathcal{E} =
(e\_1,\\ldots,e\_n~)
une base de E et A = (a\_i,\jmath) \in M\_K(n). Posons
x\_\jmath = \\sum ~
\_i=1^na\_i,\jmathe\_i. Alors A est inversible
si et seulement si
(x\_1,\\ldots,x\_n~)
est une base de E.

Démonstration Comme précédemment, le rang de la famille des x\_\jmath
est aussi le rang de son image par l'isomorphisme de E sur
K^n qui à un vecteur associe la famille de ses coordonnées
dans la base \mathcal{E}, c'est-à-dire ici de la famille des vecteurs colonnes de
A.

Définition~2.6.9 Soit \mathcal{E} et \mathcal{E}' deux bases de E. On définit
P\_\mathcal{E}^\mathcal{E}' comme étant la matrice inversible
(a\_i,\jmath) \in M\_k(n) définie par e\_\jmath'
= \\sum ~
\_i=1^na\_i,\jmathe\_i.

Remarque~2.6.5 On a donc P\_\mathcal{E}^\mathcal{E}'
= \mathrmMat~
(\mathrmId\_E,\mathcal{E}',\mathcal{E}) =\
\mathrmMat (u,\mathcal{E},\mathcal{E}) où u est défini par
u(e\_i) = e\_i'. De la première égalité on déduit
immédiatement~:

Théorème~2.6.16

\begin{itemize}
\itemsep1pt\parskip0pt\parsep0pt
\item
  (i) P\_\mathcal{E}'^\mathcal{E} = (P\_\mathcal{E}^\mathcal{E}')^-1 et
  P\_\mathcal{E}^\mathcal{E}'` =
  P\_\mathcal{E}^\mathcal{E}'P\_\mathcal{E}'^\mathcal{E}''
\item
  (ii) si X et X' sont les vecteurs colonnes des coordonnées de x \in E
  dans les bases \mathcal{E} et \mathcal{E}' et P = P\_\mathcal{E}^\mathcal{E}', on a X = PX'.
\end{itemize}

Théorème~2.6.17 Soit u \in L(E,F), \mathcal{E} et \mathcal{E}' deux bases de E, ℱ et ℱ' deux
bases de F. On pose P = P\_\mathcal{E}^\mathcal{E}', Q =
P\_ℱ^ℱ', M =\
\mathrmMat (u,\mathcal{E},ℱ), M' =\
\mathrmMat (u,\mathcal{E}',ℱ'). Alors on a M' =
Q^-1MP.

Démonstration y = u(x) \Leftrightarrow Y = MX
\Leftrightarrow QY `= MPX' \mathrel\Leftrightarrow Y
`= Q^-1MPX'. L'unicité de la matrice d'une application
linéaire permet de conclure que M' = Q^-1MP.

Définition~2.6.10 On dit que M,M' \in M\_K(m,n) sont équivalentes
s'il existe Q \in GL\_K(m) et P \in GL\_K(n) telles que M' =
Q^-1MP.

Remarque~2.6.6 Ceci revient à dire que M et M' sont les matrices d'une
même application linéaire dans des bases ''différentes''~; sous cette
forme, il est clair qu'il s'agit d'une relation d'équivalence.

Lemme~2.6.18 Si M est de rang r, elle est équivalente à J\_r =
\left
(\matrix\,I\_r&0
\cr 0 &0\right )

Démonstration Soit M =\
\mathrmMat (u,\mathcal{E},ℱ). Soit V un supplémentaire de
\mathrmKer~u dans E,
(e\_1',\\ldots,e\_r~')
une base de V ,
(e\_r+1',\\ldots,e\_n~')
une base de \mathrmKer~u.
Comme u\textbar{}\_V  est un isomorphisme de V sur
\mathrmIm~u,
(f\_1',\\ldots,f\_r~')
=
(u(e\_1'),\\ldots,u(e\_r~'))
est une base de \mathrmIm~u
que l'on peut compléter en
(f\_1',\\ldots,f\_m~')
base de F. On a alors
\mathrmMat~ (u,\mathcal{E}',ℱ') =
J\_r, d'où le résultat.

Théorème~2.6.19 Deux matrices de M\_K(m,n) sont équivalentes si
et seulement si elles ont même rang.

Démonstration La condition est bien évidemment nécessaire, et le lemme
précédent montre qu'elle est suffisante.

Théorème~2.6.20 \mathrmrg~A
= \mathrmrg^t~A.
Autrement dit, le rang d'une matrice est aussi égal au rang de la
famille de ses vecteurs lignes.

Démonstration En effet si A est équivalente à J\_r,
^tA est équivalente à ^tJ\_r qui est aussi
de rang r.

Remarque~2.6.7 Dans le cas d'un endomorphisme, on a en général \mathcal{E} = ℱ et
\mathcal{E}' = ℱ' d'où le théorème

Théorème~2.6.21 Soit u \in L(E), \mathcal{E} et \mathcal{E}' deux bases de E. On pose P =
P\_\mathcal{E}^\mathcal{E}', M =\
\mathrmMat (u,\mathcal{E}), M' =\
\mathrmMat (u,\mathcal{E}'). Alors on a M' =
P^-1MP.

Définition~2.6.11 On dit que M,M' \in M\_K(n) sont semblables s'il
existe P \in GL\_K(n) telles que M' = P^-1MP.

Remarque~2.6.8 Cela revient à dire que M et M' sont les matrices d'un
même endomorphisme dans des bases ''différentes''~; sous cette forme, il
est clair qu'il s'agit d'une relation d'équivalence.

Remarque~2.6.9 On remarque que deux matrices semblables ont même trace
ce qui permet de définir

Définition~2.6.12 Soit u \in L(E). On pose
\mathrm{tr}~u
=\
\mathrm{tr}\mathrmMat~
(u,\mathcal{E}), indépendant du choix de la base \mathcal{E} de E.

\paragraph{2.6.7 Produit des matrices par blocs}

Soit M \in M\_K(m,n),M' \in M\_K(n,p), 1 \leq q \leq m, 1 \leq r \leq n,
1 \leq s \leq p. On écrit

M =\left ( \includegraphics{cours1x.png}
\,\right ),\quad M'
=\left (

\includegraphics{cours2x.png} \,\right )

avec A \in M\_K(q,r),B \in M\_K(q,n - r),C \in M\_K(m
- q,r),D \in M\_K(m - q,n - r),A' \in M\_K(r,s),B' \in
M\_K(r,p - s),C' \in M\_K(n - r,s),D' \in M\_K(n -
r,p - s) Alors

MM' = \left (\matrix\,AA'
+ BC'&AB' + BD' \cr CA' + DC'&CB' +
DD'\right )

Démonstration Par le calcul (attention aux décalages d'indices).

{[}
{[}
{[}
{[}

\end{document}

% \documentclass[]{article}
\usepackage[T1]{fontenc}
\usepackage{lmodern}
\usepackage{amssymb,amsmath}
\usepackage{ifxetex,ifluatex}
\usepackage{fixltx2e} % provides \textsubscript
% use upquote if available, for straight quotes in verbatim environments
\IfFileExists{upquote.sty}{\usepackage{upquote}}{}
\ifnum 0\ifxetex 1\fi\ifluatex 1\fi=0 % if pdftex
  \usepackage[utf8]{inputenc}
\else % if luatex or xelatex
  \ifxetex
    \usepackage{mathspec}
    \usepackage{xltxtra,xunicode}
  \else
    \usepackage{fontspec}
  \fi
  \defaultfontfeatures{Mapping=tex-text,Scale=MatchLowercase}
  \newcommand{\euro}{€}
\fi
% use microtype if available
\IfFileExists{microtype.sty}{\usepackage{microtype}}{}
\usepackage{graphicx}
% Redefine \includegraphics so that, unless explicit options are
% given, the image width will not exceed the width of the page.
% Images get their normal width if they fit onto the page, but
% are scaled down if they would overflow the margins.
\makeatletter
\def\ScaleIfNeeded{%
  \ifdim\Gin@nat@width>\linewidth
    \linewidth
  \else
    \Gin@nat@width
  \fi
}
\makeatother
\let\Oldincludegraphics\includegraphics
{%
 \catcode`\@=11\relax%
 \gdef\includegraphics{\@ifnextchar[{\Oldincludegraphics}{\Oldincludegraphics[width=\ScaleIfNeeded]}}%
}%
\ifxetex
  \usepackage[setpagesize=false, % page size defined by xetex
              unicode=false, % unicode breaks when used with xetex
              xetex]{hyperref}
\else
  \usepackage[unicode=true]{hyperref}
\fi
\hypersetup{breaklinks=true,
            bookmarks=true,
            pdfauthor={},
            pdftitle={Determinants},
            colorlinks=true,
            citecolor=blue,
            urlcolor=blue,
            linkcolor=magenta,
            pdfborder={0 0 0}}
\urlstyle{same}  % don't use monospace font for urls
\setlength{\parindent}{0pt}
\setlength{\parskip}{6pt plus 2pt minus 1pt}
\setlength{\emergencystretch}{3em}  % prevent overfull lines
\setcounter{secnumdepth}{0}
 
/* start css.sty */
.cmr-5{font-size:50%;}
.cmr-7{font-size:70%;}
.cmmi-5{font-size:50%;font-style: italic;}
.cmmi-7{font-size:70%;font-style: italic;}
.cmmi-10{font-style: italic;}
.cmsy-5{font-size:50%;}
.cmsy-7{font-size:70%;}
.cmex-7{font-size:70%;}
.cmex-7x-x-71{font-size:49%;}
.msbm-7{font-size:70%;}
.cmtt-10{font-family: monospace;}
.cmti-10{ font-style: italic;}
.cmbx-10{ font-weight: bold;}
.cmr-17x-x-120{font-size:204%;}
.cmsl-10{font-style: oblique;}
.cmti-7x-x-71{font-size:49%; font-style: italic;}
.cmbxti-10{ font-weight: bold; font-style: italic;}
p.noindent { text-indent: 0em }
td p.noindent { text-indent: 0em; margin-top:0em; }
p.nopar { text-indent: 0em; }
p.indent{ text-indent: 1.5em }
@media print {div.crosslinks {visibility:hidden;}}
a img { border-top: 0; border-left: 0; border-right: 0; }
center { margin-top:1em; margin-bottom:1em; }
td center { margin-top:0em; margin-bottom:0em; }
.Canvas { position:relative; }
li p.indent { text-indent: 0em }
.enumerate1 {list-style-type:decimal;}
.enumerate2 {list-style-type:lower-alpha;}
.enumerate3 {list-style-type:lower-roman;}
.enumerate4 {list-style-type:upper-alpha;}
div.newtheorem { margin-bottom: 2em; margin-top: 2em;}
.obeylines-h,.obeylines-v {white-space: nowrap; }
div.obeylines-v p { margin-top:0; margin-bottom:0; }
.overline{ text-decoration:overline; }
.overline img{ border-top: 1px solid black; }
td.displaylines {text-align:center; white-space:nowrap;}
.centerline {text-align:center;}
.rightline {text-align:right;}
div.verbatim {font-family: monospace; white-space: nowrap; text-align:left; clear:both; }
.fbox {padding-left:3.0pt; padding-right:3.0pt; text-indent:0pt; border:solid black 0.4pt; }
div.fbox {display:table}
div.center div.fbox {text-align:center; clear:both; padding-left:3.0pt; padding-right:3.0pt; text-indent:0pt; border:solid black 0.4pt; }
div.minipage{width:100%;}
div.center, div.center div.center {text-align: center; margin-left:1em; margin-right:1em;}
div.center div {text-align: left;}
div.flushright, div.flushright div.flushright {text-align: right;}
div.flushright div {text-align: left;}
div.flushleft {text-align: left;}
.underline{ text-decoration:underline; }
.underline img{ border-bottom: 1px solid black; margin-bottom:1pt; }
.framebox-c, .framebox-l, .framebox-r { padding-left:3.0pt; padding-right:3.0pt; text-indent:0pt; border:solid black 0.4pt; }
.framebox-c {text-align:center;}
.framebox-l {text-align:left;}
.framebox-r {text-align:right;}
span.thank-mark{ vertical-align: super }
span.footnote-mark sup.textsuperscript, span.footnote-mark a sup.textsuperscript{ font-size:80%; }
div.tabular, div.center div.tabular {text-align: center; margin-top:0.5em; margin-bottom:0.5em; }
table.tabular td p{margin-top:0em;}
table.tabular {margin-left: auto; margin-right: auto;}
div.td00{ margin-left:0pt; margin-right:0pt; }
div.td01{ margin-left:0pt; margin-right:5pt; }
div.td10{ margin-left:5pt; margin-right:0pt; }
div.td11{ margin-left:5pt; margin-right:5pt; }
table[rules] {border-left:solid black 0.4pt; border-right:solid black 0.4pt; }
td.td00{ padding-left:0pt; padding-right:0pt; }
td.td01{ padding-left:0pt; padding-right:5pt; }
td.td10{ padding-left:5pt; padding-right:0pt; }
td.td11{ padding-left:5pt; padding-right:5pt; }
table[rules] {border-left:solid black 0.4pt; border-right:solid black 0.4pt; }
.hline hr, .cline hr{ height : 1px; margin:0px; }
.tabbing-right {text-align:right;}
span.TEX {letter-spacing: -0.125em; }
span.TEX span.E{ position:relative;top:0.5ex;left:-0.0417em;}
a span.TEX span.E {text-decoration: none; }
span.LATEX span.A{ position:relative; top:-0.5ex; left:-0.4em; font-size:85%;}
span.LATEX span.TEX{ position:relative; left: -0.4em; }
div.float img, div.float .caption {text-align:center;}
div.figure img, div.figure .caption {text-align:center;}
.marginpar {width:20%; float:right; text-align:left; margin-left:auto; margin-top:0.5em; font-size:85%; text-decoration:underline;}
.marginpar p{margin-top:0.4em; margin-bottom:0.4em;}
.equation td{text-align:center; vertical-align:middle; }
td.eq-no{ width:5%; }
table.equation { width:100%; } 
div.math-display, div.par-math-display{text-align:center;}
math .texttt { font-family: monospace; }
math .textit { font-style: italic; }
math .textsl { font-style: oblique; }
math .textsf { font-family: sans-serif; }
math .textbf { font-weight: bold; }
.partToc a, .partToc, .likepartToc a, .likepartToc {line-height: 200%; font-weight:bold; font-size:110%;}
.chapterToc a, .chapterToc, .likechapterToc a, .likechapterToc, .appendixToc a, .appendixToc {line-height: 200%; font-weight:bold;}
.index-item, .index-subitem, .index-subsubitem {display:block}
.caption td.id{font-weight: bold; white-space: nowrap; }
table.caption {text-align:center;}
h1.partHead{text-align: center}
p.bibitem { text-indent: -2em; margin-left: 2em; margin-top:0.6em; margin-bottom:0.6em; }
p.bibitem-p { text-indent: 0em; margin-left: 2em; margin-top:0.6em; margin-bottom:0.6em; }
.subsectionHead, .likesubsectionHead { margin-top:2em; font-weight: bold;}
.sectionHead, .likesectionHead { font-weight: bold;}
.quote {margin-bottom:0.25em; margin-top:0.25em; margin-left:1em; margin-right:1em; text-align:justify;}
.verse{white-space:nowrap; margin-left:2em}
div.maketitle {text-align:center;}
h2.titleHead{text-align:center;}
div.maketitle{ margin-bottom: 2em; }
div.author, div.date {text-align:center;}
div.thanks{text-align:left; margin-left:10%; font-size:85%; font-style:italic; }
div.author{white-space: nowrap;}
.quotation {margin-bottom:0.25em; margin-top:0.25em; margin-left:1em; }
h1.partHead{text-align: center}
.sectionToc, .likesectionToc {margin-left:2em;}
.subsectionToc, .likesubsectionToc {margin-left:4em;}
.sectionToc, .likesectionToc {margin-left:6em;}
.frenchb-nbsp{font-size:75%;}
.frenchb-thinspace{font-size:75%;}
.figure img.graphics {margin-left:10%;}
/* end css.sty */

\title{Determinants}
\author{}
\date{}

\begin{document}
\maketitle

\textbf{Warning: 
requires JavaScript to process the mathematics on this page.\\ If your
browser supports JavaScript, be sure it is enabled.}

\begin{center}\rule{3in}{0.4pt}\end{center}

[
[
[]
[

\section{2.7 Déterminants}

\subsection{2.7.1 Formes p-linéaires}

Définition~2.7.1 On appelle forme p-linéaire sur le Kespace vectoriel E
toute application \phi : E^p \rightarrow~ K vérifiant
\forall~(a_1,\\\ldots,a_p~)
\in K^p, \forall~~i \in [1,p],
x_i\mapsto~\phi(a_1,\\ldots,a_i-1,x_i,a_i+1,\\\ldots,a_p~)
est linéaire de E dans K.

Définition~2.7.2 Soit f une forme p-linéaire sur E. On dit qu'elle est

\begin{itemize}
\item
  (i) alternée si
  \forall~(x_1\\\ldots,x_p~)
  \in E^p,

  \left (\exists~(i,j) \in
  [1,p]^2,i\neq~j\text
  et x_ i = x_j\right ) \rigtharrow~
  f(x_1,\\ldots,x_p~)
  = 0
\item
  (ii) antisymétrique si
  \forall~(x_1\\\ldots,x_p~)
  \in E^p,

  \forall~\sigma \inS_p~,
  f(x_\sigma(1),\\ldots,x_\sigma(p)~)
  =
  \epsilon(\sigma)f(x_1,\\ldots,x_p~)
\end{itemize}

Proposition~2.7.1 Toute forme alternée est antisymétrique. Si
carK\mathrel\neq~~2, toute forme
antisymétrique est alternée.

Démonstration On remarque que si f est alternée, on a

\begin{align*} 0& =&
f(\\ldots,x_i~
+
x_j,\\ldots,x_i~
+
x_j,\\ldots~)
\%& \\ & =&
f(\\ldots,x_i,\\\ldots,x_j,\\\ldots~)
+
f(\\ldots,x_j,\\\ldots,x_i,\\\ldots~)\%&
\\ \end{align*}

On a donc
f(x_\sigma(1),\\ldots,x_\sigma(p)~)
=
\epsilon(\sigma)f(x_1,\\ldots,x_p~)
lorsque \sigma est la transposition \tau_i,j. Comme les transpositions
engendrent S_p, f est antisymétrique. Inversement si f est
antisymétrique et si x_i = x_j, soit \sigma la
transposition \tau_i,j. On a alors

\begin{align*}
f(\\ldots,x_i,\\\ldotsx_j,\\\ldots~)&
=&
-f(\\ldots,x_j,\\\ldotsx_i,\\\ldots~)\%&
\\ & =&
-f(\\ldots,x_i,\\\ldotsx_j,\\\ldots~)\%&
\\ \end{align*}

soit
f(\\ldots,x_i,\\\ldotsx_j,\\\ldots~)
= 0 si carK\mathrel\neq~~2.

Définition~2.7.3 On note A_p(E) l'espace vectoriel des formes
p-linéaires alternées sur E.

Théorème~2.7.2 Toute forme alternée est nulle sur une famille liée.

Démonstration Il suffit d'écrire un terme comme combinaison linéaire des
autres et de développer. Dans tous les termes obtenus figurent deux
termes identiques et donc chaque terme est nul~: si x_k
= \\sum ~
_j\neq~k\alpha_jx_j, on a

[} \varphi (x_1,\ldots
,x_k,\ldots
,x_p)=\sum_j\ne
k\alpha _j\varphi
(x_1,\ldots,\overbracex_j^j,\ldots
,\overbracex_j^k,\ldots ,x_p)=0
]}

où l'on a indiqué au dessus de x_j l'indice de sa position.

\subsection{2.7.2 Déterminant d'une famille de vecteurs}

Définition~2.7.4 Soit E un K-espace vectoriel de dimension n, \mathcal{E} =
(e_1,\\ldots,e_n~)
une base de E de base duale \mathcal{E}^∗ =
(e_1^∗,\\ldots,e_n^∗~),
et
(x_1,\\ldots,x_n~)
une famille de vecteurs de E. On pose
\mathrm{det}~
_\mathcal{E}(x_1,\\ldots,x_n~)
= \\sum ~
_\sigma\inS_n\epsilon(\sigma)\\∏
 _j=1^ne_j^∗(x_\sigma(j)).

Théorème~2.7.3 \mathrm{det}~
_\mathcal{E}\in A_n(E). L'espace vectoriel A_n(E) est de
dimension 1~: pour toute f \in A_n(E), on a f =
\lambda_f \mathrm{det}~
_\mathcal{E} avec \lambda_f =
f(e_1,\\ldots,e_n~).
L'application \mathrm{det}~
_\mathcal{E} est l'unique forme n-linéaire alternée vérifiant
f(e_1,\\ldots,e_n~)
= 1.

Démonstration \mathrm{det}~
_\mathcal{E} est clairement n-linéaire. Supposons que x_i =
x_j et écrivons S_n = A \cup \tau_i,jA où A est
l'ensemble des permutations de signature +1. On a donc, en tenant compte
de \epsilon(\tau_i,j\sigma) = -\epsilon(\sigma)

\begin{align*}
\mathrm{det}~
_\mathcal{E}(x_1,\\ldots,x_n~)&
=& \\sum
_\sigma\inA\epsilon(\sigma)\∏
_k=1^ne_ k^∗(x_ \sigma(k)) \%&
\\ & -& \\sum
_\sigma\inA\epsilon(\sigma)\∏
_k=1^ne_ k^∗(x_
\tau_i,j\sigma(k))\%& \\
\end{align*}

Mais on a pour tout k \in [1,n], x_\tau_i,j\sigma(k) =
x_\sigma(k)~: c'est évident si
\sigma(k)∉\i,j\
car alors \tau_i,j\sigma(k) = \sigma(k), et si par exemple \sigma(k) = i, on a
x_\tau_i,j\sigma(k) = x_j = x_i =
x_\sigma(k). On en déduit que dans la différence précédente, les
deux sommes sont égales, et donc
\mathrm{det}~
_\mathcal{E}(x_1,\\ldots,x_n~)
= 0. Ceci montre le caractère alterné de
\mathrm{det} _\mathcal{E}~.

Soit maintenant f \in A_n(E). On pose x_j
= \\sum ~
_i=1^n\xi_i,je_i. On a alors

f(x_1,\\ldots,x_n~)
= \\sum
_i_1,\ldots,i_n\in\mathbb{N}~\xi_i_1,1\\ldots\xi_i_n,nf(e_i_1,\\ldots,e_i_n~)

En fait dans cette somme on peut se limiter aux
i_1,\\ldots,i_n~
distincts, car sinon
f(e_i_1,\\ldots,e_i_n~)
= 0. Posant i_1 =
\sigma(1),\\ldots,i_n~
= \sigma(n) où \sigma est une permutation de [1,n], on obtient (compte tenu de
f(e_i_1,\\ldots,e_i_n~)
=
f(e_\sigma(1),\\ldots,e_\sigma(n)~)
=
\epsilon(\sigma)f(e_1,\\ldots,e_n~))

\begin{align*}
f(x_1,\\ldots,x_n~)&
=&
f(e_1,\\ldots,e_n~)\\sum
_\sigma\inS_n\epsilon(\sigma)\xi_\sigma(1),1\ldots\xi_\sigma(n),n~\%&
\\ & =&
f(e_1,\\ldots,e_n)f_0(x_1,\\\ldots,x_n~)
\%& \\ \end{align*}

avec une définition évidente de f_0.

Ceci montre clairement que dim A_n~(E)
\leq 1. Comme d'autre part,
\mathrm{det} _\mathcal{E}~ est
non nul (car on vérifie immédiatement que
\mathrm{det}~
_\mathcal{E}(e_1,\\ldots,e_n~)
= 1~: il y a un seul terme non nul dans la somme), c'est qu'il est bien
de dimension 1. Le reste en résulte immédiatement (ainsi que le fait que
f_0 = \mathrm{det}~
_\mathcal{E}).

Théorème~2.7.4 Soit \mathcal{E} =
(e_1,\\ldots,e_n~)
une base de E et
(x_1,\\ldots,x_n~)
une famille de E. Alors c'est une base de E si et seulement si
\mathrm{det}~
_\mathcal{E}(x_1,\\ldots,x_n)\mathrel\neq~~0.

Démonstration En effet, si c'est une base X, on a
\mathrm{det} _\mathcal{E}~
= \mathrm{det}~
_\mathcal{E}(x_1,\\ldots,x_n)\\mathrm{det}~
_X et comme
\mathrm{det}~
_\mathcal{E}\neq~0, on a
\mathrm{det}~
_\mathcal{E}(x_1,\\ldots,x_n)\mathrel\neq~~0~;
si ce n'est pas une base, c'est que la famille est liée (son cardinal
est n) et donc \mathrm{det}~
_\mathcal{E}(x_1,\\ldots,x_n~)
= 0.

\subsection{2.7.3 Déterminant d'un endomorphisme}

Théorème~2.7.5 Soit u \in L(E). Il existe un unique scalaire noté
\mathrm{det}~ u vérifiant
\forall~f \in A_n~(E),
\forall~(x_1,\\\ldots,x_n~)
\in E^n~;

f(u(x_1),\\ldots,u(x_n~))
= \mathrm{det}~ u
f(x_1,\\ldots,x_n~)

Démonstration Soit \phi_u : A_n(E) \rightarrow~ A_n(E)
définie par
\phi_u(f)(x_1,\\ldots,x_n~)
=
f(u(x_1),\\ldots,u(x_n~)).
C'est un endomorphisme d'un espace vectoriel de dimension 1, donc une
homothétie~; on note
\mathrm{det}~ u son rapport.

Proposition~2.7.6

\begin{itemize}
\item
  (i) Soit \mathcal{E} =
  (e_1,\\ldots,e_n~)
  une base de E, alors

  \mathrm{det}~ u
  = \mathrm{det}~
  _\mathcal{E}(u(e_1),\\ldots,u(e_n~))
\item
  (ii) \mathrm{det}~
  \mathrmId = 1,
  \mathrm{det}~ \lambda~u =
  \lambda~^n \mathrm{det}~
  u, \mathrm{det}~
  ^tu = \mathrm{det}~
  u
\item
  (iii) \mathrm{det}~ v \cdot u
  = \mathrm{det}~
  v\mathrm{det}~ u
\item
  (iv) u est un automorphisme de E si et seulement si
  \mathrm{det}~
  u\neq~0.
\end{itemize}

Démonstration Tout est presque immédiat~; (iii) découle de
\phi_v\cdotu = \phi_u \cdot \phi_v et du fait que le rapport
du produit de deux homothéties est le produit des rapports.

\subsection{2.7.4 Déterminant d'une matrice}

Définition~2.7.5 Soit A \in M_k(n). On appelle déterminant de A
le déterminant de la famille de ses vecteurs colonnes dans la base
canonique. On a donc
\mathrm{det}~ A
= \\sum ~
_\sigma\inS_n\epsilon(\sigma)\\∏
 _j=1^na_j,\sigma(j).

Proposition~2.7.7

\begin{itemize}
\itemsep1pt\parskip0pt\parsep0pt
\item
  (i) Soit \mathcal{E} =
  (e_1,\\ldots,e_n~)
  une base de E, u \in L(E), alors
  \mathrm{det}~ u
  = \mathrm{det}~
  \mathrmMat~ (u,\mathcal{E})
\item
  (ii) \mathrm{det}~
  I_n = 1,
  \mathrm{det}~ \lambda~A =
  \lambda~^n \mathrm{det}~
  A, \mathrm{det}~
  ^tA = \mathrm{det}~
  A
\item
  (iii) \mathrm{det}~ AB
  = \mathrm{det}~
  A\mathrm{det}~ B
\item
  (iv) A est inversible si et seulement si
  \mathrm{det}~
  A\neq~0.
\end{itemize}

Démonstration On a
\mathrm{det}~ u
= \mathrm{det}~
_\mathcal{E}(u(e_1),\\ldots,u(e_n~))
ce qui n'est autre que
\mathrm{det}~
\mathrmMat~ (u,\mathcal{E}) puisque
les vecteurs colonnes de cette matrice sont constitués des coordonnées
des u(e_j) dans la base \mathcal{E}, ce qui montre (i). La formule
\mathrm{det} ^t~A
= \mathrm{det}~ A résulte de
la remarque que nous avons faite lors de la démonstration du fait que
A_n(E) est de dimension 1~: f_0
= \mathrm{det} _\mathcal{E}~,
soit encore

\\sum
_\sigma\inS_n\epsilon(\sigma)\xi_\sigma(1),1\ldots\xi_\sigma(n),n~
= \\sum
_\sigma\inS_n\epsilon(\sigma)\xi_1,\sigma(1)\ldots\xi_n,\sigma(n)~

Tout le reste s'obtient en traduisant les résultats similaires sur les
endomorphismes.

Proposition~2.7.8 Le déterminant d'une matrice dépend linéairement de
chacun de ses vecteurs colonnes (resp lignes), le déterminant d'une
matrice ne change pas si on ajoute à une colonne (resp. ligne) une
combinaison linéaire des autres colonnes (resp. lignes). Si on effectue
une permutation \sigma sur les colonnes (resp. lignes) d'une matrice, son
déterminant est multiplié par \epsilon(\sigma). Application Calcul par la méthode du
pivot.

Démonstration Evident de par la définition du déterminant d'une matrice
comme déterminant de la famille de ses vecteurs colonnes (ou de ses
vecteurs lignes par transposition)

Théorème~2.7.9 (calcul des déterminants par blocs).
\mathrm{det}~
\left ( \includegraphics{cours3x.png}
\,\right ) = \
\mathrm{det}
A.\mathrm{det}~ C.

Démonstration Notons M = (a_i,j)_1\leqi,j\leqn =
\left
(\matrix\,A&B\cr 0
&C\right ), si bien que l'on a a_i,j = 0 si i
≥ p + 1 et j \leq p. On a alors
\mathrm{det}~ M
= \\sum ~
_\sigma\inS_n\epsilon(\sigma)\\∏
 _k=1^na_k,\sigma(k). Soit \sigma \inS_n~; s'il
existe k_0 \in [p + 1,n] tel que \sigma(k_0) \in [1,p],
on a alors a_k_0,\sigma(k_0) = 0 et donc
\∏ ~
_k=1^na_k,\sigma(k) = 0. Autrement dit, les seules
permutations qui peuvent donner une contribution non nulle au
déterminant sont les permutations \sigma \inS_n telles que \sigma([p +
1,n]) \subset~ [p + 1,n], c'est-à-dire vérifiant \sigma([p + 1,n]) = [p
+ 1,n] et donc aussi \sigma([1,p]) = [1,p]. Une telle permutation \sigma
est entièrement définie par ses restrictions \sigma_1 à [1,p] et
\sigma_2 à [p + 1,n], l'application
\sigma\mapsto~(\sigma_1,\sigma_2) étant bijective
de l'ensemble de ces permutations sur S_p \timesS_n-p.
D'autre part, on voit immédiatement que toute décomposition de
\sigma_1 et \sigma_2 en produit de transpositions, fournit une
décomposition de \sigma en produit de transpositions, ce qui montre que \epsilon(\sigma)
= \epsilon(\sigma_1)\epsilon(\sigma_2). On obtient donc~:

\begin{align*}
\mathrm{det}~ M& =&
\sum _ \sigma_1\inS_p~
\atop \sigma_2\inS_n-p
\epsilon(\sigma_1)\epsilon(\sigma_2)\∏
_k=1^pa_ k,\sigma_1(k)
∏ _k=p+1^na_
k,\sigma_2(k) \%& \\ & =&
\\sum
_\sigma_1\inS_p\epsilon(\sigma_1)\∏
_k=1^pa_ k,\sigma_1(k)
\\sum
_\sigma_2\inS_n-p\epsilon(\sigma_2)\∏
_k=p+1^na_ k,\sigma_2(k)\%&
\\ & =&
\mathrm{det}~
A.\mathrm{det}~ C \%&
\\ \end{align*}

Corollaire~2.7.10 Le déterminant d'une matrice triangulaire par blocs
est égal au produit des déterminants des blocs diagonaux. Le déterminant
d'une matrice triangulaire est égal au produit de ses éléments
diagonaux.

Démonstration Récurrence évidente.

Définition~2.7.6 Soit A = (a_i,j) \in M_K(n). On notera
A_k,l = (-1)^k+l\
\mathrm{det}
(a_i,j)_i\neq~k,j\mathrel\neq~l
(cofacteur d'indice (k,l)). La matrice (A_i,j) \in
M_K(n) est appelée la comatrice de la matrice A.

Théorème~2.7.11 (développement d'un déterminant). On a

\forall~~i \in [1,n],\quad
\mathrm{det}~ A =
\sum _k=1^na_
i,kA_i,k

(développement suivant la i-ième ligne)

\forall~~j \in [1,n],\quad
\mathrm{det}~ A =
\sum _k=1^na_
k,jA_k,j

(développement suivant la j-ième colonne)

Démonstration Par exemple sur les colonnes~; soit
(\epsilon_1,\\ldots,\epsilon_n~)
la base canonique de K^n,
(c_1,\\ldotsc_n~)
les vecteurs colonnes de la matrice A. On a c_j
= \\sum ~
_k=1^na_k,j\epsilon_k, d'où

\begin{align*}
\mathrm{det}~ A& =&
\mathrm{det}~
(c_1,\\ldots,c_n~)
\%& \\ & =& \\sum
_k=1^na_ k,j \mathrm{det}
(c_1,\ldots,c_j-1,\epsilon_k,c_j+1,\\ldots,c_n~)\%&
\\ & =& \\sum
_k=1^na_ k,j\Delta_k,j \%&
\\ \end{align*}

Par combinaisons linéaires de colonnes (pour éliminer les termes de la
k-ième ligne) puis par échange de lignes et de colonnes, on obtient

\Delta_k,l = (-1)^k+l\left
\matrix\,1&0\\ldots~0
\cr \matrix\,0
\cr \⋮~
\cr
0&(a_i,j)_i\neq~k,j\mathrel\neq~l\right
 = (-1)^k+lA_ k,l

Corollaire~2.7.12 A^t com~A =
^t com~A A =
(\mathrm{det} A)I_n~.
Si A est inversible, A^-1 = 1 \over
\mathrm{det} A~
^t com~A.

Démonstration En effet (A^t\
comA)_i,j =\ \\sum
 _k=1^na_i,kA_j,k. Mais ceci n'est
autre que le développement suivant la j-ième ligne du déterminant de la
matrice B obtenue à partir de A en rempla\ccant la
j-ième ligne par la i-ième. Si i = j, c'est donc
\mathrm{det}~ A. Si
i\neq~j, la matrice B a deux lignes identiques,
donc son déterminant est nul.

\subsection{2.7.5 Application des déterminants à la recherche du rang}

Lemme~2.7.13 Soit A = (a_i,j) \in M_K(m,n) et soit B =
(a_i,j)_i\inI,j\inJ une sous-matrice de A (avec I \subset~
[1,m] et J \subset~ [1,n]. Alors
\mathrmrg~B
\leq\mathrmrg~A.

Démonstration Soit C = (a_i,j)_i\in[1,m],j\inJ et soit
c_1,\\ldots,c_n~
les vecteurs colonnes de A. Alors on a
\mathrmrg~C
= \mathrmrg(c_j~, j
\in J)
\leq\mathrmrg(c_1,\\\ldots,c_n~)
= \mathrmrg~A. Mais soit
d'autre part
l_1',\\ldots,l_m~'
les vecteurs lignes de la matrice C. On a
\mathrmrg~B
= \mathrmrg(l_i~',
i \in I)
\leq\mathrmrg(l_1',\\\ldots,l_m~')
= \mathrmrg~C, d'où
\mathrmrg~B
\leq\mathrmrg~A.

Soit alors r le rang de A. D'après le lemme précédent, toute
sous-matrice inversible B de A a une taille (un ordre) plus petit que r.
On a alors le théorème suivant

Théorème~2.7.14 Soit A = (a_i,j) \in M_K(m,n) de rang r
et soit B = (a_i,j)_i\inI,j\inJ une sous-matrice de A
carrée inversible avec I = J
< r. Alors il existe i_0 \in [1,m] \diagdown I,j_0
\in [1,n] \diagdown J tels que la matrice B' =
(a_i,j)_i\inI\cup\i_0\,j\inJ\cup\j_0\
(matrice bordante de B dans A) soit encore inversible.

Démonstration Soit C = (a_i,j)_i\in[1,m],j\inJ et soit
c_1,\\ldots,c_n~
les vecteurs colonnes de A. On a
\mathrmrg~C
\leqJ (car C a J vecteurs colonnes)
et \mathrmrg~C
≥\mathrmrg~B =
J (car B est une sous-matrice de C). Donc
\mathrmrg~C =
J < r. Ceci montre que la famille
(c_j)_j\inJ est libre. D'autre part dans V
=\
\mathrmVect(c_1,\\ldots,c_n~),
la famille
(c_1,\\ldots,c_n~)
est génératrice. Par le théorème de la base incomplète, il existe J' tel
que J \subset~ J' \subset~ [1,n] avec (c_j)_j\inJ' base de V .
Mais J' = dim~ V = r
> J donc on peut prendre un
j_0 \in J' \diagdown J et la famille
(c_j)_j\inJ\cup\j_0\
est encore libre. Soit D =
(a_i,j)_i\in[1,m],j\inJ\cup\j_0\.
Le rang de D est donc I + 1. Soit
l_1',\\ldots,l_m~'
les vecteurs colonnes de la matrice D. La matrice
(a_i,j)_i\inI,j\inJ\cup\j_0\
est de rang I (elle a I lignes
et contient la matrice B de rang I), donc la famille
(l_i')_i\inI est de rang I alors que
la famille (l_i')_i\in[1,m] est de rang
I + 1. Le même argument à base de théorème de la
base incomplète montre que l'on peut trouver i_0 \in [1,m] \diagdown
I tel que la famille
(l_i')_i\inI\cup\i_0\
soit encore libre. La matrice B' =
(a_i,j)_i\inI\cup\i_0\,j\inJ\cup\j_0\
est donc inversible.

Remarque~2.7.1 Le théorème précédent montre donc que toute sous-matrice
inversible de taille strictement inférieure à r peut être complétée en
une autre sous-matrice inversible. On en déduit

Théorème~2.7.15 Soit A = (a_i,j) \in M_K(m,n) de rang r.
Alors A contient des sous-matrices carrées inversibles de rang r
(sous-matrices principales). Une sous-matrice carrée inversible est une
sous-matrice principale si et seulement si toutes ses matrices bordantes
sont non inversibles.

Remarque~2.7.2 Ceci permet de rechercher théoriquement le rang d'une
matrice à l'aide de déterminants, en augmentant au fur et à mesure la
taille des sous-matrices inversibles.

\subsection{2.7.6 Formes p-linéaires alternées}

Proposition~2.7.16 Soit E un K-espace vectoriel,
f_1,\\ldots,f_p~
\in E^∗. Alors f_1
∧\\ldots~ ∧
f_p : E^p \rightarrow~ K définie par
(x_1,\\ldots,x_p)\mapsto~\\mathrm{det}~
(f_i(x_j))_1\leqi\leqp,1\leqj\leqp est une forme p-
linéaire alternée sur E.

Ceci va nous permettre d'exhiber une base de A_p(E) en
utilisant les deux lemmes suivants. Pour cela soit E un K-espace
vectoriel de dimension n et \mathcal{E} =
(e_1,\\ldots,e_n~)
une base de E de base duale \mathcal{E}^∗ =
(e_1^∗,\\ldots,e_n^∗~).

Lemme~2.7.17 Soit f,g \in A_p(E) telles que pour toute famille
(i_1,\\ldots,i_p~)
vérifiant 1 \leq i_1 < i_2 <
\\ldots~ <
i_p \leq n, on ait
f(e_i_1,\\ldots,e_i_p~)
=
g(e_i_1,\\ldots,e_i_p~).
Alors f = g.

Démonstration La relation
f(e_i_1,\\ldots,e_i_p~)
=
g(e_i_1,\\ldots,e_i_p~)
reste encore vraie si
i_1,\\ldots,i_p~
sont distincts mais non ordonnés (il suffit de les réordonner par une
permutation \sigma, ce qui ne fait que multiplier les deux côtés par \epsilon(\sigma)).
Elle est triviale si
i_1,\\ldots,i_p~
ne sont pas distincts car alors les deux membres valent 0. Mais alors,
on a en posant x_j =\
\sum ~
_i=1^n\xi_i,je_i

f(x_1,\\ldots,x_p~)
= \\sum
_i_1,\ldots,i_p\in\mathbb{N}~\xi_i_1,1\\ldots\xi_i_p,pf(e_i_1,\\ldots,e_i_p~)

et la même chose pour g. Donc f = g.

Lemme~2.7.18 Soit 1 \leq i_1 < i_2 <
\\ldots~ <
i_p \leq n et 1 \leq j_1 < j_2
< \\ldots~
< j_p \leq n. Alors

e_i_1^∗∧\\ldots~
∧ e_ i_p^∗(e_
j_1,\\ldots,e_j_p~)
=
\delta_i_1,\\ldots,i_p^j_1,\\\ldots,j_p~


(symbole de Kronecker)

Démonstration Il est clair que
e_i_1^∗∧\\ldots~
∧
e_i_1^∗(e_i_1,\\ldots,e_i_p~)
= 1 (la matrice ''f_i(x_j)'' est l'identité).
Supposons donc que i_1 =
j_1,\\ldots,i_k-1~
= j_k-1,i_k\neq~j_k.
Si i_k < j_k, on a pour tout l \in
[1,p],i_k\neq~j_l soit
e_j_l^∗(e_i_k) = 0. La
matrice ''f_i(x_j)'' a donc sa k-ième ligne nulle et
son déterminant est donc nul. Si j_k < i_k,
de manière similaire, la k-ième colonne de la matrice est nulle. Dans
les deux cas, on trouve donc 0 comme résultat.

Théorème~2.7.19 La famille des
(e_i_1^∗∧\\ldots~
∧
e_i_p^∗)_1\leqi_1<i_2<\\ldots<i_p\leqn~
est une base de A_p(E) (qui est donc de dimension
C_n^p).

Démonstration Montrons que la famille est génératrice. Soit f \in
A_p(E) et

g = \\sum ~
_1\leqj_1<j_2<\\ldots<j_p\leqnf(e_j_1,\\\ldots,e_j_p)e_j_1^∗∧\\\ldots~
∧ e_j_p^∗. Grâce au lemme 2, si 1 \leq
i_1 < i_2 <
\\ldots~ <
i_p \leq n, on a
g(e_i_1,\\ldots,e_i_p~)
=
f(e_i_1,\\ldots,e_i_p~).
D'après le lemme 1, on a f = g. Il reste à montrer que la famille est
libre. Supposons que \\\sum

_1\leqj_1<j_2<\\ldots<j_p\leqn\lambda_j_1,\\\ldots,j_pe_j_1^∗∧\\\ldots~
∧ e_j_p^∗ = 0. Grâce au lemme 2, si 1 \leq
i_1 < i_2 <
\\ldots~ <
i_p \leq n on a

\begin{align*} 0& =&
0(e_i_1,\\ldots,e_i_p~)
\%& \\ & =& \\sum
_1\leqj_1<j_2<\ldots<j_p\leqn\lambda_j_1,\\ldots,j_pe_j_1^∗∧\\ldots~
∧ e_ j_p^∗(e_
i_1,\ldots,e_i_p~)
=
\lambda_i_1,\ldots,i_p~\%&
\\ \end{align*}

ce qui montre que la famille est libre.

[
[
[
[

\end{document}

% \documentclass[]{article}
\usepackage[T1]{fontenc}
\usepackage{lmodern}
\usepackage{amssymb,amsmath}
\usepackage{ifxetex,ifluatex}
\usepackage{fixltx2e} % provides \textsubscript
% use upquote if available, for straight quotes in verbatim environments
\IfFileExists{upquote.sty}{\usepackage{upquote}}{}
\ifnum 0\ifxetex 1\fi\ifluatex 1\fi=0 % if pdftex
  \usepackage[utf8]{inputenc}
\else % if luatex or xelatex
  \ifxetex
    \usepackage{mathspec}
    \usepackage{xltxtra,xunicode}
  \else
    \usepackage{fontspec}
  \fi
  \defaultfontfeatures{Mapping=tex-text,Scale=MatchLowercase}
  \newcommand{\euro}{€}
\fi
% use microtype if available
\IfFileExists{microtype.sty}{\usepackage{microtype}}{}
\ifxetex
  \usepackage[setpagesize=false, % page size defined by xetex
              unicode=false, % unicode breaks when used with xetex
              xetex]{hyperref}
\else
  \usepackage[unicode=true]{hyperref}
\fi
\hypersetup{breaklinks=true,
            bookmarks=true,
            pdfauthor={},
            pdftitle={Syst`emes lineaires},
            colorlinks=true,
            citecolor=blue,
            urlcolor=blue,
            linkcolor=magenta,
            pdfborder={0 0 0}}
\urlstyle{same}  % don't use monospace font for urls
\setlength{\parindent}{0pt}
\setlength{\parskip}{6pt plus 2pt minus 1pt}
\setlength{\emergencystretch}{3em}  % prevent overfull lines
\setcounter{secnumdepth}{0}
 
/* start css.sty */
.cmr-5{font-size:50%;}
.cmr-7{font-size:70%;}
.cmmi-5{font-size:50%;font-style: italic;}
.cmmi-7{font-size:70%;font-style: italic;}
.cmmi-10{font-style: italic;}
.cmsy-5{font-size:50%;}
.cmsy-7{font-size:70%;}
.cmex-7{font-size:70%;}
.cmex-7x-x-71{font-size:49%;}
.msbm-7{font-size:70%;}
.cmtt-10{font-family: monospace;}
.cmti-10{ font-style: italic;}
.cmbx-10{ font-weight: bold;}
.cmr-17x-x-120{font-size:204%;}
.cmsl-10{font-style: oblique;}
.cmti-7x-x-71{font-size:49%; font-style: italic;}
.cmbxti-10{ font-weight: bold; font-style: italic;}
p.noindent { text-indent: 0em }
td p.noindent { text-indent: 0em; margin-top:0em; }
p.nopar { text-indent: 0em; }
p.indent{ text-indent: 1.5em }
@media print {div.crosslinks {visibility:hidden;}}
a img { border-top: 0; border-left: 0; border-right: 0; }
center { margin-top:1em; margin-bottom:1em; }
td center { margin-top:0em; margin-bottom:0em; }
.Canvas { position:relative; }
li p.indent { text-indent: 0em }
.enumerate1 {list-style-type:decimal;}
.enumerate2 {list-style-type:lower-alpha;}
.enumerate3 {list-style-type:lower-roman;}
.enumerate4 {list-style-type:upper-alpha;}
div.newtheorem { margin-bottom: 2em; margin-top: 2em;}
.obeylines-h,.obeylines-v {white-space: nowrap; }
div.obeylines-v p { margin-top:0; margin-bottom:0; }
.overline{ text-decoration:overline; }
.overline img{ border-top: 1px solid black; }
td.displaylines {text-align:center; white-space:nowrap;}
.centerline {text-align:center;}
.rightline {text-align:right;}
div.verbatim {font-family: monospace; white-space: nowrap; text-align:left; clear:both; }
.fbox {padding-left:3.0pt; padding-right:3.0pt; text-indent:0pt; border:solid black 0.4pt; }
div.fbox {display:table}
div.center div.fbox {text-align:center; clear:both; padding-left:3.0pt; padding-right:3.0pt; text-indent:0pt; border:solid black 0.4pt; }
div.minipage{width:100%;}
div.center, div.center div.center {text-align: center; margin-left:1em; margin-right:1em;}
div.center div {text-align: left;}
div.flushright, div.flushright div.flushright {text-align: right;}
div.flushright div {text-align: left;}
div.flushleft {text-align: left;}
.underline{ text-decoration:underline; }
.underline img{ border-bottom: 1px solid black; margin-bottom:1pt; }
.framebox-c, .framebox-l, .framebox-r { padding-left:3.0pt; padding-right:3.0pt; text-indent:0pt; border:solid black 0.4pt; }
.framebox-c {text-align:center;}
.framebox-l {text-align:left;}
.framebox-r {text-align:right;}
span.thank-mark{ vertical-align: super }
span.footnote-mark sup.textsuperscript, span.footnote-mark a sup.textsuperscript{ font-size:80%; }
div.tabular, div.center div.tabular {text-align: center; margin-top:0.5em; margin-bottom:0.5em; }
table.tabular td p{margin-top:0em;}
table.tabular {margin-left: auto; margin-right: auto;}
div.td00{ margin-left:0pt; margin-right:0pt; }
div.td01{ margin-left:0pt; margin-right:5pt; }
div.td10{ margin-left:5pt; margin-right:0pt; }
div.td11{ margin-left:5pt; margin-right:5pt; }
table[rules] {border-left:solid black 0.4pt; border-right:solid black 0.4pt; }
td.td00{ padding-left:0pt; padding-right:0pt; }
td.td01{ padding-left:0pt; padding-right:5pt; }
td.td10{ padding-left:5pt; padding-right:0pt; }
td.td11{ padding-left:5pt; padding-right:5pt; }
table[rules] {border-left:solid black 0.4pt; border-right:solid black 0.4pt; }
.hline hr, .cline hr{ height : 1px; margin:0px; }
.tabbing-right {text-align:right;}
span.TEX {letter-spacing: -0.125em; }
span.TEX span.E{ position:relative;top:0.5ex;left:-0.0417em;}
a span.TEX span.E {text-decoration: none; }
span.LATEX span.A{ position:relative; top:-0.5ex; left:-0.4em; font-size:85%;}
span.LATEX span.TEX{ position:relative; left: -0.4em; }
div.float img, div.float .caption {text-align:center;}
div.figure img, div.figure .caption {text-align:center;}
.marginpar {width:20%; float:right; text-align:left; margin-left:auto; margin-top:0.5em; font-size:85%; text-decoration:underline;}
.marginpar p{margin-top:0.4em; margin-bottom:0.4em;}
.equation td{text-align:center; vertical-align:middle; }
td.eq-no{ width:5%; }
table.equation { width:100%; } 
div.math-display, div.par-math-display{text-align:center;}
math .texttt { font-family: monospace; }
math .textit { font-style: italic; }
math .textsl { font-style: oblique; }
math .textsf { font-family: sans-serif; }
math .textbf { font-weight: bold; }
.partToc a, .partToc, .likepartToc a, .likepartToc {line-height: 200%; font-weight:bold; font-size:110%;}
.chapterToc a, .chapterToc, .likechapterToc a, .likechapterToc, .appendixToc a, .appendixToc {line-height: 200%; font-weight:bold;}
.index-item, .index-subitem, .index-subsubitem {display:block}
.caption td.id{font-weight: bold; white-space: nowrap; }
table.caption {text-align:center;}
h1.partHead{text-align: center}
p.bibitem { text-indent: -2em; margin-left: 2em; margin-top:0.6em; margin-bottom:0.6em; }
p.bibitem-p { text-indent: 0em; margin-left: 2em; margin-top:0.6em; margin-bottom:0.6em; }
.paragraphHead, .likeparagraphHead { margin-top:2em; font-weight: bold;}
.subparagraphHead, .likesubparagraphHead { font-weight: bold;}
.quote {margin-bottom:0.25em; margin-top:0.25em; margin-left:1em; margin-right:1em; text-align:justify;}
.verse{white-space:nowrap; margin-left:2em}
div.maketitle {text-align:center;}
h2.titleHead{text-align:center;}
div.maketitle{ margin-bottom: 2em; }
div.author, div.date {text-align:center;}
div.thanks{text-align:left; margin-left:10%; font-size:85%; font-style:italic; }
div.author{white-space: nowrap;}
.quotation {margin-bottom:0.25em; margin-top:0.25em; margin-left:1em; }
h1.partHead{text-align: center}
.sectionToc, .likesectionToc {margin-left:2em;}
.subsectionToc, .likesubsectionToc {margin-left:4em;}
.subsubsectionToc, .likesubsubsectionToc {margin-left:6em;}
.frenchb-nbsp{font-size:75%;}
.frenchb-thinspace{font-size:75%;}
.figure img.graphics {margin-left:10%;}
/* end css.sty */

\title{Syst`emes lineaires}
\author{}
\date{}

\begin{document}
\maketitle

\textbf{Warning: \href{http://www.math.union.edu/locate/jsMath}{jsMath}
requires JavaScript to process the mathematics on this page.\\ If your
browser supports JavaScript, be sure it is enabled.}

\begin{center}\rule{3in}{0.4pt}\end{center}

{[}\href{coursse13.html}{prev}{]}
{[}\href{coursse13.html\#tailcoursse13.html}{prev-tail}{]}
{[}\hyperref[tailcoursse14.html]{tail}{]}
{[}\href{coursch3.html\#coursse14.html}{up}{]}

\subsubsection{2.8 Systèmes linéaires}

\paragraph{2.8.1 Position du problème}

Soit A = \{(\{a\}\_\{i,j\})\}\_\{1≤i≤m,1≤j≤n\} ∈ \{M\}\_\{K\}(m,n) et
\{b\}\_\{1\},\textbackslash{}mathop\{\textbackslash{}mathop\{\ldots{}\}\},\{b\}\_\{m\}
∈ K. On considère le système d'équations aux inconnues
\{x\}\_\{1\},\textbackslash{}mathop\{\textbackslash{}mathop\{\ldots{}\}\},\{x\}\_\{n\}
∈ K

(L)\textbackslash{}quad \textbackslash{}left
\textbackslash{}\{\textbackslash{}matrix\{\textbackslash{},\{a\}\_\{1,1\}\{x\}\_\{1\}
+ \textbackslash{}mathop\{\textbackslash{}mathop\{\ldots{}\}\} +
\{a\}\_\{1,n\}\{x\}\_\{n\} = \{b\}\_\{1\} \textbackslash{}cr
\textbackslash{}mathop\{\textbackslash{}mathop\{\ldots{}\}\}
\textbackslash{}cr \{a\}\_\{m,1\}\{x\}\_\{1\} +
\textbackslash{}mathop\{\textbackslash{}mathop\{\ldots{}\}\} +
\{a\}\_\{m,n\}\{x\}\_\{n\} = \{b\}\_\{m\}\}\textbackslash{}right .

On a diverses interprétations possibles d'un tel système (en notant
\{\textbackslash{}text\{Can\}\}\_\{n\} la base canonique de
\{K\}\^{}\{n\})~:

\begin{itemize}
\itemsep1pt\parskip0pt\parsep0pt
\item
  Interprétation matricielle~: AX = B si X = \textbackslash{}left
  (\textbackslash{}matrix\{\textbackslash{},\{x\}\_\{1\}
  \textbackslash{}cr
  \textbackslash{}mathop\{\textbackslash{}mathop\{\ldots{}\}\}
  \textbackslash{}cr \{x\}\_\{n\}\}\textbackslash{}right ) et B =
  \textbackslash{}left
  (\textbackslash{}matrix\{\textbackslash{},\{b\}\_\{1\}
  \textbackslash{}cr
  \textbackslash{}mathop\{\textbackslash{}mathop\{\ldots{}\}\}
  \textbackslash{}cr \{b\}\_\{m\}\}\textbackslash{}right ).
\item
  Interprétation linéaire~: u(x) = b si u : \{K\}\^{}\{n\} →
  \{K\}\^{}\{m\} est telle que
  \textbackslash{}mathop\{\textbackslash{}mathrm\{Mat\}\}
  (u,\{\textbackslash{}text\{Can\}\}\_\{n\},\{\textbackslash{}text\{Can\}\}\_\{m\})
  et si x et b admettent
  (\{x\}\_\{1\},\textbackslash{}mathop\{\textbackslash{}mathop\{\ldots{}\}\},\{x\}\_\{n\})
  et
  (\{b\}\_\{1\},\textbackslash{}mathop\{\textbackslash{}mathop\{\ldots{}\}\},\{b\}\_\{m\})
  comme coordonnées dans les bases canoniques
  \{\textbackslash{}text\{Can\}\}\_\{n\} et
  \{\textbackslash{}text\{Can\}\}\_\{m\}.
\item
  Interprétation vectorielle~: b = \{x\}\_\{1\}\{c\}\_\{1\} +
  \textbackslash{}mathop\{\textbackslash{}mathop\{\ldots{}\}\} +
  \{x\}\_\{n\}\{c\}\_\{n\}, si
  \{c\}\_\{1\},\textbackslash{}mathop\{\textbackslash{}mathop\{\ldots{}\}\},\{c\}\_\{n\}
  ∈ \{K\}\^{}\{m\} sont les vecteurs colonnes de la matrice A
\item
  Interprétation duale~: \{f\}\_\{1\}(x) =
  \{b\}\_\{1\},\textbackslash{}mathop\{\textbackslash{}mathop\{\ldots{}\}\},\{f\}\_\{m\}(x)
  = \{b\}\_\{m\} si \{f\}\_\{j\} désigne la forme linéaire sur
  \{K\}\^{}\{n\},
  (\{x\}\_\{1\},\textbackslash{}mathop\{\textbackslash{}mathop\{\ldots{}\}\},\{x\}\_\{n\})\textbackslash{}mathrel\{↦\}\{a\}\_\{j,1\}\{x\}\_\{1\}
  + \textbackslash{}mathop\{\textbackslash{}mathop\{\ldots{}\}\} +
  \{a\}\_\{j,n\}\{x\}\_\{n\}.
\end{itemize}

Remarque~2.8.1 En remarquant que A est la matrice de u dans les bases
canoniques, mais aussi la matrice des coordonnées de
(\{c\}\_\{1\},\textbackslash{}mathop\{\textbackslash{}mathop\{\ldots{}\}\},\{c\}\_\{n\})
dans la base canonique de \{K\}\^{}\{m\} et la transposée de la matrice
de
(\{f\}\_\{1\},\textbackslash{}mathop\{\textbackslash{}mathop\{\ldots{}\}\},\{f\}\_\{m\})
dans la base duale de la base canonique de \{K\}\^{}\{n\}, on obtient

Théorème~2.8.1 Avec les notations ci-dessus, on a
\textbackslash{}mathop\{\textbackslash{}mathrm\{rg\}\}A
=\textbackslash{}mathop\{ \textbackslash{}mathrm\{rg\}\}u
=\textbackslash{}mathop\{
\textbackslash{}mathrm\{rg\}\}(\{c\}\_\{1\},\textbackslash{}mathop\{\textbackslash{}mathop\{\ldots{}\}\},\{c\}\_\{n\})
=\textbackslash{}mathop\{
\textbackslash{}mathrm\{rg\}\}(\{f\}\_\{1\},\textbackslash{}mathop\{\textbackslash{}mathop\{\ldots{}\}\},\{f\}\_\{m\}).
Cette valeur commune est appelée le rang du système.

\paragraph{2.8.2 Systèmes de Cramer}

Définition~2.8.1 On appelle système de Cramer un système d'équations
linéaires vérifiant les conditions équivalentes (qui toutes impliquent
que m = n)

\begin{itemize}
\itemsep1pt\parskip0pt\parsep0pt
\item
  (i) A est une matrice inversible
\item
  (ii) u est un isomorphisme de \{K\}\^{}\{n\} sur \{K\}\^{}\{m\}
\item
  (iii)
  (\{c\}\_\{1\},\textbackslash{}mathop\{\textbackslash{}mathop\{\ldots{}\}\},\{c\}\_\{n\})
  est une base de \{K\}\^{}\{m\}
\item
  (iv)
  (\{f\}\_\{1\},\textbackslash{}mathop\{\textbackslash{}mathop\{\ldots{}\}\},\{f\}\_\{m\})
  est une base de \{(\{K\}\^{}\{n\})\}\^{}\{∗\}
\end{itemize}

Théorème~2.8.2 (Résolution d'un système de Cramer) Un système de Cramer
admet une solution unique donnée par

\begin{itemize}
\item
  (i) Interprétation matricielle~: X = \{A\}\^{}\{−1\}B
\item
  (ii) Interprétation linéaire~: x = \{u\}\^{}\{−1\}(b)
\item
  (iii) Interprétation vectorielle~:

  \{x\}\_\{j\} =\{
  \textbackslash{}mathop\{\textbackslash{}mathrm\{det\}\}
  (\{c\}\_\{1\},\textbackslash{}mathop\{\textbackslash{}mathop\{\ldots{}\}\},\{c\}\_\{j−1\},b,\{c\}\_\{j+1\},\textbackslash{}mathop\{\textbackslash{}mathop\{\ldots{}\}\},\{c\}\_\{n\})
  \textbackslash{}over
  \textbackslash{}mathop\{\textbackslash{}mathrm\{det\}\}
  (\{c\}\_\{1\},\textbackslash{}mathop\{\textbackslash{}mathop\{\ldots{}\}\},\{c\}\_\{n\})\}

  (formules de Cramer)
\item
  (iv) Soit
  (\{v\}\_\{1\},\textbackslash{}mathop\{\textbackslash{}mathop\{\ldots{}\}\},\{v\}\_\{n\})
  la base duale de
  (\{f\}\_\{1\},\textbackslash{}mathop\{\textbackslash{}mathop\{\ldots{}\}\},\{f\}\_\{n\}),
  alors x = \{b\}\_\{1\}\{v\}\_\{1\} +
  \textbackslash{}mathop\{\textbackslash{}mathop\{\ldots{}\}\} +
  \{b\}\_\{n\}\{v\}\_\{n\}.
\end{itemize}

Démonstration Tout est évident sauf le (iii). Mais on a b =
\{x\}\_\{1\}\{c\}\_\{1\} +
\textbackslash{}mathop\{\textbackslash{}mathop\{\ldots{}\}\} +
\{x\}\_\{n\}\{c\}\_\{n\}, soit

\textbackslash{}begin\{eqnarray*\}
\textbackslash{}mathop\{\textbackslash{}mathrm\{det\}\}
(\{c\}\_\{1\},\textbackslash{}mathop\{\textbackslash{}mathop\{\ldots{}\}\},\{c\}\_\{j−1\},b,\{c\}\_\{j+1\},\textbackslash{}mathop\{\textbackslash{}mathop\{\ldots{}\}\},\{c\}\_\{n\})\&\&
\%\& \textbackslash{}\textbackslash{} \& =\&
\textbackslash{}mathop\{\textbackslash{}mathrm\{det\}\}
(\{c\}\_\{1\},\textbackslash{}mathop\{\textbackslash{}mathop\{\ldots{}\}\},\{c\}\_\{j−1\},\{x\}\_\{1\}\{c\}\_\{1\}
+ \textbackslash{}mathop\{\textbackslash{}mathop\{\ldots{}\}\} +
\{x\}\_\{n\}\{c\}\_\{n\},\{c\}\_\{j+1\},\textbackslash{}mathop\{\textbackslash{}mathop\{\ldots{}\}\},\{c\}\_\{n\})\%\&
\textbackslash{}\textbackslash{} \& =\&
\{x\}\_\{j\}\textbackslash{}mathop\{ \textbackslash{}mathrm\{det\}\}
(\{c\}\_\{1\},\textbackslash{}mathop\{\textbackslash{}mathop\{\ldots{}\}\},\{c\}\_\{n\})
\%\& \textbackslash{}\textbackslash{} \textbackslash{}end\{eqnarray*\}

en développant suivant la j-ième colonne (tous les autres déterminants
sont nuls car contenant deux fois la même colonne \{c\}\_\{i\}).

\paragraph{2.8.3 Théorème de Rouché-Fontené}

Définition~2.8.2 On associe au système (L) le système dit homogène

(H)\textbackslash{}quad \textbackslash{}left
\textbackslash{}\{\textbackslash{}matrix\{\textbackslash{},\{a\}\_\{1,1\}\{x\}\_\{1\}
+ \textbackslash{}mathop\{\textbackslash{}mathop\{\ldots{}\}\} +
\{a\}\_\{1,n\}\{x\}\_\{n\} = 0 \textbackslash{}cr
\textbackslash{}mathop\{\textbackslash{}mathop\{\ldots{}\}\}
\textbackslash{}cr \{a\}\_\{m,1\}\{x\}\_\{1\} +
\textbackslash{}mathop\{\textbackslash{}mathop\{\ldots{}\}\} +
\{a\}\_\{m,n\}\{x\}\_\{n\} = 0\}\textbackslash{}right .

On appelle \{S\}\_\{L\} l'ensemble des solutions du système linéaire,
\{S\}\_\{H\} celui du système homogène.

Proposition~2.8.3 \{S\}\_\{H\} est un sous-espace vectoriel de
\{K\}\^{}\{n\} de dimension égale à n
−\textbackslash{}mathop\{\textbackslash{}mathrm\{rg\}\}L. \{S\}\_\{L\}
est soit vide, soit un sous-espace affine de direction \{S\}\_\{H\}~;
dans ce cas on obtient la solution générale de (L) en ajoutant à une
solution particulière, la solution générale de (H).

Démonstration On a \{S\}\_\{H\} =\textbackslash{}mathop\{
\textbackslash{}mathrm\{Ker\}\}u d'où le fait qu'il est un sous-espace
vectoriel~; sa dimension est donnée par le théorème du rang. De plus, si
v ∈ \{S\}\_\{L\}, on a x ∈ \{S\}\_\{L\} \textbackslash{}mathrel\{⇔\}
u(x) = b = u(v) \textbackslash{}mathrel\{⇔\} u(x − v) = 0
\textbackslash{}mathrel\{⇔\} x − v ∈ \{S\}\_\{H\}.

Définition~2.8.3 Soit P = \{(\{a\}\_\{i,j\})\}\_\{i∈I,j∈J\} une
sous-matrice principale de la matrice du système (L). On appelle
déterminants caractéristiques du système associés à la sous-matrice
principale P les m − r déterminants

\{Δ\}\_\{\{i\}\_\{0\}\} =\textbackslash{}mathop\{
\textbackslash{}mathrm\{det\}\} \textbackslash{}left
(\textbackslash{}matrix\{\textbackslash{},\{(\{a\}\_\{i,j\})\}\_\{i∈I,j∈J\}\&\{(\{b\}\_\{i\})\}\_\{i∈I\}
\textbackslash{}cr \{(\{a\}\_\{\{i\}\_\{0\},j\})\}\_\{j∈J\}
\&\{b\}\_\{\{i\}\_\{0\}\} \}\textbackslash{}right )

avec \{i\}\_\{0\} ∈ {[}1,m{]} ∖ I

Théorème~2.8.4 (Rouché-Fontené). Soit P une sous-matrice principale de
la matrice du système (L). Alors le système a des solutions si et
seulement si les m − r déterminants caractéristiques associés à P sont
nuls. Dans ce cas, l'ensemble des solutions de (L) est le même que
l'ensemble des solutions du système L' obtenu en éliminant les équations
non principales. On résout ce système (L'), en donnant des valeurs
arbitraires aux inconnues non principales et en déterminant les valeurs
correspondantes des inconnues principales par la résolution du système
de Cramer (de matrice P) ainsi obtenu.

Démonstration Etudions tout d'abord la compatibilité du système. On a

\textbackslash{}begin\{eqnarray*\}\{
S\}\_\{L\}\textbackslash{}mathrel\{≠\}∅\& \textbackslash{}mathrel\{⇔\}
\& b
∈\textbackslash{}mathop\{\textbackslash{}mathrm\{Vect\}\}(\{c\}\_\{1\},\textbackslash{}mathop\{\textbackslash{}mathop\{\ldots{}\}\},\{c\}\_\{n\})
\%\& \textbackslash{}\textbackslash{} \& \textbackslash{}mathrel\{⇔\} \&
\textbackslash{}mathop\{dim\}
\textbackslash{}mathop\{\textbackslash{}mathrm\{Vect\}\}(\{c\}\_\{1\},\textbackslash{}mathop\{\textbackslash{}mathop\{\ldots{}\}\},\{c\}\_\{n\},b)
=\textbackslash{}mathop\{ dim\}
\textbackslash{}mathop\{\textbackslash{}mathrm\{Vect\}\}(\{c\}\_\{1\},\textbackslash{}mathop\{\textbackslash{}mathop\{\ldots{}\}\},\{c\}\_\{n\})\%\&
\textbackslash{}\textbackslash{} \& \textbackslash{}mathrel\{⇔\} \&
\textbackslash{}mathop\{\textbackslash{}mathrm\{rg\}\}\textbackslash{}left
(\textbackslash{}matrix\{\textbackslash{},A\&B\}\textbackslash{}right )
=\textbackslash{}mathop\{ \textbackslash{}mathrm\{rg\}\}A \%\&
\textbackslash{}\textbackslash{} \textbackslash{}end\{eqnarray*\}

Soit donc P = \{(\{a\}\_\{i,j\})\}\_\{i∈I,j∈J\} une sous-matrice
principale de A (avec \textbar{}I\textbar{} = \textbar{}J\textbar{} =
r). Le système a des solutions si et seulement si P est encore une
sous-matrice principale de \textbackslash{}left
(\textbackslash{}matrix\{\textbackslash{},A\&B\}\textbackslash{}right ),
c'est-à-dire si et seulement si toutes les sous matrices bordantes de P
dans \textbackslash{}left
(\textbackslash{}matrix\{\textbackslash{},A\&B\}\textbackslash{}right )
sont non inversibles~; mais ces matrices bordantes sont de deux types~:
soit des matrices bordantes dans A qui sont forcément non inversibles,
soit des matrices de la forme \textbackslash{}left
(\textbackslash{}matrix\{\textbackslash{},\{(\{a\}\_\{i,j\})\}\_\{i∈I,j∈J\}\&\{(\{b\}\_\{i\})\}\_\{i∈I\}
\textbackslash{}cr \{(\{a\}\_\{\{i\}\_\{0\},j\})\}\_\{j∈J\}
\&\{b\}\_\{\{i\}\_\{0\}\} \}\textbackslash{}right ), avec \{i\}\_\{0\} ∈
{[}1,m{]} ∖ I. On a noté \{Δ\}\_\{\{i\}\_\{0\}\} le déterminant d'une
telle matrice~: les \{Δ\}\_\{\{i\}\_\{0\}\}, i ∈ {[}1,m{]} ∖ I sont les
déterminants caractéristiques du système (il y en a m − r). Le système a
des solutions si et seulement si ces déterminants caractéristiques sont
tous nuls.

Supposons la condition réalisée, soit (L') le système
\textbackslash{}left
\textbackslash{}\{\textbackslash{}matrix\{\textbackslash{},\{a\}\_\{i,1\}\{x\}\_\{1\}
+ \textbackslash{}mathop\{\textbackslash{}mathop\{\ldots{}\}\} +
\{a\}\_\{i,n\}\{x\}\_\{n\} = \{b\}\_\{i\}\}\textbackslash{}right
.,\textbackslash{}quad i ∈ I (système d'équations principales associées
à P). On a clairement \{S\}\_\{L\} ⊂ \{S\}\_\{L'\}. Mais les deux
systèmes ont même rang, si bien que \textbackslash{}mathop\{dim\}
\{S\}\_\{L\} =\textbackslash{}mathop\{ dim\} \{S\}\_\{L'\}. On a donc
\{S\}\_\{L\} = \{S\}\_\{L'\}. Or le système L' se résout facilement en
l'écrivant sous la forme

\{\textbackslash{}mathop\{∑ \}\}\_\{j∈J\}\{a\}\_\{i,j\}\{x\}\_\{j\} =
\{b\}\_\{i\} −\{\textbackslash{}mathop\{∑
\}\}\_\{j\textbackslash{}mathrel\{∉\}J\}\{a\}\_\{i,j\}\{x\}\_\{j\},\textbackslash{}quad
i ∈ I

On donne des valeurs arbitraires aux inconnues \{x\}\_\{j\},
j\textbackslash{}mathrel\{∉\}J (inconnues non principales associées à la
matrice P)~; on obtient alors un système de Cramer de matrice P qui
permet de déterminer les \{x\}\_\{j\}, j ∈ J (les inconnues principales
associées à P).

\paragraph{2.8.4 Méthode du pivot}

Appliquons la méthode du pivot sur les lignes de la matrice A en
effectuant les opérations correspondantes sur la matrice B. On obtient
un nouveau système du type

\textbackslash{}left
\textbackslash{}\{\textbackslash{}matrix\{\textbackslash{},\{α\}\_\{1,\{i\}\_\{1\}\}\{x\}\_\{\{i\}\_\{1\}\}
+ \{α\}\_\{1,\{i\}\_\{1\}+1\}\{x\}\_\{\{i\}\_\{1\}+1\} +
\textbackslash{}quad \textbackslash{}quad
\textbackslash{}mathop\{\textbackslash{}mathop\{\ldots{}\}\}\textbackslash{}quad
\textbackslash{}quad + \{α\}\_\{1,n\}\{x\}\_\{n\} = \{β\}\_\{1\}
\textbackslash{}cr \{α\}\_\{2,\{i\}\_\{2\}\}\{x\}\_\{\{i\}\_\{2\}\} +
\{α\}\_\{2,\{i\}\_\{2\}+1\}\{x\}\_\{\{i\}\_\{2\}+1\} +
\textbackslash{}quad
\textbackslash{}mathop\{\textbackslash{}mathop\{\ldots{}\}\}\textbackslash{}quad
+ \{α\}\_\{2,n\}\{x\}\_\{n\} = \{β\}\_\{2\} \textbackslash{}cr
\textbackslash{}mathop\{\textbackslash{}mathop\{\ldots{}\}\}
\textbackslash{}cr \{α\}\_\{r,\{i\}\_\{r\}\}\{x\}\_\{\{i\}\_\{r\}\} +
\{α\}\_\{1,\{i\}\_\{r\}+1\}\{x\}\_\{\{i\}\_\{r\}+1\} +
\textbackslash{}mathop\{\textbackslash{}mathop\{\ldots{}\}\} +
\{α\}\_\{r,n\}\{x\}\_\{n\} = \{β\}\_\{r\} \textbackslash{}cr 0 =
\{β\}\_\{r+1\} \textbackslash{}cr
\textbackslash{}mathop\{\textbackslash{}mathop\{\ldots{}\}\}
\textbackslash{}cr 0 = \{β\}\_\{m\}\}\textbackslash{}right .

avec \{i\}\_\{1\} \textless{} \{i\}\_\{2\} \textless{}
\textbackslash{}mathop\{\textbackslash{}mathop\{\ldots{}\}\} \textless{}
\{i\}\_\{r\}, \{α\}\_\{k,\{i\}\_\{k\}\}\textbackslash{}mathrel\{≠\}0
pour k ∈ {[}1,r{]}

Un tel système se résout immédiatement~: il a des solutions si et
seulement si \{β\}\_\{r+1\} =
\textbackslash{}mathop\{\textbackslash{}mathop\{\ldots{}\}\} =
\{β\}\_\{m\} = 0. Dans ce cas il est équivalent au système formé des r
premières équations. Celui-ci se résout en donnant des valeurs
arbitraires aux \{x\}\_\{j\} pour
j\textbackslash{}mathrel\{∉\}\textbackslash{}\{\{i\}\_\{1\},\textbackslash{}mathop\{\textbackslash{}mathop\{\ldots{}\}\},\{i\}\_\{r\}\textbackslash{}\}
et en calculant les \{x\}\_\{\{i\}\_\{k\}\} par résolution d'un système
triangulaire supérieur.

Application~: recherche de l'inverse d'une matrice~: AX = Y
\textbackslash{}mathrel\{⇔\} X = \{A\}\^{}\{−1\}Y .

{[}\href{coursse13.html}{prev}{]}
{[}\href{coursse13.html\#tailcoursse13.html}{prev-tail}{]}
{[}\href{coursse14.html}{front}{]}
{[}\href{coursch3.html\#coursse14.html}{up}{]}

\end{document}

% \documentclass[]{article}
\usepackage[T1]{fontenc}
\usepackage{lmodern}
\usepackage{amssymb,amsmath}
\usepackage{ifxetex,ifluatex}
\usepackage{fixltx2e} % provides \textsubscript
% use upquote if available, for straight quotes in verbatim environments
\IfFileExists{upquote.sty}{\usepackage{upquote}}{}
\ifnum 0\ifxetex 1\fi\ifluatex 1\fi=0 % if pdftex
  \usepackage[utf8]{inputenc}
\else % if luatex or xelatex
  \ifxetex
    \usepackage{mathspec}
    \usepackage{xltxtra,xunicode}
  \else
    \usepackage{fontspec}
  \fi
  \defaultfontfeatures{Mapping=tex-text,Scale=MatchLowercase}
  \newcommand{\euro}{€}
\fi
% use microtype if available
\IfFileExists{microtype.sty}{\usepackage{microtype}}{}
\usepackage{graphicx}
% Redefine \includegraphics so that, unless explicit options are
% given, the image width will not exceed the width of the page.
% Images get their normal width if they fit onto the page, but
% are scaled down if they would overflow the margins.
\makeatletter
\def\ScaleIfNeeded{%
  \ifdim\Gin@nat@width>\linewidth
    \linewidth
  \else
    \Gin@nat@width
  \fi
}
\makeatother
\let\Oldincludegraphics\includegraphics
{%
 \catcode`\@=11\relax%
 \gdef\includegraphics{\@ifnextchar[{\Oldincludegraphics}{\Oldincludegraphics[width=\ScaleIfNeeded]}}%
}%
\ifxetex
  \usepackage[setpagesize=false, % page size defined by xetex
              unicode=false, % unicode breaks when used with xetex
              xetex]{hyperref}
\else
  \usepackage[unicode=true]{hyperref}
\fi
\hypersetup{breaklinks=true,
            bookmarks=true,
            pdfauthor={},
            pdftitle={Valeurs propres. Vecteurs propres},
            colorlinks=true,
            citecolor=blue,
            urlcolor=blue,
            linkcolor=magenta,
            pdfborder={0 0 0}}
\urlstyle{same}  % don't use monospace font for urls
\setlength{\parindent}{0pt}
\setlength{\parskip}{6pt plus 2pt minus 1pt}
\setlength{\emergencystretch}{3em}  % prevent overfull lines
\setcounter{secnumdepth}{0}
 
/* start css.sty */
.cmr-5{font-size:50%;}
.cmr-7{font-size:70%;}
.cmmi-5{font-size:50%;font-style: italic;}
.cmmi-7{font-size:70%;font-style: italic;}
.cmmi-10{font-style: italic;}
.cmsy-5{font-size:50%;}
.cmsy-7{font-size:70%;}
.cmex-7{font-size:70%;}
.cmex-7x-x-71{font-size:49%;}
.msbm-7{font-size:70%;}
.cmtt-10{font-family: monospace;}
.cmti-10{ font-style: italic;}
.cmbx-10{ font-weight: bold;}
.cmr-17x-x-120{font-size:204%;}
.cmsl-10{font-style: oblique;}
.cmti-7x-x-71{font-size:49%; font-style: italic;}
.cmbxti-10{ font-weight: bold; font-style: italic;}
p.noindent { text-indent: 0em }
td p.noindent { text-indent: 0em; margin-top:0em; }
p.nopar { text-indent: 0em; }
p.indent{ text-indent: 1.5em }
@media print {div.crosslinks {visibility:hidden;}}
a img { border-top: 0; border-left: 0; border-right: 0; }
center { margin-top:1em; margin-bottom:1em; }
td center { margin-top:0em; margin-bottom:0em; }
.Canvas { position:relative; }
li p.indent { text-indent: 0em }
.enumerate1 {list-style-type:decimal;}
.enumerate2 {list-style-type:lower-alpha;}
.enumerate3 {list-style-type:lower-roman;}
.enumerate4 {list-style-type:upper-alpha;}
div.newtheorem { margin-bottom: 2em; margin-top: 2em;}
.obeylines-h,.obeylines-v {white-space: nowrap; }
div.obeylines-v p { margin-top:0; margin-bottom:0; }
.overline{ text-decoration:overline; }
.overline img{ border-top: 1px solid black; }
td.displaylines {text-align:center; white-space:nowrap;}
.centerline {text-align:center;}
.rightline {text-align:right;}
div.verbatim {font-family: monospace; white-space: nowrap; text-align:left; clear:both; }
.fbox {padding-left:3.0pt; padding-right:3.0pt; text-indent:0pt; border:solid black 0.4pt; }
div.fbox {display:table}
div.center div.fbox {text-align:center; clear:both; padding-left:3.0pt; padding-right:3.0pt; text-indent:0pt; border:solid black 0.4pt; }
div.minipage{width:100%;}
div.center, div.center div.center {text-align: center; margin-left:1em; margin-right:1em;}
div.center div {text-align: left;}
div.flushright, div.flushright div.flushright {text-align: right;}
div.flushright div {text-align: left;}
div.flushleft {text-align: left;}
.underline{ text-decoration:underline; }
.underline img{ border-bottom: 1px solid black; margin-bottom:1pt; }
.framebox-c, .framebox-l, .framebox-r { padding-left:3.0pt; padding-right:3.0pt; text-indent:0pt; border:solid black 0.4pt; }
.framebox-c {text-align:center;}
.framebox-l {text-align:left;}
.framebox-r {text-align:right;}
span.thank-mark{ vertical-align: super }
span.footnote-mark sup.textsuperscript, span.footnote-mark a sup.textsuperscript{ font-size:80%; }
div.tabular, div.center div.tabular {text-align: center; margin-top:0.5em; margin-bottom:0.5em; }
table.tabular td p{margin-top:0em;}
table.tabular {margin-left: auto; margin-right: auto;}
div.td00{ margin-left:0pt; margin-right:0pt; }
div.td01{ margin-left:0pt; margin-right:5pt; }
div.td10{ margin-left:5pt; margin-right:0pt; }
div.td11{ margin-left:5pt; margin-right:5pt; }
table[rules] {border-left:solid black 0.4pt; border-right:solid black 0.4pt; }
td.td00{ padding-left:0pt; padding-right:0pt; }
td.td01{ padding-left:0pt; padding-right:5pt; }
td.td10{ padding-left:5pt; padding-right:0pt; }
td.td11{ padding-left:5pt; padding-right:5pt; }
table[rules] {border-left:solid black 0.4pt; border-right:solid black 0.4pt; }
.hline hr, .cline hr{ height : 1px; margin:0px; }
.tabbing-right {text-align:right;}
span.TEX {letter-spacing: -0.125em; }
span.TEX span.E{ position:relative;top:0.5ex;left:-0.0417em;}
a span.TEX span.E {text-decoration: none; }
span.LATEX span.A{ position:relative; top:-0.5ex; left:-0.4em; font-size:85%;}
span.LATEX span.TEX{ position:relative; left: -0.4em; }
div.float img, div.float .caption {text-align:center;}
div.figure img, div.figure .caption {text-align:center;}
.marginpar {width:20%; float:right; text-align:left; margin-left:auto; margin-top:0.5em; font-size:85%; text-decoration:underline;}
.marginpar p{margin-top:0.4em; margin-bottom:0.4em;}
.equation td{text-align:center; vertical-align:middle; }
td.eq-no{ width:5%; }
table.equation { width:100%; } 
div.math-display, div.par-math-display{text-align:center;}
math .texttt { font-family: monospace; }
math .textit { font-style: italic; }
math .textsl { font-style: oblique; }
math .textsf { font-family: sans-serif; }
math .textbf { font-weight: bold; }
.partToc a, .partToc, .likepartToc a, .likepartToc {line-height: 200%; font-weight:bold; font-size:110%;}
.chapterToc a, .chapterToc, .likechapterToc a, .likechapterToc, .appendixToc a, .appendixToc {line-height: 200%; font-weight:bold;}
.index-item, .index-subitem, .index-subsubitem {display:block}
.caption td.id{font-weight: bold; white-space: nowrap; }
table.caption {text-align:center;}
h1.partHead{text-align: center}
p.bibitem { text-indent: -2em; margin-left: 2em; margin-top:0.6em; margin-bottom:0.6em; }
p.bibitem-p { text-indent: 0em; margin-left: 2em; margin-top:0.6em; margin-bottom:0.6em; }
.paragraphHead, .likeparagraphHead { margin-top:2em; font-weight: bold;}
.subparagraphHead, .likesubparagraphHead { font-weight: bold;}
.quote {margin-bottom:0.25em; margin-top:0.25em; margin-left:1em; margin-right:1em; text-align:justify;}
.verse{white-space:nowrap; margin-left:2em}
div.maketitle {text-align:center;}
h2.titleHead{text-align:center;}
div.maketitle{ margin-bottom: 2em; }
div.author, div.date {text-align:center;}
div.thanks{text-align:left; margin-left:10%; font-size:85%; font-style:italic; }
div.author{white-space: nowrap;}
.quotation {margin-bottom:0.25em; margin-top:0.25em; margin-left:1em; }
h1.partHead{text-align: center}
.sectionToc, .likesectionToc {margin-left:2em;}
.subsectionToc, .likesubsectionToc {margin-left:4em;}
.subsubsectionToc, .likesubsubsectionToc {margin-left:6em;}
.frenchb-nbsp{font-size:75%;}
.frenchb-thinspace{font-size:75%;}
.figure img.graphics {margin-left:10%;}
/* end css.sty */

\title{Valeurs propres. Vecteurs propres}
\author{}
\date{}

\begin{document}
\maketitle

\textbf{Warning: 
requires JavaScript to process the mathematics on this page.\\ If your
browser supports JavaScript, be sure it is enabled.}

\begin{center}\rule{3in}{0.4pt}\end{center}

[
[]
[

\subsubsection{3.1 Valeurs propres. Vecteurs propres}

\paragraph{3.1.1 Sous-espaces stables}

Définition~3.1.1 Soit E un K-espace vectoriel , u \in L(E). On dit qu'un
sous-espace F de E est stable si u(F) \subset~ F.

Remarque~3.1.1 Dans ce cas on peut considérer l'application (évidemment
linéaire) v : F \rightarrow~ F, x\mapsto~u(x). C'est un
endomorphisme de F appelé l'endomorphisme induit par u.

Proposition~3.1.1 Soit E un K-espace vectoriel de dimension finie, F un
sous-espace de E,
(e_1,\\ldots,e_p~)
une base de F complétée en une base \mathcal{E} =
(e_1,\\ldots,e_n~)
de E. Soit u \in L(E). Alors F est stable par u si et seulement si la
matrice de u dans la base \mathcal{E} est de la forme \left (
\includegraphics{cours4x.png} \,\right ).

Démonstration En effet F est stable par u si et seulement si
\forall~j \in [1,p], u(e_j~)
\in\mathrmVect(e_1,\\\ldots,e_p~),
ce que traduit exactement la forme de la matrice.

Remarque~3.1.2 Dans ce cas la matrice A n'est autre que la matrice dans
la base
(e_1,\\ldots,e_p~)
de l'endomorphisme v de F induit par u.

Proposition~3.1.2 Soit E un K-espace vectoriel de dimension finie,
E_1,\\ldots,E_p~
une famille de sous-espaces vectoriels de E tels que E = E_1
\oplus~⋯ \oplus~ E_p, soit \mathcal{E} une base de E
adaptée à cette décomposition en somme directe. Alors chacun des
E_i est stable par u si et seulement si la matrice de u dans la
base \mathcal{E} est de la forme

\left
(\matrix\,A_1& &0
\cr &⋱& \cr 0
& &A_p\right )

Démonstration La même~; la forme de la matrice traduit exactement que

\forall~i \in [1,p], u(\mathcal{E}_i~)
\subset~\mathrmVect(\mathcal{E}_i~)
= E_i

où l'on désigne par \mathcal{E}_i la base de E_i extraite de \mathcal{E}.

Définition~3.1.2 Soit E un K-espace vectoriel de dimension finie n~; on
appelle drapeau de E une suite \0\ =
E_0 \subset~ E_1 \subset~⋯ \subset~ E_n
= E de sous-espaces de E tels que dim~
E_i = i.

Proposition~3.1.3 Soit E un K-espace vectoriel de dimension finie n,
\0\ = E_0 \subset~ E_1
\subset~⋯ \subset~ E_n = E un drapeau de E et \mathcal{E} =
(e_1,\\ldots,e_n~)
une base de E adaptée à ce drapeau (c'est-à-dire que pour tout i \in
[1,n],
(e_1,\\ldots,e_i~)
est une base de E_i). Soit u \in L(E). Alors on a équivalence
de~:

\begin{itemize}
\itemsep1pt\parskip0pt\parsep0pt
\item
  \forall~i \in [1,n], u(E_i~) \subset~
  E_i
\item
  la matrice de u dans la base \mathcal{E} est triangulaire supérieure.
\end{itemize}

Démonstration En effet, on a évidemment au vu des inclusions
E_i-1 \subset~ E_i

\begin{align*} \forall~~i \in
[1,n], u(E_i) \subset~ E_i&& \%&
\\ & \Leftrightarrow &
\forall~i \in [1,p], u(e_i) \in E_i~
=\
\mathrmVect(e_1,\\ldots,e_i~)\%&
\\ \end{align*}

ce que traduit exactement la forme de la matrice.

\paragraph{3.1.2 Valeurs propres, vecteurs propres}

Définition~3.1.3 Soit E un K-espace vectoriel et u \in L(E). On dit que \lambda~
\in K est valeur propre de u s'il existe x \in E,
x\neq~0 tel que u(x) = \lambda~x. On dit alors que x est
un vecteur propre de u associé à la valeur propre \lambda~. L'ensemble des
valeurs propres de u est appelé le spectre de u et noté
\mathrm{Sp}~(u).

Remarque~3.1.3 On a u(x) = \lambda~x \Leftrightarrow (u -
\lambda~\mathrmId_E)(x) = 0. On en déduit que \lambda~ est
valeur propre de u si et seulement si u -
\lambda~\mathrmId_E est non injectif. Ceci amène
aussi à la définition suivante

Définition~3.1.4 Soit \lambda~
\in\mathrm{Sp}~(u). On appelle
sous-espace propre associé à \lambda~ le sous espace vectoriel E_u(\lambda~)
= \mathrmKer~(u -
\lambda~\mathrmId_E) (composé des vecteurs propres
associés à \lambda~ et du vecteur nul).

Remarque~3.1.4 On remarque bien entendu qu'un vecteur propre est associé
à une seule valeur propre (soit E_u(\lambda~) \bigcap E_u(\mu) =
\0\). En fait ce résultat peut être
précisé à l'aide du théorème essentiel suivant

Théorème~3.1.4 Soit E un K-espace vectoriel et u \in L(E). Soit
\lambda_1,\\ldots,\lambda_k~
des valeurs propres distinctes de u. Alors les sous-espaces
E_u(\lambda_i) sont en somme directe.

Démonstration On va démontrer par récurrence sur n que x_1 +
\\ldots~ +
x_n = 0 \rigtharrow~\forall~i, x_i~ = 0 si
x_i \in E_u(\lambda_i). C'est vrai pour n = 1. On
suppose le résultat vrai pour n - 1 et soit x_1 +
\\ldots~ +
x_n = 0. Appliquant u on obtient

\begin{align*} 0& =& u(x_1) +
\\ldots~ +
u(x_n) = \lambda_1x_1 +
\\ldots~ +
\lambda_nx_n\%& \\ & =&
\lambda_1x_1 +
\\ldots~ +
\lambda_nx_n - \lambda_n(x_1 +
\\ldots~ +
x_n) \%& \\ & =& (\lambda_1
- \lambda_n)x_1 +
\\ldots~ +
(\lambda_n-1 - \lambda_n)x_n-1 \%&
\\ \end{align*}

L'hypothèse de récurrence implique que \forall~~i \in
[1,n - 1], (\lambda_i - \lambda_n)x_i = 0 soit
x_i = 0 (car
\lambda_i\neq~\lambda_n). La relation de
départ donne en plus x_n = 0.

On en déduit immédiatement

Corollaire~3.1.5 Soit (x_i)_i\inI une famille de
vecteurs propres de u associés à des valeurs propres \lambda_i deux à
deux distinctes. Alors la famille est libre.

Exemple~3.1.1 La famille d'applications C^\infty~, f_\lambda~ : \mathbb{R}~
\rightarrow~ \mathbb{C}, t\mapsto~e^\lambda~t est composée de
vecteurs propres de l'opérateur de dérivation (dans l'espace vectoriel
des fonctions C^\infty~ de \mathbb{R}~ dans \mathbb{C})~: Df_\lambda~ =
\lambda~f_\lambda~. On en déduit qu'elle est libre.

Enfin le résultat suivant est souvent fort utile

Proposition~3.1.6 Soit u et v deux endomorphismes de E tels que u \cdot v =
v \cdot u. Alors tout sous-espace propre de u est stable par v.

Démonstration Si u(x) = \lambda~x, alors u(v(x)) = v(u(x)) = \lambda~v(x), donc v(x) \in
E_u(\lambda~).

\paragraph{3.1.3 Polynôme caractéristique}

Remarque~3.1.5 Soit E un K-espace vectoriel de dimension finie et u \in
L(E). On a vu que \lambda~ est valeur propre de u si et seulement si
\lambda~\mathrmId_E - u est non injectif, ce qui en
dimension finie signifie que
\mathrm{det}~
(\lambda~\mathrmId_E - u) = 0. On va donc
introduire un polynôme \chi_u(X) tel que
\forall~\lambda~ \in K,\chi_u~(\lambda~)
= \mathrm{det}~
(\lambda~\mathrmId_E - u).

Définition~3.1.5 Soit M \in M_K(n). On appelle polynôme
caractéristique de la matrice M le déterminant \chi_M(X) de la
matrice XI_n - M \in M_K[X](n).

Proposition~3.1.7

\begin{itemize}
\itemsep1pt\parskip0pt\parsep0pt
\item
  (i) si M et M' sont deux matrices semblables, alors \chi_M' =
  \chi_M
\item
  (ii) \chi_^tM = \chi_M
\item
  (iii) \chi_M(X) = X^n
  -\mathrm{tr}(M)X^n-1~
  + \\ldots~ +
  (-1)^n\
  \mathrm{det} M
\end{itemize}

Démonstration

\begin{itemize}
\itemsep1pt\parskip0pt\parsep0pt
\item
  (i) Si M' = P^-1MP, alors XI_n - M' =
  XI_n - P^-1MP = P^-1(XI_n -
  M)P et donc \mathrm{det}~
  (XI_n - M) =\
  \mathrm{det} (XI_n - M').
\item
  (ii) découle de la même fa\ccon de
  ^t(XI_n - M) = XI_n -^tM
\item
  (iii) Le coefficient du terme constant est \chi_M(0)
  = \mathrm{det}~ (-M) =
  (-1)^n\
  \mathrm{det} M. Pour les coefficients de plus haut
  degré, on écrit \chi_M(X) =\
  \sum ~
  _\sigma\inS_n\epsilon(\sigma)\\∏
   _i=1^n(\delta_i^\sigma(i)X -
  a_i,\sigma(i)). Or le degré de
  \∏ ~
  _i=1^n(\delta_i^\sigma(i)X - a_i,\sigma(i))
  est le nombre de points fixes de \sigma, c'est-à-dire soit n pour \sigma =
  \mathrmId, soit inférieur ou égal à n - 2. Donc
  \chi_M(X) =\ \∏
   _i=1^n(X - a_i,i) + R(X) avec
  deg~ R \leq n - 2. Le résultat en découle
  immédiatement.
\end{itemize}

Remarque~3.1.6 La partie (i) nous montre que si u \in L(E) et si \mathcal{E} est une
base de E, le polynôme caractéristique de la matrice
\mathrmMat~ (u,\mathcal{E}) est
indépendant du choix de \mathcal{E}.

Définition~3.1.6 Soit u \in L(E) où dim~ E
< +\infty~. On appelle polynôme caractéristique de u le polynôme
caractéristique de sa matrice dans n'importe quelle base de E.

Proposition~3.1.8

\begin{itemize}
\itemsep1pt\parskip0pt\parsep0pt
\item
  (i) \chi_u(X) = X^n
  -\mathrm{tr}(u)X^n-1~
  + \\ldots~ +
  (-1)^n\
  \mathrm{det} u
\item
  (ii) \chi_^tu = \chi_u
\item
  (iii) \lambda~ \in\mathrm{Sp}~(u)
  \Leftrightarrow \chi_u(\lambda~) = 0
\end{itemize}

Définition~3.1.7 Soit \lambda~ une valeur propre de u. On appelle multiplicité
de \lambda~ le nombre m_u(\lambda~) égal à la multiplicité de \lambda~ comme racine
de \chi_u.

Lemme~3.1.9 Soit u \in L(E) et F un sous-espace vectoriel de E stable par
u. Soit u' : F \rightarrow~ F défini par u'(x) = u(x) pour x \in F. Alors
\chi_u'(X) divise \chi_u(X).

Démonstration Soit ℱ =
(e_1,\\ldots,e_p~)
une base de F que l'on complète en \mathcal{E} =
(e_1,\\ldots,e_n~)
base de E. Alors M =\
\mathrmMat (u,\mathcal{E}) = \left
(\matrix\,A&B\cr 0
&C\right ) où A =\
\mathrmMat (u',ℱ). On a alors par un calcul de
déterminants par blocs \chi_M(X) = \chi_A(X)\chi_C(X)
ce qui montre que \chi_u'(X) = \chi_A(X) divise
\chi_u(X) = \chi_M(X).

Théorème~3.1.10 Soit u \in L(E), \lambda~
\in\mathrm{Sp}~(u),
m_u(\lambda~) la multiplicité de la valeur propre \lambda~ et E_u(\lambda~)
le sous-espace propre associé à \lambda~. Alors 1 \leq\
dim E_u(\lambda~) \leq m_u(\lambda~).

Démonstration E_u(\lambda~) est stable par u et la restriction u' de u
à E_u(\lambda~) est l'homothétie de rapport \lambda~ dont le polynôme
caractéristique est \chi_u'(X) = (X -
\lambda~)^dim E_u(\lambda~)~. Le lemme
précédent implique donc que dim~
E_u(\lambda~) \leq m_u(\lambda~). De plus
E_u(\lambda~)\neq~\0\,
donc 1 \leq dim E_u~(\lambda~).

Remarque~3.1.7 On a donc m_u(\lambda~) = 1 \rigtharrow~\
dim E_u(\lambda~) = 1.

\paragraph{3.1.4 Endomorphismes diagonalisables}

Définition~3.1.8 Soit E un K-espace vectoriel de dimension finie et u \in
L(E). On dit que u est diagonalisable s'il vérifie les conditions
équivalentes

\begin{itemize}
\itemsep1pt\parskip0pt\parsep0pt
\item
  (i) il existe une base \mathcal{E} de E telle que
  \mathrmMat~ (u,\mathcal{E}) soit
  diagonale
\item
  (ii) il existe une base \mathcal{E} de E formée de vecteurs propres de u
\item
  (iii) E est somme (directe) des sous-espaces propres de u
\end{itemize}

Démonstration (i) et (ii) sont évidemment équivalents. Réunissant des
bases des sous-espaces propres de u, on a bien évidemment (iii) \rigtharrow~(ii).
Supposons maintenant que (i) est vrai. Quitte à permuter la base, on
peut supposer que
\mathrmMat~ (u,\mathcal{E})
=\
\mathrmdiag(\lambda_1,\\ldots,\lambda_1,\lambda_2,\\\ldots,\lambda_2,\\\ldots,\\\ldots,\lambda_k,\\\ldots,\lambda_k~)
avec
\lambda_1,\\ldots,\lambda_k~
deux à deux distincts, \lambda_i figurant m_i fois. On a
alors dim E_u(\lambda_i~) ≥
m_i (on a m_i vecteurs de base dans cet espace), soit

dim~ \\oplus~
_\lambda~\in\mathrm{Sp}(u)E_u(\lambda~) =
\sum _i=1^k dim E_
u(\lambda_i) ≥\\sum
_i=1^km_ i = dim E

et donc E = \\oplus~ ~
_\lambda~\in\mathrm{Sp}(u)E_u~(\lambda~).
Donc (i) \rigtharrow~(iii).

Théorème~3.1.11 Soit E un K-espace vectoriel de dimension finie et u \in
L(E). Alors les conditions suivantes sont équivalentes

\begin{itemize}
\itemsep1pt\parskip0pt\parsep0pt
\item
  (i) u est diagonalisable
\item
  (ii) \chi_u(X) est scindé sur K et pour toute valeur propre \lambda~ de
  u la dimension du sous-espace propre associé est égale à la
  multiplicité de la valeur propre.
\end{itemize}

Démonstration Supposons u diagonalisable et soit \mathcal{E} une base de E telle
que \mathrmMat~ (u,\mathcal{E}) = D
=\
\mathrmdiag(\lambda_1,\\ldots,\lambda_1,\lambda_2,\\\ldots,\lambda_2,\\\ldots,\\\ldots,\lambda_k,\\\ldots,\lambda_k~)
avec
\lambda_1,\\ldots,\lambda_k~
deux à deux distincts, \lambda_i figurant m_i fois. Alors
\chi_u(X) = \chi_D(X) =\
∏  _i=1^k~(X -
\lambda_i)^m_i ce qui montre déjà que \chi_u
est scindé et que les valeurs propres de u sont exactement
\lambda_1,\\ldots,\lambda_k~.
De plus dim E_u(\lambda_i~) ≥
m_i = m_u(\lambda_i) puisque
E_u(\lambda_i) contient une famille libre de cardinal
m_i. On a donc dim~
E_u(\lambda_i) = m_u(\lambda_i), soit (i) \rigtharrow~(ii).
Inversement supposons (ii) vérifié. On a alors

\begin{align*} dim~
\oplus~ _i=1^kE_
u(\lambda_i)& =& \\sum
_i=1^km_ u(\lambda_i) = deg
\chi_u(X)\%& \\ & =&
dim~ E \%& \\
\end{align*}

puisque le polynôme est scindé. Soit E =\
\oplus~ ~
_i=1^kE_u(\lambda_i).

Corollaire~3.1.12 Soit E un K-espace vectoriel de dimension finie et u \in
L(E) tel que \chi_u soit scindé à racines simples. Alors u est
diagonalisable.

Remarque~3.1.8 Pratique de la diagonalisation

\begin{itemize}
\itemsep1pt\parskip0pt\parsep0pt
\item
  (i) calculer le polynôme caractéristique et en chercher les racines
  avec leurs multiplicités
\item
  (ii) pour chaque racine déterminer le sous-espace propre
  correspondant, défini par l'équation (u -
  \lambda~\mathrmId)(x) = 0~; comparer dimension du
  sous-espace propre et multiplicité de la valeur propre
\item
  (iii) déterminer une base de chaque sous-espace propre et les réunir
  en une base de E.
\end{itemize}

\paragraph{3.1.5 Matrices diagonalisables}

Définition~3.1.9 Soit M \in M_K(n). On définit de manière
évidente les valeurs propres et vecteurs propres de M~: MX = \lambda~X avec
X\neq~0.

Définition~3.1.10 Soit M \in M_K(n). On dit que M est
diagonalisable si elle vérifie les conditions équivalentes

\begin{itemize}
\itemsep1pt\parskip0pt\parsep0pt
\item
  (i) M est la matrice d'un endomorphisme diagonalisable dans une
  certaine base
\item
  (ii) M est semblable à une matrice diagonale
\item
  (iii) il existe une base de K^n ∼ M_K(n,1) formée
  de vecteurs propres de M
\item
  (iii) K^n ∼ M_K(n,1) est somme directe des
  sous-espaces propres de M
\end{itemize}

Démonstration Tout ceci est élémentaire.

On a immédiatement

Théorème~3.1.13 Soit M \in M_K(n). Alors les conditions suivantes
sont équivalentes

\begin{itemize}
\itemsep1pt\parskip0pt\parsep0pt
\item
  (i) M est diagonalisable
\item
  (ii) \chi_M(X) est scindé sur K et pour toute valeur propre \lambda~ de
  M la dimension du sous-espace propre associé est égale à la
  multiplicité de la valeur propre.
\end{itemize}

Corollaire~3.1.14 Soit M \in M_K(n) telle que \chi_M soit
scindé à racines simples. Alors M est diagonalisable.

Remarque~3.1.9 Pratique de la diagonalisation

\begin{itemize}
\itemsep1pt\parskip0pt\parsep0pt
\item
  (i) calculer le polynôme caractéristique et en chercher les racines
\item
  (ii) pour chaque racine déterminer le sous-espace propre
  correspondant, défini par l'équation (M - \lambda~I_n)X = 0~; ceci
  conduit à un système homogène de rang r_M(\lambda~)~; on a
  dim E_M(\lambda~) = n - r_M~(\lambda~)~;
  comparer dimension du sous-espace propre et multiplicité de la valeur
  propre
\item
  (iii) déterminer une base de chaque sous-espace propre~; soit P la
  matrice qui admet ces vecteurs propres comme vecteurs colonnes~; alors
  P^-1MP est diagonale.
\end{itemize}

\paragraph{3.1.6 Endomorphismes et matrices trigonalisables}

Définition~3.1.11 Soit E un K-espace vectoriel de dimension finie. On
dit que u est trigonalisable s'il vérifie les conditions équivalentes

\begin{itemize}
\itemsep1pt\parskip0pt\parsep0pt
\item
  (i) il existe une base \mathcal{E} de E telle que
  \mathrmMat~ (u,\mathcal{E}) soit
  triangulaire (supérieure)
\item
  (ii) il existe une base \mathcal{E} de E telle que \forall~~i,
  u(e_i)
  \in\mathrmVect(e_1,\\\ldots,e_i~)
\item
  (iii) il existe une suite \0\ =
  F_0 \subset~ F_1 \subset~⋯ \subset~
  F_n = E de sous-espaces de E tels que
  dim F_i = i et u(F_i~) \subset~
  F_i.
\end{itemize}

Démonstration

\begin{itemize}
\itemsep1pt\parskip0pt\parsep0pt
\item
  (i) et (ii) sont trivialement équivalents
\item
  (i) \rigtharrow~(iii)~: prendre F_i =\
  \mathrmVect(e_1,\\ldots,e_i~)
\item
  (iii) \rigtharrow~(i)~: construire par applications successives du théorème de la
  base incomplète une base
  (e_1,\\ldots,e_n~)
  telle que F_i =\
  \mathrmVect(e_1,\\ldots,e_i~).
\end{itemize}

Théorème~3.1.15 Soit E un K-espace vectoriel de dimension finie et u \in
L(E). Alors les conditions suivantes sont équivalentes

\begin{itemize}
\itemsep1pt\parskip0pt\parsep0pt
\item
  (i) u est trigonalisable
\item
  (ii) \chi_u(X) est scindé sur K (ce qui est automatiquement
  vérifié si K est algébriquement clos)
\end{itemize}

Démonstration

\begin{itemize}
\itemsep1pt\parskip0pt\parsep0pt
\item
  (i) \rigtharrow~(ii)~: si M =\
  \mathrmMat (u,\mathcal{E}) = \left
  (\matrix\,a_1,1&\\ldots~
  &\\ldots&\\\ldots~
  \cr 0
  &a_2,2&\\ldots&\\\ldots~
  \cr &
  &⋱&\\ldots~
  \cr 0
  &\\ldots~
  &0&a_n,n\right ), on a \chi_u(X) =
  \chi_M(X) =\ \∏
   _i=1^n(X - a_i,i). Donc \chi_u est
  scindé.
\item
  (ii) \rigtharrow~ (i). Par récurrence sur n~; il n'y a rien à démontrer si n = 1.
  Supposons \chi_u scindé, et soit \lambda~ une racine de \chi_u.
  Soit e_1 un vecteur propre associé à \lambda~, que l'on complète en
  (e_1,\\ldots,e_n~)
  base de E. Soit F =\
  \mathrmVect(e_2,\\ldots,e_n~),
  p la projection sur F parallèlement à Ke_1 et v : F \rightarrow~ F
  défini par v(x) = p(u(x)) si x \in F. Alors M =\
  \mathrmMat (u,\mathcal{E}) = \left
  (\matrix\,\lambda~&∗∗∗ \cr
  \matrix\,0 \cr
  \⋮~
  \cr 0&A \right ) avec A
  = \mathrmMat~
  (v,(e_2,\\ldots,e_n~)).
  On en déduit que \chi_u(X) = (X - \lambda~)\chi_v(X). Donc
  \chi_v est aussi scindé. Par hypothèse de récurrence, il existe
  une base
  (\epsilon_2,\\ldots,\epsilon_n~)
  de F telle que \mathrmMat~
  (v,(\epsilon_2,\\ldots,\epsilon_n~))
  soit triangulaire supérieure et alors
  \mathrmMat~
  (u,(e_1,\epsilon_2,\\ldots,\epsilon_n~))
  = \left (\matrix\,\lambda~&∗
  \cr \matrix\,0
  \cr
  \⋮~
  \cr
  0&\mathrmMat~
  (v,(\epsilon_2,\\ldots,\epsilon_n))~\right
  ) est triangulaire supérieure.
\end{itemize}

Remarque~3.1.10 Comme pour la diagonalisation, ces notions passent
immédiatement aux matrices

Définition~3.1.12 Soit M \in M_K(n). On dit que M est
trigonalisable si elle vérifie les conditions équivalentes

\begin{itemize}
\itemsep1pt\parskip0pt\parsep0pt
\item
  (i) M est la matrice d'un endomorphisme trigonalisable dans une
  certaine base
\item
  (ii) M est semblable à une matrice triangulaire (supérieure).
\end{itemize}

Théorème~3.1.16 Soit M \in M_K(n). Alors les conditions suivantes
sont équivalentes

\begin{itemize}
\itemsep1pt\parskip0pt\parsep0pt
\item
  (i) M est trigonalisable
\item
  (ii) \chi_M(X) est scindé sur K (ce qui est automatiquement
  vérifié si K est algébriquement clos)
\end{itemize}

Corollaire~3.1.17 L'ensemble des matrices diagonalisables est dense dans
M_\mathbb{C}(n).

Démonstration Soit M \in M_\mathbb{C}(n) et P inversible telle que

P^-1MP = T = \left
(\matrix\,a_1,1&\\ldots~
&\\ldots&\\\ldots~
\cr 0
&a_2,2&\\ldots&\\\ldots~
\cr &
&⋱&\\ldots~
\cr 0
&\\ldots~
&0&a_n,n\right )

et posons pour p \in \mathbb{N}~,

T_p = \left
(\matrix\,a_1,1 + 1
\over p
&\\ldots~
&\\ldots&\\\ldots~
\cr 0 &a_2,2 + 1 \over
p^2
&\\ldots&\\\ldots~
\cr &
&⋱&\\ldots~
\cr 0
&\\ldots~
&0&a_n,n + 1 \over p^n
\right )

Il n'y a qu'un nombre fini de p pour lesquels on peut avoir
a_i,i + 1 \over p^i =
a_j,j + 1 \over p^j (il s'agit en
effet d'une équation polynomiale en  1 \over p ). On
en déduit que pour tous les p sauf en nombre fini, T_p a un
polynôme caractéristique scindé à racines simples, donc est
diagonalisable. Il en est donc de même de M_p =
PT_pP^-1. Or
lim_p\rightarrow~+\infty~M_p~ =
PTP^-1 = M. Donc M est limite d'une suite de matrices
diagonalisables.

[
[

\end{document}

% \documentclass[]{article}
\usepackage[T1]{fontenc}
\usepackage{lmodern}
\usepackage{amssymb,amsmath}
\usepackage{ifxetex,ifluatex}
\usepackage{fixltx2e} % provides \textsubscript
% use upquote if available, for straight quotes in verbatim environments
\IfFileExists{upquote.sty}{\usepackage{upquote}}{}
\ifnum 0\ifxetex 1\fi\ifluatex 1\fi=0 % if pdftex
  \usepackage[utf8]{inputenc}
\else % if luatex or xelatex
  \ifxetex
    \usepackage{mathspec}
    \usepackage{xltxtra,xunicode}
  \else
    \usepackage{fontspec}
  \fi
  \defaultfontfeatures{Mapping=tex-text,Scale=MatchLowercase}
  \newcommand{\euro}{€}
\fi
% use microtype if available
\IfFileExists{microtype.sty}{\usepackage{microtype}}{}
\ifxetex
  \usepackage[setpagesize=false, % page size defined by xetex
              unicode=false, % unicode breaks when used with xetex
              xetex]{hyperref}
\else
  \usepackage[unicode=true]{hyperref}
\fi
\hypersetup{breaklinks=true,
            bookmarks=true,
            pdfauthor={},
            pdftitle={Polynomes d'endomorphismes},
            colorlinks=true,
            citecolor=blue,
            urlcolor=blue,
            linkcolor=magenta,
            pdfborder={0 0 0}}
\urlstyle{same}  % don't use monospace font for urls
\setlength{\parindent}{0pt}
\setlength{\parskip}{6pt plus 2pt minus 1pt}
\setlength{\emergencystretch}{3em}  % prevent overfull lines
\setcounter{secnumdepth}{0}
 
/* start css.sty */
.cmr-5{font-size:50%;}
.cmr-7{font-size:70%;}
.cmmi-5{font-size:50%;font-style: italic;}
.cmmi-7{font-size:70%;font-style: italic;}
.cmmi-10{font-style: italic;}
.cmsy-5{font-size:50%;}
.cmsy-7{font-size:70%;}
.cmex-7{font-size:70%;}
.cmex-7x-x-71{font-size:49%;}
.msbm-7{font-size:70%;}
.cmtt-10{font-family: monospace;}
.cmti-10{ font-style: italic;}
.cmbx-10{ font-weight: bold;}
.cmr-17x-x-120{font-size:204%;}
.cmsl-10{font-style: oblique;}
.cmti-7x-x-71{font-size:49%; font-style: italic;}
.cmbxti-10{ font-weight: bold; font-style: italic;}
p.noindent { text-indent: 0em }
td p.noindent { text-indent: 0em; margin-top:0em; }
p.nopar { text-indent: 0em; }
p.indent{ text-indent: 1.5em }
@media print {div.crosslinks {visibility:hidden;}}
a img { border-top: 0; border-left: 0; border-right: 0; }
center { margin-top:1em; margin-bottom:1em; }
td center { margin-top:0em; margin-bottom:0em; }
.Canvas { position:relative; }
li p.indent { text-indent: 0em }
.enumerate1 {list-style-type:decimal;}
.enumerate2 {list-style-type:lower-alpha;}
.enumerate3 {list-style-type:lower-roman;}
.enumerate4 {list-style-type:upper-alpha;}
div.newtheorem { margin-bottom: 2em; margin-top: 2em;}
.obeylines-h,.obeylines-v {white-space: nowrap; }
div.obeylines-v p { margin-top:0; margin-bottom:0; }
.overline{ text-decoration:overline; }
.overline img{ border-top: 1px solid black; }
td.displaylines {text-align:center; white-space:nowrap;}
.centerline {text-align:center;}
.rightline {text-align:right;}
div.verbatim {font-family: monospace; white-space: nowrap; text-align:left; clear:both; }
.fbox {padding-left:3.0pt; padding-right:3.0pt; text-indent:0pt; border:solid black 0.4pt; }
div.fbox {display:table}
div.center div.fbox {text-align:center; clear:both; padding-left:3.0pt; padding-right:3.0pt; text-indent:0pt; border:solid black 0.4pt; }
div.minipage{width:100%;}
div.center, div.center div.center {text-align: center; margin-left:1em; margin-right:1em;}
div.center div {text-align: left;}
div.flushright, div.flushright div.flushright {text-align: right;}
div.flushright div {text-align: left;}
div.flushleft {text-align: left;}
.underline{ text-decoration:underline; }
.underline img{ border-bottom: 1px solid black; margin-bottom:1pt; }
.framebox-c, .framebox-l, .framebox-r { padding-left:3.0pt; padding-right:3.0pt; text-indent:0pt; border:solid black 0.4pt; }
.framebox-c {text-align:center;}
.framebox-l {text-align:left;}
.framebox-r {text-align:right;}
span.thank-mark{ vertical-align: super }
span.footnote-mark sup.textsuperscript, span.footnote-mark a sup.textsuperscript{ font-size:80%; }
div.tabular, div.center div.tabular {text-align: center; margin-top:0.5em; margin-bottom:0.5em; }
table.tabular td p{margin-top:0em;}
table.tabular {margin-left: auto; margin-right: auto;}
div.td00{ margin-left:0pt; margin-right:0pt; }
div.td01{ margin-left:0pt; margin-right:5pt; }
div.td10{ margin-left:5pt; margin-right:0pt; }
div.td11{ margin-left:5pt; margin-right:5pt; }
table[rules] {border-left:solid black 0.4pt; border-right:solid black 0.4pt; }
td.td00{ padding-left:0pt; padding-right:0pt; }
td.td01{ padding-left:0pt; padding-right:5pt; }
td.td10{ padding-left:5pt; padding-right:0pt; }
td.td11{ padding-left:5pt; padding-right:5pt; }
table[rules] {border-left:solid black 0.4pt; border-right:solid black 0.4pt; }
.hline hr, .cline hr{ height : 1px; margin:0px; }
.tabbing-right {text-align:right;}
span.TEX {letter-spacing: -0.125em; }
span.TEX span.E{ position:relative;top:0.5ex;left:-0.0417em;}
a span.TEX span.E {text-decoration: none; }
span.LATEX span.A{ position:relative; top:-0.5ex; left:-0.4em; font-size:85%;}
span.LATEX span.TEX{ position:relative; left: -0.4em; }
div.float img, div.float .caption {text-align:center;}
div.figure img, div.figure .caption {text-align:center;}
.marginpar {width:20%; float:right; text-align:left; margin-left:auto; margin-top:0.5em; font-size:85%; text-decoration:underline;}
.marginpar p{margin-top:0.4em; margin-bottom:0.4em;}
.equation td{text-align:center; vertical-align:middle; }
td.eq-no{ width:5%; }
table.equation { width:100%; } 
div.math-display, div.par-math-display{text-align:center;}
math .texttt { font-family: monospace; }
math .textit { font-style: italic; }
math .textsl { font-style: oblique; }
math .textsf { font-family: sans-serif; }
math .textbf { font-weight: bold; }
.partToc a, .partToc, .likepartToc a, .likepartToc {line-height: 200%; font-weight:bold; font-size:110%;}
.chapterToc a, .chapterToc, .likechapterToc a, .likechapterToc, .appendixToc a, .appendixToc {line-height: 200%; font-weight:bold;}
.index-item, .index-subitem, .index-subsubitem {display:block}
.caption td.id{font-weight: bold; white-space: nowrap; }
table.caption {text-align:center;}
h1.partHead{text-align: center}
p.bibitem { text-indent: -2em; margin-left: 2em; margin-top:0.6em; margin-bottom:0.6em; }
p.bibitem-p { text-indent: 0em; margin-left: 2em; margin-top:0.6em; margin-bottom:0.6em; }
.paragraphHead, .likeparagraphHead { margin-top:2em; font-weight: bold;}
.subparagraphHead, .likesubparagraphHead { font-weight: bold;}
.quote {margin-bottom:0.25em; margin-top:0.25em; margin-left:1em; margin-right:1em; text-align:\jmathustify;}
.verse{white-space:nowrap; margin-left:2em}
div.maketitle {text-align:center;}
h2.titleHead{text-align:center;}
div.maketitle{ margin-bottom: 2em; }
div.author, div.date {text-align:center;}
div.thanks{text-align:left; margin-left:10%; font-size:85%; font-style:italic; }
div.author{white-space: nowrap;}
.quotation {margin-bottom:0.25em; margin-top:0.25em; margin-left:1em; }
h1.partHead{text-align: center}
.sectionToc, .likesectionToc {margin-left:2em;}
.subsectionToc, .likesubsectionToc {margin-left:4em;}
.subsubsectionToc, .likesubsubsectionToc {margin-left:6em;}
.frenchb-nbsp{font-size:75%;}
.frenchb-thinspace{font-size:75%;}
.figure img.graphics {margin-left:10%;}
/* end css.sty */

\title{Polynomes d'endomorphismes}
\author{}
\date{}

\begin{document}
\maketitle

\textbf{Warning: 
requires JavaScript to process the mathematics on this page.\\ If your
browser supports JavaScript, be sure it is enabled.}

\begin{center}\rule{3in}{0.4pt}\end{center}

{[}
{[}
{[}{]}
{[}

\subsubsection{3.2 Polynômes d'endomorphismes}

\paragraph{3.2.1 Généralités}

Soit E un K-espace vectoriel et u \in L(E). On pose u^0 =
\mathrmId\_E et pour k ≥ 1, u^k = u
\cdot u^k-1. Si P \in K{[}X{]}, P =\
\sum ~
\_k=0^da\_kX^k, on pose

P(u) = \sum \_k=0^da~\_
ku^k

Proposition~3.2.1 L'application P\mapsto~P(u) est un
morphisme de K-algèbres de K{[}X{]} dans L(E). Son image K{[}u{]} est la
plus petite sous-algèbre de L(E) contenant u~; elle est commutative.

Démonstration Vérifications immédiates.

\paragraph{3.2.2 Idéal annulateur. Polynôme minimal}

Définition~3.2.1 On appelle idéal annulateur de u \in L(E) l'idéal
I\_u = \P \in
K{[}X{]}∣P(u) = 0\. On dit
que u admet un polynôme minimal si
I\_u\neq~\0\.
Dans ce cas I\_u est engendré par un unique polynôme normalisé
\mu\_u(X) appelé le polynôme minimal de u.

Exemple~3.2.1 Si E = K{[}X{]} et u :
P(X)\mapsto~XP(X), on a Q(u) :
P(X)\mapsto~Q(X)P(X) soit
Q(u)\neq~0 et donc u n'admet pas de polynôme
minimal. Ce phénomène ne peut pas se produire en dimension finie

Théorème~3.2.2 Soit E un K-espace vectoriel de dimension finie et u \in
L(E). Alors u admet un polynôme minimal.

Démonstration Premier argument~: comme dim~
K{[}X{]} = +\infty~ et dim~ L(E) \textless{} +\infty~,
l'application P(X)\mapsto~P(u) ne peut être
in\jmathective. Deuxième argument~: comme dim~ L(E)
= n^2 (où n = dim~ E), la famille de
cardinal n^2 + 1,
(u^k)\_0\leqk\leqn^2 doit être liée~; il existe
donc
\lambda~\_0,\\ldots,\lambda~\_n^2~
non tous nuls tels que \lambda~\_0u^0 +
\\ldots~ +
\lambda~\_n^2u^n^2  = 0~; le polynôme
\lambda~\_01 +
\\ldots~ +
\lambda~\_n^2X^n^2  n'est pas nul et
il est dans I\_u.

On suppose désormais dim~ E \textless{} +\infty~

Proposition~3.2.3 Soit F un sous-espace de E stable par u et v
l'endomorphisme de F induit par u. Alors \mu\_v divise
\mu\_u.

Démonstration On vérifie facilement que pour tout k dans \mathbb{N}~,
v^k est l'endomorphisme de F induit par u^k, donc
si P(X) \in K{[}X{]}, P(v) l'endomorphisme de F induit par P(u). En
particulier \mu\_u(v) est l'endomorphisme de F induit par
\mu\_u(u) = 0 donc \mu\_u(v) = 0 et \mu\_v divise
\mu\_u.

Théorème~3.2.4 Les racines de \mu\_u sont exactement les valeurs
propres de u.

Démonstration Soit \lambda~ une valeur propre de u et x un vecteur propre
associé. On a \forall~k \in \mathbb{N}~, u^k~(x) =
\lambda~^kx et donc \forall~~P \in K{[}X{]}, P(u)(x)
= P(\lambda~)x. En particulier 0 = \mu\_u(u)(x) = \mu\_u(\lambda~)x et
comme x\neq~0, on a \mu\_u(\lambda~) = 0.
Inversement soit \lambda~ une racine de \mu\_u et écrivons
\mu\_u(X) = (X - \lambda~)Q(X). Supposons que \lambda~ n'est pas valeur propre
de u. On a 0 = \mu\_u(u) = (u -
\lambda~\mathrmId\_E) \cdot Q(u). Mais comme \lambda~ n'est pas
valeur propre de u, u - \lambda~\mathrmId\_E est
in\jmathectif et donc Q(u) = 0. Ceci impose que \mu\_u divise Q ce qui
est impossible pour une question de degrés.

\paragraph{3.2.3 Théorème de Cayley-Hamilton}

Théorème~3.2.5 Soit E un K-espace vectoriel de dimension finie et u \in
L(E). Alors \chi\_u(u) = 0.

Remarque~3.2.1 Une version équivalente est~: soit M \in M\_K(n).
Alors \chi\_M(M) = 0.

Démonstration Démonstration 1. Soit x \in E,
x\neq~0 et soit d =\
max\k∣(x,u(x),\\ldots,u^k-1~(x))\text
libre \. On a alors u^d(x) = \lambda~\_0x
+ \\ldots~ +
\lambda~\_d-1u^d-1(x). Soit E\_x
=\
\mathrmVect(x,u(x),\\ldots,u^d-1~(x)).
Alors E\_x est stable par u. Soit v la restriction de u à
E\_x. \mathcal{E}\_x =
(x,u(x),\\ldots,u^d-1~(x))
est une base de E\_x et

\mathrmMat (v,\mathcal{E}\_x~)
= \left
(\matrix\,0&0&\\ldots&\\\ldots&\lambda~\_0~
\cr
1&0&\\ldots&\\\ldots&\lambda~\_1~
\cr
\\ldots&⋱&\mathrel⋱&\\\ldots&\\\ldots~
\cr
\\ldots&\\\ldots&⋱&\mathrel⋱&\\\ldots~
\cr
0&\\ldots&\\\ldots&1&\lambda~\_d-1~\right
)

Un calcul simple montre que \chi\_v(X) = X^d -
\lambda~\_d-1X^d-1
-\\ldots~ -
\lambda~\_0. On a donc \chi\_v(v)(x) = 0 et donc
\chi\_v(u)(x) = 0. Mais \chi\_v divise \chi\_u et donc on
a aussi \chi\_u(u)(x) = 0. Comme x est quelconque, la relation
\chi\_u(u)(x) = 0 étant évidente si x = 0, on a \chi\_u(u) =
0.

Démonstration 2. Montrons que si K est un sous-corps de \mathbb{C} et M \in
M\_K(n), alors \chi\_M(M) = 0. Il suffit bien entendu de le
montrer lorsque M \in M\_\mathbb{C}(n). Si M est diagonalisable, on a M =
P^-1DP avec D =\
\mathrmdiag(\lambda~\_1,\\ldots,\lambda~\_n~).
On a alors \chi\_M(M) = \chi\_D(M) =
P^-1\chi\_D(D)P =
P^-1\
\mathrmdiag(\chi\_D(\lambda~\_1),\\ldots,\chi\_D(\lambda~\_n~))P
= P^-10P = 0. Si M n'est pas diagonalisable, soit
M\_n une suite de matrices diagonalisables qui converge vers M.
Comme l'application A\mapsto~\chi\_A(A) est
polynomiale en les coefficients de A, elle est continue et on a

\chi\_M(M) =\
lim\chi\_M\_n(M\_n) =\
lim0 = 0

Corollaire~3.2.6 Soit E un K-espace vectoriel de dimension finie et u \in
L(E). Alors \mu\_u divise \chi\_u.

\paragraph{3.2.4 Polynôme annulateur et trigonalisation}

Théorème~3.2.7 Soit E un K-espace vectoriel de dimension finie et u \in
L(E). Alors u est trigonalisable si et seulement si il existe un
polynôme P \in K{[}X{]} \diagdown 0, scindé sur K tel que P(u) = 0.

Démonstration Si u est trigonalisable, le polynôme caractéristique
\chi\_u de u est scindé et vérifie \chi\_u(u) = 0.

Inversement, supposons qu'il existe un polynôme non nul, scindé P tel
que P(u) = 0. On peut supposer que P est normalisé si bien que l'on peut
écrire P = \∏ ~
\_i=1^k(X - \lambda~\_i). On a 0
= \∏ ~
\_i=1^k(u - \lambda~\_i\mathrmId), si
bien que l'un au moins des u - \lambda~\_i\mathrmId
est non in\jmathectif. Donc u possède au moins une valeur propre.

Montrons donc le résultat par récurrence sur n =\
dim E~; il n'y a rien à démontrer si n = 1. Soit \lambda~ une valeur propre
de u. Soit e\_1 un vecteur propre associé à \lambda~, que l'on complète
en
(e\_1,\\ldots,e\_n~)
base de E. Soit F =\
\mathrmVect(e\_2,\\ldots,e\_n~),
p la pro\jmathection sur F parallèlement à Ke\_1 et v : F \rightarrow~ F défini
par v(x) = p(u(x)) si x \in F. Alors M =\
\mathrmMat (u,\mathcal{E}) = \left
(\matrix\,\lambda~&∗∗∗ \cr
\matrix\,0 \cr
\⋮~ \cr
0&A \right ) avec A =\
\mathrmMat
(v,(e\_2,\\ldots,e\_n~)).
Le produit de matrices par blocs et une récurrence élémentaire montrent
que \forall~~p \in \mathbb{N}~,

M^p = \left
(\matrix\,\lambda~&∗∗∗ \cr
\matrix\,0 \cr
\⋮~ \cr
0&A^p \right )

et par combinaisons linéaires que

P(M) = \left
(\matrix\,\lambda~&∗ ∗ ∗ \cr
\matrix\,0 \cr
\⋮~ \cr
0&P(A)\right )

On en déduit que P(A) = 0 et comme P(A) =\
\mathrmMat
(P(v),(e\_2,\\ldots,e\_n~)),
on a aussi P(v) = 0. Par hypothèse de récurrence, il existe une base
(\epsilon\_2,\\ldots,\epsilon\_n~)
de F telle que \mathrmMat~
(v,(\epsilon\_2,\\ldots,\epsilon\_n~))
soit triangulaire supérieure et alors
\mathrmMat~
(u,(e\_1,\epsilon\_2,\\ldots,\epsilon\_n~))
= \left (\matrix\,\lambda~&∗
\cr \matrix\,0
\cr \⋮~
\cr
0&\mathrmMat~
(v,(\epsilon\_2,\\ldots,\epsilon\_n))~\right
) est triangulaire supérieure.

\paragraph{3.2.5 Décomposition des noyaux}

Théorème~3.2.8 Soit E un K-espace vectoriel et u \in L(E). Soit P \in
K{[}X{]} et P =
P\_1\\ldotsP\_k~
une décomposition de P en produit de polynômes deux à deux premiers
entre eux. Alors
\mathrmKer~P(u)
= \mathrmKerP\_1~(u)
\oplus~⋯
\oplus~\mathrmKerP\_k~(u).

Démonstration Par récurrence sur k ≥ 2. Pour k = 2, écrivons P =
P\_1P\_2 avec P\_1 et P\_2 premiers
entre eux. Soit U et V tels que UP\_1 + V P\_2 = 1
(Bezout). On a dé\jmathà
\mathrmKerP\_i~(u)
\subset~\mathrmKer~P(u). De plus on
a

U(u)P\_1(u) + V (u)P\_2(u) =
\mathrmId\_E

Soit x
\in\mathrmKerP\_1~(u)
\bigcap\mathrmKerP\_2~(u).
On a x = U(u)P\_1(u)(x) + V (u)P\_2(u)(x) = 0 + 0 = 0,
donc
\mathrmKerP\_1~(u)
\bigcap\mathrmKerP\_2~(u)
= \0\. Soit maintenant x
\in\mathrmKer~P(u). On a
encore x = x\_1 + x\_2 avec x\_1 = V
(u)P\_2(u)(x) et x\_2 = U(u)P\_1(u)(x). Mais
alors

\begin{align*} P\_1(u)(x\_1)& =&
P\_1(u)V (u)P\_2(u)(x) = V
(u)(P\_1P\_2)(u)(x)\%& \\
& =& V (u)P(u)(x) = 0 \%& \\
\end{align*}

donc x\_1
\in\mathrmKerP\_1~(u).
De même x\_2
\in\mathrmKerP\_2~(u)
et donc \mathrmKer~P(u)
= \mathrmKerP\_1~(u)
\oplus~\mathrmKerP\_2~(u).
Supposons donc le résultat vrai pour k - 1. Comme P\_k et
P\_1\\ldotsP\_k-1~
sont premiers entre eux, le cas k = 2 nous donne
\mathrmKer~P(u)
=\
\mathrmKerP\_1\\ldotsP\_k-1~(u)
\oplus~\mathrmKerP\_k~(u)
puis grâce à l'hypothèse de récurrence
\mathrmKer~P(u)
= \mathrmKerP\_1~(u)
\oplus~⋯
\oplus~\mathrmKerP\_k~(u).

Corollaire~3.2.9 Soit E un K-espace vectoriel et u \in L(E). Soit P \in
K{[}X{]} et P =
P\_1\\ldotsP\_k~
une décomposition de P en produit de polynômes deux à deux premiers
entre eux. On suppose que P(u) = 0. Alors E =\
\mathrmKerP\_1(u)
\oplus~⋯
\oplus~\mathrmKerP\_k~(u).

Proposition~3.2.10

\begin{itemize}
\itemsep1pt\parskip0pt\parsep0pt
\item
  (i) Soit E un K-espace vectoriel et u \in L(E) tel que u^2 =
  u. Alors u est un pro\jmathecteur
\item
  (ii) Soit E un K-espace vectoriel et u \in L(E) tel que u^2 =
  \mathrmId. Si
  carK\mathrel\neq~~2, alors u
  est la symétrie par rapport à un sous-espace V parallèlement à un
  supplémentaire.
\end{itemize}

Démonstration

\begin{itemize}
\itemsep1pt\parskip0pt\parsep0pt
\item
  (i) On écrit X^2 - X = X(X - 1)~: les polynômes X et X - 1
  sont premiers entre eux d'où E =\
  \mathrmKer(u -\mathrmId)
  \oplus~\mathrmKer~u. Alors u est
  la pro\jmathection sur
  \mathrmKer~(u
  -\mathrmId) parallèlement à
  \mathrmKer~u.
\item
  (ii) On écrit X^2 - 1 = (X - 1)(X + 1)~: les polynômes X +
  1 et X - 1 sont premiers entre eux (si
  carK\mathrel\neq~~2) d'où E
  = \mathrmKer~(u
  -\mathrmId)
  \oplus~\mathrmKer~(u +
  \mathrmId). Alors u est la symétrie par rapport à
  \mathrmKer~(u
  -\mathrmId) parallèlement à
  \mathrmKer~(u +
  \mathrmId).
\end{itemize}

Théorème~3.2.11 Soit E un K-espace vectoriel de dimension finie et u \in
L(E). Alors u est diagonalisable si et seulement si il existe P \in
K{[}X{]} scindé à racines simples tel que P(u) = 0.

Démonstration

\begin{itemize}
\itemsep1pt\parskip0pt\parsep0pt
\item
  ( \rigtharrow~). Soit \mathcal{E} une base de E telle que D =\
  \mathrmMat (u,\mathcal{E}) soit diagonale, D
  =\
  \mathrmdiag(\lambda~\_1,\\ldots,\lambda~\_n~).
  Supposons que
  \lambda~\_1,\\ldots,\lambda~\_k~
  sont distinctes et que pour i ≥ k + 1, \lambda~\_i
  \in\\lambda~\_1,\\ldots\lambda~\_k\~.
  Soit P = \∏ ~
  \_i=1^k(X - \lambda~\_i). On a P(D)
  =\
  \mathrmdiag(P(\lambda~\_1),\\ldots,P(\lambda~\_n~))
  = 0 et donc P(u) = 0 avec P scindé à racines simples.
\item
  ( ⇐) Soit P(X) =\ \∏
   \_i=1^k(X - \lambda~\_i). On a donc E
  = \mathrmKer~(u -
  \lambda~\_1\mathrmId)
  \oplus~⋯
  \oplus~\mathrmKer~(u -
  \lambda~\_k\mathrmId). En réunissant des bases de
  tous ces sous-espaces, on obtient une base de E formée de vecteurs
  propres de u. Donc u est diagonalisable.
\end{itemize}

Remarque~3.2.2 Ce théorème peut encore s'exprimer sous la forme~: u est
diagonalisable si et seulement si \mu\_u est scindé à racines
simples.

\paragraph{3.2.6 Sous-espaces caractéristiques}

Remarque~3.2.3 Soit E un K-espace vectoriel de dimension finie, u \in L(E)
et P tel que P(u) = 0. Soit P =
P\_1\\ldotsP\_k~
une décomposition de \chi\_u en produit de polynômes deux à deux
premiers entre eux. Soit E\_i =\
\mathrmKerP\_i(u). On a E = E\_1
\oplus~⋯ \oplus~ E\_k et chacun des sous-espaces
E\_i est stable par u. Soit u\_i la restriction de u à
E\_i. Dans une base \mathcal{E} = \mathcal{E}\_1
\cup\\ldots~
\cup\mathcal{E}\_k adaptée à la décomposition en somme directe, on a M
= \mathrmMat~ (u,\mathcal{E})
=\
\mathrmdiag(M\_1,\\ldots,M\_k~)
avec M\_i =\
\mathrmMat (u\_i,\mathcal{E}\_i). On en
déduit (calcul par blocs d'un déterminant) que \chi\_u
= \∏ ~
\chi\_u\_i. De même, on a, si Q \in K{[}X{]}, Q(M)
=\
\mathrmdiag(Q(M\_1),\\ldots,Q(M\_k~))
et donc Q(u) = 0 \Leftrightarrow
\forall~i, Q(u\_i~) = 0. On en déduit que
\mu\_u = ppcm\mu\_u\_i~.
Mais P\_i(u\_i) = 0 et donc \mu\_u\_i
divise P\_i. On en déduit que les \mu\_u\_i sont
deux à deux premiers entre eux et donc \mu\_u
= \∏ ~
\mu\_u\_i.

Supposons maintenant que le polynôme caractéristique de u est scindé,
\chi\_u(X) = \\∏
 \_i=1^k(X - \lambda~\_i)^m\_i avec
\lambda~\_1,\\ldots,\lambda~\_k~
distincts. Appliquons les résultats précédents avec P = \chi\_u et
P\_i = (X - \lambda~\_i)^m\_i. Ceci nous
conduit à la définition~:

Définition~3.2.2 Soit u \in L(E) et \lambda~ une valeur propre de u de
multiplicité m. Le sous-espace
\mathrmKer~(u -
\lambda~\mathrmId)^m est appelé sous-espace
caractéristique de u associé à \lambda~, il est stable par u.

Remarque~3.2.4 Appelons donc E\_i le sous-espace caractéristique
de u associé à \lambda~\_i, et u\_i la restriction de u à
E\_i. On sait que \mu\_u\_i divise (X -
\lambda~\_i)^m\_i, donc \mu\_u\_i = (X
- \lambda~\_i)^r\_i. Dans ce cas, \mu\_u(X)
= \∏ ~
\_i=1^k(X - \lambda~\_i)^r\_i. De plus,
\chi\_u(X) = \∏ ~
\chi\_u\_i, chacun des \chi\_u\_i est scindé
et a les mêmes racines que \mu\_u\_i. On en déduit que
\chi\_u\_i(X) = (X -
\lambda~\_i)^dim E\_i~. Mais
alors \chi\_u(X) =\
∏  \_i=1^k~(X -
\lambda~\_i)^m\_i =\
∏ \_i=1^k~(X -
\lambda~\_i)^dim E\_i~. Ceci
démontre que dim E\_i = m\_i~.
D'où le théorème

Théorème~3.2.12 Soit E un K-espace vectoriel de dimension finie et u \in
L(E) dont le polynôme caractéristique est scindé sur K. Alors E est
somme directe des sous-espaces caractéristiques de u~; chaque
sous-espace caractéristique a pour dimension la multiplicité de la
valeur propre correspondante.

\paragraph{3.2.7 Application~: récurrences linéaires d'ordre 2}

Théorème~3.2.13 Soit E un K-espace vectoriel de dimension 2 et u \in L(E)
dont le polynôme caractéristique est scindé. Alors, il existe une base \mathcal{E}
de E telle que la matrice de u dans cette base soit de l'une des deux
formes suivantes~: \left
(\matrix\,\lambda~\_1&0
\cr 0 &\lambda~\_2\right ) ou
\left
(\matrix\,\lambda~&1\cr 0
&\lambda~\right ).

Démonstration Si u est diagonalisable, il existe une base \mathcal{E} de E telle
que la matrice de u dans cette base soit \left
(\matrix\,\lambda~\_1&0
\cr 0 &\lambda~\_2\right ). Supposons
donc u non diagonalisable. Le polynôme caractéristique de u a
nécessairement une racine double \lambda~ (sinon u serait diagonalisable) et le
sous-espace propre associé E\_u(\lambda~) est nécessairement de
dimension 1 (pour la même raison). Soit donc e\_2 \in E \diagdown
E\_u(\lambda~) et posons e\_1 = u(e\_2) - \lambda~e\_2
= (u - \lambda~\mathrmId)(e\_2)~; le vecteur
e\_1 est non nul car e\_2 n'est pas vecteur propre de u.
Le théorème de Cayley Hamilton garantit que (u -
\lambda~\mathrmId)^2 = 0 et donc (u -
\lambda~\mathrmId)(e\_1) = 0. Donc e\_1 est
vecteur propre de u. Ceci garantit que (e\_1,e\_2) est
libre (puisque e\_2 n'est pas vecteur propre de u), et donc est
une base de E. On a u(e\_1) = \lambda~e\_1 et u(e\_2) =
e\_1 + \lambda~e\_2 et donc la matrice de u dans cette base est
\left
(\matrix\,\lambda~&1\cr 0
&\lambda~\right ).

Proposition~3.2.14 Soit n \in \mathbb{N}~, alors

 \left
(\matrix\,\lambda~\_1&0
\cr 0 &\lambda~\_2\right )^n
= \left
(\matrix\,\lambda~\_1^n&0
\cr 0 &\lambda~\_2^n\right )

et

 \left
(\matrix\,\lambda~&1\cr 0
&\lambda~\right )^n = \left
(\matrix\,\lambda~^n&n\lambda~^n-1
\cr 0 &\lambda~^n \right )

Démonstration La première formule est évidente~; la deuxième peut se
montrer soit par récurrence, soit en appliquant la formule du binôme~;
on écrit que

\left
(\matrix\,\lambda~&1\cr 0
&\lambda~\right ) = \lambda~I\_2 + \left
(\matrix\,0&1 \cr
0&0\right )

et on remarque que \left
(\matrix\,0&1 \cr
0&0\right )^2 = 0.

Considérons a,b \in \mathbb{C}, b\neq~0 et soit E l'ensemble
des suites (u\_n)\_n\in\mathbb{N}~ de nombres complexes vérifiant

\forall~n \in \mathbb{N}~, u\_n+2 = au\_n+1~ +
bu\_n

Proposition~3.2.15 E est un \mathbb{C}-espace vectoriel et l'application E \rightarrow~
\mathbb{C}^2,
(u\_n)\_n\in\mathbb{N}~\mapsto~(u\_0,u\_1)
est un isomorphisme d'espaces vectoriels~; en particulier,
dim~ E = 2.

Démonstration La vérification du premier point est élémentaire.
L'application
(u\_n)\_n\in\mathbb{N}~\mapsto~(u\_0,u\_1)
est visiblement linéaire et elle est bi\jmathective car une suite de E est
entièrement déterminée par la donnée de ses deux premiers éléments.

Soit (u\_n)\_n\in \mathbb{N}~ \in E et posons U\_n =
\left
(\matrix\,u\_n
\cr u\_n+1\right ). On a alors

\begin{align*} U\_n+1& =&
\left
(\matrix\,u\_n+1
\cr u\_n+2\right ) =
\left
(\matrix\,u\_n+1
\cr au\_n+1 +
bu\_n\right )\%&
\\ & =& \left
(\matrix\,0&1 \cr
b&a\right )\left
(\matrix\,u\_n
\cr u\_n+1\right ) =
AU\_n \%& \\
\end{align*}

avec A = \left
(\matrix\,0&1 \cr
b&a\right ). On en déduit que U\_n =
A^nU\_0. Comme \mathbb{C} est algébriquement clos,
\chi\_A est scindé. Si A est diagonalisable, il existe P inversible
telle que A = P^-1\left
(\matrix\,\lambda~\_1&0
\cr 0 &\lambda~\_2\right )P, d'où

U\_n = P^-1\left
(\matrix\,\lambda~\_1^n&0
\cr 0 &\lambda~\_2^n\right
)PU\_0

et en prenant la première coordonnée, u\_n =
\alpha~\lambda~\_1^n + \beta~\lambda~\_2^n. Si par contre, A n'est
pas diagonalisable, il existe P inversible telle que A =
P^-1\left
(\matrix\,\lambda~&1\cr 0
&\lambda~\right )P, d'où

U\_n = P^-1\left
(\matrix\,\lambda~^n&n\lambda~^n-1
\cr 0 &\lambda~^n \right
)PU\_0

et en prenant la première coordonnée, u\_n = \alpha~\lambda~^n +
\beta~n\lambda~^n.

On a donc E \subset~ F avec F =\
\mathrmVect((\lambda~\_1^n)\_n\in\mathbb{N}~,(\lambda~\_2^n)\_n\in\mathbb{N}~)
ou F =\
\mathrmVect((\lambda~^n)\_n\in\mathbb{N}~,(n\lambda~^n)\_n\in\mathbb{N}~)
suivant le cas. Comme dim~ E = 2 et
dim~ F \leq 2, on a nécessairement égalité.
Remarquons alors que, pour \lambda~\neq~0,

(\lambda~^n)\_ n\in\mathbb{N}~ \in E\quad
\Leftrightarrow \lambda~^2 = a\lambda~ + b
\Leftrightarrow \chi\_ A(\lambda~) = 0

Ceci nous conduit à la méthode suivante de résolution de la récurrence
linéaire \forall~n \in \mathbb{N}~, u\_n+2~ =
au\_n+1 + bu\_n

\begin{itemize}
\itemsep1pt\parskip0pt\parsep0pt
\item
  rechercher les solutions particulières de la forme u\_n =
  \lambda~^n ceci conduit à une équation du second degré en \lambda~, P(\lambda~)
  = 0
\item
  si cette équation a deux racines simples \lambda~\_1 et \lambda~\_2,
  les solutions sont les suites de la forme u\_n =
  \alpha~\lambda~\_1^n + \beta~\lambda~\_2^n
\item
  si cette équation a une racine double \lambda~, les solutions sont les suites
  de la forme u\_n = \alpha~\lambda~^n + \beta~n\lambda~^n.
\end{itemize}

{[}
{[}
{[}
{[}

\end{document}

% \documentclass[]{article}
\usepackage[T1]{fontenc}
\usepackage{lmodern}
\usepackage{amssymb,amsmath}
\usepackage{ifxetex,ifluatex}
\usepackage{fixltx2e} % provides \textsubscript
% use upquote if available, for straight quotes in verbatim environments
\IfFileExists{upquote.sty}{\usepackage{upquote}}{}
\ifnum 0\ifxetex 1\fi\ifluatex 1\fi=0 % if pdftex
  \usepackage[utf8]{inputenc}
\else % if luatex or xelatex
  \ifxetex
    \usepackage{mathspec}
    \usepackage{xltxtra,xunicode}
  \else
    \usepackage{fontspec}
  \fi
  \defaultfontfeatures{Mapping=tex-text,Scale=MatchLowercase}
  \newcommand{\euro}{€}
\fi
% use microtype if available
\IfFileExists{microtype.sty}{\usepackage{microtype}}{}
\ifxetex
  \usepackage[setpagesize=false, % page size defined by xetex
              unicode=false, % unicode breaks when used with xetex
              xetex]{hyperref}
\else
  \usepackage[unicode=true]{hyperref}
\fi
\hypersetup{breaklinks=true,
            bookmarks=true,
            pdfauthor={},
            pdftitle={A propos de Jordan},
            colorlinks=true,
            citecolor=blue,
            urlcolor=blue,
            linkcolor=magenta,
            pdfborder={0 0 0}}
\urlstyle{same}  % don't use monospace font for urls
\setlength{\parindent}{0pt}
\setlength{\parskip}{6pt plus 2pt minus 1pt}
\setlength{\emergencystretch}{3em}  % prevent overfull lines
\setcounter{secnumdepth}{0}
 
/* start css.sty */
.cmr-5{font-size:50%;}
.cmr-7{font-size:70%;}
.cmmi-5{font-size:50%;font-style: italic;}
.cmmi-7{font-size:70%;font-style: italic;}
.cmmi-10{font-style: italic;}
.cmsy-5{font-size:50%;}
.cmsy-7{font-size:70%;}
.cmex-7{font-size:70%;}
.cmex-7x-x-71{font-size:49%;}
.msbm-7{font-size:70%;}
.cmtt-10{font-family: monospace;}
.cmti-10{ font-style: italic;}
.cmbx-10{ font-weight: bold;}
.cmr-17x-x-120{font-size:204%;}
.cmsl-10{font-style: oblique;}
.cmti-7x-x-71{font-size:49%; font-style: italic;}
.cmbxti-10{ font-weight: bold; font-style: italic;}
p.noindent { text-indent: 0em }
td p.noindent { text-indent: 0em; margin-top:0em; }
p.nopar { text-indent: 0em; }
p.indent{ text-indent: 1.5em }
@media print {div.crosslinks {visibility:hidden;}}
a img { border-top: 0; border-left: 0; border-right: 0; }
center { margin-top:1em; margin-bottom:1em; }
td center { margin-top:0em; margin-bottom:0em; }
.Canvas { position:relative; }
li p.indent { text-indent: 0em }
.enumerate1 {list-style-type:decimal;}
.enumerate2 {list-style-type:lower-alpha;}
.enumerate3 {list-style-type:lower-roman;}
.enumerate4 {list-style-type:upper-alpha;}
div.newtheorem { margin-bottom: 2em; margin-top: 2em;}
.obeylines-h,.obeylines-v {white-space: nowrap; }
div.obeylines-v p { margin-top:0; margin-bottom:0; }
.overline{ text-decoration:overline; }
.overline img{ border-top: 1px solid black; }
td.displaylines {text-align:center; white-space:nowrap;}
.centerline {text-align:center;}
.rightline {text-align:right;}
div.verbatim {font-family: monospace; white-space: nowrap; text-align:left; clear:both; }
.fbox {padding-left:3.0pt; padding-right:3.0pt; text-indent:0pt; border:solid black 0.4pt; }
div.fbox {display:table}
div.center div.fbox {text-align:center; clear:both; padding-left:3.0pt; padding-right:3.0pt; text-indent:0pt; border:solid black 0.4pt; }
div.minipage{width:100%;}
div.center, div.center div.center {text-align: center; margin-left:1em; margin-right:1em;}
div.center div {text-align: left;}
div.flushright, div.flushright div.flushright {text-align: right;}
div.flushright div {text-align: left;}
div.flushleft {text-align: left;}
.underline{ text-decoration:underline; }
.underline img{ border-bottom: 1px solid black; margin-bottom:1pt; }
.framebox-c, .framebox-l, .framebox-r { padding-left:3.0pt; padding-right:3.0pt; text-indent:0pt; border:solid black 0.4pt; }
.framebox-c {text-align:center;}
.framebox-l {text-align:left;}
.framebox-r {text-align:right;}
span.thank-mark{ vertical-align: super }
span.footnote-mark sup.textsuperscript, span.footnote-mark a sup.textsuperscript{ font-size:80%; }
div.tabular, div.center div.tabular {text-align: center; margin-top:0.5em; margin-bottom:0.5em; }
table.tabular td p{margin-top:0em;}
table.tabular {margin-left: auto; margin-right: auto;}
div.td00{ margin-left:0pt; margin-right:0pt; }
div.td01{ margin-left:0pt; margin-right:5pt; }
div.td10{ margin-left:5pt; margin-right:0pt; }
div.td11{ margin-left:5pt; margin-right:5pt; }
table[rules] {border-left:solid black 0.4pt; border-right:solid black 0.4pt; }
td.td00{ padding-left:0pt; padding-right:0pt; }
td.td01{ padding-left:0pt; padding-right:5pt; }
td.td10{ padding-left:5pt; padding-right:0pt; }
td.td11{ padding-left:5pt; padding-right:5pt; }
table[rules] {border-left:solid black 0.4pt; border-right:solid black 0.4pt; }
.hline hr, .cline hr{ height : 1px; margin:0px; }
.tabbing-right {text-align:right;}
span.TEX {letter-spacing: -0.125em; }
span.TEX span.E{ position:relative;top:0.5ex;left:-0.0417em;}
a span.TEX span.E {text-decoration: none; }
span.LATEX span.A{ position:relative; top:-0.5ex; left:-0.4em; font-size:85%;}
span.LATEX span.TEX{ position:relative; left: -0.4em; }
div.float img, div.float .caption {text-align:center;}
div.figure img, div.figure .caption {text-align:center;}
.marginpar {width:20%; float:right; text-align:left; margin-left:auto; margin-top:0.5em; font-size:85%; text-decoration:underline;}
.marginpar p{margin-top:0.4em; margin-bottom:0.4em;}
.equation td{text-align:center; vertical-align:middle; }
td.eq-no{ width:5%; }
table.equation { width:100%; } 
div.math-display, div.par-math-display{text-align:center;}
math .texttt { font-family: monospace; }
math .textit { font-style: italic; }
math .textsl { font-style: oblique; }
math .textsf { font-family: sans-serif; }
math .textbf { font-weight: bold; }
.partToc a, .partToc, .likepartToc a, .likepartToc {line-height: 200%; font-weight:bold; font-size:110%;}
.chapterToc a, .chapterToc, .likechapterToc a, .likechapterToc, .appendixToc a, .appendixToc {line-height: 200%; font-weight:bold;}
.index-item, .index-subitem, .index-subsubitem {display:block}
.caption td.id{font-weight: bold; white-space: nowrap; }
table.caption {text-align:center;}
h1.partHead{text-align: center}
p.bibitem { text-indent: -2em; margin-left: 2em; margin-top:0.6em; margin-bottom:0.6em; }
p.bibitem-p { text-indent: 0em; margin-left: 2em; margin-top:0.6em; margin-bottom:0.6em; }
.paragraphHead, .likeparagraphHead { margin-top:2em; font-weight: bold;}
.subparagraphHead, .likesubparagraphHead { font-weight: bold;}
.quote {margin-bottom:0.25em; margin-top:0.25em; margin-left:1em; margin-right:1em; text-align:justify;}
.verse{white-space:nowrap; margin-left:2em}
div.maketitle {text-align:center;}
h2.titleHead{text-align:center;}
div.maketitle{ margin-bottom: 2em; }
div.author, div.date {text-align:center;}
div.thanks{text-align:left; margin-left:10%; font-size:85%; font-style:italic; }
div.author{white-space: nowrap;}
.quotation {margin-bottom:0.25em; margin-top:0.25em; margin-left:1em; }
h1.partHead{text-align: center}
.sectionToc, .likesectionToc {margin-left:2em;}
.subsectionToc, .likesubsectionToc {margin-left:4em;}
.subsubsectionToc, .likesubsubsectionToc {margin-left:6em;}
.frenchb-nbsp{font-size:75%;}
.frenchb-thinspace{font-size:75%;}
.figure img.graphics {margin-left:10%;}
/* end css.sty */

\title{A propos de Jordan}
\author{}
\date{}

\begin{document}
\maketitle

\textbf{Warning: 
requires JavaScript to process the mathematics on this page.\\ If your
browser supports JavaScript, be sure it is enabled.}

\begin{center}\rule{3in}{0.4pt}\end{center}

[
[
[]
[

\subsubsection{3.3 A propos de Jordan}

\paragraph{3.3.1 Décomposition de Jordan}

Soit E un K-espace vectoriel de dimension finie et u \in L(E) dont le
polynôme caractéristique est scindé sur K,
E_1,\\ldots,E_k~
les sous-espaces caractéristiques de u associés aux valeurs propres
\lambda_1,\\ldots,\lambda_k~.
Soit u_i la restriction de u à E_i et n_i =
u_i -
\lambda_i\mathrmId_E_i. Avec les
notations précédentes, on a n_i^r_i = 0, donc
n_i est nilpotent. Soit d : E \rightarrow~ E définie par d(x_1 +
\\ldots~ +
x_k) = \lambda_1x_1 +
\\ldots~ +
\lambda_kx_k et n : E \rightarrow~ E défini par n(x_1 +
\\ldots~ +
x_k) = \\sum ~
_in_i(x_i). L'endomorphisme d est
diagonalisable (ses sous-espaces propres sont les E_i), n est
nilpotent (n^max(r_i)~ = 0)
et on a u = d + n. De plus, si d_i =
\lambda_i\mathrmId_E_i désigne
la restriction de d à E_i, on a d_i \cdot n_i =
n_i \cdot d_i et on en déduit donc que d \cdot n = d \cdot n.

Théorème~3.3.1 (décomposition de Jordan). Soit E un K-espace vectoriel
de dimension finie et u \in L(E) dont le polynôme caractéristique est
scindé sur K. Alors u s'écrit de manière unique sous la forme u = d + n
avec d diagonalisable, n nilpotent et d \cdot n = n \cdot d.

Démonstration L'existence de la décomposition vient d'être démontrée.
Soit u = d' + n' une autre décomposition vérifiant les conditions
imposées. Alors d' et n' commutent à u, donc à tous les P(u) et donc
laissent stables leurs noyaux. En particulier ils laissent stables les
sous-espaces caractéristiques de u. Soit d_i' et n_i'
les restrictions de d' et n' à E_i. n_i' est bien
entendu nilpotent. De plus, puisque d' est diagonalisable, il existe P
scindé à racines simples tel que P(d') = 0 et on a encore
P(d_i') = 0 ce qui montre que d_i' est diagonalisable.
On a d_i + n_i = d_i' + n_i' soit
encore \lambda_i\mathrmId + n_i =
d_i' + n_i' ou encore
\lambda_i\mathrmId_E_i -
d_i' = n_i' - n_i. Comme n_i'
commute à
\lambda_i\mathrmId_E_i,d_i'
et n_i', il commute à n_i et donc n_i -
n_i' est encore nilpotent. De plus
\lambda_i\mathrmId - d_i' est clairement
diagonalisable. Un endomorphisme à la fois diagonalisable et nilpotent
est nul puisqu'il doit avoir une matrice nulle dans une certaine base,
donc d_i = d_i' et n_i = n_i'.
Alors, d et d' coïncident sur des espaces dont la somme est E, donc ils
sont égaux. Ceci implique alors que n = n'. D'où l'unicité de la
décomposition.

\paragraph{3.3.2 Applications}

Puissances d'un endomorphisme

Soit E un K-espace vectoriel de dimension finie et u \in L(E) dont le
polynôme caractéristique est scindé sur K,
E_1,\\ldots,E_k~
les sous-espaces caractéristiques de u associés aux valeurs propres
\lambda_1,\\ldots,\lambda_k~.
Soit u_i la restriction de u à E_i et n_i =
u_i -
\lambda_i\mathrmId_E_i.
L'endomorphisme n_i est nilpotent d'indice de nilpotence
r_i. On a u_i =
\lambda_i\mathrmId_E_i +
n_i et donc si q \in \mathbb{N}~

u_i^q = \\sum
_p=0^qC_ q^p\lambda_
i^q-pn_ i^p = \\sum
_p=0^min(q,r_i-1)C_
q^p\lambda_ i^q-pn_ i^p

Supposons désormais que \lambda_i\neq~0. On
obtient

\begin{align*} u_i^q& =& \lambda_
i^q \\sum
_p=0^min(q,r_i-1) q(q -
1)\ldots~(q - p + 1) \over
p! \lambda_i^-pn_ i^p\%&
\\ & =& \lambda_i^q
\sum _p=0^r_i-1~ q(q -
1)\ldots~(q - p + 1) \over
p! \lambda_i^-pn_ i^p \%&
\\ \end{align*}

(puisque 
q(q-1)\\ldots~(q-p+1)
\over p! = 0 si p > q). Posons alors, pour
t \in K

\sum _p=0^r_i-1~ t(t -
1)\ldots~(t - p + 1) \over
p! \lambda_i^-pn_ i^p =
\sum _p=0^r_i-1w_
i,pt^p

avec w_i,p \in L(E_i)~; c'est une fonction polynomiale
en t à valeurs dans L(E_i) de degré inférieur ou égal à
r_i - 1. On obtient

\forall~~q \in \mathbb{N}~,\quad
u_i^q = \lambda_ i^q
\sum _p=0^r_i-1w_
i,pq^p

Supposons maintenant que u est inversible, si bien que
\forall~~i \in [1,k],
\lambda_i\neq~0. Soit \pi_i(x) la
projection sur E_i parallèlement à
\\oplus~ ~
_j\neq~iE_j. On a
\\sum ~
_i\pi_i = \mathrmId_E si bien
que u = u \cdot (\\sum ~
_i\pi_i) =\
\sum  _iu_i \cdot \pi_i~.

Lemme~3.3.2 \forall~q \in \mathbb{N}~, u^q~
= \\sum ~
u_i^q \cdot \pi_i.

Démonstration Evident par récurrence sur q en remarquant que les
E_i sont stables par u et les u_i.

On en déduit donc que \forall~~q \in
\mathbb{N}~,\quad u^q =\
\sum ~
_i=1^k\lambda_i^q\
\sum ~
_p=0^r_i-1q^pw_i,p \cdot
\pi_i, d'où le théorème (en remarquant que r_i \leq
m_i)

Théorème~3.3.3 Soit u \in L(E) inversible dont le polynôme caractéristique
est scindé sur K,
\lambda_1,\\ldots,\lambda_k~
ses valeurs propres de multiplicités respectives
m_1,\\ldots,m_k~.
Alors il existe une famille
(v_i,p)_1\leqi\leqk,0\leqp\leqm_i-1 d'endomorphismes de E
tels que

\forall~q \in \mathbb{N}~,\quad u^q~ =
\sum _i=1^k\lambda_ i^q~
\\sum
_p=0^m_i-1q^pv_ i,p

Remarque~3.3.1 Bien entendu, on a un résultat similaire pour les
matrices inversibles

Théorème~3.3.4 Soit A \in M_K(n) inversible dont le polynôme
caractéristique est scindé sur K,
\lambda_1,\\ldots,\lambda_k~
ses valeurs propres de multiplicités respectives
m_1,\\ldots,m_k~.
Alors il existe une famille
(B_i,p)_1\leqi\leqk,0\leqp\leqm_i-1 de matrices carrées
d'ordre n telles que

\forall~q \in \mathbb{N}~,\quad A^q~ =
\sum _i=1^k\lambda_ i^q~
\\sum
_p=0^m_i-1q^pB_ i,p

Suites à récurrence linéaire

Remarque~3.3.2 Soit p \in \mathbb{N}~,
a_0,\\ldots,a_p-1~
une famille d'éléments de K et

V = \(u_n)_n\in\mathbb{N}~ \in
K^\mathbb{N}~∣\forall~~n \in
\mathbb{N}~, u_ n+p = a_p-1u_n+p-1 +
\\ldots~ +
a_0u_n\

V est un sous-espace vectoriel de K^\mathbb{N}~. Il est clair que la
donnée de
u_0,\\ldots,u_p-1~
détermine parfaitement un élément de V et on a donc

Théorème~3.3.5 L'application V \rightarrow~ K^p,
(u_n)_n\in\mathbb{N}~\mapsto~(u_0,\\ldots,u_p-1~)
est un isomorphisme d'espaces vectoriels. On a en particulier
dim~ V = p.

Remarque~3.3.3 Il est clair que l'on peut se limiter à étudier le cas où
a_0\neq~0 sinon notre récurrence
linéaire d'ordre p se réduit à une récurrence linéaire d'ordre k \leq p
valable pour n ≥ n_0.

Soit (u_n)_n\in\mathbb{N}~ \in V et considérons la suite (V
_n) définie par V _n = \left
(\matrix\,u_n
\cr u_n+1 \cr
\⋮~ \cr
u_n+p-1\right ) \in K^p. On a
clairement V _n+1 = AV _n avec

A = \left (\matrix\,0 &1
&0&\\ldots~&0
\cr &⋱
&⋱&
&\⋮~
\cr &
&⋱&\mathrel⋱&\⋮~
\cr 0
&\\ldots~
&\\ldots~&0&1
\cr
a_0&a_1&\\ldots&\\\ldots&a_p-1~\right
) \in M_K(p)

et donc V _n = A^nV _0. On a
\mathrm{det}~ A =
(-1)^n-1a_0\neq~0 et donc la
matrice est inversible. On peut donc appliquer le résultat précédent.
Soit \chi le polynôme caractéristique de la matrice A (encore appelé
polynôme caractéristique de la récurrence linéaire). Un calcul simple
donne

Lemme~3.3.6 On a \chi(X) = X^p - a_p-1X^p-1
-\\ldots~ -
a_0. Pour \lambda~ \in K^∗, on a

\chi(\lambda~) = 0 \Leftrightarrow (\lambda~^n)_ n\in\mathbb{N}~ \in V

Soit
\lambda_1,\\ldots,\lambda_k~
les racines de \chi de multiplicités respectives
m_1,\\ldots,m_k~.
On sait qu'il existe une famille
(B_i,q)_1\leqi\leqk,0\leqq\leqm_i-1 de matrices carrées
d'ordre p telles que

\forall~n \in \mathbb{N}~,\quad A^n~ =
\sum _i=1^k\lambda_ i^n~
\\sum
_q=0^m_i-1n^qB_ i,q

On a donc en particulier A^nV _0
= \\sum ~
_i=1^k\lambda_i^n\
\sum ~
_q=0^m_i-1n^qB_i,qV
_0 et en prenant la première coordonnée,

u_n = \\sum
_i=1^k\lambda_ i^n \\sum
_q=0^m_i-1\alpha_ i,qn^q

Soit alors W le sous-espace de K^\mathbb{N}~ engendré par les suites
(\lambda_i^nn^q)_1\leqi\leqk,0\leqq\leqm_i-1.
On a dim~ W
\leq\\sum  m_i~ = p
et V \subset~ W avec dim~ V = p. On en déduit que V =
W et que la famille
(\lambda_i^nn^q)_1\leqi\leqk,0\leqq\leqm_i-1
est une base de V . On a donc le théorème suivant

Théorème~3.3.7 Soit p \in \mathbb{N}~,
a_0,\\ldots,a_p-1~
une famille d'éléments de K avec
a_0\neq~0, et V l'espace des suites
vérifiant la récurrence linéaire \forall~~n \in \mathbb{N}~,
u_n+p = a_p-1u_n+p-1 +
\\ldots~ +
a_0u_n\. Soit \chi(X) = X^p -
a_p-1X^p-1
-\\ldots~ -
a_0 le polynôme caractéristique de la récurrence linéaire
(obtenu en recherchant des solutions particulières de la forme
u_n = \lambda~^n),
\lambda_1,\\ldots,\lambda_k~
les racines de \chi de multiplicités respectives
m_1,\\ldots,m_k~.
Alors la famille
(\lambda_i^nn^q)_1\leqi\leqk,0\leqq\leqm_i-1
est une base de V . Les solutions de la récurrence linéaire sont
exactement les suites qui s'écrivent sous la forme

u_n = \\sum
_i=1^k\lambda_ i^nP_
i(n),\quad P_i \in K[X], deg P_i \leq
m_i - 1

Retour aux puissances d'un endomorphisme

Soit u \in L(E) et soit P(X) = X^p -
a_p-1X^p-1
-\\ldots~ -
a_0 un polynôme qui annule u (par exemple le polynôme
caractéristique). On a immédiatement

Lemme~3.3.8 (u^n)_0\leqn\leqp-1 est une famille
génératrice de
\mathrmVect(u^n~,n
\in \mathbb{N}~).

On peut donc chercher à exprimer u^n sous la forme
u^n = \alpha_n^(p-1)u^p-1 +
\\ldots~ +
\alpha_n^(0)\mathrmId.

Théorème~3.3.9 Soit
(\alpha_n^(p-1))_n\in\mathbb{N}~,\\ldots,(\alpha_n^0)_n\in\mathbb{N}~~
les suites solutions de la récurrence linéaire \alpha_n+p =
a_p-1\alpha_n+p-1 +
\\ldots~ +
a_0\alpha_n vérifiant

\forall~~i \in [0,p - 1],
\forall~~j \in [0,p - 1],\quad
\alpha_i^(j) = \delta_ i^j

Alors

\forall~n \in \mathbb{N}~,\quad u^n~ =
\alpha_ n^(p-1)u^p-1 +
\\ldots + \alpha~_
n^(0)\mathrmId

Démonstration Par récurrence sur n. C'est manifestement vérifié si n \leq p
- 1. De plus, si n ≥ p la relation u^p =
a_p-1u^p-1 +
\\ldots~ +
a_0\mathrmId donne u^n =
a_p-1u^n-1 +
\\ldots~ +
a_0u^n-p soit par l'hypothèse de récurrence

\begin{align*} u^n& =&
\sum _i=1^pa_
p-iu^n-i = \\sum
_i=1^pa_ p-i \\sum
_j=0^p-1\alpha_ n-i^(j)u^j \%&
\\ & =& \\sum
_j=0^p-1(\\sum
_i=1^pa_
p-i\alpha_n-i^(j))u^j =
\sum _j=0^p-1\alpha~_
n^(j)u^j\%& \\
\end{align*}

d'après la relation vérifiée par les (\alpha_n^(j)). Ceci
achève la démonstration.

\paragraph{3.3.3 Réduction des endomorphismes nilpotents}

Définition~3.3.1 Soit E un K-espace vectoriel et u \in L(E). On dit que u
est nilpotent d'indice de nilpotence r si u^r = 0 et
u^r-1\neq~0.

Remarque~3.3.4 Remarquons que la seule valeur propre d'un endomorphisme
nilpotent est 0, car si u(x) = \lambda~x, on a 0 = u^r(x) =
\lambda~^rx.

Proposition~3.3.10 Soit E un K-espace vectoriel de dimension n et u \in
L(E). Alors u est nilpotent si et seulement si \chi_u(X) =
X^n.

Démonstration ( \rigtharrow~) Supposons que u est nilpotent. Comme u est annulé par
le polynôme scindé X^r (si u^r = 0), u est
trigonalisable. Mais u admet comme seule valeur propre 0. On en déduit
que \chi_u(X) = X^n. Pour la réciproque, on peut par
exemple utiliser le théorème de Cayley Hamilton, ou trigonaliser u.

Remarque~3.3.5 On en déduit que l'indice de nilpotence r est inférieur
ou égal à n. On a bien entendu \mu_u(X) = X^r.

Définition~3.3.2 Soit p ≥ 1. On appelle matrice élémentaire de Jordan
d'ordre p la matrice

J_p = \left
(\matrix\,0&1&0&\\ldots~&0
\cr
\⋮&⋱&\mathrel⋱&\mathrel⋱&\\⋮~
\cr
\⋮~&
&⋱&\mathrel⋱&\⋮~
\cr
0&\\ldots&\\\ldots~&0&1
\cr
0&\\ldots&\\\ldots&\\\ldots&0~\right
)

Soit E un K-espace vectoriel de dimension n et u \in L(E) nilpotent
d'indice de nilpotence r. Supposons par exemple que r = n. On a donc
u^n-1\neq~0 avec u^n = 0.
Soit a \in E tel que u^n-1(a)\neq~0 et
posons e_i = u^n-i(a) pour 1 \leq i \leq n. Montrons que
(e_1,\\ldots,e_n~)
est une base de E. Il suffit de montrer que c'est une famille libre.
Pour cela supposons que \lambda_1e_1 +
\\ldots~ +
\lambda_ne_n = 0, soit encore

\lambda_1u^n-1(a) +
\\ldots + \lambda~_
n-1u(a) + \lambda_na = 0

Appliquons aux deux membres u^n-1 en tenant compte de
u^n(a) =
\\ldots~ =
u^2n-2(a) = 0~; on obtient \lambda_nu^n-1(a) =
0 soit \lambda_n = 0. Supposons montré que \lambda_n =
\lambda_n-1 =
\\ldots~ =
\lambda_n-k+1 = 0 si bien que l'on a

\lambda_1u^n-1(a) +
\\ldots + \lambda~_
n-k-1u^k+1(a) + \lambda_ n-ku^k(a) = 0

Appliquons aux deux membres u^n-k-1 en tenant compte de
u^n(a) =
\\ldots~ =
u^2n-k-2(a) = 0~; on obtient
\lambda_n-ku^n-1(a) = 0 soit \lambda_n-k = 0. Par
récurrence, on a bien \forall~i, \lambda_i~ = 0.
Donc
(e_1,\\ldots,e_n~)
est une base de E. Dans cette base, la matrice de u est clairement
J_n~: on a u(e_i) = e_i-1 si i ≥ 2 et
u(e_1) = 0. Ce cas particulier est à la base du résultat
suivant

Théorème~3.3.11 Soit E un K-espace vectoriel de dimension n et u \in L(E)
nilpotent. Alors il existe une base \mathcal{E} de E telle que la matrice de u
dans la base \mathcal{E} soit un tableau diagonal de matrices élémentaires de
Jordan

\mathrmMat~ (u,\mathcal{E})
=\
\mathrmdiag(J_p_1,\\ldots,J_p_k~)

Démonstration Elle va faire l'objet des deux sections suivantes

\paragraph{3.3.4 Première démonstration}

Par récurrence sur n = dim~ E. Le résultat est
évident pour n = 1. Supposons le vrai pour tous les endomorphismes
nilpotents d'espaces de dimensions inférieures ou égales à n - 1. Soit r
l'indice de nilpotence de u. Si r = n, on a déjà vu que le résultat
était vrai (avec une seule matrice élémentaire de Jordan). On peut donc
supposer que r < n. Puisque
u^r-1\neq~0, soit a \in E tel que
u^r-1(a)\neq~0. Comme précédemment la
famille \mathcal{E}_1 =
(u^r-1(a),\\ldots~,u(a),a)
est libre et il est clair que le sous-espace F =\
\mathrmVect(u^r-1(a),\\ldots~,u(a),a)
est stable par u (chaque vecteur est décalé d'un cran vers la gauche,
sauf le premier qui est annulé par u). On a
\mathrmMat~
(u_F,\mathcal{E}_1) = J_r.

Puisque u^r-1(a)\neq~0 on peut trouver
une forme linéaire f telle que
f(u^r-1(a))\neq~0. Soit

\begin{align*} G& =& \⋂
_k=0^r-1 \mathrmKerf \cdot u^k
\%& \\ & =& \x \in
E∣f(x) = f(u(x)) =
\\ldots~ =
f(u^r-1(x) = 0\\%&
\\ \end{align*}

Lemme~3.3.12 G est un supplémentaire de F stable par u.

Démonstration La stabilité par u est claire, car si x \in G on a

\begin{align*} f(u(x)) = 0,f(u(u(x))) =
0,f(u^r-2(u(x)) = f(u^r-1(x) = 0,& & \%&
\\ f(u^r-1(u(x))) =
f(u^r(x)) = f(0) = 0& & \%&
\\ \end{align*}

Montrons que F \bigcap G = \0\. Pour cela
soit x = \lambda_1u^r-1(a) +
\\ldots~ +
\lambda_r-1u(a) + \lambda_ra \in F et supposons que x appartienne à
G. On a 0 = f(u^r-1(x)) = \lambda_1f(u^2r-2(a))
+ \\ldots~ +
\lambda_r-1f(u^r(a)) + \lambda_rf(u^r-1(a))
et tenant compte de u^r(a) =
\\ldots~ =
u^2r-2(a) = 0 on obtient \lambda_rf(u^r-1(a)) =
0 soit \lambda_r = 0. Comme précédemment une récurrence descendante
montre que \lambda_r = \lambda_r-1 =
\\ldots~ =
\lambda_1 = 0 soit x = 0. Donc F et G sont en somme directe. Mais G =
\bigcap_k=0^r-1\
\mathrmKerf \cdot u^k, et donc

dim~ G = n
-\mathrmrg~(f \cdot
u^k, 0 \leq k \leq r - 1) ≥ n - r = dim~ E
- dim~ F

On a donc E = F \oplus~ G.

(Fin de la démonstration) On peut maintenant terminer la démonstration
du théorème. En appliquant notre hypothèse de récurrence à
l'endomorphisme nilpotent u_G de G, on peut trouver
une base de G telle que
\mathrmMat~
(u_G,\mathcal{E}_2) =\
\mathrmdiag(J_p_2,\\ldots,J_p_k~).
Alors \mathcal{E} = \mathcal{E}_1 \cup\mathcal{E}_2 est une base de E dans laquelle
\mathrmMat~ (u,\mathcal{E})
=\
\mathrmdiag(J_r,J_p_2,\\ldots,J_p_k~),
ce qui achève la démonstration.

\paragraph{3.3.5 Deuxième démonstration}

Posons V _i =\
\mathrmKeru^i.

Lemme~3.3.13 On a \0\ = V _0
\subset~ V _1
\subset~\\ldots~ \subset~ V
_r = E avec une suite strictement croissante.

Démonstration Les inclusions sont claires. Supposons que V _i =
V _i+1 pour i \leq r - 1. Soit x \in E. On a 0 = u^r(x) =
u^i+1(u^r-i-1(x)) donc u^r-i-1(x) \in V
_i+1 = V _i et donc u^r-1(x) =
u^i(u^r-i-1(x)) = 0. On aurait donc
u^r-1 = 0 ce qui est exclu.

Soit W_1 un supplémentaire de V _r-1 dans E = V
_r.

Lemme~3.3.14 On peut construire une suite de sous-espaces
W_2,\\ldots,W_r~
de E vérifiant

\begin{itemize}
\itemsep1pt\parskip0pt\parsep0pt
\item
  (i) \forall~~k \in [1,r],\quad V
  _r-k+1 = V _r-k \oplus~ W_k
\item
  (ii) \forall~~k \in [2,r],\quad
  u(W_k-1) \subset~ W_k
\item
  (iii) \forall~~k \in [1,r -
  1],\quad u_W_k est
  injective
\item
  On a alors E = W_1 \oplus~⋯ \oplus~
  W_r.
\end{itemize}

Démonstration On va construire W_k par récurrence sur k. Pour
ce qui concerne k = 1, il suffit de montrer que
u_W_1 est injective. Mais si x \in
W_1 \diagdown\0\, on a
x∉V _r-1, donc
u^r-1(x)\neq~0 et donc
u(x)\neq~0. Supposons donc
W_1,\\ldots,W_k-1~
construits. Soit x \in W_k-1
\diagdown\0\. On a
x∉V _r-k+1, donc
u^r-k+1(x)\neq~0, soit
u^r-k(u(x))\neq~0 et donc
u(x)∉V _r-k. On a ainsi
u(W_k-1) \bigcap V _r-k =
\0\. Mais d'autre part x \in
W_k-1 \subset~ V _r-k+2 et donc u(x) \in V _r-k+1. On
a donc u(W_k-1) \subset~ V _r-k+1, V _r-k \subset~ V
_r-k+1 avec u(W_k-1) \bigcap V _r-k =
\0\. On peut donc trouver un
supplémentaire W_k de V _r-k dans V _r-k+1
tel que u(W_k-1) \subset~ W_k. Alors, si x \in W_k
\diagdown\0\, x∉V
_r-k, soit u^r-k(x)\neq~0 et
donc si k < r, u(x)\neq~0. Ceci montre
bien que u_W_k est injective. On a donc bien
construit notre suite W_k. Il est clair par récurrence que V
_k = W_r-k+1
\oplus~\\ldots~ \oplus~
W_r et donc E = V _r = W_1
\oplus~⋯ \oplus~ W_r.

Soit alors maintenant (e_i,1)_1\leqi\leqs_1 une
base de W_1. Comme u_W_1 est
injective, (e_i,2 =
u(e_i,1))_1\leqi\leqs_1 est une base de
u(W_1) que l'on peut compléter en une base
(e_i,2)_1\leqi\leqm_2 de W_2. Une
récurrence immédiate nous permet de construire des bases
(e_i,k)_1\leqi\leqm_k des W_k telles que
pour k \leq r - 1, et 1 \leq i \leq m_k, u(e_i,k) =
e_i,k+1. On a e_i,r \in W_r \subset~ V _1
= \mathrmKer~u, donc
u(e_i,r) = 0. On obtient ainsi une base (e_i,j) de E.
Si on ordonne cette base en posant que (i,j) < (i',j')
\Leftrightarrow j > j'\text
ou (j = j'\text et i < i'), la matrice de
u est un tableau diagonal de matrices de Jordan.

\paragraph{3.3.6 Réduction de Jordan}

Soit E un K-espace vectoriel de dimension finie et u \in L(E) dont le
polynôme caractéristique est scindé sur K,
E_1,\\ldots,E_k~
les sous-espaces caractéristiques de u associés aux valeurs propres
\lambda_1,\\ldots,\lambda_k~.
Soit u_i la restriction de u à E_i et n_i =
u_i -
\lambda_i\mathrmId_E_i. Avec les
notations précédentes, on a n_i^r_i = 0, donc
n_i est nilpotent. On peut donc trouver une base \mathcal{E}_i
de E_i dans laquelle la matrice de n_i est
\mathrmdiag(J_p_1,\\\ldots,J_p_k~)
et alors la matrice de u_i dans cette base est
\mathrmdiag(J_p_1(\lambda_i),\\\ldots,J_p_k(\lambda_i~))
avec

J_p(\lambda~) = \left
(\matrix\,\lambda~&1&0&\\ldots~&0
\cr
0&\lambda~&1&\\ldots~&0
\cr
\⋮~&
&⋱&\mathrel⋱&\⋮~
\cr
0&\\ldots&\\\ldots~&\lambda~&1
\cr
0&\\ldots&\\\ldots&0&\lambda~~\right
) = \lambda~I_p + J_p \in M_K(p)

En réunissant ces bases on obtient

Théorème~3.3.15 Soit E un K-espace vectoriel de dimension finie et u \in
L(E) dont le polynôme caractéristique est scindé sur K. Alors il existe
une base \mathcal{E} de E, des scalaires
\mu_1,\\ldots,\mu_l~
(non nécessairement distincts) et des entiers
n_1,\\ldots,n_l~
tels que \mathrmMat~ (u,\mathcal{E})
=\
\mathrmdiag(J_n_1(\mu_1),\\ldots,J_n_l(\mu_l~)).

[
[
[
[

\end{document}

% \documentclass[]{article}
\usepackage[T1]{fontenc}
\usepackage{lmodern}
\usepackage{amssymb,amsmath}
\usepackage{ifxetex,ifluatex}
\usepackage{fixltx2e} % provides \textsubscript
% use upquote if available, for straight quotes in verbatim environments
\IfFileExists{upquote.sty}{\usepackage{upquote}}{}
\ifnum 0\ifxetex 1\fi\ifluatex 1\fi=0 % if pdftex
  \usepackage[utf8]{inputenc}
\else % if luatex or xelatex
  \ifxetex
    \usepackage{mathspec}
    \usepackage{xltxtra,xunicode}
  \else
    \usepackage{fontspec}
  \fi
  \defaultfontfeatures{Mapping=tex-text,Scale=MatchLowercase}
  \newcommand{\euro}{€}
\fi
% use microtype if available
\IfFileExists{microtype.sty}{\usepackage{microtype}}{}
\ifxetex
  \usepackage[setpagesize=false, % page size defined by xetex
              unicode=false, % unicode breaks when used with xetex
              xetex]{hyperref}
\else
  \usepackage[unicode=true]{hyperref}
\fi
\hypersetup{breaklinks=true,
            bookmarks=true,
            pdfauthor={},
            pdftitle={Elements de topologie generale},
            colorlinks=true,
            citecolor=blue,
            urlcolor=blue,
            linkcolor=magenta,
            pdfborder={0 0 0}}
\urlstyle{same}  % don't use monospace font for urls
\setlength{\parindent}{0pt}
\setlength{\parskip}{6pt plus 2pt minus 1pt}
\setlength{\emergencystretch}{3em}  % prevent overfull lines
\setcounter{secnumdepth}{0}
 
/* start css.sty */
.cmr-5{font-size:50%;}
.cmr-7{font-size:70%;}
.cmmi-5{font-size:50%;font-style: italic;}
.cmmi-7{font-size:70%;font-style: italic;}
.cmmi-10{font-style: italic;}
.cmsy-5{font-size:50%;}
.cmsy-7{font-size:70%;}
.cmex-7{font-size:70%;}
.cmex-7x-x-71{font-size:49%;}
.msbm-7{font-size:70%;}
.cmtt-10{font-family: monospace;}
.cmti-10{ font-style: italic;}
.cmbx-10{ font-weight: bold;}
.cmr-17x-x-120{font-size:204%;}
.cmsl-10{font-style: oblique;}
.cmti-7x-x-71{font-size:49%; font-style: italic;}
.cmbxti-10{ font-weight: bold; font-style: italic;}
p.noindent { text-indent: 0em }
td p.noindent { text-indent: 0em; margin-top:0em; }
p.nopar { text-indent: 0em; }
p.indent{ text-indent: 1.5em }
@media print {div.crosslinks {visibility:hidden;}}
a img { border-top: 0; border-left: 0; border-right: 0; }
center { margin-top:1em; margin-bottom:1em; }
td center { margin-top:0em; margin-bottom:0em; }
.Canvas { position:relative; }
li p.indent { text-indent: 0em }
.enumerate1 {list-style-type:decimal;}
.enumerate2 {list-style-type:lower-alpha;}
.enumerate3 {list-style-type:lower-roman;}
.enumerate4 {list-style-type:upper-alpha;}
div.newtheorem { margin-bottom: 2em; margin-top: 2em;}
.obeylines-h,.obeylines-v {white-space: nowrap; }
div.obeylines-v p { margin-top:0; margin-bottom:0; }
.overline{ text-decoration:overline; }
.overline img{ border-top: 1px solid black; }
td.displaylines {text-align:center; white-space:nowrap;}
.centerline {text-align:center;}
.rightline {text-align:right;}
div.verbatim {font-family: monospace; white-space: nowrap; text-align:left; clear:both; }
.fbox {padding-left:3.0pt; padding-right:3.0pt; text-indent:0pt; border:solid black 0.4pt; }
div.fbox {display:table}
div.center div.fbox {text-align:center; clear:both; padding-left:3.0pt; padding-right:3.0pt; text-indent:0pt; border:solid black 0.4pt; }
div.minipage{width:100%;}
div.center, div.center div.center {text-align: center; margin-left:1em; margin-right:1em;}
div.center div {text-align: left;}
div.flushright, div.flushright div.flushright {text-align: right;}
div.flushright div {text-align: left;}
div.flushleft {text-align: left;}
.underline{ text-decoration:underline; }
.underline img{ border-bottom: 1px solid black; margin-bottom:1pt; }
.framebox-c, .framebox-l, .framebox-r { padding-left:3.0pt; padding-right:3.0pt; text-indent:0pt; border:solid black 0.4pt; }
.framebox-c {text-align:center;}
.framebox-l {text-align:left;}
.framebox-r {text-align:right;}
span.thank-mark{ vertical-align: super }
span.footnote-mark sup.textsuperscript, span.footnote-mark a sup.textsuperscript{ font-size:80%; }
div.tabular, div.center div.tabular {text-align: center; margin-top:0.5em; margin-bottom:0.5em; }
table.tabular td p{margin-top:0em;}
table.tabular {margin-left: auto; margin-right: auto;}
div.td00{ margin-left:0pt; margin-right:0pt; }
div.td01{ margin-left:0pt; margin-right:5pt; }
div.td10{ margin-left:5pt; margin-right:0pt; }
div.td11{ margin-left:5pt; margin-right:5pt; }
table[rules] {border-left:solid black 0.4pt; border-right:solid black 0.4pt; }
td.td00{ padding-left:0pt; padding-right:0pt; }
td.td01{ padding-left:0pt; padding-right:5pt; }
td.td10{ padding-left:5pt; padding-right:0pt; }
td.td11{ padding-left:5pt; padding-right:5pt; }
table[rules] {border-left:solid black 0.4pt; border-right:solid black 0.4pt; }
.hline hr, .cline hr{ height : 1px; margin:0px; }
.tabbing-right {text-align:right;}
span.TEX {letter-spacing: -0.125em; }
span.TEX span.E{ position:relative;top:0.5ex;left:-0.0417em;}
a span.TEX span.E {text-decoration: none; }
span.LATEX span.A{ position:relative; top:-0.5ex; left:-0.4em; font-size:85%;}
span.LATEX span.TEX{ position:relative; left: -0.4em; }
div.float img, div.float .caption {text-align:center;}
div.figure img, div.figure .caption {text-align:center;}
.marginpar {width:20%; float:right; text-align:left; margin-left:auto; margin-top:0.5em; font-size:85%; text-decoration:underline;}
.marginpar p{margin-top:0.4em; margin-bottom:0.4em;}
.equation td{text-align:center; vertical-align:middle; }
td.eq-no{ width:5%; }
table.equation { width:100%; } 
div.math-display, div.par-math-display{text-align:center;}
math .texttt { font-family: monospace; }
math .textit { font-style: italic; }
math .textsl { font-style: oblique; }
math .textsf { font-family: sans-serif; }
math .textbf { font-weight: bold; }
.partToc a, .partToc, .likepartToc a, .likepartToc {line-height: 200%; font-weight:bold; font-size:110%;}
.chapterToc a, .chapterToc, .likechapterToc a, .likechapterToc, .appendixToc a, .appendixToc {line-height: 200%; font-weight:bold;}
.index-item, .index-subitem, .index-subsubitem {display:block}
.caption td.id{font-weight: bold; white-space: nowrap; }
table.caption {text-align:center;}
h1.partHead{text-align: center}
p.bibitem { text-indent: -2em; margin-left: 2em; margin-top:0.6em; margin-bottom:0.6em; }
p.bibitem-p { text-indent: 0em; margin-left: 2em; margin-top:0.6em; margin-bottom:0.6em; }
.paragraphHead, .likeparagraphHead { margin-top:2em; font-weight: bold;}
.subparagraphHead, .likesubparagraphHead { font-weight: bold;}
.quote {margin-bottom:0.25em; margin-top:0.25em; margin-left:1em; margin-right:1em; text-align:\jmathustify;}
.verse{white-space:nowrap; margin-left:2em}
div.maketitle {text-align:center;}
h2.titleHead{text-align:center;}
div.maketitle{ margin-bottom: 2em; }
div.author, div.date {text-align:center;}
div.thanks{text-align:left; margin-left:10%; font-size:85%; font-style:italic; }
div.author{white-space: nowrap;}
.quotation {margin-bottom:0.25em; margin-top:0.25em; margin-left:1em; }
h1.partHead{text-align: center}
.sectionToc, .likesectionToc {margin-left:2em;}
.subsectionToc, .likesubsectionToc {margin-left:4em;}
.subsubsectionToc, .likesubsubsectionToc {margin-left:6em;}
.frenchb-nbsp{font-size:75%;}
.frenchb-thinspace{font-size:75%;}
.figure img.graphics {margin-left:10%;}
/* end css.sty */

\title{Elements de topologie generale}
\author{}
\date{}

\begin{document}
\maketitle

\textbf{Warning: 
requires JavaScript to process the mathematics on this page.\\ If your
browser supports JavaScript, be sure it is enabled.}

\begin{center}\rule{3in}{0.4pt}\end{center}

{[}
{[}{]}
{[}

\subsubsection{4.1 Eléments de topologie générale}

\paragraph{4.1.1 Espaces topologiques}

Définition~4.1.1 Soit E un ensemble. On appelle topologie sur E toute
partie T de P(E) vérifiant les propriétés

\begin{itemize}
\itemsep1pt\parskip0pt\parsep0pt
\item
  (i) \varnothing~\inT et E \inT
\item
  (ii) A,B \inT \rigtharrow~ A \bigcap B \inT
\item
  (iii) Pour toute famille (U\_i)\_i\inI d'éléments de T,
  \⋃ ~
  \_i\inIU\_i appartient à T.
\end{itemize}

Les éléments de T s'appellent les ouverts de la topologie. On appelle
espace topologique un couple (E,T ) d'un ensemble E et d'une topologie T
sur E.

Remarque~4.1.1 On déduit de la propriété (ii) que toute intersection
finie d'ouverts est encore un ouvert.

Exemple~4.1.1 \\varnothing~,E\ est une topologie
sur E appelée la topologie grossière~; de même P(E) est une topologie
sur E appelée la topologie discrète.

\paragraph{4.1.2 La topologie de \mathbb{R}~}

Définition~4.1.2 On dit qu'une partie I de \mathbb{R}~ est un intervalle ouvert si
elle est de l'une des formes suivantes

\begin{itemize}
\itemsep1pt\parskip0pt\parsep0pt
\item
  (i) I ={]}a,b{[}= \x \in \mathbb{R}~∣a
  \textless{} x \textless{} b\
\item
  (ii) I ={]}a,+\infty~{[}= \x \in
  \mathbb{R}~∣a \textless{} x\ ou I
  ={]} -\infty~,a{[}= \x \in \mathbb{R}~∣x
  \textless{} a\
\item
  (iii) I ={]} -\infty~,+\infty~{[}= \mathbb{R}~
\end{itemize}

On vérifie facilement que cet ensemble noté ℐ est stable par
intersection finie (car on a un ordre total). Soit alors T l'ensemble
des réunions de familles d'intervalles ouverts. On vérifie facilement la
proposition suivante

Proposition~4.1.1 T est une topologie sur \mathbb{R}~ appelée la topologie
usuelle.

Théorème~4.1.2 Soit U une partie de \mathbb{R}~. On a équivalence de

\begin{itemize}
\itemsep1pt\parskip0pt\parsep0pt
\item
  (i) U est un ouvert pour la topologie usuelle
\item
  (ii) \forall~~x \in U,
  \existsI\_x~ \inℐ,\quad x \in
  I\_x \subset~ U
\end{itemize}

Démonstration ((i) \rigtharrow~ (ii)) Si U =\
⋃  \_k\inKI\_k~ et x \in U, alors
\existsk \in K, x \in I\_k et I\_x~ =
I\_k convient.

((ii) \rigtharrow~ (i)) Soit V =\
⋃  \_x\inUI\_x~. V est une
réunion de parties de U donc V \subset~ U~; mais \forall~~x \in
U,x \in I\_x \subset~ V , donc U \subset~ V . On a donc U = V \inT.

Corollaire~4.1.3 Dans \mathbb{R}~, une partie U est ouverte si et seulement si
elle vérifie

\forall~x \in U, \\exists~\epsilon
\textgreater{} 0,\quad {]}x - \epsilon,x + \epsilon{[}\subset~ U

Démonstration En effet {]}x - \epsilon,x + \epsilon{[} est un intervalle ouvert
contenant x, et inversement tout intervalle ouvert contenant x contient
un {]}x - \epsilon,x + \epsilon{[}, pour un \epsilon \textgreater{} 0 assez petit.

\paragraph{4.1.3 Fermés et voisinages}

Définition~4.1.3 Soit (E,T ) un espace topologique. On dit qu'une partie
A de E est fermée si son complémentaire est ouvert.

Proposition~4.1.4 (i) \varnothing~ et E sont fermés

\begin{itemize}
\itemsep1pt\parskip0pt\parsep0pt
\item
  (ii) si A,B sont des fermés, A \cup B est fermé (par récurrence, la
  réunion d'un nombre fini de fermés est fermée).
\item
  (iii) Pour toute famille (F\_i)\_i\inI de fermés,
  \⋂ ~
  \_i\inIF\_i est fermée.
\end{itemize}

Démonstration Par passage au complémentaire à partir des trois
propriétés des ouverts.

Remarque~4.1.2 Les parties \varnothing~ et E sont à la fois ouvertes et fermées~;
dans \mathbb{R}~ muni de la topologie usuelle, la partie {]}0,1{]} n'est ni
ouverte, ni fermée. Fermé n'est en aucun cas le contraire d'ouvert.

Définition~4.1.4 Soit (E,T ) un espace topologique, x \in E et V une
partie contenant x. On dit que V est un voisinage de x si il existe un
ouvert U tel que x \in U \subset~ V .

Proposition~4.1.5 Toute intersection finie de voisinages de x est un
voisinage de x~; toute partie contenant un voisinage de x est un
voisinage de x.

Démonstration Elémentaire.

Exemple~4.1.2 Dans \mathbb{R}~, V est un voisinage de x si et seulement si,
\exists~\epsilon \textgreater{} 0, {]}x - \epsilon,x + \epsilon{[}\subset~ V .

Théorème~4.1.6 Soit (E,T ) un espace topologique. Une partie U de E est
ouverte si et seulement si U est voisinage de tous ses points.

Démonstration Si U est ouverte, on a \forall~~x \in U, x
\in U \subset~ U, donc U est un voisinage de tous ses points. Inversement, si U
est voisinage de tous ses points, pour chaque x \in U, il existe
U\_x ouvert tel que x \in U\_x \subset~ U. Soit V
= \⋃ ~
\_x\inUU\_x. U est une réunion d'ouverts, donc un ouvert. On
a V \subset~ U comme réunion de parties de U et U \subset~ V car
\forall~x \in U,x \in U\_x~ \subset~ V . Donc U = V et U
est ouvert.

Définition~4.1.5 On notera V (a) l'ensemble des voisinages de a.

\paragraph{4.1.4 Intérieur, adhérence, frontière}

Proposition~4.1.7 Soit A une partie de E (espace topologique).

\begin{itemize}
\itemsep1pt\parskip0pt\parsep0pt
\item
  (i) L'ensemble des ouverts contenus dans A a un plus grand élément
  appelé l'intérieur de A et noté A^o.
\item
  (ii) L'ensemble des fermés contenant A a un plus petit élément appelé
  l'adhérence de A et noté \overlineA.
\end{itemize}

Démonstration \⋃ ~
\_ U\textouvert \atop U\subset~A U
est un ouvert contenu dans A et c'est bien entendu le plus grand. De
même \⋂  ~\_
F\textfermé \atop A\subset~F F est un
fermé contenant A et c'est évidemment le plus petit.

Proposition~4.1.8 c(\overlineA) = (cA)^o
et c(A^o) = \overlinecA

Démonstration Il suffit de remarquer que les ouverts sont les
complémentaires des fermés et que U \subset~ A \Leftrightarrow cA
\subset~cU et que A \subset~ F \Leftrightarrow cF \subset~cA.

Théorème~4.1.9

\begin{itemize}
\itemsep1pt\parskip0pt\parsep0pt
\item
  (i) A^o = \x \in
  A∣A \in V (x)\
\item
  (ii) \overlineA = \x \in
  E∣\forall~~V \in V (x), V \bigcap
  A\neq~\varnothing~\
\end{itemize}

Démonstration (i) Soit U = \x \in
A∣A \in V (x)\. On a U \subset~ A. On
remarque que U est ouvert~; en effet si x \in U, on a A \in V (x), donc il
existe U\_o ouvert tel que x \in U\_o \subset~ A~; mais alors
\forall~y \in U\_o, y \in U\_o~ \subset~ A, donc
A \in V (y) soit x \in U\_o \subset~ U et U \in V (x). Donc U est voisinage
de tous ses points, il est donc ouvert. On a donc U \subset~ A^o.
Mais inversement, si x \in A^o, comme A^o est
ouvert, on a x \in A^o \subset~ A, donc A \in V (x). On a donc
A^o \subset~ U soit A^o = U.

(ii) On a donc c(\overlineA) = (cA)^o =
\x∣cA \in V
(x)\ =
\x∣\exists~V
\in V (x), V \subset~cA\. On en déduit que
\overlineA = \x \in
E∣\forall~~V \in V (x), V \bigcap
A\neq~\varnothing~\.

Remarque~4.1.3 Pour la topologie naturelle de \mathbb{R}~, on a
(\overlineℚ)^o = \mathbb{R}~ et
\overlineℚ^o = \varnothing~~; ces applications ne
sont donc en rien réciproques.

Proposition~4.1.10 A est ouvert si et seulement si~A^o = A. A
est fermé si et seulement si~\overlineA = A.

Démonstration Evident.

Définition~4.1.6 Une partie A de E est dite dense dans E si elle vérifie
les conditions équivalentes

\begin{itemize}
\itemsep1pt\parskip0pt\parsep0pt
\item
  (i) \overlineA = E
\item
  (ii) \forall~x \in E, \\forall~~V \in
  V (x),\quad V \bigcap A\neq~\varnothing~.
\item
  (iii) Tout ouvert non vide de E contient un point de A
\end{itemize}

Démonstration L'équivalence de (i) et (ii) est claire d'après le
théorème précédent~; l'équivalence entre (ii) et (iii) est tout à fait
élémentaire~: tout ouvert est un voisinage, tout voisinage contient un
ouvert.

Définition~4.1.7 La frontière d'une partie A est
\mathrmFr~(A) =
\overlineA \diagdown A^o =
\overlineA \bigcap\overlinecA. C'est un
fermé de E.

Démonstration Elle est fermée comme intersection de deux fermés.

\paragraph{4.1.5 Topologie induite}

Définition~4.1.8 Soit (E,T ) un espace topologique, F une partie de E et
soit T\_F = \U \bigcap F∣U
\inT\. Alors T\_F est une topologie sur F appelée
la topologie induite par celle de E

Proposition~4.1.11 Soit A \subset~ F.

\begin{itemize}
\itemsep1pt\parskip0pt\parsep0pt
\item
  (i) A est fermée dans F si et seulement si~il existe B fermé de E tel
  que A = B \bigcap F.
\item
  (ii) Soit a \in A~; A est un voisinage de a dans F si et seulement si~il
  existe B voisinage de a dans E tel que A = B \bigcap F.
\end{itemize}

Démonstration ((i) \rigtharrow~) Supposons que A est fermé dans F. Alors F \diagdown A est
ouvert dans F et donc il existe U ouvert de E tel que F \diagdown A = U \bigcap F.
Mais on a alors A = (E \diagdown U) \bigcap F et donc A est l'intersection avec F d'un
fermé de E.

((i) ⇐) Si A = B \bigcap F, on a F \diagdown A = (E \diagdown B) \bigcap F donc F \diagdown A est ouvert
dans F, donc A est fermé dans F.

((ii) \rigtharrow~) Si A est un voisinage de a dans F, il existe U ouvert de F tel
que a \in U \subset~ A. Mais on a U = V \bigcap F, où V est un ouvert de E. Alors V \cup A
est un voisinage de a dans E tel que (V \cup A) \bigcap F = U \cup A = A.

((ii) ⇐) Si A = B \bigcap F où B est un voisinage de a dans E, il existe V
ouvert de E tel que a \in V \subset~ B. On a alors a \in V \bigcap F \subset~ A, ce qui montre
que A est un voisinage de a dans F.

Remarque~4.1.4 On prendra soin de ne pas confondre ouvert dans F et
ouvert dans E, fermé dans F et fermé dans E, etc.

{[}
{[}

\end{document}

% \documentclass[]{article}
\usepackage[T1]{fontenc}
\usepackage{lmodern}
\usepackage{amssymb,amsmath}
\usepackage{ifxetex,ifluatex}
\usepackage{fixltx2e} % provides \textsubscript
% use upquote if available, for straight quotes in verbatim environments
\IfFileExists{upquote.sty}{\usepackage{upquote}}{}
\ifnum 0\ifxetex 1\fi\ifluatex 1\fi=0 % if pdftex
  \usepackage[utf8]{inputenc}
\else % if luatex or xelatex
  \ifxetex
    \usepackage{mathspec}
    \usepackage{xltxtra,xunicode}
  \else
    \usepackage{fontspec}
  \fi
  \defaultfontfeatures{Mapping=tex-text,Scale=MatchLowercase}
  \newcommand{\euro}{€}
\fi
% use microtype if available
\IfFileExists{microtype.sty}{\usepackage{microtype}}{}
\ifxetex
  \usepackage[setpagesize=false, % page size defined by xetex
              unicode=false, % unicode breaks when used with xetex
              xetex]{hyperref}
\else
  \usepackage[unicode=true]{hyperref}
\fi
\hypersetup{breaklinks=true,
            bookmarks=true,
            pdfauthor={},
            pdftitle={Espaces metriques},
            colorlinks=true,
            citecolor=blue,
            urlcolor=blue,
            linkcolor=magenta,
            pdfborder={0 0 0}}
\urlstyle{same}  % don't use monospace font for urls
\setlength{\parindent}{0pt}
\setlength{\parskip}{6pt plus 2pt minus 1pt}
\setlength{\emergencystretch}{3em}  % prevent overfull lines
\setcounter{secnumdepth}{0}
 
/* start css.sty */
.cmr-5{font-size:50%;}
.cmr-7{font-size:70%;}
.cmmi-5{font-size:50%;font-style: italic;}
.cmmi-7{font-size:70%;font-style: italic;}
.cmmi-10{font-style: italic;}
.cmsy-5{font-size:50%;}
.cmsy-7{font-size:70%;}
.cmex-7{font-size:70%;}
.cmex-7x-x-71{font-size:49%;}
.msbm-7{font-size:70%;}
.cmtt-10{font-family: monospace;}
.cmti-10{ font-style: italic;}
.cmbx-10{ font-weight: bold;}
.cmr-17x-x-120{font-size:204%;}
.cmsl-10{font-style: oblique;}
.cmti-7x-x-71{font-size:49%; font-style: italic;}
.cmbxti-10{ font-weight: bold; font-style: italic;}
p.noindent { text-indent: 0em }
td p.noindent { text-indent: 0em; margin-top:0em; }
p.nopar { text-indent: 0em; }
p.indent{ text-indent: 1.5em }
@media print {div.crosslinks {visibility:hidden;}}
a img { border-top: 0; border-left: 0; border-right: 0; }
center { margin-top:1em; margin-bottom:1em; }
td center { margin-top:0em; margin-bottom:0em; }
.Canvas { position:relative; }
li p.indent { text-indent: 0em }
.enumerate1 {list-style-type:decimal;}
.enumerate2 {list-style-type:lower-alpha;}
.enumerate3 {list-style-type:lower-roman;}
.enumerate4 {list-style-type:upper-alpha;}
div.newtheorem { margin-bottom: 2em; margin-top: 2em;}
.obeylines-h,.obeylines-v {white-space: nowrap; }
div.obeylines-v p { margin-top:0; margin-bottom:0; }
.overline{ text-decoration:overline; }
.overline img{ border-top: 1px solid black; }
td.displaylines {text-align:center; white-space:nowrap;}
.centerline {text-align:center;}
.rightline {text-align:right;}
div.verbatim {font-family: monospace; white-space: nowrap; text-align:left; clear:both; }
.fbox {padding-left:3.0pt; padding-right:3.0pt; text-indent:0pt; border:solid black 0.4pt; }
div.fbox {display:table}
div.center div.fbox {text-align:center; clear:both; padding-left:3.0pt; padding-right:3.0pt; text-indent:0pt; border:solid black 0.4pt; }
div.minipage{width:100%;}
div.center, div.center div.center {text-align: center; margin-left:1em; margin-right:1em;}
div.center div {text-align: left;}
div.flushright, div.flushright div.flushright {text-align: right;}
div.flushright div {text-align: left;}
div.flushleft {text-align: left;}
.underline{ text-decoration:underline; }
.underline img{ border-bottom: 1px solid black; margin-bottom:1pt; }
.framebox-c, .framebox-l, .framebox-r { padding-left:3.0pt; padding-right:3.0pt; text-indent:0pt; border:solid black 0.4pt; }
.framebox-c {text-align:center;}
.framebox-l {text-align:left;}
.framebox-r {text-align:right;}
span.thank-mark{ vertical-align: super }
span.footnote-mark sup.textsuperscript, span.footnote-mark a sup.textsuperscript{ font-size:80%; }
div.tabular, div.center div.tabular {text-align: center; margin-top:0.5em; margin-bottom:0.5em; }
table.tabular td p{margin-top:0em;}
table.tabular {margin-left: auto; margin-right: auto;}
div.td00{ margin-left:0pt; margin-right:0pt; }
div.td01{ margin-left:0pt; margin-right:5pt; }
div.td10{ margin-left:5pt; margin-right:0pt; }
div.td11{ margin-left:5pt; margin-right:5pt; }
table[rules] {border-left:solid black 0.4pt; border-right:solid black 0.4pt; }
td.td00{ padding-left:0pt; padding-right:0pt; }
td.td01{ padding-left:0pt; padding-right:5pt; }
td.td10{ padding-left:5pt; padding-right:0pt; }
td.td11{ padding-left:5pt; padding-right:5pt; }
table[rules] {border-left:solid black 0.4pt; border-right:solid black 0.4pt; }
.hline hr, .cline hr{ height : 1px; margin:0px; }
.tabbing-right {text-align:right;}
span.TEX {letter-spacing: -0.125em; }
span.TEX span.E{ position:relative;top:0.5ex;left:-0.0417em;}
a span.TEX span.E {text-decoration: none; }
span.LATEX span.A{ position:relative; top:-0.5ex; left:-0.4em; font-size:85%;}
span.LATEX span.TEX{ position:relative; left: -0.4em; }
div.float img, div.float .caption {text-align:center;}
div.figure img, div.figure .caption {text-align:center;}
.marginpar {width:20%; float:right; text-align:left; margin-left:auto; margin-top:0.5em; font-size:85%; text-decoration:underline;}
.marginpar p{margin-top:0.4em; margin-bottom:0.4em;}
.equation td{text-align:center; vertical-align:middle; }
td.eq-no{ width:5%; }
table.equation { width:100%; } 
div.math-display, div.par-math-display{text-align:center;}
math .texttt { font-family: monospace; }
math .textit { font-style: italic; }
math .textsl { font-style: oblique; }
math .textsf { font-family: sans-serif; }
math .textbf { font-weight: bold; }
.partToc a, .partToc, .likepartToc a, .likepartToc {line-height: 200%; font-weight:bold; font-size:110%;}
.chapterToc a, .chapterToc, .likechapterToc a, .likechapterToc, .appendixToc a, .appendixToc {line-height: 200%; font-weight:bold;}
.index-item, .index-subitem, .index-subsubitem {display:block}
.caption td.id{font-weight: bold; white-space: nowrap; }
table.caption {text-align:center;}
h1.partHead{text-align: center}
p.bibitem { text-indent: -2em; margin-left: 2em; margin-top:0.6em; margin-bottom:0.6em; }
p.bibitem-p { text-indent: 0em; margin-left: 2em; margin-top:0.6em; margin-bottom:0.6em; }
.paragraphHead, .likeparagraphHead { margin-top:2em; font-weight: bold;}
.subparagraphHead, .likesubparagraphHead { font-weight: bold;}
.quote {margin-bottom:0.25em; margin-top:0.25em; margin-left:1em; margin-right:1em; text-align:justify;}
.verse{white-space:nowrap; margin-left:2em}
div.maketitle {text-align:center;}
h2.titleHead{text-align:center;}
div.maketitle{ margin-bottom: 2em; }
div.author, div.date {text-align:center;}
div.thanks{text-align:left; margin-left:10%; font-size:85%; font-style:italic; }
div.author{white-space: nowrap;}
.quotation {margin-bottom:0.25em; margin-top:0.25em; margin-left:1em; }
h1.partHead{text-align: center}
.sectionToc, .likesectionToc {margin-left:2em;}
.subsectionToc, .likesubsectionToc {margin-left:4em;}
.subsubsectionToc, .likesubsubsectionToc {margin-left:6em;}
.frenchb-nbsp{font-size:75%;}
.frenchb-thinspace{font-size:75%;}
.figure img.graphics {margin-left:10%;}
/* end css.sty */

\title{Espaces metriques}
\author{}
\date{}

\begin{document}
\maketitle

\textbf{Warning: \href{http://www.math.union.edu/locate/jsMath}{jsMath}
requires JavaScript to process the mathematics on this page.\\ If your
browser supports JavaScript, be sure it is enabled.}

\begin{center}\rule{3in}{0.4pt}\end{center}

{[}\href{coursse20.html}{next}{]} {[}\href{coursse18.html}{prev}{]}
{[}\href{coursse18.html\#tailcoursse18.html}{prev-tail}{]}
{[}\hyperref[tailcoursse19.html]{tail}{]}
{[}\href{coursch5.html\#coursse19.html}{up}{]}

\subsubsection{4.2 Espaces métriques}

\paragraph{4.2.1 Distances}

Définition~4.2.1 Soit E un ensemble. On appelle distance sur E toute
application d : E × E → \{ℝ\}\^{}\{+\} vérifiant pour tout x,y,z ∈ E

\begin{itemize}
\itemsep1pt\parskip0pt\parsep0pt
\item
  (i) d(x,y) = 0 \textbackslash{}mathrel\{⇔\} x = y (propriété de
  séparation)
\item
  (ii) d(x,y) = d(y,x) (propriété de symétrie)
\item
  (iii) d(x,z) ≤ d(x,y) + d(y,z) (inégalité triangulaire)
\end{itemize}

On appelle espace métrique un couple (E,d) d'un ensemble E et d'une
distance d sur E.

Proposition~4.2.1 Soit d une distance sur E Alors

\textbackslash{}mathop\{∀\}x,y,z ∈ E, \textbar{}d(x,z) −
d(y,z)\textbar{}≤ d(x,y)

Démonstration On a d(x,z) − d(y,z) ≤ d(x,y) d'après l'inégalité
triangulaire. En échangeant x et y, on a aussi d(y,z) − d(x,z) ≤ d(x,y),
d'où le résultat.

Exemple~4.2.1 Sur tout ensemble, d(x,y) = \textbackslash{}left
\textbackslash{}\{ \textbackslash{}cases\{ 1\&si
x\textbackslash{}mathrel\{≠\}y \textbackslash{}cr 0\&si x = y
\textbackslash{}cr \} \textbackslash{}right . est une distance sur E
appelée la distance discrète. Sur K = ℝ ou K = ℂ, d(x,y) = \textbar{}x −
y\textbar{} est une distance appelée la distance usuelle. Sur
\{K\}\^{}\{n\} on trouve classiquement trois distances utiles
\{d\}\_\{1\}(x,y) =\{\textbackslash{}mathop\{ \textbackslash{}mathop\{∑
\}\} \}\_\{i\}\textbar{}\{x\}\_\{i\} − \{y\}\_\{i\}\textbar{},
\{d\}\_\{2\}(x,y) =
\textbackslash{}sqrt\{\{\textbackslash{}mathop\{\textbackslash{}mathop\{∑
\}\} \}\_\{i\}\textbar{}\{x\}\_\{i\} −
\{y\}\_\{i\}\{\textbar{}\}\^{}\{2\}\} et \{d\}\_\{∞\}(x,y)
=\{\textbackslash{}mathop\{ max\}\}\_\{i\}\textbar{}\{x\}\_\{i\} −
\{y\}\_\{i\}\textbar{} si x =
(\{x\}\_\{1\},\textbackslash{}mathop\{\textbackslash{}mathop\{\ldots{}\}\},\{x\}\_\{n\})
et y =
(\{y\}\_\{1\},\textbackslash{}mathop\{\textbackslash{}mathop\{\ldots{}\}\},\{y\}\_\{n\}).

Définition~4.2.2 On appelle boule ouverte de centre a de rayon r
\textgreater{} 0~: B(a,r) = \textbackslash{}\{x ∈
E\textbackslash{}mathrel\{∣\}d(a,x) \textless{} r\textbackslash{}\}.

On appelle boule fermée de centre a de rayon r \textgreater{} 0~:
B'(a,r) = \textbackslash{}\{x ∈ E\textbackslash{}mathrel\{∣\}d(a,x) ≤
r\textbackslash{}\}.

On appelle sphère de centre a de rayon r \textgreater{} 0~: S(a,r) =
\textbackslash{}\{x ∈ E\textbackslash{}mathrel\{∣\}d(a,x) =
r\textbackslash{}\}

Définition~4.2.3 Soit (E,d) un espace métrique et \{d\}\_\{F\} la
restriction de d à F × F. Alors \{d\}\_\{F\} est encore une distance sur
F appelée la distance induite par d.

Remarque~4.2.1 On a clairement \{B\}\_\{\{d\}\_\{F\}\}(a,r) =
\{B\}\_\{d\}(a,r) ∩ F et le résultat similaire pour les boules fermées,
si a ∈ F.

Définition~4.2.4 Soit
(\{E\}\_\{1\},\{d\}\_\{1\}),\textbackslash{}mathop\{\textbackslash{}mathop\{\ldots{}\}\},(\{E\}\_\{k\},\{d\}\_\{k\})
des espaces métriques. Soit E = \{E\}\_\{1\}
×\textbackslash{}mathrel\{⋯\} × \{E\}\_\{k\}. On définit alors sur E une
distance produit par d(x,y) =\{\textbackslash{}mathop\{
max\}\}\_\{i\}\{d\}\_\{i\}(\{x\}\_\{i\},\{y\}\_\{i\}) si x =
(\{x\}\_\{1\},\textbackslash{}mathop\{\textbackslash{}mathop\{\ldots{}\}\},\{x\}\_\{k\})
et y =
(\{y\}\_\{1\},\textbackslash{}mathop\{\textbackslash{}mathop\{\ldots{}\}\},\{y\}\_\{k\}).

Définition~4.2.5

\begin{itemize}
\itemsep1pt\parskip0pt\parsep0pt
\item
  (i) Soit x ∈ E et A ⊂ E, A\textbackslash{}mathrel\{≠\}∅. On appelle
  distance de x à A le réel d(x,A) =\textbackslash{}mathop\{ inf\}
  \textbackslash{}\{d(x,a)\textbackslash{}mathrel\{∣\}a ∈
  A\textbackslash{}\}
\item
  (ii) A,B ⊂ E non vides. On appelle distance de A et B le réel d(A,B)
  =\textbackslash{}mathop\{ inf\}
  \textbackslash{}\{d(a,b)\textbackslash{}mathrel\{∣\}a ∈ A,b ∈
  B\textbackslash{}\}
\item
  (iii) On appelle diamètre de A ⊂ E, A\textbackslash{}mathrel\{≠\}∅, le
  nombre δ(A) =\textbackslash{}mathop\{
  sup\}\textbackslash{}\{d(a,a')\textbackslash{}mathrel\{∣\}a,a' ∈
  A\textbackslash{}\} ∈ ℝ ∪\textbackslash{}\{ + ∞\textbackslash{}\}~; on
  dit que A est bornée si δ(A) \textless{} +∞.
\end{itemize}

Définition~4.2.6 Soit (E,d) et (F,δ) deux espaces métriques. On appelle
isométrie de E sur F toute application f : E → F bijective qui conserve
la distance~:

\textbackslash{}mathop\{∀\}x,y ∈ E, δ(f(x),f(y)) = d(x,y)

\paragraph{4.2.2 Topologie définie par une distance}

Définition~4.2.7 Soit (E,d) un espace métrique. On appelle topologie
définie sur E par la distance d l'ensemble des parties U de E (les
ouverts de la topologie) vérifiant

\textbackslash{}mathop\{∀\}x ∈ U, \textbackslash{}mathop\{∃\}r
\textgreater{} 0,\textbackslash{}quad B(x,r) ⊂ U

Démonstration C'est bien une topologie~: clairement E et ∅ sont des
ouverts~; si U et U' sont des ouverts et x ∈ U ∩ U', il existe r
\textgreater{} 0 et r' \textgreater{} 0 tels que B(x,r) ⊂ U et B(x,r') ⊂
U' et alors \{r\}\_\{0\} =\textbackslash{}mathop\{ min\}(r,r')
\textgreater{} 0 est tel que B(x,\{r\}\_\{0\}) ⊂ U ∩ U'. Si les
\{U\}\_\{i\}, i ∈ I sont des ouverts, soit x
∈\{\textbackslash{}mathop\{\textbackslash{}mathop\{⋃ \}\}
\}\_\{i∈I\}\{U\}\_\{i\}. Il existe \{i\}\_\{0\} tel que x ∈
\{U\}\_\{\{i\}\_\{0\}\} puis r \textgreater{} 0 tel que B(x,r) ⊂
\{U\}\_\{\{i\}\_\{0\}\}. On a alors B(x,r)
⊂\{\textbackslash{}mathop\{\textbackslash{}mathop\{⋃ \}\}
\}\_\{i∈I\}\{U\}\_\{i\}.

Proposition~4.2.2 Dans un espace métrique, toute boule ouverte est un
ouvert, toute boule fermée est un fermé.

Démonstration Soit x ∈ B(a,r) et ρ = r − d(a,x) \textgreater{} 0. Si y ∈
B(x,ρ), on a d(a,y) ≤ d(a,x) + d(x,y) \textless{} d(a,x) + ρ = r soit
B(x,ρ) ⊂ B(a,r). De même on montre que si
x\textbackslash{}mathrel\{∉\}B'(a,r) et si ρ = d(a,x) − r \textgreater{}
0, alors B(x,ρ) ⊂ E ∖ B'(a,r). Donc E ∖ B'(a,r) est ouvert et B'(a,r)
est fermé.

Remarque~4.2.2

\begin{itemize}
\itemsep1pt\parskip0pt\parsep0pt
\item
  (i) V ∈ V (a) \textbackslash{}mathrel\{⇔\}
  \textbackslash{}mathop\{∃\}r \textgreater{} 0, B(a,r) ⊂ V
\item
  (ii) a ∈ \{A\}\^{}\{o\} \textbackslash{}mathrel\{⇔\}
  \textbackslash{}mathop\{∃\}r \textgreater{} 0, B(a,r) ⊂ A
\item
  (iii) \textbackslash{}overline\{A\} = \textbackslash{}\{x ∈
  E\textbackslash{}mathrel\{∣\}\textbackslash{}mathop\{∀\}r
  \textgreater{} 0, B(x,r) ∩
  A\textbackslash{}mathrel\{≠\}∅\textbackslash{}\}
\item
  (iv) \textbackslash{}mathop\{\textbackslash{}mathrm\{Fr\}\}(A) =
  \textbackslash{}\{x ∈
  E\textbackslash{}mathrel\{∣\}\textbackslash{}mathop\{∀\}r
  \textgreater{} 0, B(x,r) ∩
  A\textbackslash{}mathrel\{≠\}∅\textbackslash{}text\{ et \}B(x,r)
  ∩cA\textbackslash{}mathrel\{≠\}∅\textbackslash{}\}
\end{itemize}

Proposition~4.2.3 Soit (E,d) un espace métrique et F ⊂ E. Alors la
topologie induite sur F est la topologie définie par la distance
\{d\}\_\{F\}.

Démonstration On remarque que si a ∈ F, \{B\}\_\{\{d\}\_\{F\}\}(a,r) =
\{B\}\_\{d\}(a,r) ∩ F. Soit V un ouvert pour la topologie induite, soit
U ouvert de E tel que V = U ∩ F. On a a ∈ U, donc il existe r
\textgreater{} 0 tel que \{B\}\_\{d\}(a,r) ⊂ U. On a alors
\{B\}\_\{\{d\}\_\{F\}\}(a,r) = \{B\}\_\{d\}(a,r) ∩ F ⊂ U ∩ F ⊂ U.
Inversement, soit V un ouvert pour la distance \{d\}\_\{F\}. Pour tout x
∈ V , il existe \{r\}\_\{x\} \textgreater{} 0 tel que
\{B\}\_\{\{d\}\_\{F\}\}(x,\{r\}\_\{x\}) ⊂ V . On a alors V
=\{\textbackslash{}mathop\{ \textbackslash{}mathop\{⋃ \}\} \}\_\{x∈V
\}\{B\}\_\{\{d\}\_\{F\}\}(x,\{r\}\_\{x\}) (cette réunion contient V de
manière évidente et est contenue dans V car réunion de parties de V ).
On pose alors U =\{\textbackslash{}mathop\{ \textbackslash{}mathop\{⋃
\}\} \}\_\{x∈V \}\{B\}\_\{d\}(x,\{r\}\_\{x\}). C'est un ouvert de E et
on a V = U ∩ F.

Remarque~4.2.3 Ceci montre que la topologie définie par la distance
\{d\}\_\{F\} ne dépend que de la topologie sur E et pas vraiment de la
distance d. Montrons de même que la topologie définie par la distance
produit ne dépend que des topologies sur les espaces et pas des
distances elles-mêmes

Proposition~4.2.4 Soit (\{E\}\_\{1\},\{d\}\_\{1\}) et
(\{E\}\_\{2\},\{d\}\_\{2\}) deux espaces métriques et (\{E\}\_\{1\} ×
\{E\}\_\{2\},δ) l'espace métrique produit. Soit U ⊂ \{E\}\_\{1\} ×
\{E\}\_\{2\}. Alors U est ouvert si et seulement si~

\textbackslash{}mathop\{∀\}(\{a\}\_\{1\},\{a\}\_\{2\}) ∈ U,
\textbackslash{}mathop\{∃\}\{V \}\_\{1\} ∈ V (\{a\}\_\{1\}),
\textbackslash{}mathop\{∃\}\{V \}\_\{2\} ∈ V
(\{a\}\_\{2\}),\textbackslash{}quad \{V \}\_\{1\} × \{V \}\_\{2\} ⊂ U

Démonstration Supposons que U est ouvert pour la distance produit. Si
(\{a\}\_\{1\},\{a\}\_\{2\}) ∈ U, il existe r \textgreater{} 0 tel que
\{B\}\_\{δ\}((\{a\}\_\{1\},\{a\}\_\{2\}),r) ⊂ U. Mais on a

\textbackslash{}begin\{eqnarray*\}\{
B\}\_\{δ\}((\{a\}\_\{1\},\{a\}\_\{2\}),r)\& =\&
\textbackslash{}\{(\{x\}\_\{1\},\{x\}\_\{2\})\textbackslash{}mathrel\{∣\}\textbackslash{}mathop\{max\}(\{d\}\_\{1\}(\{x\}\_\{1\},\{a\}\_\{1\}),\{d\}\_\{2\}(\{a\}\_\{2\},\{r\}\_\{2\}))
\textless{} r\textbackslash{}\}\%\& \textbackslash{}\textbackslash{} \&
=\& \{B\}\_\{\{d\}\_\{1\}\}(\{a\}\_\{1\},r) ×
\{B\}\_\{\{d\}\_\{2\}\}(\{a\}\_\{2\},r) \%\&
\textbackslash{}\textbackslash{} \textbackslash{}end\{eqnarray*\}

et donc \{V \}\_\{1\} = \{B\}\_\{\{d\}\_\{1\}\}(\{a\}\_\{1\},r) et \{V
\}\_\{2\} = \{B\}\_\{\{d\}\_\{2\}\}(\{a\}\_\{2\},r) sont des voisinages
de \{a\}\_\{1\} et \{a\}\_\{2\} tels que \{V \}\_\{1\} × \{V \}\_\{2\} ⊂
U. Inversement, si U vérifie cette propriété, soit
(\{a\}\_\{1\},\{a\}\_\{2\}) ∈ U et soit \{V \}\_\{1\} ∈ V
(\{a\}\_\{1\}), \{V \}\_\{2\} ∈ V (\{a\}\_\{2\}) tels que \{V \}\_\{1\}
× \{V \}\_\{2\} ⊂ U. Il existe \{r\}\_\{1\} \textgreater{} 0 et
\{r\}\_\{2\} \textgreater{} 0 tels que
\{B\}\_\{\{d\}\_\{i\}\}(\{a\}\_\{i\},\{r\}\_\{i\}) ⊂ \{V \}\_\{i\}. Soit
r =\textbackslash{}mathop\{ min\}(\{r\}\_\{1\},\{r\}\_\{2\})
\textgreater{} 0. On a

\textbackslash{}begin\{eqnarray*\}\{
B\}\_\{δ\}((\{a\}\_\{1\},\{a\}\_\{2\}),r)\& =\&
\{B\}\_\{\{d\}\_\{1\}\}(\{a\}\_\{1\},r) ×
\{B\}\_\{\{d\}\_\{2\}\}(\{a\}\_\{2\},r) \%\&
\textbackslash{}\textbackslash{} \& ⊂\&
\{B\}\_\{\{d\}\_\{1\}\}(\{a\}\_\{1\},\{r\}\_\{1\}) ×
\{B\}\_\{\{d\}\_\{2\}\}(\{a\}\_\{2\},\{r\}\_\{2\}) ⊂ \{V \}\_\{1\} × \{V
\}\_\{2\} ⊂ U\%\& \textbackslash{}\textbackslash{}
\textbackslash{}end\{eqnarray*\}

donc U est un ouvert pour δ, ce qui achève la démonstration.

Remarque~4.2.4 En particulier, si \{U\}\_\{1\} et \{U\}\_\{2\} sont des
ouverts de \{E\}\_\{1\} et \{E\}\_\{2\}, alors \{U\}\_\{1\} ×
\{U\}\_\{2\} est un ouvert de \{E\}\_\{1\} × \{E\}\_\{2\}~; un tel
ouvert sera dit ouvert élémentaire.

\paragraph{4.2.3 Points isolés, points d'accumulation}

Soit toujours F une partie de E et x ∈\textbackslash{}overline\{F\}. On
sait que \textbackslash{}mathop\{∀\}V ∈ V (x) V ∩
F\textbackslash{}mathrel\{≠\}∅. On a alors deux possibilités suivant que
V ∩ F peut être réduit à \textbackslash{}\{x\textbackslash{}\} ou non.

Définition~4.2.8

\begin{itemize}
\itemsep1pt\parskip0pt\parsep0pt
\item
  (i) On dit que x ∈ F est point isolé de F, s'il existe V voisinage de
  x dans E tel que V ∩ F = \textbackslash{}\{x\textbackslash{}\}
\item
  (ii) On dit que x ∈ E est point d'accumulation de F si pour tout
  voisinage V de x dans E, V ∩ F
  ∖\textbackslash{}\{x\textbackslash{}\}\textbackslash{}mathrel\{≠\}∅.
\end{itemize}

Théorème~4.2.5 Soit E un espace métrique.

\begin{itemize}
\itemsep1pt\parskip0pt\parsep0pt
\item
  (i) x ∈ F est point isolé de F si et seulement
  si~\textbackslash{}\{x\textbackslash{}\} est ouvert dans F
\item
  (ii) x ∈ E est point d'accumulation de F si et seulement si~pour tout
  voisinage V de x dans E, V ∩ F est infini.
\end{itemize}

Démonstration (i) est tout à fait élémentaire et résulte de la
définition de la topologie induite. En ce qui concerne (ii), la partie (
⇐) est évidente. Montrons donc la partie ( ⇒). Soit x un point
d'accumulation de F, V un voisinage de x et r \textgreater{} 0 tel que
B(x,r) ⊂ V . Alors (B(x,r) ∖\textbackslash{}\{x\textbackslash{}\}) ∩
F\textbackslash{}mathrel\{≠\}∅. Soit \{x\}\_\{1\} ∈ (B(x,r)
∖\textbackslash{}\{x\textbackslash{}\}) ∩ F. Si \{x\}\_\{n\} est supposé
construit, on pose \{r\}\_\{n\} = d(x,\{x\}\_\{n\}) \textgreater{} 0 et
on choisit \{x\}\_\{n+1\} ∈ (B(x,\{r\}\_\{n\})
∖\textbackslash{}\{x\textbackslash{}\}) ∩
F\textbackslash{}mathrel\{≠\}∅. Alors la suite (d(x,\{x\}\_\{n\})) est
strictement décroissante, ce qui montre que les \{x\}\_\{n\} sont deux à
deux distincts. Ils sont tous dans F et dans B(x,r) donc dans V .

\paragraph{4.2.4 Propriété de séparation}

Théorème~4.2.6 Soit E un espace métrique, a et b deux points distincts
de E. Alors il existe U ouvert contenant a et V ouvert contenant b tels
que U ∩ V = ∅.

Démonstration Soit r =\{ 1 \textbackslash{}over 3\} d(a,b), U = B(a,r)
et V = B(b,r) conviennent.

Corollaire~4.2.7 Soit E un espace métrique et Δ =
\textbackslash{}\{(x,x) ∈ E × E\textbackslash{}\}. Alors Δ est fermée
dans E × E.

Démonstration Soit (a,b) ∈ E × E ∖ Δ. On a donc
a\textbackslash{}mathrel\{≠\}b. Il existe U ouvert contenant a et V
ouvert contenant b tels que U ∩ V = ∅. Alors U × V est un ouvert de E ×
E (élémentaire) et (U × V ) ∩ Δ = ∅, soit U × V ⊂ E × E ∖ Δ. Donc E × E
∖ Δ est voisinage de tous ses points et il est ouvert. Donc Δ est
fermée.

\paragraph{4.2.5 Changement de distances}

Définition~4.2.9 Soit E un ensemble. On dit que deux distances
\{d\}\_\{1\} et \{d\}\_\{2\} sur E sont topologiquement équivalentes si
elles définissent la même topologie (il s'agit clairement d'une relation
d'équivalence).

Théorème~4.2.8 Soit E un ensemble, d et d' deux distances sur E. Ces
distances sont topologiquement équivalentes si et seulement si~elles
vérifient

\begin{itemize}
\itemsep1pt\parskip0pt\parsep0pt
\item
  (i) \textbackslash{}mathop\{∀\}a ∈ E, \textbackslash{}mathop\{∀\}r
  \textgreater{} 0, \textbackslash{}mathop\{∃\}r' \textgreater{}
  0,\textbackslash{}quad \{B\}\_\{d'\}(a,r') ⊂ \{B\}\_\{d\}(a,r)
\item
  (ii) \textbackslash{}mathop\{∀\}a ∈ E, \textbackslash{}mathop\{∀\}r'
  \textgreater{} 0, \textbackslash{}mathop\{∃\}r \textgreater{}
  0,\textbackslash{}quad \{B\}\_\{d\}(a,r) ⊂ \{B\}\_\{d'\}(a,r')
\end{itemize}

Démonstration Ces conditions sont évidemment nécessaires puisque les
boules ouvertes pour d doivent être des ouverts pour d' et
réciproquement. Supposons maintenant que (i) est vérifiée et soit U un
ouvert pour d. Soit a ∈ U. Il existe r \textgreater{} 0 tel que
\{B\}\_\{d\}(a,r) ⊂ U. Alors \textbackslash{}mathop\{∃\}r'
\textgreater{} 0,\textbackslash{}quad \{B\}\_\{d'\}(a,r') ⊂
\{B\}\_\{d\}(a,r) ⊂ U. On en déduit que U est ouvert pour d', donc
\{T\}\_\{d\} ⊂\{T\}\_\{d'\}. De même (ii) traduit l'inclusion
\{T\}\_\{d'\} ⊂\{T\}\_\{d\}.

Définition~4.2.10 Soit E un ensemble. On dit que deux distances
\{d\}\_\{1\} et \{d\}\_\{2\} sur E sont équivalentes s'il existe α et β
strictement positifs tels que

\textbackslash{}mathop\{∀\}x,y ∈ E,\textbackslash{}quad
α\{d\}\_\{1\}(x,y) ≤ \{d\}\_\{2\}(x,y) ≤ β\{d\}\_\{1\}(x,y)

Proposition~4.2.9 Deux distances équivalentes sont topologiquement
équivalentes.

Démonstration On a \{d\}\_\{1\}(a,x) \textless{}\{ r
\textbackslash{}over β\} ⇒ \{d\}\_\{2\}(x,y) \textless{} r soit
\{B\}\_\{\{d\}\_\{1\}\}(a,\{ r \textbackslash{}over β\} ) ⊂
\{B\}\_\{\{d\}\_\{2\}\}(a,r). De même \{B\}\_\{\{d\}\_\{2\}\}(a,αr) ⊂
\{B\}\_\{\{d\}\_\{1\}\}(a,r).

Remarque~4.2.5 Soit d une distance sur E et posons d'(x,y)
=\textbackslash{}mathop\{ min\}(1,d(x,y)). On vérifie facilement que d
est une distance sur E, que d et d' sont topologiquement équivalentes
(\{B\}\_\{d\}(a,r) ⊂ \{B\}\_\{d'\}(a,r) et
\{B\}\_\{d'\}(a,\textbackslash{}mathop\{min\}(\{ 1 \textbackslash{}over
2\} ,r)) ⊂ \{B\}\_\{d\}(a,r)). Mais en général, d et d' ne sont pas
équivalentes (d' est toujours bornée alors que d ne l'est pas en
général).

\paragraph{4.2.6 La droite numérique achevée}

On pose \textbackslash{}overline\{ℝ\} = ℝ
∪\textbackslash{}\{−∞,+∞\textbackslash{}\} muni de la relation d'ordre
évidente. Les intervalles ouverts sont donc les intervalles de la forme

\begin{itemize}
\itemsep1pt\parskip0pt\parsep0pt
\item
  (i) I ={]}a,b{[}= \textbackslash{}\{x ∈ ℝ\textbackslash{}mathrel\{∣\}a
  \textless{} x \textless{} b\textbackslash{}\} pour a,b
  ∈\textbackslash{}overline\{ℝ\}
\item
  (ii) I ={]}a,+∞{]} = \textbackslash{}\{x
  ∈\textbackslash{}overline\{ℝ\}\textbackslash{}mathrel\{∣\}a
  \textless{} x\textbackslash{}\} ou I = {[}−∞,a{[}= \textbackslash{}\{x
  ∈\textbackslash{}overline\{ℝ\}\textbackslash{}mathrel\{∣\}x
  \textless{} a\textbackslash{}\}
\item
  (iii) I = {[}−∞,+∞{]} = \textbackslash{}overline\{ℝ\}
\end{itemize}

Comme sur ℝ, ces intervalles ouverts engendrent une topologie appelée la
topologie usuelle de \textbackslash{}overline\{ℝ\}. On a alors

\begin{itemize}
\item
  (i) si a ∈ ℝ, V ∈ V (a) \textbackslash{}mathrel\{⇔\}
  \textbackslash{}mathop\{∃\}ε \textgreater{} 0,\textbackslash{}quad
  {]}a − ε,a + ε{[}⊂ V
\item
  (ii) si a = +∞,

  V ∈ V (+∞) \textbackslash{}mathrel\{⇔\} \textbackslash{}mathop\{∃\}A
  \textgreater{} 0,\textbackslash{}quad {]}A,+∞{]} ⊂ V
\item
  (iii) si a = −∞,

  V ∈ V (−∞) \textbackslash{}mathrel\{⇔\} \textbackslash{}mathop\{∃\}A
  \textless{} 0,\textbackslash{}quad {[}−∞,A{[}⊂ V
\end{itemize}

Théorème~4.2.10 La topologie usuelle sur \textbackslash{}overline\{ℝ\}
est définie par une distance.

Démonstration Soit φ : \textbackslash{}overline\{ℝ\} → {[}−1,1{]}
définie par φ(x) = \textbackslash{}left \textbackslash{}\{
\textbackslash{}cases\{ \{ x \textbackslash{}over
1+\textbar{}x\textbar{}\} \&si x ∈ ℝ \textbackslash{}cr 1 \&si x = +∞
\textbackslash{}cr −1 \&si x = −∞ \textbackslash{}cr \}
\textbackslash{}right .. L'application φ est une bijection strictement
croissante donc respecte les intervalles ouverts, donc les topologies
usuelles~: si U ⊂\textbackslash{}overline\{ℝ\}, U est ouvert dans
\textbackslash{}overline\{ℝ\} si et seulement si~φ(U) est ouvert dans
{[}−1,1{]}~; comme la topologie sur {[}−1,1{]} est définie par la
distance \textbar{}x − y\textbar{}, la topologie sur
\textbackslash{}overline\{ℝ\} est définie par la distance d(x,y) =
\textbar{}φ(x) − φ(y)\textbar{} (pour cette distance, φ devient une
isométrie).

Remarque~4.2.6 On vérifie immédiatement que la topologie usuelle de
\textbackslash{}overline\{ℝ\} induit sur ℝ la topologie usuelle de ℝ.

{[}\href{coursse20.html}{next}{]} {[}\href{coursse18.html}{prev}{]}
{[}\href{coursse18.html\#tailcoursse18.html}{prev-tail}{]}
{[}\href{coursse19.html}{front}{]}
{[}\href{coursch5.html\#coursse19.html}{up}{]}

\end{document}

% \documentclass[]{article}
\usepackage[T1]{fontenc}
\usepackage{lmodern}
\usepackage{amssymb,amsmath}
\usepackage{ifxetex,ifluatex}
\usepackage{fixltx2e} % provides \textsubscript
% use upquote if available, for straight quotes in verbatim environments
\IfFileExists{upquote.sty}{\usepackage{upquote}}{}
\ifnum 0\ifxetex 1\fi\ifluatex 1\fi=0 % if pdftex
  \usepackage[utf8]{inputenc}
\else % if luatex or xelatex
  \ifxetex
    \usepackage{mathspec}
    \usepackage{xltxtra,xunicode}
  \else
    \usepackage{fontspec}
  \fi
  \defaultfontfeatures{Mapping=tex-text,Scale=MatchLowercase}
  \newcommand{\euro}{€}
\fi
% use microtype if available
\IfFileExists{microtype.sty}{\usepackage{microtype}}{}
\ifxetex
  \usepackage[setpagesize=false, % page size defined by xetex
              unicode=false, % unicode breaks when used with xetex
              xetex]{hyperref}
\else
  \usepackage[unicode=true]{hyperref}
\fi
\hypersetup{breaklinks=true,
            bookmarks=true,
            pdfauthor={},
            pdftitle={Suites},
            colorlinks=true,
            citecolor=blue,
            urlcolor=blue,
            linkcolor=magenta,
            pdfborder={0 0 0}}
\urlstyle{same}  % don't use monospace font for urls
\setlength{\parindent}{0pt}
\setlength{\parskip}{6pt plus 2pt minus 1pt}
\setlength{\emergencystretch}{3em}  % prevent overfull lines
\setcounter{secnumdepth}{0}
 
/* start css.sty */
.cmr-5{font-size:50%;}
.cmr-7{font-size:70%;}
.cmmi-5{font-size:50%;font-style: italic;}
.cmmi-7{font-size:70%;font-style: italic;}
.cmmi-10{font-style: italic;}
.cmsy-5{font-size:50%;}
.cmsy-7{font-size:70%;}
.cmex-7{font-size:70%;}
.cmex-7x-x-71{font-size:49%;}
.msbm-7{font-size:70%;}
.cmtt-10{font-family: monospace;}
.cmti-10{ font-style: italic;}
.cmbx-10{ font-weight: bold;}
.cmr-17x-x-120{font-size:204%;}
.cmsl-10{font-style: oblique;}
.cmti-7x-x-71{font-size:49%; font-style: italic;}
.cmbxti-10{ font-weight: bold; font-style: italic;}
p.noindent { text-indent: 0em }
td p.noindent { text-indent: 0em; margin-top:0em; }
p.nopar { text-indent: 0em; }
p.indent{ text-indent: 1.5em }
@media print {div.crosslinks {visibility:hidden;}}
a img { border-top: 0; border-left: 0; border-right: 0; }
center { margin-top:1em; margin-bottom:1em; }
td center { margin-top:0em; margin-bottom:0em; }
.Canvas { position:relative; }
li p.indent { text-indent: 0em }
.enumerate1 {list-style-type:decimal;}
.enumerate2 {list-style-type:lower-alpha;}
.enumerate3 {list-style-type:lower-roman;}
.enumerate4 {list-style-type:upper-alpha;}
div.newtheorem { margin-bottom: 2em; margin-top: 2em;}
.obeylines-h,.obeylines-v {white-space: nowrap; }
div.obeylines-v p { margin-top:0; margin-bottom:0; }
.overline{ text-decoration:overline; }
.overline img{ border-top: 1px solid black; }
td.displaylines {text-align:center; white-space:nowrap;}
.centerline {text-align:center;}
.rightline {text-align:right;}
div.verbatim {font-family: monospace; white-space: nowrap; text-align:left; clear:both; }
.fbox {padding-left:3.0pt; padding-right:3.0pt; text-indent:0pt; border:solid black 0.4pt; }
div.fbox {display:table}
div.center div.fbox {text-align:center; clear:both; padding-left:3.0pt; padding-right:3.0pt; text-indent:0pt; border:solid black 0.4pt; }
div.minipage{width:100%;}
div.center, div.center div.center {text-align: center; margin-left:1em; margin-right:1em;}
div.center div {text-align: left;}
div.flushright, div.flushright div.flushright {text-align: right;}
div.flushright div {text-align: left;}
div.flushleft {text-align: left;}
.underline{ text-decoration:underline; }
.underline img{ border-bottom: 1px solid black; margin-bottom:1pt; }
.framebox-c, .framebox-l, .framebox-r { padding-left:3.0pt; padding-right:3.0pt; text-indent:0pt; border:solid black 0.4pt; }
.framebox-c {text-align:center;}
.framebox-l {text-align:left;}
.framebox-r {text-align:right;}
span.thank-mark{ vertical-align: super }
span.footnote-mark sup.textsuperscript, span.footnote-mark a sup.textsuperscript{ font-size:80%; }
div.tabular, div.center div.tabular {text-align: center; margin-top:0.5em; margin-bottom:0.5em; }
table.tabular td p{margin-top:0em;}
table.tabular {margin-left: auto; margin-right: auto;}
div.td00{ margin-left:0pt; margin-right:0pt; }
div.td01{ margin-left:0pt; margin-right:5pt; }
div.td10{ margin-left:5pt; margin-right:0pt; }
div.td11{ margin-left:5pt; margin-right:5pt; }
table[rules] {border-left:solid black 0.4pt; border-right:solid black 0.4pt; }
td.td00{ padding-left:0pt; padding-right:0pt; }
td.td01{ padding-left:0pt; padding-right:5pt; }
td.td10{ padding-left:5pt; padding-right:0pt; }
td.td11{ padding-left:5pt; padding-right:5pt; }
table[rules] {border-left:solid black 0.4pt; border-right:solid black 0.4pt; }
.hline hr, .cline hr{ height : 1px; margin:0px; }
.tabbing-right {text-align:right;}
span.TEX {letter-spacing: -0.125em; }
span.TEX span.E{ position:relative;top:0.5ex;left:-0.0417em;}
a span.TEX span.E {text-decoration: none; }
span.LATEX span.A{ position:relative; top:-0.5ex; left:-0.4em; font-size:85%;}
span.LATEX span.TEX{ position:relative; left: -0.4em; }
div.float img, div.float .caption {text-align:center;}
div.figure img, div.figure .caption {text-align:center;}
.marginpar {width:20%; float:right; text-align:left; margin-left:auto; margin-top:0.5em; font-size:85%; text-decoration:underline;}
.marginpar p{margin-top:0.4em; margin-bottom:0.4em;}
.equation td{text-align:center; vertical-align:middle; }
td.eq-no{ width:5%; }
table.equation { width:100%; } 
div.math-display, div.par-math-display{text-align:center;}
math .texttt { font-family: monospace; }
math .textit { font-style: italic; }
math .textsl { font-style: oblique; }
math .textsf { font-family: sans-serif; }
math .textbf { font-weight: bold; }
.partToc a, .partToc, .likepartToc a, .likepartToc {line-height: 200%; font-weight:bold; font-size:110%;}
.chapterToc a, .chapterToc, .likechapterToc a, .likechapterToc, .appendixToc a, .appendixToc {line-height: 200%; font-weight:bold;}
.index-item, .index-subitem, .index-subsubitem {display:block}
.caption td.id{font-weight: bold; white-space: nowrap; }
table.caption {text-align:center;}
h1.partHead{text-align: center}
p.bibitem { text-indent: -2em; margin-left: 2em; margin-top:0.6em; margin-bottom:0.6em; }
p.bibitem-p { text-indent: 0em; margin-left: 2em; margin-top:0.6em; margin-bottom:0.6em; }
.paragraphHead, .likeparagraphHead { margin-top:2em; font-weight: bold;}
.subparagraphHead, .likesubparagraphHead { font-weight: bold;}
.quote {margin-bottom:0.25em; margin-top:0.25em; margin-left:1em; margin-right:1em; text-align:\jmathustify;}
.verse{white-space:nowrap; margin-left:2em}
div.maketitle {text-align:center;}
h2.titleHead{text-align:center;}
div.maketitle{ margin-bottom: 2em; }
div.author, div.date {text-align:center;}
div.thanks{text-align:left; margin-left:10%; font-size:85%; font-style:italic; }
div.author{white-space: nowrap;}
.quotation {margin-bottom:0.25em; margin-top:0.25em; margin-left:1em; }
h1.partHead{text-align: center}
.sectionToc, .likesectionToc {margin-left:2em;}
.subsectionToc, .likesubsectionToc {margin-left:4em;}
.subsubsectionToc, .likesubsubsectionToc {margin-left:6em;}
.frenchb-nbsp{font-size:75%;}
.frenchb-thinspace{font-size:75%;}
.figure img.graphics {margin-left:10%;}
/* end css.sty */

\title{Suites}
\author{}
\date{}

\begin{document}
\maketitle

\textbf{Warning: 
requires JavaScript to process the mathematics on this page.\\ If your
browser supports JavaScript, be sure it is enabled.}

\begin{center}\rule{3in}{0.4pt}\end{center}

{[}
{[}
{[}{]}
{[}

\subsubsection{4.3 Suites}

\paragraph{4.3.1 Suites convergentes, limites}

Définition~4.3.1 Soit E un espace métrique et
(x\_n)\_n\in\mathbb{N}~ une suite de E. On dit qu'elle est
convergente s'il existe \ell \in E vérifiant les conditions équivalentes

\begin{itemize}
\itemsep1pt\parskip0pt\parsep0pt
\item
  (i) \forall~V \in V (\ell), \\exists~N
  \in \mathbb{N}~, n ≥ N \rigtharrow~ x\_n \in V
\item
  (ii) \forall~~\epsilon \textgreater{} 0,
  \existsN \in \mathbb{N}~, n ≥ N \rigtharrow~ d(x\_n~,\ell)
  \textless{} \epsilon
\end{itemize}

Une suite qui n'est pas convergente est dite divergente.

Démonstration La propriété (ii) ne fait que traduire (i) pour V =
B(\ell,\epsilon). Or toute boule est un voisinage et tout voisinage contient une
boule. Donc (i) et (ii) sont équivalents.

Proposition~4.3.1 Si la suite (x\_n)\_n\in\mathbb{N}~ est
convergente, l'élément \ell de E est unique~; on l'appelle la limite de la
suite. On note \ell = limx\_n~.

Démonstration Si \ell et \ell' conviennent avec \ell\neq~\ell', il existe d'après la
propriété de séparation U ouvert contenant \ell et V ouvert contenant \ell'
tels que U \bigcap V = \varnothing~. Mais
\existsN,\quad n ≥ N \rigtharrow~ x\_n~ \in
U et \existsN', n ≥ N' \rigtharrow~ x\_n~ \in V . Pour n
≥ max(N,N'), on a x\_n~ \in U \bigcap V . C'est
absurde.

Remarque~4.3.1 On prendra garde à ne pas introduire le symbole
limx\_n~ de manière opératoire avant
d'avoir démontré son existence. On remarquera d'autre part que les
notions de convergence et de limites sont purement topologiques
puisqu'on peut les exprimer en terme de voisinages~; elles sont donc
inchangées par changement de distance topologiquement équivalente.

\paragraph{4.3.2 Sous suites, valeurs d'adhérences}

Définition~4.3.2 Soit (x\_n)\_n\in\mathbb{N}~ une suite d'éléments
de E et soit \phi : \mathbb{N}~ \rightarrow~ \mathbb{N}~ strictement croissante. On dit alors que la suite
(x\_\phi(n))\_n\in\mathbb{N}~ est une sous suite de la suite
(x\_n)\_n\in\mathbb{N}~.

Théorème~4.3.2 Si la suite (x\_n)\_n\in\mathbb{N}~ est convergente
de limite \ell, alors toute sous suite converge et a la même limite.

C'est une conséquence du lemme suivant qui se démontre à l'aide d'une
récurrence facile.

Lemme~4.3.3 Soit \phi : \mathbb{N}~ \rightarrow~ \mathbb{N}~ strictement croissante. Alors
\forall~~n \in \mathbb{N}~, \phi(n) ≥ n.

Démonstration (du théorème) Soit V voisinage de \ell et N \in \mathbb{N}~ tel que n ≥ N
\rigtharrow~ x\_n \in V . Alors pour n ≥ N on a \phi(n) ≥ n ≥ N donc
x\_\phi(n) \in V . Donc \ell est encore limite de la suite
(x\_\phi(n))\_n\in\mathbb{N}~.

Définition~4.3.3 Soit (x\_n)\_n\in\mathbb{N}~ une suite d'éléments
de E et a \in E. On dit que a est valeur d'adhérence de la suite si on a
les conditions équivalentes

\begin{itemize}
\itemsep1pt\parskip0pt\parsep0pt
\item
  (i) \forall~V \in V (a), \\forall~~N
  \in \mathbb{N}~, \exists~n ≥ N,\quad
  x\_n \in V
\item
  (i)' \forall~~\epsilon \textgreater{} 0,
  \forall~N \in \mathbb{N}~, \\exists~n ≥
  N,\quad d(x\_n,a) \textless{} \epsilon
\item
  (ii) \forall~~V \in V (a), \n \in
  \mathbb{N}~∣x\_n \in V \ est
  infini.
\item
  (ii)' \forall~~\epsilon \textgreater{} 0,
  \n \in \mathbb{N}~∣d(x\_n,a)
  \textless{} \epsilon\ est infini.
\item
  (iii) a est limite d'une sous suite (x\_\phi(n)) de la suite
  (x\_n).
\end{itemize}

Démonstration (i)' n'est qu'une reformulation de (i) en termes de
boules, comme (ii)' vis à vis de (ii). Si \n \in
\mathbb{N}~∣x\_n \in V \ est
fini, il existe N \in \mathbb{N}~ tel que n ≥ N \rigtharrow~
x\_n∉V et donc (i) n'est pas vérifié.
Ceci montre que (i) \rigtharrow~(ii). De même, si \n \in
\mathbb{N}~∣x\_n \in V \ est
infini, il contient des éléments d'indices arbitrairement grands, donc
(ii) \rigtharrow~(i). Si a est limite de la sous suite (x\_\phi(n)) et V est
un voisinage de a, il existe N \in \mathbb{N}~ tel que n ≥ N \rigtharrow~ x\_\phi(n) \in V .
Donc \n \in \mathbb{N}~∣x\_n \in V
\ contient \phi({[}N,+\infty~{[}), il est donc infini, soit
(iii) \rigtharrow~(ii). Montrons maintenant que (i)' \rigtharrow~(iii), ce qui achèvera la
démonstration. On construit \phi(n) par récurrence de la manière suivante~:
on prend \epsilon = 1 \over n+1 et N = \left
\ \cases 0 &si n = 0
\cr \phi(n - 1) + 1&si n ≥ 1 \cr 
\right .~; il existe alors p ≥ \phi(n - 1) + 1 tel que
d(a,x\_p) \textless{} 1 \over n+1 ~; on pose
\phi(n) = p. On construit ainsi une fonction strictement croissante de \mathbb{N}~
dans \mathbb{N}~ vérifiant d(a,x\_\phi(n)) \textless{} 1
\over n+1 . D'où une sous suite de limite a.

Remarque~4.3.2 On a donc vu qu'une suite convergente admet une unique
valeur d'adhérence, sa limite. Il est clair qu'une valeur d'adhérence
d'une sous suite est encore une valeur d'adhérence de la suite.

\paragraph{4.3.3 Caractérisation des fermés d'un espace métrique}

Théorème~4.3.4 Soit E un espace métrique, A une partie de E et a \in E.
Alors a est dans l'adhérence de A si et seulement si~a est limite (dans
E) d'une suite d'éléments de A.

Démonstration Si a est limite d'une suite (a\_n)\_n\in\mathbb{N}~
d'éléments de A, soit V un voisinage de a. Alors
\existsN \in \mathbb{N}~, n ≥ N \rigtharrow~ a\_n~ \in V . En
particulier a\_n \in V \bigcap A qui est donc non vide. Donc a
appartient à \overlineA. Inversement, soit a
\in\overlineA. Alors, pour tout \epsilon \textgreater{} 0, A \bigcap
B(a,\epsilon)\neq~\varnothing~. Pour \epsilon = 1 \over
n+1 on peut donc trouver a\_n \in A tel que d(a,a\_n)
\textless{} 1 \over n+1 . Donc a est limite de la
suite (a\_n) d'éléments de A.

Corollaire~4.3.5 Soit E un espace métrique. Une partie A de E est fermée
si et seulement si~toute suite d'éléments de A qui converge dans E a une
limite qui appartient à A.

Démonstration Ceci traduit simplement l'inclusion
\overlineA \subset~ A.

{[}
{[}
{[}
{[}

\end{document}

% \documentclass[]{article}
\usepackage[T1]{fontenc}
\usepackage{lmodern}
\usepackage{amssymb,amsmath}
\usepackage{ifxetex,ifluatex}
\usepackage{fixltx2e} % provides \textsubscript
% use upquote if available, for straight quotes in verbatim environments
\IfFileExists{upquote.sty}{\usepackage{upquote}}{}
\ifnum 0\ifxetex 1\fi\ifluatex 1\fi=0 % if pdftex
  \usepackage[utf8]{inputenc}
\else % if luatex or xelatex
  \ifxetex
    \usepackage{mathspec}
    \usepackage{xltxtra,xunicode}
  \else
    \usepackage{fontspec}
  \fi
  \defaultfontfeatures{Mapping=tex-text,Scale=MatchLowercase}
  \newcommand{\euro}{€}
\fi
% use microtype if available
\IfFileExists{microtype.sty}{\usepackage{microtype}}{}
\ifxetex
  \usepackage[setpagesize=false, % page size defined by xetex
              unicode=false, % unicode breaks when used with xetex
              xetex]{hyperref}
\else
  \usepackage[unicode=true]{hyperref}
\fi
\hypersetup{breaklinks=true,
            bookmarks=true,
            pdfauthor={},
            pdftitle={Limites de fonctions},
            colorlinks=true,
            citecolor=blue,
            urlcolor=blue,
            linkcolor=magenta,
            pdfborder={0 0 0}}
\urlstyle{same}  % don't use monospace font for urls
\setlength{\parindent}{0pt}
\setlength{\parskip}{6pt plus 2pt minus 1pt}
\setlength{\emergencystretch}{3em}  % prevent overfull lines
\setcounter{secnumdepth}{0}
 
/* start css.sty */
.cmr-5{font-size:50%;}
.cmr-7{font-size:70%;}
.cmmi-5{font-size:50%;font-style: italic;}
.cmmi-7{font-size:70%;font-style: italic;}
.cmmi-10{font-style: italic;}
.cmsy-5{font-size:50%;}
.cmsy-7{font-size:70%;}
.cmex-7{font-size:70%;}
.cmex-7x-x-71{font-size:49%;}
.msbm-7{font-size:70%;}
.cmtt-10{font-family: monospace;}
.cmti-10{ font-style: italic;}
.cmbx-10{ font-weight: bold;}
.cmr-17x-x-120{font-size:204%;}
.cmsl-10{font-style: oblique;}
.cmti-7x-x-71{font-size:49%; font-style: italic;}
.cmbxti-10{ font-weight: bold; font-style: italic;}
p.noindent { text-indent: 0em }
td p.noindent { text-indent: 0em; margin-top:0em; }
p.nopar { text-indent: 0em; }
p.indent{ text-indent: 1.5em }
@media print {div.crosslinks {visibility:hidden;}}
a img { border-top: 0; border-left: 0; border-right: 0; }
center { margin-top:1em; margin-bottom:1em; }
td center { margin-top:0em; margin-bottom:0em; }
.Canvas { position:relative; }
li p.indent { text-indent: 0em }
.enumerate1 {list-style-type:decimal;}
.enumerate2 {list-style-type:lower-alpha;}
.enumerate3 {list-style-type:lower-roman;}
.enumerate4 {list-style-type:upper-alpha;}
div.newtheorem { margin-bottom: 2em; margin-top: 2em;}
.obeylines-h,.obeylines-v {white-space: nowrap; }
div.obeylines-v p { margin-top:0; margin-bottom:0; }
.overline{ text-decoration:overline; }
.overline img{ border-top: 1px solid black; }
td.displaylines {text-align:center; white-space:nowrap;}
.centerline {text-align:center;}
.rightline {text-align:right;}
div.verbatim {font-family: monospace; white-space: nowrap; text-align:left; clear:both; }
.fbox {padding-left:3.0pt; padding-right:3.0pt; text-indent:0pt; border:solid black 0.4pt; }
div.fbox {display:table}
div.center div.fbox {text-align:center; clear:both; padding-left:3.0pt; padding-right:3.0pt; text-indent:0pt; border:solid black 0.4pt; }
div.minipage{width:100%;}
div.center, div.center div.center {text-align: center; margin-left:1em; margin-right:1em;}
div.center div {text-align: left;}
div.flushright, div.flushright div.flushright {text-align: right;}
div.flushright div {text-align: left;}
div.flushleft {text-align: left;}
.underline{ text-decoration:underline; }
.underline img{ border-bottom: 1px solid black; margin-bottom:1pt; }
.framebox-c, .framebox-l, .framebox-r { padding-left:3.0pt; padding-right:3.0pt; text-indent:0pt; border:solid black 0.4pt; }
.framebox-c {text-align:center;}
.framebox-l {text-align:left;}
.framebox-r {text-align:right;}
span.thank-mark{ vertical-align: super }
span.footnote-mark sup.textsuperscript, span.footnote-mark a sup.textsuperscript{ font-size:80%; }
div.tabular, div.center div.tabular {text-align: center; margin-top:0.5em; margin-bottom:0.5em; }
table.tabular td p{margin-top:0em;}
table.tabular {margin-left: auto; margin-right: auto;}
div.td00{ margin-left:0pt; margin-right:0pt; }
div.td01{ margin-left:0pt; margin-right:5pt; }
div.td10{ margin-left:5pt; margin-right:0pt; }
div.td11{ margin-left:5pt; margin-right:5pt; }
table[rules] {border-left:solid black 0.4pt; border-right:solid black 0.4pt; }
td.td00{ padding-left:0pt; padding-right:0pt; }
td.td01{ padding-left:0pt; padding-right:5pt; }
td.td10{ padding-left:5pt; padding-right:0pt; }
td.td11{ padding-left:5pt; padding-right:5pt; }
table[rules] {border-left:solid black 0.4pt; border-right:solid black 0.4pt; }
.hline hr, .cline hr{ height : 1px; margin:0px; }
.tabbing-right {text-align:right;}
span.TEX {letter-spacing: -0.125em; }
span.TEX span.E{ position:relative;top:0.5ex;left:-0.0417em;}
a span.TEX span.E {text-decoration: none; }
span.LATEX span.A{ position:relative; top:-0.5ex; left:-0.4em; font-size:85%;}
span.LATEX span.TEX{ position:relative; left: -0.4em; }
div.float img, div.float .caption {text-align:center;}
div.figure img, div.figure .caption {text-align:center;}
.marginpar {width:20%; float:right; text-align:left; margin-left:auto; margin-top:0.5em; font-size:85%; text-decoration:underline;}
.marginpar p{margin-top:0.4em; margin-bottom:0.4em;}
.equation td{text-align:center; vertical-align:middle; }
td.eq-no{ width:5%; }
table.equation { width:100%; } 
div.math-display, div.par-math-display{text-align:center;}
math .texttt { font-family: monospace; }
math .textit { font-style: italic; }
math .textsl { font-style: oblique; }
math .textsf { font-family: sans-serif; }
math .textbf { font-weight: bold; }
.partToc a, .partToc, .likepartToc a, .likepartToc {line-height: 200%; font-weight:bold; font-size:110%;}
.chapterToc a, .chapterToc, .likechapterToc a, .likechapterToc, .appendixToc a, .appendixToc {line-height: 200%; font-weight:bold;}
.index-item, .index-subitem, .index-subsubitem {display:block}
.caption td.id{font-weight: bold; white-space: nowrap; }
table.caption {text-align:center;}
h1.partHead{text-align: center}
p.bibitem { text-indent: -2em; margin-left: 2em; margin-top:0.6em; margin-bottom:0.6em; }
p.bibitem-p { text-indent: 0em; margin-left: 2em; margin-top:0.6em; margin-bottom:0.6em; }
.paragraphHead, .likeparagraphHead { margin-top:2em; font-weight: bold;}
.subparagraphHead, .likesubparagraphHead { font-weight: bold;}
.quote {margin-bottom:0.25em; margin-top:0.25em; margin-left:1em; margin-right:1em; text-align:\\jmathmathustify;}
.verse{white-space:nowrap; margin-left:2em}
div.maketitle {text-align:center;}
h2.titleHead{text-align:center;}
div.maketitle{ margin-bottom: 2em; }
div.author, div.date {text-align:center;}
div.thanks{text-align:left; margin-left:10%; font-size:85%; font-style:italic; }
div.author{white-space: nowrap;}
.quotation {margin-bottom:0.25em; margin-top:0.25em; margin-left:1em; }
h1.partHead{text-align: center}
.sectionToc, .likesectionToc {margin-left:2em;}
.subsectionToc, .likesubsectionToc {margin-left:4em;}
.subsubsectionToc, .likesubsubsectionToc {margin-left:6em;}
.frenchb-nbsp{font-size:75%;}
.frenchb-thinspace{font-size:75%;}
.figure img.graphics {margin-left:10%;}
/* end css.sty */

\title{Limites de fonctions}
\author{}
\date{}

\begin{document}
\maketitle

\textbf{Warning: 
requires JavaScript to process the mathematics on this page.\\ If your
browser supports JavaScript, be sure it is enabled.}

\begin{center}\rule{3in}{0.4pt}\end{center}

{[}
{[}
{[}{]}
{[}

\subsubsection{4.4 Limites de fonctions}

Définition~4.4.1 Si E et F sont deux ensembles, on appellera fonction de
E vers F toute application d'une partie X de E (le domaine de définition
de la fonction) dans F. On notera Def~ (f) le
domaine de définition de la fonction f.

\paragraph{4.4.1 Notion de limite suivant une partie}

Définition~4.4.2 Soit E et F deux espaces métriques, A \subset~ F , a
\in\overlineA. Soit f une fonction de E vers F telle
que A \subset~ Def~ (f). On dit que f admet une limite
en a suivant A s'il existe \ell \in F vérifiant les conditions équivalentes

\begin{itemize}
\itemsep1pt\parskip0pt\parsep0pt
\item
  (i) \forall~V \in V (\ell), \\exists~U
  \in V (a),\quad f(U \bigcap A) \subset~ V
\item
  (ii) \forall~~\epsilon \textgreater{} 0,
  \exists~\eta \textgreater{} 0,\quad (x
  \in A\text et d(x,a) \textless{} \eta) \rigtharrow~ d(f(x),\ell)
  \textless{} \epsilon.
\end{itemize}

Démonstration De nouveau, (ii) n'est qu'une reformulation de (i) en
termes de boules~: toute boule est un voisinage, tout voisinage contient
une boule.

Remarque~4.4.1 Sans la condition a \in\overlineA, on
pourrait avoir U \bigcap A = \varnothing~ et la notion deviendrait triviale, tout élément
\ell vérifiant la condition.

Proposition~4.4.1 Si la fonction f admet une limite en a suivant A,
l'élément \ell de E est unique~; on l'appelle la limite de la fonction en a
suivant A. On pose \ell =\
lim_x\rightarrow~a,x\inAf(x).

Démonstration Si \ell et \ell' conviennent avec \ell\neq~\ell', il existe d'après la
propriété de séparation V ouvert contenant \ell et V ' ouvert contenant \ell'
tels que V \bigcap V ' = \varnothing~. Mais \exists~U \in V (a), f(U \bigcap
A) \subset~ V et \exists~U' \in V (a), f(U' \bigcap A) \subset~ V '. On a
alors f(U \bigcap U' \bigcap A) \subset~ V \bigcap V ' = \varnothing~ alors que U \bigcap U' \bigcap
A\neq~\varnothing~ puisque U \bigcap U' est un voisinage de a et
que a \in\overlineA. C'est absurde.

Remarque~4.4.2 On prendra garde à ne pas introduire le symbole
lim_x\rightarrow~a,x\inA~f(x) de manière opératoire
avant d'avoir démontré son existence. On remarquera d'autre part que les
notions de convergence et de limites sont purement topologiques
puisqu'on peut les exprimer en terme de voisinages~; elles sont donc
inchangées par changement de distance topologiquement équivalente.

Théorème~4.4.2 Soit U_0 un ouvert contenant a. Alors f admet
une limite en a suivant A si et seulement si il admet une limite suivant
U_0 \bigcap A et dans ce cas la limite est la même (on dit que la
notion de limite est une notion locale).

Démonstration Si f(U \bigcap A) \subset~ V , on a à fortiori f(U \bigcap U_0 \bigcap A)
\subset~ V . Inversement, si f(U \bigcap U_0 \bigcap A) \subset~ V , U' = U \bigcap
U_0 est un voisinage de a tel que f(U' \bigcap A) \subset~ V .

Exemple~4.4.1

\begin{enumerate}
\itemsep1pt\parskip0pt\parsep0pt
\item
  Pour E = \overline\mathbb{R}~, A = \mathbb{N}~, a = +\infty~ et f(n) =
  x_n on retrouve le cas particulier des suites.
\item
  Pour A ={]}a,+\infty~{[}\bigcapDef~ (f) on trouve le cas
  particulier d'une limite à droite (si cela a un sens, c'est à dire si
  a est dans l'adhérence de cet ensemble)~; de même pour les limites à
  gauche.
\item
  Pour A = Def~ (f)
  \diagdown\a\ on trouve le cas important de
  limite quand x tend vers a en étant distinct de a.
\item
  Pour a \in Def~ (f) et A
  = Def~ (f), la seule limite possible est f(a)
  (facile).
\end{enumerate}

\paragraph{4.4.2 Propriétés élémentaires}

Proposition~4.4.3 Si f admet \ell pour limite en a suivant A, alors \ell
\in\overlinef(A).

Démonstration Si V \in V (\ell), il existe U \in V (a) tel que f(U \bigcap A) \subset~ V
(\bigcapf(A)) et comme U \bigcap A\neq~\varnothing~, on a V \bigcap
f(A)\neq~\varnothing~. Donc \ell
\in\overlinef(A).

Remarque~4.4.3 Si f admet \ell pour limite en a suivant A et si A' \subset~ A est
tel que a \in\overlineA', il est clair que f(U \bigcap A') \subset~
f(U \bigcap A) et donc f admet encore \ell comme limite en a suivant A'. La
réciproque est évidemment fausse mais on a

Théorème~4.4.4 Soit A et A' deux parties de E telles que A \cup A'
\subset~ Def (f) et a \in\overlineA~
\bigcap\overlineA'. Alors on a équivalence de

\begin{itemize}
\itemsep1pt\parskip0pt\parsep0pt
\item
  (i) f admet une limite en a suivant A \cup A'
\item
  (ii) f admet une limite suivant A, une limite suivant A' et ces
  limites sont égales.
\end{itemize}

Démonstration D'après la remarque précédente, on a (i) \rigtharrow~(ii).
Inversement supposons que \ell =\
lim_Af(x) = lim_A'~f(x). Soit
V un voisinage de \ell. Soit U \in V (a) tel que f(U \bigcap A) \subset~ V et U' \in V (a)
tel que f(U' \bigcap A') \subset~ V . Alors U \bigcap U' est un voisinage de a et f((U \bigcap
U') \bigcap (A \cup A')) \subset~ V (facile). Donc \ell est limite suivant A \cup A'.

Exemple~4.4.2 Une suite (x_n) converge si et seulement si~les
deux sous suites (x_2n) et (x_2n+1) convergent et ont
la même limite. De même, une fonction admet une limite en a si et
seulement si~elle a une limite à gauche et une limite à droite et ces
limites sont égales (à condition que tout cela ait un sens).

\paragraph{4.4.3 Composition des limites}

Théorème~4.4.5 Soit E,F et G trois espaces métriques, f fonction de E
vers F, g une fonction de F vers G. Soit A une partie de E et B une
partie de F. On suppose que A \subset~ Def~ (f), B
\subset~ Def~ (g) et f(A) \subset~ B (si bien que g \cdot f est
définie sur A). Si f admet une limite b en a suivant A et si g admet une
limite \ell en b suivant B, alors g \cdot f admet \ell pour limite en a suivant A.

Démonstration Remarquons que b \in\overlinef(A)
\subset~\overlineB. Soit alors W \in V (\ell). Il existe V \in V
(b) tel que g(V \bigcap B) \subset~ W. Pour ce voisinage V , il existe U \in V (a) tel
que f(U \bigcap A) \subset~ V . Mais on a f(U \bigcap A) \subset~ f(A) \subset~ B et donc f(U \bigcap A) \subset~ V \bigcap
B soit g \cdot f(U \bigcap A) \subset~ W, ce qui achève la démonstration.

Proposition~4.4.6 Soit
E,F_1,\\ldots,F_p~
des espaces métriques, F = F_1 \times⋯ \times
F_p, p_i la pro\\jmathmathection de F sur F_i définie
par
p_i(y_1,\\ldots,y_p~)
= y_i. Soit f une fonction de E vers F, f_i =
p_i \cdot f si bien que f(x) =
(f_1(x),\\ldots,f_p~(x)).
Alors f admet une limite \ell en a suivant A si et seulement si~chacune des
f_i admet une limite \ell_i en a suivant A. dans ce cas \ell
=
(\ell_1,\\ldots,\ell_p~).

Lemme~4.4.7 Pour tout b \in F on a
lim_y\rightarrow~bp_i~(y) =
p_i(b).

Démonstration Soit V _i un voisinage ouvert de b_i =
p_i(b). Alors U = F_1 \times⋯ \times
V _i \times⋯ \times F_p est un ouvert
contenant b tel que p_i(U) \subset~ V .

Démonstration La condition est nécessaire d'après le théorème de
composition des limites~: si f admet \ell pour limite, alors p_i \cdot
f admet pour limite p_i(\ell) que l'on nomme \ell_i. On a
alors bien entendu, \ell =
(\ell_1,\\ldots,\ell_p~).
Réciproquement, supposons que chacune des f_i admet
\ell_i pour limite en a suivant A et soit \ell =
(\ell_1,\\ldots,\ell_p~).
Soit V un voisinage de \ell. Il existe alors un ouvert élémentaire V
_1 \times⋯ \times V _p tel que
(\ell_1,\\ldots,\ell_p~)
\subset~ V _1 \times⋯ \times V _p \subset~ V . Pour
chaque i, il existe U_i voisinage de a tel que
f_i(U_i \bigcap A) \subset~ V _i. Soit U = U_1
\bigcap\\ldots~ \bigcap
U_p. On a alors f(U \bigcap A) \subset~ V _1
\times⋯ \times V _p \subset~ V , ce qui montre que f
a \ell pour limite en a suivant A.

\paragraph{4.4.4 Limites et suites}

Théorème~4.4.8 Soit E et F deux espaces métriques. Alors f admet \ell pour
limite en a suivant A si et seulement si~pour toute suite (a_n)
d'éléments de A de limite a, la suite (f(a_n))_n\in\mathbb{N}~
admet \ell pour limite.

Démonstration Le fait que la condition soit nécessaire résulte du
théorème de composition des limites~: soit V voisinage de \ell~; il existe
U \in V (a) tel que f(U \bigcap A) \subset~ V ~; il existe N \in \mathbb{N}~ tel que n ≥ N \rigtharrow~
a_n \in U(\bigcapA)~; alors, pour n ≥ N, on a f(a_n) \in V ,
donc \ell est limite de la suite (f(a_n)). Supposons maintenant
que f n'admet pas \ell pour limite en a suivant A~; ceci signifie que

\exists~\epsilon \textgreater{} 0,
\forall~~\eta \textgreater{}
0\exists~a' \in A\text tel que
d(a,a') \textless{} \eta\text et d(f(x),\ell) ≥ \epsilon

Pour \eta = 1 \over n+1 , on a donc a_n \in A tel
que d(a,a_n) \textless{} 1 \over n+1 avec
d(\ell,f(a_n)) ≥ \epsilon, d'où une suite d'éléments de A de limite a
telle que la suite (f(a_n)) n'admet pas \ell pour limite. Ceci
démontre la réciproque par contraposition.

Corollaire~4.4.9 Avec les mêmes notations, f admet une limite en a
suivant A si et seulement si~pour toute suite (a_n) d'éléments
de A de limite a, la suite (f(a_n)) converge.

Démonstration La condition est évidemment nécessaire. Pour la
réciproque, il suffit de montrer que la limite de la suite
(f(a_n)) ne dépend pas de la suite (a_n)~; or si
(a_n) et (b_n) sont deux telles suites, on définit
(c_n) par c_2n = a_n et c_2n+1 =
b_n~; cette suite converge vers a, donc la suite
(f(c_n)) converge et donc ses deux sous suites
(f(a_n)) et (f(b_n)) ont la même limite.

Remarque~4.4.4 Ce corollaire permet d'assurer l'existence d'une limite
sans expliciter celle-ci à condition d'avoir un critère de convergence
des suites (par exemple le critère de Cauchy).

{[}
{[}
{[}
{[}

\end{document}

% \documentclass[]{article}
\usepackage[T1]{fontenc}
\usepackage{lmodern}
\usepackage{amssymb,amsmath}
\usepackage{ifxetex,ifluatex}
\usepackage{fixltx2e} % provides \textsubscript
% use upquote if available, for straight quotes in verbatim environments
\IfFileExists{upquote.sty}{\usepackage{upquote}}{}
\ifnum 0\ifxetex 1\fi\ifluatex 1\fi=0 % if pdftex
  \usepackage[utf8]{inputenc}
\else % if luatex or xelatex
  \ifxetex
    \usepackage{mathspec}
    \usepackage{xltxtra,xunicode}
  \else
    \usepackage{fontspec}
  \fi
  \defaultfontfeatures{Mapping=tex-text,Scale=MatchLowercase}
  \newcommand{\euro}{€}
\fi
% use microtype if available
\IfFileExists{microtype.sty}{\usepackage{microtype}}{}
\ifxetex
  \usepackage[setpagesize=false, % page size defined by xetex
              unicode=false, % unicode breaks when used with xetex
              xetex]{hyperref}
\else
  \usepackage[unicode=true]{hyperref}
\fi
\hypersetup{breaklinks=true,
            bookmarks=true,
            pdfauthor={},
            pdftitle={Continuite},
            colorlinks=true,
            citecolor=blue,
            urlcolor=blue,
            linkcolor=magenta,
            pdfborder={0 0 0}}
\urlstyle{same}  % don't use monospace font for urls
\setlength{\parindent}{0pt}
\setlength{\parskip}{6pt plus 2pt minus 1pt}
\setlength{\emergencystretch}{3em}  % prevent overfull lines
\setcounter{secnumdepth}{0}
 
/* start css.sty */
.cmr-5{font-size:50%;}
.cmr-7{font-size:70%;}
.cmmi-5{font-size:50%;font-style: italic;}
.cmmi-7{font-size:70%;font-style: italic;}
.cmmi-10{font-style: italic;}
.cmsy-5{font-size:50%;}
.cmsy-7{font-size:70%;}
.cmex-7{font-size:70%;}
.cmex-7x-x-71{font-size:49%;}
.msbm-7{font-size:70%;}
.cmtt-10{font-family: monospace;}
.cmti-10{ font-style: italic;}
.cmbx-10{ font-weight: bold;}
.cmr-17x-x-120{font-size:204%;}
.cmsl-10{font-style: oblique;}
.cmti-7x-x-71{font-size:49%; font-style: italic;}
.cmbxti-10{ font-weight: bold; font-style: italic;}
p.noindent { text-indent: 0em }
td p.noindent { text-indent: 0em; margin-top:0em; }
p.nopar { text-indent: 0em; }
p.indent{ text-indent: 1.5em }
@media print {div.crosslinks {visibility:hidden;}}
a img { border-top: 0; border-left: 0; border-right: 0; }
center { margin-top:1em; margin-bottom:1em; }
td center { margin-top:0em; margin-bottom:0em; }
.Canvas { position:relative; }
li p.indent { text-indent: 0em }
.enumerate1 {list-style-type:decimal;}
.enumerate2 {list-style-type:lower-alpha;}
.enumerate3 {list-style-type:lower-roman;}
.enumerate4 {list-style-type:upper-alpha;}
div.newtheorem { margin-bottom: 2em; margin-top: 2em;}
.obeylines-h,.obeylines-v {white-space: nowrap; }
div.obeylines-v p { margin-top:0; margin-bottom:0; }
.overline{ text-decoration:overline; }
.overline img{ border-top: 1px solid black; }
td.displaylines {text-align:center; white-space:nowrap;}
.centerline {text-align:center;}
.rightline {text-align:right;}
div.verbatim {font-family: monospace; white-space: nowrap; text-align:left; clear:both; }
.fbox {padding-left:3.0pt; padding-right:3.0pt; text-indent:0pt; border:solid black 0.4pt; }
div.fbox {display:table}
div.center div.fbox {text-align:center; clear:both; padding-left:3.0pt; padding-right:3.0pt; text-indent:0pt; border:solid black 0.4pt; }
div.minipage{width:100%;}
div.center, div.center div.center {text-align: center; margin-left:1em; margin-right:1em;}
div.center div {text-align: left;}
div.flushright, div.flushright div.flushright {text-align: right;}
div.flushright div {text-align: left;}
div.flushleft {text-align: left;}
.underline{ text-decoration:underline; }
.underline img{ border-bottom: 1px solid black; margin-bottom:1pt; }
.framebox-c, .framebox-l, .framebox-r { padding-left:3.0pt; padding-right:3.0pt; text-indent:0pt; border:solid black 0.4pt; }
.framebox-c {text-align:center;}
.framebox-l {text-align:left;}
.framebox-r {text-align:right;}
span.thank-mark{ vertical-align: super }
span.footnote-mark sup.textsuperscript, span.footnote-mark a sup.textsuperscript{ font-size:80%; }
div.tabular, div.center div.tabular {text-align: center; margin-top:0.5em; margin-bottom:0.5em; }
table.tabular td p{margin-top:0em;}
table.tabular {margin-left: auto; margin-right: auto;}
div.td00{ margin-left:0pt; margin-right:0pt; }
div.td01{ margin-left:0pt; margin-right:5pt; }
div.td10{ margin-left:5pt; margin-right:0pt; }
div.td11{ margin-left:5pt; margin-right:5pt; }
table[rules] {border-left:solid black 0.4pt; border-right:solid black 0.4pt; }
td.td00{ padding-left:0pt; padding-right:0pt; }
td.td01{ padding-left:0pt; padding-right:5pt; }
td.td10{ padding-left:5pt; padding-right:0pt; }
td.td11{ padding-left:5pt; padding-right:5pt; }
table[rules] {border-left:solid black 0.4pt; border-right:solid black 0.4pt; }
.hline hr, .cline hr{ height : 1px; margin:0px; }
.tabbing-right {text-align:right;}
span.TEX {letter-spacing: -0.125em; }
span.TEX span.E{ position:relative;top:0.5ex;left:-0.0417em;}
a span.TEX span.E {text-decoration: none; }
span.LATEX span.A{ position:relative; top:-0.5ex; left:-0.4em; font-size:85%;}
span.LATEX span.TEX{ position:relative; left: -0.4em; }
div.float img, div.float .caption {text-align:center;}
div.figure img, div.figure .caption {text-align:center;}
.marginpar {width:20%; float:right; text-align:left; margin-left:auto; margin-top:0.5em; font-size:85%; text-decoration:underline;}
.marginpar p{margin-top:0.4em; margin-bottom:0.4em;}
.equation td{text-align:center; vertical-align:middle; }
td.eq-no{ width:5%; }
table.equation { width:100%; } 
div.math-display, div.par-math-display{text-align:center;}
math .texttt { font-family: monospace; }
math .textit { font-style: italic; }
math .textsl { font-style: oblique; }
math .textsf { font-family: sans-serif; }
math .textbf { font-weight: bold; }
.partToc a, .partToc, .likepartToc a, .likepartToc {line-height: 200%; font-weight:bold; font-size:110%;}
.chapterToc a, .chapterToc, .likechapterToc a, .likechapterToc, .appendixToc a, .appendixToc {line-height: 200%; font-weight:bold;}
.index-item, .index-subitem, .index-subsubitem {display:block}
.caption td.id{font-weight: bold; white-space: nowrap; }
table.caption {text-align:center;}
h1.partHead{text-align: center}
p.bibitem { text-indent: -2em; margin-left: 2em; margin-top:0.6em; margin-bottom:0.6em; }
p.bibitem-p { text-indent: 0em; margin-left: 2em; margin-top:0.6em; margin-bottom:0.6em; }
.paragraphHead, .likeparagraphHead { margin-top:2em; font-weight: bold;}
.subparagraphHead, .likesubparagraphHead { font-weight: bold;}
.quote {margin-bottom:0.25em; margin-top:0.25em; margin-left:1em; margin-right:1em; text-align:justify;}
.verse{white-space:nowrap; margin-left:2em}
div.maketitle {text-align:center;}
h2.titleHead{text-align:center;}
div.maketitle{ margin-bottom: 2em; }
div.author, div.date {text-align:center;}
div.thanks{text-align:left; margin-left:10%; font-size:85%; font-style:italic; }
div.author{white-space: nowrap;}
.quotation {margin-bottom:0.25em; margin-top:0.25em; margin-left:1em; }
h1.partHead{text-align: center}
.sectionToc, .likesectionToc {margin-left:2em;}
.subsectionToc, .likesubsectionToc {margin-left:4em;}
.subsubsectionToc, .likesubsubsectionToc {margin-left:6em;}
.frenchb-nbsp{font-size:75%;}
.frenchb-thinspace{font-size:75%;}
.figure img.graphics {margin-left:10%;}
/* end css.sty */

\title{Continuite}
\author{}
\date{}

\begin{document}
\maketitle

\textbf{Warning: 
requires JavaScript to process the mathematics on this page.\\ If your
browser supports JavaScript, be sure it is enabled.}

\begin{center}\rule{3in}{0.4pt}\end{center}

[
[
[]
[

\subsubsection{4.5 Continuité}

\paragraph{4.5.1 Continuité en un point}

Définition~4.5.1 Soit f une fonction de E vers F et a
\in Def~ (f). On dit que f est continue au point
a si elle vérifie les conditions équivalentes

\begin{itemize}
\itemsep1pt\parskip0pt\parsep0pt
\item
  (i) lim_x\rightarrow~a~f(x) = f(a)
\item
  (ii) \forall~~V \in V (f(a)),
  \exists~U \in V (a), f(U) \subset~ V
\item
  (iii) \forall~~\epsilon > 0,
  \exists~\eta > 0, d(x,a) < \eta \rigtharrow~
  d(f(x),f(a)) < \epsilon
\end{itemize}

Remarque~4.5.1 On a bien entendu toutes les propriétés des limites, en
particulier

Proposition~4.5.1 La continuité est une notion locale~: si U_0
est un ouvert contenant a, f est continue au point a si et seulement
si~f_U_0 est continue au point a.

Proposition~4.5.2 Si f est une fonction de E vers F_1
\times⋯ \times F_k, f =
(f_1,\\ldots,f_k~),
alors f est continue au point a si et seulement si~chacune des
f_i est continue au point a.

Proposition~4.5.3 Si f est continue au point a et g continue au point
f(a), alors g \cdot f est continue au point a (on suppose que
\mathrmIm~f
\subset~ Def~ (g)).

Théorème~4.5.4 f est continue au point a si et seulement si~pour toute
suite (a_n) de Def~ (f) de limite a,
la suite (f(a_n)) admet f(a) pour limite.

\paragraph{4.5.2 Continuité sur un espace}

Définition~4.5.2 Soit E et F deux espaces métriques. On dit que f : E \rightarrow~
F est continue si elle est continue en tout point de E.

Remarque~4.5.2 On a donc toutes les propriétés des limites et des
fonctions continues~; en particulier, la composée de deux applications
continues est continue.

Théorème~4.5.5 Soit E et F deux espaces métriques et f : E \rightarrow~ F. On a
équivalence de

\begin{itemize}
\itemsep1pt\parskip0pt\parsep0pt
\item
  (i) f est continue (sur E)
\item
  (ii) pour tout ouvert V de F, f^-1(V ) est un ouvert de E
\item
  (iii) pour tout fermé K de F, f^-1(K) est un fermé de E
\end{itemize}

Démonstration (ii) et (iii) sont équivalents puisque pour toute partie B
de F on a f^-1(cB) = cf^-1(B).

((i) \rigtharrow~(ii)) Supposons que f est continue et soit V un ouvert de F et a \in
f^-1(V ). On a f(a) \in V et V est un voisinage de f(a). Donc
il existe U \in V (a) tel que f(U) \subset~ V , soit U \subset~ f^-1(V ) et
donc f^-1(V ) est un voisinage de a. Puisque
f^-1(V ) est voisinage de tous ses points il est ouvert.

(ii) \rigtharrow~ (i)Inversement, supposons que l'image réciproque de tout ouvert
est un ouvert et soit a \in E et V \in V (f(a)). Il existe V _0
ouvert tel que f(a) \in V _0 \subset~ V . Alors U_0 =
f^-1(V _0) est un ouvert contenant a et on a
f(U_0) \subset~ V _0 \subset~ V . Donc f est continue en a.

Théorème~4.5.6 Soit E et F deux espaces métriques, f,g : E \rightarrow~ F deux
applications continues. Alors \x \in
E∣f(x) = g(x)\ est fermé
dans E.

Démonstration Soit \phi : E \rightarrow~ F \times F,
x\mapsto~(f(x),g(x)). Comme f et g sont continues, \phi
est continue. Or \x \in E∣f(x)
= g(x)\ = \phi^-1(\Delta) où \Delta =
\(y,y)∣y \in
F\. Or on sait, d'après la propriété de séparation que
\Delta est un fermé de F \times F. Son image réciproque par \phi est donc un fermé de
E.

Corollaire~4.5.7 Soit E et F deux espaces métriques, f,g : E \rightarrow~ F deux
applications continues. On suppose qu'il existe une partie A de E, dense
dans E telle que \forall~~x \in A, f(x) = g(x). Alors f
= g.

Démonstration \x \in E∣f(x) =
g(x)\ est un fermé contenant A donc
\overlineA = E~; donc f = g.

\paragraph{4.5.3 Homéomorphismes}

Définition~4.5.3 Soit E et F deux espaces métriques. on dit que f : E \rightarrow~
F est un homéomorphisme si f est bijective et si f et f^-1
sont continues.

Remarque~4.5.3 Deux espaces homéomorphes ont exactement les mêmes
propriétés topologiques (toutes celles qui peuvent s'exprimer sans faire
intervenir de distances, uniquement en termes d'ouverts, de fermés et de
voisinages).

Exemple~4.5.1 Soit f : [0,2\pi~[\rightarrow~ U = \z \in
\mathbb{C}∣z =
1\, t\mapsto~e^it. Alors
f est continue bijective, mais sa réciproque n'est pas continue au point
1 (faire tendre z vers 1 par parties imaginaires négatives,
f^-1(z) tend vers 2\pi~\neq~0 =
f^-1(1)).

[
[
[
[

\end{document}

% \documentclass[]{article}
\usepackage[T1]{fontenc}
\usepackage{lmodern}
\usepackage{amssymb,amsmath}
\usepackage{ifxetex,ifluatex}
\usepackage{fixltx2e} % provides \textsubscript
% use upquote if available, for straight quotes in verbatim environments
\IfFileExists{upquote.sty}{\usepackage{upquote}}{}
\ifnum 0\ifxetex 1\fi\ifluatex 1\fi=0 % if pdftex
  \usepackage[utf8]{inputenc}
\else % if luatex or xelatex
  \ifxetex
    \usepackage{mathspec}
    \usepackage{xltxtra,xunicode}
  \else
    \usepackage{fontspec}
  \fi
  \defaultfontfeatures{Mapping=tex-text,Scale=MatchLowercase}
  \newcommand{\euro}{€}
\fi
% use microtype if available
\IfFileExists{microtype.sty}{\usepackage{microtype}}{}
\ifxetex
  \usepackage[setpagesize=false, % page size defined by xetex
              unicode=false, % unicode breaks when used with xetex
              xetex]{hyperref}
\else
  \usepackage[unicode=true]{hyperref}
\fi
\hypersetup{breaklinks=true,
            bookmarks=true,
            pdfauthor={},
            pdftitle={Continuite uniforme},
            colorlinks=true,
            citecolor=blue,
            urlcolor=blue,
            linkcolor=magenta,
            pdfborder={0 0 0}}
\urlstyle{same}  % don't use monospace font for urls
\setlength{\parindent}{0pt}
\setlength{\parskip}{6pt plus 2pt minus 1pt}
\setlength{\emergencystretch}{3em}  % prevent overfull lines
\setcounter{secnumdepth}{0}
 
/* start css.sty */
.cmr-5{font-size:50%;}
.cmr-7{font-size:70%;}
.cmmi-5{font-size:50%;font-style: italic;}
.cmmi-7{font-size:70%;font-style: italic;}
.cmmi-10{font-style: italic;}
.cmsy-5{font-size:50%;}
.cmsy-7{font-size:70%;}
.cmex-7{font-size:70%;}
.cmex-7x-x-71{font-size:49%;}
.msbm-7{font-size:70%;}
.cmtt-10{font-family: monospace;}
.cmti-10{ font-style: italic;}
.cmbx-10{ font-weight: bold;}
.cmr-17x-x-120{font-size:204%;}
.cmsl-10{font-style: oblique;}
.cmti-7x-x-71{font-size:49%; font-style: italic;}
.cmbxti-10{ font-weight: bold; font-style: italic;}
p.noindent { text-indent: 0em }
td p.noindent { text-indent: 0em; margin-top:0em; }
p.nopar { text-indent: 0em; }
p.indent{ text-indent: 1.5em }
@media print {div.crosslinks {visibility:hidden;}}
a img { border-top: 0; border-left: 0; border-right: 0; }
center { margin-top:1em; margin-bottom:1em; }
td center { margin-top:0em; margin-bottom:0em; }
.Canvas { position:relative; }
li p.indent { text-indent: 0em }
.enumerate1 {list-style-type:decimal;}
.enumerate2 {list-style-type:lower-alpha;}
.enumerate3 {list-style-type:lower-roman;}
.enumerate4 {list-style-type:upper-alpha;}
div.newtheorem { margin-bottom: 2em; margin-top: 2em;}
.obeylines-h,.obeylines-v {white-space: nowrap; }
div.obeylines-v p { margin-top:0; margin-bottom:0; }
.overline{ text-decoration:overline; }
.overline img{ border-top: 1px solid black; }
td.displaylines {text-align:center; white-space:nowrap;}
.centerline {text-align:center;}
.rightline {text-align:right;}
div.verbatim {font-family: monospace; white-space: nowrap; text-align:left; clear:both; }
.fbox {padding-left:3.0pt; padding-right:3.0pt; text-indent:0pt; border:solid black 0.4pt; }
div.fbox {display:table}
div.center div.fbox {text-align:center; clear:both; padding-left:3.0pt; padding-right:3.0pt; text-indent:0pt; border:solid black 0.4pt; }
div.minipage{width:100%;}
div.center, div.center div.center {text-align: center; margin-left:1em; margin-right:1em;}
div.center div {text-align: left;}
div.flushright, div.flushright div.flushright {text-align: right;}
div.flushright div {text-align: left;}
div.flushleft {text-align: left;}
.underline{ text-decoration:underline; }
.underline img{ border-bottom: 1px solid black; margin-bottom:1pt; }
.framebox-c, .framebox-l, .framebox-r { padding-left:3.0pt; padding-right:3.0pt; text-indent:0pt; border:solid black 0.4pt; }
.framebox-c {text-align:center;}
.framebox-l {text-align:left;}
.framebox-r {text-align:right;}
span.thank-mark{ vertical-align: super }
span.footnote-mark sup.textsuperscript, span.footnote-mark a sup.textsuperscript{ font-size:80%; }
div.tabular, div.center div.tabular {text-align: center; margin-top:0.5em; margin-bottom:0.5em; }
table.tabular td p{margin-top:0em;}
table.tabular {margin-left: auto; margin-right: auto;}
div.td00{ margin-left:0pt; margin-right:0pt; }
div.td01{ margin-left:0pt; margin-right:5pt; }
div.td10{ margin-left:5pt; margin-right:0pt; }
div.td11{ margin-left:5pt; margin-right:5pt; }
table[rules] {border-left:solid black 0.4pt; border-right:solid black 0.4pt; }
td.td00{ padding-left:0pt; padding-right:0pt; }
td.td01{ padding-left:0pt; padding-right:5pt; }
td.td10{ padding-left:5pt; padding-right:0pt; }
td.td11{ padding-left:5pt; padding-right:5pt; }
table[rules] {border-left:solid black 0.4pt; border-right:solid black 0.4pt; }
.hline hr, .cline hr{ height : 1px; margin:0px; }
.tabbing-right {text-align:right;}
span.TEX {letter-spacing: -0.125em; }
span.TEX span.E{ position:relative;top:0.5ex;left:-0.0417em;}
a span.TEX span.E {text-decoration: none; }
span.LATEX span.A{ position:relative; top:-0.5ex; left:-0.4em; font-size:85%;}
span.LATEX span.TEX{ position:relative; left: -0.4em; }
div.float img, div.float .caption {text-align:center;}
div.figure img, div.figure .caption {text-align:center;}
.marginpar {width:20%; float:right; text-align:left; margin-left:auto; margin-top:0.5em; font-size:85%; text-decoration:underline;}
.marginpar p{margin-top:0.4em; margin-bottom:0.4em;}
.equation td{text-align:center; vertical-align:middle; }
td.eq-no{ width:5%; }
table.equation { width:100%; } 
div.math-display, div.par-math-display{text-align:center;}
math .texttt { font-family: monospace; }
math .textit { font-style: italic; }
math .textsl { font-style: oblique; }
math .textsf { font-family: sans-serif; }
math .textbf { font-weight: bold; }
.partToc a, .partToc, .likepartToc a, .likepartToc {line-height: 200%; font-weight:bold; font-size:110%;}
.chapterToc a, .chapterToc, .likechapterToc a, .likechapterToc, .appendixToc a, .appendixToc {line-height: 200%; font-weight:bold;}
.index-item, .index-subitem, .index-subsubitem {display:block}
.caption td.id{font-weight: bold; white-space: nowrap; }
table.caption {text-align:center;}
h1.partHead{text-align: center}
p.bibitem { text-indent: -2em; margin-left: 2em; margin-top:0.6em; margin-bottom:0.6em; }
p.bibitem-p { text-indent: 0em; margin-left: 2em; margin-top:0.6em; margin-bottom:0.6em; }
.paragraphHead, .likeparagraphHead { margin-top:2em; font-weight: bold;}
.subparagraphHead, .likesubparagraphHead { font-weight: bold;}
.quote {margin-bottom:0.25em; margin-top:0.25em; margin-left:1em; margin-right:1em; text-align:\jmathustify;}
.verse{white-space:nowrap; margin-left:2em}
div.maketitle {text-align:center;}
h2.titleHead{text-align:center;}
div.maketitle{ margin-bottom: 2em; }
div.author, div.date {text-align:center;}
div.thanks{text-align:left; margin-left:10%; font-size:85%; font-style:italic; }
div.author{white-space: nowrap;}
.quotation {margin-bottom:0.25em; margin-top:0.25em; margin-left:1em; }
h1.partHead{text-align: center}
.sectionToc, .likesectionToc {margin-left:2em;}
.subsectionToc, .likesubsectionToc {margin-left:4em;}
.subsubsectionToc, .likesubsubsectionToc {margin-left:6em;}
.frenchb-nbsp{font-size:75%;}
.frenchb-thinspace{font-size:75%;}
.figure img.graphics {margin-left:10%;}
/* end css.sty */

\title{Continuite uniforme}
\author{}
\date{}

\begin{document}
\maketitle

\textbf{Warning: 
requires JavaScript to process the mathematics on this page.\\ If your
browser supports JavaScript, be sure it is enabled.}

\begin{center}\rule{3in}{0.4pt}\end{center}

{[}
{[}
{[}{]}
{[}

\subsubsection{4.6 Continuité uniforme}

\paragraph{4.6.1 Applications uniformément continues}

La continuité de f : E \rightarrow~ F s'exprime par

\begin{align*} \forall~~a \in
E,\forall~~\epsilon \textgreater{} 0,
\exists~\eta(a,\epsilon) \textgreater{} 0,&& \%&
\\ & & d(x,a) \textless{} \eta(a,\epsilon) \rigtharrow~
d(f(x),f(a)) \textless{} \epsilon\%& \\
\end{align*}

Remarque~4.6.1 En général, \eta dépend de \epsilon mais aussi de a. On dira que f
est uniformément continue sur E si on peut choisir un \eta ne dépendant pas
de a. Ceci se traduit par

Définition~4.6.1 Soit E et F deux espaces métriques. On dit que f : E \rightarrow~
F est uniformément continue si on a

\begin{align*} \forall~~\epsilon
\textgreater{} 0,\exists~\eta \textgreater{}
0,\quad \forall~~x,x' \in E,&& \%&
\\ & & d(x,x') \textless{} \eta \rigtharrow~
d(f(x),f(x')) \textless{} \epsilon\%& \\
\end{align*}

Remarque~4.6.2 Toute application uniformément continue est donc
continue. Il s'agit d'une notion métrique et non topologique (elle ne
peut pas se traduire en termes d'ouverts).

Proposition~4.6.1 La composée de deux applications uniformément
continues est uniformément continue.

Démonstration Evident.

Remarque~4.6.3 Le lemme suivant peut rendre des services pour montrer
que certaines applications ne sont pas uniformément continues~:

Lemme~4.6.2 Soit E et F deux espaces métriques et f : E \rightarrow~ F. Alors f est
uniformément continue si et seulement si~pour tout couple de suites
(a\_n),(b\_n) de points de E tels que
limd(a\_n,b\_n~) = 0, on a
limd(f(a\_n),f(b\_n~)) = 0.

Démonstration (i) \rigtharrow~(ii) Soit \epsilon \textgreater{} 0. Alors
\exists~\eta \textgreater{} 0,\quad
\forall~~x,x' \in E, d(x,x') \textless{} \eta \rigtharrow~
d(f(x),f(x')) \textless{} \epsilon. Pour ce \eta, il existe N \in \mathbb{N}~ tel que n ≥ N \rigtharrow~
d(a\_n,b\_n) \textless{} \eta. Alors n ≥ N \rigtharrow~
d(f(a\_n),f(b\_n)) \textless{} \epsilon ce qui montre que
limd(f(a\_n),f(b\_n~)) = 0

(ii) \rigtharrow~(i) Nous allons montrer la contraposée. Supposons f non
uniformément continue. Alors

\exists~\epsilon \textgreater{} 0,
\forall~~\eta \textgreater{} 0,\quad
\exists~a,b \in E, d(a,b) \textless{}
\eta\text et d(f(a),f(b)) ≥ \epsilon

en prenant \eta = 1 \over n+1 , on trouve a\_n
et b\_n tels que d(a\_n,b\_n) \textless{} 1
\over n+1 alors que d(f(a\_n),f(b\_n))
≥ \epsilon, et donc (ii) n'est pas vérifiée.

Exemple~4.6.1 L'application f : \mathbb{R}~ \rightarrow~ \mathbb{R}~,
x\mapsto~x^2 est continue, mais par
uniformément continue~: pour a\_n = n,b\_n = n + 1
\over n , on a
lim\textbar{}a\_n~ -
b\_n\textbar{} = 0, mais
lim\textbar{}a\_n^2~ -
b\_n^2\textbar{} = 2.

\paragraph{4.6.2 Applications lipschitziennes}

Définition~4.6.2 Soit E et F deux espaces métriques. On dit que f : E \rightarrow~
F est lipschitzienne de rapport k ≥ 0 si

\forall~~x,x' \in E,\quad d(f(x),f(x')) \leq
kd(x,x')

Théorème~4.6.3 Toute application lipschitzienne est uniformément
continue.

Démonstration Si k = 0, f est constante et sinon

d(x,x') \textless{} \epsilon \over k \rigtharrow~ d(f(x),f(x'))
\textless{} \epsilon

Remarque~4.6.4 On montrera souvent qu'une application est lipschitzienne
par application d'un théorème des accroissements finis.

{[}
{[}
{[}
{[}

\end{document}

% \documentclass[]{article}
\usepackage[T1]{fontenc}
\usepackage{lmodern}
\usepackage{amssymb,amsmath}
\usepackage{ifxetex,ifluatex}
\usepackage{fixltx2e} % provides \textsubscript
% use upquote if available, for straight quotes in verbatim environments
\IfFileExists{upquote.sty}{\usepackage{upquote}}{}
\ifnum 0\ifxetex 1\fi\ifluatex 1\fi=0 % if pdftex
  \usepackage[utf8]{inputenc}
\else % if luatex or xelatex
  \ifxetex
    \usepackage{mathspec}
    \usepackage{xltxtra,xunicode}
  \else
    \usepackage{fontspec}
  \fi
  \defaultfontfeatures{Mapping=tex-text,Scale=MatchLowercase}
  \newcommand{\euro}{€}
\fi
% use microtype if available
\IfFileExists{microtype.sty}{\usepackage{microtype}}{}
\ifxetex
  \usepackage[setpagesize=false, % page size defined by xetex
              unicode=false, % unicode breaks when used with xetex
              xetex]{hyperref}
\else
  \usepackage[unicode=true]{hyperref}
\fi
\hypersetup{breaklinks=true,
            bookmarks=true,
            pdfauthor={},
            pdftitle={Espaces complets},
            colorlinks=true,
            citecolor=blue,
            urlcolor=blue,
            linkcolor=magenta,
            pdfborder={0 0 0}}
\urlstyle{same}  % don't use monospace font for urls
\setlength{\parindent}{0pt}
\setlength{\parskip}{6pt plus 2pt minus 1pt}
\setlength{\emergencystretch}{3em}  % prevent overfull lines
\setcounter{secnumdepth}{0}
 
/* start css.sty */
.cmr-5{font-size:50%;}
.cmr-7{font-size:70%;}
.cmmi-5{font-size:50%;font-style: italic;}
.cmmi-7{font-size:70%;font-style: italic;}
.cmmi-10{font-style: italic;}
.cmsy-5{font-size:50%;}
.cmsy-7{font-size:70%;}
.cmex-7{font-size:70%;}
.cmex-7x-x-71{font-size:49%;}
.msbm-7{font-size:70%;}
.cmtt-10{font-family: monospace;}
.cmti-10{ font-style: italic;}
.cmbx-10{ font-weight: bold;}
.cmr-17x-x-120{font-size:204%;}
.cmsl-10{font-style: oblique;}
.cmti-7x-x-71{font-size:49%; font-style: italic;}
.cmbxti-10{ font-weight: bold; font-style: italic;}
p.noindent { text-indent: 0em }
td p.noindent { text-indent: 0em; margin-top:0em; }
p.nopar { text-indent: 0em; }
p.indent{ text-indent: 1.5em }
@media print {div.crosslinks {visibility:hidden;}}
a img { border-top: 0; border-left: 0; border-right: 0; }
center { margin-top:1em; margin-bottom:1em; }
td center { margin-top:0em; margin-bottom:0em; }
.Canvas { position:relative; }
li p.indent { text-indent: 0em }
.enumerate1 {list-style-type:decimal;}
.enumerate2 {list-style-type:lower-alpha;}
.enumerate3 {list-style-type:lower-roman;}
.enumerate4 {list-style-type:upper-alpha;}
div.newtheorem { margin-bottom: 2em; margin-top: 2em;}
.obeylines-h,.obeylines-v {white-space: nowrap; }
div.obeylines-v p { margin-top:0; margin-bottom:0; }
.overline{ text-decoration:overline; }
.overline img{ border-top: 1px solid black; }
td.displaylines {text-align:center; white-space:nowrap;}
.centerline {text-align:center;}
.rightline {text-align:right;}
div.verbatim {font-family: monospace; white-space: nowrap; text-align:left; clear:both; }
.fbox {padding-left:3.0pt; padding-right:3.0pt; text-indent:0pt; border:solid black 0.4pt; }
div.fbox {display:table}
div.center div.fbox {text-align:center; clear:both; padding-left:3.0pt; padding-right:3.0pt; text-indent:0pt; border:solid black 0.4pt; }
div.minipage{width:100%;}
div.center, div.center div.center {text-align: center; margin-left:1em; margin-right:1em;}
div.center div {text-align: left;}
div.flushright, div.flushright div.flushright {text-align: right;}
div.flushright div {text-align: left;}
div.flushleft {text-align: left;}
.underline{ text-decoration:underline; }
.underline img{ border-bottom: 1px solid black; margin-bottom:1pt; }
.framebox-c, .framebox-l, .framebox-r { padding-left:3.0pt; padding-right:3.0pt; text-indent:0pt; border:solid black 0.4pt; }
.framebox-c {text-align:center;}
.framebox-l {text-align:left;}
.framebox-r {text-align:right;}
span.thank-mark{ vertical-align: super }
span.footnote-mark sup.textsuperscript, span.footnote-mark a sup.textsuperscript{ font-size:80%; }
div.tabular, div.center div.tabular {text-align: center; margin-top:0.5em; margin-bottom:0.5em; }
table.tabular td p{margin-top:0em;}
table.tabular {margin-left: auto; margin-right: auto;}
div.td00{ margin-left:0pt; margin-right:0pt; }
div.td01{ margin-left:0pt; margin-right:5pt; }
div.td10{ margin-left:5pt; margin-right:0pt; }
div.td11{ margin-left:5pt; margin-right:5pt; }
table[rules] {border-left:solid black 0.4pt; border-right:solid black 0.4pt; }
td.td00{ padding-left:0pt; padding-right:0pt; }
td.td01{ padding-left:0pt; padding-right:5pt; }
td.td10{ padding-left:5pt; padding-right:0pt; }
td.td11{ padding-left:5pt; padding-right:5pt; }
table[rules] {border-left:solid black 0.4pt; border-right:solid black 0.4pt; }
.hline hr, .cline hr{ height : 1px; margin:0px; }
.tabbing-right {text-align:right;}
span.TEX {letter-spacing: -0.125em; }
span.TEX span.E{ position:relative;top:0.5ex;left:-0.0417em;}
a span.TEX span.E {text-decoration: none; }
span.LATEX span.A{ position:relative; top:-0.5ex; left:-0.4em; font-size:85%;}
span.LATEX span.TEX{ position:relative; left: -0.4em; }
div.float img, div.float .caption {text-align:center;}
div.figure img, div.figure .caption {text-align:center;}
.marginpar {width:20%; float:right; text-align:left; margin-left:auto; margin-top:0.5em; font-size:85%; text-decoration:underline;}
.marginpar p{margin-top:0.4em; margin-bottom:0.4em;}
.equation td{text-align:center; vertical-align:middle; }
td.eq-no{ width:5%; }
table.equation { width:100%; } 
div.math-display, div.par-math-display{text-align:center;}
math .texttt { font-family: monospace; }
math .textit { font-style: italic; }
math .textsl { font-style: oblique; }
math .textsf { font-family: sans-serif; }
math .textbf { font-weight: bold; }
.partToc a, .partToc, .likepartToc a, .likepartToc {line-height: 200%; font-weight:bold; font-size:110%;}
.chapterToc a, .chapterToc, .likechapterToc a, .likechapterToc, .appendixToc a, .appendixToc {line-height: 200%; font-weight:bold;}
.index-item, .index-subitem, .index-subsubitem {display:block}
.caption td.id{font-weight: bold; white-space: nowrap; }
table.caption {text-align:center;}
h1.partHead{text-align: center}
p.bibitem { text-indent: -2em; margin-left: 2em; margin-top:0.6em; margin-bottom:0.6em; }
p.bibitem-p { text-indent: 0em; margin-left: 2em; margin-top:0.6em; margin-bottom:0.6em; }
.paragraphHead, .likeparagraphHead { margin-top:2em; font-weight: bold;}
.subparagraphHead, .likesubparagraphHead { font-weight: bold;}
.quote {margin-bottom:0.25em; margin-top:0.25em; margin-left:1em; margin-right:1em; text-align:\\jmathmathustify;}
.verse{white-space:nowrap; margin-left:2em}
div.maketitle {text-align:center;}
h2.titleHead{text-align:center;}
div.maketitle{ margin-bottom: 2em; }
div.author, div.date {text-align:center;}
div.thanks{text-align:left; margin-left:10%; font-size:85%; font-style:italic; }
div.author{white-space: nowrap;}
.quotation {margin-bottom:0.25em; margin-top:0.25em; margin-left:1em; }
h1.partHead{text-align: center}
.sectionToc, .likesectionToc {margin-left:2em;}
.subsectionToc, .likesubsectionToc {margin-left:4em;}
.subsubsectionToc, .likesubsubsectionToc {margin-left:6em;}
.frenchb-nbsp{font-size:75%;}
.frenchb-thinspace{font-size:75%;}
.figure img.graphics {margin-left:10%;}
/* end css.sty */

\title{Espaces complets}
\author{}
\date{}

\begin{document}
\maketitle

\textbf{Warning: 
requires JavaScript to process the mathematics on this page.\\ If your
browser supports JavaScript, be sure it is enabled.}

\begin{center}\rule{3in}{0.4pt}\end{center}

{[}
{[}
{[}{]}
{[}

\subsubsection{4.7 Espaces complets}

\paragraph{4.7.1 Suites de Cauchy}

Définition~4.7.1 Soit (E,d) un espace métrique. Une suite (x_n)
de E est dite suite de Cauchy si elle vérifie

\forall~~\epsilon \textgreater{} 0,
\exists~N \in \mathbb{N}~,\quad p,q ≥ N \rigtharrow~
d(x_p,x_q) \textless{} \epsilon

Remarque~4.7.1 On peut sans nuire à la généralité remplacer par

\forall~~\epsilon \textgreater{} 0,
\exists~N \in \mathbb{N}~,\quad q \textgreater{} p
≥ N \rigtharrow~ d(x_p,x_q) \textless{} \epsilon

Remarque~4.7.2 Il s'agit là d'une notion métrique et non topologique. La
suite (n) est une suite de Cauchy dans \mathbb{R}~ muni de la distance de
\overline\mathbb{R}~, mais pas de Cauchy dans \mathbb{R}~ muni de la
distance usuelle. Par contre, pour deux distances équivalentes, les
suites de Cauchy sont les mêmes.

Théorème~4.7.1

\begin{itemize}
\itemsep1pt\parskip0pt\parsep0pt
\item
  (i) Toute suite convergente est une suite de Cauchy
\item
  (ii) Toute suite de Cauchy qui a une valeur d'adhérence est
  convergente.
\item
  (iii) Toute suite de Cauchy est bornée.
\item
  (iv) L'image par une application uniformément continue d'une suite de
  Cauchy est une suite de Cauchy.
\end{itemize}

Démonstration (i) Soit \ell = limx_n~ et
\epsilon \textgreater{} 0. Il existe N \in \mathbb{N}~ tel que n ≥ N \rigtharrow~ d(x_n,\ell)
\textless{} \epsilon\diagup2. Alors p,q ≥ N \rigtharrow~ d(x_p,x_q) \leq
d(x_p,\ell),+d(\ell,x_q) \textless{} \epsilon. Donc (x_n)
est une suite de Cauchy.

(ii) Soit \ell une valeur d'adhérence de la suite de Cauchy (x_n)
et \epsilon \textgreater{} 0. Il existe N \in \mathbb{N}~ tel que p,q ≥ N \rigtharrow~
d(x_p,x_q) \textless{} \epsilon\diagup2. De plus il existe
n_0 ≥ N tel que d(x_n_0,\ell) \textless{} \epsilon\diagup2.
Pour n ≥ N, on a d(x_n,\ell) \leq
d(x_n,x_n_0) + d(x_n_0,\ell)
\textless{} \epsilon\diagup2 + \epsilon\diagup2 = \epsilon. Donc \ell est limite de (x_n).

(iii) Il existe N \in \mathbb{N}~ tel que p,q ≥ N \rigtharrow~ d(x_p,x_q)
\textless{} 1. Alors
\x_n∣n \in
\mathbb{N}~\
\subset~\x_0,\\ldots,x_N-1\~
\cup B'(x_N,1) qui est un ensemble borné.

(iv) Soit \epsilon \textgreater{} 0. Il existe \eta \textgreater{} 0 tel que
d(x,x') \textless{} \eta \rigtharrow~ d(f(x),f(x')) \textless{} \epsilon. Pour ce \eta, il
existe N \in \mathbb{N}~ tel que p,q ≥ N \rigtharrow~ d(x_p,x_q) \textless{}
\eta. Alors p,q ≥ N \rigtharrow~ d(f(x_p),f(x_q)) \textless{} \epsilon.

Remarque~4.7.3 Une suite de Cauchy a donc soit aucune valeur d'adhérence
(si elle diverge), soit une valeur d'adhérence si elle converge.

L'image par une application continue d'une suite de Cauchy n'est pas en
général une suite de Cauchy~: prendre f :{]}0,+\infty~{[}\rightarrow~{]}0,+\infty~{[},
x\mapsto~1\diagupx et x_n = 1\diagupn.

\paragraph{4.7.2 Espaces complets}

Définition~4.7.2 Un espace métrique (E,d) est dit complet si toute suite
de Cauchy de E converge dans E

Remarque~4.7.4 Il s'agit d'une notion métrique et non topologique~: bien
que la topologie soit la même, \mathbb{R}~ est complet pour la distance usuelle,
mais pas pour la distance de \overline\mathbb{R}~ (la suite (n)
est une suite de Cauchy non convergente)

Remarque~4.7.5 L'intérêt essentiel d'un espace complet est que dans un
tel espace, on peut assurer la convergence d'une suite sans exhiber au
préalable sa limite.

Théorème~4.7.2 Soit (E,d) un espace métrique et F une partie de E. (i)
Si (F,d_F) est complet, alors F est fermé dans E (ii)
Inversement, si E est complet et F fermé dans E alors (F,d_F)
est complet

Démonstration (i) Soit x \in E qui est limite d'une suite (x_n)
de F. La suite (x_n) est une suite de Cauchy~dans E, donc dans
F, donc admet une limite \ell dans F. mais l'unicité de la limite dans E
garantit que x = \ell \in F. Donc F est fermé dans E.

(ii) Soit (x_n) une suite de Cauchy~dans F~; c'est aussi une
suite de Cauchy~dans E donc elle converge vers x \in E (car E est
complet)~; mais comme F est fermé et x est limite d'une suite d'éléments
de F, x appartient à F et il est évidemment limite dans F de la suite
(x_n).

Proposition~4.7.3 Si (E_1,d_1) et
(E_2,d_2) sont deux espaces métriques complets, alors
l'espace métrique produit est complet.

Démonstration Soit (z_n) une suite de Cauchy dans E =
E_1 \times E_2, z_n = (x_n,y_n).
On a d(z_p,z_q) =\
max(d_1(x_p,x_q),d_2(y_p,y_q))
et donc d_1(x_p,x_q) \leq
d(z_p,z_q). On en déduit que (x_n) est une
suite de Cauchy~de E_1 et donc converge vers x \in E_1.
De même (y_n) converge vers y dans E_2 et alors
(z_n) converge vers (x,y) \in E.

\paragraph{4.7.3 Propriétés des espaces complets}

Théorème~4.7.4 (théorème des fermés emboîtés) Soit (E,d) un espace
métrique complet et (F_n)_n\in\mathbb{N}~ une suite de parties
fermées non vides de E vérifiant

\begin{itemize}
\itemsep1pt\parskip0pt\parsep0pt
\item
  (i) \forall~n, F_n+1 \subset~ F_n~
\item
  (ii) lim_n\rightarrow~+\infty~\delta(F_n~) = 0
\end{itemize}

Alors \⋂ ~
_n\in\mathbb{N}~F_n est un singleton (et en particulier non vide).

Démonstration Choisissons x_n \in F_n. Si q ≥ p, on a
x_p,x_q \in F_p et donc
d(x_p,x_q) \leq \delta(F_p) ce qui montre que la
suite (x_n) est une suite de Cauchy. Par conséquent elle
converge dans E, soit a sa limite. On a a =\
lim_n\rightarrow~+\infty~x_n+p et \forall~~n,
x_n+p \in F_p. Comme F_p est fermé, a
appartient à F_p et donc a
\in\⋂ ~
_p\in\mathbb{N}~F_p. Maintenant, si a,b
\in\⋂ ~
_p\in\mathbb{N}~F_p, on a d(a,b) \leq \delta(F_p) pour tout p et
donc d(a,b) = 0, soit a = b.

Remarque~4.7.6 La condition
lim_n\rightarrow~+\infty~\delta(F_n~) = 0 est
essentielle pour démontrer que l'intersection est non vide comme le
montre l'exemple F_n = {[}n,+\infty~{[} dans \mathbb{R}~, où l'intersection est
vide.

Théorème~4.7.5 (Critère de Cauchy pour les fonctions) Soit E un espace
métrique, (F,d) un espace métrique complet, A \subset~ E, a
\in\overlineA, f une fonction de E vers F telle que A
\subset~ Def~ (f). Alors f admet une limite en a
suivant A si et seulement si~elle vérifie

\forall~~\epsilon \textgreater{} 0,
\exists~U \in V (a),\quad x,x' \in U \bigcap A \rigtharrow~
d(f(x),f(x')) \textless{} \epsilon

Démonstration La condition est nécessaire car si \ell
= lim_x\rightarrow~a,x\inA~f(x) et \epsilon \textgreater{}
0, il existe U \in V (a) tel que x \in U \bigcap A \rigtharrow~ d(f(x),\ell) \textless{} \epsilon\diagup2.
Alors, pour x,x' \in U \bigcap A on a d(f(x),f(x')) \leq d(f(x),\ell) + d(\ell,f(x'))
\textless{} \epsilon\diagup2 + \epsilon\diagup2 = \epsilon. Montrons maintenant qu'elle est suffisante.
Pour cela, soit (a_n) une suite de A convergeant vers a et \epsilon
\textgreater{} 0 et soit U \in V (a) tel que x,x' \in U \bigcap A \rigtharrow~ d(f(x),f(x'))
\textless{} \epsilon~; il existe N \in \mathbb{N}~ tel que n ≥ N \rigtharrow~ a_n \in U. Pour n
≥ N, on a a_n \in U \bigcap A et donc

p,q ≥ N \rigtharrow~ a_p,a_q \in U \bigcap A \rigtharrow~
d(f(a_p),f(a_q)) \textless{} \epsilon

La suite (f(a_n)) est donc une suite de Cauchy de F, donc elle
converge. On a donc montré que pour toute suite (a_n) de A de
limite a, la suite (f(a_n)) converge~; on en déduit que f a une
limite en a suivant A.

Théorème~4.7.6 (théorème du point fixe). Soit (E,d) un espace métrique
complet et f : E \rightarrow~ E une application contractante (lipschitzienne de
rapport strictement inférieur à 1). Alors f a un unique point fixe qui
est limite de toutes les suites (x_n) définies par la
récurrence~: x_0 \in E et x_n+1 = f(x_n).

Démonstration Ecrivons d(f(x),f(y)) \leq kd(x,y) avec k \textless{} 1. Pour
l'unicité, supposons que f(x) = x et f(y) = y. On a d(x,y) \leq kd(x,y)
avec k \textless{} 1 et d(x,y) ≥ 0. ce n'est possible que si d(x,y) = 0
et donc x = y. En ce qui concerne l'existence, soit x_0 \in E et
la suite définie par la récurrence x_n+1 = f(x_n). Si
n ≥ 1, on a d(x_n+1,x_n) =
d(f(x_n),f(x_n-1)) \leq kd(x_n,x_n-1)
d'où en définitive d(x_n+1,x_n) \leq
k^nd(x_0,x_1). Mais alors, si q
\textgreater{} p on a

\begin{align*} d(x_p,x_q)& \leq&
d(x_p,x_p+1) +
\\ldots~ +
d(x_q-1,x_q) \%& \\ &
\leq& (k^p +
\\ldots~ +
k^q-1)d(x_ 0,x_1) \leq k^p
d(x_0,x_1) \over 1 - k \%&
\\ \end{align*}

Comme k \textless{} 1, on a
limk^p~
d(x_0,x_1) \over 1-k = 0, et donc la
suite est une suite de Cauchy~; elle admet donc une limite x. On a x
= limx_n+1~ =\
limf(x_n) = f(limx_n~) =
f(x) car f est continue. Donc x est point fixe de f.

{[}
{[}
{[}
{[}

\end{document}

% \documentclass[]{article}
\usepackage[T1]{fontenc}
\usepackage{lmodern}
\usepackage{amssymb,amsmath}
\usepackage{ifxetex,ifluatex}
\usepackage{fixltx2e} % provides \textsubscript
% use upquote if available, for straight quotes in verbatim environments
\IfFileExists{upquote.sty}{\usepackage{upquote}}{}
\ifnum 0\ifxetex 1\fi\ifluatex 1\fi=0 % if pdftex
  \usepackage[utf8]{inputenc}
\else % if luatex or xelatex
  \ifxetex
    \usepackage{mathspec}
    \usepackage{xltxtra,xunicode}
  \else
    \usepackage{fontspec}
  \fi
  \defaultfontfeatures{Mapping=tex-text,Scale=MatchLowercase}
  \newcommand{\euro}{€}
\fi
% use microtype if available
\IfFileExists{microtype.sty}{\usepackage{microtype}}{}
\ifxetex
  \usepackage[setpagesize=false, % page size defined by xetex
              unicode=false, % unicode breaks when used with xetex
              xetex]{hyperref}
\else
  \usepackage[unicode=true]{hyperref}
\fi
\hypersetup{breaklinks=true,
            bookmarks=true,
            pdfauthor={},
            pdftitle={Espaces et parties compactes},
            colorlinks=true,
            citecolor=blue,
            urlcolor=blue,
            linkcolor=magenta,
            pdfborder={0 0 0}}
\urlstyle{same}  % don't use monospace font for urls
\setlength{\parindent}{0pt}
\setlength{\parskip}{6pt plus 2pt minus 1pt}
\setlength{\emergencystretch}{3em}  % prevent overfull lines
\setcounter{secnumdepth}{0}
 
/* start css.sty */
.cmr-5{font-size:50%;}
.cmr-7{font-size:70%;}
.cmmi-5{font-size:50%;font-style: italic;}
.cmmi-7{font-size:70%;font-style: italic;}
.cmmi-10{font-style: italic;}
.cmsy-5{font-size:50%;}
.cmsy-7{font-size:70%;}
.cmex-7{font-size:70%;}
.cmex-7x-x-71{font-size:49%;}
.msbm-7{font-size:70%;}
.cmtt-10{font-family: monospace;}
.cmti-10{ font-style: italic;}
.cmbx-10{ font-weight: bold;}
.cmr-17x-x-120{font-size:204%;}
.cmsl-10{font-style: oblique;}
.cmti-7x-x-71{font-size:49%; font-style: italic;}
.cmbxti-10{ font-weight: bold; font-style: italic;}
p.noindent { text-indent: 0em }
td p.noindent { text-indent: 0em; margin-top:0em; }
p.nopar { text-indent: 0em; }
p.indent{ text-indent: 1.5em }
@media print {div.crosslinks {visibility:hidden;}}
a img { border-top: 0; border-left: 0; border-right: 0; }
center { margin-top:1em; margin-bottom:1em; }
td center { margin-top:0em; margin-bottom:0em; }
.Canvas { position:relative; }
li p.indent { text-indent: 0em }
.enumerate1 {list-style-type:decimal;}
.enumerate2 {list-style-type:lower-alpha;}
.enumerate3 {list-style-type:lower-roman;}
.enumerate4 {list-style-type:upper-alpha;}
div.newtheorem { margin-bottom: 2em; margin-top: 2em;}
.obeylines-h,.obeylines-v {white-space: nowrap; }
div.obeylines-v p { margin-top:0; margin-bottom:0; }
.overline{ text-decoration:overline; }
.overline img{ border-top: 1px solid black; }
td.displaylines {text-align:center; white-space:nowrap;}
.centerline {text-align:center;}
.rightline {text-align:right;}
div.verbatim {font-family: monospace; white-space: nowrap; text-align:left; clear:both; }
.fbox {padding-left:3.0pt; padding-right:3.0pt; text-indent:0pt; border:solid black 0.4pt; }
div.fbox {display:table}
div.center div.fbox {text-align:center; clear:both; padding-left:3.0pt; padding-right:3.0pt; text-indent:0pt; border:solid black 0.4pt; }
div.minipage{width:100%;}
div.center, div.center div.center {text-align: center; margin-left:1em; margin-right:1em;}
div.center div {text-align: left;}
div.flushright, div.flushright div.flushright {text-align: right;}
div.flushright div {text-align: left;}
div.flushleft {text-align: left;}
.underline{ text-decoration:underline; }
.underline img{ border-bottom: 1px solid black; margin-bottom:1pt; }
.framebox-c, .framebox-l, .framebox-r { padding-left:3.0pt; padding-right:3.0pt; text-indent:0pt; border:solid black 0.4pt; }
.framebox-c {text-align:center;}
.framebox-l {text-align:left;}
.framebox-r {text-align:right;}
span.thank-mark{ vertical-align: super }
span.footnote-mark sup.textsuperscript, span.footnote-mark a sup.textsuperscript{ font-size:80%; }
div.tabular, div.center div.tabular {text-align: center; margin-top:0.5em; margin-bottom:0.5em; }
table.tabular td p{margin-top:0em;}
table.tabular {margin-left: auto; margin-right: auto;}
div.td00{ margin-left:0pt; margin-right:0pt; }
div.td01{ margin-left:0pt; margin-right:5pt; }
div.td10{ margin-left:5pt; margin-right:0pt; }
div.td11{ margin-left:5pt; margin-right:5pt; }
table[rules] {border-left:solid black 0.4pt; border-right:solid black 0.4pt; }
td.td00{ padding-left:0pt; padding-right:0pt; }
td.td01{ padding-left:0pt; padding-right:5pt; }
td.td10{ padding-left:5pt; padding-right:0pt; }
td.td11{ padding-left:5pt; padding-right:5pt; }
table[rules] {border-left:solid black 0.4pt; border-right:solid black 0.4pt; }
.hline hr, .cline hr{ height : 1px; margin:0px; }
.tabbing-right {text-align:right;}
span.TEX {letter-spacing: -0.125em; }
span.TEX span.E{ position:relative;top:0.5ex;left:-0.0417em;}
a span.TEX span.E {text-decoration: none; }
span.LATEX span.A{ position:relative; top:-0.5ex; left:-0.4em; font-size:85%;}
span.LATEX span.TEX{ position:relative; left: -0.4em; }
div.float img, div.float .caption {text-align:center;}
div.figure img, div.figure .caption {text-align:center;}
.marginpar {width:20%; float:right; text-align:left; margin-left:auto; margin-top:0.5em; font-size:85%; text-decoration:underline;}
.marginpar p{margin-top:0.4em; margin-bottom:0.4em;}
.equation td{text-align:center; vertical-align:middle; }
td.eq-no{ width:5%; }
table.equation { width:100%; } 
div.math-display, div.par-math-display{text-align:center;}
math .texttt { font-family: monospace; }
math .textit { font-style: italic; }
math .textsl { font-style: oblique; }
math .textsf { font-family: sans-serif; }
math .textbf { font-weight: bold; }
.partToc a, .partToc, .likepartToc a, .likepartToc {line-height: 200%; font-weight:bold; font-size:110%;}
.chapterToc a, .chapterToc, .likechapterToc a, .likechapterToc, .appendixToc a, .appendixToc {line-height: 200%; font-weight:bold;}
.index-item, .index-subitem, .index-subsubitem {display:block}
.caption td.id{font-weight: bold; white-space: nowrap; }
table.caption {text-align:center;}
h1.partHead{text-align: center}
p.bibitem { text-indent: -2em; margin-left: 2em; margin-top:0.6em; margin-bottom:0.6em; }
p.bibitem-p { text-indent: 0em; margin-left: 2em; margin-top:0.6em; margin-bottom:0.6em; }
.paragraphHead, .likeparagraphHead { margin-top:2em; font-weight: bold;}
.subparagraphHead, .likesubparagraphHead { font-weight: bold;}
.quote {margin-bottom:0.25em; margin-top:0.25em; margin-left:1em; margin-right:1em; text-align:\jmathustify;}
.verse{white-space:nowrap; margin-left:2em}
div.maketitle {text-align:center;}
h2.titleHead{text-align:center;}
div.maketitle{ margin-bottom: 2em; }
div.author, div.date {text-align:center;}
div.thanks{text-align:left; margin-left:10%; font-size:85%; font-style:italic; }
div.author{white-space: nowrap;}
.quotation {margin-bottom:0.25em; margin-top:0.25em; margin-left:1em; }
h1.partHead{text-align: center}
.sectionToc, .likesectionToc {margin-left:2em;}
.subsectionToc, .likesubsectionToc {margin-left:4em;}
.subsubsectionToc, .likesubsubsectionToc {margin-left:6em;}
.frenchb-nbsp{font-size:75%;}
.frenchb-thinspace{font-size:75%;}
.figure img.graphics {margin-left:10%;}
/* end css.sty */

\title{Espaces et parties compactes}
\author{}
\date{}

\begin{document}
\maketitle

\textbf{Warning: 
requires JavaScript to process the mathematics on this page.\\ If your
browser supports JavaScript, be sure it is enabled.}

\begin{center}\rule{3in}{0.4pt}\end{center}

{[}
{[}
{[}{]}
{[}

\subsubsection{4.8 Espaces et parties compactes}

\paragraph{4.8.1 Propriété de Bolzano-Weierstrass}

Définition~4.8.1 Soit E un espace métrique. On dit que E est compact
s'il vérifie la propriété de Bolzano Weierstrass~: toute suite de E a
une valeur d'adhérence dans E. On dit qu'une partie A de E est compacte
si toute suite de A a une valeur d'adhérence dans A.

Remarque~4.8.1 La compacité est une notion purement topologique et non
métrique. Le fait pour une partie A d'être compacte ne dépend que de la
topologie de la partie et pas de l'espace ambiant E (comparer avec le
fait pour A d'être ouverte ou fermée qui dépend de E).

Théorème~4.8.1 Soit E un espace métrique et F une partie de E. (i) Si F
est compacte, alors F est fermée et bornée dans E (ii) Inversement, si E
est compact et F fermée dans E alors F est compacte

Démonstration (i) Soit x \in E qui est limite d'une suite (x\_n)
de F. La suite (x\_n) est une suite dans F, donc admet une
valeur d'adhérence \ell dans F. Mais la suite étant convergente, a une
seule valeur d'adhérence dans E, on a x = \ell \in F. Donc F est fermé dans
E. Le fait d'être borné résultera du lemme suivant

Lemme~4.8.2 Soit F une partie compacte~; alors pour tout \epsilon
\textgreater{} 0, F peut être recouvert par un nombre fini de boules de
rayon \epsilon (propriété de précompacité)

Démonstration Supposons que F ne peut pas être recouvert par un nombre
fini de boules de rayon \epsilon et soit x\_0 \in F~; on a
F⊄B'(x\_0,1)~; soit x\_1 \in F \diagdown B'(x\_0,\epsilon)~;
supposons
x\_0,\\ldots,x\_n~
construits~; alors F⊄B(x\_0,\epsilon)
\cup\\ldots~ \cup
B(x\_n,\epsilon) et on prend x\_n+1 \in F \diagdown\left
(B(x\_0,\epsilon)
\cup\\ldots~ \cup
B(x\_n,\epsilon)\right ). On construit ainsi une suite
(x\_n) telle que \forall~~p,q,
p\neq~q \rigtharrow~ d(x\_p,x\_q) ≥ \epsilon. Cette
suite n'admet aucune sous suite de Cauchy, donc aucune sous suite
convergente, donc pas de valeur d'adhérence. C'est absurde.

(ii) Soit (x\_n) une suite dans F~; c'est aussi une suite dans E
donc elle admet une valeur d'adhérence x \in E (car E est compact), x
= limx\_\phi(n)~~; mais comme F est fermé
et x est limite d'une suite d'éléments de F, x appartient à F et il est
évidemment limite dans F de la suite (x\_\phi(n)). Donc la suite
admet une valeur d'adhérence dans F et F est compacte.

Théorème~4.8.3 Soit f : E \rightarrow~ F continue. Pour toute partie compacte A de
E, f(A) est une partie compacte de F (et en particulier elle est fermée
et bornée).

Démonstration Soit (b\_n) une suite de f(A). On pose
b\_n = f(a\_n), a\_n \in A. Alors a\_n
admet une valeur d'adhérence dans A, a =\
lima\_\phi(n). Par continuité de f au point a, on a f(a)
= limf(a\_\phi(n)~) et donc la suite
(b\_n) a une valeur d'adhérence dans f(A).

Corollaire~4.8.4 Soit E un espace métrique compact et f : E \rightarrow~ F
bi\jmathective et continue. Alors f est un homéomorphisme.

Démonstration Il faut montrer que f^-1 est continue autrement
dit que pour tout fermé A de E, (f^-1)^-1(A) =
f(A) est fermée dans F~; mais une telle partie A est fermée dans un
compact, donc compacte et donc f(A) est compacte dans F donc fermée.
Ceci montre la continuité de f^-1.

Proposition~4.8.5 Si E\_1 et E\_2 sont deux espaces
métriques compacts, alors l'espace métrique produit est compact.

Démonstration Soit (z\_n) une suite dans E = E\_1 \times
E\_2, z\_n = (x\_n,y\_n). La suite
(x\_n) est une suite dans E compact, donc admet une sous suite
convergente (x\_\phi(n)). La suite (y\_\phi(n)) est une suite
dans E\_2 compact, donc admet une sous suite convergente
(y\_\phi(\psi(n))). La suite (x\_\phi(\psi(n))) est une sous suite
d'une suite convergente, donc encore convergente et donc la suite
(z\_\phi(\psi(n))) est convergente. Toute suite de E admet bien une
valeur d'adhérence.

Théorème~4.8.6 (Heine). Soit E un espace métrique compact et f : E \rightarrow~ F
continue. Alors f est uniformément continue.

Démonstration Supposons f non uniformément continue. Alors

\exists~\epsilon \textgreater{} 0,
\forall~~\eta \textgreater{} 0,\quad
\exists~a,b \in E, d(a,b) \textless{}
\eta\text et d(f(a),f(b)) ≥ \epsilon

en prenant \eta = 1 \over n+1 , on trouve a\_n
et b\_n tels que d(a\_n,b\_n) \textless{} 1
\over n+1 alors que d(f(a\_n),f(b\_n))
≥ \epsilon. La suite (a\_n) admet une sous suite convergente
(a\_\phi(n)) de limite a~; comme d(a\_\phi(n),b\_\phi(n))
\textless{} 1 \over \phi(n)+1 on a aussi
limb\_\phi(n)~ = a. Cependant
d(f(a\_\phi(n)),f(b\_\phi(n))) ≥ \epsilon, ce qui montre que la suite
(d(f(a\_\phi(n)),f(b\_\phi(n)))) ne tend pas vers 0, alors que
les deux suites f(a\_\phi(n)),f(b\_\phi(n)) admettent la même
limite f(a) (continuité de f au point a). C'est absurde.

\paragraph{4.8.2 Propriété de Borel Lebesgue}

Définition~4.8.2 On dit qu'un espace topologique E vérifie la propriété
de Borel Lebesgue si on a les conditions équivalentes (i) Pour toute
famille d'ouverts (U\_i)\_i\inI telle que E
= \⋃ ~
\_i\inIU\_i, il existe
i\_1,\\ldots,i\_k~
\in I tels que E =\ \⋃
 \_p=1^kU\_i\_p (ii) Pour toute famille
de fermés (F\_i)\_i\inI telle que
\⋂ ~
\_i\inIF\_i = \varnothing~, il existe
i\_1,\\ldots,i\_k~
\in I tels que \⋂ ~
\_p=1^kF\_i\_p = \varnothing~

Démonstration Ces deux propriétés sont équivalentes par passage au
complémentaire.

Remarque~4.8.2 On peut formuler (i) sous la forme~: de tout recouvrement
de E par des ouverts, on peut extraire un sous recouvrement fini.

On a le lemme suivant, qui nous servira pour la démonstration du
théorème~:

Lemme~4.8.7 Soit (E,d) un espace métrique compact et
(U\_i)\_i\inI une famille d'ouverts telle que E
= \⋃ ~
\_i\inIU\_i. Alors, il existe \epsilon \textgreater{} 0 tel que

\forall~~x \in E,
\existsi\_x~ \in I, B(x,\epsilon) \subset~
U\_i\_x

Démonstration Par l'absurde~; supposons que

\forall~~\epsilon \textgreater{} 0,
\existsx \in E, \\forall~~i \in I,
B(x,\epsilon)⊄U\_i

Prenons \epsilon = 1 \over n+1 et x\_n
correspondant. La suite (x\_n) a donc une valeur d'adhérence x.
Il existe i\_0 \in I tel que x \in U\_i\_0 et donc
un \eta \textgreater{} 0 tel que B(x,\eta) \subset~ U\_i\_0. Mais x
est valeur d'adhérence de la suite x\_n et donc il existe n
\textgreater{} 2\diagup\eta tel que x\_n \in B(x,\eta\diagup2). Alors, si y \in
B(x\_n, 1 \over n+1 ), on a d(y,x) \leq
d(y,x\_n) + d(x\_n,x) \textless{} 1
\over n+1 + \eta \over 2 \textless{} \eta
soit B(x\_n, 1 \over n+1 ) \subset~ B(x,\eta) \subset~
U\_i\_0. Mais ceci contredit la définition de
x\_n~: \forall~i \in I, B(x\_n~, 1
\over n+1 )⊄U\_i. C'est absurde.

Théorème~4.8.8 Un espace métrique E est compact si et seulement si~il
vérifie la propriété de Borel-Lebesgue.

Démonstration ⇐ Supposons que E vérifie la propriété de Borel-Lebesgue,
et soit (x\_n) une suite de E. Pour N \in \mathbb{N}~, posons X\_N =
\x\_n∣n ≥
N\. On a

\begin{align*} x\text valeur
d'adhérence de (x\_n)&& \%&
\\ & \Leftrightarrow &
\forall~V \in V (x), \\forall~~N \in \mathbb{N}~,
\existsn ≥ N, x\_n~ \in V \%&
\\ & \Leftrightarrow &
\forall~V \in V (x), \\forall~~N \in \mathbb{N}~,
V \bigcap X\_N\neq~\varnothing~ \%&
\\ & \Leftrightarrow &
\forall~~N \in \mathbb{N}~, x
\in\overlineX\_N \%&
\\ & \Leftrightarrow & x
\in\⋂
\_N\in\mathbb{N}~\overlineX\_N \%&
\\ \end{align*}

Supposons donc que la suite n'a pas de valeur d'adhérence~; on a alors
\⋂ ~
\_N\in\mathbb{N}~\overlineX\_N = \varnothing~ et comme ce sont
des fermés de E qui vérifie la propriété de Borel-Lebesgue, il existe
N\_1,\\ldots,N\_k~
tels que \⋂ ~
\_p=1^k\overlineX\_N\_p
= \varnothing~. Mais la suite (X\_N) est décroissante, et donc la suite
(\overlineX\_N) aussi. On a donc
\⋂ ~
\_p=1^k\overlineX\_N\_p
=
\overlineX\_max(N\_p)\mathrel\neq~~\varnothing~.
C'est absurde. Donc E est compact.

\rigtharrow~ Soit (U\_i)\_i\inI une famille d'ouverts telle que E
= \⋃ ~
\_i\inIU\_i. Alors, il existe \epsilon \textgreater{} 0 tel que

\forall~~x \in E,
\existsi\_x~ \in I, B(x,\epsilon) \subset~
U\_i\_x

Par le lemme de précompacité, on peut recouvrir E par un nombre fini de
boules de rayon \epsilon~: E = B(x\_1,\epsilon)
\cup\\ldots~ \cup
B(x\_k,\epsilon). Mais alors E \subset~ U\_i\_x\_ 1
\cup\\ldots~ \cup
U\_i\_x\_ k \subset~ E, ce qui démontre que l'on peut
recouvrir E par un nombre fini de U\_i.

\paragraph{4.8.3 Compacts de \mathbb{R}~ et \mathbb{R}~^n}

Lemme~4.8.9 Tout segment {[}a,b{]} de \mathbb{R}~ est compact.

Démonstration Soit (x\_n) une suite de {[}a,b{]}. On définit
deux suites (a\_p) et (b\_p) de la manière suivante~:
a\_0 = a et b\_0 = b~; si a\_p et b\_p
sont construits, on pose a\_p+1 = a\_p et b\_p+1
= a\_p+b\_p \over 2 si
\n \in \mathbb{N}~∣x\_n \in
{[}a\_p, a\_p+b\_p \over 2
{]}\ est infini~; sinon on pose a\_p+1 =
a\_p+b\_p \over 2 et b\_p+1 =
b\_p. On a évidemment~: (a\_p) croissante,
(b\_p) décroissante, b\_p - a\_p = b-a
\over 2^p et \n \in
\mathbb{N}~∣x\_n \in
{[}a\_p,b\_p{]}\ est infini. Les deux
suites étant ad\jmathacentes, soit \ell leur limite commune et \epsilon \textgreater{}
0. Il existe n \in \mathbb{N}~ tel que \ell - \epsilon \textless{} a\_n \leq \ell \leq
b\_n \textless{} \ell + \epsilon et donc \n \in
\mathbb{N}~∣x\_n \in{]}\ell - \epsilon,\ell +
\epsilon{[}\ est infini. Donc \ell est valeur d'adhérence de la
suite (x\_n).

Théorème~4.8.10 Les parties compactes de \mathbb{R}~ et \mathbb{R}~^n sont les
parties à la fois fermées et bornées pour une des distances usuelles.

Démonstration On sait dé\jmathà qu'une partie compacte doit être fermée et
bornée. Inversement soit A une partie fermée et bornée de \mathbb{R}~. Il existe
a,b \in \mathbb{R}~ tels que A \subset~ {[}a,b{]}. Alors A = A \bigcap {[}a,b{]} est fermé dans
{[}a,b{]} donc compacte. Même chose dans \mathbb{R}~^n en
rempla\ccant {[}a,b{]} par
{[}a\_1,b\_1{]} \times⋯ \times
{[}a\_n,b\_n{]} qui est compact comme produit de
compacts.

Corollaire~4.8.11 Soit E un espace métrique compact. Toute application
continue de E dans \mathbb{R}~ est bornée et atteint ses bornes inférieure et
supérieure.

Démonstration f(E) est compacte donc bornée et fermée (donc contient ses
bornes).

Corollaire~4.8.12 \mathbb{R}~ est complet.

Démonstration Une suite de Cauchy est bornée, donc peut être incluse
dans un segment qui est compact~; elle y admet donc une valeur
d'adhérence et donc elle converge.

Remarque~4.8.3 Bien entendu la validité de cette démonstration dépend de
la construction de \mathbb{R}~ qui est employée.

{[}
{[}
{[}
{[}

\end{document}

% \documentclass[]{article}
\usepackage[T1]{fontenc}
\usepackage{lmodern}
\usepackage{amssymb,amsmath}
\usepackage{ifxetex,ifluatex}
\usepackage{fixltx2e} % provides \textsubscript
% use upquote if available, for straight quotes in verbatim environments
\IfFileExists{upquote.sty}{\usepackage{upquote}}{}
\ifnum 0\ifxetex 1\fi\ifluatex 1\fi=0 % if pdftex
  \usepackage[utf8]{inputenc}
\else % if luatex or xelatex
  \ifxetex
    \usepackage{mathspec}
    \usepackage{xltxtra,xunicode}
  \else
    \usepackage{fontspec}
  \fi
  \defaultfontfeatures{Mapping=tex-text,Scale=MatchLowercase}
  \newcommand{\euro}{€}
\fi
% use microtype if available
\IfFileExists{microtype.sty}{\usepackage{microtype}}{}
\ifxetex
  \usepackage[setpagesize=false, % page size defined by xetex
              unicode=false, % unicode breaks when used with xetex
              xetex]{hyperref}
\else
  \usepackage[unicode=true]{hyperref}
\fi
\hypersetup{breaklinks=true,
            bookmarks=true,
            pdfauthor={},
            pdftitle={Espaces et parties connexes},
            colorlinks=true,
            citecolor=blue,
            urlcolor=blue,
            linkcolor=magenta,
            pdfborder={0 0 0}}
\urlstyle{same}  % don't use monospace font for urls
\setlength{\parindent}{0pt}
\setlength{\parskip}{6pt plus 2pt minus 1pt}
\setlength{\emergencystretch}{3em}  % prevent overfull lines
\setcounter{secnumdepth}{0}
 
/* start css.sty */
.cmr-5{font-size:50%;}
.cmr-7{font-size:70%;}
.cmmi-5{font-size:50%;font-style: italic;}
.cmmi-7{font-size:70%;font-style: italic;}
.cmmi-10{font-style: italic;}
.cmsy-5{font-size:50%;}
.cmsy-7{font-size:70%;}
.cmex-7{font-size:70%;}
.cmex-7x-x-71{font-size:49%;}
.msbm-7{font-size:70%;}
.cmtt-10{font-family: monospace;}
.cmti-10{ font-style: italic;}
.cmbx-10{ font-weight: bold;}
.cmr-17x-x-120{font-size:204%;}
.cmsl-10{font-style: oblique;}
.cmti-7x-x-71{font-size:49%; font-style: italic;}
.cmbxti-10{ font-weight: bold; font-style: italic;}
p.noindent { text-indent: 0em }
td p.noindent { text-indent: 0em; margin-top:0em; }
p.nopar { text-indent: 0em; }
p.indent{ text-indent: 1.5em }
@media print {div.crosslinks {visibility:hidden;}}
a img { border-top: 0; border-left: 0; border-right: 0; }
center { margin-top:1em; margin-bottom:1em; }
td center { margin-top:0em; margin-bottom:0em; }
.Canvas { position:relative; }
li p.indent { text-indent: 0em }
.enumerate1 {list-style-type:decimal;}
.enumerate2 {list-style-type:lower-alpha;}
.enumerate3 {list-style-type:lower-roman;}
.enumerate4 {list-style-type:upper-alpha;}
div.newtheorem { margin-bottom: 2em; margin-top: 2em;}
.obeylines-h,.obeylines-v {white-space: nowrap; }
div.obeylines-v p { margin-top:0; margin-bottom:0; }
.overline{ text-decoration:overline; }
.overline img{ border-top: 1px solid black; }
td.displaylines {text-align:center; white-space:nowrap;}
.centerline {text-align:center;}
.rightline {text-align:right;}
div.verbatim {font-family: monospace; white-space: nowrap; text-align:left; clear:both; }
.fbox {padding-left:3.0pt; padding-right:3.0pt; text-indent:0pt; border:solid black 0.4pt; }
div.fbox {display:table}
div.center div.fbox {text-align:center; clear:both; padding-left:3.0pt; padding-right:3.0pt; text-indent:0pt; border:solid black 0.4pt; }
div.minipage{width:100%;}
div.center, div.center div.center {text-align: center; margin-left:1em; margin-right:1em;}
div.center div {text-align: left;}
div.flushright, div.flushright div.flushright {text-align: right;}
div.flushright div {text-align: left;}
div.flushleft {text-align: left;}
.underline{ text-decoration:underline; }
.underline img{ border-bottom: 1px solid black; margin-bottom:1pt; }
.framebox-c, .framebox-l, .framebox-r { padding-left:3.0pt; padding-right:3.0pt; text-indent:0pt; border:solid black 0.4pt; }
.framebox-c {text-align:center;}
.framebox-l {text-align:left;}
.framebox-r {text-align:right;}
span.thank-mark{ vertical-align: super }
span.footnote-mark sup.textsuperscript, span.footnote-mark a sup.textsuperscript{ font-size:80%; }
div.tabular, div.center div.tabular {text-align: center; margin-top:0.5em; margin-bottom:0.5em; }
table.tabular td p{margin-top:0em;}
table.tabular {margin-left: auto; margin-right: auto;}
div.td00{ margin-left:0pt; margin-right:0pt; }
div.td01{ margin-left:0pt; margin-right:5pt; }
div.td10{ margin-left:5pt; margin-right:0pt; }
div.td11{ margin-left:5pt; margin-right:5pt; }
table[rules] {border-left:solid black 0.4pt; border-right:solid black 0.4pt; }
td.td00{ padding-left:0pt; padding-right:0pt; }
td.td01{ padding-left:0pt; padding-right:5pt; }
td.td10{ padding-left:5pt; padding-right:0pt; }
td.td11{ padding-left:5pt; padding-right:5pt; }
table[rules] {border-left:solid black 0.4pt; border-right:solid black 0.4pt; }
.hline hr, .cline hr{ height : 1px; margin:0px; }
.tabbing-right {text-align:right;}
span.TEX {letter-spacing: -0.125em; }
span.TEX span.E{ position:relative;top:0.5ex;left:-0.0417em;}
a span.TEX span.E {text-decoration: none; }
span.LATEX span.A{ position:relative; top:-0.5ex; left:-0.4em; font-size:85%;}
span.LATEX span.TEX{ position:relative; left: -0.4em; }
div.float img, div.float .caption {text-align:center;}
div.figure img, div.figure .caption {text-align:center;}
.marginpar {width:20%; float:right; text-align:left; margin-left:auto; margin-top:0.5em; font-size:85%; text-decoration:underline;}
.marginpar p{margin-top:0.4em; margin-bottom:0.4em;}
.equation td{text-align:center; vertical-align:middle; }
td.eq-no{ width:5%; }
table.equation { width:100%; } 
div.math-display, div.par-math-display{text-align:center;}
math .texttt { font-family: monospace; }
math .textit { font-style: italic; }
math .textsl { font-style: oblique; }
math .textsf { font-family: sans-serif; }
math .textbf { font-weight: bold; }
.partToc a, .partToc, .likepartToc a, .likepartToc {line-height: 200%; font-weight:bold; font-size:110%;}
.chapterToc a, .chapterToc, .likechapterToc a, .likechapterToc, .appendixToc a, .appendixToc {line-height: 200%; font-weight:bold;}
.index-item, .index-subitem, .index-subsubitem {display:block}
.caption td.id{font-weight: bold; white-space: nowrap; }
table.caption {text-align:center;}
h1.partHead{text-align: center}
p.bibitem { text-indent: -2em; margin-left: 2em; margin-top:0.6em; margin-bottom:0.6em; }
p.bibitem-p { text-indent: 0em; margin-left: 2em; margin-top:0.6em; margin-bottom:0.6em; }
.subsectionHead, .likesubsectionHead { margin-top:2em; font-weight: bold;}
.sectionHead, .likesectionHead { font-weight: bold;}
.quote {margin-bottom:0.25em; margin-top:0.25em; margin-left:1em; margin-right:1em; text-align:justify;}
.verse{white-space:nowrap; margin-left:2em}
div.maketitle {text-align:center;}
h2.titleHead{text-align:center;}
div.maketitle{ margin-bottom: 2em; }
div.author, div.date {text-align:center;}
div.thanks{text-align:left; margin-left:10%; font-size:85%; font-style:italic; }
div.author{white-space: nowrap;}
.quotation {margin-bottom:0.25em; margin-top:0.25em; margin-left:1em; }
h1.partHead{text-align: center}
.sectionToc, .likesectionToc {margin-left:2em;}
.subsectionToc, .likesubsectionToc {margin-left:4em;}
.sectionToc, .likesectionToc {margin-left:6em;}
.frenchb-nbsp{font-size:75%;}
.frenchb-thinspace{font-size:75%;}
.figure img.graphics {margin-left:10%;}
/* end css.sty */

\title{Espaces et parties connexes}
\author{}
\date{}

\begin{document}
\maketitle

\textbf{Warning: 
requires JavaScript to process the mathematics on this page.\\ If your
browser supports JavaScript, be sure it is enabled.}

\begin{center}\rule{3in}{0.4pt}\end{center}

[
[
[]
[

\section{4.9 Espaces et parties connexes}

\subsection{4.9.1 Notion de connexe}

Définition~4.9.1 Soit E un espace topologique. On dit que E n'est pas
connexe s'il vérifie les conditions équivalentes

\begin{itemize}
\itemsep1pt\parskip0pt\parsep0pt
\item
  (i) E est réunion de deux ouverts non vides disjoints
\item
  (ii) E est réunion de deux fermés non vides disjoints
\item
  (iii) il existe une partie de E distincte de \varnothing~ et de E qui est à la
  fois ouverte et fermée dans E.
\end{itemize}

Démonstration (i) \rigtharrow~(ii). Si E = U_1 \cup U_2 avec
U_1 et U_2 ouverts non vides disjoints, U_1
et U_2 sont aussi fermés puisque U_1 = cU_2
et U_2 = cU_1, d'où la propriété (ii). On montre de
même que (ii) \rigtharrow~(i).

(i) \rigtharrow~(iii). Si E = U_1 \cup U_2 avec U_1 et
U_2 ouverts non vides disjoints, alors U_1 est ouvert,
fermé (car U_1 = cU_2), distinct de \varnothing~ et de E.

(iii) \rigtharrow~(i). Si A est à la fois ouverte et fermée, distincte de \varnothing~ et de
E, on écrit E = A \cupcA, avec cA ouvert (complémentaire d'un fermé), les
deux étant non vides et disjoints.

Définition~4.9.2 Soit E un espace topologique et F une partie de E. On
dit que F est connexe si F muni de la topologie induite est connexe.

\subsection{4.9.2 Propriétés des connexes}

Théorème~4.9.1 Soit E,F deux espaces topologiques, f : E \rightarrow~ F continue.
Si E est connexe, alors f(E) est une partie connexe de F.

Démonstration En effet si f(E) est réunion de deux ouverts non vides
disjoints U_1 et U_2 de f(E), alors E est réunion des
deux ouverts f^-1(U_1) et
f^-1(U_2) qui sont encore disjoints et non vides

Corollaire~4.9.2 Soit E,F deux espaces topologiques, f : E \rightarrow~ F continue
et A une partie connexe de E. Alors f(A) est une partie connexe de F.

Démonstration En effet la restriction de f à A est encore continue, et
on peut lui appliquer le théorème précédent.

Proposition~4.9.3 Soit A une partie connexe de E. Alors toute partie B
telle que A \subset~ B \subset~\overlineA est connexe.

Démonstration En effet si B est réunion de deux ouverts de B non vides
disjoints U_1 et U_2, alors A est réunion des deux
ouverts de A disjoints~: U_1 \bigcap A et U_2 \bigcap A~; or ces
deux ouverts sont non vides car, A étant dense dans B, tout ouvert non
vide de B rencontre A. C'est absurde. Donc \overlineA
est connexe.

Proposition~4.9.4 Soit (A_i)_i\inI une famille de
parties connexes de E telle que
\⋂ ~
_i\inIA_i\neq~\varnothing~. Alors
\⋃ ~
_i\inIA_i est connexe.

Démonstration Soit a
\in\⋂ ~
_i\inIA_i. Si A =\
⋃  _i\inIA_i = U_1~ \cup
U_2 avec U_1 et U_2 ouverts disjoints de A,
alors chacun des A_i doit être contenu soit dans U_1,
soit dans U_2, sinon A_i serait réunion des deux
ouverts non vides disjoints A_i \bigcap U_1 et A_i
\bigcap U_2. Comme A_i contient a, il est forcément contenu
dans celui des deux ouverts U_1 et U_2 qui contient a.
Mais alors A lui-même est contenu dans cet ouvert, et donc l'autre est
vide.

Corollaire~4.9.5 Soit E un espace topologique et a un point de E. Alors
l'ensemble des connexes contenant a a un plus grand élément appelé la
composante connexe de a dans E~; deux composantes connexes sont soit
disjointes soit confondues~; toute composante connexe est fermée.

Démonstration La composante connexe de a est bien entendu
\⋃  _a\inA,
A\textconnexeA~; c'est un connexe d'après la
proposition précédente et c'est bien entendu le plus grand~; si A est la
composante connexe de a, B celle de b et si A \bigcap
B\neq~\varnothing~, alors A \cup B est connexe et donc A \cup B \subset~
A, soit B \subset~ A et de même A \subset~ B soit A = B. D'autre part,
\overlineA est encore connexe contenant a, donc
\overlineA \subset~ A et donc A est fermé.

Proposition~4.9.6 Soit
E_1,\\ldots,E_k~
des espaces connexes. Alors E_1 \times⋯ \times
E_k est connexe.

Démonstration Il suffit évidemment de montrer le résultat pour k = 2.
Soit E_1 \times E_2 = U_1 \cup U_2 avec
U_1 et U_2 ouverts disjoints. Remarquons que si a \in
E_1, l'application y\mapsto~(a,y) est un
homéomorphisme de E_2 sur \a\
\times E_2 qui est donc aussi connexe. Donc V doit être contenu soit
dans U_1 soit dans U_2 (sinon il serait réunion des
deux ouverts non vides disjoints (\a\
\times E_2) \bigcap U_1 et (\a\
\times E_2) \bigcap U_2). Soit b \in E_2~; pour la même
raison, on a par exemple E_1
\times\b\ \subset~ U_1. Alors, pour tout
a \in E_1, comme (a,b) \in\a\ \times
E_2 et E_1 \times\b\ \subset~
U_1, on a nécessairement \a\
\times E_2 \subset~ U_1 et donc E_1 \times E_2 \subset~
U_1, soit encore U_2 = \varnothing~.

\subsection{4.9.3 Connexes de \mathbb{R}~}

Définition~4.9.3 Une partie A de \mathbb{R}~ est dite convexe si
\forall~~a,b \in \mathbb{R}~, [a,b] \subset~ \mathbb{R}~.

Proposition~4.9.7 Les parties convexes de \mathbb{R}~ sont les intervalles.

Démonstration Il est clair que tout intervalle est convexe. L'ensemble
vide est bien entendu un intervalle. Soit donc A une partie convexe non
vide, m = inf~ A \in \mathbb{R}~
\cup\-\infty~\ et M =\
supA \in \mathbb{R}~ \cup\ + \infty~\. On a A \subset~
[m,M]. Pour montrer que A est un intervalle, il suffit de montrer
que ]m,M[\subset~ A. Or, soit x \in]m,M[. Il existe a \in A tel que m \leq a
< x (propriété caractéristique de la borne inférieure) et de
même, il existe b \in A tel que x < b \leq M. On a donc x
\in]a,b[\subset~ A~; d'où l'inclusion et le résultat.

Théorème~4.9.8 Les parties connexes de \mathbb{R}~ sont les intervalles.

Démonstration Soit A une partie connexe~; si A n'était pas convexe, il
existerait a,b \in A tel que ]a,b[⊄A (car a,b sont dans A)~; soit x
\in]a,b[ tel que x∉A. On a alors A = (A\bigcap]
-\infty~,x[) \cup (A\bigcap]x,+\infty~[), réunion de deux ouverts de A non vides et
disjoints~; c'est absurde. Donc A est convexe et donc un intervalle.

Inversement, soit I un intervalle~; alors il existe J intervalle ouvert
tel que J \subset~ I \subset~\overlineJ donc il suffit de montrer
que les intervalles ouverts sont connexes.

Soit I =]a,b[ un intervalle ouvert de \mathbb{R}~ et A une partie ouverte et
fermée, non vide et distincte de I. Soit x \in I \diagdown A. Alors A =
(A\bigcap]a,x[) \cup (A\bigcap]x,b[)~; au moins une des deux parties est non
vide, par exemple B = A\bigcap]x,b[= A \bigcap [x,b[. Cette partie est à la
fois ouverte et fermée dans I (intersection de deux ouverts de I et
aussi de deux fermés de I). Soit m = inf~ B ≥
x. On a m \in I et comme B est fermé dans I, on a m \in B. Mais alors
\exists~\epsilon > 0, ]m - \epsilon,m + \epsilon[\subset~ B, ce
qui contredit la définition de la borne inférieure. C'est absurde, donc
I est connexe.

Corollaire~4.9.9 (théorème des valeurs intermédiaires). Soit E un espace
topologique connexe et f : E \rightarrow~ \mathbb{R}~ continue. Alors
\mathrmIm~f est un
intervalle de \mathbb{R}~.

Démonstration f(E) est connexe, donc un intervalle.

\subsection{4.9.4 Connexité par arcs}

Définition~4.9.4 Soit E un espace topologique, a,b \in E. On appelle
chemin d'origine a et d'extrémité b dans E toute application continue \gamma
: [0,1] \rightarrow~ E telle que \gamma(0) = a et \gamma(1) = b.

Proposition~4.9.10 Soit E un espace topologique. La relation ''il existe
un chemin d'origine a et d'extrémité b'' est une relation d'équivalence
sur E.

Démonstration Cette relation est clairement réflexive (prendre \gamma
constant) et symétrique (prendre \gamma_1(t) = \gamma(1 - t)). Pour la
transitivité, soit \gamma_1 une chemin de a à b et \gamma_2 un
chemin de b à c. On définit \gamma : [0,1] \rightarrow~ E par \gamma(t) =
\left \ \cases
\gamma_1(2t) &si t \in [0,1\diagup2] \cr \gamma_2(2t
- 1)&si t \in [1\diagup2,1] \cr  \right ..
Alors \gamma est un chemin de a à c.

Définition~4.9.5 On dit que E est connexe par arcs si, pour tout couple
(a,b) \in E^2 il existe un chemin de a à b dans E.

Proposition~4.9.11 Tout espace topologique connexe par arcs est connexe.

Démonstration Soit a \in E et pour x \in E soit \gamma_x un chemin
d'origine a et d'extrémité x. Les \gamma_x([0,1]) sont des
images de connexes par une application continue, ils sont donc connexes.
Leur intersection contient a, et donc leur réunion est connexe. Mais on
a évidemment E =\ \⋃
 _x\inE\gamma_x([0,1]) (une réunion de parties de E est
contenue dans E et de plus tout élément x de E appartient à
\gamma_x([0,1])). Donc E est connexe.

[
[
[
[

\end{document}

% \documentclass[]{article}
\usepackage[T1]{fontenc}
\usepackage{lmodern}
\usepackage{amssymb,amsmath}
\usepackage{ifxetex,ifluatex}
\usepackage{fixltx2e} % provides \textsubscript
% use upquote if available, for straight quotes in verbatim environments
\IfFileExists{upquote.sty}{\usepackage{upquote}}{}
\ifnum 0\ifxetex 1\fi\ifluatex 1\fi=0 % if pdftex
  \usepackage[utf8]{inputenc}
\else % if luatex or xelatex
  \ifxetex
    \usepackage{mathspec}
    \usepackage{xltxtra,xunicode}
  \else
    \usepackage{fontspec}
  \fi
  \defaultfontfeatures{Mapping=tex-text,Scale=MatchLowercase}
  \newcommand{\euro}{€}
\fi
% use microtype if available
\IfFileExists{microtype.sty}{\usepackage{microtype}}{}
\ifxetex
  \usepackage[setpagesize=false, % page size defined by xetex
              unicode=false, % unicode breaks when used with xetex
              xetex]{hyperref}
\else
  \usepackage[unicode=true]{hyperref}
\fi
\hypersetup{breaklinks=true,
            bookmarks=true,
            pdfauthor={},
            pdftitle={Notion d'espace vectoriel norme},
            colorlinks=true,
            citecolor=blue,
            urlcolor=blue,
            linkcolor=magenta,
            pdfborder={0 0 0}}
\urlstyle{same}  % don't use monospace font for urls
\setlength{\parindent}{0pt}
\setlength{\parskip}{6pt plus 2pt minus 1pt}
\setlength{\emergencystretch}{3em}  % prevent overfull lines
\setcounter{secnumdepth}{0}
 
/* start css.sty */
.cmr-5{font-size:50%;}
.cmr-7{font-size:70%;}
.cmmi-5{font-size:50%;font-style: italic;}
.cmmi-7{font-size:70%;font-style: italic;}
.cmmi-10{font-style: italic;}
.cmsy-5{font-size:50%;}
.cmsy-7{font-size:70%;}
.cmex-7{font-size:70%;}
.cmex-7x-x-71{font-size:49%;}
.msbm-7{font-size:70%;}
.cmtt-10{font-family: monospace;}
.cmti-10{ font-style: italic;}
.cmbx-10{ font-weight: bold;}
.cmr-17x-x-120{font-size:204%;}
.cmsl-10{font-style: oblique;}
.cmti-7x-x-71{font-size:49%; font-style: italic;}
.cmbxti-10{ font-weight: bold; font-style: italic;}
p.noindent { text-indent: 0em }
td p.noindent { text-indent: 0em; margin-top:0em; }
p.nopar { text-indent: 0em; }
p.indent{ text-indent: 1.5em }
@media print {div.crosslinks {visibility:hidden;}}
a img { border-top: 0; border-left: 0; border-right: 0; }
center { margin-top:1em; margin-bottom:1em; }
td center { margin-top:0em; margin-bottom:0em; }
.Canvas { position:relative; }
li p.indent { text-indent: 0em }
.enumerate1 {list-style-type:decimal;}
.enumerate2 {list-style-type:lower-alpha;}
.enumerate3 {list-style-type:lower-roman;}
.enumerate4 {list-style-type:upper-alpha;}
div.newtheorem { margin-bottom: 2em; margin-top: 2em;}
.obeylines-h,.obeylines-v {white-space: nowrap; }
div.obeylines-v p { margin-top:0; margin-bottom:0; }
.overline{ text-decoration:overline; }
.overline img{ border-top: 1px solid black; }
td.displaylines {text-align:center; white-space:nowrap;}
.centerline {text-align:center;}
.rightline {text-align:right;}
div.verbatim {font-family: monospace; white-space: nowrap; text-align:left; clear:both; }
.fbox {padding-left:3.0pt; padding-right:3.0pt; text-indent:0pt; border:solid black 0.4pt; }
div.fbox {display:table}
div.center div.fbox {text-align:center; clear:both; padding-left:3.0pt; padding-right:3.0pt; text-indent:0pt; border:solid black 0.4pt; }
div.minipage{width:100%;}
div.center, div.center div.center {text-align: center; margin-left:1em; margin-right:1em;}
div.center div {text-align: left;}
div.flushright, div.flushright div.flushright {text-align: right;}
div.flushright div {text-align: left;}
div.flushleft {text-align: left;}
.underline{ text-decoration:underline; }
.underline img{ border-bottom: 1px solid black; margin-bottom:1pt; }
.framebox-c, .framebox-l, .framebox-r { padding-left:3.0pt; padding-right:3.0pt; text-indent:0pt; border:solid black 0.4pt; }
.framebox-c {text-align:center;}
.framebox-l {text-align:left;}
.framebox-r {text-align:right;}
span.thank-mark{ vertical-align: super }
span.footnote-mark sup.textsuperscript, span.footnote-mark a sup.textsuperscript{ font-size:80%; }
div.tabular, div.center div.tabular {text-align: center; margin-top:0.5em; margin-bottom:0.5em; }
table.tabular td p{margin-top:0em;}
table.tabular {margin-left: auto; margin-right: auto;}
div.td00{ margin-left:0pt; margin-right:0pt; }
div.td01{ margin-left:0pt; margin-right:5pt; }
div.td10{ margin-left:5pt; margin-right:0pt; }
div.td11{ margin-left:5pt; margin-right:5pt; }
table[rules] {border-left:solid black 0.4pt; border-right:solid black 0.4pt; }
td.td00{ padding-left:0pt; padding-right:0pt; }
td.td01{ padding-left:0pt; padding-right:5pt; }
td.td10{ padding-left:5pt; padding-right:0pt; }
td.td11{ padding-left:5pt; padding-right:5pt; }
table[rules] {border-left:solid black 0.4pt; border-right:solid black 0.4pt; }
.hline hr, .cline hr{ height : 1px; margin:0px; }
.tabbing-right {text-align:right;}
span.TEX {letter-spacing: -0.125em; }
span.TEX span.E{ position:relative;top:0.5ex;left:-0.0417em;}
a span.TEX span.E {text-decoration: none; }
span.LATEX span.A{ position:relative; top:-0.5ex; left:-0.4em; font-size:85%;}
span.LATEX span.TEX{ position:relative; left: -0.4em; }
div.float img, div.float .caption {text-align:center;}
div.figure img, div.figure .caption {text-align:center;}
.marginpar {width:20%; float:right; text-align:left; margin-left:auto; margin-top:0.5em; font-size:85%; text-decoration:underline;}
.marginpar p{margin-top:0.4em; margin-bottom:0.4em;}
.equation td{text-align:center; vertical-align:middle; }
td.eq-no{ width:5%; }
table.equation { width:100%; } 
div.math-display, div.par-math-display{text-align:center;}
math .texttt { font-family: monospace; }
math .textit { font-style: italic; }
math .textsl { font-style: oblique; }
math .textsf { font-family: sans-serif; }
math .textbf { font-weight: bold; }
.partToc a, .partToc, .likepartToc a, .likepartToc {line-height: 200%; font-weight:bold; font-size:110%;}
.chapterToc a, .chapterToc, .likechapterToc a, .likechapterToc, .appendixToc a, .appendixToc {line-height: 200%; font-weight:bold;}
.index-item, .index-subitem, .index-subsubitem {display:block}
.caption td.id{font-weight: bold; white-space: nowrap; }
table.caption {text-align:center;}
h1.partHead{text-align: center}
p.bibitem { text-indent: -2em; margin-left: 2em; margin-top:0.6em; margin-bottom:0.6em; }
p.bibitem-p { text-indent: 0em; margin-left: 2em; margin-top:0.6em; margin-bottom:0.6em; }
.paragraphHead, .likeparagraphHead { margin-top:2em; font-weight: bold;}
.subparagraphHead, .likesubparagraphHead { font-weight: bold;}
.quote {margin-bottom:0.25em; margin-top:0.25em; margin-left:1em; margin-right:1em; text-align:justify;}
.verse{white-space:nowrap; margin-left:2em}
div.maketitle {text-align:center;}
h2.titleHead{text-align:center;}
div.maketitle{ margin-bottom: 2em; }
div.author, div.date {text-align:center;}
div.thanks{text-align:left; margin-left:10%; font-size:85%; font-style:italic; }
div.author{white-space: nowrap;}
.quotation {margin-bottom:0.25em; margin-top:0.25em; margin-left:1em; }
h1.partHead{text-align: center}
.sectionToc, .likesectionToc {margin-left:2em;}
.subsectionToc, .likesubsectionToc {margin-left:4em;}
.subsubsectionToc, .likesubsubsectionToc {margin-left:6em;}
.frenchb-nbsp{font-size:75%;}
.frenchb-thinspace{font-size:75%;}
.figure img.graphics {margin-left:10%;}
/* end css.sty */

\title{Notion d'espace vectoriel norme}
\author{}
\date{}

\begin{document}
\maketitle

\textbf{Warning: \href{http://www.math.union.edu/locate/jsMath}{jsMath}
requires JavaScript to process the mathematics on this page.\\ If your
browser supports JavaScript, be sure it is enabled.}

\begin{center}\rule{3in}{0.4pt}\end{center}

{[}\href{coursse28.html}{next}{]}
{[}\hyperref[tailcoursse27.html]{tail}{]}
{[}\href{coursch6.html\#coursse27.html}{up}{]}

\subsubsection{5.1 Notion d'espace vectoriel normé}

\paragraph{5.1.1 Norme et distance associée}

Définition~5.1.1 Soit E un K-espace vectoriel . On appelle norme sur E
toute application
x\textbackslash{}mathrel\{↦\}\textbackslash{}\textbar{}x\textbackslash{}\textbar{}
de E dans \{ℝ\}\^{}\{+\} vérifiant

\begin{itemize}
\itemsep1pt\parskip0pt\parsep0pt
\item
  (i) \textbackslash{}\textbar{}x\textbackslash{}\textbar{} = 0
  \textbackslash{}mathrel\{⇔\} x = 0 (séparation)
\item
  (ii) \textbackslash{}\textbar{}λx\textbackslash{}\textbar{} =
  \textbar{}λ\textbar{}\textbackslash{}\textbar{}x\textbackslash{}\textbar{}
  (homogénéité)
\item
  (iii) \textbackslash{}\textbar{}x + y\textbackslash{}\textbar{}
  ≤\textbackslash{}\textbar{} x\textbackslash{}\textbar{}
  +\textbackslash{}\textbar{} y\textbackslash{}\textbar{} (inégalité
  triangulaire)
\end{itemize}

On appelle espace vectoriel normé un couple
(E,\textbackslash{}\textbar{}.\textbackslash{}\textbar{}) d'un K-espace
vectoriel et d'une norme sur E.

Exemple~5.1.1 Sur \{K\}\^{}\{n\}, on définit trois normes usuelles,
\textbackslash{}\textbar{}\{x\textbackslash{}\textbar{}\}\_\{1\}
=\textbackslash{}mathop\{ \textbackslash{}mathop\{∑ \}\}
\textbar{}\{x\}\_\{i\}\textbar{},
\textbackslash{}\textbar{}\{x\textbackslash{}\textbar{}\}\_\{2\} =
\textbackslash{}sqrt\{\textbackslash{}mathop\{\textbackslash{}mathop\{∑
\}\} \textbar{}\{x\}\_\{i\}\{\textbar{}\}\^{}\{2\}\},
\textbackslash{}\textbar{}\{x\textbackslash{}\textbar{}\}\_\{∞\}
=\textbackslash{}mathop\{ sup\}\textbar{}\{x\}\_\{i\}\textbar{}. De la
même fa\textbackslash{}c\{c\}on, on définit sur l'espace vectoriel des
fonctions continues de {[}0,1{]} dans K, C({[}0,1{]},K), trois normes
usuelles,
\textbackslash{}\textbar{}\{f\textbackslash{}\textbar{}\}\_\{1\}
=\{\textbackslash{}mathop\{∫ \}
\}\_\{0\}\^{}\{1\}\textbar{}f(t)\textbar{} dt,
\textbackslash{}\textbar{}\{f\textbackslash{}\textbar{}\}\_\{2\} =
\textbackslash{}sqrt\{\{\textbackslash{}mathop\{∫ \}
\}\_\{0\}\^{}\{1\}\textbar{}f(t)\{\textbar{}\}\^{}\{2\} dt\},
\textbackslash{}\textbar{}\{f\textbackslash{}\textbar{}\}\_\{∞\}
=\{\textbackslash{}mathop\{
sup\}\}\_\{x∈{[}0,1{]}\}\textbar{}f(x)\textbar{}.

Proposition~5.1.1 Soit E un K-espace vectoriel normé. L'application d :
E × E → \{ℝ\}\^{}\{+\} définie par d(x,y) =\textbackslash{}\textbar{} x
− y\textbackslash{}\textbar{} est une distance sur E appelée distance
associée à la norme. La topologie associée à cette distance est appelée
topologie définie par la norme.

Remarque~5.1.1 Si E est un espace vectoriel normé, on dispose de deux
familles importantes d'homéomorphismes de E sur lui même~: les
translations \{t\}\_\{v\} : x\textbackslash{}mathrel\{↦\}x + v et les
homothéties \{h\}\_\{λ\} : x\textbackslash{}mathrel\{↦\}λx
(λ\textbackslash{}mathrel\{≠\}0). On constate que tous les points ont
les mêmes propriétés topologiques et que deux boules ouvertes sont
toujours homéomorphes.

Définition~5.1.2 On appelle espace de Banach un espace vectoriel normé
complet (pour la distance associée).

Définition~5.1.3 Soit
(\{E\}\_\{1\},\textbackslash{}\textbar{}\{.\textbackslash{}\textbar{}\}\_\{1\}),\textbackslash{}mathop\{\textbackslash{}mathop\{\ldots{}\}\},(\{E\}\_\{k\},\textbackslash{}\textbar{}\{.\textbackslash{}\textbar{}\}\_\{k\})
des espaces vectoriels normés. On définit une norme sur le produit E =
\{E\}\_\{1\} ×\textbackslash{}mathrel\{⋯\} × \{E\}\_\{k\} en posant
\textbackslash{}\textbar{}x\textbackslash{}\textbar{}
=\textbackslash{}mathop\{
max\}\textbackslash{}\textbar{}\{x\{\}\_\{i\}\textbackslash{}\textbar{}\}\_\{i\}.
L'espace vectoriel
normé~(E,\textbackslash{}\textbar{}.\textbackslash{}\textbar{}) est
appelé l'espace vectoriel normé produit. Il est complet si chacun des
\{E\}\_\{i\} est complet.

\paragraph{5.1.2 Convexes, connexes}

Définition~5.1.4 Soit E un espace vectoriel normé, a,b ∈ E. On pose
{[}a,b{]} = \textbackslash{}\{ta + (1 − t)b\textbackslash{}mathrel\{∣\}t
∈ {[}0,1{]}\textbackslash{}\}. On dit qu'une partie A de E est convexe
si \textbackslash{}mathop\{∀\}a,b ∈ A,\textbackslash{}quad {[}a,b{]} ⊂
A.

Remarque~5.1.2 Le théorème d'associativité des barycentres montre
immédiatement par récurrence que si A est convexe,
\{a\}\_\{1\},\textbackslash{}mathop\{\textbackslash{}mathop\{\ldots{}\}\},\{a\}\_\{n\}
∈ A et
\{λ\}\_\{1\},\textbackslash{}mathop\{\textbackslash{}mathop\{\ldots{}\}\},\{λ\}\_\{n\}
sont des réels positifs de somme 1, alors \{λ\}\_\{1\}\{a\}\_\{1\} +
\textbackslash{}mathop\{\textbackslash{}mathop\{\ldots{}\}\} +
\{λ\}\_\{n\}\{a\}\_\{n\} est encore dans A.

Proposition~5.1.2 Toute partie convexe est connexe par arcs (et donc
connexe).

Démonstration γ(t) = (1 − t)a + tb est un chemin continu (l'application
est \textbackslash{}\textbar{}b −
a\textbackslash{}\textbar{}-lipschitzienne) d'origine a et d'extrémité
b.

Proposition~5.1.3 Dans un espace vectoriel normé, les boules sont
convexes (et donc connexes).

Démonstration Montrons le par exemple pour une boule ouverte B(a,r).
Soit x,y ∈ B(a,r) et t ∈ {[}0,1{]}. On a alors

\textbackslash{}begin\{eqnarray*\} \textbackslash{}\textbar{}tx + (1 −
t)y − a\textbackslash{}\textbar{}\& =\& \textbackslash{}\textbar{}t(x −
a) + (1 − t)(y − a)\textbackslash{}\textbar{}\%\&
\textbackslash{}\textbackslash{} \& ≤\& t\textbackslash{}\textbar{}x −
a\textbackslash{}\textbar{} + (1 − t)\textbackslash{}\textbar{}y −
a\textbackslash{}\textbar{} \%\& \textbackslash{}\textbackslash{} \&
\textless{}\& tr + (1 − t)r = r \%\& \textbackslash{}\textbackslash{}
\textbackslash{}end\{eqnarray*\}

car t ≥ 0,1 − t ≥ 0 et soit t, soit 1 − t est non nul.

Théorème~5.1.4 Dans un espace vectoriel normé, tout ouvert connexe est
connexe par arcs.

Démonstration Soit U un ouvert connexe. Soit ℛ la relation d'équivalence
sur U~: aℛb s'il existe un chemin d'origine a et d'extrémité b. Soit
C(a) la classe d'équivalence de a et montrons que C(a) est ouverte. Pour
cela soit b ∈ C(a) ⊂ U. Il existe r \textgreater{} 0 tel que B(b,r) ⊂ U.
Mais la boule, étant convexe, est connexe par arcs et donc pour tout x
de B(b,r) on a x ∈ C(b) = C(a), soit B(b,r) ⊂ C(a). Les classes
d'équivalences sont donc ouvertes dans U. Mais on a alors
\{\}\^{}\{c\}C(a) =\{\textbackslash{}mathop\{ \textbackslash{}mathop\{⋃
\}\} \}\_\{x\textbackslash{}mathrel\{∉\}C(a)\}C(x) est ouvert dans U et
donc C(a) est fermé dans U. Comme U est connexe, les seules parties
ouvertes et fermées dans U sont ∅ et U, soit C(a) = U et donc U est
connexe par arcs.

\paragraph{5.1.3 Continuité des opérations algébriques}

Théorème~5.1.5 Soit E un espace vectoriel normé. L'application s : E × E
→ E, (x,y)\textbackslash{}mathrel\{↦\}x + y est uniformément continue,
et l'application p : K × E → E, (λ,x)\textbackslash{}mathrel\{↦\}λx est
continue.

Démonstration On a \textbackslash{}\textbar{}s(x,y)
−s(x',y')\textbackslash{}\textbar{} =\textbackslash{}\textbar{} (x−x') +
(y −y')\textbackslash{}\textbar{} ≤\textbackslash{}\textbar{}
x−x'\textbackslash{}\textbar{} +\textbackslash{}\textbar{} y
−y'\textbackslash{}\textbar{} ≤
2\textbackslash{}mathop\{max\}(\textbackslash{}\textbar{}x−x'\textbackslash{}\textbar{},\textbackslash{}\textbar{}y
−y'\textbackslash{}\textbar{}) = 2\textbackslash{}\textbar{}(x,y) −
(x',y')\textbackslash{}\textbar{}, ce qui montre que s est
2-lipschitzienne.

Soit (\{λ\}\_\{0\},\{x\}\_\{0\}) ∈ K × E, λ ∈ K et x ∈ E. On a

\textbackslash{}begin\{eqnarray*\} \textbackslash{}\textbar{}p(λ,x) −
p(\{λ\}\_\{0\},\{x\}\_\{0\})\textbackslash{}\textbar{}\& =\&
\textbackslash{}\textbar{}λx −
\{λ\}\_\{0\}\{x\}\_\{0\}\textbackslash{}\textbar{} \%\&
\textbackslash{}\textbackslash{} \& =\& \textbackslash{}\textbar{}λ(x −
\{x\}\_\{0\}) + (λ −
\{λ\}\_\{0\})\{x\}\_\{0\}\textbackslash{}\textbar{}\%\&
\textbackslash{}\textbackslash{} \& ≤\&
\textbar{}λ\textbar{}\textbackslash{},\textbackslash{}\textbar{}x −
\{x\}\_\{0\}\textbackslash{}\textbar{} + \textbar{}λ −
\{λ\}\_\{0\}\textbar{}\textbackslash{},\textbackslash{}\textbar{}\{x\}\_\{0\}\textbackslash{}\textbar{}
\%\& \textbackslash{}\textbackslash{} \textbackslash{}end\{eqnarray*\}

Pour η ≤ 1, on a \textbar{}λ − \{λ\}\_\{0\}\textbar{} \textless{} η
⇒\textbar{}λ\textbar{}≤\textbar{}\{λ\}\_\{0\}\textbar{} + 1. Soit donc ε
\textgreater{} 0 et η =\textbackslash{}mathop\{ max\}(1,\{ ε
\textbackslash{}over 2(1+\textbar{}\{λ\}\_\{0\}\textbar{})\} ,\{ ε
\textbackslash{}over
2(1+\textbackslash{}\textbar{}\{x\}\_\{0\}\textbackslash{}\textbar{})\}
). Alors

\textbackslash{}begin\{eqnarray*\} \textbackslash{}\textbar{}(λ,x) −
(\{λ\}\_\{0\},\{x\}\_\{0\})\textbackslash{}\textbar{}
=\textbackslash{}mathop\{ max\}(\textbar{}λ −
\{λ\}\_\{0\}\textbar{},\textbackslash{}\textbar{}x −
\{x\}\_\{0\}\textbackslash{}\textbar{}) \textless{} η\&\&\%\&
\textbackslash{}\textbackslash{} \& ⇒\& \textbackslash{}\textbar{}p(λ,x)
− p(\{λ\}\_\{0\},\{x\}\_\{0\})\textbackslash{}\textbar{} ≤\{ ε
\textbackslash{}over 2\} +\{ ε \textbackslash{}over 2\} = ε\%\&
\textbackslash{}\textbackslash{} \textbackslash{}end\{eqnarray*\}

ce qui montre la continuité de p au point (\{λ\}\_\{0\},\{x\}\_\{0\}).

Corollaire~5.1.6 Soit X un espace métrique, E un espace vectoriel normé,
A une partie de X et a ∈\textbackslash{}overline\{A\}. (i) Si f et g
sont deux fonctions de X vers E telles que A ⊂\textbackslash{}mathop\{
Def\} (f) ∩\textbackslash{}mathop\{ Def\} (g), si f et g ont toutes deux
des limites en a suivant A et si α,β ∈ K, alors αf + βg a une limite en
a suivant A et on a

\{\textbackslash{}mathop\{lim\}\}\_\{x→a,x∈A\}(αf(x) + βg(x)) =
α\{\textbackslash{}mathop\{lim\}\}\_\{x→a,x∈A\}f(x) +
β\{\textbackslash{}mathop\{lim\}\}\_\{x→a,x∈A\}g(x)

(ii) Si φ est une fonction de X vers K et f une fonction de X vers E
telles que A ⊂\textbackslash{}mathop\{ Def\} (f)
∩\textbackslash{}mathop\{ Def\} (φ), si f et φ ont toutes deux des
limites en a suivant A, alors φf a une limite en a suivant A et on a

\{\textbackslash{}mathop\{lim\}\}\_\{x→a,x∈A\}(φ(x)f(x))
=\{\textbackslash{}mathop\{
lim\}\}\_\{x→a,x∈A\}φ(x)\{\textbackslash{}mathop\{lim\}\}\_\{x→a,x∈A\}f(x)

Remarque~5.1.3 Dans le cas de E = ℝ, les opérations algébriques sur ℝ ×
ℝ ne peuvent pas s'étendre de manière continue à
\textbackslash{}overline\{ℝ\} ×\textbackslash{}overline\{ℝ\}, ce qui
fait que certaines opérations sur les limites ne sont pas valides en
général. On a cependant le théorème suivant qui permet d'étendre les
opérations sur les limites sauf dans les cas d'indéterminations ''∞−∞''
et ''0 ×∞''

Théorème~5.1.7 (i) L'application s : ℝ × ℝ → ℝ,
(x,y)\textbackslash{}mathrel\{↦\}x + y s'étend en une application
continue de \textbackslash{}overline\{ℝ\} ×\textbackslash{}overline\{ℝ\}
∖\textbackslash{}\{(−∞,+∞),(+∞,−∞)\textbackslash{}\} dans
\textbackslash{}overline\{ℝ\} en posant x + (+∞) = +∞ si
x\textbackslash{}mathrel\{≠\} −∞ et x + (−∞) = −∞ si
x\textbackslash{}mathrel\{≠\} + ∞. (ii) L'application p : ℝ × ℝ → ℝ,
(x,y)\textbackslash{}mathrel\{↦\}xy s'étend en une application continue
de

\textbackslash{}overline\{ℝ\} ×\textbackslash{}overline\{ℝ\}
∖\textbackslash{}\{(0,+∞),(+∞,0),(0,−∞),(−∞,0)\textbackslash{}\}

dans \textbackslash{}overline\{ℝ\} en posant x × (+∞)
=\textbackslash{}mathop\{ sgn\}(x)∞ si x\textbackslash{}mathrel\{≠\}0 et
x × (−∞) = −\textbackslash{}mathop\{sgn\}(x)∞ si
x\textbackslash{}mathrel\{≠\}0.

Démonstration La vérification de la continuité est tout à fait
élémentaire. Remarquons que puisque ℝ × ℝ est dense dans
\textbackslash{}overline\{ℝ\} ×\textbackslash{}overline\{ℝ\}, ces
prolongements sont uniques.

{[}\href{coursse28.html}{next}{]} {[}\href{coursse27.html}{front}{]}
{[}\href{coursch6.html\#coursse27.html}{up}{]}

\end{document}

% \documentclass[]{article}
\usepackage[T1]{fontenc}
\usepackage{lmodern}
\usepackage{amssymb,amsmath}
\usepackage{ifxetex,ifluatex}
\usepackage{fixltx2e} % provides \textsubscript
% use upquote if available, for straight quotes in verbatim environments
\IfFileExists{upquote.sty}{\usepackage{upquote}}{}
\ifnum 0\ifxetex 1\fi\ifluatex 1\fi=0 % if pdftex
  \usepackage[utf8]{inputenc}
\else % if luatex or xelatex
  \ifxetex
    \usepackage{mathspec}
    \usepackage{xltxtra,xunicode}
  \else
    \usepackage{fontspec}
  \fi
  \defaultfontfeatures{Mapping=tex-text,Scale=MatchLowercase}
  \newcommand{\euro}{€}
\fi
% use microtype if available
\IfFileExists{microtype.sty}{\usepackage{microtype}}{}
\ifxetex
  \usepackage[setpagesize=false, % page size defined by xetex
              unicode=false, % unicode breaks when used with xetex
              xetex]{hyperref}
\else
  \usepackage[unicode=true]{hyperref}
\fi
\hypersetup{breaklinks=true,
            bookmarks=true,
            pdfauthor={},
            pdftitle={Applications lineaires continues},
            colorlinks=true,
            citecolor=blue,
            urlcolor=blue,
            linkcolor=magenta,
            pdfborder={0 0 0}}
\urlstyle{same}  % don't use monospace font for urls
\setlength{\parindent}{0pt}
\setlength{\parskip}{6pt plus 2pt minus 1pt}
\setlength{\emergencystretch}{3em}  % prevent overfull lines
\setcounter{secnumdepth}{0}
 
/* start css.sty */
.cmr-5{font-size:50%;}
.cmr-7{font-size:70%;}
.cmmi-5{font-size:50%;font-style: italic;}
.cmmi-7{font-size:70%;font-style: italic;}
.cmmi-10{font-style: italic;}
.cmsy-5{font-size:50%;}
.cmsy-7{font-size:70%;}
.cmex-7{font-size:70%;}
.cmex-7x-x-71{font-size:49%;}
.msbm-7{font-size:70%;}
.cmtt-10{font-family: monospace;}
.cmti-10{ font-style: italic;}
.cmbx-10{ font-weight: bold;}
.cmr-17x-x-120{font-size:204%;}
.cmsl-10{font-style: oblique;}
.cmti-7x-x-71{font-size:49%; font-style: italic;}
.cmbxti-10{ font-weight: bold; font-style: italic;}
p.noindent { text-indent: 0em }
td p.noindent { text-indent: 0em; margin-top:0em; }
p.nopar { text-indent: 0em; }
p.indent{ text-indent: 1.5em }
@media print {div.crosslinks {visibility:hidden;}}
a img { border-top: 0; border-left: 0; border-right: 0; }
center { margin-top:1em; margin-bottom:1em; }
td center { margin-top:0em; margin-bottom:0em; }
.Canvas { position:relative; }
li p.indent { text-indent: 0em }
.enumerate1 {list-style-type:decimal;}
.enumerate2 {list-style-type:lower-alpha;}
.enumerate3 {list-style-type:lower-roman;}
.enumerate4 {list-style-type:upper-alpha;}
div.newtheorem { margin-bottom: 2em; margin-top: 2em;}
.obeylines-h,.obeylines-v {white-space: nowrap; }
div.obeylines-v p { margin-top:0; margin-bottom:0; }
.overline{ text-decoration:overline; }
.overline img{ border-top: 1px solid black; }
td.displaylines {text-align:center; white-space:nowrap;}
.centerline {text-align:center;}
.rightline {text-align:right;}
div.verbatim {font-family: monospace; white-space: nowrap; text-align:left; clear:both; }
.fbox {padding-left:3.0pt; padding-right:3.0pt; text-indent:0pt; border:solid black 0.4pt; }
div.fbox {display:table}
div.center div.fbox {text-align:center; clear:both; padding-left:3.0pt; padding-right:3.0pt; text-indent:0pt; border:solid black 0.4pt; }
div.minipage{width:100%;}
div.center, div.center div.center {text-align: center; margin-left:1em; margin-right:1em;}
div.center div {text-align: left;}
div.flushright, div.flushright div.flushright {text-align: right;}
div.flushright div {text-align: left;}
div.flushleft {text-align: left;}
.underline{ text-decoration:underline; }
.underline img{ border-bottom: 1px solid black; margin-bottom:1pt; }
.framebox-c, .framebox-l, .framebox-r { padding-left:3.0pt; padding-right:3.0pt; text-indent:0pt; border:solid black 0.4pt; }
.framebox-c {text-align:center;}
.framebox-l {text-align:left;}
.framebox-r {text-align:right;}
span.thank-mark{ vertical-align: super }
span.footnote-mark sup.textsuperscript, span.footnote-mark a sup.textsuperscript{ font-size:80%; }
div.tabular, div.center div.tabular {text-align: center; margin-top:0.5em; margin-bottom:0.5em; }
table.tabular td p{margin-top:0em;}
table.tabular {margin-left: auto; margin-right: auto;}
div.td00{ margin-left:0pt; margin-right:0pt; }
div.td01{ margin-left:0pt; margin-right:5pt; }
div.td10{ margin-left:5pt; margin-right:0pt; }
div.td11{ margin-left:5pt; margin-right:5pt; }
table[rules] {border-left:solid black 0.4pt; border-right:solid black 0.4pt; }
td.td00{ padding-left:0pt; padding-right:0pt; }
td.td01{ padding-left:0pt; padding-right:5pt; }
td.td10{ padding-left:5pt; padding-right:0pt; }
td.td11{ padding-left:5pt; padding-right:5pt; }
table[rules] {border-left:solid black 0.4pt; border-right:solid black 0.4pt; }
.hline hr, .cline hr{ height : 1px; margin:0px; }
.tabbing-right {text-align:right;}
span.TEX {letter-spacing: -0.125em; }
span.TEX span.E{ position:relative;top:0.5ex;left:-0.0417em;}
a span.TEX span.E {text-decoration: none; }
span.LATEX span.A{ position:relative; top:-0.5ex; left:-0.4em; font-size:85%;}
span.LATEX span.TEX{ position:relative; left: -0.4em; }
div.float img, div.float .caption {text-align:center;}
div.figure img, div.figure .caption {text-align:center;}
.marginpar {width:20%; float:right; text-align:left; margin-left:auto; margin-top:0.5em; font-size:85%; text-decoration:underline;}
.marginpar p{margin-top:0.4em; margin-bottom:0.4em;}
.equation td{text-align:center; vertical-align:middle; }
td.eq-no{ width:5%; }
table.equation { width:100%; } 
div.math-display, div.par-math-display{text-align:center;}
math .texttt { font-family: monospace; }
math .textit { font-style: italic; }
math .textsl { font-style: oblique; }
math .textsf { font-family: sans-serif; }
math .textbf { font-weight: bold; }
.partToc a, .partToc, .likepartToc a, .likepartToc {line-height: 200%; font-weight:bold; font-size:110%;}
.chapterToc a, .chapterToc, .likechapterToc a, .likechapterToc, .appendixToc a, .appendixToc {line-height: 200%; font-weight:bold;}
.index-item, .index-subitem, .index-subsubitem {display:block}
.caption td.id{font-weight: bold; white-space: nowrap; }
table.caption {text-align:center;}
h1.partHead{text-align: center}
p.bibitem { text-indent: -2em; margin-left: 2em; margin-top:0.6em; margin-bottom:0.6em; }
p.bibitem-p { text-indent: 0em; margin-left: 2em; margin-top:0.6em; margin-bottom:0.6em; }
.paragraphHead, .likeparagraphHead { margin-top:2em; font-weight: bold;}
.subparagraphHead, .likesubparagraphHead { font-weight: bold;}
.quote {margin-bottom:0.25em; margin-top:0.25em; margin-left:1em; margin-right:1em; text-align:justify;}
.verse{white-space:nowrap; margin-left:2em}
div.maketitle {text-align:center;}
h2.titleHead{text-align:center;}
div.maketitle{ margin-bottom: 2em; }
div.author, div.date {text-align:center;}
div.thanks{text-align:left; margin-left:10%; font-size:85%; font-style:italic; }
div.author{white-space: nowrap;}
.quotation {margin-bottom:0.25em; margin-top:0.25em; margin-left:1em; }
h1.partHead{text-align: center}
.sectionToc, .likesectionToc {margin-left:2em;}
.subsectionToc, .likesubsectionToc {margin-left:4em;}
.subsubsectionToc, .likesubsubsectionToc {margin-left:6em;}
.frenchb-nbsp{font-size:75%;}
.frenchb-thinspace{font-size:75%;}
.figure img.graphics {margin-left:10%;}
/* end css.sty */

\title{Applications lineaires continues}
\author{}
\date{}

\begin{document}
\maketitle

\textbf{Warning: 
requires JavaScript to process the mathematics on this page.\\ If your
browser supports JavaScript, be sure it is enabled.}

\begin{center}\rule{3in}{0.4pt}\end{center}

[
[
[]
[

\subsubsection{5.2 Applications linéaires continues}

\paragraph{5.2.1 Caractérisations et normes des applications linéaires
continues}

Théorème~5.2.1 Soit E et F deux espaces vectoriels normés et u une
application linéaire de E dans F. Alors les conditions suivantes sont
équivalentes

\begin{itemize}
\itemsep1pt\parskip0pt\parsep0pt
\item
  (i) u est continue
\item
  (ii) u est continue au point 0
\item
  (iii) u est bornée sur la boule unité B'(0,1)
\item
  (iv) u est bornée sur la sphère unité S(0,1)
\item
  (v) il existe k ≥ 0 tel que \forall~~x \in E,
  \u(x)\ \leq
  k\x\
\item
  (vi) u est lipschitzienne
\end{itemize}

Démonstration (i) \rigtharrow~(ii) est évident.

(ii) \rigtharrow~(iii) Puisque u(0) = 0 et que u est continue en 0, il existe \eta
> 0 tel que
\x\ < \eta
\rigtharrow~\ u(x)\ \leq 1~; si x \in
B'(0,1), on a \ \eta \over 2
x\ \leq \eta \over 2 < \eta
soit \u( \eta \over 2
x)\ \leq 1 soit encore
\u(x)\ \leq 2
\over \eta .

(iii) \rigtharrow~(iv) est évident puisque S(0,1) \subset~ B'(0,1)

(iv) \rigtharrow~(v) soit k =\
sup_x\inS(0,1)\u(x)\
et soit x \in E. Si x = 0, on a
\u(x)\ \leq
k\x\~; si
x\neq~0, on a  x \over
\x\ \in S(0,1), soit
\u( x \over
\x\
)\ \leq k, soit
\u(x)\ \leq
k\x\.

(v) \rigtharrow~(vi) on a \u(x) -
u(y)\ =\ u(x -
y)\ \leq k\x -
y\, donc u est k-lipschitzienne.

(vi) \rigtharrow~(i) est évident

Théorème~5.2.2 Soit u : E \rightarrow~ F une application linéaire continue. Alors
on a l'égalité

\begin{align*}
sup_x\neq~0~
\u(x)\
\over
\x\ & =&
sup_\x\\leq1~\u(x)\
=\
sup_\x\=1\u(x)\
\%& \\ & =&
inf~ \k ≥
0∣\forall~~x \in E,
\u(x)\ \leq
k\x\\\%&
\\ \end{align*}

Ce nombre est appelé la norme de l'application linéaire u et noté
\u\~; on a
\forall~~x \in E,
\u(x)\
\leq\
u\\,\x\.

Démonstration Appelons M_1,M_2,M_3 et
M_4 les nombres en question dans l'ordre ci dessus. On a
clairement M_1 = M_3 puisque S(0,1) =
\ x \over
\x\
∣x\mathrel\neq~0\.
Comme S(0,1) \subset~ B'(0,1), on a M_3 \leq M_2. La
démonstration ci dessus de (iv) \rigtharrow~(v) nous a montré que
\forall~~x \in E,
\u(x)\ \leq
M_2\x\, soit
M_4 \leq M_2. Remarquons de plus que

\k ≥
0∣\forall~~x \in E,
\u(x)\ \leq
k\x\\
= \⋂
_x\neq~0[
\u(x)\
\over
\x\ ,+\infty~[

est un fermé de \mathbb{R}~ (intersection de fermés), donc contient sa borne
inférieure~; on a donc \forall~~x \in E,
\u(x)\ \leq
M_4\x\, soit
pour x\neq~0, 
\u(x)\
\over
\x\ \leq M_4
et donc M_1 \leq M_4. Si on récapitule, on a montré que
M_1 \leq M_4 \leq M_2 \leq M_3 = M_1
ce qui montre les égalités. On a vu de plus que
\forall~~x \in E,
\u(x)\ \leq
M_4\x\, soit
encore \forall~~x \in E,
\u(x)\
\leq\
u\\,\x\.

\paragraph{5.2.2 L'espace vectoriel normé des applications linéaires
continues de E dans F}

Théorème~5.2.3 L'application
u\mapsto~\u\
est une norme sur l'espace vectoriel des applications linéaires
continues de E dans F.

Démonstration La formule \forall~~x \in E,
\u(x)\
\leq\
u\\,\x\
montre que \u\ = 0 \rigtharrow~ u
= 0, la réciproque étant évidente. La formule
\u\
=\
sup_\x\=1\u(x)\
permet de montrer sans problème l'homogénéité et l'inégalité
triangulaire.

Théorème~5.2.4 Soit u : E \rightarrow~ F et v : F \rightarrow~ G linéaires continues. Alors
\v \cdot u\
\leq\
v\\,\u\.

Démonstration On a \v \cdot
u(x)\ \leq\
v\\,\u(x)\
\leq\
v\\,\u\\,\x\
et on sait que \v \cdot u\
est le plus petit k tel que \forall~~x \in E,
\v \cdot u(x)\ \leq
k\x\. Soit
\v \cdot u\
\leq\
v\\,\u\.

Remarque~5.2.1 Cette propriété (dite de sous multiplicativité) de la
norme est particulièrement commode pour ce type de norme sur l'espace
vectoriel normé des applications linéaires continues de E dans F (normes
dites subordonnées à des normes sur E et F)~; c'est ainsi que dans un
espace de matrices, on aura souvent intérêt à poser
\A\
=\
sup_\X\=1\AX\
de fa\ccon à disposer d'une inégalité
\AB\
\leq\
A\\,\B\.

Théorème~5.2.5 Si F est un espace vectoriel normé complet, l'espace
vectoriel normé des applications linéaires continues de E dans F est
complet.

Démonstration Soit (u_n) une suite d'applications linéaires
continues qui est une suite de Cauchy pour la norme que l'on vient de
définir. Soit x \in E, on a \u_p(x) -
u_q(x)\ \leq\
u_p -
u_q\\,\x\,
ce qui montre que la suite (u_n(x)) est une suite de Cauchy
dans F~; comme F est complet, elle converge vers une limite qui dépend
de x et que nous noterons u(x). La relation u_n(\alpha~x + \beta~y) =
\alpha~u_n(x) + \beta~u_n(y) donne par passage à la limite u(\alpha~x +
\beta~y) = \alpha~u(x) + \beta~u(y) ce qui montre que u est linéaire. Soit \epsilon
> 0 et N \in \mathbb{N}~ tel que p,q ≥ N \rigtharrow~\
u_p - u_q\ < \epsilon.
Pour x \in E, q > n ≥ N, on a
\u_q(x) -
u_n(x)\ \leq
\epsilon\x\. Faisons tendre q
vers + \infty~~; on obtient \u(x) -
u_n(x)\ \leq
\epsilon\x\~; ceci montre
tout d'abord que u - u_n est continue et donc u = (u -
u_n) + u_n aussi, et que n ≥ N
\rigtharrow~\ u - u_n\ \leq
\epsilon, et donc que u_n converge vers u au sens de la norme des
applications linéaires continues~; ceci achève la démonstration.

\paragraph{5.2.3 Equivalence des normes}

Définition~5.2.1 Soit E un K-espace vectoriel . On dit que deux normes
\._1 et
\._2 sur E
sont équivalentes si les distances associées sont équivalentes,
c'est-à-dire s'il existe \alpha~,\beta~ > 0 tels que

\forall~~x \in E,\quad
\alpha~\x_1
\leq\ x_2 \leq
\beta~\x_1

Théorème~5.2.6 Deux normes sont équivalentes si et seulement si~elles
définissent la même topologie.

Démonstration Si deux normes sont équivalentes, les distances associées
aussi et donc elles définissent la même topologie. Inversement,
supposons que les deux normes définissent la même topologie. Alors
\mathrmId_E :
(E,\._1) \rightarrow~
(E,\._2) est
linéaire continue, donc il existe k ≥ 0 tel que
\forall~~x \in E,
\\mathrmId_E(x)_2
\leq k\x_1 soit
encore \forall~~x \in E,
\x_2 \leq
k\x_1 et il
est alors clair que k\neq~0. De même avec
\mathrmId_E :
(E,\._2) \rightarrow~
(E,\._1), on
peut trouver un k' > 0 tel que \forall~~x
\in E, \x_1 \leq
k'\x_2, soit
\forall~x \in E, 1 \over k'~
\x_1
\leq\ x_2 \leq
k\x_1.

\paragraph{5.2.4 Caractérisation des applications bilinéaires continues}

Théorème~5.2.7 Soit E_1,E_2 et F des espaces
vectoriels normés et u une application bilinéaire de E_1 \times
E_2 dans F. Alors les conditions suivantes sont équivalentes

\begin{itemize}
\itemsep1pt\parskip0pt\parsep0pt
\item
  (i) u est continue
\item
  (ii) u est continue au point (0,0)
\item
  (iii) u est bornée sur B'(0,1) \times B'(0,1)
\item
  (iv) u est bornée sur S(0,1) \times S(0,1)
\item
  (v) il existe k ≥ 0 tel que
  \forall~(x_1,x_2) \in E_1~
  \times E_2,
  \u(x_1,x_2)\
  \leq
  k\x_1\\,\x_2\
\end{itemize}

Démonstration Tout à fait similaire à celle effectuée pour les
applications linéaires, sauf en ce qui concerne (v) \rigtharrow~(i) (car il n'y a
plus l'intermédiaire (vi)). Mais on a, si (a_1,a_2) \in
E_1 \times E_2

\begin{align*} u(x_1,x_2) -
u(a_1,a_2)& =& u(a_1 -
x_1,a_2 - x_2) + u(x_1 -
a_1,a_2)\%& \\
\text & & +u(a_1,x_2 -
a_2) \%& \\
\end{align*}

(facile) et donc, si (v) est vérifiée

\begin{align*}
\u(x_1,x_2) -
u(a_1,a_2)& \leq&
k\a_1 -
x_1\\,\a_2
- x_2\ \%&
\\ & +&
k\x_1 -
a_1\\,\a_2\
+
k\a_1\\,\x_2
- a_2\\%&
\\ \end{align*}

ce qui montre clairement la continuité (non uniforme) de u au point
(a_1,a_2).

Exemple~5.2.1 La formule \v \cdot
u\ \leq\
v\\,\u\
montre que l'application bilinéaire de composition est continue sur les
espaces d'applications linéaires continues adéquats.

[
[
[
[

\end{document}

% \documentclass[]{article}
\usepackage[T1]{fontenc}
\usepackage{lmodern}
\usepackage{amssymb,amsmath}
\usepackage{ifxetex,ifluatex}
\usepackage{fixltx2e} % provides \textsubscript
% use upquote if available, for straight quotes in verbatim environments
\IfFileExists{upquote.sty}{\usepackage{upquote}}{}
\ifnum 0\ifxetex 1\fi\ifluatex 1\fi=0 % if pdftex
  \usepackage[utf8]{inputenc}
\else % if luatex or xelatex
  \ifxetex
    \usepackage{mathspec}
    \usepackage{xltxtra,xunicode}
  \else
    \usepackage{fontspec}
  \fi
  \defaultfontfeatures{Mapping=tex-text,Scale=MatchLowercase}
  \newcommand{\euro}{€}
\fi
% use microtype if available
\IfFileExists{microtype.sty}{\usepackage{microtype}}{}
\ifxetex
  \usepackage[setpagesize=false, % page size defined by xetex
              unicode=false, % unicode breaks when used with xetex
              xetex]{hyperref}
\else
  \usepackage[unicode=true]{hyperref}
\fi
\hypersetup{breaklinks=true,
            bookmarks=true,
            pdfauthor={},
            pdftitle={Espaces vectoriels normes de dimensions finies},
            colorlinks=true,
            citecolor=blue,
            urlcolor=blue,
            linkcolor=magenta,
            pdfborder={0 0 0}}
\urlstyle{same}  % don't use monospace font for urls
\setlength{\parindent}{0pt}
\setlength{\parskip}{6pt plus 2pt minus 1pt}
\setlength{\emergencystretch}{3em}  % prevent overfull lines
\setcounter{secnumdepth}{0}
 
/* start css.sty */
.cmr-5{font-size:50%;}
.cmr-7{font-size:70%;}
.cmmi-5{font-size:50%;font-style: italic;}
.cmmi-7{font-size:70%;font-style: italic;}
.cmmi-10{font-style: italic;}
.cmsy-5{font-size:50%;}
.cmsy-7{font-size:70%;}
.cmex-7{font-size:70%;}
.cmex-7x-x-71{font-size:49%;}
.msbm-7{font-size:70%;}
.cmtt-10{font-family: monospace;}
.cmti-10{ font-style: italic;}
.cmbx-10{ font-weight: bold;}
.cmr-17x-x-120{font-size:204%;}
.cmsl-10{font-style: oblique;}
.cmti-7x-x-71{font-size:49%; font-style: italic;}
.cmbxti-10{ font-weight: bold; font-style: italic;}
p.noindent { text-indent: 0em }
td p.noindent { text-indent: 0em; margin-top:0em; }
p.nopar { text-indent: 0em; }
p.indent{ text-indent: 1.5em }
@media print {div.crosslinks {visibility:hidden;}}
a img { border-top: 0; border-left: 0; border-right: 0; }
center { margin-top:1em; margin-bottom:1em; }
td center { margin-top:0em; margin-bottom:0em; }
.Canvas { position:relative; }
li p.indent { text-indent: 0em }
.enumerate1 {list-style-type:decimal;}
.enumerate2 {list-style-type:lower-alpha;}
.enumerate3 {list-style-type:lower-roman;}
.enumerate4 {list-style-type:upper-alpha;}
div.newtheorem { margin-bottom: 2em; margin-top: 2em;}
.obeylines-h,.obeylines-v {white-space: nowrap; }
div.obeylines-v p { margin-top:0; margin-bottom:0; }
.overline{ text-decoration:overline; }
.overline img{ border-top: 1px solid black; }
td.displaylines {text-align:center; white-space:nowrap;}
.centerline {text-align:center;}
.rightline {text-align:right;}
div.verbatim {font-family: monospace; white-space: nowrap; text-align:left; clear:both; }
.fbox {padding-left:3.0pt; padding-right:3.0pt; text-indent:0pt; border:solid black 0.4pt; }
div.fbox {display:table}
div.center div.fbox {text-align:center; clear:both; padding-left:3.0pt; padding-right:3.0pt; text-indent:0pt; border:solid black 0.4pt; }
div.minipage{width:100%;}
div.center, div.center div.center {text-align: center; margin-left:1em; margin-right:1em;}
div.center div {text-align: left;}
div.flushright, div.flushright div.flushright {text-align: right;}
div.flushright div {text-align: left;}
div.flushleft {text-align: left;}
.underline{ text-decoration:underline; }
.underline img{ border-bottom: 1px solid black; margin-bottom:1pt; }
.framebox-c, .framebox-l, .framebox-r { padding-left:3.0pt; padding-right:3.0pt; text-indent:0pt; border:solid black 0.4pt; }
.framebox-c {text-align:center;}
.framebox-l {text-align:left;}
.framebox-r {text-align:right;}
span.thank-mark{ vertical-align: super }
span.footnote-mark sup.textsuperscript, span.footnote-mark a sup.textsuperscript{ font-size:80%; }
div.tabular, div.center div.tabular {text-align: center; margin-top:0.5em; margin-bottom:0.5em; }
table.tabular td p{margin-top:0em;}
table.tabular {margin-left: auto; margin-right: auto;}
div.td00{ margin-left:0pt; margin-right:0pt; }
div.td01{ margin-left:0pt; margin-right:5pt; }
div.td10{ margin-left:5pt; margin-right:0pt; }
div.td11{ margin-left:5pt; margin-right:5pt; }
table[rules] {border-left:solid black 0.4pt; border-right:solid black 0.4pt; }
td.td00{ padding-left:0pt; padding-right:0pt; }
td.td01{ padding-left:0pt; padding-right:5pt; }
td.td10{ padding-left:5pt; padding-right:0pt; }
td.td11{ padding-left:5pt; padding-right:5pt; }
table[rules] {border-left:solid black 0.4pt; border-right:solid black 0.4pt; }
.hline hr, .cline hr{ height : 1px; margin:0px; }
.tabbing-right {text-align:right;}
span.TEX {letter-spacing: -0.125em; }
span.TEX span.E{ position:relative;top:0.5ex;left:-0.0417em;}
a span.TEX span.E {text-decoration: none; }
span.LATEX span.A{ position:relative; top:-0.5ex; left:-0.4em; font-size:85%;}
span.LATEX span.TEX{ position:relative; left: -0.4em; }
div.float img, div.float .caption {text-align:center;}
div.figure img, div.figure .caption {text-align:center;}
.marginpar {width:20%; float:right; text-align:left; margin-left:auto; margin-top:0.5em; font-size:85%; text-decoration:underline;}
.marginpar p{margin-top:0.4em; margin-bottom:0.4em;}
.equation td{text-align:center; vertical-align:middle; }
td.eq-no{ width:5%; }
table.equation { width:100%; } 
div.math-display, div.par-math-display{text-align:center;}
math .texttt { font-family: monospace; }
math .textit { font-style: italic; }
math .textsl { font-style: oblique; }
math .textsf { font-family: sans-serif; }
math .textbf { font-weight: bold; }
.partToc a, .partToc, .likepartToc a, .likepartToc {line-height: 200%; font-weight:bold; font-size:110%;}
.chapterToc a, .chapterToc, .likechapterToc a, .likechapterToc, .appendixToc a, .appendixToc {line-height: 200%; font-weight:bold;}
.index-item, .index-subitem, .index-subsubitem {display:block}
.caption td.id{font-weight: bold; white-space: nowrap; }
table.caption {text-align:center;}
h1.partHead{text-align: center}
p.bibitem { text-indent: -2em; margin-left: 2em; margin-top:0.6em; margin-bottom:0.6em; }
p.bibitem-p { text-indent: 0em; margin-left: 2em; margin-top:0.6em; margin-bottom:0.6em; }
.paragraphHead, .likeparagraphHead { margin-top:2em; font-weight: bold;}
.subparagraphHead, .likesubparagraphHead { font-weight: bold;}
.quote {margin-bottom:0.25em; margin-top:0.25em; margin-left:1em; margin-right:1em; text-align:\jmathustify;}
.verse{white-space:nowrap; margin-left:2em}
div.maketitle {text-align:center;}
h2.titleHead{text-align:center;}
div.maketitle{ margin-bottom: 2em; }
div.author, div.date {text-align:center;}
div.thanks{text-align:left; margin-left:10%; font-size:85%; font-style:italic; }
div.author{white-space: nowrap;}
.quotation {margin-bottom:0.25em; margin-top:0.25em; margin-left:1em; }
h1.partHead{text-align: center}
.sectionToc, .likesectionToc {margin-left:2em;}
.subsectionToc, .likesubsectionToc {margin-left:4em;}
.subsubsectionToc, .likesubsubsectionToc {margin-left:6em;}
.frenchb-nbsp{font-size:75%;}
.frenchb-thinspace{font-size:75%;}
.figure img.graphics {margin-left:10%;}
/* end css.sty */

\title{Espaces vectoriels normes de dimensions finies}
\author{}
\date{}

\begin{document}
\maketitle

\textbf{Warning: 
requires JavaScript to process the mathematics on this page.\\ If your
browser supports JavaScript, be sure it is enabled.}

\begin{center}\rule{3in}{0.4pt}\end{center}

{[}
{[}
{[}{]}
{[}

\subsubsection{5.3 Espaces vectoriels normés de dimensions finies}

\paragraph{5.3.1 Equivalence des normes}

Lemme~5.3.1 Toutes les normes sur \mathbb{R}~^n sont équivalentes.

Démonstration Posons
\\textbar{}x\\textbar{}
= max\textbar{}x\_i~\textbar{} et
montrons que toute autre norme N est équivalente à cette norme. Soit
(e\_1,\\ldotse\_n~)
la base canonique de \mathbb{R}~^n et x \in \mathbb{R}~^n. On a N(x) =
N(\\sum ~
x\_ie\_i)
\leq\\sum ~
\textbar{}x\_i\textbar{}N(e\_i)
\leq\
max\textbar{}x\_i\textbar{}\\\sum
 \_iN(e\_i) =
\beta~\\textbar{}x\\textbar{}. On en déduit que
\textbar{}N(x) - N(y)\textbar{}\leq N(x - y) \leq \beta~\\textbar{}x
- y\\textbar{} ce qui démontre que l'application N :
(\mathbb{R}~^n,\\textbar{}.\\textbar{}) \rightarrow~
\mathbb{R}~ est continue. Soit S = \x \in
\mathbb{R}~^n∣\\textbar{}x\\textbar{}
= 1\~; S est une partie compacte de
(\mathbb{R}~^n,\\textbar{}.\\textbar{})
(fermée bornée), donc l'application N y atteint sa borne inférieure.
Soit \alpha~ = inf \_x\inS~N(x) =
N(x\_0). On a x\_0\neq~0 (car
x\_0 \in S) donc \alpha~ \textgreater{} 0. Alors, si x \in \mathbb{R}~^n,
x\neq~0, on a  x \over
\\textbar{}x\\textbar{} \in S soit N( x
\over
\\textbar{}x\\textbar{} ) ≥ \alpha~ soit
encore N(x) ≥ \alpha~\\textbar{}x\\textbar{}. On
a donc trouvé \alpha~ et \beta~ strictement positifs tels que
\forall~x \in \mathbb{R}~^n~,
\alpha~\\textbar{}x\\textbar{} \leq N(x) \leq
\beta~\\textbar{}x\\textbar{}, ce qu'il fallait
démontrer.

Théorème~5.3.2 Sur un K-espace vectoriel normé~de dimension finie toutes
les normes sont équivalentes.

Démonstration Tout \mathbb{C}-espace vectoriel normé~étant aussi un \mathbb{R}~-espace
vectoriel normé, il suffit de le montrer lorsque le corps de base est \mathbb{R}~.
Soit N\_1 et N\_2 deux normes sur E~; soit \mathcal{E} =
(e\_1,\\ldots,e\_n~)
une base de E et u : \mathbb{R}~^n \rightarrow~ E définie par
u(x\_1,\\ldots,x\_n~)
= \\sum ~
x\_ie\_i (u est un isomorphisme d'espaces vectoriels).
Alors N\_1 \cdot u et N\_2 \cdot u sont deux normes sur
\mathbb{R}~^n (facile), elles sont donc équivalentes, et donc il existe
\alpha~ et \beta~ strictement positifs tels que \forall~~x \in
\mathbb{R}~^n, \alpha~N\_1(u(x)) \leq N\_2(u(x)) \leq
\beta~N\_1(u(x)). Mais tout élément de E s'écrivant sous la forme
u(x), on a, \forall~y \in E, \alpha~N\_1~(y) \leq
N\_2(y) \leq \beta~N\_1(y), ce qu'il fallait démontrer.

\paragraph{5.3.2 Propriétés topologiques et métriques des espaces
vectoriels normés de dimension finie}

Remarque~5.3.1 Tout \mathbb{C}-espace vectoriel normé~étant aussi un \mathbb{R}~-espace
vectoriel normé, il suffit de considérer le cas où le corps de base est
\mathbb{R}~. Soit (E,\\textbar{}.\\textbar{}) un
espace vectoriel normé~de dimension finie, soit \mathcal{E} =
(e\_1,\\ldots,e\_n~)
une base de E et u : \mathbb{R}~^n \rightarrow~ E définie par
u(x\_1,\\ldots,x\_n~)
= \\sum ~
x\_ie\_i (u est un isomorphisme d'espaces vectoriels).
Alors N :
x\mapsto~\\textbar{}u(x)\\textbar{}
est une norme sur \mathbb{R}~^n qui est équivalente à la norme
\\textbar{}.\\textbar{}\_\infty~~; de
plus l'application u : (\mathbb{R}~^n,N) \rightarrow~
(E,\\textbar{}.\\textbar{}) est une
isométrie~; on en déduit que
(E,\\textbar{}.\\textbar{}) a, en tant
qu'espace vectoriel normé, les mêmes propriétés que
(\mathbb{R}~^n,\\textbar{}.\\textbar{}\_\infty~)
c'est-à-dire

Théorème~5.3.3 Tout espace vectoriel normé de dimension finie est
complet~; les parties compactes en sont les fermés bornés.

Corollaire~5.3.4 Tout sous-espace vectoriel de dimension finie d'un
espace vectoriel normé~est fermé.

Démonstration Muni de la restriction de la norme, il est complet, donc
fermé.

\paragraph{5.3.3 Continuité des applications linéaires}

Théorème~5.3.5 Soit E et F deux espaces vectoriels normés, E étant
supposé de dimension finie. Alors toute application linéaire de E dans F
est continue.

Démonstration Soit \mathcal{E} =
(e\_1,\\ldots,e\_n~)
une base de E~; comme toute les normes sur E sont équivalentes, on peut
prendre la norme définie par
\\textbar{}x\\textbar{}
= sup\textbar{}x\_i~\textbar{} si x
= \\sum ~
x\_ie\_i. On a alors

\begin{align*}
\\textbar{}u(x)\\textbar{}& =&
\\textbar{}\\sum
x\_iu(e\_i)\\textbar{}
\leq\\sum
\textbar{}x\_i\textbar{}\,\\textbar{}u(e\_i)\\textbar{}\%&
\\ & \leq&
\\textbar{}x\\textbar{}\\sum
\\textbar{}u(e\_i)\\textbar{} =
K\\textbar{}x\\textbar{} \%&
\\ \end{align*}

Ceci montre la continuité de l'application linéaire u.

Remarque~5.3.2 Ce résultat s'étend sans difficulté aux applications
bilinéaires de E\_1 \times E\_2 dans F à condition que
E\_1 et E\_2 soient de dimensions finies~; de même pour
des applications p-linéaires.

{[}
{[}
{[}
{[}

\end{document}

% \documentclass[]{article}
\usepackage[T1]{fontenc}
\usepackage{lmodern}
\usepackage{amssymb,amsmath}
\usepackage{ifxetex,ifluatex}
\usepackage{fixltx2e} % provides \textsubscript
% use upquote if available, for straight quotes in verbatim environments
\IfFileExists{upquote.sty}{\usepackage{upquote}}{}
\ifnum 0\ifxetex 1\fi\ifluatex 1\fi=0 % if pdftex
  \usepackage[utf8]{inputenc}
\else % if luatex or xelatex
  \ifxetex
    \usepackage{mathspec}
    \usepackage{xltxtra,xunicode}
  \else
    \usepackage{fontspec}
  \fi
  \defaultfontfeatures{Mapping=tex-text,Scale=MatchLowercase}
  \newcommand{\euro}{€}
\fi
% use microtype if available
\IfFileExists{microtype.sty}{\usepackage{microtype}}{}
\ifxetex
  \usepackage[setpagesize=false, % page size defined by xetex
              unicode=false, % unicode breaks when used with xetex
              xetex]{hyperref}
\else
  \usepackage[unicode=true]{hyperref}
\fi
\hypersetup{breaklinks=true,
            bookmarks=true,
            pdfauthor={},
            pdftitle={Complements : le theor`eme de Baire et ses consequences},
            colorlinks=true,
            citecolor=blue,
            urlcolor=blue,
            linkcolor=magenta,
            pdfborder={0 0 0}}
\urlstyle{same}  % don't use monospace font for urls
\setlength{\parindent}{0pt}
\setlength{\parskip}{6pt plus 2pt minus 1pt}
\setlength{\emergencystretch}{3em}  % prevent overfull lines
\setcounter{secnumdepth}{0}
 
/* start css.sty */
.cmr-5{font-size:50%;}
.cmr-7{font-size:70%;}
.cmmi-5{font-size:50%;font-style: italic;}
.cmmi-7{font-size:70%;font-style: italic;}
.cmmi-10{font-style: italic;}
.cmsy-5{font-size:50%;}
.cmsy-7{font-size:70%;}
.cmex-7{font-size:70%;}
.cmex-7x-x-71{font-size:49%;}
.msbm-7{font-size:70%;}
.cmtt-10{font-family: monospace;}
.cmti-10{ font-style: italic;}
.cmbx-10{ font-weight: bold;}
.cmr-17x-x-120{font-size:204%;}
.cmsl-10{font-style: oblique;}
.cmti-7x-x-71{font-size:49%; font-style: italic;}
.cmbxti-10{ font-weight: bold; font-style: italic;}
p.noindent { text-indent: 0em }
td p.noindent { text-indent: 0em; margin-top:0em; }
p.nopar { text-indent: 0em; }
p.indent{ text-indent: 1.5em }
@media print {div.crosslinks {visibility:hidden;}}
a img { border-top: 0; border-left: 0; border-right: 0; }
center { margin-top:1em; margin-bottom:1em; }
td center { margin-top:0em; margin-bottom:0em; }
.Canvas { position:relative; }
li p.indent { text-indent: 0em }
.enumerate1 {list-style-type:decimal;}
.enumerate2 {list-style-type:lower-alpha;}
.enumerate3 {list-style-type:lower-roman;}
.enumerate4 {list-style-type:upper-alpha;}
div.newtheorem { margin-bottom: 2em; margin-top: 2em;}
.obeylines-h,.obeylines-v {white-space: nowrap; }
div.obeylines-v p { margin-top:0; margin-bottom:0; }
.overline{ text-decoration:overline; }
.overline img{ border-top: 1px solid black; }
td.displaylines {text-align:center; white-space:nowrap;}
.centerline {text-align:center;}
.rightline {text-align:right;}
div.verbatim {font-family: monospace; white-space: nowrap; text-align:left; clear:both; }
.fbox {padding-left:3.0pt; padding-right:3.0pt; text-indent:0pt; border:solid black 0.4pt; }
div.fbox {display:table}
div.center div.fbox {text-align:center; clear:both; padding-left:3.0pt; padding-right:3.0pt; text-indent:0pt; border:solid black 0.4pt; }
div.minipage{width:100%;}
div.center, div.center div.center {text-align: center; margin-left:1em; margin-right:1em;}
div.center div {text-align: left;}
div.flushright, div.flushright div.flushright {text-align: right;}
div.flushright div {text-align: left;}
div.flushleft {text-align: left;}
.underline{ text-decoration:underline; }
.underline img{ border-bottom: 1px solid black; margin-bottom:1pt; }
.framebox-c, .framebox-l, .framebox-r { padding-left:3.0pt; padding-right:3.0pt; text-indent:0pt; border:solid black 0.4pt; }
.framebox-c {text-align:center;}
.framebox-l {text-align:left;}
.framebox-r {text-align:right;}
span.thank-mark{ vertical-align: super }
span.footnote-mark sup.textsuperscript, span.footnote-mark a sup.textsuperscript{ font-size:80%; }
div.tabular, div.center div.tabular {text-align: center; margin-top:0.5em; margin-bottom:0.5em; }
table.tabular td p{margin-top:0em;}
table.tabular {margin-left: auto; margin-right: auto;}
div.td00{ margin-left:0pt; margin-right:0pt; }
div.td01{ margin-left:0pt; margin-right:5pt; }
div.td10{ margin-left:5pt; margin-right:0pt; }
div.td11{ margin-left:5pt; margin-right:5pt; }
table[rules] {border-left:solid black 0.4pt; border-right:solid black 0.4pt; }
td.td00{ padding-left:0pt; padding-right:0pt; }
td.td01{ padding-left:0pt; padding-right:5pt; }
td.td10{ padding-left:5pt; padding-right:0pt; }
td.td11{ padding-left:5pt; padding-right:5pt; }
table[rules] {border-left:solid black 0.4pt; border-right:solid black 0.4pt; }
.hline hr, .cline hr{ height : 1px; margin:0px; }
.tabbing-right {text-align:right;}
span.TEX {letter-spacing: -0.125em; }
span.TEX span.E{ position:relative;top:0.5ex;left:-0.0417em;}
a span.TEX span.E {text-decoration: none; }
span.LATEX span.A{ position:relative; top:-0.5ex; left:-0.4em; font-size:85%;}
span.LATEX span.TEX{ position:relative; left: -0.4em; }
div.float img, div.float .caption {text-align:center;}
div.figure img, div.figure .caption {text-align:center;}
.marginpar {width:20%; float:right; text-align:left; margin-left:auto; margin-top:0.5em; font-size:85%; text-decoration:underline;}
.marginpar p{margin-top:0.4em; margin-bottom:0.4em;}
.equation td{text-align:center; vertical-align:middle; }
td.eq-no{ width:5%; }
table.equation { width:100%; } 
div.math-display, div.par-math-display{text-align:center;}
math .texttt { font-family: monospace; }
math .textit { font-style: italic; }
math .textsl { font-style: oblique; }
math .textsf { font-family: sans-serif; }
math .textbf { font-weight: bold; }
.partToc a, .partToc, .likepartToc a, .likepartToc {line-height: 200%; font-weight:bold; font-size:110%;}
.chapterToc a, .chapterToc, .likechapterToc a, .likechapterToc, .appendixToc a, .appendixToc {line-height: 200%; font-weight:bold;}
.index-item, .index-subitem, .index-subsubitem {display:block}
.caption td.id{font-weight: bold; white-space: nowrap; }
table.caption {text-align:center;}
h1.partHead{text-align: center}
p.bibitem { text-indent: -2em; margin-left: 2em; margin-top:0.6em; margin-bottom:0.6em; }
p.bibitem-p { text-indent: 0em; margin-left: 2em; margin-top:0.6em; margin-bottom:0.6em; }
.paragraphHead, .likeparagraphHead { margin-top:2em; font-weight: bold;}
.subparagraphHead, .likesubparagraphHead { font-weight: bold;}
.quote {margin-bottom:0.25em; margin-top:0.25em; margin-left:1em; margin-right:1em; text-align:justify;}
.verse{white-space:nowrap; margin-left:2em}
div.maketitle {text-align:center;}
h2.titleHead{text-align:center;}
div.maketitle{ margin-bottom: 2em; }
div.author, div.date {text-align:center;}
div.thanks{text-align:left; margin-left:10%; font-size:85%; font-style:italic; }
div.author{white-space: nowrap;}
.quotation {margin-bottom:0.25em; margin-top:0.25em; margin-left:1em; }
h1.partHead{text-align: center}
.sectionToc, .likesectionToc {margin-left:2em;}
.subsectionToc, .likesubsectionToc {margin-left:4em;}
.subsubsectionToc, .likesubsubsectionToc {margin-left:6em;}
.frenchb-nbsp{font-size:75%;}
.frenchb-thinspace{font-size:75%;}
.figure img.graphics {margin-left:10%;}
/* end css.sty */

\title{Complements : le theor`eme de Baire et ses consequences}
\author{}
\date{}

\begin{document}
\maketitle

\textbf{Warning: \href{http://www.math.union.edu/locate/jsMath}{jsMath}
requires JavaScript to process the mathematics on this page.\\ If your
browser supports JavaScript, be sure it is enabled.}

\begin{center}\rule{3in}{0.4pt}\end{center}

{[}\href{coursse31.html}{next}{]} {[}\href{coursse29.html}{prev}{]}
{[}\href{coursse29.html\#tailcoursse29.html}{prev-tail}{]}
{[}\hyperref[tailcoursse30.html]{tail}{]}
{[}\href{coursch6.html\#coursse30.html}{up}{]}

\subsubsection{5.4 Compléments~: le théorème de Baire et ses
conséquences}

\paragraph{5.4.1 Le théorème de Baire}

Théorème~5.4.1 (Baire). Soit E un espace métrique complet et
\{(\{U\}\_\{n\})\}\_\{n∈ℕ\} une suite d'ouverts denses dans E. Alors
\{\textbackslash{}mathop\{\textbackslash{}mathop\{⋂ \}\}
\}\_\{n∈ℕ\}\{U\}\_\{n\} est encore dense dans E.

Démonstration Rappelons qu'une partie est dense si et seulement si elle
rencontre tout ouvert non vide. Soit donc U un tel ouvert de E, soit
\{x\}\_\{0\} ∈ U ∩ \{U\}\_\{0\} (qui est non vide par densité de
\{U\}\_\{0\} et ouvert comme intersection de deux ouverts). Soit
\{r\}\_\{0\} \textgreater{} 0 tel que B'(\{x\}\_\{0\},\{r\}\_\{0\}) ⊂ U
∩ \{U\}\_\{0\}. Supposons \{x\}\_\{n\} et \{r\}\_\{n\} construits et
voyons comment nous allons construire \{x\}\_\{n+1\} et \{r\}\_\{n+1\}.
Comme \{U\}\_\{n+1\} est dense et B(\{x\}\_\{n\},\{r\}\_\{n\}) est un
ouvert, \{U\}\_\{n+1\} ∩ B(\{x\}\_\{n\},\{r\}\_\{n\}) est ouvert et non
vide~; soit donc \{x\}\_\{n+1\} ∈ \{U\}\_\{n+1\} ∩
B(\{x\}\_\{n\},\{r\}\_\{n\}) et \{r\}\_\{n+1\} \textless{}\{
\{r\}\_\{n\} \textbackslash{}over 2\} tel que
B'(\{x\}\_\{n+1\},\{r\}\_\{n+1\}) ⊂ \{U\}\_\{n+1\} ∩
B(\{x\}\_\{n\},\{r\}\_\{n\}) . On construit ainsi une suite de boules
fermées B'(\{x\}\_\{n\},\{r\}\_\{n\}) telles que
B'(\{x\}\_\{n+1\},\{r\}\_\{n+1\}) ⊂ B'(\{x\}\_\{n\},\{r\}\_\{n\}) avec
\{r\}\_\{n\} \textless{}\{ \{r\}\_\{0\} \textbackslash{}over
\{2\}\^{}\{n\}\} . Le théorème des fermés emboîtés nous garantit que
\{\textbackslash{}mathop\{\textbackslash{}mathop\{⋂ \}\}
\}\_\{n∈ℕ\}B'(\{x\}\_\{n\},\{r\}\_\{n\})\textbackslash{}mathrel\{≠\}∅
(car δ(B'(\{x\}\_\{n\},\{r\}\_\{n\})) \textless{} 2\{r\}\_\{n\} tend
vers 0). Mais on a B'(\{x\}\_\{0\},\{r\}\_\{0\}) ⊂ U ∩ \{U\}\_\{0\} et
pour n ≥ 1, B'(\{x\}\_\{n\},\{r\}\_\{n\}) ⊂ \{U\}\_\{n\}. On en déduit
que U ∩\{\textbackslash{}mathop\{\textbackslash{}mathop\{⋂ \}\}
\}\_\{n∈ℕ\}\{U\}\_\{n\}\textbackslash{}mathrel\{≠\}∅, ce qui achève la
démonstration.

En passant au complémentaire, on obtient une version équivalente

Théorème~5.4.2 (Baire). Soit E un espace métrique complet et
\{(\{F\}\_\{n\})\}\_\{n∈ℕ\} une suite de fermés d'intérieurs vides de E.
Alors \{\textbackslash{}mathop\{\textbackslash{}mathop\{⋃ \}\}
\}\_\{n∈ℕ\}\{F\}\_\{n\} est encore d'intérieur vide dans E.

Exemple~5.4.1 On montre facilement qu'un sous-espace vectoriel de E
distinct de E est d'intérieur vide (exercice). On en déduit que, si E,
espace vectoriel normé~de dimension infinie, est complet, E (qui n'est
pas d'intérieur vide) ne peut pas être réunion dénombrable de
sous-espaces vectoriels de dimension finie (dont on sait qu'ils sont
fermés). En particulier E ne peut pas admettre de base dénombrable.
C'est ainsi que ℝ{[}X{]} (qui admet une base dénombrable) n'est complet
pour aucune norme.

\paragraph{5.4.2 Les grands théorèmes}

Nous en citerons trois qui concernent tous des applications linéaires
dans des espaces vectoriels normés complets.

Théorème~5.4.3 (Banach-Steinhaus). Soit E un espace vectoriel
normé~complet et F un espace vectoriel normé. Soit H un ensemble
d'applications linéaires continues telles que

\textbackslash{}mathop\{∀\}x ∈ E,
\textbackslash{}mathop\{∃\}\{K\}\_\{x\} ≥ 0,
\textbackslash{}mathop\{∀\}u ∈ H,\textbackslash{}quad
\textbackslash{}\textbar{}u(x)\textbackslash{}\textbar{} ≤ \{K\}\_\{x\}

Alors il existe K ≥ 0 tel que \textbackslash{}mathop\{∀\}u ∈ H,
\textbackslash{}\textbar{}u\textbackslash{}\textbar{} ≤ K.

Démonstration Posons pour x ∈ E, p(x) =\{\textbackslash{}mathop\{
sup\}\}\_\{u∈H\}\textbackslash{}\textbar{}u(x)\textbackslash{}\textbar{}(≤
\{K\}\_\{x\}) et considérons \{E\}\_\{n\} = \textbackslash{}\{x ∈
E\textbackslash{}mathrel\{∣\}p(x) ≤ n\textbackslash{}\}. Remarquons tout
d'abord que \{E\}\_\{n\} est fermé~: en effet si (\{x\}\_\{q\}) est une
suite d'éléments de \{E\}\_\{n\} qui converge vers x ∈ E, on a pour tout
u dans H, \textbackslash{}mathop\{∀\}q ∈ ℕ,
\textbackslash{}\textbar{}u(\{x\}\_\{q\})\textbackslash{}\textbar{} ≤
n~; en faisant tendre q vers + ∞ et en utilisant la continuité de u, on
a encore \textbackslash{}\textbar{}u(x)\textbackslash{}\textbar{} ≤ n et
donc x ∈ \{E\}\_\{n\}. Maintenant notre hypothèse implique que chaque x
de E appartient à l'un des \{E\}\_\{n\} (par exemple pour n =
E(\{K\}\_\{x\}) + 1). Donc E qui est d'intérieur évidemment non vide est
réunion d'une famille de fermés. Le théorème de Baire implique que l'un
des \{E\}\_\{n\} est d'intérieur non vide~: soit donc N ∈ ℕ,
\{x\}\_\{0\} ∈ E et r \textgreater{} 0 tel que B'(\{x\}\_\{0\},r) ⊂
\{E\}\_\{N\}. Prenons alors x ∈ B'(0,1) et u ∈ H. Alors \{x\}\_\{0\} +
rx ∈ B'(\{x\}\_\{0\},r) et donc \textbackslash{}\textbar{}u(\{x\}\_\{0\}
+ rx)\textbackslash{}\textbar{} ≤ N. Mais alors
\textbackslash{}\textbar{}u(x)\textbackslash{}\textbar{} =\{ 1
\textbackslash{}over r\} \textbackslash{}\textbar{}u(\{x\}\_\{0\} + rx)
− u(\{x\}\_\{0\})\textbackslash{}\textbar{} ≤\{ 1 \textbackslash{}over
r\} (N +\textbackslash{}\textbar{}
u(\{x\}\_\{0\}\textbackslash{}\textbar{}) = K. On a donc
\textbackslash{}mathop\{∀\}u ∈ H,
\textbackslash{}\textbar{}u\textbackslash{}\textbar{} ≤ K.

Remarque~5.4.1 Sous les mêmes hypothèses, on montre alors facilement
qu'une limite simple d'applications linéaires continues est encore
continue (attention à l'hypothèse E complet)~; en effet le théorème de
Banach Steinhaus implique que la suite est équicontinue (le module de
continuité en \{x\}\_\{0\}, η(ε,\{x\}\_\{0\}), ne dépend pas de n) et on
montre simplement qu'une limite simple d'une suite équicontinue est
continue.

Théorème~5.4.4 (théorème de Banach). Soit E et F deux espaces vectoriels
normés complets, et u : E → F linéaire, continue, bijective. Alors
\{u\}\^{}\{−1\} est encore continue.

Démonstration On va montrer que u(B'(0,1)) ⊂ F contient une boule de
centre 0 dans F, B'(0,\{r\}\_\{1\}). On aura alors B'(0,\{r\}\_\{1\}) ⊂
u(B'(0,1)), soit \{u\}\^{}\{−1\}(B'(0,\{r\}\_\{1\})) ⊂ B'(0,1) et donc
si y ∈ F avec \textbackslash{}\textbar{}y\textbackslash{}\textbar{} ≤ 1,
on aura \{u\}\^{}\{−1\}(\{ \{r\}\_\{1\} \textbackslash{}over 2\} y) ∈
B'(0,1) soit encore
\textbackslash{}\textbar{}\{u\}\^{}\{−1\}(y)\textbackslash{}\textbar{}
≤\{ 2 \textbackslash{}over \{r\}\_\{1\}\} ce qui montrera que
\{u\}\^{}\{−1\} est continue.

Soit r \textgreater{} 0. On a E =\{\textbackslash{}mathop\{
\textbackslash{}mathop\{⋃ \}\} \}\_\{n∈ℕ\}nB'(0,r), on en déduit que F =
u(E) =\{\textbackslash{}mathop\{ \textbackslash{}mathop\{⋃ \}\}
\}\_\{n∈ℕ\}nu(B'(0,r)) et a fortiori F =\{\textbackslash{}mathop\{
\textbackslash{}mathop\{⋃ \}\}
\}\_\{n∈ℕ\}n\textbackslash{}overline\{u(B'(0,r))\}. L'espace vectoriel
normé complet F qui est son propre intérieur est réunion d'une famille
dénombrable de fermés~; donc l'un d'entre eux (Baire) est d'intérieur
non vide. Mais si n\textbackslash{}overline\{u(B'(0,r))\} est
d'intérieur non vide, il en est de même de
\textbackslash{}overline\{u(B'(0,r))\}. Soit donc \{y\}\_\{0\} ∈ F et ρ
\textgreater{} 0 tel que B'(\{y\}\_\{0\},ρ)
⊂\textbackslash{}overline\{u(B'(0,r))\}. On a aussi (puisque
l'application x\textbackslash{}mathrel\{↦\} − x laisse invariante
B'(0,r)), B'(−\{y\}\_\{0\},ρ) ⊂\textbackslash{}overline\{u(B'(0,r))\},
et alors, si y ∈ B'(0,ρ),

2y = (y − \{y\}\_\{0\}) + (y + \{y\}\_\{0\}) ∈ B'(−\{y\}\_\{0\},ρ) +
B(\{y\}\_\{0\},\{ρ\}\_\{0\})

or

B'(−\{y\}\_\{0\},ρ) + B(\{y\}\_\{0\},\{ρ\}\_\{0\})
⊂\textbackslash{}overline\{u(B'(0,r))\} +
\textbackslash{}overline\{u(B'(0,r))\}
⊂\textbackslash{}overline\{u(B'(0,2r))\}

(facile) et donc y ∈\textbackslash{}overline\{u(B'(0,r))\}. On a donc
trouvé, pour tout r \textgreater{} 0 un ρ \textgreater{} 0 tel que
B'(0,ρ) ⊂\textbackslash{}overline\{u(B'(0,r))\}. Les translations étant
des homéomorphismes, on a évidemment pour tout x ∈ E, B'(u(x),ρ)
⊂\textbackslash{}overline\{u(B'(x,r))\}.

Montrons alors que sous ces hypothèses B'(0,ρ) ⊂ u(B'(0,2r)). Soit en
effet y ∈ B'(0,ρ). Soit \{ρ\}\_\{n\} le réel associé à \{ r
\textbackslash{}over \{2\}\^{}\{n\}\} par la propriété ci dessus. Quitte
à remplacer les \{ρ\}\_\{n\} par des réels plus petits, on peut supposer
que \{ρ\}\_\{n\} tend vers 0. On va construire un élément \{x\}\_\{n\}
de E par récurrence de manière à vérifier
\textbackslash{}\textbar{}\{x\}\_\{n+1\} −
\{x\}\_\{n\}\textbackslash{}\textbar{} ≤\{ r \textbackslash{}over
\{2\}\^{}\{n\}\} et \textbackslash{}\textbar{}y −
u(\{x\}\_\{n\})\textbackslash{}\textbar{} ≤ \{ρ\}\_\{n\}. On pose
\{x\}\_\{0\} = 0~; supposons \{x\}\_\{n\} construit. On a donc y ∈
B'(u(\{x\}\_\{n\}),\{ρ\}\_\{n\})
⊂\textbackslash{}overline\{u(B'(\{x\}\_\{n\},\{ r \textbackslash{}over
\{2\}\^{}\{n\}\} ))\} et donc on peut trouver un point \{x\}\_\{n+1\} ∈
B'(\{x\}\_\{n\},\{ r \textbackslash{}over \{2\}\^{}\{n\}\} ) tel que
\textbackslash{}\textbar{}y −
u(\{x\}\_\{n+1\})\textbackslash{}\textbar{} ≤ \{ρ\}\_\{n+1\}, soit y ∈
B'(u(\{x\}\_\{n+1\}),\{ρ\}\_\{n+1\}), ce qui achève la construction par
récurrence. On a donc pour tout n,
\textbackslash{}\textbar{}\{x\}\_\{n+1\} −
\{x\}\_\{n\}\textbackslash{}\textbar{} ≤\{ r \textbackslash{}over
\{2\}\^{}\{n\}\} et \textbackslash{}\textbar{}y −
u(\{x\}\_\{n\})\textbackslash{}\textbar{} ≤ \{ρ\}\_\{n\}. On a
\textbackslash{}\textbar{}\{x\}\_\{n+p\} −
\{x\}\_\{n\}\textbackslash{}\textbar{} ≤\{ r \textbackslash{}over
\{2\}\^{}\{n\}\} +\{ r \textbackslash{}over \{2\}\^{}\{n+1\}\} +
\textbackslash{}mathop\{\textbackslash{}mathop\{\ldots{}\}\} +\{ r
\textbackslash{}over \{2\}\^{}\{n+p−1\}\} ≤\{ r \textbackslash{}over
\{2\}\^{}\{n−1\}\} , ce qui montre que la suite (\{x\}\_\{n\}) est une
suite de Cauchy. Comme E est complet, elle converge. Soit x sa limite.
On a \textbackslash{}\textbar{}x −
\{x\}\_\{0\}\textbackslash{}\textbar{} ≤ 2r d'après l'inégalité ci
dessus pour n = 0 et p tendant vers + ∞. D'autre part l'inégalité
\textbackslash{}\textbar{}y − u(\{x\}\_\{n\})\textbackslash{}\textbar{}
≤ \{ρ\}\_\{n\} et la continuité de u nous montrent que y = u(x), donc y
appartient à u(B'(0,2r)).

On a alors aussi B'(0,\{ ρ \textbackslash{}over 2r\} ) ⊂ u(B'(0,1)), ce
qui montre comme on l'a remarqué, que \{u\}\^{}\{−1\} est continue.

Théorème~5.4.5 (théorème du graphe fermé). Soit E et F deux espaces
vectoriels normés complets, et u : E → F linéaire. Alors u est continue
si et seulement si~son graphe est fermé dans E × F.

Démonstration Supposons tout d'abord que u est continue et soit
(\{x\}\_\{n\},u(\{x\}\_\{n\})) une suite du graphe qui converge vers
(x,y) ∈ E × F. Alors \textbackslash{}mathop\{lim\}\{x\}\_\{n\} = x et
par continuité de u, \textbackslash{}mathop\{lim\}u(\{x\}\_\{n\}) =
u(x)~; mais alors l'unicité de la limite nécessite y = u(x), donc (x,y)
est encore dans le graphe de u, ce qui montre bien que le graphe est
fermé (il s'agit là d'une propriété tout à fait générale des espaces
métriques, mais la réciproque est fausse en général). Supposons
maintenant que u est linéaire de graphe Γ fermé. Alors Γ est un
sous-espace vectoriel fermé de E × F, donc il est complet. L'application
Γ → E, (x,u(x))\textbackslash{}mathrel\{↦\}x est linéaire continue et
bijective. D'après le théorème de Banach, sa réciproque
x\textbackslash{}mathrel\{↦\}(x,u(x)) est continue et donc
x\textbackslash{}mathrel\{↦\}u(x) aussi.

Remarque~5.4.2 Il s'agit d'une technique importante~; il est en effet
considérablement plus facile de montrer qu'un graphe est fermé plutôt
qu'une continuité~; si (\{x\}\_\{n\}) est une suite de limite x, il
s'agit de montrer non plus que la suite u(\{x\}\_\{n\}) converge vers
u(x) mais plutôt que la suite u(\{x\}\_\{n\}) ne peut pas avoir d'autre
limite que u(x)~; un exemple typique d'application linéaire de graphe
fermé est la dérivation pour la topologie de la convergence uniforme~:
le théorème de dérivation des suites uniformément convergentes ne fait
que traduire la fermeture du graphe (si la suite des dérivées converge
uniformément, alors c'est vers la dérivée de la limite)~; attention
cependant que la dérivation n'est pas continue pour la topologie de la
convergence uniforme (le théorème du graphe fermé ne s'applique pas car
l'espace des applications \{C\}\^{}\{1\} n'est pas complet).

{[}\href{coursse31.html}{next}{]} {[}\href{coursse29.html}{prev}{]}
{[}\href{coursse29.html\#tailcoursse29.html}{prev-tail}{]}
{[}\href{coursse30.html}{front}{]}
{[}\href{coursch6.html\#coursse30.html}{up}{]}

\end{document}

% \documentclass[]{article}
\usepackage[T1]{fontenc}
\usepackage{lmodern}
\usepackage{amssymb,amsmath}
\usepackage{ifxetex,ifluatex}
\usepackage{fixltx2e} % provides \textsubscript
% use upquote if available, for straight quotes in verbatim environments
\IfFileExists{upquote.sty}{\usepackage{upquote}}{}
\ifnum 0\ifxetex 1\fi\ifluatex 1\fi=0 % if pdftex
  \usepackage[utf8]{inputenc}
\else % if luatex or xelatex
  \ifxetex
    \usepackage{mathspec}
    \usepackage{xltxtra,xunicode}
  \else
    \usepackage{fontspec}
  \fi
  \defaultfontfeatures{Mapping=tex-text,Scale=MatchLowercase}
  \newcommand{\euro}{€}
\fi
% use microtype if available
\IfFileExists{microtype.sty}{\usepackage{microtype}}{}
\usepackage{graphicx}
% Redefine \includegraphics so that, unless explicit options are
% given, the image width will not exceed the width of the page.
% Images get their normal width if they fit onto the page, but
% are scaled down if they would overflow the margins.
\makeatletter
\def\ScaleIfNeeded{%
  \ifdim\Gin@nat@width>\linewidth
    \linewidth
  \else
    \Gin@nat@width
  \fi
}
\makeatother
\let\Oldincludegraphics\includegraphics
{%
 \catcode`\@=11\relax%
 \gdef\includegraphics{\@ifnextchar[{\Oldincludegraphics}{\Oldincludegraphics[width=\ScaleIfNeeded]}}%
}%
\ifxetex
  \usepackage[setpagesize=false, % page size defined by xetex
              unicode=false, % unicode breaks when used with xetex
              xetex]{hyperref}
\else
  \usepackage[unicode=true]{hyperref}
\fi
\hypersetup{breaklinks=true,
            bookmarks=true,
            pdfauthor={},
            pdftitle={Complements : convexite dans les espaces vectoriels normes},
            colorlinks=true,
            citecolor=blue,
            urlcolor=blue,
            linkcolor=magenta,
            pdfborder={0 0 0}}
\urlstyle{same}  % don't use monospace font for urls
\setlength{\parindent}{0pt}
\setlength{\parskip}{6pt plus 2pt minus 1pt}
\setlength{\emergencystretch}{3em}  % prevent overfull lines
\setcounter{secnumdepth}{0}
 
/* start css.sty */
.cmr-5{font-size:50%;}
.cmr-7{font-size:70%;}
.cmmi-5{font-size:50%;font-style: italic;}
.cmmi-7{font-size:70%;font-style: italic;}
.cmmi-10{font-style: italic;}
.cmsy-5{font-size:50%;}
.cmsy-7{font-size:70%;}
.cmex-7{font-size:70%;}
.cmex-7x-x-71{font-size:49%;}
.msbm-7{font-size:70%;}
.cmtt-10{font-family: monospace;}
.cmti-10{ font-style: italic;}
.cmbx-10{ font-weight: bold;}
.cmr-17x-x-120{font-size:204%;}
.cmsl-10{font-style: oblique;}
.cmti-7x-x-71{font-size:49%; font-style: italic;}
.cmbxti-10{ font-weight: bold; font-style: italic;}
p.noindent { text-indent: 0em }
td p.noindent { text-indent: 0em; margin-top:0em; }
p.nopar { text-indent: 0em; }
p.indent{ text-indent: 1.5em }
@media print {div.crosslinks {visibility:hidden;}}
a img { border-top: 0; border-left: 0; border-right: 0; }
center { margin-top:1em; margin-bottom:1em; }
td center { margin-top:0em; margin-bottom:0em; }
.Canvas { position:relative; }
li p.indent { text-indent: 0em }
.enumerate1 {list-style-type:decimal;}
.enumerate2 {list-style-type:lower-alpha;}
.enumerate3 {list-style-type:lower-roman;}
.enumerate4 {list-style-type:upper-alpha;}
div.newtheorem { margin-bottom: 2em; margin-top: 2em;}
.obeylines-h,.obeylines-v {white-space: nowrap; }
div.obeylines-v p { margin-top:0; margin-bottom:0; }
.overline{ text-decoration:overline; }
.overline img{ border-top: 1px solid black; }
td.displaylines {text-align:center; white-space:nowrap;}
.centerline {text-align:center;}
.rightline {text-align:right;}
div.verbatim {font-family: monospace; white-space: nowrap; text-align:left; clear:both; }
.fbox {padding-left:3.0pt; padding-right:3.0pt; text-indent:0pt; border:solid black 0.4pt; }
div.fbox {display:table}
div.center div.fbox {text-align:center; clear:both; padding-left:3.0pt; padding-right:3.0pt; text-indent:0pt; border:solid black 0.4pt; }
div.minipage{width:100%;}
div.center, div.center div.center {text-align: center; margin-left:1em; margin-right:1em;}
div.center div {text-align: left;}
div.flushright, div.flushright div.flushright {text-align: right;}
div.flushright div {text-align: left;}
div.flushleft {text-align: left;}
.underline{ text-decoration:underline; }
.underline img{ border-bottom: 1px solid black; margin-bottom:1pt; }
.framebox-c, .framebox-l, .framebox-r { padding-left:3.0pt; padding-right:3.0pt; text-indent:0pt; border:solid black 0.4pt; }
.framebox-c {text-align:center;}
.framebox-l {text-align:left;}
.framebox-r {text-align:right;}
span.thank-mark{ vertical-align: super }
span.footnote-mark sup.textsuperscript, span.footnote-mark a sup.textsuperscript{ font-size:80%; }
div.tabular, div.center div.tabular {text-align: center; margin-top:0.5em; margin-bottom:0.5em; }
table.tabular td p{margin-top:0em;}
table.tabular {margin-left: auto; margin-right: auto;}
div.td00{ margin-left:0pt; margin-right:0pt; }
div.td01{ margin-left:0pt; margin-right:5pt; }
div.td10{ margin-left:5pt; margin-right:0pt; }
div.td11{ margin-left:5pt; margin-right:5pt; }
table[rules] {border-left:solid black 0.4pt; border-right:solid black 0.4pt; }
td.td00{ padding-left:0pt; padding-right:0pt; }
td.td01{ padding-left:0pt; padding-right:5pt; }
td.td10{ padding-left:5pt; padding-right:0pt; }
td.td11{ padding-left:5pt; padding-right:5pt; }
table[rules] {border-left:solid black 0.4pt; border-right:solid black 0.4pt; }
.hline hr, .cline hr{ height : 1px; margin:0px; }
.tabbing-right {text-align:right;}
span.TEX {letter-spacing: -0.125em; }
span.TEX span.E{ position:relative;top:0.5ex;left:-0.0417em;}
a span.TEX span.E {text-decoration: none; }
span.LATEX span.A{ position:relative; top:-0.5ex; left:-0.4em; font-size:85%;}
span.LATEX span.TEX{ position:relative; left: -0.4em; }
div.float img, div.float .caption {text-align:center;}
div.figure img, div.figure .caption {text-align:center;}
.marginpar {width:20%; float:right; text-align:left; margin-left:auto; margin-top:0.5em; font-size:85%; text-decoration:underline;}
.marginpar p{margin-top:0.4em; margin-bottom:0.4em;}
.equation td{text-align:center; vertical-align:middle; }
td.eq-no{ width:5%; }
table.equation { width:100%; } 
div.math-display, div.par-math-display{text-align:center;}
math .texttt { font-family: monospace; }
math .textit { font-style: italic; }
math .textsl { font-style: oblique; }
math .textsf { font-family: sans-serif; }
math .textbf { font-weight: bold; }
.partToc a, .partToc, .likepartToc a, .likepartToc {line-height: 200%; font-weight:bold; font-size:110%;}
.chapterToc a, .chapterToc, .likechapterToc a, .likechapterToc, .appendixToc a, .appendixToc {line-height: 200%; font-weight:bold;}
.index-item, .index-subitem, .index-subsubitem {display:block}
.caption td.id{font-weight: bold; white-space: nowrap; }
table.caption {text-align:center;}
h1.partHead{text-align: center}
p.bibitem { text-indent: -2em; margin-left: 2em; margin-top:0.6em; margin-bottom:0.6em; }
p.bibitem-p { text-indent: 0em; margin-left: 2em; margin-top:0.6em; margin-bottom:0.6em; }
.paragraphHead, .likeparagraphHead { margin-top:2em; font-weight: bold;}
.subparagraphHead, .likesubparagraphHead { font-weight: bold;}
.quote {margin-bottom:0.25em; margin-top:0.25em; margin-left:1em; margin-right:1em; text-align:justify;}
.verse{white-space:nowrap; margin-left:2em}
div.maketitle {text-align:center;}
h2.titleHead{text-align:center;}
div.maketitle{ margin-bottom: 2em; }
div.author, div.date {text-align:center;}
div.thanks{text-align:left; margin-left:10%; font-size:85%; font-style:italic; }
div.author{white-space: nowrap;}
.quotation {margin-bottom:0.25em; margin-top:0.25em; margin-left:1em; }
h1.partHead{text-align: center}
.sectionToc, .likesectionToc {margin-left:2em;}
.subsectionToc, .likesubsectionToc {margin-left:4em;}
.subsubsectionToc, .likesubsubsectionToc {margin-left:6em;}
.frenchb-nbsp{font-size:75%;}
.frenchb-thinspace{font-size:75%;}
.figure img.graphics {margin-left:10%;}
/* end css.sty */

\title{Complements : convexite dans les espaces vectoriels normes}
\author{}
\date{}

\begin{document}
\maketitle

\textbf{Warning: \href{http://www.math.union.edu/locate/jsMath}{jsMath}
requires JavaScript to process the mathematics on this page.\\ If your
browser supports JavaScript, be sure it is enabled.}

\begin{center}\rule{3in}{0.4pt}\end{center}

{[}\href{coursse30.html}{prev}{]}
{[}\href{coursse30.html\#tailcoursse30.html}{prev-tail}{]}
{[}\hyperref[tailcoursse31.html]{tail}{]}
{[}\href{coursch6.html\#coursse31.html}{up}{]}

\subsubsection{5.5 Compléments~: convexité dans les espaces vectoriels
normés}

\paragraph{5.5.1 Jauge d'un convexe}

Soit E un espace vectoriel normé réel et K un convexe borné qui contient
0 dans son intérieur. On définit alors une application \{j\}\_\{K\} de E
dans \{ℝ\}\^{}\{+\} par

\{j\}\_\{K\}(x) =\textbackslash{}mathop\{ inf\} \textbackslash{}\{λ
\textgreater{} 0\textbackslash{}mathrel\{∣\}\{ x \textbackslash{}over
λ\} ∈ K\textbackslash{}\}

Cette définition a bien un sens, car si B(0,r) ⊂ K, on a \{ x
\textbackslash{}over λ\} ∈ B(0,r) ∈ K dès que λ \textgreater{}\{
\textbackslash{}\textbar{}x\textbackslash{}\textbar{}
\textbackslash{}over r\} .

Définition~5.5.1 La fonction \{j\}\_\{K\} est appelée la jauge du
convexe K.

Proposition~5.5.1 Soit E un espace vectoriel normé réel et K un convexe
borné qui contient 0 dans son intérieur. Alors l'application
\{j\}\_\{K\} vérifie

\begin{itemize}
\itemsep1pt\parskip0pt\parsep0pt
\item
  (i) \{j\}\_\{K\}(x) = 0 \textbackslash{}mathrel\{⇔\} x = 0
\item
  (ii) \{j\}\_\{K\}(μx) = μ\{j\}\_\{K\}(x) si μ ≥ 0
\item
  (iii) \{j\}\_\{K\}(x + y) ≤ \{j\}\_\{K\}(x) + \{j\}\_\{K\}(y)
\end{itemize}

Si de plus K = −K, alors \{j\}\_\{K\} est une norme.

Démonstration (i) Puisque K est borné, soit M ≥ 0 tel que
\textbackslash{}mathop\{∀\}y ∈ K,
\textbackslash{}\textbar{}y\textbackslash{}\textbar{} ≤ M. Si
\{j\}\_\{K\}(x) = 0, il existe une suite \{λ\}\_\{n\} tendant vers 0
telle que \{ x \textbackslash{}over \{λ\}\_\{n\}\} ∈ K, soit
\textbackslash{}\textbar{}x\textbackslash{}\textbar{} ≤ M\{λ\}\_\{n\}.
On a donc x = 0. La réciproque est évidente.

(ii) est évident puisque \{ x \textbackslash{}over λ\} ∈ K
\textbackslash{}mathrel\{⇔\}\{ μx \textbackslash{}over μλ\} ∈ K

(iii) Supposons que \{ x \textbackslash{}over λ\} et \{ y
\textbackslash{}over μ\} appartiennent à K. Comme K est convexe, λ et μ
positifs, on a aussi \{ 1 \textbackslash{}over λ+μ\} (λ\{ x
\textbackslash{}over λ\} + μ\{ y \textbackslash{}over μ\} ) ∈ K soit
encore \{ x+y \textbackslash{}over λ+μ\} ∈ K. On a donc

\textbackslash{}\{λ \textgreater{} 0\textbackslash{}mathrel\{∣\}\{ x
\textbackslash{}over λ\} ∈ K\textbackslash{}\} + \textbackslash{}\{μ
\textgreater{} 0\textbackslash{}mathrel\{∣\}\{ y \textbackslash{}over
μ\} ∈ K\textbackslash{}\} ⊂\textbackslash{}\{ν \textgreater{}
0\textbackslash{}mathrel\{∣\}\{ x + y \textbackslash{}over ν\} ∈
K\textbackslash{}\}

En prenant les bornes inférieures on a donc \{j\}\_\{K\}(x + y) ≤
\{j\}\_\{K\}(x) + \{j\}\_\{K\}(y).

Si de plus, K = −K, on a \{j\}\_\{K\}(−x) = \{j\}\_\{K\}(x) et donc
\textbackslash{}mathop\{∀\}μ ∈ ℝ, \{j\}\_\{K\}(μx) =
\textbar{}μ\textbar{}\{j\}\_\{K\}(x) qui était la seule propriété des
normes qui manquait.

Remarque~5.5.1 On a évidemment, x ∈ K ⇒ \{j\}\_\{K\}(x) ≤ 1 et
\{j\}\_\{K\}(x) \textless{} 1 ⇒ x ∈ K, autrement dit
\{B\}\_\{\{j\}\_\{K\}\}(0,1) ⊂ K ⊂ \{B'\}\_\{\{j\}\_\{K\}\}(0,1)~; si on
suppose de plus que K est fermé, on a facilement K =
\{B'\}\_\{\{j\}\_\{K\}\}(0,1)~; autrement dit un convexe, fermé, borné
et équilibré (K = −K) est une boule fermée pour une certaine norme~; la
réciproque étant évidente.

\paragraph{5.5.2 Projection sur un convexe fermé}

Théorème~5.5.2 Soit E un espace euclidien et K une partie non vide,
convexe fermée de E~; pour tout x de E, il existe un unique élément
\{p\}\_\{K\}(x) de K tel que d(x,\{p\}\_\{K\}(x)) = d(x,K). Pour y ∈ K,
on a

y = \{p\}\_\{K\}(x) \textbackslash{}mathrel\{⇔\}
\textbackslash{}mathop\{∀\}z ∈ K,\textbackslash{}quad (x −
y\textbackslash{}mathrel\{∣\}z − y) ≤ 0

Démonstration Nous allons donner une démonstration de ce résultat qui ne
fera pas appel à la dimension finie de E, mais uniquement au fait qu'il
est complet. Soit (\{y\}\_\{n\}) une suite de K qui vérifie
\textbackslash{}\textbar{}x −
\{y\{\}\_\{n\}\textbackslash{}\textbar{}\}\^{}\{2\} ≤
d\{(x,K)\}\^{}\{2\} +\{ 1 \textbackslash{}over n\} . L'égalité de la
médiane nous donne alors

\textbackslash{}begin\{eqnarray*\}
\textbackslash{}\textbar{}\{y\}\_\{p\} −
\{y\{\}\_\{q\}\textbackslash{}\textbar{}\}\^{}\{2\}\&\& \%\&
\textbackslash{}\textbackslash{} \& =\&
\textbackslash{}\textbar{}(\{y\}\_\{p\} − x) − \{(\{y\}\_\{q\} −
x)\textbackslash{}\textbar{}\}\^{}\{2\} \%\&
\textbackslash{}\textbackslash{} \& =\&
2\textbackslash{}\textbar{}\{y\}\_\{p\} −
\{x\textbackslash{}\textbar{}\}\^{}\{2\} +
2\textbackslash{}\textbar{}\{y\}\_\{ q\} −
\{x\textbackslash{}\textbar{}\}\^{}\{2\} −\textbackslash{}\textbar{}
(\{y\}\_\{ p\} − x) + \{(\{y\}\_\{q\} −
x)\textbackslash{}\textbar{}\}\^{}\{2\}\%\&
\textbackslash{}\textbackslash{} \& =\&
2\textbackslash{}\textbar{}\{y\}\_\{p\} −
\{x\textbackslash{}\textbar{}\}\^{}\{2\} +
2\textbackslash{}\textbar{}\{y\}\_\{ q\} −
\{x\textbackslash{}\textbar{}\}\^{}\{2\} − 4\textbackslash{}\textbar{}\{
\{y\}\_\{p\} + \{y\}\_\{q\} \textbackslash{}over 2\} −
x\{)\textbackslash{}\textbar{}\}\^{}\{2\} \%\&
\textbackslash{}\textbackslash{} \textbackslash{}end\{eqnarray*\}

avec \textbackslash{}\textbar{}x −
\{y\{\}\_\{p\}\textbackslash{}\textbar{}\}\^{}\{2\} ≤
d\{(x,K)\}\^{}\{2\} +\{ 1 \textbackslash{}over p\} et
\textbackslash{}\textbar{}x −
\{y\{\}\_\{q\}\textbackslash{}\textbar{}\}\^{}\{2\} ≤
d\{(x,K)\}\^{}\{2\} +\{ 1 \textbackslash{}over q\} . Mais comme K est
convexe, \{ \{y\}\_\{p\}+\{y\}\_\{q\} \textbackslash{}over 2\} ∈ K et
donc \textbackslash{}\textbar{}\{ \{y\}\_\{p\}+\{y\}\_\{q\}
\textbackslash{}over 2\} − x\{)\textbackslash{}\textbar{}\}\^{}\{2\} ≥
d\{(x,K)\}\^{}\{2\}. On a donc \textbackslash{}\textbar{}\{y\}\_\{p\} −
\{y\{\}\_\{q\}\textbackslash{}\textbar{}\}\^{}\{2\} ≤ 2(\{ 1
\textbackslash{}over p\} +\{ 1 \textbackslash{}over q\} ). La suite
(\{y\}\_\{n\}) est une suite de Cauchy dans E, donc elle converge. Soit
y sa limite dans E. Comme K est fermé, on a y ∈ K et on a évidemment en
passant à la limite à partir de d\{(x,K)\}\^{}\{2\}
≤\textbackslash{}\textbar{} x −
\{y\{\}\_\{n\}\textbackslash{}\textbar{}\}\^{}\{2\} ≤
d\{(x,K)\}\^{}\{2\} +\{ 1 \textbackslash{}over n\} , l'égalité d(x,K) =
d(x,y).

Soit y ainsi trouvé et soit z ∈ K. Pour tout t ∈ {[}0,1{]}, (1 − t)y +
tz ∈ K et donc \textbackslash{}\textbar{}x − (1 − t)y −
t\{z\textbackslash{}\textbar{}\}\^{}\{2\} ≥\textbackslash{}\textbar{} x
− \{y\textbackslash{}\textbar{}\}\^{}\{2\}. En développant, on obtient
\{t\}\^{}\{2\}\textbackslash{}\textbar{}y −
\{z\textbackslash{}\textbar{}\}\^{}\{2\} − 2t(x −
y\textbackslash{}mathrel\{∣\}z − y) ≥ 0. Pour t ∈{]}0,1{]} on a donc
t\textbackslash{}\textbar{}y − \{z\textbackslash{}\textbar{}\}\^{}\{2\}
− 2(x − y\textbackslash{}mathrel\{∣\}z − y) ≥ 0 et en faisant tendre t
vers 0, on obtient (x − y\textbackslash{}mathrel\{∣\}z − y) ≤ 0.

Inversement supposons que \textbackslash{}mathop\{∀\}z ∈
K,\textbackslash{}quad (x − y\textbackslash{}mathrel\{∣\}z − y) ≤ 0.
Alors

\textbackslash{}begin\{eqnarray*\} \textbackslash{}\textbar{}x −
\{z\textbackslash{}\textbar{}\}\^{}\{2\}\& =\&
\textbackslash{}\textbar{}(x − y) − \{(z −
y)\textbackslash{}\textbar{}\}\^{}\{2\} \%\&
\textbackslash{}\textbackslash{} \& =\& \textbackslash{}\textbar{}x −
\{y\textbackslash{}\textbar{}\}\^{}\{2\} +\textbackslash{}\textbar{} z −
\{y\textbackslash{}\textbar{}\}\^{}\{2\} − 2(x −
y\textbackslash{}mathrel\{∣\}z − y)\%\& \textbackslash{}\textbackslash{}
\& ≥\& \textbackslash{}\textbar{}x −
\{y\textbackslash{}\textbar{}\}\^{}\{2\} \%\&
\textbackslash{}\textbackslash{} \textbackslash{}end\{eqnarray*\}

avec égalité si et seulement si z = y. Ceci montre à la fois que d(x,y)
= d(x,K) et que y est unique.

Remarque~5.5.2 La condition (x − y\textbackslash{}mathrel\{∣\}z − y) ≤ 0
correspond géométriquement à~: l'angle
(\textbackslash{}overrightarrow\{yx\},\textbackslash{}overrightarrow\{yz\})
est obtus.

\includegraphics{cours5x.png}

\paragraph{5.5.3 Hahn-Banach (version géométrique)}

Remarque~5.5.3 Il existe plusieurs théorèmes à la Hahn Banach. Certains
sont de type analytique et concernent des propriétés de prolongement de
formes linéaires ou de semi-normes d'un sous-espace vectoriel à l'espace
tout entier. D'autres sont de type géométrique et concernent des
propriétés de séparation d'un convexe et d'un point ou de deux convexes.
Nous avons choisi ici d'en présenter une version géométrique simple.

Théorème~5.5.3 Soit E un espace vectoriel normé de dimension finie, K un
convexe fermé non vide et x\textbackslash{}mathrel\{∉\}K. Alors il
existe un hyperplan affine qui sépare strictement x et K, c'est-à-dire
que x et K sont dans les deux demi-espaces ouverts définis par
l'hyperplan.

Démonstration Puisque toutes les normes sont équivalentes, on peut
supposer que E est muni d'une norme euclidienne. Soit alors y la
projection de x sur le convexe K. L'hyperplan médiateur du segment
{[}x,y{]} convient évidemment. ~~ \includegraphics{cours6x.png}

Remarque~5.5.4 Une autre fa\textbackslash{}c\{c\}on de formuler le
théorème est de dire que si K est un convexe fermé et
x\textbackslash{}mathrel\{∉\}K, il existe une forme linéaire f sur E
telle que f(x) \textless{}\{\textbackslash{}mathop\{ inf\}
\}\_\{y∈K\}f(y).

Corollaire~5.5.4 Soit E un espace vectoriel normé de dimension finie,
\{K\}\_\{1\} un convexe compact non vide et \{K\}\_\{2\} un convexe
fermé non vide tels que \{K\}\_\{1\} ∩ \{K\}\_\{2\} = ∅. Alors (i) il
existe un hyperplan H qui sépare strictement \{K\}\_\{1\} et
\{K\}\_\{2\} (ii) il existe une forme linéaire f telle que
\{\textbackslash{}mathop\{sup\}\}\_\{x∈\{K\}\_\{1\}\}f(x)
\textless{}\{\textbackslash{}mathop\{ inf\} \}\_\{x∈\{K\}\_\{2\}\}f(x)

Démonstration La fonction x\textbackslash{}mathrel\{↦\}d(x,\{K\}\_\{2\})
est continue sur le compact \{K\}\_\{1\}, donc atteint sa borne
inférieure en \{x\}\_\{0\}. Il suffit alors d'appliquer la méthode
précédente à \{x\}\_\{0\} et à \{K\}\_\{2\}.

\paragraph{5.5.4 L'enveloppe convexe~: Carathéodory et Krein Millman}

Définition~5.5.2 Soit E un ℝ-espace vectoriel et A une partie de E.
L'ensemble des convexes contenant A admet un plus petit élément appelé
l'enveloppe convexe de A~: c'est encore l'ensemble des barycentres à
coefficients positifs de points de A.

Démonstration L'intersection de tous les convexes contenant A est encore
un convexe contenant A et c'est le plus petit. L'ensemble des
barycentres à coefficients positifs de points de A est un convexe (les
barycentres à coefficients positifs de barycentres à coefficients
positifs sont encore des barycentres à coefficients positifs) contenant
A donc il contient l'enveloppe convexe~; mais comme celle-ci est stable
par barycentrage à coefficients positifs, elle doit contenir tout
barycentre à coefficients positifs de points de A, d'où l'égalité.

Théorème~5.5.5 (Carathéodory). Soit n =\textbackslash{}mathop\{ dim\} E.
Alors l'enveloppe convexe de A est encore l'ensemble des barycentres à
coefficients positifs de n + 1 points de A.

Démonstration Il suffit évidemment de démontrer que si x est barycentre
à coefficients positifs de p ≥ n + 2 points de A, c'est encore un
barycentre à coefficients positifs de p − 1 points de A. Soit donc x
=\{\textbackslash{}mathop\{ \textbackslash{}mathop\{∑ \}\}
\}\_\{i=1\}\^{}\{p\}\{λ\}\_\{i\}\{x\}\_\{i\} avec \{λ\}\_\{i\} ≥ 0 et
\textbackslash{}mathop\{\textbackslash{}mathop\{∑ \}\} \{λ\}\_\{i\} = 1.
La famille \{(\{x\}\_\{i\} − \{x\}\_\{p\})\}\_\{1≤i≤p−1\} de E est une
famille de p − 1 ≥ n + 1 éléments dans E de dimension n, donc elle est
liée. On peut trouver
\{α\}\_\{1\},\textbackslash{}mathop\{\textbackslash{}mathop\{\ldots{}\}\},\{α\}\_\{p−1\}
non tous nuls tels que
\{\textbackslash{}mathop\{\textbackslash{}mathop\{∑ \}\}
\}\_\{i=1\}\^{}\{p−1\}\{α\}\_\{i\}(\{x\}\_\{i\} − \{x\}\_\{p\}) = 0.
Posons \{α\}\_\{p\} = −(\{α\}\_\{1\} +
\textbackslash{}mathop\{\textbackslash{}mathop\{\ldots{}\}\} +
\{α\}\_\{p−1\}). On a donc
\{\textbackslash{}mathop\{\textbackslash{}mathop\{∑ \}\}
\}\_\{i=1\}\^{}\{p\}\{α\}\_\{i\}\{x\}\_\{i\} = 0 avec
\{\textbackslash{}mathop\{\textbackslash{}mathop\{∑ \}\}
\}\_\{i=1\}\^{}\{p\}\{α\}\_\{i\} = 0. Soit t ∈ \{ℝ\}\^{}\{+\}. On a
alors x =\{\textbackslash{}mathop\{ \textbackslash{}mathop\{∑ \}\}
\}\_\{i=1\}\^{}\{p\}(\{λ\}\_\{i\} − t\{α\}\_\{i\})\{x\}\_\{i\} avec
\{\textbackslash{}mathop\{\textbackslash{}mathop\{∑ \}\}
\}\_\{i=1\}\^{}\{p\}(\{λ\}\_\{i\} − t\{α\}\_\{i\}) = 1. Il suffit alors
de choisir t de telle sorte que \textbackslash{}mathop\{∀\}i,
\{λ\}\_\{i\} − t\{α\}\_\{i\} ≥ 0 avec pour un certain \{i\}\_\{0\},
\{λ\}\_\{\{i\}\_\{0\}\} − t\{α\}\_\{\{i\}\_\{0\}\} = 0 pour aboutir au
résultat souhaité. Or, si \{α\}\_\{i\} ≤ 0, on a évidemment \{λ\}\_\{i\}
− t\{α\}\_\{i\} ≥ 0. Il suffit donc de considérer les \{α\}\_\{i\}
\textgreater{} 0 et de prendre t =\textbackslash{}mathop\{
min\}\textbackslash{}\{\{ \{λ\}\_\{i\} \textbackslash{}over
\{α\}\_\{i\}\} \textbackslash{}mathrel\{∣\}\{α\}\_\{i\} \textgreater{}
0\textbackslash{}\} =\{ \{λ\}\_\{\{i\}\_\{0\}\} \textbackslash{}over
\{α\}\_\{\{i\}\_\{0\}\}\} .

Corollaire~5.5.6 Soit E un ℝ-espace vectoriel de dimension finie et A
une partie compacte de E. Alors l'enveloppe convexe de A est encore
compacte.

Démonstration Soit n =\textbackslash{}mathop\{ dim\} E,

C =
\textbackslash{}\{(\{λ\}\_\{1\},\textbackslash{}mathop\{\textbackslash{}mathop\{\ldots{}\}\},\{λ\}\_\{n+1\})
∈
\{ℝ\}\^{}\{n+1\}\textbackslash{}mathrel\{∣\}\textbackslash{}mathop\{∀\}i,
\{λ\}\_\{ i\} ≥ 0\textbackslash{}text\{ et \}\textbackslash{}mathop\{∑
\}\{λ\}\_\{i\} = 1\textbackslash{}\}

C est une partie compacte de \{ℝ\}\^{}\{n+1\} (car fermée et bornée dans
un espace vectoriel normé~de dimension finie) et l'enveloppe convexe de
A est l'image de l'application continue φ : C × \{A\}\^{}\{n+1\} → E
définie par
φ(\{λ\}\_\{1\},\textbackslash{}mathop\{\textbackslash{}mathop\{\ldots{}\}\},\{λ\}\_\{n+1\},\{x\}\_\{1\},\textbackslash{}mathop\{\textbackslash{}mathop\{\ldots{}\}\},\{x\}\_\{n+1\})
=\{\textbackslash{}mathop\{ \textbackslash{}mathop\{∑ \}\}
\}\_\{i=1\}\^{}\{n+1\}\{λ\}\_\{i\}\{x\}\_\{i\}. Comme C ×
\{A\}\^{}\{n+1\} est compacte, cette image est compacte.

Remarque~5.5.5 Soit maintenant K un convexe. On peut essayer de trouver
une partie minimale de K qui engendre K, c'est-à-dire dont K soit
l'enveloppe convexe. Une telle partie doit évidemment contenir les
points de K qui ne sont pas barycentres d'autres points de K (autrement
que de fa\textbackslash{}c\{c\}on triviale). Nous allons voir que pour
un convexe compact, ces points suffisent presque à engendrer K.

Définition~5.5.3 Soit K un convexe. Un point x de K est dit un point
extrémal de K si on a

\textbackslash{}mathop\{∀\}y,z ∈ K,\textbackslash{}quad x ∈ {[}y,z{]} ⇒
x = y\textbackslash{}text\{ ou \}x = z

Un sous-ensemble S de K est dit extrémal si

\textbackslash{}mathop\{∀\}y,z ∈ K,\textbackslash{}quad
{]}y,z{[}∩S\textbackslash{}mathrel\{≠\}∅⇒ y ∈ S\textbackslash{}text\{ et
\}z ∈ S

Lemme~5.5.7 Soit K un convexe compact, f une forme linéaire sur E, μ
=\{\textbackslash{}mathop\{ sup\}\}\_\{x∈K\}f(x). Alors K' =
\textbackslash{}\{x ∈ K\textbackslash{}mathrel\{∣\}f(x) =
μ\textbackslash{}\} est un sous-ensemble compact extrémal de K.

Démonstration En effet, soit y,z ∈ K, x ∈{]}y,z{[}∩K'~; on a f(x) = μ
avec x = ty + (1 − t)z et t ∈{]}0,1{[}. Alors μ = tf(y) + (1 − t)f(z)
avec t \textgreater{} 0, 1 − t \textgreater{} 0, f(y) ≤ μ, f(z) ≤ μ~;
ceci n'est possible que si f(y) = μ et f(z) = μ, soit y ∈ K' et z ∈ K'.

Théorème~5.5.8 (Krein-Millman). Soit E un ℝ-espace vectoriel de
dimension finie et K un convexe compact de E. Alors K est l'adhérence de
l'enveloppe convexe de ses points extrémaux.

Démonstration Nous montrerons ce résultat par récurrence sur
\textbackslash{}mathop\{dim\} E (le cas de la dimension 1 est laissé au
lecteur). Soit P l'ensemble des compacts extrémaux non vides de K.
Remarquons que tout intersection d'éléments de P est soit vide, soit
encore dans P. Soit S ∈P. Montrons tout d'abord que S contient un point
extrémal. Si toute forme linéaire f est constante sur S, alors S est un
singleton réduit à un point extrémal. Sinon, soit f une forme linéaire
non constante sur S, μ =\{\textbackslash{}mathop\{ sup\}\}\_\{x∈S\}f(x)
et S' = \textbackslash{}\{x ∈ S\textbackslash{}mathrel\{∣\}f(x) =
μ\textbackslash{}\}. Alors S' est un sous ensemble convexe compact de
l'hyperplan H d'équation f(x) = μ. En vectorialisant cet hyperplan, on
obtient par récurrence que S' admet un point extrémal x.

Montrons par l'absurde que x est un point extrémal de K. Si x ∈{]}y,z{[}
avec y,z ∈ K, on a {]}y,z{[}∩S\textbackslash{}mathrel\{≠\}∅, donc y ∈ S
et z ∈ S. Mais comme S' est un sous-ensemble extrémal de S et
{]}y,z{[}∩S'\textbackslash{}mathrel\{≠\}∅, on a y,z ∈ S'~; ceci
contredit le fait que x soit un point extrémal de S'. On a donc montré
que toute partie compacte extrémale contenait un point extrémal.

Soit donc \{K\}\_\{0\} l'adhérence de l'enveloppe convexe des points
extrémaux de K. On a \{K\}\_\{0\} ⊂ K et puisque tout ensemble extrémal
contient un point extrémal, \{K\}\_\{0\} rencontre tout ensemble
extrémal. Supposons que \{K\}\_\{0\}\textbackslash{}mathrel\{≠\}K et
soit x ∈ K ∖ \{K\}\_\{0\}. D'après le théorème de Hahn Banach, il existe
une forme linéaire f telle que f(x)
\textgreater{}\{\textbackslash{}mathop\{
sup\}\}\_\{y∈\{K\}\_\{0\}\}f(y). Soit μ =\{\textbackslash{}mathop\{
sup\}\}\_\{z∈K\}f(z) et S = \textbackslash{}\{z ∈
K\textbackslash{}mathrel\{∣\}f(z) = μ\textbackslash{}\}. S est non vide
(une fonction continue sur un compact atteint sa borne supérieure),
extrémal d'après le lemme précédent et S ∩ \{K\}\_\{0\} = ∅ (car si y ∈
\{K\}\_\{0\}, f(y) \textless{} f(x) ≤ μ). Donc S est un sous-ensemble
extrémal qui ne contient aucun point extrémal. C'est absurde. Donc K =
\{K\}\_\{0\}.

Exemple~5.5.1 Un polygone et plus généralement un polyèdre est enveloppe
convexe de ses sommets.

Remarque~5.5.6 En dimension finie, on peut affiner le résultat en
montrant qu'en fait K est l'enveloppe convexe de ses points extrémaux,
et pas seulement l'adhérence de l'enveloppe convexe. Ceci nécessite une
version plus fine de Hahn-Banach.

{[}\href{coursse30.html}{prev}{]}
{[}\href{coursse30.html\#tailcoursse30.html}{prev-tail}{]}
{[}\href{coursse31.html}{front}{]}
{[}\href{coursch6.html\#coursse31.html}{up}{]}

\end{document}

% \documentclass[]{article}
\usepackage[T1]{fontenc}
\usepackage{lmodern}
\usepackage{amssymb,amsmath}
\usepackage{ifxetex,ifluatex}
\usepackage{fixltx2e} % provides \textsubscript
% use upquote if available, for straight quotes in verbatim environments
\IfFileExists{upquote.sty}{\usepackage{upquote}}{}
\ifnum 0\ifxetex 1\fi\ifluatex 1\fi=0 % if pdftex
  \usepackage[utf8]{inputenc}
\else % if luatex or xelatex
  \ifxetex
    \usepackage{mathspec}
    \usepackage{xltxtra,xunicode}
  \else
    \usepackage{fontspec}
  \fi
  \defaultfontfeatures{Mapping=tex-text,Scale=MatchLowercase}
  \newcommand{\euro}{€}
\fi
% use microtype if available
\IfFileExists{microtype.sty}{\usepackage{microtype}}{}
\ifxetex
  \usepackage[setpagesize=false, % page size defined by xetex
              unicode=false, % unicode breaks when used with xetex
              xetex]{hyperref}
\else
  \usepackage[unicode=true]{hyperref}
\fi
\hypersetup{breaklinks=true,
            bookmarks=true,
            pdfauthor={},
            pdftitle={Relations de comparaison},
            colorlinks=true,
            citecolor=blue,
            urlcolor=blue,
            linkcolor=magenta,
            pdfborder={0 0 0}}
\urlstyle{same}  % don't use monospace font for urls
\setlength{\parindent}{0pt}
\setlength{\parskip}{6pt plus 2pt minus 1pt}
\setlength{\emergencystretch}{3em}  % prevent overfull lines
\setcounter{secnumdepth}{0}
 
/* start css.sty */
.cmr-5{font-size:50%;}
.cmr-7{font-size:70%;}
.cmmi-5{font-size:50%;font-style: italic;}
.cmmi-7{font-size:70%;font-style: italic;}
.cmmi-10{font-style: italic;}
.cmsy-5{font-size:50%;}
.cmsy-7{font-size:70%;}
.cmex-7{font-size:70%;}
.cmex-7x-x-71{font-size:49%;}
.msbm-7{font-size:70%;}
.cmtt-10{font-family: monospace;}
.cmti-10{ font-style: italic;}
.cmbx-10{ font-weight: bold;}
.cmr-17x-x-120{font-size:204%;}
.cmsl-10{font-style: oblique;}
.cmti-7x-x-71{font-size:49%; font-style: italic;}
.cmbxti-10{ font-weight: bold; font-style: italic;}
p.noindent { text-indent: 0em }
td p.noindent { text-indent: 0em; margin-top:0em; }
p.nopar { text-indent: 0em; }
p.indent{ text-indent: 1.5em }
@media print {div.crosslinks {visibility:hidden;}}
a img { border-top: 0; border-left: 0; border-right: 0; }
center { margin-top:1em; margin-bottom:1em; }
td center { margin-top:0em; margin-bottom:0em; }
.Canvas { position:relative; }
li p.indent { text-indent: 0em }
.enumerate1 {list-style-type:decimal;}
.enumerate2 {list-style-type:lower-alpha;}
.enumerate3 {list-style-type:lower-roman;}
.enumerate4 {list-style-type:upper-alpha;}
div.newtheorem { margin-bottom: 2em; margin-top: 2em;}
.obeylines-h,.obeylines-v {white-space: nowrap; }
div.obeylines-v p { margin-top:0; margin-bottom:0; }
.overline{ text-decoration:overline; }
.overline img{ border-top: 1px solid black; }
td.displaylines {text-align:center; white-space:nowrap;}
.centerline {text-align:center;}
.rightline {text-align:right;}
div.verbatim {font-family: monospace; white-space: nowrap; text-align:left; clear:both; }
.fbox {padding-left:3.0pt; padding-right:3.0pt; text-indent:0pt; border:solid black 0.4pt; }
div.fbox {display:table}
div.center div.fbox {text-align:center; clear:both; padding-left:3.0pt; padding-right:3.0pt; text-indent:0pt; border:solid black 0.4pt; }
div.minipage{width:100%;}
div.center, div.center div.center {text-align: center; margin-left:1em; margin-right:1em;}
div.center div {text-align: left;}
div.flushright, div.flushright div.flushright {text-align: right;}
div.flushright div {text-align: left;}
div.flushleft {text-align: left;}
.underline{ text-decoration:underline; }
.underline img{ border-bottom: 1px solid black; margin-bottom:1pt; }
.framebox-c, .framebox-l, .framebox-r { padding-left:3.0pt; padding-right:3.0pt; text-indent:0pt; border:solid black 0.4pt; }
.framebox-c {text-align:center;}
.framebox-l {text-align:left;}
.framebox-r {text-align:right;}
span.thank-mark{ vertical-align: super }
span.footnote-mark sup.textsuperscript, span.footnote-mark a sup.textsuperscript{ font-size:80%; }
div.tabular, div.center div.tabular {text-align: center; margin-top:0.5em; margin-bottom:0.5em; }
table.tabular td p{margin-top:0em;}
table.tabular {margin-left: auto; margin-right: auto;}
div.td00{ margin-left:0pt; margin-right:0pt; }
div.td01{ margin-left:0pt; margin-right:5pt; }
div.td10{ margin-left:5pt; margin-right:0pt; }
div.td11{ margin-left:5pt; margin-right:5pt; }
table[rules] {border-left:solid black 0.4pt; border-right:solid black 0.4pt; }
td.td00{ padding-left:0pt; padding-right:0pt; }
td.td01{ padding-left:0pt; padding-right:5pt; }
td.td10{ padding-left:5pt; padding-right:0pt; }
td.td11{ padding-left:5pt; padding-right:5pt; }
table[rules] {border-left:solid black 0.4pt; border-right:solid black 0.4pt; }
.hline hr, .cline hr{ height : 1px; margin:0px; }
.tabbing-right {text-align:right;}
span.TEX {letter-spacing: -0.125em; }
span.TEX span.E{ position:relative;top:0.5ex;left:-0.0417em;}
a span.TEX span.E {text-decoration: none; }
span.LATEX span.A{ position:relative; top:-0.5ex; left:-0.4em; font-size:85%;}
span.LATEX span.TEX{ position:relative; left: -0.4em; }
div.float img, div.float .caption {text-align:center;}
div.figure img, div.figure .caption {text-align:center;}
.marginpar {width:20%; float:right; text-align:left; margin-left:auto; margin-top:0.5em; font-size:85%; text-decoration:underline;}
.marginpar p{margin-top:0.4em; margin-bottom:0.4em;}
.equation td{text-align:center; vertical-align:middle; }
td.eq-no{ width:5%; }
table.equation { width:100%; } 
div.math-display, div.par-math-display{text-align:center;}
math .texttt { font-family: monospace; }
math .textit { font-style: italic; }
math .textsl { font-style: oblique; }
math .textsf { font-family: sans-serif; }
math .textbf { font-weight: bold; }
.partToc a, .partToc, .likepartToc a, .likepartToc {line-height: 200%; font-weight:bold; font-size:110%;}
.chapterToc a, .chapterToc, .likechapterToc a, .likechapterToc, .appendixToc a, .appendixToc {line-height: 200%; font-weight:bold;}
.index-item, .index-subitem, .index-subsubitem {display:block}
.caption td.id{font-weight: bold; white-space: nowrap; }
table.caption {text-align:center;}
h1.partHead{text-align: center}
p.bibitem { text-indent: -2em; margin-left: 2em; margin-top:0.6em; margin-bottom:0.6em; }
p.bibitem-p { text-indent: 0em; margin-left: 2em; margin-top:0.6em; margin-bottom:0.6em; }
.subsectionHead, .likesubsectionHead { margin-top:2em; font-weight: bold;}
.sectionHead, .likesectionHead { font-weight: bold;}
.quote {margin-bottom:0.25em; margin-top:0.25em; margin-left:1em; margin-right:1em; text-align:justify;}
.verse{white-space:nowrap; margin-left:2em}
div.maketitle {text-align:center;}
h2.titleHead{text-align:center;}
div.maketitle{ margin-bottom: 2em; }
div.author, div.date {text-align:center;}
div.thanks{text-align:left; margin-left:10%; font-size:85%; font-style:italic; }
div.author{white-space: nowrap;}
.quotation {margin-bottom:0.25em; margin-top:0.25em; margin-left:1em; }
h1.partHead{text-align: center}
.sectionToc, .likesectionToc {margin-left:2em;}
.subsectionToc, .likesubsectionToc {margin-left:4em;}
.sectionToc, .likesectionToc {margin-left:6em;}
.frenchb-nbsp{font-size:75%;}
.frenchb-thinspace{font-size:75%;}
.figure img.graphics {margin-left:10%;}
/* end css.sty */

\title{Relations de comparaison}
\author{}
\date{}

\begin{document}
\maketitle

\textbf{Warning: 
requires JavaScript to process the mathematics on this page.\\ If your
browser supports JavaScript, be sure it is enabled.}

\begin{center}\rule{3in}{0.4pt}\end{center}

[
[]
[

\section{6.1 Relations de comparaison}

\subsection{6.1.1 Notations}

Soit A \subset~ \mathbb{R}~ et a \in\overline\mathbb{R}~ tel que a
\in\overlineA (adhérence dans
\overline\mathbb{R}~). Si E est un espace vectoriel normé, on
notera ℱ_a,A(E) l'ensemble des fonctions de \mathbb{R}~ vers E telles
qu'il existe V \in V (a) tel que f soit définie sur V \bigcap A (autrement dit f
est définie sur A au voisinage de a).

Exemple~6.1.1 Si A = \mathbb{N}~ et a = +\infty~, ℱ_a,A(E) est l'ensemble des
suites (x_n)_n≥n_0 d'éléments de E.

\subsection{6.1.2 Domination, prépondérance}

Définition~6.1.1 Soit f \inℱ_a,A(E) et g \inℱ_a,A(F).

\begin{itemize}
\item
  (i) On dit que f est dominée par g au voisinage de a suivant A et on
  note f = O(g) si

  \existsK ≥ 0, \\exists~V \in V
  (a), \forall~~t \in V \bigcap A,
  \f(t)\ \leq
  K\g(t)\
\item
  (ii) On dit que g est prépondérante devant f (ou que f est négligeable
  devant g) au voisinage de a suivant A et on note f = o(g) si

  \forall~~\epsilon > 0,
  \existsV \in V (a), \\forall~~t \in V
  \bigcap A, \f(t)\ \leq
  \epsilon\g(t)\
\end{itemize}

Remarque~6.1.1 Il est clair que f = o(g) \rigtharrow~ f = O(g). De plus, si f =
O(g) et si X est choisi comme ci dessus, on voit que g(t) = 0 \rigtharrow~ f(t) =
0~; on constate donc que

\begin{itemize}
\itemsep1pt\parskip0pt\parsep0pt
\item
  (i) aux indéterminations près de type  0 \over 0 ,
  f = O(g) \Leftrightarrow
  \f\
  \over
  \g\ est bornée (au
  voisinage de a suivant A)
\item
  (ii) aux indéterminations près de type  0 \over 0 ,
  f = o(g) \Leftrightarrow
  lim_t\rightarrow~a,t\inA~
  \f(t)\
  \over
  \g(t)\ = 0
\end{itemize}

Exemple~6.1.2

\begin{itemize}
\itemsep1pt\parskip0pt\parsep0pt
\item
  f = O(1) \Leftrightarrow f est bornée au voisinage de a
  suivant A~;
\item
  f = o(1) \Leftrightarrow
  lim_t\rightarrow~a,t\inA~f(t) = 0.
\end{itemize}

Proposition~6.1.1

\begin{itemize}
\item
  (i) f_1 = O(g)\text et f_2 = O(g)
  \rigtharrow~ \alpha~f_1 + \muf_2 = O(g)
\item
  (ii) f_1 = o(g)\text et f_2 =
  o(g) \rigtharrow~ \alpha~f_1 + \muf_2 = o(g)
\item
  (iii) soit \phi,\psi \inℱ_a,A(K) et f,g \inℱ_a,A(E), alors

  \begin{align*} \phi = O(\psi)\text et
  f = O(g)& \rigtharrow~& \phif = O(\psig)\%& \\ \phi =
  o(\psi)\text et f = O(g)& \rigtharrow~& \phif = o(\psig) \%&
  \\ \phi = O(\psi)\text et f
  = o(g)& \rigtharrow~& \phif = o(\psig) \%& \\
  \end{align*}
\item
  (iv) f \inℱ_a,A(E),g \inℱ_a,A(F),h \inℱ_a,A(G)~;
  alors

  \begin{align*} f = O(g)\text et
  g = O(h)& \rigtharrow~& f = O(h)\%& \\ f =
  O(g)\text et g = o(h)& \rigtharrow~& f = o(h) \%&
  \\ f = o(g)\text et g
  = O(h)& \rigtharrow~& f = o(h) \%& \\
  \end{align*}
\end{itemize}

Démonstration Facile

\subsection{6.1.3 Equivalence}

Lemme~6.1.2 Soit f,g \inℱ_a,A(E). Alors f - g = o(g) \rigtharrow~ g = O(f).

Démonstration Il existe V \in V (a) tel que \forall~~t \in
V \bigcap A, \f(t) - g(t)\
\leq 1 \over 2
\g(t)\. Pour t \in V \bigcap
A, on a donc \g(t)\
=\ g(t) - f(t) + f(t)\
\leq\ g(t) - f(t)\
+\ f(t)\ \leq 1
\over 2
\g(t)\
+\ f(t)\ soit encore
\g(t)\ \leq
2\f(t)\, et donc g =
O(f).

Théorème~6.1.3 Pour f,g \inℱ_a,A(E), on pose f ∼ g si f - g =
o(g). Il s'agit d'une relation d'équivalence appelée l'équivalence des
fonctions (au voisinage de a suivant A).

Démonstration La réflexivité est claire puisque f - f = 0 = o(f). Si f ∼
g, on a f - g = o(g) et aussi d'après le lemme, g = O(f), d'où f - g =
o(f) et donc aussi g - f = o(f), soit g ∼ f. La relation est donc
symétrique. Si f ∼ g et g ∼ h, on a f - g = o(g) et g - h = o(h). Mais
on a h ∼ g, soit h - g = o(g) soit g = O(h). Alors f - g = o(g) et g =
O(h) donne f - g = o(h) et donc f - h = (f - g) + (g - h) = o(h), d'où f
∼ h, ce qui démontre la transitivité.

Proposition~6.1.4

\begin{itemize}
\itemsep1pt\parskip0pt\parsep0pt
\item
  (i) f ∼ g \rigtharrow~ f = O(g)\text et g = O(f)
\item
  (ii) \phi,\psi \inℱ_a,A(K), f,g \inℱ_a,A(E), alors \phi ∼
  \psi\text et f ∼ g \rigtharrow~ \phif ∼ \psig
\end{itemize}

Démonstration (i) est évident d'après le lemme ci dessus et la symétrie
de la relation. Pour (ii), on écrit \phif - \psig = (\phi - \psi)f + \psi(f - g). On a
\phi - \psi = o(\psi)\text et f = O(g) \rigtharrow~ (\phi - \psi)f = o(\psig) et f
- g = o(g) \rigtharrow~ \psi(f - g) = o(\psig), d'où \phif - \psig = o(\psig) et \phif ∼ \psig.

Remarque~6.1.2 La relation d'équivalence est donc compatible avec la
multiplication~; par contre, elle n'est pas compatible avec l'addition~:
f_1 ∼ g_1\text et f_2 ∼
g_2\rigtharrow~̸f_1 + f_2 ∼ g_1 + g_2
comme le montre l'exemple a = 0, f_1(t) = 1 + t,g_1(t)
= 1 + t^2,f_2(t) = g_2(t) = -1~; on a
f_1 ∼ g_1\text et f_2 =
g_2, pourtant f_1(t) + f_2(t) = t et
g_1(t) + g_2(t) = t^2 ne sont pas
équivalentes au voisinage de 0.

Lemme~6.1.5 Soit f,g \inℱ_a,A(K). Alors on a équivalence de

\begin{itemize}
\itemsep1pt\parskip0pt\parsep0pt
\item
  (i) f ∼ g
\item
  (ii) il existe \phi \inℱ_a,A(K) telle que f = g\phi et
  lim_t\rightarrow~a,t\inA~\phi(t) = 1
\end{itemize}

Démonstration (ii) \rigtharrow~(i). On écrit f - g = g(\phi - 1) avec
lim_t\rightarrow~a,t\inA~(\phi(t) - 1) = 0, d'où f - g
= o(g) et f ∼ g.

(i) \rigtharrow~(ii). On a f = O(g). D'après une remarque précédente, il existe V \in
V (a) tel que \forall~~t \in V \bigcap A, g(t) = 0 \rigtharrow~ f(t) = 0.
Définissons \phi sur V \bigcap A de la manière suivante~: \phi(t) =
\left \ \cases  f(t)
\over g(t) &si g(t)\neq~0
\cr 1 &si g(t) = 0  \right .~; si
g(t)\neq~0, on a f(t) = \phi(t)g(t) de manière
évidente et cela reste vrai si g(t) = 0 puisque alors on a aussi f(t) =
0. Montrons que lim_t\rightarrow~a,t\inA~\phi(t) = 1.
Soit \epsilon > 0~; il existe V _0 \in V (a) tel que
\forall~t \in V _0~ \bigcap A, f(t) -
g(t)\leq \epsilong(t) soit encore pour t \in V
_0 \bigcap V \bigcap A, \phi(t) -
1\,g(t)\leq
\epsilong(t). Si g(t)\neq~0 on a
donc \phi(t) - 1\leq \epsilon mais cela reste vrai si g(t) = 0
puisqu'alors \phi(t) = 1. On a donc bien
lim_t\rightarrow~a,t\inA~\phi(t) = 1.

Théorème~6.1.6

\begin{itemize}
\item
  (i) si f,g \inℱ_a,A(K) et n \in \mathbb{N}~, alors f ∼ g \rigtharrow~ f^n ∼
  g^n
\item
  (ii) si f,g \inℱ_a,A(\mathbb{R}~) et s'il existe V \in V (a) tel que
  \forall~~t \in V \bigcap A, g(t) ≥ 0 (resp. >
  0) alors

  \forall~\alpha~ \in \mathbb{R}~^+~\text
  (resp. \$\forall~\alpha~ \in \mathbb{R}~\$) f ∼ g \rigtharrow~ f^\alpha~~
  ∼ g^\alpha~
\end{itemize}

Démonstration Résulte immédiatement du lemme précédent en remarquant
pour (ii) que si \phi tend vers 1, elle est strictement positive au
voisinage de a et que lim\phi^\alpha~~ = 1.

Remarque~6.1.3 La relation d'équivalence est donc compatible avec les
puissances entières ou réelles~; par contre elle n'est pas compatible
avec l'exponentielle~: en fait on a e^f ∼ e^g
\Leftrightarrow lim~(f - g) = 0.

Le théorème suivant justifie l'intérêt de l'utilisation des équivalents
pour les recherches de limites

Théorème~6.1.7 Soit f,g \inℱ_a,A(E) telles que f ∼ g. Si g admet
une limite \ell en a suivant A, f admet la même limite en a suivant A.

Démonstration Puisque lim_t\rightarrow~a,t\inA~g(t)
= \ell, il existe V _0 \in V (a) tel que t \in V _0 \bigcap A
\rigtharrow~\ g(t) - \ell\
< 1 soit
\g(t)\ \leq 1
+\ \ell\. Soit alors \epsilon
> 0. Il existe V \in V (a) tel que
\forall~~t \in V \bigcap A, \g(t) -
\ell\ < \epsilon \over 2 et
il existe V ' \in V (a) tel que \forall~~t \in V ' \bigcap A,
\f(t) - g(t)\ \leq \epsilon
\over
2(1+\\ell\)
\g(t)\. Pour t \in V
_0 \bigcap V \bigcap V ' \bigcap A on a

\begin{align*} \f(t) -
\ell& \leq& \f(t) -
g(t)\ +\ g(t) -
\ell\\%& \\ &
\leq& \epsilon \over 2(1 +\
\ell\) (1 +\
\ell\) + \epsilon \over 2 \%&
\\ & =& \epsilon \%&
\\ \end{align*}

et donc f admet \ell pour limite en a suivant A.

\subsection{6.1.4 Changement de variables}

Soit A,B \subset~ \mathbb{R}~ et a,b \in\overline\mathbb{R}~ tel que a
\in\overlineA et b \in\overlineB.

Soit \phi une fonction de \mathbb{R}~ vers \mathbb{R}~ telle que \phi(A) \subset~ B et
lim_t\rightarrow~a,t\inA~\phi(t) = b. Par définition
de la notion de limite on a aussitôt

Lemme~6.1.8 \forall~~V `\in V (b),
\exists~V \in V (a), \phi(V \bigcap A) \subset~ V' \bigcap B.

Il en découle immédiatement le théorème suivant

Théorème~6.1.9 Soit f,g \inℱ_b,B(E). Alors

\begin{itemize}
\itemsep1pt\parskip0pt\parsep0pt
\item
  (i) f = O_b,B(g) \rigtharrow~ f \cdot \phi = O_a,A(g \cdot \phi)
\item
  (i) f = o_b,B(g) \rigtharrow~ f \cdot \phi = o_a,A(g \cdot \phi)
\item
  (i) f ∼_b,Bg \rigtharrow~ f \cdot \phi ∼_a,Ag \cdot \phi
\end{itemize}

autrement dit on peut faire tout changement de variable raisonnable dans
des relations de comparaison.

[
[

\end{document}

\documentclass[]{article}
\usepackage[T1]{fontenc}
\usepackage{lmodern}
\usepackage{amssymb,amsmath}
\usepackage{ifxetex,ifluatex}
\usepackage{fixltx2e} % provides \textsubscript
% use upquote if available, for straight quotes in verbatim environments
\IfFileExists{upquote.sty}{\usepackage{upquote}}{}
\ifnum 0\ifxetex 1\fi\ifluatex 1\fi=0 % if pdftex
  \usepackage[utf8]{inputenc}
\else % if luatex or xelatex
  \ifxetex
    \usepackage{mathspec}
    \usepackage{xltxtra,xunicode}
  \else
    \usepackage{fontspec}
  \fi
  \defaultfontfeatures{Mapping=tex-text,Scale=MatchLowercase}
  \newcommand{\euro}{€}
\fi
% use microtype if available
\IfFileExists{microtype.sty}{\usepackage{microtype}}{}
\ifxetex
  \usepackage[setpagesize=false, % page size defined by xetex
              unicode=false, % unicode breaks when used with xetex
              xetex]{hyperref}
\else
  \usepackage[unicode=true]{hyperref}
\fi
\hypersetup{breaklinks=true,
            bookmarks=true,
            pdfauthor={},
            pdftitle={Developpements limites},
            colorlinks=true,
            citecolor=blue,
            urlcolor=blue,
            linkcolor=magenta,
            pdfborder={0 0 0}}
\urlstyle{same}  % don't use monospace font for urls
\setlength{\parindent}{0pt}
\setlength{\parskip}{6pt plus 2pt minus 1pt}
\setlength{\emergencystretch}{3em}  % prevent overfull lines
\setcounter{secnumdepth}{0}
 
/* start css.sty */
.cmr-5{font-size:50%;}
.cmr-7{font-size:70%;}
.cmmi-5{font-size:50%;font-style: italic;}
.cmmi-7{font-size:70%;font-style: italic;}
.cmmi-10{font-style: italic;}
.cmsy-5{font-size:50%;}
.cmsy-7{font-size:70%;}
.cmex-7{font-size:70%;}
.cmex-7x-x-71{font-size:49%;}
.msbm-7{font-size:70%;}
.cmtt-10{font-family: monospace;}
.cmti-10{ font-style: italic;}
.cmbx-10{ font-weight: bold;}
.cmr-17x-x-120{font-size:204%;}
.cmsl-10{font-style: oblique;}
.cmti-7x-x-71{font-size:49%; font-style: italic;}
.cmbxti-10{ font-weight: bold; font-style: italic;}
p.noindent { text-indent: 0em }
td p.noindent { text-indent: 0em; margin-top:0em; }
p.nopar { text-indent: 0em; }
p.indent{ text-indent: 1.5em }
@media print {div.crosslinks {visibility:hidden;}}
a img { border-top: 0; border-left: 0; border-right: 0; }
center { margin-top:1em; margin-bottom:1em; }
td center { margin-top:0em; margin-bottom:0em; }
.Canvas { position:relative; }
li p.indent { text-indent: 0em }
.enumerate1 {list-style-type:decimal;}
.enumerate2 {list-style-type:lower-alpha;}
.enumerate3 {list-style-type:lower-roman;}
.enumerate4 {list-style-type:upper-alpha;}
div.newtheorem { margin-bottom: 2em; margin-top: 2em;}
.obeylines-h,.obeylines-v {white-space: nowrap; }
div.obeylines-v p { margin-top:0; margin-bottom:0; }
.overline{ text-decoration:overline; }
.overline img{ border-top: 1px solid black; }
td.displaylines {text-align:center; white-space:nowrap;}
.centerline {text-align:center;}
.rightline {text-align:right;}
div.verbatim {font-family: monospace; white-space: nowrap; text-align:left; clear:both; }
.fbox {padding-left:3.0pt; padding-right:3.0pt; text-indent:0pt; border:solid black 0.4pt; }
div.fbox {display:table}
div.center div.fbox {text-align:center; clear:both; padding-left:3.0pt; padding-right:3.0pt; text-indent:0pt; border:solid black 0.4pt; }
div.minipage{width:100%;}
div.center, div.center div.center {text-align: center; margin-left:1em; margin-right:1em;}
div.center div {text-align: left;}
div.flushright, div.flushright div.flushright {text-align: right;}
div.flushright div {text-align: left;}
div.flushleft {text-align: left;}
.underline{ text-decoration:underline; }
.underline img{ border-bottom: 1px solid black; margin-bottom:1pt; }
.framebox-c, .framebox-l, .framebox-r { padding-left:3.0pt; padding-right:3.0pt; text-indent:0pt; border:solid black 0.4pt; }
.framebox-c {text-align:center;}
.framebox-l {text-align:left;}
.framebox-r {text-align:right;}
span.thank-mark{ vertical-align: super }
span.footnote-mark sup.textsuperscript, span.footnote-mark a sup.textsuperscript{ font-size:80%; }
div.tabular, div.center div.tabular {text-align: center; margin-top:0.5em; margin-bottom:0.5em; }
table.tabular td p{margin-top:0em;}
table.tabular {margin-left: auto; margin-right: auto;}
div.td00{ margin-left:0pt; margin-right:0pt; }
div.td01{ margin-left:0pt; margin-right:5pt; }
div.td10{ margin-left:5pt; margin-right:0pt; }
div.td11{ margin-left:5pt; margin-right:5pt; }
table[rules] {border-left:solid black 0.4pt; border-right:solid black 0.4pt; }
td.td00{ padding-left:0pt; padding-right:0pt; }
td.td01{ padding-left:0pt; padding-right:5pt; }
td.td10{ padding-left:5pt; padding-right:0pt; }
td.td11{ padding-left:5pt; padding-right:5pt; }
table[rules] {border-left:solid black 0.4pt; border-right:solid black 0.4pt; }
.hline hr, .cline hr{ height : 1px; margin:0px; }
.tabbing-right {text-align:right;}
span.TEX {letter-spacing: -0.125em; }
span.TEX span.E{ position:relative;top:0.5ex;left:-0.0417em;}
a span.TEX span.E {text-decoration: none; }
span.LATEX span.A{ position:relative; top:-0.5ex; left:-0.4em; font-size:85%;}
span.LATEX span.TEX{ position:relative; left: -0.4em; }
div.float img, div.float .caption {text-align:center;}
div.figure img, div.figure .caption {text-align:center;}
.marginpar {width:20%; float:right; text-align:left; margin-left:auto; margin-top:0.5em; font-size:85%; text-decoration:underline;}
.marginpar p{margin-top:0.4em; margin-bottom:0.4em;}
.equation td{text-align:center; vertical-align:middle; }
td.eq-no{ width:5%; }
table.equation { width:100%; } 
div.math-display, div.par-math-display{text-align:center;}
math .texttt { font-family: monospace; }
math .textit { font-style: italic; }
math .textsl { font-style: oblique; }
math .textsf { font-family: sans-serif; }
math .textbf { font-weight: bold; }
.partToc a, .partToc, .likepartToc a, .likepartToc {line-height: 200%; font-weight:bold; font-size:110%;}
.chapterToc a, .chapterToc, .likechapterToc a, .likechapterToc, .appendixToc a, .appendixToc {line-height: 200%; font-weight:bold;}
.index-item, .index-subitem, .index-subsubitem {display:block}
.caption td.id{font-weight: bold; white-space: nowrap; }
table.caption {text-align:center;}
h1.partHead{text-align: center}
p.bibitem { text-indent: -2em; margin-left: 2em; margin-top:0.6em; margin-bottom:0.6em; }
p.bibitem-p { text-indent: 0em; margin-left: 2em; margin-top:0.6em; margin-bottom:0.6em; }
.paragraphHead, .likeparagraphHead { margin-top:2em; font-weight: bold;}
.subparagraphHead, .likesubparagraphHead { font-weight: bold;}
.quote {margin-bottom:0.25em; margin-top:0.25em; margin-left:1em; margin-right:1em; text-align:justify;}
.verse{white-space:nowrap; margin-left:2em}
div.maketitle {text-align:center;}
h2.titleHead{text-align:center;}
div.maketitle{ margin-bottom: 2em; }
div.author, div.date {text-align:center;}
div.thanks{text-align:left; margin-left:10%; font-size:85%; font-style:italic; }
div.author{white-space: nowrap;}
.quotation {margin-bottom:0.25em; margin-top:0.25em; margin-left:1em; }
h1.partHead{text-align: center}
.sectionToc, .likesectionToc {margin-left:2em;}
.subsectionToc, .likesubsectionToc {margin-left:4em;}
.subsubsectionToc, .likesubsubsectionToc {margin-left:6em;}
.frenchb-nbsp{font-size:75%;}
.frenchb-thinspace{font-size:75%;}
.figure img.graphics {margin-left:10%;}
/* end css.sty */

\title{Developpements limites}
\author{}
\date{}

\begin{document}
\maketitle

\textbf{Warning: \href{http://www.math.union.edu/locate/jsMath}{jsMath}
requires JavaScript to process the mathematics on this page.\\ If your
browser supports JavaScript, be sure it is enabled.}

\begin{center}\rule{3in}{0.4pt}\end{center}

{[}\href{coursse34.html}{next}{]} {[}\href{coursse32.html}{prev}{]}
{[}\href{coursse32.html\#tailcoursse32.html}{prev-tail}{]}
{[}\hyperref[tailcoursse33.html]{tail}{]}
{[}\href{coursch7.html\#coursse33.html}{up}{]}

\subsubsection{6.2 Développements limités}

\paragraph{6.2.1 Notion de développement limité}

Définition~6.2.1 Soit I un intervalle de ℝ et a ∈ I. Soit f : I → E et n
∈ ℕ. On dit que f admet en a un développement limité à l'ordre n s'il
existe
\{a\}\_\{0\},\{a\}\_\{1\},\textbackslash{}mathop\{\textbackslash{}mathop\{\ldots{}\}\},\{a\}\_\{n\}
∈ E tels que, au voisinage de a, f(t) = \{a\}\_\{0\} + \{a\}\_\{1\}(t −
a) + \textbackslash{}mathop\{\textbackslash{}mathop\{\ldots{}\}\} +
\{a\}\_\{n\}\{(t − a)\}\^{}\{n\} + o(\{(t − a)\}\^{}\{n\}).

Remarque~6.2.1 On notera aussi f(t) = P(t − a) + o(\{(t − a)\}\^{}\{n\})
et on parlera un peu abusivement du polynôme P.

Proposition~6.2.1 Si f admet en a un développement limité à l'ordre n,
alors celui ci est unique.

Démonstration Supposons que l'on ait deux développements distincts~:
f(t) = \{a\}\_\{0\} + \{a\}\_\{1\}(t−a) +
\textbackslash{}mathop\{\textbackslash{}mathop\{\ldots{}\}\} +
\{a\}\_\{n\}\{(t−a)\}\^{}\{n\} + o(\{(t−a)\}\^{}\{n\}) = \{b\}\_\{0\} +
\{b\}\_\{1\}(t−a) +
\textbackslash{}mathop\{\textbackslash{}mathop\{\ldots{}\}\} +
\{b\}\_\{n\}\{(t−a)\}\^{}\{n\} + o(\{(t−a)\}\^{}\{n\})et soit p
=\textbackslash{}mathop\{
min\}\textbackslash{}\{k\textbackslash{}mathrel\{∣\}\{a\}\_\{k\}\textbackslash{}mathrel\{≠\}\{b\}\_\{k\}\textbackslash{}\}.
Alors on a par soustraction (\{a\}\_\{p\} − \{b\}\_\{p\})\{(t −
a)\}\^{}\{p\} = o(\{(t − a)\}\^{}\{n\}) ce qui est absurde.

Proposition~6.2.2 Si f admet en a un développement limité à l'ordre n,
alors f est continue en a. Si n ≥ 1, alors f est dérivable en a.

Démonstration On a bien entendu f(a) = \{a\}\_\{0\} et
\{\textbackslash{}mathop\{lim\}\}\_\{t→a\}f(t) = \{a\}\_\{0\} d'où la
continuité. Si n ≥ 1, on a \{ f(t)−f(a) \textbackslash{}over t−a\} =\{
f(t)−\{a\}\_\{0\} \textbackslash{}over t−a\} = \{a\}\_\{1\} + o(1) de
limite \{a\}\_\{1\} quand t tend vers a.

Remarque~6.2.2 Ceci ne s'étend pas à des ordres supérieurs~; la fonction
f(t) = \{t\}\^{}\{100\}\textbackslash{}mathop\{ sin\} (\{ 1
\textbackslash{}over \{t\}\^{}\{100\}\} ) si
t\textbackslash{}mathrel\{≠\}0, f(0) = 0 admet en 0 un développement
limité à l'ordre 99 puisque f(t) = o(\{t\}\^{}\{99\}) (la fonction
\textbackslash{}mathop\{sin\} étant bornée)~; pourtant f n'est pas 2
fois dérivable en 0 puisque sa dérivée est définie par f'(0) = 0 et
f'(x) = 100\{t\}\^{}\{99\}\textbackslash{}mathop\{ sin\} (\{ 1
\textbackslash{}over \{t\}\^{}\{100\}\} ) −\{ 100 \textbackslash{}over
t\} \textbackslash{}mathop\{ cos\} (\{ 1 \textbackslash{}over
\{t\}\^{}\{100\}\} )~; elle n'est pas continue en 0, donc pas dérivable.
Par contre on a

Théorème~6.2.3 Si f : I → E est n fois dérivable au point a, alors f
admet en a le développement limité à l'ordre n

f(t) = f(a) +\{ \textbackslash{}mathop\{∑ \}\}\_\{k=1\}\^{}\{n\}\{
\{f\}\^{}\{(k)\}(a) \textbackslash{}over k!\} \{(t − a)\}\^{}\{k\} +
o(\{(t − a)\}\^{}\{n\})

Démonstration C'est la formule de Taylor Young, démontrée dans le
chapitre sur les fonctions d'une variable réelle.

Remarque~6.2.3 Ce théorème permet, en connaissant les dérivées
successives de la fonction f (ce qui est finalement assez rare), de
calculer un développement limité~; mais cela permet également en
connaissant un développement limité à l'ordre n de la fonction f en a
(par exemple à l'aide des méthodes du paragraphe suivant), d'en déduire
les dérivées successives de la fonction f en a.

\paragraph{6.2.2 Opérations sur les développements limités}

Proposition~6.2.4 Si f,g : I → E admettent en a des développements
limités à l'ordre n, f(t) = P(t − a) + o(\{(t − a)\}\^{}\{n\}),g(t) =
Q(t − a) + o(\{(t − a)\}\^{}\{n\}), alors αf + βg admet en a le
développement limité à l'ordre n, (αf + βg)(t) = (αP + βQ)(t − a) +
o(\{(t − a)\}\^{}\{n\}).

Démonstration Découle immédiatement des propriétés de la relation de
prépondérance.

Proposition~6.2.5 Si f,g : I → K admettent en a des développements
limités à l'ordre n, f(t) = P(t − a) + o(\{(t − a)\}\^{}\{n\}),g(t) =
Q(t − a) + o(\{(t − a)\}\^{}\{n\}), alors fg admet en a le développement
limité à l'ordre n, f(t)g(t) = R(t − a) + o(\{(t − a)\}\^{}\{n\}), où R
est le polynôme obtenu en tronquant à l'ordre n le polynôme PQ.

Démonstration On a f(t)g(t) =
P(t−a)Q(t−a)+P(t−a)\{(t−a)\}\^{}\{n\}\{ε\}\_\{2\}(t−a)+Q(t−a)\{(t−a)\}\^{}\{n\}\{ε\}\_\{1\}(t−a)+\{(t−a)\}\^{}\{2n\}\{ε\}\_\{1\}(t−a)\{ε\}\_\{2\}(t−a)
avec \{\textbackslash{}mathop\{lim\}\}\_\{t→a\}\{ε\}\_\{i\}(t − a) = 0.
On a donc f(t)g(t) = P(t − a)Q(t − a) + o(\{(t − a)\}\^{}\{n\}). Mais on
a P(X)Q(X) = R(X) + \{X\}\^{}\{n+1\}S(X), d'où P(t − a)Q(t − a) = R(t −
a) + o(\{(t − a)\}\^{}\{n\}), et donc f(t)g(t) = R(t − a) + o(\{(t −
a)\}\^{}\{n\}).

Proposition~6.2.6 Si f,g : I → K admettent en a des développements
limités à l'ordre n, f(t) = P(t − a) + o(\{(t − a)\}\^{}\{n\}),g(t) =
Q(t − a) + o(\{(t − a)\}\^{}\{n\}), et si
g(a)\textbackslash{}mathrel\{≠\}0, alors \{ f \textbackslash{}over g\}
admet en a le développement limité à l'ordre n, \{ f(t)
\textbackslash{}over g(t)\} = R(t − a) + o(\{(t − a)\}\^{}\{n\}), où R
est le quotient de la division suivant les puissances croissantes à
l'ordre n du polynôme P(X) par le polynôme R(X).

Démonstration Remarquons que g(a) = Q(0), donc
Q(0)\textbackslash{}mathrel\{≠\}0. On a

\textbackslash{}begin\{eqnarray*\}\{ f(t) \textbackslash{}over g(t)\}
−\{ P(t − a) \textbackslash{}over Q(t − a)\} \&\& \%\&
\textbackslash{}\textbackslash{} \& =\&\{ (f(t) − P(t − a))Q(t − a) +
P(t − a)(Q(t − a) − g(t)) \textbackslash{}over Q(t − a)g(t)\} \%\&
\textbackslash{}\textbackslash{} \& =\& o(\{(t − a)\}\^{}\{n\}) \%\&
\textbackslash{}\textbackslash{} \textbackslash{}end\{eqnarray*\}

puisque f(t) − P(t − a) = o(\{(t − a)\}\^{}\{n\}), Q(t − a) = O(1), g(t)
− Q(t − a) = o(\{(t − a)\}\^{}\{n\}), P(t − a) = O(1) et
\{\textbackslash{}mathop\{lim\}\}\_\{t→a\}\{ 1 \textbackslash{}over
Q(t−a)g(t)\} =\{ 1 \textbackslash{}over g\{(a)\}\^{}\{2\}\} . Ecrivons
alors P(X) = Q(X)R(X) + \{X\}\^{}\{n+1\}S(X) (division suivant les
puissances croissantes de P par Q à l'ordre n, possible car
Q(0)\textbackslash{}mathrel\{≠\}0). On a alors \{ P(t−a)
\textbackslash{}over Q(t−a)\} = R(t − a) + \{(t − a)\}\^{}\{n+1\}\{
S(t−a) \textbackslash{}over Q(t−a)\} = R(t − a) + o(\{(t −
a)\}\^{}\{n\}) puisque \{\textbackslash{}mathop\{lim\}\}\_\{t→a\}\{
S(t−a) \textbackslash{}over Q(t−a)\} =\{ S(0) \textbackslash{}over
Q(0)\} . En définitive \{ f(t) \textbackslash{}over g(t)\} = R(t − a) +
o(\{(t − a)\}\^{}\{n\}).

Le théorème suivant sera uniquement formulé en 0 pour des raisons de
commodité~; on se ramène immédiatement à cette situation par des
translations sur les variables.

Théorème~6.2.7 Soit I,J deux intervalles de ℝ contenant 0, φ : I → J
vérifiant φ(0) = 0 et admettant en 0 un développement limité à l'ordre
n, φ(t) = P(t) + o(\{t\}\^{}\{n\})~; soit f : J → E admettant en 0 un
développement limité à l'ordre n, f(u) = Q(u) + o(\{u\}\^{}\{n\}). Alors
f ∘ φ admet en 0 un développement limité à l'ordre n, f ∘ φ(t) = R(t) +
o(\{t\}\^{}\{n\}) où R(X) est le polynôme obtenu en tronquant à l'ordre
n le polynôme Q(P(X)).

Démonstration On écrit f(φ(t)) = \{a\}\_\{0\} + \{a\}\_\{1\}φ(t) +
\textbackslash{}mathop\{\textbackslash{}mathop\{\ldots{}\}\} +
\{a\}\_\{n\}φ\{(t)\}\^{}\{n\} + φ\{(t)\}\^{}\{n\}ε(φ(t)). Mais chacune
des fonctions φ\{(t)\}\^{}\{i\} admet d'après la proposition précédente
un développement φ\{(t)\}\^{}\{i\} = P\{(t)\}\^{}\{i\} +
o(\{t\}\^{}\{n\}). On a donc f(φ(t)) = \{a\}\_\{0\} + \{a\}\_\{1\}P(t) +
\textbackslash{}mathop\{\textbackslash{}mathop\{\ldots{}\}\} +
\{a\}\_\{n\}P\{(t)\}\^{}\{n\} + o(\{t\}\^{}\{n\}) +
φ\{(t)\}\^{}\{n\}ε(φ(t)). Mais comme φ admet en 0 un développement
limité à l'ordre 1 et que φ(0) = 0, on a φ(t) = O(t) et donc
φ\{(t)\}\^{}\{n\}ε(φ(t)) = o(\{t\}\^{}\{n\}). On obtient donc f ∘ φ(t) =
Q(P(t)) + o(\{t\}\^{}\{n\}) = R(t) + \{t\}\^{}\{n+1\}S(t) +
o(\{t\}\^{}\{n\}) = R(t) + o(\{t\}\^{}\{n\}).

Les deux résultats suivants découlent immédiatement de la formule de
Taylor-Young et de l'unicité du développement limité

Proposition~6.2.8 Soit f : I → E une fonction n fois dérivable au point
a ∈ I, admettant en a le développement limité à l'ordre n, f(t) =
\{a\}\_\{0\} + \{a\}\_\{1\}(t − a) +
\textbackslash{}mathop\{\textbackslash{}mathop\{\ldots{}\}\} +
\{a\}\_\{n\}\{(t − a)\}\^{}\{n\} + o(\{(t − a)\}\^{}\{n\}). Soit F : I →
E une fonction dérivable telle que F' = f. Alors F admet en a le
développement limité à l'ordre n + 1, F(t) = F(a) + \{a\}\_\{0\}(t − a)
+\{ \{a\}\_\{1\} \textbackslash{}over 2\} \{(t − a)\}\^{}\{2\} +
\textbackslash{}mathop\{\textbackslash{}mathop\{\ldots{}\}\} +\{
\{a\}\_\{n\} \textbackslash{}over n+1\} \{(t − a)\}\^{}\{n+1\} + o(\{(t
− a)\}\^{}\{n+1\}).

Proposition~6.2.9 Soit f : I → ℝ une fonction continue strictement
monotone, n fois dérivable au point 0 telle que f(0) = 0 et
f'(0)\textbackslash{}mathrel\{≠\}0. Soit J l'intervalle f(I). Alors g =
\{f\}\^{}\{−1\} : J → ℝ admet en 0 un développement limité à l'ordre n~:
g(t) = \{b\}\_\{1\}t +
\textbackslash{}mathop\{\textbackslash{}mathop\{\ldots{}\}\} +
\{b\}\_\{n\}\{t\}\^{}\{n\} + o(\{t\}\^{}\{n\})~; on obtient ce
développement limité en identifiant le développement limité de g(f(t))
au polynôme t, ce qui conduit à un système triangulaire en les inconnues
\{b\}\_\{1\},\textbackslash{}mathop\{\textbackslash{}mathop\{\ldots{}\}\},\{b\}\_\{n\}.

\paragraph{6.2.3 Développements limités classiques}

On part d'un certain nombre de développements limités classiques obtenus
par la formule de Taylor-Young et on en déduit d'autres par changements
de variables et intégration. On obtient les développements suivants en 0

\textbackslash{}begin\{eqnarray*\} \{e\}\^{}\{t\}\& =\& 1 + t +\{
\{t\}\^{}\{2\} \textbackslash{}over 2\} +
\textbackslash{}mathop\{\textbackslash{}mathop\{\ldots{}\}\} +\{
\{t\}\^{}\{n\} \textbackslash{}over n!\} + o(\{t\}\^{}\{n\}) \%\&
\textbackslash{}\textbackslash{} \textbackslash{}mathop\{cos\} t\& =\& 1
−\{ \{t\}\^{}\{2\} \textbackslash{}over 2!\} +
\textbackslash{}mathop\{\textbackslash{}mathop\{\ldots{}\}\} +
\{(−1)\}\^{}\{n\}\{ \{t\}\^{}\{2n\} \textbackslash{}over (2n)!\} +
o(\{t\}\^{}\{2n+1\}) \%\& \textbackslash{}\textbackslash{}
\textbackslash{}mathop\{sin\} t\& =\& t −\{ \{t\}\^{}\{3\}
\textbackslash{}over 3!\} +
\textbackslash{}mathop\{\textbackslash{}mathop\{\ldots{}\}\} +
\{(−1)\}\^{}\{n\}\{ \{t\}\^{}\{2n+1\} \textbackslash{}over (2n + 1)!\} +
o(\{t\}\^{}\{2n+2\})\%\& \textbackslash{}\textbackslash{}
\textbackslash{}mathop\{\textbackslash{}mathrm\{ch\}\} t\& =\& 1 +\{
\{t\}\^{}\{2\} \textbackslash{}over 2!\} +
\textbackslash{}mathop\{\textbackslash{}mathop\{\ldots{}\}\} +\{
\{t\}\^{}\{2n\} \textbackslash{}over (2n)!\} + o(\{t\}\^{}\{2n+1\}) \%\&
\textbackslash{}\textbackslash{}
\textbackslash{}mathop\{\textbackslash{}mathrm\{sh\}\} t\& =\& t +\{
\{t\}\^{}\{3\} \textbackslash{}over 3!\} +
\textbackslash{}mathop\{\textbackslash{}mathop\{\ldots{}\}\} +\{
\{t\}\^{}\{2n+1\} \textbackslash{}over (2n + 1)!\} +
o(\{t\}\^{}\{2n+2\}) \%\& \textbackslash{}\textbackslash{} \{(1 +
t)\}\^{}\{α\}\& =\& 1 + αt +\{ α(α − 1) \textbackslash{}over 2!\}
\{t\}\^{}\{2\} +
\textbackslash{}mathop\{\textbackslash{}mathop\{\ldots{}\}\} \%\&
\textbackslash{}\textbackslash{} \& \textbackslash{}text\{\} \& +\{ α(α
− 1)\textbackslash{}mathop\{\textbackslash{}mathop\{\ldots{}\}\}(α − n +
1) \textbackslash{}over n!\} \{t\}\^{}\{n\} + o(\{t\}\^{}\{n\}) \%\&
\textbackslash{}\textbackslash{} \{ 1 \textbackslash{}over 1 + t\} \&
=\& 1 − t + \{t\}\^{}\{2\} +
\textbackslash{}mathop\{\textbackslash{}mathop\{\ldots{}\}\} +
\{(−1)\}\^{}\{n\}\{t\}\^{}\{n\} + o(\{t\}\^{}\{n\}) \%\&
\textbackslash{}\textbackslash{} \{ 1 \textbackslash{}over 1 − t\} \&
=\& 1 + t + \{t\}\^{}\{2\} +
\textbackslash{}mathop\{\textbackslash{}mathop\{\ldots{}\}\} +
\{t\}\^{}\{n\} + o(\{t\}\^{}\{n\}) \%\& \textbackslash{}\textbackslash{}
\textbackslash{}mathop\{log\} (1 + t)\& =\& t −\{ \{t\}\^{}\{2\}
\textbackslash{}over 2\} +
\textbackslash{}mathop\{\textbackslash{}mathop\{\ldots{}\}\} +
\{(−1)\}\^{}\{n\}\{ \{t\}\^{}\{n\} \textbackslash{}over n\} +
o(\{t\}\^{}\{n\}) \%\& \textbackslash{}\textbackslash{}
\textbackslash{}mathop\{log\} (1 − t)\& =\& −t −\{ \{t\}\^{}\{2\}
\textbackslash{}over 2\}
−\textbackslash{}mathop\{\textbackslash{}mathop\{\ldots{}\}\} −\{
\{t\}\^{}\{n\} \textbackslash{}over n\} + o(\{t\}\^{}\{n\}) \%\&
\textbackslash{}\textbackslash{}
\textbackslash{}mathop\{\textbackslash{}mathrm\{arctg\}\} t\& =\& t −\{
\{t\}\^{}\{3\} \textbackslash{}over 3\} +
\textbackslash{}mathop\{\textbackslash{}mathop\{\ldots{}\}\} +
\{(−1)\}\^{}\{n\}\{ \{t\}\^{}\{2n+1\} \textbackslash{}over 2n + 1\} +
o(\{t\}\^{}\{2n+2\}) \%\& \textbackslash{}\textbackslash{}
\textbackslash{}mathop\{arg\}
\textbackslash{}mathop\{\textbackslash{}mathrm\{th\}\} t\& =\& t +\{
\{t\}\^{}\{3\} \textbackslash{}over 3\} +
\textbackslash{}mathop\{\textbackslash{}mathop\{\ldots{}\}\} +\{
\{t\}\^{}\{2n+1\} \textbackslash{}over 2n + 1\} + o(\{t\}\^{}\{2n+2\})
\%\& \textbackslash{}\textbackslash{} \textbackslash{}mathop\{arcsin\}
t\& =\& t +\{ \{t\}\^{}\{3\} \textbackslash{}over 6\} +
\textbackslash{}mathop\{\textbackslash{}mathop\{\ldots{}\}\} \%\&
\textbackslash{}\textbackslash{} \& \textbackslash{}text\{\} \& +\{
1.3\textbackslash{}mathop\{\textbackslash{}mathop\{\ldots{}\}\}(2n − 1)
\textbackslash{}over
2.4\textbackslash{}mathop\{\textbackslash{}mathop\{\ldots{}\}\}(2n)\} \{
\{t\}\^{}\{2n+1\} \textbackslash{}over 2n + 1\} + o(\{t\}\^{}\{2n+2\})
\%\& \textbackslash{}\textbackslash{} \textbackslash{}mathop\{arg\}
\textbackslash{}mathop\{\textbackslash{}mathrm\{sh\}\} t\& =\& t −\{
\{t\}\^{}\{3\} \textbackslash{}over 6\} +
\textbackslash{}mathop\{\textbackslash{}mathop\{\ldots{}\}\} \%\&
\textbackslash{}\textbackslash{} \& \textbackslash{}text\{\} \&
+\{(−1)\}\^{}\{n\}\{
1.3\textbackslash{}mathop\{\textbackslash{}mathop\{\ldots{}\}\}(2n − 1)
\textbackslash{}over
2.4\textbackslash{}mathop\{\textbackslash{}mathop\{\ldots{}\}\}(2n)\} \{
\{t\}\^{}\{2n+1\} \textbackslash{}over 2n + 1\} + o(\{t\}\^{}\{2n+2\})
\%\& \textbackslash{}\textbackslash{} \textbackslash{}end\{eqnarray*\}

{[}\href{coursse34.html}{next}{]} {[}\href{coursse32.html}{prev}{]}
{[}\href{coursse32.html\#tailcoursse32.html}{prev-tail}{]}
{[}\href{coursse33.html}{front}{]}
{[}\href{coursch7.html\#coursse33.html}{up}{]}

\end{document}

% \documentclass[]{article}
\usepackage[T1]{fontenc}
\usepackage{lmodern}
\usepackage{amssymb,amsmath}
\usepackage{ifxetex,ifluatex}
\usepackage{fixltx2e} % provides \textsubscript
% use upquote if available, for straight quotes in verbatim environments
\IfFileExists{upquote.sty}{\usepackage{upquote}}{}
\ifnum 0\ifxetex 1\fi\ifluatex 1\fi=0 % if pdftex
  \usepackage[utf8]{inputenc}
\else % if luatex or xelatex
  \ifxetex
    \usepackage{mathspec}
    \usepackage{xltxtra,xunicode}
  \else
    \usepackage{fontspec}
  \fi
  \defaultfontfeatures{Mapping=tex-text,Scale=MatchLowercase}
  \newcommand{\euro}{€}
\fi
% use microtype if available
\IfFileExists{microtype.sty}{\usepackage{microtype}}{}
\ifxetex
  \usepackage[setpagesize=false, % page size defined by xetex
              unicode=false, % unicode breaks when used with xetex
              xetex]{hyperref}
\else
  \usepackage[unicode=true]{hyperref}
\fi
\hypersetup{breaklinks=true,
            bookmarks=true,
            pdfauthor={},
            pdftitle={Developpements asymptotiques},
            colorlinks=true,
            citecolor=blue,
            urlcolor=blue,
            linkcolor=magenta,
            pdfborder={0 0 0}}
\urlstyle{same}  % don't use monospace font for urls
\setlength{\parindent}{0pt}
\setlength{\parskip}{6pt plus 2pt minus 1pt}
\setlength{\emergencystretch}{3em}  % prevent overfull lines
\setcounter{secnumdepth}{0}
 
/* start css.sty */
.cmr-5{font-size:50%;}
.cmr-7{font-size:70%;}
.cmmi-5{font-size:50%;font-style: italic;}
.cmmi-7{font-size:70%;font-style: italic;}
.cmmi-10{font-style: italic;}
.cmsy-5{font-size:50%;}
.cmsy-7{font-size:70%;}
.cmex-7{font-size:70%;}
.cmex-7x-x-71{font-size:49%;}
.msbm-7{font-size:70%;}
.cmtt-10{font-family: monospace;}
.cmti-10{ font-style: italic;}
.cmbx-10{ font-weight: bold;}
.cmr-17x-x-120{font-size:204%;}
.cmsl-10{font-style: oblique;}
.cmti-7x-x-71{font-size:49%; font-style: italic;}
.cmbxti-10{ font-weight: bold; font-style: italic;}
p.noindent { text-indent: 0em }
td p.noindent { text-indent: 0em; margin-top:0em; }
p.nopar { text-indent: 0em; }
p.indent{ text-indent: 1.5em }
@media print {div.crosslinks {visibility:hidden;}}
a img { border-top: 0; border-left: 0; border-right: 0; }
center { margin-top:1em; margin-bottom:1em; }
td center { margin-top:0em; margin-bottom:0em; }
.Canvas { position:relative; }
li p.indent { text-indent: 0em }
.enumerate1 {list-style-type:decimal;}
.enumerate2 {list-style-type:lower-alpha;}
.enumerate3 {list-style-type:lower-roman;}
.enumerate4 {list-style-type:upper-alpha;}
div.newtheorem { margin-bottom: 2em; margin-top: 2em;}
.obeylines-h,.obeylines-v {white-space: nowrap; }
div.obeylines-v p { margin-top:0; margin-bottom:0; }
.overline{ text-decoration:overline; }
.overline img{ border-top: 1px solid black; }
td.displaylines {text-align:center; white-space:nowrap;}
.centerline {text-align:center;}
.rightline {text-align:right;}
div.verbatim {font-family: monospace; white-space: nowrap; text-align:left; clear:both; }
.fbox {padding-left:3.0pt; padding-right:3.0pt; text-indent:0pt; border:solid black 0.4pt; }
div.fbox {display:table}
div.center div.fbox {text-align:center; clear:both; padding-left:3.0pt; padding-right:3.0pt; text-indent:0pt; border:solid black 0.4pt; }
div.minipage{width:100%;}
div.center, div.center div.center {text-align: center; margin-left:1em; margin-right:1em;}
div.center div {text-align: left;}
div.flushright, div.flushright div.flushright {text-align: right;}
div.flushright div {text-align: left;}
div.flushleft {text-align: left;}
.underline{ text-decoration:underline; }
.underline img{ border-bottom: 1px solid black; margin-bottom:1pt; }
.framebox-c, .framebox-l, .framebox-r { padding-left:3.0pt; padding-right:3.0pt; text-indent:0pt; border:solid black 0.4pt; }
.framebox-c {text-align:center;}
.framebox-l {text-align:left;}
.framebox-r {text-align:right;}
span.thank-mark{ vertical-align: super }
span.footnote-mark sup.textsuperscript, span.footnote-mark a sup.textsuperscript{ font-size:80%; }
div.tabular, div.center div.tabular {text-align: center; margin-top:0.5em; margin-bottom:0.5em; }
table.tabular td p{margin-top:0em;}
table.tabular {margin-left: auto; margin-right: auto;}
div.td00{ margin-left:0pt; margin-right:0pt; }
div.td01{ margin-left:0pt; margin-right:5pt; }
div.td10{ margin-left:5pt; margin-right:0pt; }
div.td11{ margin-left:5pt; margin-right:5pt; }
table[rules] {border-left:solid black 0.4pt; border-right:solid black 0.4pt; }
td.td00{ padding-left:0pt; padding-right:0pt; }
td.td01{ padding-left:0pt; padding-right:5pt; }
td.td10{ padding-left:5pt; padding-right:0pt; }
td.td11{ padding-left:5pt; padding-right:5pt; }
table[rules] {border-left:solid black 0.4pt; border-right:solid black 0.4pt; }
.hline hr, .cline hr{ height : 1px; margin:0px; }
.tabbing-right {text-align:right;}
span.TEX {letter-spacing: -0.125em; }
span.TEX span.E{ position:relative;top:0.5ex;left:-0.0417em;}
a span.TEX span.E {text-decoration: none; }
span.LATEX span.A{ position:relative; top:-0.5ex; left:-0.4em; font-size:85%;}
span.LATEX span.TEX{ position:relative; left: -0.4em; }
div.float img, div.float .caption {text-align:center;}
div.figure img, div.figure .caption {text-align:center;}
.marginpar {width:20%; float:right; text-align:left; margin-left:auto; margin-top:0.5em; font-size:85%; text-decoration:underline;}
.marginpar p{margin-top:0.4em; margin-bottom:0.4em;}
.equation td{text-align:center; vertical-align:middle; }
td.eq-no{ width:5%; }
table.equation { width:100%; } 
div.math-display, div.par-math-display{text-align:center;}
math .texttt { font-family: monospace; }
math .textit { font-style: italic; }
math .textsl { font-style: oblique; }
math .textsf { font-family: sans-serif; }
math .textbf { font-weight: bold; }
.partToc a, .partToc, .likepartToc a, .likepartToc {line-height: 200%; font-weight:bold; font-size:110%;}
.chapterToc a, .chapterToc, .likechapterToc a, .likechapterToc, .appendixToc a, .appendixToc {line-height: 200%; font-weight:bold;}
.index-item, .index-subitem, .index-subsubitem {display:block}
.caption td.id{font-weight: bold; white-space: nowrap; }
table.caption {text-align:center;}
h1.partHead{text-align: center}
p.bibitem { text-indent: -2em; margin-left: 2em; margin-top:0.6em; margin-bottom:0.6em; }
p.bibitem-p { text-indent: 0em; margin-left: 2em; margin-top:0.6em; margin-bottom:0.6em; }
.paragraphHead, .likeparagraphHead { margin-top:2em; font-weight: bold;}
.subparagraphHead, .likesubparagraphHead { font-weight: bold;}
.quote {margin-bottom:0.25em; margin-top:0.25em; margin-left:1em; margin-right:1em; text-align:\jmathustify;}
.verse{white-space:nowrap; margin-left:2em}
div.maketitle {text-align:center;}
h2.titleHead{text-align:center;}
div.maketitle{ margin-bottom: 2em; }
div.author, div.date {text-align:center;}
div.thanks{text-align:left; margin-left:10%; font-size:85%; font-style:italic; }
div.author{white-space: nowrap;}
.quotation {margin-bottom:0.25em; margin-top:0.25em; margin-left:1em; }
h1.partHead{text-align: center}
.sectionToc, .likesectionToc {margin-left:2em;}
.subsectionToc, .likesubsectionToc {margin-left:4em;}
.subsubsectionToc, .likesubsubsectionToc {margin-left:6em;}
.frenchb-nbsp{font-size:75%;}
.frenchb-thinspace{font-size:75%;}
.figure img.graphics {margin-left:10%;}
/* end css.sty */

\title{Developpements asymptotiques}
\author{}
\date{}

\begin{document}
\maketitle

\textbf{Warning: 
requires JavaScript to process the mathematics on this page.\\ If your
browser supports JavaScript, be sure it is enabled.}

\begin{center}\rule{3in}{0.4pt}\end{center}

{[}
{[}
{[}{]}
{[}

\subsubsection{6.3 Développements asymptotiques}

\paragraph{6.3.1 Echelles de comparaison, parties principales}

Définition~6.3.1 On appelle échelle de comparaison en a suivant A toute
famille (\phi\_i)\_i\inI de fonctions de ℱ\_a,A(\mathbb{R}~)
vérifiant

\begin{itemize}
\itemsep1pt\parskip0pt\parsep0pt
\item
  (i) \forall~i \in I, \\forall~~V \in V
  (a), \exists~t \in V \bigcap A,
  \phi\_i(t)\neq~0~; autrement dit, aucune
  des \phi\_i n'est identiquement nulle au voisinage de a suivant A
\item
  (ii) si i\neq~\jmath, l'une des deux fonctions
  \phi\_i ou \phi\_\jmath est négligeable devant l'autre
\end{itemize}

Remarque~6.3.1 On obtient une relation d'ordre strict sur I en posant i
\textless{} \jmath \Leftrightarrow \phi\_\jmath =
o(\phi\_i).

Exemple~6.3.1 Au voisinage d'un point a \in \mathbb{R}~, on a plusieurs échelles de
comparaison classiques

\begin{itemize}
\item
  a) la famille des t\mapsto~(t - a)^n, n
  \in \mathbb{N}~ (l'échelle qui conduit aux développements limités)
\item
  b) la famille des t\mapsto~(t - a)^n, n
  \in ℤ
\item
  c) la famille des t\mapsto~\textbar{}t -
  a\textbar{}^\alpha~, \alpha~ \in \mathbb{R}~
\item
  d) la famille des t\mapsto~\textbar{}t -
  a\textbar{}^\alpha~\textbar{}log~
  t\textbar{}^\beta~, \alpha~,\beta~ \in \mathbb{R}~~: on a alors

  (\alpha~,\beta~) \textless{} (\alpha~',\beta~') \Leftrightarrow
  \bigl (\alpha~ \textless{} \alpha~'\text ou (\alpha~
  = \alpha~'\text et \beta~ \textgreater{}
  \beta~')\bigr )
\end{itemize}

Exemple~6.3.2 Au voisinage de a = +\infty~, on a plusieurs échelles de
comparaison classiques

\begin{itemize}
\item
  a) la famille des t\mapsto~t^n, n \in ℤ
  (l'ordre obtenu est l'inverse de l'ordre naturel)
\item
  b) la famille des t\mapsto~t^\alpha~, \alpha~ \in \mathbb{R}~
  (l'ordre obtenu est l'inverse de l'ordre naturel)
\item
  c) la famille des
  t\mapsto~t^\alpha~(log~
  t)^\beta~, \alpha~,\beta~ \in \mathbb{R}~~: on a alors

  (\alpha~,\beta~) \textless{} (\alpha~',\beta~') \Leftrightarrow
  \bigl (\alpha~ \textgreater{} \alpha~'\text ou
  (\alpha~ = \alpha~'\text et \beta~ \textgreater{}
  \beta~')\bigr )
\item
  c) la famille des
  t\mapsto~e^P(t)t^\alpha~(log~
  t)^\beta~, P \in \mathbb{R}~{[}X{]},\alpha~,\beta~ \in \mathbb{R}~~: on a alors

  (P,\alpha~,\beta~) \textless{} (Q,\alpha~',\beta~') \Leftrightarrow
  \left \\cases
  lim\_t\rightarrow~+\infty~~(Q(t) - P(t)) = +\infty~
  \cr \cr \textou &
  \cr P = Q\text et \alpha~ \textgreater{}
  \alpha~' \cr \textou & \cr
  P = Q\text et \alpha~ = \alpha~'\text et \beta~
  \textgreater{} \beta~'  \right .
\end{itemize}

Définition~6.3.2 Soit (\phi\_i)\_i\inI une échelle de
comparaison en a suivant A et f \inℱ\_a,A(E). On dit que f admet
une partie principale suivant l'échelle de comparaison s'il existe i \in I
et a\_i \in E \diagdown\0\ tels que f(t)
∼ a\_i\phi\_i(t). Une telle partie principale si elle
existe est unique.

Démonstration Si on a f(t) ∼ a\_i\phi\_i(t) ∼
b\_\jmath\phi\_\jmath(t), on a nécessairement i = \jmath car sinon une des
deux fonctions serait négligeable devant l'autre. On a alors
(a\_i - b\_i)\phi\_i = o(\phi\_i) ce qui n'est
possible que si a\_i = b\_i.

\paragraph{6.3.2 Développements asymptotiques}

Définition~6.3.3 Soit (\phi\_i)\_i\inI une échelle de
comparaison en a suivant A et f \inℱ\_a,A(E). On dit que f admet
un développement asymptotique à la précision \phi\_\jmath suivant
l'échelle de comparaison s'il existe i\_0 \textless{}
i\_1 \textless{}
\\ldots~ \textless{}
i\_p \leq \jmath et
a\_0,a\_1,\\ldots,a\_p~
\in E \diagdown\0\ tels que

f(t) = a\_0\phi\_i\_0(t) +
\\ldots~ +
a\_p\phi\_i\_p(t) + o(\phi\_\jmath(t))

Remarque~6.3.2 Un tel développement est nécessairement unique puisque
a\_0\phi\_i\_0(t) est nécessairement la partie
principale de f(t), a\_1\phi\_i\_1(t) celle de f(t)
- a\_0\phi\_i\_0(t) et ainsi de suite \jmathusqu'à
a\_p\phi\_i\_p(t) qui doit être la partie
principale de f(t) - a\_0\phi\_i\_0(t)
-\\ldots~ -
a\_p-1\phi\_i\_p-1(t).

\paragraph{6.3.3 Opérations sur les développements asymptotiques}

Il est clair que si f et g admettent des développements asymptotiques à
la précision \phi\_\jmath et \phi\_\jmath', alors \alpha~f + \beta~g admet un
développement asymptotique à la précision
\phi\_min(\jmath,\jmath')~ obtenu de la manière
évidente (additionner les deux et supprimer les termes non
significatifs).

Si l'échelle de comparaison est stable par produit, en faisant le
produit de deux développements asymptotiques on obtient un développement
asymptotique du produit des deux fonctions, à une précision à évaluer
suivant les cas. Ceci peut permettre également de composer
développements asymptotiques et développements limités.

De plus, les théorèmes de comparaison des intégrales impropres peuvent
permettre d'intégrer des développements asymptotiques~:

à condition que la fonction g soit positive au voisinage de a et que son
intégrale converge au point a,

f = o(g) \rigtharrow~\int  \_a^x~f =
o(\int  \_a^x~g)

Exemple~6.3.3 ~: on a pour x \textgreater{} 0 au voisinage du point 0,

\begin{align*} d \over dx
(arcsin~ (1 - x))& =& - 1
\over \sqrt2x - x^2 = -
1 \over \sqrt2x  1
\over \sqrt1 - x \over
2  \%& \\ & =& - 1
\over \sqrt2x (1 + x
\over 4 + x^2 \over 16 +
o(x^2)) \%& \\ & =& -
\sqrt2 \over
2\sqrtx - \sqrt2
\over 8 \sqrtx -
\sqrt2 \over 32 x^3\diagup2 +
o(x^3\diagup2)\%& \\
\end{align*}

En intégrant de 0 à x on va obtenir, en tenant compte de \phi(t) =
o(t^3\diagup2) \rigtharrow~\int ~
\_0^x\phi(t) dt = o(\int ~
\_0^xt^3\diagup2 dt)

arcsin (1 - x) = \pi~ \over 2~
-\sqrt2x - \sqrt2
\over 12 x^3\diagup2 - \sqrt2
\over 80 x^5\diagup2 + o(x^5\diagup2)

{[}
{[}
{[}
{[}

\end{document}

% \documentclass[]{article}
\usepackage[T1]{fontenc}
\usepackage{lmodern}
\usepackage{amssymb,amsmath}
\usepackage{ifxetex,ifluatex}
\usepackage{fixltx2e} % provides \textsubscript
% use upquote if available, for straight quotes in verbatim environments
\IfFileExists{upquote.sty}{\usepackage{upquote}}{}
\ifnum 0\ifxetex 1\fi\ifluatex 1\fi=0 % if pdftex
  \usepackage[utf8]{inputenc}
\else % if luatex or xelatex
  \ifxetex
    \usepackage{mathspec}
    \usepackage{xltxtra,xunicode}
  \else
    \usepackage{fontspec}
  \fi
  \defaultfontfeatures{Mapping=tex-text,Scale=MatchLowercase}
  \newcommand{\euro}{€}
\fi
% use microtype if available
\IfFileExists{microtype.sty}{\usepackage{microtype}}{}
\ifxetex
  \usepackage[setpagesize=false, % page size defined by xetex
              unicode=false, % unicode breaks when used with xetex
              xetex]{hyperref}
\else
  \usepackage[unicode=true]{hyperref}
\fi
\hypersetup{breaklinks=true,
            bookmarks=true,
            pdfauthor={},
            pdftitle={Convergence des suites},
            colorlinks=true,
            citecolor=blue,
            urlcolor=blue,
            linkcolor=magenta,
            pdfborder={0 0 0}}
\urlstyle{same}  % don't use monospace font for urls
\setlength{\parindent}{0pt}
\setlength{\parskip}{6pt plus 2pt minus 1pt}
\setlength{\emergencystretch}{3em}  % prevent overfull lines
\setcounter{secnumdepth}{0}
 
/* start css.sty */
.cmr-5{font-size:50%;}
.cmr-7{font-size:70%;}
.cmmi-5{font-size:50%;font-style: italic;}
.cmmi-7{font-size:70%;font-style: italic;}
.cmmi-10{font-style: italic;}
.cmsy-5{font-size:50%;}
.cmsy-7{font-size:70%;}
.cmex-7{font-size:70%;}
.cmex-7x-x-71{font-size:49%;}
.msbm-7{font-size:70%;}
.cmtt-10{font-family: monospace;}
.cmti-10{ font-style: italic;}
.cmbx-10{ font-weight: bold;}
.cmr-17x-x-120{font-size:204%;}
.cmsl-10{font-style: oblique;}
.cmti-7x-x-71{font-size:49%; font-style: italic;}
.cmbxti-10{ font-weight: bold; font-style: italic;}
p.noindent { text-indent: 0em }
td p.noindent { text-indent: 0em; margin-top:0em; }
p.nopar { text-indent: 0em; }
p.indent{ text-indent: 1.5em }
@media print {div.crosslinks {visibility:hidden;}}
a img { border-top: 0; border-left: 0; border-right: 0; }
center { margin-top:1em; margin-bottom:1em; }
td center { margin-top:0em; margin-bottom:0em; }
.Canvas { position:relative; }
li p.indent { text-indent: 0em }
.enumerate1 {list-style-type:decimal;}
.enumerate2 {list-style-type:lower-alpha;}
.enumerate3 {list-style-type:lower-roman;}
.enumerate4 {list-style-type:upper-alpha;}
div.newtheorem { margin-bottom: 2em; margin-top: 2em;}
.obeylines-h,.obeylines-v {white-space: nowrap; }
div.obeylines-v p { margin-top:0; margin-bottom:0; }
.overline{ text-decoration:overline; }
.overline img{ border-top: 1px solid black; }
td.displaylines {text-align:center; white-space:nowrap;}
.centerline {text-align:center;}
.rightline {text-align:right;}
div.verbatim {font-family: monospace; white-space: nowrap; text-align:left; clear:both; }
.fbox {padding-left:3.0pt; padding-right:3.0pt; text-indent:0pt; border:solid black 0.4pt; }
div.fbox {display:table}
div.center div.fbox {text-align:center; clear:both; padding-left:3.0pt; padding-right:3.0pt; text-indent:0pt; border:solid black 0.4pt; }
div.minipage{width:100%;}
div.center, div.center div.center {text-align: center; margin-left:1em; margin-right:1em;}
div.center div {text-align: left;}
div.flushright, div.flushright div.flushright {text-align: right;}
div.flushright div {text-align: left;}
div.flushleft {text-align: left;}
.underline{ text-decoration:underline; }
.underline img{ border-bottom: 1px solid black; margin-bottom:1pt; }
.framebox-c, .framebox-l, .framebox-r { padding-left:3.0pt; padding-right:3.0pt; text-indent:0pt; border:solid black 0.4pt; }
.framebox-c {text-align:center;}
.framebox-l {text-align:left;}
.framebox-r {text-align:right;}
span.thank-mark{ vertical-align: super }
span.footnote-mark sup.textsuperscript, span.footnote-mark a sup.textsuperscript{ font-size:80%; }
div.tabular, div.center div.tabular {text-align: center; margin-top:0.5em; margin-bottom:0.5em; }
table.tabular td p{margin-top:0em;}
table.tabular {margin-left: auto; margin-right: auto;}
div.td00{ margin-left:0pt; margin-right:0pt; }
div.td01{ margin-left:0pt; margin-right:5pt; }
div.td10{ margin-left:5pt; margin-right:0pt; }
div.td11{ margin-left:5pt; margin-right:5pt; }
table[rules] {border-left:solid black 0.4pt; border-right:solid black 0.4pt; }
td.td00{ padding-left:0pt; padding-right:0pt; }
td.td01{ padding-left:0pt; padding-right:5pt; }
td.td10{ padding-left:5pt; padding-right:0pt; }
td.td11{ padding-left:5pt; padding-right:5pt; }
table[rules] {border-left:solid black 0.4pt; border-right:solid black 0.4pt; }
.hline hr, .cline hr{ height : 1px; margin:0px; }
.tabbing-right {text-align:right;}
span.TEX {letter-spacing: -0.125em; }
span.TEX span.E{ position:relative;top:0.5ex;left:-0.0417em;}
a span.TEX span.E {text-decoration: none; }
span.LATEX span.A{ position:relative; top:-0.5ex; left:-0.4em; font-size:85%;}
span.LATEX span.TEX{ position:relative; left: -0.4em; }
div.float img, div.float .caption {text-align:center;}
div.figure img, div.figure .caption {text-align:center;}
.marginpar {width:20%; float:right; text-align:left; margin-left:auto; margin-top:0.5em; font-size:85%; text-decoration:underline;}
.marginpar p{margin-top:0.4em; margin-bottom:0.4em;}
.equation td{text-align:center; vertical-align:middle; }
td.eq-no{ width:5%; }
table.equation { width:100%; } 
div.math-display, div.par-math-display{text-align:center;}
math .texttt { font-family: monospace; }
math .textit { font-style: italic; }
math .textsl { font-style: oblique; }
math .textsf { font-family: sans-serif; }
math .textbf { font-weight: bold; }
.partToc a, .partToc, .likepartToc a, .likepartToc {line-height: 200%; font-weight:bold; font-size:110%;}
.chapterToc a, .chapterToc, .likechapterToc a, .likechapterToc, .appendixToc a, .appendixToc {line-height: 200%; font-weight:bold;}
.index-item, .index-subitem, .index-subsubitem {display:block}
.caption td.id{font-weight: bold; white-space: nowrap; }
table.caption {text-align:center;}
h1.partHead{text-align: center}
p.bibitem { text-indent: -2em; margin-left: 2em; margin-top:0.6em; margin-bottom:0.6em; }
p.bibitem-p { text-indent: 0em; margin-left: 2em; margin-top:0.6em; margin-bottom:0.6em; }
.paragraphHead, .likeparagraphHead { margin-top:2em; font-weight: bold;}
.subparagraphHead, .likesubparagraphHead { font-weight: bold;}
.quote {margin-bottom:0.25em; margin-top:0.25em; margin-left:1em; margin-right:1em; text-align:justify;}
.verse{white-space:nowrap; margin-left:2em}
div.maketitle {text-align:center;}
h2.titleHead{text-align:center;}
div.maketitle{ margin-bottom: 2em; }
div.author, div.date {text-align:center;}
div.thanks{text-align:left; margin-left:10%; font-size:85%; font-style:italic; }
div.author{white-space: nowrap;}
.quotation {margin-bottom:0.25em; margin-top:0.25em; margin-left:1em; }
h1.partHead{text-align: center}
.sectionToc, .likesectionToc {margin-left:2em;}
.subsectionToc, .likesubsectionToc {margin-left:4em;}
.subsubsectionToc, .likesubsubsectionToc {margin-left:6em;}
.frenchb-nbsp{font-size:75%;}
.frenchb-thinspace{font-size:75%;}
.figure img.graphics {margin-left:10%;}
/* end css.sty */

\title{Convergence des suites}
\author{}
\date{}

\begin{document}
\maketitle

\textbf{Warning: \href{http://www.math.union.edu/locate/jsMath}{jsMath}
requires JavaScript to process the mathematics on this page.\\ If your
browser supports JavaScript, be sure it is enabled.}

\begin{center}\rule{3in}{0.4pt}\end{center}

{[}\href{coursse36.html}{next}{]}
{[}\hyperref[tailcoursse35.html]{tail}{]}
{[}\href{coursch8.html\#coursse35.html}{up}{]}

\subsubsection{7.1 Convergence des suites}

\paragraph{7.1.1 Monotonie (suites à termes réels)}

Théorème~7.1.1 Soit (\{x\}\_\{n\}) une suite croissante de nombres
réels. Alors la suite est convergente si et seulement si~elle est
majorée. Dans ce cas on a \textbackslash{}mathop\{lim\}\{x\}\_\{n\}
=\textbackslash{}mathop\{ sup\}\{x\}\_\{n\}.

Démonstration On sait déjà que toute suite convergente est bornée, donc
majorée. Inversement, si la suite est majorée, soit l
=\textbackslash{}mathop\{ sup\}\{x\}\_\{n\} et ε \textgreater{} 0. Par
définition de la borne supérieure, il existe \{n\}\_\{0\} tel que l − ε
\textless{} \{x\}\_\{\{n\}\_\{0\}\} ≤ l. Pour n ≥ \{n\}\_\{0\}, on a l −
ε \textless{} \{x\}\_\{\{n\}\_\{0\}\} ≤ \{x\}\_\{n\} ≤ l ce qui montre
que la suite converge vers l.

Remarque~7.1.1 On a un résultat analogue avec les suites décroissantes
et minorées.

Corollaire~7.1.2 Soit (\{a\}\_\{n\}) et (\{b\}\_\{n\}) deux suites de
nombres réels vérifiant

\begin{itemize}
\itemsep1pt\parskip0pt\parsep0pt
\item
  (i) (\{a\}\_\{n\}) est croissante et (\{b\}\_\{n\}) décroissante
\item
  (ii) \textbackslash{}mathop\{∀\}n ∈ ℕ, \{a\}\_\{n\} ≤ \{b\}\_\{n\}
\item
  (iii) \textbackslash{}mathop\{lim\}(\{b\}\_\{n\} − \{a\}\_\{n\}) = 0
\end{itemize}

Alors les suites (\{a\}\_\{n\}) et (\{b\}\_\{n\}) convergent et ont la
même limite ℓ qui vérifie

\textbackslash{}mathop\{∀\}n ∈ ℕ, \{a\}\_\{n\} ≤ ℓ ≤ \{b\}\_\{n\}

On dit que deux telles suites sont adjacentes.

Démonstration On remarque que \textbackslash{}mathop\{∀\}p,q,
\{a\}\_\{p\} ≤ \{b\}\_\{q\}~; en effet si p ≤ q on a \{a\}\_\{p\} ≤
\{a\}\_\{q\} ≤ \{b\}\_\{q\} et si p \textgreater{} q, on a \{a\}\_\{p\}
≤ \{b\}\_\{p\} ≤ \{a\}\_\{q\}. La suite (\{a\}\_\{n\}) est croissante
majorée par \{b\}\_\{0\}, donc converge. De même la suite (\{b\}\_\{n\})
converge et la propriété (iii) implique qu'elles ont la même limite.

Exemple~7.1.1 Posons \{u\}\_\{n\} = 1 +\{ 1 \textbackslash{}over 2\} +
\textbackslash{}mathop\{\textbackslash{}mathop\{\ldots{}\}\} +\{ 1
\textbackslash{}over n\} −\textbackslash{}mathop\{ log\} n. On a
\{u\}\_\{n+1\} − \{u\}\_\{n\} =\{ 1 \textbackslash{}over n+1\}
−\textbackslash{}mathop\{ log\} (n + 1) −\textbackslash{}mathop\{ log\}
n =\{ 1 \textbackslash{}over n+1\} −\{\textbackslash{}mathop\{∫ \}
\}\_\{n\}\^{}\{n+1\}\{ dt \textbackslash{}over t\}
=\{\textbackslash{}mathop\{∫ \} \}\_\{n\}\^{}\{n+1\}(\{ 1
\textbackslash{}over n+1\} −\{ 1 \textbackslash{}over t\} ) dt ≤ 0.
Posons \{v\}\_\{n\} = 1 +\{ 1 \textbackslash{}over 2\} +
\textbackslash{}mathop\{\textbackslash{}mathop\{\ldots{}\}\} +\{ 1
\textbackslash{}over n−1\} −\textbackslash{}mathop\{ log\} n. On a de
même \{v\}\_\{n+1\} − \{v\}\_\{n\} =\{\textbackslash{}mathop\{∫ \}
\}\_\{n\}\^{}\{n+1\}(\{ 1 \textbackslash{}over n\} −\{ 1
\textbackslash{}over t\} ) dt ≥ 0. On a donc (\{u\}\_\{n\})
décroissante, (\{v\}\_\{n\}) croissante, \{v\}\_\{n\} ≤ \{u\}\_\{n\},
\textbackslash{}mathop\{lim\}(\{v\}\_\{n\} − \{u\}\_\{n\}) = 0. Donc les
suites convergent. Soit γ leur limite commune (la constante d'Euler). On
a donc 1 +\{ 1 \textbackslash{}over 2\} +
\textbackslash{}mathop\{\textbackslash{}mathop\{\ldots{}\}\} +\{ 1
\textbackslash{}over n\} =\textbackslash{}mathop\{ log\} n + γ +
\{ε\}\_\{n\} avec \textbackslash{}mathop\{lim\}\{ε\}\_\{n\} = 0.

\paragraph{7.1.2 Critère de Cauchy}

Dans un espace métrique complet, une suite converge si et seulement
si~elle vérifie le critère de Cauchy. Cela peut servir aussi bien comme
critère de convergence (exemple des suites \{x\}\_\{n+1\} =
f(\{x\}\_\{n\}) où f est contractante) que comme critère de divergence.

Exemple~7.1.2 Posons \{x\}\_\{n\} = 1 +\{ 1 \textbackslash{}over 2\} +
\textbackslash{}mathop\{\textbackslash{}mathop\{\ldots{}\}\} +\{ 1
\textbackslash{}over n\} . On a \{x\}\_\{2n\} − \{x\}\_\{n\} =\{ 1
\textbackslash{}over n+1\} +
\textbackslash{}mathop\{\textbackslash{}mathop\{\ldots{}\}\} +\{ 1
\textbackslash{}over 2n\} ≥ n ×\{ 1 \textbackslash{}over 2n\} =\{ 1
\textbackslash{}over 2\} . La suite ne vérifie donc pas le critère de
Cauchy (bien que \textbackslash{}mathop\{lim\}(\{x\}\_\{n+1\} −
\{x\}\_\{n\}) = 0), donc elle ne converge pas.

\paragraph{7.1.3 Valeurs d'adhérences, limites inférieures et
supérieures}

Proposition~7.1.3 Soit E un espace métrique et (\{x\}\_\{n\}) une suite
de E. L'ensemble de ses valeurs d'adhérences est fermé dans E.

Démonstration On a vu dans le chapitre sur les compacts que l'ensemble X
des valeurs d'adhérences de la suite (\{x\}\_\{n\}) est
\{\textbackslash{}mathop\{\textbackslash{}mathop\{⋂ \}\}
\}\_\{N∈ℕ\}\textbackslash{}overline\{\{X\}\_\{N\}\} avec \{X\}\_\{N\} =
\textbackslash{}\{\{x\}\_\{n\}\textbackslash{}mathrel\{∣\}n ≥
N\textbackslash{}\}. Comme intersection de fermés, c'est un fermé. On
peut aussi le redémontrer directement. Soit x
∈\textbackslash{}overline\{X\} et V ∈ V (x). Soit U ouvert tel que x ∈ U
⊂ V . On a U ∩ X\textbackslash{}mathrel\{≠\}∅. Soit y ∈ U ∩ X. Comme U
est ouvert, U est un voisinage de la valeur d'adhérence y et donc
\textbackslash{}\{n ∈ ℕ\textbackslash{}mathrel\{∣\}\{x\}\_\{n\} ∈
U\textbackslash{}\} est infini~; il en est de même a fortiori de
\textbackslash{}\{n ∈ ℕ\textbackslash{}mathrel\{∣\}\{x\}\_\{n\} ∈ V
\textbackslash{}\}, donc x est encore valeur d'adhérence de la suite.

Théorème~7.1.4 Soit E un espace métrique compact et (\{x\}\_\{n\}) une
suite de E.

\begin{itemize}
\itemsep1pt\parskip0pt\parsep0pt
\item
  (i) La suite a au moins une valeur d'adhérence
\item
  (ii) La suite converge si et seulement si~elle a une unique valeur
  d'adhérence.
\end{itemize}

Démonstration L'affirmation (i) n'est autre que la définition d'un
compact.

(ii) La condition est évidemment nécessaire. Supposons la remplie et
soit ℓ cette unique valeur d'adhérence. Supposons que ℓ n'est pas limite
de la suite. Ceci signifie qu'il existe U ouvert contenant ℓ tel que
\textbackslash{}mathop\{∀\}N ∈ ℕ, \textbackslash{}mathop\{∃\}n ≥ N tel
que \{x\}\_\{n\}\textbackslash{}mathrel\{∉\}U. On construit ainsi
facilement une sous suite (\{x\}\_\{φ(n)\}) telle que
\textbackslash{}mathop\{∀\}n, \{x\}\_\{φ(n)\} ∈ E ∖ U (prendre φ(n) le
plus petit entier supérieur à N = φ(n − 1) + 1 vérifiant la condition).
Comme E ∖ U est fermé dans un compact, c'est un compact et la suite
(\{x\}\_\{φ(n)\}) doit avoir une valeur d'adhérence ℓ' ∈ E ∖ U. Mais
alors la suite (\{x\}\_\{n\}) a deux valeurs d'adhérences
ℓ\textbackslash{}mathrel\{≠\}ℓ'. C'est absurde.

Soit donc (\{x\}\_\{n\}) une suite de \textbackslash{}overline\{ℝ\}.
Soit X l'ensemble de ses valeurs d'adhérences. C'est un fermé non vide
de \textbackslash{}overline\{ℝ\}, donc il contient sa borne supérieure
et sa borne inférieure.

Définition~7.1.1 Soit (\{x\}\_\{n\}) une suite de
\textbackslash{}overline\{ℝ\}. Soit X l'ensemble de ses valeurs
d'adhérences. On pose \textbackslash{}mathop\{limsup\}\{x\}\_\{n\}
=\textbackslash{}mathop\{ max\}X et \textbackslash{}mathop\{liminf\}
\{x\}\_\{n\} =\textbackslash{}mathop\{ min\}X. La suite converge (dans
\textbackslash{}overline\{ℝ\}) si et seulement
si~\textbackslash{}mathop\{limsup\}\{x\}\_\{n\}
=\textbackslash{}mathop\{ liminf\} \{x\}\_\{n\}.

Théorème~7.1.5 Soit (\{x\}\_\{n\}) une suite de
\textbackslash{}overline\{ℝ\} et ℓ ∈\textbackslash{}overline\{ℝ\}. On a
équivalence de

\begin{itemize}
\itemsep1pt\parskip0pt\parsep0pt
\item
  (i) ℓ =\textbackslash{}mathop\{ limsup\}\{x\}\_\{n\}
\item
  (ii) ℓ est valeur d'adhérence de la suite et
  \textbackslash{}mathop\{∀\}c \textgreater{} ℓ, \textbackslash{}\{n ∈
  ℕ\textbackslash{}mathrel\{∣\}\{x\}\_\{n\} ≥ c\textbackslash{}\} est
  fini.
\item
  (iii) ℓ =\{\textbackslash{}mathop\{
  lim\}\}\_\{p→+∞\}(\{\textbackslash{}mathop\{sup\}\}\_\{n≥p\}\{x\}\_\{n\})
\end{itemize}

Démonstration (i) ⇒(ii) Soit ℓ =\textbackslash{}mathop\{
limsup\}\{x\}\_\{n\}. Alors ℓ est valeur d'adhérence de la suite. Si
\textbackslash{}\{n ∈ ℕ\textbackslash{}mathrel\{∣\}\{x\}\_\{n\} ≥
c\textbackslash{}\} est infini, on peut construire une sous suite dans
{[}c,+∞{]} qui est compact~; cette suite doit admettre une valeur
d'adhérence ℓ' ∈ {[}c,+∞{]}. On a donc ℓ' ∈ X avec ℓ'
\textgreater{}\textbackslash{}mathop\{ sup\}X. C'est absurde.

(ii) ⇒(iii) Remarquons que la suite \{y\}\_\{p\}
=\{\textbackslash{}mathop\{ sup\}\}\_\{n≥p\}\{x\}\_\{n\} est
décroissante, donc convergente dans \textbackslash{}overline\{ℝ\}. Soit
ℓ' sa limite. Soit c \textgreater{} ℓ. Il existe N ∈ ℕ tel que n ≥ N ⇒
\{x\}\_\{n\} \textless{} c. Donc pour n ≥ N, on a \{y\}\_\{n\} ≤ c et
donc ℓ' ≤ c. Comme c est quelconque ( \textgreater{} ℓ), on a ℓ' ≤ ℓ.
Mais d'autre part on sait que ℓ est valeur d'adhérence de la suite
(\{x\}\_\{n\}) d'où ℓ =\textbackslash{}mathop\{ lim\}\{x\}\_\{φ(n)\}
≤\textbackslash{}mathop\{ lim\}\{y\}\_\{φ(n)\} = ℓ'. Donc ℓ = ℓ'.

(iii) ⇒(i) Posons toujours \{y\}\_\{p\} =\{\textbackslash{}mathop\{
sup\}\}\_\{n≥p\}\{x\}\_\{n\}. Si ℓ' est une valeur d'adhérence de la
suite (\{x\}\_\{n\}), on a ℓ' =\textbackslash{}mathop\{
lim\}\{x\}\_\{φ(n)\} ≤\textbackslash{}mathop\{ lim\}\{y\}\_\{φ(n)\} = ℓ,
donc \textbackslash{}mathop\{limsup\}\{x\}\_\{n\} ≤ ℓ. Mais d'autre
part, soit U un ouvert contenant ℓ, on peut trouver un N tel que p ≥ N ⇒
\{y\}\_\{p\} ∈ U. Pour un tel p, comme U ∈ V (\{y\}\_\{p\}), on peut
trouver un n ≥ p tel que \{x\}\_\{n\} ∈ U. Ceci montre que ℓ est valeur
d'adhérence de la suite (\{x\}\_\{n\}) soit ℓ ≤\textbackslash{}mathop\{
limsup\}\{x\}\_\{n\} et donc l'égalité.

Proposition~7.1.6

\begin{itemize}
\itemsep1pt\parskip0pt\parsep0pt
\item
  (i) \textbackslash{}mathop\{limsup\}(\{u\}\_\{n\} + \{v\}\_\{n\})
  ≤\textbackslash{}mathop\{ limsup\}\{u\}\_\{n\}
  +\textbackslash{}mathop\{ limsup\}\{v\}\_\{n\} (avec égalité si l'une
  des suites est convergente)
\item
  (ii) si (\{u\}\_\{n\}) et (\{v\}\_\{n\}) sont deux suites positives,
  \textbackslash{}mathop\{limsup\}(\{u\}\_\{n\}\{v\}\_\{n\})
  ≤\textbackslash{}mathop\{ limsup\}\{u\}\_\{n\}\textbackslash{}mathop\{
  limsup\}\{v\}\_\{n\} (avec égalité si l'une des suites est
  convergente)
\item
  (iii) si λ \textgreater{} 0,
  \textbackslash{}mathop\{limsup\}(λ\{x\}\_\{n\}) =
  λ\textbackslash{}mathop\{limsup\}\{x\}\_\{n\}
\item
  (iv) si f est continue,
  f(\textbackslash{}mathop\{limsup\}\{x\}\_\{n\})
  ≤\textbackslash{}mathop\{ limsup\}f(\{x\}\_\{n\}) (avec égalité si f
  est croissante)
\end{itemize}

Démonstration (i) On pose ℓ =\textbackslash{}mathop\{
limsup\}\{u\}\_\{n\}, v =\textbackslash{}mathop\{ limsup\}\{v\}\_\{n\}.
Soit ε \textgreater{} 0. Il existe N ∈ ℕ tel que n ≥ N ⇒ \{u\}\_\{n\}
\textless{} ℓ + ε. De même, il existe N' tel que n ≥ N' ⇒ \{v\}\_\{n\}
\textless{} ℓ' + ε. Alors n ≥\textbackslash{}mathop\{ max\}(N,N') ⇒
\{u\}\_\{n\} + \{v\}\_\{n\} \textless{} ℓ + ℓ' + 2ε, ce qui montre que
\textbackslash{}mathop\{limsup\}(\{u\}\_\{n\} + \{v\}\_\{n\}) ≤ ℓ + ℓ'.
Si la suite \{u\}\_\{n\} converge, on a par exemple ℓ'
=\textbackslash{}mathop\{ lim\}\{v\}\_\{φ(n)\}, d'où ℓ + ℓ'
=\textbackslash{}mathop\{ lim\}(\{u\}\_\{φ(n)\} + \{v\}\_\{φ(n)\}) est
encore valeur d'adhérence de la suite (\{u\}\_\{n\} + \{v\}\_\{n\})~;
donc \textbackslash{}mathop\{limsup\}(\{u\}\_\{n\} + \{v\}\_\{n\}) = ℓ +
ℓ'. La démonstration de (ii) est tout à fait similaire.

(iii) est tout à fait élémentaire.

(iv) soit ℓ =\textbackslash{}mathop\{ limsup\}\{x\}\_\{n\}. On a ℓ
=\textbackslash{}mathop\{ lim\}\{x\}\_\{φ(n)\}, donc f(ℓ)
=\textbackslash{}mathop\{ lim\}f(\{x\}\_\{φ(n)\}) est valeur d'adhérence
de la suite (f(\{x\}\_\{n\})). On en déduit que f(ℓ)
≤\textbackslash{}mathop\{ limsup\}f(\{x\}\_\{n\}). Supposons maintenant
f croissante et supposons que f(ℓ) \textless{}\textbackslash{}mathop\{
limsup\}f(\{x\}\_\{n\}) = ℓ'. Soit α tel que f(ℓ) \textless{} α
\textless{} ℓ'. Le réel ℓ' est valeur d'adhérence de la suite
f(\{x\}\_\{n\}), donc on peut trouver N tel que f(\{x\}\_\{N\})
\textgreater{} α(\textgreater{} f(ℓ)). Le théorème des valeurs
intermédiaires assure qu'il existe a tel que α = f(a). Comme f est
croissante, on a a \textgreater{} ℓ. On a ℓ =\textbackslash{}mathop\{
lim\}f(\{x\}\_\{φ(n)\}) donc il existe N' tel que n ≥ N' ⇒
f(\{x\}\_\{φ(n)\}) \textgreater{} α = f(a). Mais alors n ≥ N' ⇒
\{x\}\_\{φ(n)\} \textgreater{} a \textgreater{} ℓ. Ceci contredit le
fait qu'il n'y a qu'un nombre fini de n tels que \{x\}\_\{n\}
\textgreater{} a. On a donc f(ℓ) = ℓ'.

Remarque~7.1.2 L'exemple \{u\}\_\{n\} = \{(−1)\}\^{}\{n\}, \{v\}\_\{n\}
= −\{u\}\_\{n\} montre que l'on n'a pas généralement d'égalité dans (i).
En ce qui concerne (iii), si λ \textless{} 0 on a évidemment
\textbackslash{}mathop\{limsup\}(λ\{x\}\_\{n\}) =
λ\textbackslash{}mathop\{liminf\} \{x\}\_\{n\}. De même pour (iv), si f
est décroissante, on a f(\textbackslash{}mathop\{limsup\}\{x\}\_\{n\})
=\textbackslash{}mathop\{ liminf\} f(\{x\}\_\{n\}), ce qui montre qu'en
général on n'a pas d'égalité dans (iv).

Les résultats concernant la limite inférieure sont tout à fait
similaires, les inégalités changeant de sens

Exemple~7.1.3 Soit f :{]}0,+∞{[}→{]}0,+∞{[} continue croissante~; on
suppose que l'équation f(x) =\{ x \textbackslash{}over 2\} a une unique
solution ℓ, que x \textless{} ℓ ⇒ f(x) \textgreater{}\{ x
\textbackslash{}over 2\} et x \textgreater{} ℓ ⇒ f(x) \textless{}\{ x
\textbackslash{}over 2\} ~; on considère la suite (\{x\}\_\{n\}) définie
par \{x\}\_\{n+1\} = f(\{x\}\_\{n\}) + f(\{x\}\_\{n−1\}). On vérifie
facilement que si a =\textbackslash{}mathop\{
min\}(ℓ,\{x\}\_\{0\},\{x\}\_\{1\}), b =\textbackslash{}mathop\{
max\}(ℓ,\{x\}\_\{0\},\{x\}\_\{1\}), alors \textbackslash{}mathop\{∀\}n ∈
ℕ, \{x\}\_\{n\} ∈ {[}a,b{]}. Posons M =\textbackslash{}mathop\{
limsup\}\{x\}\_\{n\} et m =\textbackslash{}mathop\{ liminf\}
\{x\}\_\{n\}. On a alors M =\textbackslash{}mathop\{
limsup\}(f(\{x\}\_\{n−1\}) + f(\{x\}\_\{n−2\}))
≤\textbackslash{}mathop\{ limsup\}f(\{x\}\_\{n−1\})
+\textbackslash{}mathop\{ limsup\}f(\{x\}\_\{n−2\}) = 2f(M). On en
déduit que M ≤ ℓ. On montre de même que m ≥ ℓ d'où m = M = ℓ et la suite
converge.

\paragraph{7.1.4 Récurrences d'ordre 1}

Soit D une partie de ℝ et f : D → ℝ une fonction continue. On considère
\{x\}\_\{0\} ∈ D et la suite (\{x\}\_\{n\}) définie par récurrence par
\{x\}\_\{n+1\} = f(\{x\}\_\{n\}). On note D' =
\textbackslash{}\{\{x\}\_\{0\} ∈
D\textbackslash{}mathrel\{∣\}\{(\{x\}\_\{n\})\}\_\{n∈ℕ\}\textbackslash{}text\{
est définie \}\textbackslash{}\} (on montre facilement que D'
=\{\textbackslash{}mathop\{ \textbackslash{}mathop\{⋃ \}\}
\}\_\{A⊂D,f(A)⊂A\}A). On remarque immédiatement que D contient tous les
points fixes de f.

Proposition~7.1.7 Si la suite (\{x\}\_\{n\}) converge vers un point ℓ ∈
D, alors f(ℓ) = ℓ.

Démonstration On a alors ℓ =\textbackslash{}mathop\{ lim\}\{x\}\_\{n+1\}
=\textbackslash{}mathop\{ lim\}f(\{x\}\_\{n\}) =
f(\textbackslash{}mathop\{lim\}\{x\}\_\{n\}) = f(ℓ) par continuité de f
au point ℓ.

Proposition~7.1.8 Soit ℓ ∈ \{D\}\^{}\{o\} tel que f(ℓ) = ℓ et supposons
f dérivable au point ℓ.

\begin{itemize}
\itemsep1pt\parskip0pt\parsep0pt
\item
  (i) Si \textbar{}f'(ℓ)\textbar{} \textless{} 1 (point fixe attractif),
  il existe un η \textgreater{} 0 tel que

  \begin{itemize}
  \itemsep1pt\parskip0pt\parsep0pt
  \item
    (a) f({]}ℓ − η,ℓ + η{[}) ⊂{]}ℓ − η,ℓ + η{[}⊂ D'
  \item
    (b) \textbackslash{}left (\textbackslash{}mathop\{∃\}\{n\}\_\{0\} ∈
    ℕ, \{x\}\_\{\{n\}\_\{0\}\} ∈{]}ℓ − η,ℓ + η{[}\textbackslash{}right )
    ⇒\textbackslash{}mathop\{ lim\}\{x\}\_\{n\} = ℓ
  \end{itemize}
\item
  (ii) Si \textbar{}f'(ℓ)\textbar{} \textgreater{} 1 (point fixe
  répulsif) et si \textbackslash{}mathop\{lim\}\{x\}\_\{n\} = ℓ, alors
  la suite est stationnaire en ℓ.
\end{itemize}

Démonstration (i) Soit k tel que \textbar{}f'(ℓ)\textbar{} \textless{} k
\textless{} 1. Comme

\{\textbackslash{}mathop\{lim\}\}\_\{x→ℓ,x\textbackslash{}mathrel\{≠\}ℓ\}\textbackslash{}left
\textbar{}\{ f(x) − f(ℓ) \textbackslash{}over x − ℓ\}
\textbackslash{}right \textbar{} =\{\textbackslash{}mathop\{
lim\}\}\_\{x→ℓ,x\textbackslash{}mathrel\{≠\}ℓ\}\textbackslash{}left
\textbar{}\{ f(x) − ℓ \textbackslash{}over x − ℓ\} \textbackslash{}right
\textbar{} = \textbar{}f'(ℓ)\textbar{} \textless{} k

il existe η \textgreater{} 0 tel que \textbar{}x − ℓ\textbar{}
\textless{} η ⇒\textbar{}f(x) − ℓ\textbar{}≤ k\textbar{}x − ℓ\textbar{}.
On a alors évidemment f({]}ℓ − η,ℓ + η{[}) ⊂{]}ℓ − η,ℓ + η{[}⊂ D'. Soit
\{n\}\_\{0\} tel que \{x\}\_\{\{n\}\_\{0\}\} ∈{]}ℓ − η,ℓ + η{[}. Alors
pour tout n ≥ \{n\}\_\{0\} on a \{x\}\_\{n\} ∈{]}ℓ − η,ℓ + η{[} et
\textbar{}\{x\}\_\{n+1\} − ℓ\textbar{}≤ k\textbar{}\{x\}\_\{n\} −
ℓ\textbar{}. On a alors \textbar{}\{x\}\_\{n\} − ℓ\textbar{}≤
\{k\}\^{}\{n−\{n\}\_\{0\}\}\textbar{}\{x\}\_\{\{n\}\_\{ 0\}\} −
ℓ\textbar{} ce qui montre que \textbackslash{}mathop\{lim\}\{x\}\_\{n\}
= ℓ.

(ii) Une méthode similaire montre que si \textbar{}f'(ℓ)\textbar{}
\textgreater{} k \textgreater{} 1, alors il existe η \textgreater{} 0
tel que \textbar{}x − ℓ\textbar{} \textless{} η ⇒\textbar{}f(x) −
ℓ\textbar{}≥ k\textbar{}x − ℓ\textbar{}. Si
\textbackslash{}mathop\{lim\}\{x\}\_\{n\} = ℓ, il existe \{n\}\_\{0\}
tel que n ≥ \{n\}\_\{0\} ⇒\textbar{}\{x\}\_\{n\} − ℓ\textbar{}
\textless{} η. On a alors \textbar{}\{x\}\_\{n+1\} − ℓ\textbar{}≥
k\textbar{}\{x\}\_\{n\} − ℓ\textbar{}, soit encore
\textbar{}\{x\}\_\{n\} − ℓ\textbar{}≥
\{k\}\^{}\{n−\{n\}\_\{0\}\}\textbar{}\{x\}\_\{\{n\}\_\{ 0\}\} −
ℓ\textbar{} avec k \textgreater{} 1. Ce n'est compatible avec le fait
que \{x\}\_\{n\} − ℓ tend vers 0 que si \{x\}\_\{\{n\}\_\{0\}\} − ℓ = 0,
et la suite est alors stationnaire.

Les deux propositions précédentes permettent de conclure dans un certain
nombre de cas. Une étude plus fine relève en général de propriétés de
monotonie de la fonction f.

Proposition~7.1.9 Soit I un intervalle stable par f sur lequel f est
monotone. On suppose qu'il existe \{n\}\_\{0\} ∈ ℕ tel que
\{x\}\_\{\{n\}\_\{0\}\} ∈ I. Alors \textbackslash{}mathop\{∀\}n ≥
\{n\}\_\{0\}, \{x\}\_\{n\} ∈ I et de plus

\begin{itemize}
\itemsep1pt\parskip0pt\parsep0pt
\item
  (i) si f est croissante sur I, la suite
  \{(\{x\}\_\{n\})\}\_\{n≥\{n\}\_\{0\}\} est monotone (le sens étant
  déterminé par le signe de \{x\}\_\{\{n\}\_\{0\}+1\} −
  \{x\}\_\{\{n\}\_\{0\}\} = f(\{x\}\_\{\{n\}\_\{0\}\}) −
  \{x\}\_\{\{n\}\_\{0\}\})
\item
  (ii) si f est décroissante sur I, les deux sous suites (\{x\}\_\{2n\})
  et (\{x\}\_\{2n+1\}) sont monotones et de sens contraire à partir de
  l'indice \{n\}\_\{0\}.
\end{itemize}

Démonstration Supposons f croissante et par exemple
f(\{x\}\_\{\{n\}\_\{0\}\}) = \{x\}\_\{\{n\}\_\{0\}+1\} ≤
\{x\}\_\{\{n\}\_\{0\}\}, alors \{x\}\_\{n\} ≤ \{x\}\_\{n−1\} ⇒
f(\{x\}\_\{n\}) ≤ f(\{x\}\_\{n−1\}) ⇒ \{x\}\_\{n+1\} ≤ \{x\}\_\{n\} ce
qui montre par récurrence que \textbackslash{}mathop\{∀\}n ≥
\{n\}\_\{0\}, \{x\}\_\{n+1\} ≤ \{x\}\_\{n\} et la suite est décroissante
à partir de \{n\}\_\{0\}. De même, si f(\{x\}\_\{\{n\}\_\{0\}\}) =
\{x\}\_\{\{n\}\_\{0\}+1\} ≥ \{x\}\_\{\{n\}\_\{0\}\}, la suite est
croissante à partir de \{n\}\_\{0\}. Supposons maintenant f décroissante
sur I et f(I) ⊂ I. Alors f ∘ f est croissante sur I et donc les deux
sous suites (\{x\}\_\{2n\}) et (\{x\}\_\{2n+1\}) sont monotones, car
elles vérifient la relation \{y\}\_\{n+1\} = f ∘ f(\{y\}\_\{n\}). De
plus elles sont de sens contraire car \{x\}\_\{2n+3\} − \{x\}\_\{2n+1\}
= f(\{x\}\_\{2n+2\}) − f(\{x\}\_\{2n\}) et f est décroissante.

Remarque~7.1.3 Supposons que l'on est dans la situation de la
proposition avec f croissante~; soit ℓ ∈ I tel que f(ℓ) = ℓ. On constate
immédiatement que le signe de ℓ − \{x\}\_\{n\} = f(ℓ) −
f(\{x\}\_\{n−1\}) est constant, si bien que ℓ fournit soit un majorant,
soit un minorant de la suite.

{[}\href{coursse36.html}{next}{]} {[}\href{coursse35.html}{front}{]}
{[}\href{coursch8.html\#coursse35.html}{up}{]}

\end{document}

% \documentclass[]{article}
\usepackage[T1]{fontenc}
\usepackage{lmodern}
\usepackage{amssymb,amsmath}
\usepackage{ifxetex,ifluatex}
\usepackage{fixltx2e} % provides \textsubscript
% use upquote if available, for straight quotes in verbatim environments
\IfFileExists{upquote.sty}{\usepackage{upquote}}{}
\ifnum 0\ifxetex 1\fi\ifluatex 1\fi=0 % if pdftex
  \usepackage[utf8]{inputenc}
\else % if luatex or xelatex
  \ifxetex
    \usepackage{mathspec}
    \usepackage{xltxtra,xunicode}
  \else
    \usepackage{fontspec}
  \fi
  \defaultfontfeatures{Mapping=tex-text,Scale=MatchLowercase}
  \newcommand{\euro}{€}
\fi
% use microtype if available
\IfFileExists{microtype.sty}{\usepackage{microtype}}{}
\ifxetex
  \usepackage[setpagesize=false, % page size defined by xetex
              unicode=false, % unicode breaks when used with xetex
              xetex]{hyperref}
\else
  \usepackage[unicode=true]{hyperref}
\fi
\hypersetup{breaklinks=true,
            bookmarks=true,
            pdfauthor={},
            pdftitle={Generalites sur les series},
            colorlinks=true,
            citecolor=blue,
            urlcolor=blue,
            linkcolor=magenta,
            pdfborder={0 0 0}}
\urlstyle{same}  % don't use monospace font for urls
\setlength{\parindent}{0pt}
\setlength{\parskip}{6pt plus 2pt minus 1pt}
\setlength{\emergencystretch}{3em}  % prevent overfull lines
\setcounter{secnumdepth}{0}
 
/* start css.sty */
.cmr-5{font-size:50%;}
.cmr-7{font-size:70%;}
.cmmi-5{font-size:50%;font-style: italic;}
.cmmi-7{font-size:70%;font-style: italic;}
.cmmi-10{font-style: italic;}
.cmsy-5{font-size:50%;}
.cmsy-7{font-size:70%;}
.cmex-7{font-size:70%;}
.cmex-7x-x-71{font-size:49%;}
.msbm-7{font-size:70%;}
.cmtt-10{font-family: monospace;}
.cmti-10{ font-style: italic;}
.cmbx-10{ font-weight: bold;}
.cmr-17x-x-120{font-size:204%;}
.cmsl-10{font-style: oblique;}
.cmti-7x-x-71{font-size:49%; font-style: italic;}
.cmbxti-10{ font-weight: bold; font-style: italic;}
p.noindent { text-indent: 0em }
td p.noindent { text-indent: 0em; margin-top:0em; }
p.nopar { text-indent: 0em; }
p.indent{ text-indent: 1.5em }
@media print {div.crosslinks {visibility:hidden;}}
a img { border-top: 0; border-left: 0; border-right: 0; }
center { margin-top:1em; margin-bottom:1em; }
td center { margin-top:0em; margin-bottom:0em; }
.Canvas { position:relative; }
li p.indent { text-indent: 0em }
.enumerate1 {list-style-type:decimal;}
.enumerate2 {list-style-type:lower-alpha;}
.enumerate3 {list-style-type:lower-roman;}
.enumerate4 {list-style-type:upper-alpha;}
div.newtheorem { margin-bottom: 2em; margin-top: 2em;}
.obeylines-h,.obeylines-v {white-space: nowrap; }
div.obeylines-v p { margin-top:0; margin-bottom:0; }
.overline{ text-decoration:overline; }
.overline img{ border-top: 1px solid black; }
td.displaylines {text-align:center; white-space:nowrap;}
.centerline {text-align:center;}
.rightline {text-align:right;}
div.verbatim {font-family: monospace; white-space: nowrap; text-align:left; clear:both; }
.fbox {padding-left:3.0pt; padding-right:3.0pt; text-indent:0pt; border:solid black 0.4pt; }
div.fbox {display:table}
div.center div.fbox {text-align:center; clear:both; padding-left:3.0pt; padding-right:3.0pt; text-indent:0pt; border:solid black 0.4pt; }
div.minipage{width:100%;}
div.center, div.center div.center {text-align: center; margin-left:1em; margin-right:1em;}
div.center div {text-align: left;}
div.flushright, div.flushright div.flushright {text-align: right;}
div.flushright div {text-align: left;}
div.flushleft {text-align: left;}
.underline{ text-decoration:underline; }
.underline img{ border-bottom: 1px solid black; margin-bottom:1pt; }
.framebox-c, .framebox-l, .framebox-r { padding-left:3.0pt; padding-right:3.0pt; text-indent:0pt; border:solid black 0.4pt; }
.framebox-c {text-align:center;}
.framebox-l {text-align:left;}
.framebox-r {text-align:right;}
span.thank-mark{ vertical-align: super }
span.footnote-mark sup.textsuperscript, span.footnote-mark a sup.textsuperscript{ font-size:80%; }
div.tabular, div.center div.tabular {text-align: center; margin-top:0.5em; margin-bottom:0.5em; }
table.tabular td p{margin-top:0em;}
table.tabular {margin-left: auto; margin-right: auto;}
div.td00{ margin-left:0pt; margin-right:0pt; }
div.td01{ margin-left:0pt; margin-right:5pt; }
div.td10{ margin-left:5pt; margin-right:0pt; }
div.td11{ margin-left:5pt; margin-right:5pt; }
table[rules] {border-left:solid black 0.4pt; border-right:solid black 0.4pt; }
td.td00{ padding-left:0pt; padding-right:0pt; }
td.td01{ padding-left:0pt; padding-right:5pt; }
td.td10{ padding-left:5pt; padding-right:0pt; }
td.td11{ padding-left:5pt; padding-right:5pt; }
table[rules] {border-left:solid black 0.4pt; border-right:solid black 0.4pt; }
.hline hr, .cline hr{ height : 1px; margin:0px; }
.tabbing-right {text-align:right;}
span.TEX {letter-spacing: -0.125em; }
span.TEX span.E{ position:relative;top:0.5ex;left:-0.0417em;}
a span.TEX span.E {text-decoration: none; }
span.LATEX span.A{ position:relative; top:-0.5ex; left:-0.4em; font-size:85%;}
span.LATEX span.TEX{ position:relative; left: -0.4em; }
div.float img, div.float .caption {text-align:center;}
div.figure img, div.figure .caption {text-align:center;}
.marginpar {width:20%; float:right; text-align:left; margin-left:auto; margin-top:0.5em; font-size:85%; text-decoration:underline;}
.marginpar p{margin-top:0.4em; margin-bottom:0.4em;}
.equation td{text-align:center; vertical-align:middle; }
td.eq-no{ width:5%; }
table.equation { width:100%; } 
div.math-display, div.par-math-display{text-align:center;}
math .texttt { font-family: monospace; }
math .textit { font-style: italic; }
math .textsl { font-style: oblique; }
math .textsf { font-family: sans-serif; }
math .textbf { font-weight: bold; }
.partToc a, .partToc, .likepartToc a, .likepartToc {line-height: 200%; font-weight:bold; font-size:110%;}
.chapterToc a, .chapterToc, .likechapterToc a, .likechapterToc, .appendixToc a, .appendixToc {line-height: 200%; font-weight:bold;}
.index-item, .index-subitem, .index-subsubitem {display:block}
.caption td.id{font-weight: bold; white-space: nowrap; }
table.caption {text-align:center;}
h1.partHead{text-align: center}
p.bibitem { text-indent: -2em; margin-left: 2em; margin-top:0.6em; margin-bottom:0.6em; }
p.bibitem-p { text-indent: 0em; margin-left: 2em; margin-top:0.6em; margin-bottom:0.6em; }
.subsectionHead, .likesubsectionHead { margin-top:2em; font-weight: bold;}
.sectionHead, .likesectionHead { font-weight: bold;}
.quote {margin-bottom:0.25em; margin-top:0.25em; margin-left:1em; margin-right:1em; text-align:justify;}
.verse{white-space:nowrap; margin-left:2em}
div.maketitle {text-align:center;}
h2.titleHead{text-align:center;}
div.maketitle{ margin-bottom: 2em; }
div.author, div.date {text-align:center;}
div.thanks{text-align:left; margin-left:10%; font-size:85%; font-style:italic; }
div.author{white-space: nowrap;}
.quotation {margin-bottom:0.25em; margin-top:0.25em; margin-left:1em; }
h1.partHead{text-align: center}
.sectionToc, .likesectionToc {margin-left:2em;}
.subsectionToc, .likesubsectionToc {margin-left:4em;}
.sectionToc, .likesectionToc {margin-left:6em;}
.frenchb-nbsp{font-size:75%;}
.frenchb-thinspace{font-size:75%;}
.figure img.graphics {margin-left:10%;}
/* end css.sty */

\title{Generalites sur les series}
\author{}
\date{}

\begin{document}
\maketitle

\textbf{Warning: 
requires JavaScript to process the mathematics on this page.\\ If your
browser supports JavaScript, be sure it is enabled.}

\begin{center}\rule{3in}{0.4pt}\end{center}

[
[
[]
[

\section{7.2 Généralités sur les séries}

\subsection{7.2.1 Notion de série}

Définition~7.2.1 Soit E un espace vectoriel normé~et (x_n) une
suite de E. On appelle sommes partielles de la série
\\sum  x_n~ les
S_n = \\sum ~
_p=0^nx_p (notée S_n(x) s'il y a risque
de confusion). On dit que la série converge si la suite des sommes
partielles converge dans E~; sa limite est alors appelée la somme de la
série et notée \\sum ~
_n=0^+\infty~x_n =\
lim_n\rightarrow~+\infty~\\\sum
 _p=0^nx_p. Une série non convergente est
dite divergente.

Remarque~7.2.1 Soit (a_n) une suite de E. Définissons une suite
(x_n) par x_0 = a_0 et pour n ≥ 1,
x_n = a_n - a_n-1. On a immédiatement
S_n(x) = a_n et donc la série
\\sum  x_n~
converge si et seulement si~la suite (a_n) converge~; dans ce
cas on a d'ailleurs lima_n~
= \\sum ~
_n=0^+\infty~x_n. Ceci peut permettre dans certains cas
de ramener une étude de convergence de suite à une étude de convergence
de série.

Proposition~7.2.1 Soit E un espace vectoriel normé,
\\sum  x_n~ et
\\sum  y_n~ deux
séries d'éléments de E. On suppose qu'il existe N \in \mathbb{N}~ tel que n ≥ N \rigtharrow~
x_n = y_n (autrement dit les deux suites ne diffèrent
que par un nombre fini de termes). Alors les deux séries sont de même
nature (simultanément convergentes ou divergentes).

Démonstration Pour n ≥ N, on a S_n(x) = S_n(y) +
(S_N(x) - S_N(y)) donc l'une des suites S_n
converge si et seulement si~l'autre converge.

Remarque~7.2.2 En faisant tendre n vers + \infty~, on obtient S(x) = S(y) +
(S_N(x) - S_N(y)).

Définition~7.2.2 Soit E un espace vectoriel normé,
\\sum  x_n~ une
série convergente et p \in \mathbb{N}~. Alors la série
\\sum ~
_n≥p+1x_n est convergente~; sa somme est notée
R_p (ou R_p(x)). On a par définition S_n +
R_n = \\sum ~
_p=0^+\infty~x_p et
limR_n~ = 0.

Proposition~7.2.2 Soit E un espace vectoriel normé. Alors l'ensemble des
suites (x_n) telles que la série
\\sum  x_n~
convergent est un sous-espace vectoriel de E^\mathbb{N}~. L'application
(x_n)_n\in\mathbb{N}~\mapsto~\\\sum
 _n=0^+\infty~x_n est linéaire de ce sous-espace
vectoriel dans E.

Démonstration Il suffit de remarquer que si \alpha~ et \beta~ sont des scalaires,
S_n(\alpha~x + \beta~y) = \alpha~S_n(x) + \beta~S_n(y).

\subsection{7.2.2 Terme général, critère de Cauchy}

Théorème~7.2.3 Si la série
\\sum  x_n~
converge, alors la suite (x_n) admet 0 pour limite.

Démonstration x_n = S_n - S_n-1 et les deux
suites ont la même limite S =\
\sum  _n=0^+\infty~x_n~.

Théorème~7.2.4 (critère de Cauchy pour les séries). Soit E un espace
vectoriel normé~complet et
\\sum  x_n~ une
série à termes de E. La série
\\sum  x_n~
converge si et seulement si~elle vérifie

\forall~~\epsilon > 0,
\exists~N \in \mathbb{N}~, q ≥ p ≥ N
\rigtharrow~\\\sum
_n=p^qx_ n\
< \epsilon

Démonstration C'est simplement le critère de Cauchy pour la suite
(S_n) des sommes partielles puisque
\\sum ~
_n=p^qx_n = S_q - S_p-1.

Exemple~7.2.1 La série harmonique
\\sum  _n≥1~ 1
\over n diverge puisque  1 \over n+1
+ \\ldots~ + 1
\over 2n ≥ n \times 1 \over 2n = 1
\over 2 . La série ne vérifie donc pas le critère de
Cauchy (bien que lim~ 1 \over
n = 0), donc elle diverge.

[
[
[
[

\end{document}

% \documentclass[]{article}
\usepackage[T1]{fontenc}
\usepackage{lmodern}
\usepackage{amssymb,amsmath}
\usepackage{ifxetex,ifluatex}
\usepackage{fixltx2e} % provides \textsubscript
% use upquote if available, for straight quotes in verbatim environments
\IfFileExists{upquote.sty}{\usepackage{upquote}}{}
\ifnum 0\ifxetex 1\fi\ifluatex 1\fi=0 % if pdftex
  \usepackage[utf8]{inputenc}
\else % if luatex or xelatex
  \ifxetex
    \usepackage{mathspec}
    \usepackage{xltxtra,xunicode}
  \else
    \usepackage{fontspec}
  \fi
  \defaultfontfeatures{Mapping=tex-text,Scale=MatchLowercase}
  \newcommand{\euro}{€}
\fi
% use microtype if available
\IfFileExists{microtype.sty}{\usepackage{microtype}}{}
\ifxetex
  \usepackage[setpagesize=false, % page size defined by xetex
              unicode=false, % unicode breaks when used with xetex
              xetex]{hyperref}
\else
  \usepackage[unicode=true]{hyperref}
\fi
\hypersetup{breaklinks=true,
            bookmarks=true,
            pdfauthor={},
            pdftitle={Series `a termes reels positifs},
            colorlinks=true,
            citecolor=blue,
            urlcolor=blue,
            linkcolor=magenta,
            pdfborder={0 0 0}}
\urlstyle{same}  % don't use monospace font for urls
\setlength{\parindent}{0pt}
\setlength{\parskip}{6pt plus 2pt minus 1pt}
\setlength{\emergencystretch}{3em}  % prevent overfull lines
\setcounter{secnumdepth}{0}
 
/* start css.sty */
.cmr-5{font-size:50%;}
.cmr-7{font-size:70%;}
.cmmi-5{font-size:50%;font-style: italic;}
.cmmi-7{font-size:70%;font-style: italic;}
.cmmi-10{font-style: italic;}
.cmsy-5{font-size:50%;}
.cmsy-7{font-size:70%;}
.cmex-7{font-size:70%;}
.cmex-7x-x-71{font-size:49%;}
.msbm-7{font-size:70%;}
.cmtt-10{font-family: monospace;}
.cmti-10{ font-style: italic;}
.cmbx-10{ font-weight: bold;}
.cmr-17x-x-120{font-size:204%;}
.cmsl-10{font-style: oblique;}
.cmti-7x-x-71{font-size:49%; font-style: italic;}
.cmbxti-10{ font-weight: bold; font-style: italic;}
p.noindent { text-indent: 0em }
td p.noindent { text-indent: 0em; margin-top:0em; }
p.nopar { text-indent: 0em; }
p.indent{ text-indent: 1.5em }
@media print {div.crosslinks {visibility:hidden;}}
a img { border-top: 0; border-left: 0; border-right: 0; }
center { margin-top:1em; margin-bottom:1em; }
td center { margin-top:0em; margin-bottom:0em; }
.Canvas { position:relative; }
li p.indent { text-indent: 0em }
.enumerate1 {list-style-type:decimal;}
.enumerate2 {list-style-type:lower-alpha;}
.enumerate3 {list-style-type:lower-roman;}
.enumerate4 {list-style-type:upper-alpha;}
div.newtheorem { margin-bottom: 2em; margin-top: 2em;}
.obeylines-h,.obeylines-v {white-space: nowrap; }
div.obeylines-v p { margin-top:0; margin-bottom:0; }
.overline{ text-decoration:overline; }
.overline img{ border-top: 1px solid black; }
td.displaylines {text-align:center; white-space:nowrap;}
.centerline {text-align:center;}
.rightline {text-align:right;}
div.verbatim {font-family: monospace; white-space: nowrap; text-align:left; clear:both; }
.fbox {padding-left:3.0pt; padding-right:3.0pt; text-indent:0pt; border:solid black 0.4pt; }
div.fbox {display:table}
div.center div.fbox {text-align:center; clear:both; padding-left:3.0pt; padding-right:3.0pt; text-indent:0pt; border:solid black 0.4pt; }
div.minipage{width:100%;}
div.center, div.center div.center {text-align: center; margin-left:1em; margin-right:1em;}
div.center div {text-align: left;}
div.flushright, div.flushright div.flushright {text-align: right;}
div.flushright div {text-align: left;}
div.flushleft {text-align: left;}
.underline{ text-decoration:underline; }
.underline img{ border-bottom: 1px solid black; margin-bottom:1pt; }
.framebox-c, .framebox-l, .framebox-r { padding-left:3.0pt; padding-right:3.0pt; text-indent:0pt; border:solid black 0.4pt; }
.framebox-c {text-align:center;}
.framebox-l {text-align:left;}
.framebox-r {text-align:right;}
span.thank-mark{ vertical-align: super }
span.footnote-mark sup.textsuperscript, span.footnote-mark a sup.textsuperscript{ font-size:80%; }
div.tabular, div.center div.tabular {text-align: center; margin-top:0.5em; margin-bottom:0.5em; }
table.tabular td p{margin-top:0em;}
table.tabular {margin-left: auto; margin-right: auto;}
div.td00{ margin-left:0pt; margin-right:0pt; }
div.td01{ margin-left:0pt; margin-right:5pt; }
div.td10{ margin-left:5pt; margin-right:0pt; }
div.td11{ margin-left:5pt; margin-right:5pt; }
table[rules] {border-left:solid black 0.4pt; border-right:solid black 0.4pt; }
td.td00{ padding-left:0pt; padding-right:0pt; }
td.td01{ padding-left:0pt; padding-right:5pt; }
td.td10{ padding-left:5pt; padding-right:0pt; }
td.td11{ padding-left:5pt; padding-right:5pt; }
table[rules] {border-left:solid black 0.4pt; border-right:solid black 0.4pt; }
.hline hr, .cline hr{ height : 1px; margin:0px; }
.tabbing-right {text-align:right;}
span.TEX {letter-spacing: -0.125em; }
span.TEX span.E{ position:relative;top:0.5ex;left:-0.0417em;}
a span.TEX span.E {text-decoration: none; }
span.LATEX span.A{ position:relative; top:-0.5ex; left:-0.4em; font-size:85%;}
span.LATEX span.TEX{ position:relative; left: -0.4em; }
div.float img, div.float .caption {text-align:center;}
div.figure img, div.figure .caption {text-align:center;}
.marginpar {width:20%; float:right; text-align:left; margin-left:auto; margin-top:0.5em; font-size:85%; text-decoration:underline;}
.marginpar p{margin-top:0.4em; margin-bottom:0.4em;}
.equation td{text-align:center; vertical-align:middle; }
td.eq-no{ width:5%; }
table.equation { width:100%; } 
div.math-display, div.par-math-display{text-align:center;}
math .texttt { font-family: monospace; }
math .textit { font-style: italic; }
math .textsl { font-style: oblique; }
math .textsf { font-family: sans-serif; }
math .textbf { font-weight: bold; }
.partToc a, .partToc, .likepartToc a, .likepartToc {line-height: 200%; font-weight:bold; font-size:110%;}
.chapterToc a, .chapterToc, .likechapterToc a, .likechapterToc, .appendixToc a, .appendixToc {line-height: 200%; font-weight:bold;}
.index-item, .index-subitem, .index-subsubitem {display:block}
.caption td.id{font-weight: bold; white-space: nowrap; }
table.caption {text-align:center;}
h1.partHead{text-align: center}
p.bibitem { text-indent: -2em; margin-left: 2em; margin-top:0.6em; margin-bottom:0.6em; }
p.bibitem-p { text-indent: 0em; margin-left: 2em; margin-top:0.6em; margin-bottom:0.6em; }
.paragraphHead, .likeparagraphHead { margin-top:2em; font-weight: bold;}
.subparagraphHead, .likesubparagraphHead { font-weight: bold;}
.quote {margin-bottom:0.25em; margin-top:0.25em; margin-left:1em; margin-right:1em; text-align:\\jmathmathustify;}
.verse{white-space:nowrap; margin-left:2em}
div.maketitle {text-align:center;}
h2.titleHead{text-align:center;}
div.maketitle{ margin-bottom: 2em; }
div.author, div.date {text-align:center;}
div.thanks{text-align:left; margin-left:10%; font-size:85%; font-style:italic; }
div.author{white-space: nowrap;}
.quotation {margin-bottom:0.25em; margin-top:0.25em; margin-left:1em; }
h1.partHead{text-align: center}
.sectionToc, .likesectionToc {margin-left:2em;}
.subsectionToc, .likesubsectionToc {margin-left:4em;}
.subsubsectionToc, .likesubsubsectionToc {margin-left:6em;}
.frenchb-nbsp{font-size:75%;}
.frenchb-thinspace{font-size:75%;}
.figure img.graphics {margin-left:10%;}
/* end css.sty */

\title{Series `a termes reels positifs}
\author{}
\date{}

\begin{document}
\maketitle

\textbf{Warning: 
requires JavaScript to process the mathematics on this page.\\ If your
browser supports JavaScript, be sure it is enabled.}

\begin{center}\rule{3in}{0.4pt}\end{center}

{[}
{[}
{[}{]}
{[}

\subsubsection{7.3 Séries à termes réels positifs}

\paragraph{7.3.1 Convergence des séries à termes réels positifs}

Théorème~7.3.1 Soit \\\sum
 x_n une série à termes réels positifs. Alors la suite des
sommes partielles est une suite croissante~; la série converge si et
seulement si~ses sommes partielles sont ma\\jmathmathorées~:
\existsM \in \mathbb{R}~, \\forall~~n \in \mathbb{N}~,
S_n \leq M.

Démonstration On a S_n - S_n-1 = x_n ≥ 0 donc
la suite (S_n) est croissante~; par suite, elle converge si et
seulement si~elle est ma\\jmathmathorée.

Remarque~7.3.1 Si une série à termes positifs diverge, on a donc
nécessairement limS_n~ = +\infty~ (puisque
la suite (S_n) est croissante).

Corollaire~7.3.2 Soit \\\sum
 x_n et \\\sum
 y_n deux séries à termes réels telles que 0 \leq x_n
\leq y_n. Alors

\begin{itemize}
\itemsep1pt\parskip0pt\parsep0pt
\item
  (i) si la série \\sum ~
  y_n converge, la série
  \\sum  x_n~
  converge également
\item
  (ii) si la série \\\sum
   x_n diverge, la série
  \\sum  y_n~
  diverge
\end{itemize}

Démonstration On a S_n(x) \leq S_n(y) donc tout ma\\jmathmathorant
de la suite (S_n(y)) est aussi un ma\\jmathmathorant de la suite
(S_n(x)), d'où (i). L'énoncé (ii) n'en est que la contraposée.

Remarque~7.3.2 Pour que l'énoncé précédent soit valable, il suffit
évidemment qu'il existe k \textgreater{} 0 et N \in \mathbb{N}~ tels que n ≥ N \rigtharrow~ 0 \leq
x_n \leq ky_n, c'est-à-dire que x_n ≥ 0,
y_n ≥ 0 et x_n = O(y_n).

\paragraph{7.3.2 Comparaison des séries à termes réels positifs}

Théorème~7.3.3 Soit \\\sum
 x_n et \\\sum
 y_n deux séries à termes réels positifs telles que
x_n = O(y_n) (resp. x_n = o(y_n)).
Alors

\begin{itemize}
\itemsep1pt\parskip0pt\parsep0pt
\item
  (i) si la série \\sum ~
  y_n converge, la série
  \\sum  x_n~
  converge également et R_n(x) = O(R_n(y)) (resp.
  R_n(x) = o(R_n(y)))
\item
  (ii) si la série \\\sum
   x_n diverge, la série
  \\sum  y_n~
  diverge et S_n(x) = O(S_n(y)) (resp. S_n(x)
  = o(S_n(y)))
\end{itemize}

Démonstration Les convergences et divergences résultent immédiatement de
la remarque qui suit le corollaire précédent et du fait que x_n
= o(y_n) \rigtharrow~ x_n = O(y_n). Montrons par exemple
les énoncés sur les relations de comparaison dans le cas x_n =
o(y_n) (des modifications évidentes de \epsilon en k ou 2k permettent
de traiter le cas x_n = O(y_n)).

(i) Soit \epsilon \textgreater{} 0~; il existe N \in \mathbb{N}~ tel que n ≥ N \rigtharrow~ 0 \leq
x_n \leq \epsilony_n. Alors pour n ≥ N, on a 0
\leq\\sum ~
_p=n+1^+\infty~x_p \leq
\epsilon\\sum ~
_p=n+1^+\infty~y_p, soit 0 \leq R_n(x) \leq
\epsilonR_n(y). On a donc R_n(x) = o(R_n(y)).

(ii) Soit \epsilon \textgreater{} 0~; il existe N \in \mathbb{N}~ tel que n ≥ N \rigtharrow~ 0 \leq
x_n \leq \epsilon \over 2 y_n. Alors pour n
\textgreater{} N, on a 0
\leq\\sum ~
_p=N+1^nx_p \leq \epsilon \over 2
 \\sum ~
_p=N+1^ny_p, soit S_n(x) -
S_N(x) \leq \epsilon \over 2 (S_n(y) -
S_N(y)) ou encore 0 \leq S_n(x) \leq \epsilon
\over 2 S_n(y) + (S_N(x) - \epsilon
\over 2 S_N(y)). Mais comme la série
\\sum  y_n~ est
à termes positifs divergente, ses sommes partielles tendent vers + \infty~ et
donc il existe N' \in \mathbb{N}~ tel que n ≥ N' \rigtharrow~ \epsilon \over 2
S_n(y) ≥ S_N(x) - \epsilon \over 2
S_N(y). Alors pour n \textgreater{}\
max(N,N'), on a 0 \leq S_n(x) \leq \epsilon \over 2
S_n(y) + \epsilon \over 2 S_n(y) =
\epsilonS_n(y) et donc S_n(x) = o(S_n(y)).

Corollaire~7.3.4 Soit \\\sum
 x_n et \\\sum
 y_n deux séries à termes réels strictement positifs telles
que

\existsN \in \mathbb{N}~, n ≥ N \rigtharrow~ x_n+1~
\over x_n \leq y_n+1
\over y_n

Alors x_n = O(y_n) et en particulier

\begin{itemize}
\itemsep1pt\parskip0pt\parsep0pt
\item
  (i) si la série \\sum ~
  y_n converge, la série
  \\sum  x_n~
  converge
\item
  (ii) si la série \\\sum
   x_n diverge, la série
  \\sum  y_n~
  diverge
\end{itemize}

Démonstration On vérifie immédiatement par récurrence que pour n ≥ N on
a x_n \leq x_N \over y_N
y_n et donc x_n = O(y_n).

Théorème~7.3.5 Soit \\\sum
 x_n et \\\sum
 y_n deux séries à termes réels telles que y_n ≥ 0
et x_n ∼ y_n. Alors les deux séries sont de même
nature et

\begin{itemize}
\itemsep1pt\parskip0pt\parsep0pt
\item
  (i) si la série \\sum ~
  y_n converge, la série
  \\sum  x_n~
  converge également et R_n(x) ∼ R_n(y)
\item
  (ii) si la série \\\sum
   y_n diverge, la série
  \\sum  x_n~
  diverge et S_n(x) ∼ S_n(y)
\end{itemize}

Démonstration Soit \epsilon \textless{} 1. Il existe N \in \mathbb{N}~ tel que n ≥ N \rigtharrow~ (1 -
\epsilon)y_n \leq x_n \leq (1 + \epsilon)y_n et donc x_n
≥ 0 pour n ≥ N. On a à la fois x_n = O(y_n) et
y_n = O(x_n) ce qui d'après le théorème précédent
montre que les deux séries convergent ou divergent simultanément.
Supposons alors les séries convergentes. On a x_n -
y_n = o(y_n), on en déduit donc la
convergence de \\sum ~
x_n - y_n et que
R_n(x - y) = o(R_n(y)). Mais
R_n(x) - R_n(y)\leq
R_n(x - y) donc R_n(x) -
R_n(y) = o(R_n(y)) et donc R_n(x) ∼
R_n(y). Supposons maintenant les séries divergentes. Alors,
soit la série \\sum ~
x_n - y_n converge et comme
limS_n~(y) = +\infty~ on a
S_n(x - y) = o(S_n(y)), soit elle
diverge et le théorème précédent assure que S_n(x -
y) = o(S_n(y)). Mais alors S_n(x)
- S_n(y)\leq S_n(x - y) =
o(S_n(y)), soit S_n(x) ∼ S_n(y).

\paragraph{7.3.3 Séries de Riemann et de Bertrand}

Théorème~7.3.6 (séries de Riemann). Soit \alpha~ \in \mathbb{R}~. La série
\\sum ~  1
\over n^\alpha~ converge si et seulement si~\alpha~
\textgreater{} 1.

Si \alpha~ \textgreater{} 1, on a R_n ∼ 1 \over
\alpha~-1  1 \over n^\alpha~-1 ~; si \alpha~ \textless{}
1, on a S_n ∼ n^1-\alpha~ \over 1-\alpha~ ~;
si \alpha~ = 1, S_n ∼ log~ n.

Démonstration Soit \alpha~\neq~1. Posons x_n
= 1 \over n^\alpha~ et y_n = 1
\over n^\alpha~-1 - 1 \over
(n+1)^\alpha~-1 . On a

 y_n \over x_n = - (1 + 1
\over n )^1-\alpha~ - 1 \over  1
\over n 

qui admet pour limite l'opposé de la dérivée en 0 de
x\mapsto~(1 + x)^1-\alpha~ soit \alpha~ - 1. On a
donc x_n ∼ 1 \over \alpha~-1 y_n
\textgreater{} 0. Les deux séries sont donc de même nature. Or
S_n(y) = 1 - 1 \over (n+1)^\alpha~-1
admet une limite finie si et seulement si~\alpha~ \textgreater{} 1. Si \alpha~
\textgreater{} 1, on a R_n(x) ∼ 1 \over \alpha~-1
R_n(y) = 1 \over \alpha~-1  1
\over n^\alpha~-1 . Si \alpha~ \textless{} 1, on a
S_n(x) ∼ 1 \over \alpha~-1 S_n(y) = 1
\over 1-\alpha~ ((n + 1)^1-\alpha~ - 1) ∼
n^1-\alpha~ \over 1-\alpha~ . Enfin, si \alpha~ = 1, on
aboutit à une étude similaire avec y_n
= log (n + 1) -\ log~
n = log (1 + 1 \over n~ )
∼ 1 \over n .

Corollaire~7.3.7 (séries de Bertrand). Soit \alpha~,\beta~ \in \mathbb{R}~. La série
\\sum  _n≥2~ 1
\over n^\alpha~(log~
n)^\beta~ converge si et seulement si~\alpha~ \textgreater{} 1 ou \alpha~ =
1,\beta~ \textgreater{} 1.

Démonstration Soit x_n = 1 \over
n^\alpha~(log n)^\beta~~ . Si \alpha~
\textgreater{} 1, soit \gamma tel que \alpha~ \textgreater{} \gamma \textgreater{} 1 et
y_n = 1 \over n^\gamma . La série
\\sum  y_n~
converge et  x_n \over y_n = 1
\over n^\alpha~-\gamma(log~
n)^\beta~ tend vers 0 car \alpha~ - \gamma \textgreater{} 0. On a donc
x_n = o(y_n) et la série
\\sum  x_n~
converge. Si \alpha~ \textless{} 1, soit \gamma tel que \alpha~ \textless{} \gamma \textless{}
1 et y_n = 1 \over n^\gamma . La série
\\sum  y_n~
diverge et  y_n \over x_n =
(log n)^\beta~~ \over
n^\gamma-\alpha~ tend vers 0 car \gamma - \alpha~ \textgreater{} 0. On a donc
y_n = o(x_n) et la série
\\sum  x_n~
converge. Le cas \alpha~ = 1 résulte facilement du paragraphe suivant.

\paragraph{7.3.4 Comparaison à des intégrales}

Théorème~7.3.8 Soit f : {[}0,+\infty~{[}\rightarrow~ \mathbb{R}~ continue par morceaux,
décroissante, positive. Posons w_n =\\int
 _n-1^nf(t) dt - f(n). Alors la série
\\sum  w_n~ est
convergente.

Démonstration On a w_n =\int ~
_n-1^n(f(t) - f(n)) dt. Comme f est décroissante,
\forall~~t \in {[}n - 1,n{]}, f(t) ≥ f(n) et donc
w_n ≥ 0. Mais d'autre part

0 \leq w_n \leq\int  _n-1^n~f(n
- 1) dt - f(n) = f(n - 1) - f(n)

On a \\sum ~
_p=1^n(f(p - 1) - f(p)) = f(0) - f(n) qui admet une limite
quand p tend vers + \infty~ (car f admet une limite en + \infty~~: elle est
décroissante et positive). Ceci montre que la série
\\sum ~ (f(p - 1) - f(p))
converge. Il en est donc de même de la série
\\sum  w_n~.

Corollaire~7.3.9 Soit f : {[}0,+\infty~{[}\rightarrow~ \mathbb{R}~ continue décroissante positive.
Alors la série \\sum ~
f(n) converge si et seulement si f est intégrable sur {[}0,+\infty~{[}.

Démonstration En effet, on déduit du théorème précédent que les deux
séries \\sum ~ f(n) et
\\sum ~
\int  _n-1^n~f(t) dt convergent ou
divergent simultanément, car leur différence est une série convergente.
Mais on a \\sum ~
_p=1^n\int ~
_p-1^pf(t) dt =\int ~
_0^nf(t) dt =\int ~
_{[}0,n{]}f. Si f est intégrable, comme la suite
({[}0,n{]})_n\in\mathbb{N}~ est une suite croissante de segments dont la
réunion est {[}0,+\infty~{[}, la suite (\int ~
_{[}0,n{]}f) est convergente de limite
\int  _{[}0,+\infty~{[}~f, donc la série
\\sum ~
\int  _n-1^n~f(t) dt converge et
il en est de même de \\\sum
 f(n). Si \\sum ~
_f(n) converge, il en est de même de
\\sum ~
\int  _n-1^n~f(t) dt, et si
{[}a,b{]} est un segment contenu dans {[}0,+\infty~{[} les ma\\jmathmathorations

\int  _{[}a,b{]}~f
\leq\int  _0^{[}b{]}+1~f =
\sum _p=0^{[}b{]}+1~
\\int  ~
_p-1^pf(t) dt \leq\\sum
_p=0^+\infty~\\\int
  _p-1^pf(t) dt

et le fait que f soit positive, montrent que f est intégrable sur
{[}0,+\infty~{[}.

Remarque~7.3.3 Bien entendu, il suffit que la condition de décroissance
soit vérifiée sur un certain {[}t_0,+\infty~{[}.

Dans le cas d'une série divergente, l'encadrement

\int  _0^n+1~f(t) dt
\leq\sum _p=0^n~f(p) \leq f(0) +
\\int  ~
_0^nf(t) dt

permet souvent d'obtenir un équivalent de la somme partielle de la
série. Dans le cas d'une série convergente, on a de même

\int  _n+1^+\infty~~f(t) dt
\leq\sum _p=n+1^+\infty~~f(p)
\leq\\int  ~
_n^+\infty~f(t) dt

ce qui permet souvent d'obtenir une ma\\jmathmathoration ou un équivalent du reste
de la série.

Exemple~7.3.1 Dans le cas limite des séries de Bertrand,
\\sum ~  1
\over n(log n)^\beta~~ ,
la fonction f(t) = 1 \over
t(log t)^\beta~~ est continue
décroissante (pour t assez grand) de limite 0. Donc la série est de même
nature que l'intégrale \int ~
_3^+\infty~ dt \over
t(log t)^\beta~~ . Mais on a
\int  _3^x~ dt
\over t(log t)^\beta~~
=\int  _\log~
3^log x~ du \over
u^\beta~ (poser u = log~ t) qui admet
une limite finie quand x tend vers + \infty~ si et seulement si \beta~
\textgreater{} 1. Ceci achève la démonstration du critère de convergence
des séries de Bertrand.

{[}
{[}
{[}
{[}

\end{document}

% \documentclass[]{article}
\usepackage[T1]{fontenc}
\usepackage{lmodern}
\usepackage{amssymb,amsmath}
\usepackage{ifxetex,ifluatex}
\usepackage{fixltx2e} % provides \textsubscript
% use upquote if available, for straight quotes in verbatim environments
\IfFileExists{upquote.sty}{\usepackage{upquote}}{}
\ifnum 0\ifxetex 1\fi\ifluatex 1\fi=0 % if pdftex
  \usepackage[utf8]{inputenc}
\else % if luatex or xelatex
  \ifxetex
    \usepackage{mathspec}
    \usepackage{xltxtra,xunicode}
  \else
    \usepackage{fontspec}
  \fi
  \defaultfontfeatures{Mapping=tex-text,Scale=MatchLowercase}
  \newcommand{\euro}{€}
\fi
% use microtype if available
\IfFileExists{microtype.sty}{\usepackage{microtype}}{}
\ifxetex
  \usepackage[setpagesize=false, % page size defined by xetex
              unicode=false, % unicode breaks when used with xetex
              xetex]{hyperref}
\else
  \usepackage[unicode=true]{hyperref}
\fi
\hypersetup{breaklinks=true,
            bookmarks=true,
            pdfauthor={},
            pdftitle={Series absolument convergentes},
            colorlinks=true,
            citecolor=blue,
            urlcolor=blue,
            linkcolor=magenta,
            pdfborder={0 0 0}}
\urlstyle{same}  % don't use monospace font for urls
\setlength{\parindent}{0pt}
\setlength{\parskip}{6pt plus 2pt minus 1pt}
\setlength{\emergencystretch}{3em}  % prevent overfull lines
\setcounter{secnumdepth}{0}
 
/* start css.sty */
.cmr-5{font-size:50%;}
.cmr-7{font-size:70%;}
.cmmi-5{font-size:50%;font-style: italic;}
.cmmi-7{font-size:70%;font-style: italic;}
.cmmi-10{font-style: italic;}
.cmsy-5{font-size:50%;}
.cmsy-7{font-size:70%;}
.cmex-7{font-size:70%;}
.cmex-7x-x-71{font-size:49%;}
.msbm-7{font-size:70%;}
.cmtt-10{font-family: monospace;}
.cmti-10{ font-style: italic;}
.cmbx-10{ font-weight: bold;}
.cmr-17x-x-120{font-size:204%;}
.cmsl-10{font-style: oblique;}
.cmti-7x-x-71{font-size:49%; font-style: italic;}
.cmbxti-10{ font-weight: bold; font-style: italic;}
p.noindent { text-indent: 0em }
td p.noindent { text-indent: 0em; margin-top:0em; }
p.nopar { text-indent: 0em; }
p.indent{ text-indent: 1.5em }
@media print {div.crosslinks {visibility:hidden;}}
a img { border-top: 0; border-left: 0; border-right: 0; }
center { margin-top:1em; margin-bottom:1em; }
td center { margin-top:0em; margin-bottom:0em; }
.Canvas { position:relative; }
li p.indent { text-indent: 0em }
.enumerate1 {list-style-type:decimal;}
.enumerate2 {list-style-type:lower-alpha;}
.enumerate3 {list-style-type:lower-roman;}
.enumerate4 {list-style-type:upper-alpha;}
div.newtheorem { margin-bottom: 2em; margin-top: 2em;}
.obeylines-h,.obeylines-v {white-space: nowrap; }
div.obeylines-v p { margin-top:0; margin-bottom:0; }
.overline{ text-decoration:overline; }
.overline img{ border-top: 1px solid black; }
td.displaylines {text-align:center; white-space:nowrap;}
.centerline {text-align:center;}
.rightline {text-align:right;}
div.verbatim {font-family: monospace; white-space: nowrap; text-align:left; clear:both; }
.fbox {padding-left:3.0pt; padding-right:3.0pt; text-indent:0pt; border:solid black 0.4pt; }
div.fbox {display:table}
div.center div.fbox {text-align:center; clear:both; padding-left:3.0pt; padding-right:3.0pt; text-indent:0pt; border:solid black 0.4pt; }
div.minipage{width:100%;}
div.center, div.center div.center {text-align: center; margin-left:1em; margin-right:1em;}
div.center div {text-align: left;}
div.flushright, div.flushright div.flushright {text-align: right;}
div.flushright div {text-align: left;}
div.flushleft {text-align: left;}
.underline{ text-decoration:underline; }
.underline img{ border-bottom: 1px solid black; margin-bottom:1pt; }
.framebox-c, .framebox-l, .framebox-r { padding-left:3.0pt; padding-right:3.0pt; text-indent:0pt; border:solid black 0.4pt; }
.framebox-c {text-align:center;}
.framebox-l {text-align:left;}
.framebox-r {text-align:right;}
span.thank-mark{ vertical-align: super }
span.footnote-mark sup.textsuperscript, span.footnote-mark a sup.textsuperscript{ font-size:80%; }
div.tabular, div.center div.tabular {text-align: center; margin-top:0.5em; margin-bottom:0.5em; }
table.tabular td p{margin-top:0em;}
table.tabular {margin-left: auto; margin-right: auto;}
div.td00{ margin-left:0pt; margin-right:0pt; }
div.td01{ margin-left:0pt; margin-right:5pt; }
div.td10{ margin-left:5pt; margin-right:0pt; }
div.td11{ margin-left:5pt; margin-right:5pt; }
table[rules] {border-left:solid black 0.4pt; border-right:solid black 0.4pt; }
td.td00{ padding-left:0pt; padding-right:0pt; }
td.td01{ padding-left:0pt; padding-right:5pt; }
td.td10{ padding-left:5pt; padding-right:0pt; }
td.td11{ padding-left:5pt; padding-right:5pt; }
table[rules] {border-left:solid black 0.4pt; border-right:solid black 0.4pt; }
.hline hr, .cline hr{ height : 1px; margin:0px; }
.tabbing-right {text-align:right;}
span.TEX {letter-spacing: -0.125em; }
span.TEX span.E{ position:relative;top:0.5ex;left:-0.0417em;}
a span.TEX span.E {text-decoration: none; }
span.LATEX span.A{ position:relative; top:-0.5ex; left:-0.4em; font-size:85%;}
span.LATEX span.TEX{ position:relative; left: -0.4em; }
div.float img, div.float .caption {text-align:center;}
div.figure img, div.figure .caption {text-align:center;}
.marginpar {width:20%; float:right; text-align:left; margin-left:auto; margin-top:0.5em; font-size:85%; text-decoration:underline;}
.marginpar p{margin-top:0.4em; margin-bottom:0.4em;}
.equation td{text-align:center; vertical-align:middle; }
td.eq-no{ width:5%; }
table.equation { width:100%; } 
div.math-display, div.par-math-display{text-align:center;}
math .texttt { font-family: monospace; }
math .textit { font-style: italic; }
math .textsl { font-style: oblique; }
math .textsf { font-family: sans-serif; }
math .textbf { font-weight: bold; }
.partToc a, .partToc, .likepartToc a, .likepartToc {line-height: 200%; font-weight:bold; font-size:110%;}
.chapterToc a, .chapterToc, .likechapterToc a, .likechapterToc, .appendixToc a, .appendixToc {line-height: 200%; font-weight:bold;}
.index-item, .index-subitem, .index-subsubitem {display:block}
.caption td.id{font-weight: bold; white-space: nowrap; }
table.caption {text-align:center;}
h1.partHead{text-align: center}
p.bibitem { text-indent: -2em; margin-left: 2em; margin-top:0.6em; margin-bottom:0.6em; }
p.bibitem-p { text-indent: 0em; margin-left: 2em; margin-top:0.6em; margin-bottom:0.6em; }
.subsectionHead, .likesubsectionHead { margin-top:2em; font-weight: bold;}
.sectionHead, .likesectionHead { font-weight: bold;}
.quote {margin-bottom:0.25em; margin-top:0.25em; margin-left:1em; margin-right:1em; text-align:justify;}
.verse{white-space:nowrap; margin-left:2em}
div.maketitle {text-align:center;}
h2.titleHead{text-align:center;}
div.maketitle{ margin-bottom: 2em; }
div.author, div.date {text-align:center;}
div.thanks{text-align:left; margin-left:10%; font-size:85%; font-style:italic; }
div.author{white-space: nowrap;}
.quotation {margin-bottom:0.25em; margin-top:0.25em; margin-left:1em; }
h1.partHead{text-align: center}
.sectionToc, .likesectionToc {margin-left:2em;}
.subsectionToc, .likesubsectionToc {margin-left:4em;}
.sectionToc, .likesectionToc {margin-left:6em;}
.frenchb-nbsp{font-size:75%;}
.frenchb-thinspace{font-size:75%;}
.figure img.graphics {margin-left:10%;}
/* end css.sty */

\title{Series absolument convergentes}
\author{}
\date{}

\begin{document}
\maketitle

\textbf{Warning: 
requires JavaScript to process the mathematics on this page.\\ If your
browser supports JavaScript, be sure it is enabled.}

\begin{center}\rule{3in}{0.4pt}\end{center}

[
[
[]
[

\section{7.4 Séries absolument convergentes}

\subsection{7.4.1 Notion de convergence absolue}

Définition~7.4.1 Soit E un espace vectoriel normé. On dit que la série
\\sum  x_n~ est
absolument convergente si la série à termes réels positifs
\\sum ~
\x_n\
converge.

Théorème~7.4.1 Soit E un espace vectoriel normé~complet. Alors toute
série absolument convergente à terme général dans E est convergente.

Démonstration On a
\\\\sum
 _n=p^qx_n\
\leq\\sum ~
_n=p^q\x_n\.
Si la série \\sum ~
\x_n\
converge, elle vérifie le critère de Cauchy, il en est donc de même de
la série \\sum ~
x_n et donc celle-ci converge.

Remarque~7.4.1 L'avantage est bien entendu de ramener l'étude à celle
d'une série à termes réels positifs.

\subsection{7.4.2 Critères de convergence absolue}

Théorème~7.4.2 Soit \\\sum
 x_n et \\\sum
 y_n deux séries telles que x_n = O(y_n)
et \\sum  y_n~
est absolument convergente. Alors
\\sum  x_n~
converge absolument.

Démonstration On a x_n = O(y_n)
\Leftrightarrow
\x_n\ =
O(\y_n\) et
il suffit d'appliquer le théorème de comparaison pour les séries à
termes réels positifs.

Remarque~7.4.2 Le théorème ci-dessus reste valable même si les termes
généraux x_n et y_n ne sont pas dans le même espace
vectoriel normé. En particulier, la série étalon
\\sum  y_n~ sera
le plus souvent une série à termes réels positifs.

En ce qui concerne les équivalents, on a un résultat plus fort

Théorème~7.4.3 Soit \\\sum
 x_n une série à terme général dans l'espace vectoriel
normé~E et \\sum ~
y_n une série à termes réels positifs. On suppose qu'il existe
\ell \in E \diagdown\0\ tel que x_n ∼
\elly_n. Alors les deux séries sont simultanément convergentes ou
divergentes.

Démonstration Si \\sum ~
y_n converge, on a x_n = O(y_n) et
y_n ≥ 0, donc la série
\\sum  x_n~ est
absolument convergente. Inversement, supposons que la série
\\sum  x_n~
converge. Puisque x_n - \elly_n = o(\elly_n), il
existe N \in \mathbb{N}~ tel que n ≥ N \rigtharrow~\ x_n -
\elly_n\ \leq 1 \over 2
\\elly_n\ = 1
\over 2
\\ell\y_n. En
sommant on obtient
\\\\sum
 _n=p^qx_n -
\ell\\sum ~
_n=p^qy_n\ \leq 1
\over 2
\\ell\\\\sum
 _n=p^qy_n. On en déduit

\begin{align*}
\\ell\\\sum
_n=p^qy_ n& =&
\\ell\\sum
_n=p^qy_ n\
\leq\ \ell\\sum
_n=p^qy_ n -\\sum
_n=p^qx_ n\
+\ \\sum
_n=p^qx_ n\\%&
\\ & \leq& 1 \over 2
\\ell\\\sum
_n=p^qy_ n +\
\sum _n=p^qx_
n\ \%& \\
\end{align*}

d'où en définitive \\\sum
 _n=p^qy_n \leq 2 \over
\\ell\
\\
\sum ~
_n=p^qx_n\. La série
\\sum  x_n~
converge, donc vérifie le critère de Cauchy. Il en est donc de même de
la série \\sum ~
y_n, qui est par suite convergente.

\subsection{7.4.3 Règles classiques}

Il suffit maintenant d'appliquer ces résultats à des séries étalons,
comme les séries de Riemann ou les séries géométriques.

Lemme~7.4.4 Soit a un nombre complexe. La série
\\sum  a^n~
converge si et seulement si~a < 1.

Démonstration La condition est évidemment nécessaire puisque le terme
général doit tendre vers 0. Supposons la vérifiée. On a
\\sum ~
_p=0^na^p = 1-a^n+1
\over 1-a qui admet la limite  1 \over
1-a . Donc la série converge.

Théorème~7.4.5 (règle de d'Alembert). Soit E un espace vectoriel normé
complet. Soit \\sum ~
x_n une série à termes dans E telle que pour tout n \in \mathbb{N}~,
x_n\neq~0 et telle que la suite (
\x_n+1\
\over
\x_n\ )
admet une limite \ell \in \mathbb{R}~ \cup\ + \infty~\. Alors

\begin{itemize}
\itemsep1pt\parskip0pt\parsep0pt
\item
  (i) si \ell < 1, la série converge absolument
\item
  (ii) si \ell > 1, la série diverge
\end{itemize}

Démonstration (i) Si \ell < 1, soit \rho tel que \ell < \rho
< 1~; il existe N \in \mathbb{N}~ tel que n ≥ N \rigtharrow~
\x_n+1\
\over
\x_n\ \leq \rho
soit \x_n+1\
\leq \rho\x_n\. On
a donc alors par récurrence
\x_n\ \leq
\rho^n-N\x_N\
= O(\rho^n). Comme la série
\\sum  \rho^n~
converge, la série \\\sum
 x_n converge absolument.

(ii) Si \ell > 1, il existe N \in \mathbb{N}~ tel que n ≥ N \rigtharrow~
\x_n+1\
\over
\x_n\
> 1 soit
\x_n+1\
>\
x_n\. On a donc alors par récurrence
\x_n\
>\
x_N\. La suite (x_n) ne peut
donc pas avoir 0 pour limite et la série diverge.

Remarque~7.4.3 Si \ell = 1 on ne peut rien conclure comme le montre
l'exemple des séries de Riemann. Lorsque la règle de d'Alembert
s'applique, elle conduit à des convergences rapides (de type
exponentielle) ou des divergences grossières (le terme général ne tend
pas vers 0).

Théorème~7.4.6 (règle de Riemann). Soit E un espace vectoriel normé.
Soit \\sum  x_n~
une série à termes dans E.

\begin{itemize}
\itemsep1pt\parskip0pt\parsep0pt
\item
  (i) S'il existe \alpha~ > 1 tel que x_n = O( 1
  \over n^\alpha~ ), alors la série converge
  absolument
\item
  (ii) S'il existe \alpha~ \in \mathbb{R}~ et \ell \in E \diagdown\0\
  tels que x_n ∼ \ell \over n^\alpha~
  alors la série converge absolument si \alpha~ > 1 et diverge si
  \alpha~ \leq 1.
\item
  (iii) Si E = \mathbb{R}~ et x_n ≥ 0, et s'il existe \alpha~ \leq 1 et \ell
  > 0 (y compris + \infty~) tel que
  limn^\alpha~x_n~ = \ell, alors la
  série diverge.
\end{itemize}

Démonstration (i) et (ii) résultent de ce qui précède. Pour (iii), il
suffit de remarquer que les hypothèses impliquent que  1
\over n^\alpha~ = O(x_n). Comme \alpha~ \leq 1, la
série \\sum ~  1
\over n^\alpha~ diverge et donc aussi la série
\\sum  x_n~.

\subsection{7.4.4 Règles complémentaires}

Théorème~7.4.7 (règle de Cauchy). Soit E un espace vectoriel normé
complet. Soit \\sum ~
x_n une série à termes dans E telle que la suite
\left
(\rootn\of\x_n\\right
) admet une limite \ell \in \mathbb{R}~ \cup\ + \infty~\.
Alors

\begin{itemize}
\itemsep1pt\parskip0pt\parsep0pt
\item
  (i) si \ell < 1, la série converge absolument
\item
  (ii) si \ell > 1, la série diverge
\end{itemize}

Démonstration (i) Si \ell < 1, soit \rho tel que \ell < \rho
< 1~; il existe N \in \mathbb{N}~ tel que n ≥ N
\rigtharrow~\rootn\of\x_n\
\leq \rho soit
\x_n\ \leq
\rho^n. Comme la série
\\sum  \rho^n~
converge, la série \\\sum
 x_n converge absolument.

(ii) Si \ell > 1, il existe N \in \mathbb{N}~ tel que n ≥ N
\rigtharrow~\rootn\of\x_n\
> 1 soit
\x_n\
> 1. La suite (x_n) ne peut donc pas avoir 0 pour
limite et la série diverge.

Théorème~7.4.8 (règle de Duhamel). Soit
\\sum  x_n~ une
série à termes dans \mathbb{R}~^+ telle que pour tout n \in \mathbb{N}~,
x_n\neq~0 et telle que  x_n+1
\over x_n = 1 - \lambda~ \over n +
o( 1 \over n ) . Alors

\begin{itemize}
\itemsep1pt\parskip0pt\parsep0pt
\item
  (i) si \lambda~ > 1, la série converge
\item
  (ii) si \lambda~ < 1, la série diverge
\end{itemize}

Démonstration Posons y_n = 1 \over
n^\alpha~ . On a  y_n+1 \over
y_n = 1 - \alpha~ \over n + o( 1
\over n ). On en déduit que si
\alpha~\neq~\lambda~,  x_n+1 \over
x_n - y_n+1 \over y_n
∼ \alpha~-\lambda~ \over n est pour n assez grand du signe de \alpha~ -
\lambda~. Si \lambda~ < 1, soit \alpha~ tel que \lambda~ < \alpha~ < 1. On
a donc pour n ≥ N,  x_n+1 \over x_n
≥ y_n+1 \over y_n et comme la série
\\sum  y_n~
diverge (car \alpha~ < 1), la série
\\sum  x_n~
diverge. Si \lambda~ > 1, soit \alpha~ tel que \lambda~ > \alpha~
> 1. On a donc pour n ≥ N,  x_n+1
\over x_n \leq y_n+1
\over y_n et comme la série
\\sum  y_n~
converge (car \alpha~ > 1), la série
\\sum  x_n~
converge.

\subsection{7.4.5 Comparaison à une intégrale}

Théorème~7.4.9 Soit f : [0,+\infty~[\rightarrow~ \mathbb{C} de classe \mathcal{C}^1 telle que
f' soit intégrable sur [0,+\infty~[. Posons w_n
=\int  _n-1^n~f(t) dt - f(n).
Alors la série \\sum ~
w_n est absolument convergente.

Démonstration On a par une intégration par parties

\begin{align*} \int ~
_n-1^n(t - n + 1)f'(t) dt& =& \left
[(t - n + 1)f(t)\right ]_ n-1^n
-\int  _n-1^n~f(t) dt\%&
\\ & =& -w_n \%&
\\ \end{align*}

On en déduit que

w_n\leq\int ~
_n-1^n(t - n + 1)f'(t) dt
\leq\int ~
_n-1^nf'(t) dt

et donc

\sum _p=1^nw_
p\leq\\int  ~
_0^nf'(t) dt
\leq\\int  ~
_0^+\infty~f'(t) dt

ce qui montre la convergence de la série à termes positifs
\\sum ~
w_n et donc la convergence absolue de la
série.

Corollaire~7.4.10 Soit f : [0,+\infty~[\rightarrow~ \mathbb{C} de classe \mathcal{C}^1 telle
que f et f' soient intégrables sur [0,+\infty~[. Alors la série
\\sum ~ f(n) est
absolument convergente.

Démonstration En effet la série
\\sum ~
\int ~
_n-1^nf(t) dt est convergente car

\sum _p=1^n~
\\int  ~
_n-1^nf(t) dt =
\\int  ~
_0^nf(t) dt
\leq\\int  ~
_0^+\infty~f(t) dt

et comme \left \int ~
_n-1^nf(t) dt\right
\leq\int ~
_n-1^nf(t) dt, la série
\\sum ~
\int  _n-1^n~f(t) dt est
absolument convergente. Comme
\\sum  w_n~ est
également absolument convergente, il en est de même de la série
\\sum ~ f(n).

[
[
[
[

\end{document}

% \documentclass[]{article}
\usepackage[T1]{fontenc}
\usepackage{lmodern}
\usepackage{amssymb,amsmath}
\usepackage{ifxetex,ifluatex}
\usepackage{fixltx2e} % provides \textsubscript
% use upquote if available, for straight quotes in verbatim environments
\IfFileExists{upquote.sty}{\usepackage{upquote}}{}
\ifnum 0\ifxetex 1\fi\ifluatex 1\fi=0 % if pdftex
  \usepackage[utf8]{inputenc}
\else % if luatex or xelatex
  \ifxetex
    \usepackage{mathspec}
    \usepackage{xltxtra,xunicode}
  \else
    \usepackage{fontspec}
  \fi
  \defaultfontfeatures{Mapping=tex-text,Scale=MatchLowercase}
  \newcommand{\euro}{€}
\fi
% use microtype if available
\IfFileExists{microtype.sty}{\usepackage{microtype}}{}
\ifxetex
  \usepackage[setpagesize=false, % page size defined by xetex
              unicode=false, % unicode breaks when used with xetex
              xetex]{hyperref}
\else
  \usepackage[unicode=true]{hyperref}
\fi
\hypersetup{breaklinks=true,
            bookmarks=true,
            pdfauthor={},
            pdftitle={Series semi-convergentes},
            colorlinks=true,
            citecolor=blue,
            urlcolor=blue,
            linkcolor=magenta,
            pdfborder={0 0 0}}
\urlstyle{same}  % don't use monospace font for urls
\setlength{\parindent}{0pt}
\setlength{\parskip}{6pt plus 2pt minus 1pt}
\setlength{\emergencystretch}{3em}  % prevent overfull lines
\setcounter{secnumdepth}{0}
 
/* start css.sty */
.cmr-5{font-size:50%;}
.cmr-7{font-size:70%;}
.cmmi-5{font-size:50%;font-style: italic;}
.cmmi-7{font-size:70%;font-style: italic;}
.cmmi-10{font-style: italic;}
.cmsy-5{font-size:50%;}
.cmsy-7{font-size:70%;}
.cmex-7{font-size:70%;}
.cmex-7x-x-71{font-size:49%;}
.msbm-7{font-size:70%;}
.cmtt-10{font-family: monospace;}
.cmti-10{ font-style: italic;}
.cmbx-10{ font-weight: bold;}
.cmr-17x-x-120{font-size:204%;}
.cmsl-10{font-style: oblique;}
.cmti-7x-x-71{font-size:49%; font-style: italic;}
.cmbxti-10{ font-weight: bold; font-style: italic;}
p.noindent { text-indent: 0em }
td p.noindent { text-indent: 0em; margin-top:0em; }
p.nopar { text-indent: 0em; }
p.indent{ text-indent: 1.5em }
@media print {div.crosslinks {visibility:hidden;}}
a img { border-top: 0; border-left: 0; border-right: 0; }
center { margin-top:1em; margin-bottom:1em; }
td center { margin-top:0em; margin-bottom:0em; }
.Canvas { position:relative; }
li p.indent { text-indent: 0em }
.enumerate1 {list-style-type:decimal;}
.enumerate2 {list-style-type:lower-alpha;}
.enumerate3 {list-style-type:lower-roman;}
.enumerate4 {list-style-type:upper-alpha;}
div.newtheorem { margin-bottom: 2em; margin-top: 2em;}
.obeylines-h,.obeylines-v {white-space: nowrap; }
div.obeylines-v p { margin-top:0; margin-bottom:0; }
.overline{ text-decoration:overline; }
.overline img{ border-top: 1px solid black; }
td.displaylines {text-align:center; white-space:nowrap;}
.centerline {text-align:center;}
.rightline {text-align:right;}
div.verbatim {font-family: monospace; white-space: nowrap; text-align:left; clear:both; }
.fbox {padding-left:3.0pt; padding-right:3.0pt; text-indent:0pt; border:solid black 0.4pt; }
div.fbox {display:table}
div.center div.fbox {text-align:center; clear:both; padding-left:3.0pt; padding-right:3.0pt; text-indent:0pt; border:solid black 0.4pt; }
div.minipage{width:100%;}
div.center, div.center div.center {text-align: center; margin-left:1em; margin-right:1em;}
div.center div {text-align: left;}
div.flushright, div.flushright div.flushright {text-align: right;}
div.flushright div {text-align: left;}
div.flushleft {text-align: left;}
.underline{ text-decoration:underline; }
.underline img{ border-bottom: 1px solid black; margin-bottom:1pt; }
.framebox-c, .framebox-l, .framebox-r { padding-left:3.0pt; padding-right:3.0pt; text-indent:0pt; border:solid black 0.4pt; }
.framebox-c {text-align:center;}
.framebox-l {text-align:left;}
.framebox-r {text-align:right;}
span.thank-mark{ vertical-align: super }
span.footnote-mark sup.textsuperscript, span.footnote-mark a sup.textsuperscript{ font-size:80%; }
div.tabular, div.center div.tabular {text-align: center; margin-top:0.5em; margin-bottom:0.5em; }
table.tabular td p{margin-top:0em;}
table.tabular {margin-left: auto; margin-right: auto;}
div.td00{ margin-left:0pt; margin-right:0pt; }
div.td01{ margin-left:0pt; margin-right:5pt; }
div.td10{ margin-left:5pt; margin-right:0pt; }
div.td11{ margin-left:5pt; margin-right:5pt; }
table[rules] {border-left:solid black 0.4pt; border-right:solid black 0.4pt; }
td.td00{ padding-left:0pt; padding-right:0pt; }
td.td01{ padding-left:0pt; padding-right:5pt; }
td.td10{ padding-left:5pt; padding-right:0pt; }
td.td11{ padding-left:5pt; padding-right:5pt; }
table[rules] {border-left:solid black 0.4pt; border-right:solid black 0.4pt; }
.hline hr, .cline hr{ height : 1px; margin:0px; }
.tabbing-right {text-align:right;}
span.TEX {letter-spacing: -0.125em; }
span.TEX span.E{ position:relative;top:0.5ex;left:-0.0417em;}
a span.TEX span.E {text-decoration: none; }
span.LATEX span.A{ position:relative; top:-0.5ex; left:-0.4em; font-size:85%;}
span.LATEX span.TEX{ position:relative; left: -0.4em; }
div.float img, div.float .caption {text-align:center;}
div.figure img, div.figure .caption {text-align:center;}
.marginpar {width:20%; float:right; text-align:left; margin-left:auto; margin-top:0.5em; font-size:85%; text-decoration:underline;}
.marginpar p{margin-top:0.4em; margin-bottom:0.4em;}
.equation td{text-align:center; vertical-align:middle; }
td.eq-no{ width:5%; }
table.equation { width:100%; } 
div.math-display, div.par-math-display{text-align:center;}
math .texttt { font-family: monospace; }
math .textit { font-style: italic; }
math .textsl { font-style: oblique; }
math .textsf { font-family: sans-serif; }
math .textbf { font-weight: bold; }
.partToc a, .partToc, .likepartToc a, .likepartToc {line-height: 200%; font-weight:bold; font-size:110%;}
.chapterToc a, .chapterToc, .likechapterToc a, .likechapterToc, .appendixToc a, .appendixToc {line-height: 200%; font-weight:bold;}
.index-item, .index-subitem, .index-subsubitem {display:block}
.caption td.id{font-weight: bold; white-space: nowrap; }
table.caption {text-align:center;}
h1.partHead{text-align: center}
p.bibitem { text-indent: -2em; margin-left: 2em; margin-top:0.6em; margin-bottom:0.6em; }
p.bibitem-p { text-indent: 0em; margin-left: 2em; margin-top:0.6em; margin-bottom:0.6em; }
.paragraphHead, .likeparagraphHead { margin-top:2em; font-weight: bold;}
.subparagraphHead, .likesubparagraphHead { font-weight: bold;}
.quote {margin-bottom:0.25em; margin-top:0.25em; margin-left:1em; margin-right:1em; text-align:justify;}
.verse{white-space:nowrap; margin-left:2em}
div.maketitle {text-align:center;}
h2.titleHead{text-align:center;}
div.maketitle{ margin-bottom: 2em; }
div.author, div.date {text-align:center;}
div.thanks{text-align:left; margin-left:10%; font-size:85%; font-style:italic; }
div.author{white-space: nowrap;}
.quotation {margin-bottom:0.25em; margin-top:0.25em; margin-left:1em; }
h1.partHead{text-align: center}
.sectionToc, .likesectionToc {margin-left:2em;}
.subsectionToc, .likesubsectionToc {margin-left:4em;}
.subsubsectionToc, .likesubsubsectionToc {margin-left:6em;}
.frenchb-nbsp{font-size:75%;}
.frenchb-thinspace{font-size:75%;}
.figure img.graphics {margin-left:10%;}
/* end css.sty */

\title{Series semi-convergentes}
\author{}
\date{}

\begin{document}
\maketitle

\textbf{Warning: 
requires JavaScript to process the mathematics on this page.\\ If your
browser supports JavaScript, be sure it is enabled.}

\begin{center}\rule{3in}{0.4pt}\end{center}

[
[
[]
[

\subsubsection{7.5 Séries semi-convergentes}

\paragraph{7.5.1 Séries alternées}

Théorème~7.5.1 (convergence des séries alternées). Soit (a_n)
une suite de nombres réels, décroissante, de limite 0. Alors la série
\\sum ~
(-1)^na_n converge~; le reste d'ordre n est du signe
de son premier terme (c'est-à-dire (-1)^n+1) et sa valeur
absolue est majorée par la valeur absolue de ce premier terme
(c'est-à-dire a_n+1).

Démonstration On a S_2n+2 - S_2n = a_2n+2 -
a_2n+1 \leq 0 et S_2n+3 - S_2n+1 =
a_2n+2 - a_2n+3 ≥ 0. La suite (S_2n) est donc
décroissante, la suite S_2n+1 est croissante~; comme
S_2n - S_2n+1 = a_2n+1 est positif et tend
vers 0, ces deux suites forment un couple de suites adjacentes~; elles
admettent donc une limite commune S qui est limite de la suite
S_n. On a pour tout n, S_2n-1 \leq S_2n+1 \leq S \leq
S_2n. Ceci nous montre que 0 \leq-R_2n = S_2n -
S \leq S_2n - S_2n+1 = a_2n+1 et que 0 \leq
R_2n-1 = S - S_2n-1 \leq S_2n - S_2n-1
= a_2n d'où les assertions sur le reste.

\paragraph{7.5.2 Etude de séries semi-convergentes}

Les théorèmes de comparaison ne s'appliquent pas aux séries quelconques.
Ainsi on a  (-1)^n \over
\sqrtn ∼ (-1)^n \over
\sqrtn + 1 \over n alors que la
première est convergente et la deuxième divergente (somme d'une série
convergente et d'une série divergente). Pour une série à termes réels,
on peut envisager le plan suivant

(i) regarder si le critère de convergence des séries alternées
s'applique (a_n =
(-1)^na_n avec
a_n décroissant de limite 0).

(ii) si a_n = (-1)^na_n
mais si on ne peut pas appliquer le critère de convergence des séries
alternées, on peut essayer de trouver une série alternée
\\sum  b_n~ qui
relève de ce critère telle que a_n ∼ b_n~; alors,
comme la série \\sum ~
b_n converge, la nature de la série
\\sum  a_n~ sera
la même que celle de la série
\\sum  (a_n~ -
b_n), avec a_n - b_n = o(a_n)~; on
essayera de poursuivre le processus jusqu'à tomber soit sur une série
divergente, soit sur une série absolument convergente

(iii) si a_n n'est pas alterné en signes, on peut utiliser une
sommation par paquets (cf plus loin)~: en regroupant les termes
consécutifs de même signe, on aboutira à une série alternée en signe à
laquelle on pourra appliquer l'une des méthodes précédentes

Enfin, pour une série à termes non réels ou qui ne relève pas d'une des
méthodes précédentes, on pourra utiliser un théorème d'Abel comme le
suivant

Théorème~7.5.2 Soit (a_n) une suite de nombres réels et
(x_n) une suite de l'espace vectoriel normé~complet E telles
que

\begin{itemize}
\itemsep1pt\parskip0pt\parsep0pt
\item
  (i) \existsM ≥ 0, \\forall~~n \in
  \mathbb{N}~,
  \\\\sum
   _p=0^nx_p\ \leq M
\item
  (ii) la suite (a_n) tend vers 0 en décroissant.
\end{itemize}

Alors la série \\sum ~
a_nx_n converge.

Démonstration On a

\begin{align*} \\sum
_n=p^qa_ nx_n& =&
\sum _n=p^qa_
n(S_n(x) - S_n-1(x)) \%&
\\ & =& \\sum
_n=p^qa_ nS_n(x)
-\sum _n=p^qa_
nS_n-1(x) \%& \\ & =&
\sum _n=p^qa_
nS_n(x) -\\sum
_n=p-1^q-1a_ n+1S_n(x) \%&
\\ \text (changement
d'indices \$n - 1\mapsto~n\$)&& \%&
\\ & & \%&
\\ & =& a_qS_q(x) -
a_pS_p-1(x) + \\sum
_n=p^q-1(a_ n -
a_n+1)S_n(x)\%& \\
\end{align*}

On a effectué ici une transformation dite transformation d'Abel.

Comme \forall~~n,
\S_n(x)\ \leq M
on a

\\\sum
_n=p^qa_
nx_n\ \leq
M(a_q + a_p
+ \\sum
_n=p^q-1a_ n -
a_n+1) = 2Ma_p

en tenant compte de a_n ≥ 0 et a_n - a_n+1 ≥
0. Comme lima_p~ = 0, la série
\\sum ~
a_nx_n vérifie le critère de Cauchy, donc elle
converge.

[
[
[
[

\end{document}

% \documentclass[]{article}
\usepackage[T1]{fontenc}
\usepackage{lmodern}
\usepackage{amssymb,amsmath}
\usepackage{ifxetex,ifluatex}
\usepackage{fixltx2e} % provides \textsubscript
% use upquote if available, for straight quotes in verbatim environments
\IfFileExists{upquote.sty}{\usepackage{upquote}}{}
\ifnum 0\ifxetex 1\fi\ifluatex 1\fi=0 % if pdftex
  \usepackage[utf8]{inputenc}
\else % if luatex or xelatex
  \ifxetex
    \usepackage{mathspec}
    \usepackage{xltxtra,xunicode}
  \else
    \usepackage{fontspec}
  \fi
  \defaultfontfeatures{Mapping=tex-text,Scale=MatchLowercase}
  \newcommand{\euro}{€}
\fi
% use microtype if available
\IfFileExists{microtype.sty}{\usepackage{microtype}}{}
\ifxetex
  \usepackage[setpagesize=false, % page size defined by xetex
              unicode=false, % unicode breaks when used with xetex
              xetex]{hyperref}
\else
  \usepackage[unicode=true]{hyperref}
\fi
\hypersetup{breaklinks=true,
            bookmarks=true,
            pdfauthor={},
            pdftitle={Operations sur les series},
            colorlinks=true,
            citecolor=blue,
            urlcolor=blue,
            linkcolor=magenta,
            pdfborder={0 0 0}}
\urlstyle{same}  % don't use monospace font for urls
\setlength{\parindent}{0pt}
\setlength{\parskip}{6pt plus 2pt minus 1pt}
\setlength{\emergencystretch}{3em}  % prevent overfull lines
\setcounter{secnumdepth}{0}
 
/* start css.sty */
.cmr-5{font-size:50%;}
.cmr-7{font-size:70%;}
.cmmi-5{font-size:50%;font-style: italic;}
.cmmi-7{font-size:70%;font-style: italic;}
.cmmi-10{font-style: italic;}
.cmsy-5{font-size:50%;}
.cmsy-7{font-size:70%;}
.cmex-7{font-size:70%;}
.cmex-7x-x-71{font-size:49%;}
.msbm-7{font-size:70%;}
.cmtt-10{font-family: monospace;}
.cmti-10{ font-style: italic;}
.cmbx-10{ font-weight: bold;}
.cmr-17x-x-120{font-size:204%;}
.cmsl-10{font-style: oblique;}
.cmti-7x-x-71{font-size:49%; font-style: italic;}
.cmbxti-10{ font-weight: bold; font-style: italic;}
p.noindent { text-indent: 0em }
td p.noindent { text-indent: 0em; margin-top:0em; }
p.nopar { text-indent: 0em; }
p.indent{ text-indent: 1.5em }
@media print {div.crosslinks {visibility:hidden;}}
a img { border-top: 0; border-left: 0; border-right: 0; }
center { margin-top:1em; margin-bottom:1em; }
td center { margin-top:0em; margin-bottom:0em; }
.Canvas { position:relative; }
li p.indent { text-indent: 0em }
.enumerate1 {list-style-type:decimal;}
.enumerate2 {list-style-type:lower-alpha;}
.enumerate3 {list-style-type:lower-roman;}
.enumerate4 {list-style-type:upper-alpha;}
div.newtheorem { margin-bottom: 2em; margin-top: 2em;}
.obeylines-h,.obeylines-v {white-space: nowrap; }
div.obeylines-v p { margin-top:0; margin-bottom:0; }
.overline{ text-decoration:overline; }
.overline img{ border-top: 1px solid black; }
td.displaylines {text-align:center; white-space:nowrap;}
.centerline {text-align:center;}
.rightline {text-align:right;}
div.verbatim {font-family: monospace; white-space: nowrap; text-align:left; clear:both; }
.fbox {padding-left:3.0pt; padding-right:3.0pt; text-indent:0pt; border:solid black 0.4pt; }
div.fbox {display:table}
div.center div.fbox {text-align:center; clear:both; padding-left:3.0pt; padding-right:3.0pt; text-indent:0pt; border:solid black 0.4pt; }
div.minipage{width:100%;}
div.center, div.center div.center {text-align: center; margin-left:1em; margin-right:1em;}
div.center div {text-align: left;}
div.flushright, div.flushright div.flushright {text-align: right;}
div.flushright div {text-align: left;}
div.flushleft {text-align: left;}
.underline{ text-decoration:underline; }
.underline img{ border-bottom: 1px solid black; margin-bottom:1pt; }
.framebox-c, .framebox-l, .framebox-r { padding-left:3.0pt; padding-right:3.0pt; text-indent:0pt; border:solid black 0.4pt; }
.framebox-c {text-align:center;}
.framebox-l {text-align:left;}
.framebox-r {text-align:right;}
span.thank-mark{ vertical-align: super }
span.footnote-mark sup.textsuperscript, span.footnote-mark a sup.textsuperscript{ font-size:80%; }
div.tabular, div.center div.tabular {text-align: center; margin-top:0.5em; margin-bottom:0.5em; }
table.tabular td p{margin-top:0em;}
table.tabular {margin-left: auto; margin-right: auto;}
div.td00{ margin-left:0pt; margin-right:0pt; }
div.td01{ margin-left:0pt; margin-right:5pt; }
div.td10{ margin-left:5pt; margin-right:0pt; }
div.td11{ margin-left:5pt; margin-right:5pt; }
table[rules] {border-left:solid black 0.4pt; border-right:solid black 0.4pt; }
td.td00{ padding-left:0pt; padding-right:0pt; }
td.td01{ padding-left:0pt; padding-right:5pt; }
td.td10{ padding-left:5pt; padding-right:0pt; }
td.td11{ padding-left:5pt; padding-right:5pt; }
table[rules] {border-left:solid black 0.4pt; border-right:solid black 0.4pt; }
.hline hr, .cline hr{ height : 1px; margin:0px; }
.tabbing-right {text-align:right;}
span.TEX {letter-spacing: -0.125em; }
span.TEX span.E{ position:relative;top:0.5ex;left:-0.0417em;}
a span.TEX span.E {text-decoration: none; }
span.LATEX span.A{ position:relative; top:-0.5ex; left:-0.4em; font-size:85%;}
span.LATEX span.TEX{ position:relative; left: -0.4em; }
div.float img, div.float .caption {text-align:center;}
div.figure img, div.figure .caption {text-align:center;}
.marginpar {width:20%; float:right; text-align:left; margin-left:auto; margin-top:0.5em; font-size:85%; text-decoration:underline;}
.marginpar p{margin-top:0.4em; margin-bottom:0.4em;}
.equation td{text-align:center; vertical-align:middle; }
td.eq-no{ width:5%; }
table.equation { width:100%; } 
div.math-display, div.par-math-display{text-align:center;}
math .texttt { font-family: monospace; }
math .textit { font-style: italic; }
math .textsl { font-style: oblique; }
math .textsf { font-family: sans-serif; }
math .textbf { font-weight: bold; }
.partToc a, .partToc, .likepartToc a, .likepartToc {line-height: 200%; font-weight:bold; font-size:110%;}
.chapterToc a, .chapterToc, .likechapterToc a, .likechapterToc, .appendixToc a, .appendixToc {line-height: 200%; font-weight:bold;}
.index-item, .index-subitem, .index-subsubitem {display:block}
.caption td.id{font-weight: bold; white-space: nowrap; }
table.caption {text-align:center;}
h1.partHead{text-align: center}
p.bibitem { text-indent: -2em; margin-left: 2em; margin-top:0.6em; margin-bottom:0.6em; }
p.bibitem-p { text-indent: 0em; margin-left: 2em; margin-top:0.6em; margin-bottom:0.6em; }
.paragraphHead, .likeparagraphHead { margin-top:2em; font-weight: bold;}
.subparagraphHead, .likesubparagraphHead { font-weight: bold;}
.quote {margin-bottom:0.25em; margin-top:0.25em; margin-left:1em; margin-right:1em; text-align:justify;}
.verse{white-space:nowrap; margin-left:2em}
div.maketitle {text-align:center;}
h2.titleHead{text-align:center;}
div.maketitle{ margin-bottom: 2em; }
div.author, div.date {text-align:center;}
div.thanks{text-align:left; margin-left:10%; font-size:85%; font-style:italic; }
div.author{white-space: nowrap;}
.quotation {margin-bottom:0.25em; margin-top:0.25em; margin-left:1em; }
h1.partHead{text-align: center}
.sectionToc, .likesectionToc {margin-left:2em;}
.subsectionToc, .likesubsectionToc {margin-left:4em;}
.subsubsectionToc, .likesubsubsectionToc {margin-left:6em;}
.frenchb-nbsp{font-size:75%;}
.frenchb-thinspace{font-size:75%;}
.figure img.graphics {margin-left:10%;}
/* end css.sty */

\title{Operations sur les series}
\author{}
\date{}

\begin{document}
\maketitle

\textbf{Warning: 
requires JavaScript to process the mathematics on this page.\\ If your
browser supports JavaScript, be sure it is enabled.}

\begin{center}\rule{3in}{0.4pt}\end{center}

[
[
[]
[

\subsubsection{7.6 Opérations sur les séries}

\paragraph{7.6.1 Combinaisons linéaires}

Proposition~7.6.1 Soit E un espace vectoriel normé,
\\sum  a_n~ et
\\sum  b_n~ deux
séries à termes dans E, \alpha~ et \beta~ deux scalaires. Si
\\sum  a_n~ et
\\sum  b_n~ sont
convergentes (resp. absolument convergentes), il en est de même de la
série \\sum ~
(\alpha~a_n + \beta~b_n) et alors

\sum _n=0^+\infty~(\alpha~a_ n~ +
\beta~b_n) = \alpha~\\sum
_n=0^+\infty~a_ n + \beta~\\sum
_n=0^+\infty~b_ n

Démonstration Le résultat a déjà été vu pour la convergence~; pour la
convergence absolue, il résulte de
\\alpha~a_n +
\beta~b_n\
\leq\alpha~\,\a_n\
+
\beta~\,\b_n\

Corollaire~7.6.2 Soit (z_n) une suite de nombres complexes,
z_n = x_n + iy_n, x_n,y_n \in
\mathbb{R}~. Alors la série \\sum ~
z_n est convergente (resp. absolument convergente) si et
seulement si~les deux séries
\\sum  x_n~ et
\\sum  y_n~ le
sont.

Démonstration Le sens direct résulte de x_n = 1
\over 2 (z_n +
\overlinez_n) et y_n = 1
\over 2i (z_n
-\overlinez_n). La réciproque est évidente.

\paragraph{7.6.2 Sommation par paquets}

Théorème~7.6.3 (Sommation par paquets) Soit E un espace vectoriel normé,
\\sum  x_n~ une
série à termes dans E, \phi une application strictement croissante de \mathbb{N}~
dans \mathbb{N}~. On pose y_0 =\
\sum  _k=0^\phi(0)x_k~ et
pour n ≥ 1, y_n =\
\sum ~
_k=\phi(n-1)+1^\phi(n)x_k. Alors

\begin{itemize}
\itemsep1pt\parskip0pt\parsep0pt
\item
  (i) si la série \\sum ~
  x_n converge, la série
  \\sum  y_n~
  converge et a même somme
\item
  (ii) la réciproque est vraie dans les deux cas suivants

  \begin{itemize}
  \itemsep1pt\parskip0pt\parsep0pt
  \item
    (a) la suite x_n tend vers 0 et la suite \phi(n + 1) - \phi(n)
    (la taille des ''paquets'') est majorée
  \item
    (b) E = \mathbb{R}~ et à l'intérieur de chaque ''paquet'' (k \in [\phi(n - 1) +
    1,\phi(n)]), tous les x_k, sont de même signe.
  \end{itemize}
\end{itemize}

Démonstration On a d'abord

S_n(y) = \\sum
_p=0^n(\\sum
_k=\phi(n-1)+1^\phi(n)x_ k) =
\sum _k=0^\phi(n)x_ k~ =
S_\phi(n)(x)

(en convenant que \phi(-1) = -1). La suite S_n(y) est donc une
sous suite de la suite S_n(x), ce qui montre l'assertion (i).

(ii.a) Soit S = \\sum ~
_n=0^+\infty~y_n et K tel que
\forall~~n, \phi(n + 1) - \phi(n) \leq K. Soit n \in \mathbb{N}~ et p
l'unique entier tel que \phi(p - 1) < n \leq \phi(p). On a alors

S_p(y) - S_n(x) = S_\phi(p)(x) - S_n(x)
= \sum _k=n+1^\phi(p)x_ k~

Soit alors \epsilon > 0 et N \in \mathbb{N}~ tel que n ≥ N
\rigtharrow~\ x_n\
< \epsilon \over 2K . Alors pour n ≥ N, on a
\S_p(y) -
S_n(x)\
\leq\\sum ~
_k=n+1^\phi(p)\x_k\
\leq (\phi(p) - n) \epsilon \over 2K \leq \epsilon \over
2 . Mais il existe N' tel que q ≥ N' \rigtharrow~\ S -
S_q(y)\ < \epsilon
\over 2 . Si on choisit n ≥\
max(N,\phi(N')), on a p ≥ N' et donc

\S - S_n(x)\
\leq\ S -
S_p(y)\ +\
S_p(y) - S_n(x)\ <
\epsilon

ce qui montre que la série
\\sum  x_n~
converge.

(ii.b) La démonstration est similaire mais on remarque que

\begin{align*} S_p(y) -
S_n(x)& =& \\sum
_k=n+1^\phi(p)x_ k =
\sum _k=n+1^\phi(p)x_
k \%& \\ & \leq&
\\sum
_k=\phi(p-1)+1^\phi(p)x_ k =
\\sum
_k=\phi(p-1)+1^\phi(p)x_ k\%&
\\ & =& y_p
\%& \\ \end{align*}

(car tous les x_k sont de même signe). Comme la série
\\sum  y_q~
converge, pour q ≥ N on a y_q <
\epsilon \over 2 . Alors pour n ≥ \phi(N), on a p ≥ N et donc
S_p(y) -
S_n(x)\leqy_p <
\epsilon \over 2 . On achève alors la démonstration comme dans
le cas précédent.

Remarque~7.6.1 La réciproque de (i) n'est pas valable en toute
généralité comme le montre l'exemple de la série
\\sum  (-1)^n~
et de \phi(n) = 2n. On a alors y_n = 0, la série
\\sum  y_n~
converge alors que la série
\\sum  x_n~ est
divergente. La réciproque (ii.b) est particulièrement intéressante pour
le cas de séries de nombres réels qui ne sont pas de signe constant~; en
regroupant ensemble les termes consécutifs de même signe, on obtient une
série de même nature que la série initiale et dont les termes sont
alternés en signe.

\paragraph{7.6.3 Modification de l'ordre des termes}

Nous allons ici étudier l'effet d'une permutation sur les termes d'une
série convergente. Pour cela nous aurons besoin du lemme suivant.

Théorème~7.6.4 Soit \\\sum
 x_n une série à termes réels ou complexes absolument
convergente et soit \sigma : \mathbb{N}~ \rightarrow~ \mathbb{N}~ bijective, une permutation de \mathbb{N}~. Alors la
série \\sum ~
x_\sigma(n) est absolument convergente et
\\sum ~
_n=0^+\infty~x_\sigma(n) =\
\sum  _n=0^+\infty~x_n~.

Démonstration Premier cas~: série à termes réels positifs. Pour n \in \mathbb{N}~,
soit N_n le plus grand élément de \sigma([0,n]). On a alors

\sum _k=0^nx_ \sigma(k)~
\leq\sum _p=0^N_n~
x_p \leq\\sum
_p=0^+\infty~x_ p

ce qui montre que la série à termes réels positifs
\\sum  x_\sigma(k)~
converge et que \\sum ~
_n=0^+\infty~x_\sigma(n)
\leq\\sum ~
_n=0^+\infty~x_n. Mais les deux séries jouent un rôle
symétrique puisque x_n = x_\sigma^-1(\sigma(n)), et
donc on a aussi \\sum ~
_n=0^+\infty~x_n
\leq\\sum ~
_n=0^+\infty~x_\sigma(n) ce qui nous donne l'égalité.

Deuxième cas~: séries à termes réels On introduit, comme d'habitude,
pour x \in \mathbb{R}~, x^+ = max~(x,0) \in
\mathbb{R}~^+ et x^- = max~(-x,0) \in
\mathbb{R}~^+ si bien que x = x^+ - x^-,
x = x^+ + x^-. On a alors 0 \leq
x_n^+ \leqx_n et 0 \leq
x_n^-\leqx_n, ce qui montre
que les deux séries à termes positifs
\\sum ~
x_n^+ et
\\sum ~
x_n^- sont convergentes. D'après le premier cas de la
démonstration, les deux séries
\\sum ~
x_\sigma(n)^+ et
\\sum ~
x_\sigma(n)^- sont convergentes et on a

\sum _n=0^+\infty~x_
\sigma(n)^+ = \\sum
_n=0^+\infty~x_ n^+,\quad
\sum _n=0^+\infty~x_
\sigma(n)^- = \\sum
_n=0^+\infty~x_ n^-

Comme x_\sigma(n) = x_\sigma(n)^+ +
x_\sigma(n)^-, la série
\\sum ~
x_\sigma(n) converge, donc la série
\\sum  x_\sigma(n)~
est absolument convergente, et comme x_\sigma(n) =
x_\sigma(n)^+ - x_\sigma(n)^-, on a

\sum _n=0^+\infty~x_ \sigma(n)~ =
\sum _n=0^+\infty~x_
\sigma(n)^+-\\sum
_n=0^+\infty~x_ \sigma(n)^- =
\sum _n=0^+\infty~x_
n^+-\\sum
_n=0^+\infty~x_ n^- =
\sum _n=0^+\infty~x_ n~

Troisième cas~: séries à termes complexes On travaille de la même
fa\ccon avec les parties réelles et parties
imaginaires. On a 0
\leq\mathrmRe(x_n)\leqx_n~
et 0
\leq\mathrmIm(x_n)\leqx_n~,
ce qui montre que les deux séries
\\sum ~
\mathrmRe(x_n~) et
\\sum ~
\mathrmIm(x_n~)
sont absolument convergentes. D'après le deuxième cas de la
démonstration, les deux séries
\\sum ~
\mathrmRe(x_\sigma(n)~)
et \\sum ~
\mathrmIm(x_\sigma(n)~)
sont absolument convergentes et on a

\\sum
_n=0^+\infty~\mathrmRe(x_ \sigma(n))
= \\sum
_n=0^+\infty~\mathrmRe(x_
n),\quad \\sum
_n=0^+\infty~\mathrmIm(x_ \sigma(n))
= \\sum
_n=0^+\infty~\mathrmIm(x_ n)

Comme
x_\sigma(n)\leq\mathrmRe(x_\sigma(n)~)
+
\mathrmIm(x_\sigma(n)~),
la série \\sum ~
x_\sigma(n) converge, donc la série
\\sum  x_\sigma(n)~
est absolument convergente, et comme x_\sigma(n)
=\
\mathrmRe(x_\sigma(n)) +
i\mathrmRe(x_\sigma(n)~),
on a

\sum _n=0^+\infty~x_ \sigma(n)~ =
\\sum
_n=0^+\infty~\mathrmRe(x_
\sigma(n))+i\\sum
_n=0^+\infty~\mathrmRe(x_ \sigma(n))
= \\sum
_n=0^+\infty~\mathrmRe(x_
n)+i\\sum
_n=0^+\infty~\mathrmIm(x_ n) =
\sum _n=0^+\infty~x_ n~

Remarque~7.6.2 La condition de convergence absolue est indispensable à
la validité du théorème. Considérons la série semi convergente
\\sum  x_n~ avec
x_n = (-1)^n-1 \over n et soit S
sa somme (on peut montrer que S = log~ 2). Soit
\phi : \mathbb{N}~^∗\rightarrow~ \mathbb{N}~^∗ définie par \phi(3k + 1) = 2k + 1, \phi(3k
+ 2) = 4k + 2 et \phi(3k + 3) = 4k + 4. On vérifie facilement que \phi est une
bijection de \mathbb{N}~ dans \mathbb{N}~ (la bijection réciproque est définie par des
congruences modulo 4). Sommons alors par paquets de 3 la série
\\sum  x_\phi(n)~.
On a

\begin{align*} x_\phi(3k+1) +
x_\phi(3k+2) + x_\phi(3k+3)&& \%&
\\ & =& 1 \over 2k +
1 - 1 \over 4k + 2 - 1 \over 4k +
4 = 1 \over 4k + 2 - 1 \over 4k +
4 \%& \\ & =& 1
\over 2 \left (x_2k+1 +
x_2k+2\right ) \%&
\\ \end{align*}

Ceci montre (réciproque du théorème de sommation par paquets, la taille
des paquets étant bornée et le terme général tendant vers 0) que la
nouvelle série converge encore, mais que sa somme est la moitié de la
somme de la série initiale.

\paragraph{7.6.4 Produit de Cauchy}

Définition~7.6.1 Soit \\\sum
 a_n et \\\sum
 b_n deux séries à termes réels ou complexes. On appelle
produit de Cauchy (ou produit de convolution) des deux séries, la série
\\sum  c_n~ avec

\forall~n \in \mathbb{N}~, c_n~ =
\sum _k=0^na_
kb_n-k = \\sum
_p+q=na_pb_q

Théorème~7.6.5 Soit \\\sum
 a_n et \\\sum
 b_n deux séries à termes réels ou complexes, absolument
convergentes. Alors leur produit de Cauchy
\\sum  c_n~ est
une série absolument convergente et on a

\sum _n=0^+\infty~c_ n~ =
\left (\\sum
_n=0^+\infty~a_ n\right
)\left (\\sum
_n=0^+\infty~b_ n\right )

Démonstration Cas particulier~: les deux séries sont à termes réels
positifs. Notons K_n = [0,n] \times [0,n] \subset~ \mathbb{N}~^2
et T_n = \(p,q) \in
\mathbb{N}~^2∣p + q \leq n\.
On a évidemment T_n \subset~ K_n \subset~ T_2n. On a alors

\begin{align*} \\sum
_k=0^nc_ k& =& \\sum
_k=0^n \\sum
_p+q=ka_pb_q = \\sum
_(p,q)\inT_na_pb_q
\leq\\sum
_(p,q)\inK_na_pb_q\%&
\\ & =& \\sum
_p=0^na_ p \\sum
_q=0^nb_ q \leq\\sum
_p=0^+\infty~a_ p \\sum
_q=0^+\infty~b_ q \%&
\\ \end{align*}

La série \\sum ~
c_n est une série à termes réels positifs dont les sommes
partielles sont majorées, donc elle converge. De plus les inclusions
T_n \subset~ K_n \subset~ T_2n se traduisent par
S_n(c) \leq S_n(a)S_n(b) \leq S_2n(c) et
en faisant tendre n vers + \infty~, on obtient S(c) = S(a)S(b) ce qui est la
formule souhaitée.

Cas général Posons a_n' = a_n,
b_n' = b_n et c_n'
= \\sum ~
_p+q=na_pb_q
leur produit de Cauchy, et désignons par
S_n(a'),S_n(b') et S_n(c') les sommes
partielles d'indice n de ces trois séries. Puisque les séries
\\sum  a_n~' et
\\sum  b_n~'
sont convergentes, le cas particulier ci dessus montre que la série
\\sum  c_n~' est
convergente et que sa somme est le produit des sommes de ces deux
séries. Mais, comme c_n\leq c_n', on
en déduit la convergence absolue de la série
\\sum  c_n~. On
a alors

\begin{align*} \left
S_n(a)S_n(b) -
S_n(c)\right & =&
\left \\sum
_(p,q)\inK_na_pb_q
-\\sum
_(p,q)\inT_na_pb_q\right
 = \left \\sum
_(p,q)\inK_n\diagdownT_na_pb_q\right
 \%& \\ & \leq&
\\sum
_(p,q)\inK_n\diagdownT_na_pb_q
= \\sum
_(p,q)\inK_na_pb_q-\\sum
_(p,q)\inT_na_pb_q
= S_n(a')S_n(b') -
S_n(c')\%&\\
\end{align*}

Puisque la somme de la série
\\sum  c_n~' est
le produit des sommes des deux séries
\\sum  a_n~' et
\\sum  b_n~', on
a
lim_n\rightarrow~+\infty~(S_n(a')S_n~(b')
- S_n(c')) = 0 et donc par la majoration ci-dessus
lim_n\rightarrow~+\infty~(S_n(a)S_n~(b)
- S_n(c)) = 0, ce qui montre que la somme de la série
\\sum  c_n~ est
le produit des sommes des deux séries
\\sum  a_n~ et
\\sum  b_n~ et
achève la démonstration.

Remarque~7.6.3 On aurait pu passer aussi du cas réel positif au cas
complexe en utilisant, comme dans le théorème de permutation des termes,
les parties positives x^+ et x^- d'un réel x, puis
les parties réelle et imaginaire d'un nombre complexe, mais la
démonstration n'aurait pas pu se généraliser comme nous le ferons
ci-dessous au cas d'une application bilinéaire plus générale.

Remarque~7.6.4 Le théorème ci dessus n'est plus valable pour des séries
convergentes~: posons a_n = b_n = (-1)^n
\over \sqrtn+1 . On a
c_n =\
\sum  _k=0^n~ 1
\over \sqrt(k+1)(n-k+1) . Mais pour
k \in [0,n], (k + 1)(n - k + 1) \leq ( n \over 2 +
1)^2 (facile). Donc c_n≥ n+1
\over  n \over 2 +1 qui tend vers
2~; donc la suite (c_n) ne tend pas vers 0 et la série
\\sum  c_n~
diverge.

On a une généralisation du théorème précédent sous la forme suivante qui
nous sera utile quand nous considérerons des séries d'endomorphismes.

Théorème~7.6.6 Soit E, F et G sont trois espaces vectoriels normés, u :
E \times F \rightarrow~ G une application bilinéaire continue,
\\sum  a_n~ une
série à termes dans E absolument convergente,
\\sum  b_n~ une
série à termes dans F absolument convergente, et si l'on pose
c_n = \\sum ~
_p+q=nu(a_p,b_q), alors la série
\\sum  c_n~ est
absolument convergente et on a

\sum _n=0^+\infty~c_ n~ =
u\left (\\sum
_n=0^+\infty~a_ n,\\sum
_n=0^+\infty~b_ n\right )

Démonstration La démonstration est tout à fait similaire~: utiliser
l'existence d'un réel positif K tel que
\u(x,y)\ \leq
K\x\
\y\ pour montrer que
\left S_n(a)S_n(b) -
S_n(c)\right \leq K\left
(S_n(a')S_n(b') -
S_n(c')\right ) en posant a_n'
=\ a_n\,
b_n' =\
b_n\ et c_n'
= \\sum ~
_p+q=n\a_p\\b_q\

[
[
[
[

\end{document}

% \documentclass[]{article}
\usepackage[T1]{fontenc}
\usepackage{lmodern}
\usepackage{amssymb,amsmath}
\usepackage{ifxetex,ifluatex}
\usepackage{fixltx2e} % provides \textsubscript
% use upquote if available, for straight quotes in verbatim environments
\IfFileExists{upquote.sty}{\usepackage{upquote}}{}
\ifnum 0\ifxetex 1\fi\ifluatex 1\fi=0 % if pdftex
  \usepackage[utf8]{inputenc}
\else % if luatex or xelatex
  \ifxetex
    \usepackage{mathspec}
    \usepackage{xltxtra,xunicode}
  \else
    \usepackage{fontspec}
  \fi
  \defaultfontfeatures{Mapping=tex-text,Scale=MatchLowercase}
  \newcommand{\euro}{€}
\fi
% use microtype if available
\IfFileExists{microtype.sty}{\usepackage{microtype}}{}
\ifxetex
  \usepackage[setpagesize=false, % page size defined by xetex
              unicode=false, % unicode breaks when used with xetex
              xetex]{hyperref}
\else
  \usepackage[unicode=true]{hyperref}
\fi
\hypersetup{breaklinks=true,
            bookmarks=true,
            pdfauthor={},
            pdftitle={Series doubles},
            colorlinks=true,
            citecolor=blue,
            urlcolor=blue,
            linkcolor=magenta,
            pdfborder={0 0 0}}
\urlstyle{same}  % don't use monospace font for urls
\setlength{\parindent}{0pt}
\setlength{\parskip}{6pt plus 2pt minus 1pt}
\setlength{\emergencystretch}{3em}  % prevent overfull lines
\setcounter{secnumdepth}{0}
 
/* start css.sty */
.cmr-5{font-size:50%;}
.cmr-7{font-size:70%;}
.cmmi-5{font-size:50%;font-style: italic;}
.cmmi-7{font-size:70%;font-style: italic;}
.cmmi-10{font-style: italic;}
.cmsy-5{font-size:50%;}
.cmsy-7{font-size:70%;}
.cmex-7{font-size:70%;}
.cmex-7x-x-71{font-size:49%;}
.msbm-7{font-size:70%;}
.cmtt-10{font-family: monospace;}
.cmti-10{ font-style: italic;}
.cmbx-10{ font-weight: bold;}
.cmr-17x-x-120{font-size:204%;}
.cmsl-10{font-style: oblique;}
.cmti-7x-x-71{font-size:49%; font-style: italic;}
.cmbxti-10{ font-weight: bold; font-style: italic;}
p.noindent { text-indent: 0em }
td p.noindent { text-indent: 0em; margin-top:0em; }
p.nopar { text-indent: 0em; }
p.indent{ text-indent: 1.5em }
@media print {div.crosslinks {visibility:hidden;}}
a img { border-top: 0; border-left: 0; border-right: 0; }
center { margin-top:1em; margin-bottom:1em; }
td center { margin-top:0em; margin-bottom:0em; }
.Canvas { position:relative; }
li p.indent { text-indent: 0em }
.enumerate1 {list-style-type:decimal;}
.enumerate2 {list-style-type:lower-alpha;}
.enumerate3 {list-style-type:lower-roman;}
.enumerate4 {list-style-type:upper-alpha;}
div.newtheorem { margin-bottom: 2em; margin-top: 2em;}
.obeylines-h,.obeylines-v {white-space: nowrap; }
div.obeylines-v p { margin-top:0; margin-bottom:0; }
.overline{ text-decoration:overline; }
.overline img{ border-top: 1px solid black; }
td.displaylines {text-align:center; white-space:nowrap;}
.centerline {text-align:center;}
.rightline {text-align:right;}
div.verbatim {font-family: monospace; white-space: nowrap; text-align:left; clear:both; }
.fbox {padding-left:3.0pt; padding-right:3.0pt; text-indent:0pt; border:solid black 0.4pt; }
div.fbox {display:table}
div.center div.fbox {text-align:center; clear:both; padding-left:3.0pt; padding-right:3.0pt; text-indent:0pt; border:solid black 0.4pt; }
div.minipage{width:100%;}
div.center, div.center div.center {text-align: center; margin-left:1em; margin-right:1em;}
div.center div {text-align: left;}
div.flushright, div.flushright div.flushright {text-align: right;}
div.flushright div {text-align: left;}
div.flushleft {text-align: left;}
.underline{ text-decoration:underline; }
.underline img{ border-bottom: 1px solid black; margin-bottom:1pt; }
.framebox-c, .framebox-l, .framebox-r { padding-left:3.0pt; padding-right:3.0pt; text-indent:0pt; border:solid black 0.4pt; }
.framebox-c {text-align:center;}
.framebox-l {text-align:left;}
.framebox-r {text-align:right;}
span.thank-mark{ vertical-align: super }
span.footnote-mark sup.textsuperscript, span.footnote-mark a sup.textsuperscript{ font-size:80%; }
div.tabular, div.center div.tabular {text-align: center; margin-top:0.5em; margin-bottom:0.5em; }
table.tabular td p{margin-top:0em;}
table.tabular {margin-left: auto; margin-right: auto;}
div.td00{ margin-left:0pt; margin-right:0pt; }
div.td01{ margin-left:0pt; margin-right:5pt; }
div.td10{ margin-left:5pt; margin-right:0pt; }
div.td11{ margin-left:5pt; margin-right:5pt; }
table[rules] {border-left:solid black 0.4pt; border-right:solid black 0.4pt; }
td.td00{ padding-left:0pt; padding-right:0pt; }
td.td01{ padding-left:0pt; padding-right:5pt; }
td.td10{ padding-left:5pt; padding-right:0pt; }
td.td11{ padding-left:5pt; padding-right:5pt; }
table[rules] {border-left:solid black 0.4pt; border-right:solid black 0.4pt; }
.hline hr, .cline hr{ height : 1px; margin:0px; }
.tabbing-right {text-align:right;}
span.TEX {letter-spacing: -0.125em; }
span.TEX span.E{ position:relative;top:0.5ex;left:-0.0417em;}
a span.TEX span.E {text-decoration: none; }
span.LATEX span.A{ position:relative; top:-0.5ex; left:-0.4em; font-size:85%;}
span.LATEX span.TEX{ position:relative; left: -0.4em; }
div.float img, div.float .caption {text-align:center;}
div.figure img, div.figure .caption {text-align:center;}
.marginpar {width:20%; float:right; text-align:left; margin-left:auto; margin-top:0.5em; font-size:85%; text-decoration:underline;}
.marginpar p{margin-top:0.4em; margin-bottom:0.4em;}
.equation td{text-align:center; vertical-align:middle; }
td.eq-no{ width:5%; }
table.equation { width:100%; } 
div.math-display, div.par-math-display{text-align:center;}
math .texttt { font-family: monospace; }
math .textit { font-style: italic; }
math .textsl { font-style: oblique; }
math .textsf { font-family: sans-serif; }
math .textbf { font-weight: bold; }
.partToc a, .partToc, .likepartToc a, .likepartToc {line-height: 200%; font-weight:bold; font-size:110%;}
.chapterToc a, .chapterToc, .likechapterToc a, .likechapterToc, .appendixToc a, .appendixToc {line-height: 200%; font-weight:bold;}
.index-item, .index-subitem, .index-subsubitem {display:block}
.caption td.id{font-weight: bold; white-space: nowrap; }
table.caption {text-align:center;}
h1.partHead{text-align: center}
p.bibitem { text-indent: -2em; margin-left: 2em; margin-top:0.6em; margin-bottom:0.6em; }
p.bibitem-p { text-indent: 0em; margin-left: 2em; margin-top:0.6em; margin-bottom:0.6em; }
.subsectionHead, .likesubsectionHead { margin-top:2em; font-weight: bold;}
.sectionHead, .likesectionHead { font-weight: bold;}
.quote {margin-bottom:0.25em; margin-top:0.25em; margin-left:1em; margin-right:1em; text-align:justify;}
.verse{white-space:nowrap; margin-left:2em}
div.maketitle {text-align:center;}
h2.titleHead{text-align:center;}
div.maketitle{ margin-bottom: 2em; }
div.author, div.date {text-align:center;}
div.thanks{text-align:left; margin-left:10%; font-size:85%; font-style:italic; }
div.author{white-space: nowrap;}
.quotation {margin-bottom:0.25em; margin-top:0.25em; margin-left:1em; }
h1.partHead{text-align: center}
.sectionToc, .likesectionToc {margin-left:2em;}
.subsectionToc, .likesubsectionToc {margin-left:4em;}
.sectionToc, .likesectionToc {margin-left:6em;}
.frenchb-nbsp{font-size:75%;}
.frenchb-thinspace{font-size:75%;}
.figure img.graphics {margin-left:10%;}
/* end css.sty */

\title{Series doubles}
\author{}
\date{}

\begin{document}
\maketitle

\textbf{Warning: 
requires JavaScript to process the mathematics on this page.\\ If your
browser supports JavaScript, be sure it is enabled.}

\begin{center}\rule{3in}{0.4pt}\end{center}

[
[
[]
[

\section{7.7 Séries doubles}

En anticipant un peu sur le chapitre concernant les séries de fonctions,
nous ferons appel au lemme suivant pour la démonstration du théorème
fondamental sur les séries doubles.

Lemme~7.7.1 (Weierstrass~: théorème de convergence dominée pour les
séries) Soit (x_n,q)_(n,q)\in\mathbb{N}~\times\mathbb{N}~ une famille de nombres
réels ou complexes indexée qar \mathbb{N}~ \times \mathbb{N}~. On fait les hypothèses suivantes

\begin{itemize}
\itemsep1pt\parskip0pt\parsep0pt
\item
  il existe une séries à termes réels positifs
  \\sum  \alpha_n~
  convergente telle que \forall~~q \in
  \mathbb{N}~,x_n,q\leq \alpha_n
\item
  pour chaque n \in \mathbb{N}~,
  lim_q\rightarrow~+\infty~x_n,q~ existe (on
  appelle y_n cette limite)
\end{itemize}

Alors, pour chaque q \in \mathbb{N}~, la série
\\sum ~
_nx_n,q est absolument convergente ainsi que la série
\\sum ~
_ny_n, la suite \left
(\\sum ~
_n=0^+\infty~x_n,q\right )_q\in\mathbb{N}~
admet une limite quand q tend vers + \infty~ et on a

lim_q\rightarrow~+\infty~~\\sum
_n=0^+\infty~x_ n,q = \\sum
_n=0^+\infty~y_ n

autrement dit

lim_q\rightarrow~+\infty~~\\sum
_n=0^+\infty~x_ n,q = \\sum
_n=0^+\infty~lim_ q\rightarrow~+\infty~x_n,q

(interversion de la limite et du signe somme)

Démonstration L'inégalité x_n,q\leq
\alpha_n, celle qui s'en déduit par passage à la limite
y_n\leq \alpha_n et la convergence de la
série \\sum ~
\alpha_n montrent les convergences absolues des séries
\\sum ~
_nx_n,q et
\\sum ~
_ny_n. Prenons donc \epsilon > 0 et choisissons M
tel que \\sum ~
_n=M+1^+\infty~\alpha_n < \epsilon\over
4. On a alors

\begin{align*} \left
\sum _n=0^+\infty~y_
n -\sum _n=0^+\infty~x_
n,q\right & \leq& \\sum
_n=0^+\infty~y_ n -
x_n,q\leq\\sum
_n=0^My_ n - x_n,q
+ \\sum
_n=M+1^+\infty~(y_ n +
x_n,q)\%& \\
& \leq& \\sum
_n=0^My_ n - x_n,q
+ 2\sum _n=M+1^+\infty~\alpha_ n~
\leq\sum _n=0^My_
n - x_n,q + \epsilon\over 2
\%&\\ \end{align*}

Maintenant, on a
lim_q\rightarrow~+\infty~~\\\sum
 _n=0^My_n -
x_n,q = 0 (chacun des termes de cette somme admet 0
pour limite), et donc il existe N \in \mathbb{N}~ tel que q ≥ N
\rigtharrow~\\sum ~
_n=0^My_n - x_n,q
< \epsilon\over 2. On a donc

q ≥ N \rigtharrow~\left \\sum
_n=0^+\infty~y_ n -\\sum
_n=0^+\infty~x_ n,q\right \leq
\epsilon\over 2 + \epsilon\over 2 = \epsilon

ce qui montre que la suite \left
(\\sum ~
_n=0^+\infty~x_n,q\right )_q\in\mathbb{N}~
admet la limite \\sum ~
_n=0^+\infty~y_n quand q tend vers + \infty~.

Remarque~7.7.1 Le lecteur qui a déjà des connaissances sur les séries de
fonctions, remarquera qu'il s'agit là tout simplement du théorème
d'interversion des limites dans le cas de convergence normale (donc
uniforme) d'une série de fonctions.

Nous pouvons maintenant démontrer le théorème d'interversion des signes
somme dans les séries doubles.

Théorème~7.7.2 Soit u = (u_n,p)_(n,p)\in\mathbb{N}~\times\mathbb{N}~ une famille
de nombres réels ou complexes indexée par \mathbb{N}~ \times \mathbb{N}~. On suppose que

\begin{itemize}
\itemsep1pt\parskip0pt\parsep0pt
\item
  pour tout entier n la série
  \\sum ~
  _pu_n,p est absolument convergente
\item
  la série \\sum ~
  _n \\sum ~
  _p=0^+\infty~u_n,p est
  convergente
\end{itemize}

Alors les séries \\\sum
 _n\left
(\\sum ~
_p=0^+\infty~u_n,p\right ) et
\\sum ~
_p\left
(\\sum ~
_n=0^+\infty~u_n,p\right ) sont
convergentes et on a

\sum _n=0^+\infty~~\left
(\sum _p=0^+\infty~u_
n,p\right ) = \\sum
_p=0^+\infty~\left (\\sum
_n=0^+\infty~u_ n,p\right )

Démonstration Nous allons appliquer le lemme précédent en posant
x_n,q =\ \\sum
 _p=0^qu_n,p et \alpha_n
= \\sum ~
_p=0^+\infty~u_n,p et bien entendu
y_n = \\sum ~
_p=0^+\infty~u_n,p =\
lim_q\rightarrow~+\infty~x_n,q. Les hypothèses du lemme étant
évidemment vérifiées, on sait que
\\sum ~
_n=0^+\infty~x_n,q admet la limite
\\sum ~
_n=0^+\infty~y_n quand q tend vers + \infty~. Mais, puisque
l'on a l'égalité
u_n,p\leq\\\sum
 _p=0^+\infty~u_n,p, la série
\\sum ~
_nu_n,p est absolument convergente pour tout p \in \mathbb{N}~ et
donc, par linéarité de la somme,

\sum _n=0^+\infty~x_ n,q~ =
\\sum
_n=0^+\infty~\\sum
_p=0^qu_ n,p = \\sum
_p=0^q \\sum
_n=0^+\infty~u_ n,p

L'existence de
lim_q\rightarrow~+\infty~~\\\sum
 _n=0^+\infty~x_n,q montre donc que la série
\\sum ~
_p \\sum ~
_n=0^+\infty~u_n,p est convergente et a pour somme
\\sum ~
_n=0^+\infty~y_n =\
\sum ~
_n=0^+\infty~\\\sum
 _p=0^+\infty~u_n,p autrement dit que

\sum _n=0^+\infty~~\left
(\sum _p=0^+\infty~u_
n,p\right ) = \\sum
_p=0^+\infty~\left (\\sum
_n=0^+\infty~u_ n,p\right )

Remarque~7.7.2 En appliquant le théorème à la suite u' =
(u_n,p)_(n,p)\in\mathbb{N}~\times\mathbb{N}~, on constate que
la série \\sum ~
_p\left
(\\sum ~
_n=0^+\infty~u_n,p\right
) est convergente, ce qui implique la convergence absolue de la série
\\sum ~
_p \\sum ~
_n=0^+\infty~u_n,p.

Remarque~7.7.3 On pourra retenir le théorème précédent sous la forme
suivante

\sum _n=0^+\infty~~\left
(\sum _p=0^+\infty~u_
n,p\right ) <
+\infty~\rigtharrow~\\sum
_n=0^+\infty~\left (\\sum
_p=0^+\infty~u_ n,p\right ) =
\sum _p=0^+\infty~~\left
(\sum _n=0^+\infty~u_
n,p\right )

[
[
[
[

\end{document}

% \documentclass[]{article}
\usepackage[T1]{fontenc}
\usepackage{lmodern}
\usepackage{amssymb,amsmath}
\usepackage{ifxetex,ifluatex}
\usepackage{fixltx2e} % provides \textsubscript
% use upquote if available, for straight quotes in verbatim environments
\IfFileExists{upquote.sty}{\usepackage{upquote}}{}
\ifnum 0\ifxetex 1\fi\ifluatex 1\fi=0 % if pdftex
  \usepackage[utf8]{inputenc}
\else % if luatex or xelatex
  \ifxetex
    \usepackage{mathspec}
    \usepackage{xltxtra,xunicode}
  \else
    \usepackage{fontspec}
  \fi
  \defaultfontfeatures{Mapping=tex-text,Scale=MatchLowercase}
  \newcommand{\euro}{€}
\fi
% use microtype if available
\IfFileExists{microtype.sty}{\usepackage{microtype}}{}
\ifxetex
  \usepackage[setpagesize=false, % page size defined by xetex
              unicode=false, % unicode breaks when used with xetex
              xetex]{hyperref}
\else
  \usepackage[unicode=true]{hyperref}
\fi
\hypersetup{breaklinks=true,
            bookmarks=true,
            pdfauthor={},
            pdftitle={Espaces de suites},
            colorlinks=true,
            citecolor=blue,
            urlcolor=blue,
            linkcolor=magenta,
            pdfborder={0 0 0}}
\urlstyle{same}  % don't use monospace font for urls
\setlength{\parindent}{0pt}
\setlength{\parskip}{6pt plus 2pt minus 1pt}
\setlength{\emergencystretch}{3em}  % prevent overfull lines
\setcounter{secnumdepth}{0}
 
/* start css.sty */
.cmr-5{font-size:50%;}
.cmr-7{font-size:70%;}
.cmmi-5{font-size:50%;font-style: italic;}
.cmmi-7{font-size:70%;font-style: italic;}
.cmmi-10{font-style: italic;}
.cmsy-5{font-size:50%;}
.cmsy-7{font-size:70%;}
.cmex-7{font-size:70%;}
.cmex-7x-x-71{font-size:49%;}
.msbm-7{font-size:70%;}
.cmtt-10{font-family: monospace;}
.cmti-10{ font-style: italic;}
.cmbx-10{ font-weight: bold;}
.cmr-17x-x-120{font-size:204%;}
.cmsl-10{font-style: oblique;}
.cmti-7x-x-71{font-size:49%; font-style: italic;}
.cmbxti-10{ font-weight: bold; font-style: italic;}
p.noindent { text-indent: 0em }
td p.noindent { text-indent: 0em; margin-top:0em; }
p.nopar { text-indent: 0em; }
p.indent{ text-indent: 1.5em }
@media print {div.crosslinks {visibility:hidden;}}
a img { border-top: 0; border-left: 0; border-right: 0; }
center { margin-top:1em; margin-bottom:1em; }
td center { margin-top:0em; margin-bottom:0em; }
.Canvas { position:relative; }
li p.indent { text-indent: 0em }
.enumerate1 {list-style-type:decimal;}
.enumerate2 {list-style-type:lower-alpha;}
.enumerate3 {list-style-type:lower-roman;}
.enumerate4 {list-style-type:upper-alpha;}
div.newtheorem { margin-bottom: 2em; margin-top: 2em;}
.obeylines-h,.obeylines-v {white-space: nowrap; }
div.obeylines-v p { margin-top:0; margin-bottom:0; }
.overline{ text-decoration:overline; }
.overline img{ border-top: 1px solid black; }
td.displaylines {text-align:center; white-space:nowrap;}
.centerline {text-align:center;}
.rightline {text-align:right;}
div.verbatim {font-family: monospace; white-space: nowrap; text-align:left; clear:both; }
.fbox {padding-left:3.0pt; padding-right:3.0pt; text-indent:0pt; border:solid black 0.4pt; }
div.fbox {display:table}
div.center div.fbox {text-align:center; clear:both; padding-left:3.0pt; padding-right:3.0pt; text-indent:0pt; border:solid black 0.4pt; }
div.minipage{width:100%;}
div.center, div.center div.center {text-align: center; margin-left:1em; margin-right:1em;}
div.center div {text-align: left;}
div.flushright, div.flushright div.flushright {text-align: right;}
div.flushright div {text-align: left;}
div.flushleft {text-align: left;}
.underline{ text-decoration:underline; }
.underline img{ border-bottom: 1px solid black; margin-bottom:1pt; }
.framebox-c, .framebox-l, .framebox-r { padding-left:3.0pt; padding-right:3.0pt; text-indent:0pt; border:solid black 0.4pt; }
.framebox-c {text-align:center;}
.framebox-l {text-align:left;}
.framebox-r {text-align:right;}
span.thank-mark{ vertical-align: super }
span.footnote-mark sup.textsuperscript, span.footnote-mark a sup.textsuperscript{ font-size:80%; }
div.tabular, div.center div.tabular {text-align: center; margin-top:0.5em; margin-bottom:0.5em; }
table.tabular td p{margin-top:0em;}
table.tabular {margin-left: auto; margin-right: auto;}
div.td00{ margin-left:0pt; margin-right:0pt; }
div.td01{ margin-left:0pt; margin-right:5pt; }
div.td10{ margin-left:5pt; margin-right:0pt; }
div.td11{ margin-left:5pt; margin-right:5pt; }
table[rules] {border-left:solid black 0.4pt; border-right:solid black 0.4pt; }
td.td00{ padding-left:0pt; padding-right:0pt; }
td.td01{ padding-left:0pt; padding-right:5pt; }
td.td10{ padding-left:5pt; padding-right:0pt; }
td.td11{ padding-left:5pt; padding-right:5pt; }
table[rules] {border-left:solid black 0.4pt; border-right:solid black 0.4pt; }
.hline hr, .cline hr{ height : 1px; margin:0px; }
.tabbing-right {text-align:right;}
span.TEX {letter-spacing: -0.125em; }
span.TEX span.E{ position:relative;top:0.5ex;left:-0.0417em;}
a span.TEX span.E {text-decoration: none; }
span.LATEX span.A{ position:relative; top:-0.5ex; left:-0.4em; font-size:85%;}
span.LATEX span.TEX{ position:relative; left: -0.4em; }
div.float img, div.float .caption {text-align:center;}
div.figure img, div.figure .caption {text-align:center;}
.marginpar {width:20%; float:right; text-align:left; margin-left:auto; margin-top:0.5em; font-size:85%; text-decoration:underline;}
.marginpar p{margin-top:0.4em; margin-bottom:0.4em;}
.equation td{text-align:center; vertical-align:middle; }
td.eq-no{ width:5%; }
table.equation { width:100%; } 
div.math-display, div.par-math-display{text-align:center;}
math .texttt { font-family: monospace; }
math .textit { font-style: italic; }
math .textsl { font-style: oblique; }
math .textsf { font-family: sans-serif; }
math .textbf { font-weight: bold; }
.partToc a, .partToc, .likepartToc a, .likepartToc {line-height: 200%; font-weight:bold; font-size:110%;}
.chapterToc a, .chapterToc, .likechapterToc a, .likechapterToc, .appendixToc a, .appendixToc {line-height: 200%; font-weight:bold;}
.index-item, .index-subitem, .index-subsubitem {display:block}
.caption td.id{font-weight: bold; white-space: nowrap; }
table.caption {text-align:center;}
h1.partHead{text-align: center}
p.bibitem { text-indent: -2em; margin-left: 2em; margin-top:0.6em; margin-bottom:0.6em; }
p.bibitem-p { text-indent: 0em; margin-left: 2em; margin-top:0.6em; margin-bottom:0.6em; }
.paragraphHead, .likeparagraphHead { margin-top:2em; font-weight: bold;}
.subparagraphHead, .likesubparagraphHead { font-weight: bold;}
.quote {margin-bottom:0.25em; margin-top:0.25em; margin-left:1em; margin-right:1em; text-align:\jmathustify;}
.verse{white-space:nowrap; margin-left:2em}
div.maketitle {text-align:center;}
h2.titleHead{text-align:center;}
div.maketitle{ margin-bottom: 2em; }
div.author, div.date {text-align:center;}
div.thanks{text-align:left; margin-left:10%; font-size:85%; font-style:italic; }
div.author{white-space: nowrap;}
.quotation {margin-bottom:0.25em; margin-top:0.25em; margin-left:1em; }
h1.partHead{text-align: center}
.sectionToc, .likesectionToc {margin-left:2em;}
.subsectionToc, .likesubsectionToc {margin-left:4em;}
.subsubsectionToc, .likesubsubsectionToc {margin-left:6em;}
.frenchb-nbsp{font-size:75%;}
.frenchb-thinspace{font-size:75%;}
.figure img.graphics {margin-left:10%;}
/* end css.sty */

\title{Espaces de suites}
\author{}
\date{}

\begin{document}
\maketitle

\textbf{Warning: 
requires JavaScript to process the mathematics on this page.\\ If your
browser supports JavaScript, be sure it is enabled.}

\begin{center}\rule{3in}{0.4pt}\end{center}

{[}
{[}
{[}{]}
{[}

\subsubsection{7.8 Espaces de suites}

Définition~7.8.1 On dit qu'une suite (x\_n)\_n\in\mathbb{N}~ de
nombres réels ou complexes est sommable si la série
\\sum  x\_n~ est
absolument convergente.

Proposition~7.8.1 L'ensemble \ell^1(\mathbb{N}~) des suites sommables de
nombres complexes est un sous espace vectoriel de \mathbb{C}^\mathbb{N}~.
L'application u =
(u\_n)\_n\in\mathbb{N}~\mapsto~\\textbar{}u\\textbar{}\_1
= \\sum ~
\_n=0^+\infty~\textbar{}u\_n\textbar{} est une norme sur
cet espace vectoriel. L'application
u\mapsto~\\\sum
 \_n\in\mathbb{N}~u\_n est linéaire de \ell^1(\mathbb{N}~) dans \mathbb{C}.

Démonstration Si (u\_n) et (v\_n) sont deux suites
sommables et \alpha~,\beta~ \in \mathbb{C}, les suites (\textbar{}u\_n\textbar{}) et
(\textbar{}v\_n\textbar{}) sont sommables~; il en est donc de
même de la suite (\textbar{}\alpha~\textbar{}\textbar{}u\_n\textbar{}
+ \textbar{}\beta~\textbar{}\textbar{}v\_n\textbar{}) (résultat sur
les séries à réels positifs) et donc de la suite
(\textbar{}\alpha~u\_n + \beta~v\_n\textbar{}) puisque
\textbar{}\alpha~u\_n +
\beta~v\_n\textbar{}\leq\textbar{}\alpha~\textbar{}\textbar{}u\_n\textbar{}
+ \textbar{}\beta~\textbar{}\textbar{}v\_n\textbar{}. Donc la suite
(\alpha~u\_n + \beta~v\_n) est sommable. La suite nulle étant de
surcroît sommable, l'ensemble \ell^1(\mathbb{N}~) des suites sommables de
nombres complexes est un sous espace vectoriel de \mathbb{C}^\mathbb{N}~. La
vérification des propriétés d'une norme est élémentaire. On a alors

\begin{align*} \\sum
\_n=0^+\infty~(\alpha~u\_ n + \beta~v\_n)& =&
lim\_p\rightarrow~+\infty~~\\sum
\_n=0^p(\alpha~u\_ n + \beta~v\_n) \%&
\\ & =&
\alpha~lim\_p\rightarrow~+\infty~~\\sum
\_n=0^pu\_ n +
\beta~lim\_p\rightarrow~+\infty~\\sum
\_n=0^pv\_ n\%& \\
& =& \alpha~\sum \_n=0^+\infty~u\_ n~
+ \beta~\sum \_n=0^+\infty~v\_ n~ \%&
\\ \end{align*}

d'où la linéarité de
u\mapsto~\\\sum
 \_n=0^+\infty~u\_n.

Proposition~7.8.2 L'ensemble \ell^2(\mathbb{N}~) des suites de nombres
complexes dont les carrés forment une suite sommable est un sous-espace
vectoriel de \mathbb{C}^\mathbb{N}~. L'application (u,v) = \left
((u\_n)\_n\in\mathbb{N}~,(v\_n)\_n\in\mathbb{N}~\right
)\mapsto~(u\mathrel∣v)
= \\sum ~
\_n=0^+\infty~\overlineu\_nv\_n
est un produit scalaire hermitien sur cet espace~; en conséquence
l'application u =
(u\_n)\_n\in\mathbb{N}~\mapsto~\\textbar{}u\\textbar{}\_2
= \left
(\\sum ~
\_n=0^+\infty~\textbar{}u\_n\textbar{}^2\right
)^1\diagup2 est une norme sur cet espace vectoriel.

Démonstration Il est clair que si (u\_n) est de carré sommable,
il en est de même de \alpha~(u\_n) = (\alpha~u\_n). Si
(u\_n) et (v\_n) sont de carré sommable, l'inégalité
élémentaire \textbar{}u\_n + v\_n\textbar{}^2
\leq 2\textbar{}u\_n\textbar{}^2 +
2\textbar{}v\_n\textbar{}^2 montre que la suite
(u\_n + v\_n) est de carré sommable. La suite nulle
étant de surcroît de carré sommable, les suites de carrés sommables
forment donc bien un sous-espace vectoriel de \mathbb{C}^\mathbb{N}~. Si
(u\_n) et (v\_n) sont de carré sommable, l'inégalité
élémentaire
\textbar{}\overlineu\_nv\_n\textbar{}\leq
1 \over 2 \textbar{}u\_n\textbar{}^2
+ 1 \over 2
\textbar{}v\_n\textbar{}^2 montre que la suite
(\overlineu\_nv\_n) est sommable. On
peut donc poser (u∣v)
= \\sum ~
\_n=0^+\infty~\overlineu\_nv\_n.
Il est clair que
(u,v)\mapsto~(u\mathrel∣v) est
sesquilinéaire hermitienne. De plus, si u\neq~0,
(u∣u) \in \mathbb{R}~^+∗ ce qui montre que
cette forme sesquilinéaire est définie positive~; on a donc un produit
scalaire hermitien et la norme associée est
\\textbar{}u\\textbar{}\_2^2
= (u∣u).

Remarque~7.8.1 Le théorème ci dessus n'est plus valable pour des séries
convergentes~: posons a\_n = b\_n = (-1)^n
\over \sqrtn+1 . On a
\textbar{}c\_n\textbar{} =\
\sum  \_k=0^n~ 1
\over \sqrt(k+1)(n-k+1) . Mais pour
k \in {[}0,n{]}, (k + 1)(n - k + 1) \leq ( n \over 2 +
1)^2 (facile). Donc \textbar{}c\_n\textbar{}≥ n+1
\over  n \over 2 +1 qui tend vers
2~; donc la suite (c\_n) ne tend pas vers 0 et la série
\\sum  c\_n~
diverge.

{[}
{[}
{[}
{[}

\end{document}

% \documentclass[]{article}
\usepackage[T1]{fontenc}
\usepackage{lmodern}
\usepackage{amssymb,amsmath}
\usepackage{ifxetex,ifluatex}
\usepackage{fixltx2e} % provides \textsubscript
% use upquote if available, for straight quotes in verbatim environments
\IfFileExists{upquote.sty}{\usepackage{upquote}}{}
\ifnum 0\ifxetex 1\fi\ifluatex 1\fi=0 % if pdftex
  \usepackage[utf8]{inputenc}
\else % if luatex or xelatex
  \ifxetex
    \usepackage{mathspec}
    \usepackage{xltxtra,xunicode}
  \else
    \usepackage{fontspec}
  \fi
  \defaultfontfeatures{Mapping=tex-text,Scale=MatchLowercase}
  \newcommand{\euro}{€}
\fi
% use microtype if available
\IfFileExists{microtype.sty}{\usepackage{microtype}}{}
\ifxetex
  \usepackage[setpagesize=false, % page size defined by xetex
              unicode=false, % unicode breaks when used with xetex
              xetex]{hyperref}
\else
  \usepackage[unicode=true]{hyperref}
\fi
\hypersetup{breaklinks=true,
            bookmarks=true,
            pdfauthor={},
            pdftitle={Complements : developpements asymptotiques, analyse numerique},
            colorlinks=true,
            citecolor=blue,
            urlcolor=blue,
            linkcolor=magenta,
            pdfborder={0 0 0}}
\urlstyle{same}  % don't use monospace font for urls
\setlength{\parindent}{0pt}
\setlength{\parskip}{6pt plus 2pt minus 1pt}
\setlength{\emergencystretch}{3em}  % prevent overfull lines
\setcounter{secnumdepth}{0}
 
/* start css.sty */
.cmr-5{font-size:50%;}
.cmr-7{font-size:70%;}
.cmmi-5{font-size:50%;font-style: italic;}
.cmmi-7{font-size:70%;font-style: italic;}
.cmmi-10{font-style: italic;}
.cmsy-5{font-size:50%;}
.cmsy-7{font-size:70%;}
.cmex-7{font-size:70%;}
.cmex-7x-x-71{font-size:49%;}
.msbm-7{font-size:70%;}
.cmtt-10{font-family: monospace;}
.cmti-10{ font-style: italic;}
.cmbx-10{ font-weight: bold;}
.cmr-17x-x-120{font-size:204%;}
.cmsl-10{font-style: oblique;}
.cmti-7x-x-71{font-size:49%; font-style: italic;}
.cmbxti-10{ font-weight: bold; font-style: italic;}
p.noindent { text-indent: 0em }
td p.noindent { text-indent: 0em; margin-top:0em; }
p.nopar { text-indent: 0em; }
p.indent{ text-indent: 1.5em }
@media print {div.crosslinks {visibility:hidden;}}
a img { border-top: 0; border-left: 0; border-right: 0; }
center { margin-top:1em; margin-bottom:1em; }
td center { margin-top:0em; margin-bottom:0em; }
.Canvas { position:relative; }
li p.indent { text-indent: 0em }
.enumerate1 {list-style-type:decimal;}
.enumerate2 {list-style-type:lower-alpha;}
.enumerate3 {list-style-type:lower-roman;}
.enumerate4 {list-style-type:upper-alpha;}
div.newtheorem { margin-bottom: 2em; margin-top: 2em;}
.obeylines-h,.obeylines-v {white-space: nowrap; }
div.obeylines-v p { margin-top:0; margin-bottom:0; }
.overline{ text-decoration:overline; }
.overline img{ border-top: 1px solid black; }
td.displaylines {text-align:center; white-space:nowrap;}
.centerline {text-align:center;}
.rightline {text-align:right;}
div.verbatim {font-family: monospace; white-space: nowrap; text-align:left; clear:both; }
.fbox {padding-left:3.0pt; padding-right:3.0pt; text-indent:0pt; border:solid black 0.4pt; }
div.fbox {display:table}
div.center div.fbox {text-align:center; clear:both; padding-left:3.0pt; padding-right:3.0pt; text-indent:0pt; border:solid black 0.4pt; }
div.minipage{width:100%;}
div.center, div.center div.center {text-align: center; margin-left:1em; margin-right:1em;}
div.center div {text-align: left;}
div.flushright, div.flushright div.flushright {text-align: right;}
div.flushright div {text-align: left;}
div.flushleft {text-align: left;}
.underline{ text-decoration:underline; }
.underline img{ border-bottom: 1px solid black; margin-bottom:1pt; }
.framebox-c, .framebox-l, .framebox-r { padding-left:3.0pt; padding-right:3.0pt; text-indent:0pt; border:solid black 0.4pt; }
.framebox-c {text-align:center;}
.framebox-l {text-align:left;}
.framebox-r {text-align:right;}
span.thank-mark{ vertical-align: super }
span.footnote-mark sup.textsuperscript, span.footnote-mark a sup.textsuperscript{ font-size:80%; }
div.tabular, div.center div.tabular {text-align: center; margin-top:0.5em; margin-bottom:0.5em; }
table.tabular td p{margin-top:0em;}
table.tabular {margin-left: auto; margin-right: auto;}
div.td00{ margin-left:0pt; margin-right:0pt; }
div.td01{ margin-left:0pt; margin-right:5pt; }
div.td10{ margin-left:5pt; margin-right:0pt; }
div.td11{ margin-left:5pt; margin-right:5pt; }
table[rules] {border-left:solid black 0.4pt; border-right:solid black 0.4pt; }
td.td00{ padding-left:0pt; padding-right:0pt; }
td.td01{ padding-left:0pt; padding-right:5pt; }
td.td10{ padding-left:5pt; padding-right:0pt; }
td.td11{ padding-left:5pt; padding-right:5pt; }
table[rules] {border-left:solid black 0.4pt; border-right:solid black 0.4pt; }
.hline hr, .cline hr{ height : 1px; margin:0px; }
.tabbing-right {text-align:right;}
span.TEX {letter-spacing: -0.125em; }
span.TEX span.E{ position:relative;top:0.5ex;left:-0.0417em;}
a span.TEX span.E {text-decoration: none; }
span.LATEX span.A{ position:relative; top:-0.5ex; left:-0.4em; font-size:85%;}
span.LATEX span.TEX{ position:relative; left: -0.4em; }
div.float img, div.float .caption {text-align:center;}
div.figure img, div.figure .caption {text-align:center;}
.marginpar {width:20%; float:right; text-align:left; margin-left:auto; margin-top:0.5em; font-size:85%; text-decoration:underline;}
.marginpar p{margin-top:0.4em; margin-bottom:0.4em;}
.equation td{text-align:center; vertical-align:middle; }
td.eq-no{ width:5%; }
table.equation { width:100%; } 
div.math-display, div.par-math-display{text-align:center;}
math .texttt { font-family: monospace; }
math .textit { font-style: italic; }
math .textsl { font-style: oblique; }
math .textsf { font-family: sans-serif; }
math .textbf { font-weight: bold; }
.partToc a, .partToc, .likepartToc a, .likepartToc {line-height: 200%; font-weight:bold; font-size:110%;}
.chapterToc a, .chapterToc, .likechapterToc a, .likechapterToc, .appendixToc a, .appendixToc {line-height: 200%; font-weight:bold;}
.index-item, .index-subitem, .index-subsubitem {display:block}
.caption td.id{font-weight: bold; white-space: nowrap; }
table.caption {text-align:center;}
h1.partHead{text-align: center}
p.bibitem { text-indent: -2em; margin-left: 2em; margin-top:0.6em; margin-bottom:0.6em; }
p.bibitem-p { text-indent: 0em; margin-left: 2em; margin-top:0.6em; margin-bottom:0.6em; }
.paragraphHead, .likeparagraphHead { margin-top:2em; font-weight: bold;}
.subparagraphHead, .likesubparagraphHead { font-weight: bold;}
.quote {margin-bottom:0.25em; margin-top:0.25em; margin-left:1em; margin-right:1em; text-align:justify;}
.verse{white-space:nowrap; margin-left:2em}
div.maketitle {text-align:center;}
h2.titleHead{text-align:center;}
div.maketitle{ margin-bottom: 2em; }
div.author, div.date {text-align:center;}
div.thanks{text-align:left; margin-left:10%; font-size:85%; font-style:italic; }
div.author{white-space: nowrap;}
.quotation {margin-bottom:0.25em; margin-top:0.25em; margin-left:1em; }
h1.partHead{text-align: center}
.sectionToc, .likesectionToc {margin-left:2em;}
.subsectionToc, .likesubsectionToc {margin-left:4em;}
.subsubsectionToc, .likesubsubsectionToc {margin-left:6em;}
.frenchb-nbsp{font-size:75%;}
.frenchb-thinspace{font-size:75%;}
.figure img.graphics {margin-left:10%;}
/* end css.sty */

\title{Complements : developpements asymptotiques, analyse numerique}
\author{}
\date{}

\begin{document}
\maketitle

\textbf{Warning: \href{http://www.math.union.edu/locate/jsMath}{jsMath}
requires JavaScript to process the mathematics on this page.\\ If your
browser supports JavaScript, be sure it is enabled.}

\begin{center}\rule{3in}{0.4pt}\end{center}

{[}\href{coursse42.html}{prev}{]}
{[}\href{coursse42.html\#tailcoursse42.html}{prev-tail}{]}
{[}\hyperref[tailcoursse43.html]{tail}{]}
{[}\href{coursch8.html\#coursse43.html}{up}{]}

\subsubsection{7.9 Compléments~: développements asymptotiques, analyse
numérique}

\paragraph{7.9.1 Calcul approché de la somme d'une série}

L'idée naturelle est d'approcher la somme S de la série convergente
\textbackslash{}mathop\{\textbackslash{}mathop\{∑ \}\} \{x\}\_\{n\} par
une somme partielle \{S\}\_\{N\} =\{\textbackslash{}mathop\{
\textbackslash{}mathop\{∑ \}\} \}\_\{n=0\}\^{}\{N\}\{x\}\_\{n\}.
L'erreur de méthode est évidemment égale à \{R\}\_\{N\}
=\{\textbackslash{}mathop\{ \textbackslash{}mathop\{∑ \}\}
\}\_\{n=N+1\}\^{}\{+∞\}\{x\}\_\{n\}. Bien entendu, à cette erreur de
méthode vient s'ajouter une erreur de calcul de la somme \{S\}\_\{N\}
que l'on peut estimer majorée par Nε où ε est la précision de
l'instrument de calcul. Entre la valeur cherchée S et la valeur calculée
\textbackslash{}overline\{\{S\}\_\{N\}\} il y a donc une erreur du type
\textbar{}S
−\textbackslash{}overline\{\{S\}\_\{N\}\}\textbar{}≤\textbar{}\{R\}\_\{N\}\textbar{}
+ Nε = δ(N) que l'on cherchera donc à minimiser (la fonction δ tend
manifestement vers + ∞ quand N croît indéfiniment).

Etudions pour cela deux cas. Dans le premier cas, la série est à
convergence géométrique~: \textbar{}\{x\}\_\{n\}\textbar{}≤
A\{ρ\}\^{}\{n\} avec ρ \textless{} 1. Alors \{R\}\_\{N\} ≤
B\{ρ\}\^{}\{N\} et δ(N) ≤ \{δ\}\_\{1\}(N) = B\{ρ\}\^{}\{N\} + Nε. On a
\{δ\}\_\{1\}'(t) = B(\textbackslash{}mathop\{log\} ρ)\{ρ\}\^{}\{t\} + ε
qui s'annule pour t = \{t\}\_\{0\} =\{ 1 \textbackslash{}over ρ\}
\textbackslash{}mathop\{ log\} \textbackslash{}left \textbar{}\{ ε
\textbackslash{}over B\textbackslash{}mathop\{ log\} ρ\}
\textbackslash{}right \textbar{}. On a intérêt à choisir N aussi proche
que possible de \{t\}\_\{0\} où la fonction \{δ\}\_\{1\} atteint son
minimum.

Exemple~7.9.1 ~: ε = 1\{0\}\^{}\{−8\},B = 1,ρ =\{ 9 \textbackslash{}over
10\} . On trouve un N de l'ordre de 150 pour une erreur de l'ordre de
1\{0\}\^{}\{−5\}. C'est parfaitement raisonnable.

Dans le second cas, la série est à convergence polynomiale~:
\textbar{}\{x\}\_\{n\}\textbar{}≤\{ A \textbackslash{}over
\{n\}\^{}\{α\}\} avec α \textgreater{} 1. Alors \{R\}\_\{N\} ≤\{ B
\textbackslash{}over \{n\}\^{}\{α−1\}\} et δ(N) ≤ \{δ\}\_\{1\}(N) =\{ B
\textbackslash{}over \{N\}\^{}\{α−1\}\} + Nε. On a \{δ\}\_\{1\}'(t) =
B(1 − α)\{t\}\^{}\{−α\} + ε qui s'annule pour t = \{t\}\_\{0\} =\{
\textbackslash{}left (\{ B(α−1) \textbackslash{}over ε\}
\textbackslash{}right )\}\^{}\{\{ 1 \textbackslash{}over α\} \}. On a
intérêt à choisir N aussi proche que possible de \{t\}\_\{0\} où la
fonction \{δ\}\_\{1\} atteint son minimum.

Exemple~7.9.2 ~: ε = 1\{0\}\^{}\{−8\},B = 1,α =\{ 11
\textbackslash{}over 10\} . On trouve un N de l'ordre de 1\{0\}\^{}\{7\}
pour une erreur de l'ordre de 0,25. On voit que la méthode fournit un
résultat très médiocre en un temps très long~; elle demande donc à être
améliorée par une accélération de convergence.

\paragraph{7.9.2 Accélération de la convergence}

Supposons que \{x\}\_\{n\} admet un développement asymptotique de la
forme

\{x\}\_\{n\} =\{ \{a\}\_\{o\} \textbackslash{}over \{n\}\^{}\{K\}\} +\{
\{a\}\_\{1\} \textbackslash{}over \{n\}\^{}\{K+1\}\} +
\textbackslash{}mathop\{\textbackslash{}mathop\{\ldots{}\}\} +\{
\{a\}\_\{N\} \textbackslash{}over \{n\}\^{}\{K+N\}\} + \{ε\}\_\{n\}

avec \textbar{}\{ε\}\_\{n\}\textbar{}≤\{ A \textbackslash{}over
\{n\}\^{}\{K+N+1\}\} . Posons \{u\}\_\{n\} =\{ \{b\}\_\{o\}
\textbackslash{}over \{n\}\^{}\{K−1\}\} +
\textbackslash{}mathop\{\textbackslash{}mathop\{\ldots{}\}\} +\{
\{b\}\_\{N\} \textbackslash{}over \{n\}\^{}\{K+N−1\}\} (où
\{b\}\_\{o\},\textbackslash{}mathop\{\textbackslash{}mathop\{\ldots{}\}\},\{b\}\_\{N\}
sont des coefficients à déterminer) puis \{y\}\_\{n\} = \{u\}\_\{n\} −
\{u\}\_\{n+1\} , et cherchons à déterminer les \{b\}\_\{i\} de telle
sorte que \textbar{}\{x\}\_\{n\} − \{y\}\_\{n\}\textbar{}≤\{ B
\textbackslash{}over \{n\}\^{}\{K+N+1\}\} (pour une certaine constante
B), c'est-à-dire, \{x\}\_\{n\} − \{y\}\_\{n\} = O(\{ 1
\textbackslash{}over \{n\}\^{}\{K+N+1\}\} ). On a \{u\}\_\{n\}
=\{\textbackslash{}mathop\{ \textbackslash{}mathop\{∑ \}\}
\}\_\{i=0\}\^{}\{N\}\{ \{b\}\_\{i\} \textbackslash{}over
\{n\}\^{}\{K+i−1\}\} , d'où

\textbackslash{}begin\{eqnarray*\}\{ y\}\_\{n\}\& =\&
\{\textbackslash{}mathop\{∑ \}\}\_\{i=0\}\^{}\{N\}\{b\}\_\{
i\}\textbackslash{}left (\{ 1 \textbackslash{}over \{n\}\^{}\{K+i−1\}\}
−\{ 1 \textbackslash{}over \{(n + 1)\}\^{}\{K+i−1\}\}
\textbackslash{}right ) \%\& \textbackslash{}\textbackslash{} \& =\&
\{\textbackslash{}mathop\{∑ \}\}\_\{i=0\}\^{}\{N\}\{b\}\_\{ i\}\{ 1
\textbackslash{}over \{n\}\^{}\{K+i−1\}\} \textbackslash{}left (1 − \{(1
+\{ 1 \textbackslash{}over n\} )\}\^{}\{1−K−i\}\textbackslash{}right
)\%\& \textbackslash{}\textbackslash{} \textbackslash{}end\{eqnarray*\}

On sait que la fonction \{f\}\_\{α\}(x) = \{(1 + x)\}\^{}\{α\} admet au
voisinage de 0 un développement limité \{f\}\_\{α\}(x) = 1
+\{\textbackslash{}mathop\{ \textbackslash{}mathop\{∑ \}\}
\}\_\{k=1\}\^{}\{p\}\{c\}\_\{k\}\^{}\{(α)\}\{x\}\^{}\{k\} +
O(\{x\}\^{}\{p+1\}) avec \{c\}\_\{k\}\^{}\{(α)\} =\{
α(α−1)\textbackslash{}mathop\{\textbackslash{}mathop\{\ldots{}\}\}(α−k+1)
\textbackslash{}over k!\} . On en déduit que

1 − \{(1 +\{ 1 \textbackslash{}over n\} )\}\^{}\{1−K−i\} =
−\{\textbackslash{}mathop\{∑ \}\}\_\{k=1\}\^{}\{N+1−i\}\{c\}\_\{
k\}\^{}\{(1−K−i)\}\{ 1 \textbackslash{}over \{n\}\^{}\{k\}\} + O(\{ 1
\textbackslash{}over \{n\}\^{}\{N+2−i\}\} )

soit

\textbackslash{}begin\{eqnarray*\}\{ 1 \textbackslash{}over
\{n\}\^{}\{K+i−1\}\} \textbackslash{}left (1 − \{(1 +\{ 1
\textbackslash{}over n\} )\}\^{}\{1−K−i\}\textbackslash{}right )\& =\&
−\{\textbackslash{}mathop\{∑ \}\}\_\{k=1\}\^{}\{N+1−i\}\{c\}\_\{
k\}\^{}\{(1−K−i)\}\{ 1 \textbackslash{}over \{n\}\^{}\{k+K+i−1\}\} +
O(\{ 1 \textbackslash{}over \{n\}\^{}\{N+K+1\}\} )\%\&
\textbackslash{}\textbackslash{} \& =\& −\{\textbackslash{}mathop\{∑
\}\}\_\{k=i\}\^{}\{N\}\{c\}\_\{ k+1−i\}\^{}\{(1−K−i)\}\{ 1
\textbackslash{}over \{n\}\^{}\{k+K\}\} + O(\{ 1 \textbackslash{}over
\{n\}\^{}\{N+K+1\}\} ) \%\& \textbackslash{}\textbackslash{}
\textbackslash{}end\{eqnarray*\}

après changement d'indices. On en déduit

\textbackslash{}begin\{eqnarray*\}\{ y\}\_\{n\}\& =\&
−\{\textbackslash{}mathop\{∑ \}\}\_\{i=0\}\^{}\{N\}\{b\}\_\{ i\}\{
\textbackslash{}mathop\{∑ \}\}\_\{k=i\}\^{}\{N\}\{c\}\_\{
k+1−i\}\^{}\{(1−K−i)\}\{ 1 \textbackslash{}over \{n\}\^{}\{k+K\}\} +
O(\{ 1 \textbackslash{}over \{n\}\^{}\{N+K+1\}\} )\%\&
\textbackslash{}\textbackslash{} \& =\& −\{\textbackslash{}mathop\{∑
\}\}\_\{k=0\}\^{}\{N\}\{ 1 \textbackslash{}over \{n\}\^{}\{k+K\}\} \{
\textbackslash{}mathop\{∑ \}\}\_\{i=0\}\^{}\{k\}\{b\}\_\{
i\}\{c\}\_\{k+1−i\}\^{}\{(1−K−i)\} + O(\{ 1 \textbackslash{}over
\{n\}\^{}\{N+K+1\}\} )\%\& \textbackslash{}\textbackslash{}
\textbackslash{}end\{eqnarray*\}

Donc

\{x\}\_\{n\} − \{y\}\_\{n\} = O(\{ 1 \textbackslash{}over
\{n\}\^{}\{K+N+1\}\} ) \textbackslash{}mathrel\{⇔\}
\textbackslash{}mathop\{∀\}k ∈ {[}0,n{]}, \{a\}\_\{k\} +\{
\textbackslash{}mathop\{∑ \}\}\_\{i=0\}\^{}\{k\}\{b\}\_\{
i\}\{c\}\_\{1−k−i\}\^{}\{(1−K−i)\} = 0

Il s'agit d'un système triangulaire en les inconnues \{b\}\_\{i\} qui
admet une unique solution. En faisant le changement d'indice j = k + 1 −
i, on obtient le système

\textbackslash{}mathop\{∀\}k ∈ {[}0,n{]}, \{a\}\_\{k\} +\{
\textbackslash{}mathop\{∑ \}\}\_\{j=1\}\^{}\{k+1\}\{b\}\_\{
k+1−j\}\{c\}\_\{j\}\^{}\{(−K−k+j)\} = 0

On calcule donc les \{b\}\_\{k\} à l'aide de la formule de récurrence
\{c\}\_\{1\}\^{}\{(−K−k+1)\}\{b\}\_\{k\} = −\{a\}\_\{k\}
−\{\textbackslash{}mathop\{\textbackslash{}mathop\{∑ \}\}
\}\_\{j=2\}\^{}\{k+1\}\{b\}\_\{k+1−j\}\{c\}\_\{j\}\^{}\{(−K−k+j)\} où
les \{c\}\_\{j\}\^{}\{(t+j)\} sont définis par récurrence par
\{c\}\_\{1\}\^{}\{(t+1)\} = t + 1 et \{c\}\_\{j+1\}\^{}\{(t+j+1)\} =\{
t+j+1 \textbackslash{}over j+1\} \{c\}\_\{j\}\^{}\{(t+j)\}. Supposons
les \{b\}\_\{i\} déterminés. Il existe une constante B telle que
\textbar{}\{x\}\_\{n\} − \{y\}\_\{n\}\textbar{}≤\{ B
\textbackslash{}over \{n\}\^{}\{K+N+1\}\} . L'erreur faite en approchant
la somme de la série \textbackslash{}mathop\{\textbackslash{}mathop\{∑
\}\} (\{x\}\_\{n\} − \{y\}\_\{n\}) par sa somme partielle d'indice n est
donc majorée par \{ B \textbackslash{}over K+N\} \{ 1
\textbackslash{}over \{n\}\^{}\{K+N\}\} . Mais la somme partielle
d'indice n de la série est

\{\textbackslash{}mathop\{∑ \}\}\_\{k=1\}\^{}\{n\}(\{x\}\_\{ k\} −
\{y\}\_\{k\}) =\{ \textbackslash{}mathop\{∑
\}\}\_\{k=1\}\^{}\{n\}\{x\}\_\{ k\} −\{\textbackslash{}mathop\{∑
\}\}\_\{k=1\}\^{}\{n\}(\{u\}\_\{ k\} − \{u\}\_\{k+1\}) = \{S\}\_\{n\} +
\{u\}\_\{1\} − \{u\}\_\{n+1\}

et la somme de la série est

\{\textbackslash{}mathop\{∑ \}\}\_\{n=1\}\^{}\{+∞\}(\{x\}\_\{ n\} −
\{y\}\_\{n\}) =\{ \textbackslash{}mathop\{∑
\}\}\_\{n=1\}\^{}\{+∞\}\{x\}\_\{ n\} −\{\textbackslash{}mathop\{∑
\}\}\_\{n=1\}\^{}\{+∞\}(\{u\}\_\{ n\} − \{u\}\_\{n+1\}) = S −
\{u\}\_\{1\}

(puisque \textbackslash{}mathop\{lim\}\{u\}\_\{n\} = 0). On a donc
\textbar{}S − \{S\}\_\{n\} + \{u\}\_\{n+1\}\textbar{}≤\{ B
\textbackslash{}over K+N\} \{ 1 \textbackslash{}over \{n\}\^{}\{K+N\}\}
et \{S\}\_\{n\} − \{u\}\_\{n+1\} est donc une bien meilleure valeur
approchée de S que \{S\}\_\{n\}.

Bien entendu ces méthodes peuvent se généraliser à d'autres types de
développements asymptotiques~: l'idée générale étant de trouver une
suite \{u\}\_\{n\} telle que la série \{x\}\_\{n\} − (\{u\}\_\{n\} −
\{u\}\_\{n+1\}) ait une décroissance vers 0 aussi rapide que possible.
Alors \{S\}\_\{n\} − \{u\}\_\{n+1\} est donc une bien meilleure valeur
approchée de S que \{S\}\_\{n\}. Cette méthode fournira également des
développements asymptotiques de restes de séries car si \{x\}\_\{n\} −
(\{u\}\_\{n\} − \{u\}\_\{n+1\}) = o(\{v\}\_\{n\}), on aura
\{R\}\_\{n\}(x) + \{u\}\_\{n+1\} = o(\{R\}\_\{n\}(v)) et donc le
développement \{R\}\_\{n\}(x) = −\{u\}\_\{n+1\} + o(\{R\}\_\{n\}(v)).

En ce qui concerne les développements asymptotiques de sommes partielles
de séries divergentes, on se ramènera à la situation précédente en
rempla\textbackslash{}c\{c\}ant la série \{x\}\_\{n\} par une série du
type \{y\}\_\{n\} = \{x\}\_\{n\} − (\{v\}\_\{n\} − \{v\}\_\{n−1\}) de
telle sorte que la série
\textbackslash{}mathop\{\textbackslash{}mathop\{∑ \}\} \{y\}\_\{n\}
converge. On aura alors \{S\}\_\{n\}(x) = \{v\}\_\{n\} − \{v\}\_\{0\} +
\{S\}\_\{n\}(y) = \{v\}\_\{n\} + A + \{R\}\_\{n\}(y) où A = S(y) −
\{v\}\_\{0\} est une constante (sa valeur ne pourra pas être obtenue
directement par cette méthode). Il suffira ensuite d'appliquer la
méthode précédente pour obtenir un développement asymptotique de
\{R\}\_\{n\}(y) à la précision souhaitée, et donc aussi un développement
asymptotique de \{R\}\_\{n\}(x).

Nous allons traiter deux exemples importants des techniques ci dessus.

Exemple~7.9.3 On recherche un développement asymptotique de
\{\textbackslash{}mathop\{\textbackslash{}mathop\{∑ \}\}
\}\_\{k=1\}\^{}\{n\}\{ 1 \textbackslash{}over k\} . Posons \{x\}\_\{n\}
=\{ 1 \textbackslash{}over n\} et \{y\}\_\{n\} =\textbackslash{}mathop\{
log\} (n) −\textbackslash{}mathop\{ log\} (n − 1) =
−\textbackslash{}mathop\{log\} (1 −\{ 1 \textbackslash{}over n\} ). On a
\{z\}\_\{n\} = \{x\}\_\{n\} − \{y\}\_\{n\} =\{ 1 \textbackslash{}over
n\} −\textbackslash{}mathop\{ log\} (1 −\{ 1 \textbackslash{}over n\} )
= −\{ 1 \textbackslash{}over 2\{n\}\^{}\{2\}\} + O(\{ 1
\textbackslash{}over \{n\}\^{}\{3\}\} ). On en déduit que la série
\textbackslash{}mathop\{\textbackslash{}mathop\{∑ \}\} \{z\}\_\{n\}
converge. On a alors

\textbackslash{}begin\{eqnarray*\} \{\textbackslash{}mathop\{∑
\}\}\_\{k=1\}\^{}\{n\}\{x\}\_\{ k\}\& =\& 1 +\{
\textbackslash{}mathop\{∑ \}\}\_\{k=2\}\^{}\{n\}\{z\}\_\{ k\} +\{
\textbackslash{}mathop\{∑ \}\}\_\{k=2\}\^{}\{n\}\{y\}\_\{ k\} = 1 +\{
\textbackslash{}mathop\{∑ \}\}\_\{k=2\}\^{}\{n\}\{z\}\_\{ k\} +\{
\textbackslash{}mathop\{∑ \}\}\_\{k=2\}\^{}\{n\}(log k − log (k −
1))\%\& \textbackslash{}\textbackslash{} \& =\&
\textbackslash{}mathop\{log\} n + (1 +\{ \textbackslash{}mathop\{∑
\}\}\_\{k=2\}\^{}\{+∞\}\{z\}\_\{ k\}) − \{R\}\_\{n\}(z)
\%\&\textbackslash{}\textbackslash{} \textbackslash{}end\{eqnarray*\}

Mais les théorèmes de comparaison des séries à termes de signes
constants assurent que puisque \{z\}\_\{n\} ∼−\{ 1 \textbackslash{}over
2\{n\}\^{}\{2\}\} , on a \{R\}\_\{n\}(z) ∼−\{ 1 \textbackslash{}over 2\}
\{\textbackslash{}mathop\{ \textbackslash{}mathop\{∑ \}\}
\}\_\{k=n+1\}\^{}\{+∞\}\{ 1 \textbackslash{}over \{k\}\^{}\{2\}\} ∼−\{ 1
\textbackslash{}over 2n\} . Posons alors γ = 1
+\{\textbackslash{}mathop\{ \textbackslash{}mathop\{∑ \}\}
\}\_\{k=2\}\^{}\{+∞\}\{z\}\_\{k\} (la constante d'Euler)~; on obtient

\{\textbackslash{}mathop\{∑ \}\}\_\{k=1\}\^{}\{n\}\{ 1
\textbackslash{}over k\} = log n + γ +\{ 1 \textbackslash{}over 2n\} +
o(\{ 1 \textbackslash{}over n\} )

(en fait il est clair que les techniques ci dessus permettent d'obtenir
un développement à un ordre arbitraire).

Exemple~7.9.4 Nous allons maintenant montrer la formule de Stirling, n!
∼\textbackslash{}sqrt\{2πn\}\{ \{n\}\^{}\{n\} \textbackslash{}over
\{e\}\^{}\{n\}\} . Pour cela posons \{a\}\_\{n\} =\{ n!\{e\}\^{}\{n\}
\textbackslash{}over \{n\}\^{}\{n+1∕2\}\} et \{b\}\_\{n\}
=\textbackslash{}mathop\{ log\} \{a\}\_\{n\} −\textbackslash{}mathop\{
log\} \{a\}\_\{n−1\} (pour n ≥ 2). On a

\textbackslash{}begin\{eqnarray*\}\{ b\}\_\{n\}\& =\&
\textbackslash{}mathop\{log\} \{ \{a\}\_\{n\} \textbackslash{}over
\{a\}\_\{n−1\}\} =\textbackslash{}mathop\{ log\} \{ n!\{e\}\^{}\{n\}\{(n
− 1)\}\^{}\{n−1∕2\} \textbackslash{}over (n −
1)!\{e\}\^{}\{n−1\}\{n\}\^{}\{n+1∕2\}\} \%\&
\textbackslash{}\textbackslash{} \& =\& \textbackslash{}mathop\{log\}
\textbackslash{}left (e\{ \{(n − 1)\}\^{}\{n−1∕2\} \textbackslash{}over
\{n\}\^{}\{n−1∕2\}\} \textbackslash{}right ) = 1 + (n −\{ 1
\textbackslash{}over 2\} )\textbackslash{}mathop\{log\} (1 −\{ 1
\textbackslash{}over n\} )\%\& \textbackslash{}\textbackslash{}
\textbackslash{}end\{eqnarray*\}

d'où \{b\}\_\{n\} = 1 + (n −\{ 1 \textbackslash{}over 2\} )(−\{ 1
\textbackslash{}over n\} −\{ 1 \textbackslash{}over 2\{n\}\^{}\{n\}\}
−\{ 1 \textbackslash{}over 3\{n\}\^{}\{3\}\} + O(\{ 1
\textbackslash{}over \{n\}\^{}\{4\}\} )) = −\{ 1 \textbackslash{}over
12\{n\}\^{}\{2\}\} + O(\{ 1 \textbackslash{}over \{n\}\^{}\{3\}\} ) On
en déduit que la série \textbackslash{}mathop\{\textbackslash{}mathop\{∑
\}\} \{b\}\_\{n\} converge. Soit S sa somme. On a alors
\{\textbackslash{}mathop\{\textbackslash{}mathop\{∑ \}\}
\}\_\{k=2\}\^{}\{n\}\{b\}\_\{k\} = S − \{R\}\_\{n\}(b), mais comme
\{b\}\_\{n\} ∼−\{ 1 \textbackslash{}over 12\{n\}\^{}\{2\}\} , on a
\{R\}\_\{n\}(b) ∼−\{ 1 \textbackslash{}over 12\}
\{\textbackslash{}mathop\{ \textbackslash{}mathop\{∑ \}\}
\}\_\{k=n+1\}\^{}\{+∞\}\{ 1 \textbackslash{}over \{k\}\^{}\{2\}\} ∼−\{ 1
\textbackslash{}over 12n\} . On a d'autre part
\{\textbackslash{}mathop\{\textbackslash{}mathop\{∑ \}\}
\}\_\{k=2\}\^{}\{n\}\{b\}\_\{k\} =\textbackslash{}mathop\{ log\}
\{a\}\_\{n\} −\textbackslash{}mathop\{ log\} \{a\}\_\{1\}, d'où
finalement \textbackslash{}mathop\{log\} \{a\}\_\{n\}
=\{\textbackslash{}mathop\{ \textbackslash{}mathop\{∑ \}\}
\}\_\{k=2\}\^{}\{n\}\{b\}\_\{k\} +\textbackslash{}mathop\{ log\}
\{a\}\_\{1\} = S +\textbackslash{}mathop\{ log\} \{a\}\_\{1\} +\{ 1
\textbackslash{}over 12n\} + o(\{ 1 \textbackslash{}over n\} ) et donc
\{a\}\_\{n\} = \{e\}\^{}\{S+\textbackslash{}mathop\{log\}
\{a\}\_\{1\}\}\textbackslash{}mathop\{ exp\} (\{ 1 \textbackslash{}over
12n\} + o(\{ 1 \textbackslash{}over n\} )) = ℓ(1 +\{ 1
\textbackslash{}over 12n\} + o(\{ 1 \textbackslash{}over n\} )) en
posant ℓ = \{e\}\^{}\{S+\textbackslash{}mathop\{log\} \{a\}\_\{1\}\}
\textgreater{} 0, soit encore

n! = ℓ\{ \{n\}\^{}\{n+1∕2\} \textbackslash{}over n!\}
\textbackslash{}left (1 +\{ 1 \textbackslash{}over 12n\} + o(\{ 1
\textbackslash{}over n\} )\textbackslash{}right )

La méthode précédente ne permet pas d'obtenir la valeur de ℓ~; on
obtient celle ci classiquement à l'aide des intégrales de Wallis~:
\{I\}\_\{n\} =\{\textbackslash{}mathop\{∫ \}
\}\_\{0\}\^{}\{π∕2\}\{\textbackslash{}mathop\{ sin\} \}\^{}\{n\}x dx.
Pour n ≥ 2, on écrit à l'aide d'une intégration par parties, en
intégrant \textbackslash{}mathop\{sin\} x et en dérivant
\{\textbackslash{}mathop\{sin\} \}\^{}\{n−1\}x

\textbackslash{}begin\{eqnarray*\}\{ I\}\_\{n\}\& =\&
\{\textbackslash{}mathop\{∫ \}
\}\_\{0\}\^{}\{π∕2\}\{\textbackslash{}mathop\{ sin\}
\}\^{}\{n−1\}x\textbackslash{}mathop\{sin\} x dx \%\&
\textbackslash{}\textbackslash{} \& =\&\{ \textbackslash{}left
{[}−\textbackslash{}mathop\{cos\} x\{\textbackslash{}mathop\{sin\}
\}\^{}\{n−1\}x\textbackslash{}right {]}\}\_\{ 0\}\^{}\{π∕2\} + (n −
1)\{\textbackslash{}mathop\{∫ \}
\}\_\{0\}\^{}\{π∕2\}\{\textbackslash{}mathop\{ sin\}
\}\^{}\{n−2\}x\{\textbackslash{}mathop\{cos\} \}\^{}\{2\}x dx \%\&
\textbackslash{}\textbackslash{} \& =\& (n −
1)\{\textbackslash{}mathop\{∫ \}
\}\_\{0\}\^{}\{π∕2\}\{\textbackslash{}mathop\{ sin\} \}\^{}\{n−2\}x(1
−\{\textbackslash{}mathop\{ sin\} \}\^{}\{2\}x) dx = (n − 1)(\{I\}\_\{
n−2\} − \{I\}\_\{n\})\%\& \textbackslash{}\textbackslash{}
\textbackslash{}end\{eqnarray*\}

d'où \{I\}\_\{n\} =\{ n−1 \textbackslash{}over n\} \{I\}\_\{n−2\}. En
tenant compte de \{I\}\_\{0\} =\{ π \textbackslash{}over 2\} et
\{I\}\_\{1\} = 1, on a alors

\{I\}\_\{2p\} =\{ (2p − 1)(2p −
3)\textbackslash{}mathop\{\textbackslash{}mathop\{\ldots{}\}\}3.1
\textbackslash{}over (2p)(2p −
2)\textbackslash{}mathop\{\textbackslash{}mathop\{\ldots{}\}\}4.2\} \{ π
\textbackslash{}over 2\} =\{ (2p)! \textbackslash{}over
\{2\}\^{}\{p\}\{(p!)\}\^{}\{2\}\} \{ π \textbackslash{}over 2\}

en multipliant numérateur et dénominateur par (2p)(2p −
2)\textbackslash{}mathop\{\textbackslash{}mathop\{\ldots{}\}\}4.2 de
manière à rétablir les facteurs manquant au numérateur. De même

\{I\}\_\{2p+1\} =\{ (2p)(2p −
2)\textbackslash{}mathop\{\textbackslash{}mathop\{\ldots{}\}\}4.2
\textbackslash{}over (2p + 1)(2p −
1)\textbackslash{}mathop\{\textbackslash{}mathop\{\ldots{}\}\}3\} =\{
\{2\}\^{}\{p\}\{(p!)\}\^{}\{2\} \textbackslash{}over (2p + 1)!\}

On en déduit en utilisant n! ∼ ℓ\textbackslash{}sqrt\{n\}\{
\{n\}\^{}\{n\} \textbackslash{}over n!\}

\{ \{I\}\_\{2p\} \textbackslash{}over \{I\}\_\{2p+1\}\} =\{ (2p +
1)(2p)\{!\}\^{}\{2\} \textbackslash{}over
\{2\}\^{}\{4p\}p\{!\}\^{}\{4\}\} \{ π \textbackslash{}over 2\} ∼\{ (2p +
1)\{ℓ\}\^{}\{2\}(2p)\{(2p)\}\^{}\{4p\}\{e\}\^{}\{4p\}
\textbackslash{}over
\{2\}\^{}\{4p\}\{e\}\^{}\{4p\}\{ℓ\}\^{}\{4\}\{p\}\^{}\{2\}\{p\}\^{}\{4p\}\}
\{ π \textbackslash{}over 2\} ∼\{ 2π \textbackslash{}over
\{ℓ\}\^{}\{2\}\}

Mais d'autre part, on a \textbackslash{}mathop\{∀\}x ∈ {[}0,\{ π
\textbackslash{}over 2\} {]}, 0 ≤\{\textbackslash{}mathop\{ sin\}
\}\^{}\{n+1\}x ≤\{\textbackslash{}mathop\{ sin\} \}\^{}\{n\}x
≤\{\textbackslash{}mathop\{ sin\} \}\^{}\{n−1\}x, soit en intégrant 0 ≤
\{I\}\_\{n+1\} ≤ \{I\}\_\{n\} ≤ \{I\}\_\{n−1\} et en tenant compte de \{
\{I\}\_\{n−1\} \textbackslash{}over \{I\}\_\{n+1\}\} =\{ n+1
\textbackslash{}over n\} , on obtient 1 ≤\{ \{I\}\_\{n\}
\textbackslash{}over \{I\}\_\{n+1\}\} ≤\{ n+1 \textbackslash{}over n\}
soit encore \textbackslash{}mathop\{lim\}\{ \{I\}\_\{n\}
\textbackslash{}over \{I\}\_\{n+1\}\} = 1. On en déduit que \{ 2π
\textbackslash{}over \{ℓ\}\^{}\{2\}\} = 1 et comme ℓ \textgreater{} 0, ℓ
= \textbackslash{}sqrt\{2π\} ce qui achève la démonstration.

{[}\href{coursse42.html}{prev}{]}
{[}\href{coursse42.html\#tailcoursse42.html}{prev-tail}{]}
{[}\href{coursse43.html}{front}{]}
{[}\href{coursch8.html\#coursse43.html}{up}{]}

\end{document}

% \documentclass[]{article}
\usepackage[T1]{fontenc}
\usepackage{lmodern}
\usepackage{amssymb,amsmath}
\usepackage{ifxetex,ifluatex}
\usepackage{fixltx2e} % provides \textsubscript
% use upquote if available, for straight quotes in verbatim environments
\IfFileExists{upquote.sty}{\usepackage{upquote}}{}
\ifnum 0\ifxetex 1\fi\ifluatex 1\fi=0 % if pdftex
  \usepackage[utf8]{inputenc}
\else % if luatex or xelatex
  \ifxetex
    \usepackage{mathspec}
    \usepackage{xltxtra,xunicode}
  \else
    \usepackage{fontspec}
  \fi
  \defaultfontfeatures{Mapping=tex-text,Scale=MatchLowercase}
  \newcommand{\euro}{€}
\fi
% use microtype if available
\IfFileExists{microtype.sty}{\usepackage{microtype}}{}
\ifxetex
  \usepackage[setpagesize=false, % page size defined by xetex
              unicode=false, % unicode breaks when used with xetex
              xetex]{hyperref}
\else
  \usepackage[unicode=true]{hyperref}
\fi
\hypersetup{breaklinks=true,
            bookmarks=true,
            pdfauthor={},
            pdftitle={Monotonie, continuite},
            colorlinks=true,
            citecolor=blue,
            urlcolor=blue,
            linkcolor=magenta,
            pdfborder={0 0 0}}
\urlstyle{same}  % don't use monospace font for urls
\setlength{\parindent}{0pt}
\setlength{\parskip}{6pt plus 2pt minus 1pt}
\setlength{\emergencystretch}{3em}  % prevent overfull lines
\setcounter{secnumdepth}{0}
 
/* start css.sty */
.cmr-5{font-size:50%;}
.cmr-7{font-size:70%;}
.cmmi-5{font-size:50%;font-style: italic;}
.cmmi-7{font-size:70%;font-style: italic;}
.cmmi-10{font-style: italic;}
.cmsy-5{font-size:50%;}
.cmsy-7{font-size:70%;}
.cmex-7{font-size:70%;}
.cmex-7x-x-71{font-size:49%;}
.msbm-7{font-size:70%;}
.cmtt-10{font-family: monospace;}
.cmti-10{ font-style: italic;}
.cmbx-10{ font-weight: bold;}
.cmr-17x-x-120{font-size:204%;}
.cmsl-10{font-style: oblique;}
.cmti-7x-x-71{font-size:49%; font-style: italic;}
.cmbxti-10{ font-weight: bold; font-style: italic;}
p.noindent { text-indent: 0em }
td p.noindent { text-indent: 0em; margin-top:0em; }
p.nopar { text-indent: 0em; }
p.indent{ text-indent: 1.5em }
@media print {div.crosslinks {visibility:hidden;}}
a img { border-top: 0; border-left: 0; border-right: 0; }
center { margin-top:1em; margin-bottom:1em; }
td center { margin-top:0em; margin-bottom:0em; }
.Canvas { position:relative; }
li p.indent { text-indent: 0em }
.enumerate1 {list-style-type:decimal;}
.enumerate2 {list-style-type:lower-alpha;}
.enumerate3 {list-style-type:lower-roman;}
.enumerate4 {list-style-type:upper-alpha;}
div.newtheorem { margin-bottom: 2em; margin-top: 2em;}
.obeylines-h,.obeylines-v {white-space: nowrap; }
div.obeylines-v p { margin-top:0; margin-bottom:0; }
.overline{ text-decoration:overline; }
.overline img{ border-top: 1px solid black; }
td.displaylines {text-align:center; white-space:nowrap;}
.centerline {text-align:center;}
.rightline {text-align:right;}
div.verbatim {font-family: monospace; white-space: nowrap; text-align:left; clear:both; }
.fbox {padding-left:3.0pt; padding-right:3.0pt; text-indent:0pt; border:solid black 0.4pt; }
div.fbox {display:table}
div.center div.fbox {text-align:center; clear:both; padding-left:3.0pt; padding-right:3.0pt; text-indent:0pt; border:solid black 0.4pt; }
div.minipage{width:100%;}
div.center, div.center div.center {text-align: center; margin-left:1em; margin-right:1em;}
div.center div {text-align: left;}
div.flushright, div.flushright div.flushright {text-align: right;}
div.flushright div {text-align: left;}
div.flushleft {text-align: left;}
.underline{ text-decoration:underline; }
.underline img{ border-bottom: 1px solid black; margin-bottom:1pt; }
.framebox-c, .framebox-l, .framebox-r { padding-left:3.0pt; padding-right:3.0pt; text-indent:0pt; border:solid black 0.4pt; }
.framebox-c {text-align:center;}
.framebox-l {text-align:left;}
.framebox-r {text-align:right;}
span.thank-mark{ vertical-align: super }
span.footnote-mark sup.textsuperscript, span.footnote-mark a sup.textsuperscript{ font-size:80%; }
div.tabular, div.center div.tabular {text-align: center; margin-top:0.5em; margin-bottom:0.5em; }
table.tabular td p{margin-top:0em;}
table.tabular {margin-left: auto; margin-right: auto;}
div.td00{ margin-left:0pt; margin-right:0pt; }
div.td01{ margin-left:0pt; margin-right:5pt; }
div.td10{ margin-left:5pt; margin-right:0pt; }
div.td11{ margin-left:5pt; margin-right:5pt; }
table[rules] {border-left:solid black 0.4pt; border-right:solid black 0.4pt; }
td.td00{ padding-left:0pt; padding-right:0pt; }
td.td01{ padding-left:0pt; padding-right:5pt; }
td.td10{ padding-left:5pt; padding-right:0pt; }
td.td11{ padding-left:5pt; padding-right:5pt; }
table[rules] {border-left:solid black 0.4pt; border-right:solid black 0.4pt; }
.hline hr, .cline hr{ height : 1px; margin:0px; }
.tabbing-right {text-align:right;}
span.TEX {letter-spacing: -0.125em; }
span.TEX span.E{ position:relative;top:0.5ex;left:-0.0417em;}
a span.TEX span.E {text-decoration: none; }
span.LATEX span.A{ position:relative; top:-0.5ex; left:-0.4em; font-size:85%;}
span.LATEX span.TEX{ position:relative; left: -0.4em; }
div.float img, div.float .caption {text-align:center;}
div.figure img, div.figure .caption {text-align:center;}
.marginpar {width:20%; float:right; text-align:left; margin-left:auto; margin-top:0.5em; font-size:85%; text-decoration:underline;}
.marginpar p{margin-top:0.4em; margin-bottom:0.4em;}
.equation td{text-align:center; vertical-align:middle; }
td.eq-no{ width:5%; }
table.equation { width:100%; } 
div.math-display, div.par-math-display{text-align:center;}
math .texttt { font-family: monospace; }
math .textit { font-style: italic; }
math .textsl { font-style: oblique; }
math .textsf { font-family: sans-serif; }
math .textbf { font-weight: bold; }
.partToc a, .partToc, .likepartToc a, .likepartToc {line-height: 200%; font-weight:bold; font-size:110%;}
.chapterToc a, .chapterToc, .likechapterToc a, .likechapterToc, .appendixToc a, .appendixToc {line-height: 200%; font-weight:bold;}
.index-item, .index-subitem, .index-subsubitem {display:block}
.caption td.id{font-weight: bold; white-space: nowrap; }
table.caption {text-align:center;}
h1.partHead{text-align: center}
p.bibitem { text-indent: -2em; margin-left: 2em; margin-top:0.6em; margin-bottom:0.6em; }
p.bibitem-p { text-indent: 0em; margin-left: 2em; margin-top:0.6em; margin-bottom:0.6em; }
.paragraphHead, .likeparagraphHead { margin-top:2em; font-weight: bold;}
.subparagraphHead, .likesubparagraphHead { font-weight: bold;}
.quote {margin-bottom:0.25em; margin-top:0.25em; margin-left:1em; margin-right:1em; text-align:justify;}
.verse{white-space:nowrap; margin-left:2em}
div.maketitle {text-align:center;}
h2.titleHead{text-align:center;}
div.maketitle{ margin-bottom: 2em; }
div.author, div.date {text-align:center;}
div.thanks{text-align:left; margin-left:10%; font-size:85%; font-style:italic; }
div.author{white-space: nowrap;}
.quotation {margin-bottom:0.25em; margin-top:0.25em; margin-left:1em; }
h1.partHead{text-align: center}
.sectionToc, .likesectionToc {margin-left:2em;}
.subsectionToc, .likesubsectionToc {margin-left:4em;}
.subsubsectionToc, .likesubsubsectionToc {margin-left:6em;}
.frenchb-nbsp{font-size:75%;}
.frenchb-thinspace{font-size:75%;}
.figure img.graphics {margin-left:10%;}
/* end css.sty */

\title{Monotonie, continuite}
\author{}
\date{}

\begin{document}
\maketitle

\textbf{Warning: \href{http://www.math.union.edu/locate/jsMath}{jsMath}
requires JavaScript to process the mathematics on this page.\\ If your
browser supports JavaScript, be sure it is enabled.}

\begin{center}\rule{3in}{0.4pt}\end{center}

{[}\href{coursse45.html}{next}{]}
{[}\hyperref[tailcoursse44.html]{tail}{]}
{[}\href{coursch9.html\#coursse44.html}{up}{]}

\subsubsection{8.1 Monotonie, continuité}

\paragraph{8.1.1 Limites et monotonie}

Proposition~8.1.1 Soit I un intervalle de ℝ et f : I → ℝ croissante.
Alors

\begin{itemize}
\itemsep1pt\parskip0pt\parsep0pt
\item
  (i) f admet en tout point a de I (dans la mesure où cela a un sens)
  une limite à gauche f(a−) et une limite à droite f(a+) dans ℝ avec
  f(a−) ≤ f(a) ≤ f(a+)
\item
  (ii) f admet en l'extrémité droite b de I une limite si et seulement
  si~elle est majorée sur I~; dans le cas contraire
  \{\textbackslash{}mathop\{lim\}\}\_\{x→b,x\textless{}b\}f(x) = +∞
\item
  (iii) f admet en l'extrémité gauche a de I une limite si et seulement
  si~elle est minorée sur I~; dans le cas contraire
  \{\textbackslash{}mathop\{lim\}\}\_\{x→a,x\textgreater{}a\}f(x) = −∞
\end{itemize}

Démonstration (i) Supposons que a n'est pas l'extrémité gauche de I.
Pour x \textless{} a on a f(x) ≤ f(a). Soit m
=\{\textbackslash{}mathop\{ sup\}\}\_\{x\textless{}a\}f(x) ≤ f(a). Soit
ε \textgreater{} 0. Il existe \{x\}\_\{0\} \textless{} a tel que m − ε
\textless{} f(\{x\}\_\{0\}) ≤ m. Alors x ∈{]}\{x\}\_\{0\},a{[}⇒ m − ε
\textless{} f(\{x\}\_\{0\}) ≤ f(x) ≤ m et donc m
=\{\textbackslash{}mathop\{ lim\}\}\_\{x→a,x\textless{}a\}f(x). De même,
si a n'est pas l'extrémité gauche de I et si M
=\{\textbackslash{}mathop\{ inf\} \}\_\{x\textgreater{}a\}f(x) ≥ f(a),
on a M =\{\textbackslash{}mathop\{
lim\}\}\_\{x→a,x\textgreater{}a\}f(x).

(ii) f admet de toute fa\textbackslash{}c\{c\}on dans
\textbackslash{}overline\{ℝ\} la limite
\{\textbackslash{}mathop\{sup\}\}\_\{x∈I\}f(x) (comme ci dessus)~; cette
limite est dans ℝ si et seulement si~f est majorée sur I~; similaire
pour (iii).

Remarque~8.1.1 On a un résultat similaire pour les applications
décroissantes~:

Proposition~8.1.2 Soit I un intervalle de ℝ et f : I → ℝ décroissante.
Alors (i) f admet en tout point a de I (dans la mesure où cela a un
sens) une limite à gauche f(a−) et une limite à droite f(a+) dans ℝ avec
f(a−) ≥ f(a) ≥ f(a+) (ii) f admet en l'extrémité droite b de I une
limite si et seulement si~elle est minorée sur I~; dans le cas contraire
\{\textbackslash{}mathop\{lim\}\}\_\{x→b,x\textless{}b\}f(x) = −∞ (iii)
f admet en l'extrémité gauche a de I une limite si et seulement si~elle
est majorée sur I~; dans le cas contraire
\{\textbackslash{}mathop\{lim\}\}\_\{x→a,x\textgreater{}a\}f(x) = +∞

\paragraph{8.1.2 Continuité et monotonie}

Lemme~8.1.3 Soit I un intervalle de ℝ et f : I → ℝ monotone. Alors f est
continue si et seulement si~f(I) est un intervalle.

Démonstration La condition est évidemment nécessaire d'après le théorème
des valeurs intermédiaires. Inversement supposons que f(I) est un
intervalle et a ∈ I. On peut par exemple supposer que f est croissante.
Supposons que f(a) \textless{} f(a+) (ce qui sous entend que a n'est pas
l'extrémité droite de I). Soit x \textgreater{} a. On a alors f(a)
\textless{} f(a+) ≤ f(x) (puisque f(a+) =\{\textbackslash{}mathop\{
inf\} \}\_\{t\textgreater{}a\}f(t)). En particulier {]}f(a),f(a+){[}⊂
{[}f(a),f(x){]} ⊂ f(I) (convexité des intervalles). Soit alors y
∈{]}f(a),f(a+){[}~; on peut poser y = f(t) pour t ∈ I. Mais si t ≤ a, on
a y = f(t) ≤ f(a) et si t \textgreater{} a on a y = f(t) ≥ f(a+). C'est
absurde. Donc f(a) = f(a+). On montre de même que si a n'est pas
l'extrémité gauche de I, f(a) = f(a−). Donc f est continue sur I.

Théorème~8.1.4 Soit I un intervalle de ℝ et f : I → ℝ continue
strictement monotone. Alors J = f(I) est un intervalle de ℝ et f induit
un homéomorphisme de I sur J.

Démonstration On sait déjà que J est un intervalle~; alors
\{f\}\^{}\{−1\} : J → I est encore strictement monotone et
\{f\}\^{}\{−1\}(J) = I est un intervalle, donc \{f\}\^{}\{−1\} est
continue. Donc f induit un homéomorphisme de I sur J.

Le théorème suivant montre que réciproquement, la condition de stricte
monotonie est une condition nécessaire pour un homéomorphisme d'un
intervalle sur un autre.

Théorème~8.1.5 Soit I un intervalle de ℝ et f : I → ℝ continue. Alors f
est injective si et seulement si~elle est strictement monotone.

Démonstration La condition est évidemment suffisante. Inversement,
supposons f continue et injective. Soit X = \textbackslash{}\{(x,y) ∈ I
× I\textbackslash{}mathrel\{∣\}x \textless{} y\textbackslash{}\} et g :
X → ℝ définie par g(x,y) = f(y) − f(x). Alors X est connexe (car
convexe) et g est continue. L'ensemble g(X) est donc un intervalle de ℝ
et cet intervalle ne contient pas 0 car g est injective. Donc soit g(X)
⊂{]}0,+∞{[} (auquel cas f est strictement croissante), soit g(X) ⊂{]}
−∞,0{[} (auquel cas f est strictement décroissante).

{[}\href{coursse45.html}{next}{]} {[}\href{coursse44.html}{front}{]}
{[}\href{coursch9.html\#coursse44.html}{up}{]}

\end{document}

% \documentclass[]{article}
\usepackage[T1]{fontenc}
\usepackage{lmodern}
\usepackage{amssymb,amsmath}
\usepackage{ifxetex,ifluatex}
\usepackage{fixltx2e} % provides \textsubscript
% use upquote if available, for straight quotes in verbatim environments
\IfFileExists{upquote.sty}{\usepackage{upquote}}{}
\ifnum 0\ifxetex 1\fi\ifluatex 1\fi=0 % if pdftex
  \usepackage[utf8]{inputenc}
\else % if luatex or xelatex
  \ifxetex
    \usepackage{mathspec}
    \usepackage{xltxtra,xunicode}
  \else
    \usepackage{fontspec}
  \fi
  \defaultfontfeatures{Mapping=tex-text,Scale=MatchLowercase}
  \newcommand{\euro}{€}
\fi
% use microtype if available
\IfFileExists{microtype.sty}{\usepackage{microtype}}{}
\ifxetex
  \usepackage[setpagesize=false, % page size defined by xetex
              unicode=false, % unicode breaks when used with xetex
              xetex]{hyperref}
\else
  \usepackage[unicode=true]{hyperref}
\fi
\hypersetup{breaklinks=true,
            bookmarks=true,
            pdfauthor={},
            pdftitle={Derivee},
            colorlinks=true,
            citecolor=blue,
            urlcolor=blue,
            linkcolor=magenta,
            pdfborder={0 0 0}}
\urlstyle{same}  % don't use monospace font for urls
\setlength{\parindent}{0pt}
\setlength{\parskip}{6pt plus 2pt minus 1pt}
\setlength{\emergencystretch}{3em}  % prevent overfull lines
\setcounter{secnumdepth}{0}
 
/* start css.sty */
.cmr-5{font-size:50%;}
.cmr-7{font-size:70%;}
.cmmi-5{font-size:50%;font-style: italic;}
.cmmi-7{font-size:70%;font-style: italic;}
.cmmi-10{font-style: italic;}
.cmsy-5{font-size:50%;}
.cmsy-7{font-size:70%;}
.cmex-7{font-size:70%;}
.cmex-7x-x-71{font-size:49%;}
.msbm-7{font-size:70%;}
.cmtt-10{font-family: monospace;}
.cmti-10{ font-style: italic;}
.cmbx-10{ font-weight: bold;}
.cmr-17x-x-120{font-size:204%;}
.cmsl-10{font-style: oblique;}
.cmti-7x-x-71{font-size:49%; font-style: italic;}
.cmbxti-10{ font-weight: bold; font-style: italic;}
p.noindent { text-indent: 0em }
td p.noindent { text-indent: 0em; margin-top:0em; }
p.nopar { text-indent: 0em; }
p.indent{ text-indent: 1.5em }
@media print {div.crosslinks {visibility:hidden;}}
a img { border-top: 0; border-left: 0; border-right: 0; }
center { margin-top:1em; margin-bottom:1em; }
td center { margin-top:0em; margin-bottom:0em; }
.Canvas { position:relative; }
li p.indent { text-indent: 0em }
.enumerate1 {list-style-type:decimal;}
.enumerate2 {list-style-type:lower-alpha;}
.enumerate3 {list-style-type:lower-roman;}
.enumerate4 {list-style-type:upper-alpha;}
div.newtheorem { margin-bottom: 2em; margin-top: 2em;}
.obeylines-h,.obeylines-v {white-space: nowrap; }
div.obeylines-v p { margin-top:0; margin-bottom:0; }
.overline{ text-decoration:overline; }
.overline img{ border-top: 1px solid black; }
td.displaylines {text-align:center; white-space:nowrap;}
.centerline {text-align:center;}
.rightline {text-align:right;}
div.verbatim {font-family: monospace; white-space: nowrap; text-align:left; clear:both; }
.fbox {padding-left:3.0pt; padding-right:3.0pt; text-indent:0pt; border:solid black 0.4pt; }
div.fbox {display:table}
div.center div.fbox {text-align:center; clear:both; padding-left:3.0pt; padding-right:3.0pt; text-indent:0pt; border:solid black 0.4pt; }
div.minipage{width:100%;}
div.center, div.center div.center {text-align: center; margin-left:1em; margin-right:1em;}
div.center div {text-align: left;}
div.flushright, div.flushright div.flushright {text-align: right;}
div.flushright div {text-align: left;}
div.flushleft {text-align: left;}
.underline{ text-decoration:underline; }
.underline img{ border-bottom: 1px solid black; margin-bottom:1pt; }
.framebox-c, .framebox-l, .framebox-r { padding-left:3.0pt; padding-right:3.0pt; text-indent:0pt; border:solid black 0.4pt; }
.framebox-c {text-align:center;}
.framebox-l {text-align:left;}
.framebox-r {text-align:right;}
span.thank-mark{ vertical-align: super }
span.footnote-mark sup.textsuperscript, span.footnote-mark a sup.textsuperscript{ font-size:80%; }
div.tabular, div.center div.tabular {text-align: center; margin-top:0.5em; margin-bottom:0.5em; }
table.tabular td p{margin-top:0em;}
table.tabular {margin-left: auto; margin-right: auto;}
div.td00{ margin-left:0pt; margin-right:0pt; }
div.td01{ margin-left:0pt; margin-right:5pt; }
div.td10{ margin-left:5pt; margin-right:0pt; }
div.td11{ margin-left:5pt; margin-right:5pt; }
table[rules] {border-left:solid black 0.4pt; border-right:solid black 0.4pt; }
td.td00{ padding-left:0pt; padding-right:0pt; }
td.td01{ padding-left:0pt; padding-right:5pt; }
td.td10{ padding-left:5pt; padding-right:0pt; }
td.td11{ padding-left:5pt; padding-right:5pt; }
table[rules] {border-left:solid black 0.4pt; border-right:solid black 0.4pt; }
.hline hr, .cline hr{ height : 1px; margin:0px; }
.tabbing-right {text-align:right;}
span.TEX {letter-spacing: -0.125em; }
span.TEX span.E{ position:relative;top:0.5ex;left:-0.0417em;}
a span.TEX span.E {text-decoration: none; }
span.LATEX span.A{ position:relative; top:-0.5ex; left:-0.4em; font-size:85%;}
span.LATEX span.TEX{ position:relative; left: -0.4em; }
div.float img, div.float .caption {text-align:center;}
div.figure img, div.figure .caption {text-align:center;}
.marginpar {width:20%; float:right; text-align:left; margin-left:auto; margin-top:0.5em; font-size:85%; text-decoration:underline;}
.marginpar p{margin-top:0.4em; margin-bottom:0.4em;}
.equation td{text-align:center; vertical-align:middle; }
td.eq-no{ width:5%; }
table.equation { width:100%; } 
div.math-display, div.par-math-display{text-align:center;}
math .texttt { font-family: monospace; }
math .textit { font-style: italic; }
math .textsl { font-style: oblique; }
math .textsf { font-family: sans-serif; }
math .textbf { font-weight: bold; }
.partToc a, .partToc, .likepartToc a, .likepartToc {line-height: 200%; font-weight:bold; font-size:110%;}
.chapterToc a, .chapterToc, .likechapterToc a, .likechapterToc, .appendixToc a, .appendixToc {line-height: 200%; font-weight:bold;}
.index-item, .index-subitem, .index-subsubitem {display:block}
.caption td.id{font-weight: bold; white-space: nowrap; }
table.caption {text-align:center;}
h1.partHead{text-align: center}
p.bibitem { text-indent: -2em; margin-left: 2em; margin-top:0.6em; margin-bottom:0.6em; }
p.bibitem-p { text-indent: 0em; margin-left: 2em; margin-top:0.6em; margin-bottom:0.6em; }
.paragraphHead, .likeparagraphHead { margin-top:2em; font-weight: bold;}
.subparagraphHead, .likesubparagraphHead { font-weight: bold;}
.quote {margin-bottom:0.25em; margin-top:0.25em; margin-left:1em; margin-right:1em; text-align:\jmathustify;}
.verse{white-space:nowrap; margin-left:2em}
div.maketitle {text-align:center;}
h2.titleHead{text-align:center;}
div.maketitle{ margin-bottom: 2em; }
div.author, div.date {text-align:center;}
div.thanks{text-align:left; margin-left:10%; font-size:85%; font-style:italic; }
div.author{white-space: nowrap;}
.quotation {margin-bottom:0.25em; margin-top:0.25em; margin-left:1em; }
h1.partHead{text-align: center}
.sectionToc, .likesectionToc {margin-left:2em;}
.subsectionToc, .likesubsectionToc {margin-left:4em;}
.subsubsectionToc, .likesubsubsectionToc {margin-left:6em;}
.frenchb-nbsp{font-size:75%;}
.frenchb-thinspace{font-size:75%;}
.figure img.graphics {margin-left:10%;}
/* end css.sty */

\title{Derivee}
\author{}
\date{}

\begin{document}
\maketitle

\textbf{Warning: 
requires JavaScript to process the mathematics on this page.\\ If your
browser supports JavaScript, be sure it is enabled.}

\begin{center}\rule{3in}{0.4pt}\end{center}

{[}
{[}
{[}{]}
{[}

\subsubsection{8.2 Dérivée}

\paragraph{8.2.1 Notion de dérivée}

Définition~8.2.1 Soit I un intervalle de \mathbb{R}~, E un espace vectoriel
normé~et f : I \rightarrow~ E. On dit que f est dérivable en a \in I si existe
lim\_x\rightarrow~a,x\neq~a~
f(x)-f(a) \over x-a ~; dans ce cas cette limite est
appelée la dérivée de f au point a et notée f'(a).

Remarque~8.2.1 Comme toute notion de limite, il s'agit d'une notion
locale~: f : I \rightarrow~ E est dérivable au point a si et seulement si~sa
restriction à {]}a - \eta,a + \eta{[}\bigcapI est dérivable au point a.

Proposition~8.2.1 Si f est dérivable au point a \in I elle est continue au
point a.

Démonstration On écrit pour x\neq~a,  f(x)-f(a)
\over x-a = f'(a) + \epsilon(x - a) avec
lim\_h\rightarrow~0~\epsilon(h) = 0~; on a donc f(x) =
f(a) + (x - a)f'(a) + (x - a)\epsilon(x - a) ce qui montre que
lim\_x\rightarrow~a,x\neq~a~f(x)
= f(a)~; donc f est continue au point a.

Définition~8.2.2 Soit I un intervalle de \mathbb{R}~, E un espace vectoriel
normé~et f : I \rightarrow~ E. On dit que f est dérivable si elle est dérivable en
tout point de I~; l'application f' : a\mapsto~f'(a)
est appelée la dérivée de f.

Remarque~8.2.2 On a donc~: f dérivable \rigtharrow~ f continue.

\paragraph{8.2.2 Opérations sur les dérivées}

Théorème~8.2.2 Soit I un intervalle de \mathbb{R}~, E un espace vectoriel normé~et
f et g des applications de I dans E dérivables au point a~; si \alpha~ et \beta~
sont des scalaires, \alpha~f + \beta~g est dérivable au point a et (\alpha~f + \beta~g)'(a) =
\alpha~f'(a) + \beta~g'(a).

Démonstration Il suffit de remarquer que  (\alpha~f+\beta~g)(x)-(\alpha~f+\beta~g)(a)
\over x-a = \alpha~ f(x)-f(a) \over x-a +
\beta~ g(x)-g(a) \over x-a et d'appliquer les théorèmes
sur les limites.

Théorème~8.2.3 Soit I un intervalle de \mathbb{R}~, E,F,G trois espaces vectoriels
normés, f : I \rightarrow~ E, g : I \rightarrow~ F~; soit u : E \times F \rightarrow~ G une application
bilinéaire continue et h : I \rightarrow~ G,
t\mapsto~u(f(t),g(t)). Si f et g sont dérivables au
point a, alors h est dérivable au point a et h'(a) = u(f'(a),g(a)) +
u(f(a),g'(a)).

Démonstration On vérifie immédiatement que  h(x)-h(a)
\over x-a = u( f(x)-f(a) \over x-a
,g(x)) + u(f(a), g(x)-g(a) \over x-a ). Or
lim\_x\rightarrow~a,x\neq~a~
f(x)-f(a) \over x-a = f'(a),
lim\_x\rightarrow~a,x\neq~a~g(x)
= g(a) et
lim\_x\rightarrow~a,x\neq~a~
g(x)-g(a) \over x-a = g'(a). Comme u est continue, on a
lim\_x\rightarrow~a,x\neq~a~
h(x)-h(a) \over x-a = u(f'(a),g(a)) + u(f(a),g'(a)).

Remarque~8.2.3 Ce théorème s'étend immédiatement au cas d'une
application p-linéaire continue u : E\_1
\times⋯ \times E\_p dans G. Dans ce cas on a

h'(a) = \sum \_i=1^pu(f~\_
1(a),\ldots,f\_i-1(a),f\_i'(a),f\_i+1(a),\\ldots,f\_p~(a))

En particulier, dans le cas d'un déterminant on retiendra

\begin{align*}
{[}\mathrm{det}~
\_\mathcal{E}(f\_1,\\ldots,f\_n~){]}'(a)&&
\%& \\ & =& \\sum
\_i=1^n \mathrm{det} \_
\mathcal{E}(f\_1(a),\ldots,f\_i-1(a),f\_i'(a),f\_i+1(a),\\ldots,f\_n~(a))\%&
\\ \end{align*}

Théorème~8.2.4 Soit \phi : I \rightarrow~ \mathbb{R}~ et f : J \rightarrow~ E avec \phi(I) \subset~ J~; soit a \in I.
Si \phi est dérivable au point a et si f est dérivable au point f(a), alors
f \cdot \phi est dérivable au point a et (f \cdot \phi)'(a) = \phi'(a)f'(\phi(a)).

Démonstration En effet la dérivabilité de f au point \phi(a) peut se
traduire par

f(x) - f(\phi(a)) = (x - \phi(a))f'(\phi(a)) + (x - \phi(a))\epsilon(x)

avec lim\_x\rightarrow~\phi(a)~\epsilon(x) = 0. On a donc

f(\phi(t)) - f(\phi(a)) = (\phi(t) - \phi(a))f'(\phi(a)) + (\phi(t) - \phi(a))\epsilon(\phi(t))

avec lim\_t\rightarrow~a~\epsilon(\phi(t)) = 0 puisque \phi est
continue au point a.

De même la dérivabilité de \phi au point a se traduit par

\phi(t) - \phi(a) = (t - a)\phi'(a) + o(t - a)

On obtient alors en rempla\ccant

f(\phi(t)) - f(\phi(a)) = (t - a)\phi'(a)f'(\phi(a)) + o(t - a)

ce qui montre que

lim\_t\rightarrow~a,t\neq~a~f(\phi(t))
- f(\phi(a))\over t - a = \phi'(a)f'(\phi(a))

Théorème~8.2.5 Soit f : I \rightarrow~ \mathbb{R}~, a \in I tel que
f(a)\neq~0. Si f est dérivable au point a, il
existe \epsilon \textgreater{} 0 tel que f ne s'annule pas sur J = I\bigcap{]}a - \epsilon,a
+ \epsilon{[}. La fonction  1 \over f est dérivable au point
a et \left ( 1 \over f
\right )'(a) = - f'(a) \over
f(a)^2 .

Démonstration La fonction f étant continue au point a, il existe \epsilon
\textgreater{} 0 tel que t \in I\bigcap{]}a - \epsilon,a + \epsilon{[}\rigtharrow~\textbar{}f(t) -
f(a)\textbar{} \textless{} \textbar{}f(a)\textbar{}
\over 2 ~; on en déduit que t \in J \rigtharrow~
f(t)\neq~0. Pour t \in J
\diagdown\a\ on a  1 \over
t-a \left ( 1 \over f (t) - 1
\over f (a)\right ) = - 1
\over f(t)f(a)  f(t)-f(a) \over t-a
qui tend vers - f'(a) \over f(a)^2 quand t
tend vers a.

\paragraph{8.2.3 Dérivées d'ordre supérieur}

Définition~8.2.3 Soit I un intervalle de \mathbb{R}~, E un espace vectoriel
normé~et f : I \rightarrow~ E. Soit n ≥ 1. On dit que f est n fois dérivable au
point a \in I s'il existe \eta \textgreater{} 0 tel que f est n - 1 fois
dérivable sur I\bigcap{]}a - \eta,a + \eta{[} et si l'application f^(n-1)
est dérivable au point a~; on pose alors f^(n)(a) =
(f^(n-1))'(a). On dit que f est n fois dérivable sur I si
elle est n fois dérivable en tout point de I~; on dit que f est de
classe C^n si elle est n fois dérivable sur I et si
f^(n) est continue sur I~; on dit que f est C^\infty~ si
elle est de classe C^n pour tout n.

Remarque~8.2.4 Puisque toute fonction dérivable est continue, si f est n
fois dérivable, elle est de classe C^n-1.

Théorème~8.2.6 (Leibnitz). Soit I un intervalle de \mathbb{R}~, E,F,G trois
espaces vectoriels normés, f : I \rightarrow~ E, g : I \rightarrow~ F~; soit u : E \times F \rightarrow~ G une
application bilinéaire continue et h : I \rightarrow~ G,
t\mapsto~u(f(t),g(t)). Si f et g sont n fois
dérivables au point a, alors h est n fois dérivable au point a et

h^(n)(a) = \\sum
\_p=0^nC\_
n^pu(f^(p)(a),g^(n-p)(a))

Démonstration Par récurrence sur n~; le résultat a dé\jmathà été vu pour n =
1~; supposons le vrai pour n - 1 et soit \epsilon \textgreater{} 0 tel que f et
g soient n - 1 fois dérivables sur I\bigcap{]}a - \eta,a + \eta{[}. L'hypothèse de
récurrence implique que h est n - 1 fois dérivable sur I\bigcap{]}a - \eta,a +
\eta{[} et que sa dérivée (n - 1)^\textième
est donnée par

h^(n-1)(t) = \\sum
\_p=0^n-1C\_
n-1^pu(f^(p)(t),g^(n-1-p)(t))

Mais toutes les applications f^(p), g^(n-1-p) sont
dérivables au point a~; il en est donc de même des applications
t\mapsto~u(f^(p)(t),g^(n-1-p)(t)),
et donc de h^(n-1). Donc h est n fois dérivable au point a et

\begin{align*} h^(n)(a)& =&
\sum \_p=0^n-1C~\_
n-1^p(u(f^(p+1)(a),g^(n-1-p)(a))\%&
\\ & \text &
\quad + u(f^(p)(a),g^(n-p)(a)))
\%& \\ & =& \\sum
\_p=1^nC\_
n-1^p-1u(f^(p)(a),g^(n-p)(a)) \%&
\\ & \text &
\quad + \\sum
\_p=0^n-1C\_
n-1^pu(f^(p)(a),g^(n-p)(a)) \%&
\\ \end{align*}

en changeant dans la première somme p + 1 en p~; puis

\begin{align*} h^(n)(a)& =&
u(f(a),g^(n)(a)) \%& \\ &
\text & +\\sum
\_p=1^n-1(C\_ n-1^p-1 + C\_
n-1^p)u(f^(p)(a),g^(n-p)(a))\%&
\\ & \text &
+u(f^(n)(a),g(a)) \%& \\ &
=& \sum \_p=0^nC~\_
n^pu(f^(p)(a),g^(n-p)(a)) \%&
\\ \end{align*}

ce qui achève la récurrence.

Corollaire~8.2.7 Sous les mêmes hypothèses, si f et g sont de classe
C^n, h est de classe C^n.

Démonstration C'est clair d'après la formule ci dessus.

Théorème~8.2.8 Soit \phi : I \rightarrow~ \mathbb{R}~ et f : J \rightarrow~ E avec \phi(I) \subset~ J~; soit a \in I.
Si \phi est n fois dérivable au point a et si f est n fois dérivable au
point f(a), alors f \cdot \phi est n fois dérivable au point a.

Démonstration Par récurrence sur n~; le résultat a dé\jmathà été vu pour n =
1~; supposons le vrai pour n - 1 et soit \eta \textgreater{} 0 tel que f \cdot
\phi soit dérivable sur I\bigcap{]}a - \eta,a + \eta{[} avec (f \cdot \phi)' = \phi'(f' \cdot \phi).
Comme f' et \phi sont n - 1 fois dérivables en a, l'hypothèse de récurrence
implique que f' \cdot \phi est n - 1 fois dérivable en a~; comme \phi' l'est
également, le théorème de Leibnitz appliqué au produit ordinaire assure
que (f \cdot \phi)' = \phi'(f' \cdot \phi) est n - 1 fois dérivable au point a, donc que
f \cdot \phi est n fois dérivable au point a.

Corollaire~8.2.9 Sous les mêmes hypothèses, si f et \phi sont de classe
C^n, f \cdot \phi est de classe C^n.

{[}
{[}
{[}
{[}

\end{document}

% 
\subsubsection{8.3 Fonctions réelles d'une variable réelle}

\subsubsection{Théorème de Rolle, formule des accroissements finis}
\label{sec:theoreme-de-rolle}



Lemme~8.3.1 Soit f : I \rightarrow~ \mathbb{R}~~; si f admet en c \in I^o un
extremum local et si f est dérivable au point c, alors f'(c) = 0.

Démonstration Supposons par exemple que f a en c un maximum local. Pour
c - \eta < x < c, on a  f(x)-f(c)
\over x-c ≥ 0 d'où en faisant tendre x vers c, f'(c) ≥
0. Pour c < x < c + \eta, on a  f(x)-f(c)
\over x-c \leq 0 d'où en faisant tendre x vers c, f'(c) \leq
0. On a donc f'(c) = 0.

\begin{thm}[Rolle]
   Soit f : [a,b] \rightarrow~ \mathbb{R}~, continue sur [a,b],
dérivable sur ]a,b[ telle que f(a) = f(b). Alors il existe c
\in]a,b[ tel que f'(c) = 0.

\end{thm}
Démonstration Si f est constante sur [a,b], n'importe quel c
\in]a,b[ convient. Sinon, par exemple, il existe x \in [a,b] tel que
f(x) > f(a) = f(b). La fonction f est continue sur le
compact [a,b] donc elle est bornée et atteint ses bornes. Soit c \in
[a,b] tel que f(c) =\
sup\f(t)∣t \in
[a,b]\. On a f(c) ≥ f(x) > f(a) =
f(b), donc c \in]a,b[. Mais alors, le lemme ci dessus garantit que
f'(c) = 0.

Corollaire~8.3.3 (formule des accroissements finis). Soit f : [a,b]
\rightarrow~ \mathbb{R}~, continue sur [a,b], dérivable sur ]a,b[. Alors il existe c
\in]a,b[ tel que f(b) - f(a) = (b - a)f'(c).

Démonstration On applique le théorème de Rolle à g(t) = f(t) -
f(b)-f(a) \over b-a (t - a). On a g(b) = g(a) = f(a), g
est, comme f, continue sur [a,b] et dérivable sur ]a,b[. Donc il
existe c \in]a,b[ tel que g'(c) = 0~; mais g'(c) = f'(c) - f(b)-f(a)
\over b-a d'où le résultat.

\paragraph{8.3.2 Monotonie et dérivation}

Théorème~8.3.4 Soit I un intervalle de \mathbb{R}~, f : I \rightarrow~ \mathbb{R}~ continue sur I et
dérivable sur I^o. Alors (i) f est constante sur I si et
seulement si~\forall~t \in I^o~, f'(t) = 0
(ii) f est croissante sur I si et seulement
si~\forall~t \in I^o~, f'(t) ≥ 0 (iii) f est
décroissante sur I si et seulement si~\forall~~t \in
I^o, f'(t) \leq 0

Démonstration La définition de la dérivée f'(t)
=\
lim_x\rightarrow~t,x\neq~t f(x)-f(t)
\over x-t montre clairement que les conditions sont
nécessaires (prendre x > t et faire tendre x vers t).
Inversement, si x,y \in I avec x < y, f est continue sur
[x,y] \subset~ I et dérivable sur ]x,y[\subset~ I^o et donc la
formule des accroissements finis assure qu'il existe z \in]x,y[\subset~
I^o tel que f(y) - f(x) = (y - x)f'(z), ce qui montre
immédiatement que les conditions sont suffisantes.

Corollaire~8.3.5 Soit I un intervalle de \mathbb{R}~, f : I \rightarrow~ \mathbb{R}~ continue sur I et
dérivable sur I^o. Alors on a équivalence de (i) f est
strictement croissante (ii) \forall~~t \in
I^o, f'(t) ≥ 0 et \t \in
I^o∣f'(t) = 0\
est d'intérieur vide.

Démonstration (i) \rigtharrow~(ii) Si f est strictement croissante, alors
\forall~t \in I^o~, f'(t) ≥ 0~; supposons que
\t \in I^o∣f'(t) =
0\ n'est pas d'intérieur vide~; alors il contient un
segment [a,b] avec a < b~; mais alors d'après le théorème
précédent, f est constante sur [a,b] ce qui contredit la stricte
monotonie de f.

(ii) \rigtharrow~(i) On sait que si \forall~t \in I^o~,
f'(t) ≥ 0, f est croissante~; supposons qu'elle n'est pas strictement
croissante~; alors il existe a,b \in I tels que a < b et f(a) =
f(b)~; en conséquence f est constante sur ]a,b[\subset~ I^o et
donc \forall~~t \in]a,b[, f'(t) = 0~; donc
l'intervalle ouvert ]a,b[ est contenu dans l'intérieur de
\t \in I^o∣f'(t) =
0\, c'est absurde.

\paragraph{8.3.3 Difféomorphismes}

Théorème~8.3.6 Soit I et J deux intervalles de \mathbb{R}~ et f : I \rightarrow~ J un
homéomorphisme. Soit a \in I un point où f est dérivable. Alors
f^-1 est dérivable au point f(a) si et seulement
si~f'(a)\neq~0. Dans ce cas,
(f^-1)'(f(a)) = 1 \over f'(a) .

Démonstration Posons g = f^-1. On a g \cdot f =
\mathrmId_I. Si f est dérivable au point a et
g dérivable au point f(a), le théorème de dérivation des fonctions
composées assure que 1 = (\mathrmId_I)'(a) =
(g \cdot f)'(a) = g'(f(a))f'(a), donc f'(a)\neq~0 et
g'(f(a)) = 1 \over f'(a) . Inversement supposons que
f'(a)\neq~0. On a alors
lim_t\rightarrow~a,t\neq~a~
t-a \over f(t)-f(a) = 1 \over f'(a)
. Appliquons le théorème de composition des limites en posant t = g(u)
(avec a = g(f(a))), en remarquant que u\neq~f(a)
\rigtharrow~ g(u)\neq~a. On a donc, puisque g est continue
au point f(a),

lim_u\rightarrow~f(a),u\neq~f(a)~
g(u) - g(f(a)) \over u - f(a) = 1
\over f'(a)

Donc g est dérivable au point f(a).

Définition~8.3.1 Soit I et J deux intervalles de \mathbb{R}~~; on dit que f : I \rightarrow~
J est un difféomorphisme de classe C^n (n ≥ 1) si f est
bijective et f et f^-1 sont de classe C^n.

Théorème~8.3.7 Soit n ≥ 1, f : I \rightarrow~ \mathbb{R}~. On a équivalence de (i) f est un
C^n difféomorphisme de I sur f(I) (ii) f est de classe
C^n et f' ne s'annule pas.

Démonstration (i) \rigtharrow~(ii) est clair d'après le théorème précédent.
Inversement, supposons que f est de classe C^n et que f' ne
s'annule pas. Alors f' garde un signe constant (elle est continue), et
donc f est strictement monotone. Donc f définit un homéomorphisme de I
sur J = f(I). Le théorème précédent assure que f^-1 est
dérivable sur I et que (f^-1)' = 1 \over
f'\cdotf^-1 ce qui garantit déjà la continuité de
(f^-1)'. Supposons alors que f^-1 est de classe
C^k avec k < n. Comme f' est de classe
C^k, f' \cdot f^-1 est de classe C^k~; il
en est donc de même de  1 \over f'\cdotf^-1 ,
donc de (f^-1)' et donc f^-1 est de classe
C^k+1~; par récurrence, on en déduit que f^-1 est
de classe C^n.

\paragraph{8.3.4 Formule de Taylor Lagrange}

Théorème~8.3.8 (Taylor-Lagrange). Soit f : [a,b] \rightarrow~ \mathbb{R}~ de classe
C^n sur [a,b] et n + 1 fois dérivable sur ]a,b[.
Alors il existe c \in]a,b[ tel que

f(b) = f(a) + \sum _k=1^n~
f^(k)(a) \over k! (b - a)^k +
f^(n+1)(c) \over (n + 1)! (b -
a)^n+1

Démonstration Posons \phi(t) = f(b) - f(t)
-\\sum ~
_k=1^n f^(k)(t) \over k!
(b - t)^k - \lambda~(b - t)^n+1 où \lambda~ est choisi de telle
sorte que \phi(a) = 0 (c'est évidemment possible). Il est clair que \phi est
continue sur [a,b], dérivable sur ]a,b[ comme toutes les
fonctions f^(k), 0 \leq k \leq n. De plus

\begin{align*} \phi'(t)& =& -f'(t)
-\sum _k=1^n~
f^(k+1)(t) \over k! (b - t)^k
\%& \\ & \text &
+\sum _k=1^n~
f^(k)(t) \over (k - 1)! (b -
t)^k-1 + \lambda~(n + 1)(b - t)^n\%&
\\ & =& -f'(t)
-\sum _l=2^n+1~
f^(l)(t) \over (l - 1)! (b -
t)^l-1 \%& \\ &
\text & +\\sum
_k=1^n f^(k)(t) \over (k -
1)! (b - t)^k-1 + \lambda~(n + 1)(b - t)^n\%&
\\ & =& (b -
t)^n\left ((n + 1)\lambda~ - f^(n+1)(t)
\over n! \right ) \%&
\\ \end{align*}

(tous les autres termes se détruisent deux à deux). D'après le théorème
de Rolle, il existe c \in]a,b[ tel que \phi'(c) = 0, soit (b -
c)^n\left ((n + 1)\lambda~ - f^(n+1)(c)
\over n! \right ) = 0. Comme b -
c\neq~0, on a \lambda~ = f^(n+1)(c)
\over (n+1)! . En écrivant que \phi(a) = 0, on obtient
alors la formule souhaitée.

Remarque~8.3.1 Pour n = 0, on trouve comme cas particulier la formule
des accroissements finis. La même formule est encore valable si on prend
f : [b,a] \rightarrow~ \mathbb{R}~.

\paragraph{8.3.5 Extensions du théorème des accroissements finis}

Théorème~8.3.9 Soit f,g : [a,b] \rightarrow~ \mathbb{R}~ continues sur [a,b],
dérivables sur ]a,b[. Alors, il existe c \in]a,b[ tel que
\left
\matrix\,f(b) - f(a)&f'(c)
\cr g(b) - g(a)&g'(c)\right 
= 0.

Démonstration Posons

\phi(t) = \left
\matrix\,f(b) - f(a)&f(t) -
f(a) \cr g(b) - g(a)&g(t) - g(a)\right


La fonction \phi est continue sur [a,b], dérivable sur ]a,b[ avec
\phi'(t) = \left
\matrix\,f(b) - f(a)&f'(t)
\cr g(b) - g(a)&g'(t)\right .
Comme \phi(a) = \phi(b) = 0, le théorème de Rolle garantit l'existence d'un c
\in]a,b[ tel que \phi'(c) = 0.

Corollaire~8.3.10 (règle de L'Hôpital). Soit f,g : I \rightarrow~ \mathbb{R}~ continues sur
I, dérivables sur I \diagdown\a\. On suppose
qu'il existe \eta > 0 tel que g' ne s'annule pas sur ]a -
\eta,a + \eta[\diagdown\a\ et que  f'
\over g' a une limite \ell au point a. Alors  f(t)-f(a)
\over g(t)-g(a) admet la même limite au point a.

Démonstration Le théorème de Rolle garantit déjà que g(t) - g(a) ne
s'annule pas sur ]a - \eta,a +
\eta[\diagdown\a\. De plus le théorème
précédent montre que pour t \in]a - \eta,a +
\eta[\diagdown\a\, il existe c_t
\in]a,t[ (ou ]t,a[) tel que \left
\matrix\,f(t) -
f(a)&f'(c_t) \cr g(t) -
g(a)&g'(c_t)\right  = 0 soit 
f(t)-f(a) \over g(t)-g(a) = f'(c_t)
\over g'(c_t) . Quand t tend vers a, il en est
de même de c_t et le théorème de composition des limites donne

lim_t\rightarrow~a,t\neq~a~
f(t) - f(a) \over g(t) - g(a) = \ell

\paragraph{8.3.6 Fonctions convexes de classe \mathcal{C}^1}

Définition~8.3.2 Soit I un intervalle de \mathbb{R}~ et f : I \rightarrow~ \mathbb{R}~ une fonction de
classe \mathcal{C}^1. On dit que f est convexe si f' est croissante.

Remarque~8.3.2 Si f est de classe C^2, f est convexe si et
seulement si~f'' est positive.

Théorème~8.3.11 Soit I un intervalle de \mathbb{R}~ et f : I \rightarrow~ \mathbb{R}~ une fonction de
classe \mathcal{C}^1 convexe. Alors (i) \forall~~a,b
\in I, \forall~~t \in [0,1], f(ta + (1 - t)b) \leq tf(a)
+ (1 - t)f(b) (ii) \Gamma = \(x,y) \in
\mathbb{R}~^2∣x \in I\text et
y ≥ f(x)\ est une partie convexe de \mathbb{R}~^2
(iii) \forall~~a,b \in I, f(b) ≥ f(a) + (b - a)f'(a)
(iv) si a \in I, l'application I \diagdown\a\
dans \mathbb{R}~, t\mapsto~p_a(t) = f(t)-f(a)
\over t-a est croissante (v)
\forall~~a,b,c \in I, a < b < c \rigtharrow~
f(b)-f(a) \over b-a \leq f(c)-f(a) \over
c-a \leq f(c)-f(b) \over c-b

Démonstration (i) On peut évidemment supposer a < b. D'après
le théorème des accroissements finis, il existe c \in]a,b[ tel que
f(b) - f(a) = (b - a)f'(c). Posons c = t_0a + (1 -
t_0)b. Soit \phi(t) = tf(a) + (1 - t)f(b) - f(ta + (1 - t)b) pour
t \in [0,1]. Alors \phi est de classe \mathcal{C}^1 et \phi'(t) = f(a) -
f(b) - (a - b)f'(ta + (1 - t)b) = (b - a)(f'(ta + (1 - t)b) -
f'(t_0a + (1 - t_0)b). Comme f' est croissante et
t\mapsto~ta + (1 - t)b est décroissante, la composée
est décroissante et donc on a le tableau de variation

\begin{center}\rule{3in}{0.4pt}\end{center}

\begin{center}\rule{3in}{0.4pt}\end{center}

\begin{center}\rule{3in}{0.4pt}\end{center}

\begin{center}\rule{3in}{0.4pt}\end{center}

\begin{center}\rule{3in}{0.4pt}\end{center}

\begin{center}\rule{3in}{0.4pt}\end{center}

t

0

t_0

1

\begin{center}\rule{3in}{0.4pt}\end{center}

\begin{center}\rule{3in}{0.4pt}\end{center}

\begin{center}\rule{3in}{0.4pt}\end{center}

\begin{center}\rule{3in}{0.4pt}\end{center}

\begin{center}\rule{3in}{0.4pt}\end{center}

\begin{center}\rule{3in}{0.4pt}\end{center}

\phi'(t)

+

0

-

\begin{center}\rule{3in}{0.4pt}\end{center}

\begin{center}\rule{3in}{0.4pt}\end{center}

\begin{center}\rule{3in}{0.4pt}\end{center}

\begin{center}\rule{3in}{0.4pt}\end{center}

\begin{center}\rule{3in}{0.4pt}\end{center}

\begin{center}\rule{3in}{0.4pt}\end{center}

\phi(t)

0

\nearrow

\searrow

0

\begin{center}\rule{3in}{0.4pt}\end{center}

\begin{center}\rule{3in}{0.4pt}\end{center}

\begin{center}\rule{3in}{0.4pt}\end{center}

\begin{center}\rule{3in}{0.4pt}\end{center}

\begin{center}\rule{3in}{0.4pt}\end{center}

\begin{center}\rule{3in}{0.4pt}\end{center}

ce qui montre que la fonction \phi est positive sur [0,1].

(ii) Soit (x_1,y_1) et (x_2,y_2)
dans \Gamma et t \in [0,1]. On a

ty_1 + (1 - t)y_2 ≥ tf(x_1) + (1 -
t)f(x_2) ≥ f(tx_1 + (1 - t)x_2)

donc t(x_1,y_1) + (1 - t)(x_2,y_2) \in
\Gamma. Donc \Gamma est convexe.

(iii) Posons \phi(t) = f(t) - f(a) - (t - a)f'(a). La fonction \phi est de
classe \mathcal{C}^1 et \phi'(t) = f'(t) - f'(a). Comme f' est croissante,
on a le tableau de variation

\begin{center}\rule{3in}{0.4pt}\end{center}

\begin{center}\rule{3in}{0.4pt}\end{center}

\begin{center}\rule{3in}{0.4pt}\end{center}

\begin{center}\rule{3in}{0.4pt}\end{center}

\begin{center}\rule{3in}{0.4pt}\end{center}

\begin{center}\rule{3in}{0.4pt}\end{center}

t

a

\begin{center}\rule{3in}{0.4pt}\end{center}

\begin{center}\rule{3in}{0.4pt}\end{center}

\begin{center}\rule{3in}{0.4pt}\end{center}

\begin{center}\rule{3in}{0.4pt}\end{center}

\begin{center}\rule{3in}{0.4pt}\end{center}

\begin{center}\rule{3in}{0.4pt}\end{center}

\phi'(t)

+

0

-

\begin{center}\rule{3in}{0.4pt}\end{center}

\begin{center}\rule{3in}{0.4pt}\end{center}

\begin{center}\rule{3in}{0.4pt}\end{center}

\begin{center}\rule{3in}{0.4pt}\end{center}

\begin{center}\rule{3in}{0.4pt}\end{center}

\begin{center}\rule{3in}{0.4pt}\end{center}

\phi(t)

\searrow

0

\nearrow

\begin{center}\rule{3in}{0.4pt}\end{center}

\begin{center}\rule{3in}{0.4pt}\end{center}

\begin{center}\rule{3in}{0.4pt}\end{center}

\begin{center}\rule{3in}{0.4pt}\end{center}

\begin{center}\rule{3in}{0.4pt}\end{center}

\begin{center}\rule{3in}{0.4pt}\end{center}

ce qui montre que la fonction \phi est positive sur I.

(iv) Posons p_a(t) = f(t)-f(a) \over t-a si
t\neq~a et p_a(a) = f'(a). La fonction
p_a est continue sur I, dérivable sur I
\diagdown\a\ et p_a'(t) =
f(a)-f(t)-(a-t)f'(t) \over (t-a)^2 ≥ 0
d'après (iii). On en déduit que p_a est croissante.

(v) D'après (iv), on a p_a(b) \leq p_a(c) =
p_c(a) \leq p_c(b) ce qui est le résultat souhaité.

Théorème~8.3.12 Soit f : I \rightarrow~ \mathbb{R}~ de classe \mathcal{C}^1 convexe. Alors,
pour tout
(x_1,\\ldots,x_n~)
\in I^n, pour toute famille
(\alpha_1,\\ldots,\alpha_n~)
\in (\mathbb{R}~^+)^n telle que \alpha_1 +
\\ldots~ +
\alpha_n = 1, on a

f(\sum _i=1^n\alpha~_
ix_i) \leq\\sum
_i=1^n\alpha_ if(x_i)

Démonstration Par récurrence sur n. Si n = 2, on a \alpha_2 = 1 -
\alpha_1 et \alpha_1 \in [0,1]. L'inégalité se réduit à
l'assertion (i) du théorème précédent. Supposons le résultat vrai pour n
- 1 et montrons le pour n. Si \alpha_n = 0, on est immédiatement
ramené au cas n - 1. On peut donc supposer
\alpha_n\neq~0. Si \alpha_n = 1, alors
tous les autres \alpha_i sont nuls et l'inégalité est triviale. On
peut donc supposer \alpha_n \in]0,1[. On écrit alors
\\sum ~
_i=1^n\alpha_ix_i = \alpha_nx_n
+ (1 - \alpha_n)y avec y =
\alpha_1x_1+\\ldots+\alpha_n-1x_n-1~
\over
\alpha_1+\\ldots+\alpha_n-1~
= \beta_1x_1 +
\\ldots\beta_n-1x_n-1~
\in I. On a alors \beta_i ≥ 0 et
\\sum ~
_i=1^n-1\beta_i = 1. On peut donc écrire (par
l'hypothèse de récurrence) f(y)
\leq\\sum ~
_i=1^n-1\beta_if(x_i) soit

\begin{align*} f(\\sum
_i=1^n\alpha_ ix_i)& =&
f(\alpha_nx_n + (1 - \alpha_n)y) \leq
\alpha_nf(x_n) + (1 - \alpha_n)f(y) \%&
\\ & \leq& \alpha_nf(x_n) + (1
- \alpha_n)\\sum
_i=1^n-1\beta_ if(x_i) =
\sum _i=1^n\alpha~_
if(x_i)\%& \\
\end{align*}

puisque (1 - \alpha_n)\beta_i = \alpha_i.

Corollaire~8.3.13 (inégalité de Hölder). Soit p,q \in \mathbb{R}~^+∗ tels
que  1 \over p + 1 \over q = 1.
Pour toute famille
a_1,\\ldots,a_n,b_1,\\\ldots,b_n~
de réels positifs, on a

\sum _i=1^na_
ib_i \leq\left (\\sum
_i=1^na_ i^p\right
)^1\diagupp\left (\\sum
_i=1^nb_ i^q\right
)^1\diagupq

Démonstration Posons A = \left
(\\sum ~
_i=1^na_i^p\right
)^1\diagupp, B = \left
(\\sum ~
_i=1^nb_i^q\right
)^1\diagupq. La fonction exponentielle étant convexe sur \mathbb{R}~, on a
\forall~s,t \in \mathbb{R}~, e~^ s \over
p + t \over q  \leq 1 \over p
e^s + 1 \over q e^t. Si
a_i et b_i sont non nuls, en appliquant ceci à s =
plog  a_i \over A~
et t = qlog  b_i~
\over B , on obtient  a_i
\over A  b_i \over B \leq 1
\over p  a_i^p \over
A^p + 1 \over q  b_i^q
\over B^q , inégalité qui reste vrai si
a_ib_i = 0~; en sommant de i = 1 jusque n on obtient

 1 \over AB \\sum
_i=1^na_ ib_i \leq 1
\over pA^p  \\sum
_i=1^na_ i^p + 1 \over
qB^q  \\sum
_i=1^nb_ i^q = 1 \over
p + 1 \over q = 1

soit \\sum ~
_i=1^na_ib_i \leq AB ce qu'on voulait
démontrer.

Corollaire~8.3.14 (inégalité de Minkowski). Soit p ≥ 1. Pour toute
famille
a_1,\\ldots,a_n,b_1,\\\ldots,b_n~
de réels positifs, on a

 \left (\\sum
_i=1^n(a_ i +
b_i)^p\right )^1\diagupp
\leq\left (\\sum
_i=1^na_ i^p\right
)^1\diagupp + \left (\\sum
_i=1^nb_ i^p\right
)^1\diagupp

Démonstration C'est évident si p = 1~; si p > 1,
définissons q par la condition  1 \over p + 1
\over q = 1~; on écrit (a_i +
b_i)^p = a_i(a_i +
b_i)^p-1 + b_i(a_i +
b_i)^p-1 et on applique deux fois l'inégalité de
Hölder. On obtient alors

\begin{align*} \\sum
_i=1^n(a_ i + b_i)^p&
\leq& \left (\\sum
_i=1^na_ i^p\right
)^1\diagupp\left (\\sum
_i=1^n(a_ i +
b_i)^(p-1)q\right )^1\diagupq \%&
\\ & \text &
+\left (\\sum
_i=1^nb_ i^p\right
)^1\diagupp\left (\\sum
_i=1^n(a_ i +
b_i)^(p-1)q\right )^1\diagupq\%&
\\ \end{align*}

Mais (p - 1)q = p et l'inégalité ci dessus s'écrit donc après mise en
facteur

\begin{align*} \left
(\sum _i=1^n(a_ i~ +
b_i)^p\right )^1\diagupp&& \%&
\\ & \leq& \left
(\left (\\sum
_i=1^na_ i^p\right
)^1\diagupp + \left (\\sum
_i=1^nb_ i^p\right
)^1\diagupp\right )\left
(\sum _i=1^n(a_ i~ +
b_i)^p\right )^1\diagupq\%&
\\ \end{align*}

Si \\sum ~
_i=1^n(a_i + b_i)^p = 0,
l'inégalité cherchée est évidente~; sinon, on peut diviser les deux
membres par \left
(\\sum ~
_i=1^n(a_i +
b_i)^p\right )^1\diagupq et on
obtient (en tenant compte de 1 - 1 \over p = 1
\over q )

 \left (\\sum
_i=1^n(a_ i +
b_i)^p\right )^1\diagupp
\leq\left (\\sum
_i=1^na_ i^p\right
)^1\diagupp + \left (\\sum
_i=1^nb_ i^p\right
)^1\diagupp

[
[
[
[

\end{document}

% \documentclass[]{article}
\usepackage[T1]{fontenc}
\usepackage{lmodern}
\usepackage{amssymb,amsmath}
\usepackage{ifxetex,ifluatex}
\usepackage{fixltx2e} % provides \textsubscript
% use upquote if available, for straight quotes in verbatim environments
\IfFileExists{upquote.sty}{\usepackage{upquote}}{}
\ifnum 0\ifxetex 1\fi\ifluatex 1\fi=0 % if pdftex
  \usepackage[utf8]{inputenc}
\else % if luatex or xelatex
  \ifxetex
    \usepackage{mathspec}
    \usepackage{xltxtra,xunicode}
  \else
    \usepackage{fontspec}
  \fi
  \defaultfontfeatures{Mapping=tex-text,Scale=MatchLowercase}
  \newcommand{\euro}{€}
\fi
% use microtype if available
\IfFileExists{microtype.sty}{\usepackage{microtype}}{}
\ifxetex
  \usepackage[setpagesize=false, % page size defined by xetex
              unicode=false, % unicode breaks when used with xetex
              xetex]{hyperref}
\else
  \usepackage[unicode=true]{hyperref}
\fi
\hypersetup{breaklinks=true,
            bookmarks=true,
            pdfauthor={},
            pdftitle={Fonctions vectorielles d'une variable reelle},
            colorlinks=true,
            citecolor=blue,
            urlcolor=blue,
            linkcolor=magenta,
            pdfborder={0 0 0}}
\urlstyle{same}  % don't use monospace font for urls
\setlength{\parindent}{0pt}
\setlength{\parskip}{6pt plus 2pt minus 1pt}
\setlength{\emergencystretch}{3em}  % prevent overfull lines
\setcounter{secnumdepth}{0}
 
/* start css.sty */
.cmr-5{font-size:50%;}
.cmr-7{font-size:70%;}
.cmmi-5{font-size:50%;font-style: italic;}
.cmmi-7{font-size:70%;font-style: italic;}
.cmmi-10{font-style: italic;}
.cmsy-5{font-size:50%;}
.cmsy-7{font-size:70%;}
.cmex-7{font-size:70%;}
.cmex-7x-x-71{font-size:49%;}
.msbm-7{font-size:70%;}
.cmtt-10{font-family: monospace;}
.cmti-10{ font-style: italic;}
.cmbx-10{ font-weight: bold;}
.cmr-17x-x-120{font-size:204%;}
.cmsl-10{font-style: oblique;}
.cmti-7x-x-71{font-size:49%; font-style: italic;}
.cmbxti-10{ font-weight: bold; font-style: italic;}
p.noindent { text-indent: 0em }
td p.noindent { text-indent: 0em; margin-top:0em; }
p.nopar { text-indent: 0em; }
p.indent{ text-indent: 1.5em }
@media print {div.crosslinks {visibility:hidden;}}
a img { border-top: 0; border-left: 0; border-right: 0; }
center { margin-top:1em; margin-bottom:1em; }
td center { margin-top:0em; margin-bottom:0em; }
.Canvas { position:relative; }
li p.indent { text-indent: 0em }
.enumerate1 {list-style-type:decimal;}
.enumerate2 {list-style-type:lower-alpha;}
.enumerate3 {list-style-type:lower-roman;}
.enumerate4 {list-style-type:upper-alpha;}
div.newtheorem { margin-bottom: 2em; margin-top: 2em;}
.obeylines-h,.obeylines-v {white-space: nowrap; }
div.obeylines-v p { margin-top:0; margin-bottom:0; }
.overline{ text-decoration:overline; }
.overline img{ border-top: 1px solid black; }
td.displaylines {text-align:center; white-space:nowrap;}
.centerline {text-align:center;}
.rightline {text-align:right;}
div.verbatim {font-family: monospace; white-space: nowrap; text-align:left; clear:both; }
.fbox {padding-left:3.0pt; padding-right:3.0pt; text-indent:0pt; border:solid black 0.4pt; }
div.fbox {display:table}
div.center div.fbox {text-align:center; clear:both; padding-left:3.0pt; padding-right:3.0pt; text-indent:0pt; border:solid black 0.4pt; }
div.minipage{width:100%;}
div.center, div.center div.center {text-align: center; margin-left:1em; margin-right:1em;}
div.center div {text-align: left;}
div.flushright, div.flushright div.flushright {text-align: right;}
div.flushright div {text-align: left;}
div.flushleft {text-align: left;}
.underline{ text-decoration:underline; }
.underline img{ border-bottom: 1px solid black; margin-bottom:1pt; }
.framebox-c, .framebox-l, .framebox-r { padding-left:3.0pt; padding-right:3.0pt; text-indent:0pt; border:solid black 0.4pt; }
.framebox-c {text-align:center;}
.framebox-l {text-align:left;}
.framebox-r {text-align:right;}
span.thank-mark{ vertical-align: super }
span.footnote-mark sup.textsuperscript, span.footnote-mark a sup.textsuperscript{ font-size:80%; }
div.tabular, div.center div.tabular {text-align: center; margin-top:0.5em; margin-bottom:0.5em; }
table.tabular td p{margin-top:0em;}
table.tabular {margin-left: auto; margin-right: auto;}
div.td00{ margin-left:0pt; margin-right:0pt; }
div.td01{ margin-left:0pt; margin-right:5pt; }
div.td10{ margin-left:5pt; margin-right:0pt; }
div.td11{ margin-left:5pt; margin-right:5pt; }
table[rules] {border-left:solid black 0.4pt; border-right:solid black 0.4pt; }
td.td00{ padding-left:0pt; padding-right:0pt; }
td.td01{ padding-left:0pt; padding-right:5pt; }
td.td10{ padding-left:5pt; padding-right:0pt; }
td.td11{ padding-left:5pt; padding-right:5pt; }
table[rules] {border-left:solid black 0.4pt; border-right:solid black 0.4pt; }
.hline hr, .cline hr{ height : 1px; margin:0px; }
.tabbing-right {text-align:right;}
span.TEX {letter-spacing: -0.125em; }
span.TEX span.E{ position:relative;top:0.5ex;left:-0.0417em;}
a span.TEX span.E {text-decoration: none; }
span.LATEX span.A{ position:relative; top:-0.5ex; left:-0.4em; font-size:85%;}
span.LATEX span.TEX{ position:relative; left: -0.4em; }
div.float img, div.float .caption {text-align:center;}
div.figure img, div.figure .caption {text-align:center;}
.marginpar {width:20%; float:right; text-align:left; margin-left:auto; margin-top:0.5em; font-size:85%; text-decoration:underline;}
.marginpar p{margin-top:0.4em; margin-bottom:0.4em;}
.equation td{text-align:center; vertical-align:middle; }
td.eq-no{ width:5%; }
table.equation { width:100%; } 
div.math-display, div.par-math-display{text-align:center;}
math .texttt { font-family: monospace; }
math .textit { font-style: italic; }
math .textsl { font-style: oblique; }
math .textsf { font-family: sans-serif; }
math .textbf { font-weight: bold; }
.partToc a, .partToc, .likepartToc a, .likepartToc {line-height: 200%; font-weight:bold; font-size:110%;}
.chapterToc a, .chapterToc, .likechapterToc a, .likechapterToc, .appendixToc a, .appendixToc {line-height: 200%; font-weight:bold;}
.index-item, .index-subitem, .index-subsubitem {display:block}
.caption td.id{font-weight: bold; white-space: nowrap; }
table.caption {text-align:center;}
h1.partHead{text-align: center}
p.bibitem { text-indent: -2em; margin-left: 2em; margin-top:0.6em; margin-bottom:0.6em; }
p.bibitem-p { text-indent: 0em; margin-left: 2em; margin-top:0.6em; margin-bottom:0.6em; }
.subsectionHead, .likesubsectionHead { margin-top:2em; font-weight: bold;}
.sectionHead, .likesectionHead { font-weight: bold;}
.quote {margin-bottom:0.25em; margin-top:0.25em; margin-left:1em; margin-right:1em; text-align:justify;}
.verse{white-space:nowrap; margin-left:2em}
div.maketitle {text-align:center;}
h2.titleHead{text-align:center;}
div.maketitle{ margin-bottom: 2em; }
div.author, div.date {text-align:center;}
div.thanks{text-align:left; margin-left:10%; font-size:85%; font-style:italic; }
div.author{white-space: nowrap;}
.quotation {margin-bottom:0.25em; margin-top:0.25em; margin-left:1em; }
h1.partHead{text-align: center}
.sectionToc, .likesectionToc {margin-left:2em;}
.subsectionToc, .likesubsectionToc {margin-left:4em;}
.sectionToc, .likesectionToc {margin-left:6em;}
.frenchb-nbsp{font-size:75%;}
.frenchb-thinspace{font-size:75%;}
.figure img.graphics {margin-left:10%;}
/* end css.sty */

\title{Fonctions vectorielles d'une variable reelle}
\author{}
\date{}

\begin{document}
\maketitle

\textbf{Warning: 
requires JavaScript to process the mathematics on this page.\\ If your
browser supports JavaScript, be sure it is enabled.}

\begin{center}\rule{3in}{0.4pt}\end{center}

[
[
[]
[

\section{8.4 Fonctions vectorielles d'une variable réelle}

\subsection{8.4.1 Inégalité des accroissements finis}

Dans le cas d'une fonction vectorielle d'une variable réelle, le
théorème de Rolle et le théorème des accroissements finis ne sont plus
valables comme le montre l'exemple de la fonction f :
t\mapsto~e^it entre 0 et 2\pi~ (on a f(0) =
f(2\pi~) et cependant la dérivée f'(t) = ie^it ne s'annule pas).
On obtient uniquement une inégalité que l'on peut mettre sous une forme
plus générale

Théorème~8.4.1 (inégalité des accroissements finis) . Soit f : [a,b]
\rightarrow~ E et g : [a,b] \rightarrow~ \mathbb{R}~. On suppose que f et g sont continues sur
[a,b] et dérivables sur ]a,b[ avec \forall~~t
\in]a,b[, \f'(t)\ \leq
g'(t). Alors \f(b) -
f(a)\ \leq g(b) - g(a).

Démonstration (Première démonstration) Si on suppose en plus que f et g
sont de classe \mathcal{C}^1 sur ]a,b[, on peut écrire pour a
< x < y < b,

\begin{align*} \f(y) -
f(x)& =&
\\int ~
_x^yf'(t) dt\ \%&
\\ & \leq& \int ~
_x^y\f'(t)\
dt \leq\int  _x^y~g'(t) dt = g(y) -
g(x)\%& \\
\end{align*}

Il ne reste plus qu'à faire tendre x vers a et y vers b pour obtenir
l'inégalité souhaitée.

Remarque~8.4.1 Attention~! L'utilisation inconsidérée de cette première
démonstration peut provoquer un cercle vicieux dans l'exposé~: la
formule f(y) - f(x) =\int ~
_x^yf'(t) dt fait appel de manière cachée au fait qu'une
fonction de dérivée nulle est constante, ce qui est en général vu comme
une conséquence de l'inégalité des accroissements finis~!

Démonstration (Deuxième démonstration) Dans le cas général, soit \epsilon
> 0, soit \phi_\epsilon(t) =\ f(t)
- f(a)\ - (g(t) - g(a)) - \epsilon(t - a) et
X_\epsilon = \t \in
[a,b]∣\phi_\epsilon(t) \leq
\epsilon\. On a X_\epsilon = \phi_\epsilon^-1(]
-\infty~,\epsilon]) et comme \phi_\epsilon est continue, X_\epsilon est fermé
(image réciproque d'un fermé). De plus \phi_\epsilon(a) = 0, donc il
existe \eta > 0 tel que [a,a + \eta] \subset~ X_\epsilon. Soit c
= supX_\epsilon~. Supposons que c <
b. Comme c ≥ a + \eta, on a c \in]a,b[ et donc f et g sont dérivables au
point c. On a \f'(c)\
= lim_t\rightarrow~c~\
f(t)-f(c) \over t-c \ et g'(c)
= lim_t\rightarrow~c~ g(t)-g(c)
\over t-c . Donc il existe \alpha~ > 0 tel que
pour t \in]c,c + \alpha~[ on ait à la fois \
f(t)-f(c) \over t-c \
\leq\ f'(c)\ + \epsilon
\over 2 et  g(t)-g(c) \over t-c ≥
g'(c) - \epsilon \over 2 . Tenant compte de
\f'(c)\ \leq g'(c), on
obtient

\ f(t) - f(c) \over t - c
\ \leq\
f'(c)\ + \epsilon \over 2 \leq g'(c)
+ \epsilon \over 2 \leq g(t) - g(c) \over t -
c + \epsilon

soit encore \f(t) -
f(c)\ \leq g(t) - g(c) + \epsilon(t - c). Comme c \in
X_\epsilon, on a \f(c) -
f(a)\ \leq g(c) - g(a) + \epsilon(c - a) + \epsilon et donc pour
t \in]c,c + \alpha~[, \f(t) -
f(a)\ \leq\ f(t) -
f(c)\ +\ f(c) -
f(a)\ \leq g(t) - g(a) + \epsilon(t - a) + \epsilon, soit encore
\phi_\epsilon(t) \leq \epsilon. On a donc ]c,c + \alpha~[\subset~ X_\epsilon ce qui
contredit la définition de c =\
supX_\epsilon. On a donc b = c \in X_\epsilon, soit encore
\f(b) - f(a)\ \leq g(b) -
g(a) + \epsilon(b - a) + \epsilon. En faisant tendre \epsilon vers 0, on trouve alors
l'inégalité \f(b) -
f(a)\ \leq g(b) - g(a).

En fait on utilisera la plupart du temps la version suivante du théorème
précédent

Corollaire~8.4.2 Soit f : [a,b] \rightarrow~ E, continue sur [a,b]
dérivable sur ]a,b[ telle que \forall~~t
\in]a,b[, \f'(t)\ \leq
M. Alors \f(b) - f(a)\
\leq M(b - a).

Démonstration Il suffit d'appliquer l'inégalité des accroissements finis
à g(t) = Mt pour laquelle on a g'(t) = M et g(b) - g(a) = M(b - a).

\subsection{8.4.2 Applications de l'inégalité des accroissements finis}

Théorème~8.4.3 Soit f : I \rightarrow~ E, continue sur I, dérivable sur
I^o. Alors f est k-lipschitzienne si et seulement
si~\forall~t \in I^o~,
\f'(t)\ \leq k.

Démonstration Si f est k-lipschitzienne, on a pour a \in I^o et
t\neq~a, \ f(t)-f(a)
\over t-a \ \leq k d'où en
faisant tendre t vers a,
\f'(a)\ \leq k. La
condition est donc nécessaire. Réciproquement, supposons que
\forall~t \in I^o~,
\f'(t)\ \leq k, soit a,b
\in I tels que a < b. Alors [a,b] \subset~ I et ]a,b[\subset~
I^o, on peut donc appliquer le corollaire de l'inégalité des
accroissements finis, \f(b) -
f(a)\ \leq k(b - a), ce qui montre que f est
k-lipschitzienne.

Remarque~8.4.2 Ce théorème peut permettre en particulier de caractériser
les applications contractantes.

Théorème~8.4.4 Soit E un espace vectoriel normé complet et f :]a,b[\rightarrow~
E dérivable, telle que la fonction f' admet une limite \ell au point a.
Alors f se prolonge en une application continue
\tildef : [a,b[\rightarrow~ E. L'application
\tildef est dérivable sur [a,b[ et
\tildef'(a) = \ell.

Démonstration Il existe \eta_0 > 0 tel que
\forall~t \in]a,a + \eta_0~[,
\f'(t)\
\leq\ \ell\ + 1. Donc f est
lipschitzienne sur ]a,a + \eta_0[. En particulier si une suite
(x_n) de ]a,b[ admet la limite a, c'est une suite de
Cauchy, donc son image par f est encore une suite de Cauchy. Comme E est
complet, la suite (f(x_n)) est donc convergente. Pour toute
suite (x_n) de limite a, la suite (f(x_n)) admet une
limite, donc f admet une limite L au point a. Si l'on pose alors
\tildef(a) = L et \tildef(t) =
f(t) si t \in]a,b[, la fonction \tildef est donc
continue sur [a,b[ et dérivable sur ]a,b[ avec
\tildef'(t) = f'(t). Montrons que
\tildef est dérivable au point a. Posons g(t)
=\tilde f(t) - \ellt et soit \epsilon > 0. Il
existe \eta > 0 tel que t \in]a,a +
\eta[\rigtharrow~\ g'(t)\
=\ f'(t) - \ell\
< \epsilon. On en déduit que g est \epsilon-lipschitzienne sur [a,a + \eta]
et en particulier pour t \in [a,a + \eta], \g(t)
- g(a)\ \leq \epsilon(t - a) soit encore (après division
par t - a), \
\tildef(t)-\tildef(a)
\over t-a - \ell\ \leq \epsilon, ce qui
montre que \tildef est dérivable au point a et que sa
dérivée en ce point est \ell.

Remarque~8.4.3 Si f est continue sur [a,b[, dérivable sur ]a,b[
et si f' admet la limite \ell au point a, on a évidemment L = f(a) et donc
\tildef = f. Travaillant séparément à gauche de a et
à droite de a, on obtient le corollaire suivant

Corollaire~8.4.5 Soit f : I \rightarrow~ E, a \in I. On suppose que f est continue
sur I, dérivable sur I \diagdown\a\ et que f'
admet une limite \ell au point a. Alors f est dérivable au point a et f'(a)
= \ell.

Remarque~8.4.4 Une récurrence évidente à partir du théorème ci dessus
montre que si f est continue sur [a,b[, n fois dérivable sur
]a,b[ et si f^(n) admet une limite \ell au point a, alors
toutes les dérivées intermédiaires f^(k) admettent une limite
au point a~; ceci permet alors d'appliquer le corollaire ci dessus qui
garantira que f est n fois dérivable au point a (et que toutes les
dérivées f^(k) sont continues au point a). Par contre la même
méthode ne peut s'appliquer sur I
\diagdown\a\, rien ne garantissant que les
limites à droite et à gauche de f^(k) sont les mêmes~:
l'exemple de la fonction x \rightarrow~x dont la dérivée
seconde sur \mathbb{R}~ \diagdown\0\ est nulle fournit
un contre exemple évident.

\subsection{8.4.3 Formules de Taylor}

Théorème~8.4.6 (inégalité de Taylor-Lagrange). Soit f : [a,b] \rightarrow~ E
(resp. f : [b,a] \rightarrow~ E) de classe C^n~; on suppose que f
est n + 1 fois dérivable sur ]a,b[ (resp. ]a,b[) et que
\f^(n+1)(t)\
\leq M. Alors

\f(b) - f(a) -\\sum
_k=1^n f^(k)(a) \over k!
(b - a)^k\ \leq M
\over (n + 1)! (b - a)^n+1

Démonstration Posons \phi(t) = f(b) - f(t)
-\\sum ~
_k=1^n f^(k)(t) \over k!
(b - t)^k. Il est clair que \phi est continue sur [a,b],
dérivable sur ]a,b[ comme toutes les fonctions f^(k), 0 \leq
k \leq n. De plus

\begin{align*} \phi'(t)&& \%&
\\ & =& -f'(t)
-\sum _k=1^n~
f^(k+1)(t) \over k! (b - t)^k +
\sum _k=1^n f^(k)~(t)
\over (k - 1)! (b - t)^k-1 \%&
\\ & =& -f'(t)
-\sum _l=2^n+1~
f^(l)(t) \over (l - 1)! (b -
t)^l-1 + \sum _k=1^n~
f^(k)(t) \over (k - 1)! (b -
t)^k-1\%& \\ & =& -
f^(n+1)(t) \over n! (b - t)^n
\%& \\ \end{align*}

(tous les autres termes se détruisent deux à deux). On a donc
\\phi'(t)\ \leq M
(b-t)^n \over n! = \psi'(t) pour \psi(t) = -M
(b-t)^n+1 \over (n+1)! . L'inégalité des
accroissements finis assure que \\phi(b) -
\phi(a)\ \leq \psi(b) - \psi(a), soit encore
\\phi(a)\ \leq-\psi(a) ce qui
n'est autre que l'inégalité à démontrer.

Théorème~8.4.7 (formule de Taylor Young). Soit I un intervalle de \mathbb{R}~, a \in
I et f : I \rightarrow~ E, n fois dérivable au point a. Alors, au voisinage de a,

f(t) = f(a) + \sum _k=1^n~
f^(k)(a) \over k! (t - a)^k +
o((t - a)^n)

Démonstration On montre le résultat par récurrence sur n. Pour n = 1, il
s'agit seulement de la définition de la dérivée~: en posant \epsilon(t - a) =
f(t)-f(a) \over t-a - f'(a), on a f(t) = f(a) + (t -
a)f'(a) + (t - a)\epsilon(t - a) avec
lim_t\rightarrow~a~\epsilon(t - a) = 0. Supposons donc
le résultat vrai pour n - 1 et soit \eta_0 > 0 tel
que f soit n - 1 fois dérivable sur ]a,-\eta_0,a +
\eta_0[\bigcapI. On peut alors appliquer notre hypothèse de récurrence
à la fonction f' sur ]a,-\eta_0,a + \eta_0[\bigcapI puisque
celle ci est n - 1 fois dérivable au point a. On a donc f'(t) = f'(a)
+ \\sum ~
_k=1^n-1 f^(k+1)(a) \over k!
(t - a)^k + o((t - a)^n-1). Etant donné \epsilon
> 0, il existe donc \eta > 0 tel que, pour t
\in]a,-\eta,a + \eta[\bigcapI, \f'(t) - f'(a)
+ \\sum ~
_k=1^n-1 f^(k+1)(a) \over k!
(t - a)^k\ \leq \epsilont -
a^n-1. Pour t \in]a,a + \eta[, ceci s'écrit encore
\\phi'(t)\ \leq \psi'(t) avec
\phi(t) = f(t) - f(a) -\\\sum
 _k=1^n f^(k)(a) \over
k! (t - a)^k et \psi(t) = \epsilon(t-a)^n
\over n . L'inégalité des accroissements finis (dont
les conditions de validité sur [a,t] sont évidemment vérifiées)
assure que \\phi(t) -
\phi(a)\ \leq \psi(t) - \psi(a) soit encore

\f(t) - f(a) -\\sum
_k=1^n f^(k)(a) \over k!
(t - a)^k\ \leq \epsilon(t - a)^n
\over n

On a donc f(t) - f(a)
-\\sum ~
_k=1^n f^(k)(a) \over k!
(t - a)^k = o((t - a)^n) en a, à droite de a. On
montre de manière similaire avec \psi(t) = (-1)^n
\epsilon(t-a)^n \over n , le même résultat à gauche
de a.

Remarque~8.4.5 On prendra soin de ne pas confondre l'inégalité de Taylor
Lagrange (ou la formule de Taylor Lagrange pour les fonctions à valeurs
réelles) qui donne une estimation globale de la fonction f sur tout un
intervalle, avec la formule de Taylor Young qui donne un comportement
local de la fonction (en fait un développement limité).

On montre en intégration le résultat suivant (d'où l'on peut d'ailleurs
déduire facilement l'inégalité de Taylor Lagrange, mais avec des
conditions plus fortes de validité)~; la démonstration consiste
simplement en n intégrations par parties successives.

Théorème~8.4.8 (formule de Taylor avec reste intégral). Soit f : I \rightarrow~ E
de classe C^n+1. Alors, \forall~~a,b \in I,

f(b) = f(a) + \sum _k=1^n~
f^(k)(a) \over k! (b - a)^k +
\\int  ~
_a^b (b - t)^n \over n!
f^(n+1)(t) dt

[
[
[
[

\end{document}

% \documentclass[]{article}
\usepackage[T1]{fontenc}
\usepackage{lmodern}
\usepackage{amssymb,amsmath}
\usepackage{ifxetex,ifluatex}
\usepackage{fixltx2e} % provides \textsubscript
% use upquote if available, for straight quotes in verbatim environments
\IfFileExists{upquote.sty}{\usepackage{upquote}}{}
\ifnum 0\ifxetex 1\fi\ifluatex 1\fi=0 % if pdftex
  \usepackage[utf8]{inputenc}
\else % if luatex or xelatex
  \ifxetex
    \usepackage{mathspec}
    \usepackage{xltxtra,xunicode}
  \else
    \usepackage{fontspec}
  \fi
  \defaultfontfeatures{Mapping=tex-text,Scale=MatchLowercase}
  \newcommand{\euro}{€}
\fi
% use microtype if available
\IfFileExists{microtype.sty}{\usepackage{microtype}}{}
\ifxetex
  \usepackage[setpagesize=false, % page size defined by xetex
              unicode=false, % unicode breaks when used with xetex
              xetex]{hyperref}
\else
  \usepackage[unicode=true]{hyperref}
\fi
\hypersetup{breaklinks=true,
            bookmarks=true,
            pdfauthor={},
            pdftitle={Fonctions classiques},
            colorlinks=true,
            citecolor=blue,
            urlcolor=blue,
            linkcolor=magenta,
            pdfborder={0 0 0}}
\urlstyle{same}  % don't use monospace font for urls
\setlength{\parindent}{0pt}
\setlength{\parskip}{6pt plus 2pt minus 1pt}
\setlength{\emergencystretch}{3em}  % prevent overfull lines
\setcounter{secnumdepth}{0}
 
/* start css.sty */
.cmr-5{font-size:50%;}
.cmr-7{font-size:70%;}
.cmmi-5{font-size:50%;font-style: italic;}
.cmmi-7{font-size:70%;font-style: italic;}
.cmmi-10{font-style: italic;}
.cmsy-5{font-size:50%;}
.cmsy-7{font-size:70%;}
.cmex-7{font-size:70%;}
.cmex-7x-x-71{font-size:49%;}
.msbm-7{font-size:70%;}
.cmtt-10{font-family: monospace;}
.cmti-10{ font-style: italic;}
.cmbx-10{ font-weight: bold;}
.cmr-17x-x-120{font-size:204%;}
.cmsl-10{font-style: oblique;}
.cmti-7x-x-71{font-size:49%; font-style: italic;}
.cmbxti-10{ font-weight: bold; font-style: italic;}
p.noindent { text-indent: 0em }
td p.noindent { text-indent: 0em; margin-top:0em; }
p.nopar { text-indent: 0em; }
p.indent{ text-indent: 1.5em }
@media print {div.crosslinks {visibility:hidden;}}
a img { border-top: 0; border-left: 0; border-right: 0; }
center { margin-top:1em; margin-bottom:1em; }
td center { margin-top:0em; margin-bottom:0em; }
.Canvas { position:relative; }
li p.indent { text-indent: 0em }
.enumerate1 {list-style-type:decimal;}
.enumerate2 {list-style-type:lower-alpha;}
.enumerate3 {list-style-type:lower-roman;}
.enumerate4 {list-style-type:upper-alpha;}
div.newtheorem { margin-bottom: 2em; margin-top: 2em;}
.obeylines-h,.obeylines-v {white-space: nowrap; }
div.obeylines-v p { margin-top:0; margin-bottom:0; }
.overline{ text-decoration:overline; }
.overline img{ border-top: 1px solid black; }
td.displaylines {text-align:center; white-space:nowrap;}
.centerline {text-align:center;}
.rightline {text-align:right;}
div.verbatim {font-family: monospace; white-space: nowrap; text-align:left; clear:both; }
.fbox {padding-left:3.0pt; padding-right:3.0pt; text-indent:0pt; border:solid black 0.4pt; }
div.fbox {display:table}
div.center div.fbox {text-align:center; clear:both; padding-left:3.0pt; padding-right:3.0pt; text-indent:0pt; border:solid black 0.4pt; }
div.minipage{width:100%;}
div.center, div.center div.center {text-align: center; margin-left:1em; margin-right:1em;}
div.center div {text-align: left;}
div.flushright, div.flushright div.flushright {text-align: right;}
div.flushright div {text-align: left;}
div.flushleft {text-align: left;}
.underline{ text-decoration:underline; }
.underline img{ border-bottom: 1px solid black; margin-bottom:1pt; }
.framebox-c, .framebox-l, .framebox-r { padding-left:3.0pt; padding-right:3.0pt; text-indent:0pt; border:solid black 0.4pt; }
.framebox-c {text-align:center;}
.framebox-l {text-align:left;}
.framebox-r {text-align:right;}
span.thank-mark{ vertical-align: super }
span.footnote-mark sup.textsuperscript, span.footnote-mark a sup.textsuperscript{ font-size:80%; }
div.tabular, div.center div.tabular {text-align: center; margin-top:0.5em; margin-bottom:0.5em; }
table.tabular td p{margin-top:0em;}
table.tabular {margin-left: auto; margin-right: auto;}
div.td00{ margin-left:0pt; margin-right:0pt; }
div.td01{ margin-left:0pt; margin-right:5pt; }
div.td10{ margin-left:5pt; margin-right:0pt; }
div.td11{ margin-left:5pt; margin-right:5pt; }
table[rules] {border-left:solid black 0.4pt; border-right:solid black 0.4pt; }
td.td00{ padding-left:0pt; padding-right:0pt; }
td.td01{ padding-left:0pt; padding-right:5pt; }
td.td10{ padding-left:5pt; padding-right:0pt; }
td.td11{ padding-left:5pt; padding-right:5pt; }
table[rules] {border-left:solid black 0.4pt; border-right:solid black 0.4pt; }
.hline hr, .cline hr{ height : 1px; margin:0px; }
.tabbing-right {text-align:right;}
span.TEX {letter-spacing: -0.125em; }
span.TEX span.E{ position:relative;top:0.5ex;left:-0.0417em;}
a span.TEX span.E {text-decoration: none; }
span.LATEX span.A{ position:relative; top:-0.5ex; left:-0.4em; font-size:85%;}
span.LATEX span.TEX{ position:relative; left: -0.4em; }
div.float img, div.float .caption {text-align:center;}
div.figure img, div.figure .caption {text-align:center;}
.marginpar {width:20%; float:right; text-align:left; margin-left:auto; margin-top:0.5em; font-size:85%; text-decoration:underline;}
.marginpar p{margin-top:0.4em; margin-bottom:0.4em;}
.equation td{text-align:center; vertical-align:middle; }
td.eq-no{ width:5%; }
table.equation { width:100%; } 
div.math-display, div.par-math-display{text-align:center;}
math .texttt { font-family: monospace; }
math .textit { font-style: italic; }
math .textsl { font-style: oblique; }
math .textsf { font-family: sans-serif; }
math .textbf { font-weight: bold; }
.partToc a, .partToc, .likepartToc a, .likepartToc {line-height: 200%; font-weight:bold; font-size:110%;}
.chapterToc a, .chapterToc, .likechapterToc a, .likechapterToc, .appendixToc a, .appendixToc {line-height: 200%; font-weight:bold;}
.index-item, .index-subitem, .index-subsubitem {display:block}
.caption td.id{font-weight: bold; white-space: nowrap; }
table.caption {text-align:center;}
h1.partHead{text-align: center}
p.bibitem { text-indent: -2em; margin-left: 2em; margin-top:0.6em; margin-bottom:0.6em; }
p.bibitem-p { text-indent: 0em; margin-left: 2em; margin-top:0.6em; margin-bottom:0.6em; }
.paragraphHead, .likeparagraphHead { margin-top:2em; font-weight: bold;}
.subparagraphHead, .likesubparagraphHead { font-weight: bold;}
.quote {margin-bottom:0.25em; margin-top:0.25em; margin-left:1em; margin-right:1em; text-align:\\jmathmathustify;}
.verse{white-space:nowrap; margin-left:2em}
div.maketitle {text-align:center;}
h2.titleHead{text-align:center;}
div.maketitle{ margin-bottom: 2em; }
div.author, div.date {text-align:center;}
div.thanks{text-align:left; margin-left:10%; font-size:85%; font-style:italic; }
div.author{white-space: nowrap;}
.quotation {margin-bottom:0.25em; margin-top:0.25em; margin-left:1em; }
h1.partHead{text-align: center}
.sectionToc, .likesectionToc {margin-left:2em;}
.subsectionToc, .likesubsectionToc {margin-left:4em;}
.subsubsectionToc, .likesubsubsectionToc {margin-left:6em;}
.frenchb-nbsp{font-size:75%;}
.frenchb-thinspace{font-size:75%;}
.figure img.graphics {margin-left:10%;}
/* end css.sty */

\title{Fonctions classiques}
\author{}
\date{}

\begin{document}
\maketitle

\textbf{Warning: 
requires JavaScript to process the mathematics on this page.\\ If your
browser supports JavaScript, be sure it is enabled.}

\begin{center}\rule{3in}{0.4pt}\end{center}

{[}
{[}
{[}{]}
{[}

\subsubsection{8.5 Fonctions classiques}

\paragraph{8.5.1 Fonctions circulaires réciproques}

Le lecteur démontrera sans difficulté les résultats suivants qui
découlent immédiatement des caractérisations des homéomorphismes et des
difféomorphismes d'un intervalle sur un autre intervalle de \mathbb{R}~.

Théorème~8.5.1 (i)
x\mapsto~cos~ x est un
homéomorphisme décroissant de {[}0,\pi~{]} sur {[}-1,1{]}~;
l'homéomorphisme réciproque est noté arccos~ :
{[}-1,1{]} \rightarrow~ {[}0,\pi~{]}~; arccos~ est
C^\infty~ sur {]} - 1,1{[} et arccos~ '(x)
= - 1 \over \sqrt1-x^2
. (ii) x\mapsto~sin~ x est
un homéomorphisme croissant de {[}-\pi~\diagup2,\pi~\diagup2{]} sur {[}-1,1{]}~;
l'homéomorphisme réciproque est noté arcsin~ :
{[}-1,1{]} \rightarrow~ {[}-\pi~\diagup2,\pi~\diagup2{]}~; arcsin~ est
C^\infty~ sur {]} - 1,1{[} et arcsin~ '(x)
= 1 \over \sqrt1-x^2 .
(iii) x\mapsto~tan~ x est
un C^\infty~ difféomorphisme croissant de {]} - \pi~\diagup2,\pi~\diagup2{[} sur {]}
-\infty~,+\infty~{[}~; le difféomorphisme réciproque est noté
arctan~ :{]} -\infty~,+\infty~{[}\rightarrow~{]} - \pi~\diagup2,\pi~\diagup2{[} et
arctan~ '(x) = 1 \over
1+x^2 .

Remarque~8.5.1 Pour t \in {[}-1,1{]},

x = arccos t \mathrel\Leftrightarrow~ t
= cos x\text et ~x \in
{[}0,\pi~{]}

x = arcsin t \mathrel\Leftrightarrow~ t
= sin x\text et ~x \in
{[}-\pi~\diagup2,\pi~\diagup2{]}

Pour t \in \mathbb{R}~,

x = arctan t \mathrel\Leftrightarrow~ t
= tan x\text et ~x \in{]} -
\pi~\diagup2,\pi~\diagup2{[}

cos (\arccos~ t) = t,
sin (\arccos~ t) =
\sqrt1 - t^2,
tan (\arccos~ t) =
\\ldots~

cos (\arcsin~ t) =
\sqrt1 - t^2,
sin (\arcsin~ t) = t,
tan (\arcsin~ t) =
\\ldots~

cos (\arctan~ t) = 1
\over \sqrt1 + t^2 ,
sin (\arctan~ t) =
\\ldots~,
tan (\arctan~ t) = t

\paragraph{8.5.2 Fonctions hyperboliques directes}

\mathrmch~ x = 1
\over 2 (e^x + e^-x),
\mathrmsh~ x = 1
\over 2 (e^x - e^-x),
\mathrmth~ x =
\mathrmsh~ x
\over
\mathrmch x~ =
e^2x-1 \over e^2x+1

 \mathrmch~
`= \mathrmsh~ ,
\mathrmsh~'
= \mathrmch~ ,
\mathrmth~ ' = 1
-\mathrmth ^2~

 \mathrmch~ x
+ \mathrmsh~ x =
e^x, \mathrmch~ x
-\mathrmsh~ x =
e^-x

 \mathrmch ^2~x
-\mathrmsh ^2~x =
1

\mathrmch~ (a + b)
= \mathrmch~
a\mathrmch~ b
+ \mathrmsh~
a\mathrmsh~ b

\mathrmch~ (a - b)
= \mathrmch~
a\mathrmch~ b
-\mathrmsh~
a\mathrmsh~ b

\mathrmsh~ (a + b)
= \mathrmsh~
a\mathrmch~ b
+ \mathrmch~
a\mathrmsh~ b

\mathrmsh~ (a - b)
= \mathrmsh~
a\mathrmch~ b
-\mathrmch~
a\mathrmsh~ b

\mathrmch~ 2a =
2\mathrmch ^2~a -
1 = 1 + 2\mathrmsh~
^2a = \mathrmch~
^2a + \mathrmsh~
^2a

\mathrmsh~ 2a =
2\mathrmsh~
a\mathrmch~ a,
\mathrmth~ 2a =
2 \mathrmth~ a
\over
1+\mathrmth~
^2a

Si t = \mathrmth~ ( x
\over 2 ),

\mathrmch~ x = 1 +
t^2 \over 1 - t^2 ,
\mathrmsh~ x = 2t
\over 1 - t^2 ,
\mathrmth~ x = 2t
\over 1 + t^2

\paragraph{8.5.3 Fonctions hyperboliques réciproques}

Théorème~8.5.2 (i)
x\mapsto~\mathrmch~
x est un homéomorphisme croissant de {[}0,+\infty~{[} sur {[}1,+\infty~{[}~;
l'homéomorphisme réciproque est noté arg~
\mathrmch~ : {[}1,+\infty~{[}\rightarrow~
{[}0,+\infty~{[}~; arg~
\mathrmch~ est
C^\infty~ sur {]}1,+\infty~{[} et arg~
\mathrmch~ '(x) = 1
\over \sqrtx^2  -1 . (ii)
x\mapsto~\mathrmsh~
x est un C^\infty~ difféomorphisme croissant de {]} -\infty~,+\infty~{[} sur
{]} -\infty~,+\infty~{[}~; le difféomorphisme réciproque est noté
arg~
\mathrmsh~ :{]} -\infty~,+\infty~{[}\rightarrow~{]}
-\infty~,+\infty~{[} et on a arg~
\mathrmsh~ '(x) = 1
\over \sqrt1+x^2 . (iii)
x\mapsto~\mathrmth~
x est un C^\infty~ difféomorphisme croissant de {]} -\infty~,+\infty~{[} sur
{]} - 1,+1{[}~; le difféomorphisme réciproque est noté
arg~
\mathrmth~ :{]} - 1,1{[}\rightarrow~{]}
-\infty~,+\infty~{[} et on a arg~
\mathrmth~ '(x) = 1
\over 1-x^2 .

Pour t ≥ 1,\quad x = arg~
\mathrmch~ t
\Leftrightarrow t =\
\mathrmch x\text et x ≥ 0.

Pour t \in \mathbb{R}~,\quad x = arg~
\mathrmsh~ t
\Leftrightarrow t =\
\mathrmsh x.

Pour t \in{]} - 1,1{[}, \quad x =\
arg \mathrmth~ t
\Leftrightarrow t =\
\mathrmth x.

arg~
\mathrmch~ t
= log (t + \sqrtt~^2
 - 1),\quad arg~
\mathrmsh~ t
= log (t + \sqrtt~^2
 + 1),\quad arg~
\mathrmth~ t = 1
\over 2  log~ ( 1+t
\over 1-t )

{[}
{[}
{[}
{[}

\end{document}

% \documentclass[]{article}
\usepackage[T1]{fontenc}
\usepackage{lmodern}
\usepackage{amssymb,amsmath}
\usepackage{ifxetex,ifluatex}
\usepackage{fixltx2e} % provides \textsubscript
% use upquote if available, for straight quotes in verbatim environments
\IfFileExists{upquote.sty}{\usepackage{upquote}}{}
\ifnum 0\ifxetex 1\fi\ifluatex 1\fi=0 % if pdftex
  \usepackage[utf8]{inputenc}
\else % if luatex or xelatex
  \ifxetex
    \usepackage{mathspec}
    \usepackage{xltxtra,xunicode}
  \else
    \usepackage{fontspec}
  \fi
  \defaultfontfeatures{Mapping=tex-text,Scale=MatchLowercase}
  \newcommand{\euro}{€}
\fi
% use microtype if available
\IfFileExists{microtype.sty}{\usepackage{microtype}}{}
\ifxetex
  \usepackage[setpagesize=false, % page size defined by xetex
              unicode=false, % unicode breaks when used with xetex
              xetex]{hyperref}
\else
  \usepackage[unicode=true]{hyperref}
\fi
\hypersetup{breaklinks=true,
            bookmarks=true,
            pdfauthor={},
            pdftitle={Analyse numerique des fonctions d'une variable},
            colorlinks=true,
            citecolor=blue,
            urlcolor=blue,
            linkcolor=magenta,
            pdfborder={0 0 0}}
\urlstyle{same}  % don't use monospace font for urls
\setlength{\parindent}{0pt}
\setlength{\parskip}{6pt plus 2pt minus 1pt}
\setlength{\emergencystretch}{3em}  % prevent overfull lines
\setcounter{secnumdepth}{0}
 
/* start css.sty */
.cmr-5{font-size:50%;}
.cmr-7{font-size:70%;}
.cmmi-5{font-size:50%;font-style: italic;}
.cmmi-7{font-size:70%;font-style: italic;}
.cmmi-10{font-style: italic;}
.cmsy-5{font-size:50%;}
.cmsy-7{font-size:70%;}
.cmex-7{font-size:70%;}
.cmex-7x-x-71{font-size:49%;}
.msbm-7{font-size:70%;}
.cmtt-10{font-family: monospace;}
.cmti-10{ font-style: italic;}
.cmbx-10{ font-weight: bold;}
.cmr-17x-x-120{font-size:204%;}
.cmsl-10{font-style: oblique;}
.cmti-7x-x-71{font-size:49%; font-style: italic;}
.cmbxti-10{ font-weight: bold; font-style: italic;}
p.noindent { text-indent: 0em }
td p.noindent { text-indent: 0em; margin-top:0em; }
p.nopar { text-indent: 0em; }
p.indent{ text-indent: 1.5em }
@media print {div.crosslinks {visibility:hidden;}}
a img { border-top: 0; border-left: 0; border-right: 0; }
center { margin-top:1em; margin-bottom:1em; }
td center { margin-top:0em; margin-bottom:0em; }
.Canvas { position:relative; }
li p.indent { text-indent: 0em }
.enumerate1 {list-style-type:decimal;}
.enumerate2 {list-style-type:lower-alpha;}
.enumerate3 {list-style-type:lower-roman;}
.enumerate4 {list-style-type:upper-alpha;}
div.newtheorem { margin-bottom: 2em; margin-top: 2em;}
.obeylines-h,.obeylines-v {white-space: nowrap; }
div.obeylines-v p { margin-top:0; margin-bottom:0; }
.overline{ text-decoration:overline; }
.overline img{ border-top: 1px solid black; }
td.displaylines {text-align:center; white-space:nowrap;}
.centerline {text-align:center;}
.rightline {text-align:right;}
div.verbatim {font-family: monospace; white-space: nowrap; text-align:left; clear:both; }
.fbox {padding-left:3.0pt; padding-right:3.0pt; text-indent:0pt; border:solid black 0.4pt; }
div.fbox {display:table}
div.center div.fbox {text-align:center; clear:both; padding-left:3.0pt; padding-right:3.0pt; text-indent:0pt; border:solid black 0.4pt; }
div.minipage{width:100%;}
div.center, div.center div.center {text-align: center; margin-left:1em; margin-right:1em;}
div.center div {text-align: left;}
div.flushright, div.flushright div.flushright {text-align: right;}
div.flushright div {text-align: left;}
div.flushleft {text-align: left;}
.underline{ text-decoration:underline; }
.underline img{ border-bottom: 1px solid black; margin-bottom:1pt; }
.framebox-c, .framebox-l, .framebox-r { padding-left:3.0pt; padding-right:3.0pt; text-indent:0pt; border:solid black 0.4pt; }
.framebox-c {text-align:center;}
.framebox-l {text-align:left;}
.framebox-r {text-align:right;}
span.thank-mark{ vertical-align: super }
span.footnote-mark sup.textsuperscript, span.footnote-mark a sup.textsuperscript{ font-size:80%; }
div.tabular, div.center div.tabular {text-align: center; margin-top:0.5em; margin-bottom:0.5em; }
table.tabular td p{margin-top:0em;}
table.tabular {margin-left: auto; margin-right: auto;}
div.td00{ margin-left:0pt; margin-right:0pt; }
div.td01{ margin-left:0pt; margin-right:5pt; }
div.td10{ margin-left:5pt; margin-right:0pt; }
div.td11{ margin-left:5pt; margin-right:5pt; }
table[rules] {border-left:solid black 0.4pt; border-right:solid black 0.4pt; }
td.td00{ padding-left:0pt; padding-right:0pt; }
td.td01{ padding-left:0pt; padding-right:5pt; }
td.td10{ padding-left:5pt; padding-right:0pt; }
td.td11{ padding-left:5pt; padding-right:5pt; }
table[rules] {border-left:solid black 0.4pt; border-right:solid black 0.4pt; }
.hline hr, .cline hr{ height : 1px; margin:0px; }
.tabbing-right {text-align:right;}
span.TEX {letter-spacing: -0.125em; }
span.TEX span.E{ position:relative;top:0.5ex;left:-0.0417em;}
a span.TEX span.E {text-decoration: none; }
span.LATEX span.A{ position:relative; top:-0.5ex; left:-0.4em; font-size:85%;}
span.LATEX span.TEX{ position:relative; left: -0.4em; }
div.float img, div.float .caption {text-align:center;}
div.figure img, div.figure .caption {text-align:center;}
.marginpar {width:20%; float:right; text-align:left; margin-left:auto; margin-top:0.5em; font-size:85%; text-decoration:underline;}
.marginpar p{margin-top:0.4em; margin-bottom:0.4em;}
.equation td{text-align:center; vertical-align:middle; }
td.eq-no{ width:5%; }
table.equation { width:100%; } 
div.math-display, div.par-math-display{text-align:center;}
math .texttt { font-family: monospace; }
math .textit { font-style: italic; }
math .textsl { font-style: oblique; }
math .textsf { font-family: sans-serif; }
math .textbf { font-weight: bold; }
.partToc a, .partToc, .likepartToc a, .likepartToc {line-height: 200%; font-weight:bold; font-size:110%;}
.chapterToc a, .chapterToc, .likechapterToc a, .likechapterToc, .appendixToc a, .appendixToc {line-height: 200%; font-weight:bold;}
.index-item, .index-subitem, .index-subsubitem {display:block}
.caption td.id{font-weight: bold; white-space: nowrap; }
table.caption {text-align:center;}
h1.partHead{text-align: center}
p.bibitem { text-indent: -2em; margin-left: 2em; margin-top:0.6em; margin-bottom:0.6em; }
p.bibitem-p { text-indent: 0em; margin-left: 2em; margin-top:0.6em; margin-bottom:0.6em; }
.paragraphHead, .likeparagraphHead { margin-top:2em; font-weight: bold;}
.subparagraphHead, .likesubparagraphHead { font-weight: bold;}
.quote {margin-bottom:0.25em; margin-top:0.25em; margin-left:1em; margin-right:1em; text-align:\\jmathmathustify;}
.verse{white-space:nowrap; margin-left:2em}
div.maketitle {text-align:center;}
h2.titleHead{text-align:center;}
div.maketitle{ margin-bottom: 2em; }
div.author, div.date {text-align:center;}
div.thanks{text-align:left; margin-left:10%; font-size:85%; font-style:italic; }
div.author{white-space: nowrap;}
.quotation {margin-bottom:0.25em; margin-top:0.25em; margin-left:1em; }
h1.partHead{text-align: center}
.sectionToc, .likesectionToc {margin-left:2em;}
.subsectionToc, .likesubsectionToc {margin-left:4em;}
.subsubsectionToc, .likesubsubsectionToc {margin-left:6em;}
.frenchb-nbsp{font-size:75%;}
.frenchb-thinspace{font-size:75%;}
.figure img.graphics {margin-left:10%;}
/* end css.sty */

\title{Analyse numerique des fonctions d'une variable}
\author{}
\date{}

\begin{document}
\maketitle

\textbf{Warning: 
requires JavaScript to process the mathematics on this page.\\ If your
browser supports JavaScript, be sure it is enabled.}

\begin{center}\rule{3in}{0.4pt}\end{center}

{[}
{[}
{[}{]}
{[}

\subsubsection{8.6 Analyse numérique des fonctions d'une variable}

\paragraph{8.6.1 Interpolation linéaire, interpolation polynomiale}

Considérons f une fonction de classe C^n sur l'intervalle
{[}a,b{]}, soit x_1 \textless{}
\\ldots~ \textless{}
x_n des points de {[}a,b{]} et considérons l'unique polynôme P
\in \mathbb{R}_n-1{[}X{]} vérifiant P(x_i) = f(x_i) pour
1 \leq i \leq n (polynôme d'interpolation de Lagrange).

Lemme~8.6.1 Pour tout x \in {[}a,b{]}, \exists~\zeta
\in{]}a,b{[}, f(x) - P(x) =
(x-x_1)\\ldots(x-x_n~)
\over n! f^(n)(\zeta).

Démonstration Si x est l'un des x_i, n'importe quel \zeta convient.
On peut donc supposer que f n'est pas l'un des x_i. Considérons
alors g : t\mapsto~f(t) - P(t) - \lambda~
(t-x_1)\\ldots(t-x_n~)
\over n! où \lambda~ est choisi de telle sorte que g(x) = 0
(c'est évidemment possible). Alors g est de classe C^n et a n
+ 1 zéros distincts sur l'intervalle {[}a,b{]}. Des applications
répétées du théorème de Rolle assurent que la fonction g' a n zéros
distincts sur l'intervalle {]}a,b{[} (un entre deux zéros de g au sens
strict), puis que g'' a n - 1 zéros distincts \\jmathmathusque g^(n)
qui a au moins un zéro. Soit donc \zeta tel que g^(n)(\zeta) = 0. On
a donc 0 = g^(n)(\zeta) = f^(n)(\zeta) - \lambda~ car
P^(n) = 0 (vu que deg~ P \leq n - 1) et
 d^n \over dt^n ((t -
x_1)\\ldots~(t
- x_n)) = n!. on a donc \lambda~ = f^(n)(\zeta) et en écrivant
que g(x) = 0, on obtient f(x) - P(x) =
(x-x_1)\\ldots(x-x_n~)
\over n! f^(n)(\zeta).

Considérons le cas particulier où n = 2 et où l'on prend x_1 =
a et x_2 = b~; c'est le cas de l'interpolation linéaire entre a
et b où l'on remplace la courbe y = f(x) par la corde \\jmathmathoignant les
points (a,f(a)) et (b,f(b)). On a alors P(t) = f(a) + f(b)-f(a)
\over b-a (t - a). Si on appelle M_2
=\
sup_t\in{[}a,b{]}f''(t), on obtient

Proposition~8.6.2 (erreur dans une interpolation linéaire). Soit f :
{[}a,b{]} \rightarrow~ \mathbb{R}~ de classe C^2, M_2
=\
sup_t\in{[}a,b{]}f''(t). Alors

\forall~~t \in {[}a,b{]}, f(t)
-\left (f(a) + f(b) - f(a) \over b - a
(t - a)\right )\leq M_2 (b -
a)^2 \over 8

Démonstration On a en effet

\begin{align*} f(x) - P(x)&
=& (x - a)(b - x) \over 2
f''(\zeta)\leq M_2 (x - a)(b - x)
\over 2 \%& \\ & \leq&
M_2 (b - a)^2 \over 8 \%&
\\ \end{align*}

car (x - a)(b - x) \leq (b-a)^2 \over 4 pour
x \in {[}a,b{]}.

\paragraph{8.6.2 Dérivation numérique}

Nous nous limiterons au calcul approché des dérivées d'ordre 1 et 2
d'une fonction numérique. Pour la dérivée d'ordre 1 nous utiliserons la
formule f(x + h) = f(x) + hf'(x) + h\epsilon(h) avec
lim_h\rightarrow~0~\epsilon(h) = 0. On en déduit que
f'(x) est peu différent de \Delta_hf(x) = f(x+h)-f(x)
\over h quand h est petit. Mathématiquement, plus h est
petit, plus \Delta_hf(x) est proche de f'(x). Mais qu'en est-il de
la valeur calculée \overline\Delta_hf(x))~? Le
calcul de f(x + h) - f(x) se fait avec une erreur absolue de l'ordre de
2\deltaf(x), où \delta est la précision de l'instrument de calcul (
10^-16 par exemple). On a donc \Delta_hf(x)
-\overline\Delta_hf(x))\leq
2\deltaf(x) \over h , qui tend vers + \infty~
quand h tend vers 0. Il faut donc trouver un compromis pour la valeur de
h à choisir. Supposons f de classe C^2. Alors on a f(x + h) =
f(x) + hf'(x) + h^2 \over 2 f''(x) +
h^2\epsilon(h) et donc f'(x) - \Delta_hf(x)
est peu différent de  h \over 2
f''(x). On a donc f'(x)
-\overline\Delta_hf(x))\leq h
\over 2 f''(x) +
2\deltaf(x) \over h . Le deuxième membre
est une fonction de h qui est minimale pour h =
2\sqrt \deltaf(x) \over
f''(x)  et qui vaut alors
2\sqrt\deltaf(x)f''(x). Avec les
fonctions usuelles, on retiendra qu'il faut choisir un h de l'ordre de
\sqrt\delta (plutôt un peu trop grand, qu'un peu trop
petit) et que l'on obtient alors une erreur de l'ordre de
\sqrt \delta. Ainsi on choisira par exemple h =
10^-8 et on pourra espérer avoir 7 ou 8 chiffres
significatifs dans le calcul de la dérivée (ce qui est le plus souvent
largement suffisant, par exemple pour une étude de fonction).

Une autre méthode de calcul de la dérivée qui donne des valeurs un peu
plus précises (mais peut poser des problèmes de définition de la
fonction aux bornes de l'intervalle) est de prendre comme valeur
approchée de la dérivée l'expression  f(x+h)-f(x-h)
\over 2h (dérivée symétrique). L'erreur est alors en 
h^2 \over 6
f^(3)(x) + 2\deltaf(x)
\over h , elle est minimale pour un h de l'ordre de
\root3\of\delta et elle est alors de
l'ordre de \delta^2\diagup3 (prendre h = 10^-5 pour obtenir
environ 9 à 10 chiffres significatifs).

Pour le calcul de la dérivée seconde, reprenons la formule de
Taylor-Young f(x + h) = f(x) + hf'(x) + h^2
\over 2 f''(x) + h^2\epsilon(h). On obtient alors
lim_h\rightarrow~0~ f(x+h)+f(x-h)-2f(x)
\over h^2 = f''(x). On utilisera comme
valeur approchée de f''(x) l'expression \Delta_h^(2)f(x) =
f(x+h)+f(x-h)-2f(x) \over h^2 . Utilisons la
même méthode pour évaluer l'erreur entre la valeur calculée de
\Delta_h^(2)f(x) et f''(x). Supposons f de classe
C^4. On a la formule de Taylor Young à l'ordre 4, f(x + h) =
f(x) + hf'(x) + h^2 \over 2 f''(x) +
h^3 \over 6 f^(3)(x) +
h^4 \over 24 f^(4)(x) +
h^4\epsilon(h) d'où \Delta_h^(2)f(x) -
f''(x) est peu différent de  h^2
\over 12 f^(4)(x). On
obtient donc une ma\\jmathmathoration du type
\overline\Delta_h^(2)f(x) -
f''(x)\leq h^2 \over 12
f^(4)(x) + 3\delta \over
h^2 f(x). L'erreur est minimale pour
une valeur de h de l'ordre de
\root4\of\delta. On choisira donc un h
de l'ordre de 10^-4 pour obtenir un résultat optimal pour les
fonctions usuelles.

\paragraph{8.6.3 Recherche des zéros d'une fonction}

On suppose dans toutes les méthodes qui suivent que l'on a effectué la
séparation des zéros de la fonction f, c'est-à-dire que l'on a trouvé un
intervalle {[}a,b{]} sur lequel f est strictement monotone avec f(a)f(b)
\textless{} 0. On suppose également que f est suffisamment dérivable. On
sait alors que f a un unique zéro r sur l'intervalle {[}a,b{]}.

Méthode de dichotomie

Soit c un point de {]}a,b{[}. Alors, soit f(a) et f(c) sont de signe
contraire, auquel cas r \in{]}a,c{[}, soit f(a) et f(c) sont de même signe
et dans ce cas r \in{]}c,a{[}. En prenant c = a+b \over
2 et en itérant le procédé, on construit une suite de segments
emboîtés {[}a_n,b_n{]} tels que
\forall~n \in \mathbb{N}~, r \in {[}a_n,b_n~{]} et
b_n - a_n = 1 \over 2
(b_n-1 - a_n-1), soit b_n - a_n =
1 \over 2^n (b - a). Le théorème des
segments emboîtés garantit alors que
\⋂ ~
_n\in\mathbb{N}~{[}a_n,b_n{]} = r. Si l'on prend
a_n comme valeur approchée de r par exemple, on a r -
a_n\leq 1 \over 2^n (b -
a).

Méthode de Lagrange

La méthode de Lagrange consiste à assimiler sur le segment {[}a,b{]} la
courbe y = f(x) à la droite passant par les points (a,f(a)) et (b,f(b)),
c'est-à-dire à approcher f par la fonction P(x) = f(a) + f(b)-f(a)
\over b-a (x - a) et à prendre comme valeur approchée
de r le réel \barr vérifiant
P(\barr) = 0.

Etudions l'erreur commise dans cette méthode. Soit P(x) = f(a) +
f(b)-f(a) \over b-a (x - a) (interpolation linéaire).
La ma\\jmathmathoration de l'erreur dans une interpolation linéaire nous garantit
que, en posant M_2 =\
sup_t\in{[}a,b{]}f''(t), on a f(r)
- P(r)\leq M_2 (b-a)^2 \over
8 . Mais f(r) = 0 et  P(r) \over
r-\barr = P(r)-P(\barr)
\over r-\barr = f(b)-f(a)
\over b-a = f'(c) pour un c \in{]}a,b{[}. On en déduit
que \barr - r =
\left  P(r) \over f'(c)
\right \leq M_2(b-a)^2
\over 8m_1 si l'on pose m_1
= inf~
_t\in{[}a,b{]}f'(t) (que nous supposerons
strictement positif). D'où la ma\\jmathmathoration de l'erreur dans la méthode de
Lagrange

\barr - r\leq M_2(b -
a)^2 \over 8m_1

On pourra par exemple combiner la méthode de dichotomie et la méthode de
Lagrange pour trouver rapidement une bonne approximation de r.
Remarquons de plus que si on suppose en outre que f est de concavité
constante sur {[}a,b{]}, alors on connait le signe de r
-\bar r. En effet

(f convexe croissante, ou f concave décroissante))
\rigtharrow~\bar r \textless{} r

(f convexe décroissante, ou f concave croissante))
\rigtharrow~\bar r \textgreater{} r

Méthode de Newton

La méthode de Newton consiste à assimiler la courbe y = f(x) à la
tangente en un point c \in {[}a,b{]}. Cette tangente a pour équation y -
f(c) = f'(c)(x - c). Elle coupe donc l'axe des x au point d'abscisse r'
= c - f(c) \over f'(c) .

Cherchons à ma\\jmathmathorer l'erreur commise r - r'. On a
d'après la formule de Taylor Lagrange 0 = f(r) = f(c) + (r - c)f'(c) +
(r-c)^2 \over 2 f''(x) pour un certain x
\in{]}r,c{[}. De plus on a f(c) + (r' - c)f'(c) = 0. En soustrayant membre
à membre les deux égalités on trouve (r' - r)f'(c) -
(r-c)^2 \over 2 f''(x) = 0, soit r' - r =
(r-c)^2 \over 2  f'`(x)
\over f'(c) et donc (avec les notations données ci
dessus dans la méthode de Lagrange)

r' - r\leq M_2(b - a)^2
\over 2m_1

Remarquons que la ma\\jmathmathoration de l'erreur obtenue est 4 fois plus grande
que dans la méthode de Lagrange. Ce n'est évidemment pas que la méthode
de Newton soit moins bonne que la méthode de Lagrange, c'est simplement
la ma\\jmathmathoration de l'erreur qui est un peu moins fine. Remarquons que si
l'on suppose en outre que f est de concavité constante sur {[}a,b{]},
alors on connait le signe de r' - r. En effet

(f convexe croissante, ou f concave décroissante)) \rigtharrow~ r' \textgreater{} r

(f convexe décroissante, ou f concave croissante)) \rigtharrow~ r' \textless{} r

Les inégalités sont donc en sens contraire de celles obtenues par la
méthode de Lagrange. Dans le cas où f est monotone et de concavité
constante sur {[}a,b{]}, la combinaison de la méthode de Lagrange et de
la méthode de Newton fournit un encadrement de r, ce qui est meilleur
qu'une ma\\jmathmathoration d'erreur.

Dans la pratique on ne s'arrête pas après avoir appliqué une fois la
méthode de Newton, mais au contraire on applique de nouveau la méthode
de Newton, mais cette fois ci au point r'. Cela revient à construire une
suite (x_n) par récurrence par x_o = c et
x_n+1 = x_n - f(x_n) \over
f'(x_n) . Les questions qui se posent naturellement sont

\begin{itemize}
\itemsep1pt\parskip0pt\parsep0pt
\item
  (i) est ce que tous les x_n sont dans {[}a,b{]}~? ,
\item
  (ii) est ce que la suite (x_n) converge~?
\item
  (iii) dans ce cas, sa limite est-elle r~?
\end{itemize}

Il est clair que dans la mesure où les réponses aux questions (i) et
(ii) sont oui, la réponse à la question (iii) est aussi oui, puisque si
l'on appelle L la limite de la suite, on doit avoir L = L - f(L)
\over f'(L) , soit f(L) = 0. Il nous reste donc à
répondre aux deux premières questions.

Nous nous placerons sous les hypothèses suivantes~: f est de classe
C^2 sur {[}a,b{]}, f' ne s'annule pas sur {[}a,b{]} et f''
est de signe constant sur {[}a,b{]} (donc f est strictement monotone et
de concavité constante). Dans un premier temps nous supposerons pour
faire les raisonnements que f est croissante convexe sur {[}a,b{]}.
Prenons x_o = c \textgreater{} r (par exemple c = b). On va
alors montrer que \forall~n, x_n~ \in {[}r,b{]}
et que la suite (x_n) est décroissante, ce qui permettra de
répondre positivement aux questions (i) et (ii). Pour cela considérons
la fonction g définie par g(x) = x - f(x) \over f'(x)
= xf'(x)-f(x) \over f'(x) . On a g'(x) = f(x)f'`(x)
\over f'(x)^2 ≥ 0 donc g est croissante sur
{[}a,b{]}. Supposons que x_n \in {[}r,b{]}. Comme g(r) = r, on a
x_n+1 = g(x_n) ≥ g(r) = r. De plus f étant strictement
croissante, elle est positive sur {[}r,b{]} et donc x_n+1 =
x_n - f(x_n) \over f'(x_n)
\leq x_n, d'où r \leq x_n+1 \leq x_n \leq b. On en déduit
que les réponses aux questions (i) et (ii) sont positives et fournissent
donc un moyen d'approximation de r. Remarquons qu'il est fondamental
pour cela d'avoir choisi un c \textgreater{} r, car g est décroissante
sur {[}a,r{]} et donc si x_o = c \textless{} r, x_1 =
g(x_o) ≥ g(r) = r, mais plus rien ne garantit que x_1
appartient tou\\jmathmathours à {[}a,b{]}. Dans les cas de monotonie ou de
concavité différents on a les conclusions suivantes

(f convexe croissante ou f concave décroissante)~: choisir x_o
\textgreater{} r~; la suite (x_n) est décroissante et converge
vers r

(f convexe décroissante ou f concave croissante)~: choisir x_o
\textless{} r~; la suite (x_n) est croissante et converge vers
r

Il nous reste à savoir avec quelle vitesse la suite converge. On a g'(r)
= 0. Puisque g est continue, si l'on se donne K \textless{} 1, il existe
h \textgreater{} 0 tel que \forall~~x \in {[}r,r + h{]},
g'(x) \textless{} K. Alors soit N tel que n
\textgreater{} N \rigtharrow~ x_n \in {[}r,r + h{]}. On a alors pour n
\textgreater{} N, x_n+1 - r =
g(x_n) - g(r)\leq Kx_n -
r, d'où pour n \textgreater{} N, x_n -
r\leq K^n-Nx_N - r. On en
déduit que la suite (x_n - r) est négligeable devant la suite
K^n pour tout K \textless{} 1, et on a donc une convergence
extrêmement rapide dès que l'on se rapproche de r. On peut d'ailleurs
trouver un équivalent de x_n - r dans le
cas où g''(r)\neq~0 et montrer que
x_n - r∼ K^2^n  avec
un K \textless{} 1. On a donc une convergence très rapide (de type
hyperexponentiel)~: en gros le nombre de décimales double à chaque
itération dans la limite de la précision de la machine.

Méthode des approximations successives

Elle consiste à transformer une équation du type f(x) = 0 en une
équation du type g(x) = x à laquelle on essayera d'appliquer une méthode
de point fixe~: on construit une suite x_n+1 =
g(x_n)~; si cette suite converge, elle converge vers un point
fixe de g. Le lecteur pourra remarquer que la méthode de Newton en est
un cas particulier avec g(x) = x - f(x) \over f'(x) .

{[}
{[}
{[}
{[}

\end{document}

% \documentclass[]{article}
\usepackage[T1]{fontenc}
\usepackage{lmodern}
\usepackage{amssymb,amsmath}
\usepackage{ifxetex,ifluatex}
\usepackage{fixltx2e} % provides \textsubscript
% use upquote if available, for straight quotes in verbatim environments
\IfFileExists{upquote.sty}{\usepackage{upquote}}{}
\ifnum 0\ifxetex 1\fi\ifluatex 1\fi=0 % if pdftex
  \usepackage[utf8]{inputenc}
\else % if luatex or xelatex
  \ifxetex
    \usepackage{mathspec}
    \usepackage{xltxtra,xunicode}
  \else
    \usepackage{fontspec}
  \fi
  \defaultfontfeatures{Mapping=tex-text,Scale=MatchLowercase}
  \newcommand{\euro}{€}
\fi
% use microtype if available
\IfFileExists{microtype.sty}{\usepackage{microtype}}{}
\ifxetex
  \usepackage[setpagesize=false, % page size defined by xetex
              unicode=false, % unicode breaks when used with xetex
              xetex]{hyperref}
\else
  \usepackage[unicode=true]{hyperref}
\fi
\hypersetup{breaklinks=true,
            bookmarks=true,
            pdfauthor={},
            pdftitle={Subdivisions, approximation des fonctions},
            colorlinks=true,
            citecolor=blue,
            urlcolor=blue,
            linkcolor=magenta,
            pdfborder={0 0 0}}
\urlstyle{same}  % don't use monospace font for urls
\setlength{\parindent}{0pt}
\setlength{\parskip}{6pt plus 2pt minus 1pt}
\setlength{\emergencystretch}{3em}  % prevent overfull lines
\setcounter{secnumdepth}{0}
 
/* start css.sty */
.cmr-5{font-size:50%;}
.cmr-7{font-size:70%;}
.cmmi-5{font-size:50%;font-style: italic;}
.cmmi-7{font-size:70%;font-style: italic;}
.cmmi-10{font-style: italic;}
.cmsy-5{font-size:50%;}
.cmsy-7{font-size:70%;}
.cmex-7{font-size:70%;}
.cmex-7x-x-71{font-size:49%;}
.msbm-7{font-size:70%;}
.cmtt-10{font-family: monospace;}
.cmti-10{ font-style: italic;}
.cmbx-10{ font-weight: bold;}
.cmr-17x-x-120{font-size:204%;}
.cmsl-10{font-style: oblique;}
.cmti-7x-x-71{font-size:49%; font-style: italic;}
.cmbxti-10{ font-weight: bold; font-style: italic;}
p.noindent { text-indent: 0em }
td p.noindent { text-indent: 0em; margin-top:0em; }
p.nopar { text-indent: 0em; }
p.indent{ text-indent: 1.5em }
@media print {div.crosslinks {visibility:hidden;}}
a img { border-top: 0; border-left: 0; border-right: 0; }
center { margin-top:1em; margin-bottom:1em; }
td center { margin-top:0em; margin-bottom:0em; }
.Canvas { position:relative; }
li p.indent { text-indent: 0em }
.enumerate1 {list-style-type:decimal;}
.enumerate2 {list-style-type:lower-alpha;}
.enumerate3 {list-style-type:lower-roman;}
.enumerate4 {list-style-type:upper-alpha;}
div.newtheorem { margin-bottom: 2em; margin-top: 2em;}
.obeylines-h,.obeylines-v {white-space: nowrap; }
div.obeylines-v p { margin-top:0; margin-bottom:0; }
.overline{ text-decoration:overline; }
.overline img{ border-top: 1px solid black; }
td.displaylines {text-align:center; white-space:nowrap;}
.centerline {text-align:center;}
.rightline {text-align:right;}
div.verbatim {font-family: monospace; white-space: nowrap; text-align:left; clear:both; }
.fbox {padding-left:3.0pt; padding-right:3.0pt; text-indent:0pt; border:solid black 0.4pt; }
div.fbox {display:table}
div.center div.fbox {text-align:center; clear:both; padding-left:3.0pt; padding-right:3.0pt; text-indent:0pt; border:solid black 0.4pt; }
div.minipage{width:100%;}
div.center, div.center div.center {text-align: center; margin-left:1em; margin-right:1em;}
div.center div {text-align: left;}
div.flushright, div.flushright div.flushright {text-align: right;}
div.flushright div {text-align: left;}
div.flushleft {text-align: left;}
.underline{ text-decoration:underline; }
.underline img{ border-bottom: 1px solid black; margin-bottom:1pt; }
.framebox-c, .framebox-l, .framebox-r { padding-left:3.0pt; padding-right:3.0pt; text-indent:0pt; border:solid black 0.4pt; }
.framebox-c {text-align:center;}
.framebox-l {text-align:left;}
.framebox-r {text-align:right;}
span.thank-mark{ vertical-align: super }
span.footnote-mark sup.textsuperscript, span.footnote-mark a sup.textsuperscript{ font-size:80%; }
div.tabular, div.center div.tabular {text-align: center; margin-top:0.5em; margin-bottom:0.5em; }
table.tabular td p{margin-top:0em;}
table.tabular {margin-left: auto; margin-right: auto;}
div.td00{ margin-left:0pt; margin-right:0pt; }
div.td01{ margin-left:0pt; margin-right:5pt; }
div.td10{ margin-left:5pt; margin-right:0pt; }
div.td11{ margin-left:5pt; margin-right:5pt; }
table[rules] {border-left:solid black 0.4pt; border-right:solid black 0.4pt; }
td.td00{ padding-left:0pt; padding-right:0pt; }
td.td01{ padding-left:0pt; padding-right:5pt; }
td.td10{ padding-left:5pt; padding-right:0pt; }
td.td11{ padding-left:5pt; padding-right:5pt; }
table[rules] {border-left:solid black 0.4pt; border-right:solid black 0.4pt; }
.hline hr, .cline hr{ height : 1px; margin:0px; }
.tabbing-right {text-align:right;}
span.TEX {letter-spacing: -0.125em; }
span.TEX span.E{ position:relative;top:0.5ex;left:-0.0417em;}
a span.TEX span.E {text-decoration: none; }
span.LATEX span.A{ position:relative; top:-0.5ex; left:-0.4em; font-size:85%;}
span.LATEX span.TEX{ position:relative; left: -0.4em; }
div.float img, div.float .caption {text-align:center;}
div.figure img, div.figure .caption {text-align:center;}
.marginpar {width:20%; float:right; text-align:left; margin-left:auto; margin-top:0.5em; font-size:85%; text-decoration:underline;}
.marginpar p{margin-top:0.4em; margin-bottom:0.4em;}
.equation td{text-align:center; vertical-align:middle; }
td.eq-no{ width:5%; }
table.equation { width:100%; } 
div.math-display, div.par-math-display{text-align:center;}
math .texttt { font-family: monospace; }
math .textit { font-style: italic; }
math .textsl { font-style: oblique; }
math .textsf { font-family: sans-serif; }
math .textbf { font-weight: bold; }
.partToc a, .partToc, .likepartToc a, .likepartToc {line-height: 200%; font-weight:bold; font-size:110%;}
.chapterToc a, .chapterToc, .likechapterToc a, .likechapterToc, .appendixToc a, .appendixToc {line-height: 200%; font-weight:bold;}
.index-item, .index-subitem, .index-subsubitem {display:block}
.caption td.id{font-weight: bold; white-space: nowrap; }
table.caption {text-align:center;}
h1.partHead{text-align: center}
p.bibitem { text-indent: -2em; margin-left: 2em; margin-top:0.6em; margin-bottom:0.6em; }
p.bibitem-p { text-indent: 0em; margin-left: 2em; margin-top:0.6em; margin-bottom:0.6em; }
.paragraphHead, .likeparagraphHead { margin-top:2em; font-weight: bold;}
.subparagraphHead, .likesubparagraphHead { font-weight: bold;}
.quote {margin-bottom:0.25em; margin-top:0.25em; margin-left:1em; margin-right:1em; text-align:\\jmathmathustify;}
.verse{white-space:nowrap; margin-left:2em}
div.maketitle {text-align:center;}
h2.titleHead{text-align:center;}
div.maketitle{ margin-bottom: 2em; }
div.author, div.date {text-align:center;}
div.thanks{text-align:left; margin-left:10%; font-size:85%; font-style:italic; }
div.author{white-space: nowrap;}
.quotation {margin-bottom:0.25em; margin-top:0.25em; margin-left:1em; }
h1.partHead{text-align: center}
.sectionToc, .likesectionToc {margin-left:2em;}
.subsectionToc, .likesubsectionToc {margin-left:4em;}
.subsubsectionToc, .likesubsubsectionToc {margin-left:6em;}
.frenchb-nbsp{font-size:75%;}
.frenchb-thinspace{font-size:75%;}
.figure img.graphics {margin-left:10%;}
/* end css.sty */

\title{Subdivisions, approximation des fonctions}
\author{}
\date{}

\begin{document}
\maketitle

\textbf{Warning: 
requires JavaScript to process the mathematics on this page.\\ If your
browser supports JavaScript, be sure it is enabled.}

\begin{center}\rule{3in}{0.4pt}\end{center}

{[}
{[}{]}
{[}

\subsubsection{9.1 Subdivisions, approximation des fonctions}

\paragraph{9.1.1 Subdivisions d'un segment}

Définition~9.1.1 Soit {[}a,b{]} un segment de \mathbb{R}~~; on appelle subdivision
de {[}a,b{]} toute famille \sigma = (a_i)_0\leqi\leqn telle que
a_0 = a, a_n = b et \forall~~i \in
{[}1,n{]}, a_i-1 \textless{} a_i. On appelle pas de la
subdivision \sigma le nombre réel \delta(\sigma) =\
min_i\in{[}1,n{]}(a_i - a_i-1).

On notera \mathrmPt~(\sigma) =
\a_i∣0 \leq i \leq
n\ et S({[}a,b{]}) l'ensemble des subdivisions de
{[}a,b{]}. On définit une relation d'ordre partielle sur S(I) en disant
que \sigma' est plus fine que \sigma si
\mathrmPt~(\sigma)
\subset~\mathrmPt~(\sigma') on a alors
clairement \delta(\sigma') \leq \delta(\sigma)). On notera \sigma \cup \sigma' la subdivision définie par
\mathrmPt~(\sigma \cup \sigma')
= \mathrmPt~(\sigma)
\cup\mathrmPt~(\sigma'). Elle est
plus fine que \sigma et que \sigma'.

\paragraph{9.1.2 Propriétés liées aux subdivisions}

Définition~9.1.2 On dit que f : {[}a,b{]} \rightarrow~ E est une fonction en
escalier (resp. affine par morceaux) s'il existe une subdivision \sigma =
(a_i)_0\leqi\leqn de {[}a,b{]} (que l'on dira adaptée à f)
telle que f soit constante (resp. affine) sur chacun des intervalles
ouverts {]}a_i-1,a_i{[}.

Remarque~9.1.1 Il est clair que toute fonction affine par morceaux est
bornée et que toute fonction en escalier est affine par morceaux.

Définition~9.1.3 On dit que f : {[}a,b{]} \rightarrow~ E est une fonction de classe
C^k par morceaux s'il existe une subdivision \sigma =
(a_i)_0\leqi\leqn de {[}a,b{]} (que l'on dira adaptée à f)
telle que l'on ait les conditions équivalentes (i) pour chaque i \in
{[}1,n{]}, il existe une fonction f_i :
{[}a_i-1,a_i{]} \rightarrow~ E de classe C^k telle que
\forall~t \in{]}a_i-1,a_i~{[}, f(t) =
f_i(t) (ii) la fonction f est de classe C^k sur
{[}a,b{]}
\diagdown\a_0,\\ldots,a_n\~
et
f,f',\\ldots,f^(k)~
admettent des limites à gauche et à droite en tous les points
a_i où cela a un sens.

Démonstration (i) \rigtharrow~(ii) est clair puisque dans ce cas
lim_t\rightarrow~a_i^+f^(p)~(t)
=\
lim_t\rightarrow~a_i^+f_i+1^(p)(t) =
f_i+1^(p)(a_i) et
lim_t\rightarrow~a_i^-f^(p)~(t)
=\
lim_t\rightarrow~a_i^-f_i^(p)(t) =
f_i^(p)(a_i).

En ce qui concerne (ii) \rigtharrow~(i) posons

 f_i(t) = \left \
\cases f(t) &si a_i-1 \textless{} t
\textless{} a_i \cr
f(a_i-1^+)&si t = a_i-1 \cr
f(a_i^-) &si t = a_i  \right
.

alors la fonction f_i est continue sur
{[}a_i-1,a_i{]}, de classe C^k sur
{]}a_i-1,a_i{[} et toutes les dérivées
f_i^(p) = f^(p) admettent des limites aux
points a_i-1 et a_i. On a vu dans le chapitre sur les
fonctions d'une variable qu'une telle fonction était de classe
C^k.

Remarque~9.1.2 Une fonction de classe C^k par morceaux n'est
pas nécessairement continue (en particulier f(a_i) peut être
distinct de la limite à gauche et de la limite à droite au point
a_i)~; cependant, comme chacune des f_i est continue
sur un compact donc bornée, et que les a_i sont en nombre fini,
une fonction de classe C^k par morceaux est bornée~;
remarquons également qu'une fonction affine par morceaux (et a fortiori
une fonction en escalier) est de classe C^\infty~ par morceaux.

Remarque~9.1.3 Si f est en escalier, ou affine par morceaux, ou
C^k par morceaux et si \sigma \inS({[}a,b{]}) est adaptée à f, alors
toute subdivision plus fine est encore adaptée à f~; en particulier si \sigma
est adaptée à f et \sigma' adaptée à g, alors \sigma \cup \sigma' est adaptée à la fois à
f et à g, d'où l'on déduit immédiatement la proposition suivante~:

Proposition~9.1.1 L'ensemble des applications en escalier (resp. affine
par morceaux, resp. C^k par morceaux) est un sous-espace
vectoriel ~de l'ensemble des applications de {[}a,b{]} dans E.

Définition~9.1.4 (Extension).Si I est un intervalle de \mathbb{R}~, on dira que f
: I \rightarrow~ E est en escalier (resp. affine par morceaux, resp. C^k
par morceaux) si sa restriction à tout segment {[}a,b{]} contenu dans I
est en escalier (resp. affine par morceaux, resp. C^k par
morceaux).

\paragraph{9.1.3 Approximation des fonctions}

Soit {[}a,b{]} un segment de \mathbb{R}~ et ℬ({[}a,b{]},E) l'espace vectoriel des
fonctions bornées de {[}a,b{]} dans E. Pour f \inℬ({[}a,b{]},E), on notera
\f\\infty~
=\
sup_x\in{[}a,b{]}\f(x)\.

Définition~9.1.5 On dit que f : {[}a,b{]} \rightarrow~ E est réglée si elle vérifie
les conditions équivalentes (i) pour tout \epsilon \textgreater{} 0, il existe
\phi : {[}a,b{]} \rightarrow~ E en escalier telle que
sup_t\in{[}a,b{]}~\f(t)
- \phi(t)\ \leq \epsilon (ii) il existe une suite
\phi_n de fonctions en escalier de {[}a,b{]} dans E telle que
lim_n\rightarrow~+\infty~~\left
(sup_t\in{[}a,b{]}~\f(t)
- \phi_n(t)\\right ) = 0.
L'ensemble des fonctions réglées de {[}a,b{]} dans E est un sous-espace
vectoriel de ℬ({[}a,b{]},E).

Démonstration Les deux définitions sont clairement équivalentes (prendre
\epsilon = 1\diagupn pour (i) \rigtharrow~(ii)). La définition (i) implique évidemment que f - \phi
est bornée et comme \phi l'est également, une fonction réglée sur un
segment est nécessairement bornée. Autrement dit, le sous-espace
vectoriel des fonctions réglées de {[}a,b{]} dans E (car il est
évidemment stable par combinaisons linéaires) n'est autre que
l'adhérence de l'espace vectoriel des fonctions en escalier pour la
norme \.\\infty~.

Théorème~9.1.2 Toute fonction continue par morceaux est réglée.

Démonstration Commen\ccons par le démontrer pour une
fonction continue sur un segment. Une telle fonction est uniformément
continue. Soit \epsilon \textgreater{} 0. Il existe \eta \textgreater{} 0 tel que
\forall~~t,t' \in {[}a,b{]}, t - t'
\textless{} \eta \rigtharrow~\ f(t) -
f(t')\ \textless{} \epsilon. Soit alors \sigma =
(a_i)_0\leqi\leqn une subdivision de pas plus petit que \eta et
définissons \phi : {[}a,b{]} \rightarrow~ E par \phi(a_i) = f(a_i) pour
i \in {[}0,n{]} et \phi(t) = f( a_i-1+a_i
\over 2 ) si t \in{]}a_i-1,a_i{[} avec
i \in {[}1,n{]}. Alors \f(t) -
\phi(t)\ vaut 0 si t est l'un des a_i et
\f(t) - f( a_i-1+a_i
\over 2 )\ \leq \epsilon si t
\in{]}a_i-1,a_i{[} puisque alors t -
a_i-1+a_i \over 2 
\textless{} \delta(\sigma) \textless{} \eta. On a bien une fonction \phi en escalier
telle que
sup_t\in{[}a,b{]}~\f(t)
- \phi(t)\ \leq \epsilon.

Si maintenant f est continue par morceaux, soit \sigma =
(a_i)_0\leqi\leqn une subdivision adaptée à f et soit
f_i : {[}a_i-1,a_i{]} \rightarrow~ E de classe
C^o telle que \forall~~t
\in{]}a_i-1,a_i{[}, f(t) = f_i(t). On peut
appliquer le cas précédent à f_i et trouver \phi_i en
escalier telle que
sup_t\in{[}a_i-1,a_i{]}\f_i~(t)
- \phi_i(t)\ \leq \epsilon. On définit alors une
fonction en escalier \phi : {[}a,b{]} \rightarrow~ E par \phi(a_i) =
f(a_i) et \phi(t) = \phi_i(t) si t
\in{]}a_i-1,a_i{[}. Alors on a
sup_t\in{[}a,b{]}~\f(t)
- \phi(t)\ \leq\
max_i\in{[}1,n{]}\left
(sup_t\in{[}a_i-1,a_i{]}\f_i~(t)
- \phi_i(t)\\right ) \leq \epsilon,
ce qui montre que f est réglée.

En fait, on peut montrer le résultat plus général suivant (que nous
n'utiliserons pas par la suite)

Théorème~9.1.3 Une fonction f : {[}a,b{]} \rightarrow~ E (espace vectoriel
normé~complet) est réglée si et seulement si~elle admet en tout point de
{[}a,b{]} (où cela a un sens) une limite à gauche et une limite à
droite.

Démonstration Supposons tout d'abord f réglée~; soit x_o un
point de {[}a,b{]} et montrons que, si
x_o\neq~b, f a une limite à droite au
point x_o à l'aide du critère de Cauchy pour les fonctions.
Soit donc \epsilon \textgreater{} 0 et \phi en escalier telle que
\forall~~t \in {[}a,b{]}, \f(t)
- \phi(t)\ \textless{} \epsilon \over
3 . Soit (a_i) une subdivision adaptée à \phi et soit \eta
\textgreater{} 0 tel que {]}x_o,x_o +
\eta{[}\subset~{]}a_i-1,a_i{[}. Pour t,t'
\in{]}x_o,x_o + \eta{[}, on a
\f(t) - f(t')\
=\ (f(t) - \phi(t)) + (\phi(t') -
f(t'))\ (car \phi(t) = \phi(t')), soit
\f(t) - f(t')\ \leq \epsilon
\over 3 + \epsilon \over 3 \textless{} \epsilon.
La fonction f vérifie donc le critère de Cauchy en x_o à
droite, et donc elle admet une limite à droite.

Inversement, supposons que f admette en tout point de {[}a,b{]} une
limite à gauche et une limite à droite et soit \epsilon \textgreater{} 0.
Alors, pour tout x \in {[}a,b{]}, il existe \eta_x \textgreater{} 0
tel que \forall~t,t' \in{]}x,x + \eta_x~{[},
\f(t) - f(t')\
\textless{} \epsilon et \forall~~t,t' \in{]}x -
\eta_x,x{[}, \f(t) -
f(t')\ \textless{} \epsilon (critère de Cauchy comme
condition nécessaire d'existence des limites). On a alors {[}a,b{]}
\subset~\⋃ ~
_x\in{[}a,b{]}{]}x - \eta_x,x + \eta_x{[} (recouvrement
de {[}a,b{]} par des ouverts). D'après le théorème de Borel Lebesgue, on
peut trouver
x_1,\\ldots,x_p~
tels que {[}a,b{]} \subset~{]}x_1 -
\eta_x_1,x_1 +
\eta_x_1{[}\cup\\ldots\cup{]}x_p~
- \eta_x_p,x_p + \eta_x_p{[}.
Soit alors (a_i)_0\leqi\leqn une subdivision de {[}a,b{]}
telle que, pour tout i \in {[}1,n{]}, il existe \\jmathmath \in {[}1,p{]} tel que
{]}a_i-1,a_i{[}\subset~{]}x_\\jmathmath -
\eta_x_\\jmathmath,x_\\jmathmath{[} ou
{]}a_i-1,a_i{[}\subset~{]}x_\\jmathmath,x_\\jmathmath +
\eta_x_\\jmathmath{[}. On a alors \forall~~t,t'
\in{]}a_i-1,a_i{[}, \f(t) -
f(t')\ \textless{} \epsilon. On définit alors une
fonction \phi : {[}a,b{]} \rightarrow~ E par \phi(a_i) = f(a_i) et \phi(t)
= f( a_i-1+a_i \over 2 ) si t
\in{]}a_i-1,a_i{[} avec i \in {[}1,n{]}. Alors on a
\forall~~t \in {[}a,b{]}, \f(t)
- \phi(t)\ \leq \epsilon. Donc f est réglée.

Remarque~9.1.4 En particulier, on retrouve que les fonctions continues
par morceaux, mais aussi les fonctions monotones, sont réglées.

Enfin, pour terminer sur le problème de l'approximation des fonctions
nous donnerons les résultat suivants permettant d'approcher une fonction
continue soit par une fonction continue et affine par morceaux, soit par
une fonction polynomiale, soit par un polynôme trigonométrique.

Théorème~9.1.4 Soit f : {[}a,b{]} \rightarrow~ \mathbb{R}~ continue. Alors, pour tout \epsilon
\textgreater{} 0, il existe une fonction \phi : {[}a,b{]} \rightarrow~ E continue et
affine par morceaux telle que \forall~~t \in {[}a,b{]},
\f(t) - \phi(t)\
\textless{} \epsilon.

Démonstration Une telle fonction est uniformément continue. Soit \epsilon
\textgreater{} 0. Il existe \eta \textgreater{} 0 tel que
\forall~~t,t' \in {[}a,b{]}, t - t'
\textless{} \eta \rigtharrow~\ f(t) -
f(t')\ \textless{} \epsilon \over 2
. Soit alors \sigma = (a_i)_0\leqi\leqn une subdivision de pas
plus petit que \eta et définissons \phi : {[}a,b{]} \rightarrow~ E par \phi(a_i) =
f(a_i) pour i \in {[}0,n{]}, \phi(t) = f(a_i-1) +
f(a_i)-f(a_i-1 \over
a_i-a_i-1 (t - a_i-1) si t
\in{]}a_i-1,a_i{[}. Alors \phi est clairement affine par
morceaux et continue. Alors \f(t) -
\phi(t)\ vaut 0 si t est l'un des a_i et
si t \in{]}a_i-1,a_i{[}, on a (en tenant compte de
f(a_i-1) = \phi(a_i-1))

\begin{align*} \f(t) -
\phi(t)& \leq& \f(t) -
f(a_i-1)\ +\
\phi(a_i-1) - \phi(t)\ \%&
\\ & \leq& \f(t) -
f(a_i-1)\ +\
\phi(a_i-1) - \phi(a_i)\\%&
\\ & =& \f(t) -
f(a_i-1)\ +\
f(a_i-1) - f(a_i)\\%&
\\ & \textless{}& 2 \epsilon
\over 2 = \epsilon \%& \\
\end{align*}

puisque, \phi étant affine sur {[}a_i-1,a_i{]}, on a
\\phi(a_i-1) -
\phi(t)\ \leq\
\phi(a_i-1) - \phi(a_i)\. Ceci
termine la démonstration.

Théorème~9.1.5 (premier théorème de Weierstrass). Soit f : {[}a,b{]} \rightarrow~ \mathbb{C}
continue. Alors, pour tout \epsilon \textgreater{} 0, il existe un polynôme P \in
\mathbb{C}{[}X{]} tel que \forall~~t \in {[}a,b{]},
\f(t) - P(t)\
\textless{} \epsilon.

Démonstration Ce résultat pourra être admis. Nous en donnerons cependant
une démonstration qui construit effectivement un tel polynôme (appelé
polynôme de Bernstein, de tels polynômes \\jmathmathouent un rôle important en
infographie). Il suffit évidemment de montrer ce résultat lorsque a = 0
et b = 1 (on fait ensuite un changement de variable affine qui
transforme {[}0,1{]} en {[}a,b{]}). Pour cela on part des identités
élémentaires suivantes (la première est la formule du binôme, les deux
autres s'en déduisent en dérivant par rapport à u)

\begin{align*} (u + v)^n& =&
\sum _k=0^nC_
n^ku^kv^n-k \%&
\\ n(u + v)^n-1& =&
\sum _k=1^nC_
n^kku^k-1v^n-k \%&
\\ n(n - 1)(u + v)^n-2& =&
\sum _k=2^nC_
n^kk(k - 1)u^k-2v^n-k\%&
\\ \end{align*}

Changeant u en x et v en 1 - x, on en déduit que

\begin{align*} 1& =& \\sum
_k=0^nC_ n^kx^k(1 -
x)^n-k \%& \\ n& =&
\sum _k=1^nC_
n^kkx^k-1(1 - x)^n-k \%&
\\ n(n - 1)& =&
\sum _k=2^nC_
n^kk(k - 1)x^k-2(1 - x)^n-k\%&
\\ \end{align*}

Soit a \in \mathbb{R}~. Ecrivons alors que

 \left (a - k \over n
\right )^2 = a^2 + ( 1
\over n^2 - 2 a \over n
)k + 1 \over n^2 k(k - 1)

On en déduit que

\begin{align*} \\sum
_k=0^nC_ n^k\left (a
- k \over n \right
)^2x^k(1 - x)^n-k&& \%&
\\ & =& a^2
\sum _k=0^nC_
n^kx^k(1 - x)^n-k + ( 1
\over n^2 - 2 a \over n
)\sum _k=0^nC_
n^kkx^k(1 - x)^n-k \%&
\\ & \text & + 1
\over n^2  \\sum
_k=0^nC_ n^kk(k - 1)x^k(1 -
x)^n-k \%& \\ & =&
a^2 \\sum
_k=0^nC_ n^kx^k(1 -
x)^n-k + ( 1 \over n^2 - 2 a
\over n )\\sum
_k=1^nC_ n^kkx^k(1 -
x)^n-k \%& \\ &
\text & + 1 \over n^2
 \sum _k=2^nC_
n^kk(k - 1)x^k(1 - x)^n-k \%&
\\ & =& a^2
\sum _k=0^nC_
n^kx^k(1 - x)^n-k + ( 1
\over n^2 - 2 a \over n
)x\sum _k=1^nC_
n^kkx^k-1(1 - x)^n-k\%&
\\ & \text & + 1
\over n^2 x^2
\sum _k=2^nC_
n^kk(k - 1)x^k-2(1 - x)^n-k \%&
\\ & =& a^2 + ( 1
\over n^2 - 2 a \over n
)xn + 1 \over n^2 x^2n(n - 1) =
(x - a)^2 + x(1 - x) \over n \%&
\\ \end{align*}

puis en rempla\ccant a par x

\sum _k=0^nC_
n^k\left (x - k \over n
\right )^2x^k(1 - x)^n-k
= x(1 - x) \over n

Soit \delta \textgreater{} 0. On a donc, pour x \in {[}0,1{]},

\begin{align*} \delta^2
\sum _\left x- k
\over n \right
≥\deltaC_n^kx^k(1 -
x)^n-k&& \%& \\ & \leq&
\sum _\left x-
k \over n \right
≥\deltaC_n^k\left (x - k
\over n \right
)^2x^k(1 - x)^n-k\%&
\\ & \leq& \\sum
_k=0^nC_ n^k\left (x
- k \over n \right
)^2x^k(1 - x)^n-k \%&
\\ & =& x(1 - x) \over
n \leq 1 \over 4n \%&
\\ \end{align*}

soit encore

\sum _\left x-
k \over n \right
≥\deltaC_n^kx^k(1 - x)^n-k
\leq 1 \over 4n\delta^2

Soit f : {[}0,1{]} \rightarrow~ \mathbb{C} continue et posons B_n(x)
= \\sum ~
_k=0^nf( k \over n
)C_n^kx^k(1 - x)^n-k (polynôme en
x de degré inférieur ou égal à n). Ecrivons

f(x) = f(x)1 = \\sum
_k=0^nf(x)C_ n^kx^k(1 -
x)^n-k

On a alors

\begin{align*} f(x) -
B_n(x)& =& \left
\sum _k=0^n~(f(x) - f(
k \over n ))C_n^kx^k(1 -
x)^n-k\right \%&
\\ & \leq& \\sum
_k=0^n\left f(x) - f( k
\over n )\right
C_n^kx^k(1 - x)^n-k \%&
\\ \end{align*}

Soit \epsilon \textgreater{} 0, puisque f est continue sur {[}0,1{]}, elle est
uniformément continue et donc il existe \delta \textgreater{} 0 tel que, pour
tout couple t,t' vérifiant t - t' \textless{} \delta, on
ait f(t) - f(t') \textless{} \epsilon
\over 2 . On a alors en ma\\jmathmathorant suivant les cas
\left f(x) - f( k \over n
)\right  par  \epsilon \over 2 ou
par 2\f_\infty~

\begin{align*} f(x) -
B_n(x)& \leq& \\sum
_\left x- k \over n
\right \textless{}\delta\left
f(x) - f( k \over n )\right
C_n^kx^k(1 - x)^n-k \%&
\\ & \text &
+\sum _\left x-
k \over n \right
≥\delta\left f(x) - f( k
\over n )\right
C_n^kx^k(1 - x)^n-k\%&
\\ & \leq& \epsilon \over 2
\sum _\left x-
k \over n \right
\textless{}\deltaC_n^kx^k(1 -
x)^n-k \%& \\ &
\text &
+2\f_\infty~\\sum
_\left x- k \over n
\right
≥\deltaC_n^kx^k(1 - x)^n-k
\%& \\ & \leq& \epsilon \over
2 \sum _k=0^nC_
n^kx^k(1 - x)^n-k \%&
\\ & \text &
+2\f_\infty~\\sum
_\left x- k \over n
\right
≥\deltaC_n^kx^k(1 - x)^n-k
\%& \\ & \leq& \epsilon \over
2 +
2\f_\infty~ 1
\over 4n\delta^2 \%&
\\ \end{align*}

Prenons alors n assez grand pour que
2\f_\infty~ 1
\over 4n\delta^2 \textless{} \epsilon
\over 2 . On a alors, pour tout x \in {[}0,1{]},
f(x) - B_n(x) \textless{} \epsilon, ce qui achève
la démonstration.

Définition~9.1.6 On appelle polynôme trigonométrique de coefficients
a_p \in \mathbb{C}, p =
-N,\\ldots~,N la
fonction périodique de période 2\pi~ de \mathbb{R}~ dans \mathbb{C},
t\mapsto~\\\sum
 _p=-N^Na_pe^ipt.

Remarque~9.1.5 Les coefficients a_p sont uniquement déterminés
puisque les fonctions t\mapsto~e^ipt
forment une famille libre (ce sont par exemple des vecteurs propres de
l'opérateur de dérivation).

On vérifie immédiatement que les polynômes trigonométriques forment une
sous algèbre de l'algèbre des fonctions de classe C^\infty~ de \mathbb{R}~
dans \mathbb{C}.

Théorème~9.1.6 (deuxième théorème de Weierstrass). Soit f : \mathbb{R}~ \rightarrow~ \mathbb{C}
continue, périodique de période 2\pi~. Alors, pour tout \epsilon \textgreater{} 0,
il existe un polynôme trigonométrique g : \mathbb{R}~ \rightarrow~ \mathbb{C} tel que
\forall~~t \in \mathbb{R}~, \f(t) -
g(t)\ \textless{} \epsilon.

Démonstration Ce théorème sera admis~: c'est une conséquence facile du
théorème de Dirichlet qui dit qu'une fonction continue, de classe
\mathcal{C}^1 par morceaux, périodique de période 2\pi~ est somme de sa
série de Fourier qui converge normalement, donc peut être approchée à 
\epsilon \over 2 près par un polynôme trigonométrique. Il
suffit donc d'approcher notre fonction continue à  \epsilon
\over 2 près par une fonction continue, de classe
\mathcal{C}^1 par morceaux, périodique de période 2\pi~, et pour cela
d'approcher la restriction de f à {[}0,2\pi~{]} par une fonction continue
affine par morceaux prenant la même valeur en 0 et 2\pi~, ce qui se fait
comme ci-dessus.

{[}
{[}

\end{document}

% \documentclass[]{article}
\usepackage[T1]{fontenc}
\usepackage{lmodern}
\usepackage{amssymb,amsmath}
\usepackage{ifxetex,ifluatex}
\usepackage{fixltx2e} % provides \textsubscript
% use upquote if available, for straight quotes in verbatim environments
\IfFileExists{upquote.sty}{\usepackage{upquote}}{}
\ifnum 0\ifxetex 1\fi\ifluatex 1\fi=0 % if pdftex
  \usepackage[utf8]{inputenc}
\else % if luatex or xelatex
  \ifxetex
    \usepackage{mathspec}
    \usepackage{xltxtra,xunicode}
  \else
    \usepackage{fontspec}
  \fi
  \defaultfontfeatures{Mapping=tex-text,Scale=MatchLowercase}
  \newcommand{\euro}{€}
\fi
% use microtype if available
\IfFileExists{microtype.sty}{\usepackage{microtype}}{}
\ifxetex
  \usepackage[setpagesize=false, % page size defined by xetex
              unicode=false, % unicode breaks when used with xetex
              xetex]{hyperref}
\else
  \usepackage[unicode=true]{hyperref}
\fi
\hypersetup{breaklinks=true,
            bookmarks=true,
            pdfauthor={},
            pdftitle={Integrale des fonctions reglees sur un segment},
            colorlinks=true,
            citecolor=blue,
            urlcolor=blue,
            linkcolor=magenta,
            pdfborder={0 0 0}}
\urlstyle{same}  % don't use monospace font for urls
\setlength{\parindent}{0pt}
\setlength{\parskip}{6pt plus 2pt minus 1pt}
\setlength{\emergencystretch}{3em}  % prevent overfull lines
\setcounter{secnumdepth}{0}
 
/* start css.sty */
.cmr-5{font-size:50%;}
.cmr-7{font-size:70%;}
.cmmi-5{font-size:50%;font-style: italic;}
.cmmi-7{font-size:70%;font-style: italic;}
.cmmi-10{font-style: italic;}
.cmsy-5{font-size:50%;}
.cmsy-7{font-size:70%;}
.cmex-7{font-size:70%;}
.cmex-7x-x-71{font-size:49%;}
.msbm-7{font-size:70%;}
.cmtt-10{font-family: monospace;}
.cmti-10{ font-style: italic;}
.cmbx-10{ font-weight: bold;}
.cmr-17x-x-120{font-size:204%;}
.cmsl-10{font-style: oblique;}
.cmti-7x-x-71{font-size:49%; font-style: italic;}
.cmbxti-10{ font-weight: bold; font-style: italic;}
p.noindent { text-indent: 0em }
td p.noindent { text-indent: 0em; margin-top:0em; }
p.nopar { text-indent: 0em; }
p.indent{ text-indent: 1.5em }
@media print {div.crosslinks {visibility:hidden;}}
a img { border-top: 0; border-left: 0; border-right: 0; }
center { margin-top:1em; margin-bottom:1em; }
td center { margin-top:0em; margin-bottom:0em; }
.Canvas { position:relative; }
li p.indent { text-indent: 0em }
.enumerate1 {list-style-type:decimal;}
.enumerate2 {list-style-type:lower-alpha;}
.enumerate3 {list-style-type:lower-roman;}
.enumerate4 {list-style-type:upper-alpha;}
div.newtheorem { margin-bottom: 2em; margin-top: 2em;}
.obeylines-h,.obeylines-v {white-space: nowrap; }
div.obeylines-v p { margin-top:0; margin-bottom:0; }
.overline{ text-decoration:overline; }
.overline img{ border-top: 1px solid black; }
td.displaylines {text-align:center; white-space:nowrap;}
.centerline {text-align:center;}
.rightline {text-align:right;}
div.verbatim {font-family: monospace; white-space: nowrap; text-align:left; clear:both; }
.fbox {padding-left:3.0pt; padding-right:3.0pt; text-indent:0pt; border:solid black 0.4pt; }
div.fbox {display:table}
div.center div.fbox {text-align:center; clear:both; padding-left:3.0pt; padding-right:3.0pt; text-indent:0pt; border:solid black 0.4pt; }
div.minipage{width:100%;}
div.center, div.center div.center {text-align: center; margin-left:1em; margin-right:1em;}
div.center div {text-align: left;}
div.flushright, div.flushright div.flushright {text-align: right;}
div.flushright div {text-align: left;}
div.flushleft {text-align: left;}
.underline{ text-decoration:underline; }
.underline img{ border-bottom: 1px solid black; margin-bottom:1pt; }
.framebox-c, .framebox-l, .framebox-r { padding-left:3.0pt; padding-right:3.0pt; text-indent:0pt; border:solid black 0.4pt; }
.framebox-c {text-align:center;}
.framebox-l {text-align:left;}
.framebox-r {text-align:right;}
span.thank-mark{ vertical-align: super }
span.footnote-mark sup.textsuperscript, span.footnote-mark a sup.textsuperscript{ font-size:80%; }
div.tabular, div.center div.tabular {text-align: center; margin-top:0.5em; margin-bottom:0.5em; }
table.tabular td p{margin-top:0em;}
table.tabular {margin-left: auto; margin-right: auto;}
div.td00{ margin-left:0pt; margin-right:0pt; }
div.td01{ margin-left:0pt; margin-right:5pt; }
div.td10{ margin-left:5pt; margin-right:0pt; }
div.td11{ margin-left:5pt; margin-right:5pt; }
table[rules] {border-left:solid black 0.4pt; border-right:solid black 0.4pt; }
td.td00{ padding-left:0pt; padding-right:0pt; }
td.td01{ padding-left:0pt; padding-right:5pt; }
td.td10{ padding-left:5pt; padding-right:0pt; }
td.td11{ padding-left:5pt; padding-right:5pt; }
table[rules] {border-left:solid black 0.4pt; border-right:solid black 0.4pt; }
.hline hr, .cline hr{ height : 1px; margin:0px; }
.tabbing-right {text-align:right;}
span.TEX {letter-spacing: -0.125em; }
span.TEX span.E{ position:relative;top:0.5ex;left:-0.0417em;}
a span.TEX span.E {text-decoration: none; }
span.LATEX span.A{ position:relative; top:-0.5ex; left:-0.4em; font-size:85%;}
span.LATEX span.TEX{ position:relative; left: -0.4em; }
div.float img, div.float .caption {text-align:center;}
div.figure img, div.figure .caption {text-align:center;}
.marginpar {width:20%; float:right; text-align:left; margin-left:auto; margin-top:0.5em; font-size:85%; text-decoration:underline;}
.marginpar p{margin-top:0.4em; margin-bottom:0.4em;}
.equation td{text-align:center; vertical-align:middle; }
td.eq-no{ width:5%; }
table.equation { width:100%; } 
div.math-display, div.par-math-display{text-align:center;}
math .texttt { font-family: monospace; }
math .textit { font-style: italic; }
math .textsl { font-style: oblique; }
math .textsf { font-family: sans-serif; }
math .textbf { font-weight: bold; }
.partToc a, .partToc, .likepartToc a, .likepartToc {line-height: 200%; font-weight:bold; font-size:110%;}
.chapterToc a, .chapterToc, .likechapterToc a, .likechapterToc, .appendixToc a, .appendixToc {line-height: 200%; font-weight:bold;}
.index-item, .index-subitem, .index-subsubitem {display:block}
.caption td.id{font-weight: bold; white-space: nowrap; }
table.caption {text-align:center;}
h1.partHead{text-align: center}
p.bibitem { text-indent: -2em; margin-left: 2em; margin-top:0.6em; margin-bottom:0.6em; }
p.bibitem-p { text-indent: 0em; margin-left: 2em; margin-top:0.6em; margin-bottom:0.6em; }
.subsectionHead, .likesubsectionHead { margin-top:2em; font-weight: bold;}
.sectionHead, .likesectionHead { font-weight: bold;}
.quote {margin-bottom:0.25em; margin-top:0.25em; margin-left:1em; margin-right:1em; text-align:justify;}
.verse{white-space:nowrap; margin-left:2em}
div.maketitle {text-align:center;}
h2.titleHead{text-align:center;}
div.maketitle{ margin-bottom: 2em; }
div.author, div.date {text-align:center;}
div.thanks{text-align:left; margin-left:10%; font-size:85%; font-style:italic; }
div.author{white-space: nowrap;}
.quotation {margin-bottom:0.25em; margin-top:0.25em; margin-left:1em; }
h1.partHead{text-align: center}
.sectionToc, .likesectionToc {margin-left:2em;}
.subsectionToc, .likesubsectionToc {margin-left:4em;}
.sectionToc, .likesectionToc {margin-left:6em;}
.frenchb-nbsp{font-size:75%;}
.frenchb-thinspace{font-size:75%;}
.figure img.graphics {margin-left:10%;}
/* end css.sty */

\title{Integrale des fonctions reglees sur un segment}
\author{}
\date{}

\begin{document}
\maketitle

\textbf{Warning: 
requires JavaScript to process the mathematics on this page.\\ If your
browser supports JavaScript, be sure it is enabled.}

\begin{center}\rule{3in}{0.4pt}\end{center}

[
[
[]
[

\section{9.2 Intégrale des fonctions réglées sur un segment}

\subsection{9.2.1 Intégrale des applications en escalier}

Théorème~9.2.1 Soit f : [a,b] \rightarrow~ E en escalier et \sigma =
(a_i)_0\leqi\leqn une subdivision adaptée à f~; alors la
somme \\sum ~
_i=1^n(a_i - a_i-1)f_i (où l'on
désigne par f_i la constante telle que
\forall~t \in]a_i-1,a_i~[, f(t) =
f_i) est indépendante du choix de \sigma~; on la note
\int  _a^b~f et on l'appelle
l'intégrale sur [a,b] de la fonction en escalier f.

Démonstration Soit \sigma' une subdivision adaptée à f telle que
\mathrmPt~(\sigma')
= \mathrmPt~(\sigma)
\cup\c\. Alors, si c
\in]a_k-1,a_k], la somme relative à \sigma' est

\begin{align*} \\sum
_i=1^k-1(a_ i - a_i-1)f_i +
(c - a_k-1)f_k + (a_k - c)f_k +
\sum _i=k+1^n(a_ i~ -
a_i-1)&&\%& \\ & =&
\sum _i=1^k-1(a_ i~ -
a_i-1)f_i + (a_k -
a_k-1)f_i + \\sum
_i=k+1^n(a_ i - a_i-1)\%&
\\ & =& \\sum
_i=1^n(a_ i - a_i-1)f_i \%&
\\ \end{align*}

ce qui est encore la somme relative à \sigma~; une récurrence évidente montre
que si \sigma' est plus fine que \sigma (autrement dit si on a ajouté un nombre
fini de points à \sigma), la somme relative à \sigma' est égale à celle relative à
\sigma. Maintenant si \sigma et \sigma' sont deux subdivisions adaptées à f, la
subdivision \sigma \cup \sigma' est encore adaptée à f et elle est plus fine que \sigma et
que \sigma'~; la somme relative à \sigma' est donc égale à celle relative à \sigma \cup \sigma'
et donc à celle relative à \sigma.

Les propriétés de l'intégrale des fonctions en escalier sont tout à fait
élémentaires à partir de la définition. On obtient

Théorème~9.2.2 (i) L'application
f\mapsto~\int ~
_a^bf est linéaire de l'espace vectoriel ~des applications
en escalier de [a,b] dans E dans l'espace vectoriel ~E. (ii) Soit u
: E \rightarrow~ F linéaire et f : [a,b] \rightarrow~ E en escalier~; alors u \cdot f :
[a,b] \rightarrow~ F est en escalier et \int ~
_a^bu \cdot f = u\left
(\int  _a^b~f\right
) (iii) Soit f : [a,b] \rightarrow~ E en escalier~; alors
\f\ : [a,b] \rightarrow~ \mathbb{R}~,
t\mapsto~\f(t)\
est en escalier et \\\int
 _a^bf\
\leq\int ~
_a^b\f\
(iv) Soit f : [a,b] \rightarrow~ E en escalier, alors
\\int ~
_a^bf\ \leq (b -
a)\f\\infty~. (v) Si c
\in]a,b[ et si f : [a,b] \rightarrow~ E est en escalier, alors
f__[a,c] et
f__[c,b] sont en escalier et
\int  _a^b~f
=\int  _a^c~f
+\int  _c^b~f.

Démonstration En prenant une subdivision adaptée à la fois à f et à g,
on a \int  _a^b~(\alpha~f + \beta~g)
= \\sum ~
_i=1^n(a_i - a_i-1)(\alpha~f_i +
\beta~g_i) = \alpha~\int  _a^b~f +
\beta~\int  _a^b~g. Si \sigma est adaptée à
f elle est aussi adaptée à u \cdot f et à
\f\ et on a
\int  _a^b~u \cdot f
= \\sum ~
_i=1^n(a_i - a_i-1)u(f_i) =
u(\int  _a^b~f) et
\\int ~
_a^bf\
=\\
\sum  _i=1^n(a_i~ -
a_i-1)f_i\
\leq\\sum ~
_i=1^n(a_i -
a_i-1)\f_i\
=\int ~
_a^b\f\.
Pour montrer (iv) on écrit

\begin{align*}
\\int ~
_a^bf& =&
\\\sum
_i=1^n(a_ i -
a_i-1)f_i\
\leq\sum _i=1^n(a_ i~ -
a_i-1)\f_i\\%&
\\ & \leq&
\f\\infty~\\sum
_i=1^n(a_ i - a_i-1) = (b -
a)\f\\infty~ \%&
\\ \end{align*}

En ce qui concerne (v), il suffit d'introduire une subdivision \sigma adaptée
à f, de lui ajouter éventuellement le point c pour obtenir encore une
subdivision adaptée à f~; on coupe alors la somme en deux au point c.

\subsection{9.2.2 Intégrale des fonctions réglées}

On suppose désormais que E est un espace vectoriel normé~complet

Théorème~9.2.3 Soit f : [a,b] \rightarrow~ \mathbb{R}~ une fonction réglée et
(\phi_n)_n\in\mathbb{N}~ une suite de fonctions en escalier telles
que lim~\f -
\phi_n\\infty~ = 0. Alors la suite
\left (\int ~
_a^b\phi_n\right )_n\in\mathbb{N}~
converge~; sa limite est indépendante du choix de la suite
(\phi_n)_n\in\mathbb{N}~~; on l'appelle l'intégrale de f sur le
segment [a,b] et on la note \int ~
_a^bf.

Démonstration Soit \epsilon > 0 et N \in \mathbb{N}~ tel que n ≥ N
\rigtharrow~\ f - \phi_n\\infty~
< \epsilon \over 2(b-a) . Pour p,q ≥ N on a
\\phi_p -
\phi_q\\infty~ \leq\
\phi_p - f\\infty~ +\
f - \phi_q\\infty~ < \epsilon
\over b-a et donc
\\int ~
_a^b\phi_p -\int ~
_a^b\phi_q\
=\\int ~
_a^b(\phi_p -
\phi_q)\ \leq (b -
a)\\phi_p -
\phi_q\\infty~ < \epsilon. La suite
(\int  _a^b\phi_n~) vérifie
donc le critère de Cauchy, donc elle converge (E étant complet). Soit
(\psi_n) une autre suite en escalier telle que
lim~\f -
\psi_n\\infty~ = 0. Comme
\\phi_n -
\psi_n\\infty~ \leq\
\phi_n - f\\infty~ +\
f - \psi_n\\infty~, on a
lim\\phi_n~ -
\psi_n\\infty~ = 0 et la majoration
\\int ~
_a^b\phi_n -\int ~
_a^b\psi_n\
=\\int ~
_a^b(\phi_n -
\psi_n)\ \leq (b -
a)\\phi_n -
\psi_n\\infty~ montre que les deux suites
(\int  _a^b\phi_n~) et
(\int  _a^b\psi_n~) (dont on
sait déjà qu'elles convergent) ont la même limite.

Remarque~9.2.1 La méthode ci dessus est la méthode classique de
prolongement d'une application uniformément continue
(f\mapsto~\int ~
_a^bf) d'un sous-ensemble (celui des fonctions en
escalier) à son adhérence (les fonctions réglées). Remarquons également
que si f est une fonction en escalier, on peut prendre pour tout n ,
\phi_n = f et que donc son intégrale en tant que fonction en
escalier est la même que son intégrale en tant que fonction réglée. En
particulier l'intégrale d'une constante m est m(b - a).

Théorème~9.2.4 (i) L'application
f\mapsto~\int ~
_a^bf est linéaire de l'espace vectoriel ~des applications
réglées de [a,b] dans E dans l'espace vectoriel ~E. (ii) Soit u : E
\rightarrow~ F linéaire continue et f : [a,b] \rightarrow~ E réglée~; alors u \cdot f :
[a,b] \rightarrow~ F est réglée et \int ~
_a^bu \cdot f = u\left
(\int  _a^b~f\right
) (iii) Soit f : [a,b] \rightarrow~ E réglée~; alors
\f\ : [a,b] \rightarrow~ \mathbb{R}~,
t\mapsto~\f(t)\
est réglée et \\int ~
_a^bf\
\leq\int ~
_a^b\f\
(iv) Si c \in]a,b[ et si f : [a,b] \rightarrow~ E est réglée, alors
f__[a,c] et
f__[c,b] sont réglées et
\int  _a^b~f
=\int  _a^c~f
+\int  _c^b~f.

Démonstration (i) Soit f et g de [a,b] dans E réglées, soit
(\phi_n) et (\psi_n) deux suites de fonctions en escalier
telles que lim~\f -
\phi_n\\infty~ = 0 et
lim~\g -
\psi_n\\infty~ = 0. Si \alpha~,\beta~ \in K, on a
\(\alpha~f + \beta~g) - (\alpha~\phi_n +
\beta~\psi_n)\\infty~
\leq\alpha~\f -
\phi_n\\infty~ +
\beta~\g -
\psi_n\\infty~ et donc
lim~\(\alpha~f + \beta~g) -
(\alpha~\phi_n + \beta~\psi_n)\\infty~ = 0. On en
déduit que \alpha~f + \beta~g est encore réglée et que

\begin{align*} \int ~
_a^b(\alpha~f + \beta~g)& =&
lim\\int ~
_a^b(\alpha~\phi_ n + \beta~\psi_n) \%&
\\ & =&
\alpha~lim\\int ~
_a^b\phi_ n +
\beta~lim\\int ~
_a^b\psi_ n\%& \\ &
=& \alpha~\int  _a^b~f +
\beta~\int  _a^b~g \%&
\\ \end{align*}

(ii) Soit u : E \rightarrow~ F linéaire continue et f : [a,b] \rightarrow~ E réglée, soit
(\phi_n) une suite de fonctions en escalier telle que
lim~\f -
\phi_n\\infty~ = 0. Alors u \cdot \phi_n est
en escalier (toute subdivision adaptée à \phi_n l'est encore à u \cdot
\phi_n) et \u \cdot f(t) - u \cdot
\phi_n(t)\ =\
u(f(t) - \phi_n(t))\
\leq\
u\\,\f(t)
- \phi_n(t)\ d'où
\u \cdot f - u \cdot
\phi_n\\infty~ \leq\
u\\,\f
- \phi_n\\infty~. Ceci montre que u \cdot f est
encore réglée et on a

\begin{align*} \int ~
_a^bu \cdot f& =&
lim\\int ~
_a^bu \cdot \phi_ n =\
limu(\int  _a^b\phi_
n)\%& \\ & =&
u(lim\\int ~
_a^b\phi_ n) = u(\int ~
_a^bf) \%& \\
\end{align*}

puisque u est continue et \int ~
_a^bu \cdot \phi_n = u(\int ~
_a^b\phi_n) (intégrale des fonctions en escalier).

(iii) La démonstration est similaire en remarquant que
\\phi_n\ est
encore en escalier et que
\f(t)\
-\
\phi_n(t)\\leq\
f(t) - \phi_n(t)\, soit
\\f\
-\
\phi_n\\\infty~
\leq\ f - \phi_n\\infty~.
On a donc \f\ réglée
et

\begin{align*} \int ~
_a^b\f&
=& lim\\int ~
_a^b\\phi_
n\ ≥\
lim\\int ~
_a^b\phi_ n\\%&
\\ & =&
\lim~\\int
 _a^b\phi_ n\
=\\int ~
_a^bf\ \%&
\\ \end{align*}

puisque
x\mapsto~\x\
est continue et \int ~
_a^b\\phi_n\
≥\\int ~
_a^b\phi_n\ (intégrale des
fonctions en escalier).

(iv) On remarque ici que
(\phi_n)__[a,c] et
(\phi_n)__[c,b] sont encore en
escalier et que
\f__[a,c] -
(\phi_n)__[a,c]\\infty~
\leq\ f - \phi_n\\infty~
et de même sur [c,b]. On a donc

\begin{align*} \int ~
_a^bf& =&
lim(\\int ~
_a^c\phi_ n +\int ~
_c^b\phi_ n) =\
lim\int  _a^c\phi_ n~
+ lim\\int ~
_c^b\phi_ n\%& \\ &
=& \int  _a^c~f
+\int  _c^b~f \%&
\\ \end{align*}

puisque l'existence de toutes les limites est garantie.

La propriété (iii) a un certain nombre de conséquences extrêmement
importantes

Théorème~9.2.5 (i) Soit f : [a,b] \rightarrow~ \mathbb{R}~ réglée positive~; alors
\int  _a^b~f ≥ 0. (ii) Soit f et g
des applications réglées de [a,b] dans \mathbb{R}~ telles que f \leq g. Alors
\int  _a^b~f
\leq\int  _a^b~g (iii) Soit f :
[a,b] \rightarrow~ E réglée~; alors
\\int ~
_a^bf\ \leq (b -
a)\f\\infty~.

Démonstration (i) On a d'après l'assertion (iii) du théorème précédent
\left \int ~
_a^bf\right 
\leq\int  _a^b~f
=\int  _a^b~f~; ceci exige
\int  _a^b~f ≥ 0.

(ii) La fonction g - f est réglée positive, donc
\int  _a^b~g
-\int  _a^b~f
=\int  _a^b~(g - f) ≥ 0

(iii) On a \\int ~
_a^bf\
\leq\int ~
_a^b\f\
\leq\int ~
_a^b\f\\infty~
= (b - a)\f\\infty~ puisque
\forall~~t,
\f(t)\
\leq\ f\\infty~.

En fait le résultat (i) peut être précisé de la manière suivante

Théorème~9.2.6 Soit f : [a,b] \rightarrow~ \mathbb{R}~ réglée positive~; on suppose qu'il
existe t_0 \in [a,b] tel que
f(t_0)\neq~0 et f continue au point
t_0. Alors \int  _a^b~f
> 0.

Démonstration Supposons par exemple
t_0\neq~b. On peut alors trouver \eta
> 0 tel que t_0 + \eta < b et tel que
\forall~t \in [t_0,t_0~ + \eta],
f(t) - f(t_0) < f(t_0)
\over 2 , soit encore f(t) >
f(t_0) \over 2 . Comme
\int  _a^t_0~f ≥ 0 et
\int  _t_0+\eta^b~f ≥ 0, on
a

\int  _a^b~f
≥\int ~
_t_0^t_0+\etaf ≥\\int
 _t_0^t_0+\eta f(t_0)
\over 2 = \etaf(t_0) \over 2
> 0

Corollaire~9.2.7 Soit f : [a,b] \rightarrow~ \mathbb{R}~ continue positive~; si
\int  _a^b~f = 0, alors f = 0.

Théorème~9.2.8 (première formule de la moyenne). Soit f,g : [a,b] \rightarrow~
\mathbb{R}~. On suppose que f est continue et que g est réglée positive. Alors, il
existe c \in [a,b] tel que \int ~
_a^bfg = f(c)\int ~
_a^bg.

Démonstration L'image de [a,b] par f est à la fois connexe et
compact dans \mathbb{R}~, c'est donc un segment de \mathbb{R}~. Soit f([a,b]) =
[m,M]. On a \forall~~t \in [a,b],m \leq f(t) \leq M et
donc, puisque g(t) ≥ 0, on a \forall~~t \in
[a,b],mg(t) \leq f(t)g(t) \leq Mg(t). En intégrant, on a alors
m\int  _a^b~g
\leq\int  _a^b~fg \leq
M\int  _a^b~g. Si
\int  _a^b~g = 0, on en déduit que
\int  _a^b~fg = 0 et n'importe
quel c \in [a,b] convient. Sinon, on a m \leq
\int  _a^b~fg \over
\int  _a^bg~ \leq M et donc

\exists~c \in [a,b], \\int
 _a^bfg \over \\int
 _a^bg = f(c)

ce que nous voulions démontrer.

\subsection{9.2.3 Convention de Chasles}

Définition~9.2.1 Soit I un intervalle de \mathbb{R}~~; on dit que f est réglée sur
I si sa restriction à tout segment inclus dans I est réglée.

Dans ce cas, si a et b sont dans I et a < b, on peut définir
\int  _a^b~f. On étendra la
définition en posant \int  _a^b~f
= 0 si a = b et \int  _a^b~f =
-\int  _b^a~f si a >
b. On a alors

Proposition~9.2.9 (relation de Chasles). Soit f : I \rightarrow~ E réglée. Alors

\forall~a,b,c \in I, \\int ~
_a^cf =\int  _a^b~f
+\int  _b^c~f

Démonstration Examiner toutes les configurations possibles de a,b,c.

Remarque~9.2.2 Le lecteur prendra garde à ne pas utiliser les diverses
majorations ou minorations rencontrées auparavant dans les cas où a
> b~; dans ce cas, toutes les inégalités précédentes sont
changées de sens.

\subsection{9.2.4 Sommes de Riemann}

Soit \sigma = (a_i)_0\leqi\leqn une subdivision de [a,b], \xi =
(\xi_i)_1\leqi\leqn une famille de points de [a,b] tels
que \forall~i \in [1,n], \xi_i~ \in
[a_i-1,a_i]. Si f est une application de [a,b]
dans E, on posera

S(f,\sigma,\xi) = \sum _i=1^n(a_
i - a_i-1)f(\xi_i)

Définition~9.2.2 On dira que S(f,\sigma,\xi) est une somme de Riemann associée
à la subdivision \sigma et à la famille de points \xi.

Théorème~9.2.10 Soit f : [a,b] \rightarrow~ E réglée~; alors les sommes de
Riemann de f tendent vers l'intégrale de f quand le pas de la
subdivision tend vers 0, plus précisément

\forall~~\epsilon >
0,\exists~\eta > 0,
\forall~~(\sigma,\xi),\quad \delta(\sigma) < \eta
\rigtharrow~\\int ~
_a^bf - S(f,\sigma,\xi)\ < \epsilon

Démonstration Nous allons tout d'abord démontrer ce résultat pour une
fonction \phi : [a,b] \rightarrow~ E en escalier. Soit
(x_k)_0\leqk\leqK une subdivision adaptée à \phi. Soit \sigma =
(a_i)_0\leqi\leqn une subdivision de [a,b], \xi =
(\xi_i)_1\leqi\leqn une famille de points de [a,b] tels
que \forall~i \in [1,n], \xi_i~ \in
[a_i-1,a_i]. On écrit alors

\begin{align*} \int ~
_a^b\phi - S(\phi,\sigma,\xi)& =& \\sum
_i=1^n\left
(\\int  ~
_a_i-1^a_i \phi - (a_i -
a_i-1)\phi(\xi_i)\right )\%&
\\ & =& \\sum
_i=1^n
\\int  ~
_a_i-1^a_i (\phi - \phi(\xi_i)) \%&
\\ \end{align*}

Notons H = \i \in
[1,n]∣\exists~k \in
[0,K], x_k \in
[a_i-1,a_i]\. On remarque tout
d'abord que chaque x_k ne peut appartenir qu'à au plus 2
intervalles [a_i-1,a_i] et que donc le cardinal de
H est plus petit que 2K + 2. Soit i \in [1,n]. Deux cas sont
possibles~; si i∉H, la fonction \phi est
constante sur [a_i-1,a_i] et donc
\int ~
_a_i-1^a_i(\phi - \phi(\xi_i)) = 0. Si
par contre, i \in H, on a

\\int ~
_a_i-1^a_i (\phi -
\phi(\xi_i))\ \leq\\int
 _a_i-1^a_i
2\\phi\\infty~ \leq
2\delta(\sigma)\\phi\\infty~

On en déduit que

\\int ~
_a^b\phi - S(\phi,\sigma,\xi)\ \leq
2\delta(\sigma)\\phi\
\infty~Card~H \leq 4(K +
1)\delta(\sigma)\\phi\\infty~

On voit donc que \delta(\sigma) < \epsilon \over
4(K+1)\\phi\\infty~
\rigtharrow~\\int ~
_a^b\phi - S(\phi,\sigma,\xi)\ < \epsilon.

Supposons maintenant que f est réglée, et soit \epsilon > 0. Il
existe une fonction \phi en escalier telle que \f
- \phi\\infty~ < \epsilon \over
4(b-a) . On a alors
\\int ~
_a^bf -\int ~
_a^b\phi\ \leq \epsilon \over
4 et

\begin{align*} \S(f,\sigma,\xi) -
S(\phi,\sigma,\xi)& \leq& \\sum
_i=1^n(a_ i -
a_i-1)\f(\xi_i) -
\phi(\xi_i)\\%&
\\ & \leq& (b -
a)\f - \phi\\infty~
< \epsilon \over 4 \%&
\\ \end{align*}

Mais pour la fonction en escalier \phi il existe \eta > 0 tel que
\forall~~(\sigma,\xi), \delta(\sigma) < \eta
\rigtharrow~\\int ~
_a^b\phi - S(\phi,\sigma,\xi)\ \leq \epsilon
\over 2 . Alors, \delta(\sigma) < \eta implique

\begin{align*}
\\int ~
_a^bf - S(f,\sigma,\xi)&& \%&
\\ & \leq&
\\int ~
_a^bf -\int ~
_a^b\phi\
+\\int ~
_a^b\phi - S(\phi,\sigma,\xi)\\%&
\\ & \text &
+\S(\phi,\sigma,\xi) - S(f,\sigma,\xi)\
\%& \\ & \leq& \epsilon \%&
\\ \end{align*}

ce qu'on voulait démontrer.

Remarque~9.2.3 On a vu ici une des techniques essentielles pour l'étude
des fonctions réglées (ou même continues)~: on commence par démontrer le
résultat cherché pour des fonctions en escalier et on en déduit le
résultat général par approximation.

Remarque~9.2.4 L'intérêt essentiel de ce résultat est non pas de
calculer des intégrales (il est bien rare que l'on y arrive par cette
méthode) ni même de calculer des valeurs approchées d'intégrales (la
convergence étant beaucoup trop lente), mais plutôt de trouver les
limites de certaines suites en les identifiant comme sommes de Riemann
d'une certaine fonction réglée. De ce point de vue, les subdivisions
courantes sont les subdivisions régulières \sigma_n = (a + i b-a
\over n )_0\leqi\leqn avec divers choix possibles de
\xi_i, \xi_i = a_i-1 ou \xi_i =
a_i ou plus rarement \xi_i =
a_i-1+a_i \over 2 .

\subsection{9.2.5 Sommes de Darboux}

Pour une fonction réglée f : [a,b] \rightarrow~ \mathbb{R}~ et \sigma =
(a_i)_0\leqi\leqn une subdivision de [a,b], d'autres
sommes se présentent naturellement~; puisque f est bornée, on peut poser
m_i = inf~
_t\in[a_i-1,a_i]f(t) et M_i
=\
sup_t\in[a_i-1,a_i]f(t). On introduit alors
les sommes de Darboux inférieures et supérieures d(f,\sigma)
= \\sum ~
_i=1^n(a_i - a_i-1)m_i et
D(f,\sigma) = \\sum ~
_i=1^n(a_i - a_i-1)M_i~;
remarquons que si f est continue, la fonction f atteint ses bornes et on
retombe sur de classiques sommes de Riemann. Dans le cas général, on a

Proposition~9.2.11 (i) d(f,\sigma) \leq\int ~
_a^bf \leq D(f,\sigma) (ii) les sommes de Darboux de f tendent
vers l'intégrale de f quand le pas de la subdivision tend vers 0.

Démonstration (i) On écrit \int ~
_a^bf - d(f,\sigma) =\
\sum  _i=1^n~\left
(\int ~
_a_i-1^a_if - (a_i -
a_i-1)m_i\right )
= \\sum~
_i=1^n\int ~
_a_i-1^a_i(f - m_i) ≥ 0 et de
même pour la somme de Darboux supérieure.

(ii) Soit \epsilon > 0~; soit \eta > 0 tel que
\forall~~(\sigma,\xi),\quad \delta(\sigma) < \eta
\rigtharrow~\left \int ~
_a^bf - S(f,\sigma,\xi)\right 
< \epsilon \over 2 , et \sigma =
(a_i)_0\leqi\leqn une subdivision de [a,b] de pas plus
petit que \eta~; pour chaque i \in [1,n], soit \xi_i \in
[a_i-1,a_i] tel que m_i \leq f(\xi_i)
< m_i + \epsilon \over 2(b-a) . En
multipliant par (a_i - a_i-1) et en sommant les
inégalités obtenues, on a d(f,\sigma) \leq S(f,\sigma,\xi) \leq d(f,\sigma) + \epsilon
\over 2 et donc \left
\int  _a^b~f -
d(f,\sigma)\right \leq\left
\int  _a^b~f -
S(f,\sigma,\xi)\right  + S(f,\sigma,\xi) -
d(f,\sigma) < \epsilon.

[
[
[
[

\end{document}

% \documentclass[]{article}
\usepackage[T1]{fontenc}
\usepackage{lmodern}
\usepackage{amssymb,amsmath}
\usepackage{ifxetex,ifluatex}
\usepackage{fixltx2e} % provides \textsubscript
% use upquote if available, for straight quotes in verbatim environments
\IfFileExists{upquote.sty}{\usepackage{upquote}}{}
\ifnum 0\ifxetex 1\fi\ifluatex 1\fi=0 % if pdftex
  \usepackage[utf8]{inputenc}
\else % if luatex or xelatex
  \ifxetex
    \usepackage{mathspec}
    \usepackage{xltxtra,xunicode}
  \else
    \usepackage{fontspec}
  \fi
  \defaultfontfeatures{Mapping=tex-text,Scale=MatchLowercase}
  \newcommand{\euro}{€}
\fi
% use microtype if available
\IfFileExists{microtype.sty}{\usepackage{microtype}}{}
\ifxetex
  \usepackage[setpagesize=false, % page size defined by xetex
              unicode=false, % unicode breaks when used with xetex
              xetex]{hyperref}
\else
  \usepackage[unicode=true]{hyperref}
\fi
\hypersetup{breaklinks=true,
            bookmarks=true,
            pdfauthor={},
            pdftitle={Primitives et integrales},
            colorlinks=true,
            citecolor=blue,
            urlcolor=blue,
            linkcolor=magenta,
            pdfborder={0 0 0}}
\urlstyle{same}  % don't use monospace font for urls
\setlength{\parindent}{0pt}
\setlength{\parskip}{6pt plus 2pt minus 1pt}
\setlength{\emergencystretch}{3em}  % prevent overfull lines
\setcounter{secnumdepth}{0}
 
/* start css.sty */
.cmr-5{font-size:50%;}
.cmr-7{font-size:70%;}
.cmmi-5{font-size:50%;font-style: italic;}
.cmmi-7{font-size:70%;font-style: italic;}
.cmmi-10{font-style: italic;}
.cmsy-5{font-size:50%;}
.cmsy-7{font-size:70%;}
.cmex-7{font-size:70%;}
.cmex-7x-x-71{font-size:49%;}
.msbm-7{font-size:70%;}
.cmtt-10{font-family: monospace;}
.cmti-10{ font-style: italic;}
.cmbx-10{ font-weight: bold;}
.cmr-17x-x-120{font-size:204%;}
.cmsl-10{font-style: oblique;}
.cmti-7x-x-71{font-size:49%; font-style: italic;}
.cmbxti-10{ font-weight: bold; font-style: italic;}
p.noindent { text-indent: 0em }
td p.noindent { text-indent: 0em; margin-top:0em; }
p.nopar { text-indent: 0em; }
p.indent{ text-indent: 1.5em }
@media print {div.crosslinks {visibility:hidden;}}
a img { border-top: 0; border-left: 0; border-right: 0; }
center { margin-top:1em; margin-bottom:1em; }
td center { margin-top:0em; margin-bottom:0em; }
.Canvas { position:relative; }
li p.indent { text-indent: 0em }
.enumerate1 {list-style-type:decimal;}
.enumerate2 {list-style-type:lower-alpha;}
.enumerate3 {list-style-type:lower-roman;}
.enumerate4 {list-style-type:upper-alpha;}
div.newtheorem { margin-bottom: 2em; margin-top: 2em;}
.obeylines-h,.obeylines-v {white-space: nowrap; }
div.obeylines-v p { margin-top:0; margin-bottom:0; }
.overline{ text-decoration:overline; }
.overline img{ border-top: 1px solid black; }
td.displaylines {text-align:center; white-space:nowrap;}
.centerline {text-align:center;}
.rightline {text-align:right;}
div.verbatim {font-family: monospace; white-space: nowrap; text-align:left; clear:both; }
.fbox {padding-left:3.0pt; padding-right:3.0pt; text-indent:0pt; border:solid black 0.4pt; }
div.fbox {display:table}
div.center div.fbox {text-align:center; clear:both; padding-left:3.0pt; padding-right:3.0pt; text-indent:0pt; border:solid black 0.4pt; }
div.minipage{width:100%;}
div.center, div.center div.center {text-align: center; margin-left:1em; margin-right:1em;}
div.center div {text-align: left;}
div.flushright, div.flushright div.flushright {text-align: right;}
div.flushright div {text-align: left;}
div.flushleft {text-align: left;}
.underline{ text-decoration:underline; }
.underline img{ border-bottom: 1px solid black; margin-bottom:1pt; }
.framebox-c, .framebox-l, .framebox-r { padding-left:3.0pt; padding-right:3.0pt; text-indent:0pt; border:solid black 0.4pt; }
.framebox-c {text-align:center;}
.framebox-l {text-align:left;}
.framebox-r {text-align:right;}
span.thank-mark{ vertical-align: super }
span.footnote-mark sup.textsuperscript, span.footnote-mark a sup.textsuperscript{ font-size:80%; }
div.tabular, div.center div.tabular {text-align: center; margin-top:0.5em; margin-bottom:0.5em; }
table.tabular td p{margin-top:0em;}
table.tabular {margin-left: auto; margin-right: auto;}
div.td00{ margin-left:0pt; margin-right:0pt; }
div.td01{ margin-left:0pt; margin-right:5pt; }
div.td10{ margin-left:5pt; margin-right:0pt; }
div.td11{ margin-left:5pt; margin-right:5pt; }
table[rules] {border-left:solid black 0.4pt; border-right:solid black 0.4pt; }
td.td00{ padding-left:0pt; padding-right:0pt; }
td.td01{ padding-left:0pt; padding-right:5pt; }
td.td10{ padding-left:5pt; padding-right:0pt; }
td.td11{ padding-left:5pt; padding-right:5pt; }
table[rules] {border-left:solid black 0.4pt; border-right:solid black 0.4pt; }
.hline hr, .cline hr{ height : 1px; margin:0px; }
.tabbing-right {text-align:right;}
span.TEX {letter-spacing: -0.125em; }
span.TEX span.E{ position:relative;top:0.5ex;left:-0.0417em;}
a span.TEX span.E {text-decoration: none; }
span.LATEX span.A{ position:relative; top:-0.5ex; left:-0.4em; font-size:85%;}
span.LATEX span.TEX{ position:relative; left: -0.4em; }
div.float img, div.float .caption {text-align:center;}
div.figure img, div.figure .caption {text-align:center;}
.marginpar {width:20%; float:right; text-align:left; margin-left:auto; margin-top:0.5em; font-size:85%; text-decoration:underline;}
.marginpar p{margin-top:0.4em; margin-bottom:0.4em;}
.equation td{text-align:center; vertical-align:middle; }
td.eq-no{ width:5%; }
table.equation { width:100%; } 
div.math-display, div.par-math-display{text-align:center;}
math .texttt { font-family: monospace; }
math .textit { font-style: italic; }
math .textsl { font-style: oblique; }
math .textsf { font-family: sans-serif; }
math .textbf { font-weight: bold; }
.partToc a, .partToc, .likepartToc a, .likepartToc {line-height: 200%; font-weight:bold; font-size:110%;}
.chapterToc a, .chapterToc, .likechapterToc a, .likechapterToc, .appendixToc a, .appendixToc {line-height: 200%; font-weight:bold;}
.index-item, .index-subitem, .index-subsubitem {display:block}
.caption td.id{font-weight: bold; white-space: nowrap; }
table.caption {text-align:center;}
h1.partHead{text-align: center}
p.bibitem { text-indent: -2em; margin-left: 2em; margin-top:0.6em; margin-bottom:0.6em; }
p.bibitem-p { text-indent: 0em; margin-left: 2em; margin-top:0.6em; margin-bottom:0.6em; }
.subsectionHead, .likesubsectionHead { margin-top:2em; font-weight: bold;}
.sectionHead, .likesectionHead { font-weight: bold;}
.quote {margin-bottom:0.25em; margin-top:0.25em; margin-left:1em; margin-right:1em; text-align:justify;}
.verse{white-space:nowrap; margin-left:2em}
div.maketitle {text-align:center;}
h2.titleHead{text-align:center;}
div.maketitle{ margin-bottom: 2em; }
div.author, div.date {text-align:center;}
div.thanks{text-align:left; margin-left:10%; font-size:85%; font-style:italic; }
div.author{white-space: nowrap;}
.quotation {margin-bottom:0.25em; margin-top:0.25em; margin-left:1em; }
h1.partHead{text-align: center}
.sectionToc, .likesectionToc {margin-left:2em;}
.subsectionToc, .likesubsectionToc {margin-left:4em;}
.sectionToc, .likesectionToc {margin-left:6em;}
.frenchb-nbsp{font-size:75%;}
.frenchb-thinspace{font-size:75%;}
.figure img.graphics {margin-left:10%;}
/* end css.sty */

\title{Primitives et integrales}
\author{}
\date{}

\begin{document}
\maketitle

\textbf{Warning: 
requires JavaScript to process the mathematics on this page.\\ If your
browser supports JavaScript, be sure it is enabled.}

\begin{center}\rule{3in}{0.4pt}\end{center}

[
[
[]
[

\section{9.3 Primitives et intégrales}

\subsection{9.3.1 Continuité et dérivabilité par rapport à une borne}

Théorème~9.3.1 Soit I un intervalle de \mathbb{R}~, f : I \rightarrow~ E réglée, a \in I. Pour
t \in I, on pose F(t) =\int  _a^t~f.
Alors l'application F est continue sur I~; elle est dérivable en tout
point t_0 où f est continue et on a alors F'(t_0) =
f(t_0).

Démonstration Soit t_0 \in I. Supposons tout d'abord que
t_0 n'est pas une extrémité de I et soit \eta > 0 tel
que [t_0 - \eta,t_0 + \eta] \subset~ I. Alors f est réglée sur
[t_0 - \eta,t_0 + \eta] donc bornée par une constante M
≥ 0. Pour t \in [t_0 - \eta,t_0 + \eta], on a alors
\F(t) -
F(t_0)\
=\\int ~
_t_0^tf\ \leq Mt
- t_0 ce qui montre que F est continue au point
t_0. On montre le résultat d'une manière similaire si
t_0 est une extrémité de I en introduisant selon le cas
[t_0,t_0 + \eta] ou [t_0 -
\eta,t_0].

Supposons maintenant que f est continue au point t_0~; soit \epsilon
> 0 et soit \eta > 0 tel que t -
t_0 < \eta \rigtharrow~\ f(t) -
f(t_0)\ \leq \epsilon. Pour t -
t_0 < \eta, on a

\begin{align*} \F(t) -
F(t_0) - (t -
t_0)f(t_0)\
=\\int ~
_t_0^tf - (t - t_
0)f(t_0)&&\%&
\\ & =&
\\int ~
_t_0^t(f - f(t_
0))\ \leq sgn~(t -
t_0)\int ~
_t_0^t\f - f(t_
0)\\%& \\ &
\leq& \epsilont - t_0 \%&
\\ \end{align*}

ce qui peut encore s'écrire \
F(t)-F(t_0) \over t-t_0 -
f(t_0)\ \leq \epsilon. Ceci montre que F est
dérivable au point t_0 et que F'(t_0) =
f(t_0).

Remarque~9.3.1 De la même fa\ccon, on montre que la
continuité à gauche de f implique la dérivabilité à gauche de F~; le
même résultat est évidemment encore valable à droite.

\subsection{9.3.2 Primitives}

Définition~9.3.1 Soit f : I \rightarrow~ E une application~; on dit que F : I \rightarrow~ E
est une primitive de f si F est dérivable et F' = f.

En remarquant que F' = G' \Leftrightarrow F - G est
constante sur l'intervalle I, on obtient immédiatement

Proposition~9.3.2 Si f : I \rightarrow~ E admet une primitive F, elle en admet une
infinité qui sont exactement les t\mapsto~F(t) + k
où k \in E.

Remarque~9.3.2 On a vu que si F' admet une limite en un point
t_0, alors nécessairement F' était continue au point
t_0~; ceci permet d'exhiber facilement une fonction qui n'admet
pas de primitive (toute fonction qui admet une limite en un point sans
que ce soit la valeur de cette fonction en ce point)~; ceci montre
d'autre part qu'une fonction réglée qui admet une primitive est
nécessairement continue, puisqu'elle doit admettre en tout point une
limite à gauche et une limite à droite, qui ne peuvent être que la
valeur de la fonction en ce point (étudier séparément ce qui se passe à
gauche et à droite du point).

Théorème~9.3.3 Soit f : I \rightarrow~ E une fonction continue~; alors f admet des
primitives sur I. Si F est une telle primitive, on a
\forall~a,b \in I, \\int ~
_a^bf = F(b) - F(a) = \left
[F(t)\right ]_a^b.

Démonstration Soit \alpha~ \in I et posons G(t) =\int ~
_\alpha~^tf~; puisque f est continue, G est dérivable et G' = f.
Donc G est une primitive de f. Si F est une autre primitive de f, on a F
= G + k et donc

\int  _a^b~f
=\int  _\alpha~^b~f
-\int  _\alpha~^a~f = G(b) - G(a) = F(b)
- F(a)

Remarque~9.3.3 Ce dernier résultat est un des moyens les plus simples et
les plus généraux de calcul d'intégrales~; il ramène le calcul d'une
intégrale à la recherche d'une primitive de la fonction f.

\subsection{9.3.3 Changement de variable, intégration par parties}

En ce qui concerne le changement de variable, on est confronté à un
choix~: soit autoriser des fonctions très générales (des fonctions
réglées) et se limiter à des changements de variables très particuliers
(mais néanmoins fort utiles)~; soit restreindre le champ d'application
aux fonctions continues et autoriser des changements de variables plus
généraux (par exemple de classe \mathcal{C}^1).

Le premier théorème se montre trivialement pour les fonctions en
escalier et ensuite par un simple passage à la limite pour les fonctions
réglées sur un segment.

Théorème~9.3.4 Soit f : [a,b] \rightarrow~ E une fonction réglée. (i) Soit T \in
\mathbb{R}~ et g : [a - T,b - T] \rightarrow~ E, t\mapsto~f(t + T).
Alors g est réglée et \int ~
_a-T^b-Tg =\int ~
_a^bf. (ii) Soit \lambda~ \in \mathbb{R}~^∗ et soit g :
t\mapsto~f(\lambda~t) définie sur [ a
\over \lambda~ , b \over \lambda~ ] si \lambda~
> 0 et sur [ b \over \lambda~ , a
\over \lambda~ ] si \lambda~ < 0. Alors g est réglée et
(avec la convention de Chasles) \int ~
_a\diagup\lambda~^b\diagup\lambda~g = 1 \over \lambda~
\int  _a^b~f.

Théorème~9.3.5 (changement de variables). Soit f : I \rightarrow~ E continue et
soit \phi : J \rightarrow~ I de classe \mathcal{C}^1 (où I et J sont deux intervalles
de \mathbb{R}~). Alors,

\forall~a,b \in J, \\int ~
_a^bf \cdot \phi \phi' =\int ~
_\phi(a)^\phi(b)f

Démonstration Soit F une primitive de f sur I, alors (F \cdot \phi)' = f \cdot \phi \phi'
et donc F \cdot \phi est une primitive de f \cdot \phi \phi' sur J~; on a donc
\int  _a^b~f \cdot \phi \phi' = F \cdot \phi(b) - F
\cdot \phi(a) =\int  _\phi(a)^\phi(b)~f.

Remarque~9.3.4 Notation On introduira la notation différentielle
\int  _a^b~f
=\int  _a^b~f(t) dt où t est une
variable muette. De cette manière, faire le changement de variables t =
\phi(u) dans l'intégrale \int ~
_\phi(a)^\phi(b)f(t) dt c'est faire varier u de a à b (pour que
t varie de \phi(a) à \phi(b)) et remplacer f(t) par (f \cdot \phi)(u) et dt par \phi'(u)
du si bien que la formule ci dessus s'écrit \\int
 _a^b(f \cdot \phi)(u)\phi'(u) du =\int ~
_\phi(a)^\phi(b)f(t) dt. De même, faire le changement de
variable inverse t = \phi(u) dans l'intégrale \\int
 _a^bf \cdot \phi(u)\phi'(u) du c'est faire varier t de \phi(a) à
\phi(b) (puisque u varie de a à b), remplacer f(\phi(u)) par f(t) et \phi'(u) du
par dt.

Exemple~9.3.1 Les deux sens du théorème de changement de variables sont
utiles comme nous allons le voir sur deux exemples~:
\int ~
_0^xtsin (t^2~) dt =
1 \over 2 \int ~
_0^x^2  sin~ (u) du
= 1-cos x^2~ \over
2 en posant u = t^2~; de même \\int
 _0^1\sqrt1 - x^2 dx
=\int ~
_0^\pi~\diagup2\sqrt1 -\
sin ^2  tcos~ t dt
=\int ~
_0^\pi~\diagup2 cos ^2~t dt
=\int  _0^\pi~\diagup2~
1+cos (2t) \over 2~ dt = \pi~
\over 4 en posant x = cos~ t.

Théorème~9.3.6 (intégration par parties). Soit f,g : [a,b] \rightarrow~ \mathbb{C} de
classe \mathcal{C}^1~; alors

\int  _a^b~f(t)g'(t) dt =
\left [f(t)g(t)\right ]_
a^b -\int  _a^b~f'(t)g(t)
dt

Démonstration Il suffit de remarquer que fg est une primitive de la
fonction continue f'g + fg' et que en conséquence
\int  _a^b~(fg' + f'g) =
\left [f(t)g(t)\right
]_a^b.

Remarque~9.3.5 Le résultat s'étend à n'importe quelle application
bilinéaire continue \phi (produit scalaire, produit vectoriel, produit
matriciel) et on obtient la formule

\int  _a^b~\phi(f(t),g'(t)) dt =
\left [\phi(f(t),g(t))\right ]_
a^b -\int ~
_a^b\phi(f'(t),g(t)) dt

avec la même démonstration.

Corollaire~9.3.7 (formule de Taylor avec reste intégral). Soit f : I \rightarrow~ E
de classe C^n+1 et a \in I. Alors \forall~~b
\in I,

f(b) = f(a) + \sum _k=1^n~
f^(k)(a) \over k! (b - a)^k +
\\int  ~
_a^b (b - t)^n \over n!
f^(n+1)(t) dt

Démonstration Par récurrence sur n. Si n = 0, il s'agit simplement de la
formule f(b) = f(a) +\int ~
_a^bf'(t) dt pour f de classe \mathcal{C}^1. De plus une
intégration par parties (en intégrant  (b-t)^n-1
\over (n-1)! et en dérivant f^(n)(t)) donne

\begin{align*} \int ~
_a^b (b - t)^n-1 \over (n -
1)! f^(n)(t) dt&& \%& \\
& =& \left [- (b - t)^n
\over n! f^(n)(t)\right
]_ a^b +\int ~
_a^b (b - t)^n \over n!
f^(n+1)(t) dt\%& \\ & =&
f^(n)(a) \over n! (b - a)^n
+\int  _a^b~ (b -
t)^n \over n! f^(n+1)(t) dt \%&
\\ \end{align*}

ce qui permet de passer de n - 1 à n.

\subsection{9.3.4 Deuxième formule de la moyenne}

Théorème~9.3.8 (deuxième formule de la moyenne). Soit f,g : [a,b] \rightarrow~
\mathbb{R}~. On suppose que f est de classe \mathcal{C}^1, positive et
décroissante et que g est continue. Alors, il existe c \in [a,b] tel
que \int  _a^b~fg =
f(a)\int  _a^c~g.

Démonstration Posons G(x) =\int ~
_a^xg. On sait que G est continue. L'image de [a,b]
par G est à la fois connexe et compact dans \mathbb{R}~, c'est donc un segment de
\mathbb{R}~. Soit G([a,b]) = [m,M]. On a \forall~~t \in
[a,b],m \leq G(t) \leq M. Supposons démontré que mf(a)
\leq\int  _a^b~fg \leq Mf(a). Alors soit
f(a) = 0, auquel cas \int  _a^b~fg
= 0 et n'importe quel c convient, soit f(a)\neq~0
et donc  1 \over f(a) \int ~
_a^bfg \in [m,M] et donc \exists~c \in
[a,b], 1 \over f(a) \\int
 _a^bfg = G(c), ce que l'on voulait démontrer. Nous
allons donc montrer que mf(a) \leq\int ~
_a^bfg \leq Mf(a).

On peut faire une intégration par parties et on a (en tenant compte de
G(a) = 0)

\begin{align*} \int ~
_a^bfg& =& \int ~
_a^bfG' = \left
[f(t)G(t)\right ]_ a^b
-\int  _a^b~f'(t)G(t) dt\%&
\\ & =& f(b)G(b)
+\int  _a^b~(-f'(t))G(t) dt \%&
\\ \end{align*}

Comme - f' est positive, on peut appliquer la première formule de la
moyenne et il existe d \in [a,b] tel que \\int
 _a^b(-f'(t))G(t) dt = G(d)\\int
 _a^b(-f'(t)) dt = (f(a) - f(b))G(d). On a donc
\int  _a^b~fg = f(b)G(b) + (f(a) -
f(b))G(d). Mais m \leq G(b) \leq M, m \leq G(d) \leq M, f(b) ≥ 0 et f(a) - f(b) ≥ 0.
On a donc

mf(a) = f(b)m + (f(a) - f(b))m \leq\int ~
_a^bfg \leq f(b)M + (f(a) - f(b))M = Mf(a)

ce qui achève la démonstration.

[
[
[
[

\end{document}

% \documentclass[]{article}
\usepackage[T1]{fontenc}
\usepackage{lmodern}
\usepackage{amssymb,amsmath}
\usepackage{ifxetex,ifluatex}
\usepackage{fixltx2e} % provides \textsubscript
% use upquote if available, for straight quotes in verbatim environments
\IfFileExists{upquote.sty}{\usepackage{upquote}}{}
\ifnum 0\ifxetex 1\fi\ifluatex 1\fi=0 % if pdftex
  \usepackage[utf8]{inputenc}
\else % if luatex or xelatex
  \ifxetex
    \usepackage{mathspec}
    \usepackage{xltxtra,xunicode}
  \else
    \usepackage{fontspec}
  \fi
  \defaultfontfeatures{Mapping=tex-text,Scale=MatchLowercase}
  \newcommand{\euro}{€}
\fi
% use microtype if available
\IfFileExists{microtype.sty}{\usepackage{microtype}}{}
\ifxetex
  \usepackage[setpagesize=false, % page size defined by xetex
              unicode=false, % unicode breaks when used with xetex
              xetex]{hyperref}
\else
  \usepackage[unicode=true]{hyperref}
\fi
\hypersetup{breaklinks=true,
            bookmarks=true,
            pdfauthor={},
            pdftitle={Recherches de primitives},
            colorlinks=true,
            citecolor=blue,
            urlcolor=blue,
            linkcolor=magenta,
            pdfborder={0 0 0}}
\urlstyle{same}  % don't use monospace font for urls
\setlength{\parindent}{0pt}
\setlength{\parskip}{6pt plus 2pt minus 1pt}
\setlength{\emergencystretch}{3em}  % prevent overfull lines
\setcounter{secnumdepth}{0}
 
/* start css.sty */
.cmr-5{font-size:50%;}
.cmr-7{font-size:70%;}
.cmmi-5{font-size:50%;font-style: italic;}
.cmmi-7{font-size:70%;font-style: italic;}
.cmmi-10{font-style: italic;}
.cmsy-5{font-size:50%;}
.cmsy-7{font-size:70%;}
.cmex-7{font-size:70%;}
.cmex-7x-x-71{font-size:49%;}
.msbm-7{font-size:70%;}
.cmtt-10{font-family: monospace;}
.cmti-10{ font-style: italic;}
.cmbx-10{ font-weight: bold;}
.cmr-17x-x-120{font-size:204%;}
.cmsl-10{font-style: oblique;}
.cmti-7x-x-71{font-size:49%; font-style: italic;}
.cmbxti-10{ font-weight: bold; font-style: italic;}
p.noindent { text-indent: 0em }
td p.noindent { text-indent: 0em; margin-top:0em; }
p.nopar { text-indent: 0em; }
p.indent{ text-indent: 1.5em }
@media print {div.crosslinks {visibility:hidden;}}
a img { border-top: 0; border-left: 0; border-right: 0; }
center { margin-top:1em; margin-bottom:1em; }
td center { margin-top:0em; margin-bottom:0em; }
.Canvas { position:relative; }
li p.indent { text-indent: 0em }
.enumerate1 {list-style-type:decimal;}
.enumerate2 {list-style-type:lower-alpha;}
.enumerate3 {list-style-type:lower-roman;}
.enumerate4 {list-style-type:upper-alpha;}
div.newtheorem { margin-bottom: 2em; margin-top: 2em;}
.obeylines-h,.obeylines-v {white-space: nowrap; }
div.obeylines-v p { margin-top:0; margin-bottom:0; }
.overline{ text-decoration:overline; }
.overline img{ border-top: 1px solid black; }
td.displaylines {text-align:center; white-space:nowrap;}
.centerline {text-align:center;}
.rightline {text-align:right;}
div.verbatim {font-family: monospace; white-space: nowrap; text-align:left; clear:both; }
.fbox {padding-left:3.0pt; padding-right:3.0pt; text-indent:0pt; border:solid black 0.4pt; }
div.fbox {display:table}
div.center div.fbox {text-align:center; clear:both; padding-left:3.0pt; padding-right:3.0pt; text-indent:0pt; border:solid black 0.4pt; }
div.minipage{width:100%;}
div.center, div.center div.center {text-align: center; margin-left:1em; margin-right:1em;}
div.center div {text-align: left;}
div.flushright, div.flushright div.flushright {text-align: right;}
div.flushright div {text-align: left;}
div.flushleft {text-align: left;}
.underline{ text-decoration:underline; }
.underline img{ border-bottom: 1px solid black; margin-bottom:1pt; }
.framebox-c, .framebox-l, .framebox-r { padding-left:3.0pt; padding-right:3.0pt; text-indent:0pt; border:solid black 0.4pt; }
.framebox-c {text-align:center;}
.framebox-l {text-align:left;}
.framebox-r {text-align:right;}
span.thank-mark{ vertical-align: super }
span.footnote-mark sup.textsuperscript, span.footnote-mark a sup.textsuperscript{ font-size:80%; }
div.tabular, div.center div.tabular {text-align: center; margin-top:0.5em; margin-bottom:0.5em; }
table.tabular td p{margin-top:0em;}
table.tabular {margin-left: auto; margin-right: auto;}
div.td00{ margin-left:0pt; margin-right:0pt; }
div.td01{ margin-left:0pt; margin-right:5pt; }
div.td10{ margin-left:5pt; margin-right:0pt; }
div.td11{ margin-left:5pt; margin-right:5pt; }
table[rules] {border-left:solid black 0.4pt; border-right:solid black 0.4pt; }
td.td00{ padding-left:0pt; padding-right:0pt; }
td.td01{ padding-left:0pt; padding-right:5pt; }
td.td10{ padding-left:5pt; padding-right:0pt; }
td.td11{ padding-left:5pt; padding-right:5pt; }
table[rules] {border-left:solid black 0.4pt; border-right:solid black 0.4pt; }
.hline hr, .cline hr{ height : 1px; margin:0px; }
.tabbing-right {text-align:right;}
span.TEX {letter-spacing: -0.125em; }
span.TEX span.E{ position:relative;top:0.5ex;left:-0.0417em;}
a span.TEX span.E {text-decoration: none; }
span.LATEX span.A{ position:relative; top:-0.5ex; left:-0.4em; font-size:85%;}
span.LATEX span.TEX{ position:relative; left: -0.4em; }
div.float img, div.float .caption {text-align:center;}
div.figure img, div.figure .caption {text-align:center;}
.marginpar {width:20%; float:right; text-align:left; margin-left:auto; margin-top:0.5em; font-size:85%; text-decoration:underline;}
.marginpar p{margin-top:0.4em; margin-bottom:0.4em;}
.equation td{text-align:center; vertical-align:middle; }
td.eq-no{ width:5%; }
table.equation { width:100%; } 
div.math-display, div.par-math-display{text-align:center;}
math .texttt { font-family: monospace; }
math .textit { font-style: italic; }
math .textsl { font-style: oblique; }
math .textsf { font-family: sans-serif; }
math .textbf { font-weight: bold; }
.partToc a, .partToc, .likepartToc a, .likepartToc {line-height: 200%; font-weight:bold; font-size:110%;}
.chapterToc a, .chapterToc, .likechapterToc a, .likechapterToc, .appendixToc a, .appendixToc {line-height: 200%; font-weight:bold;}
.index-item, .index-subitem, .index-subsubitem {display:block}
.caption td.id{font-weight: bold; white-space: nowrap; }
table.caption {text-align:center;}
h1.partHead{text-align: center}
p.bibitem { text-indent: -2em; margin-left: 2em; margin-top:0.6em; margin-bottom:0.6em; }
p.bibitem-p { text-indent: 0em; margin-left: 2em; margin-top:0.6em; margin-bottom:0.6em; }
.paragraphHead, .likeparagraphHead { margin-top:2em; font-weight: bold;}
.subparagraphHead, .likesubparagraphHead { font-weight: bold;}
.quote {margin-bottom:0.25em; margin-top:0.25em; margin-left:1em; margin-right:1em; text-align:justify;}
.verse{white-space:nowrap; margin-left:2em}
div.maketitle {text-align:center;}
h2.titleHead{text-align:center;}
div.maketitle{ margin-bottom: 2em; }
div.author, div.date {text-align:center;}
div.thanks{text-align:left; margin-left:10%; font-size:85%; font-style:italic; }
div.author{white-space: nowrap;}
.quotation {margin-bottom:0.25em; margin-top:0.25em; margin-left:1em; }
h1.partHead{text-align: center}
.sectionToc, .likesectionToc {margin-left:2em;}
.subsectionToc, .likesubsectionToc {margin-left:4em;}
.subsubsectionToc, .likesubsubsectionToc {margin-left:6em;}
.frenchb-nbsp{font-size:75%;}
.frenchb-thinspace{font-size:75%;}
.figure img.graphics {margin-left:10%;}
/* end css.sty */

\title{Recherches de primitives}
\author{}
\date{}

\begin{document}
\maketitle

\textbf{Warning: 
requires JavaScript to process the mathematics on this page.\\ If your
browser supports JavaScript, be sure it is enabled.}

\begin{center}\rule{3in}{0.4pt}\end{center}

[
[
[]
[

\subsubsection{9.4 Recherches de primitives}

\paragraph{9.4.1 Position du problème}

Soit f une fonction de \mathbb{R}~ vers \mathbb{R}~ ou \mathbb{C}. On cherche à déterminer des
intervalles (maximaux) I sur lesquels f est continue et sur un tel
intervalle, une primitive F de f. La notation F(t)
=\int ~ f(t) dt + k, t \in I signifiera~: f est
continue sur I et F est une primitive de f sur I

Remarque~9.4.1 On prendra garde que dans cette notation, et
contrairement à la notation différentielle des intégrales, la variable t
n'est pas muette. C'est bien le même t qui figure dans F(t) et dans
\int ~ f(t) dt

\paragraph{9.4.2 Techniques usuelles}

Si F est une primitive de f sur I et si G est une primitive de g sur I,
alors \alpha~F + \beta~G est une primitive de \alpha~f + \beta~g sur I ce qu'on écrira

\int ~ (\alpha~f(t) + \beta~g(t)) dt =
\alpha~\int  f(t) dt + \beta~\\int ~
g(t) dt, t \in I

Sur le même modèle on écrira le théorème de changement de variables avec
\phi : I \rightarrow~ J de classe \mathcal{C}^1

\int ~ f(\phi(t))\phi'(t) dt =\\int
 f(u) du, u = \phi(t), t \in I

et le théorème d'intégrations par parties pour deux fonctions f et g de
classe \mathcal{C}^1

\int ~ f(t)g'(t) dt = f(t)g(t)
-\int ~ f'(t)g(t) dt, t \in I

théorèmes dont la démonstration est évidente.

\paragraph{9.4.3 Primitives usuelles}

On posera I_n =] - \pi~ \over 2 + n\pi~, \pi~
\over 2 + n\pi~[ et J_n =]n\pi~,(n + 1)\pi~[ pour
n \in \mathbb{N}~.

\array \int ~
cos t dt =\ sin~ t +
k, t \in \mathbb{R}~; &\int  \sin~ t
dt = -cos~ t + k, t \in \mathbb{R}~ \cr
\int ~  dt \over
cos ^2t~ =\
\mathrmtg t + k, t \in I_n;
&\int ~  dt \over
sin ^2t~ =
-\mathrmcotg~ t + k, t \in
J_n \cr \int ~  dt
\over cos t~
= log~ \left
\mathrmtg~ ( t
\over 2 + \pi~ \over 4
)\right  + k, t \in
I_n;&\int ~  dt \over
sin t =\ log~
\left
\mathrmtg~  t
\over 2 \right , t \in
J_n \cr \int ~
\mathrmtg~ t dt =
-log~ \left
cos~ t\right
 + k, t \in I_n; &\int ~
\mathrmcotg~ t dt
= log~ \left
sin~ t\right
, t \in J_n \cr
\int ~
\mathrmch~ t dt
= \mathrmsh~ t + k, t \in \mathbb{R}~;
&\int ~
\mathrmsh~ t dt
= \mathrmch~ t + k, t \in \mathbb{R}~
\cr \int ~  dt
\over
\mathrmch ^2t~
= \mathrmth~ t + k, t \in \mathbb{R}~;
&\int ~  dt \over
\mathrmsh ^2t~
= -coth~ t + k, t \in]
-\infty~,0[\text ou t \in]0,+\infty~[ \cr
\int ~  dt \over
\mathrmch t~ =
2\mathrmarctg~
e^t + k, t \in \mathbb{R}~; &\int~  dt
\over
\mathrmsh t~
= log~ \left
\mathrmth~  t
\over 2 \right , t \in]
-\infty~,0[\text ou t \in]0,+\infty~[ \cr
\int ~
\mathrmth~ t dt
= log~
\mathrmch~ t + k, t \in \mathbb{R}~;
&\int  \coth~ t dt
= log~ \left
\mathrmsh~
t\right , t \in] -\infty~,0[\text
ou t \in]0,+\infty~[ \cr \int ~
t^\alpha~ dt = t^\alpha~+1 \over \alpha~+1 + k,
(\alpha~\neq~ - 1) &\int ~  dt
\over t = log~
t + k, t \in] -\infty~,0[\text ou t
\in]0,+\infty~[ 

\begin{align*} \int ~  dt
\over t^2 + a^2 & =& 1
\over a
\mathrmarctg~  t
\over a , t \in \mathbb{R}~ \%& \\
\int   dt \over a^2~ -
t^2 & =& 1 \over 2a
log~ \left  t + a
\over t - a \right  = 1
\over a arg~
\mathrmth~  t
\over a ,t \in]
-a,a[\text
pour la dernière expression  \%& \\
\int ~  dt \over
\sqrta^2  - t^2 & =&
arcsin~  t \over
a + k,t \in]
-a,a[ \%&
\\ \int ~  dt
\over \sqrtt^2  +
a^2 & =& arg~
\mathrmsh~  t
\over a + k
= log (t + \sqrtt~^2
 + a^2) + k', t \in \mathbb{R}~ \%& \\
\int ~  dt \over
\sqrtt^2  - a^2 & =&
log~ \left t +
\sqrtt^2  -
a^2\right  + k =
\left \ \cases
arg~
\mathrmch~  t
\over a + k&si t
\in]a,+\infty~[ \cr
-arg~
\mathrmch~ 
t \over a +
k&si t \in] -\infty~,-a[ \cr 
\right .\%&\\
\end{align*}

\paragraph{9.4.4 Fractions rationnelles}

On rappelle le résultat suivant

Théorème~9.4.1 Soit R(X) = A(X) \over B(X) une
fraction rationnelle à coefficients complexes, B(X) =
b\∏ ~
_i=1^k(X - a_i)^m_i la
décomposition du dénominateur en facteurs du premier degré. Alors R(X)
s'écrit de manière unique sous la forme

R(X) = E(X) + \\sum
_i=1^k\left ( \alpha_i,1
\over X - a_i +
\ldots + \alpha_i,m_i~
\over (X - a_i)^m_i
\right )

Démonstration E(X) est évidemment le quotient de la division euclidienne
de A(X) par B(X).

On montre que si A(X),B_1(X),B_2(X) sont trois
polynômes tels que B_1(X) et B_2(X) sont premiers
entre eux, alors il existe des polynômes U(X) et V (X) tels que

 A(X) \over B_1(X)B_2(X) = U(X)
\over B_1(X) + V (X) \over
B_2(X)

en effet puisque B_1 et B_2 sont premiers entre eux,
on a \mathbb{C}[X] = B_1(X)\mathbb{C}[X] + B_2(X)\mathbb{C}[X], donc
A(X) peut s'écrire sous la forme A(X) = U(X)B_2(X) + V
(X)B_1(X) et en divisant par B_1(X)B_2(X) on
obtient la décomposition souhaitée. De plus, si un couple (U,V )
convient, il est clair que tout couple (U - B_1Q,V +
B_2Q) convient. En rempla\ccant U par le
reste de sa division euclidienne par B_1, on peut donc supposer
que deg~ U <\
deg B_1~; on voit alors immédiatement que si
deg~ A <\
deg B_1B_2, on a aussi deg~
V < deg B_2~ (l'ensemble des
fractions rationnelles dont le degré du numérateur est strictement
inférieur au degré du numérateur est une sous algèbre de \mathbb{C}(X)). Une
récurrence évidente permet donc d'écrire

 A(X) \over B(X) = E(X) + \\sum
_i=1^k A_i(X) \over (X -
a_i)^m_i

avec deg A_i~ <
m_i. On écrit alors la formule de Taylor pour le polynôme
A_i au point a_i, soit A_i(X) =
\alpha_i,m_i + \alpha_i,m_i-1(X -
a_i) +
\\ldots~ +
\alpha_i,1(X - a_i)^m_i-1 (car
deg A_i \leq m_i~ - 1) d'où la
décomposition souhaitée. L'unicité de la décomposition découle
immédiatement du lemme suivant

Lemme~9.4.2 Le polynôme \alpha_i,1X +
\\ldots~ +
\alpha_i,m_iX^m_i est l'unique polynôme
P(X) sans terme constant tel que  A(X) \over B(X) -
P( 1 \over X-a_i ) n'admette pas
a_i comme pôle.

Démonstration Il est clair que ce polynôme convient. Si P_1 et
P_2 sont deux tels polynômes, alors (P_1 -
P_2)( 1 \over X-a_i ) =
\left ( A(X) \over B(X) -
P_2( 1 \over X-a_i
)\right ) -\left ( A(X)
\over B(X) - P_1( 1 \over
X-a_i )\right ) est la différence de deux
fractions rationnelles qui n'admettent pas le pôle a_i donc
c'est une fraction rationnelle qui n'admet pas le pôle a_i.
Ceci n'est possible que si P_1 - P_2 est constant,
mais comme P_1 et P_2 sont sans terme constant, on a
P_1 = P_2.

Méthode de calcul E(X) est le quotient de la division euclidienne de
A(X) par B(X). En ce qui concerne les parties polaires  \alpha_i,1
\over X-a_i +
\\ldots~ +
\alpha_i,m_i \over
(X-a_i)^m_i on peut procéder de la
manière suivante~:

\begin{itemize}
\item
  si m_i = 1 (pôle simple) on peut poser B(X) = (X -
  a_i)B_1(X)~; en multipliant les deux membres de la
  décomposition par X - a_i et en substituant a_i à X,
  on obtient (en remarquant que B'(X) = B_1(X) + (X -
  a_i)B_1'(X))

  \alpha_i,1 = A(a_i) \over
  B_1(a_i) = A(a_i) \over
  B'(a_i)
\item
  si m_i > 1, on écrit  A(X+a_i)
  \over B(X+a_i) = P(X) \over
  X^m_iQ(X) avec
  Q(0)\neq~0. On effectue la division suivant les
  puissances croissantes de P par Q à l'ordre m_i, d'où P(X) =
  S(X)Q(X) + X^m_iT(X) avec
  deg S \leq m_i~ - 1. On obtient alors
   P(X) \over X^m_iQ(X) = S(X)
  \over X^m_i + T(X)
  \over Q(X) = \alpha_i,1 \over
  X + \\ldots~ +
  \alpha_i,m_i \over
  X^m_i + T(X) \over Q(X) et
  donc

   A(X) \over B(X) = \alpha_i,1
  \over X - a_i +
  \\ldots~ +
  \alpha_i,m_i \over (X -
  a_i)^m_i + T(X - a_i)
  \over Q(X - a_i)

  Comme  T(X-a_i) \over Q(X-a_i)
  n'admet pas a_i comme pôle, c'est que l'on a déterminé la
  partie polaire relative au pôle a_i.
\end{itemize}

Pour chercher une primitive d'une fraction rationnelle  A(X)
\over B(X) dont on connaît la décomposition en éléments
simples

R(X) = E(X) + \\sum
_i=1^k\left ( \alpha_i,1
\over X - a_i +
\ldots + \alpha_i,m_i~
\over (X - a_i)^m_i
\right )

il suffit donc de savoir chercher une primitive du polynôme E(X) (ce qui
est élémentaire) et de chacun des éléments simples  1
\over (X-a_i)^k .

Théorème~9.4.3 (i) Une primitive de t\mapsto~ 1
\over (t-a)^k ,
k\neq~1, est - 1 \over k-1 
1 \over (t-a)^k-1 (ii) Une primitive de
t\mapsto~ 1 \over t-a est
log~ t - a si a \in \mathbb{R}~,
log~ t - a +
i\mathrmarctg~ ( t-\alpha~
\over \beta~ ) si a = \alpha~ + i\beta~ \in \mathbb{C} \diagdown \mathbb{R}~.

Démonstration Le premier point et le deuxième sont évidents~; si a = \alpha~ +
i\beta~ \in \mathbb{C} \diagdown \mathbb{R}~, on écrit  1 \over t-a = 1
\over t-\alpha~-i\beta~ = t-\alpha~ \over
(t-\alpha~)^2+\beta~^2 + i \beta~ \over
(t-\alpha~)^2+\beta~^2 dont une primitive est  1
\over 2  log~ ((t -
\alpha~)^2 + \beta~^2) +
i\mathrmarctg~ ( t-\alpha~
\over \beta~ ).

\paragraph{9.4.5 Fractions rationnelles en sinus et cosinus}

On cherche une primitive d'une fonction du type f :
t\mapsto~R(cos~
t,sin~ t) où R est une fraction rationnelle.

Dans le cas où R est un polynôme, la linéarisation de f(t) en utilisant
les formules de trigonométrie et en particulier
cos t = e^it+e^-it~
\over 2 , sin~ t =
e^it-e^-it \over 2i permettra de
calculer une primitive.

Pour une fraction rationnelle, nous utiliserons à plusieurs reprises le
lemme suivant

Lemme~9.4.4 Soit R(X,Y ) une fraction rationnelle à deux variables.
Alors il existe deux fractions rationnelles R_1 et R_2
à deux variables telles que R(X,Y ) = R_1(X^2,Y ) +
XR_2(X^2,Y )

Démonstration On écrit, en séparant au dénominateur, les puissances
paires de X des puissances impaires,

\begin{align*} R(X,Y )& =& A(X,Y )
\over B_1(X^2,Y ) +
XB_2(X^2,Y ) \%& \\
& =& A(X,Y )(B_1(X^2,Y ) -
XB_2(X^2,Y )) \over
B_1(X^2,Y )^2 -
X^2B_2(X^2,Y )^2 \%&
\\ & =& C(X,Y ) \over
D(X^2,Y ) = C_1(X^2,Y ) +
XC_2(X^2,Y ) \over D(X^2,Y
) \%& \\ & =&
R_1(X^2,Y ) + XR_ 2(X^2,Y ) \%&
\\ \end{align*}

En appliquant ce lemme, nous constatons que nous pouvons écrire

\begin{align*} f(t)& =&
R_1(cos~
^2t,sin~ t) +\
cos tR_ 2(cos~
^2t,sin~ t) \%&
\\ & =& R_1(1
- sin~
^2t,sin~ t) +\
cos t R_ 2(1 - sin~
^2t,sin~ t)\%&
\\ & =&
f_1(sin~ t) +\
cos t f_2(sin~ t) \%&
\\ \end{align*}

où f_1 et f_2 sont des fractions rationnelles à une
variable. Si f_1 = 0, on a alors

\int  f(t) dt =\\int ~
f_2(sin~
t)cos t dt =\\int ~
f_2(u) du

avec u = sin~ t. On est donc ramené à la
recherche d'une primitive de fraction rationnelle, ce que nous savons
faire. Or on constate facilement que, puisque
cos (\pi~ - t) = -\cos~ t
et sin (\pi~ - t) =\ sin~
t, on a f_1 = 0 \Leftrightarrow
\forall~~t \in \mathbb{R}~, f(\pi~ - t) = -f(t).

De même on peut écrire f(t) = f_3(cos~
t) + sin~
tf_4(cos~ t) (en intervertissant le
rôle du sinus et du cosinus, ou en changeant t en  \pi~
\over 2 - t) et si f_3 = 0, on a
\int  f(t) dt =\\int ~
f_4(cos~
t)sin t dt = -\\int ~
f_4(u) du avec u = cos~ t. Or comme ci
dessus, f_3 = 0 \Leftrightarrow
\forall~~t \in \mathbb{R}~, f(-t) = -f(t).

Mais on peut encore écrire f(t) = R(cos~
t,sin t) = R(\cos~
t,\mathrmtg~
tcos t) = S(\cos~
t,\mathrmtg~ t) et en
appliquant de nouveau le lemme, f(t) =
R_3(cos~
^2t,\mathrmtg~ t)
+ cos~
tR_4(cos~
^2t,\mathrmtg~ t).
Mais cos ^2~t = 1
\over
1+\mathrmtg~
^2t ce qui permet d'écrire f(t) =
f_5(\mathrmtg~ t)
+ cos~
tf_6(\mathrmtg~ t).
Alors, si f_6 = 0, le changement de variables u
= \mathrmtg~ t pour t \in]
- \pi~ \over 2 + n\pi~, \pi~ \over 2 +
n\pi~[, conduira à \int ~ f(t) dt
=\int ~
f_5(\mathrmtg~ t)
dt =\int   f_4~(u) \over
1+u^2 du, c'est-à-dire encore à une primitive de fraction
rationnelle. Or f_6 = 0 \Leftrightarrow
\forall~~t \in \mathbb{R}~, f(t + \pi~) = f(t).

Dans tous les autres cas, le changement de variable u
= \mathrmtg~  t
\over 2 , t \in](2n - 1)\pi~,(2n + 1)\pi~[ conduit à

\int  R(\cos~
t,sin t) dt =\\int ~ R(
1 - u^2 \over 1 + u^2 , 2u
\over 1 + u^2 ) 2du \over 1
+ u^2

c'est-à-dire encore à une primitive de fraction rationnelle.

On déduit de cette étude que

Proposition~9.4.5 Soit f(t) une fraction rationnelle en
sin t et \cos~ t

\begin{itemize}
\itemsep1pt\parskip0pt\parsep0pt
\item
  (i) si \forall~~t \in \mathbb{R}~, f(\pi~ - t) = -f(t), le
  changement de variable u = sin~ t conduit à
  la recherche d'une primitive de fraction rationnelle
\item
  (ii) si \forall~~t \in \mathbb{R}~, f(-t) = -f(t), le changement
  de variable u = cos~ t conduit à la recherche
  d'une primitive de fraction rationnelle
\item
  (iii) si \forall~~t \in \mathbb{R}~, f(t + \pi~) = f(t), le
  changement de variable u =\
  \mathrmtg t, t \in] - \pi~ \over
  2 + n\pi~, \pi~ \over 2 + n\pi~[, conduit à la recherche
  d'une primitive de fraction rationnelle
\item
  (iv) dans tous les autres cas, le changement de variable u
  = \mathrmtg~  t
  \over 2 , t \in](2n - 1)\pi~,(2n + 1)\pi~[, conduit à la
  recherche d'une primitive de fraction rationnelle.
\end{itemize}

Remarque~9.4.2 Les règles (i),(ii) et (iii) doivent toujours être
utilisées de préférence à la règle (iv) car elles conduisent à une
fraction rationnelle dont les degrés des numérateurs et dénominateurs
sont plus petits que dans la règle (iv). Le lecteur prendra garde à ne
pas appliquer les règles (iii) et (iv) en dehors de leurs intervalles de
validité respectifs (t \in] - \pi~ \over 2 + n\pi~, \pi~
\over 2 + n\pi~[ ou t \in](2n - 1)\pi~,(2n + 1)\pi~[) sous
peine d'erreurs difficilement décelables.

\paragraph{9.4.6 Fractions rationnelles en sinus et cosinus
hyperboliques}

On cherche une primitive d'une fonction du type f :
t\mapsto~R(\mathrmch~
t,\mathrmsh~ t) où R est une
fraction rationnelle.

Une première méthode est de rechercher le changement de variable que
l'on ferait pour calculer une primitive de g(t) =
R(cos t,\sin~ t)
(c'est-à-dire en transformant toutes les fonctions hyperboliques en
leurs analogues circulaires) et de faire le changement de variable
analogue u = \mathrmsh~ t, u
= \mathrmch~ t, u
= \mathrmth~ t ou u
= \mathrmth~  t
\over 2 .

Une deuxième méthode est de remarquer que f(t) est de la forme
S(e^t) où S est une fraction rationnelle à une variable. Le
changement de variable u = e^t conduit alors à
\int  f(t) dt =\\int ~
S(e^t) dt =\int ~  S(u)
\over u du c'est-à-dire encore à une primitive de
fraction rationnelle.

\paragraph{9.4.7 Intégrales abéliennes}

On cherche une primitive d'une fonction du type g :
x\mapsto~R(x,f(x)) où R est une fraction rationnelle
et f une fonction telle que la courbe d'équation y = f(x) puisse être
paramétrée par x = \phi(t),y = \psi(t) où \phi et \psi sont des fractions
rationnelles (où éventuellement des fonctions trigonométriques).

On a alors \int ~ g(x) dx
=\int  R(x,f(x)) dx =\\int ~
R(\phi(t),\psi(t))\phi'(t) dt par le changement de variable x = \phi(t) ce qui
conduit donc à une primitive de fractions rationnelles~; le paramètre t
doit varier de telle sorte que y = f(x) \Leftrightarrow x =
\phi(t), y = \psi(t)

Le cas le plus important est le cas des intégrales abéliennes où f est
une fonction algébrique~; autrement dit où la courbe y = f(x) est une
partie d'une courbe algébrique \Gamma d'équation P(x,y) = 0 où P est un
polynôme à deux variables. Une telle courbe, paramétrable par deux
fractions rationnelles x = \phi(t),y = \psi(t) est appelée une courbe
unicursale.

Remarque~9.4.3 L'exemple le plus simple de courbe algébrique non
unicursale est une courbe elliptique d'équation y^2 =
x^3 + px + q~; c'est ainsi que le calcul des primitives du
type \int  R(x,\sqrtx~^3
 + px + q) dx ne relèvera pas en général de la théorie précédente.

Nous allons étudier tout particulièrement deux exemples de fonctions
algébriques f.

Premier exemple~: f(x) = \rootn
\ofax+b \over cx+d  avec ad -
bc\neq~0. La courbe \Gamma est alors la courbe (cx +
d)y^n - (ax + b) = 0. On peut la paramétrer en posant y = t
auquel cas on obtient x = dt^n-b \over
-ct^n+a ~; d'où dx = nt^n-1 ad-bc
\over (ct^n-a)^2 . On obtient
donc

\int ~ R(x,\rootn
\ofax + b \over cx + d ) dx
=\int  R( dt^n~ - b
\over -ct^n + a ,t)nt^n-1 ad -
bc \over (ct^n - a)^2 dt

en posant t = \rootn \ofax+b
\over cx+d  ce qui conduit à la recherche d'une
primitive de fraction rationnelle.

Deuxième exemple~: f(x) = \sqrtax^2  + bx +
c avec a\neq~0 (sinon on retombe sur l'exemple
précédent avec n = 2, c = 0 et d = 1). La courbe \Gamma est alors la courbe
d'équation y^2 = ax^2 + bx + c, il s'agit soit
d'une ellipse (si a < 0) soit d'une hyperbole (si a
> 0). Bien entendu on doit se limiter à la portion de cette
conique située dans le demi plan supérieur~: y ≥ 0. Introduisons \Delta =
b^2 - 4ac que l'on peut manifestement supposer non nul, car
sinon ax^2 + bx + c est un carré parfait.

Premier cas~: a < 0~; on peut se limiter cas où \Delta
> 0 car sinon \forall~~x \in \mathbb{R}~,
ax^2 + bx + c < 0 et la fonction n'est jamais
définie. On écrit ax^2 + bx + c = a(x - \alpha~)(x - \beta~) = a((x -
p)^2 - q^2) en introduisant d'une part les racines \alpha~
et \beta~ du trinome, d'autre part sa forme canonique. La fonction f est
définie sur [\alpha~,\beta~].

Une première manière de paramétrer \Gamma est d'écrire son équation sous la
forme (x - p)^2 + y^2 \over
a = q^2 ce qui conduit au paramétrage x
- p = qcos~ t, y =
q\sqrtasin~
t et donc à \int ~
R(x,\sqrtax^2  + bx + c) dx
=\int  R(p + q\cos~
t,q\sqrtasin~
t)(-qsin~ t) dt, fraction rationnelle en
sin et \cos~ ~; le
paramètre t varie dans [0,\pi~] de telle manière que y ≥ 0.

Une deuxième manière est de couper l'ellipse \Gamma par une droite variable
passant par un point de l'ellipse, par exemple le point (\alpha~,0). On pose
donc y = t(x - \alpha~). Ceci conduit à y^2 = t^2(x -
\alpha~)^2 = a(x - \alpha~)(x - \beta~), soit t^2(x - \alpha~) = a(x - \beta~),
soit x = \alpha~t^2-a\beta~ \over t^2-a ,
puis y = t(x - \alpha~) = at(\beta~-\alpha~) \over t^2-a ~;
on obtient ainsi un paramétrage unicursal de \Gamma et on aboutit à une
recherche de primitive de fraction rationnelle~; le paramètre t varie de
telle sorte que y ≥ 0, soit t ≥ 0.

Deuxième cas~: a > 0, \Delta < 0. La fonction f(x) =
\sqrtax^2  + bx + c est définie sur \mathbb{R}~. On
écrit ax^2 + bx + c = a((x - p)^2 +
q^2) en introduisant sa forme canonique.

Une première manière de paramétrer \Gamma est d'écrire son équation sous la
forme  y^2 \over a - (x - p)^2
= q^2 ce qui conduit au paramétrage x - p =
q\mathrmsh~ t, y =
q\sqrta\mathrmch~
t et donc à \int ~
R(x,\sqrtax^2  + bx + c) dx
=\int ~ R(p +
q\mathrmsh~
t,q\sqrta\mathrmch~
t)(q\mathrmch~ t) dt,
fraction rationnelle en
\mathrmsh~ et
\mathrmch~ ~; le paramètre t
varie dans \mathbb{R}~.

Une deuxième manière est de couper l'hyperbole \Gamma par une droite variable
parallèle à l'une de ses asymptotes (de telles droites ne coupant \Gamma
qu'en un seul point), par exemple y = \sqrtax + t. On
a alors y^2 = (\sqrtax + t)^2 =
ax^2 + bx + c soit 2tx\sqrta +
t^2 = bx + c soit encore x = c-t^2
\over 2t\sqrta-b puis y =
\sqrtax + t =
\\ldots~~; on
aboutit à une recherche de primitive de fraction rationnelle~; le
paramètre t varie de telle sorte que y ≥ 0.

Troisième cas~: a > 0, \Delta > 0. On écrit
ax^2 + bx + c = a(x - \alpha~)(x - \beta~) = a((x - p)^2 -
q^2) en introduisant d'une part les racines \alpha~ et \beta~ du
trinome, d'autre part sa forme dite canonique. La fonction f est définie
sur ] -\infty~,\alpha~] et sur [\beta~,+\infty~[.

Une première manière de paramétrer \Gamma est d'écrire son équation sous la
forme (x - p)^2 - y^2 \over a =
q^2 ce qui conduit au paramétrage x - p =
q\epsilon\mathrmch~ t, y =
q\sqrta\mathrmsh~
t, avec \epsilon = ±1 = sgn~(x - p), et donc à
\int  R(x,\sqrtax^2 ~
+ bx + c) dx =\int ~ R(p +
q\epsilon\mathrmch~
t,q\sqrta\mathrmsh~
t)(\epsilonq\mathrmsh~ t) dt,
fraction rationnelle en
\mathrmsh~ et
\mathrmch~ ~; le paramètre t
varie dans [0,+\infty~[ de telle manière que y ≥ 0.

Une deuxième manière est de couper l'hyperbole \Gamma par une droite variable
passant par un point de l'hyperbole, par exemple le point (\alpha~,0). On pose
donc y = t(x - \alpha~). Ceci conduit à y^2 = t^2(x -
\alpha~)^2 = a(x - \alpha~)(x - \beta~), soit t^2(x - \alpha~) = a(x - \beta~),
soit x = \alpha~t^2-a\beta~ \over t^2-a ,
puis y = t(x - \alpha~) = at(\beta~-\alpha~) \over t^2-a ~;
on obtient ainsi un paramétrage unicursal de \Gamma et on aboutit à une
recherche de primitive de fraction rationnelle~; le paramètre t varie de
telle sorte que y ≥ 0.

Une troisième manière est de couper l'hyperbole \Gamma par une droite
variable parallèle à l'une de ses asymptotes (de telles droites ne
coupant \Gamma qu'en un seul point), par exemple y =
\sqrtax + t. On a alors y^2 =
(\sqrtax + t)^2 = ax^2 + bx + c
soit 2tx\sqrta + t^2 = bx + c soit encore
x = c-t^2 \over
2t\sqrta-b puis y = \sqrtax + t
= \\ldots~~; on
aboutit à une recherche de primitive de fraction rationnelle~; le
paramètre t varie de telle sorte que y ≥ 0.

[
[
[
[

\end{document}

% \documentclass[]{article}
\usepackage[T1]{fontenc}
\usepackage{lmodern}
\usepackage{amssymb,amsmath}
\usepackage{ifxetex,ifluatex}
\usepackage{fixltx2e} % provides \textsubscript
% use upquote if available, for straight quotes in verbatim environments
\IfFileExists{upquote.sty}{\usepackage{upquote}}{}
\ifnum 0\ifxetex 1\fi\ifluatex 1\fi=0 % if pdftex
  \usepackage[utf8]{inputenc}
\else % if luatex or xelatex
  \ifxetex
    \usepackage{mathspec}
    \usepackage{xltxtra,xunicode}
  \else
    \usepackage{fontspec}
  \fi
  \defaultfontfeatures{Mapping=tex-text,Scale=MatchLowercase}
  \newcommand{\euro}{€}
\fi
% use microtype if available
\IfFileExists{microtype.sty}{\usepackage{microtype}}{}
\ifxetex
  \usepackage[setpagesize=false, % page size defined by xetex
              unicode=false, % unicode breaks when used with xetex
              xetex]{hyperref}
\else
  \usepackage[unicode=true]{hyperref}
\fi
\hypersetup{breaklinks=true,
            bookmarks=true,
            pdfauthor={},
            pdftitle={Integration sur un intervalle quelconque : fonctions `a valeurs reelles positives},
            colorlinks=true,
            citecolor=blue,
            urlcolor=blue,
            linkcolor=magenta,
            pdfborder={0 0 0}}
\urlstyle{same}  % don't use monospace font for urls
\setlength{\parindent}{0pt}
\setlength{\parskip}{6pt plus 2pt minus 1pt}
\setlength{\emergencystretch}{3em}  % prevent overfull lines
\setcounter{secnumdepth}{0}
 
/* start css.sty */
.cmr-5{font-size:50%;}
.cmr-7{font-size:70%;}
.cmmi-5{font-size:50%;font-style: italic;}
.cmmi-7{font-size:70%;font-style: italic;}
.cmmi-10{font-style: italic;}
.cmsy-5{font-size:50%;}
.cmsy-7{font-size:70%;}
.cmex-7{font-size:70%;}
.cmex-7x-x-71{font-size:49%;}
.msbm-7{font-size:70%;}
.cmtt-10{font-family: monospace;}
.cmti-10{ font-style: italic;}
.cmbx-10{ font-weight: bold;}
.cmr-17x-x-120{font-size:204%;}
.cmsl-10{font-style: oblique;}
.cmti-7x-x-71{font-size:49%; font-style: italic;}
.cmbxti-10{ font-weight: bold; font-style: italic;}
p.noindent { text-indent: 0em }
td p.noindent { text-indent: 0em; margin-top:0em; }
p.nopar { text-indent: 0em; }
p.indent{ text-indent: 1.5em }
@media print {div.crosslinks {visibility:hidden;}}
a img { border-top: 0; border-left: 0; border-right: 0; }
center { margin-top:1em; margin-bottom:1em; }
td center { margin-top:0em; margin-bottom:0em; }
.Canvas { position:relative; }
li p.indent { text-indent: 0em }
.enumerate1 {list-style-type:decimal;}
.enumerate2 {list-style-type:lower-alpha;}
.enumerate3 {list-style-type:lower-roman;}
.enumerate4 {list-style-type:upper-alpha;}
div.newtheorem { margin-bottom: 2em; margin-top: 2em;}
.obeylines-h,.obeylines-v {white-space: nowrap; }
div.obeylines-v p { margin-top:0; margin-bottom:0; }
.overline{ text-decoration:overline; }
.overline img{ border-top: 1px solid black; }
td.displaylines {text-align:center; white-space:nowrap;}
.centerline {text-align:center;}
.rightline {text-align:right;}
div.verbatim {font-family: monospace; white-space: nowrap; text-align:left; clear:both; }
.fbox {padding-left:3.0pt; padding-right:3.0pt; text-indent:0pt; border:solid black 0.4pt; }
div.fbox {display:table}
div.center div.fbox {text-align:center; clear:both; padding-left:3.0pt; padding-right:3.0pt; text-indent:0pt; border:solid black 0.4pt; }
div.minipage{width:100%;}
div.center, div.center div.center {text-align: center; margin-left:1em; margin-right:1em;}
div.center div {text-align: left;}
div.flushright, div.flushright div.flushright {text-align: right;}
div.flushright div {text-align: left;}
div.flushleft {text-align: left;}
.underline{ text-decoration:underline; }
.underline img{ border-bottom: 1px solid black; margin-bottom:1pt; }
.framebox-c, .framebox-l, .framebox-r { padding-left:3.0pt; padding-right:3.0pt; text-indent:0pt; border:solid black 0.4pt; }
.framebox-c {text-align:center;}
.framebox-l {text-align:left;}
.framebox-r {text-align:right;}
span.thank-mark{ vertical-align: super }
span.footnote-mark sup.textsuperscript, span.footnote-mark a sup.textsuperscript{ font-size:80%; }
div.tabular, div.center div.tabular {text-align: center; margin-top:0.5em; margin-bottom:0.5em; }
table.tabular td p{margin-top:0em;}
table.tabular {margin-left: auto; margin-right: auto;}
div.td00{ margin-left:0pt; margin-right:0pt; }
div.td01{ margin-left:0pt; margin-right:5pt; }
div.td10{ margin-left:5pt; margin-right:0pt; }
div.td11{ margin-left:5pt; margin-right:5pt; }
table[rules] {border-left:solid black 0.4pt; border-right:solid black 0.4pt; }
td.td00{ padding-left:0pt; padding-right:0pt; }
td.td01{ padding-left:0pt; padding-right:5pt; }
td.td10{ padding-left:5pt; padding-right:0pt; }
td.td11{ padding-left:5pt; padding-right:5pt; }
table[rules] {border-left:solid black 0.4pt; border-right:solid black 0.4pt; }
.hline hr, .cline hr{ height : 1px; margin:0px; }
.tabbing-right {text-align:right;}
span.TEX {letter-spacing: -0.125em; }
span.TEX span.E{ position:relative;top:0.5ex;left:-0.0417em;}
a span.TEX span.E {text-decoration: none; }
span.LATEX span.A{ position:relative; top:-0.5ex; left:-0.4em; font-size:85%;}
span.LATEX span.TEX{ position:relative; left: -0.4em; }
div.float img, div.float .caption {text-align:center;}
div.figure img, div.figure .caption {text-align:center;}
.marginpar {width:20%; float:right; text-align:left; margin-left:auto; margin-top:0.5em; font-size:85%; text-decoration:underline;}
.marginpar p{margin-top:0.4em; margin-bottom:0.4em;}
.equation td{text-align:center; vertical-align:middle; }
td.eq-no{ width:5%; }
table.equation { width:100%; } 
div.math-display, div.par-math-display{text-align:center;}
math .texttt { font-family: monospace; }
math .textit { font-style: italic; }
math .textsl { font-style: oblique; }
math .textsf { font-family: sans-serif; }
math .textbf { font-weight: bold; }
.partToc a, .partToc, .likepartToc a, .likepartToc {line-height: 200%; font-weight:bold; font-size:110%;}
.chapterToc a, .chapterToc, .likechapterToc a, .likechapterToc, .appendixToc a, .appendixToc {line-height: 200%; font-weight:bold;}
.index-item, .index-subitem, .index-subsubitem {display:block}
.caption td.id{font-weight: bold; white-space: nowrap; }
table.caption {text-align:center;}
h1.partHead{text-align: center}
p.bibitem { text-indent: -2em; margin-left: 2em; margin-top:0.6em; margin-bottom:0.6em; }
p.bibitem-p { text-indent: 0em; margin-left: 2em; margin-top:0.6em; margin-bottom:0.6em; }
.paragraphHead, .likeparagraphHead { margin-top:2em; font-weight: bold;}
.subparagraphHead, .likesubparagraphHead { font-weight: bold;}
.quote {margin-bottom:0.25em; margin-top:0.25em; margin-left:1em; margin-right:1em; text-align:justify;}
.verse{white-space:nowrap; margin-left:2em}
div.maketitle {text-align:center;}
h2.titleHead{text-align:center;}
div.maketitle{ margin-bottom: 2em; }
div.author, div.date {text-align:center;}
div.thanks{text-align:left; margin-left:10%; font-size:85%; font-style:italic; }
div.author{white-space: nowrap;}
.quotation {margin-bottom:0.25em; margin-top:0.25em; margin-left:1em; }
h1.partHead{text-align: center}
.sectionToc, .likesectionToc {margin-left:2em;}
.subsectionToc, .likesubsectionToc {margin-left:4em;}
.subsubsectionToc, .likesubsubsectionToc {margin-left:6em;}
.frenchb-nbsp{font-size:75%;}
.frenchb-thinspace{font-size:75%;}
.figure img.graphics {margin-left:10%;}
/* end css.sty */

\title{Integration sur un intervalle quelconque : fonctions `a valeurs reelles
positives}
\author{}
\date{}

\begin{document}
\maketitle

\textbf{Warning: 
requires JavaScript to process the mathematics on this page.\\ If your
browser supports JavaScript, be sure it is enabled.}

\begin{center}\rule{3in}{0.4pt}\end{center}

[
[
[]
[

\subsubsection{9.5 Intégration sur un intervalle quelconque~: fonctions
à valeurs réelles positives}

\paragraph{9.5.1 Fonctions intégrables à valeurs réelles positives}

Définition~9.5.1 Soit I un intervalle de \mathbb{R}~, f : I \rightarrow~ \mathbb{R}~ positive et
continue par morceaux. On dit que f est intégrable sur I s'il existe une
constante M ≥ 0 telle que, pour tout segment [a,b] \subset~ I, on ait
\int  _a^b~f \leq M . On note alors
\int  _I~f =\
sup_[a,b]\subset~I\int ~
_a^bf.

Proposition~9.5.1 Soit I un intervalle de \mathbb{R}~, f : I \rightarrow~ \mathbb{R}~ positive et
continue par morceaux, intégrable sur I. Alors f est intégrable sur tout
intervalle I' inclus dans I et \int ~
_I'f \leq\int  _I~f.

Démonstration En effet tout segment inclus dans I' est également un
segment inclus dans I, donc le même M convient comme majorant.

Proposition~9.5.2 Soit f,g : I \rightarrow~ \mathbb{R}~ positives et continues par morceaux
telles que 0 \leq f \leq g. Si g est intégrable sur I il en est de même de f
et \int  _I~f
\leq\int  _I~g.

Démonstration Evident d'après la définition.

Proposition~9.5.3 Soit I un intervalle de \mathbb{R}~, f : I \rightarrow~ \mathbb{R}~ positive et
continue par morceaux. Alors f est intégrable sur I si et seulement si
il existe une suite ([a_n,b_n])_n\in\mathbb{N}~
croissante de segments contenus dans I, dont la réunion est égale à I,
et une constante positive M telle que \forall~~n \in \mathbb{N}~,
\int  _a_n^b_n~f
\leq M. Dans ce cas, on a

\int  _I~f =\
sup_n\int ~
_a_n^b_n f =\
lim_n\rightarrow~+\infty~\int ~
_a_n^b_n f

Démonstration La condition est bien évidemment nécessaire~: prendre
n'importe quelle suite ([a_n,b_n]) vérifiant les
conditions voulues. Inversement supposons qu'il existe une telle suite
([a_n,b_n]) et une constante M ≥ 0. Soit J =
[a,b] un segment inclus dans I et posons J_n =
[a_n,b_n]. Si b = sup~I,
alors supI \in\cupJ_n~ et donc il existe N
\in \mathbb{N}~ tel que supI \in J_N~ auquel cas
supI \in J_n~ pour tout n ≥ N. Si par
contre, b < sup~I
= limb_n~, alors il existe N \in \mathbb{N}~ tel
que n ≥ N \rigtharrow~ b_n > b. Dans les deux cas il existe N
\in \mathbb{N}~ tel que n ≥ N \rigtharrow~ b_n ≥ b. De même, il existe N' \in \mathbb{N}~ tel que
n ≥ N' \rigtharrow~ a_n \leq a. Soit n = max~(N,N'),
on a alors J = [a,b] \subset~ [a_n,b_n] =
J_n et donc

\int  _a^b~f
\leq\int  _a_n^b_n~
f \leq M

ce qui montre que f est intégrable sur I.

La démonstration précédente montre clairement dans sa première partie
que sup_n~\\int
 _a_n^b_nf \leq\
sup_[a,b]\subset~I\int ~
_a^bf =\int  _I~f et dans
sa deuxième partie que
sup_[a,b]\subset~I~\\int
 _a^bf \leq\
sup_n\int ~
_a_n^b_nf, et donc l'égalité
\int  _I~f =\
sup_n\int ~
_a_n^b_nf. Mais comme la suite
\left (\int ~
_a_n^b_nf\right
)_n\in\mathbb{N}~ est croissante majorée, sa borne supérieure est aussi sa
limite.

Proposition~9.5.4 Soit I = [a,b] un segment de \mathbb{R}~, f : I \rightarrow~ \mathbb{R}~ positive
et continue par morceaux. Alors f est intégrable sur I et
\int  _I~f =\\int
 _a^bf. De plus f est intégrable sur ]a,b[,
[a,b[ et ]a,b], toutes ces intégrales étant égales.

Démonstration Si J = [c,d] est un segment inclus dans [a,b], on
a \int  _c^d~f
\leq\int  _a^b~f, donc f est
intégrable et \int  _I~f
\leq\int  _a^b~f. Mais d'autre part,
[a,b] est lui même un segment inclus dans I, donc
\int  _a^b~f
\leq\int  _I~f, et donc l'égalité. On sait
alors que f est intégrable sur tout intervalle inclus dans I et en
particulier sur ]a,b[, [a,b[ et ]a,b]. De plus, si
a_n = a + 1 \over n et b_n = b -
1 \over n , J_n =
[a_n,b_n] est une suite croissante de segments
dont la réunion est ]a,b[, donc

\int  _]a,b[~f
= lim\\int ~
_a_n^b_n f =\\int
 _a^bf =\int ~
_[a,b]f

par continuité de l'intégrale par rapport à ses bornes. Comme on a
]a,b[\subset~ [a,b[\subset~ [a,b], on a aussi \\int
 _]a,b[f \leq\int  _[a,b[~f
\leq\int  _[a,b]~f, d'où l'égalité des
trois nombres. Il en est de même de \int ~
_]a,b]f.

Proposition~9.5.5 Soit f : I \rightarrow~ \mathbb{R}~ continue positive intégrable, telle que
\int  _I~f = 0. Alors f = 0.

Démonstration Pour tout segment J \subset~ I, on a 0
\leq\int  _J~f \leq\\int
 _If = 0, donc \int  _J~f = 0 ce
qui implique que f est nulle sur J. La fonction f est donc nulle sur
tout segment inclus dans I, donc elle est nulle.

Proposition~9.5.6 Soit f,g : I \rightarrow~ \mathbb{R}~ positives et continues par morceaux,
soit \alpha~ \in \mathbb{R}~^+. Si f et g sont intégrables sur I, il en est de
même de f + g et de \alpha~f et on a

\int  _I~(f + g)
=\int  _I~f +\\int
 _Ig\text et \int ~
_I(\alpha~f) = \alpha~\int  _I~f

Démonstration L'intégrabilité est évidente à partir de la définition.
Pour les égalités, il suffit de prendre une suite (J_n)
croissante de segments de réunion I et de passer à la limite dans les
formules

\int  _J_n~(f + g)
=\int  _J_n~f
+\int ~
_J_ng\text et
\int  _J_n~(\alpha~f) =
\alpha~\int  _J_n~f

Proposition~9.5.7 Soit I un intervalle de \mathbb{R}~, f : I \rightarrow~ \mathbb{R}~ positive et
continue par morceaux. Soit a \in I^o. Alors f est intégrable
sur I si et seulement si elle est intégrable sur I\bigcap] -\infty~,a] et sur I
\bigcap [a,+\infty~[. Dans ce cas,

\int  _I~f =\\int
 _I\bigcap]-\infty~,a]f +\int ~
_I\bigcap[a,+\infty~[

Démonstration Si f est intégrable sur I, elle est intégrable sur tout
sous intervalle de I et donc sur I\bigcap] -\infty~,a] et sur I \bigcap [a,+\infty~[.
Inversement, si f est intégrable sur ces deux sous intervalles, soit
M_1 et M_2 les majorants des intégrales sur les sous
segments de I\bigcap] -\infty~,a] et I \bigcap [a,+\infty~[. Si J est un segment inclus
dans I on a

\int  _J~f \leq\left
\ \cases M_1 &si
supJ \leq a \cr M_1~ +
M_2&si a \in J \cr M_2 &si a
\leq inf J ~ \right .

et dans tous les cas \int  _J~f \leq
M_1 + M_2. Donc f est intégrable sur I. Soit alors
J_n = [a_n,b_n] une suite croissante de
segments de réunion I. Pour n assez grand, on a a_n \leq a \leq
b_n car a est dans l'intérieur de I. Mais
([a_n,a]) est une suite croissante de segments de réunion
I\bigcap] -\infty~,a] et ([a,b_n]) est une suite croissante de
segments de réunion I \bigcap [a,+\infty~[. On peut donc passer à la limite dans
la formule \int ~
_[a_n,b_n]f =\int ~
_[a_n,a]f +\int ~
_[a,b_n]f, et on obtient

\int  _I~f =\\int
 _I\bigcap]-\infty~,a]f +\int ~
_I\bigcap[a,+\infty~[

Proposition~9.5.8 Soit -\infty~ < a < b \leq +\infty~, et f :
[a,b[\rightarrow~ \mathbb{R}~ positive et continue par morceaux. Pour x \in [a,b[,
posons F(x) =\int  _a^x~f(t) dt.
Alors f est intégrable sur [a,b[ si et seulement si F admet une
limite au point b. Dans ce cas, \int ~
_[a,b[f = lim_x\rightarrow~b~F(x) -
F(a)

Démonstration Soit b_n une suite croissante de [a,b[ de
limite b. Alors [a,b_n] est une suite croissante de
segments dont la réunion est [a,b[. Donc f est intégrable si et
seulement si la suite \int ~
_a^b_nf = F(b_n) - F(a) admet une
limite, donc si et seulement si la suite (F(b_n)) est
convergente. Mais comme F est croissante, ceci équivaut à l'existence de
la limite de F en b.

Remarque~9.5.1 Si f n'est pas intégrable sur [a,b[, alors F, qui est
croissante, admet + \infty~ comme limite au point b.

Remarque~9.5.2 De même, si -\infty~\leq a < b < +\infty~, et f
:]a,b] \rightarrow~ \mathbb{R}~ positive et continue par morceaux. Pour x \in]a,b],
posons F(x) =\int  _x^b~f(t) dt.
Alors f est intégrable sur ]a,b] si et seulement si F (qui est cette
fois décroissante) admet une limite au point a. Dans ce cas,
\int  _]a,b]~f = F(b)
- lim_x\rightarrow~a~F(x)

\paragraph{9.5.2 Règles de comparaison}

Théorème~9.5.9 Soit f,g : [a,b[\rightarrow~ \mathbb{R}~ continues par morceaux positives.
On suppose qu'au voisinage de b on a f = O(g) (resp. f = o(g)). Alors
(i) si g est intégrable sur [a,b[, il en est de même de f et
\int  _[x,b[~f(t) dt =
O(\int  _[x,b[~g(t) dt) (resp.
\int  _[x,b[~f(t) dt =
o(\int  _[x,b[~g(t) dt)) (ii) si f
n'est pas intégrable sur [a,b[, g ne l'est pas non plus et
\int  _a^x~f(t) dt =
O(\int  _a^x~g(t) dt) (resp.
\int  _a^x~f(t) dt =
o(\int  _a^x~g(t) dt))

Démonstration Les convergences et divergences découlent immédiatement de
l'inégalité 0 \leq f \leq Kg qui est vraie sur [c,b[ et du fait que f et g
sont intégrables sur [a,c] (car continues par morceaux sur ce
segment). De plus f = o(g) \rigtharrow~ f = O(g). En ce qui concerne la comparaison
des restes ou des intégrales partielles, la démonstration est tout à
fait similaire à celle du théorème analogue sur les séries. Nous allons
la faire dans le cas f = o(g), la démonstration étant analogue pour f =
O(g) en changeant \epsilon en K ou en 2K.

(i) Supposons f = o(g) et g intégrable. Soit \epsilon > 0. Il
existe c \in [a,b[ tel que t ≥ c \rigtharrow~ 0 \leq f(t) \leq \epsilong(t). Alors pour x ≥ c,
on a (en intégrant l'inégalité de x à b), 0 \leq\\int
 _[x,b[f(t) dt \leq \epsilon\int ~
_[x,b[g(t) dt et donc \int ~
_[x,b[f(t) dt = o(\int ~
_[x,b[g(t) dt).

(ii) Supposons f = o(g) et f non intégrable sur [a,b[. Soit \epsilon
> 0. Il existe c \in [a,b[ tel que t ≥ c \rigtharrow~ 0 \leq f(t) \leq \epsilon
\over 2 g(t). Alors pour x ≥ c, on a (en intégrant
l'inégalité de c à x), \int ~
_c^xf(t) dt \leq \epsilon \over 2
\int  _c^x~g(t) dt, soit encore à
l'aide de la relation de Chasles

0 \leq\int  _a^x~f(t) dt \leq \epsilon
\over 2 \int ~
_a^xg(t) dt + \left
(\int  _a^c~f(t) dt - \epsilon
\over 2 \int ~
_a^cg(t) dt\right )

Mais comme on sait que g n'est pas intégrable sur [a,b[ et que g ≥
0, on a
lim_x\rightarrow~b\\int ~
_a^xg(t) dt = +\infty~. Donc il existe c' \in [a,b[ tel que x
≥ c' \rigtharrow~ \epsilon \over 2 \int ~
_a^xg(t) dt >\int ~
_a^cf(t) dt - \epsilon \over 2
\int  _a^c~g(t) dt. Alors, pour x
≥ max~(c,c'), on a

0 \leq\int  _a^x~f(t) dt \leq \epsilon
\over 2 \int ~
_a^xg(t) dt + \epsilon \over 2
\int  _a^x~g(t) dt =
\epsilon\int  _a^x~g(t) dt

et donc \int  _a^x~f(t) dt =
o(\int  _a^x~g(t) dt).

Remarque~9.5.3 Il suffit pour appliquer le théorème précédent que la
condition de positivité de f et g soit vérifiée dans un voisinage de b.

Théorème~9.5.10 Soit f,g : [a,b[\rightarrow~ \mathbb{R}~ continues par morceaux. On
suppose que g est positive et que au voisinage de b, on a f ∼ g. Alors f
et g sont simultanément intégrables ou non intégrables sur [a,b[.
Plus précisément (i) Si g est intégrable sur [a,b[, alors f
également et \int  _[x,b[~f(t) dt
∼\int  _[x,b[~g(t) dt (ii) Si g est
non intégrable sur [a,b[, alors f également et
\int  _a^x~f(t) dt
∼\int  _a^x~g(t) dt.

Démonstration Puisque f(t) ∼ g(t), il existe c \in [a,b[ tel que x
> c \rigtharrow~ 1 \over 2 g(t) \leq f(t) \leq 3
\over 2 g(t) ce qui montre que f est positive au
voisinage de b et que l'on a à la fois f = O(g) et g = O(f). Le théorème
précédent assure alors que f est intégrable sur [a,b[ si et
seulement si~g l'est. Pla\ccons nous dans le cas
d'intégrabilité. On a f - g = o(g), on en déduit que
f - g est intégrable et que
\int  _[x,b[~f(t) -
g(t) dt = o(\int ~
_[x,b[g(t) dt). Mais bien évidemment \left
\int  _[x,b[~f(t) dt
-\int  _[x,b[~g(t)
dt\right \leq\int ~
_[x,b[f(t) - g(t) dt. On a donc
\int  _[x,b[~f(t) dt
-\int  _[x,b[~g(t) dt =
o(\int  _[x,b[~g(t) dt) et donc
\int  _[x,b[~f(t) dt
∼\int  _[x,b[~g(t) dt. Dans le cas de
non intégrabilité, deux cas se présentent. Si f - g
est non intégrable, le théorème précédent assure que
\int  _a^x~f(t) -
g(t) dt = o(\int ~
_a^xg(t) dt)~; si par contre elle est intégrable,
\int  _a^x~f(t) -
g(t) dt admet une limite finie en b alors que
\int  _a^x~g(t) dt tend vers + \infty~
et on a donc encore \int ~
_a^xf(t) - g(t) dt =
o(\int  _a^x~g(t) dt). L'inégalité
\left \int ~
_a^xf(t) dt -\int ~
_a^xg(t) dt\right
\leq\int ~
_a^xf(t) - g(t) dt donne alors
\int  _a^x~f(t) dt
-\int  _a^x~g(t) dt =
o(\int  _a^x~g(t) dt) et donc
\int  _a^x~f(t) dt
∼\int  _a^x~g(t) dt.

\paragraph{9.5.3 Exemples fondamentaux}

L'idée générale est d'obtenir une famille de fonctions étalons.

Proposition~9.5.11 La fonction
t\mapsto~t^\alpha~ est intégrable sur
[a,+\infty~[ (avec a > 0) si et seulement si~\alpha~ >
1.

Démonstration On a

\int  _1^x~ dt
\over t^\alpha~ = \left
\ \cases  1 \over
\alpha~-1 (1 - x^1-\alpha~)&si \alpha~\neq~1
\cr \cr log~ x
&si \alpha~ = 1 \cr  \right .

qui admet une limite finie en + \infty~ si et seulement si~\alpha~ > 1.

Exemple~9.5.1 Intégrales de Bertrand \int ~
_e^+\infty~ dt \over
t^\alpha~(log t)^\beta~~ . Si \alpha~
> 1, soit \gamma tel que 1 < \alpha~ < \gamma. On a
alors  1 \over
t^\alpha~(log t)^\beta~~ = o( 1
\over t^\gamma ) et donc
t\mapsto~ 1 \over
t^\alpha~(log t)^\beta~~ est
intégrable sur [e,+\infty~[. Si \alpha~ < 1, soit \gamma tel que \alpha~
< \gamma < 1~; on a alors  1 \over
t^\gamma = o( 1 \over
t^\alpha~(log t)^\beta~~ ) et
comme t\mapsto~ 1 \over
t^\gamma n'est pas intégrable sur [e,+\infty~[,
t\mapsto~ 1 \over
t^\alpha~(log t)^\beta~~ n'est pas
intégrable sur [e,+\infty~[. Si \alpha~ = 1, on a par le changement de variables
u = log~ t,

\begin{align*} \int ~
_e^x dt \over
t(log t)^\beta~~ & =&
\int ~
_1^log x~ du
\over u^\beta~ \%&
\\ & =& \left
\ \cases  1 \over
\beta~-1 (1 - (log x)^1-\beta~~)&si
\alpha~\neq~1 \cr \cr
log \log~ x &si \alpha~ = 1
 \right .\%&\\
\end{align*}

qui admet une limite en + \infty~ si et seulement si~\beta~ > 1. En
définitive t\mapsto~ 1 \over
t^\alpha~(log t)^\beta~~ est
intégrable sur [e,+\infty~[ si et seulement si~\alpha~ > 1 ou (\alpha~ =
1 et \beta~ > 1).

Proposition~9.5.12 La fonction
t\mapsto~t^\alpha~ est intégrable sur ]0,a]
(avec a > 0) si et seulement si~\alpha~ < 1.

Démonstration On a

\int  _x^a~ dt
\over t^\alpha~ = \left
\ \cases  1 \over
1-\alpha~ (a^1-\alpha~ - x^1-\alpha~)&si
\alpha~\neq~1 \cr \cr
log a -\ log~ x&si \alpha~
= 1  \right .

qui admet une limite au point 0 si et seulement si~\alpha~ < 1.

Exemple~9.5.2 Intégrales de Bertrand \int ~
_0^1\diagupet^\alpha~log~
t^\beta~ dt. Si \alpha~ > -1, soit \gamma tel que \alpha~
> \gamma > -1. On a alors en 0,
t^\alpha~log~
t^\beta~ = o(t^\gamma) (car 
t^\alpha~ log~
t^\beta~ \over t^\gamma =
t^\alpha~-\gammalog~
t^\beta~ tend vers 0 quand t tend vers 0) et comme
t\mapsto~t^\gamma est intégrable sur
]0,1\diagupe], il en est de même de
t\mapsto~t^\alpha~log~
t^\beta~. Si \alpha~ < -1, soit \gamma tel que \alpha~
< \gamma < -1. Alors t^\gamma =
o(t^\alpha~log~
t^\beta~) et comme
t\mapsto~t^\gamma n'est pas intégrable sur
]0,1\diagupe], il en est de même de
t\mapsto~t^\alpha~log~
t^\beta~. Si \alpha~ = -1, le changement de variables u =
-log~ t conduit à

\int  _x^1\diagupe~
log t^\beta~~
\over t dt =\int ~
_1^- log xu^\beta~~ du

qui admet une limite quand x tend vers 0 si et seulement si~\beta~
< -1. En définitive,
t\mapsto~t^\alpha~log~
t^\beta~ est intégrable sur [0,1\diagupe[ si et seulement
si~\alpha~ > -1 ou (\alpha~ = -1 et \beta~ < -1).

[
[
[
[

\end{document}

% \documentclass[]{article}
\usepackage[T1]{fontenc}
\usepackage{lmodern}
\usepackage{amssymb,amsmath}
\usepackage{ifxetex,ifluatex}
\usepackage{fixltx2e} % provides \textsubscript
% use upquote if available, for straight quotes in verbatim environments
\IfFileExists{upquote.sty}{\usepackage{upquote}}{}
\ifnum 0\ifxetex 1\fi\ifluatex 1\fi=0 % if pdftex
  \usepackage[utf8]{inputenc}
\else % if luatex or xelatex
  \ifxetex
    \usepackage{mathspec}
    \usepackage{xltxtra,xunicode}
  \else
    \usepackage{fontspec}
  \fi
  \defaultfontfeatures{Mapping=tex-text,Scale=MatchLowercase}
  \newcommand{\euro}{€}
\fi
% use microtype if available
\IfFileExists{microtype.sty}{\usepackage{microtype}}{}
\ifxetex
  \usepackage[setpagesize=false, % page size defined by xetex
              unicode=false, % unicode breaks when used with xetex
              xetex]{hyperref}
\else
  \usepackage[unicode=true]{hyperref}
\fi
\hypersetup{breaklinks=true,
            bookmarks=true,
            pdfauthor={},
            pdftitle={Integration sur un intervalle quelconque : fonctions `a valeurs complexes},
            colorlinks=true,
            citecolor=blue,
            urlcolor=blue,
            linkcolor=magenta,
            pdfborder={0 0 0}}
\urlstyle{same}  % don't use monospace font for urls
\setlength{\parindent}{0pt}
\setlength{\parskip}{6pt plus 2pt minus 1pt}
\setlength{\emergencystretch}{3em}  % prevent overfull lines
\setcounter{secnumdepth}{0}
 
/* start css.sty */
.cmr-5{font-size:50%;}
.cmr-7{font-size:70%;}
.cmmi-5{font-size:50%;font-style: italic;}
.cmmi-7{font-size:70%;font-style: italic;}
.cmmi-10{font-style: italic;}
.cmsy-5{font-size:50%;}
.cmsy-7{font-size:70%;}
.cmex-7{font-size:70%;}
.cmex-7x-x-71{font-size:49%;}
.msbm-7{font-size:70%;}
.cmtt-10{font-family: monospace;}
.cmti-10{ font-style: italic;}
.cmbx-10{ font-weight: bold;}
.cmr-17x-x-120{font-size:204%;}
.cmsl-10{font-style: oblique;}
.cmti-7x-x-71{font-size:49%; font-style: italic;}
.cmbxti-10{ font-weight: bold; font-style: italic;}
p.noindent { text-indent: 0em }
td p.noindent { text-indent: 0em; margin-top:0em; }
p.nopar { text-indent: 0em; }
p.indent{ text-indent: 1.5em }
@media print {div.crosslinks {visibility:hidden;}}
a img { border-top: 0; border-left: 0; border-right: 0; }
center { margin-top:1em; margin-bottom:1em; }
td center { margin-top:0em; margin-bottom:0em; }
.Canvas { position:relative; }
li p.indent { text-indent: 0em }
.enumerate1 {list-style-type:decimal;}
.enumerate2 {list-style-type:lower-alpha;}
.enumerate3 {list-style-type:lower-roman;}
.enumerate4 {list-style-type:upper-alpha;}
div.newtheorem { margin-bottom: 2em; margin-top: 2em;}
.obeylines-h,.obeylines-v {white-space: nowrap; }
div.obeylines-v p { margin-top:0; margin-bottom:0; }
.overline{ text-decoration:overline; }
.overline img{ border-top: 1px solid black; }
td.displaylines {text-align:center; white-space:nowrap;}
.centerline {text-align:center;}
.rightline {text-align:right;}
div.verbatim {font-family: monospace; white-space: nowrap; text-align:left; clear:both; }
.fbox {padding-left:3.0pt; padding-right:3.0pt; text-indent:0pt; border:solid black 0.4pt; }
div.fbox {display:table}
div.center div.fbox {text-align:center; clear:both; padding-left:3.0pt; padding-right:3.0pt; text-indent:0pt; border:solid black 0.4pt; }
div.minipage{width:100%;}
div.center, div.center div.center {text-align: center; margin-left:1em; margin-right:1em;}
div.center div {text-align: left;}
div.flushright, div.flushright div.flushright {text-align: right;}
div.flushright div {text-align: left;}
div.flushleft {text-align: left;}
.underline{ text-decoration:underline; }
.underline img{ border-bottom: 1px solid black; margin-bottom:1pt; }
.framebox-c, .framebox-l, .framebox-r { padding-left:3.0pt; padding-right:3.0pt; text-indent:0pt; border:solid black 0.4pt; }
.framebox-c {text-align:center;}
.framebox-l {text-align:left;}
.framebox-r {text-align:right;}
span.thank-mark{ vertical-align: super }
span.footnote-mark sup.textsuperscript, span.footnote-mark a sup.textsuperscript{ font-size:80%; }
div.tabular, div.center div.tabular {text-align: center; margin-top:0.5em; margin-bottom:0.5em; }
table.tabular td p{margin-top:0em;}
table.tabular {margin-left: auto; margin-right: auto;}
div.td00{ margin-left:0pt; margin-right:0pt; }
div.td01{ margin-left:0pt; margin-right:5pt; }
div.td10{ margin-left:5pt; margin-right:0pt; }
div.td11{ margin-left:5pt; margin-right:5pt; }
table[rules] {border-left:solid black 0.4pt; border-right:solid black 0.4pt; }
td.td00{ padding-left:0pt; padding-right:0pt; }
td.td01{ padding-left:0pt; padding-right:5pt; }
td.td10{ padding-left:5pt; padding-right:0pt; }
td.td11{ padding-left:5pt; padding-right:5pt; }
table[rules] {border-left:solid black 0.4pt; border-right:solid black 0.4pt; }
.hline hr, .cline hr{ height : 1px; margin:0px; }
.tabbing-right {text-align:right;}
span.TEX {letter-spacing: -0.125em; }
span.TEX span.E{ position:relative;top:0.5ex;left:-0.0417em;}
a span.TEX span.E {text-decoration: none; }
span.LATEX span.A{ position:relative; top:-0.5ex; left:-0.4em; font-size:85%;}
span.LATEX span.TEX{ position:relative; left: -0.4em; }
div.float img, div.float .caption {text-align:center;}
div.figure img, div.figure .caption {text-align:center;}
.marginpar {width:20%; float:right; text-align:left; margin-left:auto; margin-top:0.5em; font-size:85%; text-decoration:underline;}
.marginpar p{margin-top:0.4em; margin-bottom:0.4em;}
.equation td{text-align:center; vertical-align:middle; }
td.eq-no{ width:5%; }
table.equation { width:100%; } 
div.math-display, div.par-math-display{text-align:center;}
math .texttt { font-family: monospace; }
math .textit { font-style: italic; }
math .textsl { font-style: oblique; }
math .textsf { font-family: sans-serif; }
math .textbf { font-weight: bold; }
.partToc a, .partToc, .likepartToc a, .likepartToc {line-height: 200%; font-weight:bold; font-size:110%;}
.chapterToc a, .chapterToc, .likechapterToc a, .likechapterToc, .appendixToc a, .appendixToc {line-height: 200%; font-weight:bold;}
.index-item, .index-subitem, .index-subsubitem {display:block}
.caption td.id{font-weight: bold; white-space: nowrap; }
table.caption {text-align:center;}
h1.partHead{text-align: center}
p.bibitem { text-indent: -2em; margin-left: 2em; margin-top:0.6em; margin-bottom:0.6em; }
p.bibitem-p { text-indent: 0em; margin-left: 2em; margin-top:0.6em; margin-bottom:0.6em; }
.paragraphHead, .likeparagraphHead { margin-top:2em; font-weight: bold;}
.subparagraphHead, .likesubparagraphHead { font-weight: bold;}
.quote {margin-bottom:0.25em; margin-top:0.25em; margin-left:1em; margin-right:1em; text-align:\jmathustify;}
.verse{white-space:nowrap; margin-left:2em}
div.maketitle {text-align:center;}
h2.titleHead{text-align:center;}
div.maketitle{ margin-bottom: 2em; }
div.author, div.date {text-align:center;}
div.thanks{text-align:left; margin-left:10%; font-size:85%; font-style:italic; }
div.author{white-space: nowrap;}
.quotation {margin-bottom:0.25em; margin-top:0.25em; margin-left:1em; }
h1.partHead{text-align: center}
.sectionToc, .likesectionToc {margin-left:2em;}
.subsectionToc, .likesubsectionToc {margin-left:4em;}
.subsubsectionToc, .likesubsubsectionToc {margin-left:6em;}
.frenchb-nbsp{font-size:75%;}
.frenchb-thinspace{font-size:75%;}
.figure img.graphics {margin-left:10%;}
/* end css.sty */

\title{Integration sur un intervalle quelconque : fonctions `a valeurs
complexes}
\author{}
\date{}

\begin{document}
\maketitle

\textbf{Warning: 
requires JavaScript to process the mathematics on this page.\\ If your
browser supports JavaScript, be sure it is enabled.}

\begin{center}\rule{3in}{0.4pt}\end{center}

{[}
{[}
{[}{]}
{[}

\subsubsection{9.6 Intégration sur un intervalle quelconque~: fonctions
à valeurs complexes}

\paragraph{9.6.1 Fonctions à valeurs complexes intégrables}

Définition~9.6.1 Soit I un intervalle de \mathbb{R}~ et f : I \rightarrow~ \mathbb{C} continue par
morceaux. On dit que f est intégrable sur I si la fonction à valeurs
réelles positives \textbar{}f\textbar{} est intégrable sur I.

Théorème~9.6.1 Soit f : I \rightarrow~ \mathbb{C} continue par morceaux et intégrable. Alors
pour toute suite (J\_n) croissante de segments contenus dans I
de réunion I, la suite (\int ~
\_J\_nf)\_n\in\mathbb{N}~ est convergente. Sa limite est
indépendante de la suite (J\_n) et notée
\int  \_I~f.

Démonstration Soit q \textgreater{} p. On a

\left \textbar{}\int ~
\_J\_qf -\int ~
\_J\_pf\right \textbar{} =
\left \textbar{}\int ~
\_J\_q\diagdownJ\_pf\right
\textbar{}\leq\int ~
\_J\_q\diagdownJ\_p\textbar{}f\textbar{}
=\int ~
\_J\_q\textbar{}f\textbar{}-\int ~
\_J\_p\textbar{}f\textbar{}

(avec un tout petit abus d'écriture en notant J\_q \diagdown
J\_p = {[}a\_q,a\_p{]} \cup
{[}b\_p,b\_q{]} la réunion de deux segments dis\jmathoints ).
Comme \textbar{}f\textbar{} est intégrable, la suite
(\int ~
\_J\_n\textbar{}f\textbar{}) converge, donc c'est une
suite de Cauchy, et par conséquent il en est de même de la suite
(\int  \_J\_n~f) qui est donc
convergente. Si (K\_n) est une autre suite de segments vérifiant
les mêmes propriétés, deux cas se présentent. Si
\forall~n, J\_n \subset~ K\_n~, alors

\left \textbar{}\int ~
\_K\_nf -\int ~
\_J\_nf\right \textbar{} =
\left \textbar{}\int ~
\_K\_n\diagdownJ\_nf\right
\textbar{}\leq\int ~
\_K\_n\diagdownJ\_n\textbar{}f\textbar{}
=\int ~
\_K\_n\textbar{}f\textbar{}-\int ~
\_J\_n\textbar{}f\textbar{}

Mais les deux suites (\int ~
\_J\_n\textbar{}f\textbar{}) et
(\int ~
\_K\_n\textbar{}f\textbar{}) ont la même limite et donc
leur différence tend vers 0. Il en est donc de même de la différence des
suites (\int  \_J\_n~f) et
(\int  \_K\_n~f), qui, étant
convergentes, ont donc la même limite. Si J\_n n'est pas
forcément inclus dans K\_n il suffit d'écrire

lim\\int ~
\_J\_nf =\
lim\int  \_J\_n\cupK\_n~f
= lim\\int ~
\_K\_nf

Corollaire~9.6.2 Soit f : I \rightarrow~ \mathbb{C} continue par morceaux et intégrable.
Alors \left \textbar{}\int ~
\_If\right \textbar{}\leq\\int
 \_I\textbar{}f\textbar{}.

Démonstration Il suffit de passer à la limite à partir de l'inégalité
\left \textbar{}\int ~
\_J\_nf\right \textbar{}
\leq\int ~
\_J\_n\textbar{}f\textbar{}.

Théorème~9.6.3 Soit -\infty~ \textless{} a \textless{} b \leq +\infty~ et f :
{[}a,b{[}\rightarrow~ \mathbb{C} intégrable. Alors la fonction
x\mapsto~\int ~
\_a^xf(t) dt admet la limite \int ~
\_{[}a,b{[}f au point b.

Démonstration Soit a \textless{} x \textless{} y \textless{} b~; on a

\left \textbar{}\int ~
\_a^yf -\int ~
\_a^xf\right \textbar{} =
\left \textbar{}\int ~
\_x^yf\right
\textbar{}\leq\int ~
\_x^y\textbar{}f\textbar{} =\int ~
\_a^y\textbar{}f\textbar{}-\int ~
\_a^x\textbar{}f\textbar{}

Comme l'application
x\mapsto~\int ~
\_a^x\textbar{}f\textbar{} admet une limite au point b,
elle vérifie le critère de Cauchy~: pour tout \epsilon \textgreater{} 0, il
existe c \in {[}a,b{[} tel que c \textless{} x \textless{} y \textless{} b
\rigtharrow~\left \textbar{}\int ~
\_a^y\textbar{}f\textbar{}-\int ~
\_a^x\textbar{}f\textbar{}\right \textbar{}
\textless{} \epsilon~; alors l'inégalité ci dessus montre que
x\mapsto~\int ~
\_a^xf(t) dt vérifie également ce critère de Cauchy, donc
admet une limite au point b. Soit alors b\_n une suite
croissante de limite b. On a

lim\_x\rightarrow~b\\int ~
\_a^xf(t) dt = lim~\_
n\rightarrow~+\infty~\int  \_a^b\_n ~f(t)
dt =\
lim\_n\rightarrow~+\infty~\int ~
\_{[}a,b\_n{]}f =\int ~
\_{[}a,b{[}f

Proposition~9.6.4 Soit I un intervalle de \mathbb{R}~, f : I \rightarrow~ \mathbb{C} continue par
morceaux, intégrable sur I. Alors f est intégrable sur tout intervalle
I' inclus dans I.

Démonstration En effet l'intégrabilité de f équivaut à celle de
\textbar{}f\textbar{}.

Proposition~9.6.5 Soit f : I \rightarrow~ \mathbb{C} et \phi : I \rightarrow~ \mathbb{R}~^+ continues par
morceaux telles que 0 \leq\textbar{}f\textbar{}\leq \phi. Si \phi est intégrable sur
I il en est de même de f et \left
\textbar{}\int  \_I~f\right
\textbar{}\leq\int  \_I~\phi.

Démonstration Evident d'après les définitions.

Corollaire~9.6.6 Soit I un intervalle borné de \mathbb{R}~ et soit f : I \rightarrow~ \mathbb{C}
continue par morceaux et bornée. Alors f est intégrable sur I.

Démonstration Appliquer la proposition précédente avec \phi constante
ma\jmathorant \textbar{}f\textbar{}.

Proposition~9.6.7 Soit I = {[}a,b{]} un segment de \mathbb{R}~, f : I \rightarrow~ \mathbb{C} continue
par morceaux. Alors f est intégrable sur I et
\int  \_I~f =\\int
 \_a^bf. De plus f est intégrable sur {]}a,b{[},
{[}a,b{[} et {]}a,b{]}, toutes ces intégrales étant égales.

Démonstration La fonction \textbar{}f\textbar{} est positive et continue
par morceaux, donc intégrable sur {[}a,b{]}. Donc f l'est également. On
sait alors que \textbar{}f\textbar{} est intégrable sur tout intervalle
inclus dans I et en particulier sur {]}a,b{[}, {[}a,b{[} et {]}a,b{]}~;
il en est donc de même pour f. De plus, si a\_n = a + 1
\over n et b\_n = b - 1 \over
n , J\_n = {[}a\_n,b\_n{]} est une suite
croissante de segments dont la réunion est {]}a,b{[}, donc

\int  \_{]}a,b{[}~f
= lim\\int ~
\_a\_n^b\_n f =\\int
 \_a^bf

par continuité de l'intégrale par rapport à ses bornes. On fait une
démonstration similaire pour {[}a,b{[} avec {[}a,b\_n{]} et
{]}a,b{]} avec {[}a\_n,b{]}. Pour {[}a,b{]}, on prend
a\_n = a et b\_n = b.

Théorème~9.6.8 Soit f,g : I \rightarrow~ \mathbb{C} continues par morceaux, soit \alpha~,\beta~ \in \mathbb{C}. Si
f et g sont intégrables sur I, il en est de même de \alpha~f + \beta~g et on a

\int  \_I~(\alpha~f + \beta~g) =
\alpha~\int  \_I~f +
\beta~\int  \_I~g

Autrement dit, l'ensemble des applications de I dans \mathbb{C} qui sont
intégrables sur I est un sous-espace vectoriel de l'espace vectoriel des
applications de I dans \mathbb{C} et l'application
f\mapsto~\int  \_I~f
est linéaire.

Démonstration L'intégrabilité est évidente à partir de l'inégalité
\textbar{}\alpha~f + \beta~g\textbar{}\leq\textbar{}\alpha~\textbar{}\textbar{}f\textbar{} +
\textbar{}\beta~\textbar{}\textbar{}g\textbar{} et du fait que
\textbar{}f\textbar{} et \textbar{}g\textbar{} étant intégrables, il en
est de même de \textbar{}\alpha~\textbar{}\textbar{}f\textbar{} +
\textbar{}\beta~\textbar{}\textbar{}g\textbar{}. Pour les égalités, il suffit
de prendre une suite (J\_n) croissante de segments de réunion I
et de passer à la limite dans les formules

\int  \_J\_n~(\alpha~f + \beta~g) =
\alpha~\int  \_J\_n~f +
\beta~\int  \_J\_n~g

Proposition~9.6.9 Soit I un intervalle de \mathbb{R}~, f : I \rightarrow~ \mathbb{R}~ continue par
morceaux. Soit a \in I^o. Alors f est intégrable sur I si et
seulement si elle est intégrable sur I\bigcap{]} -\infty~,a{]} et sur I \bigcap
{[}a,+\infty~{[}. Dans ce cas,

\int  \_I~f =\\int
 \_I\bigcap{]}-\infty~,a{]}f +\int ~
\_I\bigcap{[}a,+\infty~{[}f

Démonstration Le résultat similaire dé\jmathà démontré pour
\textbar{}f\textbar{} démontre l'équivalence entre les diverses
intégrabilités. Soit alors J\_n =
{[}a\_n,b\_n{]} une suite croissante de segments de
réunion I. Pour n assez grand, on a a\_n \leq a \leq b\_n car
a est dans l'intérieur de I. Mais ({[}a\_n,a{]}) est une suite
croissante de segments de réunion I\bigcap{]} -\infty~,a{]} et
({[}a,b\_n{]}) est une suite croissante de segments de réunion I
\bigcap {[}a,+\infty~{[}. On peut donc passer à la limite dans la formule
\int  \_{[}a\_n,b\_n{]}~f
=\int  \_{[}a\_n,a{]}~f
+\int  \_{[}a,b\_n{]}~f, et on
obtient

\int  \_I~f =\\int
 \_I\bigcap{]}-\infty~,a{]}f +\int ~
\_I\bigcap{[}a,+\infty~{[}

\paragraph{9.6.2 Décomposition des fonctions à valeurs complexes}

Soit x \in \mathbb{R}~. On pose x^+ = max~(x,0)
et x^- = max~(-x,0). On a
x^+,x^-\in \mathbb{R}~^+, x = x^+ -
x^-, \textbar{}x\textbar{} = x^+ + x^-,
x^+ = 1 \over 2 (\textbar{}x\textbar{} +
x) et x^- = 1 \over 2
(\textbar{}x\textbar{}- x).

Remarque~9.6.1 Si f : I \rightarrow~ \mathbb{R}~, on peut ainsi lui associer des fonctions
f^+ et f^- à valeurs dans \mathbb{R}~^+. On a
f^+,f^-\in \mathbb{R}~^+, f = f^+ -
f^-, \textbar{}f\textbar{} = f^+ + f^-,
f^+ = 1 \over 2 (\textbar{}f\textbar{} +
f) et f^- = 1 \over 2
(\textbar{}f\textbar{}- f). Ces deux dernières formules montrent
clairement que si f est continue par morceaux, il en est de même de
f^+ et f^-.

Théorème~9.6.10 Soit f : I \rightarrow~ \mathbb{R}~ continue par morceaux. Alors f est
intégrable sur I si et seulement si les fonctions (à valeurs réelles
positives) f^+ et f^- le sont. Dans ce cas

\int  \_I~f =\\int
 \_If^+ -\int ~
\_If^-\text et
\int  \_I~\textbar{}f\textbar{}
=\int  \_If^+~
+\int  \_If^-~

Démonstration Si f est intégrable sur I, il en est de même pour
\textbar{}f\textbar{} et donc pour f^+ et f^-
puisque 0 \leq f^+ \leq\textbar{}f\textbar{} et 0 \leq
f^-\leq\textbar{}f\textbar{}. Inversement, si f^+ et
f^- sont intégrables, leur différence f l'est également. Les
formules proviennent de la linéarité de l'intégrale.

Théorème~9.6.11 Soit f : I \rightarrow~ \mathbb{C} continue par morceaux. Alors f est
intégrable sur I si et seulement si les fonctions (à valeurs réelles)
\mathrmRe~f et
\mathrmIm~f le sont. Dans ce
cas

\int  \_I~f =\\int
 \_I \mathrmRe~f +
i\int  \_I~\
\mathrmImf,\quad
\int  \_I\overlinef~ =
\overline\int  \_If~

Démonstration Si f est intégrable sur I, il en est de même pour
\textbar{}f\textbar{} et donc pour
\mathrmRe~f et
\mathrmIm~f puisque 0
\leq\textbar{}\mathrmRe~f\textbar{}\leq\textbar{}f\textbar{}
et 0
\leq\textbar{}\mathrmIm~f\textbar{}\leq\textbar{}f\textbar{}.
Inversement, si \mathrmRe~f
et \mathrmIm~f sont
intégrables, alors f =\
\mathrmRef +
i\mathrmIm~f l'est
également. Les formules proviennent de la linéarité de l'intégrale.

Remarque~9.6.2 La combinaison de ces deux théorèmes peut permettre de
ramener un problème sur des fonctions à valeurs complexes à des
problèmes sur des fonctions à valeurs réelles positives.

\paragraph{9.6.3 Convention et relation de Chasles}

Définition~9.6.2 Soit I un intervalle de \mathbb{R}~, f : I \rightarrow~ \mathbb{C} continue par
morceaux et intégrable. Soit a,b \in\overlineI. Alors
on posera

\int  \_a^b~f(t) dt =
\left \ \cases
\int  \_{]}a,b{[}~f &si a \textless{} b
\cr 0 &si a = b \cr
-\int  \_{]}b,a{[}~f&si b \textless{} a
 \right .

La définition a bien un sens puisque f est intégrable sur {]}a,b{[}\subset~ I
ou {]}b,a{[}\subset~ I suivant le cas.

Théorème~9.6.12 Soit I un intervalle de \mathbb{R}~, f : I \rightarrow~ \mathbb{C} continue par
morceaux et intégrable. Soit a,b,c \in\overlineI. Alors
on a

\int  \_a^c~f
=\int  \_a^b~f
+\int  \_b^c~f

Démonstration Etudier toutes les positions relatives de a,b et c.

\paragraph{9.6.4 Règles de comparaison}

Théorème~9.6.13 Soit f : {[}a,b{[}\rightarrow~ \mathbb{C} continue par morceaux et g :
{[}a,b{[}\rightarrow~ \mathbb{R}~^+ continue par morceaux, positive et intégrable.
On suppose qu'au voisinage de b on a f = O(g) (resp. f = o(g)). Alors f
est intégrable sur {[}a,b{[} et \int ~
\_{[}x,b{[}f(t) dt = O(\int ~
\_{[}x,b{[}g(t) dt) (resp. \int ~
\_{[}x,b{[}f(t) dt = o(\int ~
\_{[}x,b{[}g(t) dt))

Démonstration On a en effet \textbar{}f\textbar{} = O(g) (resp.
\textbar{}f\textbar{} = o(g)) et \left
\textbar{}\int ~
\_{[}x,b{[}f\right
\textbar{}\leq\int ~
\_{[}x,b{[}\textbar{}f\textbar{}. Il suffit donc d'appliquer le
théorème de comparaison à \textbar{}f\textbar{} et g.

Remarque~9.6.3 Il suffit pour appliquer le théorème précédent que la
condition de positivité de g soit vérifiée dans un voisinage de b.

\paragraph{9.6.5 Espaces de fonctions continues}

Théorème~9.6.14 Soit I un intervalle de \mathbb{R}~. L'ensemble des fonctions
continues et intégrables sur I à valeurs complexes est un sous-espace
vectoriel de l'espace C(I, \mathbb{C}). L'application
f\mapsto~\\textbar{}f\\textbar{}\_1
=\int  \_I~\textbar{}f\textbar{} est une
norme sur cet espace (appelée la norme de la convergence en moyenne).

Démonstration Vérification immédiate à partir des résultats précédents.

Théorème~9.6.15 Soit I un intervalle de \mathbb{R}~. L'ensemble des fonctions
continues à valeurs complexes dont le carré est intégrable sur I est un
sous-espace vectoriel de l'espace C(I, \mathbb{C}). L'application
(f,g)\mapsto~(f\mathrel∣g)
=\int  \_I\overlinef~g
est un produit scalaire hermitien sur cet espace. En particulier,
l'application
f\mapsto~\\textbar{}f\\textbar{}\_2
= (f∣f)^1\diagup2 est une norme sur cet
espace et on a l'inégalité de Cauchy-Schwarz
\textbar{}(f∣g)\textbar{}\leq\\textbar{}
f\\textbar{}\_2\\textbar{}g\\textbar{}\_2.

Démonstration Il est clair que si f est de carré intégrable, il en est
de même de \alpha~f pour \alpha~ \in \mathbb{C}. De plus l'inégalité élémentaire \textbar{}f +
g\textbar{}^2 \leq 2\textbar{}f\textbar{}^2 +
2\textbar{}g\textbar{}^2 montre que si f et g sont de carré
intégrables, il en est de même de f + g. Comme de surcroît il existe des
fonctions de carré intégrables (par exemple la fonction nulle),
celles-ci forment un sous-espace vectoriel de C(I, \mathbb{C}). L'inégalité
élémentaire \textbar{}\overlinefg\textbar{}\leq 1
\over 2 \textbar{}f\textbar{}^2 + 1
\over 2 \textbar{}g\textbar{}^2 montre que
si f et g sont de carré intégrables, \overlinegf est
intégrable ce qui permet de définir (f∣g)
=\int  \_I\overlinef~g.
L'application est visiblement sesquilinéaire hermitienne, on a
(f∣f) =\int ~
\_I\textbar{}f\textbar{}^2 ≥ 0 avec égalité si et
seulement si \textbar{}f\textbar{}^2 = 0, soit f = 0, puisque
f est continue. Les autres affirmations sont des conséquences des
résultats sur les produits scalaires hermitiens.

\paragraph{9.6.6 Notion d'intégrale impropre}

Définition~9.6.3 Soit -\infty~ \textless{} a \textless{} b \leq +\infty~ et f :
{[}a,b{[}\rightarrow~ E continue par morceaux. On dit que l'intégrale
\int  \_a^b~f(t) dt converge si
existe
lim\_x\rightarrow~b,x\textless{}b~\\int
 \_a^xf(t) dt. Dans ce cas on pose
\int  \_a^b~f(t) dt
=\
lim\_x\rightarrow~b,x\textless{}b\int ~
\_a^xf(t) dt.

On a une notion similaire avec -\infty~\leq a \textless{} b \textless{} +\infty~ et f
:{]}a,b{]} \rightarrow~ E continue par morceaux.

Remarque~9.6.4 Si l'intégrale ne converge pas, elle est dite divergente.
Si b \textless{} +\infty~ et si f est la restriction à {[}a,b{[} d'une
fonction réglée sur {[}a,b{]}, alors l'application
x\mapsto~\int ~
\_a^xf(t) dt est continue au point b~; l'intégrale impropre
est donc convergente et la valeur de l'intégrale impropre est donc la
valeur de l'intégrale, si bien qu'il n'y a pas d'ambiguïté dans la
notation \int  \_a^b~f(t) dt~; dans
ce cas nous parlerons d'une intégrale faussement impropre. Un exemple
typique est celui de \int  \_0^1~
sin t \over t~ dt qui est a
priori impropre en 0, mais qui est la restriction à {]}0,1{]} de la
fonction continue f(t) = \left \
\cases  sin~ t
\over t &si t\neq~0
\cr 1 &si t = 0 \cr 
\right ..

Proposition~9.6.16 Soit f : {[}a,b{[}\rightarrow~ E une fonction continue par
morceaux et c \in {[}a,b{[}. Alors l'intégrale \\int
 \_a^bf(t) dt converge si et seulement si~l'intégrale
\int  \_c^b~f(t) dt converge.

Démonstration On a \int  \_a^x~f(t)
dt =\int  \_a^c~f(t) dt
+\int  \_c^x~f(t) dt ce qui montre
que \int  \_a^x~f(t) dt a une
limite en b si et seulement si~\int ~
\_c^xf(t) dt en a une.

Remarque~9.6.5 Cette propriété montre que si f : {[}a,b{[}\rightarrow~ E est une
fonction continue par morceaux, la convergence de
\int  \_a^b~f(t) dt ne dépend que
de la restriction de f à un voisinage de b~; il s'agit donc d'une notion
locale en b.

Théorème~9.6.17 Si f est intégrable sur {[}a,b{[}, alors
\int  \_a^b~f(t) dt converge. Mais
la réciproque est fausse dans le cas général (mais vraie pour les
fonctions à valeurs dans \mathbb{R}~^+).

Démonstration On a vu que si f est intégrable sur {[}a,b{[}, alors
x\mapsto~\int ~
\_a^xf(t) dt admet la limite \int ~
\_If au point b. L'exemple suivant montre que la réciproque est
fausse.

Exemple~9.6.1 Etude de l'intégrale \int ~
\_1^+\infty~ sin~ t
\over t^\alpha~ dt pour \alpha~ \textgreater{} 0. On a
 sin t \over t^\alpha~~
= O( 1 \over t^\alpha~ ), donc si \alpha~
\textgreater{} 1 la fonction est intégrable.

Si 0 \textless{} \alpha~ \leq 1, on a après intégration par parties

\int  \_1^x~
sin t \over t^\alpha~~ dt
= cos 1 - \cos~ x
\over x^\alpha~ +\int ~
\_1^x cos~ t
\over t^\alpha~+1 dt

Mais lim\_x\rightarrow~+\infty~~
cos x \over x^\alpha~~ =
0 et la fonction t\mapsto~
cos t \over t^\alpha~+1~
est intégrable puisque  cos~ t
\over t^\alpha~+1 = O( 1 \over
t^\alpha~+1 ). On en déduit que le terme de droite de l'égalité
ci dessus a une limite en + \infty~, et donc le terme de gauche aussi. En
conséquence, l'intégrale impropre \int ~
\_1^+\infty~ sin~ t
\over t^\alpha~ dt converge. Montrons que la
fonction n'est pas intégrable~; on a

\begin{align*} \int ~
\_1^x \textbar{}sin~ t\textbar{}
\over t^\alpha~ & ≥& \\int
 \_1^x sin ^2~t
\over t^\alpha~ dt \%&
\\ & =& 1 \over 2
\int  \_1^x~ 1
- cos~ (2t) \over
t^\alpha~ dt \%& \\ & =& 1
\over 2 \int ~
\_1^x 1 \over t^\alpha~ dt - 1
\over 2 \int ~
\_1^x cos~ (2t)
\over t^\alpha~ dt\%&
\\ \end{align*}

Mais l'intégrale \int  \_1^x~ 1
\over t^\alpha~ dt admet pour limite + \infty~ (car \alpha~ \leq
1), alors que l'intégrale \int ~
\_1^x cos~ (2t)
\over t^\alpha~ dt converge (même méthode
d'intégration par parties). On en déduit que
lim\_x\rightarrow~+\infty~~\\int
 \_1^x sin ^2~t
\over t^\alpha~ dt = +\infty~ et donc aussi
lim\_x\rightarrow~+\infty~~\\int
 \_1^x \textbar{} sin~
t\textbar{} \over t^\alpha~ dt = +\infty~.

{[}
{[}
{[}
{[}

\end{document}

% \documentclass[]{article}
\usepackage[T1]{fontenc}
\usepackage{lmodern}
\usepackage{amssymb,amsmath}
\usepackage{ifxetex,ifluatex}
\usepackage{fixltx2e} % provides \textsubscript
% use upquote if available, for straight quotes in verbatim environments
\IfFileExists{upquote.sty}{\usepackage{upquote}}{}
\ifnum 0\ifxetex 1\fi\ifluatex 1\fi=0 % if pdftex
  \usepackage[utf8]{inputenc}
\else % if luatex or xelatex
  \ifxetex
    \usepackage{mathspec}
    \usepackage{xltxtra,xunicode}
  \else
    \usepackage{fontspec}
  \fi
  \defaultfontfeatures{Mapping=tex-text,Scale=MatchLowercase}
  \newcommand{\euro}{€}
\fi
% use microtype if available
\IfFileExists{microtype.sty}{\usepackage{microtype}}{}
\usepackage{graphicx}
% Redefine \includegraphics so that, unless explicit options are
% given, the image width will not exceed the width of the page.
% Images get their normal width if they fit onto the page, but
% are scaled down if they would overflow the margins.
\makeatletter
\def\ScaleIfNeeded{%
  \ifdim\Gin@nat@width>\linewidth
    \linewidth
  \else
    \Gin@nat@width
  \fi
}
\makeatother
\let\Oldincludegraphics\includegraphics
{%
 \catcode`\@=11\relax%
 \gdef\includegraphics{\@ifnextchar[{\Oldincludegraphics}{\Oldincludegraphics[width=\ScaleIfNeeded]}}%
}%
\ifxetex
  \usepackage[setpagesize=false, % page size defined by xetex
              unicode=false, % unicode breaks when used with xetex
              xetex]{hyperref}
\else
  \usepackage[unicode=true]{hyperref}
\fi
\hypersetup{breaklinks=true,
            bookmarks=true,
            pdfauthor={},
            pdftitle={Developpements asymptotiques et analyse numerique},
            colorlinks=true,
            citecolor=blue,
            urlcolor=blue,
            linkcolor=magenta,
            pdfborder={0 0 0}}
\urlstyle{same}  % don't use monospace font for urls
\setlength{\parindent}{0pt}
\setlength{\parskip}{6pt plus 2pt minus 1pt}
\setlength{\emergencystretch}{3em}  % prevent overfull lines
\setcounter{secnumdepth}{0}
 
/* start css.sty */
.cmr-5{font-size:50%;}
.cmr-7{font-size:70%;}
.cmmi-5{font-size:50%;font-style: italic;}
.cmmi-7{font-size:70%;font-style: italic;}
.cmmi-10{font-style: italic;}
.cmsy-5{font-size:50%;}
.cmsy-7{font-size:70%;}
.cmex-7{font-size:70%;}
.cmex-7x-x-71{font-size:49%;}
.msbm-7{font-size:70%;}
.cmtt-10{font-family: monospace;}
.cmti-10{ font-style: italic;}
.cmbx-10{ font-weight: bold;}
.cmr-17x-x-120{font-size:204%;}
.cmsl-10{font-style: oblique;}
.cmti-7x-x-71{font-size:49%; font-style: italic;}
.cmbxti-10{ font-weight: bold; font-style: italic;}
p.noindent { text-indent: 0em }
td p.noindent { text-indent: 0em; margin-top:0em; }
p.nopar { text-indent: 0em; }
p.indent{ text-indent: 1.5em }
@media print {div.crosslinks {visibility:hidden;}}
a img { border-top: 0; border-left: 0; border-right: 0; }
center { margin-top:1em; margin-bottom:1em; }
td center { margin-top:0em; margin-bottom:0em; }
.Canvas { position:relative; }
li p.indent { text-indent: 0em }
.enumerate1 {list-style-type:decimal;}
.enumerate2 {list-style-type:lower-alpha;}
.enumerate3 {list-style-type:lower-roman;}
.enumerate4 {list-style-type:upper-alpha;}
div.newtheorem { margin-bottom: 2em; margin-top: 2em;}
.obeylines-h,.obeylines-v {white-space: nowrap; }
div.obeylines-v p { margin-top:0; margin-bottom:0; }
.overline{ text-decoration:overline; }
.overline img{ border-top: 1px solid black; }
td.displaylines {text-align:center; white-space:nowrap;}
.centerline {text-align:center;}
.rightline {text-align:right;}
div.verbatim {font-family: monospace; white-space: nowrap; text-align:left; clear:both; }
.fbox {padding-left:3.0pt; padding-right:3.0pt; text-indent:0pt; border:solid black 0.4pt; }
div.fbox {display:table}
div.center div.fbox {text-align:center; clear:both; padding-left:3.0pt; padding-right:3.0pt; text-indent:0pt; border:solid black 0.4pt; }
div.minipage{width:100%;}
div.center, div.center div.center {text-align: center; margin-left:1em; margin-right:1em;}
div.center div {text-align: left;}
div.flushright, div.flushright div.flushright {text-align: right;}
div.flushright div {text-align: left;}
div.flushleft {text-align: left;}
.underline{ text-decoration:underline; }
.underline img{ border-bottom: 1px solid black; margin-bottom:1pt; }
.framebox-c, .framebox-l, .framebox-r { padding-left:3.0pt; padding-right:3.0pt; text-indent:0pt; border:solid black 0.4pt; }
.framebox-c {text-align:center;}
.framebox-l {text-align:left;}
.framebox-r {text-align:right;}
span.thank-mark{ vertical-align: super }
span.footnote-mark sup.textsuperscript, span.footnote-mark a sup.textsuperscript{ font-size:80%; }
div.tabular, div.center div.tabular {text-align: center; margin-top:0.5em; margin-bottom:0.5em; }
table.tabular td p{margin-top:0em;}
table.tabular {margin-left: auto; margin-right: auto;}
div.td00{ margin-left:0pt; margin-right:0pt; }
div.td01{ margin-left:0pt; margin-right:5pt; }
div.td10{ margin-left:5pt; margin-right:0pt; }
div.td11{ margin-left:5pt; margin-right:5pt; }
table[rules] {border-left:solid black 0.4pt; border-right:solid black 0.4pt; }
td.td00{ padding-left:0pt; padding-right:0pt; }
td.td01{ padding-left:0pt; padding-right:5pt; }
td.td10{ padding-left:5pt; padding-right:0pt; }
td.td11{ padding-left:5pt; padding-right:5pt; }
table[rules] {border-left:solid black 0.4pt; border-right:solid black 0.4pt; }
.hline hr, .cline hr{ height : 1px; margin:0px; }
.tabbing-right {text-align:right;}
span.TEX {letter-spacing: -0.125em; }
span.TEX span.E{ position:relative;top:0.5ex;left:-0.0417em;}
a span.TEX span.E {text-decoration: none; }
span.LATEX span.A{ position:relative; top:-0.5ex; left:-0.4em; font-size:85%;}
span.LATEX span.TEX{ position:relative; left: -0.4em; }
div.float img, div.float .caption {text-align:center;}
div.figure img, div.figure .caption {text-align:center;}
.marginpar {width:20%; float:right; text-align:left; margin-left:auto; margin-top:0.5em; font-size:85%; text-decoration:underline;}
.marginpar p{margin-top:0.4em; margin-bottom:0.4em;}
.equation td{text-align:center; vertical-align:middle; }
td.eq-no{ width:5%; }
table.equation { width:100%; } 
div.math-display, div.par-math-display{text-align:center;}
math .texttt { font-family: monospace; }
math .textit { font-style: italic; }
math .textsl { font-style: oblique; }
math .textsf { font-family: sans-serif; }
math .textbf { font-weight: bold; }
.partToc a, .partToc, .likepartToc a, .likepartToc {line-height: 200%; font-weight:bold; font-size:110%;}
.chapterToc a, .chapterToc, .likechapterToc a, .likechapterToc, .appendixToc a, .appendixToc {line-height: 200%; font-weight:bold;}
.index-item, .index-subitem, .index-subsubitem {display:block}
.caption td.id{font-weight: bold; white-space: nowrap; }
table.caption {text-align:center;}
h1.partHead{text-align: center}
p.bibitem { text-indent: -2em; margin-left: 2em; margin-top:0.6em; margin-bottom:0.6em; }
p.bibitem-p { text-indent: 0em; margin-left: 2em; margin-top:0.6em; margin-bottom:0.6em; }
.subsectionHead, .likesubsectionHead { margin-top:2em; font-weight: bold;}
.sectionHead, .likesectionHead { font-weight: bold;}
.quote {margin-bottom:0.25em; margin-top:0.25em; margin-left:1em; margin-right:1em; text-align:justify;}
.verse{white-space:nowrap; margin-left:2em}
div.maketitle {text-align:center;}
h2.titleHead{text-align:center;}
div.maketitle{ margin-bottom: 2em; }
div.author, div.date {text-align:center;}
div.thanks{text-align:left; margin-left:10%; font-size:85%; font-style:italic; }
div.author{white-space: nowrap;}
.quotation {margin-bottom:0.25em; margin-top:0.25em; margin-left:1em; }
h1.partHead{text-align: center}
.sectionToc, .likesectionToc {margin-left:2em;}
.subsectionToc, .likesubsectionToc {margin-left:4em;}
.sectionToc, .likesectionToc {margin-left:6em;}
.frenchb-nbsp{font-size:75%;}
.frenchb-thinspace{font-size:75%;}
.figure img.graphics {margin-left:10%;}
/* end css.sty */

\title{Developpements asymptotiques et analyse numerique}
\author{}
\date{}

\begin{document}
\maketitle

\textbf{Warning: 
requires JavaScript to process the mathematics on this page.\\ If your
browser supports JavaScript, be sure it is enabled.}

\begin{center}\rule{3in}{0.4pt}\end{center}

[
[
[]
[

\section{9.7 Développements asymptotiques et analyse numérique}

\subsection{9.7.1 La formule d'Euler-Mac Laurin}

Proposition~9.7.1 Il existe une unique famille de polynômes
B_n(X) (polynômes de Bernoulli) dans \mathbb{R}~[X] vérifiant les
relations

\begin{itemize}
\itemsep1pt\parskip0pt\parsep0pt
\item
  (i) B_0(X) = 1, B_1(X) = X - 1
  \over 2 ,
\item
  (ii) B_n'(X) = nB_n-1(X) pour n ≥ 1
\item
  (iii) B_2n+1(0) = B_2n+1(1) = 0 pour n ≥ 1
\end{itemize}

Démonstration La relation (ii) définit B_n à une constante près
et la relation (iii) fixe les deux constantes d'intégration qui se sont
introduites pour le passage de B_2n-1 à B_2n+1.

Théorème~9.7.2

\begin{itemize}
\itemsep1pt\parskip0pt\parsep0pt
\item
  (i) B_n est un polynôme normalisé de degré n.
\item
  (ii) On a B_n(1 - X) = (-1)^nB_n(X) et en
  particulier B_2n+1( 1 \over 2 ) = 0,
  B_2n(1) = B_2n(0) (noté b_2n)
\item
  (iii) B_n(X + 1) - B_n(X) = nX^n-1
\end{itemize}

Démonstration (i) est évident par récurrence à partir de
B_n'(X) = nB_n-1(X). Pour démontrer (ii), il suffit de
démontrer que, si l'on pose C_n(X) =
(-1)^nB_n(1 - X), la suite (C_n) vérifie
les mêmes relations que la suite (B_n(X)), ce qui est immédiat.
On montre (iii) par récurrence sur n. La relation est vérifiée pour n =
1 et si elle est vérifiée pour n - 1, soit P(X) = B_n(X + 1) -
B_n(X) - nX^n-1. On a P'(X) = n(B_n-1(X +
1) - B_n-1(X) - (n - 1)X^n-2) = 0 par l'hypothèse de
récurrence. Mais d'autre part P(0) = B_n(1) - B_n(0) =
0 (par définition si n est impair, d'après l'assertion précédente si n
est pair), donc P est le polynôme nul.

Théorème~9.7.3 (formule d'Euler-Mac Laurin). Soit f : [0,1] \rightarrow~ E de
classe C^2n+1. Alors

\begin{align*} \int ~
_0^1f(t) dt& =& 1 \over 2 (f(1) +
f(0)) \%& \\ & \text
& -\\sum
_k=1^n(f^(2k-1)(1) -
f^(2k-1)(0)) b_2k \over (2k)!
\%& \\ & \text & -
1 \over (2n + 1)! \int ~
_0^1f^(2n+1)(t)B_ 2n+1(t) dt \%&
\\ \end{align*}

Démonstration Par récurrence sur n. Pour n = 0, on écrit

\begin{align*} \int ~
_0^1f(t) dt& =& \int ~
_0^1f(t)B_ 0(t) dt = \left
[f(t)B_1(t)\right ]_0^1
-\int  _0^1f'(t)B_ 1~(t)
dt\%& \\ & =& 1 \over
2 (f(1) + f(0)) -\int ~
_0^1f'(t)B_ 1(t) dt \%&
\\ \end{align*}

Si la formule est vérifiée pour n, deux intégrations par parties donnent

\begin{align*} 1 \over (2n + 1)!
\int ~
_0^1f^(2n+1)(t)B_ 2n+1(t)
dt\quad && \%& \\ &
=& 1 \over (2n + 1)! \left
[f^(2n+1)(t) B_2n+2(t) \over 2n +
2 \right ]_0^1 \%&
\\ & \text & - 1
\over (2n + 2)! \int ~
_0^1f^(2n+2)(t)B_ 2n+2(t) dt\%&
\\ & =& b_2n+2
\over (2n + 2)! (f^(2n+1)(1) -
f^(2n+1)(0)) \%& \\ &
\text & -\Biggl ( 1
\over (2n + 2)! \left
[f^(2n+2)(t) B_2n+3(t) \over 2n +
3 \right ]_0^1 \%&
\\ & \text & - 1
\over (2n + 3)! \int ~
_0^1f^(2n+3)(t)B_ 2n+3(t)
dt\Biggr )\%& \\ &
=& b_2n+2 \over (2n + 2)!
(f^(2n+1)(1) - f^(2n+1)(0)) \%&
\\ & \text & + 1
\over (2n + 3)! \int ~
_0^1f^(2n+3)(t)B_ 2n+3(t) dt\%&
\\ \end{align*}

en tenant compte de B_2n+2(0) = B_2n+2(1) =
b_2n+2 et de B_2n+3(0) = B_2n+3(1) = 0.

Remarque~9.7.1 On peut montrer que \forall~~x \in
[0,1], B_n(x)\leq 4e^2\pi~ n!
\over (2\pi~)^n ce qui permet d'avoir une
estimation du reste. Si f est à valeurs réelles, on peut obtenir une
autre estimation du reste en montrant par récurrence que les polynômes
B_n ont les variations suivantes

̲ ̲ ̲ ̲ ̲ ̲ ̲ ̲ ̲ ̲

x

0

1\diagup2

1 ̲ ̲ ̲ ̲ ̲ ̲ ̲ ̲ ̲ ̲

B_4n(x)

b_4n < 0

\nearrow

0

\nearrow

> 0

\searrow

0

\searrow

b_4n < 0 ̲ ̲ ̲ ̲ ̲ ̲ ̲ ̲ ̲ ̲

B_4n+1(x)

0

\searrow

\nearrow

0

\nearrow

\searrow

0 ̲ ̲ ̲ ̲ ̲ ̲ ̲ ̲ ̲ ̲

B_4n+2(x)

b_4n+2 > 0

\searrow

0

\searrow

< 0

\nearrow

0

\nearrow

b_4n+2 > 0 ̲ ̲ ̲ ̲ ̲ ̲ ̲ ̲ ̲ ̲

B_4n+3(x)

0

\nearrow

\searrow

0

\searrow

\nearrow

0 ̲ ̲ ̲ ̲ ̲ ̲ ̲ ̲ ̲ ̲

\includegraphics{cours7x.png}

Ceci montre que les polynômes B_2p - b_2p sont de
signe constant sur [0,1]. On peut donc utiliser la première formule
de la moyenne, ce qui nous donne

\begin{align*} \int ~
_0^1f^(2n+2)(t)B_ 2n+2(t)
dt\quad && \%& \\ & =&
b_2n+2\int ~
_0^1f^(2n+2)(t) dt +\\int
 _0^1f^(2n+2)(t)(B_ 2n+2(t) -
b_2n+2) dt\%& \\ & =&
b_2n+2(f^(2n+1)(1) - f^(2n+1)(0)) \%&
\\ & \text &
+f^(2n+2)(\xi)\int ~
_0^1(B_ 2n+2(t) - b_2n+2) dt \%&
\\ & =&
b_2n+2(f^(2n+1)(1) - f^(2n+1)(0)) -
b_ 2n+2f^(2n+2)(\xi) \%&
\\ \end{align*}

car \int  _0^1B_2n+2~(t)
dt = \left [ B_2n+3(t)
\over 2n+3 \right ]_0^1
= 0. On obtient, en reprenant la démonstration du lemme, la formule sous
la forme

\begin{align*} \int ~
_0^1f(t) dt& =& 1 \over 2 (f(1) +
f(0)) \%& \\ & \text
& -\\sum
_k=1^n+1(f^(2k-1)(1) -
f^(2k-1)(0)) b_2k \over (2k)!
\%& \\ & \text & +
b_2n+2 \over (2n + 1)! f^(2n+2)(\xi)
\%& \\ \end{align*}

Exemple~9.7.1 Appliquons cette formule à f(t) = 1 \over
t+p . On va obtenir

\begin{align*} log~ (p + 1)
- log (p)& =& 1 \over 2~
( 1 \over p + 1 + 1 \over p ) \%&
\\ & \text &
+\sum _k=1^n+1~( 1
\over (p + 1)^2k - 1 \over
p^2k ) b_2k \over 2k \%&
\\ & \text & + (2n
+ 2)b_2n+2 \over \xi_p^2n+3 \%&
\\ \end{align*}

avec \xi_p \in [p,p + 1] et donc \xi_p ∼ p. En sommant
de p = 1 jusque N - 1, on obtient

\begin{align*} log~ N& =&
\sum _p=1^N~ 1
\over p - 1 \over 2 - 1
\over 2N + \\sum
_k=1^n+1( 1 \over N^2k -
1) b_2k \over 2k \%&
\\ & \text & +(2n +
2)b_2n+2 \sum _p=1^N-1~
1 \over \xi_p^2n+3 \%&
\\ \end{align*}

et en utilisant \\sum ~
_p=N^+\infty~ 1 \over
\xi_p^2n+3 = O( 1 \over
N^2n+2 ) on obtient, après amalgame de tous les termes ne
dépendant pas de N en une constante \gamma,

\begin{align*} \\sum
_p=1^N 1 \over p & =&
log N + \gamma + 1 \over 2N~
-\sum _k=1^n b_2k~
\over 2kN^2k + O( 1 \over
N^2n+2 )\%& \\
\end{align*}

\subsection{9.7.2 Calcul approché d'intégrales}

Méthode des trapèzes

Soit f : [a,b] \rightarrow~ \mathbb{R}~ de classe C^2 et p \in \mathbb{N}~^∗.
Pour k \in [0,p] posons a_k = a + k b-a
\over p . On approche la fonction f par la fonction \phi :
[a,b] \rightarrow~ E qui vérifie \forall~~k \in [0,p],
\phi(a_k) = f(a_k) et qui est linéaire sur chaque
intervalle [a_k-1,a_k]. On a immédiatement
\int ~
_a_k-1^a_k\phi = (a_ k -
a_k-1) f(a_k)+f(a_k-1) \over
2 (aire d'un trapèze). D'où, \int ~
_a^b\phi = b-a \over n
\left ( f(a) \over 2
+ \\sum ~
_k=1^p-1f(a_k) + f(b) \over 2
\right ) = T_p avec les notations du subsectione
précédent. On prendra donc comme valeur approchée de I
=\int  _a^b~f,
\overlineI =\int ~
_a^b\phi = T_p.

Majoration de l'erreur~: on cherche à majorer I
-\overlineI =
\int  _a^b~(f -
\phi). Posons g = f - \phi et calculons à l'aide d'une intégration
par parties l'intégrale suivante (en remarquant que la restriction de g
à [a_k-1,a_k] est de classe C^2 avec
g'`= f'')

\begin{align*} \int ~
_a_k-1^a_k f'`(t)(t -
a_k-1)(a_k - t) dt&& \%&
\\ & =& \int ~
_a_k-1^a_k g'`(t)(t -
a_k-1)(a_k - t) dt \%&
\\ & =& \left
[g'(t)(t - a_k-1)(a_k - t)\right
]_a_k-1^a_k 
+\int  _a_k-1^a_k~
g'(t)(2t - a_k-1 - a_k) dt\%&
\\ & =& \int ~
_a_k-1^a_k g'(t)(2t - a_k-1 -
a_k) dt \%& \\ & =&
\left [g(t)(2t - a_k-1 -
a_k)\right
]_a_k-1^a_k  -
2\int  _a_k-1^a_k~
g(t) dt \%& \\ & =&
-2\int ~
_a_k-1^a_k g(t) dt \%&
\\ \end{align*}

puisque g(a_k-1) = g(a_k) = 0. On a donc

\begin{align*} \left
\int ~
_a_k-1^a_k g(t) dt\right
& =& 1 \over 2 \left
\int ~
_a_k-1^a_k f''(t)(t -
a_k-1)(a_k - t) dt\right \%&
\\ & \leq& M_2
\over 2 \int ~
_a_k-1^a_k (t -
a_k-1)(a_k - t) dt \%&
\\ & =& M_2
\over 12 (a_k - a_k-1)^3
= M_2(b - a)^3 \over
12p^3 \%& \\
\end{align*}

En sommant de k = 1 à p, on obtient

I -\overlineI\leq M_2(b
- a)^3 \over 12p^2

Application de la formule d'Euler-Mac Laurin

Soit alors f : [a,b] \rightarrow~ E de classe C^2n+1 et p \in
\mathbb{N}~^∗. Pour k \in [0,p] posons a_k = a + k b-a
\over p ~; appliquons la formule précédente à
t\mapsto~f(a_k-1 + t b-a
\over p ). On obtient alors (avec le changement de
variable x = a_k-1 + t b-a \over p )

\begin{align*} \int ~
_a_k-1^a_k f(x) dx& =& b - a
\over n \int ~
_0^1f(a_ k-1 + t b - a \over p
) dt\%& \\ & =& b - a
\over 2p (f(a_k-1) + f(a_k)) \%&
\\ & -\\sum
_k=1^n (b - a)^2k \over
p^2k (f^(2k-1)(a_ k) -
f^(2k-1)(a_ k-1)) b_2k
\over (2k)! &\%& \\ &
\text & + (b - a)^2n+2
\over p^2n+2(2n + 1)! \rho_n,k \%&
\\ \end{align*}

avec \rho_n,k =\int ~
_0^1f^(2n+1)(a_k-1 + t b-a
\over p )B_2n+1(t) dt. Posons M_2n+1
=\
sup_t\in[a,b]\f^(2n+1)(t)\.
On a alors
\\rho_n,k\ \leq
M_2n+1\int ~
_0^1B_2n+1(t) dt. Sommons
alors les égalités ci dessus, en posant

T_p = b - a \over n \left (
f(a) \over 2 + \\sum
_k=1^p-1f(a_ k) + f(b) \over
2 \right )

on obtient,

\begin{align*} \int ~
_a^bf(x) dx& =& T_ p
-\sum _k=1^n~ (b -
a)^2k \over p^2k
(f^(2k-1)(b) - f^(2k-1)(a)) b_2k
\over (2k)! \%& \\ &
\text & + (b - a)^2n+2
\over p^2n+2(2n + 1)! S_n,p \%&
\\ \end{align*}

avec S_n,p =\
\sum  _k=1^p\rho_n,k~ et
donc \S_n,p\
\leq\\sum ~
_k=1^p\\rho_n,k\
\leq pM_2n+1\int ~
_0^1B_2n+1(t) dt. On obtient
donc

Théorème~9.7.4 Soit f : [a,b] \rightarrow~ E de classe C^2n+1 et p \in
\mathbb{N}~^∗. Pour k \in [0,p] posons a_k = a + k b-a
\over p . Soit T_p = b-a
\over n \left ( f(a)
\over 2 +\
\sum  _k=1^p-1f(a_k~) +
f(b) \over 2 \right ) et M_2n+1
=\
sup_t\in[a,b]\f^(2n+1)(t)\.
Alors

\begin{align*} \int ~
_a^bf(x) dx& =& T_ p
-\sum _k=1^n b_2k~(b
- a)^2k \over p^2k(2k)!
(f^(2k-1)(b) - f^(2k-1)(a))\%&
\\ & \text & + (b
- a)^2n+2 \over p^2n+1(2n + 1)!
R_n,p \%& \\
\end{align*}

avec \R_n,p\
\leq M_2n+1\int ~
_0^1B_2n+1(t) dt.

Remarque~9.7.2 Ce théorème nous donne un développement à un ordre
arbitraire de la différence entre l'intégrale et sa valeur approchée par
la méthode des trapèzes

Méthode de Simpson

La formule d'Euler-Mac Laurin, nous montre que si f est de classe
C^4, on a

I - T_p = \lambda~ \over p^2 + O( 1
\over p^4 )

On a donc également I - T_2p = \lambda~ \over
4p^2 + O( 1 \over p^4 ) puis
4(I - T_2p) - (I - T_p) = O( 1 \over
p^4 ) ou encore

I - 1 \over 3 (4T_2p - T_p) = O(
1 \over p^4 )

Posons donc a_k = a + k b-a \over 2p , on a

\begin{align*} S_p& =& (b - a)
\over 6p (T_p + 4T_2p) \%&
\\ & =& (b - a) \over
6p (f(a) + 4f(a_1) + 2f(a_2) + 4f(a_3) +
\\ldots~\%&
\\ & \text &
+2f(a_2p-2) + 4f(a_2p-1) + f(b)) \%&
\\ \end{align*}

On sait donc que l'on a une majoration du type

I - S_p\leq M \over
p^4

Remarque~9.7.3 On peut montrer qu'en fait I -
S_p\leq M_4(b-a)^5
\over 2880p^4 avec M_4
=\
sup_t\in[a,b]\f^(4)(t)\,
majoration de peu d'intérêt dans la pratique vu la difficulté qu'il y a
habituellement à trouver un majorant de M_4.

Méthode de Romberg

Elle consiste à généraliser la méthode qui nous a fait passer de la
méthode des trapèzes à la méthode de Simpson en utilisant le calcul de
T_p,T_2p,T_4p,\\ldots,T_2^kp~
pour éliminer successivement les termes en  1 \over
p^2 , 1 \over p^4
,\\ldots~, 1
\over p^2k du développement asymptotique
donné par la formule d'Euler-Mac Laurin.

\subsection{9.7.3 La méthode de Laplace}

C'est une méthode classique de recherche d'équivalents d'intégrales
dépendant d'un paramètre (ici n) consistant à remarquer qu'un intégrande
du type f(t)e^ng(t) va privilégier, pour n grand, les valeurs
de t pour lesquelles la fonction g atteint son maximum (car si x
< y, e^nx = o(e^ny)).

Proposition~9.7.5 Soit f :]a,b[\rightarrow~ \mathbb{R}~ continue intégrable sur
]a,b[. Soit g :]a,b[\rightarrow~ \mathbb{R}~ de classe C^2. On suppose que
g atteint son maximum en un point c \in]a,b[ avec g''(c) <
0, f(c)\neq~0 et que \forall~~\eta
> 0,
sup_t-c≥\eta~g(t)
< g(c). Alors, quand n tend vers + \infty~

\int  _a^bf(t)e^ng(t)~
dt ∼ f(c)e^ng(c)\sqrt 2\pi~
\over ng''(c) 

Démonstration Quitte à changer f en - f, on peut supposer f(c)
> 0. Soit \alpha~ tel que 0 < \alpha~
<\
min(g''(c),f(c)) et soit \eta > 0 tel
que t - c\leq \eta \rigtharrow~f(t) - f(c)\leq \alpha~ et
g(t) - g(c) - (t-c)^2 \over 2
g''(c) < \alpha~ (t-c)^2 \over
2 (puisque g'(c) = 0). Sur [c - \eta,c + \eta], on a f(c) - \alpha~
< f(t) < f(c) + \alpha~ et

g(c) + (t - c)^2 \over 2 (g'`(c) - \alpha~) \leq
g(t) \leq g(c) + (t - c)^2 \over 2 (g''(c) +
\alpha~)

On obtient donc

\begin{align*} (f(c) -
\alpha~)e^ng(c)\int ~
_c-\eta^c+\eta exp~
\left (n (t - c)^2 \over 2
(g'`(c) - \alpha~)\right ) dt&& \%&
\\ & \leq& \int ~
_c-\eta^c+\etaf(t)e^ng(t) dt \%&
\\ & \leq& (f(c) +
\alpha~)e^ng(c)\int ~
_c-\eta^c+\eta exp~
\left (n (t - c)^2 \over 2
(g''(c) + \alpha~)\right ) dt\%&
\\ \end{align*}

Mais, si \lambda~ < 0, le changement de variable u =
\sqrt n\lambda~ \over 2
 (t - c) donne

\begin{align*} \int ~
_c-\eta^c+\eta exp~
\left (n\lambda~ (t - c)^2 \over 2
\right ) dt&& \%& \\ &
=& \sqrt 2 \over
n\lambda~ \int  _
-\sqrt n\lambda~ \over
2  \eta^\sqrt n\lambda~
\over 2  \etae^-u^2  du \%&
\\ & ∼& \sqrt 2
\over n\lambda~
\int ~
_-\infty~^+\infty~e^-u^2  du =
\sqrt 2\pi~ \over
n\lambda~ \%& \\
\end{align*}

Posons I_n =
\sqrtne^-ng(c)\\int
 _c-\eta^c+\etaf(t)e^ng(t) dt. On a donc
u_n \leq I_n \leq v_n avec

\begin{align*} u_n& =& (f(c) -
\alpha~)\sqrtn\int ~
_c-\eta^c+\eta exp~
\left (n (t - c)^2 \over 2
(g'`(c) - \alpha~)\right ) dt\%&
\\ & ∼& (f(c) -
\alpha~)\sqrt 2\pi~ \over g''(c) -
\alpha~  \%& \\
\end{align*}

et de même v_n ∼ (f(c) + \alpha~)\sqrt 2\pi~
\over g''(c)+\alpha~ . Donnons nous \epsilon
> 0 et soit \alpha~ tel que

\begin{itemize}
\itemsep1pt\parskip0pt\parsep0pt
\item
  (i) 0 < \alpha~ <\
  min(g''(c),f(c))
\item
  (ii) (f(c) - \alpha~)\sqrt 2\pi~ \over
  g'`(c)-\alpha~  >
  f(c)\sqrt 2\pi~ \over
  g''(c)  - \epsilon \over 2
\item
  (iii) (f(c) + \alpha~)\sqrt 2\pi~ \over
  g'`(c)+\alpha~  <
  f(c)\sqrt 2\pi~ \over
  g''(c)  + \epsilon \over 2
\end{itemize}

On prend le \eta correspondant comme ci-dessus. Alors comme
limu_n~ = (f(c) -
\alpha~)\sqrt 2\pi~ \over
g''(c)-\alpha~  et
limv_n~ = (f(c) +
\alpha~)\sqrt 2\pi~ \over
g''(c)+\alpha~ , il existe N \in \mathbb{N}~ tel que

\begin{align*} n ≥ N \rigtharrow~ f(c)\sqrt
2\pi~ \over g'`(c) - \epsilon
\over 2 < u_n \leq v_n
< f(c)\sqrt 2\pi~ \over
g''(c)  + \epsilon \over 2 & &
\%& \\ \end{align*}

Pour n ≥ N on a donc f(c)\sqrt 2\pi~
\over g'`(c)  - \epsilon
\over 2 < I_n <
f(c)\sqrt 2\pi~ \over
g''(c)  + \epsilon \over 2 .

Soit M =\
sup_t-c≥\etag(t) < g(c). On a alors

\left
\sqrtne^-ng(c)\\int
 _t-c≥\etaf(t)e^ng(t)
dt\right
\leq\sqrtne^-n(g(c)-M)\\int
 _a^bf(t) dt

qui tend vers 0 quand n tend vers + \infty~. Soit donc N' \in \mathbb{N}~ tel que n \leq N'
\rigtharrow~\sqrtne^-n(g(c)-M)\\int
 _a^bf(t) dt < \epsilon
\over 2 . Alors, pour n ≥\
max(N,N'), on a

\begin{align*} f(c)\sqrt 2\pi~
\over g'`(c) - \epsilon& <&
\sqrtne^-ng(c)\\int
 _a^bf(t)e^ng(t) dt\%&
\\ & <&
f(c)\sqrt 2\pi~ \over
g''(c)  + \epsilon \%&
\\ \end{align*}

ce qui montre le résultat.

Remarque~9.7.4 Pour appliquer la méthode précédente, il suffit en fait
qu'il existe un n_0 \in \mathbb{N}~ tel que \int ~
_a^bf(t)e^n_0g(t)
dt converge~: on peut alors écrire, en posant f_1(t) =
f(t)e^n_0g(t)

\begin{align*} \int ~
_a^bf(t)e^ng(t) dt& =&
\int  _a^bf_
1(t)e^(n-n_0)g(t) dt \%&
\\ & ∼&
f_1(c)e^(n-n_0)g(c)\sqrt
2\pi~ \over (n - n_0)g'`(c)
\%& \\ & ∼&
f(c)e^ng(c)\sqrt 2\pi~ \over
ng''(c)  \%& \\
\end{align*}

Exemple~9.7.2 Ecrivons

n! = \Gamma(n + 1) =\int ~
_0^+\infty~t^ne^-t dt =
n^n+1\int ~
_0^+\infty~(ue^-u)^n du

avec le changement de variable t = nu. Pour trouver un équivalent de
\int ~
_0^+\infty~(ue^-u)^n du, on peut appliquer
la méthode de Laplace, en tenant compte de la remarque ci-dessus avec
n_0 = 1. On prend donc f(u) = 1, g(u) =\
log (ue^-u) = -u + log~ u qui
atteint son maximum au point 1 avec g''(1) = -1, g(1) = -1. D'où

\int ~
_0^+\infty~(ue^-u)^n du ∼
e^-n\sqrt 2\pi~ \over n 

et donc n! ∼\sqrt2\pi~n n^n
\over e^n .

[
[
[
[

\end{document}

% \documentclass[]{article}
\usepackage[T1]{fontenc}
\usepackage{lmodern}
\usepackage{amssymb,amsmath}
\usepackage{ifxetex,ifluatex}
\usepackage{fixltx2e} % provides \textsubscript
% use upquote if available, for straight quotes in verbatim environments
\IfFileExists{upquote.sty}{\usepackage{upquote}}{}
\ifnum 0\ifxetex 1\fi\ifluatex 1\fi=0 % if pdftex
  \usepackage[utf8]{inputenc}
\else % if luatex or xelatex
  \ifxetex
    \usepackage{mathspec}
    \usepackage{xltxtra,xunicode}
  \else
    \usepackage{fontspec}
  \fi
  \defaultfontfeatures{Mapping=tex-text,Scale=MatchLowercase}
  \newcommand{\euro}{€}
\fi
% use microtype if available
\IfFileExists{microtype.sty}{\usepackage{microtype}}{}
\ifxetex
  \usepackage[setpagesize=false, % page size defined by xetex
              unicode=false, % unicode breaks when used with xetex
              xetex]{hyperref}
\else
  \usepackage[unicode=true]{hyperref}
\fi
\hypersetup{breaklinks=true,
            bookmarks=true,
            pdfauthor={},
            pdftitle={Generalites sur les integrales impropres},
            colorlinks=true,
            citecolor=blue,
            urlcolor=blue,
            linkcolor=magenta,
            pdfborder={0 0 0}}
\urlstyle{same}  % don't use monospace font for urls
\setlength{\parindent}{0pt}
\setlength{\parskip}{6pt plus 2pt minus 1pt}
\setlength{\emergencystretch}{3em}  % prevent overfull lines
\setcounter{secnumdepth}{0}
 
/* start css.sty */
.cmr-5{font-size:50%;}
.cmr-7{font-size:70%;}
.cmmi-5{font-size:50%;font-style: italic;}
.cmmi-7{font-size:70%;font-style: italic;}
.cmmi-10{font-style: italic;}
.cmsy-5{font-size:50%;}
.cmsy-7{font-size:70%;}
.cmex-7{font-size:70%;}
.cmex-7x-x-71{font-size:49%;}
.msbm-7{font-size:70%;}
.cmtt-10{font-family: monospace;}
.cmti-10{ font-style: italic;}
.cmbx-10{ font-weight: bold;}
.cmr-17x-x-120{font-size:204%;}
.cmsl-10{font-style: oblique;}
.cmti-7x-x-71{font-size:49%; font-style: italic;}
.cmbxti-10{ font-weight: bold; font-style: italic;}
p.noindent { text-indent: 0em }
td p.noindent { text-indent: 0em; margin-top:0em; }
p.nopar { text-indent: 0em; }
p.indent{ text-indent: 1.5em }
@media print {div.crosslinks {visibility:hidden;}}
a img { border-top: 0; border-left: 0; border-right: 0; }
center { margin-top:1em; margin-bottom:1em; }
td center { margin-top:0em; margin-bottom:0em; }
.Canvas { position:relative; }
li p.indent { text-indent: 0em }
.enumerate1 {list-style-type:decimal;}
.enumerate2 {list-style-type:lower-alpha;}
.enumerate3 {list-style-type:lower-roman;}
.enumerate4 {list-style-type:upper-alpha;}
div.newtheorem { margin-bottom: 2em; margin-top: 2em;}
.obeylines-h,.obeylines-v {white-space: nowrap; }
div.obeylines-v p { margin-top:0; margin-bottom:0; }
.overline{ text-decoration:overline; }
.overline img{ border-top: 1px solid black; }
td.displaylines {text-align:center; white-space:nowrap;}
.centerline {text-align:center;}
.rightline {text-align:right;}
div.verbatim {font-family: monospace; white-space: nowrap; text-align:left; clear:both; }
.fbox {padding-left:3.0pt; padding-right:3.0pt; text-indent:0pt; border:solid black 0.4pt; }
div.fbox {display:table}
div.center div.fbox {text-align:center; clear:both; padding-left:3.0pt; padding-right:3.0pt; text-indent:0pt; border:solid black 0.4pt; }
div.minipage{width:100%;}
div.center, div.center div.center {text-align: center; margin-left:1em; margin-right:1em;}
div.center div {text-align: left;}
div.flushright, div.flushright div.flushright {text-align: right;}
div.flushright div {text-align: left;}
div.flushleft {text-align: left;}
.underline{ text-decoration:underline; }
.underline img{ border-bottom: 1px solid black; margin-bottom:1pt; }
.framebox-c, .framebox-l, .framebox-r { padding-left:3.0pt; padding-right:3.0pt; text-indent:0pt; border:solid black 0.4pt; }
.framebox-c {text-align:center;}
.framebox-l {text-align:left;}
.framebox-r {text-align:right;}
span.thank-mark{ vertical-align: super }
span.footnote-mark sup.textsuperscript, span.footnote-mark a sup.textsuperscript{ font-size:80%; }
div.tabular, div.center div.tabular {text-align: center; margin-top:0.5em; margin-bottom:0.5em; }
table.tabular td p{margin-top:0em;}
table.tabular {margin-left: auto; margin-right: auto;}
div.td00{ margin-left:0pt; margin-right:0pt; }
div.td01{ margin-left:0pt; margin-right:5pt; }
div.td10{ margin-left:5pt; margin-right:0pt; }
div.td11{ margin-left:5pt; margin-right:5pt; }
table[rules] {border-left:solid black 0.4pt; border-right:solid black 0.4pt; }
td.td00{ padding-left:0pt; padding-right:0pt; }
td.td01{ padding-left:0pt; padding-right:5pt; }
td.td10{ padding-left:5pt; padding-right:0pt; }
td.td11{ padding-left:5pt; padding-right:5pt; }
table[rules] {border-left:solid black 0.4pt; border-right:solid black 0.4pt; }
.hline hr, .cline hr{ height : 1px; margin:0px; }
.tabbing-right {text-align:right;}
span.TEX {letter-spacing: -0.125em; }
span.TEX span.E{ position:relative;top:0.5ex;left:-0.0417em;}
a span.TEX span.E {text-decoration: none; }
span.LATEX span.A{ position:relative; top:-0.5ex; left:-0.4em; font-size:85%;}
span.LATEX span.TEX{ position:relative; left: -0.4em; }
div.float img, div.float .caption {text-align:center;}
div.figure img, div.figure .caption {text-align:center;}
.marginpar {width:20%; float:right; text-align:left; margin-left:auto; margin-top:0.5em; font-size:85%; text-decoration:underline;}
.marginpar p{margin-top:0.4em; margin-bottom:0.4em;}
.equation td{text-align:center; vertical-align:middle; }
td.eq-no{ width:5%; }
table.equation { width:100%; } 
div.math-display, div.par-math-display{text-align:center;}
math .texttt { font-family: monospace; }
math .textit { font-style: italic; }
math .textsl { font-style: oblique; }
math .textsf { font-family: sans-serif; }
math .textbf { font-weight: bold; }
.partToc a, .partToc, .likepartToc a, .likepartToc {line-height: 200%; font-weight:bold; font-size:110%;}
.chapterToc a, .chapterToc, .likechapterToc a, .likechapterToc, .appendixToc a, .appendixToc {line-height: 200%; font-weight:bold;}
.index-item, .index-subitem, .index-subsubitem {display:block}
.caption td.id{font-weight: bold; white-space: nowrap; }
table.caption {text-align:center;}
h1.partHead{text-align: center}
p.bibitem { text-indent: -2em; margin-left: 2em; margin-top:0.6em; margin-bottom:0.6em; }
p.bibitem-p { text-indent: 0em; margin-left: 2em; margin-top:0.6em; margin-bottom:0.6em; }
.subsectionHead, .likesubsectionHead { margin-top:2em; font-weight: bold;}
.sectionHead, .likesectionHead { font-weight: bold;}
.quote {margin-bottom:0.25em; margin-top:0.25em; margin-left:1em; margin-right:1em; text-align:justify;}
.verse{white-space:nowrap; margin-left:2em}
div.maketitle {text-align:center;}
h2.titleHead{text-align:center;}
div.maketitle{ margin-bottom: 2em; }
div.author, div.date {text-align:center;}
div.thanks{text-align:left; margin-left:10%; font-size:85%; font-style:italic; }
div.author{white-space: nowrap;}
.quotation {margin-bottom:0.25em; margin-top:0.25em; margin-left:1em; }
h1.partHead{text-align: center}
.sectionToc, .likesectionToc {margin-left:2em;}
.subsectionToc, .likesubsectionToc {margin-left:4em;}
.sectionToc, .likesectionToc {margin-left:6em;}
.frenchb-nbsp{font-size:75%;}
.frenchb-thinspace{font-size:75%;}
.figure img.graphics {margin-left:10%;}
/* end css.sty */

\title{Generalites sur les integrales impropres}
\author{}
\date{}

\begin{document}
\maketitle

\textbf{Warning: 
requires JavaScript to process the mathematics on this page.\\ If your
browser supports JavaScript, be sure it is enabled.}

\begin{center}\rule{3in}{0.4pt}\end{center}

[
[
[]
[

\section{9.8 Généralités sur les intégrales impropres}

Remarque~9.8.1 Ce subsectione et ceux qui suivent, qui correspondent aux
anciens programmes de classes préparatoires, font double emploi avec les
subsectiones sur les fonctions intégrables sur un intervalle. Ils ont été
maintenus ici par souci de compatibilité avec certains enseignements
universitaires.

Définition~9.8.1 On dit qu'une fonction est réglée sur un intervalle I
si elle est réglée sur tout segment inclus dans I.

\subsection{9.8.1 Notion d'intégrale impropre}

Définition~9.8.2 Soit -\infty~ < a < b \leq +\infty~ et f :
[a,b[\rightarrow~ E réglée. On dit que l'intégrale \\int
 _a^bf(t) dt converge si existe
lim_x\rightarrow~b,x<b~\\int
 _a^xf(t) dt. Dans ce cas on pose
\int  _a^b~f(t) dt
=\
lim_x\rightarrow~b,x<b\int ~
_a^xf(t) dt. On a une notion similaire avec -\infty~\leq a
< b < +\infty~ et f :]a,b] \rightarrow~ E réglée.

Remarque~9.8.2 Si l'intégrale ne converge pas, elle est dite divergente.
Si b < +\infty~ et si f est la restriction à [a,b[ d'une
fonction réglée sur [a,b], alors l'application
x\mapsto~\int ~
_a^xf(t) dt est continue au point b~; l'intégrale impropre
est donc convergente et la valeur de l'intégrale impropre est donc la
valeur de l'intégrale, si bien qu'il n'y a pas d'ambiguïté dans la
notation \int  _a^b~f(t) dt~; dans
ce cas nous parlerons d'une intégrale faussement impropre. Un exemple
typique est celui de \int  _0^1~
sin t \over t~ dt qui est a
priori impropre en 0, mais qui est la restriction à ]0,1] de la
fonction continue f(t) = \left \
\cases  sin~ t
\over t &si t\neq~0
\cr 1 &si t = 0 \cr 
\right ..

Proposition~9.8.1 Soit f : [a,b[\rightarrow~ E une fonction réglée et c \in
[a,b[. Alors l'intégrale \int ~
_a^bf(t) dt converge si et seulement si~l'intégrale
\int  _c^b~f(t) dt converge.

Démonstration On a \int  _a^x~f(t)
dt =\int  _a^c~f(t) dt
+\int  _c^x~f(t) dt ce qui montre
que \int  _a^x~f(t) dt a une
limite en b si et seulement si~\int ~
_c^xf(t) dt en a une.

Remarque~9.8.3 Cette propriété montre que si f : [a,b[\rightarrow~ E est une
fonction réglée, la convergence de \int ~
_a^bf(t) dt ne dépend que de la restriction de f à un
voisinage de b~; il s'agit donc d'une notion locale en b.

\subsection{9.8.2 Intégrales plusieurs fois impropres}

Définition~9.8.3 Soit -\infty~\leq a < b \leq +\infty~ et f :]a,b[\rightarrow~ E
réglée. On dit que l'intégrale \int ~
_a^bf(t) dt converge si on a les conditions équivalentes
(i) il existe c \in]a,b[ tel que les deux intégrales impropres
\int  _a^c~f(t) dt et
\int  _c^b~f(t) dt convergent (ii)
pour tout c \in]a,b[, les deux intégrales impropres
\int  _a^c~f(t) dt et
\int  _c^b~f(t) dt convergent
(iii) l'application
(x,y)\mapsto~\int ~
_x^yf(t) dt admet une limite quand x tend vers a et y tend
vers b indépendamment l'un de l'autre. On pose alors
\int  _a^b~f(t) dt
=\int  _a^c~f(t) dt
+\int  _c^b~f(t) dt
=\
lim_x\rightarrow~a,y\rightarrow~b\int ~
_x^yf(t) dt.

Démonstration Découle immédiatement de la relation de Chasles.

Remarque~9.8.4 L'intégrale \int ~
_-\infty~^+\infty~t dt diverge alors que
lim_x\rightarrow~+\infty~~\\int
 _-x^xt dt = 0. Il est donc impératif dans (iii)
d'introduire deux variables x et y et de les faire varier
indépendamment.

Définition~9.8.4 Soit -\infty~\leq a_0 < a_1
< \\ldots~
< a_n \leq +\infty~ et f
:]a_0,a_n[\diagdown\a_1,\\ldots,a_n-1\~
\rightarrow~ E une fonction réglée. On dit que l'intégrale
\int ~
_a_0^a_nf(t) dt converge si chacune des
intégrales \int ~
_a_i-1^a_if(t) dt converge. On pose
alors \int  _a^b~f(t) dt
= \\sum ~
_i=1^n\int ~
_a_i-1^a_if(t) dt

Avec ces définitions, toutes les questions concernant des intégrales
plusieurs fois impropres se ramènent à des problèmes sur les intégrales
une fois impropres.

\subsection{9.8.3 Opérations sur les intégrales impropres}

Théorème~9.8.2 L'ensemble des fonctions réglées de [a,b[ dans E
telles que l'intégrale impropre \int ~
_a^bf(t) dt converge est un sous espace vectoriel ~de
l'ensemble des fonctions réglées de [a,b[ dans E. L'application
f\mapsto~\int ~
_a^bf(t) dt est linéaire de cet espace vectoriel dans E.

Démonstration Il suffit d'écrire \int ~
_a^x(\alpha~f(t) + \beta~g(t)) dt = \alpha~\int ~
_a^xf(t) dt + \beta~\int ~
_a^xg(t) dt et d'utiliser les théorèmes sur les limites.

Théorème~9.8.3 Soit u : E \rightarrow~ F une application linéaire continue, f :
[a,b[\rightarrow~ E réglée telle que \int ~
_a^bf(t) dt converge. Alors \int ~
_a^bu(f(t)) dt converge et \int ~
_a^bu(f(t)) dt = u(\int ~
_a^bf(t) dt).

Démonstration Il suffit d'écrire \int ~
_a^xu(f(t)) dt = u(\int ~
_a^xf(t) dt) et d'utiliser la continuité de u.

Corollaire~9.8.4 Soit E un espace vectoriel normé~de dimension finie,
(e_1,\\ldots,e_n~)
une base de E, f : [a,b[\rightarrow~ E réglée. On écrit f(t) =
f_1(t)e_1 +
\\ldots~ +
f_n(t)e_n. Alors \int ~
_a^bf(t) dt converge si et seulement si~chacune des
intégrales \int ~
_a^bf_i(t) dt converge et alors

\int  _a^b~f(t) dt =
(\int  _a^bf_ 1~(t)
dt)e_1 +
\\ldots~ +
(\int  _a^bf_ n~(t)
dt)e_n

Démonstration Soit u : K^n \rightarrow~ E,
(x_1,\\ldots,x_n)\mapsto~x_1e_1~
+ \\ldots~ +
x_ne_n. L'application linéaire u est un isomorphisme
d'espaces vectoriels et puisque les espaces sont de dimension finie, u
et u^-1 sont continues. Il suffit alors d'appliquer le
théorème précédent en remarquant que f = u \cdot
(f_1,\\ldots,f_n~)
et que
(f_1,\\ldots,f_n~)
= u^-1 \cdot f.

Changement de variables Soit \phi : [a,b[\rightarrow~ [\alpha~,\beta~[ de classe
\mathcal{C}^1 telle que lim_u\rightarrow~b~\phi(u)
= \beta~. Soit f : [\alpha~,\beta~[\rightarrow~ E continue. Pour x \in [a,b[, on peut alors
écrire \int  _a^x~f(\phi(u))\phi'(u) du
=\int  _\phi(a)^\phi(x)~f(t) dt. On en
déduit par le théorème de composition des limites, que si l'intégrale
\int  _\alpha~^\beta~~f(t) dt converge, alors
l'intégrale \int ~
_a^bf(\phi(u))\phi'(u) du converge et que dans ce cas

\int  _a^b~f(\phi(u))\phi'(u) du
=\int  _\phi(a)^\beta~~f(t) dt

Inversement, si l'on suppose que \phi est un homéomorphisme de [a,b[
sur [\alpha~,\beta~[, on a, pour y \in [\alpha~,\beta~[, \int ~
_\alpha~^yf(t) dt =\int ~
_\phi^-1(\alpha~)^\phi^-1(y) f(\phi(u))\phi'(u) du et
alors la convergence de \int ~
_a^bf(\phi(u))\phi'(u) du implique celle de
\int  _\alpha~^\beta~~f(t) dt et l'égalité
ci-dessus. On retiendra en particulier

Théorème~9.8.5 Soit \phi : [a,b[\rightarrow~ [\alpha~,\beta~[ un homéomorphisme. On
suppose que \phi est de classe \mathcal{C}^1 et que
lim_u\rightarrow~b~\phi(u) = \beta~. Soit f : [\alpha~,\beta~[\rightarrow~
E continue. Alors les deux intégrales impropres
\int  _a^b~f(\phi(u))\phi'(u) du et
\int  _\alpha~^\beta~~f(t) dt sont de même
nature (convergentes ou divergentes) et on a l'égalité

\int  _a^b~f(\phi(u))\phi'(u) du
=\int  _\phi(a)^\beta~~f(t) dt

Intégration par parties Soit f,g : [a,b[\rightarrow~ \mathbb{C} de classe
\mathcal{C}^1. Pour x \in [a,b[, on peut alors faire une intégration
par parties et écrire

\int  _a^x~f(t)g'(t) dt =
\left [f(t)g(t)\right ]_
a^x -\int  _a^x~f'(t)g(t)
dt

Si deux des trois termes qui dépendent de x admettent une limite en b,
alors le troisième aussi et on a alors

\int  _a^b~f(t)g'(t) dt
= lim_ x\rightarrow~b~(f(x)g(x)) - f(a)g(a)
-\int  _a^b~f'(t)g(t) dt

que l'on écrit encore

\int  _a^b~f(t)g'(t) dt =
\left [f(t)g(t)\right ]_
a^b -\int  _a^b~f'(t)g(t)
dt

Sinon, on conserve les intégrales partielles de a à x jusqu'à pouvoir
lever les indéterminations éventuelles.

Remarque~9.8.5 Le lecteur devra faire preuve d'une grande prudence~: une
intégration par parties peut facilement faire passer d'une intégrale
convergente à une intégrale divergente, en particulier avec des
fonctions comme le logarithme.

\subsection{9.8.4 Intégrales et séries~: intégration par paquets}

Théorème~9.8.6 Soit f : [a,b[\rightarrow~ E réglée, (b_n) une suite
strictement croissante de [a,b[ de limite b. On pose pour n ≥ 1,
x_n =\int ~
_b_n-1^b_nf(t) dt. Alors

\begin{itemize}
\itemsep1pt\parskip0pt\parsep0pt
\item
  (i) si l'intégrale \int ~
  _a^bf(t) dt converge, la série
  \\sum ~
  _n≥1x_n converge
\item
  (ii) la réciproque est exacte dans les deux cas suivants

  \begin{itemize}
  \itemsep1pt\parskip0pt\parsep0pt
  \item
    (a) la suite (b_n - b_n-1) est bornée et
    lim_t\rightarrow~b~f(t) = 0
  \item
    (b) E = \mathbb{R}~ et la fonction f est de signe constant sur chaque
    intervalle [b_n-1,b_n].
  \end{itemize}
\end{itemize}

Démonstration (i) On a
\\sum ~
_n=1^Nx_n =\int ~
_b_0^b_Nf(t) dt = F(b_N) avec
F(x) =\int  _b_0^x~f(t)
dt. Puisque l'intégrale converge, la fonction F a une limite au point
b~; le théorème de composition des limites assure alors l'existence de
lim_N\rightarrow~+\infty~F(b_N~), donc la
convergence de la série~; on a d'ailleurs
\\sum ~
_n=1^+\infty~x_n =\int ~
_b_0^bf(t) dt.

(ii)(a) Soit x > b_0 et soit p l'unique entier tel
que b_p-1 \leq x < b_p. On a alors

\sum _n=1^px_ n~
-\\int  ~
_b_0^xf(t) dt =
\\int  ~
_x^b_p f(t) dt

Soit alors \epsilon > 0, K > 0 tel que
\forall~n \in \mathbb{N}~, b_n - b_n-1~ \leq K, c \in
[a,b[ tel que t \in [c,b[\rigtharrow~\
f(t)\ < \epsilon \over 2K
. Pour x > c, on a alors
\\\\sum
 _n=1^px_n -\int ~
_b_0^xf(t) dt\
\leq\int ~
_x^b_p\f(t)\
dt \leq K \epsilon \over 2K = \epsilon \over 2 .
Soit S = \\sum ~
_n=1^+\infty~x_n et N \in \mathbb{N}~ tel que n \rigtharrow~ N
\rigtharrow~\ S
-\\sum ~
_k=1^nx_k\ <
\epsilon \over 2 . Pour x ≥ x_N, on a p ≥ N et donc
pour x > max(c,b_N~) on a

\begin{align*} \S
-\int  _b_0^x~f(t)
dt& \leq& \S
-\sum _k=1^px_
k\ +\
\sum _n=1^px_ n~
-\\int  ~
_b_0^xf(t) dt\\%&
\\ & <& \epsilon
\over 2 + \epsilon \over 2 = \epsilon \%&
\\ \end{align*}

ce qui montre la convergence de l'intégrale.

(ii)(b) La démonstration est similaire. Mais on écrit, en utilisant le
fait que f est de signe constant sur [b_p-1,b_p]

\begin{align*} \\sum
_n=1^px_ n
-\\int  ~
_b_0^xf(t) dt& =&
\int  _x^b_p~
f(t) dt =\int ~
_x^b_p f(t) dt \%&
\\ & \leq& \int ~
_b_p-1^b_p f(t) dt =
\int ~
_b_p-1^b_p f(t) dt\%&
\\ & =& x_p
\%& \\ \end{align*}

Puisque la série converge, limx_n~ = 0
et donc, il existe M tel que n ≥ M \rigtharrow~x_n
< \epsilon \over 2 . Soit N \in \mathbb{N}~ tel que n ≥ N
\rigtharrow~S -\\sum ~
_k=1^nx_k < \epsilon
\over 2 . Pour x ≥\
max(x_N,x_M), on a p ≥\
max(N,M) et donc

S -\int ~
_b_0^xf(t) dt\leqS
-\sum _k=1^px_
k + \\sum
_n=1^px_ n
-\\int  ~
_b_0^xf(t) dt < \epsilon
\over 2 + \epsilon \over 2 = \epsilon

ce qui montre la convergence de l'intégrale.

[
[
[
[

\end{document}

% \documentclass[]{article}
\usepackage[T1]{fontenc}
\usepackage{lmodern}
\usepackage{amssymb,amsmath}
\usepackage{ifxetex,ifluatex}
\usepackage{fixltx2e} % provides \textsubscript
% use upquote if available, for straight quotes in verbatim environments
\IfFileExists{upquote.sty}{\usepackage{upquote}}{}
\ifnum 0\ifxetex 1\fi\ifluatex 1\fi=0 % if pdftex
  \usepackage[utf8]{inputenc}
\else % if luatex or xelatex
  \ifxetex
    \usepackage{mathspec}
    \usepackage{xltxtra,xunicode}
  \else
    \usepackage{fontspec}
  \fi
  \defaultfontfeatures{Mapping=tex-text,Scale=MatchLowercase}
  \newcommand{\euro}{€}
\fi
% use microtype if available
\IfFileExists{microtype.sty}{\usepackage{microtype}}{}
\ifxetex
  \usepackage[setpagesize=false, % page size defined by xetex
              unicode=false, % unicode breaks when used with xetex
              xetex]{hyperref}
\else
  \usepackage[unicode=true]{hyperref}
\fi
\hypersetup{breaklinks=true,
            bookmarks=true,
            pdfauthor={},
            pdftitle={Integrale des fonctions reelles positives},
            colorlinks=true,
            citecolor=blue,
            urlcolor=blue,
            linkcolor=magenta,
            pdfborder={0 0 0}}
\urlstyle{same}  % don't use monospace font for urls
\setlength{\parindent}{0pt}
\setlength{\parskip}{6pt plus 2pt minus 1pt}
\setlength{\emergencystretch}{3em}  % prevent overfull lines
\setcounter{secnumdepth}{0}
 
/* start css.sty */
.cmr-5{font-size:50%;}
.cmr-7{font-size:70%;}
.cmmi-5{font-size:50%;font-style: italic;}
.cmmi-7{font-size:70%;font-style: italic;}
.cmmi-10{font-style: italic;}
.cmsy-5{font-size:50%;}
.cmsy-7{font-size:70%;}
.cmex-7{font-size:70%;}
.cmex-7x-x-71{font-size:49%;}
.msbm-7{font-size:70%;}
.cmtt-10{font-family: monospace;}
.cmti-10{ font-style: italic;}
.cmbx-10{ font-weight: bold;}
.cmr-17x-x-120{font-size:204%;}
.cmsl-10{font-style: oblique;}
.cmti-7x-x-71{font-size:49%; font-style: italic;}
.cmbxti-10{ font-weight: bold; font-style: italic;}
p.noindent { text-indent: 0em }
td p.noindent { text-indent: 0em; margin-top:0em; }
p.nopar { text-indent: 0em; }
p.indent{ text-indent: 1.5em }
@media print {div.crosslinks {visibility:hidden;}}
a img { border-top: 0; border-left: 0; border-right: 0; }
center { margin-top:1em; margin-bottom:1em; }
td center { margin-top:0em; margin-bottom:0em; }
.Canvas { position:relative; }
li p.indent { text-indent: 0em }
.enumerate1 {list-style-type:decimal;}
.enumerate2 {list-style-type:lower-alpha;}
.enumerate3 {list-style-type:lower-roman;}
.enumerate4 {list-style-type:upper-alpha;}
div.newtheorem { margin-bottom: 2em; margin-top: 2em;}
.obeylines-h,.obeylines-v {white-space: nowrap; }
div.obeylines-v p { margin-top:0; margin-bottom:0; }
.overline{ text-decoration:overline; }
.overline img{ border-top: 1px solid black; }
td.displaylines {text-align:center; white-space:nowrap;}
.centerline {text-align:center;}
.rightline {text-align:right;}
div.verbatim {font-family: monospace; white-space: nowrap; text-align:left; clear:both; }
.fbox {padding-left:3.0pt; padding-right:3.0pt; text-indent:0pt; border:solid black 0.4pt; }
div.fbox {display:table}
div.center div.fbox {text-align:center; clear:both; padding-left:3.0pt; padding-right:3.0pt; text-indent:0pt; border:solid black 0.4pt; }
div.minipage{width:100%;}
div.center, div.center div.center {text-align: center; margin-left:1em; margin-right:1em;}
div.center div {text-align: left;}
div.flushright, div.flushright div.flushright {text-align: right;}
div.flushright div {text-align: left;}
div.flushleft {text-align: left;}
.underline{ text-decoration:underline; }
.underline img{ border-bottom: 1px solid black; margin-bottom:1pt; }
.framebox-c, .framebox-l, .framebox-r { padding-left:3.0pt; padding-right:3.0pt; text-indent:0pt; border:solid black 0.4pt; }
.framebox-c {text-align:center;}
.framebox-l {text-align:left;}
.framebox-r {text-align:right;}
span.thank-mark{ vertical-align: super }
span.footnote-mark sup.textsuperscript, span.footnote-mark a sup.textsuperscript{ font-size:80%; }
div.tabular, div.center div.tabular {text-align: center; margin-top:0.5em; margin-bottom:0.5em; }
table.tabular td p{margin-top:0em;}
table.tabular {margin-left: auto; margin-right: auto;}
div.td00{ margin-left:0pt; margin-right:0pt; }
div.td01{ margin-left:0pt; margin-right:5pt; }
div.td10{ margin-left:5pt; margin-right:0pt; }
div.td11{ margin-left:5pt; margin-right:5pt; }
table[rules] {border-left:solid black 0.4pt; border-right:solid black 0.4pt; }
td.td00{ padding-left:0pt; padding-right:0pt; }
td.td01{ padding-left:0pt; padding-right:5pt; }
td.td10{ padding-left:5pt; padding-right:0pt; }
td.td11{ padding-left:5pt; padding-right:5pt; }
table[rules] {border-left:solid black 0.4pt; border-right:solid black 0.4pt; }
.hline hr, .cline hr{ height : 1px; margin:0px; }
.tabbing-right {text-align:right;}
span.TEX {letter-spacing: -0.125em; }
span.TEX span.E{ position:relative;top:0.5ex;left:-0.0417em;}
a span.TEX span.E {text-decoration: none; }
span.LATEX span.A{ position:relative; top:-0.5ex; left:-0.4em; font-size:85%;}
span.LATEX span.TEX{ position:relative; left: -0.4em; }
div.float img, div.float .caption {text-align:center;}
div.figure img, div.figure .caption {text-align:center;}
.marginpar {width:20%; float:right; text-align:left; margin-left:auto; margin-top:0.5em; font-size:85%; text-decoration:underline;}
.marginpar p{margin-top:0.4em; margin-bottom:0.4em;}
.equation td{text-align:center; vertical-align:middle; }
td.eq-no{ width:5%; }
table.equation { width:100%; } 
div.math-display, div.par-math-display{text-align:center;}
math .texttt { font-family: monospace; }
math .textit { font-style: italic; }
math .textsl { font-style: oblique; }
math .textsf { font-family: sans-serif; }
math .textbf { font-weight: bold; }
.partToc a, .partToc, .likepartToc a, .likepartToc {line-height: 200%; font-weight:bold; font-size:110%;}
.chapterToc a, .chapterToc, .likechapterToc a, .likechapterToc, .appendixToc a, .appendixToc {line-height: 200%; font-weight:bold;}
.index-item, .index-subitem, .index-subsubitem {display:block}
.caption td.id{font-weight: bold; white-space: nowrap; }
table.caption {text-align:center;}
h1.partHead{text-align: center}
p.bibitem { text-indent: -2em; margin-left: 2em; margin-top:0.6em; margin-bottom:0.6em; }
p.bibitem-p { text-indent: 0em; margin-left: 2em; margin-top:0.6em; margin-bottom:0.6em; }
.subsectionHead, .likesubsectionHead { margin-top:2em; font-weight: bold;}
.sectionHead, .likesectionHead { font-weight: bold;}
.quote {margin-bottom:0.25em; margin-top:0.25em; margin-left:1em; margin-right:1em; text-align:justify;}
.verse{white-space:nowrap; margin-left:2em}
div.maketitle {text-align:center;}
h2.titleHead{text-align:center;}
div.maketitle{ margin-bottom: 2em; }
div.author, div.date {text-align:center;}
div.thanks{text-align:left; margin-left:10%; font-size:85%; font-style:italic; }
div.author{white-space: nowrap;}
.quotation {margin-bottom:0.25em; margin-top:0.25em; margin-left:1em; }
h1.partHead{text-align: center}
.sectionToc, .likesectionToc {margin-left:2em;}
.subsectionToc, .likesubsectionToc {margin-left:4em;}
.sectionToc, .likesectionToc {margin-left:6em;}
.frenchb-nbsp{font-size:75%;}
.frenchb-thinspace{font-size:75%;}
.figure img.graphics {margin-left:10%;}
/* end css.sty */

\title{Integrale des fonctions reelles positives}
\author{}
\date{}

\begin{document}
\maketitle

\textbf{Warning: 
requires JavaScript to process the mathematics on this page.\\ If your
browser supports JavaScript, be sure it is enabled.}

\begin{center}\rule{3in}{0.4pt}\end{center}

[
[
[]
[

\section{9.9 Intégrale des fonctions réelles positives}

\subsection{9.9.1 Critère de convergence des fonctions réelles positives}

Remarque~9.9.1 Dans toute la suite, sauf précision contraire on
supposera que -\infty~ < a < b \leq +\infty~ si on considère
l'intervalle [a,b[ et que -\infty~\leq a < b < +\infty~ si on
considère l'intervalle ]a,b].

Théorème~9.9.1 Soit f : [a,b[\rightarrow~ \mathbb{R}~ réglée positive. Alors
\int  _a^b~f(t) dt converge si et
seulement si~les intégrales partielles \int ~
_a^xf(t) dt sont majorées~:

\existsM ≥ 0, \\forall~~x \in
[a,b[, \int  _a^x~f(t) dt \leq M

Démonstration Si a \leq x < x' < b, on a
\int  _a^x'~f(t) dt
-\int  _a^x~f(t) dt
=\int  _x^x'~f(t) dt ≥ 0, dont
l'application x\mapsto~\\int
 _a^xf(t) dt est croissante. En conséquence, elle admet
une limite au point b si et seulement si~elle est majorée.

Remarque~9.9.2 Dans le cas d'une fonction réglée f :]a,b] \rightarrow~ \mathbb{R}~
positive, l'application
x\mapsto~\int ~
_x^bf(t) dt est cette fois-ci décroissante~; donc elle
admet une limite (à droite) au point a si et seulement si~elle est
majorée. Dans tous les cas d'intégrales impropres à gauche ou à droite
d'une fonction réelle positive, la convergence est équivalente à la
majoration des intégrales partielles~; dans ce cas, si l'intégrale
diverge, les intégrales partielles tendent vers + \infty~.

Théorème~9.9.2 Soit f,g : [a,b[\rightarrow~ \mathbb{R}~ réglées telles que
\forall~~t \in [a,b[, 0 \leq f(t) \leq g(t). Alors (i) si
l'intégrale \int  _a^b~g(t) dt
converge, l'intégrale \int ~
_a^bf(t) dt converge. (ii) si l'intégrale
\int  _a^b~f(t) dt diverge,
l'intégrale \int  _a^b~g(t) dt
diverge.

Démonstration Pour (i), il suffit de remarquer que
\int  _a^x~f(t) dt
\leq\int  _a^x~g(t) dt, donc que tout
majorant des intégrales partielles de g majore également les intégrales
partielles de f. Quant à (ii), ce n'est que la contraposée de (i).

Remarque~9.9.3 On a déjà vu que la convergence ou la divergence de
l'intégrale était une propriété locale en b. Pour appliquer le résultat
précédent, il suffit donc de supposer que sur un voisinage de b on a 0 \leq
f(t) \leq Kg(t), autrement dit que f(t) = O(g(t)) au voisinage de b.

\subsection{9.9.2 Règles de comparaison}

Théorème~9.9.3 Soit f,g : [a,b[\rightarrow~ \mathbb{R}~ réglées positives. On suppose
qu'au voisinage de b on a f = 0(g) (resp. f = o(g)). Alors (i) si
\int  _a^b~g(t) dt converge,
\int  _a^b~f(t) dt converge
également et \int  _x^b~f(t) dt =
0(\int  _x^b~g(t) dt) (resp.
\int  _x^b~f(t) dt =
o(\int  _x^b~g(t) dt)) (ii) si
\int  _a^b~f(t) dt converge,
\int  _a^b~g(t) dt converge
également et \int  _a^x~f(t) dt =
0(\int  _a^x~g(t) dt) (resp.
\int  _a^x~f(t) dt =
o(\int  _a^x~g(t) dt))

Démonstration Les convergences et divergences découlent immédiatement de
la remarque qui suit le théorème ci-dessus et du fait que f = o(g) \rigtharrow~ f =
O(g). En ce qui concerne la comparaison des restes ou des intégrales
partielles, la démonstration est tout à fait similaire à celle du
théorème analogue sur les séries. Nous allons les faire dans le cas f =
o(g), la démonstration étant analogue pour f = O(g) en changeant \epsilon en K
ou en 2K.

(i) Supposons f = o(g) et \int ~
_a^bg(t) dt convergente. Soit \epsilon > 0. Il
existe c \in [a,b[ tel que t ≥ c \rigtharrow~ 0 \leq f(t) \leq \epsilong(t). Alors pour x ≥ c,
on a (en intégrant l'inégalité de c à b), 0 \leq\\int
 _x^bf(t) dt \leq \epsilon\int ~
_x^bg(t) dt et donc \int ~
_x^bf(t) dt = o(\int ~
_x^bg(t) dt).

(ii) Supposons f = o(g) et \int ~
_a^bf(t) dt divergente. Soit \epsilon > 0. Il existe
c \in [a,b[ tel que t ≥ c \rigtharrow~ 0 \leq f(t) \leq \epsilon \over 2
g(t). Alors pour x ≥ c, on a (en intégrant l'inégalité de c à x),
\int  _c^x~f(t) dt \leq \epsilon
\over 2 \int ~
_c^xg(t) dt, soit encore à l'aide de la relation de
Chasles

0 \leq\int  _a^x~f(t) dt \leq \epsilon
\over 2 \int ~
_a^xg(t) dt + \left
(\int  _a^c~f(t) dt - \epsilon
\over 2 \int ~
_a^cg(t) dt\right )

Mais comme on sait que l'intégrale \int ~
_a^bg(t) dt diverge et que g ≥ 0, on a
lim_x\rightarrow~b\\int ~
_a^xg(t) dt = +\infty~. Donc il existe c' \in [a,b[ tel que x
≥ c' \rigtharrow~ \epsilon \over 2 \int ~
_a^xg(t) dt >\int ~
_a^cf(t) dt - \epsilon \over 2
\int  _a^c~g(t) dt. Alors, pour x
≥ max~(c,c'), on a

0 \leq\int  _a^x~f(t) dt \leq \epsilon
\over 2 \int ~
_a^xg(t) dt + \epsilon \over 2
\int  _a^x~g(t) dt =
\epsilon\int  _a^x~g(t) dt

et donc \int  _a^x~f(t) dt =
o(\int  _a^x~g(t) dt).

Remarque~9.9.4 Il suffit pour appliquer le théorème précédent que la
condition de positivité de f et g soit vérifiée dans un voisinage de b.

Théorème~9.9.4 Soit f,g : [a,b[\rightarrow~ \mathbb{R}~ réglées. On suppose que g est
positive et que au voisinage de b, on a f ∼ g. Alors les deux intégrales
\int  _a^b~f(t) dt et
\int  _a^b~g(t) dt sont de même
nature. Plus précisément (i) Si \int ~
_a^bg(t) dt converge, alors \int ~
_a^bf(t) dt converge également et
\int  _x^b~f(t) dt
∼\int  _x^b~g(t) dt (ii) Si
\int  _a^b~g(t) dt diverge, alors
\int  _a^b~f(t) dt diverge
également et \int  _a^x~f(t) dt
∼\int  _a^x~g(t) dt.

Démonstration Puisque f(t) ∼ g(t), il existe c \in [a,b[ tel que x
> c \rigtharrow~ 1 \over 2 g(t) \leq f(t) \leq 3
\over 2 g(t) ce qui montre que f est positive au
voisinage de b et que l'on a à la fois f = O(g) et g = O(f). Le théorème
précédent assure alors que \int ~
_a^bg(t) dt converge si et seulement
si~\int  _a^b~f(t) dt converge.
Pla\ccons nous dans le cas de convergence. On a
f - g = o(g), on en déduit que l'intégrale
\int  _a^b~f(t) -
g(t) dt converge et que \int ~
_x^bf(t) - g(t) dt =
o(\int  _x^b~g(t) dt). Mais bien
évidemment \left \int ~
_x^bf(t) dt -\int ~
_x^bg(t) dt\right
\leq\int ~
_x^bf(t) - g(t) dt. On a donc
\int  _x^b~f(t) dt
-\int  _x^b~g(t) dt =
o(\int  _x^b~g(t) dt) et donc
\int  _x^b~f(t) dt
∼\int  _x^b~g(t) dt. Dans le cas
de divergence, deux cas se présentent. Si l'intégrale
\int  _a^b~f(t) -
g(t) dt diverge, le théorème précédent assure que
\int  _a^x~f(t) -
g(t) dt = o(\int ~
_a^xg(t) dt)~; si par contre elle converge,
\int  _a^x~f(t) -
g(t) dt admet une limite finie en b alors que
\int  _a^x~g(t) dt tend vers + \infty~
et on a donc encore \int ~
_a^xf(t) - g(t) dt =
o(\int  _a^x~g(t) dt). L'inégalité
\left \int ~
_a^xf(t) dt -\int ~
_a^xg(t) dt\right
\leq\int ~
_a^xf(t) - g(t) dt donne alors
\int  _a^x~f(t) dt
-\int  _a^x~g(t) dt =
o(\int  _a^x~g(t) dt) et donc
\int  _a^x~f(t) dt
∼\int  _a^x~g(t) dt.

\subsection{9.9.3 Exemples fondamentaux}

L'idée générale est d'obtenir une famille de fonctions étalon.

Proposition~9.9.5 L'intégrale \int ~
_1^+\infty~ dt \over t^\alpha~ converge
si et seulement si~\alpha~ > 1.

Démonstration \int  _1^x~ dt
\over t^\alpha~ = \left
\ \cases  1 \over
\alpha~-1 (1 - x^1-\alpha~)&si \alpha~\neq~1
\cr log~ x &si \alpha~ = 1
\cr  \right . qui admet une limite finie
en + \infty~ si et seulement si~\alpha~ > 1.

Exemple~9.9.1 Intégrales de Bertrand \int ~
_e^+\infty~ dt \over
t^\alpha~(log t)^\beta~~ . Si \alpha~
> 1, soit \gamma tel que 1 < \alpha~ < \gamma. On a
alors  1 \over
t^\alpha~(log t)^\beta~~ = o( 1
\over t^\gamma ) et donc l'intégrale converge. Si
\alpha~ < 1, soit \gamma tel que \alpha~ < \gamma < 1~; on a
alors  1 \over t^\gamma = o( 1
\over t^\alpha~(log~
t)^\beta~ ) et comme \int ~
_e^+\infty~ dt \over t^\gamma diverge,
l'intégrale diverge. Si \alpha~ = 1, on a par le changement de variables u
= log~ t,

\int  _e^x~ dt
\over t(log t)^\beta~~
=\int ~
_1^log x~ du
\over u^\beta~ = \left
\ \cases  1 \over
\alpha~-1 (1 - (log x)^1-\alpha~~)&si
\alpha~\neq~1 \cr
log \log~ x &si \alpha~ = 1
 \right .

qui admet une limite en + \infty~ si et seulement si~\beta~ > 1. En
définitive l'intégrale de Bertrand \int ~
_e^+\infty~ dt \over
t^\alpha~(log t)^\beta~~ converge
si et seulement si~\alpha~ > 1 ou (\alpha~ = 1 et \beta~ > 1).

Proposition~9.9.6 Soit -\infty~ < a < b < +\infty~.
L'intégrale \int  _a^b~ dt
\over b-t^\alpha~ converge si
et seulement si~\alpha~ < 1.

Démonstration On a

\int  _a^b~ dt
\over (b - t)^\alpha~ = \left
\ \cases  1 \over
1-\alpha~ ((b - a)^1-\alpha~ - (b - x)^1-\alpha~)&si
\alpha~\neq~1 \cr
log (b - a) -\ log~ (b
- x)&si \alpha~ = 1  \right .

qui admet une limite au point b si et seulement si~\alpha~ < 1.

Exemple~9.9.2 Intégrales de Bertrand \int ~
_0^1\diagupet^\alpha~log~
t^\beta~ dt. Si \alpha~ > -1, soit \gamma tel que \alpha~
> \gamma > -1. On a alors en 0,
t^\alpha~log~
t^\beta~ = o(t^\gamma) (car 
t^\alpha~ log~
t^\beta~ \over t^\gamma =
t^\alpha~-\gammalog~
t^\beta~ tend vers 0 quand t tend vers 0) et comme
\int  _0^1\diagupet^\gamma~ dt
converge, l'intégrale converge. Si \alpha~ < -1, soit \gamma tel que \alpha~
< \gamma < -1. Alors t^\gamma =
o(t^\alpha~log~
t^\beta~) et comme \int ~
_0^1\diagupet^\gamma dt diverge, l'intégrale diverge. Si \alpha~
= -1, le changement de variables u = -log~ t
conduit à

\int  _x^1\diagupe~
log t^\beta~~
\over t dt =\int ~
_1^- log xu^\beta~~ du

qui admet une limite quand x tend vers 0 si et seulement si~\beta~
< -1. En définitive, l'intégrale de Bertrand
\int ~
_0^1\diagupet^\alpha~log~
t^\beta~ dt converge si et seulement si~\alpha~ >
-1 ou (\alpha~ = -1 et \beta~ < -1).

Remarque~9.9.5 Ce dernier exemple aurait pu également être traité à
partir des intégrales de Bertrand en + \infty~ à l'aide du changement de
variables u = 1 \over t qui est de classe
\mathcal{C}^1 et un homéomorphisme de ]0,1\diagupe] sur [e,+\infty~[ et
donc qui conserve la nature des intégrales. Ceci montre que l'intégrale
\int ~
_0^1\diagupet^\alpha~log~
t^\beta~ dt est de même nature que l'intégrale
\int  _e^+\infty~~
(log u)^\beta~~ \over
u^\alpha~  du \over u^2 qui
converge si et seulement si~2 + \alpha~ > 1 ou 2 + \alpha~ = 1 et \beta~
< -1.

[
[
[
[

\end{document}

% \documentclass[]{article}
\usepackage[T1]{fontenc}
\usepackage{lmodern}
\usepackage{amssymb,amsmath}
\usepackage{ifxetex,ifluatex}
\usepackage{fixltx2e} % provides \textsubscript
% use upquote if available, for straight quotes in verbatim environments
\IfFileExists{upquote.sty}{\usepackage{upquote}}{}
\ifnum 0\ifxetex 1\fi\ifluatex 1\fi=0 % if pdftex
  \usepackage[utf8]{inputenc}
\else % if luatex or xelatex
  \ifxetex
    \usepackage{mathspec}
    \usepackage{xltxtra,xunicode}
  \else
    \usepackage{fontspec}
  \fi
  \defaultfontfeatures{Mapping=tex-text,Scale=MatchLowercase}
  \newcommand{\euro}{€}
\fi
% use microtype if available
\IfFileExists{microtype.sty}{\usepackage{microtype}}{}
\ifxetex
  \usepackage[setpagesize=false, % page size defined by xetex
              unicode=false, % unicode breaks when used with xetex
              xetex]{hyperref}
\else
  \usepackage[unicode=true]{hyperref}
\fi
\hypersetup{breaklinks=true,
            bookmarks=true,
            pdfauthor={},
            pdftitle={Convergence absolue, semi-convergence},
            colorlinks=true,
            citecolor=blue,
            urlcolor=blue,
            linkcolor=magenta,
            pdfborder={0 0 0}}
\urlstyle{same}  % don't use monospace font for urls
\setlength{\parindent}{0pt}
\setlength{\parskip}{6pt plus 2pt minus 1pt}
\setlength{\emergencystretch}{3em}  % prevent overfull lines
\setcounter{secnumdepth}{0}
 
/* start css.sty */
.cmr-5{font-size:50%;}
.cmr-7{font-size:70%;}
.cmmi-5{font-size:50%;font-style: italic;}
.cmmi-7{font-size:70%;font-style: italic;}
.cmmi-10{font-style: italic;}
.cmsy-5{font-size:50%;}
.cmsy-7{font-size:70%;}
.cmex-7{font-size:70%;}
.cmex-7x-x-71{font-size:49%;}
.msbm-7{font-size:70%;}
.cmtt-10{font-family: monospace;}
.cmti-10{ font-style: italic;}
.cmbx-10{ font-weight: bold;}
.cmr-17x-x-120{font-size:204%;}
.cmsl-10{font-style: oblique;}
.cmti-7x-x-71{font-size:49%; font-style: italic;}
.cmbxti-10{ font-weight: bold; font-style: italic;}
p.noindent { text-indent: 0em }
td p.noindent { text-indent: 0em; margin-top:0em; }
p.nopar { text-indent: 0em; }
p.indent{ text-indent: 1.5em }
@media print {div.crosslinks {visibility:hidden;}}
a img { border-top: 0; border-left: 0; border-right: 0; }
center { margin-top:1em; margin-bottom:1em; }
td center { margin-top:0em; margin-bottom:0em; }
.Canvas { position:relative; }
li p.indent { text-indent: 0em }
.enumerate1 {list-style-type:decimal;}
.enumerate2 {list-style-type:lower-alpha;}
.enumerate3 {list-style-type:lower-roman;}
.enumerate4 {list-style-type:upper-alpha;}
div.newtheorem { margin-bottom: 2em; margin-top: 2em;}
.obeylines-h,.obeylines-v {white-space: nowrap; }
div.obeylines-v p { margin-top:0; margin-bottom:0; }
.overline{ text-decoration:overline; }
.overline img{ border-top: 1px solid black; }
td.displaylines {text-align:center; white-space:nowrap;}
.centerline {text-align:center;}
.rightline {text-align:right;}
div.verbatim {font-family: monospace; white-space: nowrap; text-align:left; clear:both; }
.fbox {padding-left:3.0pt; padding-right:3.0pt; text-indent:0pt; border:solid black 0.4pt; }
div.fbox {display:table}
div.center div.fbox {text-align:center; clear:both; padding-left:3.0pt; padding-right:3.0pt; text-indent:0pt; border:solid black 0.4pt; }
div.minipage{width:100%;}
div.center, div.center div.center {text-align: center; margin-left:1em; margin-right:1em;}
div.center div {text-align: left;}
div.flushright, div.flushright div.flushright {text-align: right;}
div.flushright div {text-align: left;}
div.flushleft {text-align: left;}
.underline{ text-decoration:underline; }
.underline img{ border-bottom: 1px solid black; margin-bottom:1pt; }
.framebox-c, .framebox-l, .framebox-r { padding-left:3.0pt; padding-right:3.0pt; text-indent:0pt; border:solid black 0.4pt; }
.framebox-c {text-align:center;}
.framebox-l {text-align:left;}
.framebox-r {text-align:right;}
span.thank-mark{ vertical-align: super }
span.footnote-mark sup.textsuperscript, span.footnote-mark a sup.textsuperscript{ font-size:80%; }
div.tabular, div.center div.tabular {text-align: center; margin-top:0.5em; margin-bottom:0.5em; }
table.tabular td p{margin-top:0em;}
table.tabular {margin-left: auto; margin-right: auto;}
div.td00{ margin-left:0pt; margin-right:0pt; }
div.td01{ margin-left:0pt; margin-right:5pt; }
div.td10{ margin-left:5pt; margin-right:0pt; }
div.td11{ margin-left:5pt; margin-right:5pt; }
table[rules] {border-left:solid black 0.4pt; border-right:solid black 0.4pt; }
td.td00{ padding-left:0pt; padding-right:0pt; }
td.td01{ padding-left:0pt; padding-right:5pt; }
td.td10{ padding-left:5pt; padding-right:0pt; }
td.td11{ padding-left:5pt; padding-right:5pt; }
table[rules] {border-left:solid black 0.4pt; border-right:solid black 0.4pt; }
.hline hr, .cline hr{ height : 1px; margin:0px; }
.tabbing-right {text-align:right;}
span.TEX {letter-spacing: -0.125em; }
span.TEX span.E{ position:relative;top:0.5ex;left:-0.0417em;}
a span.TEX span.E {text-decoration: none; }
span.LATEX span.A{ position:relative; top:-0.5ex; left:-0.4em; font-size:85%;}
span.LATEX span.TEX{ position:relative; left: -0.4em; }
div.float img, div.float .caption {text-align:center;}
div.figure img, div.figure .caption {text-align:center;}
.marginpar {width:20%; float:right; text-align:left; margin-left:auto; margin-top:0.5em; font-size:85%; text-decoration:underline;}
.marginpar p{margin-top:0.4em; margin-bottom:0.4em;}
.equation td{text-align:center; vertical-align:middle; }
td.eq-no{ width:5%; }
table.equation { width:100%; } 
div.math-display, div.par-math-display{text-align:center;}
math .texttt { font-family: monospace; }
math .textit { font-style: italic; }
math .textsl { font-style: oblique; }
math .textsf { font-family: sans-serif; }
math .textbf { font-weight: bold; }
.partToc a, .partToc, .likepartToc a, .likepartToc {line-height: 200%; font-weight:bold; font-size:110%;}
.chapterToc a, .chapterToc, .likechapterToc a, .likechapterToc, .appendixToc a, .appendixToc {line-height: 200%; font-weight:bold;}
.index-item, .index-subitem, .index-subsubitem {display:block}
.caption td.id{font-weight: bold; white-space: nowrap; }
table.caption {text-align:center;}
h1.partHead{text-align: center}
p.bibitem { text-indent: -2em; margin-left: 2em; margin-top:0.6em; margin-bottom:0.6em; }
p.bibitem-p { text-indent: 0em; margin-left: 2em; margin-top:0.6em; margin-bottom:0.6em; }
.paragraphHead, .likeparagraphHead { margin-top:2em; font-weight: bold;}
.subparagraphHead, .likesubparagraphHead { font-weight: bold;}
.quote {margin-bottom:0.25em; margin-top:0.25em; margin-left:1em; margin-right:1em; text-align:justify;}
.verse{white-space:nowrap; margin-left:2em}
div.maketitle {text-align:center;}
h2.titleHead{text-align:center;}
div.maketitle{ margin-bottom: 2em; }
div.author, div.date {text-align:center;}
div.thanks{text-align:left; margin-left:10%; font-size:85%; font-style:italic; }
div.author{white-space: nowrap;}
.quotation {margin-bottom:0.25em; margin-top:0.25em; margin-left:1em; }
h1.partHead{text-align: center}
.sectionToc, .likesectionToc {margin-left:2em;}
.subsectionToc, .likesubsectionToc {margin-left:4em;}
.subsubsectionToc, .likesubsubsectionToc {margin-left:6em;}
.frenchb-nbsp{font-size:75%;}
.frenchb-thinspace{font-size:75%;}
.figure img.graphics {margin-left:10%;}
/* end css.sty */

\title{Convergence absolue, semi-convergence}
\author{}
\date{}

\begin{document}
\maketitle

\textbf{Warning: 
requires JavaScript to process the mathematics on this page.\\ If your
browser supports JavaScript, be sure it is enabled.}

\begin{center}\rule{3in}{0.4pt}\end{center}

[
[
[]
[

\subsubsection{9.10 Convergence absolue, semi-convergence}

\paragraph{9.10.1 Critère de Cauchy pour les intégrales}

Théorème~9.10.1 (critère de Cauchy). Soit E un espace vectoriel normé
complet et -\infty~ < a < b \leq +\infty~. Soit f : [a,b[\rightarrow~ E
réglée. Alors l'intégrale \int ~
_a^bf(t) dt converge si et seulement si~

\forall~~\epsilon > 0,
\exists~c \in [a,b[, c < u < v
< b \rigtharrow~\\int ~
_u^vf(t) dt\ < \epsilon

Démonstration Si F(x) =\int ~
_a^xf(t) dt, la propriété ci dessus est équivalente à

\forall~~\epsilon > 0,
\exists~c \in [a,b[, c < u < v
< b \rigtharrow~\ F(v) -
F(u)\ < \epsilon

ce qui n'est autre que le critère de Cauchy pour l'existence de la
limite de F au point b.

\paragraph{9.10.2 Convergence absolue}

Définition~9.10.1 Soit f : [a,b[\rightarrow~ E réglée. On dit que l'intégrale
\int  _a^b~f(t) dt converge
absolument si l'intégrale \int ~
_a^b\f(t)\
dt converge.

Théorème~9.10.2 Soit E un espace vectoriel normé complet et -\infty~
< a < b \leq +\infty~. Soit f : [a,b[\rightarrow~ E réglée. Si
l'intégrale \int  _a^b~f(t) dt
converge absolument, elle converge.

Démonstration Soit \epsilon > 0. Puisque l'intégrale
\int ~
_a^b\f(t)\dt
converge, d'après le critère de Cauchy, il existe c \in [a,b[ tel que
c < u < v < b \rigtharrow~\\int

_u^v\f(t)\
dt < \epsilon. Alors c < u < v < b
\rigtharrow~\\int ~
_u^vf(t) dt\
\leq\int ~
_u^v\f(t)\
dt < \epsilon. Donc l'intégrale \int ~
_a^bf(t) dt vérifie le critère de Cauchy, par conséquent
elle converge.

Remarque~9.10.1 L'avantage est évidemment que la convergence absolue
concerne la convergence d'une intégrale de fonction à valeurs réelles
positives pour laquelle nous disposons déjà de critères simples.

\paragraph{9.10.3 Règles de convergence}

Proposition~9.10.3 Soit f : [a,b[\rightarrow~ E et g : [a,b[\rightarrow~ F réglées. On
suppose que f = 0(g) et que l'intégrale \int ~
_a^bg(t) dt converge absolument. Alors l'intégrale
\int  _a^b~f(t) dt converge
absolument.

Démonstration En effet f = 0(g) \Leftrightarrow
\f(t)\ =
O(\g(t)\) au voisinage
de b.

Remarque~9.10.2 En général la fonction étalon g sera choisie à valeurs
réelles positives.

Théorème~9.10.4 Soit f : [a,b[\rightarrow~ E et g : [a,b[\rightarrow~ \mathbb{R}~ réglées. On
suppose que g est positive et qu'il existe \ell \in E
\diagdown\0\ tel que, au voisinage de b, f(t)
∼ \ellg(t). Alors (i) si \int ~
_a^bg(t) dt converge, l'intégrale
\int  _a^b~f(t) dt converge
absolument (ii) si \int  _a^b~g(t)
dt diverge, l'intégrale \int ~
_a^bf(t) dt diverge.

Démonstration On a bien entendu, f = O(g), et d'après la proposition
précédente, si \int  _a^b~g(t) dt
converge, l'intégrale \int ~
_a^bf(t) dt converge absolument. Inversement, supposons
que l'intégrale \int  _a^b~f(t) dt
converge. On a f - \ellg =
o(\\ellg\) et donc il
existe c \in [a,b[ tel que t > c
\rigtharrow~\ f(t) - \ellg(t)\ \leq 1
\over 2
\\ellg(t)\ = 1
\over 2
\\ell\g(t). Soit alors c
< u < v < b~; on a
\\ell\\\int
 _u^vg(t) dt =\
\ell\int  _u^v~g(t)
dt\ puisque g est réelle positive. On a donc

\begin{align*}
\\ell\\\int
 _u^vg(t) dt& =&
\\int ~
_u^v(\ellg(t) - f(t)) dt +\int ~
_u^vf(t) dt\\%&
\\ & \leq& \int ~
_u^v\\ellg(t) -
f(t)\ dt
+\\int ~
_u^vf(t) dt\ \%&
\\ & \leq& 1 \over 2
\\ell\\\int
 _u^vg(t) dt
+\\int ~
_u^vf(t) dt\ \%&
\\ \end{align*}

On en déduit que \int  _u^v~g(t)
dt \leq 2 \over
\\ell\
\\int ~
_u^vf(t) dt\. Comme
\int  _a^b~f(t) dt converge,
l'intégrale vérifie le critère de Cauchy~; l'inégalité ci dessus montre
que l'intégrale \int  _a^b~g(t) dt
vérifie également le critère de Cauchy, donc converge.

En utilisant alors nos fonctions ''étalons''  1 \over
t^\alpha~ en + \infty~ et  1 \over
(b-t)^\alpha~ en b \in \mathbb{R}~, on obtient les critères suivants

Théorème~9.10.5 Soit E un espace vectoriel normé. Soit f : [a,+\infty~[\rightarrow~ E
réglée. (i) S'il existe \alpha~ > 1 tel que f(t) = 0( 1
\over t^\alpha~ ), alors l'intégrale
\int  _a^+\infty~~f(t) dt converge
absolument (ii) S'il existe \alpha~ \in \mathbb{R}~ et \ell \in E
\diagdown\0\ tels que f(t) ∼ \ell
\over t^\alpha~ alors l'intégrale
\int  _a^+\infty~~f(t) dt converge
absolument si \alpha~ > 1 et diverge si \alpha~ \leq 1. (iii) Si E = \mathbb{R}~ et
f(t) ≥ 0, et s'il existe \alpha~ \leq 1 et \ell > 0 (y compris + \infty~) tel
que lim_t\rightarrow~+\infty~t^\alpha~~f(t) = \ell,
alors l'intégrale \int  _a^+\infty~~f(t)
dt diverge.

Théorème~9.10.6 Soit E un espace vectoriel normé, b \in \mathbb{R}~. Soit f :
[a,b[\rightarrow~ E réglée. (i) S'il existe \alpha~ < 1 tel que f(t) = 0(
1 \over (b-t)^\alpha~ ), alors l'intégrale
\int  _a^b~f(t) dt converge
absolument (ii) S'il existe \alpha~ \in \mathbb{R}~ et \ell \in E
\diagdown\0\ tels que f(t) ∼ \ell
\over (b-t)^\alpha~ alors l'intégrale
\int  _a^b~f(t) dt converge
absolument si \alpha~ < 1 et diverge si \alpha~ ≥ 1. (iii) Si E = \mathbb{R}~ et
f(t) ≥ 0, et s'il existe \alpha~ ≥ 1 et \ell > 0 (y compris + \infty~) tel
que lim_t\rightarrow~b(b - t)^\alpha~~f(t) =
\ell, alors l'intégrale \int ~
_a^bf(t) dt diverge.

Exemple~9.10.1 La fonction
t\mapsto~e^-t^2  est continue
sur [0,+\infty~[ et en + \infty~ on a e^-t^2  = o( 1
\over t^2 ). Donc l'intégrale
\int ~
_0^+\infty~e^-t^2  dt converge. De même,
considérons l'intégrale \int ~
_0^+\infty~t^s-1e^-t dt. L'application
t\mapsto~t^s-1e^-t est
continue sur ]0,+\infty~[, donc l'intégrale est a priori doublement
impropre en 0 et en + \infty~. En + \infty~, on a t^s-1e^-t =
o( 1 \over t^2 ) et donc l'intégrale
converge en + \infty~. En 0, on a t^s-1e^-t ∼
t^s-1 > 0, donc l'intégrale converge si et
seulement si~s - 1 > -1 soit s > 0. En
définitive, l'intégrale converge si et seulement si~s > 0.
On pose alors \Gamma(s) =\int ~
_0^+\infty~t^s-1e^-t dt. Une intégration
par parties donne alors pour s > 0

\begin{align*} \int ~
_x^yt^se^-t dt& =&
\left
[-t^se^-t\right ]_
x^y + s\int ~
_x^yt^s-1e^-t dt \%&
\\ & =& x^se^-x -
y^se^-y + s\int ~
_x^yt^s-1e^-t dt\%&
\\ \end{align*}

et en faisant tendre x vers 0 et y vers + \infty~, on obtient l'équation
fonctionnelle \Gamma(s) = s\Gamma(s - 1). Tenant compte de \Gamma(1)
=\int  _0^+\infty~e^-t~ dt =
1, on obtient \forall~~n \in \mathbb{N}~, \Gamma(n) = (n - 1)!.
Remarquons également qu'en faisant le changement de variables t =
\sqrtu qui est de classe \mathcal{C}^1 sur [x,y]
pour 0 < x < y < +\infty~, on obtient

\int ~
_x^ye^-t^2  dt = 1
\over 2 \int ~
_\sqrtx^\sqrty
e^-u \over \sqrtu du

En faisant tendre x vers 0 et y vers + \infty~, on obtient
\int ~
_0^+\infty~e^-t^2  dt = 1
\over 2 \Gamma( 1 \over 2 ).

\paragraph{9.10.4 Semi-convergence}

On dit qu'une intégrale \int ~
_a^bf(t) dt est semi-convergente si elle converge, sans
être absolument convergente. L'outil essentiel pour montrer une
convergence non absolue est l'intégration par parties~; les autres
outils sont un théorème d'Abel ou le retour pur et simple au critère de
Cauchy.

Exemple~9.10.2 Etude de l'intégrale \int ~
_1^+\infty~ sin~ t
\over t^\alpha~ dt. On a 
sin t \over t^\alpha~~ =
O( 1 \over t^\alpha~ ), donc si \alpha~ >
1 l'intégrale converge absolument.

Si 0 < \alpha~ \leq 1, on a après intégration par parties

\int  _1^x~
sin t \over t^\alpha~~ dt
= cos 1 - \cos~ x
\over x^\alpha~ +\int ~
_1^x cos~ t
\over t^\alpha~+1 dt

Mais lim_x\rightarrow~+\infty~~
cos x \over x^\alpha~~ =
0 et l'intégrale \int  _1^+\infty~~
cos t \over t^\alpha~+1~
dt converge absolument puisque  cos~ t
\over t^\alpha~+1 = O( 1 \over
t^\alpha~+1 ). On en déduit que le terme de droite de l'égalité
ci dessus a une limite en + \infty~, et donc le terme de gauche aussi. En
conséquence, l'intégrale \int ~
_1^+\infty~ sin~ t
\over t^\alpha~ dt converge. Montrons qu'elle ne
converge pas absolument~; on a

\begin{align*} \int ~
_1^x sin~ t
\over t^\alpha~ & ≥& \\int
 _1^x sin ^2~t
\over t^\alpha~ dt = 1 \over 2
\int  _1^x~ 1
- cos~ (2t) \over
t^\alpha~ dt\%& \\ & =& 1
\over 2 \int ~
_1^x 1 \over t^\alpha~ dt - 1
\over 2 \int ~
_1^x cos~ (2t)
\over t^\alpha~ dt \%&
\\ \end{align*}

Mais l'intégrale \int  _1^+\infty~~ 1
\over t^\alpha~ dt est divergente (car \alpha~ \leq 1),
alors que l'intégrale \int ~
_1^+\infty~ cos~ (2t)
\over t^\alpha~ dt converge (même méthode
d'intégration par parties). On en déduit que
lim_x\rightarrow~+\infty~~\\int
 _1^x sin ^2~t
\over t^\alpha~ dt = +\infty~ et donc aussi
lim_x\rightarrow~+\infty~~\\int
 _1^x  sin~
t \over t^\alpha~ dt = +\infty~.

Si \alpha~ \leq 0, posons \beta~ = -\alpha~. On a (en posant t = u + n\pi~),

\begin{align*} \left
\int ~
_n\pi~^(n+1)\pi~t^\beta~ sin~ t
dt\right & =& \int ~
_0^\pi~(u + n\pi~)^\beta~ sin~ u
du \%& \\ & ≥&
(n\pi~)^\beta~\int ~
_0^\pi~ sin~ u du =
2(n\pi~)^\beta~\%& \\
\end{align*}

qui ne tend pas vers 0 quand n tend vers + \infty~~; le critère de Cauchy
n'est pas vérifié, et donc l'intégrale diverge.

Dans certains cas, le théorème d'Abel peut rendre des services (mais
très souvent, une simple intégration par parties peut s'y substituer)

Théorème~9.10.7 (Abel). Soit f : [a,b[\rightarrow~ \mathbb{R}~ de classe \mathcal{C}^1
et g : [a,b[\rightarrow~ \mathbb{R}~ continue. On suppose que

\begin{itemize}
\itemsep1pt\parskip0pt\parsep0pt
\item
  (i) f est positive, décroissante, de limite 0 en b
\item
  (ii) \existsM ≥ 0, \\forall~~x \in
  [a,b[,\quad \left
  \int  _a^x~g(t)
  dt\right \leq M
\end{itemize}

Alors l'intégrale \int ~
_a^bf(t)g(t) dt converge.

Démonstration On montre que cette intégrale impropre vérifie le critère
de Cauchy à l'aide de la deuxième formule de la moyenne (dont les
hypothèses sur [u,v] sont bien vérifiées)

\begin{align*} \left
\int  _u^v~f(t)g(t)
dt\right & =& f(u)\left
\int  _u^w~g(t)
dt\right  \%& \\
& =& f(u)\left \int ~
_a^wg(t) dt -\int ~
_a^ug(t) dt\right \leq 2Mf(u)\%&
\\ \end{align*}

Soit \epsilon > 0. Il existe c \in [a,b[ tel que c < u
< b \rigtharrow~ 2Mf(u) < \epsilon. Alors c < u <
v < b \rigtharrow~\left
\int  _u^v~f(t)g(t)
dt\right  ce qui assure la convergence de
l'intégrale.

[
[
[
[

\end{document}

% \documentclass[]{article}
\usepackage[T1]{fontenc}
\usepackage{lmodern}
\usepackage{amssymb,amsmath}
\usepackage{ifxetex,ifluatex}
\usepackage{fixltx2e} % provides \textsubscript
% use upquote if available, for straight quotes in verbatim environments
\IfFileExists{upquote.sty}{\usepackage{upquote}}{}
\ifnum 0\ifxetex 1\fi\ifluatex 1\fi=0 % if pdftex
  \usepackage[utf8]{inputenc}
\else % if luatex or xelatex
  \ifxetex
    \usepackage{mathspec}
    \usepackage{xltxtra,xunicode}
  \else
    \usepackage{fontspec}
  \fi
  \defaultfontfeatures{Mapping=tex-text,Scale=MatchLowercase}
  \newcommand{\euro}{€}
\fi
% use microtype if available
\IfFileExists{microtype.sty}{\usepackage{microtype}}{}
\ifxetex
  \usepackage[setpagesize=false, % page size defined by xetex
              unicode=false, % unicode breaks when used with xetex
              xetex]{hyperref}
\else
  \usepackage[unicode=true]{hyperref}
\fi
\hypersetup{breaklinks=true,
            bookmarks=true,
            pdfauthor={},
            pdftitle={Suites de fonctions},
            colorlinks=true,
            citecolor=blue,
            urlcolor=blue,
            linkcolor=magenta,
            pdfborder={0 0 0}}
\urlstyle{same}  % don't use monospace font for urls
\setlength{\parindent}{0pt}
\setlength{\parskip}{6pt plus 2pt minus 1pt}
\setlength{\emergencystretch}{3em}  % prevent overfull lines
\setcounter{secnumdepth}{0}
 
/* start css.sty */
.cmr-5{font-size:50%;}
.cmr-7{font-size:70%;}
.cmmi-5{font-size:50%;font-style: italic;}
.cmmi-7{font-size:70%;font-style: italic;}
.cmmi-10{font-style: italic;}
.cmsy-5{font-size:50%;}
.cmsy-7{font-size:70%;}
.cmex-7{font-size:70%;}
.cmex-7x-x-71{font-size:49%;}
.msbm-7{font-size:70%;}
.cmtt-10{font-family: monospace;}
.cmti-10{ font-style: italic;}
.cmbx-10{ font-weight: bold;}
.cmr-17x-x-120{font-size:204%;}
.cmsl-10{font-style: oblique;}
.cmti-7x-x-71{font-size:49%; font-style: italic;}
.cmbxti-10{ font-weight: bold; font-style: italic;}
p.noindent { text-indent: 0em }
td p.noindent { text-indent: 0em; margin-top:0em; }
p.nopar { text-indent: 0em; }
p.indent{ text-indent: 1.5em }
@media print {div.crosslinks {visibility:hidden;}}
a img { border-top: 0; border-left: 0; border-right: 0; }
center { margin-top:1em; margin-bottom:1em; }
td center { margin-top:0em; margin-bottom:0em; }
.Canvas { position:relative; }
li p.indent { text-indent: 0em }
.enumerate1 {list-style-type:decimal;}
.enumerate2 {list-style-type:lower-alpha;}
.enumerate3 {list-style-type:lower-roman;}
.enumerate4 {list-style-type:upper-alpha;}
div.newtheorem { margin-bottom: 2em; margin-top: 2em;}
.obeylines-h,.obeylines-v {white-space: nowrap; }
div.obeylines-v p { margin-top:0; margin-bottom:0; }
.overline{ text-decoration:overline; }
.overline img{ border-top: 1px solid black; }
td.displaylines {text-align:center; white-space:nowrap;}
.centerline {text-align:center;}
.rightline {text-align:right;}
div.verbatim {font-family: monospace; white-space: nowrap; text-align:left; clear:both; }
.fbox {padding-left:3.0pt; padding-right:3.0pt; text-indent:0pt; border:solid black 0.4pt; }
div.fbox {display:table}
div.center div.fbox {text-align:center; clear:both; padding-left:3.0pt; padding-right:3.0pt; text-indent:0pt; border:solid black 0.4pt; }
div.minipage{width:100%;}
div.center, div.center div.center {text-align: center; margin-left:1em; margin-right:1em;}
div.center div {text-align: left;}
div.flushright, div.flushright div.flushright {text-align: right;}
div.flushright div {text-align: left;}
div.flushleft {text-align: left;}
.underline{ text-decoration:underline; }
.underline img{ border-bottom: 1px solid black; margin-bottom:1pt; }
.framebox-c, .framebox-l, .framebox-r { padding-left:3.0pt; padding-right:3.0pt; text-indent:0pt; border:solid black 0.4pt; }
.framebox-c {text-align:center;}
.framebox-l {text-align:left;}
.framebox-r {text-align:right;}
span.thank-mark{ vertical-align: super }
span.footnote-mark sup.textsuperscript, span.footnote-mark a sup.textsuperscript{ font-size:80%; }
div.tabular, div.center div.tabular {text-align: center; margin-top:0.5em; margin-bottom:0.5em; }
table.tabular td p{margin-top:0em;}
table.tabular {margin-left: auto; margin-right: auto;}
div.td00{ margin-left:0pt; margin-right:0pt; }
div.td01{ margin-left:0pt; margin-right:5pt; }
div.td10{ margin-left:5pt; margin-right:0pt; }
div.td11{ margin-left:5pt; margin-right:5pt; }
table[rules] {border-left:solid black 0.4pt; border-right:solid black 0.4pt; }
td.td00{ padding-left:0pt; padding-right:0pt; }
td.td01{ padding-left:0pt; padding-right:5pt; }
td.td10{ padding-left:5pt; padding-right:0pt; }
td.td11{ padding-left:5pt; padding-right:5pt; }
table[rules] {border-left:solid black 0.4pt; border-right:solid black 0.4pt; }
.hline hr, .cline hr{ height : 1px; margin:0px; }
.tabbing-right {text-align:right;}
span.TEX {letter-spacing: -0.125em; }
span.TEX span.E{ position:relative;top:0.5ex;left:-0.0417em;}
a span.TEX span.E {text-decoration: none; }
span.LATEX span.A{ position:relative; top:-0.5ex; left:-0.4em; font-size:85%;}
span.LATEX span.TEX{ position:relative; left: -0.4em; }
div.float img, div.float .caption {text-align:center;}
div.figure img, div.figure .caption {text-align:center;}
.marginpar {width:20%; float:right; text-align:left; margin-left:auto; margin-top:0.5em; font-size:85%; text-decoration:underline;}
.marginpar p{margin-top:0.4em; margin-bottom:0.4em;}
.equation td{text-align:center; vertical-align:middle; }
td.eq-no{ width:5%; }
table.equation { width:100%; } 
div.math-display, div.par-math-display{text-align:center;}
math .texttt { font-family: monospace; }
math .textit { font-style: italic; }
math .textsl { font-style: oblique; }
math .textsf { font-family: sans-serif; }
math .textbf { font-weight: bold; }
.partToc a, .partToc, .likepartToc a, .likepartToc {line-height: 200%; font-weight:bold; font-size:110%;}
.chapterToc a, .chapterToc, .likechapterToc a, .likechapterToc, .appendixToc a, .appendixToc {line-height: 200%; font-weight:bold;}
.index-item, .index-subitem, .index-subsubitem {display:block}
.caption td.id{font-weight: bold; white-space: nowrap; }
table.caption {text-align:center;}
h1.partHead{text-align: center}
p.bibitem { text-indent: -2em; margin-left: 2em; margin-top:0.6em; margin-bottom:0.6em; }
p.bibitem-p { text-indent: 0em; margin-left: 2em; margin-top:0.6em; margin-bottom:0.6em; }
.paragraphHead, .likeparagraphHead { margin-top:2em; font-weight: bold;}
.subparagraphHead, .likesubparagraphHead { font-weight: bold;}
.quote {margin-bottom:0.25em; margin-top:0.25em; margin-left:1em; margin-right:1em; text-align:justify;}
.verse{white-space:nowrap; margin-left:2em}
div.maketitle {text-align:center;}
h2.titleHead{text-align:center;}
div.maketitle{ margin-bottom: 2em; }
div.author, div.date {text-align:center;}
div.thanks{text-align:left; margin-left:10%; font-size:85%; font-style:italic; }
div.author{white-space: nowrap;}
.quotation {margin-bottom:0.25em; margin-top:0.25em; margin-left:1em; }
h1.partHead{text-align: center}
.sectionToc, .likesectionToc {margin-left:2em;}
.subsectionToc, .likesubsectionToc {margin-left:4em;}
.subsubsectionToc, .likesubsubsectionToc {margin-left:6em;}
.frenchb-nbsp{font-size:75%;}
.frenchb-thinspace{font-size:75%;}
.figure img.graphics {margin-left:10%;}
/* end css.sty */

\title{Suites de fonctions}
\author{}
\date{}

\begin{document}
\maketitle

\textbf{Warning: 
requires JavaScript to process the mathematics on this page.\\ If your
browser supports JavaScript, be sure it is enabled.}

\begin{center}\rule{3in}{0.4pt}\end{center}

[
[]
[

\subsubsection{10.1 Suites de fonctions}

\paragraph{10.1.1 Convergence simple, convergence uniforme}

Définition~10.1.1 Soit X un ensemble, E un espace métrique,
(f_n)_n\in\mathbb{N}~ une suite d'applications de X dans E. On dit
que la suite converge simplement si pour tout x \in X, la suite
(f_n(x))_n\in\mathbb{N}~ converge dans E. Dans ce cas, on pose
f(x) = limf_n~(x) et on dit que f : X
\rightarrow~ E est limite simple de la suite (f_n).

Remarque~10.1.1 La traduction en métrique de f est limite simple de la
suite (f_n) est

\forall~x \in X, \\forall~~\epsilon
> 0, \exists~N(\epsilon,x),\quad
n ≥ N(\epsilon,x) \rigtharrow~ d(f(x),f_n(x)) < \epsilon

où l'entier N dépend à la fois de \epsilon et de x \in X.

Exemple~10.1.1 Soit f_n : [0,1] \rightarrow~ \mathbb{R}~,
x\mapsto~x^n. La suite f_n
converge simplement vers f : [0,1] \rightarrow~ \mathbb{R}~ définie par f(x) =
\left \ \cases 1&si x
= 1 \cr 0&si x\neq~1 
\right .. Pour un \epsilon < 1 donné, le meilleur
N(\epsilon,x) que l'on puisse prendre est 0 si x = 1 ou x = 0, et E(
log~ \epsilon \over
log x~ ) si 0 < x < 1. On
constate que sup_x\in[0,1]~N(\epsilon,x) =
+\infty~. Il n'est donc pas question de prendre le même N pour tous les x.

Définition~10.1.2 Soit X un ensemble, E un espace métrique,
(f_n)_n\in\mathbb{N}~ une suite d'applications de X dans E. On dit
que la suite converge uniformément s'il existe f : X \rightarrow~ E vérifiant les
conditions équivalentes

\begin{itemize}
\itemsep1pt\parskip0pt\parsep0pt
\item
  (i) \forall~~\epsilon > 0,
  \exists~N(\epsilon),\quad n ≥ N(\epsilon)
  \rigtharrow~\forall~x \in X, d(f(x),f_n~(x)) <
  \epsilon
\item
  (ii) lim_n\rightarrow~+\infty~\mu_n~ = 0 où
  l'on a posé \mu_n =\
  sup_x\inXd(f_n(x),f(x)) \in \mathbb{R}~ \cup\ +
  \infty~\.
\end{itemize}

Démonstration L'équivalence est claire puisque \mu_n <
\epsilon \rigtharrow~ (\forall~x \in X, d(f_n~(x),f(x))
< \epsilon) et qu'inversement (\forall~~x \in X,
d(f_n(x),f(x)) < \epsilon) \rigtharrow~ \mu_n \leq \epsilon.

Remarque~10.1.2 Il est clair que si la suite (f_n) converge
uniformément vers f, elle converge simplement vers f. On en déduit que
la fonction f est unique.

\paragraph{10.1.2 Plan d'étude d'une suite de fonctions}

Soit X un ensemble, E un espace métrique, (f_n)_n\in\mathbb{N}~
une suite d'applications de X dans E.

On commence par étudier la convergence simple de la suite de fonctions.
Pour chaque x \in X on étudie la suite (f_n(x)) d'éléments de E.
Dans le cas où cette suite est convergente pour chaque x \in X, on définit
f : X \rightarrow~ E par f(x) = limf_n~(x)~;
l'application f est limite simple de la suite (f_n).

On étudie ensuite la convergence uniforme de la suite (f_n)
vers f. Pour montrer une convergence uniforme, on peut soit chercher une
suite (\alpha_n) de limite 0 indépendante de x telle que
\forall~x \in X, d(f_n~(x),f(x)) \leq
\alpha_n, soit étudier directement la suite (\mu_n) où
\mu_n =\
sup_x\inXd(f_n(x),f(x)) \in \mathbb{R}~ \cup\ +
\infty~\. Pour montrer une non convergence uniforme, on peut
soit utiliser un des théorèmes suivants qui garantissent qu'un certain
nombre de propriétés des fonctions f_n sont conservées par
limite uniforme, soit utiliser la proposition suivante

Proposition~10.1.1 Soit X un ensemble, E un espace métrique,
(f_n)_n\in\mathbb{N}~ une suite d'applications de X dans E. Alors
la suite (f_n) converge uniformément vers f si et seulement
si~pour toute suite (x_n) de X, on a
limd(f(x_n),f_n(x_n~))
= 0.

Démonstration La condition est évidemment nécessaire puisque 0 \leq
d(f(x_n),f_n(x_n)) \leq \mu_n.
Inversement, si la suite ne converge pas uniformément vers f, on a, en
niant la propriété (i)

\exists~\epsilon > 0,
\forall~N \in \mathbb{N}~, \\exists~n ≥ N,
\existsx_n~ \in X,\quad
d(f(x_n),f_n(x_n)) ≥ \epsilon

Ceci définit x_n pour une infinité de n. Pour les autres, on
choisit un x_n arbitraire. On a pour une infinité de n,
d(f(x_n),f_n(x_n)) ≥ \epsilon et donc la suite
d(f(x_n),f_n(x_n)) ne tend pas vers 0.

Exemple~10.1.2 Soit f_n : [0,1] \rightarrow~ \mathbb{R}~,
x\mapsto~x^n. La suite f_n
converge simplement vers f : [0,1] \rightarrow~ \mathbb{R}~ définie par f(x) =
\left \ \cases 1&si x
= 1 \cr 0&si x\neq~1 
\right .. Prenons x_n = 1 - 1
\over n . On a f_n(x_n) -
f(x_n) = (1 - 1 \over n )^n de
limite  1 \over e et non 0. Donc la suite ne converge
pas uniformément.

Remarque~10.1.3 Lorsque la convergence n'est pas uniforme sur X tout
entier, on peut rechercher des parties de X sur lesquelles cette
convergence est uniforme.

Exemple~10.1.3 Soit f_n : [0, \pi~\over 2 ]
\rightarrow~ \mathbb{R}~ définie par f_n(t) =
n^\alpha~ sin~
^ntcos~ t. Il est clair que
\forall~t \in [0, \pi~\over 2~ ],
limf_n~(t) = 0~: si t \in [0,
\pi~\over 2 [ on a 0 \leq sin~ t
< 1 et si t = \pi~\diagup2, on a cos~ t = 0.
La suite converge simplement vers la fonction nulle. On a \mu_n
= sup_t\in[0,\pi~\over
2 ]f(t) - f_n(t)
= sup_t\in[0,\pi~\over
2 ]f_n(t). Mais f_n'(t) =
n^\alpha~ sin ^n-1~t(n - (n +
1)sin ^2~t) et on a donc le tableau
de variation, en posant t_n = arcsin~
\sqrt n\over n+1

\begin{center}\rule{3in}{0.4pt}\end{center}

\begin{center}\rule{3in}{0.4pt}\end{center}

\begin{center}\rule{3in}{0.4pt}\end{center}

\begin{center}\rule{3in}{0.4pt}\end{center}

\begin{center}\rule{3in}{0.4pt}\end{center}

\begin{center}\rule{3in}{0.4pt}\end{center}

t

0

t_n

\pi~\over 2

\begin{center}\rule{3in}{0.4pt}\end{center}

\begin{center}\rule{3in}{0.4pt}\end{center}

\begin{center}\rule{3in}{0.4pt}\end{center}

\begin{center}\rule{3in}{0.4pt}\end{center}

\begin{center}\rule{3in}{0.4pt}\end{center}

\begin{center}\rule{3in}{0.4pt}\end{center}

f_n(t)

0

\nearrow

\mu_n

\searrow

0

\begin{center}\rule{3in}{0.4pt}\end{center}

\begin{center}\rule{3in}{0.4pt}\end{center}

\begin{center}\rule{3in}{0.4pt}\end{center}

\begin{center}\rule{3in}{0.4pt}\end{center}

\begin{center}\rule{3in}{0.4pt}\end{center}

\begin{center}\rule{3in}{0.4pt}\end{center}

On a donc

\mu_n = f_n(t_n) =
n^\alpha~\left ( n\over n +
1\right )^n\diagup2 1\over
\sqrtn + 1 ∼ n^\alpha~\over
\sqrte\sqrtn

La suite converge uniformément si et seulement si~\alpha~ <
1\over 2. Par contre, soit a <
\pi~\over 2 et soit N tel que
arcsin~ \sqrt
N\over N+1 > a. Alors dès que n ≥ N,
la fonction f_n est croissante sur [0,a] et donc
sup_t\leqaf_n~(t) =
f_n(a) qui tend vers 0 quand n tend vers + \infty~. On en déduit que
la suite f_n converge uniformément vers la fonction nulle sur
tout intervalle [0,a] (mais pas sur leur réunion [0,
\pi~\over 2 [).

A titre d'introduction à ce qui suit, calculons
\int  _0~^\pi~\over
2 f_n(t) dt~; on a par un simple changement de variables u
= sin t, \\int ~
_0^\pi~\over 2 f_n(t) dt =
n^\alpha~\int ~
_0^1u^n du =
n^\alpha~\over n+1. On voit donc que bien que
\forall~t \in [0, \pi~\over 2~ ],
lim_n\rightarrow~+\infty~f_n~(t) = 0, la suite
\int  _0~^\pi~\over
2 f_n(t) dt ne converge vers 0 que si \alpha~ < 1. Si \alpha~
= 1, elle converge vers 1, et si \alpha~ > 1, elle converge vers
+ \infty~. Autrement dit, si \alpha~ ≥ 1, on a
lim_n\rightarrow~+\infty~~\\int
 _0^1f_n(t)
dt\neq~\int ~
_0^1\
lim_n\rightarrow~+\infty~f_n(t) dt. On voit qu'une convergence simple
ne permet pas d'intervertir le symbole limite et le symbole d'intégrale.

\paragraph{10.1.3 Critère de Cauchy uniforme}

Définition~10.1.3 Soit X un ensemble, E un espace métrique. On dit
qu'une suite (f_n)_n\in\mathbb{N}~ d'applications de X dans E
vérifie le critère de Cauchy uniforme si on a

\forall~~\epsilon > 0,
\existsN \in \mathbb{N}~, p,q ≥ N \rigtharrow~\\forall~~x
\in X, d(f_p(x),f_q(x)) < \epsilon

Remarque~10.1.4 Il est clair que si la suite (f_n) vérifie le
critère de Cauchy uniforme, pour chaque x \in X, la suite
(f_n(x)) d'éléments de E est une suite de Cauchy.

Théorème~10.1.2 Soit X un ensemble, E un espace métrique complet. Alors
une suite (f_n)_n\in\mathbb{N}~ d'applications de X dans E est
uniformément convergente si et seulement si~elle vérifie le critère de
Cauchy uniforme.

Démonstration Le sens direct se démontre de la manière habituelle et
n'utilise pas la complétude de E~: si (f_n) converge
uniformément vers f, soit \epsilon > 0 et N \in \mathbb{N}~ tel que n ≥ N
\rigtharrow~\forall~x \in X, d(f(x),f_n~(x)) <
\epsilon \over 2 . Alors, si p,q ≥ N, on a
\forall~x \in X, d(f_p(x),f_q~(x)) \leq
d(f_p(x),f(x)) + d(f(x),f_q(x)) < \epsilon
\over 2 + \epsilon \over 2 = \epsilon.

Pour la réciproque, supposons que E est complet et que la suite
(f_n) vérifie le critère de Cauchy uniforme. D'après la
remarque précédente, pour chaque x \in X, la suite (f_n(x))
d'éléments de E est une suite de Cauchy, donc elle converge. On pose
f(x) = limf_n~(x). Montrons que la
suite converge uniformément vers f. Soit \epsilon > 0, et soit N \in
\mathbb{N}~ tel que p,q ≥ N \rigtharrow~\forall~~x \in X,
d(f_p(x),f_q(x)) < \epsilon \over
2 . Fixons p ≥ N et faisons tendre q vers + \infty~. On obtient, en tenant
compte de limf_q~(x) = f(x) et de la
continuité de la fonction distance, \forall~~x \in X,
d(f_p(x),f(x)) \leq \epsilon \over 2 < \epsilon, ce
qui montre la convergence uniforme vers f.

\paragraph{10.1.4 Fonctions bornées, norme de la convergence uniforme}

Soit X un ensemble, E un espace vectoriel normé. On notera ℬ(X,E)
l'ensemble des applications bornées de X dans E. Pour f \inℬ(X,E), on
posera \f\\infty~
=\
sup_t\inX\f(t)\
\in \mathbb{R}~.

Proposition~10.1.3 L'application
f\mapsto~\f\\infty~
est une norme sur l'espace vectoriel ℬ(X,E). Soit (f_n) une
suite de ℬ(X,E). Alors la suite (f_n) converge uniformément si
et seulement si~elle converge dans (ℬ(X,E),\
\\infty~), avec la même limite.

Démonstration La vérification des propriétés des normes est élémentaire.
Si la suite (f_n) converge dans
(ℬ(X,E),\ \\infty~), soit f
sa limite. On a alors \mu_n =\
sup_x\inX\f(x) -
f_n(x)\ =\ f
- f_n\\infty~. On en déduit que la suite
converge uniformément vers f. Inversement, si la suite converge
uniformément vers f : X \rightarrow~ E, il existe N \in \mathbb{N}~ tel que n ≥ N \rigtharrow~
\mu_n =\
sup_x\inX\f(x) -
f_n(x)\ < 1. La fonction f -
f_N est donc bornée~; comme f_N est bornée, la
fonction f est également bornée. On a alors \f
- f_n\\infty~ = \mu_n, ce qui montre
que la suite (f_n) converge vers f dans
(ℬ(X,E),\ \\infty~).

Remarque~10.1.5 On voit en particulier qu'une suite de fonctions bornées
qui converge uniformément a une limite qui est également une fonction
bornée.

Remarque~10.1.6 Soit (f_n) une suite de ℬ(X,E). Alors la suite
(f_n) vérifie le critère de Cauchy uniforme si et seulement
si~c'est une suite de Cauchy de (ℬ(X,E),\
\\infty~) (immédiat). On en déduit, d'après un
théorème précédent, que si E est complet,
(ℬ(X,E),\ \\infty~) est lui
aussi complet.

\paragraph{10.1.5 Opérations sur les fonctions}

Bien entendu, les théorèmes de continuité des opérations algébriques
s'appliquent immédiatement aux suites simplement convergentes puisque si
f(x) = limf_n~(x) et g(x)
= limg_n~(x), on a (\alpha~f + \beta~g)(x)
= lim(\alpha~f_n + \beta~g_n~)(x) et
f(x)g(x) = limf_n(x)g_n~(x).

La convergence uniforme est stable par combinaisons linéaires comme le
montre le théorème suivant.

Théorème~10.1.4 Soit X un ensemble, E un espace vectoriel normé. Soit
(f_n) et (g_n) deux suites d'applications de X dans E
qui convergent uniformément. Soit \alpha~ et \beta~ des scalaires. Alors la suite
(\alpha~f_n + \beta~g_n)_n\in\mathbb{N}~ converge uniformément.

Démonstration Soit f = limf_n~ et g
= limg_n~. Soit \epsilon > 0, et
N \in \mathbb{N}~ tel que

n ≥ N \rigtharrow~\forall~~x \in X, \f(x)
- f_n(x)\ \leq \epsilon \over
2(1 + \alpha~)

et

\g(x) -
g_n(x)\ \leq \epsilon \over 2(1
+ \beta~)

Alors pour n ≥ N, on a \forall~~x \in X,
\(\alpha~f + \beta~g)(x) - (\alpha~f_n +
\beta~g_n)(x)\ \leq\alpha~ \epsilon
\over 2(1+\alpha~) +
\beta~ \epsilon \over
2(1+\beta~) < \epsilon.

Par contre, la convergence uniforme n'est pas stable par produit comme
le montre l'exemple suivant~:

Exemple~10.1.4 Soit f_n : \mathbb{R}~ \rightarrow~ \mathbb{R}~ définie par f_n(x) =
1 \over n . La suite (f_n) converge
uniformément vers la fonction nulle. Soit g : \mathbb{R}~ \rightarrow~ \mathbb{R}~ définie par g(x) =
x. Alors la suite (f_ng) converge simplement vers 0, mais pas
uniformément puisque
sup_x\in\mathbb{R}~f_n~(x)g(x)
= sup_x\in\mathbb{R}~~\left
 x \over n \right 
= +\infty~. A fortiori, la convergence uniforme d'une suite (f_n) et
d'une suite (g_n) n'implique pas la convergence uniforme de la
suite (f_ng_n) (prendre g_n = g). Cependant,
on a le résultat suivant

Théorème~10.1.5 Soit X un ensemble. Soit (f_n) et
(g_n) deux suites d'applications bornées de X dans K qui
convergent uniformément. Alors la suite (f_ng_n)
converge uniformément.

Démonstration Soit f = limf_n~ et g
= limg_n~. On sait déjà que f et g
sont bornées. On écrit alors

\begin{align*} f(x)g(x) -
f_n(x)g_n(x)& =& (f_n(x) -
f(x))(g_n(x) - g(x)) \%& \\ &
\text & +f(x)(g_n(x) - g(x)) +
g(x)(f_n(x) - f(x))\%& \\
\end{align*}

ce qui nous donne

\begin{align*} f(x)g(x) -
f_n(x)g_n(x)& \leq& f_n(x)
- f(x)g_n(x) - g(x) \%&
\\ & &
+f(x)g_n(x) - g(x) +
g(x)f_n(x) - f(x)\%&
\\ \end{align*}

puis \fg -
f_ng_n\\infty~
\leq\ f_n -
f\\infty~\g_n -
g\\infty~ +\
f\\infty~\f -
f_n\\infty~ +\
g\\infty~\g -
g_n\\infty~. On obtient donc
lim~\fg -
f_ng_n\\infty~ = 0 et donc
(f_ng_n) converge uniformément vers fg.

\paragraph{10.1.6 Propriétés de la convergence uniforme}

Exemple~10.1.5 Soit f_n : [0,1] \rightarrow~ \mathbb{R}~,
x\mapsto~x^n. La suite f_n
converge simplement vers f : [0,1] \rightarrow~ \mathbb{R}~ définie par f(x) =
\left \ \cases 1&si x
= 1 \cr 0&si x\neq~1 
\right .. Chacune des fonctions f_n est continue
au point 1, alors que f ne l'est pas. De nouveau, on a 1
= lim_n\rightarrow~+\infty~~\left
(lim_x\rightarrow~1^-x^n~\right
)\neq~lim_x\rightarrow~1^-~\left
(lim_n\rightarrow~+\infty~x^n~\right
) = 0.

Théorème~10.1.6~(conservation de la continuité) Soit E et F deux espaces
métriques, (f_n)_n\in\mathbb{N}~ une suite d'applications de E
dans F qui converge simplement vers f : E \rightarrow~ F. Soit a \in E. On suppose
que (i) chacune des f_n est continue au point a (ii) il existe
U voisinage de a telle que la suite (f_n) converge uniformément
sur U Alors f est continue au point a.

Démonstration Soit \epsilon > 0 et soit N \in \mathbb{N}~ tel que n ≥ N
\rigtharrow~\forall~x \in U, d(f(x),f_n~(x)) <
\epsilon \over 3 . Comme f_N est continue au point a,
il existe V voisinage de a tel que x \in V \rigtharrow~
d(f_N(x),f_N(a)) < \epsilon \over
3 . Alors, pour x \in U \bigcap V , on a d(f(x),f(a)) \leq
d(f(x),f_N(x)) + d(f_N(x),f_N(a)) +
d(f_N(a),f(a)) \leq \epsilon \over 3 + \epsilon
\over 3 + \epsilon \over 3 = \epsilon. Donc f est
continue au point a.

Corollaire~10.1.7 Soit E et F deux espaces métriques,
(f_n)_n\in\mathbb{N}~ une suite d'applications continues de E dans
F qui converge uniformément vers f : E \rightarrow~ F. Alors f est continue.

Remarque~10.1.7 Il suffit évidemment que tout point ait un voisinage sur
lequel la suite converge uniformément, ce que l'on appelle la
convergence uniforme locale.

Théorème~10.1.8~(interversion des limites) Soit E un espace métrique, F
un espace métrique complet, (f_n)_n\in\mathbb{N}~ une suite de
fonctions de E dans F. Soit a \in E, A \subset~ E tel que a
\in\overlineA et \forall~~n \in \mathbb{N}~, A
\subset~ Def (f_n~). On suppose que

\begin{itemize}
\itemsep1pt\parskip0pt\parsep0pt
\item
  (i) la suite f_n converge uniformément vers f sur A
\item
  (ii) chacune des f_n a une limite \ell_n en a suivant A
\end{itemize}

Alors la suite (\ell_n) admet une limite \ell et f admet \ell pour
limite en a suivant A, autrement dit

lim_n\rightarrow~+\infty~~\left
(lim_x\rightarrow~a,x\inAf_n~(x)\right
) = lim_x\rightarrow~a,x\inA~\left
(lim_n\rightarrow~+\infty~f_n~(x)\right
)

Démonstration Pour montrer que la suite (\ell_n) admet une limite
\ell, il suffit de montrer que c'est une suite de Cauchy. Mais, la suite
(f_n) vérifie le critère de Cauchy uniforme sur A. Soit \epsilon
> 0~; il existe N \in \mathbb{N}~ tel que p,q ≥ N
\rigtharrow~\forall~x \in A, d(f_p(x),f_q~(x))
< \epsilon. Soit p,q ≥ N~; en faisant tendre x vers a en restant dans
A, on obtient d(\ell_p,\ell_q) \leq \epsilon ce qui montre
effectivement que la suite (\ell_n) est une suite de Cauchy de F,
donc qu'elle converge.

Soit \epsilon > 0 et soit N \in \mathbb{N}~ tel que n ≥ N
\rigtharrow~\forall~x \in A, d(f(x),f_n~(x)) <
\epsilon \over 3 et soit N' \in \mathbb{N}~ tel que n ≥ N' \rigtharrow~
d(\ell_n,\ell) < \epsilon \over 3 . Soit n
= max(N,N'). Comme f_n~ admet
\ell_n pour limite en a suivant A, il existe U voisinage de a dans
E tel que x \in U \bigcap A \rigtharrow~ d(f_n(x),\ell_n) \leq \epsilon
\over 3 . Alors, pour x \in U \bigcap A, on a

\begin{align*} d(f(x),\ell)& \leq&
d(f(x),f_n(x)) + d(f_n(x),\ell_n) +
d(\ell_n,\ell)\%& \\ &
<& \epsilon \over 3 + \epsilon \over
3 + \epsilon \over 3 = \epsilon \%&
\\ \end{align*}

ce qui montre que lim_x\rightarrow~a,x\inA~f(x) =
\ell.

Remarque~10.1.8 Le résultat suivant s'applique en particulier dans le
cas où a = +\infty~ et A = \mathbb{N}~, c'est-à-dire au cas d'une suite double
(x_n,p) d'éléments de F~: avec les hypothèses

\begin{itemize}
\itemsep1pt\parskip0pt\parsep0pt
\item
  (i) lim_n\rightarrow~+\infty~x_n,p~ =
  y_p uniformément par rapport à p
\item
  (ii) lim_p\rightarrow~+\infty~x_n,p~ =
  \ell_n
\end{itemize}

Alors la suite (\ell_n) admet une limite \ell et on a
lim_p\rightarrow~+\infty~y_p~ = \ell, autrement
dit

lim_n\rightarrow~+\infty~~\left
(lim_p\rightarrow~+\infty~x_n,p~\right
) = lim_p\rightarrow~+\infty~~\left
(lim_n\rightarrow~+\infty~x_n,p~\right
)

Exemple~10.1.6 Le résultat précédent utilise de manière essentielle la
convergence uniforme par rapport à p comme le montre l'exemple
x_n,p = n \over n+p pour lequel on a

0 = lim_n\rightarrow~+\infty~~\left
(lim_p\rightarrow~+\infty~x_n,p~\right
)\neq~lim_p\rightarrow~+\infty~~\left
(lim_n\rightarrow~+\infty~x_n,p~\right
) = 1

Théorème~10.1.9~(intégration) Soit (f_n) une suite de fonctions
réglées de [a,b] dans E (espace vectoriel normé complet) qui
converge uniformément vers f : [a,b] \rightarrow~ E. Alors f est réglée et la
suite (\int  _a^bf_n~(t)
dt) admet la limite \int ~
_a^bf(t) dt.

Démonstration Soit \epsilon > 0~; il existe N \in \mathbb{N}~ tel que n ≥ N
\rigtharrow~\ f - f_n\\infty~
< \epsilon \over 2 . Mais puisque f_N est
réglée, il existe \phi : [a,b] \rightarrow~ E en escalier telle que
\f_N - \phi\\infty~
< \epsilon \over 2 . On a donc
\f - \phi\\infty~
\leq\ f - f_N\\infty~
+\ f_N - \phi\\infty~
< \epsilon ce qui montre que f est encore réglée. On a alors

\\int ~
_a^bf -\int ~
_a^bf_ n\
\leq\int ~
_a^b\f - f_
n\ \leq (b - a)\f -
f_n\\infty~

ce qui montre que la suite (\int ~
_a^bf_n(t) dt) admet la limite
\int  _a^b~f(t) dt.

Remarque~10.1.9 Comme le montre la démonstration précédente, le fait que
l'intervalle soit borné est essentiel. Le résultat précédent ne s'étend
donc pas aux intégrales sur des intervalles non bornés (voir pour cela
le paragraphe sur les fonctions intégrables). Par contre on a

Corollaire~10.1.10 Soit I un intervalle de \mathbb{R}~, (f_n) une suite
de fonctions réglées de I dans E (espace vectoriel normé complet) qui
converge uniformément vers f : I \rightarrow~ E. Alors f est réglée. Soit a \in I,
F_n(x) =\int ~
_a^xf_n(t) dt et F(x)
=\int  _a^x~f(t) dt. Alors la
suite (F_n) converge uniformément vers F sur tout segment
inclus dans I.

Démonstration Le théorème précédent montre que f est réglée sur tout
segment inclus dans I, donc réglée. De plus, si J est un segment inclus
dans I, on peut, quitte à l'agrandir, supposer qu'il contient a. On a
alors, pour x \in J,

\begin{align*} \F(x) -
F_n(x)& \leq& \left
\int ~
_a^x\f(t) - f_
n(t)\ dt\right  \%&
\\ & \leq& x -
a\f -
f_n\\infty~ \leq
\ell(J)\f -
f_n\\infty~\%&
\\ \end{align*}

(en travaillant séparément dans les cas a \leq x et a > x), ce
qui montre la convergence uniforme sur J de F_n vers F.

Par contre, la convergence uniforme d'une suite de fonctions dérivables
n'implique pas que la limite soit elle-même dérivable. C'est même de
cette manière, par limite uniforme, qu'ont été construits les premiers
exemples de fonctions continues n'admettant de dérivée en aucun point
(voir le paragraphe sur les séries de fonctions). Par contre on a

Théorème~10.1.11 Soit I un intervalle de \mathbb{R}~, (f_n) une suite de
fonctions de I dans E (espace vectoriel normé complet) qui converge
simplement vers f : I \rightarrow~ E. On suppose que (i) chacune des f_n
est de classe \mathcal{C}^1 (ii) la suite (f_n') converge
uniformément sur I vers une fonction g. Alors f est de classe
\mathcal{C}^1 et f' = g.

Démonstration Soit a \in I. Puisque chaque f_n est de classe
\mathcal{C}^1, on a \forall~x \in I, f_n~(x) =
f_n(a) +\int ~
_a^xf_n'(t) dt. D'après le théorème précédent la
suite x\mapsto~\int ~
_a^xf_n'(t) dt converge uniformément sur tout
segment inclus dans I vers \int ~
_a^xg(t) dt. On obtient donc, en faisant tendre n vers +
\infty~, \forall~~x \in I, f(x) = f(a)
+\int  _a^x~g(t) dt. Comme g est
continue (limite uniforme de fonctions continues), f est de classe
\mathcal{C}^1 et f' = g.

Remarque~10.1.10 Comme le montre la démonstration précédente, il suffit,
avec les mêmes hypothèses, que la suite (f_n) converge en un
point a pour qu'elle converge simplement sur I, cette convergence étant
d'ailleurs uniforme sur tout segment inclus dans I. On retiendra que,
pour montrer la dérivabilité d'une limite de suites de fonctions, il
faut s'attacher à la convergence uniforme de la suite des dérivées, et
non à celle de la suite elle-même.

\paragraph{10.1.7 Suites de fonctions intégrables sur un intervalle}

Remarque~10.1.11 Les théorèmes du type
lim\\int  f_n~
=\int  \limf_n~
démontrés précédemment ont des hypothèses trop restrictives~: ils
nécessitent d'une part que l'intervalle soit borné et d'autre part que
la suite de fonctions converge uniformément sur tout l'intervalle. La
théorie de Lebesgue étend ces théorèmes à des situations plus générales
que nous n'étudierons pas en détail, mais d'où nous extrairons un
certain nombre de résultats utiles, qui ne seront pas démontrés en toute
généralité, mais seulement avec quelques hypothèses supplémentaires.

Nous admettrons le résultat fondamental suivant suivant dont la
démonstration est difficile

Lemme~10.1.12 Soit J un segment de \mathbb{R}~, (f_n)_n\in\mathbb{N}~ une
suite de fonctions continues par morceaux de J dans \mathbb{R}~^+
vérifiant

\begin{itemize}
\itemsep1pt\parskip0pt\parsep0pt
\item
  il existe M ≥ 0 tel que \forall~~n \in \mathbb{N}~,
  \forall~t \in J, f_n~(t) \leq M
\item
  la suite (f_n)_n\in\mathbb{N}~ converge simplement vers 0, soit
  \forall~~t \in J,
  lim_n\rightarrow~+\infty~f_n~(t) = 0
\end{itemize}

Alors la suite (\int ~
_Jf_n)_n\in\mathbb{N}~ converge vers 0.

On en déduit le lemme suivant

Lemme~10.1.13~(Convergence bornée sur un segment) Soit J un segment de
\mathbb{R}~, (f_n)_n\in\mathbb{N}~ une suite de fonctions continues par
morceaux de J dans \mathbb{C} qui converge simplement vers f continue par
morceaux. On suppose qu'il existe M ≥ 0 tel que
\forall~n \in \mathbb{N}~, \\forall~~t \in J,
f_n(t)\leq M. Alors la suite
(\int  _Jf_n)_n\in\mathbb{N}~~
admet la limite \int  _J~f.

Démonstration On pose g_n(t) = f(t) -
f_n(t). Comme \forall~~n \in \mathbb{N}~,
\forall~t \in J, f_n~(t)\leq
M, en passant à la limite on a f(t)\leq M, soit encore
0 \leq g(t) \leq 2M. D'autre part, \forall~~t \in J,
lim_n\rightarrow~+\infty~g_n~(t) = 0. D'après
le lemme précédent, on a
lim\\int ~
_Jg_n = 0. Or

\left \int  _J~f
-\int ~
_Jf_n\right  =
\left \int ~
_J(f - f_n)\right
\leq\int  _J~f -
f_n =\int ~
_Jg_n

qui tend vers 0. Autrement dit la suite (\\int
 _Jf_n)_n\in\mathbb{N}~ admet la limite
\int  _J~f.

Théorème~10.1.14~(convergence dominée) Soit I un intervalle de \mathbb{R}~,
(f_n) une suite de fonctions de I dans \mathbb{C} continues par morceaux
qui converge simplement vers f : I \rightarrow~ \mathbb{C} continue par morceaux. On suppose
qu'il existe \phi : I \rightarrow~ \mathbb{R}~^+ continue par morceaux et intégrable
sur I telle que \forall~~n \in \mathbb{N}~,
f_n\leq \phi (hypothèse de domination). Alors les
fonctions f_n et f sont intégrables sur I et la suite
(\int  _If_n~) est convergente
de limite \int  _I~f~:

\int  _I~f =\
lim_n\rightarrow~+\infty~\int  _If_n~

Démonstration Comme f_n(t)\leq \phi(t) et que \phi
est intégrable, les fonctions f_n sont intégrables sur I. De
plus, en faisant tendre n vers + \infty~, on a aussi f(t)\leq
\phi(t), donc f est également intégrable sur I. Soit J un segment inclus
dans I. On a

\begin{align*} \left
\int  _I~f
-\int ~
_If_n\right & \leq&
\left \int  _I~f
-\int  _J~f\right
 + \left \\int
 _Jf -\int ~
_Jf_n\right  +
\left \int ~
_Jf_n -\int ~
_If_n\right \%&
\\ & =& \left
\int ~
_I\diagdownJf\right  + \left
\int  _J~f
-\int ~
_Jf_n\right  +
\left \int ~
_I\diagdownJf_n\right  \%&
\\ & \leq& \int ~
_I\diagdownJ\phi + \left
\int  _J~f
-\int ~
_Jf_n\right 
+\int  _I\diagdownJ~\phi \%&
\\ \end{align*}

Comme \phi est intégrable positive, on a \int ~
_I\phi =\
sup\\int ~
_J\phi∣J \subset~ I\. Soit donc
\epsilon > 0~; il existe J segment inclus dans I tel que
\int  _I\phi -\epsilon\over 3~
\leq\int  _J~\phi \leq\\int
 _I\phi, soit encore 0 \leq\int ~
_I\diagdownJ\phi \leq \epsilon\over 3. Fixons un tel segment J~;
sur ce segment, la suite f_n converge simplement vers f et
f_n(t)\leq \phi(t)\leq M avec M
= sup_t\inJ~\phi(t) (qui existe puisque \phi
est continue par morceaux, donc bornée sur tout segment). Le lemme de
convergence bornée sur un segment assure que \\int
 _Jf =\
lim_n\rightarrow~+\infty~\int  _Jf_n~~;
donc il existe N \in \mathbb{N}~ tel que n ≥ N \rigtharrow~\left
\int  _J~f
-\int ~
_Jf_n\right  <
\epsilon\over 3. Alors, pour n ≥ N, on a

\left \int  _I~f
-\int ~
_If_n\right \leq
2\epsilon\over 3 + \epsilon\over 3 = \epsilon

ce qui montre bien que la suite (\int ~
_If_n) est convergente de limite
\int  _I~f

Remarque~10.1.12 Il est important de constater que l'hypothèse de
domination par une fonction intégrable \phi indépendante de n sert non
seulement à garantir l'intégrabilité des f_n et de f, mais est
également un argument essentiel de la démonstration de
\int  _I~f =\
lim_n\rightarrow~+\infty~\int  _If_n~,
et donc de la validité du résultat. Comme on l'a déjà vu avec la suite
de fonctions continues sur [0, \pi~\over 2 ],
t\mapsto~n^\alpha~\
sin ^n-1tcos~ t, une suite de
fonctions intégrables peut très bien converger simplement vers une
fonction intégrable sans que l'on ait \int ~
_If =\
lim_n\rightarrow~+\infty~\int  _If_n~.

Théorème~10.1.15~(convergence monotone) Soit I un intervalle de \mathbb{R}~,
(f_n) une suite croissante de fonctions de I dans \mathbb{R}~ continues
par morceaux et intégrables sur I, qui converge simplement vers f : I \rightarrow~
\mathbb{R}~ continue par morceaux. Alors la suite (\int ~
_If_n) est majorée si et seulement si la fonction f est
intégrable. Dans ces conditions on a

\int  _I~f =\
sup_n\in\mathbb{N}~\int  _If_n~
= lim_n\rightarrow~+\infty~~\\int
 _If_n

Démonstration En rempla\ccant éventuellement
f_n par f_n - f_0 et f par f - f_0,
on peut supposer que les fonctions f_n sont positives, et donc
f également.

Supposons tout d'abord que la fonction f est intégrable. On a alors
\forall~t \in I, f_n~(t) =
f_n(t) \leq f(t), et le théorème de convergence dominée assure que
la suite (croissante) (\int ~
_If_n)_n\in\mathbb{N}~ converge vers
\int  _I~f~; en particulier elle est
majorée.

Supposons en sens inverse que que la suite
(\int  _If_n)_n\in\mathbb{N}~~ est
majorée par M. Soit J un segment inclus dans I~; on a donc 0
\leq\int  _Jf_n~
\leq\int  _If_n~ \leq M, mais d'autre
part, on a \forall~~t \in J,
f_n(t) = f_n(t) \leq f(t) et f est
intégrable sur le segment J puisqu'elle est continue par morceaux sur ce
segment. On a donc \int  _J~f
= lim\\int ~
_Jf_n \leq M par le théorème de convergence dominée. Pour
tout segment J \subset~ I, on a \int  _J~f \leq M
et f est positive, par définition même, elle est intégrable sur I, ce
qui achève la démonstration de l'équivalence

[
[

\end{document}

% \documentclass[]{article}
\usepackage[T1]{fontenc}
\usepackage{lmodern}
\usepackage{amssymb,amsmath}
\usepackage{ifxetex,ifluatex}
\usepackage{fixltx2e} % provides \textsubscript
% use upquote if available, for straight quotes in verbatim environments
\IfFileExists{upquote.sty}{\usepackage{upquote}}{}
\ifnum 0\ifxetex 1\fi\ifluatex 1\fi=0 % if pdftex
  \usepackage[utf8]{inputenc}
\else % if luatex or xelatex
  \ifxetex
    \usepackage{mathspec}
    \usepackage{xltxtra,xunicode}
  \else
    \usepackage{fontspec}
  \fi
  \defaultfontfeatures{Mapping=tex-text,Scale=MatchLowercase}
  \newcommand{\euro}{€}
\fi
% use microtype if available
\IfFileExists{microtype.sty}{\usepackage{microtype}}{}
\ifxetex
  \usepackage[setpagesize=false, % page size defined by xetex
              unicode=false, % unicode breaks when used with xetex
              xetex]{hyperref}
\else
  \usepackage[unicode=true]{hyperref}
\fi
\hypersetup{breaklinks=true,
            bookmarks=true,
            pdfauthor={},
            pdftitle={Series de fonctions},
            colorlinks=true,
            citecolor=blue,
            urlcolor=blue,
            linkcolor=magenta,
            pdfborder={0 0 0}}
\urlstyle{same}  % don't use monospace font for urls
\setlength{\parindent}{0pt}
\setlength{\parskip}{6pt plus 2pt minus 1pt}
\setlength{\emergencystretch}{3em}  % prevent overfull lines
\setcounter{secnumdepth}{0}
 
/* start css.sty */
.cmr-5{font-size:50%;}
.cmr-7{font-size:70%;}
.cmmi-5{font-size:50%;font-style: italic;}
.cmmi-7{font-size:70%;font-style: italic;}
.cmmi-10{font-style: italic;}
.cmsy-5{font-size:50%;}
.cmsy-7{font-size:70%;}
.cmex-7{font-size:70%;}
.cmex-7x-x-71{font-size:49%;}
.msbm-7{font-size:70%;}
.cmtt-10{font-family: monospace;}
.cmti-10{ font-style: italic;}
.cmbx-10{ font-weight: bold;}
.cmr-17x-x-120{font-size:204%;}
.cmsl-10{font-style: oblique;}
.cmti-7x-x-71{font-size:49%; font-style: italic;}
.cmbxti-10{ font-weight: bold; font-style: italic;}
p.noindent { text-indent: 0em }
td p.noindent { text-indent: 0em; margin-top:0em; }
p.nopar { text-indent: 0em; }
p.indent{ text-indent: 1.5em }
@media print {div.crosslinks {visibility:hidden;}}
a img { border-top: 0; border-left: 0; border-right: 0; }
center { margin-top:1em; margin-bottom:1em; }
td center { margin-top:0em; margin-bottom:0em; }
.Canvas { position:relative; }
li p.indent { text-indent: 0em }
.enumerate1 {list-style-type:decimal;}
.enumerate2 {list-style-type:lower-alpha;}
.enumerate3 {list-style-type:lower-roman;}
.enumerate4 {list-style-type:upper-alpha;}
div.newtheorem { margin-bottom: 2em; margin-top: 2em;}
.obeylines-h,.obeylines-v {white-space: nowrap; }
div.obeylines-v p { margin-top:0; margin-bottom:0; }
.overline{ text-decoration:overline; }
.overline img{ border-top: 1px solid black; }
td.displaylines {text-align:center; white-space:nowrap;}
.centerline {text-align:center;}
.rightline {text-align:right;}
div.verbatim {font-family: monospace; white-space: nowrap; text-align:left; clear:both; }
.fbox {padding-left:3.0pt; padding-right:3.0pt; text-indent:0pt; border:solid black 0.4pt; }
div.fbox {display:table}
div.center div.fbox {text-align:center; clear:both; padding-left:3.0pt; padding-right:3.0pt; text-indent:0pt; border:solid black 0.4pt; }
div.minipage{width:100%;}
div.center, div.center div.center {text-align: center; margin-left:1em; margin-right:1em;}
div.center div {text-align: left;}
div.flushright, div.flushright div.flushright {text-align: right;}
div.flushright div {text-align: left;}
div.flushleft {text-align: left;}
.underline{ text-decoration:underline; }
.underline img{ border-bottom: 1px solid black; margin-bottom:1pt; }
.framebox-c, .framebox-l, .framebox-r { padding-left:3.0pt; padding-right:3.0pt; text-indent:0pt; border:solid black 0.4pt; }
.framebox-c {text-align:center;}
.framebox-l {text-align:left;}
.framebox-r {text-align:right;}
span.thank-mark{ vertical-align: super }
span.footnote-mark sup.textsuperscript, span.footnote-mark a sup.textsuperscript{ font-size:80%; }
div.tabular, div.center div.tabular {text-align: center; margin-top:0.5em; margin-bottom:0.5em; }
table.tabular td p{margin-top:0em;}
table.tabular {margin-left: auto; margin-right: auto;}
div.td00{ margin-left:0pt; margin-right:0pt; }
div.td01{ margin-left:0pt; margin-right:5pt; }
div.td10{ margin-left:5pt; margin-right:0pt; }
div.td11{ margin-left:5pt; margin-right:5pt; }
table[rules] {border-left:solid black 0.4pt; border-right:solid black 0.4pt; }
td.td00{ padding-left:0pt; padding-right:0pt; }
td.td01{ padding-left:0pt; padding-right:5pt; }
td.td10{ padding-left:5pt; padding-right:0pt; }
td.td11{ padding-left:5pt; padding-right:5pt; }
table[rules] {border-left:solid black 0.4pt; border-right:solid black 0.4pt; }
.hline hr, .cline hr{ height : 1px; margin:0px; }
.tabbing-right {text-align:right;}
span.TEX {letter-spacing: -0.125em; }
span.TEX span.E{ position:relative;top:0.5ex;left:-0.0417em;}
a span.TEX span.E {text-decoration: none; }
span.LATEX span.A{ position:relative; top:-0.5ex; left:-0.4em; font-size:85%;}
span.LATEX span.TEX{ position:relative; left: -0.4em; }
div.float img, div.float .caption {text-align:center;}
div.figure img, div.figure .caption {text-align:center;}
.marginpar {width:20%; float:right; text-align:left; margin-left:auto; margin-top:0.5em; font-size:85%; text-decoration:underline;}
.marginpar p{margin-top:0.4em; margin-bottom:0.4em;}
.equation td{text-align:center; vertical-align:middle; }
td.eq-no{ width:5%; }
table.equation { width:100%; } 
div.math-display, div.par-math-display{text-align:center;}
math .texttt { font-family: monospace; }
math .textit { font-style: italic; }
math .textsl { font-style: oblique; }
math .textsf { font-family: sans-serif; }
math .textbf { font-weight: bold; }
.partToc a, .partToc, .likepartToc a, .likepartToc {line-height: 200%; font-weight:bold; font-size:110%;}
.chapterToc a, .chapterToc, .likechapterToc a, .likechapterToc, .appendixToc a, .appendixToc {line-height: 200%; font-weight:bold;}
.index-item, .index-subitem, .index-subsubitem {display:block}
.caption td.id{font-weight: bold; white-space: nowrap; }
table.caption {text-align:center;}
h1.partHead{text-align: center}
p.bibitem { text-indent: -2em; margin-left: 2em; margin-top:0.6em; margin-bottom:0.6em; }
p.bibitem-p { text-indent: 0em; margin-left: 2em; margin-top:0.6em; margin-bottom:0.6em; }
.subsectionHead, .likesubsectionHead { margin-top:2em; font-weight: bold;}
.sectionHead, .likesectionHead { font-weight: bold;}
.quote {margin-bottom:0.25em; margin-top:0.25em; margin-left:1em; margin-right:1em; text-align:justify;}
.verse{white-space:nowrap; margin-left:2em}
div.maketitle {text-align:center;}
h2.titleHead{text-align:center;}
div.maketitle{ margin-bottom: 2em; }
div.author, div.date {text-align:center;}
div.thanks{text-align:left; margin-left:10%; font-size:85%; font-style:italic; }
div.author{white-space: nowrap;}
.quotation {margin-bottom:0.25em; margin-top:0.25em; margin-left:1em; }
h1.partHead{text-align: center}
.sectionToc, .likesectionToc {margin-left:2em;}
.subsectionToc, .likesubsectionToc {margin-left:4em;}
.sectionToc, .likesectionToc {margin-left:6em;}
.frenchb-nbsp{font-size:75%;}
.frenchb-thinspace{font-size:75%;}
.figure img.graphics {margin-left:10%;}
/* end css.sty */

\title{Series de fonctions}
\author{}
\date{}

\begin{document}
\maketitle

\textbf{Warning: 
requires JavaScript to process the mathematics on this page.\\ If your
browser supports JavaScript, be sure it is enabled.}

\begin{center}\rule{3in}{0.4pt}\end{center}

[
[
[]
[

\section{10.2 Séries de fonctions}

\subsection{10.2.1 Différents modes de convergence}

Remarque~10.2.1 Soit E un ensemble, F un espace vectoriel normé. Soit
(u_n)_n\in\mathbb{N}~ une suite d'applications de E dans F. On
s'intéressera ici à la série
\\sum ~
_n\in\mathbb{N}~u_n(x)~; c'est à la fois, pour chaque x \in E, une
série d'éléments de F et une suite de fonctions, la suite de ses sommes
partielles S_n =\
\sum  _p=0^nu_p~. On peut
donc déjà distinguer trois modes possibles de convergence de la série de
fonctions.

Définition~10.2.1 On dit que la série
\\sum ~
_n\in\mathbb{N}~u_n d'applications de E dans F converge (i)
simplement sur E si pour chaque x \in E, la série
\\sum ~
_n\in\mathbb{N}~u_n(x) converge (ii) absolument sur E si pour chaque
x \in E, la série \\sum ~
_n\in\mathbb{N}~u_n(x) converge absolument (autrement dit la série
\\sum ~
_n\in\mathbb{N}~\u_n(x)\
converge) (iii) uniformément sur E si la suite d'applications de E dans
F, x\mapsto~S_n(x)
= \\sum ~
_p=0^nu_p(x) converge uniformément sur E

Remarque~10.2.2 Les résultats sur les séries à valeurs dans F montrent
que si F est complet, la convergence absolue implique la convergence
simple. De même, les résultats sur les suites de fonctions montrent que
la convergence uniforme implique la convergence simple. Bien entendu, le
critère de Cauchy uniforme peut s'appliquer à des séries de fonctions à
valeurs dans un espace vectoriel normé complet, et on obtient

Théorème~10.2.1 Soit E un ensemble et F un espace vectoriel normé
complet. Une série \\\sum
 _n\in\mathbb{N}~u_n d'applications de E dans F est uniformément
convergente si et seulement si~elle vérifie le critère de Cauchy
uniforme

\forall~~\epsilon > 0,
\exists~N \in \mathbb{N}~, q ≥ p ≥ N
\rigtharrow~\forall~~x \in E,
\\\sum
_n=p^qu_ n(x)\
< \epsilon

Démonstration C'est tout simplement le critère de Cauchy pour les suites
de fonctions en remarquant que
\\sum ~
_n=p^qu_n = S_q - S_p-1.

Remarque~10.2.3 Nous allons introduire un quatrième mode de convergence
plus fort que les trois autres, la convergence normale~:

Définition~10.2.2 Soit
\\sum ~
_n\in\mathbb{N}~u_n une série d'applications de E dans F. On dit
qu'elle converge normalement si elle vérifie les conditions équivalentes
(i) chaque u_n est une application bornée et la série (à termes
réels positifs) \\sum ~
_n\in\mathbb{N}~\u_n\\infty~
est convergente (ii) il existe une série à termes réels positifs
\\sum ~
_n\in\mathbb{N}~\alpha_n qui converge et qui vérifie
\forall~~x \in E,
\u_n(x)\ \leq
\alpha_n.

Démonstration (i) \rigtharrow~(ii)~: prendre \alpha_n
=\ u_n\\infty~.

(ii) \rigtharrow~(i)~: il suffit de remarquer que 0 \leq\
u_n\\infty~ \leq \alpha_n pour avoir la
convergence de \\sum ~
_n\in\mathbb{N}~\u_n\\infty~.

Remarque~10.2.4 Montrer une convergence normale, c'est donc majorer
\u_n(x)\ par
une série convergente indépendante de x. On constate que la convergence
normale n'est autre que la convergence absolue dans
(ℬ(E,F),\._\infty~).

Théorème~10.2.2 Si F est complet, la convergence normale implique à la
fois la convergence absolue et la convergence uniforme.

Démonstration Pour tout x \in E, on a 0 \leq\
u_n(x)\ \leq\
u_n\\infty~, et donc si la série
\\sum ~
_n\in\mathbb{N}~\u_n\\infty~,
la série \\sum ~
_n\in\mathbb{N}~\u_n(x)\
converge. Pour montrer la convergence uniforme, puisque F est complet,
il suffit de montrer que le critère de Cauchy uniforme est vérifié~;
mais on a, pour x \in E,

\\\sum
_n=p^qu_ n(x)\
\leq\\sum
_n=p^q\u_
n(x)\ \leq\\sum
_n=p^q\u_
n\\infty~

Comme la série \\sum ~
\u_n\\infty~
converge, pour \epsilon > 0, il existe N \in \mathbb{N}~ tel que q
> p ≥ N
\rigtharrow~\\sum ~
_n=p^q\u_n\\infty~
< \epsilon. Alors

q ≥ p ≥ N \rigtharrow~\forall~~x \in E,
\\\sum
_n=p^qu_ n(x)\
< \epsilon

Exemple~10.2.1 Soit \alpha~ > 0 et soit la série d'applications
de \mathbb{R}~^+ dans \mathbb{R}~,
\\sum  _n≥1~ 1
\over n^\alpha~(1+nx) . On a 0 \leq 1
\over n^\alpha~(1+nx) \leq 1 \over
n^\alpha~ série indépendante de x. Cette dernière série converge
si \alpha~ > 1 et donc, si \alpha~ > 1, la série
\\sum  _n≥1~ 1
\over n^\alpha~(1+nx) converge normalement sur
\mathbb{R}~^+. Si \alpha~ \leq 1, la série diverge au point 0. Elle ne peut pas
converger uniformément sur ]0,+\infty~[, sinon elle vérifierait le critère
de Cauchy uniforme et pour \epsilon > 0, il existerait N \in \mathbb{N}~ tel
que q ≥ p ≥ N \rigtharrow~\forall~~x \in]0,+\infty~[,
\\sum ~
_n=p^q 1 \over n^\alpha~(1+nx)
< \epsilon~; mais alors, pour q et p fixés, en faisant tendre x vers
0 on obtiendrait \\\sum
 _n=p^q 1 \over n^\alpha~ \leq \epsilon,
donc la série \\sum ~ 
1 \over n^\alpha~ convergerait par le critère de
Cauchy, ce qui est absurde. Par contre, l'équivalent  1
\over n^\alpha~(1+nx) ∼ 1 \over
xn^\alpha~+1 > 0 montre que si x > 0 la
série converge (car \alpha~ + 1 > 1). Donc la série converge
simplement sur ]0,+\infty~[ (et même absolument puisque c'est une série à
termes positifs). Si a > 0, on a, pour x \in [a,+\infty~[, 0
\leq 1 \over n^\alpha~(1+nx) < 1
\over n^\alpha~(1+na) ∼ 1 \over
an^\alpha~+1 > 0, ce qui montre que la série
converge normalement sur [a,+\infty~[.

Remarque~10.2.5 Le même argument utilisant le critère de Cauchy uniforme
permet de montrer que si
\\sum  u_n~ est
une série d'applications continues de E dans F (complet) qui converge
uniformément sur une partie A de E, alors elle converge encore
uniformément sur l'adhérence \overlineA de A~; la
plupart du temps, les convergences uniformes se produisent donc sur des
ensembles fermés et toute affirmation d'une convergence uniforme sur une
partie non fermée doit immédiatement susciter une inquiétude légitime
(même si parfois elle peut être infondée, une ou plusieurs des
u_n pouvant ne pas être continue).

\subsection{10.2.2 Critères supplémentaires de convergence uniforme}

A part le critère de Cauchy uniforme, il y a peu de méthodes générales
permettant de montrer des convergences uniformes qui ne sont pas des
convergences normales. On retiendra cependant les deux cas suivants qui
sont importants.

Théorème~10.2.3~(convergence uniforme des séries alternées) Soit
(u_n)_n\in\mathbb{N}~ une suite d'applications de l'ensemble E
dans \mathbb{R}~ vérifiant les hypothèses suivantes (i) pour chaque x \in E, la
suite (u_n(x))_n\in\mathbb{N}~ est décroissante (ii) la suite
(u_n) converge uniformément vers la fonction nulle. Alors la
série \\sum ~
(-1)^nu_n converge uniformément sur E.

Démonstration Pour chaque x \in E, la série
\\sum ~
(-1)^nu_n(x) est convergente d'après le théorème sur
les séries alternées. De plus si l'on désigne par S_n sa somme
partielle d'indice n et par S sa somme, on sait (par le théorème sur les
séries alternées) que S(x) - S_n(x)\leq
u_n+1(x)~; la convergence uniforme de (u_n) vers la
fonction nulle implique donc la convergence uniforme de (S_n)
vers S.

Théorème~10.2.4~(critère d'Abel uniforme) Soit (a_n) une suite
d'applications de E dans \mathbb{R}~ et (u_n) une suite d'applications de
E dans l'espace vectoriel normé~complet F telles que (i)
\existsM ≥ 0, \\forall~~n \in \mathbb{N}~,
\forall~~x \in E,
\\\\sum
 _p=0^nu_p(x)\ \leq M
(ii) la suite (a_n) converge uniformément vers 0 en
décroissant. Alors la série
\\sum ~
a_nu_n converge uniformément

Démonstration On a, en posant S_n(x)
= \\sum ~
_p=0^nu_p(x)

\begin{align*} \\sum
_n=p^qa_ n(x)u_n(x)& =&
\sum _n=p^qa_
n(x)(S_n(x) - S_n-1(x)) \%&
\\ & =& \\sum
_n=p^qa_ n(x)S_n(x)
-\sum _n=p^qa_
n(x)S_n-1(x) \%& \\ & =&
\sum _n=p^qa_
n(x)S_n(x) -\\sum
_n=p-1^q-1a_ n+1(x)S_n(x)\%&
\\ \text(changement
d'indices \$n - 1\mapsto~n\$)&& \%&
\\ & =& a_q(x)S_q(x) -
a_p(x)S_p-1(x) \%& \\
& \text & +\\sum
_n=p^q-1(a_ n(x) -
a_n+1(x))S_n(x) \%& \\
\end{align*}

On a effectué ici une transformation d'Abel. Comme
\forall~n, \\forall~~x \in E,
\S_n(x)\ \leq M
on a

\\\sum
_n=p^qa_
n(x)u_n(x)\ \leq
M(a_q(x) +
a_p(x) + \\sum
_n=p^q-1a_ n(x) -
a_n+1(x)) = 2Ma_p(x)

en tenant compte de a_n(x) ≥ 0 et a_n(x) -
a_n+1(x) ≥ 0. Comme la suite (a_n) converge
uniformément vers 0, la série
\\sum ~
a_nu_n vérifie le critère de Cauchy uniforme, donc
elle converge uniformément.

\subsection{10.2.3 Propriétés de la convergence uniforme}

Il suffit d'appliquer à la suite (S_n) d'applications de E dans
F les résultats sur les suites de fonctions pour obtenir les théorèmes
suivants

Théorème~10.2.5~(conservation de la continuité) Soit E un espace
métrique, F un espace vectoriel normé. Soit
\\sum ~
_n\in\mathbb{N}~u_n une série d'applications de E dans F qui
converge simplement, de somme S : E \rightarrow~ F,
x\mapsto~S(x) =\
\sum  _n=0^+\infty~u_n~(x).
Soit a \in E. On suppose que (i) chacune des u_n est continue au
point a (ii) il existe U voisinage de a telle que la série
\\sum  u_n~
converge uniformément sur U Alors S est continue au point a.

Démonstration Chacune des u_n étant continue en a, il en est de
même de S_n.

Corollaire~10.2.6 Soit E un espace métrique, F un espace vectoriel
normé. Soit \\sum ~
_n\in\mathbb{N}~u_n une série d'applications continues de E dans F
qui converge uniformément. Alors la somme S de la série est continue.

Remarque~10.2.6 Il suffit évidemment que tout point ait un voisinage sur
lequel la série converge uniformément, ce que l'on appelle la
convergence uniforme locale.

Théorème~10.2.7~(interversion des limites) Soit E un espace métrique, F
un espace vectoriel normé complet. Soit
\\sum ~
_n\in\mathbb{N}~u_n une série de fonctions de E dans F. Soit a \in E,
A \subset~ E tel que a \in\overlineA et
\forall~n \in \mathbb{N}~, A \subset~\ Def~
(u_n). On suppose que

\begin{itemize}
\itemsep1pt\parskip0pt\parsep0pt
\item
  (i) la série \\sum ~
  u_n converge uniformément sur A~; soit S sa somme
\item
  (ii) chacune des u_n a une limite \ell_n en a suivant A
\end{itemize}

Alors la série \\sum ~
\ell_n converge et x\mapsto~S(x) admet
\\sum ~
_n=0^+\infty~\ell_n pour limite en a suivant A, autrement
dit

\sum _n=0^+\infty~~\left
(lim_ x\rightarrow~a,x\inAu_n(x)\right ) =
lim_x\rightarrow~a,x\inA\left (\\sum
_n=0^+\infty~u_ n(x)\right )

Démonstration Il suffit de remarquer que S_n
= \\sum ~
_p=0^nu_p admet la limite
\\sum ~
_p=0^n\ell_p en a suivant A et d'appliquer le
théorème d'interversion des limites à la suite (S_n).

Remarque~10.2.7 Le résultat suivant s'applique en particulier dans le
cas où a = +\infty~ et A = \mathbb{N}~, c'est-à-dire au cas d'une suite double
(x_n,p) d'éléments de E~: avec les hypothèses

\begin{itemize}
\itemsep1pt\parskip0pt\parsep0pt
\item
  (i) la série \\sum ~
  _n\in\mathbb{N}~x_n,p converge uniformément par rapport à p
\item
  (ii) lim_p\rightarrow~+\infty~x_n,p~ =
  \ell_n
\end{itemize}

Alors la série \\sum ~
_n\in\mathbb{N}~\ell_n converge et

\sum _n=0^+\infty~~\left
(lim_ p\rightarrow~+\infty~x_n,p\right ) =
lim_p\rightarrow~+\infty~\left (\\sum
_n=0^+\infty~x_ n,p\right )

Exemple~10.2.2 Le résultat précédent utilise de manière essentielle la
convergence uniforme par rapport à p comme le montre l'exemple
x_n,p = n \over n+p - n-1
\over n+p-1 = p \over (n+p)(n+p-1)
pour lequel on a

0 = \\sum
_n=1^+\infty~\left (lim_
p\rightarrow~+\infty~x_n,p\right
)\neq~lim_p\rightarrow~+\infty~\left
(\sum _n=1^+\infty~x_
n,p\right ) = 1

Théorème~10.2.8~(intégration) Soit
\\sum  u_n~ une
suite de fonctions réglées de [a,b] dans E (espace vectoriel normé
complet) qui converge uniformément sur [a,b], de somme S : [a,b]
\rightarrow~ E. Alors S est réglée et la série
\\sum ~
_n\in\mathbb{N}~\int ~
_a^bu_n(t) dt converge, de somme
\int  _a^b~S(t) dt, autrement dit
\int  _a^b~\left
(\\sum ~
_n=0^+\infty~u_n(t)\right ) dt
= \\sum ~
_n=0^+\infty~\int ~
_a^bu_n(t) dt (interversion du signe somme et du
signe intégrale).

Démonstration Il suffit d'appliquer le théorème correspondant sur les
suites de fonctions en remarquant que S_n est réglée et que
\int  _a^bS_n~(t) dt
= \\sum ~
_p=0^n\int ~
_a^bu_p(t) dt

Remarque~10.2.8 Comme pour les suites de fonctions, le fait que
l'intervalle soit borné est essentiel. Le résultat précédent ne s'étend
donc pas aux intégrales impropres sur des intervalles non bornés. Par
contre on a

Corollaire~10.2.9 Soit I un intervalle de \mathbb{R}~,
\\sum  u_n~ une
suite de fonctions réglées de I dans E (espace vectoriel normé complet)
qui converge uniformément sur I, de somme S : I \rightarrow~ E. Alors S est réglée.
Soit a \in I, U_n(x) =\int ~
_a^xu_n(t) dt et U(x)
=\int  _a^x~S(t) dt. Alors la
série \\sum ~
_n\in\mathbb{N}~U_n converge uniformément sur tout segment inclus
dans I et elle admet U pour somme.

La convergence uniforme d'une série de fonctions dérivables n'implique
pas que la somme soit elle-même dérivable. C'est même de cette manière,
par limite uniforme, qu'ont été construits les premiers exemples de
fonctions continues n'admettant de dérivée en aucun point (voir
ci-dessous). Par contre on a

Théorème~10.2.10 Soit I un intervalle de \mathbb{R}~,
\\sum  u_n~ une
suite d'applications de I dans E qui converge simplement sur I, de somme
S : I \rightarrow~ E. On suppose que (i) chacune des u_n est de classe
\mathcal{C}^1 (ii) la série
\\sum  u_n~'
converge uniformément sur I Alors S est de classe \mathcal{C}^1 et
\forall~~x \in I, S'(x) =\
\sum  _n=0^+\infty~u_n~'(x).

Démonstration Il suffit de remarquer que les S_n sont de classe
\mathcal{C}^1 et que S_n'(x) =\
\sum  _p=0^nu_p~'(x). Il
ne reste plus qu'à appliquer le théorème correspondant sur les suites de
fonctions.

Remarque~10.2.9 Comme pour les suites de fonctions, il suffit, avec les
mêmes hypothèses, que la suite
\\sum  u_n~
converge en un point a pour qu'elle converge simplement sur I, cette
convergence étant d'ailleurs uniforme sur tout segment inclus dans I. On
retiendra donc, que pour montrer la dérivabilité d'une somme de série de
fonctions, il faut s'attacher à la convergence uniforme de la série des
dérivées, et non à celle de la série elle-même.

Exemple~10.2.3 Etude de la fonction \zeta de Riemann~: on pose, pour x
> 1, \zeta(x) =\
\sum  _n=1^+\infty~~ 1
\over n^x . Pour a > 1 et x \in
[a,+\infty~[, on a  1 \over n^x \leq 1
\over n^a qui est une série convergente
indépendante de a. Donc la série converge normalement sur [a,+\infty~[ et
la fonction \zeta est continue sur [a,+\infty~[ quel que soit a >
1~; elle est donc continue sur ]1,+\infty~[. La fonction x \rightarrow~ 1
\over n^x est de classe C^\infty~ et sa
dérivée p-ième est  (-1)^p(log~
n)^p \over n^x . Si a
> 1, la série de fonctions
\\sum  _n≥1~
(-1)^p(log n)^p~
\over n^x converge normalement sur
[a,+\infty~[ (car \left 
(-1)^p(log n)^p~
\over n^x \right
\leq (log n)^p~
\over n^a qui est une série de Bertrand
convergente, indépendante de x) et donc \zeta est de classe C^p
sur [a,+\infty~[ avec \zeta^(p)(x) =\
\sum  _n=1^+\infty~~
(-1)^p(log n)^p~
\over n^x . Comme a est quelconque avec a
> 1, \zeta est de classe C^\infty~ sur ]1,+\infty~[ et on a
la formule ci-dessus.

Exemple~10.2.4 Nous allons donner un exemple de fonction continue sur un
intervalle, qui n'est dérivable en aucun point. Posons pour cela f(x)
= \\sum ~
_n=0^+\infty~a^n cos~
(b^n\pi~x) avec 0 < a < 1 et b entier
multiple de 4. La majoration \left
a^ncos(b^n\pi~x)\right
 \leq a^n montre que la série converge normalement sur
\mathbb{R}~ et que sa somme est donc une fonction continue sur \mathbb{R}~. On a 
f(x+h)-f(x) \over h =\
\sum  _n=0^+\infty~a^n~
cos~
(b^n\pi~(x+h))-cos (b^n~\pi~x)
\over h . Prenons en particulier h = 1
\over b^p où p est un entier. On a alors,

\begin{align*} f(x + h) - f(x)
\over h & =& \\sum
_n=0^+\infty~a^n \cos
(b^n\pi~x + b^nh\pi~) - \cos
(b^n\pi~x) \over h \%&
\\ & =& \\sum
_n=0^pa^n \cos
(b^n\pi~x + b^nh\pi~) - \cos
(b^n\pi~x) \over h \%&
\\ \end{align*}

car b^nh = b^n-p est un entier pair pour n
> p. On peut donc écrire  f(x+h)-f(x)
\over h = S_p-1 -
2b^pa^p cos~
(b^p\pi~x) avec

\begin{align*} S_p-1&
=& \left \\sum
_n=0^p-1a^n \cos
(b^n\pi~x + b^nh\pi~) - \cos
(b^n\pi~x) \over h \right
\%& \\ & =&
2\left \\sum
_n=0^p-1a^n \sin
(b^n\pi~x + b^nh\pi~ \over 2
)\sin ( b^nh\pi~ \over 2 )
\over h \right  \%&
\\ & \leq& 2\\sum
_n=0^p-1a^n\left 
\sin ( b^nh\pi~ \over 2 )
\over h \right  \%&
\\ & <&
\pi~\\sum
_n=0^p-1b^na^n \%&
\\ \end{align*}

en utilisant sin~ x
< x pour x > 0. Supposons a et b choisis de telle
sorte que ba - 1 > 2\pi~~; on a alors
S_p-1 < \pi~
b^pa^p-1 \over ba-1 <
\pi~ b^pa^p \over ba-1 et donc
S_p-1 = \epsilon_pb^pa^p avec
\epsilon_p < 1 \over
2 . On a alors  f(x+h)-f(x) \over h =
a^pb^p(\epsilon_ p -
2cos (b^p~\pi~x)). En suivant la même
méthode on peutécrire  f(x+ h \over 2 )-f(x)
\over  h \over 2  =
a^pb^p(\eta_p -
2\sqrt2cos~
(b^p\pi~x + \pi~ \over 4 )) avec
\eta_p < 1 \over
2 . Mais les deux nombres b^p\pi~x et b^p\pi~x + \pi~
\over 4 différant de  \pi~ \over 4 ,
l'un des deux cosinus au moins est en valeur absolue supérieur à
sin  \pi~ \over 8~ (exercice
facile), et donc on a soit \left  f(x+h)-f(x)
\over h \right ≥
a^pb^p(2sin~  \pi~
\over 8 - 1 \over 2 ), soit
\left  f(x+ h \over 2
)-f(x) \over  h \over 2 
\right ≥
a^pb^p(2\sqrt2sin~
 \pi~ \over 8 - 1 \over 2 ). Comme
lim_p\rightarrow~+\infty~b^pa^p~
= +\infty~, on a donc
limsup_h\rightarrow~0~\left
 f(x+h)-f(x) \over h \right
 = +\infty~, donc f n'est pas dérivable au point x.

\subsection{10.2.4 Séries de fonctions intégrables sur un intervalle}

Remarque~10.2.10 Comme pour les suites de fonctions, les théorèmes du
type \\sum ~
\int  u_n =\\int ~
\\sum  u_n~
démontrés précédemment ont des hypothèses trop restrictives~: ils
nécessitent d'une part que l'intervalle soit borné et d'autre part que
la série de fonctions converge uniformément sur tout l'intervalle. La
théorie de Lebesgue étend également ces théorèmes à des situations plus
générales d'où nous extrairons un certain nombre de résultats utiles.

Théorème~10.2.11~(convergence monotone) Soit I un intervalle de \mathbb{R}~,
\\sum  u_n~ une
série de fonctions de I dans \mathbb{R}~ positives, continues par morceaux et
intégrables sur I~; on suppose que la série converge simplement sur I et
que sa somme S = \\sum ~
u_n est continue par morceaux. Alors la série
\\sum ~
_n\in\mathbb{N}~\int  _Iu_n~ converge
si et seulement si la fonction S est intégrable. Dans ces conditions on
a

\int  _I~S =\\int
 _I \\sum
_n=0^+\infty~u_ n = \\sum
_n=0^+\infty~\\\int
  _Iu_n

Démonstration Il suffit d'appliquer le théorème de convergence monotone
pour les suites de fonctions à la suite S_n
= \\sum ~
_p=0^nu_p. C'est une suite croissante de
fonctions positives, intégrables et continues par morceaux qui converge
simplement vers S. Donc la suite des intégrales
\int  _IS_n~
= \\sum ~
_p=0^n\int  _Iu_p~
converge si et seulement si S est intégrable et dans ce cas
\int  _I~S =\
lim\int  _IS_n~~; donc la
série \\sum ~
_n\in\mathbb{N}~\int  _Iu_n~ converge
si et seulement si la fonction S est intégrable et dans ces conditions
on a

\int  _I~S =\\int
 _I \\sum
_n=0^+\infty~u_ n = \\sum
_n=0^+\infty~\\\int
  _Iu_n

Théorème~10.2.12~(intégration terme à terme) Soit I un intervalle de \mathbb{R}~,
\\sum  u_n~ une
série de fonctions de I dans \mathbb{C} continues par morceaux et intégrables sur
I~; on suppose que la série converge simplement sur I et que sa somme S
= \\sum  u_n~
est continue par morceaux. Si la série
\\sum ~
_n\in\mathbb{N}~\int ~
_Iu_n est convergente, alors S est
intégrable sur I, la série
\\sum ~
_n\in\mathbb{N}~\int  _Iu_n~ converge
et on a à la fois

\int  _I~S =\\int
 _I \\sum
_n=0^+\infty~u_ n = \\sum
_n=0^+\infty~\\\int
  _Iu_n\text et
\\int  ~
_IS\leq\\sum
_n=0^+\infty~\\\int
  _Iu_n

Démonstration Soit S_n : I \rightarrow~ \mathbb{R}~^+,
t\mapsto~\\\sum
 _k=0^nu_k(t), M_n : I \rightarrow~
\mathbb{R}~^+,
t\mapsto~\\\sum
 _k=0^nu_k(t) et
h_n : I \rightarrow~ \mathbb{R}~^+,
t\mapsto~min(S(t),M_n~(t)).
La formule classique min~(x,y) =
1\over 2(x + y -x - y) montre que
h_n est continue par morceaux. Puisque chacune des fonctions
u_k est intégrable sur I, il en est de même
de M_n et donc de h_n qui est dominée par
M_n~; on a aussi

\int  _Ih_n~
\leq\int  _IM_n~ =
\sum _k=0^n~
\\int  ~
_Iu_k\leq\\sum
_k=0^+\infty~\\\int
  _Iu_k

Comme M_n+1(t) ≥ M_n(t), la suite
(h_n)_n\in\mathbb{N}~ est croissante. Fixons t \in I et \epsilon
> 0~; il existe N \in \mathbb{N}~ tel que n ≥ N \rigtharrow~S(t) -
S_n(t) < \epsilon~; on a alors, pour n ≥ N,
S(t)- \epsilon \leqS_n(t)\leq
M_n(t) et donc S(t)- \epsilon \leq h_n(t)
=\
min(S(t),M_n(t))
\leqS(t), ce qui montre que la suite
(h_n)_n\in\mathbb{N}~ converge simplement vers
S, qui est continue par morceaux. On peut donc
appliquer le théorème de convergence monotone à la suite
(h_n)_n\in\mathbb{N}~ et en déduire que S est
intégrable avec \int ~
_IS =\
lim\int  _Ih_n~
\leq\\sum ~
_k=0^+\infty~\int ~
_Iu_k. On en conclut que S est
intégrable et que \left
\int  _I~S\right
\leq\int ~
_IS\leq\\\sum
 _k=0^+\infty~\int ~
_Iu_k.

On applique ce que l'on vient de démontrer à la série
\\sum ~
_p≥n+1u_p et l'on obtient

\left \int  _I~S
-\sum _k=0^n~
\\int  ~
_Iu_k\right  =
\left \int ~
_I(S -\sum _k=0^nu_
k)\right  = \left
\int  _I~
\sum _k=n+1^+\infty~u_
k\right \leq\\sum
_k=n+1^+\infty~\\\int
  _Iu_k

qui tend vers 0 quand n tend vers + \infty~ (reste d'une série convergente).
On a donc \int  _I~S
= \\sum ~
_k=0^+\infty~\int ~
_Iu_k.

Remarque~10.2.11 Il est important de constater que l'hypothèse de
convergence de la série
\\sum ~
_n\int ~
_Iu_n sert non seulement à garantir
l'intégrabilité de S et la convergence (absolue) de la série
\\sum ~
_n\int  _Iu_n~, mais est
également un argument essentiel de la démonstration de
\int  _I~\
\sum  _n=0^+\infty~u_n~
= \\sum ~
_n=0^+\infty~\int ~
_Iu_n, et donc de la validité du résultat. La série
\\sum  u_n~ peut
très bien converger avec une somme intégrable, la série
\\sum ~
\int  _Iu_n~ convergeant (même
absolument) sans que l'on ait \int ~
_I \\sum ~
_n=0^+\infty~u_n =\
\sum ~
_n=0^+\infty~\int ~
_Iu_n

[
[
[
[

\end{document}

% \documentclass[]{article}
\usepackage[T1]{fontenc}
\usepackage{lmodern}
\usepackage{amssymb,amsmath}
\usepackage{ifxetex,ifluatex}
\usepackage{fixltx2e} % provides \textsubscript
% use upquote if available, for straight quotes in verbatim environments
\IfFileExists{upquote.sty}{\usepackage{upquote}}{}
\ifnum 0\ifxetex 1\fi\ifluatex 1\fi=0 % if pdftex
  \usepackage[utf8]{inputenc}
\else % if luatex or xelatex
  \ifxetex
    \usepackage{mathspec}
    \usepackage{xltxtra,xunicode}
  \else
    \usepackage{fontspec}
  \fi
  \defaultfontfeatures{Mapping=tex-text,Scale=MatchLowercase}
  \newcommand{\euro}{€}
\fi
% use microtype if available
\IfFileExists{microtype.sty}{\usepackage{microtype}}{}
\ifxetex
  \usepackage[setpagesize=false, % page size defined by xetex
              unicode=false, % unicode breaks when used with xetex
              xetex]{hyperref}
\else
  \usepackage[unicode=true]{hyperref}
\fi
\hypersetup{breaklinks=true,
            bookmarks=true,
            pdfauthor={},
            pdftitle={Integrales dependant d'un param`etre},
            colorlinks=true,
            citecolor=blue,
            urlcolor=blue,
            linkcolor=magenta,
            pdfborder={0 0 0}}
\urlstyle{same}  % don't use monospace font for urls
\setlength{\parindent}{0pt}
\setlength{\parskip}{6pt plus 2pt minus 1pt}
\setlength{\emergencystretch}{3em}  % prevent overfull lines
\setcounter{secnumdepth}{0}
 
/* start css.sty */
.cmr-5{font-size:50%;}
.cmr-7{font-size:70%;}
.cmmi-5{font-size:50%;font-style: italic;}
.cmmi-7{font-size:70%;font-style: italic;}
.cmmi-10{font-style: italic;}
.cmsy-5{font-size:50%;}
.cmsy-7{font-size:70%;}
.cmex-7{font-size:70%;}
.cmex-7x-x-71{font-size:49%;}
.msbm-7{font-size:70%;}
.cmtt-10{font-family: monospace;}
.cmti-10{ font-style: italic;}
.cmbx-10{ font-weight: bold;}
.cmr-17x-x-120{font-size:204%;}
.cmsl-10{font-style: oblique;}
.cmti-7x-x-71{font-size:49%; font-style: italic;}
.cmbxti-10{ font-weight: bold; font-style: italic;}
p.noindent { text-indent: 0em }
td p.noindent { text-indent: 0em; margin-top:0em; }
p.nopar { text-indent: 0em; }
p.indent{ text-indent: 1.5em }
@media print {div.crosslinks {visibility:hidden;}}
a img { border-top: 0; border-left: 0; border-right: 0; }
center { margin-top:1em; margin-bottom:1em; }
td center { margin-top:0em; margin-bottom:0em; }
.Canvas { position:relative; }
li p.indent { text-indent: 0em }
.enumerate1 {list-style-type:decimal;}
.enumerate2 {list-style-type:lower-alpha;}
.enumerate3 {list-style-type:lower-roman;}
.enumerate4 {list-style-type:upper-alpha;}
div.newtheorem { margin-bottom: 2em; margin-top: 2em;}
.obeylines-h,.obeylines-v {white-space: nowrap; }
div.obeylines-v p { margin-top:0; margin-bottom:0; }
.overline{ text-decoration:overline; }
.overline img{ border-top: 1px solid black; }
td.displaylines {text-align:center; white-space:nowrap;}
.centerline {text-align:center;}
.rightline {text-align:right;}
div.verbatim {font-family: monospace; white-space: nowrap; text-align:left; clear:both; }
.fbox {padding-left:3.0pt; padding-right:3.0pt; text-indent:0pt; border:solid black 0.4pt; }
div.fbox {display:table}
div.center div.fbox {text-align:center; clear:both; padding-left:3.0pt; padding-right:3.0pt; text-indent:0pt; border:solid black 0.4pt; }
div.minipage{width:100%;}
div.center, div.center div.center {text-align: center; margin-left:1em; margin-right:1em;}
div.center div {text-align: left;}
div.flushright, div.flushright div.flushright {text-align: right;}
div.flushright div {text-align: left;}
div.flushleft {text-align: left;}
.underline{ text-decoration:underline; }
.underline img{ border-bottom: 1px solid black; margin-bottom:1pt; }
.framebox-c, .framebox-l, .framebox-r { padding-left:3.0pt; padding-right:3.0pt; text-indent:0pt; border:solid black 0.4pt; }
.framebox-c {text-align:center;}
.framebox-l {text-align:left;}
.framebox-r {text-align:right;}
span.thank-mark{ vertical-align: super }
span.footnote-mark sup.textsuperscript, span.footnote-mark a sup.textsuperscript{ font-size:80%; }
div.tabular, div.center div.tabular {text-align: center; margin-top:0.5em; margin-bottom:0.5em; }
table.tabular td p{margin-top:0em;}
table.tabular {margin-left: auto; margin-right: auto;}
div.td00{ margin-left:0pt; margin-right:0pt; }
div.td01{ margin-left:0pt; margin-right:5pt; }
div.td10{ margin-left:5pt; margin-right:0pt; }
div.td11{ margin-left:5pt; margin-right:5pt; }
table[rules] {border-left:solid black 0.4pt; border-right:solid black 0.4pt; }
td.td00{ padding-left:0pt; padding-right:0pt; }
td.td01{ padding-left:0pt; padding-right:5pt; }
td.td10{ padding-left:5pt; padding-right:0pt; }
td.td11{ padding-left:5pt; padding-right:5pt; }
table[rules] {border-left:solid black 0.4pt; border-right:solid black 0.4pt; }
.hline hr, .cline hr{ height : 1px; margin:0px; }
.tabbing-right {text-align:right;}
span.TEX {letter-spacing: -0.125em; }
span.TEX span.E{ position:relative;top:0.5ex;left:-0.0417em;}
a span.TEX span.E {text-decoration: none; }
span.LATEX span.A{ position:relative; top:-0.5ex; left:-0.4em; font-size:85%;}
span.LATEX span.TEX{ position:relative; left: -0.4em; }
div.float img, div.float .caption {text-align:center;}
div.figure img, div.figure .caption {text-align:center;}
.marginpar {width:20%; float:right; text-align:left; margin-left:auto; margin-top:0.5em; font-size:85%; text-decoration:underline;}
.marginpar p{margin-top:0.4em; margin-bottom:0.4em;}
.equation td{text-align:center; vertical-align:middle; }
td.eq-no{ width:5%; }
table.equation { width:100%; } 
div.math-display, div.par-math-display{text-align:center;}
math .texttt { font-family: monospace; }
math .textit { font-style: italic; }
math .textsl { font-style: oblique; }
math .textsf { font-family: sans-serif; }
math .textbf { font-weight: bold; }
.partToc a, .partToc, .likepartToc a, .likepartToc {line-height: 200%; font-weight:bold; font-size:110%;}
.chapterToc a, .chapterToc, .likechapterToc a, .likechapterToc, .appendixToc a, .appendixToc {line-height: 200%; font-weight:bold;}
.index-item, .index-subitem, .index-subsubitem {display:block}
.caption td.id{font-weight: bold; white-space: nowrap; }
table.caption {text-align:center;}
h1.partHead{text-align: center}
p.bibitem { text-indent: -2em; margin-left: 2em; margin-top:0.6em; margin-bottom:0.6em; }
p.bibitem-p { text-indent: 0em; margin-left: 2em; margin-top:0.6em; margin-bottom:0.6em; }
.paragraphHead, .likeparagraphHead { margin-top:2em; font-weight: bold;}
.subparagraphHead, .likesubparagraphHead { font-weight: bold;}
.quote {margin-bottom:0.25em; margin-top:0.25em; margin-left:1em; margin-right:1em; text-align:justify;}
.verse{white-space:nowrap; margin-left:2em}
div.maketitle {text-align:center;}
h2.titleHead{text-align:center;}
div.maketitle{ margin-bottom: 2em; }
div.author, div.date {text-align:center;}
div.thanks{text-align:left; margin-left:10%; font-size:85%; font-style:italic; }
div.author{white-space: nowrap;}
.quotation {margin-bottom:0.25em; margin-top:0.25em; margin-left:1em; }
h1.partHead{text-align: center}
.sectionToc, .likesectionToc {margin-left:2em;}
.subsectionToc, .likesubsectionToc {margin-left:4em;}
.subsubsectionToc, .likesubsubsectionToc {margin-left:6em;}
.frenchb-nbsp{font-size:75%;}
.frenchb-thinspace{font-size:75%;}
.figure img.graphics {margin-left:10%;}
/* end css.sty */

\title{Integrales dependant d'un param`etre}
\author{}
\date{}

\begin{document}
\maketitle

\textbf{Warning: \href{http://www.math.union.edu/locate/jsMath}{jsMath}
requires JavaScript to process the mathematics on this page.\\ If your
browser supports JavaScript, be sure it is enabled.}

\begin{center}\rule{3in}{0.4pt}\end{center}

{[}\href{coursse61.html}{prev}{]}
{[}\href{coursse61.html\#tailcoursse61.html}{prev-tail}{]}
{[}\hyperref[tailcoursse62.html]{tail}{]}
{[}\href{coursch11.html\#coursse62.html}{up}{]}

\subsubsection{10.3 Intégrales dépendant d'un paramètre}

\paragraph{10.3.1 Position du problème}

Soit E un espace métrique, a,b ∈ ℝ, E' un espace vectoriel normé complet
et f : E × {[}a,b{]} → E', (x,t)\textbackslash{}mathrel\{↦\}f(x,t). On
suppose que \textbackslash{}mathop\{∀\}x ∈ E, l'application
t\textbackslash{}mathrel\{↦\}f(x,t) est réglée de {[}a,b{]} dans E'. On
peut donc définir une application F : E → E' par F(x)
=\{\textbackslash{}mathop\{∫ \} \}\_\{a\}\^{}\{b\}f(x,t) dt. Nous allons
nous intéresser ici aux propriétés de la fonction F (continuité,
dérivabilité, intégration) en fonction de celles de f.

\paragraph{10.3.2 Continuité}

Théorème~10.3.1~(Continuité par convergence dominée) Soit E un espace
métrique, I un intervalle de ℝ, f : E × I → ℂ,
(x,t)\textbackslash{}mathrel\{↦\}f(x,t). On suppose (i) pour chaque x ∈
E, l'application t\textbackslash{}mathrel\{↦\}f(x,t) est continue par
morceaux sur I (ii) pour chaque t ∈ I, l'application
x\textbackslash{}mathrel\{↦\}f(x,t) est continue sur E (iii) il existe
une fonction φ : I → \{ℝ\}\^{}\{+\}, intégrable, telle que
\textbackslash{}mathop\{∀\}(x,t) ∈ E × I, \textbar{}f(x,t)\textbar{}≤
φ(t) (hypothèse de domination). Alors, pour tout x ∈ E, la fonction
t\textbackslash{}mathrel\{↦\}f(x,t) est intégrable sur I et
l'application F : E → ℂ,
x\textbackslash{}mathrel\{↦\}\{\textbackslash{}mathop\{∫ \}
\}\_\{I\}f(x,t) dt est continue sur E.

Démonstration L'intégrabilité de t\textbackslash{}mathrel\{↦\}f(x,t) est
claire avec la majoration \textbar{}f(x,t)\textbar{}≤ φ(t). Soit alors x
∈ E et (\{x\}\_\{n\}) une suite de E de limite x. Posons \{g\}\_\{n\}(t)
= f(\{x\}\_\{n\},t) et g(t) = f(x,t). La suite (\{g\}\_\{n\}) est une
suite de fonctions continues par morceaux sur I qui converge vers g
continue par morceaux et on a \textbar{}\{g\}\_\{n\}\textbar{}≤ φ avec φ
intégrable~; le théorème de convergence dominée assure que
\textbackslash{}mathop\{lim\}\{\textbackslash{}mathop\{∫ \}
\}\_\{I\}\{g\}\_\{n\} =\{\textbackslash{}mathop\{∫ \} \}\_\{I\}g soit
encore \{\textbackslash{}mathop\{lim\}\}\_\{n→+∞\}F(\{x\}\_\{n\}) =
F(x). Donc F est bien continue.

Remarque~10.3.1 Pour montrer que F est continue, il suffit de montrer
que sa restriction à tout compact K contenu dans E est continue, donc
qu'à tout compact K contenu dans E, on peut associer une fonction
\{φ\}\_\{K\} intégrable telle que \textbackslash{}mathop\{∀\}(x,t) ∈ K ×
I, \textbar{}f(x,t)\textbar{}≤ \{φ\}\_\{K\}(t).

Corollaire~10.3.2 Soit U un ouvert de \{ℝ\}\^{}\{n\}, a,b ∈ ℝ et f : U ×
{[}a,b{]} → ℂ continue, (x,t)\textbackslash{}mathrel\{↦\}f(x,t). Alors
l'application F : U → ℂ définie par F(x) =\{\textbackslash{}mathop\{∫ \}
\}\_\{a\}\^{}\{b\}f(x,t) dt est continue.

Démonstration Soit \{x\}\_\{0\} ∈ U et r \textgreater{} 0 tel que la
boule fermée B'(\{x\}\_\{0\},r) soit contenue dans U. La fonction f est
continue sur le compact B'(\{x\}\_\{0\},r) × {[}a,b{]}, donc elle y est
bornée~: soit M ≥ 0 tel que \textbackslash{}mathop\{∀\}(x,t) ∈
B'(\{x\}\_\{0\},r) × {[}a,b{]}, \textbar{}f(x,t)\textbar{}≤ M. Comme la
fonction constante t\textbackslash{}mathrel\{↦\}M est intégrable sur
{[}a,b{]} et bien entendue indépendante de x, le théorème de continuité
par convergence dominée montre que F est continue sur B'(\{x\}\_\{0\},r)
et en particulier qu'elle est continue au point \{x\}\_\{0\}.

Exemple~10.3.1 Soit 0 \textless{} a \textless{} b \textless{} +∞ et soit
\{Γ\}\_\{a,b\}(x) =\{\textbackslash{}mathop\{∫ \}
\}\_\{a\}\^{}\{b\}\{t\}\^{}\{x−1\}\{e\}\^{}\{−t\} dt. On a ici, f(x,t) =
\{t\}\^{}\{x−1\}\{e\}\^{}\{−t\} =\textbackslash{}mathop\{ exp\} (−t − (x
− 1)\textbackslash{}mathop\{log\} t) qui est une fonction continue de ℝ
× {[}a,b{]} dans ℝ (composée de fonctions continues). On en déduit que
\{Γ\}\_\{a,b\} est continue sur ℝ.

\paragraph{10.3.3 Dérivabilité}

Nous supposerons ici que E = I intervalle de ℝ. Soit \{t\}\_\{0\} ∈
{[}a,b{]}~; lorsque l'application
x\textbackslash{}mathrel\{↦\}f(x,\{t\}\_\{0\}) est dérivable en un point
\{x\}\_\{0\} ∈ I, sa dérivée au point \{x\}\_\{0\} sera notée \{ ∂f
\textbackslash{}over ∂x\} (\{x\}\_\{0\},\{t\}\_\{0\}).

Théorème~10.3.3~(Dérivabilité par convergence dominée) Soit J un
intervalle de ℝ, I un intervalle de ℝ, f : J × I → ℂ, (x,t) → f(x,t)
continue, admettant une dérivée partielle par rapport à x,
(x,t)\textbackslash{}mathrel\{↦\}\{ ∂f \textbackslash{}over ∂x\} (x,t),
continue sur J × I. On suppose que pour tout x ∈ E, la fonction
t\textbackslash{}mathrel\{↦\}f(x,t) est intégrable sur I et qu'il existe
une fonction φ : I → \{ℝ\}\^{}\{+\}, intégrable, telle que
\textbackslash{}mathop\{∀\}(x,t) ∈ J × I, \textbar{}\{ ∂f
\textbackslash{}over ∂x\} (x,t)\textbar{}≤ φ(t) (hypothèse de
domination). Alors, l'application F : J → ℂ,
x\textbackslash{}mathrel\{↦\}\{\textbackslash{}mathop\{∫ \}
\}\_\{I\}f(x,t) dt est de classe \{C\}\^{}\{1\} sur J et

\textbackslash{}mathop\{∀\}x ∈ J, F'(x) =\{\textbackslash{}mathop\{∫ \}
\}\_\{I\}\{ ∂f \textbackslash{}over ∂x\} (x,t) dt

Démonstration L'intégrabilité de t\textbackslash{}mathrel\{↦\}\{ ∂f
\textbackslash{}over ∂x\} (x,t) est claire avec la majoration
\textbar{}\{ ∂f \textbackslash{}over ∂x\} (x,t)\textbar{}≤ φ(t). Soit
alors x ∈ J et \{x\}\_\{n\} une suite de J
∖\textbackslash{}\{x\textbackslash{}\} de limite x. Posons
\{g\}\_\{n\}(t) =\{ f(x,t)−f(\{x\}\_\{n\},t) \textbackslash{}over
x−\{x\}\_\{n\}\} et g(t) =\{ ∂f \textbackslash{}over ∂x\} (x,t). La
suite (\{g\}\_\{n\}) est une suite de fonctions continues sur I qui
converge vers g continue. L'inégalité des accroissements finis assure
que \textbar{}f(x,t) − f(\{x\}\_\{n\},t)\textbar{}≤\textbar{}x −
\{x\}\_\{n\}\textbar{}\{\textbackslash{}mathop\{sup\}\}\_\{y∈{]}x,\{x\}\_\{n\}{[}\}\textbackslash{}left
\textbar{}\{ ∂f \textbackslash{}over ∂x\} (y,t)\textbackslash{}right
\textbar{}≤\textbar{}x − \{x\}\_\{n\}\textbar{}φ(t), d'où

\textbar{}\{g\}\_\{n\}(t)\textbar{} = \textbackslash{}left \textbar{}\{
f(x,t) − f(\{x\}\_\{n\},t) \textbackslash{}over x − \{x\}\_\{n\}\}
\textbackslash{}right \textbar{}≤ φ(t)

avec φ intégrable. Le théorème de convergence dominée assure alors que
\textbackslash{}mathop\{lim\}\{\textbackslash{}mathop\{∫ \}
\}\_\{I\}\{g\}\_\{n\} =\{\textbackslash{}mathop\{∫ \} \}\_\{I\}g, soit
encore que \{\textbackslash{}mathop\{lim\}\}\_\{n→+∞\}\{
F(\{x\}\_\{n\})−F(x) \textbackslash{}over \{x\}\_\{n\}−x\}
=\{\textbackslash{}mathop\{∫ \} \}\_\{I\}\{ ∂f \textbackslash{}over ∂x\}
(x,t) dt. Comme la suite (\{x\}\_\{n\}) est quelconque, on a

\{\textbackslash{}mathop\{lim\}\}\_\{t→x\}\{ F(t) − F(x)
\textbackslash{}over t − x\} =\{\textbackslash{}mathop\{∫ \} \}\_\{I\}\{
∂f \textbackslash{}over ∂x\} (x,t) dt

donc F est dérivable et F'(x) =\{\textbackslash{}mathop\{∫ \}
\}\_\{I\}\{ ∂f \textbackslash{}over ∂x\} (x,t) dt. La continuité de F'
relève du théorème précédent relatif à la continuité d'une intégrale
dépendant d'un paramètre, la fonction étant dominée indépendamment du
paramètre.

Corollaire~10.3.4 Soit J un intervalle de ℝ, I un intervalle de ℝ, f : J
× I → ℂ, (x,t)\textbackslash{}mathrel\{↦\}f(x,t) continue, admettant des
dérivées partielles par rapport à x, (x,t)\textbackslash{}mathrel\{↦\}\{
\{∂\}\^{}\{i\}f \textbackslash{}over ∂\{x\}\^{}\{i\}\} (x,t), continues
sur J × I, i =
1,\textbackslash{}mathop\{\textbackslash{}mathop\{\ldots{}\}\},k. On
suppose que pour tout x ∈ E, la fonction
t\textbackslash{}mathrel\{↦\}f(x,t) est intégrable sur I et qu'il existe
des fonctions
\{φ\}\_\{1\},\textbackslash{}mathop\{\textbackslash{}mathop\{\ldots{}\}\},\{φ\}\_\{k\}
: I → \{ℝ\}\^{}\{+\}, intégrables, telles que
\textbackslash{}mathop\{∀\}(x,t) ∈ J × I, \textbar{}\{ \{∂\}\^{}\{i\}f
\textbackslash{}over ∂\{x\}\^{}\{i\}\} (x,t)\textbar{}≤ \{φ\}\_\{i\}(t)
(hypothèses de domination). Alors, l'application F : J → ℂ,
x\textbackslash{}mathrel\{↦\}\{\textbackslash{}mathop\{∫ \}
\}\_\{I\}f(x,t) dt est de classe \{C\}\^{}\{k\} sur J et

\textbackslash{}mathop\{∀\}i ∈ {[}1,k{]}, \textbackslash{}mathop\{∀\}x ∈
J, \{F\}\^{}\{(i)\}(x) =\{\textbackslash{}mathop\{∫ \} \}\_\{I\}\{
\{∂\}\^{}\{i\}f \textbackslash{}over ∂\{x\}\^{}\{i\}\} (x,t) dt

Démonstration Récurrence évidente à partir du théorème précédent

Théorème~10.3.5 Soit I un intervalle de ℝ, a,b ∈ ℝ et f : I × {[}a,b{]}
→ ℂ, (x,t)\textbackslash{}mathrel\{↦\}f(x,t). On suppose que (i) Pour
chaque x ∈ I, l'application t\textbackslash{}mathrel\{↦\}f(x,t) est
continue par morceaux sur {[}a,b{]} (ii) Pour chaque (x,t) ∈ I ×
{[}a,b{]}, f admet une dérivée partielle par rapport à x, \{ ∂f
\textbackslash{}over ∂x\} (x,t) et que l'application
(x,t)\textbackslash{}mathrel\{↦\}\{ ∂f \textbackslash{}over ∂x\} (x,t)
est continue. Alors F : I ⇒ ℂ,
x\textbackslash{}mathrel\{↦\}\{\textbackslash{}mathop\{∫ \}
\}\_\{a\}\^{}\{b\}f(x,t) dt est de classe \{C\}\^{}\{1\} et
\textbackslash{}mathop\{∀\}\{x\}\_\{0\} ∈ I, F'(\{x\}\_\{0\})
=\{\textbackslash{}mathop\{∫ \} \}\_\{a\}\^{}\{b\}\{ ∂f
\textbackslash{}over ∂x\} (\{x\}\_\{0\},t) dt.

Démonstration Il suffit, comme dans le théorème correspondant de
continuité pour une intégrale dépendant d'un paramètre sur un segment,
de prendre \{x\}\_\{0\} ∈ I, un segment K, voisinage de \{x\}\_\{0\}
dans I, et d'utiliser le fait que la fonction
(x,t)\textbackslash{}mathrel\{↦\}\{ ∂f \textbackslash{}over ∂x\} (x,t)
est continue, donc bornée par un certain M sur le compact K × {[}a,b{]}.
La fonction constante M ainsi introduite est intégrable sur le segment
{[}a,b{]} et fournit ainsi une fonction dominante de la fonction
(x,t)\textbackslash{}mathrel\{↦\}\{ ∂f \textbackslash{}over ∂x\} (x,t)
sur K × {[}a,b{]}. Par le théorème de dérivation par convergence
dominée, F est dérivable sur K et en particulier elle est dérivable au
point \{x\}\_\{0\} avec F'(\{x\}\_\{0\}) =\{\textbackslash{}mathop\{∫ \}
\}\_\{a\}\^{}\{b\}\{ ∂f \textbackslash{}over ∂x\} (\{x\}\_\{0\},t) dt.

Exemple~10.3.2 Soit 0 \textless{} a \textless{} b \textless{} +∞ et soit
\{Γ\}\_\{a,b\}(x) =\{\textbackslash{}mathop\{∫ \}
\}\_\{a\}\^{}\{b\}\{t\}\^{}\{x−1\}\{e\}\^{}\{−t\} dt. On a ici, f(x,t) =
\{t\}\^{}\{x−1\}\{e\}\^{}\{−t\} =\textbackslash{}mathop\{ exp\} (−t + (x
− 1)\textbackslash{}mathop\{log\} t) qui admet une dérivée par rapport à
x, \{ ∂f \textbackslash{}over ∂x\} (x,t) =\textbackslash{}mathop\{ log\}
t \{t\}\^{}\{x−1\}\{e\}\^{}\{−t\} qui est une fonction continue de ℝ ×
{[}a,b{]} dans ℝ (composée de fonctions continues). On en déduit que
\{Γ\}\_\{a,b\} est de classe \{C\}\^{}\{1\} sur ℝ et que
\{Γ\}\_\{a,b\}'(x) =\{\textbackslash{}mathop\{∫ \}
\}\_\{a\}\^{}\{b\}\{t\}\^{}\{x−1\}\textbackslash{}mathop\{ log\} t
\{e\}\^{}\{−t\} dt. Une récurrence évidente montrera alors que
\{Γ\}\_\{a,b\} est de classe \{C\}\^{}\{∞\} et que
\{Γ\}\_\{a,b\}\^{}\{(n)\}(x) =\{\textbackslash{}mathop\{∫ \}
\}\_\{a\}\^{}\{b\}\{t\}\^{}\{x−1\}\{(\textbackslash{}mathop\{log\}
t)\}\^{}\{n\}\{e\}\^{}\{−t\} dt.

\paragraph{10.3.4 Théorème de Fubini sur un produit de segments}

Théorème~10.3.6~(Fubini sur un produit de segments) Soit f : {[}a,b{]} ×
{[}c,d{]} → ℂ continue. Alors

\{\textbackslash{}mathop\{∫ \} \}\_\{a\}\^{}\{b\}\textbackslash{}left
(\{\textbackslash{}mathop\{∫ \} \}\_\{c\}\^{}\{d\}f(x,y)
dy\textbackslash{}right ) dx =\{\textbackslash{}mathop\{∫ \}
\}\_\{c\}\^{}\{d\}\textbackslash{}left (\{\textbackslash{}mathop\{∫ \}
\}\_\{a\}\^{}\{b\}f(x,y) dx\textbackslash{}right ) dy

Démonstration On va démontrer que \textbackslash{}mathop\{∀\}t ∈
{[}a,b{]}, F(t) = G(t) où F(t) =\{\textbackslash{}mathop\{∫ \}
\}\_\{a\}\^{}\{t\}\textbackslash{}left (\{\textbackslash{}mathop\{∫ \}
\}\_\{c\}\^{}\{d\}f(x,y) dy\textbackslash{}right ) dx et G(t)
=\{\textbackslash{}mathop\{∫ \} \}\_\{c\}\^{}\{d\}\textbackslash{}left
(\{\textbackslash{}mathop\{∫ \} \}\_\{a\}\^{}\{t\}f(x,y)
dx\textbackslash{}right ) dy. Pour t = b, on aura alors le résultat
voulu. Comme F(a) = G(a) = 0, il suffit de démontrer que F et G sont
dérivables et que F' = G'. Mais on a F(t) =\{\textbackslash{}mathop\{∫
\} \}\_\{a\}\^{}\{t\}φ(x) dx avec φ(x) =\{\textbackslash{}mathop\{∫ \}
\}\_\{c\}\^{}\{d\}f(x,y) dy. Le théorème de continuité des intégrales
dépendant d'un paramètre sur un segment (conséquence du théorème de
continuité par convergence dominée) assure que φ est continue, donc que
F est dérivable et que F'(t) = φ(t) =\{\textbackslash{}mathop\{∫ \}
\}\_\{c\}\^{}\{d\}f(t,y) dy.

On a aussi G(t) =\{\textbackslash{}mathop\{∫ \} \}\_\{c\}\^{}\{d\}ψ(t,y)
dy avec ψ(t,y) =\{\textbackslash{}mathop\{∫ \} \}\_\{a\}\^{}\{t\}f(x,y)
dx. Comme x\textbackslash{}mathrel\{↦\}f(x,y) est continue,
t\textbackslash{}mathrel\{↦\}ψ(t,y) est de classe \{C\}\^{}\{1\} et
\{∂ψ\textbackslash{}over ∂t\} (t,y) = f(t,y). Mais f étant continue sur
le compact {[}a,b{]} × {[}c,d{]}, elle y est bornée et on a donc

\textbackslash{}mathop\{∀\}(t,y) ∈ {[}a,b{]} × {[}c,d{]},
\textbackslash{}left \textbar{}\{∂ψ\textbackslash{}over ∂t\}
(t,y)\textbackslash{}right \textbar{}≤\textbackslash{}\textbar{}
\{f\textbackslash{}\textbar{}\}\_\{∞\}

qui est une fonction (constante) de la variable y, intégrable sur
{[}c,d{]} et indépendante de t. Le théorème de dérivation par
convergence dominée assure que l'application G :
t\textbackslash{}mathrel\{↦\}\{\textbackslash{}mathop\{∫ \}
\}\_\{c\}\^{}\{d\}ψ(t,y) dy est dérivable et que G'(t)
=\{\textbackslash{}mathop\{∫ \}
\}\_\{c\}\^{}\{d\}\{∂ψ\textbackslash{}over ∂t\} (t,y) dy, soit encore
G'(t) =\{\textbackslash{}mathop\{∫ \} \}\_\{c\}\^{}\{d\}f(t,y) dy = φ(t)
= F'(t), ce qui achève la démonstration.

\paragraph{10.3.5 Intégrales sur un pavé ou un rectangle}

Définition~10.3.1 On appelle pavé de \{ℝ\}\^{}\{2\} (resp. rectangle de
\{ℝ\}\^{}\{2\}) toute partie de \{ℝ\}\^{}\{2\} de la forme {[}a,b{]} ×
{[}c,d{]} (resp I × I' où I et I' sont des intervalles de ℝ).

En appliquant le théorème de Fubini pour une fonction continue sur un
produit de segments, on est amené à donner la définition suivante~:

Définition~10.3.2 Soit P = {[}a,b{]} × {[}c,d{]} un pavé de
\{ℝ\}\^{}\{2\} et f : P → ℂ une fonction continue. On appelle intégrale
de la fonction f sur le pavé P le nombre complexe noté indifféremment
\textbackslash{}mathop\{∫ \} \{\textbackslash{}mathop\{∫ \} \}\_\{P\}f
ou \textbackslash{}mathop\{∫ \} \{\textbackslash{}mathop\{∫ \}
\}\_\{P\}f(x,y) dx dy défini par

\textbackslash{}mathop\{∫ \} \{\textbackslash{}mathop\{∫ \} \}\_\{P\}f
=\textbackslash{}mathop\{∫ \} \{\textbackslash{}mathop\{∫ \}
\}\_\{P\}f(x,y) dx dy =\{\textbackslash{}mathop\{∫ \}
\}\_\{a\}\^{}\{b\}\textbackslash{}left (\{\textbackslash{}mathop\{∫ \}
\}\_\{c\}\^{}\{d\}f(x,y) dy\textbackslash{}right ) dx
=\{\textbackslash{}mathop\{∫ \} \}\_\{c\}\^{}\{d\}\textbackslash{}left
(\{\textbackslash{}mathop\{∫ \} \}\_\{a\}\^{}\{b\}f(x,y)
dx\textbackslash{}right ) dy

On peut alors répéter les définitions et résultats qui nous ont permis
de définir les fonctions intégrables sur un intervalle à partir de
l'intégrale sur un segment, pour définir des fonctions intégrables sur
un rectangle à partir de la notion de intégrale sur un pavé. On donnera
donc les définitions et propriétés suivantes sans commentaire ou
démonstration.

\begin{itemize}
\itemsep1pt\parskip0pt\parsep0pt
\item
  Soit R un rectangle et f : R → ℝ continue positive. On dit que f est
  intégrable sur R s'il existe M ≥ 0 tel que pour tout pavé P ⊂ R on ait
  \textbackslash{}mathop\{∫ \} \{\textbackslash{}mathop\{∫ \} \}\_\{P\}f
  ≤ M. On pose alors \textbackslash{}mathop\{∫ \}
  \{\textbackslash{}mathop\{∫ \} \}\_\{R\}f =\{\textbackslash{}mathop\{
  sup\}\}\_\{P⊂R\}f~; on montre que si \{(\{P\}\_\{n\})\}\_\{n∈ℕ\} est
  une suite croissante de pavés contenus dans R dont la réunion est R,
  alors \textbackslash{}mathop\{∫ \} \{\textbackslash{}mathop\{∫ \}
  \}\_\{R\}f =\{\textbackslash{}mathop\{
  lim\}\}\_\{n→+∞\}\textbackslash{}mathop\{∫ \}
  \{\textbackslash{}mathop\{∫ \} \}\_\{\{P\}\_\{n\}\}f.
\item
  Soit R un rectangle et f : R → ℂ continue. On dit que f est intégrable
  sur R si la fonction continue positive \textbar{}f\textbar{} est
  intégrable sur R~; on montre alors que si \{(\{P\}\_\{n\})\}\_\{n∈ℕ\}
  est une suite croissante de pavés contenus dans R dont la réunion est
  R, la suite \{\textbackslash{}left (\textbackslash{}mathop\{∫ \}
  \{\textbackslash{}mathop\{∫ \}
  \}\_\{\{P\}\_\{n\}\}f\textbackslash{}right )\}\_\{n∈ℕ\} converge et
  que sa limite est indépendante du choix de la suite
  \{(\{P\}\_\{n\})\}\_\{n∈ℕ\}~; on pose donc \textbackslash{}mathop\{∫
  \} \{\textbackslash{}mathop\{∫ \} \}\_\{R\}f
  =\{\textbackslash{}mathop\{ lim\}\}\_\{n→+∞\}\textbackslash{}mathop\{∫
  \} \{\textbackslash{}mathop\{∫ \} \}\_\{\{P\}\_\{n\}\}f.
\item
  L'ensemble des fonctions continues de R dans ℂ intégrables sur R est
  un sous-espace vectoriel de l'espace des fonctions continues et
  l'application f\textbackslash{}mathrel\{↦\}\textbackslash{}mathop\{∫
  \} \{\textbackslash{}mathop\{∫ \} \}\_\{R\}f est linéaire.
\item
  Si un rectangle R est la réunion de deux rectangles \{R\}\_\{1\} et
  \{R\}\_\{2\} ne se rencontrant que suivant un de leurs côtés, alors f
  est intégrable sur R si et seulement si elle est intégrable sur
  \{R\}\_\{1\} et \{R\}\_\{2\} et dans ce cas, \textbackslash{}mathop\{∫
  \} \{\textbackslash{}mathop\{∫ \} \}\_\{R\}f
  =\textbackslash{}mathop\{∫ \} \{\textbackslash{}mathop\{∫ \}
  \}\_\{\{R\}\_\{1\}\}f +\textbackslash{}mathop\{∫ \}
  \{\textbackslash{}mathop\{∫ \} \}\_\{\{R\}\_\{2\}\}f.
\end{itemize}

On utilisera plusieurs fois le lemme suivant

Lemme~10.3.7 Soit R et R' deux rectangles tels que R ⊂ R' et soit f : R'
→ ℂ continue et intégrable sur R'. Alors f est intégrable sur R et

\textbackslash{}left \textbar{}\textbackslash{}mathop\{∫ \}
\{\textbackslash{}mathop\{∫ \} \}\_\{R'\}f −\textbackslash{}mathop\{∫ \}
\{\textbackslash{}mathop\{∫ \} \}\_\{R\}f\textbackslash{}right
\textbar{}≤\textbackslash{}mathop\{∫ \} \{\textbackslash{}mathop\{∫ \}
\}\_\{R'\}\textbar{}f\textbar{}−\textbackslash{}mathop\{∫ \}
\{\textbackslash{}mathop\{∫ \} \}\_\{R\}\textbar{}f\textbar{}

Démonstration Tout pavé P inclus dans R est inclus dans R' et donc
\{\textbackslash{}mathop\{sup\}\}\_\{P⊂R\}\textbackslash{}mathop\{∫ \}
\{\textbackslash{}mathop\{∫ \}
\}\_\{P\}\textbar{}f\textbar{}≤\{\textbackslash{}mathop\{
sup\}\}\_\{P⊂R'\}\textbackslash{}mathop\{∫ \}
\{\textbackslash{}mathop\{∫ \} \}\_\{P\}\textbar{}f\textbar{}
=\textbackslash{}mathop\{∫ \} \{\textbackslash{}mathop\{∫ \}
\}\_\{R'\}\textbar{}f\textbar{} \textless{} +∞ ce qui garantit
l'intégrabilité de f sur R. De plus (avec quelques conventions
d'écritures évidentes pour l'intégrale sur la différence de deux
rectangles)

\textbackslash{}left \textbar{}\textbackslash{}mathop\{∫ \}
\{\textbackslash{}mathop\{∫ \} \}\_\{R'\}f −\textbackslash{}mathop\{∫ \}
\{\textbackslash{}mathop\{∫ \} \}\_\{R\}f\textbackslash{}right
\textbar{} = \textbackslash{}left \textbar{}\textbackslash{}mathop\{∫ \}
\{\textbackslash{}mathop\{∫ \} \}\_\{R'∖R\}f\textbackslash{}right
\textbar{}≤\textbackslash{}mathop\{∫ \} \{\textbackslash{}mathop\{∫ \}
\}\_\{R'∖R\}\textbar{}f\textbar{}≤\textbackslash{}mathop\{∫ \}
\{\textbackslash{}mathop\{∫ \}
\}\_\{R'\}\textbar{}f\textbar{}−\textbackslash{}mathop\{∫ \}
\{\textbackslash{}mathop\{∫ \} \}\_\{R\}\textbar{}f\textbar{}

Lemme~10.3.8 Soit R = I × I' un rectangle, f : R → ℂ une fonction
continue, intégrable sur R. Soit \{(\{J\}\_\{n\})\}\_\{n∈ℕ\} et
\{(\{K\}\_\{n\})\}\_\{n∈ℕ\} des suites croissantes de segments dont les
réunions sont respectivement I et I'. Alors

\textbackslash{}mathop\{∫ \} \{\textbackslash{}mathop\{∫ \} \}\_\{R\}f
=\{\textbackslash{}mathop\{ lim\}\}\_\{n→+∞\}\textbackslash{}mathop\{∫
\} \{\textbackslash{}mathop\{∫ \} \}\_\{I×\{K\}\_\{n\}\}f
=\{\textbackslash{}mathop\{ lim\}\}\_\{n→+∞\}\textbackslash{}mathop\{∫
\} \{\textbackslash{}mathop\{∫ \} \}\_\{\{J\}\_\{n\}×I'\}f

Démonstration Premier cas~: f est à valeurs réelles positives. On a
alors

\textbackslash{}mathop\{∫ \} \{\textbackslash{}mathop\{∫ \}
\}\_\{\{J\}\_\{n\}×\{K\}\_\{n\}\}f ≤\textbackslash{}mathop\{∫ \}
\{\textbackslash{}mathop\{∫ \} \}\_\{I×\{K\}\_\{n\}\}f
≤\textbackslash{}mathop\{∫ \} \{\textbackslash{}mathop\{∫ \}
\}\_\{I×I'\}f

et comme \textbackslash{}mathop\{∫ \} \{\textbackslash{}mathop\{∫ \}
\}\_\{I×I'\}f =\{\textbackslash{}mathop\{
lim\}\}\_\{n→+∞\}\textbackslash{}mathop\{∫ \}
\{\textbackslash{}mathop\{∫ \} \}\_\{\{J\}\_\{n\}×\{K\}\_\{n\}\}f (les
\{J\}\_\{n\} × \{K\}\_\{n\} forment une suite croissante de pavés dont
la réunion est I × I'), on a \textbackslash{}mathop\{∫ \}
\{\textbackslash{}mathop\{∫ \} \}\_\{I×I'\}f =\{\textbackslash{}mathop\{
lim\}\}\_\{n→+∞\}\textbackslash{}mathop\{∫ \}
\{\textbackslash{}mathop\{∫ \} \}\_\{I×\{K\}\_\{n\}\}f. On démontre
l'autre formule de manière similaire.

Deuxième cas~: f est à valeurs complexes. On remarque que, d'après le
lemme précédent

\textbackslash{}left \textbar{}\textbackslash{}mathop\{∫ \}
\{\textbackslash{}mathop\{∫ \} \}\_\{I×I'\}f −\textbackslash{}mathop\{∫
\} \{\textbackslash{}mathop\{∫ \}
\}\_\{I×\{K\}\_\{n\}\}f\textbackslash{}right
\textbar{}≤\textbackslash{}mathop\{∫ \} \{\textbackslash{}mathop\{∫ \}
\}\_\{I×I'\}\textbar{}f\textbar{}−\textbackslash{}mathop\{∫ \}
\{\textbackslash{}mathop\{∫ \}
\}\_\{I×\{K\}\_\{n\}\}\textbar{}f\textbar{}

qui tend vers 0 d'après le premier cas. Donc \textbackslash{}mathop\{∫
\} \{\textbackslash{}mathop\{∫ \} \}\_\{I×I'\}f
=\{\textbackslash{}mathop\{ lim\}\}\_\{n→+∞\}\textbackslash{}mathop\{∫
\} \{\textbackslash{}mathop\{∫ \} \}\_\{I×\{K\}\_\{n\}\}f. On démontre
l'autre formule de manière similaire.

\paragraph{10.3.6 Théorème de Fubini sur un produit d'intervalles}

Lemme~10.3.9 Soit I' un intervalle de ℝ, f : {[}a,b{]} × I' → ℂ
continue. On fait les hypothèses suivantes~:

\begin{itemize}
\itemsep1pt\parskip0pt\parsep0pt
\item
  pour tout x ∈ {[}a,b{]}, y\textbackslash{}mathrel\{↦\}f(x,y) est
  intégrable sur I'
\item
  l'application g :
  x\textbackslash{}mathrel\{↦\}\{\textbackslash{}mathop\{∫ \}
  \}\_\{I'\}f(x,y) dy est continue par morceaux sur {[}a,b{]}
\item
  f est intégrable sur le rectangle {[}a,b{]} × I'
\end{itemize}

Alors \textbackslash{}mathop\{∫ \} \{\textbackslash{}mathop\{∫ \}
\}\_\{{[}a,b{]}×I'\}f =\{\textbackslash{}mathop\{∫ \}
\}\_\{a\}\^{}\{b\}g =\{\textbackslash{}mathop\{∫ \}
\}\_\{a\}\^{}\{b\}\textbackslash{}left (\{\textbackslash{}mathop\{∫ \}
\}\_\{I'\}f(x,y) dy\textbackslash{}right )dx.

Démonstration Soit \{K\}\_\{n\} une suite croissante de segments dont la
réunion est I'. On sait que

\textbackslash{}mathop\{∫ \} \{\textbackslash{}mathop\{∫ \}
\}\_\{{[}a,b{]}×I'\}f =\{\textbackslash{}mathop\{
lim\}\}\_\{n→+∞\}\textbackslash{}mathop\{∫ \}
\{\textbackslash{}mathop\{∫ \} \}\_\{{[}a,b{]}×\{K\}\_\{n\}\}f
=\{\textbackslash{}mathop\{ lim\}\}\_\{n→+∞\}\{\textbackslash{}mathop\{∫
\} \}\_\{a\}\^{}\{b\}\textbackslash{}left (\{\textbackslash{}mathop\{∫
\} \}\_\{\{K\}\_\{n\}\}f(x,y) dy\textbackslash{}right )dx
=\{\textbackslash{}mathop\{ lim\}\}\_\{n→+∞\}\{\textbackslash{}mathop\{∫
\} \}\_\{a\}\^{}\{b\}\{g\}\_\{ n\}(x) dx

en posant \{g\}\_\{n\}(x) =\{\textbackslash{}mathop\{∫ \}
\}\_\{\{K\}\_\{n\}\}f(x,y) dy~; le théorème de continuité des intégrales
dépendant d'un paramètre sur un segment nous garantit que \{g\}\_\{n\}
est continue~; de plus, comme la fonction
y\textbackslash{}mathrel\{↦\}f(x,y) est intégrable sur I', la suite
\{(\{g\}\_\{n\})\}\_\{n∈ℕ\} converge simplement vers g et on peut écrire

\textbackslash{}begin\{eqnarray*\} \textbar{}\{g\}\_\{n+1\}(x) −
\{g\}\_\{n\}(x)\textbar{}\& =\& \textbackslash{}left
\textbar{}\{\textbackslash{}mathop\{∫ \} \}\_\{\{K\}\_\{n+1\}\}f(x,y) dy
−\{\textbackslash{}mathop\{∫ \} \}\_\{\{K\}\_\{n\}\}f(x,y)
dy\textbackslash{}right \textbar{} \%\& \textbackslash{}\textbackslash{}
\& =\& \textbackslash{}left \textbar{}\{\textbackslash{}mathop\{∫ \}
\}\_\{\{K\}\_\{n+1\}∖\{K\}\_\{n\}\}f(x,y) dy\textbackslash{}right
\textbar{} ≤\{\textbackslash{}mathop\{∫ \}
\}\_\{\{K\}\_\{n+1\}∖\{K\}\_\{n\}\}\textbar{}f(x,y)\textbar{} dy\%\&
\textbackslash{}\textbackslash{} \& =\& \{\textbackslash{}mathop\{∫ \}
\}\_\{\{K\}\_\{n+1\}\}\textbar{}f(x,y)\textbar{} dy
−\{\textbackslash{}mathop\{∫ \}
\}\_\{\{K\}\_\{n\}\}\textbar{}f(x,y)\textbar{} dy \%\&
\textbackslash{}\textbackslash{} \textbackslash{}end\{eqnarray*\}

Posons \{u\}\_\{n\} =\{\textbackslash{}mathop\{∫ \}
\}\_\{a\}\^{}\{b\}\textbackslash{}left (\{\textbackslash{}mathop\{∫ \}
\}\_\{\{K\}\_\{n\}\}\textbar{}f(x,y)\textbar{} dy\textbackslash{}right
)dx. En intégrant l'inégalité ci-dessus de a à b, on obtient

\{\textbackslash{}mathop\{∫ \} \}\_\{a\}\^{}\{b\}\textbar{}\{g\}\_\{
n+1\}(x)−\{g\}\_\{n\}(x)\textbar{} dx ≤\{\textbackslash{}mathop\{∫ \}
\}\_\{a\}\^{}\{b\}\textbackslash{}left (\{\textbackslash{}mathop\{∫ \}
\}\_\{\{K\}\_\{n+1\}\}\textbar{}f(x,y)\textbar{} dy\textbackslash{}right
)−\{\textbackslash{}mathop\{∫ \} \}\_\{a\}\^{}\{b\}\textbackslash{}left
(\{\textbackslash{}mathop\{∫ \}
\}\_\{\{K\}\_\{n\}\}\textbar{}f(x,y)\textbar{} dy\textbackslash{}right )
= \{u\}\_\{n+1\}−\{u\}\_\{n\}

Mais \{\textbackslash{}mathop\{\textbackslash{}mathop\{∑ \}\}
\}\_\{n=0\}\^{}\{N\}(\{u\}\_\{n+1\} − \{u\}\_\{n\}) = \{u\}\_\{N+1\} −
\{u\}\_\{0\} qui admet la limite \textbackslash{}mathop\{∫ \}
\{\textbackslash{}mathop\{∫ \}
\}\_\{{[}a,b{]}×I'\}\textbar{}f\textbar{}. Donc la série
\textbackslash{}mathop\{\textbackslash{}mathop\{∑ \}\} (\{u\}\_\{n+1\} −
\{u\}\_\{n\}) converge, et par conséquent, il en est de même de la série
\textbackslash{}mathop\{\textbackslash{}mathop\{∑ \}\}
\{\textbackslash{}mathop\{∫ \}
\}\_\{a\}\^{}\{b\}\textbar{}\{g\}\_\{n+1\}(x) −
\{g\}\_\{n\}(x)\textbar{} dx. Le théorème d'intégration termes à termes
pour les séries de fonctions assure que

\{\textbackslash{}mathop\{∑
\}\}\_\{n=0\}\^{}\{+∞\}\{\textbackslash{}mathop\{\textbackslash{}mathop\{∫
\} \} \}\_\{a\}\^{}\{b\}(\{g\}\_\{ n+1\} − \{g\}\_\{n\}) =\{
\textbackslash{}mathop\{\textbackslash{}mathop\{∫ \} \}
\}\_\{a\}\^{}\{b\}\{ \textbackslash{}mathop\{∑
\}\}\_\{n=0\}\^{}\{+∞\}(\{g\}\_\{ n+1\} − \{g\}\_\{n\}) =\{
\textbackslash{}mathop\{\textbackslash{}mathop\{∫ \} \}
\}\_\{a\}\^{}\{b\}(g − \{g\}\_\{ 0\})

Mais

\{\textbackslash{}mathop\{∑ \}\}\_\{n=0\}\^{}\{N−1\}\{
\textbackslash{}mathop\{\textbackslash{}mathop\{∫ \} \}
\}\_\{a\}\^{}\{b\}(\{g\}\_\{ n+1\} − \{g\}\_\{n\}) =\{
\textbackslash{}mathop\{\textbackslash{}mathop\{∫ \} \}
\}\_\{a\}\^{}\{b\}\{g\}\_\{ N\}
−\{\textbackslash{}mathop\{\textbackslash{}mathop\{∫ \} \}
\}\_\{a\}\^{}\{b\}\{g\}\_\{ 0\}

Autrement dit on a
\{\textbackslash{}mathop\{lim\}\}\_\{N→+∞\}\textbackslash{}left
(\{\textbackslash{}mathop\{∫ \} \}\_\{a\}\^{}\{b\}\{g\}\_\{N\}
−\{\textbackslash{}mathop\{∫ \}
\}\_\{a\}\^{}\{b\}\{g\}\_\{0\}\textbackslash{}right )
=\{\textbackslash{}mathop\{∫ \} \}\_\{a\}\^{}\{b\}(g − \{g\}\_\{0\}),
soit encore
\{\textbackslash{}mathop\{lim\}\}\_\{N→+∞\}\{\textbackslash{}mathop\{∫
\} \}\_\{a\}\^{}\{b\}\{g\}\_\{N\} =\{\textbackslash{}mathop\{∫ \}
\}\_\{a\}\^{}\{b\}g, c'est à dire \textbackslash{}mathop\{∫ \}
\{\textbackslash{}mathop\{∫ \} \}\_\{{[}a,b{]}×I'\}f
=\{\textbackslash{}mathop\{∫ \} \}\_\{a\}\^{}\{b\}f, ce que nous
cherchions à démontrer.

Nous sommes maintenant en mesure de démontrer un premier théorème
reliant l'intégrale sur un rectangle et les intégrales partielles sur
les intervalles projections de ce rectangle sur les deux axes.

Théorème~10.3.10 Soit R = I × I' un rectangle, f : R → ℂ continue. On
suppose que

\begin{itemize}
\itemsep1pt\parskip0pt\parsep0pt
\item
  f est intégrable sur R
\item
  pour tout x ∈ I, y\textbackslash{}mathrel\{↦\}f(x,y) est intégrable
  sur I'
\item
  les applications
  x\textbackslash{}mathrel\{↦\}\{\textbackslash{}mathop\{∫ \}
  \}\_\{I'\}\textbar{}f(x,y)\textbar{} dy et g :
  x\textbackslash{}mathrel\{↦\}\{\textbackslash{}mathop\{∫ \}
  \}\_\{I'\}f(x,y) dy sont continues par morceaux sur I
\end{itemize}

Alors g est intégrable sur I et \textbackslash{}mathop\{∫ \}
\{\textbackslash{}mathop\{∫ \} \}\_\{I×I'\}f
=\{\textbackslash{}mathop\{∫ \} \}\_\{I\}g =\{\textbackslash{}mathop\{∫
\} \}\_\{I\}\textbackslash{}left (\{\textbackslash{}mathop\{∫ \}
\}\_\{I'\}f(x,y) dy\textbackslash{}right )dx.

Démonstration Soit J un segment inclus dans I. D'après le lemme
précédent dont les hypothèses sont évidemment vérifiées,

\{\textbackslash{}mathop\{∫ \} \}\_\{J\}\textbar{}g\textbar{}
=\{\textbackslash{}mathop\{∫ \} \}\_\{I\}\textbackslash{}left
\textbar{}\{\textbackslash{}mathop\{∫ \} \}\_\{I'\}f(x,y)
dy\textbackslash{}right \textbar{} dx ≤\{\textbackslash{}mathop\{∫ \}
\}\_\{J\}\textbackslash{}left (\{\textbackslash{}mathop\{∫ \}
\}\_\{I'\}\textbar{}f(x,y)\textbar{} dy\textbackslash{}right ) dx
=\textbackslash{}mathop\{∫ \} \{\textbackslash{}mathop\{∫ \}
\}\_\{J×I'\}\textbar{}f\textbar{}≤\textbackslash{}mathop\{∫ \}
\{\textbackslash{}mathop\{∫ \} \}\_\{I×I'\}\textbar{}f\textbar{}

ce qui garantit que g est intégrable sur I.

Soit maintenant \{(\{J\}\_\{n\})\}\_\{n∈ℕ\} une suite croissante de
segments dont la réunion est I. Alors, en combinant les deux lemmes
précédents, on a

\textbackslash{}mathop\{∫ \} \{\textbackslash{}mathop\{∫ \}
\}\_\{I×I'\}f =\{\textbackslash{}mathop\{
lim\}\}\_\{n→+∞\}\textbackslash{}mathop\{∫ \}
\{\textbackslash{}mathop\{∫ \} \}\_\{\{J\}\_\{n\}×I'\}f
=\{\textbackslash{}mathop\{ lim\}\}\_\{n→+∞\}\{\textbackslash{}mathop\{∫
\} \}\_\{\{J\}\_\{n\}\}g =\{\textbackslash{}mathop\{∫ \} \}\_\{I\}g

ce que nous voulions démontrer.

Nous allons en déduire un théorème nous permettant d'intervertir les
signes d'intégration sur des intervalles.

Théorème~10.3.11 Soit I et I' deux intervalles de ℝ, f : I × I' → ℂ
continue. On suppose que

\begin{itemize}
\itemsep1pt\parskip0pt\parsep0pt
\item
  pour tout x ∈ I, y\textbackslash{}mathrel\{↦\}f(x,y) est intégrable
  sur I'
\item
  pour tout y ∈ I', x\textbackslash{}mathrel\{↦\}f(x,y) est intégrable
  sur I et l'application
  y\textbackslash{}mathrel\{↦\}\{\textbackslash{}mathop\{∫ \}
  \}\_\{I\}f(x,y) dx est continue par morceaux
\item
  l'application x\textbackslash{}mathrel\{↦\}\{\textbackslash{}mathop\{∫
  \} \}\_\{I'\}f(x,y) dy est continue par morceaux sur I et
  x\textbackslash{}mathrel\{↦\}\{\textbackslash{}mathop\{∫ \}
  \}\_\{I'\}\textbar{}f(x,y)\textbar{} dy est continue par morceaux et
  intégrable sur I
\end{itemize}

Alors l'application
y\textbackslash{}mathrel\{↦\}\{\textbackslash{}mathop\{∫ \}
\}\_\{I\}f(x,y) dx est intégrable sur I' et on a

\{\textbackslash{}mathop\{∫ \} \}\_\{I\}\textbackslash{}left
(\{\textbackslash{}mathop\{∫ \} \}\_\{I'\}f(x,y) dy\textbackslash{}right
)dx =\{\textbackslash{}mathop\{∫ \} \}\_\{I'\}\textbackslash{}left
(\{\textbackslash{}mathop\{∫ \} \}\_\{I\}f(x,y) dx\textbackslash{}right
)dy

Démonstration Soit P = J × K un pavé contenu dans I × I'. On a alors

\textbackslash{}mathop\{∫ \} \{\textbackslash{}mathop\{∫ \}
\}\_\{J×K\}\textbar{}f\textbar{} =\{\textbackslash{}mathop\{∫ \}
\}\_\{J\}\textbackslash{}left (\{\textbackslash{}mathop\{∫ \}
\}\_\{K\}\textbar{}f(x,y)\textbar{} dy\textbackslash{}right )dx
≤\{\textbackslash{}mathop\{∫ \} \}\_\{J\}\textbackslash{}left
(\{\textbackslash{}mathop\{∫ \} \}\_\{I'\}\textbar{}f(x,y)\textbar{}
dy\textbackslash{}right )dx ≤\{\textbackslash{}mathop\{∫ \}
\}\_\{I\}\textbackslash{}left (\{\textbackslash{}mathop\{∫ \}
\}\_\{I'\}\textbar{}f(x,y)\textbar{} dy\textbackslash{}right )dx

ce qui montre que \textbar{}f\textbar{} est intégrable sur le rectangle
I × I'. Il en est donc de même pour f.

Le théorème précédent assure que l'application
x\textbackslash{}mathrel\{↦\}\{\textbackslash{}mathop\{∫ \}
\}\_\{I'\}f(x,y) dy est intégrable sur I et que

\textbackslash{}mathop\{∫ \} \{\textbackslash{}mathop\{∫ \}
\}\_\{I×I'\}f =\{\textbackslash{}mathop\{∫ \}
\}\_\{I\}\textbackslash{}left (\{\textbackslash{}mathop\{∫ \}
\}\_\{I'\}f(x,y) dy\textbackslash{}right )dx

On applique à nouveau le théorème précédent en intervertissant le rôle
de I et I', donc des variables x et y. On obtient que l'application
y\textbackslash{}mathrel\{↦\}\{\textbackslash{}mathop\{∫ \}
\}\_\{I\}f(x,y) dx est intégrable sur I et que
\{\textbackslash{}mathop\{∫ \} \}\_\{I'\}\textbackslash{}left
(\{\textbackslash{}mathop\{∫ \} \}\_\{I\}f(x,y) dx\textbackslash{}right
)dy =\textbackslash{}mathop\{∫ \} \{\textbackslash{}mathop\{∫ \}
\}\_\{I×I'\}f ce qui nous donne l'égalité recherchée.

Remarque~10.3.2 Moyennant la vérification que toutes les fonctions
intégrées sont continues par morceaux, on pourra retenir ce théorème
sous la forme

\textbackslash{}left .\textbackslash{}array\{
\{\textbackslash{}mathop\{∫ \} \}\_\{I\}\textbackslash{}left
(\{\textbackslash{}mathop\{∫ \} \}\_\{I'\}\textbar{}f(x,y)\textbar{}
dy\textbackslash{}right )dx \textless{} +∞ \textbackslash{}cr
\textbackslash{}mathop\{∀\}y ∈ E, \{\textbackslash{}mathop\{∫ \}
\}\_\{I\}\textbar{}f(x,y)\textbar{} dx \textless{} +∞ \}
\textbackslash{}right \textbackslash{}\}⇒\{\textbackslash{}mathop\{∫ \}
\}\_\{I\}\textbackslash{}left (\{\textbackslash{}mathop\{∫ \}
\}\_\{I'\}f(x,y) dy\textbackslash{}right )dx
=\{\textbackslash{}mathop\{∫ \} \}\_\{I'\}\textbackslash{}left
(\{\textbackslash{}mathop\{∫ \} \}\_\{I\}f(x,y) dx\textbackslash{}right
)dy

\paragraph{10.3.7 La fonction Γ}

Définition~10.3.3 Pour x ∈{]}0,+∞{[}, on pose Γ(x)
=\{\textbackslash{}mathop\{∫ \}
\}\_\{0\}\^{}\{+∞\}\{t\}\^{}\{x−1\}\{e\}\^{}\{−t\} dt.

Démonstration En + ∞, \{t\}\^{}\{x−1\}\{e\}\^{}\{−t\} = o(\{ 1
\textbackslash{}over \{t\}\^{}\{2\}\} ) donc la fonction est intégrable
sur {[}1,+∞{[}. En 0, on a \{t\}\^{}\{x−1\}\{e\}\^{}\{−t\} ∼
\{t\}\^{}\{x−1\} \textgreater{} 0 donc la fonction est intégrable sur
{]}0,1{]} si et seulement si~x \textgreater{} 0. La fonction Γ est donc
définie pour x \textgreater{} 0.

Proposition~10.3.12 La fonction Γ est de classe \{C\}\^{}\{∞\} sur
{]}0,+∞{[} et

\textbackslash{}mathop\{∀\}k ∈ ℕ, \textbackslash{}mathop\{∀\}x
∈{]}0,+∞{[}, \{Γ\}\^{}\{(k)\}(x) =\{\textbackslash{}mathop\{∫ \}
\}\_\{0\}\^{}\{+∞\}\{(\textbackslash{}mathop\{log\}
t)\}\^{}\{k\}\{e\}\^{}\{−t\}\{t\}\^{}\{x−1\} dt

Démonstration Soit 0 \textless{} a \textless{} 1 \textless{} b
\textless{} +∞ et posons f(x,t) = \{t\}\^{}\{x−1\}\{e\}\^{}\{−t\} pour
(x,t) ∈ {[}a,b{]}×{]}0,+∞{[}. Soit J = {[}a,b{]} et I ={]}0,+∞{[}~; la
fonction f : J × I → ℂ, (x,t)\textbackslash{}mathrel\{↦\}f(x,t) est
continue et admet des dérivées partielles par rapport à x,
(x,t)\textbackslash{}mathrel\{↦\}\{\{∂\}\^{}\{i\}f\textbackslash{}over
∂\{x\}\^{}\{i\}\}(x,t) = \{(\textbackslash{}mathop\{log\}
t)\}\^{}\{i\}\{e\}\^{}\{−t\}\{t\}\^{}\{x−1\}, continues sur J × I, i =
1,\textbackslash{}mathop\{\textbackslash{}mathop\{\ldots{}\}\},k. Soit
\{φ\}\_\{i\} :{]}0,+∞{[}→ \{ℝ\}\^{}\{+\} définie par

\{ φ\}\_\{i\}(t) = \textbackslash{}left \textbackslash{}\{
\textbackslash{}cases\{ \textbar{}\textbackslash{}mathop\{log\}
t\{\textbar{}\}\^{}\{i\}\{e\}\^{}\{−t\}\{t\}\^{}\{a−1\}\&si t ∈{]}0,1{]}
\textbackslash{}cr \{(\textbackslash{}mathop\{log\}
t)\}\^{}\{i\}\{e\}\^{}\{−t\}\{t\}\^{}\{b−1\}\&si t ≥ 1 \}
\textbackslash{}right .

Alors \{φ\}\_\{i\} est continue par morceaux, intégrable sur {]}0,+∞{[}
et on a

\textbackslash{}mathop\{∀\}(x,t) ∈ {[}a,b{]}×{]}0,+∞{[}, \textbar{}\{
\{∂\}\^{}\{i\}f \textbackslash{}over ∂\{x\}\^{}\{i\}\} (x,t)\textbar{}≤
\{φ\}\_\{i\}(t)

D'après le théorème de dérivation des intégrales dépendant d'un
paramètre, Γ(x) =\{\textbackslash{}mathop\{∫ \}
\}\_\{0\}\^{}\{+∞\}f(x,t) dt est de classe \{C\}\^{}\{k\} sur {[}a,b{]}
et \{Γ\}\^{}\{(i)\}(x) =\{\textbackslash{}mathop\{∫ \}
\}\_\{{]}0,+∞{[}\}\{ \{∂\}\^{}\{i\}f \textbackslash{}over
∂\{x\}\^{}\{i\}\} (x,t) dt. Comme a et b sont quelconques, le résultat
reste valide sur la réunion des intervalles {[}a,b{]}, donc Γ est de
classe \{C\}\^{}\{k\} sur {]}0,+∞{[} et

\textbackslash{}mathop\{∀\}k ∈ ℕ, \textbackslash{}mathop\{∀\}x
∈{]}0,+∞{[}, \{Γ\}\^{}\{(k)\}(x) =\{\textbackslash{}mathop\{∫ \}
\}\_\{0\}\^{}\{+∞\}\{(\textbackslash{}mathop\{log\}
t)\}\^{}\{k\}\{e\}\^{}\{−t\}\{t\}\^{}\{x−1\} dt

Proposition~10.3.13 Pour tout x ∈{]}0,+∞{[}, Γ(x + 1) = xΓ(x). En
particulier \textbackslash{}mathop\{∀\}n ∈ ℕ, Γ(n + 1) = n!.

Démonstration Soit 0 \textless{} a \textless{} b \textless{} +∞. On a
par une intégration par parties

\{\textbackslash{}mathop\{∫ \}
\}\_\{a\}\^{}\{b\}\{e\}\^{}\{−t\}\{t\}\^{}\{x\} dt =\{
\textbackslash{}left
{[}−\{e\}\^{}\{−t\}\{t\}\^{}\{x\}\textbackslash{}right {]}\}\_\{
a\}\^{}\{b\} + x\{\textbackslash{}mathop\{∫ \}
\}\_\{a\}\^{}\{b\}\{e\}\^{}\{−t\}\{t\}\^{}\{x−1\} dt

Il suffit alors de faire tendre a vers 0 et b vers + ∞~; le crochet
admet la limite 0 et on obtient Γ(x + 1) = xΓ(x). Comme Γ(1) = 1, une
récurrence immédiate donne Γ(n + 1) = n!.

Proposition~10.3.14 Γ(\{ 1 \textbackslash{}over 2\} ) =
2\{\textbackslash{}mathop\{∫ \}
\}\_\{0\}\^{}\{+∞\}\{e\}\^{}\{−\{t\}\^{}\{2\} \} dt =
\textbackslash{}sqrt\{π\}.

Démonstration Soit 0 \textless{} a \textless{} b \textless{} +∞. Le
changement de variable t = \{u\}\^{}\{2\} donne

\{\textbackslash{}mathop\{∫ \}
\}\_\{a\}\^{}\{b\}\{e\}\^{}\{−t\}\{t\}\^{}\{−1∕2\} dt =
2\{\textbackslash{}mathop\{∫ \}
\}\_\{\textbackslash{}sqrt\{a\}\}\^{}\{\textbackslash{}sqrt\{b\}\}\{e\}\^{}\{−\{u\}\^{}\{2\}
\} du

Il suffit alors de faire tendre a vers 0 et b vers + ∞, pour avoir Γ(\{
1 \textbackslash{}over 2\} ) = 2\{\textbackslash{}mathop\{∫ \}
\}\_\{0\}\^{}\{+∞\}\{e\}\^{}\{−\{t\}\^{}\{2\} \} dt. La valeur \{
\textbackslash{}sqrt\{π\} \textbackslash{}over 2\} de cette dernière
intégrale sera admise (démontrée en exercice).

\paragraph{10.3.8 Méthodes directes}

En ce qui concerne la continuité et la dérivabilité de F, on peut aussi
tenter de majorer directement les expressions F(x) − F(\{x\}\_\{0\})
=\{\textbackslash{}mathop\{∫ \} \}\_\{a\}\^{}\{b\}(f(x,t) −
f(\{x\}\_\{0\},t)) dt et F(x) − F(\{x\}\_\{0\}) − (x −
\{x\}\_\{0\})\{\textbackslash{}mathop\{∫ \} \}\_\{a\}\^{}\{b\}\{ ∂f
\textbackslash{}over ∂x\} (\{x\}\_\{0\},t) dt
=\{\textbackslash{}mathop\{∫ \} \}\_\{a\}\^{}\{b\}\textbackslash{}left
(f(x,t) − f(\{x\}\_\{0\},t) − (x − \{x\}\_\{0\})\{ ∂f
\textbackslash{}over ∂x\} (\{x\}\_\{0\},t)\textbackslash{}right ) dten
utilisant en particulier l'inégalité des accroissements finis et
l'inégalité de Taylor Lagrange à l'ordre 2 qui nous donneront, moyennant
des hypothèses raisonnables,

\textbackslash{}\textbar{}f(x,t) −
f(\{x\}\_\{0\},t)\textbackslash{}\textbar{} ≤\textbar{}x −
\{x\}\_\{0\}\textbar{}\{\textbackslash{}mathop\{sup\}\}\_\{y∈{[}\{x\}\_\{0\},x{]}\}\textbackslash{}\textbar{}\{
∂f \textbackslash{}over ∂x\} (y,t)\textbackslash{}\textbar{}

et

\textbackslash{}begin\{eqnarray*\} \textbackslash{}\textbar{}f(x,t) −
f(\{x\}\_\{0\},t) − (x − \{x\}\_\{0\})\{ ∂f \textbackslash{}over ∂x\}
(\{x\}\_\{0\},t)\textbackslash{}\textbar{}\&\& \%\&
\textbackslash{}\textbackslash{} \& \& \textbackslash{}quad
\textbackslash{}quad ≤\{ \textbar{}x −
\{x\}\_\{0\}\{\textbar{}\}\^{}\{2\} \textbackslash{}over 2\}
\{\textbackslash{}mathop\{sup\}\}\_\{y∈{[}\{x\}\_\{0\},x{]}\}\textbackslash{}\textbar{}\{
\{∂\}\^{}\{2\}f \textbackslash{}over ∂\{x\}\^{}\{2\}\}
(y,t)\textbackslash{}\textbar{}\%\& \textbackslash{}\textbackslash{}
\textbackslash{}end\{eqnarray*\}

Des méthodes directes similaires peuvent être utilisées pour
l'intégration.

Exemple~10.3.3 Soit F(x) =\{\textbackslash{}mathop\{∫ \}
\}\_\{0\}\^{}\{+∞\}\{ \textbackslash{}mathop\{sin\} t
\textbackslash{}over t\} \{e\}\^{}\{−tx\} dt. Il est clair que la
fonction est intégrable pour x \textgreater{} 0. Montrons que F est
dérivable sur {]}0,+∞{[}. On a, pour \{x\}\_\{0\} \textgreater{} 0 et x
\textgreater{}\{ \{x\}\_\{0\} \textbackslash{}over 2\}

\{e\}\^{}\{−tx\} − \{e\}\^{}\{−t\{x\}\_\{0\} \} + (x −
\{x\}\_\{0\})t\{e\}\^{}\{−t\{x\}\_\{0\} \} =\{ \{(x −
\{x\}\_\{0\})\}\^{}\{2\} \textbackslash{}over 2\}
\{t\}\^{}\{2\}\{e\}\^{}\{−tξ\}, ξ ∈ {[}\{x\}\_\{ 0\},x{]}

en appliquant la formule de Taylor Lagrange à l'application
y\textbackslash{}mathrel\{↦\}\{e\}\^{}\{−ty\} sur {[}\{x\}\_\{0\},x{]}
(ou {[}x,\{x\}\_\{0\}{]}), soit encore \textbar{}\{e\}\^{}\{−tx\} −
\{e\}\^{}\{−t\{x\}\_\{0\}\} + (x −
\{x\}\_\{0\})t\{e\}\^{}\{−t\{x\}\_\{0\}\}\textbar{}≤\{
\{(x−\{x\}\_\{0\})\}\^{}\{2\} \textbackslash{}over 2\}
\{t\}\^{}\{2\}\{e\}\^{}\{−\{ t\{x\}\_\{0\} \textbackslash{}over 2\} \}.
On en déduit que \textbar{}F(x) − F(\{x\}\_\{0\}) + (x −
\{x\}\_\{0\})\{\textbackslash{}mathop\{∫ \}
\}\_\{0\}\^{}\{+∞\}\textbackslash{}mathop\{sin\}
t\{e\}\^{}\{−t\{x\}\_\{0\}\} dt\textbar{}≤\{
\{(x−\{x\}\_\{0\})\}\^{}\{2\} \textbackslash{}over 2\}
\{\textbackslash{}mathop\{∫ \}
\}\_\{0\}\^{}\{+∞\}\textbar{}\textbackslash{}mathop\{sin\}
t\textbar{}t\{e\}\^{}\{−\{ t\{x\}\_\{0\} \textbackslash{}over 2\} \} dt,
toutes les intégrales ayant manifestement un sens. En divisant par
\textbar{}x − \{x\}\_\{0\}\textbar{}, on en déduit que F est dérivable
en \{x\}\_\{0\} et que F'(\{x\}\_\{0\}) = −\{\textbackslash{}mathop\{∫
\} \}\_\{0\}\^{}\{+∞\}\textbackslash{}mathop\{sin\}
t\{e\}\^{}\{−t\{x\}\_\{0\}\} dt = −\{ 1 \textbackslash{}over
1+\{x\}\_\{0\}\^{}\{2\}\} (facile). On obtient donc F(x) = K
−\textbackslash{}mathop\{\textbackslash{}mathrm\{arctg\}\} x, pour x
\textgreater{} 0. Nous laissons en exercice au lecteur le soin de
montrer que \{\textbackslash{}mathop\{lim\}\}\_\{x→∞\}F(x) = 0, ce qui
montrera que K =\{ π \textbackslash{}over 2\} . Bien entendu, on aurait
pu aussi utiliser le théorème de convergence dominée pour démontrer la
dérivabilité de F sur {]}0,+∞{[} (ou plutôt sur {[}a,+∞{[} pour tout a
\textgreater{} 0).

{[}\href{coursse61.html}{prev}{]}
{[}\href{coursse61.html\#tailcoursse61.html}{prev-tail}{]}
{[}\href{coursse62.html}{front}{]}
{[}\href{coursch11.html\#coursse62.html}{up}{]}

\end{document}

\newpage
\part{Séries entières}
\documentclass[]{article}
\usepackage[T1]{fontenc}
\usepackage{lmodern}
\usepackage{amssymb,amsmath}
\usepackage{ifxetex,ifluatex}
\usepackage{fixltx2e} % provides \textsubscript
% use upquote if available, for straight quotes in verbatim environments
\IfFileExists{upquote.sty}{\usepackage{upquote}}{}
\ifnum 0\ifxetex 1\fi\ifluatex 1\fi=0 % if pdftex
  \usepackage[utf8]{inputenc}
\else % if luatex or xelatex
  \ifxetex
    \usepackage{mathspec}
    \usepackage{xltxtra,xunicode}
  \else
    \usepackage{fontspec}
  \fi
  \defaultfontfeatures{Mapping=tex-text,Scale=MatchLowercase}
  \newcommand{\euro}{€}
\fi
% use microtype if available
\IfFileExists{microtype.sty}{\usepackage{microtype}}{}
\ifxetex
  \usepackage[setpagesize=false, % page size defined by xetex
              unicode=false, % unicode breaks when used with xetex
              xetex]{hyperref}
\else
  \usepackage[unicode=true]{hyperref}
\fi
\hypersetup{breaklinks=true,
            bookmarks=true,
            pdfauthor={},
            pdftitle={Convergence des series enti`eres},
            colorlinks=true,
            citecolor=blue,
            urlcolor=blue,
            linkcolor=magenta,
            pdfborder={0 0 0}}
\urlstyle{same}  % don't use monospace font for urls
\setlength{\parindent}{0pt}
\setlength{\parskip}{6pt plus 2pt minus 1pt}
\setlength{\emergencystretch}{3em}  % prevent overfull lines
\setcounter{secnumdepth}{0}
 
/* start css.sty */
.cmr-5{font-size:50%;}
.cmr-7{font-size:70%;}
.cmmi-5{font-size:50%;font-style: italic;}
.cmmi-7{font-size:70%;font-style: italic;}
.cmmi-10{font-style: italic;}
.cmsy-5{font-size:50%;}
.cmsy-7{font-size:70%;}
.cmex-7{font-size:70%;}
.cmex-7x-x-71{font-size:49%;}
.msbm-7{font-size:70%;}
.cmtt-10{font-family: monospace;}
.cmti-10{ font-style: italic;}
.cmbx-10{ font-weight: bold;}
.cmr-17x-x-120{font-size:204%;}
.cmsl-10{font-style: oblique;}
.cmti-7x-x-71{font-size:49%; font-style: italic;}
.cmbxti-10{ font-weight: bold; font-style: italic;}
p.noindent { text-indent: 0em }
td p.noindent { text-indent: 0em; margin-top:0em; }
p.nopar { text-indent: 0em; }
p.indent{ text-indent: 1.5em }
@media print {div.crosslinks {visibility:hidden;}}
a img { border-top: 0; border-left: 0; border-right: 0; }
center { margin-top:1em; margin-bottom:1em; }
td center { margin-top:0em; margin-bottom:0em; }
.Canvas { position:relative; }
li p.indent { text-indent: 0em }
.enumerate1 {list-style-type:decimal;}
.enumerate2 {list-style-type:lower-alpha;}
.enumerate3 {list-style-type:lower-roman;}
.enumerate4 {list-style-type:upper-alpha;}
div.newtheorem { margin-bottom: 2em; margin-top: 2em;}
.obeylines-h,.obeylines-v {white-space: nowrap; }
div.obeylines-v p { margin-top:0; margin-bottom:0; }
.overline{ text-decoration:overline; }
.overline img{ border-top: 1px solid black; }
td.displaylines {text-align:center; white-space:nowrap;}
.centerline {text-align:center;}
.rightline {text-align:right;}
div.verbatim {font-family: monospace; white-space: nowrap; text-align:left; clear:both; }
.fbox {padding-left:3.0pt; padding-right:3.0pt; text-indent:0pt; border:solid black 0.4pt; }
div.fbox {display:table}
div.center div.fbox {text-align:center; clear:both; padding-left:3.0pt; padding-right:3.0pt; text-indent:0pt; border:solid black 0.4pt; }
div.minipage{width:100%;}
div.center, div.center div.center {text-align: center; margin-left:1em; margin-right:1em;}
div.center div {text-align: left;}
div.flushright, div.flushright div.flushright {text-align: right;}
div.flushright div {text-align: left;}
div.flushleft {text-align: left;}
.underline{ text-decoration:underline; }
.underline img{ border-bottom: 1px solid black; margin-bottom:1pt; }
.framebox-c, .framebox-l, .framebox-r { padding-left:3.0pt; padding-right:3.0pt; text-indent:0pt; border:solid black 0.4pt; }
.framebox-c {text-align:center;}
.framebox-l {text-align:left;}
.framebox-r {text-align:right;}
span.thank-mark{ vertical-align: super }
span.footnote-mark sup.textsuperscript, span.footnote-mark a sup.textsuperscript{ font-size:80%; }
div.tabular, div.center div.tabular {text-align: center; margin-top:0.5em; margin-bottom:0.5em; }
table.tabular td p{margin-top:0em;}
table.tabular {margin-left: auto; margin-right: auto;}
div.td00{ margin-left:0pt; margin-right:0pt; }
div.td01{ margin-left:0pt; margin-right:5pt; }
div.td10{ margin-left:5pt; margin-right:0pt; }
div.td11{ margin-left:5pt; margin-right:5pt; }
table[rules] {border-left:solid black 0.4pt; border-right:solid black 0.4pt; }
td.td00{ padding-left:0pt; padding-right:0pt; }
td.td01{ padding-left:0pt; padding-right:5pt; }
td.td10{ padding-left:5pt; padding-right:0pt; }
td.td11{ padding-left:5pt; padding-right:5pt; }
table[rules] {border-left:solid black 0.4pt; border-right:solid black 0.4pt; }
.hline hr, .cline hr{ height : 1px; margin:0px; }
.tabbing-right {text-align:right;}
span.TEX {letter-spacing: -0.125em; }
span.TEX span.E{ position:relative;top:0.5ex;left:-0.0417em;}
a span.TEX span.E {text-decoration: none; }
span.LATEX span.A{ position:relative; top:-0.5ex; left:-0.4em; font-size:85%;}
span.LATEX span.TEX{ position:relative; left: -0.4em; }
div.float img, div.float .caption {text-align:center;}
div.figure img, div.figure .caption {text-align:center;}
.marginpar {width:20%; float:right; text-align:left; margin-left:auto; margin-top:0.5em; font-size:85%; text-decoration:underline;}
.marginpar p{margin-top:0.4em; margin-bottom:0.4em;}
.equation td{text-align:center; vertical-align:middle; }
td.eq-no{ width:5%; }
table.equation { width:100%; } 
div.math-display, div.par-math-display{text-align:center;}
math .texttt { font-family: monospace; }
math .textit { font-style: italic; }
math .textsl { font-style: oblique; }
math .textsf { font-family: sans-serif; }
math .textbf { font-weight: bold; }
.partToc a, .partToc, .likepartToc a, .likepartToc {line-height: 200%; font-weight:bold; font-size:110%;}
.chapterToc a, .chapterToc, .likechapterToc a, .likechapterToc, .appendixToc a, .appendixToc {line-height: 200%; font-weight:bold;}
.index-item, .index-subitem, .index-subsubitem {display:block}
.caption td.id{font-weight: bold; white-space: nowrap; }
table.caption {text-align:center;}
h1.partHead{text-align: center}
p.bibitem { text-indent: -2em; margin-left: 2em; margin-top:0.6em; margin-bottom:0.6em; }
p.bibitem-p { text-indent: 0em; margin-left: 2em; margin-top:0.6em; margin-bottom:0.6em; }
.paragraphHead, .likeparagraphHead { margin-top:2em; font-weight: bold;}
.subparagraphHead, .likesubparagraphHead { font-weight: bold;}
.quote {margin-bottom:0.25em; margin-top:0.25em; margin-left:1em; margin-right:1em; text-align:\\jmathmathustify;}
.verse{white-space:nowrap; margin-left:2em}
div.maketitle {text-align:center;}
h2.titleHead{text-align:center;}
div.maketitle{ margin-bottom: 2em; }
div.author, div.date {text-align:center;}
div.thanks{text-align:left; margin-left:10%; font-size:85%; font-style:italic; }
div.author{white-space: nowrap;}
.quotation {margin-bottom:0.25em; margin-top:0.25em; margin-left:1em; }
h1.partHead{text-align: center}
.sectionToc, .likesectionToc {margin-left:2em;}
.subsectionToc, .likesubsectionToc {margin-left:4em;}
.subsubsectionToc, .likesubsubsectionToc {margin-left:6em;}
.frenchb-nbsp{font-size:75%;}
.frenchb-thinspace{font-size:75%;}
.figure img.graphics {margin-left:10%;}
/* end css.sty */

\title{Convergence des series enti`eres}
\author{}
\date{}

\begin{document}
\maketitle

\textbf{Warning: 
requires JavaScript to process the mathematics on this page.\\ If your
browser supports JavaScript, be sure it is enabled.}

\begin{center}\rule{3in}{0.4pt}\end{center}

{[}
{[}{]}
{[}

\subsubsection{11.1 Convergence des séries entières}

\paragraph{11.1.1 Notion de série entière}

Définition~11.1.1 Soit (a_n)_n\in\mathbb{N}~ une suite de l'espace
vectoriel normé complet E. On appelle série entière associée à la suite
(a_n) la série de fonctions de \mathbb{C} (resp. \mathbb{R}~) dans E,
\\sum ~
_n≥0u_n, où l'on pose u_n(z) =
a_nz^n~; on notera simplement
\\sum ~
_n≥0a_nz^n cette série de fonctions de la
variable z.

Remarque~11.1.1 Dans le cas où E = \mathbb{R}~ ou E = \mathbb{C}, la série entière est
associée à une unique série formelle
\\sum ~
_n=0^+\infty~a_nX^n \in K{[}{[}X{]}{]}.

\paragraph{11.1.2 Rayon de convergence}

Lemme~11.1.1 (Abel). Soit E un K-espace vectoriel normé complet,
\\sum ~
a_nz^n une série entière à coefficients dans E. Soit
z_0 \in K^∗ tel que la suite
(a_nz_0^n) soit bornée. Alors la série
\\sum ~
a_nz^n converge absolument pour tout z \in K tel que
z \textless{} z_0~; la
série entière converge même normalement dans tout disque fermé D'(0,r) =
\z \in
K∣z\leq r\
pour tout nombre réel r tel que r \textless{}
z_0.

Démonstration Soit M ≥ 0 tel que \forall~~n \in \mathbb{N}~,
\a_nz_0^n\
\leq M et soit z \in K tel que z \textless{}
z_0. On a alors
\a_nz^n\
=\
a_nz_0^n\
\left  z \over z_0
\right ^n \leq M\left
 z \over z_0 \right
^n. Comme \left  z
\over z_0 \right 
\textless{} 1, la série géométrique est convergente et donc la série
\\sum ~
a_nz^n converge absolument. Pour z \in D'(0,r), on a
de la même fa\ccon
\a_nz^n\
\leq M\left  r \over
z_0 \right ^n qui est une
série convergente indépendante de z, donc la série converge normalement
sur D'(0,r).

Théorème~11.1.2 Soit E un K-espace vectoriel normé complet,
\\sum ~
a_nz^n une série entière à coefficients dans E.
Posons R_1 =\
sup\z∣\\\sum
 a_nz^n\text converge
\ \in \mathbb{R}~^+ \cup\ +
\infty~\ et R_2 =\
sup\z∣(a_nz^n)\text
est bornée \ \in \mathbb{R}~^+ \cup\ +
\infty~\. On a R_1 = R_2. En notant R la
valeur commune, la série converge absolument dans D(0,R) =
\z \in K∣z
\textless{} R\ et converge normalement dans tout disque
fermé D'(0,r) = \z \in
K∣z\leq r\
tel que r \textless{} R.

Démonstration Soit r \in {[}0,R_1{[}~; d'après la propriété
caractéristique de la borne supérieure, il existe z \in K tel que la série
\\sum ~
a_nz^n converge avec r \textless{}
z\leq R_1. Mais alors
lima_nz^n~ = 0, donc la
suite (a_nz^n) est bornée et a fortiori la suite
(a_nr^n) est bornée~; donc r \in {[}0,R_2{]},
soit {[}0,R_1{[}\subset~ {[}0,R_2{]} et donc R_1 \leq
R_2. Soit r \in {[}0,R_2{[}~; d'après la propriété
caractéristique de la borne supérieure, il existe z \in K tel que la suite
(a_nz^n) soit bornée avec r \textless{}
z\leq R_2. Mais alors, d'après le lemme
d'Abel, la série \\sum ~
a_nr^n converge absolument, r \in
{[}0,R_1{]}, soit {[}0,R_2{[}\subset~ {[}0,R_1{]} et
donc R_2 \leq R_1.

Soit alors z \in D(0,R)~; il existe z_0 \in K tel que la suite
(a_nz_0^n) soit bornée avec
z \textless{} z_0\leq R.
Mais alors, d'après le lemme d'Abel, la série
\\sum ~
a_nz^n converge absolument. De même, soit r
\textless{} R~; il existe z_0 \in K tel que la suite
(a_nz_0^n) soit bornée avec r \textless{}
z_0\leq R. Mais alors, d'après le lemme
d'Abel, la série \\sum ~
a_nz^n converge normalement sur D'(0,r).

Remarque~11.1.2 Le lecteur prendra garde au fait qu'en général la série
ne converge pas normalement sur D(0,R) ni même uniformément.

Définition~11.1.2 R est appelé le rayon de convergence de la série
entière, D(0,R) son disque ouvert de convergence, C(0,R) =
\z \in K∣z
= R\ son cercle de convergence.

Remarque~11.1.3 Par définition même au vu des résultats précédents, la
série converge absolument dans le disque ouvert de convergence (et même
uniformément dans tout disque fermé inclus dans le disque ouvert de
convergence)~; pour z \textgreater{} R la série
diverge et en fait, la suite (a_nz^n) n'est même pas
bornée. La nature de la série sur le disque fermé de convergence dépend
de la série et du point considéré.

Exemple~11.1.1 Soit \alpha~ \in \mathbb{R}~~; la série entière
\\sum   z^n~
\over n^\alpha~ a pour rayon de convergence 1~;
pour z = 1 la nature de la série dépend à la fois de
\alpha~ et de z.

\begin{itemize}
\itemsep1pt\parskip0pt\parsep0pt
\item
  (i) Pour \alpha~ \textgreater{} 1, la série converge pour tout z tel que
  z = 1
\item
  (ii) Pour \alpha~ \leq 0, la série diverge pour tout z tel que
  z = 1 (le terme général ne tend pas vers 0)
\item
  (iii) Pour 0 \textless{} \alpha~ \leq 1, la série diverge en z = 1 mais
  converge pout tout point z tel que z = 1 et
  z\neq~1 (appliquer le critère d'Abel).
\end{itemize}

\paragraph{11.1.3 Recherche du rayon de convergence}

Les deux remarques suivantes, qui découlent immédiatement des résultats
précédents peuvent rendre de grands services dans la détermination du
rayon de convergence

\begin{itemize}
\itemsep1pt\parskip0pt\parsep0pt
\item
  (i) si z \in K est tel que la série
  \\sum ~
  a_nz^n converge, alors z\leq R
\item
  (ii) si z \in K est tel que la série
  \\sum ~
  a_nz^n diverge, alors R \leqz
\end{itemize}

On pourra éventuellement, pour cette recherche, trouver refuge dans l'un
des théorèmes suivants

Théorème~11.1.3 (règle de d'Alembert). On suppose que
\forall~~n \in \mathbb{N}~,
a_n\neq~0~; si la suite 
\a_n+1\
\over
\a_n\ admet
une limite \ell \in \mathbb{R}~^+ \cup\ + \infty~\,
alors le rayon de convergence de la série entière
\\sum ~
a_nz^n est  1 \over \ell .

Démonstration Il suffit d'appliquer la règle de d'Alembert pour les
séries numériques en remarquant que 
\a_n+1z^n+1\
\over
\a_nz^n\
admet la limite \ellz et que donc la série converge
absolument pour \ellz \textless{} 1 et diverge pour
\ellz \textgreater{} 1.

Remarque~11.1.4 On prendra garde à la condition
\forall~~n \in \mathbb{N}~,
a_n\neq~0. En particulier, on ne tentera
pas d'appliquer cette règle à des séries comportant une infinité de
termes nuls comme les séries entières du type
\\sum ~
a_nz^2n ou
\\sum ~
a_nz^n^2 ~; c'est ainsi qu'une
application imprudente de la règle de d'Alembert à la série entière
\\sum ~
3^nz^2n pourrait faire croire que le rayon de
convergence est  1 \over 3 alors que l'écriture
3^nz^2n =
\left
(3z^2\right )^n
montre qu'il vaut  1 \over \sqrt3
.

Exemple~11.1.2 Pour \\\sum
  z^n \over n! , on a R = +\infty~~; pour
\\sum   z^n~
\over n^\alpha~ , on a R = 1~; pour
\\sum  n!z^n~,
on a R = 0 (la série diverge pour tout z\neq~0).

Théorème~11.1.4 (règle d'Hadamard). Le rayon de convergence de la série
entière \\sum ~
a_nz^n est égal à

 1 \over
limsup\rootn\of\a_n\~
\in \mathbb{R}~^+ \cup\ + \infty~\

Démonstration Posons \ell =\
limsup\rootn\of\a_n\
\in \mathbb{R}~^+ \cup\ + \infty~\. Soit z \in K
tel que z \textless{} 1 \over \ell .
On a alors \ell \textless{} 1 \over
z . Soit donc \rho tel que \ell \textless{} \rho
\textless{} 1 \over z . D'après
la propriété de la limite supérieure, il existe N \in \mathbb{N}~ tel que n ≥ N
\rigtharrow~\rootn\of\a_n\
\leq \rho soit encore
\a_n\ \leq
\rho^n et donc
\a_nz^n\
\leq (\rhoz)^n. Mais \rhoz
\textless{} 1 et donc la série
\\sum ~
(\rhoz)^n converge. On en déduit que la
série \\sum ~
a_nz^n converge absolument, soit R ≥ 1
\over \ell . De plus, si z
\textgreater{} 1 \over \ell , on a \ell \textgreater{} 1
\over z . Comme \ell est valeur
d'adhérence de la suite
(\rootn\of\a_n\),
il existe une infinité de n tels que
\rootn\of\a_n\
\textgreater{} 1 \over z soit
\a_nz^n\
\textgreater{} 1~; la suite (a_nz^n) ne tend pas
vers 0, donc la série diverge~; ceci montre que R \leq 1
\over \ell , ce qui achève la démonstration.

Remarque~11.1.5 En particulier, si la suite
(\rootn\of\a_n\)
converge vers \ell, on a R = 1 \over \ell .

\paragraph{11.1.4 Opérations sur les séries entières}

Proposition~11.1.5 Soit
\\sum ~
a_nz^n et
\\sum ~
b_nz^n deux séries entières à coefficients dans E de
rayons de convergence respectifs R_1 et R_2, \alpha~ et \beta~
des scalaires. Alors la série entière
\\sum  (\alpha~a_n~ +
\beta~b_n)z^n a un rayon de convergence supérieur ou égal
à min(R_1,R_2~) et

z \textless{}\
min(R_1,R_2) \rigtharrow~\\sum
_n=0^+\infty~(\alpha~a_ n + \beta~b_n)z^n =
\alpha~\sum _n=0^+\infty~a_
nz^n + \beta~\\sum
_n=0^+\infty~b_ nz^n

Démonstration En effet, si z
\textless{} min(R_1,R_2~),
les deux séries \\sum ~
a_nz^n et
\\sum ~
b_nz^n sont convergentes, et donc la série
\\sum  (\alpha~a_n~ +
\beta~b_n)z^n converge également, soit R
≥ min(R_1,R_2~). La formule
découle immédiatement du résultat similaire sur les séries numériques.

Remarque~11.1.6 L'exemple b_n = -a_n, \alpha~ = \beta~ = 1,
montre que l'on peut avoir R \textgreater{}\
min(R_1,R_2)

Proposition~11.1.6 Soit
\\sum ~
a_nz^n et
\\sum ~
b_nz^n deux séries entières à coefficients dans K de
rayons de convergence respectifs R_1 et R_2. Posons
c_n = \\sum ~
_k=0^na_kb_n-k
= \\sum ~
_p+q=na_pb_q (série entière produit). Alors la
série entière \\sum ~
c_nz^n a un rayon de convergence supérieur ou égal à
min(R_1,R_2~) et

z \textless{}\
min(R_1,R_2) \rigtharrow~\\sum
_n=0^+\infty~c_ nz^n =
\left (\\sum
_n=0^+\infty~a_ nz^n\right
)\left (\\sum
_n=0^+\infty~b_ nz^n\right
)

Démonstration En effet, si z
\textless{} min(R_1,R_2~),
les deux séries \\sum ~
a_nz^n et
\\sum ~
b_nz^n sont absolument convergentes, et donc la
série produit de Cauchy est également absolument convergente. Mais on a
\\sum ~
_p+q=n(a_pz^p)(b_qz^q) =
z^n \\sum ~
_p+q=na_pb_q = c_nz^n. On a
donc R ≥ min(R_1,R_2~). La
formule découle immédiatement du résultat similaire sur les séries
numériques (la somme du produit de Cauchy est le produit des sommes des
séries).

Proposition~11.1.7 Soit
\\sum ~
a_nz^n une série entière à coefficients dans K avec
a_0\neq~0. Il existe alors une unique
série entière \\sum ~
b_nz^n tel que le produit des deux séries entières
soit la constante 1. Si
\\sum ~
a_nz^n a un rayon de convergence non nul R, il en
est de même du rayon de convergence R' de la série entière
\\sum ~
b_nz^n. On a

z \textless{} min~(R,R')
\rigtharrow~\sum _n=0^+\infty~b_
nz^n = 1 \over \\sum
_n=0^+\infty~a_nz^n

Démonstration On doit avoir a_0b_0 = 1 et pour n ≥ 1,
\\sum ~
_k=0^na_n-kb_k = 0. La suite
(b_n) est donc définie par b_0 = 1
\over a_0 et pour n ≥ 1, b_n = - 1
\over a_0 \
\sum ~
_k=0^n-1a_n-kb_k ce qui définit
parfaitement la suite par récurrence. Remarquons que si l'on multiplie
tous les a_n par \lambda~\neq~0, tous les
b_n sont divisés par \lambda~, et les rayons de convergence ne sont
pas modifiés. Sans nuire à la généralité, on peut donc supposer que
a_0 = 1. On a alors b_0 = 1 et b_n =
-\\sum ~
_k=0^n-1a_n-kb_k. Supposons donc R
\textgreater{} 0 et soit r \textless{} R. La suite
(a_nr^n) est donc bornée. Soit M tel que
\forall~~n,
a_nr^n \leq M soit
a_n\leq Mr^-n. On va montrer que
\forall~n ≥ 1, b_n~\leq
M(M + 1)^n-1r^-n par récurrence sur n. Pour n = 1,
on a b_1 = -a_1, soit b_1
= a_1\leq Mr^-1 ce qui est bien
l'inégalité souhaitée. Supposons l'inégalité vérifiée de 1 à n - 1. On a
alors (compte tenu de b_0 = 1)

\begin{align*} b_n&
\leq& a_n + \\sum
_k=1^n-1a_
n-kb_k \%&
\\ & \leq& Mr^-n +
\sum _k=1^n-1Mr^k-n~M(M
+ 1)^k-1r^-k \%& \\
& =& Mr^-n\left (1 +
M\sum _k=1^n-1~(M +
1)^k-1\right ) \%&
\\ & =&
Mr^-n\left (1 + M (M + 1)^n-1 - 1
\over (M + 1) - 1 \right ) = M(M +
1)^n-1r^-n\%& \\
\end{align*}

ce qui achève la récurrence. Alors
b_nz^n\leq M \over
M+1 \left ( (M+1)z
\over r \right )^n ce qui
montre que la série \\\sum
 b_nz^n converge pour z
\textless{} r \over M+1 , soit R' ≥ r
\over M+1 \textgreater{} 0.

La formule découle immédiatement de la proposition précédente.

Remarque~11.1.7 L'exemple a_0 = 1, a_1 = -1,
a_n = 0 pour n ≥ 2 (c'est-à-dire de la série entière 1 - z),
pour laquelle la série inverse est la série
\\sum  z^n~
pour laquelle R' = 1, montre qu'on ne peut pas dire grand chose de la
valeur effective de R'. On peut avoir aussi bien R' \leq R que R' ≥ R
(échanger le rôle de \\\sum
 a_nz^n et
\\sum ~
b_nz^n).

{[}
{[}

\end{document}

\documentclass[]{article}
\usepackage[T1]{fontenc}
\usepackage{lmodern}
\usepackage{amssymb,amsmath}
\usepackage{ifxetex,ifluatex}
\usepackage{fixltx2e} % provides \textsubscript
% use upquote if available, for straight quotes in verbatim environments
\IfFileExists{upquote.sty}{\usepackage{upquote}}{}
\ifnum 0\ifxetex 1\fi\ifluatex 1\fi=0 % if pdftex
  \usepackage[utf8]{inputenc}
\else % if luatex or xelatex
  \ifxetex
    \usepackage{mathspec}
    \usepackage{xltxtra,xunicode}
  \else
    \usepackage{fontspec}
  \fi
  \defaultfontfeatures{Mapping=tex-text,Scale=MatchLowercase}
  \newcommand{\euro}{€}
\fi
% use microtype if available
\IfFileExists{microtype.sty}{\usepackage{microtype}}{}
\ifxetex
  \usepackage[setpagesize=false, % page size defined by xetex
              unicode=false, % unicode breaks when used with xetex
              xetex]{hyperref}
\else
  \usepackage[unicode=true]{hyperref}
\fi
\hypersetup{breaklinks=true,
            bookmarks=true,
            pdfauthor={},
            pdftitle={Somme d'une serie enti`ere},
            colorlinks=true,
            citecolor=blue,
            urlcolor=blue,
            linkcolor=magenta,
            pdfborder={0 0 0}}
\urlstyle{same}  % don't use monospace font for urls
\setlength{\parindent}{0pt}
\setlength{\parskip}{6pt plus 2pt minus 1pt}
\setlength{\emergencystretch}{3em}  % prevent overfull lines
\setcounter{secnumdepth}{0}
 
/* start css.sty */
.cmr-5{font-size:50%;}
.cmr-7{font-size:70%;}
.cmmi-5{font-size:50%;font-style: italic;}
.cmmi-7{font-size:70%;font-style: italic;}
.cmmi-10{font-style: italic;}
.cmsy-5{font-size:50%;}
.cmsy-7{font-size:70%;}
.cmex-7{font-size:70%;}
.cmex-7x-x-71{font-size:49%;}
.msbm-7{font-size:70%;}
.cmtt-10{font-family: monospace;}
.cmti-10{ font-style: italic;}
.cmbx-10{ font-weight: bold;}
.cmr-17x-x-120{font-size:204%;}
.cmsl-10{font-style: oblique;}
.cmti-7x-x-71{font-size:49%; font-style: italic;}
.cmbxti-10{ font-weight: bold; font-style: italic;}
p.noindent { text-indent: 0em }
td p.noindent { text-indent: 0em; margin-top:0em; }
p.nopar { text-indent: 0em; }
p.indent{ text-indent: 1.5em }
@media print {div.crosslinks {visibility:hidden;}}
a img { border-top: 0; border-left: 0; border-right: 0; }
center { margin-top:1em; margin-bottom:1em; }
td center { margin-top:0em; margin-bottom:0em; }
.Canvas { position:relative; }
li p.indent { text-indent: 0em }
.enumerate1 {list-style-type:decimal;}
.enumerate2 {list-style-type:lower-alpha;}
.enumerate3 {list-style-type:lower-roman;}
.enumerate4 {list-style-type:upper-alpha;}
div.newtheorem { margin-bottom: 2em; margin-top: 2em;}
.obeylines-h,.obeylines-v {white-space: nowrap; }
div.obeylines-v p { margin-top:0; margin-bottom:0; }
.overline{ text-decoration:overline; }
.overline img{ border-top: 1px solid black; }
td.displaylines {text-align:center; white-space:nowrap;}
.centerline {text-align:center;}
.rightline {text-align:right;}
div.verbatim {font-family: monospace; white-space: nowrap; text-align:left; clear:both; }
.fbox {padding-left:3.0pt; padding-right:3.0pt; text-indent:0pt; border:solid black 0.4pt; }
div.fbox {display:table}
div.center div.fbox {text-align:center; clear:both; padding-left:3.0pt; padding-right:3.0pt; text-indent:0pt; border:solid black 0.4pt; }
div.minipage{width:100%;}
div.center, div.center div.center {text-align: center; margin-left:1em; margin-right:1em;}
div.center div {text-align: left;}
div.flushright, div.flushright div.flushright {text-align: right;}
div.flushright div {text-align: left;}
div.flushleft {text-align: left;}
.underline{ text-decoration:underline; }
.underline img{ border-bottom: 1px solid black; margin-bottom:1pt; }
.framebox-c, .framebox-l, .framebox-r { padding-left:3.0pt; padding-right:3.0pt; text-indent:0pt; border:solid black 0.4pt; }
.framebox-c {text-align:center;}
.framebox-l {text-align:left;}
.framebox-r {text-align:right;}
span.thank-mark{ vertical-align: super }
span.footnote-mark sup.textsuperscript, span.footnote-mark a sup.textsuperscript{ font-size:80%; }
div.tabular, div.center div.tabular {text-align: center; margin-top:0.5em; margin-bottom:0.5em; }
table.tabular td p{margin-top:0em;}
table.tabular {margin-left: auto; margin-right: auto;}
div.td00{ margin-left:0pt; margin-right:0pt; }
div.td01{ margin-left:0pt; margin-right:5pt; }
div.td10{ margin-left:5pt; margin-right:0pt; }
div.td11{ margin-left:5pt; margin-right:5pt; }
table[rules] {border-left:solid black 0.4pt; border-right:solid black 0.4pt; }
td.td00{ padding-left:0pt; padding-right:0pt; }
td.td01{ padding-left:0pt; padding-right:5pt; }
td.td10{ padding-left:5pt; padding-right:0pt; }
td.td11{ padding-left:5pt; padding-right:5pt; }
table[rules] {border-left:solid black 0.4pt; border-right:solid black 0.4pt; }
.hline hr, .cline hr{ height : 1px; margin:0px; }
.tabbing-right {text-align:right;}
span.TEX {letter-spacing: -0.125em; }
span.TEX span.E{ position:relative;top:0.5ex;left:-0.0417em;}
a span.TEX span.E {text-decoration: none; }
span.LATEX span.A{ position:relative; top:-0.5ex; left:-0.4em; font-size:85%;}
span.LATEX span.TEX{ position:relative; left: -0.4em; }
div.float img, div.float .caption {text-align:center;}
div.figure img, div.figure .caption {text-align:center;}
.marginpar {width:20%; float:right; text-align:left; margin-left:auto; margin-top:0.5em; font-size:85%; text-decoration:underline;}
.marginpar p{margin-top:0.4em; margin-bottom:0.4em;}
.equation td{text-align:center; vertical-align:middle; }
td.eq-no{ width:5%; }
table.equation { width:100%; } 
div.math-display, div.par-math-display{text-align:center;}
math .texttt { font-family: monospace; }
math .textit { font-style: italic; }
math .textsl { font-style: oblique; }
math .textsf { font-family: sans-serif; }
math .textbf { font-weight: bold; }
.partToc a, .partToc, .likepartToc a, .likepartToc {line-height: 200%; font-weight:bold; font-size:110%;}
.chapterToc a, .chapterToc, .likechapterToc a, .likechapterToc, .appendixToc a, .appendixToc {line-height: 200%; font-weight:bold;}
.index-item, .index-subitem, .index-subsubitem {display:block}
.caption td.id{font-weight: bold; white-space: nowrap; }
table.caption {text-align:center;}
h1.partHead{text-align: center}
p.bibitem { text-indent: -2em; margin-left: 2em; margin-top:0.6em; margin-bottom:0.6em; }
p.bibitem-p { text-indent: 0em; margin-left: 2em; margin-top:0.6em; margin-bottom:0.6em; }
.paragraphHead, .likeparagraphHead { margin-top:2em; font-weight: bold;}
.subparagraphHead, .likesubparagraphHead { font-weight: bold;}
.quote {margin-bottom:0.25em; margin-top:0.25em; margin-left:1em; margin-right:1em; text-align:\jmathustify;}
.verse{white-space:nowrap; margin-left:2em}
div.maketitle {text-align:center;}
h2.titleHead{text-align:center;}
div.maketitle{ margin-bottom: 2em; }
div.author, div.date {text-align:center;}
div.thanks{text-align:left; margin-left:10%; font-size:85%; font-style:italic; }
div.author{white-space: nowrap;}
.quotation {margin-bottom:0.25em; margin-top:0.25em; margin-left:1em; }
h1.partHead{text-align: center}
.sectionToc, .likesectionToc {margin-left:2em;}
.subsectionToc, .likesubsectionToc {margin-left:4em;}
.subsubsectionToc, .likesubsubsectionToc {margin-left:6em;}
.frenchb-nbsp{font-size:75%;}
.frenchb-thinspace{font-size:75%;}
.figure img.graphics {margin-left:10%;}
/* end css.sty */

\title{Somme d'une serie enti`ere}
\author{}
\date{}

\begin{document}
\maketitle

\textbf{Warning: 
requires JavaScript to process the mathematics on this page.\\ If your
browser supports JavaScript, be sure it is enabled.}

\begin{center}\rule{3in}{0.4pt}\end{center}

{[}
{[}
{[}{]}
{[}

\subsubsection{11.2 Somme d'une série entière}

\paragraph{11.2.1 Etude sur le disque ouvert de convergence (domaine
complexe)}

Théorème~11.2.1 (continuité de la somme). Soit
\\sum ~
a\_nz^n une série entière à coefficients dans E, de
rayon de convergence R \textgreater{} 0. Alors la fonction S :
z\mapsto~\\\sum
 \_n=0^+\infty~a\_nz^n est continue sur le
disque D(0,R) = \z \in
K∣\textbar{}z\textbar{} \textless{}
R\.

Démonstration On a vu en effet que la série convergeait normalement sur
D'(0,r) pour r \textless{} R, donc S est continue sur un tel D'(0,r) et
donc finalement sur D(0,R).

Corollaire~11.2.2 (principe des zéros isolés). Soit
\\sum ~
a\_nz^n une série entière non nulle à coefficients
dans E, de rayon de convergence R \textgreater{} 0. Alors, il existe \eta
\textgreater{} 0 tel que la fonction S :
z\mapsto~\\\sum
 \_n=0^+\infty~a\_nz^n ne s'annule pas sur
D(0,\eta) \diagdown\0\.

Démonstration Soit en effet p =\
min\k \in
\mathbb{N}~∣a\_k\mathrel\neq~0\.
On a alors S(z) =\ \\sum
 \_n=p^+\infty~a\_nz^n =
z^p \\sum ~
\_n=p^+\infty~a\_nz^n-p =
z^p \\sum ~
\_n=0^+\infty~a\_n+pz^n. Mais la série entière
\\sum ~
\_na\_n+pz^n a même rayon de convergence que la
série entière \\sum ~
a\_nz^n (facile) et sa somme définit donc une
fonction s continue sur D(0,R) avec s(0) =
a\_p\neq~0. Donc il existe \eta
\textgreater{} 0 tel que \textbar{}z\textbar{} \textless{} \eta \rigtharrow~
s(z)\neq~0. Mais alors, pour 0 \textless{}
\textbar{}z\textbar{} \textless{} \eta, on a S(z) =
z^ps(z)\neq~0, ce que l'on voulait
démontrer.

Corollaire~11.2.3 (principe d'identification). Soit
\\sum ~
a\_nz^n et et
\\sum ~
b\_nz^n deux séries entières à coefficients dans E,
de rayons de convergence non nuls, de sommes S\_1 et
S\_2. Alors on a équivalence de (i) \forall~~n
\in \mathbb{N}~, a\_n = b\_n (ii) il existe \eta \textgreater{} 0 tel
que \forall~z \in D(0,\eta), S\_1~(z) =
S\_2(z) (iii) il existe une suite (z\_n) de K formée
d'éléments distincts telle que limz\_n~
= 0 et \forall~n \in \mathbb{N}~, S\_1(z\_n~) =
S\_2(z\_n)

Démonstration Il suffit d'appliquer le principe des zéros isolés à la
série entière \\sum ~
(a\_n - b\_n)z^n dont la somme est
S\_1 - S\_2 dans le disque
D(0,min(R\_1,R\_2~)).

Remarque~11.2.1 Le corollaire précédent qui garantit l'unicité du
développement en série entière d'une fonction est très souvent utilisé~;
il permet en particulier de travailler par identification. Il laisse
penser qu'il doit être possible de récupérer les valeurs des
coefficients a\_n à partir de la somme S de la série. En fait,
dans une première approche, les techniques sont très différentes suivant
que le corps de base est \mathbb{C} ou \mathbb{R}~.

Dans le cadre complexe, on a le théorème suivant qui relie les
coefficients du développement en série entière à la somme de la fonction

Théorème~11.2.4 (formules de Cauchy). Soit E un \mathbb{C}-espace vectoriel normé
complet, \\sum ~
a\_nz^n une série entière à coefficients dans E, de
rayon de convergence R \textgreater{} 0, de somme S. Alors, pour tout r
\textless{} R, on a

\forall~n \in \mathbb{N}~, a\_n~ = 1
\over 2\pi~r^n \int ~
\_0^2\pi~S(re^i\theta)e^-in\theta d\theta

Démonstration Puisque r \textless{} R, la série
\\sum ~
\\textbar{}a\_n\\textbar{}r^n
est convergente.

On a S(re^i\theta)e^-in\theta =\
\sum ~
\_p=0^+\infty~a\_pr^pe^i(p-n)\theta.
Mais l'inégalité
\\textbar{}a\_nr^pe^i(p-n)\theta\\textbar{}
\leq\\textbar{}
a\_p\\textbar{}r^p montre que la série
converge normalement par rapport à \theta. On en déduit que

\begin{align*} \int ~
\_0^2\pi~S(re^i\theta)e^-in\theta d\theta& =&
\int  \_0^2\pi~~
\sum \_p=0^+\infty~a~\_
pr^pe^i(p-n)\theta d\theta\%&
\\ & =& \\sum
\_p=0^+\infty~a\_ pr^p
\\int  ~
\_0^2\pi~e^i(p-n)\theta d\theta\%&
\\ & =& 2\pi~a\_nr^n
\%& \\ \end{align*}

car \int  \_0^2\pi~e^ik\theta~
d\theta = \left \ \cases 0
&si k\neq~0 \cr 2\pi~&si k = 0 
\right .. On obtient donc la formule ci dessus.

Théorème~11.2.5 Soit \\\sum
 a\_nz\_n une série entière de rayon de convergence
R et de somme S(z). Soit z\_0 \in \mathbb{C} tel que
\textbar{}z\_0\textbar{} \textless{} R. Alors la fonction
S(z\_0 + u) est développable en série entière de u dans le
disque ouvert \textbar{}u\textbar{} \textless{} R
-\textbar{}z\_0\textbar{}, ce que l'on traduit par~: la somme
d'une série entière est analytique dans son disque ouvert de
convergence.

Démonstration Puisque \textbar{}z\_0 +
u\textbar{}\leq\textbar{}z\_0\textbar{} + \textbar{}u\textbar{}
\textless{} R, on peut écrire

\begin{align*} S(z\_0 + u)& =&
\sum \_n=0^+\infty~a~\_
n(z\_0 + u)^n \%& \\
& =& \\sum
\_n=0^+\infty~\left (\\sum
\_m=0^nC\_ n^ma\_
nz\_0^n-mu^m\right )\%&
\\ \end{align*}

On considère alors la famille (x\_m,n)\_m,n\in\mathbb{N}~ définie
par

 x\_m,n = \left \
\cases
C\_n^ma\_nz\_0^n-mu^m&si
m \leq n \cr 0 &si m \textgreater{} n 
\right .

On a

\sum \_m=0^+\infty~\textbar{}x~\_
m,n\textbar{} = \\sum
\_m=0^n\textbar{}C\_ n^ma\_
nz\_0^n-mu^m\textbar{} = \textbar{}a\_
n\textbar{}(\textbar{}z\_0\textbar{} +
\textbar{}u\textbar{})^n

qui est une série convergente puisque la série
\\sum ~
\_n\textbar{}a\_n\textbar{}(\textbar{}z\_0\textbar{}
+ \textbar{}u\textbar{})^n converge (une série entière converge
absolument dans son disque ouvert de convergence). Ceci montre que la
famille (x\_m,n)\_m,n\in\mathbb{N}~ est sommable. On peut donc
appliquer d'interversion des sommations et on a

\begin{align*} S(z\_0 + u)& =&
\sum \_n=0^+\infty~~\left
(\sum \_m=0^+\infty~x~\_
m,n\right ) = \\sum
\_m=0^+\infty~\left (\\sum
\_n=0^+\infty~x\_ m,n\right )\%&
\\ & =& \\sum
\_m=0^+\infty~u^m\left
(\sum \_n=m^+\infty~C~\_
n^ma\_ nz\_0^n-m\right )
\%& \\ \end{align*}

ce qui montre le résultat.

\paragraph{11.2.2 Etude sur le disque ouvert de convergence (domaine
réel)}

Avant de regarder le cas réel, nous allons démontrer le lemme suivant

Lemme~11.2.6 Soit \\sum ~
a\_nz^n une série entière à coefficients dans le
K-espace vectoriel normé E, de rayon de convergence R. Soit F \in K(X) une
fraction rationnelle non nulle et N \in \mathbb{N}~ tel que F n'ait pas de pôle
entier supérieur à N. Alors la série entière
\\sum ~
F(n)a\_nz^n a encore pour rayon de convergence R.

Démonstration Soit z \in K tel que \textbar{}z\textbar{} \textless{} R et
soit r tel que \textbar{}z\textbar{} \textless{} r \textless{} R. La
suite
\\textbar{}a\_n\\textbar{}r^n
est donc bornée, par exemple ma\jmathorée par M. On a alors, pour n ≥ N,
\\textbar{}F(n)a\_nz^n\\textbar{}
\leq M\textbar{}F(n)\textbar{}\left (
\textbar{}z\textbar{} \over r \right
)^n ∼ \lambda~n^d\left (
\textbar{}z\textbar{} \over r \right
)^n (où d est le degré de la fraction rationnelle, différence
entre le degré de son numérateur et celui de son dénominateur, si bien
que \textbar{}F(t)\textbar{}∼ \lambda~t^d au voisinage de + \infty~) qui
tend vers 0 quand n tend vers + \infty~. On a donc R' ≥ R. Mais on a aussi
a\_n = 1 \over F (n)\left
(F(n)a\_n\right ) si bien que les suites
(a\_n) et (F(n)a\_n) \jmathouent ici un rôle parfaitement
symétrique. On a donc aussi R ≥ R', soit R = R'.

On va en déduire le théorème suivant

Théorème~11.2.7 Soit E un \mathbb{R}~-espace vectoriel normé complet,
\\sum ~
a\_nt^n une série entière à coefficients dans E, de
rayon de convergence R \textgreater{} 0, de somme S. Alors la fonction S
est de classe C^\infty~ sur {]} - R,R{[} et
\forall~p \in \mathbb{N}~, \\forall~~t \in{]} -
R,R{[}

\begin{align*} S^(p)(t)& =&
\sum \_n=p^+\infty~~n(n -
1)\ldots(n - p + 1)a~\_
pt^n-p\%& \\ & =&
\sum \_n=0^+\infty~~ (n + p)!
\over n! a\_n+pt^n \%&
\\ \end{align*}

Démonstration Les deux formules se déduisent l'une de l'autre par un
changement d'indice (le changement de n - p en n). Il suffit donc de
montrer la première. Mais le lemme précédent assure que la série entière
\\sum  \_n≥p~n(n
- 1)\\ldots~(n - p +
1)a\_pt^n a même rayon de convergence R que la série
de départ. Il en est donc de même de la série entière
\\sum  \_n≥p~n(n
- 1)\\ldots~(n - p +
1)a\_pt^n-p et cette série converge donc normalement
sur {[}-r,r{]} pour r \textless{} R. Montrons donc le résultat par
récurrence sur p. Pour p = 0, il n'y a rien à montrer. Supposons le
résultat vrai pour p avec \forall~~t \in{]} - R,R{[},
S^(p)(t) =\
\sum  \_n=p^+\infty~~n(n -
1)\\ldots~(n - p +
1)a\_pt^n-p. La série dérivée
\\sum ~
\_n≥p+1n(n -
1)\\ldots~(n - p +
1)(n - p)a\_pt^n-p-1 converge normalement sur
{[}-r,+r{]} et le théorème de dérivation des séries de fonctions nous
garantit que S^(p) est de classe \mathcal{C}^1 (donc S de
classe C^p+1) sur {[}-r,r{]} avec
\forall~t \in {[}-r,r{]}, S^(p)~(t)
= \\sum ~
\_n=p^+\infty~n(n -
1)\\ldots~(n - p +
1)(n - p)a\_pt^n-p-1~; mais comme r est quelconque (r
\textless{} R), S est de classe C^p+1 sur {]} - R,R{[} et la
formule ci-dessus y reste valable, ce qui achève la récurrence.

Corollaire~11.2.8 Soit E un \mathbb{R}~-espace vectoriel normé complet,
\\sum ~
a\_nt^n une série entière à coefficients dans E, de
rayon de convergence R \textgreater{} 0, de somme S. Alors

\forall~n \in \mathbb{N}~, a\_n = S^(n)~(0)
\over n!

Démonstration Faire t = 0 dans la formule précédente.

Remarque~11.2.2 Les coefficients a\_n sont donc les mêmes que
ceux qui apparaissent dans un développement limité en 0 de la fonction
S.

Le même argument de convergence normale sur {[}-r,r{]} \subset~{]} - R,R{[}
montrera le théorème suivant

Théorème~11.2.9 Soit E un \mathbb{R}~-espace vectoriel normé complet,
\\sum ~
a\_nt^n une série entière à coefficients dans E, de
rayon de convergence R \textgreater{} 0, de somme S. Alors

\forall~~t \in{]} - R,R{[}, \\int
 \_0^tS(u) du = \\sum
\_n=0^+\infty~a\_ n t^n+1
\over n + 1

\paragraph{11.2.3 Etude sur le cercle de convergence}

On a vu qu'en un point du cercle de convergence, la série pouvait aussi
bien diverger que converger. Si la série converge, la question de la
continuité de la somme en ce point se pose immédiatement. En fait, on
peut montrer que sur \mathbb{C}, la somme peut très bien être discontinue en un
tel point, mais qu'il s'agit en fait d'une discontinuité tangentielle~:
il se peut que S(z) ne tende pas vers S(z\_0) quand z tend vers
z\_0 tangentiellement au cercle de convergence. Pour nous il
nous suffira de savoir que S(z) tend vers S(z\_0) quand z tend
vers z\_0 suivant un rayon, ce que garantit le théorème suivant

Théorème~11.2.10 (Abel) Soit
\\sum ~
a\_nz^n une série entière à coefficients dans E, de
rayon de convergence R \textgreater{} 0 et S :
z\mapsto~\\\sum
 \_n=0^+\infty~a\_nz^n continue sur le
disque D(0,R) = \z \in
K∣\textbar{}z\textbar{} \textless{}
R\. Soit z\_0 \in K tel que
\textbar{}z\_0\textbar{} = R et la série
\\sum ~
a\_nz\_0^n converge. Alors

\sum \_n=0^+\infty~a~\_
nz\_0^n = lim\_
t\rightarrow~1^-S(tz\_0)

Démonstration On considère la série de fonctions
\\sum ~
a\_nz\_0^nt^n, qui converge sur
{[}0,1{]}. Nous allons démontrer sa convergence uniforme~; ceci
garantira la continuité de sa somme au point 1, ce qui n'est autre
l'assertion à démontrer.

Premier cas~: la série \\\sum
 a\_nz\_0^n converge absolument. Alors on a
\forall~~t \in {[}0,1{]},
\\textbar{}a\_nz\_0^nt^n\\textbar{}
\leq\\textbar{}
a\_nz\_0^n\\textbar{}, série
convergente indépendante de t. Donc la série converge normalement.

Deuxième cas~: le critère des séries alternées s'applique à la série
\\sum ~
a\_nz\_0^n, autrement dit
a\_nz\_0^n = (-1)^nb\_n avec
(b\_n) qui tend vers 0 en décroissant. Alors
a\_nz\_0^nt^n =
(-1)^nb\_nt^n, avec
t\mapsto~b\_nt^n qui tend
uniformément vers 0 en décroissant. Le critère de convergence uniforme
des séries alternées garantit la convergence uniforme de la série.

Cas général~: nous allons montrer que la série de fonctions vérifie le
critère de Cauchy uniforme. Pour cela posons R\_n
= \\sum ~
\_k=n^+\infty~a\_kz\_0^k. On a alors

\begin{align*} \\sum
\_n=p^qa\_ nz\_0^nt^n&
=& \sum \_n=p^q(R\_ n~ -
R\_n+1)t^n = \\sum
\_n=p^qR\_ nt^n
-\sum \_n=p^qR~\_
n+1t^n\%& \\ & =&
\sum \_n=p^qR~\_
nt^n -\\sum
\_n=p+1^q+1R\_ nt^n-1 \%&
\\ & =& R\_pt^p -
R\_ q+1t^q -\\sum
\_n=p+1^qR\_ n(t^n-1 - t^n)
\%& \\ \end{align*}

On a limR\_n~ = 0 (reste d'une série
convergente). Soit \epsilon \textgreater{} 0~; il existe N \in \mathbb{N}~ (indépendant de
t) tel que n ≥ N \rigtharrow~\\textbar{}
R\_n\\textbar{} \textless{} \epsilon
\over 2 . Alors, en tenant compte de t^p ≥
0, t^q ≥ 0 et t^n-1 - t^n ≥ 0, on a
\forall~~t \in {[}0,1{]},

\\textbar{}\\sum
\_n=p^qa\_
nz\_0^nt^n\\textbar{} \leq \epsilon
\over 2 (t^p + t^q +
\sum \_n=p+1^q(t^n-1~ -
t^n)) = 2t^p \epsilon \over 2 \leq \epsilon

ce qui montre que la série vérifie le critère de Cauchy uniforme, donc
est uniformément convergente.

Remarque~11.2.3 Une des premières utilités du théorème précédent est de
calculer la somme de certaines séries numériques du type
\\sum ~
a\_nz\_0^n~; il arrive en effet fréquemment que
la somme de la série entière
\\sum ~
a\_nz^n soit facile à calculer pour
\textbar{}z\textbar{} \textless{} R (par exemple par dérivation ou par
résolution d'une certaine équation différentielle). Il suffit alors de
passer à la limite pour calculer la somme de la série.

Exemple~11.2.1 On cherche à calculer la somme de la série alternée
\\sum ~
\_n=1^+\infty~ (-1)^n-1 \over n .
Pour \textbar{}t\textbar{} \textless{} 1, on pose f(t)
= \\sum ~
\_n=1^+\infty~ (-1)^n-1 \over n
t^n~; on sait que f est C^\infty~ sur {]} - 1,1{[} et
que f'(t) = \\sum ~
\_n=1^+\infty~(-1)^n-1t^n-1 = 1
\over 1+t . Comme f(0) = 0, on a f(t)
= log~ (1 + t). Le théorème précédent assure
que \\sum ~
\_n=1^+\infty~ (-1)^n-1 \over n
=\
lim\_t\rightarrow~1^-log~ (1 + t)
= log~ 2. Suivant le même principe, le lecteur
montrera que \\sum ~
\_n=0^+\infty~ (-1)^n \over 2n+1
= \mathrmarctg~ 1 = \pi~
\over 4 , en introduisant la série entière
\\sum ~
\_n=0^+\infty~ (-1)^n \over 2n+1
t^2n+1.

{[}
{[}
{[}
{[}

\end{document}

\documentclass[]{article}
\usepackage[T1]{fontenc}
\usepackage{lmodern}
\usepackage{amssymb,amsmath}
\usepackage{ifxetex,ifluatex}
\usepackage{fixltx2e} % provides \textsubscript
% use upquote if available, for straight quotes in verbatim environments
\IfFileExists{upquote.sty}{\usepackage{upquote}}{}
\ifnum 0\ifxetex 1\fi\ifluatex 1\fi=0 % if pdftex
  \usepackage[utf8]{inputenc}
\else % if luatex or xelatex
  \ifxetex
    \usepackage{mathspec}
    \usepackage{xltxtra,xunicode}
  \else
    \usepackage{fontspec}
  \fi
  \defaultfontfeatures{Mapping=tex-text,Scale=MatchLowercase}
  \newcommand{\euro}{€}
\fi
% use microtype if available
\IfFileExists{microtype.sty}{\usepackage{microtype}}{}
\ifxetex
  \usepackage[setpagesize=false, % page size defined by xetex
              unicode=false, % unicode breaks when used with xetex
              xetex]{hyperref}
\else
  \usepackage[unicode=true]{hyperref}
\fi
\hypersetup{breaklinks=true,
            bookmarks=true,
            pdfauthor={},
            pdftitle={Developpements en series enti`eres},
            colorlinks=true,
            citecolor=blue,
            urlcolor=blue,
            linkcolor=magenta,
            pdfborder={0 0 0}}
\urlstyle{same}  % don't use monospace font for urls
\setlength{\parindent}{0pt}
\setlength{\parskip}{6pt plus 2pt minus 1pt}
\setlength{\emergencystretch}{3em}  % prevent overfull lines
\setcounter{secnumdepth}{0}
 
/* start css.sty */
.cmr-5{font-size:50%;}
.cmr-7{font-size:70%;}
.cmmi-5{font-size:50%;font-style: italic;}
.cmmi-7{font-size:70%;font-style: italic;}
.cmmi-10{font-style: italic;}
.cmsy-5{font-size:50%;}
.cmsy-7{font-size:70%;}
.cmex-7{font-size:70%;}
.cmex-7x-x-71{font-size:49%;}
.msbm-7{font-size:70%;}
.cmtt-10{font-family: monospace;}
.cmti-10{ font-style: italic;}
.cmbx-10{ font-weight: bold;}
.cmr-17x-x-120{font-size:204%;}
.cmsl-10{font-style: oblique;}
.cmti-7x-x-71{font-size:49%; font-style: italic;}
.cmbxti-10{ font-weight: bold; font-style: italic;}
p.noindent { text-indent: 0em }
td p.noindent { text-indent: 0em; margin-top:0em; }
p.nopar { text-indent: 0em; }
p.indent{ text-indent: 1.5em }
@media print {div.crosslinks {visibility:hidden;}}
a img { border-top: 0; border-left: 0; border-right: 0; }
center { margin-top:1em; margin-bottom:1em; }
td center { margin-top:0em; margin-bottom:0em; }
.Canvas { position:relative; }
li p.indent { text-indent: 0em }
.enumerate1 {list-style-type:decimal;}
.enumerate2 {list-style-type:lower-alpha;}
.enumerate3 {list-style-type:lower-roman;}
.enumerate4 {list-style-type:upper-alpha;}
div.newtheorem { margin-bottom: 2em; margin-top: 2em;}
.obeylines-h,.obeylines-v {white-space: nowrap; }
div.obeylines-v p { margin-top:0; margin-bottom:0; }
.overline{ text-decoration:overline; }
.overline img{ border-top: 1px solid black; }
td.displaylines {text-align:center; white-space:nowrap;}
.centerline {text-align:center;}
.rightline {text-align:right;}
div.verbatim {font-family: monospace; white-space: nowrap; text-align:left; clear:both; }
.fbox {padding-left:3.0pt; padding-right:3.0pt; text-indent:0pt; border:solid black 0.4pt; }
div.fbox {display:table}
div.center div.fbox {text-align:center; clear:both; padding-left:3.0pt; padding-right:3.0pt; text-indent:0pt; border:solid black 0.4pt; }
div.minipage{width:100%;}
div.center, div.center div.center {text-align: center; margin-left:1em; margin-right:1em;}
div.center div {text-align: left;}
div.flushright, div.flushright div.flushright {text-align: right;}
div.flushright div {text-align: left;}
div.flushleft {text-align: left;}
.underline{ text-decoration:underline; }
.underline img{ border-bottom: 1px solid black; margin-bottom:1pt; }
.framebox-c, .framebox-l, .framebox-r { padding-left:3.0pt; padding-right:3.0pt; text-indent:0pt; border:solid black 0.4pt; }
.framebox-c {text-align:center;}
.framebox-l {text-align:left;}
.framebox-r {text-align:right;}
span.thank-mark{ vertical-align: super }
span.footnote-mark sup.textsuperscript, span.footnote-mark a sup.textsuperscript{ font-size:80%; }
div.tabular, div.center div.tabular {text-align: center; margin-top:0.5em; margin-bottom:0.5em; }
table.tabular td p{margin-top:0em;}
table.tabular {margin-left: auto; margin-right: auto;}
div.td00{ margin-left:0pt; margin-right:0pt; }
div.td01{ margin-left:0pt; margin-right:5pt; }
div.td10{ margin-left:5pt; margin-right:0pt; }
div.td11{ margin-left:5pt; margin-right:5pt; }
table[rules] {border-left:solid black 0.4pt; border-right:solid black 0.4pt; }
td.td00{ padding-left:0pt; padding-right:0pt; }
td.td01{ padding-left:0pt; padding-right:5pt; }
td.td10{ padding-left:5pt; padding-right:0pt; }
td.td11{ padding-left:5pt; padding-right:5pt; }
table[rules] {border-left:solid black 0.4pt; border-right:solid black 0.4pt; }
.hline hr, .cline hr{ height : 1px; margin:0px; }
.tabbing-right {text-align:right;}
span.TEX {letter-spacing: -0.125em; }
span.TEX span.E{ position:relative;top:0.5ex;left:-0.0417em;}
a span.TEX span.E {text-decoration: none; }
span.LATEX span.A{ position:relative; top:-0.5ex; left:-0.4em; font-size:85%;}
span.LATEX span.TEX{ position:relative; left: -0.4em; }
div.float img, div.float .caption {text-align:center;}
div.figure img, div.figure .caption {text-align:center;}
.marginpar {width:20%; float:right; text-align:left; margin-left:auto; margin-top:0.5em; font-size:85%; text-decoration:underline;}
.marginpar p{margin-top:0.4em; margin-bottom:0.4em;}
.equation td{text-align:center; vertical-align:middle; }
td.eq-no{ width:5%; }
table.equation { width:100%; } 
div.math-display, div.par-math-display{text-align:center;}
math .texttt { font-family: monospace; }
math .textit { font-style: italic; }
math .textsl { font-style: oblique; }
math .textsf { font-family: sans-serif; }
math .textbf { font-weight: bold; }
.partToc a, .partToc, .likepartToc a, .likepartToc {line-height: 200%; font-weight:bold; font-size:110%;}
.chapterToc a, .chapterToc, .likechapterToc a, .likechapterToc, .appendixToc a, .appendixToc {line-height: 200%; font-weight:bold;}
.index-item, .index-subitem, .index-subsubitem {display:block}
.caption td.id{font-weight: bold; white-space: nowrap; }
table.caption {text-align:center;}
h1.partHead{text-align: center}
p.bibitem { text-indent: -2em; margin-left: 2em; margin-top:0.6em; margin-bottom:0.6em; }
p.bibitem-p { text-indent: 0em; margin-left: 2em; margin-top:0.6em; margin-bottom:0.6em; }
.paragraphHead, .likeparagraphHead { margin-top:2em; font-weight: bold;}
.subparagraphHead, .likesubparagraphHead { font-weight: bold;}
.quote {margin-bottom:0.25em; margin-top:0.25em; margin-left:1em; margin-right:1em; text-align:\jmathustify;}
.verse{white-space:nowrap; margin-left:2em}
div.maketitle {text-align:center;}
h2.titleHead{text-align:center;}
div.maketitle{ margin-bottom: 2em; }
div.author, div.date {text-align:center;}
div.thanks{text-align:left; margin-left:10%; font-size:85%; font-style:italic; }
div.author{white-space: nowrap;}
.quotation {margin-bottom:0.25em; margin-top:0.25em; margin-left:1em; }
h1.partHead{text-align: center}
.sectionToc, .likesectionToc {margin-left:2em;}
.subsectionToc, .likesubsectionToc {margin-left:4em;}
.subsubsectionToc, .likesubsubsectionToc {margin-left:6em;}
.frenchb-nbsp{font-size:75%;}
.frenchb-thinspace{font-size:75%;}
.figure img.graphics {margin-left:10%;}
/* end css.sty */

\title{Developpements en series enti`eres}
\author{}
\date{}

\begin{document}
\maketitle

\textbf{Warning: 
requires JavaScript to process the mathematics on this page.\\ If your
browser supports JavaScript, be sure it is enabled.}

\begin{center}\rule{3in}{0.4pt}\end{center}

{[}
{[}
{[}{]}
{[}

\subsubsection{11.3 Développements en séries entières}

\paragraph{11.3.1 Problème local, problème global}

Définition~11.3.1 (globale). Soit U un voisinage de 0 dans K et f : U \rightarrow~
E (espace vectoriel normé complet). Soit r \textgreater{} 0 tel que
D(0,r) = \z \in
K∣\textbar{}z\textbar{} \textless{}
r\ \subset~ U. On dit que f est développable en série entière
sur le disque D(0,r) s'il existe une série entière
\\sum ~
a\_nz^n de rayon de convergence R ≥ r tel que
\forall~~z \in D(0,r), f(z) =\
\sum  a\_nz^n~.

Définition~11.3.2 (locale). Soit U un voisinage de 0 dans K et f : U \rightarrow~ E
(espace vectoriel normé complet). On dit que f est développable en série
entière au voisinage de 0 s'il existe r \textgreater{} 0 tel que D(0,r)
= \z \in
K∣\textbar{}z\textbar{} \textless{}
r\ \subset~ U et f est développable en série entière dans
D(0,r).

Remarque~11.3.1 Dans le premier cas, on impose donc le disque dans
lequel f doit être développable en série entière, alors que dans le
second cas on laisse tout latitude à r de prendre des valeurs aussi
petites que nécessaires.

Remarque~11.3.2 Il est facile de construire des fonctions non
développables en séries entières au voisinage de 0. Par exemple toute
fonction admettant 0 comme zéro non isolé ne peut être développable en
série entière au voisinage de 0. Par exemple f(x) =
e^-1\diagupx^2  sin~  1
\over x si x\neq~0, f(0) = 0~;
cette fonction, bien que C^\infty~ sur \mathbb{R}~, ne peut être développable
en série entière au voisinage de 0.

\paragraph{11.3.2 Méthodes de développement}

On peut utiliser tout d'abord les résultats sur les opérations
algébriques ou analytiques concernant les séries entières. On obtiendra
de manière évidente les résultats suivants

Théorème~11.3.1 (i) si f et g : U \rightarrow~ E sont développables en séries
entières~sur D(0,r) \subset~ U (resp. au voisinage de 0), alors \alpha~f + \beta~g est
développable en série entière~sur D(0,r) (resp. au voisinage de 0). (ii)
si f et g : U \rightarrow~ K sont développables en séries entières~sur D(0,r) \subset~ U
(resp. au voisinage de 0), alors fg est développable en série
entière~sur D(0,r) (resp. au voisinage de 0). (iii) si f : U \rightarrow~ K est
développable en série entière~au voisinage de 0 et si
f(0)\neq~0, alors  1 \over f
est développable en série entière~au voisinage de 0. (iv) si U \subset~ \mathbb{R}~, f :
U \rightarrow~ E est dérivable et si f' est développable en série entière~sur
D(0,r) \subset~ U (resp. au voisinage de 0), alors f est développable en série
entière~sur D(0,r) (resp. au voisinage de 0). (v) si U \subset~ \mathbb{R}~, f : U \rightarrow~ E
est développable en série entière~sur D(0,r) \subset~ U (resp. au voisinage de
0), alors f est C^\infty~ sur {]} - r,r{[} (resp. au voisinage de
0)et f^(p) est développable en série entière~sur D(0,r)
(resp. au voisinage de 0).

Dans un cadre général, on pourra utiliser le résultat suivant

Théorème~11.3.2 (Mac Laurin). Soit U un voisinage de 0, f : U \rightarrow~ E de
classe C^\infty~ et r \textgreater{} 0 tel que {]} - r,r{[}\subset~ U.
Alors f est développable en série entière dans {]} - r,r{[} si et
seulement si

\forall~~t \in{]} - r,r{[},
lim\_n\rightarrow~+\infty~~\left (f(t)
-\sum \_k=0^n~
f^(k)(0) \over k!
t^k\right ) = 0

Démonstration On sait en effet que f est développable en série entière
dans {]} - r,r{[} si et seulement si

\forall~~t \in{]} - r,r{[}, f(t) =
\sum \_k=0^+\infty~ f^(k)~(0)
\over k! t^k

ce qui est la même chose que l'assertion ci dessus.

Méthode~: pour démontrer que f est développable en série entière dans
{]} - r,r{[}, on peut donc chercher à estimer (par exemple par
utilisation de la formule de Taylor-Lagrange ou de la formule de Taylor
avec reste intégral) l'expression R\_n(t) = f(t)
-\\sum ~
\_k=0^n f^(k)(0) \over k!
t^k et à montrer qu'elle admet la limite 0 quand n tend vers
+ \infty~.

Exemple~11.3.1 Soit f :{]} - r,r{[}\rightarrow~ E une fonction de classe
C^\infty~ telle que \forall~~n \in \mathbb{N}~,
\forall~~t \in{]} - r,r{[},
\\textbar{}f^(n)(t)\\textbar{}
\leq M. On peut alors écrire la formule de Taylor avec reste intégral, et
on obtient, pour t \in{]} - r,r{[}, R\_n(t)
=\int  \_0^t (t-u)^n~
\over n! f^(n+1)(u) du soit encore
\\textbar{}R\_n(t)\\textbar{} \leq
M\int  \_0^t (t-u)^n~
\over n! du = M t^n+1 \over
(n+1)! qui tend vers 0 quand n tend vers + \infty~ (la série converge
d'après la règle de d'Alembert). On en déduit que la fonction f est
développable en série entière dans {]} - r,r{[}. C'est ainsi que quelle
que soit la méthode utilisée pour définir une fonction exponentielle,
qui vérifiera (exp~ )'
= exp~ , on aura (pour un r quelconque)
\forall~n \in \mathbb{N}~, \\forall~~t \in{]} -
r,r{[}, \textbar{}f^(n)(t)\textbar{}\leq e^r, donc
\forall~~t \in{]} - r,r{[},
exp~ (t) =\
\sum  \_n=0^+\infty~ t^n~
\over n! et donc \forall~~t \in \mathbb{R}~,
exp~ (t) =\
\sum  \_n=0^+\infty~ t^n~
\over n! . Une étude similaire pourrait être faite pour
les fonctions trigonométriques (cosinus et sinus). En fait, dans le
paragraphe suivant, nous allons définir directement ces fonctions comme
sommes de séries entières.

Enfin, une dernière technique utilise le théorème de Cauchy-Lipschitz
sur l'unicité des solutions d'équations différentielles à conditions
initiales données. Soit f : U \rightarrow~ K une fonction de classe C^\infty~
vérifiant une équation différentielle y^(n) =
G(t,y,y',\\ldots,y^(n-1)~).
Supposons également que nous connaissions une série entière
\\sum ~
a\_nt^n de rayon de convergence R \textgreater{} 0
telle que sa somme S vérifie la même équation différentielle avec de
plus S(0) =
f(0),\\ldots,S^(n-1)~(0)
= f^(n-1)(0). Si l'équation différentielle relève de l'un des
deux théorèmes de Cauchy Lipschitz, et si en particulier elle est
linéaire à coefficients continus, on aura \forall~~t
\in{]} - R,R{[}\bigcapU, f(t) = S(t) =\
\sum ~
\_k=0^+\infty~a\_kt^k. La démarche est alors
la suivante

\begin{itemize}
\item
  (i) trouver une équation différentielle vérifiée par f
\item
  (ii) trouver une série entière
  \\sum ~
  a\_nt^n vérifiant formellement l'équation
  différentielle en question avec

  a\_0 =
  f(0),\\ldots,a\_n-1~
  = (n - 1)!f^(n-1)(0)
\item
  (iii) montrer que cette série entière a un rayon de convergence R non
  nul
\item
  (iv) utiliser un théorème de Cauchy Lipschitz pour garantir que
  \forall~~t \in{]} - R,R{[}\bigcapU, f(t) = S(t)
  = \\sum ~
  \_k=0^+\infty~a\_kt^k
\end{itemize}

Exemple~11.3.2 Soit m \in \mathbb{R}~ et f : \mathbb{R}~ \rightarrow~ \mathbb{R}~ définie par f(x) = (x +
\sqrt1 + x^2)^m. On a

\begin{align*} f'(x)& =& m\left (1
+ x\over \sqrt1 +
x^2\right )(x + \sqrt1
+ x^2)^m-1\%& \\ &
=& m f(x)\over \sqrt1 +
x^2 \%& \\
\end{align*}

d'où

\begin{align*} f'`(x)& =& -
2mx\over  (1 + x^2)^3\diagup2f(x) +
m\over \sqrt1 +
x^2f'(x)\%& \\ & =& -
x\over 1 + x^2f'(x) +
m^2\over 1 + x^2f(x) \%&
\\ \end{align*}

en rempla\ccant simultanément m
f(x)\over \sqrt1+x^2
par f'(x) et f'(x) par m f(x)\over
\sqrt1+x^2 . Donc f vérifie l'équation
différentielle

(1 + x^2)y'`+ xy' - m^2y = 0

avec les conditions initiales f(0) = 1, f'(0) = m.

Cherchons inversement une fonction S développable en série entière sur
{]} - R,R{[} vérifiant cette équation différentielle. On a alors S(t)
= \\sum ~
\_n=0^+\infty~a\_nt^n, tS'(t)
= \\sum ~
\_n=0^+\infty~na\_nt^n, t^2S''(t)
= \\sum ~
\_n=0^+\infty~n(n - 1)a\_nt^n et S''(t)
= \\sum ~
\_n=0^+\infty~(n + 1)(n + 2)a\_n+2t^n
(vérifications faciles laissées au lecteur, en remarquant que
na\_n = 0 pour n = 0 et n(n - 1)a\_n = 0 pour n = 0 et n
= 1). On a donc

\begin{align*} (1 + t^2)S'`(t) + tS'(t) -
m^2S(t)&& \%& \\ & =&
\sum \_n=0^+\infty~~\left
((n + 1)(n + 2)a\_ n+2 + n(n - 1)a\_n + na\_n -
m^2a\_ n\right )t^n\%&
\\ & =& \\sum
\_n=0^+\infty~\left ((n + 1)(n + 2)a\_
n+2 + (n^2 - m^2)a\_
n\right )t^n \%&
\\ \end{align*}

L'unicité du développement en série entière de la fonction nulle montre
que S est solution de l'équation différentielle si et seulement si~

\forall~n \in \mathbb{N}~, (n + 1)(n + 2)a\_n+2~ +
(n^2 - m^2)a\_ n

On veut de plus que a\_0 = 1 et a\_1 = m. Ces trois
relations définissent parfaitement les deux suites
(a\_2n)\_n\in\mathbb{N}~ et (a\_2n+1)\_n\in\mathbb{N}~ (l'une
des deux, celle correspondant à la parité de m, étant nulle à partir du
rang \textbar{}m\textbar{} + 2 si m \in ℤ). Des deux séries entières
\\sum ~
a\_2nt^2n et
\\sum ~
a\_2n+1t^2n+1, l'une des deux est de rayon de
convergence 1, l'autre de rayon de convergence soit 1 soit + \infty~. Donc la
série entière \\sum ~
a\_nt^n est de rayon de convergence 1
= inf~(1,+\infty~).

Sur {]} - 1,1{[}, on peut donc définir S(t) =\
\sum ~
\_n=0^+\infty~a\_nt^n. Cette fonction vérifie
la même équation différentielle que f avec les mêmes conditions
initiales. Le théorème de Cauchy-Lipschitz pour les équations
différentielles linéaire à coefficients continus garantit que f et S
coïncident sur l'intersection de leurs intervalles de définition,
c'est-à-dire sur {]} - 1,1{[}. Donc f est développable en série entière
sur {]} - 1,1{[}.

\paragraph{11.3.3 Fonction exponentielle. Fonctions trigonométriques}

La règle de d'Alembert montre que la série entière
\\sum  \_n≥0~
z^n \over n! a un rayon de convergence
infini. Ceci \jmathustifie l'introduction de la définition suivante

Définition~11.3.3 Pour z \in \mathbb{C}, on pose exp~ (z)
= \\sum ~
\_n=0^+\infty~ z^n \over n! .

Proposition~11.3.3 (i) z\mapsto~exp(z) est un
morphisme de groupes de (\mathbb{C},+) dans (\mathbb{C}^∗,.), autrement dit
exp (0) = 1, \exp~
(z\_1 + z\_2) = exp~
z\_1 exp z\_2~,
exp (z)\mathrel\neq~~0 et
(exp (z))^-1~
= exp (-z). (ii) \\forall~~z
\in \mathbb{C}, exp (\overlinez~) =
\overlineexp z~ (iii) pour
tout z \in \mathbb{C}, l'application
t\mapsto~exp~ (tz) est
C^\infty~ de \mathbb{R}~ dans \mathbb{C} et  d^n \over
dt^n (exp~ (tz)) =
z^n exp~ (tz).

Démonstration (i) Soit z\_1,z\_2 \in \mathbb{C}. On pose
a\_n = z\_1^n \over n! et
b\_n = z\_2^n \over n! . Ces
séries sont absolument convergentes. On peut donc faire le produit de
Cauchy de ces deux séries et on a alors c\_n
= \\sum ~
\_k=0^n 1 \over k!(n-k)!
z\_1^kz\_ 2^n-k = 1 \over
n! (z\_1 + z\_2)^n d'après la formule du
binôme. On a donc

\sum \_n=0^+\infty~ (z\_1~ +
z\_2)^n \over n! =
\left (\\sum
\_n=0^+\infty~ z\_1^n \over
n! \right )\left
(\sum \_n=0^+\infty~~
z\_2^n \over n! \right )

(ii) Il suffit de faire tendre N vers + \infty~ dans la formule évidente
\\sum ~
\_n=0^N \overlinez^n
\over n! =
\overline\\\sum
 \_n=0^N z^n \over n! 

(iii) On a exp~ (tz)
= \\sum ~
\_n=0^+\infty~ z^n \over n!
t^n qui est une série entière en t de rayon de convergence
infini. On en déduit que sa somme est de classe C^\infty~ sur \mathbb{R}~ et
que

\begin{align*} d^n \over
dt^n (exp~ (tz))& =&
\sum \_k=n^+\infty~ z^k~
\over k! k(k - 1)\ldots~(k -
n + 1)t^k-n\%& \\ & =&
\sum \_k=n^+\infty~ z^k~
\over (k - n)! t^k-n = z^n exp
(tz) \%& \\
\end{align*}

après le changement de k - n en k.

Définition~11.3.4 On pose e = exp~ 1~; on a
bien entendu, \forall~~n \in ℤ,
exp (n) = e^n~. On en déduit
facilement que \forall~~r \in ℚ,
exp (r) = e^r~, ce qui \jmathustifie
ensuite la notation exp (z) = e^z~
pour z \in \mathbb{C}.

Définition~11.3.5 Pour z \in \mathbb{C}, on pose cos~ z
= e^iz+e^-iz \over 2 ,
sin z = e^iz-e^-iz~
\over 2i ,
\mathrmch~ z =
e^z+e^-z \over 2 ,
\mathrmsh~ z =
e^z-e^-z \over 2 .

Remarque~11.3.3 Il est clair que les fonctions
cos~ et
\mathrmch~ sont paires et
que les fonctions sin~ et
\mathrmsh~ sont impaires.

Proposition~11.3.4 (i) \forall~~z \in \mathbb{C},
\mathrmch~ iz
= cos~ z,
\mathrmsh~ iz =
isin z, \cos~
^2z + sin ^2~z = 1,
\mathrmch ^2~z
-\mathrmsh ^2~z =
1 (ii) \forall~~a,b \in \mathbb{C}

\begin{align*} cos~ (a +
b)& =& cos a\cos~ b
- sin a\sin~ b\%&
\\ sin~ (a + b)&
=& sin a\cos~ b
+ cos a\sin~ b\%&
\\
\mathrmch~ (a + b)& =&
\mathrmch~
a\mathrmch~ b
+ \mathrmsh~
a\mathrmsh~ b \%&
\\
\mathrmsh~ (a + b)& =&
\mathrmsh~
a\mathrmch~ b
+ \mathrmch~
a\mathrmsh~ b \%&
\\ \end{align*}

Démonstration Par le calcul à partir de la définition.

Remarque~11.3.4 On déduit de ces formules de manière évidente toutes les
formules usuelles de la trigonométrie circulaire ou hyperbolique dont
nous ne citerons que celles qu'il est absolument indispensable de
connaître par coeur~:

\forall~~a,b \in \mathbb{C}

\begin{align*} cos~ (a +
b)& =& cos a\cos~ b
- sin a\sin~ b \%&
\\ cos~ (a - b)&
=& cos a\cos~ b
+ sin a\sin~ b \%&
\\ sin~ (a + b)&
=& sin a\cos~ b
+ cos a\sin~ b \%&
\\ sin~ (a - b)&
=& sin a\cos~ b
- cos a\sin~ b \%&
\\ cos~ (2a)&
=& cos ^2~a
- sin ^2~a =
2cos ^2~a - 1 = 1 -
2sin ^2~a\%&
\\ sin~ (2a)&
=& 2sin a\cos~ a \%&
\\ cos~
^2a& =& 1 + cos~ 2a
\over 2 ,\quad
sin ^2~a = 1
- cos 2a \over 2~ \%&
\\ cos~ p
+ cos q& =& 2\cos~
 p - q \over 2 cos~  p + q
\over 2 \%& \\
cos p -\ cos~ q& =&
-2sin  p - q \over 2~
sin  p + q \over 2~ \%&
\\ sin~ p
+ sin q& =& 2\cos~
 p - q \over 2 sin~  p + q
\over 2 \%& \\
sin p -\ sin~ q& =&
2sin  p - q \over 2~
cos  p + q \over 2~ \%&
\\ \end{align*}

\forall~~a,b \in \mathbb{C}

\begin{align*}
\mathrmch~ (a + b)& =&
\mathrmch~
a\mathrmch~ b
+ \mathrmsh~
a\mathrmsh~ b \%&
\\
\mathrmch~ (a - b)& =&
\mathrmch~
a\mathrmch~ b
-\mathrmsh~
a\mathrmsh~ b \%&
\\
\mathrmsh~ (a + b)& =&
\mathrmsh~
a\mathrmch~ b
+ \mathrmch~
a\mathrmsh~ b \%&
\\
\mathrmsh~ (a - b)& =&
\mathrmsh~
a\mathrmch~ b
-\mathrmch~
a\mathrmsh~ b \%&
\\
\mathrmch~ (2a)& =&
\mathrmch ^2~a
+ \mathrmsh ^2~a
= 2\mathrmch ^2~a
- 1 = 1 + 2\mathrmsh~
^2a\%& \\
\mathrmsh~ (2a)& =&
2\mathrmsh~
a\mathrmch~ a \%&
\\ \end{align*}

Etude sur \mathbb{R}~ des fonctions exponentielle et logarithme

Théorème~11.3.5 L'application
t\mapsto~e^t est un C^\infty~
difféomorphisme de \mathbb{R}~ sur {]}0,+\infty~{[}. Le difféomorphisme réciproque est
appelé le logarithme (naturel ou népérien). Pour t \textgreater{} 0 et \alpha~
\in \mathbb{R}~, on pose t^\alpha~ = e^\alpha~ log~
t (la notation est cohérente pour \alpha~ \in ℚ).On a alors (entre autres
propriétés)

\begin{itemize}
\item
  (i) \forall~t\_1,t\_2~ \in{]}0,+\infty~{[},
  log (t\_1t\_2~)
  = log t\_1~ +\
  log t\_2
\item
  (ii)  d \over dt (log~ t)
  = 1 \over t
\item
  (iii) On a également

  \begin{align*} \forall~~\alpha~
  \textgreater{} 0, &
  lim\_t\rightarrow~+\infty~t^-\alpha~e^t~
  = +\infty~,\quad
  lim\_t\rightarrow~+\infty~t^-\alpha~~\
  log t = 0& \%& \\ &
  lim\_t\rightarrow~+\infty~t^\alpha~e^-t~
  = 0,\quad
  lim\_t\rightarrow~0t^\alpha~~\
  log t = 0 & \%& \\
  \end{align*}
\end{itemize}

Démonstration On a clairement exp~ t
\textgreater{} 0 pour t ≥ 0~; pour t \leq 0, on a
exp t = (\exp~
\textbar{}t\textbar{})^-1 \textgreater{} 0~; on en déduit que
 d \over dt (exp~ t)
= exp~ t \textgreater{} 0 pour tout t \in \mathbb{R}~ donc
t\mapsto~exp~ t est un
C^\infty~ difféomorphisme croissant de \mathbb{R}~ sur son image. Mais, pour
t ≥ 0, on a exp~ t =\
\sum  \_n=0^+\infty~ t^n~
\over n! ≥ 1 + t donc
lim\_t\rightarrow~+\infty~\exp~
t = +\infty~. Comme pour t \textless{} 0, exp~ t =
(exp \textbar{}t\textbar{})^-1~
\textgreater{} 0, on a
lim\_t\rightarrow~-\infty~\exp~
t = 0. Donc exp~ (\mathbb{R}~) ={]}0,+\infty~{[}. La cohérence
de la définition de t^\alpha~ =
e^\alpha~ log t~ est laissée aux soins du
lecteur ainsi que les propriétés évidentes de l'application
t\mapsto~t^\alpha~.

(i) Comme exp~ est un isomorphisme de (\mathbb{R}~,+) sur
(\mathbb{R}~^∗⋅,.), log~ est également un
isomorphisme de groupes de (\mathbb{R}~^∗⋅,.) sur (\mathbb{R}~,+)

(ii) Le théorème sur la dérivation des fonctions réciproques montre que
 d \over dt (log~ t) = 1
\over exp~
'(log t)~ = 1 \over
exp \ log t~ = 1
\over t .

(iii) pour t \textgreater{} 0, on a exp~ t
= \\sum ~
\_n=0^+\infty~ t^n \over n! ≥
t^N \over N! . Soit \alpha~ \textgreater{} 0, on a
alors pour N \textgreater{} \alpha~, t^-\alpha~\
exp t \textgreater{} t^N-\alpha~ \over N! , ce
qui montre que
lim\_t\rightarrow~+\infty~t^-\alpha~e^t~
= +\infty~. En passant à l'inverse, on trouve
lim\_t\rightarrow~\infty~t^\alpha~e^-t~
= 0. Comme
lim\_x\rightarrow~+\infty~\log~
x = +\infty~, le théorème de composition des limites donne
lim\_x\rightarrow~+\infty~(\log~
x)^-1\diagup\alpha~x = 0, soit encore
lim\_x\rightarrow~+\infty~x^-\alpha~~\
log x = 0~; en changeant x en 1\diagupx, on obtient alors
lim\_x\rightarrow~0x^\alpha~~\
log x = 0.

Etude sur \mathbb{R}~ des fonctions cosinus et sinus

On a pour t \in \mathbb{R}~, \overlinee^it
= exp (\overlineit~) =
e^-it. On a donc cos~ t
=\
\mathrmRe(e^it) et
sin~ t =\
\mathrmIm(e^it). On en déduit que
cos~ t =\
\sum  \_n=0^+\infty~(-1)^n~
t^2n \over (2n)! et
sin~ t =\
\sum  \_n=0^+\infty~(-1)^n~
t^2n+1 \over (2n+1)! . Ces formules (ou
celles de définition) montrent immédiatement que
cos ' = -\sin~ et
sin ' =\ cos~ . On a,
pour t \in{]}0,2{]},  sin~ t
\over t =\
\sum  \_n=0^+\infty~(-1)^n~
t^2n \over (2n+1)! qui est la somme d'une
série alternée à partir de n = 1 car

  t^2n+2 \over (2n+3)!
\over  t^2n \over (2n+1)!
 = t^2 \over (2n + 1)(2n + 3)
\textless{} 1

pour t \in{]}0,2{]} et n ≥ 1. On en déduit (théorème sur l'encadrement des
sommes d'une série alternée) que \forall~~t
\in{]}0,2{]}, sin t \over t~ ≥
1 - t^2 \over 6 \textgreater{} 0, donc
sin~ t \textgreater{} 0. Comme
cos ' = -\sin~ , la
fonction cosinus est strictement décroissante sur {[}0,2{]}. On a
cos~ 0 = 1 \textgreater{} 0 et
cos~ 2 =\
\sum  \_n=0^+\infty~(-1)^n~
2^2n \over (2n)! qui est la somme d'une
série alternée à partir de n = 1 car

  2^2n+2 \over (2n+2)!
\over  2^2n \over (2n)! 
= 4 \over (2n + 1)(2n + 2) \textless{} 1

pour n ≥ 1. On en déduit (théorème sur l'encadrement des sommes d'une
série alternée) que

cos 2 \leq 1 - 4 \over 2!~ +
16 \over 4! \textless{} 0

Le théorème des valeurs intermédiaires assure alors que la fonction
cosinus s'annule une et une seule fois sur {]}0,2{[} en un point \alpha~. On
posera \pi~ = 2\alpha~. On a donc cos~  \pi~
\over 2 = 0 et cos~ t
\textgreater{} 0 pour t \in {[}0, \pi~ \over 2 {[}. Comme
sin ' =\ cos~ ,
sin~ est strictement croissante sur {[}0, \pi~
\over 2 {]}, comme sin~ 0 = 0
et cos ^2~ \pi~ \over
2 + sin ^2~ \pi~
\over 2 = 1, on a donc sin~ 
\pi~ \over 2 = 1. Les formules d'addition donnent alors
cos (x + \pi~ \over 2~ ) =
-sin x, \sin~ (x + \pi~
\over 2 ) = cos~ x et on a
donc les variations suivantes

\begin{center}\rule{3in}{0.4pt}\end{center}

\begin{center}\rule{3in}{0.4pt}\end{center}

\begin{center}\rule{3in}{0.4pt}\end{center}

\begin{center}\rule{3in}{0.4pt}\end{center}

\begin{center}\rule{3in}{0.4pt}\end{center}

\begin{center}\rule{3in}{0.4pt}\end{center}

\begin{center}\rule{3in}{0.4pt}\end{center}

\begin{center}\rule{3in}{0.4pt}\end{center}

\begin{center}\rule{3in}{0.4pt}\end{center}

\begin{center}\rule{3in}{0.4pt}\end{center}

t

0

 \pi~ \over 2

\pi~

 3\pi~ \over 2

2\pi~

\begin{center}\rule{3in}{0.4pt}\end{center}

\begin{center}\rule{3in}{0.4pt}\end{center}

\begin{center}\rule{3in}{0.4pt}\end{center}

\begin{center}\rule{3in}{0.4pt}\end{center}

\begin{center}\rule{3in}{0.4pt}\end{center}

\begin{center}\rule{3in}{0.4pt}\end{center}

\begin{center}\rule{3in}{0.4pt}\end{center}

\begin{center}\rule{3in}{0.4pt}\end{center}

\begin{center}\rule{3in}{0.4pt}\end{center}

\begin{center}\rule{3in}{0.4pt}\end{center}

 cos~ t

1

\searrow

0

\searrow

- 1

\nearrow

0

\nearrow

1

\begin{center}\rule{3in}{0.4pt}\end{center}

\begin{center}\rule{3in}{0.4pt}\end{center}

\begin{center}\rule{3in}{0.4pt}\end{center}

\begin{center}\rule{3in}{0.4pt}\end{center}

\begin{center}\rule{3in}{0.4pt}\end{center}

\begin{center}\rule{3in}{0.4pt}\end{center}

\begin{center}\rule{3in}{0.4pt}\end{center}

\begin{center}\rule{3in}{0.4pt}\end{center}

\begin{center}\rule{3in}{0.4pt}\end{center}

\begin{center}\rule{3in}{0.4pt}\end{center}

sin~ t

0

\nearrow

1

\searrow

0

\searrow

- 1

\nearrow

0

\begin{center}\rule{3in}{0.4pt}\end{center}

\begin{center}\rule{3in}{0.4pt}\end{center}

\begin{center}\rule{3in}{0.4pt}\end{center}

\begin{center}\rule{3in}{0.4pt}\end{center}

\begin{center}\rule{3in}{0.4pt}\end{center}

\begin{center}\rule{3in}{0.4pt}\end{center}

\begin{center}\rule{3in}{0.4pt}\end{center}

\begin{center}\rule{3in}{0.4pt}\end{center}

\begin{center}\rule{3in}{0.4pt}\end{center}

\begin{center}\rule{3in}{0.4pt}\end{center}

On a cos~ (x + 2\pi~) =\
cos x et sin~ (x + 2\pi~)
= sin~ x (de nouveau par les formules
d'addition). L'ensemble des périodes des fonctions
cos et \sin~ (qui est
évidemment le même puisque chacune est au signe près la dérivée de
l'autre) est un sous-groupe fermé de \mathbb{R}~ donc de la forme aℤ avec donc a
= 2\pi~ \over k , où k \in ℤ~; le tableau de variation
montre clairement que k ne peut pas être strictement supérieur à 1~;
donc les deux fonctions sont périodiques de plus petite période 2\pi~.

\paragraph{11.3.4 Nombres complexes de module 1}

Avec l'aimable autorisation de Hervé Pépin.

Théorème~11.3.6 L'application \phi :
t\mapsto~e^it est un morphisme sur\jmathectif
de groupes de (\mathbb{R}~,+) sur (U,.) (où U désigne l'ensemble des nombres
complexes de module 1) dont le noyau est 2\pi~ℤ. Par factorisation
canonique, il définit un isomorphisme \overline\phi du
groupe quotient (\mathbb{R}~\diagup2\pi~ℤ) sur le groupe (U,.).

Démonstration Il est clair que \phi(t\_1 + t\_2) =
\phi(t\_1)\phi(t\_2), donc \phi est un morphisme de groupes. Soit
z = a + ib un nombre complexe de module 1. On a a \in {[}-1,1{]}, donc il
existe t \in \mathbb{R}~ tel que cos~ t = a. Mais alors
sin t = 1 - a^2 = b^2~ et
donc b vaut soit sin~ t soit
sin~ (-t). Dans les deux cas on a trouvé un x \in
\mathbb{R}~ tel que z = e^ix ce qui montre que \phi est sur\jmathectif. On a
alors x \in\mathrmKer~\phi
\Leftrightarrow \phi(x) = 1 \mathrel\Leftrightarrow
cos x = 1 \mathrel\Leftrightarrow~ x \in 2\pi~ℤ
ce qui assure que
\mathrmKer~\phi = 2\pi~ℤ.

Définition~11.3.6 Soit \zeta \in \mathbb{C}, z\neq~0. On pose
alors Arg~ z =
\overline\phi^-1( z \over
\textbar{}z\textbar{} ) \in \mathbb{R}~\diagup2\pi~ℤ (argument du nombre complexe z).

Remarque~11.3.5 Si Arg~ z = \theta + 2\pi~ℤ, on a par
définition z = \textbar{}z\textbar{}e^i\theta. On a bien entendu,
Arg(z\_1z\_2~)
= Arg z\_1~ +\
Arg z\_2. On peut prendre un représentant
arg z dans \mathbb{R}~ de \Arg~
z (par exemple classiquement dans {]} - \pi~,\pi~{]}), mais alors en général
arg(z\_1z\_2)\neq~\arg~
z\_1 + arg z\_2~.

Définition~11.3.7 Soit X un espace métrique et f : X \rightarrow~ U une application
continue. On dit que f est relevable s'il existe \phi : X \rightarrow~ \mathbb{R}~ continue
telle que f = e^i\phi.

Proposition~11.3.7 Soit X un espace métrique et f : X \rightarrow~ U une
application continue non sur\jmathective. Alors f est relevable.

Démonstration Soit \alpha~ \in \mathbb{R}~ tel que
e^i\alpha~∉f(X). L'application \omega :{]}\alpha~,\alpha~
+ 2\pi~{[}\rightarrow~ U \diagdown\e^i\alpha~\,
t\mapsto~e^it est un homéomorphisme. Donc
\phi = \omega^-1 \cdot f convient.

Corollaire~11.3.8 Soit X un espace métrique et f,g : X \rightarrow~ U telles que
\forall~~x \in X, \textbar{}f(x) - g(x)\textbar{}
\textless{} 2. Alors  f \over g est relevable.

Démonstration On a \forall~~x \in X, f(x)
\over g(x) \neq~ - 1 et donc 
f \over g n'est pas sur\jmathective.

Nous allons maintenant introduire un concept important en topologie,
l'homotopie.

Définition~11.3.8 Soit X et E deux espaces métriques, f,g : X \rightarrow~ E. On
dit que f et g sont homotopes s'il existe F : X \times {[}0,1{]} \rightarrow~ E continue
telle que \forall~~x \in X, f(x) = F(x,0) et g(x) =
F(x,1).

Théorème~11.3.9 Soit X un espace métrique compact et f,g : X \rightarrow~ U
homotopes. Alors  f \over g est relevable et donc (f
relevable) \Leftrightarrow (g relevable).

Démonstration Soit F comme ci dessus. Alors F est continue sur X \times
{[}0,1{]} compact, donc uniformément continue. Soit donc \eta
\textgreater{} 0 tel que \textbar{}t - t'\textbar{} \textless{} \eta
\rigtharrow~\forall~~x \in X,\textbar{}F(x,t) - F(x,t')\textbar{}
\textless{} 2 et soit t\_0 =
1,\\ldots,t\_n~
= 1 une subdivision de {[}0,1{]} de pas plus petit que \eta. Alors, si
f\_i(x) = F(x,t\_i), on a \forall~~x \in
X, \textbar{}f\_i+1(x) - f\_i(x)\textbar{} \textless{} 2
donc  f\_i \over f\_i+1 est
relevable. Mais alors  f \over g = f\_0
\over f\_n =\
∏  \_i=0^n-1 f\_i~
\over f\_i+1 est relevable.

Corollaire~11.3.10 Soit f : {[}a,b{]} \rightarrow~ U une application continue.
Alors f est relevable.

Démonstration f est évidemment homotope à l'application constante
x\mapsto~f(a) qui est relevable (car non sur\jmathective)
par F(x,t) = f(a + t(x - a)).

Corollaire~11.3.11 Soit I un intervalle de \mathbb{R}~ et f : I \rightarrow~ U continue.
Alors f est relevable.

\paragraph{11.3.5 Fonctions classiques}

Fonctions d'une variable complexe

Lemme~11.3.12 La fonction z\mapsto~ 1
\over 1-z est développable en série entière dans le
disque D(0,1) (R = 1).

Démonstration On a  1 \over 1-z = 1 + z +
⋯ + z^n + z^n+1
\over 1-z ce dernier terme tendant vers 0 quand n tend
vers + \infty~ si \textbar{}z\textbar{} \textless{} 1. On en déduit que pour
\textbar{}z\textbar{} \textless{} 1,  1 \over 1-z
= \\sum ~
\_n=0^+\infty~z^n avec un rayon de convergence
évidemment égal à 1 par la règle de d'Alembert.

Proposition~11.3.13 Pour \textbar{}z\textbar{} \textless{} 1, on a  1
\over (1-z)^k+1 =\
\sum ~
\_n=0^+\infty~C\_n+k^kz^n avec un rayon
de convergence égal à 1.

Démonstration Soit z\neq~0~; le lemme précédent
montre que pour t \in{]} - 1 \over
\textbar{}z\textbar{} , 1 \over
\textbar{}z\textbar{} {[}, on a  1 \over 1-tz
= \\sum ~
\_n=0^+\infty~z^nt^n. Dérivons k fois par
rapport à t cette série entière en t. On obtient donc

 z^kk! \over (1 - tz)^k+1 =
\sum \_n=k^+\infty~z^n~ n!
\over (n - k)! t^n-k =
\sum \_n=0^+\infty~z^n+k~ (n +
k)! \over n! t^n

après le changement de n en n + k. Si \textbar{}z\textbar{} \textless{}
1, en divisant les deux membres par z^kk! et en faisant t = 1
\in{]} - 1 \over \textbar{}z\textbar{} , 1
\over \textbar{}z\textbar{} {[}, on obtient la formule
désirée. Il est clair que le rayon de convergence de la série obtenue
est 1 (règle de d'Alembert).

Corollaire~11.3.14 Soit R(z) = P(z) \over Q(z) une
fraction rationnelle à coefficients complexes n'admettant pas 0 pour
pôle et soit \rho =\
min\\textbar{}\alpha~\textbar{}∣\alpha~\text
pôle de R\. Alors R est développable en série entière
dans le disque D(0,\rho).

Démonstration Par décomposition en éléments simples sur le corps des
nombres complexes, il suffit de montrer que si \alpha~ est un pôle de R(z), un
élément simple de la forme  1 \over
(z-\alpha~)^k est développable en série entière dans le disque
D(0,\rho). Mais

 1 \over (z - \alpha~)^k = (-1)^k
\over \alpha~^k  1 \over (1 -
z \over \alpha~ )^k = (-1)^k
\over \alpha~^k  \\sum
\_n=0^+\infty~C\_ n+k-1^k-1 z^n
\over \alpha~^n

avec un rayon de convergence égal à \textbar{}\alpha~\textbar{}≥ \rho.

Remarque~11.3.6 On montre facilement que le rayon de convergence est
exactement égal à \rho, puisque la somme de la série entière ne peut
admettre un prolongement continu en un pôle de la fraction rationnelle.

Fonctions d'une variable réelle

Lemme~11.3.15 Soit \alpha~ \in \mathbb{R}~. La fonction t\mapsto~(1
+ t)^\alpha~ est développable en série entière dans {]} - 1,1{[} si
\alpha~∉\mathbb{N}~, dans \mathbb{R}~ si \alpha~ \in \mathbb{N}~.

Démonstration Le résultat est évident si \alpha~ \in \mathbb{N}~ puisqu'alors la fonction
est polynomiale. Nous supposerons donc \alpha~∉\mathbb{N}~.
Considérons la série entière 1 +\
\sum  \_n≥1~
\alpha~(\alpha~-1)\\ldots~(\alpha~-n+1)
\over n! t^n. On a 
\alpha~(\alpha~-1)\\ldots~(\alpha~-n+1)
\over n! \neq~0 et

lim~ 
\alpha~(\alpha~-1)\\ldots~(\alpha~-n)
\over (n+1)! \over 
\alpha~(\alpha~-1)\\ldots~(\alpha~-n+1)
\over n!  = lim~ \alpha~ - n
\over n + 1 = -1

Donc son rayon de convergence est 1. Posons donc, pour t \in{]} - 1,1{[},
S(t) = 1 + \\sum ~
\_n=1^+\infty~
\alpha~(\alpha~-1)\\ldots~(\alpha~-n+1)
\over n! t^n. On a

\begin{align*} S'(t)& =&
\sum \_n=1^+\infty~~ \alpha~(\alpha~ -
1)\ldots~(\alpha~ - n + 1) \over (n
- 1)! t^n-1\%& \\ & =&
\sum \_n=0^+\infty~~ \alpha~(\alpha~ -
1)\ldots(\alpha~ - n) \over n!~
t^n \%& \\
\end{align*}

après un changement d'indice. On en déduit que

\begin{align*} (1 + t)S'(t) = S'(t) + tS'(t)&&
\%& \\ & =& \alpha~ +
\sum \_n=1^+\infty~~\left
( \alpha~(\alpha~ - 1)\ldots~(\alpha~ - n)
\over n! + \alpha~(\alpha~ -
1)\ldots~(\alpha~ - n + 1) \over (n
- 1)! \right )t^n\%&
\\ \end{align*}

en utilisant d'abord la seconde expression de S'(t) puis la première.
Mais

\begin{align*} \alpha~(\alpha~ -
1)\\ldots~(\alpha~ - n)
\over n! + \alpha~(\alpha~ -
1)\\ldots~(\alpha~ - n +
1) \over (n - 1)! && \%&
\\ & =& \alpha~(\alpha~ -
1)\\ldots~(\alpha~ - n +
1) \over (n - 1)! \left ( \alpha~ - n
\over n + 1\right )\%&
\\ & =& \alpha~(\alpha~ -
1)\\ldots~(\alpha~ - n +
1) \over (n - 1)!  \alpha~ \over n \%&
\\ \end{align*}

On en déduit que

(1 + t)S'(t) = \alpha~ + \alpha~\\sum
\_n=1^+\infty~ \alpha~(\alpha~ - 1)\ldots~(\alpha~ -
n + 1) \over n! t^n = \alpha~S(t)

Alors

 d \over dt \left ( S(t)
\over (1 + t)^\alpha~ \right ) =
(1 + t)S'(t) - \alpha~S(t) \over (1 + t)^\alpha~+1 = 0

donc la fonction est constante sur {]} - 1,1{[}. Comme elle vaut 1 au
point 0, elle est constamment égale à 1 et donc
\forall~t \in{]} - 1,1{[}, (1 + t)^\alpha~~ = S(t)
= 1 + \\sum ~
\_n=1^+\infty~
\alpha~(\alpha~-1)\\ldots~(\alpha~-n+1)
\over n! t^n.

On obtient ainsi facilement des développements en série entière de (1
+ t)^-1, (1 + t^2)^-1, (1 -
t^2)^-1, (1 - t^2)^-1\diagup2, (1
+ t^2)^-1\diagup2, puis par intégration des développements
de log~ (1 + t),
\mathrmarctg~ t,
arg~
\mathrmth~ t,
arcsin t et \arg~
\mathrmsh~ t. En
récapitulant, on obtient la table suivante de développements en série
entière

\begin{align*} e^t& =&
\sum \_n=0^+\infty~ t^n~
\over n! ,\quad R = +\infty~ \%&
\\ cos~ t& =&
\sum \_n=0^+\infty~(-1)^n~
t^2n \over (2n)! ,\quad R =
+\infty~ \%& \\ sin~
t& =& \\sum
\_n=0^+\infty~(-1)^n t^2n+1
\over (2n + 1)! ,\quad R = +\infty~ \%&
\\
\mathrmch~ t& =&
\sum \_n=0^+\infty~ t^2n~
\over (2n)! ,\quad R = +\infty~ \%&
\\
\mathrmsh~ t& =&
\sum \_n=0^+\infty~ t^2n+1~
\over (2n + 1)! ,\quad R = +\infty~ \%&
\\ (1 + t)^\alpha~& =& 1 +
\sum \_n=1^+\infty~~ \alpha~(\alpha~ -
1)\ldots~(\alpha~ - n + 1) \over
n! t^n,\quad R = 1\text ou
 + \infty~\%& \\  1 \over 1
+ t & =& \\sum
\_n=0^+\infty~(-1)^nt^n,\quad
R = 1 \%& \\  1 \over 1
- t & =& \\sum
\_n=0^+\infty~t^n,\quad R = 1 \%&
\\ log~ (1 + t)&
=& \\sum
\_n=1^+\infty~(-1)^n-1 t^n
\over n ,\quad R = 1 \%&
\\ log~ (1 - t)&
=& -\sum \_n=1^+\infty~~
t^n \over n ,\quad R = 1 \%&
\\
\mathrmarctg~ t& =&
\sum \_n=0^+\infty~(-1)^n~
t^2n+1 \over 2n + 1 ,\quad R
= 1 \%& \\ arg~
\mathrmth~ t& =&
\sum \_n=0^+\infty~ t^2n+1~
\over 2n + 1 ,\quad R = 1 \%&
\\ arcsin~ t&
=& t + \sum \_n=1^+\infty~~
1.3\ldots~(2n - 1) \over
2.4\ldots(2n)  t^2n+1~
\over 2n + 1 ,\quad R = 1 \%&
\\ arg~
\mathrmsh~ t& =& t +
\sum \_n=1^+\infty~(-1)^n~
1.3\ldots~(2n - 1) \over
2.4\ldots(2n)  t^2n+1~
\over 2n + 1 ,\quad R = 1 \%&
\\ \end{align*}

\paragraph{11.3.6 Méthodes de sommation}

Le problème est ici, étant donné une série entière
\\sum ~
a\_nt^n, de reconnaître une fonction exprimable avec
des fonctions classiques dont c'est le développement en série entière.
Bien entendu, on utilise toutes les méthodes inverses des méthodes ci
dessus (combinaisons linéaires, produits, changement de variable,
dérivation, intégration, équations différentielles). Nous décrirons ici
seulement deux cas classiques qui se traitent avec des méthodes
spécifiques.

Séries entières \\sum ~
P(n)t^n, où P est un polynôme Le rayon de convergence est
évidemment égal à 1. On a vu que

 1 \over (1 - t)^k+1 =
\sum \_n=0^+\infty~C~\_
n+k^kt^n = \\sum
\_n=0^+\infty~ (n + k)(n + k -
1)\ldots(n + 1) \over k!~
t^n

Posons donc P\_0(X) = 1 et

P\_k(X) = (X + k)(X + k -
1)\\ldots~(X + 1)
\over k!

pour k ≥ 1 et soit d = deg~ P. La famille
(P\_0,P\_1,\\ldots,P\_d~)
est une base de \mathbb{R}~\_d{[}X{]} (espace vectoriel des polynômes de
degré inférieur ou égal à d) car elle est échelonnée en degré. On peut
donc écrire P(X) = \lambda~\_0P\_0(X) +
\lambda~\_1P\_1(X) +
\\ldots~ +
\lambda~\_dP\_d(X) et alors, pour t \in{]} - 1,1{[},

\sum \_n=0^+\infty~P(n)t^n~
= \lambda~\_0 \over 1 - t + \lambda~\_1
\over (1 - t)^2 +
\ldots + \lambda~\_d~ \over
(1 - t)^d+1

Remarque~11.3.7 Cette méthode s'étend à des séries entières du type
\\sum ~  P(n)
\over n+k t^n où k est un entier. Il suffit
en effet de multiplier par t^k et de dériver pour tomber sur
une série du type précédent. On peut ensuite, par décomposition en
éléments simples sommer les séries du type
\\sum ~  P(n)
\over
(n+k\_1)\\ldots(n+k\_m)~
t^n où
k\_1,\\ldots,k\_m~
sont des entiers distincts.

Séries entières \\sum ~
P(n) t^n \over n! , où P est un polynôme
Le rayon de convergence est évidemment égal à + \infty~. On remarque ici que
\\sum ~
\_n=0^+\infty~n(n -
1)\\ldots~(n - k +
1) t^n \over n!
= \\sum ~
\_n=k^+\infty~ t^n \over (n-k)! =
t^ke^t. Posons donc P\_0(X) = 1 et
P\_k(X) = X(X -
1)\\ldots~(X - k +
1) pour k ≥ 1 et soit d = deg~ P. La famille
(P\_0,P\_1,\\ldots,P\_d~)
est une base de \mathbb{R}~\_d{[}X{]} (espace vectoriel des polynômes de
degré inférieur ou égal à d) car elle est échelonnée en degré. On peut
donc écrire P(X) = \lambda~\_0P\_0(X) +
\lambda~\_1P\_1(X) +
\\ldots~ +
\lambda~\_dP\_d(X) et alors, pour t \in \mathbb{R}~, on a

\sum \_n=0^+\infty~P(n) t^n~
\over n! = (\lambda~\_0 + \lambda~\_1t +
\ldots~ +
\lambda~\_dt^d)e^t

{[}
{[}
{[}
{[}

\end{document}

\documentclass[]{article}
\usepackage[T1]{fontenc}
\usepackage{lmodern}
\usepackage{amssymb,amsmath}
\usepackage{ifxetex,ifluatex}
\usepackage{fixltx2e} % provides \textsubscript
% use upquote if available, for straight quotes in verbatim environments
\IfFileExists{upquote.sty}{\usepackage{upquote}}{}
\ifnum 0\ifxetex 1\fi\ifluatex 1\fi=0 % if pdftex
  \usepackage[utf8]{inputenc}
\else % if luatex or xelatex
  \ifxetex
    \usepackage{mathspec}
    \usepackage{xltxtra,xunicode}
  \else
    \usepackage{fontspec}
  \fi
  \defaultfontfeatures{Mapping=tex-text,Scale=MatchLowercase}
  \newcommand{\euro}{€}
\fi
% use microtype if available
\IfFileExists{microtype.sty}{\usepackage{microtype}}{}
\ifxetex
  \usepackage[setpagesize=false, % page size defined by xetex
              unicode=false, % unicode breaks when used with xetex
              xetex]{hyperref}
\else
  \usepackage[unicode=true]{hyperref}
\fi
\hypersetup{breaklinks=true,
            bookmarks=true,
            pdfauthor={},
            pdftitle={Application aux endomorphismes continus et aux matrices},
            colorlinks=true,
            citecolor=blue,
            urlcolor=blue,
            linkcolor=magenta,
            pdfborder={0 0 0}}
\urlstyle{same}  % don't use monospace font for urls
\setlength{\parindent}{0pt}
\setlength{\parskip}{6pt plus 2pt minus 1pt}
\setlength{\emergencystretch}{3em}  % prevent overfull lines
\setcounter{secnumdepth}{0}
 
/* start css.sty */
.cmr-5{font-size:50%;}
.cmr-7{font-size:70%;}
.cmmi-5{font-size:50%;font-style: italic;}
.cmmi-7{font-size:70%;font-style: italic;}
.cmmi-10{font-style: italic;}
.cmsy-5{font-size:50%;}
.cmsy-7{font-size:70%;}
.cmex-7{font-size:70%;}
.cmex-7x-x-71{font-size:49%;}
.msbm-7{font-size:70%;}
.cmtt-10{font-family: monospace;}
.cmti-10{ font-style: italic;}
.cmbx-10{ font-weight: bold;}
.cmr-17x-x-120{font-size:204%;}
.cmsl-10{font-style: oblique;}
.cmti-7x-x-71{font-size:49%; font-style: italic;}
.cmbxti-10{ font-weight: bold; font-style: italic;}
p.noindent { text-indent: 0em }
td p.noindent { text-indent: 0em; margin-top:0em; }
p.nopar { text-indent: 0em; }
p.indent{ text-indent: 1.5em }
@media print {div.crosslinks {visibility:hidden;}}
a img { border-top: 0; border-left: 0; border-right: 0; }
center { margin-top:1em; margin-bottom:1em; }
td center { margin-top:0em; margin-bottom:0em; }
.Canvas { position:relative; }
li p.indent { text-indent: 0em }
.enumerate1 {list-style-type:decimal;}
.enumerate2 {list-style-type:lower-alpha;}
.enumerate3 {list-style-type:lower-roman;}
.enumerate4 {list-style-type:upper-alpha;}
div.newtheorem { margin-bottom: 2em; margin-top: 2em;}
.obeylines-h,.obeylines-v {white-space: nowrap; }
div.obeylines-v p { margin-top:0; margin-bottom:0; }
.overline{ text-decoration:overline; }
.overline img{ border-top: 1px solid black; }
td.displaylines {text-align:center; white-space:nowrap;}
.centerline {text-align:center;}
.rightline {text-align:right;}
div.verbatim {font-family: monospace; white-space: nowrap; text-align:left; clear:both; }
.fbox {padding-left:3.0pt; padding-right:3.0pt; text-indent:0pt; border:solid black 0.4pt; }
div.fbox {display:table}
div.center div.fbox {text-align:center; clear:both; padding-left:3.0pt; padding-right:3.0pt; text-indent:0pt; border:solid black 0.4pt; }
div.minipage{width:100%;}
div.center, div.center div.center {text-align: center; margin-left:1em; margin-right:1em;}
div.center div {text-align: left;}
div.flushright, div.flushright div.flushright {text-align: right;}
div.flushright div {text-align: left;}
div.flushleft {text-align: left;}
.underline{ text-decoration:underline; }
.underline img{ border-bottom: 1px solid black; margin-bottom:1pt; }
.framebox-c, .framebox-l, .framebox-r { padding-left:3.0pt; padding-right:3.0pt; text-indent:0pt; border:solid black 0.4pt; }
.framebox-c {text-align:center;}
.framebox-l {text-align:left;}
.framebox-r {text-align:right;}
span.thank-mark{ vertical-align: super }
span.footnote-mark sup.textsuperscript, span.footnote-mark a sup.textsuperscript{ font-size:80%; }
div.tabular, div.center div.tabular {text-align: center; margin-top:0.5em; margin-bottom:0.5em; }
table.tabular td p{margin-top:0em;}
table.tabular {margin-left: auto; margin-right: auto;}
div.td00{ margin-left:0pt; margin-right:0pt; }
div.td01{ margin-left:0pt; margin-right:5pt; }
div.td10{ margin-left:5pt; margin-right:0pt; }
div.td11{ margin-left:5pt; margin-right:5pt; }
table[rules] {border-left:solid black 0.4pt; border-right:solid black 0.4pt; }
td.td00{ padding-left:0pt; padding-right:0pt; }
td.td01{ padding-left:0pt; padding-right:5pt; }
td.td10{ padding-left:5pt; padding-right:0pt; }
td.td11{ padding-left:5pt; padding-right:5pt; }
table[rules] {border-left:solid black 0.4pt; border-right:solid black 0.4pt; }
.hline hr, .cline hr{ height : 1px; margin:0px; }
.tabbing-right {text-align:right;}
span.TEX {letter-spacing: -0.125em; }
span.TEX span.E{ position:relative;top:0.5ex;left:-0.0417em;}
a span.TEX span.E {text-decoration: none; }
span.LATEX span.A{ position:relative; top:-0.5ex; left:-0.4em; font-size:85%;}
span.LATEX span.TEX{ position:relative; left: -0.4em; }
div.float img, div.float .caption {text-align:center;}
div.figure img, div.figure .caption {text-align:center;}
.marginpar {width:20%; float:right; text-align:left; margin-left:auto; margin-top:0.5em; font-size:85%; text-decoration:underline;}
.marginpar p{margin-top:0.4em; margin-bottom:0.4em;}
.equation td{text-align:center; vertical-align:middle; }
td.eq-no{ width:5%; }
table.equation { width:100%; } 
div.math-display, div.par-math-display{text-align:center;}
math .texttt { font-family: monospace; }
math .textit { font-style: italic; }
math .textsl { font-style: oblique; }
math .textsf { font-family: sans-serif; }
math .textbf { font-weight: bold; }
.partToc a, .partToc, .likepartToc a, .likepartToc {line-height: 200%; font-weight:bold; font-size:110%;}
.chapterToc a, .chapterToc, .likechapterToc a, .likechapterToc, .appendixToc a, .appendixToc {line-height: 200%; font-weight:bold;}
.index-item, .index-subitem, .index-subsubitem {display:block}
.caption td.id{font-weight: bold; white-space: nowrap; }
table.caption {text-align:center;}
h1.partHead{text-align: center}
p.bibitem { text-indent: -2em; margin-left: 2em; margin-top:0.6em; margin-bottom:0.6em; }
p.bibitem-p { text-indent: 0em; margin-left: 2em; margin-top:0.6em; margin-bottom:0.6em; }
.paragraphHead, .likeparagraphHead { margin-top:2em; font-weight: bold;}
.subparagraphHead, .likesubparagraphHead { font-weight: bold;}
.quote {margin-bottom:0.25em; margin-top:0.25em; margin-left:1em; margin-right:1em; text-align:\jmathustify;}
.verse{white-space:nowrap; margin-left:2em}
div.maketitle {text-align:center;}
h2.titleHead{text-align:center;}
div.maketitle{ margin-bottom: 2em; }
div.author, div.date {text-align:center;}
div.thanks{text-align:left; margin-left:10%; font-size:85%; font-style:italic; }
div.author{white-space: nowrap;}
.quotation {margin-bottom:0.25em; margin-top:0.25em; margin-left:1em; }
h1.partHead{text-align: center}
.sectionToc, .likesectionToc {margin-left:2em;}
.subsectionToc, .likesubsectionToc {margin-left:4em;}
.subsubsectionToc, .likesubsubsectionToc {margin-left:6em;}
.frenchb-nbsp{font-size:75%;}
.frenchb-thinspace{font-size:75%;}
.figure img.graphics {margin-left:10%;}
/* end css.sty */

\title{Application aux endomorphismes continus et aux matrices}
\author{}
\date{}

\begin{document}
\maketitle

\textbf{Warning: 
requires JavaScript to process the mathematics on this page.\\ If your
browser supports JavaScript, be sure it is enabled.}

\begin{center}\rule{3in}{0.4pt}\end{center}

{[}
{[}
{[}{]}
{[}

\subsubsection{11.4 Application aux endomorphismes continus et aux
matrices}

\paragraph{11.4.1 Calcul fonctionnel et premières applications}

Soit E un K-espace vectoriel normé complet. Si u est un endomorphisme
continu de E, on pose
\\textbar{}u\\textbar{}
=\
sup\_x\neq~0
\\textbar{}u(x)\\textbar{}
\over
\\textbar{}x\\textbar{} . On sait que
\forall~~x \in E,
\\textbar{}u(x)\\textbar{}
\leq\\textbar{}
u\\textbar{}\,\\textbar{}x\\textbar{}.
Soit ℒ(E) l'algèbre des endomorphismes continus sur E. On sait que
(ℒ(E),\\textbar{}.\\textbar{}) est un
espace vectoriel normé complet et que \forall~~u,v
\inℒ(E), \\textbar{}v \cdot u\\textbar{}
\leq\\textbar{}
v\\textbar{}\,\\textbar{}u\\textbar{}.
En particulier, par une récurrence évidente sur n, on a

\forall~n \in \mathbb{N}~, \\forall~~u \inℒ(E),
\\textbar{}u^n\\textbar{}
\leq\\textbar{} u\\textbar{}^n

Proposition~11.4.1 Soit
\\sum ~
a\_nz^n une série entière à coefficients dans K de
rayon de convergence R \textgreater{} 0 et soit u \inℒ(E) tel que
\\textbar{}u\\textbar{} \textless{} R.
Alors la série \\sum ~
a\_nu^n est absolument convergente.

Démonstration On a
\\textbar{}a\_nu^n\\textbar{}
\leq\textbar{}a\_n\textbar{}\,\\textbar{}u\\textbar{}^n
et comme \\textbar{}u\\textbar{}
\textless{} R, la série
\\sum ~
\textbar{}a\_n\textbar{}\,\\textbar{}u\\textbar{}^n
est convergente.

Remarque~11.4.1 Bien entendu, en introduisant la somme
\\sum ~
\_n=0^+\infty~a\_nu^n, on espère que bon
nombre des propriétés formelles de la somme S(z)
= \\sum ~
\_n=0^+\infty~a\_nz^n, valables pour z \in
D(0,R), se transmettront à
\\sum ~
\_n=0^+\infty~a\_nu^n.

Donnons une première application de ce calcul fonctionnel qui généralise
l'identité (1 - z)\\\sum
 \_n=0^+\infty~z^n = 1~:

Proposition~11.4.2 L'ensemble des automorphismes continus de E est un
ouvert de ℒ(E).

Démonstration Soit u \inℒ(E) tel que
\\textbar{}u\\textbar{} \textless{} 1.
D'après la proposition précédente, la série
\\sum ~
\_n≥0u^n converge absolument. Soit s sa somme. On a

(\mathrmId\_E - u) \cdot\left
(\\sum
\_n=0^Nu^n\right ) =
\left (\\sum
\_n=0^Nu^n\right ) \cdot
(\mathrmId\_ E - u) =
\mathrmId\_E - u^n+1

En faisant tendre n vers + \infty~, on a
(\mathrmId\_E - u) \cdot s = s \cdot
(\mathrmId\_E - u) =
\mathrmId\_E, ce qui montre que
\mathrmId\_E - u est un automorphisme continu
de E d'inverse s. Soit maintenant v un automorphisme continu de E et u
\inℒ(E). On écrit v + u = v \cdot (\mathrmId\_E +
v^-1 \cdot u). D'après les préliminaires,
\mathrmId\_E + v^-1 \cdot u (et donc v
+ u) est un automorphisme continu de E dès que
\\textbar{}v^-1 \cdot u\\textbar{}
\textless{} 1 et donc dès que
\\textbar{}u\\textbar{} \textless{} 1
\over
\\textbar{}v^-1\\textbar{} .
On en déduit que la boule B(v, 1 \over
\\textbar{}v^-1\\textbar{} )
est contenue dans l'ensemble des automorphismes continus de E, qui est
donc ouvert.

\paragraph{11.4.2 Exponentielle d'un endomorphisme ou d'une matrice}

Définition~11.4.1 Si u \inℒ(E), on pose exp~ (u)
= \\sum ~
\_n=0^+\infty~ u^n \over n! (série
absolument convergente)

Démonstration La série entière
\\sum  \_n≥0~
z^n \over n! étant de rayon de convergence
infinie, la série \\\sum
 \_n≥0 u^n \over n! est
absolument convergente quelle que soit la norme de u \inℒ(E).

Remarque~11.4.2 De même, si A \in M\_p(K), on définit de la même
fa\ccon exp~ (A) =
e^A =\ \\sum
 \_n=0^+\infty~ A^n \over n! .
On a bien entendu
Mat(\exp~ (u),\mathcal{E})
= exp (\Mat~(u,\mathcal{E})) si
\mathcal{E} est une base de E de dimension finie.

Proposition~11.4.3

\begin{itemize}
\item
  (i) Pour tout automorphisme continu v de E, on a
  exp (v^-1~ \cdot u \cdot v) =
  v^-1 \cdot exp~ (u) \cdot v
\item
  (ii) si u,v \inℒ(E) commutent, alors exp~ (u +
  v) = exp (u) \cdot\ exp~
  (v) = exp~ (v) \cdot\
  exp (u)~; en particulier, pour tout u \inℒ(E),
  exp~ (u) est un automorphisme continu de E et
  (exp (u))^-1~
  = exp~ (-u)
\item
  (iii) l'application \mathbb{R}~\mapsto~ℒ(E),
  t\mapsto~exp~ (tu) est de
  classe C^\infty~ et on a

  \forall~n \in \mathbb{N}~, d^n~
  \over dt^n  exp~
  (tu) = u^n \cdot exp~ (tu)
  = exp (tu) \cdot u^n~
\end{itemize}

Démonstration (i) On a
\\sum ~
\_n=0^N (v^-1\cdotu\cdotv)^n
\over n! =\
\sum  \_n=0^N~
v^-1\cdotu^n\cdotv \over n! =
v^-1 \cdot\left
(\\sum ~\_
n=0^N u^n \over n!
\right ) \cdot v et en faisant tendre N vers + \infty~, on obtient
exp (v^-1~ \cdot u \cdot v) =
v^-1 \cdot exp~ (u) \cdot v.

(ii) Si u,v \inℒ(E) commutent, on pose a\_n = u^n
\over n! et b\_n = v^n
\over n! . Ces séries sont absolument convergentes. On
peut donc faire le produit de Cauchy de ces deux séries et on a alors
c\_n = \\sum ~
\_k=0^n 1 \over k!(n-k)!
u^kv^n-k = 1 \over n! (u +
v)^n d'après la formule du binôme (car u et v commutent). On a
donc

\sum \_n=0^+\infty~~ (u +
v)^n \over n! = \left
(\sum \_n=0^+\infty~ u^n~
\over n! \right ) \cdot\left
(\sum \_n=0^+\infty~ v^n~
\over n! \right )

formule dans laquelle on peut également échanger u et v. On a alors bien
entendu exp~ (u) \cdot\
exp (-u) = exp~ (-u)
\cdot exp (u) =\ exp~ (u -
u) = exp~ (0) =
\mathrmId\_E, ce qui montre que
exp~ (u) est un automorphisme continu de E et
que (exp (u))^-1~
= exp~ (-u)

(iii) On a exp~ (tu)
= \\sum ~
\_k=0^+\infty~ u^k \over k!
t^k, série entière en t de rayon de convergence infini
puisqu'elle converge pour tout t. Sa somme est donc de classe
C^\infty~ et (en dérivant terme à terme cette série entière) on a

\begin{align*} d^n \over
dt^n  exp~ (tu)& =&
\sum \_k=n^+\infty~ u^k~
\over (k - n)! t^k-n = u^n
\cdot\sum \_k=n^+\infty~ u^k-n~
\over (k - n)! t^k-n\%&
\\ & =& u^n
\cdot exp~ (tu) \%&
\\ \end{align*}

Mais exp~ (tu) et u commutent évidemment, d'où
 d^n \over dt^n
 exp (tu) = u^n~
\cdot exp (tu) =\ exp~
(tu) \cdot u^n.

Bien entendu, ce théorème a sa traduction matricielle et on a

Théorème~11.4.4

\begin{itemize}
\item
  (i) \forall~A \in M\_p~(K),
  \forall~P \in GL\_p~(K),

  exp (P^-1~AP) =
  P^-1 exp~ (A)P
\item
  (ii) si A,B \in M\_p(K) commutent, alors
  exp (A + B) =\ exp~
  (A)exp (B) =\ exp~
  (B)exp~ (A)~; en particulier, pour tout A \in
  M\_p(K), exp~ (A) est dans
  GL\_p(K) et (exp (A))^-1~
  = exp~ (-A)
\item
  (iii) l'application \mathbb{R}~\mapsto~M\_p(K),
  t\mapsto~exp~ (tA) est de
  classe C^\infty~ et on a

  \forall~n \in \mathbb{N}~, d^n~
  \over dt^n  exp~
  (tA) = A^n exp~ (tA)
  = exp (tA)A^n~
\end{itemize}

La première propriété montre en particulier que si A est diagonalisable,
on a A =
P\mathrmdiag(\lambda~\_1,\\\ldots,\lambda~\_p)P^-1~,
et donc exp~ (A) =
P\mathrmdiag(e^\lambda~\_1,\\\ldots,e^\lambda~\_p)P^-1~.

Si A est nilpotente d'indice r, on a exp~ (A)
= \\sum ~
\_n=0^r-1 A^n \over n! .

Si A \in M\_p(\mathbb{C}) est quelconque, on a la décomposition de Jordan A
= D + N avec D diagonalisable, N nilpotente et DN = ND. On a donc
d'après la propriété (ii) ci dessus exp~ (A)
= exp (D)\exp~ (N) ce
qui permet le calcul de exp~ (A).

Une autre manière de voir, est d'introduire les sous-espaces
caractéristiques de u \in L(E). Soit
\lambda~\_1,\\ldots,\lambda~\_k~
les valeurs propres distinctes de u et E\_i le sous-espace
caractéristique de u associé à \lambda~\_i. Soit u\_i la
restriction de u à E\_i, \pi~\_i la pro\jmathection sur
E\_i parallèlement à
\\oplus~ ~
\_\jmath\neq~iE\_\jmath. On a évidemment
exp (tu)\_\textbar{}\_E~\_
i = exp (tu\_i~) et donc
exp~ (tu) =\
\sum ~
\_i=1^k exp (tu\_i~) \cdot
\pi~\_i. Mais u\_i =
\lambda~\_i\mathrmId + n\_i avec
n\_i nilpotent. On a donc exp~
(tu\_i) = e^t\lambda~\_i\
\sum~
\_k=0^r\_i-1t^kn\_i^k. On
en déduit que exp~ (tu)
= \\sum ~
\_i=0^ke^t\lambda~\_i\
\sum~
\_k=0^r\_i-1t^kv\_i,k, avec
v\_i,k = n\_i^k \cdot \pi~\_i ce qui donne la
forme générique de exp~ (tu) sous forme de
sommes de produits de fonctions exponentielles par des fonctions
polynomiales.

\paragraph{11.4.3 Application aux systèmes différentiels homogènes à
coefficients constants}

Soit A \in M\_p(K) et le système différentiel à coefficients
constants

 dX \over dt = AX \Leftrightarrow
\left \ \cases 
dx\_1 \over dt &= a\_11x\_1 +
\\ldots~ +
a\_1px\_p \cr
\\ldots~
\cr  dx\_p \over dt &=
a\_p1x\_1 +
\\ldots~ +
a\_ppx\_p  \right .

Théorème~11.4.5 Soit X\_0 \in M\_p,1(K). L'unique solution
du système homogène  dX \over dt = AX vérifiant X(0)
= X\_0 est l'application
t\mapsto~exp~
(tA)X\_0.

Démonstration Cette application convient évidemment puisque  d
\over dt (exp~
(tA)X\_0) = Aexp (tA)X\_0~.
Soit t\mapsto~X(t) une autre solution et soit Y (t)
= exp~ (-tA)X(t). On a Y `(t) =
-exp~ (-tA)AX(t) +\
exp (-tA)X'(t) = exp~ (-tA)(X'(t) - AX(t)) =
0. On en déduit que Y est constante égale à Y (0). Mais Y (0) =
X\_0. On a donc Y (t) = X\_0 soit encore X(t)
= exp (tA)X\_0~.

Remarque~11.4.3 En particulier, si K = \mathbb{C}, la discussion précédente
montre que les fonctions
x\_1,\\ldots,x\_p~
sont des exponentielles polynômes.

{[}
{[}
{[}
{[}

\end{document}

\newpage
\part{Formes quadratiques}
% \documentclass[]{article}
\usepackage[T1]{fontenc}
\usepackage{lmodern}
\usepackage{amssymb,amsmath}
\usepackage{ifxetex,ifluatex}
\usepackage{fixltx2e} % provides \textsubscript
% use upquote if available, for straight quotes in verbatim environments
\IfFileExists{upquote.sty}{\usepackage{upquote}}{}
\ifnum 0\ifxetex 1\fi\ifluatex 1\fi=0 % if pdftex
  \usepackage[utf8]{inputenc}
\else % if luatex or xelatex
  \ifxetex
    \usepackage{mathspec}
    \usepackage{xltxtra,xunicode}
  \else
    \usepackage{fontspec}
  \fi
  \defaultfontfeatures{Mapping=tex-text,Scale=MatchLowercase}
  \newcommand{\euro}{€}
\fi
% use microtype if available
\IfFileExists{microtype.sty}{\usepackage{microtype}}{}
\ifxetex
  \usepackage[setpagesize=false, % page size defined by xetex
              unicode=false, % unicode breaks when used with xetex
              xetex]{hyperref}
\else
  \usepackage[unicode=true]{hyperref}
\fi
\hypersetup{breaklinks=true,
            bookmarks=true,
            pdfauthor={},
            pdftitle={Formes bilineaires},
            colorlinks=true,
            citecolor=blue,
            urlcolor=blue,
            linkcolor=magenta,
            pdfborder={0 0 0}}
\urlstyle{same}  % don't use monospace font for urls
\setlength{\parindent}{0pt}
\setlength{\parskip}{6pt plus 2pt minus 1pt}
\setlength{\emergencystretch}{3em}  % prevent overfull lines
\setcounter{secnumdepth}{0}
 
/* start css.sty */
.cmr-5{font-size:50%;}
.cmr-7{font-size:70%;}
.cmmi-5{font-size:50%;font-style: italic;}
.cmmi-7{font-size:70%;font-style: italic;}
.cmmi-10{font-style: italic;}
.cmsy-5{font-size:50%;}
.cmsy-7{font-size:70%;}
.cmex-7{font-size:70%;}
.cmex-7x-x-71{font-size:49%;}
.msbm-7{font-size:70%;}
.cmtt-10{font-family: monospace;}
.cmti-10{ font-style: italic;}
.cmbx-10{ font-weight: bold;}
.cmr-17x-x-120{font-size:204%;}
.cmsl-10{font-style: oblique;}
.cmti-7x-x-71{font-size:49%; font-style: italic;}
.cmbxti-10{ font-weight: bold; font-style: italic;}
p.noindent { text-indent: 0em }
td p.noindent { text-indent: 0em; margin-top:0em; }
p.nopar { text-indent: 0em; }
p.indent{ text-indent: 1.5em }
@media print {div.crosslinks {visibility:hidden;}}
a img { border-top: 0; border-left: 0; border-right: 0; }
center { margin-top:1em; margin-bottom:1em; }
td center { margin-top:0em; margin-bottom:0em; }
.Canvas { position:relative; }
li p.indent { text-indent: 0em }
.enumerate1 {list-style-type:decimal;}
.enumerate2 {list-style-type:lower-alpha;}
.enumerate3 {list-style-type:lower-roman;}
.enumerate4 {list-style-type:upper-alpha;}
div.newtheorem { margin-bottom: 2em; margin-top: 2em;}
.obeylines-h,.obeylines-v {white-space: nowrap; }
div.obeylines-v p { margin-top:0; margin-bottom:0; }
.overline{ text-decoration:overline; }
.overline img{ border-top: 1px solid black; }
td.displaylines {text-align:center; white-space:nowrap;}
.centerline {text-align:center;}
.rightline {text-align:right;}
div.verbatim {font-family: monospace; white-space: nowrap; text-align:left; clear:both; }
.fbox {padding-left:3.0pt; padding-right:3.0pt; text-indent:0pt; border:solid black 0.4pt; }
div.fbox {display:table}
div.center div.fbox {text-align:center; clear:both; padding-left:3.0pt; padding-right:3.0pt; text-indent:0pt; border:solid black 0.4pt; }
div.minipage{width:100%;}
div.center, div.center div.center {text-align: center; margin-left:1em; margin-right:1em;}
div.center div {text-align: left;}
div.flushright, div.flushright div.flushright {text-align: right;}
div.flushright div {text-align: left;}
div.flushleft {text-align: left;}
.underline{ text-decoration:underline; }
.underline img{ border-bottom: 1px solid black; margin-bottom:1pt; }
.framebox-c, .framebox-l, .framebox-r { padding-left:3.0pt; padding-right:3.0pt; text-indent:0pt; border:solid black 0.4pt; }
.framebox-c {text-align:center;}
.framebox-l {text-align:left;}
.framebox-r {text-align:right;}
span.thank-mark{ vertical-align: super }
span.footnote-mark sup.textsuperscript, span.footnote-mark a sup.textsuperscript{ font-size:80%; }
div.tabular, div.center div.tabular {text-align: center; margin-top:0.5em; margin-bottom:0.5em; }
table.tabular td p{margin-top:0em;}
table.tabular {margin-left: auto; margin-right: auto;}
div.td00{ margin-left:0pt; margin-right:0pt; }
div.td01{ margin-left:0pt; margin-right:5pt; }
div.td10{ margin-left:5pt; margin-right:0pt; }
div.td11{ margin-left:5pt; margin-right:5pt; }
table[rules] {border-left:solid black 0.4pt; border-right:solid black 0.4pt; }
td.td00{ padding-left:0pt; padding-right:0pt; }
td.td01{ padding-left:0pt; padding-right:5pt; }
td.td10{ padding-left:5pt; padding-right:0pt; }
td.td11{ padding-left:5pt; padding-right:5pt; }
table[rules] {border-left:solid black 0.4pt; border-right:solid black 0.4pt; }
.hline hr, .cline hr{ height : 1px; margin:0px; }
.tabbing-right {text-align:right;}
span.TEX {letter-spacing: -0.125em; }
span.TEX span.E{ position:relative;top:0.5ex;left:-0.0417em;}
a span.TEX span.E {text-decoration: none; }
span.LATEX span.A{ position:relative; top:-0.5ex; left:-0.4em; font-size:85%;}
span.LATEX span.TEX{ position:relative; left: -0.4em; }
div.float img, div.float .caption {text-align:center;}
div.figure img, div.figure .caption {text-align:center;}
.marginpar {width:20%; float:right; text-align:left; margin-left:auto; margin-top:0.5em; font-size:85%; text-decoration:underline;}
.marginpar p{margin-top:0.4em; margin-bottom:0.4em;}
.equation td{text-align:center; vertical-align:middle; }
td.eq-no{ width:5%; }
table.equation { width:100%; } 
div.math-display, div.par-math-display{text-align:center;}
math .texttt { font-family: monospace; }
math .textit { font-style: italic; }
math .textsl { font-style: oblique; }
math .textsf { font-family: sans-serif; }
math .textbf { font-weight: bold; }
.partToc a, .partToc, .likepartToc a, .likepartToc {line-height: 200%; font-weight:bold; font-size:110%;}
.chapterToc a, .chapterToc, .likechapterToc a, .likechapterToc, .appendixToc a, .appendixToc {line-height: 200%; font-weight:bold;}
.index-item, .index-subitem, .index-subsubitem {display:block}
.caption td.id{font-weight: bold; white-space: nowrap; }
table.caption {text-align:center;}
h1.partHead{text-align: center}
p.bibitem { text-indent: -2em; margin-left: 2em; margin-top:0.6em; margin-bottom:0.6em; }
p.bibitem-p { text-indent: 0em; margin-left: 2em; margin-top:0.6em; margin-bottom:0.6em; }
.paragraphHead, .likeparagraphHead { margin-top:2em; font-weight: bold;}
.subparagraphHead, .likesubparagraphHead { font-weight: bold;}
.quote {margin-bottom:0.25em; margin-top:0.25em; margin-left:1em; margin-right:1em; text-align:\\jmathmathustify;}
.verse{white-space:nowrap; margin-left:2em}
div.maketitle {text-align:center;}
h2.titleHead{text-align:center;}
div.maketitle{ margin-bottom: 2em; }
div.author, div.date {text-align:center;}
div.thanks{text-align:left; margin-left:10%; font-size:85%; font-style:italic; }
div.author{white-space: nowrap;}
.quotation {margin-bottom:0.25em; margin-top:0.25em; margin-left:1em; }
h1.partHead{text-align: center}
.sectionToc, .likesectionToc {margin-left:2em;}
.subsectionToc, .likesubsectionToc {margin-left:4em;}
.subsubsectionToc, .likesubsubsectionToc {margin-left:6em;}
.frenchb-nbsp{font-size:75%;}
.frenchb-thinspace{font-size:75%;}
.figure img.graphics {margin-left:10%;}
/* end css.sty */

\title{Formes bilineaires}
\author{}
\date{}

\begin{document}
\maketitle

\textbf{Warning: 
requires JavaScript to process the mathematics on this page.\\ If your
browser supports JavaScript, be sure it is enabled.}

\begin{center}\rule{3in}{0.4pt}\end{center}

{[}
{[}{]}
{[}

\subsubsection{12.1 Formes bilinéaires}

\paragraph{12.1.1 Généralités}

Définition~12.1.1 Soit E un K-espace vectoriel . On appelle forme
bilinéaire sur E toute application \phi : E \times E \rightarrow~ K telle que

\begin{itemize}
\itemsep1pt\parskip0pt\parsep0pt
\item
  (i) \forall~~x \in E,
  y\mapsto~\phi(x,y) est linéaire
\item
  (ii) \forall~~y \in E,
  x\mapsto~\phi(x,y) est linéaire
\end{itemize}

Remarque~12.1.1 On a en particulier \forall~~x,y \in E,
\phi(x,0) = \phi(0,y) = 0.

Remarque~12.1.2 Il est clair que si \phi et \psi sont deux formes bilinéaires
sur E, il en est de même de \alpha~\phi + \beta~\psi, d'où la proposition

Proposition~12.1.1 L'ensemble L_2(E) des formes bilinéaires sur
E est un sous-espace vectoriel de l'espace K^E\timesE des
applications de E \times E dans K.

Remarque~12.1.3 Soit \phi une forme bilinéaire sur E. Pour chaque x \in E,
l'application y\mapsto~\phi(x,y) est une forme linéaire
sur E donc un élément, noté g_\phi(x), du dual E^∗ de
E. De même, pour chaque y \in E, l'application
x\mapsto~\phi(x,y) est une forme linéaire sur E, donc
un élément, noté d_\phi(y), de E^∗. La relation

\begin{align*} \left
{[}g_\phi(\alpha~x + \beta~x')\right {]}(y)& =& \phi(\alpha~x +
\beta~x',y) = \alpha~\phi(x,y) + \beta~\phi(x',y)\%& \\ & =&
\left {[}\alpha~g_\phi(x) +
\beta~g_\phi(x')\right {]}(y) \%&
\\ \end{align*}

montre clairement que g_\phi :
x\mapsto~g_\phi(x) est une application
linéaire de E dans E^∗. Il en est évidemment de même de
d_\phi : y\mapsto~d_\phi(y).

Définition~12.1.2 L'application g_\phi : E \rightarrow~ E^∗ (resp.
d_\phi) est appelée l'application linéaire gauche (resp. droite)
associée à la forme bilinéaire \phi.

\paragraph{12.1.2 Formes bilinéaires symétriques, antisymétriques}

Définition~12.1.3 Soit \phi \in L_2(E). On dit que \phi est symétrique
(resp. antisymétrique) si \forall~~x,y \in E, \phi(y,x) =
\phi(x,y) (resp. = -\phi(x,y)).

Proposition~12.1.2 Soit \phi \in L_2(E). Alors \phi est symétrique
(resp. antisymétrique) si et seulement si~d_\phi = g_\phi
(resp. d_\phi = -g_\phi).

Démonstration En effet \phi(x,y) =\big
{[}g_\phi(x)\big {]}(y) et \phi(y,x)
=\big {[}d_\phi(x)\big {]}(y). Donc

\begin{align*} \forall~~x,y \in E,
\phi(y,x) = \epsilon\phi(x,y)&& \%& \\ &
\Leftrightarrow & \forall~~x,y \in E,
\big {[}g_\phi(x)\big {]}(y) =
\epsilon\big {[}d_\phi(x)\big {]}(y) \%&
\\ & \Leftrightarrow &
\forall~x \in E, g_\phi(x) = \epsilond_\phi~(x)
\Leftrightarrow g_\phi = \epsilond_\phi\%&
\\ \end{align*}

Proposition~12.1.3 L'ensemble S_2(E) (resp. A_2(E))
des formes bilinéaires symétriques (resp. antisymétriques) est un
sous-espace vectoriel de L_2(E). Si la caractéristique de K est
différente de 2, alors L_2(E) = S_2(E) \oplus~
A_2(E).

Démonstration La première affirmation est laissée aux soins du lecteur.
Si la caractéristique de K est différente de 2, on a clairement
S_2(E) \bigcap A_2(E) =
\0\ et la relation \phi = \psi + \theta avec
\psi(x,y) = 1 \over 2 (\phi(x,y) + \phi(y,x)), \theta(x,y) = 1
\over 2 (\phi(x,y) - \phi(y,x)), qui sont respectivement
symétrique et antisymétrique, montre que L_2(E) =
S_2(E) + A_2(E).

\paragraph{12.1.3 Matrice d'une forme bilinéaire}

Supposons que E est de dimension finie et soit \mathcal{E} =
(e_1,\\ldots,e_n~)
une base de E.

Définition~12.1.4 Soit \phi \in L_2(E). On appelle matrice de \phi dans
la base \mathcal{E} la matrice

\mathrmMat~ (\phi,\mathcal{E}) =
(\phi(e_i,e_\\jmathmath))_1\leqi,\\jmathmath\leqn \in M_K(n)

Proposition~12.1.4
\mathrmMat~ (\phi,\mathcal{E}) est
l'unique matrice \Omega \in M_K(n) vérifiant

\forall~(x,y) \in E \times E, \phi(x,y) = ^t~X\OmegaY

où X (resp. Y ) désigne le vecteur colonne des coordonnées de x (resp.
y) dans la base \mathcal{E}.

Démonstration Si \Omega = (\omega_i,\\jmathmath), on a

 ^tX\OmegaY = \\sum
_i=1^nx_ i(\OmegaY )_i =
\sum _i=1^nx_ i~
\sum _\\jmathmath=1^n\omega_
i,\\jmathmathy_\\jmathmath = \\sum
_i,\\jmathmath\omega_i,\\jmathmathx_iy_\\jmathmath

Mais d'autre part \phi(x,y) =
\phi(\\sum ~
_i=1^nx_ie_i,\\\sum
 _\\jmathmath=1^ny_\\jmathmathe_\\jmathmath)
= \\sum ~
_i,\\jmathmath\phi(e_i,e_\\jmathmath)x_iy_\\jmathmath en
utilisant la bilinéarité de \phi. Ceci montre que
\mathrmMat~ (\phi,\mathcal{E}) vérifie
bien la relation voulue. Inversement, si \Omega vérifie cette formule, on a
\phi(e_k,e_l) = ^tE_k\OmegaE_l
= \\sum ~
_i,\\jmathmath\omega_i,\\jmathmath\delta_i^k\delta_\\jmathmath^l =
\omega_k,l ce qui montre que \Omega =\
\mathrmMat (\phi,\mathcal{E}).

Théorème~12.1.5 L'application
\phi\mapsto~\mathrmMat~
(\phi,\mathcal{E}) est un isomorphisme d'espaces vectoriels de L_2(E) sur
M_K(n).

Démonstration Les détails sont laissés aux soins du lecteur.
L'application réciproque est bien entendu l'application qui à \Omega \in
M_K(n) associe \phi : E \times E \rightarrow~ K définie par \phi(x,y) =
^tX\OmegaY qui est clairement bilinéaire.

Corollaire~12.1.6 Si E est de dimension finie,
dim L_2~(E) =
(dim E)^2~.

Théorème~12.1.7 Soit E de dimension finie, \mathcal{E} =
(e_1,\\ldots,e_n~)
une base de E, \mathcal{E}^∗ =
(e_1^∗,\\ldots,e_n^∗~)
la base duale. Soit \phi \in L_2(E). Alors

\mathrmMat~ (\phi,\mathcal{E})
= \mathrmMat~
(d_\phi,\mathcal{E},\mathcal{E}^∗) = ^t\
\mathrmMat (g_ \phi,\mathcal{E},\mathcal{E}^∗)

Démonstration Notons \Omega =\
\mathrmMat (\phi,\mathcal{E}), A =\
\mathrmMat (d_\phi,\mathcal{E},\mathcal{E}^∗) et B
= \mathrmMat~
(g_\phi,\mathcal{E},\mathcal{E}^∗). On a

\begin{align*} \omega_i,\\jmathmath& =&
\phi(e_i,e_\\jmathmath) = \left
(d_\phi(e_\\jmathmath)\right )(e_i) \%&
\\ & =& \left
(\sum _k=1^na_
k,\\jmathmathe_k^∗\right )(e_ i) =
a_i,\\jmathmath\%& \\
\end{align*}

compte tenu de e_k^∗(e_i) =
\delta_k^i~; de même

\begin{align*} \omega_i,\\jmathmath& =&
\phi(e_i,e_\\jmathmath) = \left
(g_\phi(e_i)\right )(e_\\jmathmath) \%&
\\ & =& \left
(\sum _k=1^nb_
k,ie_k^∗\right )(e_ \\jmathmath) =
b_\\jmathmath,i\%& \\
\end{align*}

ce qui démontre le résultat.

Corollaire~12.1.8 La forme bilinéaire \phi est symétrique (resp.
antisymétrique) si et seulement si~sa matrice dans la base \mathcal{E} est
symétrique (resp. antisymétrique).

Le rang de \mathrmMat~
(d_\phi,\mathcal{E},\mathcal{E}^∗) est indépendant du choix de la base \mathcal{E}~;
il en est donc de même du rang de
\mathrmMat~ (\phi,\mathcal{E}). Ceci
conduit à la définition suivante

Définition~12.1.5 Soit E de dimension finie et \phi \in L_2(E). On
appelle rang de E le rang de sa matrice dans n'importe quelle base de E.
On a

\mathrmrg~\phi
= \mathrmrgd_\phi~
= \mathrmrgg_\phi~
=\
\mathrmrg\mathrmMat~
(\phi,\mathcal{E})

\paragraph{12.1.4 Changements de bases, discriminant}

Théorème~12.1.9 Soit E un espace vectoriel de dimension finie, \mathcal{E} et \mathcal{E}'
deux bases de E, P = P_\mathcal{E}^\mathcal{E}' la matrice de passage de \mathcal{E} à
\mathcal{E}'. Soit \phi \in L_2(E), \Omega =\
\mathrmMat (\phi,\mathcal{E}) et \Omega' =\
\mathrmMat (\phi,\mathcal{E}'). Alors

\Omega' = ^tP\OmegaP

Démonstration Si X (resp. Y ) désigne le vecteur colonne des coordonnées
de x (resp. y) dans la base \mathcal{E} et X' (resp. Y ') désigne le vecteur
colonne des coordonnées de x (resp. y) dans la base \mathcal{E}', on a X = PX', Y
= PY ', d'où

\phi(x,y) = ^t(PX')\Omega(PY ) = ^tX'(^tP\OmegaP)Y '

Comme \Omega' est l'unique matrice vérifiant \forall~~(x,y)
\in E \times E, \phi(x,y) = ^tX'\Omega'Y ', on a \Omega' = ^tP\OmegaP.

Définition~12.1.6 Soit E un espace vectoriel de dimension finie, \mathcal{E} une
base de E et \phi \in L_2(E). On appelle discriminant de \phi dans la
base \mathcal{E} le déterminant de la matrice
\mathrmMat~ (\phi,\mathcal{E}).

Remarque~12.1.4 La formule ci dessus montre que lors d'un changement de
base, le discriminant est multiplié par
(\mathrm{det}~
P)^2.

On introduit ainsi une nouvelle relation d'équivalence sur les matrices
carrées d'ordre n~: représenter une même forme bilinéaire dans des bases
différentes.

Définition~12.1.7 Soit \Omega,\Omega' \in M_K(n). On dit que ces deux
matrices sont congruentes s'il existe P \in GL_K(n) telle que \Omega'
= ^tP\OmegaP. Il s'agit d'une relation d'équivalence sur
M_K(n).

Remarque~12.1.5 Bien entendu cette relation de congruence laisse stables
les sous-espaces vectoriels des matrices symétriques ou antisymétriques.

\paragraph{12.1.5 Orthogonalité}

Soit E un K-espace vectoriel ~et \phi une forme bilinéaire sur E.

Définition~12.1.8 On dit que x est orthogonal à y (relativement à \phi), et
on pose x \bot y, si \phi(x,y) = 0.

Définition~12.1.9 Soit A une partie de E. On pose

\begin{itemize}
\itemsep1pt\parskip0pt\parsep0pt
\item
  (i) A^\bot = \x \in
  E∣\forall~~a \in A, \phi(a,x)
  = 0\
\item
  (ii) ^\bot A = \x \in
  E∣\forall~~a \in A, \phi(x,a)
  = 0\
\end{itemize}

Remarque~12.1.6 Notons A^\bot^∗  l'orthogonal de A
dans le dual E^∗ de E, c'est-à-dire l'espace vectoriel des
formes linéaires sur E qui sont nulles sur A. On a

\begin{align*} x \in A^\bot&
\Leftrightarrow & \forall~~a \in A, \phi(a,x)
= 0 \%& \\ &
\Leftrightarrow & \forall~~a \in A,
\big {[}d_\phi(x)\big {]}(a) = 0
\%& \\ & \Leftrightarrow &
d_\phi(x) \in A^\bot^∗ 
\Leftrightarrow x \in
d_\phi^-1(A^\bot^∗ )\%&
\\ \end{align*}

On en déduit que A^\bot =
d_\phi^-1(A^\bot^∗ ) et ^\bot A
= g_\phi^-1(A^\bot^∗ ).

Proposition~12.1.10 Soit A une partie de E~; alors

\begin{itemize}
\itemsep1pt\parskip0pt\parsep0pt
\item
  (i)A^\bot et ^\bot A sont des sous espaces vectoriels
  de E
\item
  (ii)A^\bot =\
  \mathrmVect(A)^\bot et ^\bot A
  = ^\bot
  \mathrmVect~(A)
\item
  (iii) A \subset~^\bot (A^\bot) et A \subset~ (^\bot
  A)^\bot
\item
  (iv) A \subset~ B \rigtharrow~ B^\bot\subset~ A^\bot et ^\bot B
  \subset~^\bot A.
\end{itemize}

Démonstration (i) découle immédiatement de la bilinéarité de \phi ou de la
remarque précédente. Il en est de même pour (ii) puisqu'un vecteur x est
orthogonal (aussi bien à gauche qu'à droite) à tout vecteur de A si et
seulement si il est orthogonal à toute combinaison linéaire de vecteurs
de A, c'est à dire à
\mathrmVect~(A). En ce qui
concerne (iii), il suffit de remarquer que tout vecteur a de A est
orthogonal à tout vecteur qui est orthogonal à tout vecteur de A. Pour
(iv), un vecteur x qui est orthogonal à tout vecteur de B est évidemment
orthogonal à tout vecteur de A.

Remarque~12.1.7 Dans le cas où \phi est symétrique ou antisymétrique, on a
\phi(x,y) = 0 \Leftrightarrow \phi(y,x) = 0, si bien que la
relation d'orthogonalité est symétrique. Dans ce cas, il n'y a pas lieu
de distinguer ^\bot A de A^\bot. Dans toute la suite
nous ferons l'hypothèse que \phi est soit symétrique, soit antisymétrique.

\paragraph{12.1.6 Formes non dégénérées}

En règle générale on posera

Définition~12.1.10 Soit E un K-espace vectoriel , \phi une forme bilinéaire
symétrique (resp. antisymétrique) sur E. On appelle noyau de \phi le
sous-espace

\mathrmKer~\phi =
\x \in
E∣\forall~~y \in E, \phi(x,y) =
0\ = E^\bot =\
\mathrmKerd_ \phi

Définition~12.1.11 Soit E un K-espace vectoriel , \phi une forme bilinéaire
symétrique (resp. antisymétrique) sur E. On dit que \phi est non dégénérée
si elle vérifie les conditions équivalentes

\begin{itemize}
\itemsep1pt\parskip0pt\parsep0pt
\item
  (i) \mathrmKer~\phi =
  E^\bot = \0\
\item
  (ii) pour x \in E on a \left
  (\forall~~y \in E, \phi(x,y) = 0\right ) \rigtharrow~
  x = 0
\item
  (iii) d_\phi (resp. g_\phi) est une application linéaire
  in\\jmathmathective de E dans E^∗.
\end{itemize}

L'équivalence entre ces trois propriétés est évidente.

Si E est un espace vectoriel de dimension finie, on sait que
dim E^∗~ =\
dim E. Si d_\phi est in\\jmathmathective, elle est nécessairement
bi\\jmathmathective et on obtient

Théorème~12.1.11 Soit E un K-espace vectoriel ~de dimension finie, \phi une
forme bilinéaire symétrique (resp. antisymétrique) non dégénérée sur E.
Alors l'application linéaire droite d_\phi est un isomorphisme
d'espace vectoriel de E sur E^∗~; autrement dit, pour toute
forme linéaire f sur E, il existe un unique vecteur v_f \in E tel
que \forall~x \in E, f(x) = \phi(x,v_f~).

Corollaire~12.1.12 Soit E un K-espace vectoriel ~de dimension finie, \phi
une forme bilinéaire symétrique (resp. antisymétrique) non dégénérée sur
E. Soit A un sous-espace vectoriel de E. Alors
dim A +\ dim~
A^\bot = dim E et A = A^\bot\bot~.

Démonstration On a en effet

dim A^\bot~ =\
dim d_ \phi^-1(A^\bot^∗ )
= dim A^\bot^∗ ~
= dim E -\ dim~ A

puisque d_\phi est un isomorphisme d'espaces vectoriels. On sait
d'autre part que A \subset~ A^\bot\bot et que
dim A^\bot\bot~ =\
dim E - dim A^\bot~
= dim~ A, d'où l'égalité.

Remarque~12.1.8 Il ne faudrait pas en déduire abusivement que A et
A^\bot sont supplémentaires~; en effet, en général A \bigcap
A^\bot\neq~\0\.
Nous nous intéresserons plus particulièrement à ce point dans le
paragraphe suivant.

Si \mathcal{E} est une base de E, alors \Omega =\
\mathrmMat (\phi,\mathcal{E}) =\
\mathrmMat (d_\phi,\mathcal{E},\mathcal{E}^∗) et
\mathrmrg~\phi
= \mathrmrg~\Omega. On en déduit

Théorème~12.1.13 Soit E un K-espace vectoriel ~de dimension finie n, \phi
une forme bilinéaire symétrique (resp. antisymétrique) sur E, \mathcal{E} une base
de E et \Omega = \mathrmMat~
(\phi,\mathcal{E}). Alors les propositions suivantes sont équivalentes

\begin{itemize}
\itemsep1pt\parskip0pt\parsep0pt
\item
  (i) \phi est non dégénérée
\item
  (ii) \Omega est une matrice inversible
\item
  (iii) \mathrmrg~\phi = n.
\end{itemize}

Remarque~12.1.9 En général,
\mathrmKer~\phi
= \mathrmKerd_\phi~,
\mathrmrg~\phi
= \mathrmrgd_\phi~, si
bien que le théorème du rang devient

Proposition~12.1.14 Soit E un K-espace vectoriel ~de dimension finie n,
\phi une forme bilinéaire symétrique (resp. antisymétrique) sur E, \mathcal{E} une
base de E. Alors dim~ E
= \mathrmrg~\phi
+ dim~
\mathrmKer~\phi.

\paragraph{12.1.7 Isotropie}

Définition~12.1.12 Soit E un K-espace vectoriel , \phi une forme bilinéaire
symétrique (resp. antisymétrique) sur E. On dit qu'un sous-espace
vectoriel A de E est non isotrope s'il vérifie les conditions
équivalentes

\begin{itemize}
\itemsep1pt\parskip0pt\parsep0pt
\item
  (i) A \bigcap A^\bot = \0\
\item
  (ii) la restriction de \phi à A \times A est non dégénérée.
\end{itemize}

Démonstration On a en effet
\mathrmKer\phi__A\timesA~
= \x \in
A∣\forall~~y \in A, \phi(x,y) =
0\ = \x \in
A∣x \in A^\bot\ = A \bigcap
A^\bot.

Définition~12.1.13 Soit E un K-espace vectoriel , \phi une forme bilinéaire
symétrique (resp. antisymétrique) sur E. On dit que x \in E est un vecteur
isotrope s'il vérifie les conditions équivalentes suivantes

\begin{itemize}
\itemsep1pt\parskip0pt\parsep0pt
\item
  (i) la droite Kx est un sous-espace isotrope ou x = 0
\item
  (ii) \phi(x,x) = 0
\end{itemize}

Démonstration (i) \rigtharrow~(ii)~: soit y \in Kx \bigcap
(Kx)^\bot\diagdown\0\~; on a y = \lambda~x
avec \lambda~\neq~0, d'où 0 = \phi(y,y) =
\lambda~^2\phi(x,x), soit \phi(x,x) = 0.

(ii) \rigtharrow~(i) si \phi(x,x) = 0, on a clairement Kx \subset~ (Kx)^\bot.

Remarque~12.1.10 Il est clair que si \phi est antisymétrique et si
\mathrmcarK\mathrel\neq~~2,
alors tout vecteur est isotrope. La notion n'est donc réellement
intéressante que pour les formes symétriques.

Exemple~12.1.1 Pour la forme bilinéaire symétrique sur \mathbb{R}~^4,
\phi(x,y) = x_1y_1 + x_2y_2 +
x_3y_3 - x_4y_4 (forme de Lorentz,
celle de la relativité), le vecteur (1,0,0,1) est isotrope~; cette forme
est bien entendu non dégénérée puisque sa matrice dans la base canonique
est la matrice
\mathrmdiag~(1,1,1,-1) qui
est inversible~; on voit donc que \phi peut être non dégénérée, alors que
sa restriction à un sous-espace est dégénérée (et même nulle).

Définition~12.1.14 On dit que la forme bilinéaire symétrique \phi sur E est
définie s'il n'existe pas de vecteur isotrope autre que 0.

Remarque~12.1.11 Si A est un sous-espace isotrope, alors tout vecteur de
A \bigcap A^\bot\diagdown\0\ est clairement
isotrope (étant orthogonal à tout vecteur de A, il est orthogonal à lui
même). On en déduit que si \phi est une forme bilinéaire symétrique
définie, alors tout sous-espace de E est non isotrope. En particulier, E
lui même est non isotrope et donc

Proposition~12.1.15 Soit \phi une forme bilinéaire symétrique définie~;
alors \phi est non dégénérée et tout sous-espace est non isotrope pour \phi.

Théorème~12.1.16 Soit E un K-espace vectoriel ~de dimension finie, \phi une
forme bilinéaire symétrique (resp. antisymétrique) non dégénérée sur E.
Soit A un sous-espace vectoriel de E. Alors on a l'équivalence de

\begin{itemize}
\itemsep1pt\parskip0pt\parsep0pt
\item
  (i) A est non isotrope
\item
  (ii) E = A \oplus~ A^\bot
\end{itemize}

Démonstration En effet on sait que dim~ A
+ dim A^\bot~ =\
dim E. On a donc

A \bigcap A^\bot = \0\
\Leftrightarrow E = A \oplus~ A^\bot

Corollaire~12.1.17 Soit E un K-espace vectoriel ~de dimension finie, \phi
une forme bilinéaire symétrique définie sur E. Soit A un sous-espace
vectoriel de E. Alors E = A \oplus~ A^\bot.

{[}
{[}

\end{document}

% \documentclass[]{article}
\usepackage[T1]{fontenc}
\usepackage{lmodern}
\usepackage{amssymb,amsmath}
\usepackage{ifxetex,ifluatex}
\usepackage{fixltx2e} % provides \textsubscript
% use upquote if available, for straight quotes in verbatim environments
\IfFileExists{upquote.sty}{\usepackage{upquote}}{}
\ifnum 0\ifxetex 1\fi\ifluatex 1\fi=0 % if pdftex
  \usepackage[utf8]{inputenc}
\else % if luatex or xelatex
  \ifxetex
    \usepackage{mathspec}
    \usepackage{xltxtra,xunicode}
  \else
    \usepackage{fontspec}
  \fi
  \defaultfontfeatures{Mapping=tex-text,Scale=MatchLowercase}
  \newcommand{\euro}{€}
\fi
% use microtype if available
\IfFileExists{microtype.sty}{\usepackage{microtype}}{}
\ifxetex
  \usepackage[setpagesize=false, % page size defined by xetex
              unicode=false, % unicode breaks when used with xetex
              xetex]{hyperref}
\else
  \usepackage[unicode=true]{hyperref}
\fi
\hypersetup{breaklinks=true,
            bookmarks=true,
            pdfauthor={},
            pdftitle={Formes quadratiques},
            colorlinks=true,
            citecolor=blue,
            urlcolor=blue,
            linkcolor=magenta,
            pdfborder={0 0 0}}
\urlstyle{same}  % don't use monospace font for urls
\setlength{\parindent}{0pt}
\setlength{\parskip}{6pt plus 2pt minus 1pt}
\setlength{\emergencystretch}{3em}  % prevent overfull lines
\setcounter{secnumdepth}{0}
 
/* start css.sty */
.cmr-5{font-size:50%;}
.cmr-7{font-size:70%;}
.cmmi-5{font-size:50%;font-style: italic;}
.cmmi-7{font-size:70%;font-style: italic;}
.cmmi-10{font-style: italic;}
.cmsy-5{font-size:50%;}
.cmsy-7{font-size:70%;}
.cmex-7{font-size:70%;}
.cmex-7x-x-71{font-size:49%;}
.msbm-7{font-size:70%;}
.cmtt-10{font-family: monospace;}
.cmti-10{ font-style: italic;}
.cmbx-10{ font-weight: bold;}
.cmr-17x-x-120{font-size:204%;}
.cmsl-10{font-style: oblique;}
.cmti-7x-x-71{font-size:49%; font-style: italic;}
.cmbxti-10{ font-weight: bold; font-style: italic;}
p.noindent { text-indent: 0em }
td p.noindent { text-indent: 0em; margin-top:0em; }
p.nopar { text-indent: 0em; }
p.indent{ text-indent: 1.5em }
@media print {div.crosslinks {visibility:hidden;}}
a img { border-top: 0; border-left: 0; border-right: 0; }
center { margin-top:1em; margin-bottom:1em; }
td center { margin-top:0em; margin-bottom:0em; }
.Canvas { position:relative; }
li p.indent { text-indent: 0em }
.enumerate1 {list-style-type:decimal;}
.enumerate2 {list-style-type:lower-alpha;}
.enumerate3 {list-style-type:lower-roman;}
.enumerate4 {list-style-type:upper-alpha;}
div.newtheorem { margin-bottom: 2em; margin-top: 2em;}
.obeylines-h,.obeylines-v {white-space: nowrap; }
div.obeylines-v p { margin-top:0; margin-bottom:0; }
.overline{ text-decoration:overline; }
.overline img{ border-top: 1px solid black; }
td.displaylines {text-align:center; white-space:nowrap;}
.centerline {text-align:center;}
.rightline {text-align:right;}
div.verbatim {font-family: monospace; white-space: nowrap; text-align:left; clear:both; }
.fbox {padding-left:3.0pt; padding-right:3.0pt; text-indent:0pt; border:solid black 0.4pt; }
div.fbox {display:table}
div.center div.fbox {text-align:center; clear:both; padding-left:3.0pt; padding-right:3.0pt; text-indent:0pt; border:solid black 0.4pt; }
div.minipage{width:100%;}
div.center, div.center div.center {text-align: center; margin-left:1em; margin-right:1em;}
div.center div {text-align: left;}
div.flushright, div.flushright div.flushright {text-align: right;}
div.flushright div {text-align: left;}
div.flushleft {text-align: left;}
.underline{ text-decoration:underline; }
.underline img{ border-bottom: 1px solid black; margin-bottom:1pt; }
.framebox-c, .framebox-l, .framebox-r { padding-left:3.0pt; padding-right:3.0pt; text-indent:0pt; border:solid black 0.4pt; }
.framebox-c {text-align:center;}
.framebox-l {text-align:left;}
.framebox-r {text-align:right;}
span.thank-mark{ vertical-align: super }
span.footnote-mark sup.textsuperscript, span.footnote-mark a sup.textsuperscript{ font-size:80%; }
div.tabular, div.center div.tabular {text-align: center; margin-top:0.5em; margin-bottom:0.5em; }
table.tabular td p{margin-top:0em;}
table.tabular {margin-left: auto; margin-right: auto;}
div.td00{ margin-left:0pt; margin-right:0pt; }
div.td01{ margin-left:0pt; margin-right:5pt; }
div.td10{ margin-left:5pt; margin-right:0pt; }
div.td11{ margin-left:5pt; margin-right:5pt; }
table[rules] {border-left:solid black 0.4pt; border-right:solid black 0.4pt; }
td.td00{ padding-left:0pt; padding-right:0pt; }
td.td01{ padding-left:0pt; padding-right:5pt; }
td.td10{ padding-left:5pt; padding-right:0pt; }
td.td11{ padding-left:5pt; padding-right:5pt; }
table[rules] {border-left:solid black 0.4pt; border-right:solid black 0.4pt; }
.hline hr, .cline hr{ height : 1px; margin:0px; }
.tabbing-right {text-align:right;}
span.TEX {letter-spacing: -0.125em; }
span.TEX span.E{ position:relative;top:0.5ex;left:-0.0417em;}
a span.TEX span.E {text-decoration: none; }
span.LATEX span.A{ position:relative; top:-0.5ex; left:-0.4em; font-size:85%;}
span.LATEX span.TEX{ position:relative; left: -0.4em; }
div.float img, div.float .caption {text-align:center;}
div.figure img, div.figure .caption {text-align:center;}
.marginpar {width:20%; float:right; text-align:left; margin-left:auto; margin-top:0.5em; font-size:85%; text-decoration:underline;}
.marginpar p{margin-top:0.4em; margin-bottom:0.4em;}
.equation td{text-align:center; vertical-align:middle; }
td.eq-no{ width:5%; }
table.equation { width:100%; } 
div.math-display, div.par-math-display{text-align:center;}
math .texttt { font-family: monospace; }
math .textit { font-style: italic; }
math .textsl { font-style: oblique; }
math .textsf { font-family: sans-serif; }
math .textbf { font-weight: bold; }
.partToc a, .partToc, .likepartToc a, .likepartToc {line-height: 200%; font-weight:bold; font-size:110%;}
.chapterToc a, .chapterToc, .likechapterToc a, .likechapterToc, .appendixToc a, .appendixToc {line-height: 200%; font-weight:bold;}
.index-item, .index-subitem, .index-subsubitem {display:block}
.caption td.id{font-weight: bold; white-space: nowrap; }
table.caption {text-align:center;}
h1.partHead{text-align: center}
p.bibitem { text-indent: -2em; margin-left: 2em; margin-top:0.6em; margin-bottom:0.6em; }
p.bibitem-p { text-indent: 0em; margin-left: 2em; margin-top:0.6em; margin-bottom:0.6em; }
.paragraphHead, .likeparagraphHead { margin-top:2em; font-weight: bold;}
.subparagraphHead, .likesubparagraphHead { font-weight: bold;}
.quote {margin-bottom:0.25em; margin-top:0.25em; margin-left:1em; margin-right:1em; text-align:\\jmathmathustify;}
.verse{white-space:nowrap; margin-left:2em}
div.maketitle {text-align:center;}
h2.titleHead{text-align:center;}
div.maketitle{ margin-bottom: 2em; }
div.author, div.date {text-align:center;}
div.thanks{text-align:left; margin-left:10%; font-size:85%; font-style:italic; }
div.author{white-space: nowrap;}
.quotation {margin-bottom:0.25em; margin-top:0.25em; margin-left:1em; }
h1.partHead{text-align: center}
.sectionToc, .likesectionToc {margin-left:2em;}
.subsectionToc, .likesubsectionToc {margin-left:4em;}
.subsubsectionToc, .likesubsubsectionToc {margin-left:6em;}
.frenchb-nbsp{font-size:75%;}
.frenchb-thinspace{font-size:75%;}
.figure img.graphics {margin-left:10%;}
/* end css.sty */

\title{Formes quadratiques}
\author{}
\date{}

\begin{document}
\maketitle

\textbf{Warning: 
requires JavaScript to process the mathematics on this page.\\ If your
browser supports JavaScript, be sure it is enabled.}

\begin{center}\rule{3in}{0.4pt}\end{center}

{[}
{[}
{[}{]}
{[}

\subsubsection{12.2 Formes quadratiques}

\paragraph{12.2.1 Notion de forme quadratique}

Soit E un K-espace vectoriel et \phi une forme bilinéaire symétrique sur E.
Soit \Phi l'application de E dans K qui à x associe \Phi(x) = \phi(x,x).

Proposition~12.2.1 On a les identités suivantes

\begin{itemize}
\itemsep1pt\parskip0pt\parsep0pt
\item
  (i) \Phi(\lambda~x) = \lambda~^2\Phi(x)
\item
  (ii) \Phi(x + y) = \Phi(x) + 2\phi(x,y) + \Phi(y) (identité de polarisation)
\item
  (iii) \Phi(x + y) + \Phi(x - y) = 2(\Phi(x) + \Phi(y)) (identité de la médiane)
\end{itemize}

Démonstration (i) \Phi(\lambda~x) = \phi(\lambda~x,\lambda~x) = \lambda~^2\phi(x,x) =
\lambda~^2\Phi(x)

(ii) \Phi(x + y) = \phi(x + y,x + y) = \Phi(x) + \phi(x,y) + \phi(y,x) + \Phi(y) = \Phi(x) +
2\phi(x,y) + \Phi(y)

(iii) changeant y en - y dans l'identité précédente, on a aussi \Phi(x - y)
= \Phi(x) - 2\phi(x,y) + \Phi(y), et en additionnant les deux on trouve \Phi(x + y)
+ \Phi(x - y) = 2(\Phi(x) + \Phi(y)).

Remarque~12.2.1 Si
\mathrmcarK\mathrel\neq~~2,
l'identité (ii) montre que l'application \phi\mapsto~\Phi
est in\\jmathmathective de S_2(E) dans K^E (espace vectoriel
des applications de E dans K) puisque la connaissance de \Phi permet de
retrouver \phi par

\phi(x,y) = 1 \over 2 (\Phi(x + y) - \Phi(x) - \Phi(y))

Ceci nous amène à poser

Définition~12.2.1 Soit K un corps de caractéristique différente de 2 et
E un K-espace vectoriel . On appelle forme quadratique sur E toute
application \Phi : E \rightarrow~ K vérifiant les deux propriétés

\begin{itemize}
\itemsep1pt\parskip0pt\parsep0pt
\item
  (i) \forall~\lambda~ \in K, \\forall~~x \in
  E, \Phi(\lambda~x) = \lambda~^2\Phi(x)
\item
  (ii) l'application \phi : E \times E \rightarrow~ K, \phi(x,y) = 1 \over
  2 (\Phi(x + y) - \Phi(x) - \Phi(y)) est une forme bilinéaire (évidemment
  symétrique) sur E.
\end{itemize}

Dans ce cas, on a \forall~~x \in E, \Phi(x) = \phi(x,x)~; \phi
est appelée la forme polaire de \Phi.

Démonstration On a \phi(x,x) = 1 \over 2 (\Phi(2x) - 2\Phi(x))
= 1 \over 2 (4\Phi(x) - 2\Phi(x)) = \Phi(x) en utilisant la
propriété (i).

Exemple~12.2.1 Sur K^n, \Phi(x) =\
\sum  _i=1^nx_i^2~
est une forme quadratique dont la forme polaire associée est \phi(x,y)
= \\sum ~
_i=1^nx_iy_i. Si K = \mathbb{R}~ ou K = \mathbb{C}, et si E
désigne l'espace vectoriel des fonctions continues de {[}a,b{]} dans K,
\Phi(f) =\int  _a^bf(t)^2~
dt est une forme quadratique dont la forme polaire est \phi(f,g)
=\int  _a^b~f(t)g(t) dt.

Proposition~12.2.2 L'ensemble Q(E) des formes quadratiques sur E est un
sous-espace vectoriel de K^E~; l'application
\phi\mapsto~\Phi est un isomorphisme d'espaces vectoriels
de S_2(E) sur Q(E).

Remarque~12.2.2 Par la suite on confondra toutes les notions relatives à
\phi et à \Phi~: orthogonalité, matrice, non dégénérescence, isotropie~; en
particulier on posera
\mathrmKer~\Phi
= \mathrmKer~\phi =
\x \in
E∣\forall~~y \in E, \phi(x,y) =
0\. On remarquera qu'en général,
\mathrmKer\Phi\mathrel\neq~~\x
\in E∣\Phi(x) = 0\.

Théorème~12.2.3 (Pythagore). Soit E un K-espace vectoriel ~et \Phi \inQ(E), \phi
la forme polaire de \Phi. Alors

x \bot_\phiy \Leftrightarrow \Phi(x + y) = \Phi(x) + \Phi(y)

Démonstration C'est une conséquence évidente de l'identité de
polarisation.

\paragraph{12.2.2 Formes quadratiques en dimension finie}

Soit E un K-espace vectoriel ~de dimension finie, \Phi \inQ(E) de forme
polaire \phi.

Théorème~12.2.4 Soit \mathcal{E} une base de E. Alors
\mathrmMat~ (\phi,\mathcal{E}) est
l'unique matrice \Omega \in M_K(n) qui est symétrique et qui vérifie

\forall~x \in E, \Phi(x) = ^t~X\OmegaX

Démonstration Il est clair que \Omega =\
\mathrmMat (\Phi,\mathcal{E}) est symétrique et vérifie \Phi(x) =
\phi(x,x) = ^tX\OmegaX. Inversement, soit \Omega une matrice symétrique
vérifiant cette propriété. On a alors

\begin{align*} \phi(x,y)& =& 1 \over
2 (\Phi(x + y) - \Phi(x) - \Phi(y)) \%& \\ &
=& 1 \over 2 (^t(X + Y )\Omega(X + Y )
-^tX\OmegaX -^tY \OmegaY \%&
\\ & =& 1 \over 2
(^tX\OmegaY + ^tY \OmegaX) \%&
\\ \end{align*}

Mais, une matrice 1 \times 1 étant forcément symétrique ^tY \OmegaX =
^t(^tY \OmegaX) = ^tX^t\OmegaY =
^tX\OmegaY puisque \Omega est symétrique. On a donc \phi(x,y) =
^tX\OmegaY ce qui montre que \Omega =\
\mathrmMat (\phi,\mathcal{E}).

Remarque~12.2.3 On prendra garde à la condition de symétrie de \Omega. Il est
en effet clair que l'on peut remplacer, dans la condition \Phi(x) =
^tX\OmegaX, la matrice \Omega par une matrice \Omega' = \Omega + A où A est
antisymétrique, puisque dans ce cas ^tXAX = 0. On aura alors
\Phi(x) = ^tX\Omega'X bien que \Omega' ne soit pas la matrice de \Phi dans la
base \mathcal{E}.

Posons \Omega = \mathrmMat~ (\phi,\mathcal{E})
= (\omega_i,\\jmathmath)_1\leqi,\\jmathmath\leqn. On a alors

\phi(x,y) = \\sum
_i,\\jmathmath\omega_i,\\jmathmathx_iy_\\jmathmath =
\\sum
_i\omega_i,ix_iy_i +
\\sum
_i\textless{}\\jmathmath\omega_i,\\jmathmath(x_iy_\\jmathmath +
x_\\jmathmathy_i)

en tenant compte de \omega_i,\\jmathmath = \omega_\\jmathmath,i. On a donc

\Phi(x) = \phi(x,x) = \\sum
_i\omega_i,ix_i^2 +
2\\sum
_i\textless{}\\jmathmath\omega_i,\\jmathmathx_ix_\\jmathmath =
P_\Phi(x_1,\ldots,x_n~)

où P_\Phi est le polynôme homogène de degré 2 à n variables
P_\Phi(X_1,\\ldots,X_n~)
= \\sum ~
_i\omega_i,iX_i^2 +\
\sum ~
_i\textless{}\\jmathmath\omega_i,\\jmathmathX_iX_\\jmathmath.
Inversement, soit P un polynôme homogène de degré 2 à n variables,
P(X_1,\\ldots,X_n~)
= \\sum ~
_i=1^na_i,iX_i^2
+ \\sum ~
_i\textless{}\\jmathmatha_i,\\jmathmathX_iX_\\jmathmath. Définissons
\phi sur E par

\phi(x,y) = \\sum
_ia_i,ix_iy_i +
\sum _i\textless{}\\jmathmath a_i,\\jmathmath~
\over 2 (x_iy_\\jmathmath +
x_\\jmathmathy_i)

si x = \\sum ~
x_ie_i et y =\
\sum  y_ie_i~. Alors \phi est
clairement une forme bilinéaire symétrique sur E et la forme quadratique
associée vérifie \Phi(x) =
P(x_1,\\ldots,x_n~).
On obtient donc

Théorème~12.2.5 Soit E un K-espace vectoriel ~de dimension finie n, \mathcal{E}
une base de E. L'application qui à une forme quadratique \Phi sur E de
matrice \Omega = (\omega_i,\\jmathmath)_1\leqi,\\jmathmath\leqn dans la base \mathcal{E} associe le
polynôme à n variables
P_\Phi(X_1,\\ldots,X_n~)
= \\sum ~
_i\omega_i,iX_i^2 +
2\\sum ~
_i\textless{}\\jmathmath\omega_i,\\jmathmathX_iX_\\jmathmath est un
isomorphisme d'espaces vectoriels de Q(E) sur l'espace
H_2(X_1,\\ldots,X_n~)
des polynômes homogènes de degré 2 à n variables. Inversement, étant
donné
P(X_1,\\ldots,X_n~)
= \\sum ~
_i=1^na_i,iX_i^2
+ \\sum ~
_i\textless{}\\jmathmatha_i,\\jmathmathX_iX_\\jmathmath \in
H_2(X_1,\\ldots,X_n~),
la forme bilinéaire symétrique associée est donnée par \phi(x,y)
= \\sum ~
_ia_i,ix_iy_i
+ \\sum ~
_i\textless{}\\jmathmath a_i,\\jmathmath \over 2
(x_iy_\\jmathmath + x_\\jmathmathy_i) (règle du
dédoublement des termes).

Remarque~12.2.4 La règle du dédoublement des termes signifie donc que
l'on obtient l'expression de \phi(x,y) à partir de l'expression polynomiale
de \Phi(x) en rempla\ccant partout les termes carrés
x_i^2 par x_iy_i et les termes
rectangles x_ix_\\jmathmath par  1 \over 2
(x_iy_\\jmathmath + x_\\jmathmathy_i). Le lecteur
vérifiera également sans difficulté que

\phi(x,y) = 1 \over 2 \\sum
_i=1^nx_ i \partial~P \over
\partial~X_i
(y_1,\ldots,y_n~)

\paragraph{12.2.3 Matrices et déterminants de Gram}

Définition~12.2.2 Soit E un K-espace vectoriel , \Phi \inQ(E), \phi la forme
polaire de \Phi. Soit
(v_1,\\ldots,v_n~)
une famille finie d'éléments de E. On appelle matrice de Gram de la
famille la matrice
Gram(v_1,\\\ldots,v_n~)
= (\phi(v_i,v_\\jmathmath))_1\leqi,\\jmathmath\leqn et déterminant de Gram
le scalaire
G(v_1,\\ldots,v_n~)
= \mathrm{det}~
Gram(v_1,\\\ldots,v_n~).

Lemme~12.2.6 Soit V =\
\mathrmVect(v_1,\\ldots,v_n~).
Alors

\mathrmrg(\Gram(v_1,\\\ldots,v_n~))
= dim V -\ dim~ (V \bigcap
V ^\bot)

Démonstration Soit \mathcal{E} =
(e_1,\\ldots,e_n~)
la base canonique de K^n et u l'application linéaire de
K^n dans E définie par u(e_i) = v_i. Alors
V = u(K^n). Soit \psi la forme bilinéaire symétrique sur
K^n définie par \psi(x,y) = \phi(u(x),u(y)). On a donc, d'après le
théorème du rang

n = dim K^n~
= \mathrmrg~\psi
+ dim~
\mathrmKer~\psi

Mais \mathrmMat~ (\psi,\mathcal{E}) =
\left (\psi(e_i,e_\\jmathmath)\right
) = \left
(\phi(u(e_i)),u(e_\\jmathmath))\right ) =
\left (\phi(v_i,v_\\jmathmath)\right
) =\
Gram(v_1,\\ldots,v_n~),
si bien que \mathrmrg~\psi
=\
\mathrmrgGram(v_1,\\\ldots,v_n~).
D'autre part

\begin{align*} x
\in\mathrmKer~\psi&
\Leftrightarrow & \forall~~y \in
K^n, \psi(x,y) = 0 \%& \\ &
\Leftrightarrow & \forall~~y \in
K^n,\phi(u(x),u(y)) = 0 \%& \\ &
\Leftrightarrow & \forall~~v \in V =
u(K^n), \phi(u(x),v) = 0\%& \\ &
\Leftrightarrow & u(x) \in V ^\bot\bigcap V \%&
\\ \end{align*}

On a donc \mathrmKer~\psi =
u^-1(V \bigcap V ^\bot), soit (puisque V \bigcap V ^\bot\subset~
V = \mathrmIm~u),

\begin{align*} dim~
\mathrmKer~\psi& =&
dim V \bigcap V ^\bot~
+ dim~
\mathrmKer~u \%&
\\ & =& dim~ V
\bigcap V ^\bot + n - dim~
\mathrmIm~u\%&
\\ & =& dim~ V
\bigcap V ^\bot + n - dim~ V \%&
\\ \end{align*}

D'où finalement

\begin{align*}
\mathrmrg(\Gram(v_1,\\\ldots,v_n~))&
=& \mathrmrg~\psi = n
- dim~
\mathrmKer~\psi \%&
\\ & =& dim~ V
- dim (V \bigcap V ^\bot~)\%&
\\ \end{align*}

Comme dim~ V \leq n, on a donc
\mathrmrg\Gram(v_1,\\\ldots,v_n~)
= n \Leftrightarrow dim~ V = n et
V \bigcap V ^\bot = \0\. On a donc

Proposition~12.2.7 Soit E un K-espace vectoriel , \Phi \inQ(E), \phi la forme
polaire de \Phi. Soit
(v_1,\\ldots,v_n~)
une famille finie d'éléments de E. Alors on a équivalence de

\begin{itemize}
\itemsep1pt\parskip0pt\parsep0pt
\item
  (i)
  G(v_1,\\ldots,v_n)\mathrel\neq~~0
\item
  (ii) la famille
  (v_1,\\ldots,v_n~)
  est libre et le sous-espace
  \mathrmVect(v_1,\\\ldots,v_n~)
  est non isotrope.
\end{itemize}

Corollaire~12.2.8 Soit E un K-espace vectoriel , \Phi \inQ(E) une forme
quadratique sur E qui est définie. Soit
(v_1,\\ldots,v_n~)
une famille finie d'éléments de E. Alors on a équivalence de

\begin{itemize}
\itemsep1pt\parskip0pt\parsep0pt
\item
  (i)
  G(v_1,\\ldots,v_n)\mathrel\neq~~0
\item
  (ii) la famille
  (v_1,\\ldots,v_n~)
  est libre.
\end{itemize}

Remarque~12.2.5 Les déterminants de Gram permettent donc, moyennant la
connaissance d'une forme quadratique définie sur E (s'il en existe), de
tester la liberté d'une famille finie, quel que soit le cardinal de
cette famille et même si l'espace vectoriel E est de dimension infinie.
C'est ainsi que pour une famille
(f_1,\\ldots,f_n~)
de fonctions continues de {[}0,1{]} dans \mathbb{R}~, on a

(f_1,\\ldots,f_n~)\text
libre  \Leftrightarrow
\mathrm{det}~
\left (\int ~
_0^1f_ if_\\jmathmath\right
)\neq~0

{[}
{[}
{[}
{[}

\end{document}

% \documentclass[]{article}
\usepackage[T1]{fontenc}
\usepackage{lmodern}
\usepackage{amssymb,amsmath}
\usepackage{ifxetex,ifluatex}
\usepackage{fixltx2e} % provides \textsubscript
% use upquote if available, for straight quotes in verbatim environments
\IfFileExists{upquote.sty}{\usepackage{upquote}}{}
\ifnum 0\ifxetex 1\fi\ifluatex 1\fi=0 % if pdftex
  \usepackage[utf8]{inputenc}
\else % if luatex or xelatex
  \ifxetex
    \usepackage{mathspec}
    \usepackage{xltxtra,xunicode}
  \else
    \usepackage{fontspec}
  \fi
  \defaultfontfeatures{Mapping=tex-text,Scale=MatchLowercase}
  \newcommand{\euro}{€}
\fi
% use microtype if available
\IfFileExists{microtype.sty}{\usepackage{microtype}}{}
\ifxetex
  \usepackage[setpagesize=false, % page size defined by xetex
              unicode=false, % unicode breaks when used with xetex
              xetex]{hyperref}
\else
  \usepackage[unicode=true]{hyperref}
\fi
\hypersetup{breaklinks=true,
            bookmarks=true,
            pdfauthor={},
            pdftitle={Reduction des formes quadratiques en dimension finie},
            colorlinks=true,
            citecolor=blue,
            urlcolor=blue,
            linkcolor=magenta,
            pdfborder={0 0 0}}
\urlstyle{same}  % don't use monospace font for urls
\setlength{\parindent}{0pt}
\setlength{\parskip}{6pt plus 2pt minus 1pt}
\setlength{\emergencystretch}{3em}  % prevent overfull lines
\setcounter{secnumdepth}{0}
 
/* start css.sty */
.cmr-5{font-size:50%;}
.cmr-7{font-size:70%;}
.cmmi-5{font-size:50%;font-style: italic;}
.cmmi-7{font-size:70%;font-style: italic;}
.cmmi-10{font-style: italic;}
.cmsy-5{font-size:50%;}
.cmsy-7{font-size:70%;}
.cmex-7{font-size:70%;}
.cmex-7x-x-71{font-size:49%;}
.msbm-7{font-size:70%;}
.cmtt-10{font-family: monospace;}
.cmti-10{ font-style: italic;}
.cmbx-10{ font-weight: bold;}
.cmr-17x-x-120{font-size:204%;}
.cmsl-10{font-style: oblique;}
.cmti-7x-x-71{font-size:49%; font-style: italic;}
.cmbxti-10{ font-weight: bold; font-style: italic;}
p.noindent { text-indent: 0em }
td p.noindent { text-indent: 0em; margin-top:0em; }
p.nopar { text-indent: 0em; }
p.indent{ text-indent: 1.5em }
@media print {div.crosslinks {visibility:hidden;}}
a img { border-top: 0; border-left: 0; border-right: 0; }
center { margin-top:1em; margin-bottom:1em; }
td center { margin-top:0em; margin-bottom:0em; }
.Canvas { position:relative; }
li p.indent { text-indent: 0em }
.enumerate1 {list-style-type:decimal;}
.enumerate2 {list-style-type:lower-alpha;}
.enumerate3 {list-style-type:lower-roman;}
.enumerate4 {list-style-type:upper-alpha;}
div.newtheorem { margin-bottom: 2em; margin-top: 2em;}
.obeylines-h,.obeylines-v {white-space: nowrap; }
div.obeylines-v p { margin-top:0; margin-bottom:0; }
.overline{ text-decoration:overline; }
.overline img{ border-top: 1px solid black; }
td.displaylines {text-align:center; white-space:nowrap;}
.centerline {text-align:center;}
.rightline {text-align:right;}
div.verbatim {font-family: monospace; white-space: nowrap; text-align:left; clear:both; }
.fbox {padding-left:3.0pt; padding-right:3.0pt; text-indent:0pt; border:solid black 0.4pt; }
div.fbox {display:table}
div.center div.fbox {text-align:center; clear:both; padding-left:3.0pt; padding-right:3.0pt; text-indent:0pt; border:solid black 0.4pt; }
div.minipage{width:100%;}
div.center, div.center div.center {text-align: center; margin-left:1em; margin-right:1em;}
div.center div {text-align: left;}
div.flushright, div.flushright div.flushright {text-align: right;}
div.flushright div {text-align: left;}
div.flushleft {text-align: left;}
.underline{ text-decoration:underline; }
.underline img{ border-bottom: 1px solid black; margin-bottom:1pt; }
.framebox-c, .framebox-l, .framebox-r { padding-left:3.0pt; padding-right:3.0pt; text-indent:0pt; border:solid black 0.4pt; }
.framebox-c {text-align:center;}
.framebox-l {text-align:left;}
.framebox-r {text-align:right;}
span.thank-mark{ vertical-align: super }
span.footnote-mark sup.textsuperscript, span.footnote-mark a sup.textsuperscript{ font-size:80%; }
div.tabular, div.center div.tabular {text-align: center; margin-top:0.5em; margin-bottom:0.5em; }
table.tabular td p{margin-top:0em;}
table.tabular {margin-left: auto; margin-right: auto;}
div.td00{ margin-left:0pt; margin-right:0pt; }
div.td01{ margin-left:0pt; margin-right:5pt; }
div.td10{ margin-left:5pt; margin-right:0pt; }
div.td11{ margin-left:5pt; margin-right:5pt; }
table[rules] {border-left:solid black 0.4pt; border-right:solid black 0.4pt; }
td.td00{ padding-left:0pt; padding-right:0pt; }
td.td01{ padding-left:0pt; padding-right:5pt; }
td.td10{ padding-left:5pt; padding-right:0pt; }
td.td11{ padding-left:5pt; padding-right:5pt; }
table[rules] {border-left:solid black 0.4pt; border-right:solid black 0.4pt; }
.hline hr, .cline hr{ height : 1px; margin:0px; }
.tabbing-right {text-align:right;}
span.TEX {letter-spacing: -0.125em; }
span.TEX span.E{ position:relative;top:0.5ex;left:-0.0417em;}
a span.TEX span.E {text-decoration: none; }
span.LATEX span.A{ position:relative; top:-0.5ex; left:-0.4em; font-size:85%;}
span.LATEX span.TEX{ position:relative; left: -0.4em; }
div.float img, div.float .caption {text-align:center;}
div.figure img, div.figure .caption {text-align:center;}
.marginpar {width:20%; float:right; text-align:left; margin-left:auto; margin-top:0.5em; font-size:85%; text-decoration:underline;}
.marginpar p{margin-top:0.4em; margin-bottom:0.4em;}
.equation td{text-align:center; vertical-align:middle; }
td.eq-no{ width:5%; }
table.equation { width:100%; } 
div.math-display, div.par-math-display{text-align:center;}
math .texttt { font-family: monospace; }
math .textit { font-style: italic; }
math .textsl { font-style: oblique; }
math .textsf { font-family: sans-serif; }
math .textbf { font-weight: bold; }
.partToc a, .partToc, .likepartToc a, .likepartToc {line-height: 200%; font-weight:bold; font-size:110%;}
.chapterToc a, .chapterToc, .likechapterToc a, .likechapterToc, .appendixToc a, .appendixToc {line-height: 200%; font-weight:bold;}
.index-item, .index-subitem, .index-subsubitem {display:block}
.caption td.id{font-weight: bold; white-space: nowrap; }
table.caption {text-align:center;}
h1.partHead{text-align: center}
p.bibitem { text-indent: -2em; margin-left: 2em; margin-top:0.6em; margin-bottom:0.6em; }
p.bibitem-p { text-indent: 0em; margin-left: 2em; margin-top:0.6em; margin-bottom:0.6em; }
.paragraphHead, .likeparagraphHead { margin-top:2em; font-weight: bold;}
.subparagraphHead, .likesubparagraphHead { font-weight: bold;}
.quote {margin-bottom:0.25em; margin-top:0.25em; margin-left:1em; margin-right:1em; text-align:justify;}
.verse{white-space:nowrap; margin-left:2em}
div.maketitle {text-align:center;}
h2.titleHead{text-align:center;}
div.maketitle{ margin-bottom: 2em; }
div.author, div.date {text-align:center;}
div.thanks{text-align:left; margin-left:10%; font-size:85%; font-style:italic; }
div.author{white-space: nowrap;}
.quotation {margin-bottom:0.25em; margin-top:0.25em; margin-left:1em; }
h1.partHead{text-align: center}
.sectionToc, .likesectionToc {margin-left:2em;}
.subsectionToc, .likesubsectionToc {margin-left:4em;}
.subsubsectionToc, .likesubsubsectionToc {margin-left:6em;}
.frenchb-nbsp{font-size:75%;}
.frenchb-thinspace{font-size:75%;}
.figure img.graphics {margin-left:10%;}
/* end css.sty */

\title{Reduction des formes quadratiques en dimension finie}
\author{}
\date{}

\begin{document}
\maketitle

\textbf{Warning: 
requires JavaScript to process the mathematics on this page.\\ If your
browser supports JavaScript, be sure it is enabled.}

\begin{center}\rule{3in}{0.4pt}\end{center}

[
[
[]
[

\subsubsection{12.3 Réduction des formes quadratiques en dimension
finie}

\paragraph{12.3.1 Familles et bases orthogonales}

Définition~12.3.1 Soit E un K-espace vectoriel ~et \Phi une forme
quadratique sur E de forme polaire \phi. Soit (e_i)_i\inI
une famille d'éléments de E. On dit que la famille est

\begin{itemize}
\itemsep1pt\parskip0pt\parsep0pt
\item
  (i) orthogonale si i\neq~j \rigtharrow~
  \phi(e_i,e_j) = 0
\item
  (ii) orthonormée si \forall~~i,j,
  \phi(e_i,e_j) = \delta_i^j
\end{itemize}

Proposition~12.3.1 Une famille orthogonale ne contenant pas de vecteur
isotrope est libre~; en particulier toute famille orthonormée est libre.

Démonstration Soit (e_i)_i\inI une famille orthogonale.
Soit (\lambda_i)_i\inI des scalaires tels que
\i \in
I∣\lambda_i\mathrel\neq~0\
est fini et \\sum ~
\lambda_ie_i = 0. Soit k \in I. On a alors

0 = \phi(e_k,\sum \lambda_ie_i~)
= \sum \lambda_i\phi(e_k,e_i~) =
\lambda_k\phi(e_k,e_k) = \lambda_k\Phi(e_k)

Comme \Phi(e_k)\neq~0, on a \lambda_k =
0 ce qui montre que la famille est libre.

En dimension finie nous nous intéresserons tout particulièrement aux
bases qui sont orthogonales ou même mieux orthonormées.

Proposition~12.3.2 Soit E un K-espace vectoriel ~de dimension finie et \Phi
une forme quadratique sur E de forme polaire \phi. Soit \mathcal{E} une base de E.
Les propositions suivantes sont équivalentes

\begin{itemize}
\itemsep1pt\parskip0pt\parsep0pt
\item
  (i) \mathcal{E} est une base orthogonale (resp. orthonormée)
\item
  (ii) \mathrmMat~ (\phi,\mathcal{E}) est
  diagonale (resp. la matrice identité)
\item
  (iii) \forall~~x \in E, \Phi(x) =\
  \sum  \alpha_ix_i^2~
  (resp. \Phi(x) = \\sum ~
  x_i^2) si x =\
  \sum  x_ie_i~.
\end{itemize}

Démonstration Evident puisque
\mathrmMat~ (\phi,\mathcal{E}) =
(\phi(e_i,e_j))_1\leqi,j\leqn.

Théorème~12.3.3 Soit E un K-espace vectoriel ~de dimension finie et \Phi
une forme quadratique sur E de forme polaire \phi. Alors il existe des
bases de E orthogonales pour \phi.

Démonstration Nous allons montrer ce résultat par récurrence sur
dim E. Pour \dim~ E =
1, il n'y a rien à démontrer toute base étant orthogonale. Supposons le
résultat démontré pour tout espace vectoriel de dimension n - 1 et soit
E de dimension n. Si \Phi = 0, alors toute base est orthogonale. Sinon,
soit a \in E tel que \Phi(a)\neq~0~; on a
a\neq~0 et on peut donc compléter a en une base
(a,v_2,\\ldots,v_n~)
de E. Posons e_i = v_i + \lambda_ia et cherchons à
déterminer \lambda_i pour que \phi(a,e_i) = 0~; ceci conduit à
l'équation \phi(a,v_i) + \lambda_i\Phi(a) = 0, soit encore
\lambda_i = - \phi(a,e_i) \over \Phi(a) . Les
\lambda_i (et donc les e_i) étant ainsi choisis, la famille
(a,e_2,\\ldots,e_n~)
est encore une base de E (on vérifie facilement qu'elle est libre et
elle a le bon cardinal), avec \forall~~i \in [2,n],
e_i \bot a. Soit H =\
\mathrmVect(e_2,\\ldotse_n~)~;
on a donc a \bot H. Par hypothèse de récurrence, H admet une base
(a_2,\\ldots,a_n~)
orthogonale pour \Phi__H (et donc pour \Phi).
Comme Ka et H sont supplémentaires dans E,
(a,a_2,\\ldots,a_n~)
est une base de E et elle est orthogonale pour \phi.

Corollaire~12.3.4 Soit A \in M_K(n) une matrice symétrique. Alors
il existe une matrice inversible P telle que ^tPAP soit
diagonale.

Démonstration Soit \Phi la forme quadratique sur K^n dont la
matrice dans la base canonique \mathcal{E} est A. Soit \mathcal{E}' une base orthogonale
pour \phi et P la matrice de passage de \mathcal{E} à \mathcal{E}'. Alors la matrice de \phi dans
la base \mathcal{E}' est ^tPAP et elle est diagonale.

Remarque~12.3.1 On prendra soin de ne pas confondre ^tPAP et
P^-1AP~; ce corollaire ne concerne aucunement une quelconque
diagonalisabilité de la matrice A (dont nous verrons qu'elle dépend
essentiellement du corps de base).

Soit E un K-espace vectoriel ~de dimension finie et \Phi une forme
quadratique sur E de forme polaire \phi. Soit \mathcal{E} une base orthogonale de E.
Alors la matrice de \phi dans la base \mathcal{E} est diagonale donc de la forme

\left
(\matrix\,\alpha_1&0&\\ldots~&0
\cr 0
&⋱&\mathrel⋱&\⋮~
\cr \⋮~
&⋱&\mathrel⋱&0
\cr 0
&\\ldots&0&\alpha_n~\right
)

et quitte à permuter les vecteurs de base on peut supposer que
\alpha_1\neq~0,\\ldots,\alpha_r\mathrel\neq~0,\alpha_r+1~
= \\ldots~ =
\alpha_n = 0 pour un certain r \in [0,n]. On a bien entendu r
=\
\mathrmrg\mathrmMat~
(\phi,\mathcal{E})) = \mathrmrg~\phi ce qui
montre que r ne dépend pas de la base choisie. Les vecteurs
e_r+1,\\ldots,e_n~
sont bien entendu dans
\mathrmKer~\phi (étant
orthogonaux à tous les autres vecteurs de base mais aussi à eux mêmes,
ils sont orthogonaux à tout vecteur de E)~; mais
dim~
\mathrmKer~\phi = n
-\mathrmrg~\phi = n - r. On en
déduit que ces vecteurs forment une base de
\mathrmKer~\phi et que donc
\mathrmKer~\phi
=\
\mathrmVect(e_r+1,\\ldots,e_n~).
Donnons nous d'autre part
\lambda_1,\\ldots,\lambda_r~
non nuls et considérons la base \mathcal{E}' =
(\lambda_1e_1,\\ldots,\lambda_re_r,e_r+1,\\\ldots,e_n~).
Il s'agit encore d'une base orthogonale et

\mathrmMat~ (\phi,\mathcal{E}) =
\left
(\matrix\,\lambda_1^2\alpha_1&0&\\ldots~
&0&0 \cr 0
&⋱&\mathrel⋱
&\\ldots&\\⋮~
\cr \⋮~
&⋱&\lambda_r^2\alpha_r&\mathrel⋱&\⋮~
\cr \⋮~
&\\ldots&\mathrel⋱~
&0&⋱ \cr
\⋮~
&\\ldots&\\\ldots~
&⋱&\mathrel⋱\right
)

On voit donc que l'on peut multiplier
\alpha_1,\\ldots,\alpha_r~
par des carrés non nuls arbitraires. En particulier, si
\alpha_1,\\ldots,\alpha_r~
sont des carrés, en prenant \lambda_i tel que \lambda_i^2
= 1 \over \alpha_i on obtient une base dans
laquelle

\mathrmMat~ (\phi,\mathcal{E}) =
\left
(\matrix\,I_r&0
\cr 0 &0\right )

Si K est algébriquement clos, tout élément de K est un carré et cette
réduction est toujours possible. On a donc

Théorème~12.3.5 Soit K un corps algébriquement clos, E un K-espace
vectoriel ~de dimension finie et \Phi une forme quadratique sur E de forme
polaire \phi, de rang r. Alors il existe des bases (orthogonales) de E
telles que \mathrmMat~ (\phi,\mathcal{E})
= \left
(\matrix\,I_r&0
\cr 0 &0\right ). En particulier, si \Phi
est non dégénérée, il existe des bases orthonormées de E.

Corollaire~12.3.6 Soit K un corps algébriquement clos et A,B \in
M_K(n) des matrices symétriques. Alors A et B sont congruentes
si et seulement si~elles ont même rang. En particulier, si A est une
matrice symétrique inversible, il existe P inversible telle que A =
^tPP.

Démonstration Le premier point résulte immédiatement du théorème. En ce
qui concerne le deuxième, si A est une matrice symétrique inversible,
elle est congruente à l'identité, donc il existe P inversible telle que
A = ^tPI_nP = ^tPP.

\paragraph{12.3.2 Décomposition en carrés. Algorithme de Gauss}

Théorème~12.3.7 (décomposition en carrés). Soit E un K-espace vectoriel
~de dimension finie et \Phi une forme quadratique sur E. Alors il existe
des formes linéaires
f_1,\\ldots,f_r~
linéairement indépendantes et des scalaires
\alpha_1,\\ldots,\alpha_r~
non nuls tels que \forall~~x \in E, \Phi(x)
= \\sum ~
_i=1^r\alpha_if_i(x)^2. Dans toute
telle décomposition, on a
\mathrmrg~\Phi = r.

Démonstration Soit \mathcal{E} =
(e_1,\\ldots,e_n~)
une base orthogonale pour \Phi, A =\
\mathrmdiag(\alpha_1,\\ldots,\alpha_n~)
la matrice de \Phi dans la base \mathcal{E}. Quitte à permuter la base, on peut
supposer que
\alpha_1\neq~0,\\ldots,\alpha_r\mathrel\neq~~0
et que \alpha_r+1 =
\\ldots~ =
\alpha_n = 0. Dans une telle base, on a, en notant
(e_1^∗,\\ldots,e_n^∗~)
la base duale de \mathcal{E},

\Phi(x) = ^tXAX = \\sum
_i=1^r\alpha_ ix_i^2 =
\sum _i=1^r\alpha~_
ie_i^∗(x)^2

ce qui montre l'existence d'une telle décomposition avec f_i =
e_i^∗ pour 1 \leq i \leq r. Inversement, considérons une telle
décomposition. Comme la famille
(f_1,\\ldots,f_r~)
est libre, on peut la compléter en une base
(f_1,\\ldots,f_n~)
de E^∗~; cette base est la base duale d'une unique base
(e_1,\\ldots,e_n~)
de E. Si x = \\sum ~
x_ie_i, alors

\Phi(x) = \sum _i=1^r\alpha~_
if_i(x)^2 = \\sum
_i=1^r\alpha_ ie_i^∗(x)^2
= \sum _i=1^r\alpha~_
ix_i^2

On en déduit que la matrice de \Phi dans la base \mathcal{E} est la matrice
\mathrmdiag(\alpha_1,\\\ldots,\alpha_r,0,\\\ldots~,0),
ce qui montre que r est le rang de \Phi.

Remarque~12.3.2 La démonstration précédente montre qu'à toute base
orthogonale de E est associée une telle décomposition en carrés de \Phi et
qu'inversement à toute telle décomposition (avec
f_1,\\ldots,f_r~
linéairement indépendantes) correspond une base orthogonale. Les
problèmes de la décomposition en carrés d'une forme quadratique ou de la
construction d'une base orthogonale sont donc équivalents (ils sont en
fait duaux l'un de l'autre). Nous allons commencer par un algorithme de
décomposition en carrés en renvoyant au paragraphe suivant un algorithme
de construction de bases orthogonales.

Pour décrire un algorithme de décomposition en carrés de formes
linéaires nous allons travailler sur les polynômes homogènes. Soit en
effet \mathcal{E} une base de E. Il existe un unique polynôme homogène de degré 2,
P \in
H_2(X_1,\\ldots,X_n~),
tel que \Phi(x) =
P(x_1,\\ldots,x_n~)
si x = \\sum ~
x_ie_i. De même, si f est une forme linéaire sur E, on
a f = \\sum ~
a_ie_i^∗, d'où f(x) =\
\sum  a_ix_i~ =
F(x_1,\\ldots,x_n~)
où F est un polynôme homogène de degré 1~; inversement à tout tel
polynôme homogène de degré 1 est associée une unique forme linéaire. Une
traduction du théorème de décomposition en carrés est donc

Proposition~12.3.8 Soit P \in
K[X_1,\\ldots,X_n~]
un polynôme homogène de degré 2. Alors il existe des polynômes homogènes
de degré 1,
P_1,\\ldots,P_r~,
linéairement indépendants et des scalaires non nuls
\alpha_1,\\ldots,\alpha_r~
tels que P = \\sum ~
_i=1^r\alpha_iP_i^2.

Algorithme de Gauss

L'algorithme de Gauss va permettre d'expliciter une telle décomposition
en travaillant par récurrence sur le nombre n de variables du polynôme P
(en considérant par convention que pour n = 0, le polynôme est nul).

Si n = 1, alors P(X_1) = \alpha_1X_1^2 et
on a soit r = 0 si \alpha_1 = 0, soit P(X_1) =
\alpha_1P_1(X_1)^2 avec
P_1(X_1) = X_1 si
\alpha_1\neq~0.

Supposons donc connue une telle décomposition pour tout polynôme à n - 1
ou n - 2 variables. Soit P \in
K[X_1,\\ldots,X_n~]
un polynôme homogène de degré 2 à n variables. On a donc

P(X_1,\\ldots,X_n~)
= \sum \omega_i,iX_i^2~ +
2\\sum
_i<j\omega_i,jX_iX_j

Distinguons alors deux cas

Premier cas~: il existe i \in [1,n] tel que
\omega_i,i\neq~0. Quitte à permuter les noms
des variables, on peut supposer que
\omega_n,n\neq~0. Utilisant une mise sous
forme canonique du trinome du second degré en la variable X_n,
on peut donc écrire

P(X_1,\\ldots,X_n~)
= \omega_n,n\left (X_n +
\sum _i=1^n-1 \omega_i,n~
\over \omega_n,n X_i\right
)^2 + Q(X_
1,\\ldots,X_n-1~)

où Q est un polynôme homogène de degré 2 en n - 1 variables. D'après
l'hypothèse de récurrence, on peut écrire Q =\
\sum ~
_i=1^k\alpha_iP_i^2 où P_i \in
K[X_1,\\ldots,X_n-1~]
\subset~
K[X_1,\\ldots,X_n~],
les P_i étant homogènes de degré 1 et linéairement
indépendants, les \alpha_i étant non nuls. En posant P_k+1
= X_n +\ \\sum
 _i=1^n-1 \omega_i,n \over
\omega_n,n X_i et \alpha_k+1 =
\omega_n,n\neq~0 on a alors P
= \\sum ~
_i=1^k+1\alpha_iP_i^2~; il nous reste
à vérifier que
P_1,\\ldots,P_k+1~
sont linéairement indépendants. Mais si \lambda_1P_1 +
\\ldots~ +
\lambda_k+1P_k+1 = 0, en considérant le coefficient de la
variable X_n (qui ne figure que dans P_k+1 avec le
coefficient 1), on a \lambda_k+1 = 0~; mais alors, comme la famille
(P_1,\\ldots,P_k~)
est libre, on a \forall~i \in [1,k], \lambda_i~ =
0, ce qui termine le traitement de ce cas.

Deuxième cas~: pour tout i \in [1,n], on a \omega_i,i = 0, mais il
existe i et j tels que i < j et
\omega_i,j\neq~0. Quitte à permuter les noms
des variables, on peut supposer que
\omega_n-1,n\neq~0. On a alors

\begin{align*}
P(X_1,\\ldots,X_n~)
= 2\\sum
_i<j\omega_i,jX_iX_j&& \%&
\\ & =&
2\omega_n-1,n\left (X_n-1 +
\sum _i=1^n-2 \omega_i,n~
\over \omega_n-1,n
X_i\right )\left (X_n
+ \sum _i=1^n-2~
\omega_i,n-1 \over \omega_n-1,n
X_i\right )\%& \\
& \text &
+Q(X_1,\\ldots,X_n-2~)
\%& \\ \end{align*}

où Q est un polynôme homogène de degré 2 en n - 2 variables. D'après
l'hypothèse de récurrence, on peut écrire Q =\
\sum ~
_i=1^k\alpha_iP_i^2 où P_i \in
K[X_1,\\ldots,X_n-2~]
\subset~
K[X_1,\\ldots,X_n~],
les P_i étant homogènes de degré 1 et linéairement
indépendants, les \alpha_i étant non nuls. Utilisant l'identité 2ab
= 1 \over 2 \left ((a +
b)^2 - (a - b)^2\right ), on obtient
alors P = \\sum ~
_i=1^k+2\alpha_iP_i^2 avec
\alpha_k+1 = -\alpha_k+2 = \omega_n-1,n
\over 2 ,

P_k+1 = X_n-1 + X_n +
\sum _i=1^n-2 \omega_i,n~ +
\omega_i,n-1 \over \omega_n-1,n X_i

et

P_k+2 = X_n-1 - X_n +
\sum _i=1^n-2 \omega_i,n~ -
\omega_i,n-1 \over \omega_n-1,n X_i

Il nous reste à vérifier que
P_1,\\ldots,P_k+2~
sont linéairement indépendants. Si \lambda_1P_1 +
\\ldots~ +
\lambda_k+2P_k+2 = 0, en considérant le coefficient de la
variable X_n-1 (qui ne figure que dans P_k+1 et
P_k+2), on a \lambda_k+1 + \lambda_k+2 = 0 et en
considérant le coefficient de la variable X_n (qui ne figure
que dans P_k+1 et P_k+2), on a \lambda_k+1 -
\lambda_k+2 = 0~; on a donc \lambda_k+1 = \lambda_k+2 = 0~;
mais alors, comme la famille
(P_1,\\ldots,P_k~)
est libre, on a \forall~i \in [1,k], \lambda_i~ =
0, ce qui termine le traitement de ce cas.

Troisième cas~: \Phi est la forme quadratique nulle, et il n'y a rien à
faire.

Ceci achève la description de l'algorithme de Gauss.

[
[
[
[

\end{document}

% \documentclass[]{article}
\usepackage[T1]{fontenc}
\usepackage{lmodern}
\usepackage{amssymb,amsmath}
\usepackage{ifxetex,ifluatex}
\usepackage{fixltx2e} % provides \textsubscript
% use upquote if available, for straight quotes in verbatim environments
\IfFileExists{upquote.sty}{\usepackage{upquote}}{}
\ifnum 0\ifxetex 1\fi\ifluatex 1\fi=0 % if pdftex
  \usepackage[utf8]{inputenc}
\else % if luatex or xelatex
  \ifxetex
    \usepackage{mathspec}
    \usepackage{xltxtra,xunicode}
  \else
    \usepackage{fontspec}
  \fi
  \defaultfontfeatures{Mapping=tex-text,Scale=MatchLowercase}
  \newcommand{\euro}{€}
\fi
% use microtype if available
\IfFileExists{microtype.sty}{\usepackage{microtype}}{}
\ifxetex
  \usepackage[setpagesize=false, % page size defined by xetex
              unicode=false, % unicode breaks when used with xetex
              xetex]{hyperref}
\else
  \usepackage[unicode=true]{hyperref}
\fi
\hypersetup{breaklinks=true,
            bookmarks=true,
            pdfauthor={},
            pdftitle={Formes quadratiques reelles},
            colorlinks=true,
            citecolor=blue,
            urlcolor=blue,
            linkcolor=magenta,
            pdfborder={0 0 0}}
\urlstyle{same}  % don't use monospace font for urls
\setlength{\parindent}{0pt}
\setlength{\parskip}{6pt plus 2pt minus 1pt}
\setlength{\emergencystretch}{3em}  % prevent overfull lines
\setcounter{secnumdepth}{0}
 
/* start css.sty */
.cmr-5{font-size:50%;}
.cmr-7{font-size:70%;}
.cmmi-5{font-size:50%;font-style: italic;}
.cmmi-7{font-size:70%;font-style: italic;}
.cmmi-10{font-style: italic;}
.cmsy-5{font-size:50%;}
.cmsy-7{font-size:70%;}
.cmex-7{font-size:70%;}
.cmex-7x-x-71{font-size:49%;}
.msbm-7{font-size:70%;}
.cmtt-10{font-family: monospace;}
.cmti-10{ font-style: italic;}
.cmbx-10{ font-weight: bold;}
.cmr-17x-x-120{font-size:204%;}
.cmsl-10{font-style: oblique;}
.cmti-7x-x-71{font-size:49%; font-style: italic;}
.cmbxti-10{ font-weight: bold; font-style: italic;}
p.noindent { text-indent: 0em }
td p.noindent { text-indent: 0em; margin-top:0em; }
p.nopar { text-indent: 0em; }
p.indent{ text-indent: 1.5em }
@media print {div.crosslinks {visibility:hidden;}}
a img { border-top: 0; border-left: 0; border-right: 0; }
center { margin-top:1em; margin-bottom:1em; }
td center { margin-top:0em; margin-bottom:0em; }
.Canvas { position:relative; }
li p.indent { text-indent: 0em }
.enumerate1 {list-style-type:decimal;}
.enumerate2 {list-style-type:lower-alpha;}
.enumerate3 {list-style-type:lower-roman;}
.enumerate4 {list-style-type:upper-alpha;}
div.newtheorem { margin-bottom: 2em; margin-top: 2em;}
.obeylines-h,.obeylines-v {white-space: nowrap; }
div.obeylines-v p { margin-top:0; margin-bottom:0; }
.overline{ text-decoration:overline; }
.overline img{ border-top: 1px solid black; }
td.displaylines {text-align:center; white-space:nowrap;}
.centerline {text-align:center;}
.rightline {text-align:right;}
div.verbatim {font-family: monospace; white-space: nowrap; text-align:left; clear:both; }
.fbox {padding-left:3.0pt; padding-right:3.0pt; text-indent:0pt; border:solid black 0.4pt; }
div.fbox {display:table}
div.center div.fbox {text-align:center; clear:both; padding-left:3.0pt; padding-right:3.0pt; text-indent:0pt; border:solid black 0.4pt; }
div.minipage{width:100%;}
div.center, div.center div.center {text-align: center; margin-left:1em; margin-right:1em;}
div.center div {text-align: left;}
div.flushright, div.flushright div.flushright {text-align: right;}
div.flushright div {text-align: left;}
div.flushleft {text-align: left;}
.underline{ text-decoration:underline; }
.underline img{ border-bottom: 1px solid black; margin-bottom:1pt; }
.framebox-c, .framebox-l, .framebox-r { padding-left:3.0pt; padding-right:3.0pt; text-indent:0pt; border:solid black 0.4pt; }
.framebox-c {text-align:center;}
.framebox-l {text-align:left;}
.framebox-r {text-align:right;}
span.thank-mark{ vertical-align: super }
span.footnote-mark sup.textsuperscript, span.footnote-mark a sup.textsuperscript{ font-size:80%; }
div.tabular, div.center div.tabular {text-align: center; margin-top:0.5em; margin-bottom:0.5em; }
table.tabular td p{margin-top:0em;}
table.tabular {margin-left: auto; margin-right: auto;}
div.td00{ margin-left:0pt; margin-right:0pt; }
div.td01{ margin-left:0pt; margin-right:5pt; }
div.td10{ margin-left:5pt; margin-right:0pt; }
div.td11{ margin-left:5pt; margin-right:5pt; }
table[rules] {border-left:solid black 0.4pt; border-right:solid black 0.4pt; }
td.td00{ padding-left:0pt; padding-right:0pt; }
td.td01{ padding-left:0pt; padding-right:5pt; }
td.td10{ padding-left:5pt; padding-right:0pt; }
td.td11{ padding-left:5pt; padding-right:5pt; }
table[rules] {border-left:solid black 0.4pt; border-right:solid black 0.4pt; }
.hline hr, .cline hr{ height : 1px; margin:0px; }
.tabbing-right {text-align:right;}
span.TEX {letter-spacing: -0.125em; }
span.TEX span.E{ position:relative;top:0.5ex;left:-0.0417em;}
a span.TEX span.E {text-decoration: none; }
span.LATEX span.A{ position:relative; top:-0.5ex; left:-0.4em; font-size:85%;}
span.LATEX span.TEX{ position:relative; left: -0.4em; }
div.float img, div.float .caption {text-align:center;}
div.figure img, div.figure .caption {text-align:center;}
.marginpar {width:20%; float:right; text-align:left; margin-left:auto; margin-top:0.5em; font-size:85%; text-decoration:underline;}
.marginpar p{margin-top:0.4em; margin-bottom:0.4em;}
.equation td{text-align:center; vertical-align:middle; }
td.eq-no{ width:5%; }
table.equation { width:100%; } 
div.math-display, div.par-math-display{text-align:center;}
math .texttt { font-family: monospace; }
math .textit { font-style: italic; }
math .textsl { font-style: oblique; }
math .textsf { font-family: sans-serif; }
math .textbf { font-weight: bold; }
.partToc a, .partToc, .likepartToc a, .likepartToc {line-height: 200%; font-weight:bold; font-size:110%;}
.chapterToc a, .chapterToc, .likechapterToc a, .likechapterToc, .appendixToc a, .appendixToc {line-height: 200%; font-weight:bold;}
.index-item, .index-subitem, .index-subsubitem {display:block}
.caption td.id{font-weight: bold; white-space: nowrap; }
table.caption {text-align:center;}
h1.partHead{text-align: center}
p.bibitem { text-indent: -2em; margin-left: 2em; margin-top:0.6em; margin-bottom:0.6em; }
p.bibitem-p { text-indent: 0em; margin-left: 2em; margin-top:0.6em; margin-bottom:0.6em; }
.paragraphHead, .likeparagraphHead { margin-top:2em; font-weight: bold;}
.subparagraphHead, .likesubparagraphHead { font-weight: bold;}
.quote {margin-bottom:0.25em; margin-top:0.25em; margin-left:1em; margin-right:1em; text-align:justify;}
.verse{white-space:nowrap; margin-left:2em}
div.maketitle {text-align:center;}
h2.titleHead{text-align:center;}
div.maketitle{ margin-bottom: 2em; }
div.author, div.date {text-align:center;}
div.thanks{text-align:left; margin-left:10%; font-size:85%; font-style:italic; }
div.author{white-space: nowrap;}
.quotation {margin-bottom:0.25em; margin-top:0.25em; margin-left:1em; }
h1.partHead{text-align: center}
.sectionToc, .likesectionToc {margin-left:2em;}
.subsectionToc, .likesubsectionToc {margin-left:4em;}
.subsubsectionToc, .likesubsubsectionToc {margin-left:6em;}
.frenchb-nbsp{font-size:75%;}
.frenchb-thinspace{font-size:75%;}
.figure img.graphics {margin-left:10%;}
/* end css.sty */

\title{Formes quadratiques reelles}
\author{}
\date{}

\begin{document}
\maketitle

\textbf{Warning: 
requires JavaScript to process the mathematics on this page.\\ If your
browser supports JavaScript, be sure it is enabled.}

\begin{center}\rule{3in}{0.4pt}\end{center}

[
[
[]
[

\subsubsection{12.4 Formes quadratiques réelles}

\paragraph{12.4.1 Formes positives, négatives}

Définition~12.4.1 Soit E un \mathbb{R}~ espace vectoriel et \Phi une forme
quadratique sur E. On dit que \Phi est positive (resp. négative) si
\forall~~x \in E, \Phi(x) ≥ 0 (resp. \leq 0).

Remarque~12.4.1 On déduit de la définition précédente que \Phi est à la
fois définie et positive si et seulement
si~\forall~x\mathrel\neq~~0, \Phi(x)
> 0.

Notons aussi la proposition suivante

Proposition~12.4.1 Soit E un \mathbb{R}~ espace vectoriel et \Phi une forme
quadratique définie sur E. Alors \Phi est soit positive, soit négative.

Démonstration Supposons que \Phi n'est ni positive, ni négative. Soit x \in E
tel que \Phi(x) < 0 (\Phi n'est pas positive) et y \in E tel que \Phi(y)
> 0 (\Phi n'est pas négative). Alors la famille (x,y) est
libre~: sinon il existerait par exemple \lambda~ \in \mathbb{R}~ tel que y = \lambda~x et on
aurait \Phi(y) = \lambda~^2\Phi(x) \leq 0. Pour t \in [0,1], posons f(t) =
\Phi((1 - t)x + ty)~; l'identité de polarisation montre que f est un
polynôme du second degré en t et l'on a f(0) = \Phi(x) < 0, f(1)
= \Phi(y) > 0. Le théorème des valeurs intermédiaires assure
qu'il existe t_0 \in [0,1] tel que f(t_0) = 0, soit
\Phi((1 - t_0)x + t_0y) = 0~; mais comme la famille (x,y)
est libre, on a (1 - t_0)x +
t_0y\neq~0, ce qui montre que \Phi n'est
pas définie.

Remarque~12.4.2 Un raisonnement similaire montre que si E est un
\mathbb{C}-espace vectoriel ~de dimension supérieure ou égale à 2, il ne peut pas
exister de forme quadratique définie sur E.

\paragraph{12.4.2 Bases de Sylvester. Signature}

Soit E un \mathbb{R}~ espace vectoriel de dimension finie et \Phi une forme
quadratique sur E. Soit \mathcal{E} une base orthogonale de E et \Omega
= \mathrmMat~ (\Phi,\mathcal{E})
=\
\mathrmdiag(\alpha_1,\\ldots,\alpha_n~)
la matrice de \Phi dans cette base (avec donc \alpha_i =
\Phi(e_i)). Quitte à permuter les vecteurs de la base, on peut
supposer que \alpha_1 >
0,\\ldots,\alpha_p~
> 0,\alpha_p+1 <
0,\\ldots,\alpha_p+q~
< 0,\alpha_p+q+1 =
\\ldots~ =
\alpha_n = 0 avec p ≥ 0,q ≥ 0,p + q \leq n.

Théorème~12.4.2 (inertie de Sylvester). Les entiers p et q sont
indépendants de la base orthogonale choisie~: l'entier p (resp. q) est
la dimension maximale des sous-espaces F de E tels que la restriction de
\Phi à F soit définie positive (resp. définie négative).

Démonstration Pour x
\in\mathrmVect(e_1,\\\ldots,e_p~),
on a \Phi(x) = \\sum ~
_i=1^p\alpha_ix_i^2 ≥ 0 avec égalité
si et seulement si~\forall~i, x_i~ = 0 soit x
= 0. Ceci montre que la restriction de \Phi à
\mathrmVect(e_1,\\\ldots,e_p~)
est définie positive. Soit maintenant F un sous-espace de E tel que la
restriction de \Phi à F soit définie positive. Pour x
\in\mathrmVect(e_p+1,\\\ldots,e_n~),
on a \Phi(x) = \\sum ~
_i=p+1^p+q\alpha_ix_i^2 \leq 0. Pour x \in
F \diagdown\0\ on a \Phi(x) > 0. On
en déduit que F
\bigcap\mathrmVect(e_p+1,\\\ldots,e_n~)
= \0\ et donc ces deux sous-espaces
sont en somme directe. En conséquence, dim~ F
+ dim~
\mathrmVect(e_p+1,\\\ldots,e_n~)
\leq n, soit encore dim~ F + n - p \leq n, d'où
dim~ F \leq p. Donc p est la dimension maximale
des sous-espaces F de E tels que la restriction de \Phi à F soit définie
positive. Le raisonnement est similaire pour l'entier q.

Définition~12.4.2 Le couple (p,q) est appelé la signature de la forme
quadratique \Phi. On a p + q =\
\mathrmrg\Phi.

Remarque~12.4.3 \Phi est positive (resp. définie positive) si et seulement
si~elle est de signature (p,0) (resp. (n,0)).

Reprenons alors notre base orthogonale \mathcal{E} avec \alpha_1
>
0,\\ldots,\alpha_p~
> 0,\alpha_p+1 <
0,\\ldots,\alpha_p+q~
< 0,\alpha_p+q+1 =
\\ldots~ =
\alpha_n = 0. Pour i \in [1,p + q] posons \epsilon_i = 1
\over \sqrt\alpha_i 
 e_i et pour i \in [p + q + 1,n], \epsilon_i
= e_i. Nous obtenons alors une nouvelle base orthogonale \mathcal{E}' de
E telle que

\Phi(\epsilon_1) =
\\ldots~ =
\Phi(\epsilon_p) = 1,\Phi(\epsilon_p+1) =
\\ldots~ =
\Phi(\epsilon_p+q) = -1

\Phi(\epsilon_p+q+1) =
\\ldots~ =
\Phi(\epsilon_n) = 0

Définition~12.4.3 Une telle base orthogonale sera appelée une base de
Sylvester de E.

Par définition même, la matrice de \Phi dans une base de Sylvester est la
matrice \left
(\matrix\,I_p&0 &0
\cr 0 &-I_q&0 \cr 0 &0
&0\right ).

Remarque~12.4.4 Bien entendu, si \Phi est définie positive, une base de
Sylvester est simplement une base orthonormée.

\paragraph{12.4.3 Inégalités}

Théorème~12.4.3 (inégalité de Schwarz). Soit E un \mathbb{R}~ espace vectoriel et
\Phi une forme quadratique positive sur E de forme polaire \phi. Alors

\forall~x,y \in E, \phi(x,y)^2~ \leq \Phi(x)\Phi(y)

Si \Phi est en plus définie, alors il y a égalité si et seulement si~la
famille (x,y) est liée.

Démonstration On écrit \forall~~t \in \mathbb{R}~, \Phi(x + ty) ≥ 0,
soit encore t^2\Phi(y) + 2t\phi(x,y) + \Phi(x) ≥ 0. Ce trinome de
degré inférieur ou égal à 2 doit donc avoir un discriminant réduit
négatif, soit \phi(x,y)^2 - \Phi(x)\Phi(y) \leq 0. Supposons que \Phi est
définie~; si on a l'égalité, deux cas sont possibles. Soit y = 0 auquel
cas la famille (x,y) est liée, soit \Phi(y)\neq~0~;
mais dans ce cas ce trinome en t a une racine double t_0, et
donc \Phi(x + t_0y) = 0 d'où x + t_0y = 0 et donc la
famille est liée. Inversement, si la famille (x,y) est liée, on a par
exemple x = \lambda~y et dans ce cas \phi(x,y)^2 =
\lambda~^2\Phi(y)^2 = \Phi(x)\Phi(y).

Corollaire~12.4.4 Soit E un \mathbb{R}~ espace vectoriel et \Phi une forme
quadratique positive sur E de forme polaire \phi. Alors le noyau de \Phi est
l'ensemble des vecteurs isotropes pour \Phi. En particulier, \Phi est non
dégénérée si et seulement si~elle est définie.

Démonstration Tout vecteur du noyau est bien entendu isotrope.
Inversement, si x est un vecteur isotrope et si y \in E, alors 0 \leq
\phi(x,y)^2 \leq \Phi(x)\Phi(y) = 0, soit \phi(x,y) = 0 et donc x est dans
le noyau de \phi.

Théorème~12.4.5 (inégalité de Minkowski). Soit E un \mathbb{R}~ espace vectoriel
et \Phi une forme quadratique positive sur E. Alors

\forall~x,y \in E, \sqrt\Phi(x + y)~
\leq\sqrt\Phi(x) + \sqrt\Phi(y)

Si \Phi est en plus définie, alors il y a égalité si et seulement si~la
famille (x,y) est positivement liée.

Démonstration On a

\begin{align*} \Phi(x + y)& = \Phi(x) + 2\phi(x,y) + \Phi(y) \leq
\Phi(x) + 2\phi(x,y) + \Phi(y)& \%&
\\ & \leq \Phi(x) +
2\sqrt\Phi(x)\Phi(y) + \Phi(y) = \left
(\sqrt\Phi(x) +
\sqrt\Phi(y)\right )^2& \%&
\\ \end{align*}

d'où \sqrt\Phi(x + y) \leq\sqrt\Phi(x) +
\sqrt\Phi(y). Si \Phi est définie, l'égalité nécessite à la
fois que \phi(x,y) = \sqrt\Phi(x)\Phi(y),
donc que (x,y) soit liée, et que \phi(x,y) ≥ 0, soit que le coefficient de
proportionnalité soit positif.

\paragraph{12.4.4 Espaces préhilbertiens réels}

Définition~12.4.4 On appelle espace préhilbertien réel un couple (E,\Phi)
d'un \mathbb{R}~-espace vectoriel ~E et d'une forme quadratique définie positive
sur E.

Théorème~12.4.6 Soit (E,\Phi) un espace préhilbertien réel. Alors
l'application x\mapsto~\sqrt\Phi(x)
est une norme sur E appelée norme euclidienne.

Démonstration La propriété de séparation provient du fait que \Phi est
définie. L'homogénéité provient de l'homogénéité de la forme
quadratique. Quant à l'inégalité triangulaire, ce n'est autre que
l'inégalité de Minkowski.

Définition~12.4.5 On notera (x∣y) = \phi(x,y) et
\x\^2 =
(x∣x) = \Phi(x)

Théorème~12.4.7 Soit E un espace préhilbertien réel et F un sous-espace
vectoriel de dimension finie de E. Alors l'orthogonal F^\bot de
F dans E est un supplémentaire de F, appelé le supplémentaire orthogonal
de F. La projection sur F parallèlement à F^\bot est appelée la
projection orthogonale sur F. On a
codimF^\bot~ =\
dim F et F^\bot\bot = F.

Démonstration Tout d'abord, si x \in F \bigcap F^\bot, on a x \bot x et
donc (x∣x) = 0 ce qui implique x = 0~; on a
donc F \bigcap F^\bot = \0\. Soit
maintenant x \in E et soit f : F \rightarrow~ \mathbb{R}~ définie par f(y) =
(x∣y). Clairement, f est une forme linéaire
sur l'espace F~; comme F est un espace vectoriel de dimension finie muni
d'une forme bilinéaire symétrique non dégénérée, il existe x_1
\in F tel que \forall~~y \in F, f(y) =
(x_1∣y). On a donc
\forall~y \in F, (x\mathrel∣~y) =
(x_1∣y) et donc
\forall~~y \in F, (x -
x_1∣y) = 0. On en déduit que x -
x_1 \in F^\bot. Comme x = x_1 + (x -
x_1), on a bien E = F + F^\bot. On en déduit que E = F
\oplus~ F^\bot, et donc que
codimF^\bot~ =\
dim F.

On a clairement F \subset~ F^\bot\bot. Inversement, soit x \in
F^\bot\bot et écrivons x = x_1 + x_2 avec
x_1 \in F et x_2 \in F^\bot~; comme x_2 \in
F^\bot et x \in F^\bot\bot, on a
(x_2∣x) = 0 soit encore
(x_2∣x_1) +
(x_2∣x_2) = 0, soit encore,
compte tenu de (x_2∣x_1) =
0, (x_2∣x_2) = 0 et donc
x_2 = 0~; ceci nous montre que x \in F, soit encore
F^\bot\bot\subset~ F et donc F^\bot\bot = F.

Théorème~12.4.8 Soit E un espace préhilbertien réel et F un sous-espace
vectoriel de dimension finie de E. Soit x \in E. Il existe un unique v \in F
tel que d(x,F) =\ x -
v\~; v est la projection orthogonale de x sur
F.

Démonstration Ecrivons x = v + w avec v \in F et y \in F^\bot, et
donc v = p_F(x). Pour y \in F, on a, en tenant compte de v - y \in
F et w \in F^\bot qui impliquent que v - y \bot w,

\begin{align*} \x -
y\^2& =&
\(v - y) +
w\^2 =\ v -
y\^2 +\
w\^2\%&
\\ & =& \v -
y\^2 +\ x -
v\^2 ≥\ x -
v\^2 \%&
\\ \end{align*}

avec égalité si et seulement si~y = v, ce qui démontre la première
partie du résultat.

Proposition~12.4.9 Si
(v_1,\\ldots,v_p~)
est une base de F, on a

d(x,F)^2 =
\mathrm{det}~
Gram(v_1,\\\ldots,v_p~,x)
\over
\mathrm{det}~
Gram(v_1,\\\ldots,v_p)~

Démonstration Si x \in F, la formule est évidente puisque les deux membres
de la formule sont nuls (la famille
(v_1,\\ldots,v_p~,x)
étant liée, son déterminant de Gram est nul). On a

\begin{align*}
\mathrm{det}~
Gram(v_1,\\\ldots,v_p~,x)&&
\%& \\ & =& \left
\matrix\,Gram(v_1,\\\ldots,v_p)&\matrix\,(v_1\mathrel∣~x)
\cr \⋮~
\cr (v_p∣x)
\cr
\matrix\,(v_1∣x)&\\ldots&(v_p\mathrel∣x)~
&\x\^2
\right  \%& \\
& =& \left
\matrix\,Gram(v_1,\\\ldots,v_p)&\matrix\,(v_1\mathrel∣~v)
\cr \⋮~
\cr (v_p∣v)
\cr
\matrix\,(v_1∣v)&\\ldots&(v_p\mathrel∣v)~
&\v\^2
+\
w\^2\right
 \%& \\ & =&
\left
\matrix\,Gram(v_1,\\\ldots,v_p)&\matrix\,(v_1\mathrel∣~v)
\cr \⋮~
\cr (v_p∣v)
\cr
\matrix\,(v_1∣v)&\\ldots&(v_p\mathrel∣v)~
&\v\^2
\right  + \left
\matrix\,Gram(v_1,\\\ldots,v_p~)&\matrix\,0
\cr \⋮~
\cr 0 \cr
\matrix\,(v_1∣v)&\\ldots&(v_p\mathrel∣v)~
&\w\^2\right
 \%& \\ & =&
\mathrm{det}~
Gram(v_1,\\\ldots,v_p~,v)
+\
w\^2\
\mathrm{det} Gram(v_
1,\\ldots,v_p~)\%&
\\ & =&
\w\^2\
\mathrm{det} Gram(v_
1,\\ldots,v_p~)
\%& \\ & =&
\mathrm{det}~
Gram(v_1,\\\ldots,v_p)d(x,F)^2~
\%& \\ \end{align*}

en remarquant que (v_i∣x) =
(v_i∣v),
\x\^2
=\ v\^2
+\ w\^2,
que v est une combinaison linéaire de
(v_1,\\ldots,v_p~)
ce qui implique que
\mathrm{det}~
Gram(v_1,\\\ldots,v_p~,v)
= 0 et en utilisant la linéarité du déterminant par rapport à sa
dernière colonne. Ceci démontre que

d(x,F)^2 =
\mathrm{det}~
Gram(v_1,\\\ldots,v_p~,x)
\over
\mathrm{det}~
Gram(v_1,\\\ldots,v_p)~

Remarque~12.4.5 En dimension 3, en tenant compte de divers résultats qui
expriment le déterminant de Gram en fonction du produit mixte ou de la
norme du produit vectoriel, on obtient les formules

d(x, \mathbb{R}~u) = \x ∧ u\
\over
\u\

et

d(x,\mathrmVect~(u,v)) =
\big  [u,v,x] \big
 \over \u ∧
v\

\paragraph{12.4.5 Espaces euclidiens}

Définition~12.4.6 On appelle espace euclidien un espace préhilbertien
réel de dimension finie.

Récapitulons les principales propriétés des espaces euclidiens qui
découlent presque immédiatement de tout ce que nous avons déjà vu
précédemment

Théorème~12.4.10 Soit E un espace euclidien.

\begin{itemize}
\itemsep1pt\parskip0pt\parsep0pt
\item
  (i) pour toute forme linéaire f sur E, il existe un unique vecteur
  v_f \in E tel que \forall~~x \in E, f(x) =
  (x∣v_f)
\item
  (ii) pour tout sous-espace vectoriel A de E, on a E = A \oplus~
  A^\bot et (A^\bot)^\bot = A
\end{itemize}

Démonstration (i) est une propriété générale des formes non dégénérées~;
(ii) est une propriété générale des formes définies.

\paragraph{12.4.6 Algorithme de Gram-Schmidt}

Théorème~12.4.11 (Algorithme de Gram-Schmidt). Soit E un espace
euclidien. Soit \mathcal{E} =
(e_1,\\ldots,e_n~)
une base de E. Alors il existe une base orthogonale \mathcal{E}' =
(\epsilon_1,\\ldots,\epsilon_n~)
de E vérifiant les conditions équivalentes suivantes

\begin{itemize}
\itemsep1pt\parskip0pt\parsep0pt
\item
  (i) \forall~k \in [1,n], \epsilon_k~
  \in\mathrmVect(e_1,\\\ldots,e_k~)
\item
  (ii) \forall~~k \in [1,n],
  \mathrmVect(\epsilon_1,\\\ldots,\epsilon_k~)
  =\
  \mathrmVect(e_1,\\ldots,e_k~)
\item
  (iii) la matrice de passage de \mathcal{E} à \mathcal{E}' est triangulaire supérieure
\end{itemize}

Si \mathcal{E}' =
(\epsilon_1,\\ldots,\epsilon_n~)
et \mathcal{E}'' =
(\eta_1,\\ldots,\eta_n~)
sont deux telles bases orthogonales, il existe des scalaires
\lambda_1,\\ldots,\lambda_n~
non nuls tels que \forall~i \in [1,n], \eta_i~
= \lambda_i\epsilon_i.

Démonstration Démontrons tout d'abord l'équivalence des trois
propriétés. Il est clair que (i) \Leftrightarrow (iii) et
que (ii) \rigtharrow~(i). De plus, si (i) est vérifié, on a
\forall~i \in [1,k], \epsilon_i~
\in\mathrmVect(e_1,\\\ldots,e_i~)
\subset~\mathrmVect(e_1,\\\ldots,e_k~)
et donc
\mathrmVect(\epsilon_1,\\\ldots,\epsilon_k~)
\subset~\mathrmVect(e_1,\\\ldots,e_k~).
Comme les deux sous-espaces ont même dimension, ils sont égaux et donc
(i) \rigtharrow~(ii). Nous allons maintenant démontrer l'existence et l'unicité à
multiplication par des scalaires non nuls près. Posons V _0 =
\0\, V _k
=\
\mathrmVect(e_1,\\ldots,e_k~)
et remarquons que le fait que la base soit orthogonale se traduit par le
fait que \epsilon_k est orthogonal à
\mathrmVect(\epsilon_1,\\\ldots,\epsilon_k-1~),
soit encore d'après (ii) à
\mathrmVect(e_1,\\\ldots,e_k-1~)
= V _k-1. On voit donc que l'on doit choisir \epsilon_k \in V
_k \bigcap V _k-1^\bot. Mais ce sous-espace n'est autre
que l'orthogonal dans V _k du sous espace V _k-1~; cet
orthogonal est de dimension k - (k - 1) = 1~; ceci démontre déjà que si
\epsilon_k et \eta_k conviennent, alors ils sont proportionnels,
d'où la partie unicité de la proposition. Pour l'existence, choisissons
pour chaque k un vecteur non nul \epsilon_k \in V _k \bigcap V
_k-1^\bot. On a alors, puisque V _k-1 est non
isotrope, V _k = V _k-1 \oplus~ K\epsilon_k~; une
récurrence évidente montre alors que V _k
=\
\mathrmVect(\epsilon_1,\\ldots,\epsilon_k~).
Donc
(\epsilon_1,\\ldots,\epsilon_n~)
est une base de E (famille génératrice de cardinal n), évidemment
orthogonale et qui vérifie les conditions voulues.

Remarque~12.4.6 Comme on l'a vu ci-dessus, la base orthogonale n'est pas
unique~; il y a divers moyens de la normaliser. L'un des plus simples
est de demander que \epsilon_k ait une coordonnée égale à 1 suivant
e_k (cette coordonnée est bien évidemment non nulle car V
_k = V _k-1 \oplus~ Ke_k et il est exclu que
\epsilon_k appartienne à V _k-1)~; dans ce cas la base est
évidemment unique. Une autre normalisation possible est de demander que
la nouvelle base soit orthonormée et que les produits scalaires
(e_i∣\epsilon_i) soient tous
positifs.

Nous allons maintenant étudier un algorithme de construction de la base
\mathcal{E}' dans le cadre de cette normalisation. Il se fonde sur la remarque
suivante~: pour chaque k, on doit avoir \epsilon_k = e_k +
v_k avec v_k \in V _k-1
=\
\mathrmVect(e_1,\\ldots,e_k-1~)
=\
\mathrmVect(\epsilon_1,\\ldots,\epsilon_k-1~)~;
ceci impose que,
\epsilon_1,\\ldots,\epsilon_k-1~
étant supposés déjà déterminés, \epsilon_k = e_k
+ \\sum ~
_i=1^k-1\alpha_i\epsilon_i, les \alpha_i étant
déterminés par la condition que \epsilon_k doit être orthogonal à
\epsilon_1,\\ldots,\epsilon_k-1~~;
or , pour j \in [1,k - 1],

(\epsilon_k∣\epsilon_j) =
(e_k∣\epsilon_j) +
\sum _i=1^k-1\alpha~_
i(\epsilon_i∣\epsilon_j) =
(e_k∣\epsilon_j) +
\alpha_j\\epsilon_j\^2

puisque (\epsilon_i∣\epsilon_j) = 0 si
i\neq~j. On doit donc poser \alpha_j = -
(e_k∣\epsilon_j)
\over
\\epsilon_j\^2
.

On obtient donc l'algorithme suivant (pour la première normalisation, où
l'on demande que \epsilon_k ait une coordonnée égale à 1 suivant
e_k)

Algorithme de Gram-Schmidt

\begin{itemize}
\itemsep1pt\parskip0pt\parsep0pt
\item
  \epsilon_1 = e_1
\item
  pour k de 2 à n faire
\item
  \quad \quad \epsilon_k = e_k
  -\\sum ~
  _i=1^k-1
  (e_k∣\epsilon_i)
  \over
  (\epsilon_i∣\epsilon_i) \epsilon_i
\end{itemize}

Pour la deuxième normalisation, qui demande que la base soit
orthonormée, on a Algorithme de Gram-Schmidt

\begin{itemize}
\itemsep1pt\parskip0pt\parsep0pt
\item
  \epsilon_1 = e_1\over
  \e_1\
\item
  pour k de 2 à n faire
\item
  \quad \quad \epsilon'_k = e_k
  -\\sum ~
  _i=1^k-1(e_k∣\epsilon_i)\epsilon_i
\item
  \quad \quad \epsilon_k =
  \epsilon_k'\over
  \\epsilon_k'\
\end{itemize}

\paragraph{12.4.7 Application~: polynômes orthogonaux}

Soit -\infty~\leq a < b \leq +\infty~ et \omega :]a,b[\rightarrow~ \mathbb{R}~ une application
continue positive non nulle telle que pour tout polynôme P \in \mathbb{R}~[X] la
fonction P\omega soit intégrable sur ]a,b[.

Théorème~12.4.12 La forme bilinéaire (P∣Q)
=\int  _]a,b[~P(t)Q(t)\omega(t) dt est
définie positive sur \mathbb{R}~[X].

Démonstration On a (P∣P)
=\int  _]a,b[P(t)^2~\omega(t)
dt ≥ 0. De plus, si (P∣P) = 0, comme
P(t)^2\omega(t) est continue positive, on a
\forall~t \in]a,b[, P(t)^2~\omega(t) = 0. Mais
comme \omega est non nulle, il existe un intervalle ]c,d[ sur lequel \omega ne
s'annule pas. On a donc \forall~~t \in]c,d[, P(t) =
0 et le polynôme P ayant une infinité de racines est le polynôme nul.

Nous pouvons donc appliquer l'algorithme de Gram-Schmidt à la base
(X^n)_n\in\mathbb{N}~ de \mathbb{R}~[X] et on obtient donc

Théorème~12.4.13 Il existe une unique famille
(P_n)_n\in\mathbb{N}~ de \mathbb{R}~[X] vérifiant les conditions
suivantes

\begin{itemize}
\itemsep1pt\parskip0pt\parsep0pt
\item
  (i) pour tout n, P_n est un polynôme normalisé de degré n
\item
  (ii) \forall~~i,j \in \mathbb{N}~,
  i\neq~j \rigtharrow~
  (P_i∣P_j) = 0
\end{itemize}

Définition~12.4.7 Les polynômes P_n sont appelés les polynômes
orthogonaux relativement au poids \omega.

Remarque~12.4.7 Pour chaque n \in \mathbb{N}~,
(P_0,\\ldots,P_n~)
est une base de l'espace vectoriel \mathbb{R}_n[X] des polynômes de
degré inférieur ou égal à n, puisque c'est une famille libre (échelonnée
en degrés) de cardinal n + 1. On a bien entendu P_n+1 \bot
\mathbb{R}_n[X].

Théorème~12.4.14 Le polynôme P_n a toutes ses racines réelles
distinctes situées dans l'intervalle ]a,b[.

Démonstration Soit
\alpha_1,\\ldots,\alpha_k~
les racines de P de multiplicités impaires situées dans l'intervalle
]a,b[ (avec k ≥ 0). Soit Q(X) =\
∏  _i=1^k(X - \alpha_i~).
Supposons que k \leq n - 1~; alors Q \in \mathbb{R}_n-1[X] et donc
(Q∣P_n) = 0, soit
\int  _]a,b[~P(t)Q(t)\omega(t) dt = 0.
Mais le polynôme PQ n'a que des racines de multiplicités paires sur
]a,b[, il est donc de signe constant et donc PQ\omega également. On en
déduit que PQ\omega = 0 sur ]a,b[, puis comme précédemment que PQ = 0, ce
qui est absurde. Donc k = n, ce qui exige que les \alpha_i soient de
multiplicités 1 et que P(X) =\
∏  _i=1^n(X - \alpha_i~).

Application à l'intégration.

Théorème~12.4.15 Les racines de P_n sont les seuls réels
\alpha_1,\\ldots,\alpha_n~
tels qu'il existe des scalaires
\lambda_1,\\ldots,\lambda_n~
vérifiant

\forall~P \in \mathbb{R}_2n-1~[X],
\int  _]a,b[~P(t)\omega(t) dt =
\sum _i=1^n\lambda~_
iP(\alpha_i)

Démonstration Soit
\alpha_1,\\ldots,\alpha_n~
les racines de P_n et \epsilon_i la forme linéaire sur
\mathbb{R}_n-1[X], P\mapsto~P(\alpha_i)~; si
\forall~i, \epsilon_i~(P) = 0, P est un polynôme de
degré au plus n - 1 qui admet n racines distinctes, il est donc nul.
Donc
P\mapsto~(\epsilon_1(P),\\ldots,\epsilon_n~(P))
est injective, donc bijective, ce qui implique
\mathrmVect(\epsilon_1,\\\ldots,\epsilon_n~)
= \mathbb{R}_n-1[X]^∗. Comme f :
P\mapsto~\int ~
_]a,b[P(t)\omega(t) dt est une forme linéaire sur
\mathbb{R}_n-1[X], elle est combinaison linéaire de
\epsilon_1,\\ldots,\epsilon_n~,
soit f = \lambda_1\epsilon_1 +
\\ldots~ +
\lambda_n\epsilon_n. On a alors

\forall~P \in \mathbb{R}_n-1~[X],
\int  _]a,b[~P(t)\omega(t) dt =
\sum _i=1^n\lambda~_
iP(\alpha_i)

Soit maintenant P \in \mathbb{R}_2n-1[X]. on peut écrire P =
QP_n + R avec Q,R \in \mathbb{R}_n-1[X]. On a bien entendu
P(\alpha_i) = R(\alpha_i) et

\begin{align*} \int ~
_]a,b[P(t)\omega(t) dt& =&
(Q∣P_n) +\\int
 _]a,b[R(t)\omega(t) dt =\int ~
_]a,b[R(t)\omega(t) dt\%& \\ & =&
\sum _i=1^n\lambda~_
iR(\alpha_i) = \\sum
_i=1^n\lambda_ iP(\alpha_i) \%&
\\ \end{align*}

ce qui montre que
\alpha_1,\\ldots,\alpha_n~
vérifient les conditions voulues. Inversement, supposons que
\alpha_1,\\ldots,\alpha_n~
sont n nombres réels vérifiant les conditions voulues et Q(X)
= \∏ ~
_i=1^n(X - \alpha_i). Alors, pour tout P \in
\mathbb{R}_n-1[X], on a PQ \in \mathbb{R}_2n-1[X] soit

(P∣Q) =\int ~
_]a,b[P(t)Q(t)\omega(t) dt = \\sum
_i=1^n\lambda_ iP(\alpha_i)Q(\alpha_i) = 0

et donc Q \bot R_n-1[X]. Comme Q est normalisé, on a Q =
P_n.

Théorème~12.4.16 Il existe des scalaires a_n,b_n tels
que

\forall~n \in \mathbb{N}~^∗, P_ n+1~ = (X +
a_n)P_n + b_nP_n-1

Démonstration Soit Q = XP_n \in \mathbb{R}_n+1[X]~; on a donc
XP_n = \\sum ~
_i=0^n+1\lambda_iP_i(X). En considérant le
terme de degré n + 1, les polynômes P_n+1 et XP_n
étant normalisés, on a \lambda_n+1 = 1. Pour n ≥ 2 et i \leq n - 2, on a

\begin{align*}
\lambda_i\P_i\^2&
=& (Q∣P_ i)
=\int  _]a,b[P_i~(t)Q(t)\omega(t)
dt =\int ~
_]a,b[P_i(t)tP_n(t) dt\%&
\\ & =&
(P_n∣XP_i) \%&
\\ \end{align*}

Mais puisque i \leq n - 2, XP_i \in \mathbb{R}_n-1[X] et donc
XP_i \bot P_n, soit \lambda_i = 0. On a donc

XP_n = P_n+1 + \lambda_nP_n +
\lambda_n-1P_n-1

ce qui donne la formule demandée.

Pour certains poids \omega particuliers, les polynômes orthogonaux vérifient
des équations différentielles linéaires d'ordre 2 que nous n'étudierons
pas de manière générale. Citons quelques cas particulièrement importants
de polynômes orthogonaux

\begin{itemize}
\itemsep1pt\parskip0pt\parsep0pt
\item
  (i) a = -1,b = 1,\omega(t) = 1~: polynômes de Legendre L_n(t) =
  \lambda_n d^n \over dt^n
  (t^2 - 1)^n
\item
  (ii) a = -\infty~,b = +\infty~,\omega(t) = e^-t^2 ~: polynômes
  d'Hermite H_n(t)e^-t^2  =
  \lambda_n d^n \over dt^n
  e^-t^2 
\item
  (iii) a = -1,b = 1,\omega(t) = 1 \over
  \sqrt1-t^2 ~: polynômes de Tchebychev
  T_n(cos~ x) =
  \lambda_n cos~ (nx)
\end{itemize}

[
[
[
[

\end{document}

% \documentclass[]{article}
\usepackage[T1]{fontenc}
\usepackage{lmodern}
\usepackage{amssymb,amsmath}
\usepackage{ifxetex,ifluatex}
\usepackage{fixltx2e} % provides \textsubscript
% use upquote if available, for straight quotes in verbatim environments
\IfFileExists{upquote.sty}{\usepackage{upquote}}{}
\ifnum 0\ifxetex 1\fi\ifluatex 1\fi=0 % if pdftex
  \usepackage[utf8]{inputenc}
\else % if luatex or xelatex
  \ifxetex
    \usepackage{mathspec}
    \usepackage{xltxtra,xunicode}
  \else
    \usepackage{fontspec}
  \fi
  \defaultfontfeatures{Mapping=tex-text,Scale=MatchLowercase}
  \newcommand{\euro}{€}
\fi
% use microtype if available
\IfFileExists{microtype.sty}{\usepackage{microtype}}{}
\ifxetex
  \usepackage[setpagesize=false, % page size defined by xetex
              unicode=false, % unicode breaks when used with xetex
              xetex]{hyperref}
\else
  \usepackage[unicode=true]{hyperref}
\fi
\hypersetup{breaklinks=true,
            bookmarks=true,
            pdfauthor={},
            pdftitle={Endomorphismes et formes quadratiques},
            colorlinks=true,
            citecolor=blue,
            urlcolor=blue,
            linkcolor=magenta,
            pdfborder={0 0 0}}
\urlstyle{same}  % don't use monospace font for urls
\setlength{\parindent}{0pt}
\setlength{\parskip}{6pt plus 2pt minus 1pt}
\setlength{\emergencystretch}{3em}  % prevent overfull lines
\setcounter{secnumdepth}{0}
 
/* start css.sty */
.cmr-5{font-size:50%;}
.cmr-7{font-size:70%;}
.cmmi-5{font-size:50%;font-style: italic;}
.cmmi-7{font-size:70%;font-style: italic;}
.cmmi-10{font-style: italic;}
.cmsy-5{font-size:50%;}
.cmsy-7{font-size:70%;}
.cmex-7{font-size:70%;}
.cmex-7x-x-71{font-size:49%;}
.msbm-7{font-size:70%;}
.cmtt-10{font-family: monospace;}
.cmti-10{ font-style: italic;}
.cmbx-10{ font-weight: bold;}
.cmr-17x-x-120{font-size:204%;}
.cmsl-10{font-style: oblique;}
.cmti-7x-x-71{font-size:49%; font-style: italic;}
.cmbxti-10{ font-weight: bold; font-style: italic;}
p.noindent { text-indent: 0em }
td p.noindent { text-indent: 0em; margin-top:0em; }
p.nopar { text-indent: 0em; }
p.indent{ text-indent: 1.5em }
@media print {div.crosslinks {visibility:hidden;}}
a img { border-top: 0; border-left: 0; border-right: 0; }
center { margin-top:1em; margin-bottom:1em; }
td center { margin-top:0em; margin-bottom:0em; }
.Canvas { position:relative; }
li p.indent { text-indent: 0em }
.enumerate1 {list-style-type:decimal;}
.enumerate2 {list-style-type:lower-alpha;}
.enumerate3 {list-style-type:lower-roman;}
.enumerate4 {list-style-type:upper-alpha;}
div.newtheorem { margin-bottom: 2em; margin-top: 2em;}
.obeylines-h,.obeylines-v {white-space: nowrap; }
div.obeylines-v p { margin-top:0; margin-bottom:0; }
.overline{ text-decoration:overline; }
.overline img{ border-top: 1px solid black; }
td.displaylines {text-align:center; white-space:nowrap;}
.centerline {text-align:center;}
.rightline {text-align:right;}
div.verbatim {font-family: monospace; white-space: nowrap; text-align:left; clear:both; }
.fbox {padding-left:3.0pt; padding-right:3.0pt; text-indent:0pt; border:solid black 0.4pt; }
div.fbox {display:table}
div.center div.fbox {text-align:center; clear:both; padding-left:3.0pt; padding-right:3.0pt; text-indent:0pt; border:solid black 0.4pt; }
div.minipage{width:100%;}
div.center, div.center div.center {text-align: center; margin-left:1em; margin-right:1em;}
div.center div {text-align: left;}
div.flushright, div.flushright div.flushright {text-align: right;}
div.flushright div {text-align: left;}
div.flushleft {text-align: left;}
.underline{ text-decoration:underline; }
.underline img{ border-bottom: 1px solid black; margin-bottom:1pt; }
.framebox-c, .framebox-l, .framebox-r { padding-left:3.0pt; padding-right:3.0pt; text-indent:0pt; border:solid black 0.4pt; }
.framebox-c {text-align:center;}
.framebox-l {text-align:left;}
.framebox-r {text-align:right;}
span.thank-mark{ vertical-align: super }
span.footnote-mark sup.textsuperscript, span.footnote-mark a sup.textsuperscript{ font-size:80%; }
div.tabular, div.center div.tabular {text-align: center; margin-top:0.5em; margin-bottom:0.5em; }
table.tabular td p{margin-top:0em;}
table.tabular {margin-left: auto; margin-right: auto;}
div.td00{ margin-left:0pt; margin-right:0pt; }
div.td01{ margin-left:0pt; margin-right:5pt; }
div.td10{ margin-left:5pt; margin-right:0pt; }
div.td11{ margin-left:5pt; margin-right:5pt; }
table[rules] {border-left:solid black 0.4pt; border-right:solid black 0.4pt; }
td.td00{ padding-left:0pt; padding-right:0pt; }
td.td01{ padding-left:0pt; padding-right:5pt; }
td.td10{ padding-left:5pt; padding-right:0pt; }
td.td11{ padding-left:5pt; padding-right:5pt; }
table[rules] {border-left:solid black 0.4pt; border-right:solid black 0.4pt; }
.hline hr, .cline hr{ height : 1px; margin:0px; }
.tabbing-right {text-align:right;}
span.TEX {letter-spacing: -0.125em; }
span.TEX span.E{ position:relative;top:0.5ex;left:-0.0417em;}
a span.TEX span.E {text-decoration: none; }
span.LATEX span.A{ position:relative; top:-0.5ex; left:-0.4em; font-size:85%;}
span.LATEX span.TEX{ position:relative; left: -0.4em; }
div.float img, div.float .caption {text-align:center;}
div.figure img, div.figure .caption {text-align:center;}
.marginpar {width:20%; float:right; text-align:left; margin-left:auto; margin-top:0.5em; font-size:85%; text-decoration:underline;}
.marginpar p{margin-top:0.4em; margin-bottom:0.4em;}
.equation td{text-align:center; vertical-align:middle; }
td.eq-no{ width:5%; }
table.equation { width:100%; } 
div.math-display, div.par-math-display{text-align:center;}
math .texttt { font-family: monospace; }
math .textit { font-style: italic; }
math .textsl { font-style: oblique; }
math .textsf { font-family: sans-serif; }
math .textbf { font-weight: bold; }
.partToc a, .partToc, .likepartToc a, .likepartToc {line-height: 200%; font-weight:bold; font-size:110%;}
.chapterToc a, .chapterToc, .likechapterToc a, .likechapterToc, .appendixToc a, .appendixToc {line-height: 200%; font-weight:bold;}
.index-item, .index-subitem, .index-subsubitem {display:block}
.caption td.id{font-weight: bold; white-space: nowrap; }
table.caption {text-align:center;}
h1.partHead{text-align: center}
p.bibitem { text-indent: -2em; margin-left: 2em; margin-top:0.6em; margin-bottom:0.6em; }
p.bibitem-p { text-indent: 0em; margin-left: 2em; margin-top:0.6em; margin-bottom:0.6em; }
.paragraphHead, .likeparagraphHead { margin-top:2em; font-weight: bold;}
.subparagraphHead, .likesubparagraphHead { font-weight: bold;}
.quote {margin-bottom:0.25em; margin-top:0.25em; margin-left:1em; margin-right:1em; text-align:justify;}
.verse{white-space:nowrap; margin-left:2em}
div.maketitle {text-align:center;}
h2.titleHead{text-align:center;}
div.maketitle{ margin-bottom: 2em; }
div.author, div.date {text-align:center;}
div.thanks{text-align:left; margin-left:10%; font-size:85%; font-style:italic; }
div.author{white-space: nowrap;}
.quotation {margin-bottom:0.25em; margin-top:0.25em; margin-left:1em; }
h1.partHead{text-align: center}
.sectionToc, .likesectionToc {margin-left:2em;}
.subsectionToc, .likesubsectionToc {margin-left:4em;}
.subsubsectionToc, .likesubsubsectionToc {margin-left:6em;}
.frenchb-nbsp{font-size:75%;}
.frenchb-thinspace{font-size:75%;}
.figure img.graphics {margin-left:10%;}
/* end css.sty */

\title{Endomorphismes et formes quadratiques}
\author{}
\date{}

\begin{document}
\maketitle

\textbf{Warning: 
requires JavaScript to process the mathematics on this page.\\ If your
browser supports JavaScript, be sure it is enabled.}

\begin{center}\rule{3in}{0.4pt}\end{center}

[
[
[]
[

\subsubsection{12.5 Endomorphismes et formes quadratiques}

\paragraph{12.5.1 Notion d'adjoint}

Soit E un K-espace vectoriel , \Phi une forme quadratique non dégénérée sur
E de forme polaire \phi.

Définition~12.5.1 Soit u,v \in L(E). On dit que u et v sont des
endomorphismes adjoints si

\forall~~x,y \in E, \phi(u(x),y) = \phi(x,v(y))

Remarque~12.5.1 La symétrie de \phi montre clairement que u et v jouent des
rôles symétriques, donc que u est adjoint de v si et seulement si~v est
adjoint de u.

Proposition~12.5.1 Si u \in L(E) admet un adjoint, celui-ci est unique.

Démonstration Si v_1 et v_2 sont adjoints de u, on a
\forall~x,y \in E, \phi(u(x),y) = \phi(x,v_1~(y)) =
\phi(x,v_2(y)). On a donc \forall~~x,y \in E,
\phi(v_1(y) - v_2(y),x) = 0, donc pour y \in E,
v_1(y) - v_2(y)
\in\mathrmKer~\phi =
\0\ et donc v_1 =
v_2.

Définition~12.5.2 Lorsque u \in L(E) admet un adjoint, nous le noterons
u^∗ et nous noterons L^∗(E) l'ensemble des
endomorphismes de E qui admettent un adjoint. Il est clair que
\mathrmId_E appartient à L^∗(E)
et que \mathrmId^∗ =
\mathrmId.

Proposition~12.5.2 L^∗(E) est un sous-espace vectoriel de
L(E). L'application u\mapsto~u^∗ est un
endomorphisme involutif de L^∗(E). Si u,v \in
L^∗(E), alors u \cdot v aussi et (u \cdot v)^∗ =
v^∗\cdot u^∗.

Démonstration On a déjà vu que la relation u et v sont adjoints était
symétrique, donc si u \in L^∗(E), u^∗ aussi et
u^∗∗ = u. Si u,v \in L^∗(E), \alpha~,\beta~ \in K, on a

\begin{align*} \phi((\alpha~u + \beta~v)(x),y)& =& \phi(\alpha~u(x) +
\beta~v(x),y) \%& \\ & =& \alpha~\phi(u(x),y) +
\beta~\phi(v(x),y) \%& \\ & =&
\alpha~\phi(x,u^∗(y)) + \beta~\phi(x,v^∗(y))\%&
\\ & =& \phi(x,(\alpha~u^∗ +
\beta~v^∗)(y)) \%& \\
\end{align*}

ce qui montre que \alpha~u + \beta~v \in L^∗(E) et que (\alpha~u +
\beta~v)^∗ = \alpha~u^∗ + \beta~v^∗~; donc
L^∗(E) est un sous-espace vectoriel de L(E) et
u\mapsto~u^∗ est linéaire. Si u,v \in
L^∗(E), on a

\phi(u \cdot v(x),y) = \phi(v(x),u^∗(y)) = \phi(x,v^∗\cdot
u^∗(y))

ce qui montre que u \cdot v admet v^∗\cdot u^∗ comme
adjoint.

Une des propriétés essentielles de l'adjoint que nous utiliserons de
fa\ccon assez systématique pour la réduction des
endomorphismes est la suivante

Théorème~12.5.3 Soit u \in L^∗(E). Soit F un sous-espace de E
stable par u~; alors F^\bot est stable par u^∗.

Démonstration Soit x \in F^\bot. Si y \in F, on a
\phi(u^∗(x),y) = \phi(x,u(y)) = 0 puisque u(y) \in F et x \in
F^\bot. Donc u^∗(x) \in F^\bot et
F^\bot est stable par u^∗.

Définition~12.5.3 On dit que u \in L(E) est symétrique ou autoadjoint si
u^∗ = u. On dit que u est antisymétrique si u^∗ =
-u.

Proposition~12.5.4 L'espace L^∗(E) est somme directe du
sous-espace des endomorphismes symétriques et du sous-espace des
endomorphismes antisymétriques.

Démonstration L'endomorphisme de L^∗(E),
u\mapsto~u^∗ étant involutif, l'espace
L^∗(E) est somme directe du sous-espace propre associé à la
valeur propre 1 (les endomorphismes symétriques) et du sous-espace
propre associé à la valeur propre -1 (les endomorphismes
antisymétriques).

\paragraph{12.5.2 Adjoint en dimension finie}

Soit E un K-espace vectoriel ~de dimension finie, \Phi une forme
quadratique non dégénérée sur E de forme polaire \phi.

Théorème~12.5.5 Tout endomorphisme de E admet un unique adjoint
u^∗ (autrement dit L^∗(E) = L(E)). Si u \in L(E), \mathcal{E}
une base de E, \Omega =\
\mathrmMat (\phi,\mathcal{E}) et A =\
\mathrmMat (u,\mathcal{E}), alors

\mathrmMat~
(u^∗,\mathcal{E}) = \Omega^-1^tA\Omega

Démonstration Soit \mathcal{E} une base de E et \Omega =\
\mathrmMat (\phi,\mathcal{E}). Comme \phi est non dégénérée, la
matrice \Omega est inversible. Soit u,v \in L(E), A =\
\mathrmMat (u,\mathcal{E}) et B =\
\mathrmMat (v,\mathcal{E}). Si x,y \in E, on a \phi(u(x),y) =
^t(AX)\OmegaY = ^tX^tA\OmegaY et \phi(x,v(y)) =
^tX\OmegaBY . L'unicité de la matrice de la forme bilinéaire
(x,y)\mapsto~\phi(u(x),y) montre que

\begin{align*} \forall~~x,y \in E,
\phi(u(x),y) = \phi(x,v(y))& \Leftrightarrow & ^tA\Omega
= \OmegaB \%& \\ &
\Leftrightarrow & B = \Omega^-1^tA\Omega\%&
\\ \end{align*}

ce qui montre à la fois l'existence (et l'unicité) de l'adjoint et la
formule voulue.

Remarque~12.5.2 Si la base \mathcal{E} est orthonormée, alors \Omega = I_n et
\mathrmMat~
(u^∗,\mathcal{E}) = ^t\
\mathrmMat (u,\mathcal{E})~; en particulier

Corollaire~12.5.6 Soit \mathcal{E} une base orthonormée de E~; alors u est
symétrique (resp. antisymétrique) si et seulement
si~\mathrmMat~ (u,\mathcal{E}) est une
matrice symétrique (resp. antisymétrique).

Corollaire~12.5.7 Si u \in L(E) est inversible, alors u^∗ est
inversible et (u^-1)^∗ =
(u^∗)^-1.

Démonstration On a u^-1 \cdot u =
\mathrmId_E d'où (u^-1 \cdot
u)^∗ = \mathrmId_E^∗, soit
u^∗\cdot (u^-1)^∗ =
\mathrmId_E. De même u \cdot u^-1 =
\mathrmId_E donne par passage à l'adjoint
(u^-1)^∗\cdot u^∗ =
\mathrmId_E. Ceci montre que u^∗
est inversible et que (u^-1)^∗ =
(u^∗)^-1

Corollaire~12.5.8
\mathrm{det} u^∗~
= \mathrm{det}~ u,
\mathrm{tr}u^∗~
= \mathrm{tr}~u,
\chi_u^∗ = \chi_u.

Démonstration Soit \mathcal{E} une base de E, \Omega =\
\mathrmMat (\phi,\mathcal{E}) et A =\
\mathrmMat (u,\mathcal{E}), alors
\mathrmMat~
(u^∗,\mathcal{E}) = \Omega^-1^tA\Omega. On a donc
\mathrm{det} u^∗~
= \mathrm{det}~
\Omega^-1^tA\Omega =\
\mathrm{det} ^tA =\
\mathrm{det} A =\
\mathrm{det} u. La démonstration est la même pour la
trace et pour le polynôme caractéristique.

Proposition~12.5.9 Soit E un espace euclidien, u \in L(E). Alors

\begin{itemize}
\itemsep1pt\parskip0pt\parsep0pt
\item
  (i)
  \mathrmKeru^∗~
  =
  (\mathrmImu)^\bot~,
  \mathrmImu^∗~ =
  (\mathrmKeru)^\bot~
\item
  (ii)
  \mathrmKeru^∗~u
  = \mathrmKer~u et
  \mathrmImu^∗~u
  = \mathrmImu^∗~
\end{itemize}

Démonstration (ii) On a

\begin{align*} x
\in\mathrmKeru^∗~&
\Leftrightarrow & u^∗(x) = 0
\Leftrightarrow \forall~~y \in E,
(u^∗(x)∣y) = 0 \%&
\\ & \Leftrightarrow &
\forall~y \in E, (x\mathrel∣~u(y)) =
0 \Leftrightarrow x \in
(\mathrmImu)^\bot~\%&
\\ \end{align*}

En appliquant ce résultat à u^∗ on obtient,
\mathrmKer~u =
(\mathrmImu^∗)^\bot~
et en prenant l'orthogonal,
\mathrmImu^∗~ =
(\mathrmKeru)^\bot~

(ii) On a visiblement u(x) = 0 \rigtharrow~ u^∗u(x) = 0, donc
\mathrmKer~u
\subset~\mathrmKeru^∗~u~;
mais d'autre part, si x
\in\mathrmKeru^∗~u,
on a

\u(x)\^2 =
(u(x)∣u(x)) =
(u^∗u(x)∣x) =
(0∣x) = 0

et donc u(x) = 0, soit
\mathrmKeru^∗~u
\subset~\mathrmKer~u et l'égalité.
On en déduit alors que

\mathrmImu^∗~u =
(\mathrmKer(u^∗u)^∗)^\bot~
=
(\mathrmKeru^∗u)^\bot~
=
(\mathrmKeru)^\bot~
= \mathrmImu^∗~

\paragraph{12.5.3 Endomorphismes symétriques et formes quadratiques}

Soit E un K-espace vectoriel ~de dimension finie, \Phi une forme
quadratique non dégénérée sur E de forme polaire \phi. A tout endomorphisme
u de E, on peut associer la forme bilinéaire \psi_u :
(x,y)\mapsto~\phi(x,u(y)). Il est clair que u est
symétrique (resp. antisymétrique) si et seulement si~\psi_u est
une forme bilinéaire symétrique (resp. antisymétrique).

Théorème~12.5.10 L'application u\mapsto~\psi_u
est un isomorphisme d'espaces vectoriels de L(E) sur l'espace
L_2(E) des formes bilinéaires sur E.

Démonstration Soit \mathcal{E} une base de E, \Omega =\
\mathrmMat (\phi,\mathcal{E}) et A =\
\mathrmMat (u,\mathcal{E}). Alors \psi_u(x,y) =
\phi(x,u(y)) = ^tX\OmegaAY et donc
\mathrmMat (\psi_u~,\mathcal{E})
= \OmegaA. Comme l'application A\mapsto~\OmegaA est un
isomorphisme, il en est de même de
u\mapsto~\psi_u.

Remarque~12.5.3 Supposons que la base \mathcal{E} est orthonormée, si bien que \Omega =
I_n. Alors
\mathrmMat (\psi_u~,\mathcal{E})
= \mathrmMat~ (u,\mathcal{E}). Une
matrice carrée est à la fois la matrice d'un endomorphisme u de E et
d'une forme bilinéaire \psi_u sur E. Mais le lecteur prendra garde
au fait que les formules de changement de bases ne sont évidemment pas
les mêmes pour l'endomorphisme
(A\mapsto~P^-1AP) et pour la forme
bilinéaire (A\mapsto~^tPAP).

\paragraph{12.5.4 Groupe orthogonal}

Soit E un K-espace vectoriel ~de dimension finie, \Phi une forme
quadratique non dégénérée sur E de forme polaire \phi.

Définition~12.5.4 On dit que u \in L(E) est un endomorphisme orthogonal si
on a les propriétés équivalentes

\begin{itemize}
\itemsep1pt\parskip0pt\parsep0pt
\item
  (i) \forall~~x \in E, \Phi(u(x)) = \Phi(x)
\item
  (ii) \forall~~x,y \in E, \phi(u(x),u(y)) = \phi(x,y)
\item
  (iii) u est inversible et u^-1 = u^∗
\item
  (iv) u \cdot u^∗ = \mathrmId_E
\item
  (v) u^∗\cdot u = \mathrmId_E
\end{itemize}

Démonstration (ii) \rigtharrow~(i) est évident (faire y = x). (i) \rigtharrow~(ii) provient de
l'identité de polarisation pour \Phi et de la linéarité de u

\begin{align*} \phi(u(x),u(y))& =& 1
\over 2 (\Phi(u(x) + u(y)) - \Phi(u(x)) - \Phi(u(y)))\%&
\\ & =& 1 \over 2
(\Phi(u(x + y)) - \Phi(u(x)) - \Phi(u(y))) \%& \\
& =& 1 \over 2 (\Phi(x + y) - \Phi(x) - \Phi(y)) = \phi(x,y)
\%& \\ \end{align*}

Pour un endomorphisme d'un espace vectoriel de dimension finie, on sait
que l'inversibilité est équivalente à l'inversibilité à gauche ou à
droite. On a donc (iii) \Leftrightarrow (iv)
\Leftrightarrow (v). Supposons (ii) vérifié. Alors \phi(x,y) =
\phi(u(x),u(y)) = \phi(x,u^∗\cdot u(y)), ce qui montre (puisque \phi est
non dégénérée) que u^∗\cdot u =
\mathrmId_E~; donc (ii) \rigtharrow~(v). De même (v)
\rigtharrow~(ii) puisque \phi(u(x),u(y)) = \phi(x,u^∗\cdot u(y)).

Remarque~12.5.4 La définition peut s'étendre au cas de la dimension
infinie, à condition d'imposer a priori que u est inversible.

Théorème~12.5.11 L'ensemble O_\Phi(E) des endomorphismes
orthogonaux de E est un sous groupe de (GL(E),\cdot). Pour tout
endomorphisme orthogonal u de E, on a
\mathrm{det}~ u = ±1.
L'ensemble SO_\Phi(E) des endomorphismes orthogonaux de
déterminant 1 est un sous-groupe distingué de O_\Phi(E) dont les
éléments sont appelés des rotations.

Démonstration On a clairement \mathrmId_E \in
O_\Phi(E). La définition (i) montre évidemment que si u et v sont
orthogonaux, il en est de même de u \cdot v. De plus, soit u \in
O_\Phi(E)~; on a \Phi(u^-1(x)) = \Phi(u(u^-1(x)))
= \Phi(x) ce qui montre que u^-1 \in O_\Phi(E). Donc
O_\Phi(E) est un sous-groupe de (GL(E),\cdot). On a alors 1
= \mathrm{det}~
\mathrmId_E =\
\mathrm{det} (u^∗\cdot u)
= \mathrm{det}~
u^∗\mathrm{det}~ u
= (\mathrm{det}~
u)^2, soit
\mathrm{det}~ u = ±1.
L'application O_\Phi(E) \rightarrow~\-
1,1\,
u\mapsto~\mathrm{det}~
u est un morphisme de groupes multiplicatifs~; son noyau
SO_\Phi(E) est donc un sous-groupe distingué.

Théorème~12.5.12 On suppose qu'il existe dans E des bases orthonormées.
Soit u \in L(E).

\begin{itemize}
\itemsep1pt\parskip0pt\parsep0pt
\item
  (i) Si u est orthogonal, il envoie toute base orthonormée sur une base
  orthonormée.
\item
  (ii) Inversement, s'il existe une base orthonormée \mathcal{E} de E telle que
  u(\mathcal{E}) soit encore orthonormée, alors u est un endomorphisme orthogonal.
\end{itemize}

Démonstration (i) On a \phi(u(e_i),u(e_j)) =
\phi(e_i,e_j) = \delta_i^j.

(ii) Soit x = \\sum ~
x_ie_i \in E. On a \Phi(x) =\
\sum  x_i^2~. Mais on a aussi
u(x) = \\sum ~
x_iu(e_i) et comme u(\mathcal{E}) est orthonormée, \Phi(u(x))
= \\sum ~
x_i^2~; on a donc \forall~~x \in E,
\Phi(u(x)) = \Phi(x).

Théorème~12.5.13 Soit u un endomorphisme orthogonal et F un sous-espace
de E stable par u. Alors F^\bot est stable par u.

Démonstration On a u(F) \subset~ F et comme u est inversible, on a
dim u(F) =\ dim~ F. On
a donc u(F) = F. Soit donc x \in F^\bot et y \in F~; il existe z \in F
tel que u(z) = y, d'où \phi(u(x),y) = \phi(u(x),u(z)) = \phi(x,z) = 0, et donc
u(x) \in F^\bot.

\paragraph{12.5.5 Matrices orthogonales}

Proposition~12.5.14 Soit E un K-espace vectoriel ~de dimension finie, \Phi
une forme quadratique non dégénérée sur E de forme polaire \phi. Soit u \in
L(E), \mathcal{E} une base de E, \Omega =\
\mathrmMat (\phi,\mathcal{E}) et A =\
\mathrmMat (u,\mathcal{E}). Alors u est un endomorphisme
orthogonal si et seulement si~^tA\OmegaA = \Omega.

Démonstration On a \phi(u(x),u(y)) = ^t(AX)\Omega(AY ) =
^tX^tA\OmegaAY . L'unicité de la matrice d'une forme
bilinéaire montre que

\forall~~x,y \in E, \phi(u(x),u(y)) = \phi(x,y)
\Leftrightarrow ^tA\OmegaA = \Omega

En particulier, si \mathcal{E} est une base orthonormée de E, u est un
endomorphisme orthogonal si et seulement si~^tAA =
I_n. Ceci conduit à la définition suivante

Définition~12.5.5 Soit A \in M_K(n)~; On dit que A est une
matrice orthogonale si elle vérifie les conditions équivalentes

\begin{itemize}
\itemsep1pt\parskip0pt\parsep0pt
\item
  (i) A est inversible et A^-1 = ^tA
\item
  (ii) ^tAA = I_n
\item
  (iii) A^tA = I_n
\end{itemize}

Théorème~12.5.15 L'ensemble O_K(n) des matrices carrées
orthogonales d'ordre n est un sous groupe de (GL_K(n),.). Pour
toute matrice orthogonale A, on a
\mathrm{det}~ A = ±1.
L'ensemble SO_K(n) des matrices orthogonales de déterminant 1
est un sous-groupe distingué de O_K(n) dont les éléments sont
appelés des matrices de rotations.

Démonstration On a clairement I_n \in O_K(n). La
définition (i) montre évidemment que si A et B sont orthogonales, il en
est de même de AB. De plus, soit A \in O_K(n)~; on a
A^-1^t(A^-1) =
A^-1^t(^tA) = A^-1A =
I_n ce qui montre que A^-1 \in O_K(n). Donc
O_K(n) est un sous-groupe de (GL_K(n),.). On a alors 1
= \mathrm{det} I_n~
= \mathrm{det}~
(^tAA) =
(\mathrm{det}~
A)^2, soit
\mathrm{det}~ A = ±1.
L'application O_K(n) \rightarrow~\-
1,1\,
A\mapsto~\mathrm{det}~
A est un morphisme de groupes multiplicatifs~; son noyau
SO_K(n) est donc un sous-groupe distingué.

Dans ce paragraphe, on munira K^n de la forme bilinéaire
symétrique naturelle (qui rend la base canonique orthonormée),
c'est-à-dire que l'on posera (x∣y)
= \\sum ~
_i=1^nx_iy_i

Théorème~12.5.16 Une matrice A \in M_K(n) est orthogonale si et
seulement si~ses vecteurs colonnes (resp. lignes) forment une base
orthonormée de K^n.

Démonstration Soit
(c_1,\\ldots,c_n~)
les vecteurs colonnes de A,
(l_1,\\ldots,l_n~)
ses vecteurs lignes. On a

\begin{align*} A \in O_K(n)&
\Leftrightarrow & ^tAA = I_ n
\Leftrightarrow \forall~~i,j,
(^tAA)_ i,j = \delta_i^j\%&
\\ & \Leftrightarrow &
\forall~~i,j, \\sum
_k=1^na_ k,ia_k,j =
\delta_i^j \%& \\ &
\Leftrightarrow & \forall~~i,j,
(c_i∣c_j) =
\delta_i^j \%& \\
\end{align*}

De la même fa\ccon, en traduisant la relation
A^tA = I_n, on obtiendrait
(l_i∣l_j) =
\delta_i^j.

Théorème~12.5.17 Soit E un K-espace vectoriel ~de dimension finie, \Phi une
forme quadratique non dégénérée sur E de forme polaire \phi. Soit \mathcal{E} une
base orthonormée de E, \mathcal{E}' une base de E. Alors on a équivalence de

\begin{itemize}
\itemsep1pt\parskip0pt\parsep0pt
\item
  (i) \mathcal{E}' est orthonormée
\item
  (ii) la matrice P_\mathcal{E}^\mathcal{E}' de passage de la base \mathcal{E} à la
  base \mathcal{E}' est orthogonale.
\end{itemize}

Démonstration On sait que P_\mathcal{E}^\mathcal{E}'
= \mathrmMat~ (u,\mathcal{E}) où u est
l'endomorphisme de E défini par \forall~~i,
u(e_i) = e_i'. Or d'après les résultats du paragraphe
précédent, u est un endomorphisme orthogonal si et seulement si~\mathcal{E}' est
orthonormée~; mais d'autre part, comme \mathcal{E} est orthonormée, u est
orthogonal si et seulement
si~\mathrmMat~ (u,\mathcal{E}) est une
matrice orthogonale, d'où l'équivalence entre (i) et (ii).

[
[
[
[

\end{document}

% \documentclass[]{article}
\usepackage[T1]{fontenc}
\usepackage{lmodern}
\usepackage{amssymb,amsmath}
\usepackage{ifxetex,ifluatex}
\usepackage{fixltx2e} % provides \textsubscript
% use upquote if available, for straight quotes in verbatim environments
\IfFileExists{upquote.sty}{\usepackage{upquote}}{}
\ifnum 0\ifxetex 1\fi\ifluatex 1\fi=0 % if pdftex
  \usepackage[utf8]{inputenc}
\else % if luatex or xelatex
  \ifxetex
    \usepackage{mathspec}
    \usepackage{xltxtra,xunicode}
  \else
    \usepackage{fontspec}
  \fi
  \defaultfontfeatures{Mapping=tex-text,Scale=MatchLowercase}
  \newcommand{\euro}{€}
\fi
% use microtype if available
\IfFileExists{microtype.sty}{\usepackage{microtype}}{}
\ifxetex
  \usepackage[setpagesize=false, % page size defined by xetex
              unicode=false, % unicode breaks when used with xetex
              xetex]{hyperref}
\else
  \usepackage[unicode=true]{hyperref}
\fi
\hypersetup{breaklinks=true,
            bookmarks=true,
            pdfauthor={},
            pdftitle={Endomorphismes d'un espace euclidien},
            colorlinks=true,
            citecolor=blue,
            urlcolor=blue,
            linkcolor=magenta,
            pdfborder={0 0 0}}
\urlstyle{same}  % don't use monospace font for urls
\setlength{\parindent}{0pt}
\setlength{\parskip}{6pt plus 2pt minus 1pt}
\setlength{\emergencystretch}{3em}  % prevent overfull lines
\setcounter{secnumdepth}{0}
 
/* start css.sty */
.cmr-5{font-size:50%;}
.cmr-7{font-size:70%;}
.cmmi-5{font-size:50%;font-style: italic;}
.cmmi-7{font-size:70%;font-style: italic;}
.cmmi-10{font-style: italic;}
.cmsy-5{font-size:50%;}
.cmsy-7{font-size:70%;}
.cmex-7{font-size:70%;}
.cmex-7x-x-71{font-size:49%;}
.msbm-7{font-size:70%;}
.cmtt-10{font-family: monospace;}
.cmti-10{ font-style: italic;}
.cmbx-10{ font-weight: bold;}
.cmr-17x-x-120{font-size:204%;}
.cmsl-10{font-style: oblique;}
.cmti-7x-x-71{font-size:49%; font-style: italic;}
.cmbxti-10{ font-weight: bold; font-style: italic;}
p.noindent { text-indent: 0em }
td p.noindent { text-indent: 0em; margin-top:0em; }
p.nopar { text-indent: 0em; }
p.indent{ text-indent: 1.5em }
@media print {div.crosslinks {visibility:hidden;}}
a img { border-top: 0; border-left: 0; border-right: 0; }
center { margin-top:1em; margin-bottom:1em; }
td center { margin-top:0em; margin-bottom:0em; }
.Canvas { position:relative; }
li p.indent { text-indent: 0em }
.enumerate1 {list-style-type:decimal;}
.enumerate2 {list-style-type:lower-alpha;}
.enumerate3 {list-style-type:lower-roman;}
.enumerate4 {list-style-type:upper-alpha;}
div.newtheorem { margin-bottom: 2em; margin-top: 2em;}
.obeylines-h,.obeylines-v {white-space: nowrap; }
div.obeylines-v p { margin-top:0; margin-bottom:0; }
.overline{ text-decoration:overline; }
.overline img{ border-top: 1px solid black; }
td.displaylines {text-align:center; white-space:nowrap;}
.centerline {text-align:center;}
.rightline {text-align:right;}
div.verbatim {font-family: monospace; white-space: nowrap; text-align:left; clear:both; }
.fbox {padding-left:3.0pt; padding-right:3.0pt; text-indent:0pt; border:solid black 0.4pt; }
div.fbox {display:table}
div.center div.fbox {text-align:center; clear:both; padding-left:3.0pt; padding-right:3.0pt; text-indent:0pt; border:solid black 0.4pt; }
div.minipage{width:100%;}
div.center, div.center div.center {text-align: center; margin-left:1em; margin-right:1em;}
div.center div {text-align: left;}
div.flushright, div.flushright div.flushright {text-align: right;}
div.flushright div {text-align: left;}
div.flushleft {text-align: left;}
.underline{ text-decoration:underline; }
.underline img{ border-bottom: 1px solid black; margin-bottom:1pt; }
.framebox-c, .framebox-l, .framebox-r { padding-left:3.0pt; padding-right:3.0pt; text-indent:0pt; border:solid black 0.4pt; }
.framebox-c {text-align:center;}
.framebox-l {text-align:left;}
.framebox-r {text-align:right;}
span.thank-mark{ vertical-align: super }
span.footnote-mark sup.textsuperscript, span.footnote-mark a sup.textsuperscript{ font-size:80%; }
div.tabular, div.center div.tabular {text-align: center; margin-top:0.5em; margin-bottom:0.5em; }
table.tabular td p{margin-top:0em;}
table.tabular {margin-left: auto; margin-right: auto;}
div.td00{ margin-left:0pt; margin-right:0pt; }
div.td01{ margin-left:0pt; margin-right:5pt; }
div.td10{ margin-left:5pt; margin-right:0pt; }
div.td11{ margin-left:5pt; margin-right:5pt; }
table[rules] {border-left:solid black 0.4pt; border-right:solid black 0.4pt; }
td.td00{ padding-left:0pt; padding-right:0pt; }
td.td01{ padding-left:0pt; padding-right:5pt; }
td.td10{ padding-left:5pt; padding-right:0pt; }
td.td11{ padding-left:5pt; padding-right:5pt; }
table[rules] {border-left:solid black 0.4pt; border-right:solid black 0.4pt; }
.hline hr, .cline hr{ height : 1px; margin:0px; }
.tabbing-right {text-align:right;}
span.TEX {letter-spacing: -0.125em; }
span.TEX span.E{ position:relative;top:0.5ex;left:-0.0417em;}
a span.TEX span.E {text-decoration: none; }
span.LATEX span.A{ position:relative; top:-0.5ex; left:-0.4em; font-size:85%;}
span.LATEX span.TEX{ position:relative; left: -0.4em; }
div.float img, div.float .caption {text-align:center;}
div.figure img, div.figure .caption {text-align:center;}
.marginpar {width:20%; float:right; text-align:left; margin-left:auto; margin-top:0.5em; font-size:85%; text-decoration:underline;}
.marginpar p{margin-top:0.4em; margin-bottom:0.4em;}
.equation td{text-align:center; vertical-align:middle; }
td.eq-no{ width:5%; }
table.equation { width:100%; } 
div.math-display, div.par-math-display{text-align:center;}
math .texttt { font-family: monospace; }
math .textit { font-style: italic; }
math .textsl { font-style: oblique; }
math .textsf { font-family: sans-serif; }
math .textbf { font-weight: bold; }
.partToc a, .partToc, .likepartToc a, .likepartToc {line-height: 200%; font-weight:bold; font-size:110%;}
.chapterToc a, .chapterToc, .likechapterToc a, .likechapterToc, .appendixToc a, .appendixToc {line-height: 200%; font-weight:bold;}
.index-item, .index-subitem, .index-subsubitem {display:block}
.caption td.id{font-weight: bold; white-space: nowrap; }
table.caption {text-align:center;}
h1.partHead{text-align: center}
p.bibitem { text-indent: -2em; margin-left: 2em; margin-top:0.6em; margin-bottom:0.6em; }
p.bibitem-p { text-indent: 0em; margin-left: 2em; margin-top:0.6em; margin-bottom:0.6em; }
.paragraphHead, .likeparagraphHead { margin-top:2em; font-weight: bold;}
.subparagraphHead, .likesubparagraphHead { font-weight: bold;}
.quote {margin-bottom:0.25em; margin-top:0.25em; margin-left:1em; margin-right:1em; text-align:\\jmathmathustify;}
.verse{white-space:nowrap; margin-left:2em}
div.maketitle {text-align:center;}
h2.titleHead{text-align:center;}
div.maketitle{ margin-bottom: 2em; }
div.author, div.date {text-align:center;}
div.thanks{text-align:left; margin-left:10%; font-size:85%; font-style:italic; }
div.author{white-space: nowrap;}
.quotation {margin-bottom:0.25em; margin-top:0.25em; margin-left:1em; }
h1.partHead{text-align: center}
.sectionToc, .likesectionToc {margin-left:2em;}
.subsectionToc, .likesubsectionToc {margin-left:4em;}
.subsubsectionToc, .likesubsubsectionToc {margin-left:6em;}
.frenchb-nbsp{font-size:75%;}
.frenchb-thinspace{font-size:75%;}
.figure img.graphics {margin-left:10%;}
/* end css.sty */

\title{Endomorphismes d'un espace euclidien}
\author{}
\date{}

\begin{document}
\maketitle

\textbf{Warning: 
requires JavaScript to process the mathematics on this page.\\ If your
browser supports JavaScript, be sure it is enabled.}

\begin{center}\rule{3in}{0.4pt}\end{center}

{[}
{[}
{[}{]}
{[}

\subsubsection{12.6 Endomorphismes d'un espace euclidien}

\paragraph{12.6.1 Droites et plans stables}

Nous utiliserons à deux reprises le lemme suivant

Lemme~12.6.1 Soit E un \mathbb{R}~-espace vectoriel ~de dimension finie et u \in
L(E). Alors u admet soit une droite stable, soit un plan stable.

Démonstration Soit P un polynôme normalisé annulateur de u et soit P =
P_1\\ldotsP_n~
la décomposition de P en polynômes normalisés irréductibles sur \mathbb{R}~. On a
0 = P(u) = P_1(u) \cdot⋯ \cdot
P_n(u). Donc l'un des P_i(u) est non in\\jmathmathectif. Soit x
\in\mathrmKerP_i~(u)
\diagdown\0\. Deux cas sont possibles~:

\begin{itemize}
\itemsep1pt\parskip0pt\parsep0pt
\item
  P_1 est de degré 1, soit P_1(X) = X - \lambda~, alors (u -
  \lambda~\mathrmId)(x) = 0, x est vecteur propre de u et la
  droite \mathbb{R}~x est stable par u~;
\item
  P_1 est de degré 2, alors P_1(X) = X^2 -
  aX - b et on a donc u^2(x) = au(x) + bx~; le sous-espace
  \mathrmVect~(x,u(x)) est
  de dimension au plus 2 (en fait il est facile de vérifier qu'elle est
  égale à 2) et il est stable par u.
\end{itemize}

\paragraph{12.6.2 Réduction des endomorphismes symétriques}

Théorème~12.6.2 Soit E un espace euclidien et u un endomorphisme de E.
Alors on a équivalence de

\begin{itemize}
\itemsep1pt\parskip0pt\parsep0pt
\item
  (i) u est un endomorphisme symétrique
\item
  (ii) il existe une base orthonormée formée de vecteurs propres de u
\item
  (iii) il existe une base orthonormée \mathcal{E} telle que
  \mathrmMat~ (u,\mathcal{E}) soit
  diagonale.
\end{itemize}

Démonstration (ii) et (iii) sont clairement équivalents. Si (iii) est
vérifiée, la matrice de u dans la base orthonormée \mathcal{E} est symétrique et
donc u est un endomorphisme symétrique. Il nous reste donc à montrer que
(i) \rigtharrow~(ii), ce que nous allons faire par récurrence sur n
= dim~ E. Montrons pour cela que u a un vecteur
propre. D'après le lemme ci dessus, u admet soit une droite, soit un
plan stable. Si u admet une droite stable, cette droite est engendrée
par un vecteur propre. Si u a un plan stable \Pi, soit u' l'endomorphisme
induit par u sur \Pi (c'est bien entendu un endomorphisme symétrique de
\Pi), \mathcal{E} = (e_1,e_2) une base orthonormée de \Pi et
\mathrmMat~ (u',\mathcal{E}) =
\left
(\matrix\,a&b\cr b
&c\right )~; alors \chi_u(X) = (X - a)(X - c) -
b^2 = X^2 - (a + b)X + ac - b^2 de
discriminant \Delta = (a + c)^2 - 4(ac - b^2) = (a -
c)^2 + 4b^2 ≥ 0~; donc u' a un vecteur propre dans \Pi
qui est également un vecteur propre de u dans E. Supposons donc que tout
endomorphisme symétrique d'un espace de dimension n - 1 admet une base
orthonormée de vecteurs propres. Soit e_1 un vecteur propre de
u (endomorphisme symétrique d'un espace euclidien de dimension n).
Quitte à remplacer e_1 par  e_1 \over
\e_1\ , on
peut supposer que
\e_1\ = 1.
Soit H = e_1^\bot. Comme Ke_1 est stable par u,
son orthogonal H est stable par u^∗ = u. La restriction u' de
u à H est un endomorphisme symétrique de H de dimension n - 1, donc
admet une base orthonormée
(e_2,\\ldots,e_n~)
formée de vecteurs propres de u' (donc de u)~; alors
(e_1,\\ldots,e_n~)
est une base orthonormée de E formée de vecteurs propres de u.

Corollaire~12.6.3 Soit E un espace euclidien et u un endomorphisme
symétrique de E~; alors E est somme directe orthogonale des sous-espaces
propres de u.

Démonstration Puisque u est diagonalisable, E est somme directe des
sous-espaces propres de u. Il suffit de montrer que ces sous-espaces
sont deux à deux orthogonaux. Mais, si x \in E_u(\lambda~) et y \in
E_u(\mu) avec \lambda~\neq~\mu, on a

\lambda~(x∣y) = (u(x)\mathrel∣y)
= (x∣u(y)) =
(x∣\muy) = \mu(x\mathrel∣y)

Comme \lambda~\neq~\mu, on a
(x∣y) = 0.

Remarque~12.6.1 Ceci permet une pratique simple de la réduction d'un
endomorphisme symétrique~; il suffit en effet de déterminer une base
orthonormée de chacun des sous-espaces propres de u et de réunir ces
bases~; on obtient une base orthonormée de E formée de vecteurs propres
de u.

Définition~12.6.1 Soit E un espace euclidien et u un endomorphisme
symétrique de E. On dit que u est un endomorphisme positif (resp. défini
positif) s'il vérifie les conditions équivalentes~:

\begin{itemize}
\itemsep1pt\parskip0pt\parsep0pt
\item
  (i) \forall~~x \in
  E,(u(x)∣x) ≥ 0 (resp.
  \forall~x\neq~0,(u(x)\mathrel∣~x)
  \textgreater{} 0)
\item
  (ii) L'application
  x\mapsto~(u(x)\mathrel∣x) est
  une forme quadratique positive sur E (resp. définie positive)
\item
  (iii) Les valeurs propres de u sont positives (resp. strictement
  positives).
\end{itemize}

Démonstration En remarquant que l'application Q_u :
x\mapsto~(u(x)\mathrel∣x) est une
forme quadratique de forme polaire
(x,y)\mapsto~(u(x)\mathrel∣y), on
a immédiatement l'équivalence de (i) et (ii). Soit \lambda~ une valeur propre
de u et x un vecteur propre associé~; on a alors
(u(x)∣x) = (\lambda~x\mathrel∣x)
= \lambda~\x\^2,
si bien que \lambda~ = (u(x)∣x)
\over
\x\^2 ~;
il en résulte que (i) \rigtharrow~(iii). Inversement, supposons (iii) vérifiée et
soit
(e_1,\\ldots,e_n~)
une base orthonormée formée de vecteurs propres de u, u(e_i) =
\lambda_ie_i. On a alors, si x =\
\sum  _ix_ie_i~,

(u(x)∣x) = (\\sum
_i\lambda_ix_ie_i∣\\sum
_ix_ie_i) = \\sum
_i\lambda_ix_i^2 ≥ 0

(resp. \textgreater{} 0 si x\neq~0) si bien que
(iii) \rigtharrow~(i).

Théorème~12.6.4 (réduction simultanée des formes quadratiques). Soit E
un \mathbb{R}~ espace vectoriel de dimension finie, \Phi une forme quadratique
définie positive, \Psi une forme quadratique sur E. Alors il existe une
base \mathcal{E} de E orthonormée pour \Phi et orthogonale pour \Psi.

Démonstration On sait qu'il existe un unique endomorphisme u de E tel
que \forall~~x,y \in E, \psi(x,y) = \phi(u(x),y). Comme \psi est
symétrique, u est un endomorphisme symétrique de l'espace euclidien
(E,\Phi) et il existe une base \mathcal{E} =
(e_1,\\ldots,e_n~)
orthonormée pour \Phi formée de vecteurs propres de u~: u(e_i) =
\lambda_ie_i. On a alors

\psi(e_i,e_\\jmathmath) = \phi(e_i,u(e_\\jmathmath)) =
\phi(e_i,\lambda_\\jmathmathe_\\jmathmath) =
\lambda_\\jmathmath\delta_i^\\jmathmath

ce qui montre que \mathcal{E} est une base orthogonale pour \psi.

\paragraph{12.6.3 Normes d'endomorphismes}

Théorème~12.6.5 Soit E un espace euclidien et u \in L(E). Alors

\u\
=\
sup_\x\\leq1,\y\\leq1(u(x)∣y)

Démonstration Supposons que
\x\ \leq
1,\y\ \leq 1. On a alors
d'après l'inégalité de Schwarz

(u(x)∣y)\leq\
u(x)\\y\
\leq\ u\

ce qui montre que \u\
≥\
sup_\x\\leq1,\y\\leq1(u(x)∣y).
Mais d'autre part, puisque la boule unité fermée est compacte, il existe
x_0 tel que
\x_0\ \leq 1
avec \u(x_0)\
=\
sup_\x\\leq1\u(x)\
=\ u\. Posons alors,
si u\neq~0, y_0 = u(x_0)
\over
\u(x_0)\ .
On a \y_0\ =
1 et

(u(x_0)∣y_0)
= (u(x_0)∣u(x_0))
\over
\u(x_0)\
=\ u(x_0)\
=\ u\

ce qui montre que \u\
\leq\
sup_\x\\leq1,\y\\leq1(u(x)∣y),
et donc l'égalité.

Corollaire~12.6.6 Soit E un espace euclidien et u \in L(E). Alors
\u\
=\ u^∗\.

Démonstration En effet

\u\
=\
sup_\x\\leq1,\y\\leq1(u(x)∣y)
=\
sup_\x\\leq1,\y\\leq1(x∣u^∗(y))
=\ u^∗\

Théorème~12.6.7 Soit E un espace euclidien et u \in L(E) symétrique
positif. Alors

\u\
=\
sup_\x\\leq1(u(x)∣x)
=\
max_\lambda~\in\mathrm{Sp}(u)~\lambda~

Démonstration Supposons que
\x\ \leq 1. On a alors
d'après l'inégalité de Schwarz

(u(x)∣x) \leq\
u(x)\\x\
\leq\ u\

ce qui montre que \u\
≥\
sup_\x\\leq1(u(x)∣x).
De plus, soit k =\
sup_\x\\leq1(u(x)∣x),
si bien que \forall~~x \in
E,(u(x)∣x) \leq
k\x\^2, et
supposons que \x\ \leq
1,\y\ \leq 1. On a alors,
puisque u est symétrique

\begin{align*}
(u(x)∣y)& =& 1
\over 4 (u(x +
y)∣x + y) - (u(x -
y)∣x - y)\%&
\\ & \leq& k \over 4
(\x + y\^2
+\ x -
y\^2) = k \over 2
(\x\^2
+\ y\^2) \leq
k \%& \\ \end{align*}

et donc

\u\
=\
sup_\x\\leq1,\y\\leq1(u(x)∣y)\leq
k

et par conséquent

\u\
=\
sup_\x\\leq1(u(x)∣x)

Soit
(e_1,\\ldots,e_n~)
une base orthonormée formée de vecteurs propres de u, u(e_i) =
\lambda_ie_i. On peut supposer que \lambda_1 \leq
\lambda_2
\leq\\ldots~ \leq
\lambda_n. On a alors, si x =\
\sum  _ix_ie_i~,

(u(x)∣x) = (\\sum
_i\lambda_ix_ie_i∣\\sum
_ix_ie_i) = \\sum
_i\lambda_ix_i^2 \leq \lambda_ n
\sum _ix_i^2 \leq \lambda~_
n

avec égalité si x = e_n. Ceci montre que

sup_\x\\leq1(u(x)\mathrel∣~x)
=\
max_\lambda~\in\mathrm{Sp}(u)~\lambda~

et achève la démonstration.

Corollaire~12.6.8 Soit E un espace euclidien et u \in L(E). Alors
u^∗u est un endomorphisme symétrique positif et
\u^∗u\
=\ u\^2.

Démonstration On a (u^∗u)^∗ =
u^∗u^∗∗ = u^∗u donc u^∗u est
symétrique. De plus (u^∗u(x)∣x) =
(u(x)∣u(x)) =\
u(x)\^2 ≥ 0, ce qui montre que
u^∗u est positif. On a alors

\u^∗u\^2
= sup_\
x\\leq1(u^∗u(x)∣x)
= sup_\
x\\leq1\u(x)\^2
=\ u\^2

Corollaire~12.6.9 \u\
est la racine carrée de la plus grande valeur propre de u^∗u.

Démonstration Résulte immédiatement des résultats précédents.

\paragraph{12.6.4 Endomorphismes orthogonaux d'un plan euclidien}

Remarque~12.6.2 Soit E un \mathbb{R}~-espace vectoriel ~de dimension finie. La
relation définie sur l'ensemble des bases de E par

\mathcal{E}\mathcal{R}\mathcal{E}'\Leftrightarrow
\mathrm{det}~
P_\mathcal{E}^\mathcal{E}' \textgreater{} 0

est une relation d'équivalence pour laquelle il y a deux classes
d'équivalence appelées orientations de l'espace. Le choix d'une de ces
classes (les bases directes) oriente l'espace E.

Soit E un espace euclidien de dimension 2.

Théorème~12.6.10 Soit u \in O(E) et \mathcal{E} une base orthonormée de E.

\begin{itemize}
\itemsep1pt\parskip0pt\parsep0pt
\item
  (i) Si u \in SO(E), alors
  \mathrmMat~ (u,\mathcal{E}) =
  \left
  (\matrix\,cos~
  \theta&-sin~ \theta\cr
  sin \theta &\cos~
  \theta\right ) pour un \theta \in \mathbb{R}~\diagup2\pi~\mathbb{Z} ne dépendant que de
  l'orientation de la base \mathcal{E} (un changement d'orientation changeant \theta en
  - \theta)~; le groupe SO(E) est commutatif, isomorphe au groupe (\mathbb{R}~\diagup2\pi~\mathbb{Z},+)
\item
  (ii) Si \mathrm{det}~ u =
  -1, alors \mathrmMat~
  (u,\mathcal{E}) = \left
  (\matrix\,cos~
  \theta&sin~ \theta \cr
  sin \theta&-\cos~
  \theta\right )~; u est une symétrie orthogonale par
  rapport à une droite.
\end{itemize}

Démonstration Posons \mathcal{E} = (e_1,e_2) et u(e_1)
= ae_1 + be_2. On a a^2 + b^2
=\
u(e_1)\^2
=\
e_1\^2 = 1, donc il existe
\theta \in \mathbb{R}~\diagup2\pi~\mathbb{Z} tel que a = cos~ \theta et b
= sin \theta. On a u(e_2~) \in
u(e_1)^\bot = \mathbb{R}~(-be_1 + ae_2). On en
déduit que u(e_2) = \lambda~(-sin~
\thetae_1 + cos \thetae_2~)~; comme
\u(e_2)\
=\ e_2\ = 1,
on doit avoir \lambda~^2 = 1, soit \lambda~ = ±1. Donc la matrice de u dans
la base \mathcal{E} est de l'une des deux formes

\left
(\matrix\,cos~
\theta&-sin~ \theta \cr
sin \theta&\cos~ \theta
\cr \right )\text ou
\left
(\matrix\,cos~
\theta&sin~ \theta \cr
sin \theta&-\cos~
\theta\right )

Il est clair que le premier cas correspond à
\mathrm{det}~ u = 1 et le
second cas à \mathrm{det}~ u =
-1. Dans le second cas, on vérifie immédiatement que u^2 =
\mathrmId_E, ce qui montre que u est une
symétrie (évidemment orthogonale). Comme
u\neq~\mathrmId et
u\neq~ -\mathrmId, c'est
nécessairement une symétrie par rapport à une droite.

Dans le premier cas, on a
\mathrm{tr}~u =
2cos~ \theta, ce qui montre que
cos~ \theta est indépendant du choix de la base \mathcal{E},
et que donc \theta \in \mathbb{R}~\diagup2\pi~\mathbb{Z} est déterminé au signe près. On vérifie
immédiatement que

\begin{align*} \left
(\matrix\,cos~
\theta&-sin~ \theta \cr
sin \theta&\cos~ \theta
\cr \right )\left
(\matrix\,cos~
\theta'&-sin~ \theta'\cr
sin \theta' &\cos~
\theta'\right )&& \%& \\ &
\quad & = \left
(\matrix\,cos~
(\theta + \theta')&-sin~ (\theta + \theta') \cr
sin (\theta + \theta')&\cos~ (\theta
+ \theta')\right )\%& \\
\end{align*}

ce qui montre que le groupe SO(2) = \R(\theta) =
\left
(\matrix\,cos~
\theta&-sin~ \theta\cr
sin \theta &\cos~
\theta\right )∣\theta \in
\mathbb{R}~\diagup2\pi~\mathbb{Z}\ est commutatif et isomorphe à (\mathbb{R}~\diagup2\pi~\mathbb{Z},+). Soit u
\in SO(E)~; si \mathcal{E} et \mathcal{E}' sont deux bases orthonormées de même sens
(c'est-à-dire que
\mathrm{det}~
P_\mathcal{E}^\mathcal{E}' \textgreater{} 0), alors P =
P_\mathcal{E}^\mathcal{E}'\in SO(2), on a

\mathrmMat~ (u,\mathcal{E}') =
P^-1 \mathrmMat~
(u,\mathcal{E})P =
P^-1P\mathrmMat~
(u,\mathcal{E}) = \mathrmMat~ (u,\mathcal{E})

puisque SO(2) est commutatif et que P et
\mathrmMat~ (u,\mathcal{E}) sont
toutes deux dans SO(2). Donc \theta ne dépend que de l'orientation de la base
\mathcal{E}. Si maintenant, \mathcal{E} et \mathcal{E}' sont deux bases orthonormées de sens contraire
(c'est-à-dire que
\mathrm{det}~
P_\mathcal{E}^\mathcal{E}' \textless{} 0), alors
\mathrmMat~ (u,\mathcal{E})P est une
matrice orthogonale de déterminant - 1. Comme on l'a vu, son carré est
nécessairement l'identité de même que le carré de P, ce qui montre que
\mathrmMat~ (u,\mathcal{E}') =
P^-1 \mathrmMat~
(u,\mathcal{E})P = P\mathrmMat~ (u,\mathcal{E})P
= \mathrmMat~
(u,\mathcal{E})^-1 = R(-\theta) (si
\mathrmMat~ (u,\mathcal{E}) = R(\theta))~:
un changement d'orientation de la base change donc \theta en - \theta.

Définition~12.6.2 Soit E un plan euclidien orienté, u \in SO(E). On
appelle mesure de la rotation u l'unique élément \theta de \mathbb{R}~\diagup2\pi~\mathbb{Z} tel que,
pour toute base orthonormée directe \mathcal{E} de E, on ait
\mathrmMat~ (u,\mathcal{E}) =
\left
(\matrix\,cos~
\theta&-sin~ \theta\cr
sin \theta &\cos~
\theta\right ).

\paragraph{12.6.5 Réduction des endomorphismes orthogonaux}

Théorème~12.6.11 Soit E un espace euclidien et u un endomorphisme
orthogonal de E. Alors il existe une base orthonormée \mathcal{E} de E telle que

\mathrmMat~ (u,\mathcal{E}) =
\left
(\matrix\,I_p&0
&\\ldots~
&\\ldots&\\\ldots~&0
\cr 0 &-I_q&0
&\\ldots&\\\ldots&\\⋮~
\cr \⋮~
&0
&A_1&0&\\ldots&\\⋮~
\cr \⋮~
&\\ldots~
&⋱
&⋱&\mathrel⋱&\⋮~
\cr \⋮~
&\\ldots~
&\\ldots~
&⋱&\mathrel⋱&0
\cr 0
&\\ldots~
&\\ldots~
&\\ldots&0&A_s~\right
)

avec A_i = \left
(\matrix\,cos~
\theta_i&-sin \theta_i~
\cr sin~
\theta_i&cos \theta_i~
\right ), \theta_i \in \mathbb{R}~ \diagdown 2\pi~\mathbb{Z}.

Démonstration Par récurrence sur n = dim~ E. Si
n = 1, alors u = ±\mathrmId_E et le résultat
est évident. Supposons le donc démontré pour tout espace euclidien de
dimension strictement inférieure à n et soit E de dimension n, u \in O(E).
Si u admet une valeur propre \lambda~, soit x un vecteur propre associé. On a
\lambda~\x\
=\ u(x)\
=\ x\, d'où \lambda~ = ±1. La
droite \mathbb{R}~x est stable par u, donc H = (\mathbb{R}~x)^\bot aussi. La
restriction v de u à H est un endomorphisme orthogonal de H et par
l'hypothèse de récurrence, il existe une base orthonormée
(e_2,\\ldots,e_n~)
de H telle que la matrice de v dans cette base soit de la forme voulue.
Alors ( x \over
\x\
,e_2,\\ldots,e_n~)
est une base orthonormée de E et à une permutation près de cette base
(si \lambda~ = -1), la matrice de u dans cette base est de la forme voulue. Si
u n'a pas de valeur propre (réelle), soit \Pi un plan stable par u, dont
l'existence est garantie par un lemme précédent. L'endomorphisme de \Pi
induit par u est un endomorphisme orthogonal de \Pi sans valeur propre,
donc une rotation d'angle \theta_1 \in \mathbb{R}~ \diagdown \pi~\mathbb{Z}. Soit
(e_1,e_2) une base orthonormée de \Pi. Le sous-espace de
dimension n - 2, H = \Pi^\bot est également stable par u et
l'endomorphisme v de H induit par u est un endomorphisme orthogonal de H
sans valeur propre. Par hypothèse de récurrence, il existe une base
orthonormée
(e_3,\\ldots,e_n~)
de H telle que la matrice de v dans cette base soit de la forme voulue
(avec p = q = 0). Alors la matrice de u dans la base orthonormée
(e_1,e_2,e_3,\\ldots,e_n~)
est de la forme voulue, ce qui achève la démonstration.

Exemple~12.6.1 Si dim~ E = 3, on a les formes
réduites possibles (en tenant compte de
\mathrm{det}~ u =
(-1)^q et de p + q + 2s = 3)

\begin{itemize}
\itemsep1pt\parskip0pt\parsep0pt
\item
  \mathrm{det}~ u = 1

  \begin{itemize}
  \itemsep1pt\parskip0pt\parsep0pt
  \item
    \left
    (\matrix\,1&0&0 \cr
    0&1&0 \cr 0&0&1\right )
    (identité),
  \item
    \left (\matrix\,1&0
    &0 \cr 0&-1&0 \cr 0&0
    &-1\right ) (retournement d'axe \mathbb{R}~e_1),
  \item
    \left (\matrix\,1&0
    &0 \cr 0&cos~
    \theta&-sin~ \theta \cr
    0&sin \theta&\cos~ \theta
    \right ) (rotation d'axe \mathbb{R}~e_1)
  \end{itemize}
\item
  \mathrm{det}~ u = -1

  \begin{itemize}
  \itemsep1pt\parskip0pt\parsep0pt
  \item
    \left (\matrix\,-1&0
    &0 \cr 0 &-1&0 \cr 0 &0
    &-1\right )\quad (
    -\mathrmId_E),
  \item
    \left
    (\matrix\,1&0&0 \cr
    0&1&0 \cr 0&0&-1\right )
    (symétrie par rapport au plan
    \mathrmVect(e_1,e_2~)),
  \item
    \left (\matrix\,-1&0
    &0 \cr 0 &cos~
    \theta&-sin~ \theta \cr 0
    &sin \theta&\cos~ \theta
    \right ) (composée de la symétrie par rapport au
    plan
    \mathrmVect(e_2,e_3~)
    et d'une rotation d'axe \mathbb{R}~e_1)
  \end{itemize}
\end{itemize}

\paragraph{12.6.6 Produit vectoriel, produit mixte}

Théorème~12.6.12 Soit E un espace euclidien orienté de dimension n.
L'application n linéaire alternée
\mathrm{det} _\mathcal{E}~ est
indépendante du choix de la base orthonormée directe \mathcal{E}.

Démonstration Si \mathcal{E}' est une autre base orthonormée directe, la matrice
de passage P_\mathcal{E}^\mathcal{E}' est à la fois orthogonale et de
déterminant strictement positif, donc de déterminant 1. Or on a

\mathrm{det} _\mathcal{E}~
= \mathrm{det}~
_\mathcal{E}(\mathcal{E}')\mathrm{det}~
_\mathcal{E}' = \mathrm{det}~
P_\mathcal{E}^\mathcal{E}'\mathrm{det}~
_ \mathcal{E}' = \mathrm{det}~
_\mathcal{E}'

Définition~12.6.3 On notera
{[}x_1,\\ldots,x_n~{]}
= \mathrm{det}~
_\mathcal{E}(x_1,\\ldots,x_n~)
et on l'appellera le produit mixte des n vecteurs
x_1,\\ldots,x_n~.

Remarque~12.6.3 Il est clair qu'un changement d'orientation de l'espace
change le produit mixte en son opposé.

Théorème~12.6.13 Soit E un espace euclidien orienté. Alors, pour toute
famille
(x_1,\\ldots,x_n~)
de E on a

\mathrm{det}~
Gram(x_1,\\\ldots,x_n~)
=
{[}x_1,\\ldots,x_n{]}^2~

Démonstration Soit \mathcal{E} une base orthonormée directe et soit A la matrice
des coordonnées de
(x_1,\\ldots,x_n~)
dans la base \mathcal{E}. On a alors
\mathrm{det}~ A =
{[}x_1,\\ldots,x_n~{]}.
D'autre part

(x_i∣x_\\jmathmath) =
\sum _k=1^na_
k,ia_k,\\jmathmath = (^tAA)_ i,\\jmathmath

si bien que
Gram(x_1,\\\ldots,x_n~)
= ^tAA. On a donc

\mathrm{det}~
Gram(x_1,\\\ldots,x_n~)
= \mathrm{det}~
^tAA =
(\mathrm{det} A)^2~
= {[}x_
1,\\ldots,x_n{]}^2~

Théorème~12.6.14 (et définition). Soit E un espace euclidien orienté de
dimension n. Soit
x_1,\\ldots,x_n-1~
\in E. Il existe un unique vecteur, appelé le produit vectoriel des n - 1
vecteurs
x_1,\\ldots,x_n-1~
et noté x_1
∧\\ldots~ ∧
x_n-1 tel que

\forall~~y \in E,
{[}x_1,\\ldots,x_n-1~,y{]}
= (x_1
∧\\ldots~ ∧
x_n-1∣y)

Démonstration L'application
y\mapsto~{[}x_1,\\ldots,x_n-1~,y{]}
est une forme linéaire sur E, donc représentée par le produit scalaire
avec un unique vecteur.

Proposition~12.6.15

\begin{itemize}
\itemsep1pt\parskip0pt\parsep0pt
\item
  (i) x_1
  ∧\\ldots~ ∧
  x_n-1 = 0 \Leftrightarrow
  (x_1,\\ldots,x_n-1~)
  est une famille liée
\item
  (ii) \forall~i \in {[}1,n - 1{]}, x_1~
  ∧\\ldots~ ∧
  x_n-1 \bot x_i
\item
  (iii) si
  (x_1,\\ldots,x_n-1~)
  est une famille libre, alors
  (x_1,\\ldots,x_n-1,x_1~
  ∧\\ldots~ ∧
  x_n-1) est une base directe de E
\item
  (iv) \x_1
  ∧\\ldots~ ∧
  x_n-1\^2
  = \mathrm{det}~
  Gram(x_1,\\\ldots,x_n-1~).
\end{itemize}

Démonstration (i) On a en effet

\begin{align*}
(x_1,\\ldots,x_n-1~)\text
libre & \Leftrightarrow & \exists~y
\in E,
(x_1,\\ldots,x_n-1~,y)\text
base de E\%& \\ &
\Leftrightarrow & \exists~y \in E,
{[}x_1,\\ldots,x_n-1,y{]}\mathrel\neq~~0
\%& \\ & \Leftrightarrow &
\existsy \in E, (x_1~
∧\\ldots~ ∧
x_n-1∣y)\mathrel\neq~0
\%& \\ & \Leftrightarrow &
x_1
∧\\ldots~ ∧
x_n-1\neq~0 \%&
\\ \end{align*}

(ii) (x_1
∧\\ldots~ ∧
x_n-1∣x_i) =
{[}x_1,\\ldots,x_i,\\\ldots,x_n-1,x_i~{]}
= 0

(iii) On a

\begin{align*}
{[}x_1,\\ldots,x_n-1,x_1~
∧\\ldots~ ∧
x_n-1{]}& =& (x_1
∧\\ldots~ ∧
x_n-1∣x_1
∧\\ldots~ ∧
x_n-1)\%& \\ & =&
\x_1
∧\\ldots~ ∧
x_n-1\^2 \textgreater{} 0
\%& \\ \end{align*}

(iv) Comme on vient de le voir,

\begin{align*}
\x_1
∧\\ldots~ ∧
x_n-1\^4&& \%&
\\ & =&
{[}x_1,\\ldots,x_n-1,x_1~
∧\\ldots~ ∧
x_n-1{]}^2 \%& \\ &
=& \mathrm{det}~
Gram(x_1,\\\ldots,x_n-1,x_1~
∧\\ldots~ ∧
x_n-1) \%& \\ & =&
\left
\matrix\,Gram(x_1,\\\ldots,x_n-1~)&0
\cr 0 &\x_1
∧\\ldots~ ∧
x_n-1\^2\right
 \%& \\ & =&
\x_1
∧\\ldots~ ∧
x_n-1\^2\
\mathrm{det} Gram(x_
1,\\ldots,x_n-1~)\%&
\\ \end{align*}

puisque (x_1
∧\\ldots~ ∧
x_n-1∣x_i) = 0. Si
(x_1,\\ldots,x_n~)
est libre, on peut simplifier par \x_1
∧\\ldots~ ∧
x_n-1\^2 et on obtient
\x_1
∧\\ldots~ ∧
x_n-1\^2
= \mathrm{det}~
Gram(x_1,\\\ldots,x_n-1~),
formule qui est encore exacte si la famille est liée puisque les deux
termes valent 0.

Remarque~12.6.4 Si
(x_1,\\ldots,x_n-1~)
est une famille libre, (ii) définit la droite engendrée par x_1
∧\\ldots~ ∧
x_n-1, (iii) définit son orientation sur cette droite et (iv)
définit sa norme, ce qui fournit une construction géométrique du produit
vectoriel~: c'est le vecteur orthogonal à l'hyperplan
\mathrmVect(x_1,\\\ldots,x_n-1~),
tel que la base
(x_1,\\ldots,x_n-1,x_1~
∧\\ldots~ ∧
x_n-1) soit une base directe de E et dont la norme est
\sqrt\\mathrm{det}
  Gram (x_1 ~ ,
\\ldots~ ,
x_n-1  ).

Coordonnées du produit vectoriel

Soit \mathcal{E} une base orthonormée directe de E, x_\\jmathmath
= \\sum ~
_i=1^n\alpha_i,\\jmathmathe_i, y
= \\sum ~
_i=1^ny_ie_i. On a alors, en développant
le déterminant suivant la dernière colonne

\begin{align*}
{[}x_1,\\ldots,x_n-1~,y{]}&
=& \left
\matrix\,\alpha_1,1&\\ldots&\alpha_1,n-1&y_1~
\cr
\\ldots~
&\\ldots&\\\ldots~
&\\ldots~
\cr
\alpha_n,1&\\ldots&\alpha_n,n-1&y_n~\right
 = \sum _i=1^n\Delta_
iy_i\%& \\ & =&
(\sum _i=1^n\Delta_
ie_i∣y) \%&
\\ \end{align*}

avec

\Delta_i = (-1)^n+i\left
\matrix\,\alpha_1,1
&\\ldots&\alpha_1,n-1~
\cr
\\ldots~
&\\ldots&\\\ldots~
\cr
\alpha_i-1,1&\\ldots&\alpha_i-1,n-1~
\cr
\alpha_i+1,1&\\ldots&\alpha_i+1,n-1~
\cr
\\ldots~
&\\ldots&\\\ldots~
\cr \alpha_n,1
&\\ldots&\alpha_n,n-1~
\right 

On en déduit que

x_1
∧\\ldots~ ∧
x_n-1 = \\sum
_i=1^n\Delta_ ie_i

Produit vectoriel en dimension 3

On a alors \x_1 ∧
x_2\^2
= \mathrm{det}~
Gram(x_1,x_2~)
=\
x_1\^2\x_2\^2
- (x_1∣x_2)^2
=\
x_1\^2\x_2\^2(1
- cos ^2~\theta) où \theta désigne l'angle non
orienté des vecteurs x_1 et x_2. On a donc alors

\x_1 ∧
x_2\ =\
x_1\\,\x_2\\
sin \theta

On a également le résultat important suivant

Théorème~12.6.16 Soit E un espace euclidien de dimension 3,
x_1,x_2,x_3 \in E. Alors

(x_1 ∧ x_2) ∧ x_3 =
(x_1∣x_3)x_2 -
(x_2∣x_3)x_1

Démonstration Si (x_1,x_2) est liée, on a par exemple
x_2 = \lambda~x_1 et on vérifie facilement que les deux
membres valent 0. Si (x_1,x_2) est libre, alors
(x_1 ∧ x_2) ∧ x_3 \in (x_1 ∧
x_2)^\bot =\
\mathrmVect(x_1,x_2), si bien
qu'a priori (x_1 ∧ x_2) ∧ x_3 = \lambda~x_1
+ \mux_2. Un calcul sur les coordonnées dans une base orthonormée
directe adéquate (par exemple ( x_1 \over
\x_1\ ,
x_1∧x_2 \over
\x_1∧x_2\
,\\ldots~)) fournit
les valeurs de \lambda~ et \mu.

Remarque~12.6.5 On pourra utiliser le moyen mnémotechnique suivant~: le
produit scalaire affecté du signe + concerne les deux termes extrêmes de
l'expression (x_1 ∧ x_2) ∧ x_3.

Corollaire~12.6.17 Soit a\neq~0. L'équation x ∧ a
= b a une solution si et seulement si~a \bot b.

Démonstration Il est clair que la condition est nécessaire. Si elle est
vérifiée, cherchons x sous la forme x_0 = \lambda~a ∧ b. On a alors

x_0 ∧ a = \lambda~(a ∧ b) ∧ a = \lambda~(a∣a)b -
\lambda~(a∣b)a =
\lambda~\a\^2b

Donc x_0 = 1 \over
\a\^2 a ∧
b est une solution.

Remarque~12.6.6 On a alors

\begin{align*} x ∧ a = b&
\Leftrightarrow & x ∧ a = x_0 ∧ a
\Leftrightarrow (x - x_0) ∧ a = 0\%&
\\ & \Leftrightarrow & x -
x_0 = \lambda~a \%& \\
\end{align*}

\paragraph{12.6.7 Angles}

On désigne par E un espace euclidien (de dimension finie), par O(E)
(resp. O^+(E)) le groupe orthogonal (resp. le groupe des
rotations de E).

Notion générale d'angles d'ob\\jmathmathets

Soit X un ensemble de parties de E stable par O(E), c'est-à-dire que

\forall~~r \in O(E)\quad
\forall~~A \in X\quad r(A) \in X.

Exemple~: X peut être l'ensemble D(E) des droites de E, ou l'ensemble
\tildeD(\mathcal{E}) des demi-droites de E, ou l'ensemble des
plans de E, ou l'ensemble des hyperplans de E.

Définition~12.6.4 On appelle angle non orienté (resp angle orienté)
d'éléments de X le quotient de X \times X par la relation \mathcal{R} définie par

\begin{align*}
(D_1,D_2)\mathcal{R}(D_1',D_2')
\Leftrightarrow&& \%& \\
& & \exists~r \in O(E) (\textresp.
O^+(E))\quad r(D_ 1) =
D_1'\text et r(D_2) =
D_2'\%& \\
\end{align*}

On notera \overline(D_1,D_2)(resp.
\widehat(D_1,D_2)) la classe
d'équivalence du couple (D_1,D_2) et on l'appellera
l'angle non orienté (resp. l'angle orienté) des ob\\jmathmathets D_1 et
D_2.

Cela revient à définir les angles par les propriétés

\overline(D_1,D_2) =
\overline(D_1',D_2')
\Leftrightarrow \exists~r \in O(E),
\quad r(D_1) = D_1',r(D_2) =
D_2'

et

\widehat(D_1,D_2)
=\widehat (D_1',D_2')
\Leftrightarrow \exists~r \in
O^+(E), \quad r(D_ 1) =
D_1',r(D_2) = D_2'

On s'intéressera par la suite uniquement au cas où X est l'ensemble D(E)
des droites de E ou l'ensemble \tildeD(\mathcal{E}) des
demi-droites de E (angles de droites ou de demi droites).

Comparaison des angles orientés et non orientés

Théorème~12.6.18 Si dim~ E ≥ 3 les notions
d'angles orientés ou non orientés coïncident aussi bien pour les droites
que pour les demi-droites, c'est-à-dire que

\existsr \in O(E)\quad r(D_1~)
= D_1'\text et r(D_2) =
D_2'

si et seulement si

\existsr \in O^+~(E)\quad
r(D_ 1) = D_1'\text et
r(D_2) = D_2'.

Démonstration L'implication '' ⇐'' est claire, et si la propriété de
gauche est vérifiée, soit r appartient à O^+(E) et c'est
terminé, soit r appartient à O^-(E), mais alors il suffit de
composer r par une symétrie s par rapport à un hyperplan contenant
D_1' et D_2' pour trouver un r' = s \cdot r \in
O^+(E) tel que r'(D_1) = D_1' et
r'(D_2) = D_2', car s laisse invariantes D_1'
et D_2'.

Mesure des angles non orientés de droites ou de demi-droites

Pour (D_1,D_2) \inD(E)^2, on définit
\phi(D_1,D_2) de la manière suivante~: soit x_1
un vecteur directeur de D_1 et x_2 un vecteur
directeur de D_2, le réel 
(x_1∣x_2)
\over
\x_1\
\x_2\ \in
{[}0,1{]} est indépendant du choix des vecteurs directeurs de
D_1 et D_2 (changer x_1 en \lambda~x_1 et
x_2 en \mux_2 avec \lambda~\neq~0 et
\mu\neq~0 ne change pas sa valeur), on le définit
comme \phi(D_1,D_2).

Théorème~12.6.19 \forall~(D_1,D_2~)
\inD(E)^2\quad
\overline(D_1,D_2) =
\overline(D_1',D_2')\quad
\Leftrightarrow \quad
\phi(D_1,D_2) = \phi(D_1',D_2').

Démonstration ( \rigtharrow~). Il suffit de remarquer que si r \in O(E) alors


(r(x_1)∣r(x_2))
\over
\r(x_1)\
\r(x_2)\
=
(x_1∣x_2)
\over
\x_1\
\x_2\ .

( ⇐). Supposons que \phi(D_1,D_2) =
\phi(D_1',D_2')\neq~1. Soit
d'abord r \in O(E) qui envoie D_1 sur D_1' et le plan
Vect(D_1,D_2) sur le plan
Vect(D_1',D_2') (la construire en prenant des bonnes
bases orthonormées). Alors si on pose D_3 = r(D_2), on
a \phi(D_1',D_3) = \phi(D_1,D_2) =
\phi(D_1',D_2'). Dans le plan
Vect(D_1',D_2') (qui contient les trois droites
D_1', D_2' et D_3) ceci impose que soit
D_3 = D_2' (et dans ce cas on a trouvé r tel que
r(D_1) = D_1' et r(D_2) = D_2'),
soit D_3 et D_2' sont symétriques par rapport à
D_1, auquel cas en composant r par la symétrie orthogonale par
rapport à l'hyperplan D_1' \oplus~ V
ect(D_1',D_2')^\bot on trouve un r' tel que
r'(D_1) = D_1' et r'(D_2) = D_2'.

Si \phi(D_1,D_2) = \phi(D_1',D_2') = 1, on
a D_1 = D_2, D_1' = D_2' et il
suffit de choisir un r tel que r(D_1) = D_1'.

Définition~12.6.5 On appelle mesure de l'angle non orienté des droites
D_1 et D_2 l'unique réel \theta \in {[}0,\pi~\diagup2{]} tel que
cos \theta = \phi(D_1,D_2~).

Le théorème précédent montre que deux angles non orientés de droites
sont égaux si et seulement si leurs mesures sont égales.

Pour les demi-droites on suit un plan analogue en posant cette fois
\phi(D_1,D_2) =
(x_1∣x_2)
\over
\x_1\
\x_2\ \in
{[}-1,1{]} qui ne dépend pas du choix des vecteurs directeurs des
demi-droites (car cette fois \lambda~ et \mu sont nécessairement positifs). On a
le même théorème (avec une démonstration analogue) et on peut donc poser

Définition~12.6.6 On appelle mesure de l'angle non orienté des
demi-droites D_1 et D_2 l'unique réel \theta \in {[}0,\pi~{]}
tel que cos \theta = \phi(D_1,D_2~).

Le théorème montre que deux angles non orientés de demi-droites sont
égaux si et seulement si leurs mesures sont égales.

Angles orientés de demi-droites dans le plan euclidien

On notera \tildeA(\mathcal{E}) l'ensemble des angles orientés
de demi-droites du plan euclidien E.

Théorème~12.6.20 Soit D \in\tildeD(\mathcal{E}). Alors
l'application f : O^+(E) \rightarrow~\tildeA(\mathcal{E}),
r\mapsto~\widehat(D,r(D)) est une
bi\\jmathmathection qui ne dépend pas du choix de D.

Démonstration Pour l'in\\jmathmathectivité, si on a f(r) = f(r') c'est qu'il
existe r'' \in O^+(E) tel que r''(D) = D et r'`\cdot r(D) = r'(D).
Mais la première relation impose que r'' = \mathrmId
(une rotation du plan euclidien qui laisse invariante une demi-droite
est l'identité) et la deuxième que r^-1 \cdot r'(D) = D soit r' =
r pour la même raison. En ce qui concerne l'indépendance de D il suffit
de remarquer que si D' \in\tildeD(\mathcal{E}), il existe
r_0 \in O^+(E) tel que D' = r_0(D) et alors

\widehat(D',r(D')) =\widehat
(r_0(D),r \cdot r_0(D)) =\widehat
(r_0(D),r_0 \cdot r(D)) =\widehat
(D,r(D))

(on voit ici le rôle essentiel de la commutativité de O^+(E)
en dimension 2). La sur\\jmathmathectivité provient alors du fait que tout angle
\widehat(D_1,D_2) est de la forme
\widehat(D_1,r(D_1)) = f(r) en
utilisant une rotation r qui envoie D_1 sur D_2.

Cette bi\\jmathmathection naturelle permet de transporter la structure de groupe
commutatif de O^+(E) à \tildeA(\mathcal{E}) en
posant~:

si \theta_1 = f(r_1) et \theta_2 = f(r_2) ,
alors on pose \theta_1 + \theta_2 = f(r_1 \cdot
r_2).

On définit de plus l'angle nul comme étant
f(\mathrmId) =\widehat (D,D),
l'angle plat comme étant f(-\mathrmId)
=\widehat (D,-D).

Théorème~12.6.21 (relation de Chasles)
\widehat(D_1,D_2)
+\widehat (D_2,D_3)
=\widehat (D_1,D_3).

Démonstration Soit r_1 \in O^+(E) telle que
r_1(D_1) = D_2 et r_2 \in
O^+(E) telle que r_2(D_2) = D_3.
Alors

\widehat(D_1,D_2)
+\widehat (D_2,D_3) =
f(r_1) + f(r_2) = f(r_2 \cdot r_1)
=\widehat (D_1,r_2 \cdot
r_1(D_1)) =\widehat
(D_1,D_3)

On retrouve alors sans difficulté toutes les propriétés des angles de
demi-droites. Comme de plus, si E est orienté, on connait un
isomorphisme de groupes abéliens entre \mathbb{R}~\diagup2\pi~\mathbb{Z} et O^+(E), on en
déduit un isomorphisme entre \tildeA(\mathcal{E}) et \mathbb{R}~\diagup2\pi~\mathbb{Z} qui
permet de mesurer les angles modulo 2\pi~ et d'effectuer les calculs (en
particulier les divisions par 2) dans \mathbb{R}~\diagup2\pi~\mathbb{Z}. On en déduit par exemple
facilement qu'un couple de demi-droites a deux bissectrices opposés.

Angles orientés de droites dans le plan euclidien

On notera A(E) l'ensemble des angles orientés de droites du plan
euclidien E. La seule différence est que f n'est plus in\\jmathmathective, mais
que f(r_1) = f(r_2) \Leftrightarrow
r_2 = ±r_1. On en déduit que f induit une bi\\jmathmathection
\tildef de O^+(E)\diagup\
±\mathrmId\ sur A(E). On peut donc
encore transporter la structure de groupe de
O^+(E)\diagup\
±\mathrmId\ sur A(E) et on obtient
encore une relation de Chasles. Si E est orienté, on obtient un
isomorphisme de A(E) avec \mathbb{R}~\diagup\pi~\mathbb{Z} qui permet de mesurer les angles de
droites modulo \pi~.

Exemple~12.6.2 L'application \theta\mapsto~2\theta est un
isomorphisme de groupes de \mathbb{R}~\diagup\pi~\mathbb{Z} sur \mathbb{R}~\diagup2\pi~\mathbb{Z}. On en déduit un isomorphisme
encore noté \theta\mapsto~2\theta de A(E) sur
\tildeA(\mathcal{E}). Soit D_1 et D_2 deux
droites de E, s_1 et s_2 les symétries orthogonales
par rapport à ces droites. Montrer que s_2 \cdot s_1 est
la rotation d'angle (de demi-droites)
2\widehat(D_1,D_2).

{[}
{[}
{[}
{[}

\end{document}

\part{Formes hermitiennes}
% \documentclass[]{article}
\usepackage[T1]{fontenc}
\usepackage{lmodern}
\usepackage{amssymb,amsmath}
\usepackage{ifxetex,ifluatex}
\usepackage{fixltx2e} % provides \textsubscript
% use upquote if available, for straight quotes in verbatim environments
\IfFileExists{upquote.sty}{\usepackage{upquote}}{}
\ifnum 0\ifxetex 1\fi\ifluatex 1\fi=0 % if pdftex
  \usepackage[utf8]{inputenc}
\else % if luatex or xelatex
  \ifxetex
    \usepackage{mathspec}
    \usepackage{xltxtra,xunicode}
  \else
    \usepackage{fontspec}
  \fi
  \defaultfontfeatures{Mapping=tex-text,Scale=MatchLowercase}
  \newcommand{\euro}{€}
\fi
% use microtype if available
\IfFileExists{microtype.sty}{\usepackage{microtype}}{}
\ifxetex
  \usepackage[setpagesize=false, % page size defined by xetex
              unicode=false, % unicode breaks when used with xetex
              xetex]{hyperref}
\else
  \usepackage[unicode=true]{hyperref}
\fi
\hypersetup{breaklinks=true,
            bookmarks=true,
            pdfauthor={},
            pdftitle={Complements sur la conjugaison},
            colorlinks=true,
            citecolor=blue,
            urlcolor=blue,
            linkcolor=magenta,
            pdfborder={0 0 0}}
\urlstyle{same}  % don't use monospace font for urls
\setlength{\parindent}{0pt}
\setlength{\parskip}{6pt plus 2pt minus 1pt}
\setlength{\emergencystretch}{3em}  % prevent overfull lines
\setcounter{secnumdepth}{0}
 
/* start css.sty */
.cmr-5{font-size:50%;}
.cmr-7{font-size:70%;}
.cmmi-5{font-size:50%;font-style: italic;}
.cmmi-7{font-size:70%;font-style: italic;}
.cmmi-10{font-style: italic;}
.cmsy-5{font-size:50%;}
.cmsy-7{font-size:70%;}
.cmex-7{font-size:70%;}
.cmex-7x-x-71{font-size:49%;}
.msbm-7{font-size:70%;}
.cmtt-10{font-family: monospace;}
.cmti-10{ font-style: italic;}
.cmbx-10{ font-weight: bold;}
.cmr-17x-x-120{font-size:204%;}
.cmsl-10{font-style: oblique;}
.cmti-7x-x-71{font-size:49%; font-style: italic;}
.cmbxti-10{ font-weight: bold; font-style: italic;}
p.noindent { text-indent: 0em }
td p.noindent { text-indent: 0em; margin-top:0em; }
p.nopar { text-indent: 0em; }
p.indent{ text-indent: 1.5em }
@media print {div.crosslinks {visibility:hidden;}}
a img { border-top: 0; border-left: 0; border-right: 0; }
center { margin-top:1em; margin-bottom:1em; }
td center { margin-top:0em; margin-bottom:0em; }
.Canvas { position:relative; }
li p.indent { text-indent: 0em }
.enumerate1 {list-style-type:decimal;}
.enumerate2 {list-style-type:lower-alpha;}
.enumerate3 {list-style-type:lower-roman;}
.enumerate4 {list-style-type:upper-alpha;}
div.newtheorem { margin-bottom: 2em; margin-top: 2em;}
.obeylines-h,.obeylines-v {white-space: nowrap; }
div.obeylines-v p { margin-top:0; margin-bottom:0; }
.overline{ text-decoration:overline; }
.overline img{ border-top: 1px solid black; }
td.displaylines {text-align:center; white-space:nowrap;}
.centerline {text-align:center;}
.rightline {text-align:right;}
div.verbatim {font-family: monospace; white-space: nowrap; text-align:left; clear:both; }
.fbox {padding-left:3.0pt; padding-right:3.0pt; text-indent:0pt; border:solid black 0.4pt; }
div.fbox {display:table}
div.center div.fbox {text-align:center; clear:both; padding-left:3.0pt; padding-right:3.0pt; text-indent:0pt; border:solid black 0.4pt; }
div.minipage{width:100%;}
div.center, div.center div.center {text-align: center; margin-left:1em; margin-right:1em;}
div.center div {text-align: left;}
div.flushright, div.flushright div.flushright {text-align: right;}
div.flushright div {text-align: left;}
div.flushleft {text-align: left;}
.underline{ text-decoration:underline; }
.underline img{ border-bottom: 1px solid black; margin-bottom:1pt; }
.framebox-c, .framebox-l, .framebox-r { padding-left:3.0pt; padding-right:3.0pt; text-indent:0pt; border:solid black 0.4pt; }
.framebox-c {text-align:center;}
.framebox-l {text-align:left;}
.framebox-r {text-align:right;}
span.thank-mark{ vertical-align: super }
span.footnote-mark sup.textsuperscript, span.footnote-mark a sup.textsuperscript{ font-size:80%; }
div.tabular, div.center div.tabular {text-align: center; margin-top:0.5em; margin-bottom:0.5em; }
table.tabular td p{margin-top:0em;}
table.tabular {margin-left: auto; margin-right: auto;}
div.td00{ margin-left:0pt; margin-right:0pt; }
div.td01{ margin-left:0pt; margin-right:5pt; }
div.td10{ margin-left:5pt; margin-right:0pt; }
div.td11{ margin-left:5pt; margin-right:5pt; }
table[rules] {border-left:solid black 0.4pt; border-right:solid black 0.4pt; }
td.td00{ padding-left:0pt; padding-right:0pt; }
td.td01{ padding-left:0pt; padding-right:5pt; }
td.td10{ padding-left:5pt; padding-right:0pt; }
td.td11{ padding-left:5pt; padding-right:5pt; }
table[rules] {border-left:solid black 0.4pt; border-right:solid black 0.4pt; }
.hline hr, .cline hr{ height : 1px; margin:0px; }
.tabbing-right {text-align:right;}
span.TEX {letter-spacing: -0.125em; }
span.TEX span.E{ position:relative;top:0.5ex;left:-0.0417em;}
a span.TEX span.E {text-decoration: none; }
span.LATEX span.A{ position:relative; top:-0.5ex; left:-0.4em; font-size:85%;}
span.LATEX span.TEX{ position:relative; left: -0.4em; }
div.float img, div.float .caption {text-align:center;}
div.figure img, div.figure .caption {text-align:center;}
.marginpar {width:20%; float:right; text-align:left; margin-left:auto; margin-top:0.5em; font-size:85%; text-decoration:underline;}
.marginpar p{margin-top:0.4em; margin-bottom:0.4em;}
.equation td{text-align:center; vertical-align:middle; }
td.eq-no{ width:5%; }
table.equation { width:100%; } 
div.math-display, div.par-math-display{text-align:center;}
math .texttt { font-family: monospace; }
math .textit { font-style: italic; }
math .textsl { font-style: oblique; }
math .textsf { font-family: sans-serif; }
math .textbf { font-weight: bold; }
.partToc a, .partToc, .likepartToc a, .likepartToc {line-height: 200%; font-weight:bold; font-size:110%;}
.chapterToc a, .chapterToc, .likechapterToc a, .likechapterToc, .appendixToc a, .appendixToc {line-height: 200%; font-weight:bold;}
.index-item, .index-subitem, .index-subsubitem {display:block}
.caption td.id{font-weight: bold; white-space: nowrap; }
table.caption {text-align:center;}
h1.partHead{text-align: center}
p.bibitem { text-indent: -2em; margin-left: 2em; margin-top:0.6em; margin-bottom:0.6em; }
p.bibitem-p { text-indent: 0em; margin-left: 2em; margin-top:0.6em; margin-bottom:0.6em; }
.paragraphHead, .likeparagraphHead { margin-top:2em; font-weight: bold;}
.subparagraphHead, .likesubparagraphHead { font-weight: bold;}
.quote {margin-bottom:0.25em; margin-top:0.25em; margin-left:1em; margin-right:1em; text-align:justify;}
.verse{white-space:nowrap; margin-left:2em}
div.maketitle {text-align:center;}
h2.titleHead{text-align:center;}
div.maketitle{ margin-bottom: 2em; }
div.author, div.date {text-align:center;}
div.thanks{text-align:left; margin-left:10%; font-size:85%; font-style:italic; }
div.author{white-space: nowrap;}
.quotation {margin-bottom:0.25em; margin-top:0.25em; margin-left:1em; }
h1.partHead{text-align: center}
.sectionToc, .likesectionToc {margin-left:2em;}
.subsectionToc, .likesubsectionToc {margin-left:4em;}
.subsubsectionToc, .likesubsubsectionToc {margin-left:6em;}
.frenchb-nbsp{font-size:75%;}
.frenchb-thinspace{font-size:75%;}
.figure img.graphics {margin-left:10%;}
/* end css.sty */

\title{Complements sur la conjugaison}
\author{}
\date{}

\begin{document}
\maketitle

\textbf{Warning: 
requires JavaScript to process the mathematics on this page.\\ If your
browser supports JavaScript, be sure it is enabled.}

\begin{center}\rule{3in}{0.4pt}\end{center}

[
[]
[

\subsubsection{13.1 Compléments sur la conjugaison}

\paragraph{13.1.1 Applications semi-linéaires}

Définition~13.1.1 Soit E et F deux \mathbb{C}-espaces vectoriels et u : E \rightarrow~ F. On
dit que u est semi-linéaire si elle vérifie

\begin{itemize}
\itemsep1pt\parskip0pt\parsep0pt
\item
  (i) \forall~~x,y \in E, u(x + y) = u(x) + u(y)
\item
  (ii) \forall~\lambda~ \in \mathbb{C}, \\forall~~x \in
  E, u(\lambda~x) = \overline\lambda~u(x).
\end{itemize}

Remarque~13.1.1 Soit E un \mathbb{C}-espace vectoriel . On munit E d'une autre
structure d'espace vectoriel, notée \checkE en posant
\lambda~ ∗ x = \overline\lambda~x. Une application semi-linéaire de
E dans F n'est autre qu'une application linéaire de E dans
\checkF. Ceci permet d'appliquer aux applications
semi-linéaires la plupart des résultats sur les applications linéaires
en tenant compte des résultats suivants dont la démonstration est
élémentaire~:

\begin{itemize}
\item
  a) une famille (x_i)_i\inI d'éléments de E est libre
  (resp. génératrice, resp. base) dans \checkE si et
  seulement si~il en est de même dans E
\item
  b) \mathrmrg~
  _\checkE(x_i)_i\inI
  = \mathrmrg~
  _E(x_i)_i\inI, dim~
  \checkE = dim~ E
\item
  c) F est un sous-espace vectoriel de \checkE si et
  seulement si~c'est un sous-espace vectoriel de E
\item
  d) le théorème du rang s'applique aux applications semi-linéaires~; en
  particulier, si u : E \rightarrow~ F est semi-linéaire entre deux espaces de même
  dimension finie, alors u est injective si et seulement si~elle est
  surjective
\item
  e) si l'on définit A =\
  \mathrmMat (u,\mathcal{E},ℱ) par u(e_j)
  = \\sum ~
  _ia_i,jf_i (notations évidentes) alors

  y = u(x) \Leftrightarrow Y =
  A\overlineX
\item
  f) la composée de deux applications semi-linéaires n'est pas
  semi-linéaire, mais au contraire linéaire.
\end{itemize}

\paragraph{13.1.2 Matrices conjuguées et transconjuguées}

Définition~13.1.2 Soit A = (a_i,j)_1\leqi\leqm,1\leqj\leqn \in
M_\mathbb{C}(m,n). On appelle matrice conjuguée de A la matrice
\overlineA =
(\overlinea_i,j)_1\leqi\leqm,1\leqj\leqn \in
M_\mathbb{C}(m,n).

Proposition~13.1.1 L'application
A\mapsto~\overlineA est un
automorphisme semi-linéaire de M_\mathbb{C}(m,n). On a
\mathrmrg\overlineA~
= \mathrmrg~A. Si A \in
M_\mathbb{C}(m,n) et B \in M_\mathbb{C}(n,p), alors
\overlineAB =
\overlineA\,\overlineB.
Dans le cadre des matrices carrées, on a
\mathrm{det}~
\overlineA =
\overline\mathrm{det}~
A,
\mathrm{tr}\overlineA~
=
\overline\mathrm{tr}A~,
\chi_\overlineA(X) =
\overline\chi_A(X), A est inversible si et
seulement si~\overlineA est inversible, et dans ce
cas (\overlineA)^-1 =
\overlineA^-1.

Démonstration Vérification élémentaire laissée au lecteur.

Définition~13.1.3 Soit A = (a_i,j)_1\leqi\leqm,1\leqj\leqn \in
M_\mathbb{C}(m,n). On appelle matrice transconjuguée (ou matrice
adjointe) de A la matrice A^∗ =
^t(\overlineA) =
\overline^tA =
(\overlinea_j,i)_1\leqi\leqm,1\leqj\leqn \in
M_\mathbb{C}(n,m).

A partir des propriétés de A\mapsto~^tA
et de A\mapsto~\overlineA on
déduit facilement les propriétés suivantes

Proposition~13.1.2 L'application
A\mapsto~A^∗ est un isomorphisme
semi-linéaire involutif de M_\mathbb{C}(m,n) sur M_\mathbb{C}(n,m). On a
\mathrmrgA^∗~
= \mathrmrg~A. Si A \in
M_\mathbb{C}(m,n) et B \in M_\mathbb{C}(n,p), alors (AB)^∗ =
B^∗A^∗. Dans le cadre des matrices carrées, on a
\mathrm{det} A^∗~ =
\overline\mathrm{det}~
A,
\mathrm{tr}A^∗~ =
\overline\mathrm{tr}A~,
\chi_A^∗(X) =
\overline\chi_A(X), A est inversible si et
seulement si~A^∗ est inversible, et dans ce cas
(A^∗)^-1 = (A^-1)^∗.

Remarque~13.1.2 On prendra garde à la relation (\lambda~A)^∗ =
\overline\lambda~A^∗ en n'oubliant pas la
conjugaison.

\paragraph{13.1.3 Matrices hermitiennes, antihermitiennes}

Définition~13.1.4 Soit A \in M_\mathbb{C}(n). on dit que A est hermitienne
(resp. antihermitienne) si A^∗ = A (resp. A^∗ =
-A).

Remarque~13.1.3 A = (a_i,j) est hermitienne si et seulement
si~\forall~i,j, a_j,i~ =
\overlinea_i,j. En particulier les
coefficients diagonaux a_i,i doivent être réels

Théorème~13.1.3 Les ensembles des matrices hermitiennes et
antihermitiennes sont des \mathbb{R}~-sous-espaces vectoriels (mais pas des
\mathbb{C}-sous-espaces vectoriels) de M_\mathbb{C}(n). On a

A\text hermitienne  \Leftrightarrow
iA\text antihermitienne

Si ℋ_n désigne le \mathbb{R}~-sous-espace vectoriel des matrices
hermitiennes, on a M_\mathbb{C}(n) = ℋ_n \oplus~ iℋ_n.

Démonstration La vérification du premier point est élémentaire. Si on a
A = A_1 + iA_2 avec A_1 et A_2
hermitiennes, alors A^∗ = A_1 - iA_2 ce qui
donne A_1 = 1 \over 2 (A + A^∗)
et A_2 = 1 \over 2i (A - A^∗) et
démontre déjà l'unicité de la décomposition. De plus la formule

A = 1 \over 2 (A + A^∗) + i 1
\over 2i (A - A^∗)

avec  1 \over 2 (A + A^∗) et  1
\over 2i (A - A^∗) qui sont hermitiennes
(facile) montre l'existence de la décomposition.

Remarque~13.1.4 On voit donc que contrairement aux matrices symétriques
ou antisymétriques qui sont de nature différentes, il n'y a pas de
différence essentielle entre matrices hermitiennes ou antihermitiennes~:
on passe des unes aux autres par multiplication par i, ce qui permet de
limiter l'étude aux matrices hermitiennes. Pour une telle matrice, les
formules \mathrm{det}~
A^∗ =
\overline\mathrm{det}~
A,
\mathrm{tr}A^∗~ =
\overline\mathrm{tr}A~,
\chi_A^∗(X) =
\overline\chi_A(X) montrent que
\mathrm{det}~ A \in \mathbb{R}~,
\mathrm{tr}~A \in \mathbb{R}~ et que
\chi_A(X) \in \mathbb{R}~[X].

[
[

\end{document}

% \documentclass[]{article}
\usepackage[T1]{fontenc}
\usepackage{lmodern}
\usepackage{amssymb,amsmath}
\usepackage{ifxetex,ifluatex}
\usepackage{fixltx2e} % provides \textsubscript
% use upquote if available, for straight quotes in verbatim environments
\IfFileExists{upquote.sty}{\usepackage{upquote}}{}
\ifnum 0\ifxetex 1\fi\ifluatex 1\fi=0 % if pdftex
  \usepackage[utf8]{inputenc}
\else % if luatex or xelatex
  \ifxetex
    \usepackage{mathspec}
    \usepackage{xltxtra,xunicode}
  \else
    \usepackage{fontspec}
  \fi
  \defaultfontfeatures{Mapping=tex-text,Scale=MatchLowercase}
  \newcommand{\euro}{€}
\fi
% use microtype if available
\IfFileExists{microtype.sty}{\usepackage{microtype}}{}
\ifxetex
  \usepackage[setpagesize=false, % page size defined by xetex
              unicode=false, % unicode breaks when used with xetex
              xetex]{hyperref}
\else
  \usepackage[unicode=true]{hyperref}
\fi
\hypersetup{breaklinks=true,
            bookmarks=true,
            pdfauthor={},
            pdftitle={Formes sesquilineaires},
            colorlinks=true,
            citecolor=blue,
            urlcolor=blue,
            linkcolor=magenta,
            pdfborder={0 0 0}}
\urlstyle{same}  % don't use monospace font for urls
\setlength{\parindent}{0pt}
\setlength{\parskip}{6pt plus 2pt minus 1pt}
\setlength{\emergencystretch}{3em}  % prevent overfull lines
\setcounter{secnumdepth}{0}
 
/* start css.sty */
.cmr-5{font-size:50%;}
.cmr-7{font-size:70%;}
.cmmi-5{font-size:50%;font-style: italic;}
.cmmi-7{font-size:70%;font-style: italic;}
.cmmi-10{font-style: italic;}
.cmsy-5{font-size:50%;}
.cmsy-7{font-size:70%;}
.cmex-7{font-size:70%;}
.cmex-7x-x-71{font-size:49%;}
.msbm-7{font-size:70%;}
.cmtt-10{font-family: monospace;}
.cmti-10{ font-style: italic;}
.cmbx-10{ font-weight: bold;}
.cmr-17x-x-120{font-size:204%;}
.cmsl-10{font-style: oblique;}
.cmti-7x-x-71{font-size:49%; font-style: italic;}
.cmbxti-10{ font-weight: bold; font-style: italic;}
p.noindent { text-indent: 0em }
td p.noindent { text-indent: 0em; margin-top:0em; }
p.nopar { text-indent: 0em; }
p.indent{ text-indent: 1.5em }
@media print {div.crosslinks {visibility:hidden;}}
a img { border-top: 0; border-left: 0; border-right: 0; }
center { margin-top:1em; margin-bottom:1em; }
td center { margin-top:0em; margin-bottom:0em; }
.Canvas { position:relative; }
li p.indent { text-indent: 0em }
.enumerate1 {list-style-type:decimal;}
.enumerate2 {list-style-type:lower-alpha;}
.enumerate3 {list-style-type:lower-roman;}
.enumerate4 {list-style-type:upper-alpha;}
div.newtheorem { margin-bottom: 2em; margin-top: 2em;}
.obeylines-h,.obeylines-v {white-space: nowrap; }
div.obeylines-v p { margin-top:0; margin-bottom:0; }
.overline{ text-decoration:overline; }
.overline img{ border-top: 1px solid black; }
td.displaylines {text-align:center; white-space:nowrap;}
.centerline {text-align:center;}
.rightline {text-align:right;}
div.verbatim {font-family: monospace; white-space: nowrap; text-align:left; clear:both; }
.fbox {padding-left:3.0pt; padding-right:3.0pt; text-indent:0pt; border:solid black 0.4pt; }
div.fbox {display:table}
div.center div.fbox {text-align:center; clear:both; padding-left:3.0pt; padding-right:3.0pt; text-indent:0pt; border:solid black 0.4pt; }
div.minipage{width:100%;}
div.center, div.center div.center {text-align: center; margin-left:1em; margin-right:1em;}
div.center div {text-align: left;}
div.flushright, div.flushright div.flushright {text-align: right;}
div.flushright div {text-align: left;}
div.flushleft {text-align: left;}
.underline{ text-decoration:underline; }
.underline img{ border-bottom: 1px solid black; margin-bottom:1pt; }
.framebox-c, .framebox-l, .framebox-r { padding-left:3.0pt; padding-right:3.0pt; text-indent:0pt; border:solid black 0.4pt; }
.framebox-c {text-align:center;}
.framebox-l {text-align:left;}
.framebox-r {text-align:right;}
span.thank-mark{ vertical-align: super }
span.footnote-mark sup.textsuperscript, span.footnote-mark a sup.textsuperscript{ font-size:80%; }
div.tabular, div.center div.tabular {text-align: center; margin-top:0.5em; margin-bottom:0.5em; }
table.tabular td p{margin-top:0em;}
table.tabular {margin-left: auto; margin-right: auto;}
div.td00{ margin-left:0pt; margin-right:0pt; }
div.td01{ margin-left:0pt; margin-right:5pt; }
div.td10{ margin-left:5pt; margin-right:0pt; }
div.td11{ margin-left:5pt; margin-right:5pt; }
table[rules] {border-left:solid black 0.4pt; border-right:solid black 0.4pt; }
td.td00{ padding-left:0pt; padding-right:0pt; }
td.td01{ padding-left:0pt; padding-right:5pt; }
td.td10{ padding-left:5pt; padding-right:0pt; }
td.td11{ padding-left:5pt; padding-right:5pt; }
table[rules] {border-left:solid black 0.4pt; border-right:solid black 0.4pt; }
.hline hr, .cline hr{ height : 1px; margin:0px; }
.tabbing-right {text-align:right;}
span.TEX {letter-spacing: -0.125em; }
span.TEX span.E{ position:relative;top:0.5ex;left:-0.0417em;}
a span.TEX span.E {text-decoration: none; }
span.LATEX span.A{ position:relative; top:-0.5ex; left:-0.4em; font-size:85%;}
span.LATEX span.TEX{ position:relative; left: -0.4em; }
div.float img, div.float .caption {text-align:center;}
div.figure img, div.figure .caption {text-align:center;}
.marginpar {width:20%; float:right; text-align:left; margin-left:auto; margin-top:0.5em; font-size:85%; text-decoration:underline;}
.marginpar p{margin-top:0.4em; margin-bottom:0.4em;}
.equation td{text-align:center; vertical-align:middle; }
td.eq-no{ width:5%; }
table.equation { width:100%; } 
div.math-display, div.par-math-display{text-align:center;}
math .texttt { font-family: monospace; }
math .textit { font-style: italic; }
math .textsl { font-style: oblique; }
math .textsf { font-family: sans-serif; }
math .textbf { font-weight: bold; }
.partToc a, .partToc, .likepartToc a, .likepartToc {line-height: 200%; font-weight:bold; font-size:110%;}
.chapterToc a, .chapterToc, .likechapterToc a, .likechapterToc, .appendixToc a, .appendixToc {line-height: 200%; font-weight:bold;}
.index-item, .index-subitem, .index-subsubitem {display:block}
.caption td.id{font-weight: bold; white-space: nowrap; }
table.caption {text-align:center;}
h1.partHead{text-align: center}
p.bibitem { text-indent: -2em; margin-left: 2em; margin-top:0.6em; margin-bottom:0.6em; }
p.bibitem-p { text-indent: 0em; margin-left: 2em; margin-top:0.6em; margin-bottom:0.6em; }
.paragraphHead, .likeparagraphHead { margin-top:2em; font-weight: bold;}
.subparagraphHead, .likesubparagraphHead { font-weight: bold;}
.quote {margin-bottom:0.25em; margin-top:0.25em; margin-left:1em; margin-right:1em; text-align:\jmathustify;}
.verse{white-space:nowrap; margin-left:2em}
div.maketitle {text-align:center;}
h2.titleHead{text-align:center;}
div.maketitle{ margin-bottom: 2em; }
div.author, div.date {text-align:center;}
div.thanks{text-align:left; margin-left:10%; font-size:85%; font-style:italic; }
div.author{white-space: nowrap;}
.quotation {margin-bottom:0.25em; margin-top:0.25em; margin-left:1em; }
h1.partHead{text-align: center}
.sectionToc, .likesectionToc {margin-left:2em;}
.subsectionToc, .likesubsectionToc {margin-left:4em;}
.subsubsectionToc, .likesubsubsectionToc {margin-left:6em;}
.frenchb-nbsp{font-size:75%;}
.frenchb-thinspace{font-size:75%;}
.figure img.graphics {margin-left:10%;}
/* end css.sty */

\title{Formes sesquilineaires}
\author{}
\date{}

\begin{document}
\maketitle

\textbf{Warning: 
requires JavaScript to process the mathematics on this page.\\ If your
browser supports JavaScript, be sure it is enabled.}

\begin{center}\rule{3in}{0.4pt}\end{center}

{[}
{[}
{[}{]}
{[}

\subsubsection{13.2 Formes sesquilinéaires}

\paragraph{13.2.1 Généralités}

Définition~13.2.1 Soit E un \mathbb{C}-espace vectoriel . On appelle forme
sesquilinéaire sur E toute application \phi : E \times E \rightarrow~ \mathbb{C} telle que

\begin{itemize}
\itemsep1pt\parskip0pt\parsep0pt
\item
  (i) \forall~~x \in E,
  y\mapsto~\phi(x,y) est linéaire
\item
  (ii) \forall~~y \in E,
  x\mapsto~\phi(x,y) est semilinéaire
\end{itemize}

Remarque~13.2.1 On a en particulier \forall~~y \in E,
\phi(y,0) = \phi(0,y) = 0~; de plus \phi(x,\lambda~y) = \lambda~\phi(x,y), \phi(\lambda~x,y) =
\overline\lambda~\phi(x,y). Plus généralement
\phi(\\sum ~
\lambda~\_ix\_i,\\\sum
 \mu\_\jmathy\_\jmath) =\
\sum ~
\_i,\jmath\overline\lambda~\_i\mu\_\jmath\phi(x\_i,y\_\jmath).

Il est clair que si \phi et \psi sont deux formes sesquilinéaires sur E, il en
est de même de \alpha~\phi + \beta~\psi, d'où la proposition

Proposition~13.2.1 L'ensemble L\_3\diagup2(E) des formes
sesquilinéaires sur E est un sous-espace vectoriel de l'espace
\mathbb{C}^E\timesE des applications de E \times E dans \mathbb{C}.

Remarque~13.2.2 Soit \phi une forme sesquilinéaire sur E. Pour chaque x \in
E, l'application y\mapsto~\phi(x,y) est une forme
linéaire sur E donc un élément, noté g\_\phi(x), du dual
E^∗ de E. De même, pour chaque y \in E, l'application
x\mapsto~\overline\phi(x,y) est une
forme linéaire sur E, donc un élément, noté d\_\phi(y), de
E^∗. La relation

\begin{align*} \left
{[}g\_\phi(\alpha~x + \beta~x')\right {]}(y)& =& \phi(\alpha~x +
\beta~x',y) = \overline\alpha~\phi(x,y) +
\overline\beta~\phi(x',y)\%&
\\ & =& \left
{[}\overline\alpha~g\_\phi(x) +
\overline\beta~g\_\phi(x')\right
{]}(y) \%& \\
\end{align*}

montre clairement que g\_\phi :
x\mapsto~g\_\phi(x) est une application
semilinéaire de E dans E^∗. Il en est de même de d\_\phi
: y\mapsto~d\_\phi(y).

Définition~13.2.2 L'application g\_\phi : E \rightarrow~ E^∗ (resp.
d\_\phi) est appelée l'application semilinéaire gauche (resp.
droite) associée à la forme sesquilinéaire \phi.

\paragraph{13.2.2 Formes sesquilinéaires hermitiennes, antihermitiennes}

Définition~13.2.3 Soit \phi \in L\_3\diagup2(E). On dit que \phi est
hermitienne (resp. antihermitienne) si \forall~~x,y \in
E, \phi(y,x) = \overline\phi(x,y) (resp. =
-\overline\phi(x,y)).

Proposition~13.2.2 \phi est hermitienne si et seulement si~i\phi est
antihermitienne.

Démonstration Evident

Remarque~13.2.3 Ceci nous permettra par la suite de ne considérer que le
cas des formes hermitiennes.

Proposition~13.2.3 Soit \phi \in L\_3\diagup2(E). Alors \phi est hermitienne
si et seulement si~d\_\phi = g\_\phi.

Démonstration En effet \phi(x,y) =\big
{[}g\_\phi(x)\big {]}(y) et
\overline\phi(y,x) =\big
{[}d\_\phi(x)\big {]}(y). Alors

\begin{align*} \forall~~x,y \in E,
\overline\phi(y,x) = \phi(x,y)&& \%&
\\ & \Leftrightarrow &
\forall~~x,y \in E, \big
{[}g\_\phi(x)\big {]}(y) = \epsilon\big
{[}d\_\phi(x)\big {]}(y)\%&
\\ & \Leftrightarrow &
\forall~x \in E, g\_\phi(x) = d\_\phi~(x)
\Leftrightarrow g\_\phi = d\_\phi \%&
\\ \end{align*}

Proposition~13.2.4 L'ensemble H(E) des formes sesquilinéaires
hermitiennes est un \mathbb{R}~-sous-espace vectoriel de L\_3\diagup2(E) (mais
pas un \mathbb{C} sous-espace vectoriel). On a L\_3\diagup2(E) = H(E) \oplus~ iH(E).

Démonstration La première affirmation est laissée aux soins du lecteur.
On a clairement H(E) \bigcap iH(E) = \0\ et
la relation \phi = \psi + i\theta avec \psi(x,y) = 1 \over 2
(\phi(x,y) + \phi(y,x)), \theta(x,y) = 1 \over 2i (\phi(x,y) -
\phi(y,x)) qui sont toutes deux hermitiennes montre que L\_3\diagup2(E) =
H(E) \oplus~ iH(E).

\paragraph{13.2.3 Matrice d'une forme sesquilinéaire}

Supposons que E est de dimension finie et soit \mathcal{E} =
(e\_1,\\ldots,e\_n~)
une base de E.

Définition~13.2.4 Soit \phi \in L\_3\diagup2(E). On appelle matrice de \phi
dans la base \mathcal{E} la matrice

\mathrmMat~ (\phi,\mathcal{E}) =
(\phi(e\_i,e\_\jmath))\_1\leqi,\jmath\leqn \in M\_\mathbb{C}(n)

Proposition~13.2.5
\mathrmMat~ (\phi,\mathcal{E}) est
l'unique matrice \Omega \in M\_\mathbb{C}(n) vérifiant

\forall~(x,y) \in E \times E, \phi(x,y) = X^∗~\OmegaY

où X (resp. Y ) désigne le vecteur colonne des coordonnées de x (resp.
y) dans la base \mathcal{E}.

Démonstration Si \Omega = (\omega\_i,\jmath), on a

X^∗\OmegaY = \\sum
\_i=1^n\overlinex\_ i(\OmegaY
)\_i = \\sum
\_i=1^n\overlinex\_ i
\sum \_\jmath=1^n\omega~\_
i,\jmathy\_\jmath = \\sum
\_i,\jmath\omega\_i,\jmath\overlinex\_iy\_\jmath

Mais d'autre part \phi(x,y) =
\phi(\\sum ~
\_i=1^nx\_ie\_i,\\\sum
 \_\jmath=1^ny\_\jmathe\_\jmath)
= \\sum ~
\_i,\jmath\phi(e\_i,e\_\jmath)\overlinex\_iy\_\jmath
en utilisant la sesquilinéarité de \phi. Ceci montre que
\mathrmMat~ (\phi,\mathcal{E}) vérifie
bien la relation voulue. Inversement, si \Omega vérifie cette formule, on a
\phi(e\_k,e\_l) = E\_k^∗\OmegaE\_l
= \\sum ~
\_i,\jmath\omega\_i,\jmath\delta\_i^k\delta\_\jmath^l =
\omega\_k,l ce qui montre que \Omega =\
\mathrmMat (\phi,\mathcal{E}).

Théorème~13.2.6 L'application
\phi\mapsto~\mathrmMat~
(\phi,\mathcal{E}) est un isomorphisme d'espaces vectoriels de L\_3\diagup2(E) sur
M\_\mathbb{C}(n).

Démonstration Les détails sont laissés aux soins du lecteur.
L'application réciproque est bien entendu l'application qui à \Omega \in
M\_\mathbb{C}(n) associe \phi : E \times E \rightarrow~ \mathbb{C} définie par \phi(x,y) =
X^∗\OmegaY qui est clairement sesquilinéaire.

Théorème~13.2.7 Soit E de dimension finie, \mathcal{E} =
(e\_1,\\ldots,e\_n~)
une base de E, \mathcal{E}^∗ =
(e\_1^∗,\\ldots,e\_n^∗~)
la base duale. Soit \phi \in L\_3\diagup2(E). Alors

\mathrmMat~ (\phi,\mathcal{E}) =
\overline\mathrmMat~
(d\_\phi,\mathcal{E},\mathcal{E}^∗) =
^t \mathrmMat~
(g\_ \phi,\mathcal{E},\mathcal{E}^∗)

Démonstration Notons \Omega =\
\mathrmMat (\phi,\mathcal{E}), A =\
\mathrmMat (d\_\phi,\mathcal{E},\mathcal{E}^∗) et B
= \mathrmMat~
(g\_\phi,\mathcal{E},\mathcal{E}^∗). On a

\begin{align*}
\overline\omega\_i,\jmath& =&
\overline\phi(e\_i,e\_\jmath) =
\left (d\_\phi(e\_\jmath)\right
)(e\_i) \%& \\ & =&
\left (\\sum
\_k=1^na\_
k,\jmathe\_k^∗\right )(e\_ i) =
a\_i,\jmath\%& \\
\end{align*}

compte tenu de e\_k^∗(e\_i) =
\delta\_k^i~; de même

\begin{align*} \omega\_i,\jmath& =&
\phi(e\_i,e\_\jmath) = \left
(g\_\phi(e\_i)\right )(e\_\jmath) \%&
\\ & =& \left
(\sum \_k=1^nb~\_
k,ie\_k^∗\right )(e\_ \jmath) =
b\_\jmath,i\%& \\
\end{align*}

ce qui démontre le résultat.

Corollaire~13.2.8 La forme sesquilinéaire \phi est hermitienne si et
seulement si~sa matrice dans la base \mathcal{E} est hermitienne.

Le rang de \mathrmMat~
(d\_\phi,\mathcal{E},\mathcal{E}^∗) est indépendant du choix de la base \mathcal{E}~;
il en est donc de même du rang de
\mathrmMat~ (\phi,\mathcal{E}). Ceci
conduit à la définition suivante

Définition~13.2.5 Soit E de dimension finie et \phi \in L\_3\diagup2(E). On
appelle rang de E le rang de sa matrice dans n'importe quelle base de E.
On a

\mathrmrg~\phi
= \mathrmrgd\_\phi~
= \mathrmrgg\_\phi~
=\
\mathrmrg\mathrmMat~
(\phi,\mathcal{E})

\paragraph{13.2.4 Changements de bases}

Théorème~13.2.9 Soit E un espace vectoriel de dimension finie, \mathcal{E} et \mathcal{E}'
deux bases de E, P = P\_\mathcal{E}^\mathcal{E}' la matrice de passage de \mathcal{E} à
\mathcal{E}'. Soit \phi \in L\_3\diagup2(E), \Omega =\
\mathrmMat (\phi,\mathcal{E}) et \Omega' =\
\mathrmMat (\phi,\mathcal{E}'). Alors

\Omega' = P^∗\OmegaP

Démonstration Si X (resp. Y ) désigne le vecteur colonne des coordonnées
de x (resp. y) dans la base \mathcal{E} et X' (resp. Y ') désigne le vecteur
colonne des coordonnées de x (resp. y) dans la base \mathcal{E}', on a X = PX', Y
= PY ', d'où

\phi(x,y) = (PX')^∗\Omega(PY `) = X'^∗(P^∗\OmegaP)Y
'

Comme \Omega' est l'unique matrice vérifiant \forall~~(x,y)
\in E \times E, \phi(x,y) = X'^∗\Omega'Y ', on a \Omega' = P^∗\OmegaP.

\paragraph{13.2.5 Orthogonalité}

Soit E un \mathbb{C}-espace vectoriel ~et \phi une forme sesquilinéaire hermitienne
sur E.

Définition~13.2.6 On dit que x est orthogonal à y (relativement à \phi), et
on pose x \bot y, si \phi(x,y) = 0.

Remarque~13.2.4 \phi étant supposée hermitienne, il s'agit visiblement
d'une relation symétrique

Définition~13.2.7 Soit A une partie de E. On pose A^\bot =
\x \in
E∣\forall~~a \in A, \phi(a,x) =
0\

Remarque~13.2.5 Notons A^\bot^∗  l'orthogonal de A
dans le dual E^∗ de E, c'est-à-dire l'espace vectoriel des
formes linéaires sur E qui sont nulles sur A. On a

\begin{align*} x \in A^\bot&
\Leftrightarrow & \forall~~a \in A, \phi(a,x)
= 0 \%& \\ &
\Leftrightarrow & \forall~~a \in A,
\big {[}d\_\phi(x)\big {]}(a) = 0
\%& \\ & \Leftrightarrow &
d\_\phi(x) \in A^\bot^∗ 
\Leftrightarrow x \in
d\_\phi^-1(A^\bot^∗ )\%&
\\ \end{align*}

On en déduit que A^\bot =
d\_\phi^-1(A^\bot^∗ ) =
g\_\phi^-1(A^\bot^∗ ).

Proposition~13.2.10 Soit A une partie de E~; alors

\begin{itemize}
\itemsep1pt\parskip0pt\parsep0pt
\item
  (i)A^\bot est un sous-espace vectoriel de E
\item
  (ii)A^\bot =\
  \mathrmVect(A)^\bot
\item
  (iii) A \subset~ (A^\bot)^\bot
\item
  (iv) A \subset~ B \rigtharrow~ B^\bot\subset~ A^\bot.
\end{itemize}

Démonstration (i) découle immédiatement de la sesquilinéarité de \phi ou de
la remarque précédente. Il en est de même pour (ii) puisqu'un vecteur x
est orthogonal à tout vecteur de A si et seulement si il est orthogonal
à toute combinaison linéaire de vecteurs de A, c'est à dire à
\mathrmVect~(A). En ce qui
concerne (iii), il suffit de remarquer que tout vecteur a de A est
orthogonal à tout vecteur qui est orthogonal à tout vecteur de A. Pour
(iv), un vecteur x qui est orthogonal à tout vecteur de B est évidemment
orthogonal à tout vecteur de A.

\paragraph{13.2.6 Formes non dégénérées}

En règle générale on posera

Définition~13.2.8 Soit E un \mathbb{C}-espace vectoriel , \phi une forme
sesquilinéaire hermitienne sur E. On appelle noyau de \phi le sous-espace

\mathrmKer~\phi =
\x \in
E∣\forall~~y \in E, \phi(x,y) =
0\ = E^\bot =\
\mathrmKerd\_ \phi

Définition~13.2.9 Soit E un \mathbb{C}-espace vectoriel , \phi une forme
sesquilinéaire hermitienne sur E. On dit que \phi est non dégénérée si elle
vérifie les conditions équivalentes

\begin{itemize}
\itemsep1pt\parskip0pt\parsep0pt
\item
  (i) \mathrmKer~\phi =
  E^\bot = \0\
\item
  (ii) pour x \in E on a \left
  (\forall~~y \in E, \phi(x,y) = 0\right ) \rigtharrow~
  x = 0
\item
  (iii) d\_\phi (resp. g\_\phi) est une application
  semilinéaire in\jmathective de E dans E^∗.
\end{itemize}

L'équivalence entre ces trois propriétés est évidente.

Si E est un espace vectoriel de dimension finie, on sait que
dim E^∗~ =\
dim E. Si g\_\phi est in\jmathective, elle est nécessairement
bi\jmathective et on obtient

Théorème~13.2.11 Soit E un \mathbb{C}-espace vectoriel ~de dimension finie, \phi une
forme sesquilinéaire hermitienne non dégénérée sur E. Alors
l'application semilinéaire gauche g\_\phi est un isomorphisme
d'espace vectoriel de E sur E^∗~; autrement dit, pour toute
forme linéaire f sur E, il existe un unique vecteur v\_f \in E tel
que \forall~x \in E, f(x) = \phi(v\_f~,x).

Corollaire~13.2.12 Soit E un \mathbb{C}-espace vectoriel ~de dimension finie, \phi
une forme sesquilinéaire hermitienne non dégénérée sur E. Soit A un
sous-espace vectoriel de E. Alors dim~ A
+ dim A^\bot~ =\
dim E et A = A^\bot\bot.

Démonstration On a en effet

dim A^\bot~ =\
dim g\_ \phi^-1(A^\bot^∗ )
= dim A^\bot^∗ ~
= dim E -\ dim~ A

puisque g\_\phi est un isomorphisme d'espaces vectoriels. On sait
d'autre part que A \subset~ A^\bot\bot et que
dim A^\bot\bot~ =\
dim E - dim A^\bot~
= dim~ A, d'où l'égalité.

Remarque~13.2.6 Il ne faudrait pas en déduire abusivement que A et
A^\bot sont supplémentaires~; en effet, en général A \bigcap
A^\bot\neq~\0\.
Nous nous intéresserons plus particulièrement à ce point dans le
paragraphe suivant.

Si \mathcal{E} est une base de E, alors \Omega =\
\mathrmMat (\phi,\mathcal{E}) =
^t \mathrmMat~
(g\_\phi,\mathcal{E},\mathcal{E}^∗) et
\mathrmrg~\phi
= \mathrmrg~\Omega. On en déduit

Théorème~13.2.13 Soit E un \mathbb{C}-espace vectoriel ~de dimension finie n, \phi
une forme sesquilinéaire hermitienne sur E, \mathcal{E} une base de E et \Omega
= \mathrmMat~ (\phi,\mathcal{E}). Alors
les propositions suivantes sont équivalentes

\begin{itemize}
\itemsep1pt\parskip0pt\parsep0pt
\item
  (i) \phi est non dégénérée
\item
  (ii) \Omega est une matrice inversible
\item
  (iii) \mathrmrg~\phi = n.
\end{itemize}

Remarque~13.2.7 En général,
\mathrmKer~\phi
= \mathrmKerg\_\phi~,
\mathrmrg~\phi
= \mathrmrgg\_\phi~, si
bien que le théorème du rang devient

Proposition~13.2.14 Soit E un \mathbb{C}-espace vectoriel ~de dimension finie n,
\phi une forme sesquilinéaire hermitienne sur E, \mathcal{E} une base de E. Alors
dim~ E =\
\mathrmrg\phi + dim~
\mathrmKer~\phi.

{[}
{[}
{[}
{[}

\end{document}

% \documentclass[]{article}
\usepackage[T1]{fontenc}
\usepackage{lmodern}
\usepackage{amssymb,amsmath}
\usepackage{ifxetex,ifluatex}
\usepackage{fixltx2e} % provides \textsubscript
% use upquote if available, for straight quotes in verbatim environments
\IfFileExists{upquote.sty}{\usepackage{upquote}}{}
\ifnum 0\ifxetex 1\fi\ifluatex 1\fi=0 % if pdftex
  \usepackage[utf8]{inputenc}
\else % if luatex or xelatex
  \ifxetex
    \usepackage{mathspec}
    \usepackage{xltxtra,xunicode}
  \else
    \usepackage{fontspec}
  \fi
  \defaultfontfeatures{Mapping=tex-text,Scale=MatchLowercase}
  \newcommand{\euro}{€}
\fi
% use microtype if available
\IfFileExists{microtype.sty}{\usepackage{microtype}}{}
\ifxetex
  \usepackage[setpagesize=false, % page size defined by xetex
              unicode=false, % unicode breaks when used with xetex
              xetex]{hyperref}
\else
  \usepackage[unicode=true]{hyperref}
\fi
\hypersetup{breaklinks=true,
            bookmarks=true,
            pdfauthor={},
            pdftitle={Formes quadratiques hermitiennes},
            colorlinks=true,
            citecolor=blue,
            urlcolor=blue,
            linkcolor=magenta,
            pdfborder={0 0 0}}
\urlstyle{same}  % don't use monospace font for urls
\setlength{\parindent}{0pt}
\setlength{\parskip}{6pt plus 2pt minus 1pt}
\setlength{\emergencystretch}{3em}  % prevent overfull lines
\setcounter{secnumdepth}{0}
 
/* start css.sty */
.cmr-5{font-size:50%;}
.cmr-7{font-size:70%;}
.cmmi-5{font-size:50%;font-style: italic;}
.cmmi-7{font-size:70%;font-style: italic;}
.cmmi-10{font-style: italic;}
.cmsy-5{font-size:50%;}
.cmsy-7{font-size:70%;}
.cmex-7{font-size:70%;}
.cmex-7x-x-71{font-size:49%;}
.msbm-7{font-size:70%;}
.cmtt-10{font-family: monospace;}
.cmti-10{ font-style: italic;}
.cmbx-10{ font-weight: bold;}
.cmr-17x-x-120{font-size:204%;}
.cmsl-10{font-style: oblique;}
.cmti-7x-x-71{font-size:49%; font-style: italic;}
.cmbxti-10{ font-weight: bold; font-style: italic;}
p.noindent { text-indent: 0em }
td p.noindent { text-indent: 0em; margin-top:0em; }
p.nopar { text-indent: 0em; }
p.indent{ text-indent: 1.5em }
@media print {div.crosslinks {visibility:hidden;}}
a img { border-top: 0; border-left: 0; border-right: 0; }
center { margin-top:1em; margin-bottom:1em; }
td center { margin-top:0em; margin-bottom:0em; }
.Canvas { position:relative; }
li p.indent { text-indent: 0em }
.enumerate1 {list-style-type:decimal;}
.enumerate2 {list-style-type:lower-alpha;}
.enumerate3 {list-style-type:lower-roman;}
.enumerate4 {list-style-type:upper-alpha;}
div.newtheorem { margin-bottom: 2em; margin-top: 2em;}
.obeylines-h,.obeylines-v {white-space: nowrap; }
div.obeylines-v p { margin-top:0; margin-bottom:0; }
.overline{ text-decoration:overline; }
.overline img{ border-top: 1px solid black; }
td.displaylines {text-align:center; white-space:nowrap;}
.centerline {text-align:center;}
.rightline {text-align:right;}
div.verbatim {font-family: monospace; white-space: nowrap; text-align:left; clear:both; }
.fbox {padding-left:3.0pt; padding-right:3.0pt; text-indent:0pt; border:solid black 0.4pt; }
div.fbox {display:table}
div.center div.fbox {text-align:center; clear:both; padding-left:3.0pt; padding-right:3.0pt; text-indent:0pt; border:solid black 0.4pt; }
div.minipage{width:100%;}
div.center, div.center div.center {text-align: center; margin-left:1em; margin-right:1em;}
div.center div {text-align: left;}
div.flushright, div.flushright div.flushright {text-align: right;}
div.flushright div {text-align: left;}
div.flushleft {text-align: left;}
.underline{ text-decoration:underline; }
.underline img{ border-bottom: 1px solid black; margin-bottom:1pt; }
.framebox-c, .framebox-l, .framebox-r { padding-left:3.0pt; padding-right:3.0pt; text-indent:0pt; border:solid black 0.4pt; }
.framebox-c {text-align:center;}
.framebox-l {text-align:left;}
.framebox-r {text-align:right;}
span.thank-mark{ vertical-align: super }
span.footnote-mark sup.textsuperscript, span.footnote-mark a sup.textsuperscript{ font-size:80%; }
div.tabular, div.center div.tabular {text-align: center; margin-top:0.5em; margin-bottom:0.5em; }
table.tabular td p{margin-top:0em;}
table.tabular {margin-left: auto; margin-right: auto;}
div.td00{ margin-left:0pt; margin-right:0pt; }
div.td01{ margin-left:0pt; margin-right:5pt; }
div.td10{ margin-left:5pt; margin-right:0pt; }
div.td11{ margin-left:5pt; margin-right:5pt; }
table[rules] {border-left:solid black 0.4pt; border-right:solid black 0.4pt; }
td.td00{ padding-left:0pt; padding-right:0pt; }
td.td01{ padding-left:0pt; padding-right:5pt; }
td.td10{ padding-left:5pt; padding-right:0pt; }
td.td11{ padding-left:5pt; padding-right:5pt; }
table[rules] {border-left:solid black 0.4pt; border-right:solid black 0.4pt; }
.hline hr, .cline hr{ height : 1px; margin:0px; }
.tabbing-right {text-align:right;}
span.TEX {letter-spacing: -0.125em; }
span.TEX span.E{ position:relative;top:0.5ex;left:-0.0417em;}
a span.TEX span.E {text-decoration: none; }
span.LATEX span.A{ position:relative; top:-0.5ex; left:-0.4em; font-size:85%;}
span.LATEX span.TEX{ position:relative; left: -0.4em; }
div.float img, div.float .caption {text-align:center;}
div.figure img, div.figure .caption {text-align:center;}
.marginpar {width:20%; float:right; text-align:left; margin-left:auto; margin-top:0.5em; font-size:85%; text-decoration:underline;}
.marginpar p{margin-top:0.4em; margin-bottom:0.4em;}
.equation td{text-align:center; vertical-align:middle; }
td.eq-no{ width:5%; }
table.equation { width:100%; } 
div.math-display, div.par-math-display{text-align:center;}
math .texttt { font-family: monospace; }
math .textit { font-style: italic; }
math .textsl { font-style: oblique; }
math .textsf { font-family: sans-serif; }
math .textbf { font-weight: bold; }
.partToc a, .partToc, .likepartToc a, .likepartToc {line-height: 200%; font-weight:bold; font-size:110%;}
.chapterToc a, .chapterToc, .likechapterToc a, .likechapterToc, .appendixToc a, .appendixToc {line-height: 200%; font-weight:bold;}
.index-item, .index-subitem, .index-subsubitem {display:block}
.caption td.id{font-weight: bold; white-space: nowrap; }
table.caption {text-align:center;}
h1.partHead{text-align: center}
p.bibitem { text-indent: -2em; margin-left: 2em; margin-top:0.6em; margin-bottom:0.6em; }
p.bibitem-p { text-indent: 0em; margin-left: 2em; margin-top:0.6em; margin-bottom:0.6em; }
.paragraphHead, .likeparagraphHead { margin-top:2em; font-weight: bold;}
.subparagraphHead, .likesubparagraphHead { font-weight: bold;}
.quote {margin-bottom:0.25em; margin-top:0.25em; margin-left:1em; margin-right:1em; text-align:justify;}
.verse{white-space:nowrap; margin-left:2em}
div.maketitle {text-align:center;}
h2.titleHead{text-align:center;}
div.maketitle{ margin-bottom: 2em; }
div.author, div.date {text-align:center;}
div.thanks{text-align:left; margin-left:10%; font-size:85%; font-style:italic; }
div.author{white-space: nowrap;}
.quotation {margin-bottom:0.25em; margin-top:0.25em; margin-left:1em; }
h1.partHead{text-align: center}
.sectionToc, .likesectionToc {margin-left:2em;}
.subsectionToc, .likesubsectionToc {margin-left:4em;}
.subsubsectionToc, .likesubsubsectionToc {margin-left:6em;}
.frenchb-nbsp{font-size:75%;}
.frenchb-thinspace{font-size:75%;}
.figure img.graphics {margin-left:10%;}
/* end css.sty */

\title{Formes quadratiques hermitiennes}
\author{}
\date{}

\begin{document}
\maketitle

\textbf{Warning: 
requires JavaScript to process the mathematics on this page.\\ If your
browser supports JavaScript, be sure it is enabled.}

\begin{center}\rule{3in}{0.4pt}\end{center}

[
[
[]
[

\subsubsection{13.3 Formes quadratiques hermitiennes}

\paragraph{13.3.1 Notion de forme quadratique hermitienne}

Soit E un \mathbb{C}-espace vectoriel et \phi une forme sesquilinéaire hermitienne
sur E. Soit \Phi l'application de E dans \mathbb{R}~ qui à x associe \Phi(x) = \phi(x,x)
(on a en effet \phi(x,x) = \overline\phi(x,x) donc \Phi(x) \in
\mathbb{R}~).

Proposition~13.3.1 On a les identités suivantes

\begin{itemize}
\itemsep1pt\parskip0pt\parsep0pt
\item
  (i) \Phi(\lambda~x) = \lambda~^2\Phi(x)
\item
  (ii) \Phi(x + y) = \Phi(x) +
  2\mathrmRe~(\phi(x,y)) + \Phi(y)
\item
  (ii)' \Phi(x + y) - \Phi(x - y) + i\Phi(x + iy) - i\Phi(x - iy) = 4\phi(y,x)
  (identité de polarisation)
\item
  (iii) \Phi(x + y) + \Phi(x - y) = 2(\Phi(x) + \Phi(y)) (identité de la médiane)
\end{itemize}

Démonstration (i) \Phi(\lambda~x) = \phi(\lambda~x,\lambda~x) =
\lambda~\overline\lambda~\phi(x,x) =
\lambda~^2\Phi(x)

(ii) \Phi(x + y) = \phi(x + y,x + y) = \Phi(x) + \phi(x,y) + \phi(y,x) + \Phi(y) = \Phi(x) +
2\mathrmRe~(\phi(x,y)) + \Phi(y)~;
(ii)' s'en déduit immédiatement par un petit calcul

(iii) changeant y en - y dans l'identité précédente, on a aussi \Phi(x - y)
= \Phi(x) - 2\phi(x,y) + \Phi(y), et en additionnant les deux on trouve \Phi(x + y)
+ \Phi(x - y) = 2(\Phi(x) + \Phi(y)).

Remarque~13.3.1 L'identité (ii)' montre que l'application
\phi\mapsto~\Phi est injective de H(E) dans \mathbb{R}~^E
(espace vectoriel des applications de E dans \mathbb{R}~) puisque la connaissance
de \Phi permet de retrouver \phi. Ceci nous amène à poser

Définition~13.3.1 Soit E un \mathbb{C}-espace vectoriel . On appelle forme
quadratique hermitienne sur E toute application \Phi : E \rightarrow~ \mathbb{R}~ telle qu'il
existe une forme sesquilinéaire hermitienne \phi : E \times E \rightarrow~ \mathbb{C} vérifiant
\forall~~x \in E, \Phi(x) = \phi(x,x). Dans ce cas, \phi est
unique et est appelée la forme polaire de \Phi.

Exemple~13.3.1 Sur \mathbb{C}^n, \Phi(x) =\
\sum ~
_i=1^nx_i^2 est
une forme quadratique hermitienne dont la forme polaire associée est
\phi(x,y) = \\sum ~
_i=1^n\overlinex_iy_i.
Si E désigne l'espace vectoriel des fonctions continues de [a,b]
dans \mathbb{C}, \Phi(f) =\int ~
_a^bf(t)^2 dt est une forme
quadratique hermitienne dont la forme polaire est \phi(f,g)
=\int ~
_a^b\overlinef(t)g(t) dt.

Proposition~13.3.2 L'ensemble Q(E) des formes quadratiques sur E est un
\mathbb{R}~-sous-espace vectoriel de \mathbb{R}~^E~; l'application
\phi\mapsto~\Phi est un isomorphisme de \mathbb{R}~-espaces
vectoriels de H(E) sur Q(E).

Remarque~13.3.2 Par la suite on confondra toutes les notions relatives à
\phi et à \Phi~: orthogonalité, matrice, non dégénérescence, isotropie~; en
particulier on posera
\mathrmKer~\Phi
= \mathrmKer~\phi =
\x \in
E∣\forall~~y \in E, \phi(x,y) =
0\. On remarquera qu'en général,
\mathrmKer\Phi\mathrel\neq~~\x
\in E∣\Phi(x) = 0\.

Théorème~13.3.3 (Pythagore). Soit E un \mathbb{C}-espace vectoriel ~et \Phi \inQ(E), \phi
la forme polaire de \Phi. Alors

x \bot_\phiy \rigtharrow~ \Phi(x + y) = \Phi(x) + \Phi(y)

Démonstration C'est une conséquence évidente de l'identité \Phi(x + y) =
\Phi(x) + 2\mathrmRe~(\phi(x,y)) +
\Phi(y). Remarquons l'absence de réciproque, contrairement au cas des
formes quadratiques.

\paragraph{13.3.2 Formes quadratiques hermitiennes en dimension finie}

Soit E un \mathbb{C}-espace vectoriel ~de dimension finie, \Phi \inQ(E) de forme
polaire \phi.

Théorème~13.3.4 Soit \mathcal{E} une base de E. Alors
\mathrmMat~ (\phi,\mathcal{E}) est
l'unique matrice \Omega \in M_\mathbb{C}(n) qui est hermitienne et qui vérifie

\forall~x \in E, \Phi(x) = X^∗~\OmegaX

Démonstration Il est clair que \Omega =\
\mathrmMat (\Phi,\mathcal{E}) est hermitienne et vérifie \Phi(x) =
\phi(x,x) = X^∗\OmegaX. Inversement, soit \Omega une matrice hermitienne
vérifiant cette propriété. On a alors

\phi(y,x) = 1 \over 4 (\Phi(x + y) - \Phi(x - y) + i\Phi(x + iy)
- i\Phi(x - iy)) = Y ^∗\OmegaX

(après un calcul un peu pénible) ce qui montre que \Omega
= \mathrmMat~ (\phi,\mathcal{E}).

Posons \Omega = \mathrmMat~ (\phi,\mathcal{E})
= (\omega_i,j)_1\leqi,j\leqn. On a alors

\phi(x,y) = \\sum
_i,j\omega_i,j\overlinex_iy_j
= \\sum
_i\omega_i,i\overlinex_iy_i
+ \\sum
_i<j(\omega_i,j\overlinex_iy_j
+ \omega_j,i\overlinex_jy_i)

En tenant compte de \omega_i,j =
\overline\omega_j,i, on a donc

\Phi(x) = \phi(x,x) = \\sum
_i\omega_i,ix_i^2 +
2\mathrmRe(\\sum
_i<j\omega_i,j\overlinex_ix_j)
=
P_\Phi(x_1,\ldots,x_n~)

Inversement, soit P de la forme
P(x_1,\\ldots,x_n~)
= \\sum ~
_i=1^na_i,ix_i^2
+
2\mathrmRe~(\\\sum

_i<ja_i,j\overlinex_ix_j).
Définissons \phi sur E par

\phi(x,y) = \\sum
_ia_i,i\overlinex_iy_i
+ \\sum
_i<j(a_i,j\overlinex_iy_j
+
\overlinea_i,j\overlinex_jy_i)

si x = \\sum ~
x_ie_i et y =\
\sum  y_ie_i~. Alors \phi est
clairement une forme sesquilinéaire hermitienne sur E et la forme
quadratique associée vérifie \Phi(x) =
P(x_1,\\ldots,x_n~).
On obtient l'expression de \phi(x,y) à partir de l'expression polynomiale
de \Phi(x) en rempla\ccant partout les termes carrés
x_i^2 par
\overlinex_iy_i et les termes
rectangles
\mathrmRe(a_i,j\overlinex_ix_j~)
par  1 \over 2
(a_i,j\overlinex_ix_j +
\overlinea_i,j\overlinex_jy_i).

Théorème~13.3.5 Si \mathcal{E} est une base orthonormée de E (c'est à dire
\phi(e_i,e_j) = \delta_i^j), alors
\mathrmMat~ (\phi,\mathcal{E}) =
I_n, \phi(x,y) = X^∗Y =\
\sum ~
_i=1^n\overlinex_iy_i
et \Phi(x) = X^∗X =\
\sum ~
_i=1^nx_i^2.

Démonstration Evident.

\paragraph{13.3.3 Formes quadratiques hermitiennes définies positives}

Définition~13.3.2 Soit E un \mathbb{C} espace vectoriel et \Phi une forme
quadratique hermitienne sur E. On dit que \Phi est définie positive si
\forall~x \in E \diagdown\0\~,
\Phi(x) > 0.

Théorème~13.3.6 (inégalité de Schwarz). Soit E un \mathbb{C} espace vectoriel et
\Phi une forme quadratique hermitienne définie positive sur E de forme
polaire \phi. Alors

\forall~~x,y \in E,
\phi(x,y)^2 \leq \Phi(x)\Phi(y)

avec égalité si et seulement si~la famille (x,y) est liée.

Démonstration L'inégalité est évidente si y = 0~; supposons donc
y\neq~0. Soit \theta \in \mathbb{R}~. On écrit
\forall~t \in \mathbb{R}~, \Phi(x + te^i\theta~y) ≥ 0, soit
encore t^2\Phi(y) +
2t\mathrmRe(e^i\theta~\phi(x,y))
+ \Phi(x) ≥ 0. Choisissons \theta tel que \phi(x,y) =
e^-i\theta\phi(x,y) (autrement dit l'opposé d'un
argument de \phi(x,y)). On a donc t^2\Phi(y) +
2t\phi(x,y) + \Phi(x) ≥ 0. Ce trinome du second degré doit
donc avoir un discriminant réduit négatif, soit
\phi(x,y)^2 - \Phi(x)\Phi(y) \leq 0. Si on a
l'égalité, deux cas sont possibles. Soit y = 0 auquel cas la famille
(x,y) est liée, soit \Phi(y)\neq~0~; mais dans ce
cas ce trinome en t a une racine double t_0, et donc \Phi(x +
t_0e^i\thetay) = 0 d'où x + t_0e^i\thetay
= 0 et donc la famille est liée. Inversement, si la famille (x,y) est
liée, on a par exemple x = \lambda~y et dans ce cas
\phi(x,y)^2 =
\lambda~^2\Phi(y)^2 = \Phi(x)\Phi(y).

Théorème~13.3.7 (inégalité de Minkowski). Soit E un \mathbb{C} espace vectoriel
et \Phi une forme quadratique hermitienne définie positive sur E. Alors

\forall~x,y \in E, \sqrt\Phi(x + y)~
\leq\sqrt\Phi(x) + \sqrt\Phi(y)

avec égalité si et seulement si~la famille (x,y) est positivement liée.

Démonstration On a

\begin{align*} \Phi(x + y)& =& \Phi(x) +
2\mathrmRe~(\phi(x,y)) +
\Phi(y)\%& \\ & \leq& \Phi(x) +
2\phi(x,y) + \Phi(y) \%& \\
& \leq& \Phi(x) + 2\sqrt\Phi(x)\Phi(y) + \Phi(y)\%&
\\ & =& \left
(\sqrt\Phi(x) +
\sqrt\Phi(y)\right )^2 \%&
\\ \end{align*}

d'où \sqrt\Phi(x + y) \leq\sqrt\Phi(x) +
\sqrt\Phi(y). L'égalité nécessite à la fois que
\phi(x,y) = \sqrt\Phi(x)\Phi(y), donc que
(x,y) soit liée, et que
\mathrmRe~(\phi(x,y)) =
\phi(x,y)≥ 0, c'est-à-dire que le coefficient de
proportionnalité soit réel et positif.

Définition~13.3.3 On appelle espace préhilbertien complexe un couple
(E,\Phi) d'un \mathbb{C}-espace vectoriel ~E et d'une forme quadratique hermitienne
définie positive sur E. On appelle espace hermitien un espace
préhilbertien complexe de dimension finie.

Théorème~13.3.8 Soit (E,\Phi) un espace préhilbertien complexe. Alors
l'application x\mapsto~\sqrt\Phi(x)
est une norme sur E appelée norme hermitienne.

Démonstration La propriété de séparation provient du fait que \Phi est
définie. L'homogénéité provient de l'homogénéité de la forme
quadratique. Quant à l'inégalité triangulaire, ce n'est autre que
l'inégalité de Minkowski.

Définition~13.3.4 On notera (x∣y) = \phi(x,y) et
\x\^2 =
(x∣x) = \Phi(x)

\paragraph{13.3.4 Espaces hermitiens}

Une forme définie positive étant clairement non dégénérée, on a bien
évidemment

Théorème~13.3.9 Soit E un espace hermitien. Pour toute forme linéaire f
sur E, il existe un unique vecteur v_f \in E tel que
\forall~~x \in E, f(x) =
(v_f∣x)

D'autre part si \Phi est définie positive, et si A est un sous-espace
vectoriel de E on a

x \in A \bigcap A^\bot\rigtharrow~ x \bot x \rigtharrow~ (x∣x) = 0 \rigtharrow~ x
= 0

Comme de plus dim~ A +\
dim A^\bot = dim~ E, on obtient

Théorème~13.3.10 Soit E un espace hermitien.Pour tout sous-espace
vectoriel A de E, on a E = A \oplus~ A^\bot et
(A^\bot)^\bot = A.

Enfin l'existence de bases orthonormées nous est garanti par
l'algorithme de Gramm-Schmidt, dont la démonstration est strictement la
même que pour les formes quadratiques~:

Théorème~13.3.11 Soit E un espace hermitien. Soit \mathcal{E} =
(e_1,\\ldots,e_n~)
une base de E. Alors il existe une base orthonormée \mathcal{E}' =
(\epsilon_1,\\ldots,\epsilon_n~)
de E vérifiant les conditions équivalentes suivantes

\begin{itemize}
\itemsep1pt\parskip0pt\parsep0pt
\item
  (i) \forall~k \in [1,n], \epsilon_k~
  \in\mathrmVect(e_1,\\\ldots,e_k~)
\item
  (ii) \forall~~k \in [1,n],
  \mathrmVect(\epsilon_1,\\\ldots,\epsilon_k~)
  =\
  \mathrmVect(e_1,\\ldots,e_k~)
\item
  (iii) la matrice de passage de \mathcal{E} à \mathcal{E}' est triangulaire supérieure
\end{itemize}

Si \mathcal{E}' =
(\epsilon_1,\\ldots,\epsilon_n~)
et \mathcal{E}'' =
(\eta_1,\\ldots,\eta_n~)
sont deux telles bases orthonormées, il existe des scalaires
\lambda_1,\\ldots,\lambda_n~
de module 1 tels que \forall~~i \in [1,n],
\eta_i = \lambda_i\epsilon_i.

[
[
[
[

\end{document}

% \documentclass[]{article}
\usepackage[T1]{fontenc}
\usepackage{lmodern}
\usepackage{amssymb,amsmath}
\usepackage{ifxetex,ifluatex}
\usepackage{fixltx2e} % provides \textsubscript
% use upquote if available, for straight quotes in verbatim environments
\IfFileExists{upquote.sty}{\usepackage{upquote}}{}
\ifnum 0\ifxetex 1\fi\ifluatex 1\fi=0 % if pdftex
  \usepackage[utf8]{inputenc}
\else % if luatex or xelatex
  \ifxetex
    \usepackage{mathspec}
    \usepackage{xltxtra,xunicode}
  \else
    \usepackage{fontspec}
  \fi
  \defaultfontfeatures{Mapping=tex-text,Scale=MatchLowercase}
  \newcommand{\euro}{€}
\fi
% use microtype if available
\IfFileExists{microtype.sty}{\usepackage{microtype}}{}
\ifxetex
  \usepackage[setpagesize=false, % page size defined by xetex
              unicode=false, % unicode breaks when used with xetex
              xetex]{hyperref}
\else
  \usepackage[unicode=true]{hyperref}
\fi
\hypersetup{breaklinks=true,
            bookmarks=true,
            pdfauthor={},
            pdftitle={Endomorphismes d'un espace hermitien},
            colorlinks=true,
            citecolor=blue,
            urlcolor=blue,
            linkcolor=magenta,
            pdfborder={0 0 0}}
\urlstyle{same}  % don't use monospace font for urls
\setlength{\parindent}{0pt}
\setlength{\parskip}{6pt plus 2pt minus 1pt}
\setlength{\emergencystretch}{3em}  % prevent overfull lines
\setcounter{secnumdepth}{0}
 
/* start css.sty */
.cmr-5{font-size:50%;}
.cmr-7{font-size:70%;}
.cmmi-5{font-size:50%;font-style: italic;}
.cmmi-7{font-size:70%;font-style: italic;}
.cmmi-10{font-style: italic;}
.cmsy-5{font-size:50%;}
.cmsy-7{font-size:70%;}
.cmex-7{font-size:70%;}
.cmex-7x-x-71{font-size:49%;}
.msbm-7{font-size:70%;}
.cmtt-10{font-family: monospace;}
.cmti-10{ font-style: italic;}
.cmbx-10{ font-weight: bold;}
.cmr-17x-x-120{font-size:204%;}
.cmsl-10{font-style: oblique;}
.cmti-7x-x-71{font-size:49%; font-style: italic;}
.cmbxti-10{ font-weight: bold; font-style: italic;}
p.noindent { text-indent: 0em }
td p.noindent { text-indent: 0em; margin-top:0em; }
p.nopar { text-indent: 0em; }
p.indent{ text-indent: 1.5em }
@media print {div.crosslinks {visibility:hidden;}}
a img { border-top: 0; border-left: 0; border-right: 0; }
center { margin-top:1em; margin-bottom:1em; }
td center { margin-top:0em; margin-bottom:0em; }
.Canvas { position:relative; }
li p.indent { text-indent: 0em }
.enumerate1 {list-style-type:decimal;}
.enumerate2 {list-style-type:lower-alpha;}
.enumerate3 {list-style-type:lower-roman;}
.enumerate4 {list-style-type:upper-alpha;}
div.newtheorem { margin-bottom: 2em; margin-top: 2em;}
.obeylines-h,.obeylines-v {white-space: nowrap; }
div.obeylines-v p { margin-top:0; margin-bottom:0; }
.overline{ text-decoration:overline; }
.overline img{ border-top: 1px solid black; }
td.displaylines {text-align:center; white-space:nowrap;}
.centerline {text-align:center;}
.rightline {text-align:right;}
div.verbatim {font-family: monospace; white-space: nowrap; text-align:left; clear:both; }
.fbox {padding-left:3.0pt; padding-right:3.0pt; text-indent:0pt; border:solid black 0.4pt; }
div.fbox {display:table}
div.center div.fbox {text-align:center; clear:both; padding-left:3.0pt; padding-right:3.0pt; text-indent:0pt; border:solid black 0.4pt; }
div.minipage{width:100%;}
div.center, div.center div.center {text-align: center; margin-left:1em; margin-right:1em;}
div.center div {text-align: left;}
div.flushright, div.flushright div.flushright {text-align: right;}
div.flushright div {text-align: left;}
div.flushleft {text-align: left;}
.underline{ text-decoration:underline; }
.underline img{ border-bottom: 1px solid black; margin-bottom:1pt; }
.framebox-c, .framebox-l, .framebox-r { padding-left:3.0pt; padding-right:3.0pt; text-indent:0pt; border:solid black 0.4pt; }
.framebox-c {text-align:center;}
.framebox-l {text-align:left;}
.framebox-r {text-align:right;}
span.thank-mark{ vertical-align: super }
span.footnote-mark sup.textsuperscript, span.footnote-mark a sup.textsuperscript{ font-size:80%; }
div.tabular, div.center div.tabular {text-align: center; margin-top:0.5em; margin-bottom:0.5em; }
table.tabular td p{margin-top:0em;}
table.tabular {margin-left: auto; margin-right: auto;}
div.td00{ margin-left:0pt; margin-right:0pt; }
div.td01{ margin-left:0pt; margin-right:5pt; }
div.td10{ margin-left:5pt; margin-right:0pt; }
div.td11{ margin-left:5pt; margin-right:5pt; }
table[rules] {border-left:solid black 0.4pt; border-right:solid black 0.4pt; }
td.td00{ padding-left:0pt; padding-right:0pt; }
td.td01{ padding-left:0pt; padding-right:5pt; }
td.td10{ padding-left:5pt; padding-right:0pt; }
td.td11{ padding-left:5pt; padding-right:5pt; }
table[rules] {border-left:solid black 0.4pt; border-right:solid black 0.4pt; }
.hline hr, .cline hr{ height : 1px; margin:0px; }
.tabbing-right {text-align:right;}
span.TEX {letter-spacing: -0.125em; }
span.TEX span.E{ position:relative;top:0.5ex;left:-0.0417em;}
a span.TEX span.E {text-decoration: none; }
span.LATEX span.A{ position:relative; top:-0.5ex; left:-0.4em; font-size:85%;}
span.LATEX span.TEX{ position:relative; left: -0.4em; }
div.float img, div.float .caption {text-align:center;}
div.figure img, div.figure .caption {text-align:center;}
.marginpar {width:20%; float:right; text-align:left; margin-left:auto; margin-top:0.5em; font-size:85%; text-decoration:underline;}
.marginpar p{margin-top:0.4em; margin-bottom:0.4em;}
.equation td{text-align:center; vertical-align:middle; }
td.eq-no{ width:5%; }
table.equation { width:100%; } 
div.math-display, div.par-math-display{text-align:center;}
math .texttt { font-family: monospace; }
math .textit { font-style: italic; }
math .textsl { font-style: oblique; }
math .textsf { font-family: sans-serif; }
math .textbf { font-weight: bold; }
.partToc a, .partToc, .likepartToc a, .likepartToc {line-height: 200%; font-weight:bold; font-size:110%;}
.chapterToc a, .chapterToc, .likechapterToc a, .likechapterToc, .appendixToc a, .appendixToc {line-height: 200%; font-weight:bold;}
.index-item, .index-subitem, .index-subsubitem {display:block}
.caption td.id{font-weight: bold; white-space: nowrap; }
table.caption {text-align:center;}
h1.partHead{text-align: center}
p.bibitem { text-indent: -2em; margin-left: 2em; margin-top:0.6em; margin-bottom:0.6em; }
p.bibitem-p { text-indent: 0em; margin-left: 2em; margin-top:0.6em; margin-bottom:0.6em; }
.paragraphHead, .likeparagraphHead { margin-top:2em; font-weight: bold;}
.subparagraphHead, .likesubparagraphHead { font-weight: bold;}
.quote {margin-bottom:0.25em; margin-top:0.25em; margin-left:1em; margin-right:1em; text-align:\jmathustify;}
.verse{white-space:nowrap; margin-left:2em}
div.maketitle {text-align:center;}
h2.titleHead{text-align:center;}
div.maketitle{ margin-bottom: 2em; }
div.author, div.date {text-align:center;}
div.thanks{text-align:left; margin-left:10%; font-size:85%; font-style:italic; }
div.author{white-space: nowrap;}
.quotation {margin-bottom:0.25em; margin-top:0.25em; margin-left:1em; }
h1.partHead{text-align: center}
.sectionToc, .likesectionToc {margin-left:2em;}
.subsectionToc, .likesubsectionToc {margin-left:4em;}
.subsubsectionToc, .likesubsubsectionToc {margin-left:6em;}
.frenchb-nbsp{font-size:75%;}
.frenchb-thinspace{font-size:75%;}
.figure img.graphics {margin-left:10%;}
/* end css.sty */

\title{Endomorphismes d'un espace hermitien}
\author{}
\date{}

\begin{document}
\maketitle

\textbf{Warning: 
requires JavaScript to process the mathematics on this page.\\ If your
browser supports JavaScript, be sure it is enabled.}

\begin{center}\rule{3in}{0.4pt}\end{center}

{[}
{[}
{[}{]}
{[}

\subsubsection{13.4 Endomorphismes d'un espace hermitien}

\paragraph{13.4.1 Notion d'ad\jmathoint}

Soit E un espace préhilbertien complexe

Définition~13.4.1 Soit E un espace préhilbertien complexe. Soit u,v \in
L(E). On dit que u et v sont des endomorphismes ad\jmathoints si

\forall~x,y \in E, (u(x)\mathrel∣~y)
= (x∣v(y))

Remarque~13.4.1 La symétrie hermitienne du produit scalaire montre
clairement que u et v \jmathouent des rôles symétriques, donc que u est
ad\jmathoint de v si et seulement si~v est ad\jmathoint de u.

Théorème~13.4.1 Soit E un espace hermitien. Tout endomorphisme de E
admet un unique ad\jmathoint u^∗. Si u \in L(E), \mathcal{E} une base de E, \Omega
= \mathrmMat~ (\phi,\mathcal{E}) et A
= \mathrmMat~ (u,\mathcal{E}), alors

\mathrmMat~
(u^∗,\mathcal{E}) = \Omega^-1A^∗\Omega

Démonstration Soit \mathcal{E} une base de E et \Omega =\
\mathrmMat (\phi,\mathcal{E}). Comme \phi est non dégénérée, la
matrice \Omega est inversible. Soit u,v \in L(E), A =\
\mathrmMat (u,\mathcal{E}) et B =\
\mathrmMat (v,\mathcal{E}). Si x,y \in E, on a
(u(x)∣y) = (AX)^∗\OmegaY =
X^∗A^∗\OmegaY et (x∣v(y)) =
X^∗\OmegaBY . L'unicité de la matrice de la forme sesquilinéaire
(x,y)\mapsto~(u(x)\mathrel∣y)
montre que

\begin{align*} \forall~~x,y \in E,
(u(x)∣y) =
(x∣v(y))&& \%&
\\ & \Leftrightarrow &
A^∗\Omega = \OmegaB \Leftrightarrow B =
\Omega^-1A^∗\Omega\%& \\
\end{align*}

ce qui montre à la fois l'existence et l'unicité de l'ad\jmathoint et la
formule voulue.

Proposition~13.4.2 Soit E un espace hermitien. L'application
u\mapsto~u^∗ est un endomorphisme
semi-linéaire involutif de L(E). Si u,v \in L(E), alors u \cdot v aussi et
(u \cdot v)^∗ = v^∗\cdot u^∗. Si u \in L(E) est
inversible, alors u^∗ est inversible et
(u^-1)^∗ = (u^∗)^-1.

Démonstration On a dé\jmathà vu que la relation u et v sont ad\jmathoints était
symétrique, donc si u \in L(E), u^∗ aussi et u^∗∗ =
u. Si u,v \in L(E), \alpha~,\beta~ \in \mathbb{C}, on a

\begin{align*} ((\alpha~u +
\beta~v)(x)∣y)& =& (\alpha~u(x) +
\beta~v(x)∣y) \%&
\\ & =&
\overline\alpha~(u(x)∣y) +
\overline\beta~(v(x)∣y) \%&
\\ & =&
\overline\alpha~(x∣u^∗(y))
+
\overline\beta~(x∣v^∗(y))\%&
\\ & =&
(x∣(\overline\alpha~u^∗
+ \overline\beta~v^∗)(y)) \%&
\\ \end{align*}

ce qui montre que (\alpha~u + \beta~v)^∗ =
\overline\alpha~u^∗ +
\overline\beta~v^∗ et donc la semilinéarité de
u\mapsto~u^∗. Si u,v \in L(E), on a

(u \cdot v(x)∣y) =
(v(x)∣u^∗(y)) =
(x∣v^∗\cdot u^∗(y))

ce qui montre que u \cdot v admet v^∗\cdot u^∗ comme
ad\jmathoint.

Si u est inversible, on a u^-1 \cdot u =
\mathrmId\_E d'où (u^-1 \cdot
u)^∗ = \mathrmId\_E^∗, soit
u^∗\cdot (u^-1)^∗ =
\mathrmId\_E. De même u \cdot u^-1 =
\mathrmId\_E donne par passage à l'ad\jmathoint
(u^-1)^∗\cdot u^∗ =
\mathrmId\_E. Ceci montre que u^∗
est inversible et que (u^-1)^∗ =
(u^∗)^-1

Proposition~13.4.3 Soit E un espace hermitien, u \in L(E). Alors

\begin{itemize}
\itemsep1pt\parskip0pt\parsep0pt
\item
  (i) \mathrm{det}~
  u^∗ =
  \overline\mathrm{det}~
  u,
  \mathrm{tr}u^∗~ =
  \overline\mathrm{tr}u~,
  \chi\_u^∗ = \overline\chi\_u
\item
  (ii)
  \mathrmKeru^∗~
  =
  (\mathrmImu)^\bot~,
  \mathrmImu^∗~ =
  (\mathrmKeru)^\bot~
\item
  (iii)
  \mathrmKeru^∗~u
  = \mathrmKer~u et
  \mathrmImu^∗~u
  = \mathrmImu^∗~
\end{itemize}

Démonstration (i) Soit \mathcal{E} une base de E, \Omega =\
\mathrmMat (\phi,\mathcal{E}) et A =\
\mathrmMat (u,\mathcal{E}), alors
\mathrmMat~
(u^∗,\mathcal{E}) = \Omega^-1A^∗\Omega. On a donc
\mathrm{det} u^∗~
= \mathrm{det}~
\Omega^-1A^∗\Omega =\
\mathrm{det} A^∗ =
\overline\mathrm{det}~
A =
\overline\mathrm{det}~
u. La démonstration est la même pour la trace et pour le polynôme
caractéristique.

(ii) On a

\begin{align*} x
\in\mathrmKeru^∗~&
\Leftrightarrow & u^∗(x) = 0
\Leftrightarrow \forall~~y \in E,
(u^∗(x)∣y) = 0 \%&
\\ & \Leftrightarrow &
\forall~y \in E, (x\mathrel∣~u(y)) =
0 \Leftrightarrow x \in
(\mathrmImu)^\bot~\%&
\\ \end{align*}

En appliquant ce résultat à u^∗ on obtient,
\mathrmKer~u =
(\mathrmImu^∗)^\bot~
et en prenant l'orthogonal,
\mathrmImu^∗~ =
(\mathrmKeru)^\bot~

(iii) On a visiblement u(x) = 0 \rigtharrow~ u^∗u(x) = 0, donc
\mathrmKer~u
\subset~\mathrmKeru^∗~u~;
mais d'autre part, si x
\in\mathrmKeru^∗~u,
on a

\\textbar{}u(x)\\textbar{}^2 =
(u(x)∣u(x)) =
(u^∗u(x)∣x) =
(0∣x) = 0

et donc u(x) = 0, soit
\mathrmKeru^∗~u
\subset~\mathrmKer~u et l'égalité.
On en déduit alors que

\mathrmImu^∗~u =
(\mathrmKer(u^∗u)^∗)^\bot~
=
(\mathrmKeru^∗u)^\bot~
=
(\mathrmKeru)^\bot~
= \mathrmImu^∗~

Une des propriétés essentielles de l'ad\jmathoint que nous utiliserons de
fa\ccon systématique pour la réduction des
endomorphismes est la suivante

Théorème~13.4.4 Soit u \in L(E). Soit F un sous-espace de E stable par u~;
alors F^\bot est stable par u^∗.

Démonstration Soit x \in F^\bot. Si y \in F, on a
\phi(u^∗(x),y) = \phi(x,u(y)) = 0 puisque u(y) \in F et x \in
F^\bot. Donc u^∗(x) \in F^\bot et
F^\bot est stable par u^∗.

\paragraph{13.4.2 Endomorphismes hermitiens}

Définition~13.4.2 Soit E un espace hermitien, u \in L(E). On dit que u est
hermitien (ou autoad\jmathoint) s'il vérifie les conditions équivalentes

\begin{itemize}
\itemsep1pt\parskip0pt\parsep0pt
\item
  (i) u^∗ = u
\item
  (ii) \forall~~x,y \in E,
  (u(x)∣y) =
  (x∣u(y))
\end{itemize}

Remarque~13.4.2 Si la base \mathcal{E} est orthonormée, alors
\mathrmMat~ ((
∣ ),\mathcal{E}) = I\_n et
\mathrmMat~
(u^∗,\mathcal{E}) =\
\mathrmMat (u,\mathcal{E})^∗~; en particulier

Théorème~13.4.5 Soit \mathcal{E} une base orthonormée de E~; alors u est hermitien
si et seulement
si~\mathrmMat~ (u,\mathcal{E}) est une
matrice hermitienne.

Proposition~13.4.6 L'ensemble H(E) des endomorphismes hermitiens est un
\mathbb{R}~-sous-espace vectoriel de L(E) (mais pas un \mathbb{C} sous-espace vectoriel).
On a L(E) = H(E) \oplus~ iH(E)

Démonstration L'endomorphisme de L^∗(E),
u\mapsto~u^∗ étant \mathbb{R}~ linéaire et
involutif, l'espace L(E) est somme directe du sous-espace propre associé
à la valeur propre 1 (les endomorphismes hermitiens) et du sous-espace
propre associé à la valeur propre -1 (les endomorphismes antihermitiens,
qui ne sont autre que les endomorphismes hermitiens multipliés par i).

\paragraph{13.4.3 Groupe unitaire}

Soit E un espace hermitien

Définition~13.4.3 On dit que u \in L(E) est un endomorphisme unitaire si
on a les propriétés équivalentes

\begin{itemize}
\itemsep1pt\parskip0pt\parsep0pt
\item
  (i) \forall~~x \in E,
  \\textbar{}u(x)\\textbar{}
  =\\textbar{} x\\textbar{}
\item
  (ii) \forall~~x,y \in E,
  (u(x)∣u(y)) =
  (x∣y)
\item
  (iii) u est inversible et u^-1 = u^∗
\item
  (iv) u \cdot u^∗ = \mathrmId\_E
\item
  (v) u^∗\cdot u = \mathrmId\_E
\end{itemize}

Démonstration (ii) \rigtharrow~(i) est évident (faire y = x). (i) \rigtharrow~(ii) provient de
l'identité de polarisation et de la linéarité de u. Pour un
endomorphisme d'un espace vectoriel de dimension finie, on sait que
l'inversibilité est équivalente à l'inversibilité à gauche ou à droite.
On a donc (iii) \Leftrightarrow (iv)
\Leftrightarrow (v). Supposons (ii) vérifié. Alors \phi(x,y) =
\phi(u(x),u(y)) = \phi(x,u^∗\cdot u(y)), ce qui montre (puisque \phi est
non dégénérée) que u^∗\cdot u =
\mathrmId\_E~; donc (ii) \rigtharrow~(v). De même (v)
\rigtharrow~(ii) puisque \phi(u(x),u(y)) = \phi(x,u^∗\cdot u(y)).

Théorème~13.4.7 L'ensemble U(E) des endomorphismes unitaires de E est un
sous-groupe de (GL(E),\cdot). Pour tout endomorphisme unitaire u de E, on a
\textbar{}\mathrm{det}~
u\textbar{} = 1. L'ensemble SU(E) des endomorphismes unitaires de
déterminant 1 est un sous-groupe distingué de U(E).

Démonstration On a clairement \mathrmId\_E \in
U(E). La définition (i) montre évidemment que si u et v sont unitaires,
il en est de même de u \cdot v. De plus, soit u \in U(E)~; on a
\\textbar{}u^-1(x)\\textbar{}
=\\textbar{}
u(u^-1(x))\\textbar{}
=\\textbar{} x\\textbar{} ce qui montre
que u^-1 \in U(E). Donc U(E) est un sous-groupe de (GL(E),\cdot).
On a alors 1 = \mathrm{det}~
\mathrmId\_E =\
\mathrm{det} (u^∗\cdot u)
= \mathrm{det}~
u^∗\mathrm{det}~ u
= \textbar{}\mathrm{det}~
u\textbar{}^2, soit
\textbar{}\mathrm{det}~
u\textbar{} = 1. L'application de U(E) dans le groupe multiplicatif des
nombres complexes de module 1,
u\mapsto~\mathrm{det}~
u est un morphisme de groupes~; son noyau SU(E) est donc un sous groupe
distingué.

Théorème~13.4.8 Soit u \in L(E).

\begin{itemize}
\itemsep1pt\parskip0pt\parsep0pt
\item
  (i) Si u est unitaire, il envoie toute base orthonormée sur une base
  orthonormée.
\item
  (ii) Inversement, s'il existe une base orthonormée \mathcal{E} de E telle que
  u(\mathcal{E}) soit encore orthonormée, alors u est un endomorphisme unitaire.
\end{itemize}

Démonstration (i) On a
(u(e\_i)∣u(e\_\jmath)) =
(e\_i∣e\_\jmath) =
\delta\_i^\jmath.

(ii) Soit x = \\sum ~
x\_ie\_i \in E. On a
\\textbar{}x\\textbar{}^2
= \\sum ~
\textbar{}x\_i\textbar{}^2. Mais on a aussi u(x)
= \\sum ~
x\_iu(e\_i) et comme u(\mathcal{E}) est orthonormée,
\\textbar{}u(x)\\textbar{}^2
= \\sum ~
\textbar{}x\_i\textbar{}^2~; on a donc
\forall~~x \in E,
\\textbar{}u(x)\\textbar{}
=\\textbar{} x\\textbar{}.

Théorème~13.4.9 Soit u un endomorphisme unitaire et F un sous-espace de
E stable par u. Alors F^\bot est stable par u.

Démonstration On a u(F) \subset~ F et comme u est inversible, on a
dim u(F) =\ dim~ F. On
a donc u(F) = F. Soit donc x \in F^\bot et y \in F~; il existe z \in F
tel que u(z) = y, d'où (u(x)∣y) =
(u(x)∣u(z)) =
(x∣z) = 0, et donc u(x) \in F^\bot.

\paragraph{13.4.4 Matrices unitaires}

Proposition~13.4.10 Soit E un espace hermitien. Soit u \in L(E), \mathcal{E} une
base de E, \Omega = \mathrmMat~
(( ∣ ),\mathcal{E}) et A =\
\mathrmMat (u,\mathcal{E}). Alors u est un endomorphisme
unitaire si et seulement si~A^∗\OmegaA = \Omega.

Démonstration On a \phi(u(x),u(y)) = (AX)^∗\Omega(AY ) =
X^∗A^∗\OmegaAY . L'unicité de la matrice d'une forme
bilinéaire montre que

\forall~~x,y \in E,
(u(x)∣u(y)) =
(x∣y) \mathrel\Leftrightarrow
A^∗\OmegaA = \Omega

En particulier, si \mathcal{E} est une base orthonormée de E, u est un
endomorphisme unitaire si et seulement si~A^∗A =
I\_n. Ceci conduit à la définition suivante

Définition~13.4.4 Soit A \in M\_\mathbb{C}(n). On dit que A est une matrice
unitaire si elle vérifie les conditions équivalentes

\begin{itemize}
\itemsep1pt\parskip0pt\parsep0pt
\item
  (i) A est inversible et A^-1 = A^∗
\item
  (ii) A^∗A = I\_n
\item
  (iii) AA^∗ = I\_n
\end{itemize}

Théorème~13.4.11 L'ensemble U(n) des matrices carrées unitaires d'ordre
n est un sous-groupe de (GL\_\mathbb{C}(n),.). Pour toute matrice
unitaire A, on a
\textbar{}\mathrm{det}~
A\textbar{} = 1. L'ensemble SU(n) des matrices unitaires de déterminant
1 est un sous-groupe distingué de U(n) .

Démonstration On a clairement I\_n \in U(n). La définition (i)
montre évidemment que si A et B sont unitaires, il en est de même de AB.
De plus, soit A \in U(n)~; on a
A^-1(A^-1)^∗ =
A^-1(A^∗)^∗ = A^-1A =
I\_n ce qui montre que A^-1 \in U(n). Donc U(n) est un
sous-groupe de (GL\_\mathbb{C}(n),.). On a alors 1
= \mathrm{det} I\_n~
= \mathrm{det}~
(A^∗A) =
\textbar{}\mathrm{det}~
A\textbar{}^2, soit
\textbar{}\mathrm{det}~
A\textbar{} = 1. L'application de U(n) dans le groupe multiplicatif des
nombres complexes de module 1,
A\mapsto~\mathrm{det}~
A est un morphisme de groupes multiplicatifs~; son noyau SU(n) est donc
un sous-groupe distingué.

Dans ce paragraphe, on munira \mathbb{C}^n de la forme sesquilinéaire
hermitienne naturelle (qui rend la base canonique orthonormée),
c'est-à-dire que l'on posera (x∣y)
= \\sum ~
\_i=1^n\overlinex\_iy\_i

Théorème~13.4.12 Une matrice A \in M\_\mathbb{C}(n) est unitaire si et
seulement si~ses vecteurs colonnes (resp. lignes) forment une base
orthonormée de \mathbb{C}^n.

Démonstration Soit
(c\_1,\\ldots,c\_n~)
les vecteurs colonnes de A,
(l\_1,\\ldots,l\_n~)
ses vecteurs lignes. On a

\begin{align*} A \in U(n)&
\Leftrightarrow & A^∗A = I\_ n
\Leftrightarrow \forall~~i,\jmath,
(A^∗A)\_ i,\jmath = \delta\_i^\jmath \%&
\\ & \Leftrightarrow &
\forall~~i,\jmath, \\sum
\_k=1^n\overlinea\_
k,ia\_k,\jmath = \delta\_i^\jmath
\Leftrightarrow \forall~i,\jmath, (c~\_
i∣c\_\jmath) = \delta\_i^\jmath\%&
\\ \end{align*}

De la même fa\ccon, en traduisant la relation
AA^∗ = I\_n, on obtiendrait
(l\_i∣l\_\jmath) =
\delta\_i^\jmath.

Théorème~13.4.13 Soit E un espace hermitien. Soit \mathcal{E} une base orthonormée
de E, \mathcal{E}' une base de E. Alors on a équivalence de

\begin{itemize}
\itemsep1pt\parskip0pt\parsep0pt
\item
  (i) \mathcal{E}' est orthonormée
\item
  (ii) la matrice P\_\mathcal{E}^\mathcal{E}' de passage de la base \mathcal{E} à la
  base \mathcal{E}' est unitaire.
\end{itemize}

Démonstration On sait que P\_\mathcal{E}^\mathcal{E}'
= \mathrmMat~ (u,\mathcal{E}) où u est
l'endomorphisme de E défini par \forall~~i,
u(e\_i) = e\_i'. Or d'après les résultats du paragraphe
précédent, u est un endomorphisme unitaire si et seulement si~\mathcal{E}' est
orthonormée~; mais d'autre part, comme \mathcal{E} est orthonormée, u est unitaire
si et seulement
si~\mathrmMat~ (u,\mathcal{E}) est une
matrice unitaire, d'où l'équivalence entre (i) et (ii).

\paragraph{13.4.5 Réduction des endomorphismes normaux}

Définition~13.4.5 Soit E un espace hermitien et u \in L(E). On dit que u
est un endomorphisme normal si

u^∗u = uu^∗

Lemme~13.4.14 Soit u un endomorphisme normal. Alors
\mathrmKeru^∗~
= \mathrmKer~u.

Démonstration On a

\begin{align*} x
\in\mathrmKeru^∗~&
\Leftrightarrow &
(u^∗(x)∣u^∗(x)) = 0
\Leftrightarrow
(uu^∗(x)∣x) = 0\%&
\\ & \Leftrightarrow &
(u^∗u(x)∣x) = 0
\Leftrightarrow (u(x)\mathrel∣u(x)) = 0
\%& \\ & \Leftrightarrow &
x \in\mathrmKer~u \%&
\\ \end{align*}

Lemme~13.4.15 2. Soit u un endomorphisme normal. Alors, pour tout \lambda~ \in \mathbb{C},
\mathrmKer(u^∗-\overline\lambda~\mathrmId\_E~)
= \mathrmKer~(u -
\lambda~\mathrmId\_E).

Démonstration Il suffit de remarquer que u -
\lambda~\mathrmId est encore normal (élémentaire) et de lui
appliquer le lemme précédent en remarquant que
u^∗-\overline\lambda~\mathrmId\_E
= (u - \lambda~\mathrmId\_E)^∗

Théorème~13.4.16 Soit u un endomorphisme d'un espace hermitien. On a
équivalence de

\begin{itemize}
\itemsep1pt\parskip0pt\parsep0pt
\item
  (i) u est normal
\item
  (ii) u est diagonalisable dans une base orthonormée.
\end{itemize}

Démonstration (ii) \rigtharrow~(i) Soit \mathcal{E} une base orthonormée de diagonalisation
de u. Alors \mathrmMat~
(u,\mathcal{E}) =\
diag(\lambda~\_1,\\ldots,\lambda~\_n~).
Comme \mathcal{E} est orthonormée, on a
\mathrmMat~
(u^∗,\mathcal{E}) =\
\mathrmMat (u,\mathcal{E})^∗
=\
diag(\overline\lambda~\_1,\\ldots,\overline\lambda~\_n~).
Les deux matrices diagonales commutant, on a uu^∗ =
u^∗u, donc u est normal.

(i) \rigtharrow~(ii) Montrons le résultat par récurrence sur
dim~ E, le résultat étant évident si
dim~ E = 1. Supposons que u est normal. Comme \mathbb{C}
est algébriquement clos, u admet une valeur propre \lambda~. Comme
\mathrmKer(u^∗-\overline\lambda~\mathrmId\_E~)
= \mathrmKer~(u -
\lambda~\mathrmId\_E), E\_u(\lambda~)
= \mathrmKer~(u -
\lambda~\mathrmId\_E) est stable par u^∗
et donc E\_u(\lambda~)^\bot est stable par u^∗∗ = u.
Mais comme E\_u(\lambda~) est stable par u, le sous-espace
E\_u(\lambda~)^\bot est stable par u^∗. Soit v =
u\_\textbar{}\_ E\_u(\lambda~)^\bot. La
relation (v(x)∣y) =
(u(x)∣y) =
(x∣u^∗(y)) pour x,y \in
E\_u(\lambda~)^\bot montre que v^∗ =
u\_\textbar{}\_ E\_u(\lambda~)^\bot^∗,
donc v^∗v = vv^∗ et donc v est un endomorphisme
normal de E\_u(\lambda~)^\bot. Par hypothèse de récurrence, il
existe une base orthonormée de E\_u(\lambda~)^\bot formée de
vecteurs propres de v donc de u. Comme E = E\_u(\lambda~) \bot \oplus~
E\_u(\lambda~)^\bot, si on réunit cette base avec une base
orthonormée de E\_u(\lambda~), on obtient une base orthonormée de E
formée évidemment de vecteurs propres de u, ce qui achève la
démonstration.

Remarque~13.4.3 Soit \mathcal{E} une telle base. Alors
\mathrmMat~ (u,\mathcal{E})
=\
diag(\lambda~\_1,\\ldots,\lambda~\_n~).
L'endomorphisme u est hermitien si et seulement si~sa matrice dans la
base orthonormée \mathcal{E} est hermitienne, c'est-à-dire si et seulement
si~\forall~i, \lambda~\_i~ \in \mathbb{R}~~; de même u est
unitaire si et seulement si~sa matrice dans la base orthonormée \mathcal{E} est
unitaire, c'est-à-dire si et seulement si~\forall~~i,
\textbar{}\lambda~\_i\textbar{} = 1. Comme il est clair que tout
endomorphisme hermitien ou unitaire est normal on obtient les deux
corollaires

Corollaire~13.4.17 Soit u un endomorphisme d'un espace hermitien. On a
équivalence de

\begin{itemize}
\itemsep1pt\parskip0pt\parsep0pt
\item
  (i) u est hermitien
\item
  (ii) u est diagonalisable dans une base orthonormée et
  \mathrm{Sp}~(u) \subset~ \mathbb{R}~
\end{itemize}

Corollaire~13.4.18 Soit u un endomorphisme d'un espace hermitien. On a
équivalence de

\begin{itemize}
\itemsep1pt\parskip0pt\parsep0pt
\item
  (i) u est unitaire
\item
  (ii) u est diagonalisable dans une base orthonormée et
  \mathrm{Sp}~(u) \subset~ U
  (ensemble des nombres complexes de module 1)
\end{itemize}

\paragraph{13.4.6 Réduction des matrices normales}

En traduisant le paragraphe précédent en terme de matrices (en utilisant
le produit hermitien canonique sur \mathbb{C}^2 défini par
(x∣y) =\
\sum ~
\_i\overlinex\_iy\_i) on
obtient la définition et les résultats suivants.

Définition~13.4.6 Soit A \in M\_\mathbb{C}(n). On dit que A est une matrice
normale si

A^∗A = AA^∗

Théorème~13.4.19 Soit A \in M\_\mathbb{C}(n). On a équivalence de

\begin{itemize}
\itemsep1pt\parskip0pt\parsep0pt
\item
  (i) A est normal
\item
  (ii) Il existe P unitaire telle que P^-1AP =
  P^∗AP soit diagonale.
\end{itemize}

Corollaire~13.4.20 Soit A \in M\_\mathbb{C}(n). On a équivalence de

\begin{itemize}
\itemsep1pt\parskip0pt\parsep0pt
\item
  (i) A est hermitienne
\item
  (ii) Il existe P unitaire telle que P^-1AP =
  P^∗AP soit diagonale réelle
\end{itemize}

Corollaire~13.4.21 Soit A \in M\_\mathbb{C}(n). On a équivalence de

\begin{itemize}
\itemsep1pt\parskip0pt\parsep0pt
\item
  (i) A est unitaire
\item
  (ii) Il existe P unitaire telle que P^-1AP =
  P^∗AP soit diagonale à éléments diagonaux dans U (ensemble
  des nombres complexes de module 1)
\end{itemize}

{[}
{[}
{[}
{[}

\end{document}

\part{Séries de Fourier}
\documentclass[]{article}
\usepackage[T1]{fontenc}
\usepackage{lmodern}
\usepackage{amssymb,amsmath}
\usepackage{ifxetex,ifluatex}
\usepackage{fixltx2e} % provides \textsubscript
% use upquote if available, for straight quotes in verbatim environments
\IfFileExists{upquote.sty}{\usepackage{upquote}}{}
\ifnum 0\ifxetex 1\fi\ifluatex 1\fi=0 % if pdftex
  \usepackage[utf8]{inputenc}
\else % if luatex or xelatex
  \ifxetex
    \usepackage{mathspec}
    \usepackage{xltxtra,xunicode}
  \else
    \usepackage{fontspec}
  \fi
  \defaultfontfeatures{Mapping=tex-text,Scale=MatchLowercase}
  \newcommand{\euro}{€}
\fi
% use microtype if available
\IfFileExists{microtype.sty}{\usepackage{microtype}}{}
\ifxetex
  \usepackage[setpagesize=false, % page size defined by xetex
              unicode=false, % unicode breaks when used with xetex
              xetex]{hyperref}
\else
  \usepackage[unicode=true]{hyperref}
\fi
\hypersetup{breaklinks=true,
            bookmarks=true,
            pdfauthor={},
            pdftitle={Introduction : transformee de Fourier sur les groupes abeliens finis},
            colorlinks=true,
            citecolor=blue,
            urlcolor=blue,
            linkcolor=magenta,
            pdfborder={0 0 0}}
\urlstyle{same}  % don't use monospace font for urls
\setlength{\parindent}{0pt}
\setlength{\parskip}{6pt plus 2pt minus 1pt}
\setlength{\emergencystretch}{3em}  % prevent overfull lines
\setcounter{secnumdepth}{0}
 
/* start css.sty */
.cmr-5{font-size:50%;}
.cmr-7{font-size:70%;}
.cmmi-5{font-size:50%;font-style: italic;}
.cmmi-7{font-size:70%;font-style: italic;}
.cmmi-10{font-style: italic;}
.cmsy-5{font-size:50%;}
.cmsy-7{font-size:70%;}
.cmex-7{font-size:70%;}
.cmex-7x-x-71{font-size:49%;}
.msbm-7{font-size:70%;}
.cmtt-10{font-family: monospace;}
.cmti-10{ font-style: italic;}
.cmbx-10{ font-weight: bold;}
.cmr-17x-x-120{font-size:204%;}
.cmsl-10{font-style: oblique;}
.cmti-7x-x-71{font-size:49%; font-style: italic;}
.cmbxti-10{ font-weight: bold; font-style: italic;}
p.noindent { text-indent: 0em }
td p.noindent { text-indent: 0em; margin-top:0em; }
p.nopar { text-indent: 0em; }
p.indent{ text-indent: 1.5em }
@media print {div.crosslinks {visibility:hidden;}}
a img { border-top: 0; border-left: 0; border-right: 0; }
center { margin-top:1em; margin-bottom:1em; }
td center { margin-top:0em; margin-bottom:0em; }
.Canvas { position:relative; }
li p.indent { text-indent: 0em }
.enumerate1 {list-style-type:decimal;}
.enumerate2 {list-style-type:lower-alpha;}
.enumerate3 {list-style-type:lower-roman;}
.enumerate4 {list-style-type:upper-alpha;}
div.newtheorem { margin-bottom: 2em; margin-top: 2em;}
.obeylines-h,.obeylines-v {white-space: nowrap; }
div.obeylines-v p { margin-top:0; margin-bottom:0; }
.overline{ text-decoration:overline; }
.overline img{ border-top: 1px solid black; }
td.displaylines {text-align:center; white-space:nowrap;}
.centerline {text-align:center;}
.rightline {text-align:right;}
div.verbatim {font-family: monospace; white-space: nowrap; text-align:left; clear:both; }
.fbox {padding-left:3.0pt; padding-right:3.0pt; text-indent:0pt; border:solid black 0.4pt; }
div.fbox {display:table}
div.center div.fbox {text-align:center; clear:both; padding-left:3.0pt; padding-right:3.0pt; text-indent:0pt; border:solid black 0.4pt; }
div.minipage{width:100%;}
div.center, div.center div.center {text-align: center; margin-left:1em; margin-right:1em;}
div.center div {text-align: left;}
div.flushright, div.flushright div.flushright {text-align: right;}
div.flushright div {text-align: left;}
div.flushleft {text-align: left;}
.underline{ text-decoration:underline; }
.underline img{ border-bottom: 1px solid black; margin-bottom:1pt; }
.framebox-c, .framebox-l, .framebox-r { padding-left:3.0pt; padding-right:3.0pt; text-indent:0pt; border:solid black 0.4pt; }
.framebox-c {text-align:center;}
.framebox-l {text-align:left;}
.framebox-r {text-align:right;}
span.thank-mark{ vertical-align: super }
span.footnote-mark sup.textsuperscript, span.footnote-mark a sup.textsuperscript{ font-size:80%; }
div.tabular, div.center div.tabular {text-align: center; margin-top:0.5em; margin-bottom:0.5em; }
table.tabular td p{margin-top:0em;}
table.tabular {margin-left: auto; margin-right: auto;}
div.td00{ margin-left:0pt; margin-right:0pt; }
div.td01{ margin-left:0pt; margin-right:5pt; }
div.td10{ margin-left:5pt; margin-right:0pt; }
div.td11{ margin-left:5pt; margin-right:5pt; }
table[rules] {border-left:solid black 0.4pt; border-right:solid black 0.4pt; }
td.td00{ padding-left:0pt; padding-right:0pt; }
td.td01{ padding-left:0pt; padding-right:5pt; }
td.td10{ padding-left:5pt; padding-right:0pt; }
td.td11{ padding-left:5pt; padding-right:5pt; }
table[rules] {border-left:solid black 0.4pt; border-right:solid black 0.4pt; }
.hline hr, .cline hr{ height : 1px; margin:0px; }
.tabbing-right {text-align:right;}
span.TEX {letter-spacing: -0.125em; }
span.TEX span.E{ position:relative;top:0.5ex;left:-0.0417em;}
a span.TEX span.E {text-decoration: none; }
span.LATEX span.A{ position:relative; top:-0.5ex; left:-0.4em; font-size:85%;}
span.LATEX span.TEX{ position:relative; left: -0.4em; }
div.float img, div.float .caption {text-align:center;}
div.figure img, div.figure .caption {text-align:center;}
.marginpar {width:20%; float:right; text-align:left; margin-left:auto; margin-top:0.5em; font-size:85%; text-decoration:underline;}
.marginpar p{margin-top:0.4em; margin-bottom:0.4em;}
.equation td{text-align:center; vertical-align:middle; }
td.eq-no{ width:5%; }
table.equation { width:100%; } 
div.math-display, div.par-math-display{text-align:center;}
math .texttt { font-family: monospace; }
math .textit { font-style: italic; }
math .textsl { font-style: oblique; }
math .textsf { font-family: sans-serif; }
math .textbf { font-weight: bold; }
.partToc a, .partToc, .likepartToc a, .likepartToc {line-height: 200%; font-weight:bold; font-size:110%;}
.chapterToc a, .chapterToc, .likechapterToc a, .likechapterToc, .appendixToc a, .appendixToc {line-height: 200%; font-weight:bold;}
.index-item, .index-subitem, .index-subsubitem {display:block}
.caption td.id{font-weight: bold; white-space: nowrap; }
table.caption {text-align:center;}
h1.partHead{text-align: center}
p.bibitem { text-indent: -2em; margin-left: 2em; margin-top:0.6em; margin-bottom:0.6em; }
p.bibitem-p { text-indent: 0em; margin-left: 2em; margin-top:0.6em; margin-bottom:0.6em; }
.paragraphHead, .likeparagraphHead { margin-top:2em; font-weight: bold;}
.subparagraphHead, .likesubparagraphHead { font-weight: bold;}
.quote {margin-bottom:0.25em; margin-top:0.25em; margin-left:1em; margin-right:1em; text-align:justify;}
.verse{white-space:nowrap; margin-left:2em}
div.maketitle {text-align:center;}
h2.titleHead{text-align:center;}
div.maketitle{ margin-bottom: 2em; }
div.author, div.date {text-align:center;}
div.thanks{text-align:left; margin-left:10%; font-size:85%; font-style:italic; }
div.author{white-space: nowrap;}
.quotation {margin-bottom:0.25em; margin-top:0.25em; margin-left:1em; }
h1.partHead{text-align: center}
.sectionToc, .likesectionToc {margin-left:2em;}
.subsectionToc, .likesubsectionToc {margin-left:4em;}
.subsubsectionToc, .likesubsubsectionToc {margin-left:6em;}
.frenchb-nbsp{font-size:75%;}
.frenchb-thinspace{font-size:75%;}
.figure img.graphics {margin-left:10%;}
/* end css.sty */

\title{Introduction : transformee de Fourier sur les groupes abeliens finis}
\author{}
\date{}

\begin{document}
\maketitle

\textbf{Warning: 
requires JavaScript to process the mathematics on this page.\\ If your
browser supports JavaScript, be sure it is enabled.}

\begin{center}\rule{3in}{0.4pt}\end{center}

[
[]
[

\subsubsection{14.1 Introduction~: transformée de Fourier sur les
groupes abéliens finis}

Ce paragraphe sert simplement d'introduction à la suite du chapitre.
Lors d'une première lecture il peut être sauté sans inconvénient.

\paragraph{14.1.1 Caractères des groupes abéliens finis}

Définition~14.1.1 Soit (G,.) un groupe abélien fini. On appelle
caractère de G tout morphisme de groupe \chi de G dans (\mathbb{C}^∗,.).
On note \hatG l'ensemble des caractères de G.

Remarque~14.1.1 On vérifie immédiatement que \hatG
est lui même muni d'une structure de groupes en posant (\chi\chi')(x) =
\chi(x)\chi'(x).

Proposition~14.1.1 Soit \chi \in\hat G. Alors
\forall~~x \in G, \chi(x) = 1.

Démonstration Puisque G est un groupe fini, tout élément est d'ordre
fini, et donc il existe n \in \mathbb{N}~ tel que x^n = e. On a donc 1 =
\chi(e) = \chi(x^n) = \chi(x)^n ce qui montre que \chi(x) est
une racine de l'unité donc de module 1.

Proposition~14.1.2 \hatG est un groupe abélien fini.

Démonstration On sait que \forall~~x \in G,
x^G = e. Le même raisonnement que ci
dessus monte que \chi(x) est une racine G-ième de
l'unité. Donc \hatG est un sous-ensemble de
l'ensemble des applications de G dans le groupe fini
\Gamma_G des racines G-ièmes
de l'unité, donc il est fini.

Lemme~14.1.3 Soit G un groupe abélien fini et H un sous-groupe de G.
Soit \psi un caractère de H. Alors il existe un caractère \chi de G dont la
restriction à H est \psi.

Démonstration Pour des raisons de cardinal, il existe un sous-groupe K
maximal auquel \psi admet un prolongement \phi \in\hat K.
Nous allons montrer par l'absurde que K = G, ce qui démontrera le lemme.
Supposons donc que K\neq~G et soit x \in G \diagdown K.
L'ensemble des n \in \mathbb{Z} tel que x^n \in K est un sous-groupe de \mathbb{Z},
donc de la forme d\mathbb{Z} pour un d > 0 (car
(x^G = e \in K) et
d\neq~1 (car x\mathrel∉K).
Soit \omega \in \mathbb{C} tel que \omega^d = \phi(x^d). Soit K' =
\x^nk∣n \in \mathbb{Z}, k \in
K\ le sous-groupe de G engendré par K et x et
prolongeons \phi à K' en posant \phi'(x^nk) = \omega^n\phi(k).
Montrons tout d'abord que \phi' est bien définie. Si l'on a x^nk
= x^mk', on a x^n-m = k'k^-1 \in K et
donc d divise n - m. Posons n - m = dq si bien que x^dq =
k'k^-1~; on a alors \phi(k')\phi(k)^-1 =
\phi(x^dq) = \phi(x^d)^q =
(\omega^d)^q = \omega^dq = \omega^n-m, si
bien que \omega^n\phi(k) = \omega^m\phi(k') ce qui montre que \phi'
est bien définie. Il est élémentaire de vérifier que \phi' est encore un
morphisme de groupes et il est clair qu'il prolonge \phi et donc qu'il
prolonge \psi. Mais ceci contredit alors la maximalité de K. On a donc K =
G et donc \psi se prolonge à G tout entier.

Théorème~14.1.4 Soit x \in G, x\neq~e. Alors il
existe \chi \in\hat G tel que
\chi(x)\neq~1.

Démonstration Soit d l'ordre de x, \omega = exp~ (
2i\pi~ \over d ), H le sous-groupe engendré par x.
L'application x^n\mapsto~\omega^n
est bien définie (car si x^n = x^p, alors d divise
n - p et donc \omega^n = \omega^p) et c'est un caractère de
H (facile). Donc il existe \chi \in\hat G qui prolonge \psi.
On a en particulier \chi(x) = \omega\neq~1.

Définition~14.1.2 Si f est une application de G dans \mathbb{C}, on notera
\int  _G~f =\
\sum  _x\inG~f(x). En posant alors
(f∣g) = 1 \over
G \int ~
_G\overlinefg, on munit l'espace E des
applications de G dans \mathbb{C} d'une structure d'espace hermitien.

Proposition~14.1.5

\begin{itemize}
\itemsep1pt\parskip0pt\parsep0pt
\item
  (i) Soit \chi \in\hat G. Alors
  \int  _G~\chi = \left
  \\cases 0 &si
  \chi\neq~1 \cr
  G&si \chi = 1  \right ..
\item
  (ii) Soit x \in G. Alors
  \\sum ~
  _\chi\in\hatG\chi(x) = \left
  \\cases 0 &si
  x\neq~e \cr
  \hatG&si x = e 
  \right .
\end{itemize}

Démonstration (i) Supposons que \chi\neq~1 et soit x
\in G tel que \chi(x)\neq~1. On a alors

\chi(x)\sum _g\inG~\chi(g) =
\sum _g\inG~\chi(xg) =
\sum _g\inG~\chi(g)

puisque g\mapsto~xg est une bijection de G sur lui
même. Comme \chi(x)\neq~1, on en déduit que
\\sum  _g\inG~\chi(g)
= 0. Si par contre, \chi = 1, on a
\\sum  _g\inG~\chi(g)
= G.

(ii) Si x\neq~e, soit \phi \in\hat
G tel que \phi(x)\neq~1. On a alors

\phi(x)\sum _\chi\in\hatG~\chi(x)
= \\sum
_\chi\in\hatG(\phi\chi)(x) =
\sum _\chi\in\hatG~\chi(x)

puisque \chi\mapsto~\phi\chi est une bijection de
\hatG sur lui même. Comme
\phi(x)\neq~1, on a
\\sum ~
_\chi\in\hatG\chi(x) = 0. Si par contre, x = e, on a
pour tout \chi, \chi(x) = 1 et donc
\\sum ~
_\chi\in\hatG\chi(x) =
\hatG.

Corollaire~14.1.6 G =
\hatG.

Démonstration Soit S =\
\sum ~
_\chi\in\hatG\
\sum  _x\inG~\chi(x). On a (en utilisant le
symbole de Kronecker \delta_a^b = \left
\ \cases 1&si a = b
\cr 0&si a\neq~b\\ 
\right . et en notant 1 le caractère constant égal à 1) S
= \\sum ~
_\chi\in\hatGG\delta_\chi^1
= G. Mais on a aussi S =\
\sum  _x\inG~\
\sum  _\chi\in\hatG~\chi(x)
= \\sum ~
_x\inG\hatG\delta_x^e
= \hatG d'où le résultat.

Corollaire~14.1.7 \hatG est une famille orthonormée
de E.

Démonstration On a (\chi∣\phi) = 1
\over G \\int
 _G\overline\chi\phi = 0 si
\overline\chi\phi\neq~1, soit
\chi\neq~\phi (puisque \overline\chi(g)
= 1 \over \chi(g) comme nombre complexe de module 1). Si
par contre, \chi = \phi, on a \overline\chi\phi =
\chi^2 = 1 et donc
(\chi∣\phi) = 1.

\paragraph{14.1.2 Transformée de Fourier sur un groupe abélien fini}

Définition~14.1.3 Soit G un groupe abélien fini et soit f : G \rightarrow~ \mathbb{C}. On
définit la transformée de Fourier de f comme étant l'application
\hatf :\hat G \rightarrow~ \mathbb{C} définie par

\forall~\chi \in\hat G~,
\hatf(\chi) = (\chi∣f) =
1\over
G\int ~
_Gf\overline\chi

Théorème~14.1.8 (cf Dirichlet). Soit f : G \rightarrow~ \mathbb{C}. Alors,

\forall~~x \in G, f(x) = \\sum
_\chi\in\hatG\hatf(\chi)\chi(x)

Démonstration On a

\begin{align*} \\sum
_\chi\in\hatG\hatf(\chi)\chi(x)&
=& 1 \over G
\sum _\chi\in\hatG~
\\sum
_y\inGf(y)\overline\chi(y)\chi(x)\%&
\\ & =& 1 \over
G \\sum
_y\inGf(y)\\sum
_\chi\in\hatG\chi(xy^-1) \%&
\\ \end{align*}

puisque \overline\chi(y) = 1 \over
\chi(y) = \chi(y^-1). Mais, d'après un résultat précédent
\\sum ~
_\chi\in\hatG\chi(xy^-1) = 0 si
xy^-1\neq~e, soit
y\neq~x et
\\sum ~
_\chi\in\hatG\chi(xy^-1) =
\hatG = G si
y = x. D'où ne persiste dans la somme que le terme pour y = x et donc
\\sum ~
_\chi\in\hatG\hatf(\chi)\chi(x) =
f(x).

Corollaire~14.1.9 \hatG est une base orthonormée de
E.

Démonstration Le théorème précédent montre que c'est une famille
génératrice et on a vu précédemment que c'est une famille orthonormée,
donc libre.

Théorème~14.1.10 (cf Parseval-Plancherel). Soit f : G \rightarrow~ \mathbb{C}. Alors

\f\^2 =
(f∣f) = \\sum
_\chi\in\hatG\hatf(\chi)^2

Démonstration On a vu précédemment que les \hatf(\chi)
sont les coordonnées de f dans la base orthonormée
\hatG de E et le carré de la norme de f dans E est la
somme des modules des carrés des coordonnées dans une base orthonormée.

[
[

\end{document}

% 
\subsubsection{14.2 Séries trigonométriques}

\paragraph{14.2.1 Rappels d'intégration}

Lemme~14.2.1 Soit f : \mathbb{R}~ \rightarrow~ \mathbb{C} périodique de période T, continue par
morceaux. Alors, pour tout a \in \mathbb{R}~, \int ~
_a^a+Tf(t) dt =\int ~
_0^Tf(t) dt

Démonstration On écrit \int ~
_a^a+Tf(t) dt =\int ~
_a^0f(t) dt+\int ~
_0^Tf(t) dt+\int ~
_T^a+Tf(t) dt =\int ~
_a^0f(t) dt+\int ~
_0^Tf(t) dt+\int ~
_0^af(u+T) du = \int ~
_a^0f(t) dt +\int ~
_0^Tf(t) dt +\int ~
_0^af(u) du =\int ~
_0^Tf(t) dt en faisant le changement de variable u = t -
T.

Lemme~14.2.2 Pour tout n \in \mathbb{Z}, \int ~
_0^2\pi~e^int dt = 2\pi~\delta_n^0.

\paragraph{14.2.2 Généralités}

Définition~14.2.1 (forme réelle). Soit (a_n)_n≥0 et
(b_n)_n≥1 deux suites de nombres complexes. On appelle
série trigonométrique associée la série de fonctions de \mathbb{R}~ dans \mathbb{C},

a_0 + \sum _n≥1(a_n~
\cos nx + b_n \sin nx)

Remarque~14.2.1 Soit n \in \mathbb{N}~^∗ et a_n et b_n
deux nombres complexes. On a alors a_n\
cos nx + b_n sin~ nx =
c_ne^inx + c_-ne^-inx avec
c_n = a_n-ib_n \over 2 et
c_-n = a_n+ib_n \over 2 .
Inversement, si on se donne deux nombres complexes c_n et
c_-n, on a c_ne^inx +
c_-ne^-inx = a_n\
sin nx + b_n cos~ nx avec
a_n = c_n + c_-n et b_n =
i(c_n - c_-n). Ceci amène également à poser

Définition~14.2.2 (forme complexe). Soit (c_n)_n\in\mathbb{Z} une
suite de nombres complexes. On appelle série trigonométrique associée la
série de fonctions de \mathbb{R}~ dans \mathbb{C},

c_0 + \\sum
_n≥1(c_ne^inx + c_
-ne^-inx)

On passe donc de la forme réelle à la forme complexe ou vice versa par
les formules

\begin{align*} a_0& =& c_0 \%&
\\ \forall~~n ≥
1,\quad c_n& =& a_n - ib_n
\over 2 ,\quad c_-n =
a_n + ib_n \over 2 \%&
\\ \forall~~n ≥
1,\quad a_n& =& c_n +
c_-n,\quad b_n = i(c_n -
c_-n)\%& \\
\end{align*}

\paragraph{14.2.3 Un cas de convergence normale}

Théorème~14.2.3 On considère une série trigonométrique vérifiant les
conditions équivalentes

\begin{itemize}
\itemsep1pt\parskip0pt\parsep0pt
\item
  (i) les deux séries \\\sum
   a_n et
  \\sum ~
  b_n sont convergentes.
\item
  (ii) les deux séries
  \\sum ~
  _n≥0c_n et
  \\sum ~
  _n≥0c_-n sont convergentes.
\end{itemize}

Alors la série trigonométrique converge normalement sur \mathbb{R}~, sa somme f
est une fonction continue périodique de période 2\pi~ et on a

\begin{align*} \forall~~n \in
\mathbb{Z},\quad c_n& =& 1 \over 2\pi~
\int  _0^2\pi~f(t)e^-int~
dt \%& \\ \forall~~n ≥
1,\quad a_n& =& 1 \over \pi~
\int ~
_0^2\pi~f(t)cos~ nt
dt,\quad b_ n = 1 \over \pi~
\int ~
_0^2\pi~f(t)sin~ nt dt\%&
\\ \end{align*}

Démonstration Les relations
a_n\leqc_n +
c_-n,
b_n\leqc_n +
c_-n, c_n\leq 1
\over 2 (a_n +
b_n) et
c_-n\leq 1 \over 2
(a_n + b_n)
(que l'on déduit facilement des relations du paragraphe précédent)
montrent clairement l'équivalence. Alors on a

\forall~x \in \mathbb{R}~, c_ne^inx~
+ c_ -ne^-inx\leqc_
n + c_-n

qui est une série convergente indépendante de x. On a donc la
convergence normale de la série et en particulier la continuité de sa
somme. Cette somme est évidemment périodique de période 2\pi~ puisque
toutes les applications
x\mapsto~c_ne^inx +
c_-ne^-inx le sont. Soit p \in \mathbb{Z}. On a aussi
\forall~~x \in \mathbb{R}~,
(c_ne^inx +
c_-ne^-inx)e^-ipx\leqc_n
+ c_-n ce qui montre que la série
c_0e^-ipx +\
\sum  _n≥1(c_ne^inx~
+ c_-ne^-inx)e^-ipx converge normalement
sur \mathbb{R}~, donc sur [0,2\pi~]. Ceci justifie donc dans le calcul suivant
l'interversion du signe d'intégrale et du signe somme

\begin{align*} \int ~
_0^2\pi~f(t)e^-ipt dt&& \%&
\\ & =& \int ~
_0^2\pi~\left (c_ 0e^-ipt
+ \sum _n≥1(c_ne^int~
+ c_ -ne^-int)e^-ipt\right
) dt \%& \\ & =&
c_0\int ~
_0^2\pi~e^-ipt dt \%&
\\ & \text &
+\sum _n=1^+\infty~~\left
(c_ n \\int  ~
_0^2\pi~e^i(n-p)t dt + c_ -n
\\int  ~
_0^2\pi~e^-i(n+p)t dt\right )\%&
\\ & =& 2\pi~\left
(c_0\delta_p^0 + \\sum
_n=1^+\infty~(c_ n\delta_p^n + c_
-n\delta_p^-n)\right ) = 2\pi~c_ p
\%& \\ \end{align*}

en distinguant les différents cas possibles p = 0, p ≥ 1 ou p \leq-1. Les
relations sur les a_n et b_n s'en déduisent facilement
par les formules du premier paragraphe.

Remarque~14.2.2 La même technique permet d'aboutir aux mêmes formules
dès que la série trigonométrique converge uniformément sur un segment de
longueur 2\pi~.


% \documentclass[]{article}
\usepackage[T1]{fontenc}
\usepackage{lmodern}
\usepackage{amssymb,amsmath}
\usepackage{ifxetex,ifluatex}
\usepackage{fixltx2e} % provides \textsubscript
% use upquote if available, for straight quotes in verbatim environments
\IfFileExists{upquote.sty}{\usepackage{upquote}}{}
\ifnum 0\ifxetex 1\fi\ifluatex 1\fi=0 % if pdftex
  \usepackage[utf8]{inputenc}
\else % if luatex or xelatex
  \ifxetex
    \usepackage{mathspec}
    \usepackage{xltxtra,xunicode}
  \else
    \usepackage{fontspec}
  \fi
  \defaultfontfeatures{Mapping=tex-text,Scale=MatchLowercase}
  \newcommand{\euro}{€}
\fi
% use microtype if available
\IfFileExists{microtype.sty}{\usepackage{microtype}}{}
\ifxetex
  \usepackage[setpagesize=false, % page size defined by xetex
              unicode=false, % unicode breaks when used with xetex
              xetex]{hyperref}
\else
  \usepackage[unicode=true]{hyperref}
\fi
\hypersetup{breaklinks=true,
            bookmarks=true,
            pdfauthor={},
            pdftitle={Serie de Fourier d'une fonction},
            colorlinks=true,
            citecolor=blue,
            urlcolor=blue,
            linkcolor=magenta,
            pdfborder={0 0 0}}
\urlstyle{same}  % don't use monospace font for urls
\setlength{\parindent}{0pt}
\setlength{\parskip}{6pt plus 2pt minus 1pt}
\setlength{\emergencystretch}{3em}  % prevent overfull lines
\setcounter{secnumdepth}{0}
 
/* start css.sty */
.cmr-5{font-size:50%;}
.cmr-7{font-size:70%;}
.cmmi-5{font-size:50%;font-style: italic;}
.cmmi-7{font-size:70%;font-style: italic;}
.cmmi-10{font-style: italic;}
.cmsy-5{font-size:50%;}
.cmsy-7{font-size:70%;}
.cmex-7{font-size:70%;}
.cmex-7x-x-71{font-size:49%;}
.msbm-7{font-size:70%;}
.cmtt-10{font-family: monospace;}
.cmti-10{ font-style: italic;}
.cmbx-10{ font-weight: bold;}
.cmr-17x-x-120{font-size:204%;}
.cmsl-10{font-style: oblique;}
.cmti-7x-x-71{font-size:49%; font-style: italic;}
.cmbxti-10{ font-weight: bold; font-style: italic;}
p.noindent { text-indent: 0em }
td p.noindent { text-indent: 0em; margin-top:0em; }
p.nopar { text-indent: 0em; }
p.indent{ text-indent: 1.5em }
@media print {div.crosslinks {visibility:hidden;}}
a img { border-top: 0; border-left: 0; border-right: 0; }
center { margin-top:1em; margin-bottom:1em; }
td center { margin-top:0em; margin-bottom:0em; }
.Canvas { position:relative; }
li p.indent { text-indent: 0em }
.enumerate1 {list-style-type:decimal;}
.enumerate2 {list-style-type:lower-alpha;}
.enumerate3 {list-style-type:lower-roman;}
.enumerate4 {list-style-type:upper-alpha;}
div.newtheorem { margin-bottom: 2em; margin-top: 2em;}
.obeylines-h,.obeylines-v {white-space: nowrap; }
div.obeylines-v p { margin-top:0; margin-bottom:0; }
.overline{ text-decoration:overline; }
.overline img{ border-top: 1px solid black; }
td.displaylines {text-align:center; white-space:nowrap;}
.centerline {text-align:center;}
.rightline {text-align:right;}
div.verbatim {font-family: monospace; white-space: nowrap; text-align:left; clear:both; }
.fbox {padding-left:3.0pt; padding-right:3.0pt; text-indent:0pt; border:solid black 0.4pt; }
div.fbox {display:table}
div.center div.fbox {text-align:center; clear:both; padding-left:3.0pt; padding-right:3.0pt; text-indent:0pt; border:solid black 0.4pt; }
div.minipage{width:100%;}
div.center, div.center div.center {text-align: center; margin-left:1em; margin-right:1em;}
div.center div {text-align: left;}
div.flushright, div.flushright div.flushright {text-align: right;}
div.flushright div {text-align: left;}
div.flushleft {text-align: left;}
.underline{ text-decoration:underline; }
.underline img{ border-bottom: 1px solid black; margin-bottom:1pt; }
.framebox-c, .framebox-l, .framebox-r { padding-left:3.0pt; padding-right:3.0pt; text-indent:0pt; border:solid black 0.4pt; }
.framebox-c {text-align:center;}
.framebox-l {text-align:left;}
.framebox-r {text-align:right;}
span.thank-mark{ vertical-align: super }
span.footnote-mark sup.textsuperscript, span.footnote-mark a sup.textsuperscript{ font-size:80%; }
div.tabular, div.center div.tabular {text-align: center; margin-top:0.5em; margin-bottom:0.5em; }
table.tabular td p{margin-top:0em;}
table.tabular {margin-left: auto; margin-right: auto;}
div.td00{ margin-left:0pt; margin-right:0pt; }
div.td01{ margin-left:0pt; margin-right:5pt; }
div.td10{ margin-left:5pt; margin-right:0pt; }
div.td11{ margin-left:5pt; margin-right:5pt; }
table[rules] {border-left:solid black 0.4pt; border-right:solid black 0.4pt; }
td.td00{ padding-left:0pt; padding-right:0pt; }
td.td01{ padding-left:0pt; padding-right:5pt; }
td.td10{ padding-left:5pt; padding-right:0pt; }
td.td11{ padding-left:5pt; padding-right:5pt; }
table[rules] {border-left:solid black 0.4pt; border-right:solid black 0.4pt; }
.hline hr, .cline hr{ height : 1px; margin:0px; }
.tabbing-right {text-align:right;}
span.TEX {letter-spacing: -0.125em; }
span.TEX span.E{ position:relative;top:0.5ex;left:-0.0417em;}
a span.TEX span.E {text-decoration: none; }
span.LATEX span.A{ position:relative; top:-0.5ex; left:-0.4em; font-size:85%;}
span.LATEX span.TEX{ position:relative; left: -0.4em; }
div.float img, div.float .caption {text-align:center;}
div.figure img, div.figure .caption {text-align:center;}
.marginpar {width:20%; float:right; text-align:left; margin-left:auto; margin-top:0.5em; font-size:85%; text-decoration:underline;}
.marginpar p{margin-top:0.4em; margin-bottom:0.4em;}
.equation td{text-align:center; vertical-align:middle; }
td.eq-no{ width:5%; }
table.equation { width:100%; } 
div.math-display, div.par-math-display{text-align:center;}
math .texttt { font-family: monospace; }
math .textit { font-style: italic; }
math .textsl { font-style: oblique; }
math .textsf { font-family: sans-serif; }
math .textbf { font-weight: bold; }
.partToc a, .partToc, .likepartToc a, .likepartToc {line-height: 200%; font-weight:bold; font-size:110%;}
.chapterToc a, .chapterToc, .likechapterToc a, .likechapterToc, .appendixToc a, .appendixToc {line-height: 200%; font-weight:bold;}
.index-item, .index-subitem, .index-subsubitem {display:block}
.caption td.id{font-weight: bold; white-space: nowrap; }
table.caption {text-align:center;}
h1.partHead{text-align: center}
p.bibitem { text-indent: -2em; margin-left: 2em; margin-top:0.6em; margin-bottom:0.6em; }
p.bibitem-p { text-indent: 0em; margin-left: 2em; margin-top:0.6em; margin-bottom:0.6em; }
.subsectionHead, .likesubsectionHead { margin-top:2em; font-weight: bold;}
.sectionHead, .likesectionHead { font-weight: bold;}
.quote {margin-bottom:0.25em; margin-top:0.25em; margin-left:1em; margin-right:1em; text-align:justify;}
.verse{white-space:nowrap; margin-left:2em}
div.maketitle {text-align:center;}
h2.titleHead{text-align:center;}
div.maketitle{ margin-bottom: 2em; }
div.author, div.date {text-align:center;}
div.thanks{text-align:left; margin-left:10%; font-size:85%; font-style:italic; }
div.author{white-space: nowrap;}
.quotation {margin-bottom:0.25em; margin-top:0.25em; margin-left:1em; }
h1.partHead{text-align: center}
.sectionToc, .likesectionToc {margin-left:2em;}
.subsectionToc, .likesubsectionToc {margin-left:4em;}
.sectionToc, .likesectionToc {margin-left:6em;}
.frenchb-nbsp{font-size:75%;}
.frenchb-thinspace{font-size:75%;}
.figure img.graphics {margin-left:10%;}
/* end css.sty */

\title{Serie de Fourier d'une fonction}
\author{}
\date{}

\begin{document}
\maketitle

\textbf{Warning: 
requires JavaScript to process the mathematics on this page.\\ If your
browser supports JavaScript, be sure it is enabled.}

\begin{center}\rule{3in}{0.4pt}\end{center}

[
[
[]
[

\section{14.3 Série de Fourier d'une fonction}

\subsection{14.3.1 Les espaces C et D}

Définition~14.3.1 On considère l'espace vectoriel C des fonctions de \mathbb{R}~
dans \mathbb{C}, continues par morceaux et périodiques de période 2\pi~. On
désignera par D le sous-espace vectoriel des applications f : \mathbb{R}~ \rightarrow~ \mathbb{C},
continues par morceaux, périodiques de période 2\pi~ et vérifiant
\forall~~x \in \mathbb{R}~, f(x) =
f(x^+)+f(x^-) \over 2 (où
f(x^+) et f(x^-) désignent respectivement les
limites à gauche et à droite de f au point x). Pour f,g \inC, on posera
(f∣g) = 1 \over 2\pi~
\int ~
_0^2\pi~\overlinef(t)g(t) dt,
\f_2 =
\sqrt(f∣ f) et
e_n : t\mapsto~e^int.

Théorème~14.3.1 L'application
(f,g)\mapsto~(f\mathrel∣g) est
une forme hermitienne positive sur C dont la restriction à D est définie
positive. La famille (e_n)_n\in\mathbb{Z} est une famille
orthonormée de C. Pour toute f \inC, on a
\f_2
\leq\ f_\infty~
(norme de la convergence uniforme).

Démonstration Le caractère sesquilinéaire et la symétrie hermitienne
sont évidents. Si f \inC, on a (f∣f) = 1
\over 2\pi~ \int ~
_0^2\pi~f(t)^2 dt ≥ 0. La
nullité de (f∣f) nécessite que f soit nulle
en tout point de [0,2\pi~] où elle est continue, soit sur [0,2\pi~]
privé d'un nombre fini de points. Si f est dans D, alors en chacun de
ces points on a f(x^+) = f(x^-) = 0 (car il existe
tout un intervalle ouvert à gauche de x sur lequel f est nul, et de même
à droite) et donc f(x) = 0, par conséquent f est la fonction nulle sur
[0,2\pi~], donc sur \mathbb{R}~.

Remarque~14.3.1 On prendra garde que si f est seulement continue par
morceaux,
\f_2 = 0
n'implique pas f = 0.

\subsection{14.3.2 Coefficients de Fourier d'une fonction continue par
morceaux}

Définition~14.3.2 Soit f : \mathbb{R}~ \rightarrow~ \mathbb{C} continue par morceaux et périodique de
période 2\pi~. On définit les coefficients de Fourier de la fonction f par

\begin{align*} \forall~~n \in
\mathbb{Z},\quad c_n(f)& =&
(e_n∣f) = 1 \over
2\pi~ \int ~
_0^2\pi~f(t)e^-int dt\%&
\\ \forall~~n ≥
0,\quad a_n(f)& =& 1 \over
\pi~ \int ~
_0^2\pi~f(t)cos~ nt dt \%&
\\ \forall~~n ≥
1,\quad b_n(f)& =& 1 \over
\pi~ \int ~
_0^2\pi~f(t)sin~ nt dt \%&
\\ \end{align*}

Remarque~14.3.2 Les fonctions intégrées étant périodiques de période 2\pi~,
on a aussi pour tout a \in \mathbb{R}~, c_n(f) = 1 \over
2\pi~ \int ~
_a^a+2\pi~f(t)e^-int dt, a_n(f) = 1
\over \pi~ \int ~
_a^a+2\pi~f(t)cos~ nt dt,
b_n(f) = 1 \over \pi~
\int ~
_a^a+2\pi~f(t)sin~ nt dt et en
particulier c_n(f) = 1 \over 2\pi~
\int  _-\pi~^\pi~f(t)e^-int~
dt, a_n(f) = 1 \over \pi~
\int ~
_-\pi~^\pi~f(t)cos~ nt dt,
b_n(f) = 1 \over \pi~
\int ~
_-\pi~^\pi~f(t)sin~ nt dt

Proposition~14.3.2 On a les relations suivantes

\begin{align*} c_0(f)& =&
a_0(f) \over 2 \%&
\\ \forall~~n ≥
1,\quad c_n(f)& =& a_n(f) -
ib_n(f) \over 2 ,\quad
c_-n(f) = a_n(f) + ib_n(f)
\over 2 \%& \\
\forall~n ≥ 1,\quad a_n~(f)&
=& c_n(f) + c_-n(f),\quad b_n
= i(c_n(f) - c_-n(f)) \%&
\\ \end{align*}

Démonstration Elémentaire

Proposition~14.3.3 Soit f : \mathbb{R}~ \rightarrow~ \mathbb{C} continue par morceaux et périodique de
période 2\pi~. Si f est à valeurs réelles, on a a_n(f) \in \mathbb{R}~,
b_n(f) \in \mathbb{R}~ et c_-n(f) =
\overlinec_n(f). Si f est paire (resp.
impaire) on a b_n(f) = 0 (resp. a_n(f) = 0).

Démonstration Si f est à valeurs réelles, il en est de même de
x\mapsto~f(x)cos~ nx et de
x\mapsto~f(x)sin~ nx ce qui
montre que a_n(f) et b_n(f) sont réels~; de plus
f(x)e^inx = \overlinef(x)e^-inx
ce qui montre que c_-n(f) =
\overlinec_n(f). Si f est paire, on a
b_n(f) = 1 \over 2\pi~
\int ~
_-\pi~^\pi~f(x)sin~ nx dx = 0 puisque
la fonction f(x)sin~ nx est impaire. Le
raisonnement est similaire si f est impaire avec les a_n(f).

Définition~14.3.3 Soit f : \mathbb{R}~ \rightarrow~ \mathbb{C} continue par morceaux et périodique de
période 2\pi~. On appelle série de Fourier de la fonction f la série
trigonométrique

\begin{align*} c_0(f) +
\sum _n≥1(c_n(f)e^inx~
+ c_ -n(f)e^-inx)&& \%&
\\ & & = a_0(f)
\over 2 + \\sum
_n≥1(a_n(f)\cos nx +
b_n(f)\sin nx)\%&
\\ \end{align*}

Définition~14.3.4 Pour n ≥ 1, on posera (sommes partielles de la série
de Fourier)

\begin{align*} S_n(f)(x)& =&
c_0(f) + \\sum
_p=1^n(c_ p(f)e^ipx + c_
-p(f)e^-ipx) \%& \\ & =&
a_0(f) \over 2 + \\sum
_p=1^n(a_ p(f)\cos px +
b_p(f)\sin px)\%&
\\ \end{align*}

\subsection{14.3.3 Inégalité de Bessel et théorème de Riemann-Lebesgue}

Définition~14.3.5 Pour N ≥ 1, on posera T_N
=\
\mathrmVect(e_-N,e_-N+1,\\ldots,e_-1,e_0,e_1,\\\ldots,e_N-1,e_N~)
(espace vectoriel des polynômes trigonométriques de degré inférieur ou
égal à N.

Remarque~14.3.3 On a également

T_N = \x\mapsto~
a_0 \over 2 + \\sum
_p=1^N(a_ p \cos px +
b_p \sin px)\

Par définition même
(e_-N,e_-N+1,\\ldots,e_-1,e_0,e_1,\\\ldots,e_N-1,e_N~)
est une base orthonormée de T_N.

Lemme~14.3.4 Soit f : \mathbb{R}~ \rightarrow~ \mathbb{C} continue par morceaux et périodique de
période 2\pi~. Alors
\S_N(f)_2^2
= \\sum ~
_k=-N^Nc_k(f)^2.

Démonstration
c_-N(f),\\ldots,c_0(f),\\\ldots,c_N~(f)
sont les coordonnées de S_N(f) dans la base orthonormée
(e_-N,e_-N+1,\\ldots,e_-1,e_0,e_1,\\\ldots,e_N-1,e_N~)~;
la norme au carré de S_N(f) est donc la somme des carrés des
modules de ces coordonnées~; d'où le résultat.

Lemme~14.3.5 Soit f : \mathbb{R}~ \rightarrow~ \mathbb{C} continue par morceaux et périodique de
période 2\pi~. Alors S_N(f) est la projection orthogonale de f sur
le sous-espace vectoriel T_N.

Démonstration Puisque S_N(f) appartient à T_N, il
suffit de montrer que f - S_N(f) \bot T_N ou encore que
\forall~~n \in [-N,N],
(e_n∣f - S_N(f)) = 0, ou
encore que \forall~~n \in [-N,N],
(e_n∣f) =
(e_n∣S_N(f)). Mais
(e_n∣S_N(f)) est la
coordonnée suivant e_n de S_N(f) (puisque la base est
orthonormée), c'est donc c_n(f) =
(e_n∣f) par définition, ce qui
montre le résultat.

Théorème~14.3.6 (Bessel). Soit f : \mathbb{R}~ \rightarrow~ \mathbb{C} continue par morceaux et
périodique de période 2\pi~. Alors la série
c_0(f)^2
+ \\sum ~
_n≥1(c_n(f)^2 +
c_-n(f)^2) est convergente et on
a

c_0(f)^2 +
\sum _n=1^+\infty~(c_
n(f)^2 + c_
-n(f)^2) \leq\
f_ 2^2

Démonstration Puisque S_N(f) est la projection orthogonale de f
sur T_N, on a f = S_N(f) + (f - S_N(f)) avec
S_N(f) \bot f - S_N(f). Le théorème de Pythagore assure
que
\f_2^2
=\
S_N(f)_2^2
+\ f -
S_N(f)_2^2, d'où
encore d'après le lemme 1

c_0(f)^2 +
\sum _n=1^N(c_
n(f)^2 + c_
-n(f)^2) =\ S_
N(f)_2^2
\leq\ f_
2^2

La série à termes positifs
c_0(f)^2
+ \\sum ~
_n≥1(c_n(f)^2 +
c_-n(f)^2) a ses sommes
partielles majorées par
\f_2^2,
donc elle converge et sa somme est majorée par
\f_2^2,
ce qui achève la démonstration.

Remarque~14.3.4 Un calcul élémentaire montre que pour n ≥ 1,

c_n(f)^2 + c_
-n(f)^2 = 1 \over 2
(a_n(f)^2 + b_
n(f)^2)

ce qui montre que les séries
\\sum ~
a_n(f)^2 et
\\sum ~
b_n(f)^2 convergent et que (en
tenant compte de a_0(f) = c_0(f)
\over 2 )

 a_0(f)^2 \over
4 + 1 \over 2 \\sum
_n=1^+\infty~(a_
n(f)^2 + b_
n(f)^2) \leq\
f\^2

Théorème~14.3.7 (Riemann-Lebesgue). Soit f : \mathbb{R}~ \rightarrow~ \mathbb{C} continue par morceaux
et périodique de période 2\pi~. Alors

lim_n\rightarrow~±\infty~c_n~(f)
= lim_n\rightarrow~+\infty~a_n~(f)
= lim_n\rightarrow~+\infty~b_n~(f) = 0

Démonstration Puisque les séries
\\sum ~
_n≥1(c_n(f)^2 +
c_-n(f)^2),
\\sum ~
a_n(f)^2 et
\\sum ~
b_n(f)^2 sont convergentes,
leurs termes généraux admettent la limite 0, ce qui montre le résultat.

\subsection{14.3.4 Les théorèmes de Dirichlet}

Nous aurons besoin par la suite du lemme suivant

Lemme~14.3.8 Pour tout entier n ≥ 1 et pour
t∉2\pi~\mathbb{Z},
\\sum ~
_k=-n^ne^ikt = sin~
(2n+1) t \over 2 \over
sin  t \over 2 ~ .

Démonstration On a en effet

\begin{align*} \\sum
_k=-n^ne^ikt& =& e^-int
\sum _k=0^2ne^ikt~ =
e^-int e^(2n+1)it - 1 \over
e^it - 1 \%& \\ & =&
e^(n+1)it - e^-int \over
e^it - 1 = e^(n+ 1 \over 2
)it - e^-(n+ 1 \over 2 )it
\over e^i t \over 2  -
e^-i t \over 2  \%&
\\ \end{align*}

en multipliant numérateur et dénominateur par e^-it\diagup2. On en
déduit immédiatement la formule souhaitée.

Théorème~14.3.9 (Dirichlet). Soit f : \mathbb{R}~ \rightarrow~ \mathbb{C} de classe \mathcal{C}^1 par
morceaux et périodique de période 2\pi~. Alors la série de Fourier de f
converge sur \mathbb{R}~ et

\forall~x \in \mathbb{R}~, f(x^+~) +
f(x^-) \over 2 = c_0(f) +
\sum _n=1^+\infty~(c_
n(f)e^inx + c_ -n(f)e^-inx)

Démonstration On a

\begin{align*} S_n(f)(x)& =& 1
\over 2\pi~ \\sum
_k=-n^ne^inx
\\int  ~
_0^2\pi~f(t)e^-int dt\%&
\\ & =& 1 \over 2\pi~
\int ~
_0^2\pi~f(t)\left (\\sum
_k=-n^ne^in(x-t)\right )
dt\%& \\ & =& 1 \over
2\pi~ \int  _0^2\pi~~f(t)
sin (2n + 1) x-t \over 2~
\over sin~  x-t
\over 2  dt \%& \\
\end{align*}

Faisons le changement de variable t = x + u, on obtient

\begin{align*} S_n(f)(x)& =& 1
\over 2\pi~ \int ~
_-x^2\pi~-xf(x + u) sin~ (2n +
1) u \over 2 \over
sin  u \over 2 ~ du\%&
\\ & =& 1 \over 2\pi~
\int  _-\pi~^\pi~~f(x + u)
sin (2n + 1) u \over 2~
\over sin~  u
\over 2  du \%& \\
\end{align*}

puisque la fonction intégrée est périodique de période 2\pi~ et que donc
son intégrale sur tout intervalle de longueur 2\pi~ est la même. Coupons
l'intégrale en deux, l'une de - \pi~ à 0, l'autre de 0 à \pi~. Dans la
première faisons le changement de variable u = -2v et dans la seconde le
changement de variable u = 2v. On obtient

\begin{align*} S_n(f)(x)& =& 1
\over \pi~ \int ~
_0^\pi~\diagup2f(x - 2v) sin~ (2n + 1)v
\over sin v~ dv \%&
\\ & \text & + 1
\over \pi~ \int ~
_0^\pi~\diagup2f(x + 2v) sin~ (2n + 1)v
\over sin v~ dv \%&
\\ & =& 1 \over \pi~
\int  _0^\pi~\diagup2~(f(x + 2v) + f(x -
2v)) sin~ (2n + 1)v \over
sin v~ dv\%&
\\ \end{align*}

Appliquons le résultat précédent à la fonction constante f_0 :
x\mapsto~1. On a bien entendu
S_n(f_0)(x) = 1 puisque c_0(f_0) = 1
et c_n(f_0) = 0 pour n\neq~0~;
on obtient

1 = 2 \over \pi~ \int ~
_0^\pi~\diagup2 sin~ (2n + 1)v
\over sin v~ dv

On en déduit que

\begin{align*} S_n(f)(x) -
f(x^+) + f(x^-) \over 2 =&&
\%& \\ & & 1 \over \pi~
\int  _0^\pi~\diagup2~ f(x + 2v) -
f(x^+) + f(x - 2v) - f(x_ -) \over
sin v \sin~ (2n +
1)v dv\%& \\
\end{align*}

Considérons la fonction g périodique de période 2\pi~ définie par

\begin{align*} g(v)& =& f(x + 2v) -
f(x^+) + f(x - 2v) - f(x_-) \over
sin v \text pour ~v
\in]0, \pi~ \over 2 ]\%&
\\ g(0)& =& 2(f'(x^+) -
f'(x^-)) \%& \\ g(v)& =&
0\text pour v \in] \pi~ \over 2
,2\pi~[ \%& \\
\end{align*}

Comme la fonction \tildef définie par
\tildef(x) = f(x^+) et
\tildef(t) = f(t) pour t > x est
dérivable à droite au point x (puisque f est de classe \mathcal{C}^1
par morceaux), on a, quand v tend vers 0 par valeurs supérieures,

\begin{align*} f(x + 2v) - f(x^+))
\over sin v~ &
∼_v\rightarrow~0,v>0& f(x + 2v) - f(x^+)
\over v \%& \\ & = &
2 \tildef(x + 2v) -\tilde f(x)
\over 2v \%& \\
\end{align*}

de limite 2f'(x^+). De même on a

lim_v\rightarrow~0,v>0~ f(x - 2v)
- f(x_-) \over sin v~
= -2f'(x^-)

ce qui montre que g est continue à droite au point 0. On en déduit
immédiatement que g est continue par morceaux. Mais alors

\begin{align*} S_n(f)(x) -
f(x^+) + f(x^-) \over 2 && \%&
\\ & =& 1 \over \pi~
\int ~
_0^2\pi~g(v)sin~ (2n + 1)v dv =
b_ 2n+1(g)\%& \\
\end{align*}

D'après le théorème de Riemann-Lebesgue, cette expression tend vers 0
quand n tend vers + \infty~, ce qui montre à la fois la convergence de la
série et donne la valeur de sa somme.

Lemme~14.3.10 Soit f : \mathbb{R}~ \rightarrow~ \mathbb{C} périodique de période 2\pi~de classe
\mathcal{C}^1 par morceaux et continue. Alors
\forall~n \in \mathbb{Z}, c_n(f') = inc_n~(f)
(où f' désigne la fonction de D égale à la dérivée de f sauf en un
nombre fini de points modulo 2\pi~).

Démonstration Soit \sigma = (a_i)_0\leqi\leqp une subdivision de
[0,2\pi~] adaptée à f. En tout point de [0,2\pi~]
\diagdown\a_0,\\ldots,a_p\~,
f'(t) est la dérivée de f et on pose f'(a_i) = 1
\over 2 (f'(a_i^+) + f'(a_
i^-)), si bien que f' \inD. Une intégration par parties donne, si
[a,b] \subset~]a_i-1,a_i[,

\begin{align*} \int ~
_a^bf'(t)e^-int dt& =& \left
[f(t)e^-int\right ]_ a^b +
in\int  _a^bf(t)e^-int~
dt \%& \\ & =& f(b)e^-inb -
f(a)e^-ina + in\int ~
_a^bf(t)e^-int dt\%&
\\ \end{align*}

En faisant tendre a vers a_i-1 et b vers a_i, en
tenant compte de la continuité de f aux points a_i-1 et
a_i on obtient

\begin{align*} \int ~
_a_i-1^a_i f'(t)e^-int dt&
=& f(a_ i)e^-ina_i  -
f(a_i-1)e^-ina_i-1 \%&
\\ & \text &
+in\int ~
_a_i-1^a_i f(t)e^-int dt \%&
\\ \end{align*}

et en sommant

\begin{align*} \int ~
_0^2\pi~f'(t)e^-int dt&& \%&
\\ & =& \\sum
_i=1^p
\\int  ~
_a_i-1^a_i f'(t)e^-int dt
\%& \\ & =& \\sum
_i=1^p\left (f(a_
i)e^-ina_i  -
f(a_i-1)e^-ina_i-1 \right )
+ in\\int  ~
_a_i-1^a_i f(t)e^-int dt\%&
\\ & =&
f(a_p)e^-ina_p  -
f(a_0)e^-ina_0  +
in\int  _a_0^a_p~
f(t)e^-int dt \%& \\ & =&
in\int ~
_0^2\pi~f(t)e^-int dt \%&
\\ \end{align*}

puisque a_0 = 0, a_p = 2\pi~,
f(a_p)e^-ina_p = f(2\pi~)e^-in2\pi~ =
f(2\pi~) = f(0) = f(a_0)e^-ina_0. En divisant
par 2\pi~, on obtient c_n(f') = inc_n(f).

Théorème~14.3.11 (Dirichlet). Soit f : \mathbb{R}~ \rightarrow~ \mathbb{C} périodique de période 2\pi~ de
classe \mathcal{C}^1 par morceaux et continue. Alors la série
c_0(f) +\
\sum ~
_n≥1(c_n(f) +
c_-n(f)) converge, la série de Fourier de f
converge normalement sur \mathbb{R}~ et on a

\forall~x \in \mathbb{R}~, f(x) = c_0~(f) +
\sum _n=1^+\infty~(c_
n(f)e^inx + c_ -n(f)e^-inx)

(autrement dit f est somme de sa série de Fourier).

Démonstration Pour a et b réels on a ab \leq 1 \over 2
(a^2 + b^2)~; on en déduit que si
n\neq~0, on a 0
\leqc_n(f) = \left 
c_n(f') \over in \right
\leq 1 \over 2
(c_n(f)^2 + 1
\over n^2 ). D'après le théorème de Bessel,
la série \\sum ~
_n≥0c_n(f')^2 converge et
d'après la théorie des séries de Riemann la série
\\sum ~  1
\over n^2 converge. On en déduit que la
série \\sum ~
_n≥1c_n(f) converge. On montre de la
même fa\ccon que la série
\\sum ~
_n≥1c_-n(f) converge, d'où la
convergence de la série
\\sum ~
_n≥1(c_n(f) +
c_-n(f)). La convergence normale de la
série de Fourier en résulte immédiatement puisque

\forall~~x \in \mathbb{R}~,
c_n(f)e^inx + c_
-n(f)e^-inx\leqc_ n(f)
+ c_-n(f)

qui est une série convergente indépendante de x. La formule résulte du
premier théorème de Dirichlet en remarquant que si f est continue, f(x)
= f(x^+)+f(x^-) \over 2 .

\subsection{14.3.5 Coefficients de Fourier des fonctions de classe
C^k}

Théorème~14.3.12 Soit f : \mathbb{R}~ \rightarrow~ \mathbb{C} périodique de période 2\pi~ de classe
C^k. Alors

\forall~n \in \mathbb{Z}, c_n~(f) =
(in)^kc_ n(f^(k))

et, quand n tend vers + \infty~, c_n(f) = o( 1
\over n^k ).

Démonstration On a vu que c_n(f') = inc_n(f) et il
suffit de faire une récurrence évidente sur k pour obtenir
c_n(f) = (in)^kc_n(f^(k)). Comme
le théorème de Riemann-Lebesgue assure que
lim_n\rightarrow~+\infty~c_n(f^(k)~)
= 0, on a c_n(f) = o( 1 \over n^k
).

Remarque~14.3.5 Autrement dit, plus la fonction est régulière, plus vite
les coefficients de Fourier tendent vers 0 à l'infini. Si f est de
classe C^\infty~, on a pour tout k \in \mathbb{N}~,
lim_n\rightarrow~+\infty~n^kc_n~(f)
= 0 (typiquement les coefficients de Fourier seront à décroissance
exponentielle).

\subsection{14.3.6 Le théorème de Parseval}

Lemme~14.3.13 Soit f : \mathbb{R}~ \rightarrow~ \mathbb{C} périodique de période 2\pi~ et continue par
morceaux. Alors, pour tout \epsilon > 0, il existe g : \mathbb{R}~ \rightarrow~ \mathbb{C}
périodique de période 2\pi~, de classe \mathcal{C}^1 par morceaux et
continue telle que \f -
g_2 < \epsilon.

Démonstration Supposons tout d'abord que f est en escalier et soit 0 =
a_0 < a_1 <
\\ldots~ <
a_p = 2\pi~ une subdivision de [0,2\pi~] adaptée à f avec f(t) =
\lambda_i pour t \in]a_i-1,a_i[. Soit \delta le pas de
la subdivision. Pour  2 \over n < \eta
définissons une fonction g_n par

\begin{itemize}
\itemsep1pt\parskip0pt\parsep0pt
\item
  (i) \forall~~i \in [0,p],
  g_n(a_i) = 0
\item
  (ii) \forall~~i \in [1,p],
  \forall~t \in [a_i-1~ + 1
  \over n ,a_i - 1 \over n
  ], g_n(t) = \lambda_i
\item
  (iii) g_n est affine sur chacun des intervalles
  [a_i-1,a_i-1 + 1 \over n ] et
  [a_i - 1 \over n ,a_i].
\end{itemize}

Il est clair que g_n est continue, affine par morceaux. Comme
de plus g_n(0) = g_n(2\pi~) = 0 elle se prolonge en une
application continue et périodique de période 2\pi~ sur \mathbb{R}~. Puisque
g_n est affine par morceaux, elle est a fortiori de classe
\mathcal{C}^1 par morceaux. On a

\begin{align*} \int ~
_a_i-1^a_i f(t) -
g_n(t)^2 dt& =&
\int ~
_a_i-1^a_i-1+ 1 \over n
f(t) - g_n(t)^2 dt + \%&
\\ & \text &
\int  _a_i~- 1
\over n ^a_i f(t) -
g_n(t)^2 dt \%&
\\ \end{align*}

Mais on a g(t) = n\lambda_i(t - a_i) pour t \in
[a_i-1,a_i-1 + 1 \over n ] et
g(t) = -n\lambda_i(t - a_i) pour t \in [a_i - 1
\over n ,a_i]. On a donc

\begin{align*} \int ~
_a_i-1^a_i f(t) -
g_n(t)^2 dt&& \%&
\\ & =&
\lambda_i^2\left
(\int ~
_a_i-1^a_i-1+ 1 \over n
(1 - n(t - a_i-1))^2 dt\right .
\%& \\ & \text &
\quad \quad + \left
.\int  _a_i~- 1
\over n ^a_i (1 + n(t -
a_i))^2 dt\right ) \%&
\\ & =&
\lambda_i^2\left
(\int  _0~^ 1 \over
n (1 - nu)^2 dt +\int  _-
1 \over n ^0(1 + nu)^2
dt\right )\%& \\ & =&
\lambda_i^2 \over 3n
\left (\left [-(1 -
nu)^3\right ]_ 0^ 1
\over n  + \left [(1 +
nu)^3\right ]_- 1 \over
n ^0\right ) \%&
\\ & =&
2\lambda_i^2 \over 3n
\%& \\ \end{align*}

soit encore

2\pi~\f -
g_n_2^2
=\int  _0^2\pi~~f(t) -
g_ n(t)^2 dt = 2 \over
3n \\sum
_i=1^p\lambda_ i^2

On en déduit que
lim_n\rightarrow~+\infty~~\f -
g_n_2 = 0 et que donc on
peut trouver un n tel que  2 \over n < \eta
avec \f -
g_n_2 < \epsilon.

Supposons maintenant que f est continue par morceaux. Sa restriction à
[0,2\pi~] est réglée et donc on peut trouver \phi en escalier sur
[0,2\pi~[ (et que l'on prolonge par périodicité) telle que
\f - \phi_\infty~
< \epsilon \over 2 . On a alors

\begin{align*} \f -
\phi_2^2& =& 1
\over 2\pi~ \int ~
_0^2\pi~f(t) - \phi(t)^2 dt \leq 1
\over 2\pi~ \int ~
_0^2\pi~\f -
\phi_ \infty~^2 dt\%&
\\ & =& \f -
\phi_\infty~^2 \%&
\\ \end{align*}

soit encore \f -
\phi_2 \leq\ f -
\phi_\infty~ < \epsilon
\over 2 . Mais d'autre part, comme \phi est en escalier,
on sait qu'on peut trouver g continue et affine par morceaux telle que
\\phi - g_2
< \epsilon \over 2 . On a alors
\f - g_2
\leq\ f - \phi_2
+\ \phi - g_2
< \epsilon ce qui démontre le lemme.

Théorème~14.3.14 (Parseval-Plancherel). Soit f : \mathbb{R}~ \rightarrow~ \mathbb{C} périodique de
période 2\pi~ et continue par morceaux. Alors

\begin{align*}
\f_2^2&
=& 1 \over 2\pi~ \int ~
_0^2\pi~f(t)^2 dt \%&
\\ & =&
c_0(f)^2 +
\sum _n=1^+\infty~(c_
n(f)^2 + c_
-n(f)^2) \%& \\ &
=& a_0(f)^2
\over 4 + 1 \over 2
\sum _n=1^+\infty~(a_
n(f)^2 + b_
n(f)^2)\%& \\
\end{align*}

Démonstration On sait que
c_0(f)^2
+ \\sum ~
_n=1^N(c_n(f)^2 +
c_-n(f)^2)
=\
S_N(f)_2^2. D'autre
part, à l'aide du théorème de Pythagore et puisque S_N(f) est
la projection orthogonale de f sur le sous-espace T_N des
polynômes trigonométriques de degré au plus N, on a
\f_2^2
=\
S_N(f)_2^2
+\ f -
S_N(f)_2^2. Le
résultat à démontrer est donc équivalent à
lim_N\rightarrow~+\infty~\S_N(f)_2^2~
=\
f_2^2, soit encore à
lim_N\rightarrow~+\infty~~\f -
S_N(f)_2 = 0.

Supposons tout d'abord que f est \mathcal{C}^1 par morceaux et
continue. On sait que la série de Fourier de f converge normalement,
donc uniformément vers f. On a donc
lim_N\rightarrow~+\infty~~\f -
S_N(f)_\infty~ = 0, mais comme ci
dessus, on a \f -
S_N(f)_2
\leq\ f -
S_N(f)_\infty~ ce qui montre que
lim_N\rightarrow~+\infty~~\f -
S_N(f)_2 = 0.

Si maintenant f est seulement continue par morceaux, soit \epsilon
> 0 et g : \mathbb{R}~ \rightarrow~ \mathbb{C} périodique de période 2\pi~, de classe
\mathcal{C}^1 par morceaux et continue telle que
\f - g_2
< \epsilon \over 2 . D'après le premier cas, on a
lim_N\rightarrow~+\infty~~\g -
S_N(g)_2 = 0 et donc il
existe N_0 \in \mathbb{N}~ tel que N ≥ N_0
\rigtharrow~\ g -
S_N(g)_2 < \epsilon
\over 2 . Mais comme S_N(g) \in T_N et
que S_N(f) est la projection orthogonale de f sur T_N,
on a \f -
S_N(f)_2
\leq\ f -
S_N(g)_2 soit encore, pour N
≥ N_0,

\begin{align*} \f -
S_N(f)_2& \leq&
\f -
S_N(g)_2
\leq\ f - g_2
+\ g -
S_N(g)_2\%&
\\ & <& \epsilon
\over 2 + \epsilon \over 2 = \epsilon \%&
\\ \end{align*}

ce qui démontre le résultat. La deuxième formule résulte d'un calcul
précédent qui montre que

\begin{align*}
c_0(f)^2& +&
\sum _n=1^+\infty~(c_
n(f)^2 + c_
-n(f)^2) \%& \\ &
=& a_0(f)^2
\over 4 + 1 \over 2
\sum _n=1^+\infty~(a_
n(f)^2 + b_
n(f)^2)\%& \\
\end{align*}

Corollaire~14.3.15 (injectivité de la transformation de Fourier). Soit f
et g deux fonctions de D telles que \forall~~n \in \mathbb{N}~,
c_n(f) = c_n(g). Alors f = g.

Démonstration On a \forall~n \in \mathbb{N}~, c_n~(f - g)
= 0, soit encore d'après le théorème de Parseval,
\f - g_2 =
0. Comme f - g appartient à D sur laquelle le produit scalaire est
défini positif, on a f - g = 0.

Remarque~14.3.6 Si on suppose seulement que f et g sont continues par
morceaux, on obtient seulement que f et g coïncident sauf en un nombre
fini de points (sur un intervalle de longueur 2\pi~).

[
[
[
[

\end{document}

% \subsubsection{14.4 Fonctions périodiques de période T}

Remarque~14.4.1 Remarquons que si f est périodique de période T, alors
\tildef définie par $\tildef(t) =
f( T \over 2\pi~ t)$ est périodique de période $2\pi~$ et l'on
$a f(x) =\tilde f( 2\pi~ \over T x)$ .
Ceci permet d'adapter tous les résultats précédents aux fonctions de
période T.

On pose

\begin{align*} (f∣g)&
=& 1 \over T \int ~
_0^T\overlinef(t)g(t) dt = 1
\over T \int ~
_a^a+T\overlinef(t)g(t) dt\%&
\\
\f_2^2&
=& (f∣f) = 1 \over T
\int ~
_0^Tf(t)^2 dt \%&
\\ e_n(t)& =&
e^2i\pi~nt\diagupT \%& \\
\end{align*}

Alors $(e_n)_n \in \mathbb{Z}$ est une famille orthonormée de C. On
définit les coefficients de Fourier de $f \in C $par

\begin{align*} \forall~~n \in
\mathbb{Z},\quad c_n(f)& =&
(e_n∣f) = 1 \over
T \int ~
_0^Tf(t)e^-2\pi~int\diagupT dt\%&
\\ \forall~~n ≥
0,\quad a_n(f)& =& 2 \over
T \int ~
_0^Tf(t)cos~  2\pi~nt
\over T dt \%& \\
\forall~n ≥ 1,\quad b_n~(f)&
=& 2 \over T \int ~
_0^Tf(t)sin~  2\pi~nt
\over T dt \%& \\
\end{align*}

la série de Fourier de f par

\begin{align*} c_0(f)& +&
\\sum
_n≥1(c_n(f)e^2\pi~inx\diagupT + c_
-n(f)e^-2\pi~inx\diagupT) \%& \\ &
=& a_0(f) \over 2 +
\\sum
_n≥1(a_n(f)\cos  2\pi~nx
\over T + b_n(f)\sin 2\pi~nx
\over T )\%& \\
\end{align*}

et on a les théorèmes

\begin{thm}[Bessel]
 Soit $f : \mathbb{R}~ \rightarrow~ \mathbb{C}$ continue par morceaux et
périodique de période T. Alors la série
\[
c_0(f)^2
+ \\sum ~
_n≥1(c_n(f)^2 +
c_-n(f)^2)
\]
 est convergente et on
a
\[
c_0(f)^2 +
\sum _n=1^+\infty~(c_
n(f)^2 + c_
-n(f)^2) \leq\
f_ 2^2
\]
\end{thm}
Théorème~14.4.2 (Dirichlet). Soit $f :  \mathbb{R}~ \rightarrow~ \mathbb{C} $ de
classe $\mathcal{C}^1 $ par
morceaux et périodique de période T. Alors la série de Fourier de f
converge sur $\mathbb{R}~ et \forall~~x \in \mathbb{R}$~,

\begin{align*} f(x^+) +
f(x^-) \over 2 && \%&
\\ & =& c_0(f) +
\sum _n=1^+\infty~(c_
n(f)e^2\pi~inx\diagupT + c_ -n(f)e^-2\pi~inx\diagupT)
\%& \\ & =& a_0(f)
\over 2 + \\sum
_n=1^+\infty~(a_ n(f)\cos  2\pi~nx
\over T + b_n(f)\sin  2\pi~nx
\over T )\%& \\
\end{align*}

\begin{thm}
  (Dirichlet). Soit $f : \mathbb{R}~ \rightarrow~ \mathbb{C}$ périodique de période Tde
classe $\mathcal{C}^1$ par morceaux et continue. Alors la série
\[
c_0(f) +\
\sum ~
_n≥1(c_n(f) +
c_-n(f))
\] converge, la série de Fourier de f
converge normalement sur $\mathbb{R}$ et on a $\forall~~x \in \mathbb{R}$,

\begin{align*} f(x)& =& c_0(f) +
\sum _n=1^+\infty~(c_
n(f)e^2\pi~inx\diagupT + c_ -n(f)e^-2\pi~inx\diagupT)
\%& \\ & =& a_0(f)
\over 2 + \\sum
_n=1^+\infty~(a_ n(f)\cos  2\pi~nx
\over T + b_n(f)\sin  2\pi~nx
\over T )\%& \\
\end{align*}

\end{thm}
(autrement dit f est somme de sa série de Fourier).

\begin{thm}
  Soit f : \mathbb{R}~ \rightarrow~ \mathbb{C} périodique de période T de classe
C^k. Alors
\[
\forall~n \in \mathbb{Z}, c_n~(f) =
\left ( 2\pi~in \over T
\right )^kc_ n(f^(k))
\]
et, quand n tend vers $+ \infty~$, $c_n(f) = o( 1
\over n^k )$.
\end{thm}

\begin{thm}[Parseval-Plancherel]
 . Soit $f : \mathbb{R}~ \rightarrow~ \mathbb{C} $ périodique de
période T et continue par morceaux. Alors
\begin{align*}
\\f_2^2& =& 1 \over T \int ~
_0^Tf(t)^2 dt \%&
\\ & =&
c_0(f)^2 +
\sum _n=1^+\infty~(c_
n(f)^2 + c_
-n(f)^2) \%& \\ &
=& a_0(f)^2
\over 4 + 1 \over 2
\sum _n=1^+\infty~(a_
n(f)^2 + b_
n(f)^2)\%& \\
\end{align*}

\end{thm}

% \documentclass[]{article}
\usepackage[T1]{fontenc}
\usepackage{lmodern}
\usepackage{amssymb,amsmath}
\usepackage{ifxetex,ifluatex}
\usepackage{fixltx2e} % provides \textsubscript
% use upquote if available, for straight quotes in verbatim environments
\IfFileExists{upquote.sty}{\usepackage{upquote}}{}
\ifnum 0\ifxetex 1\fi\ifluatex 1\fi=0 % if pdftex
  \usepackage[utf8]{inputenc}
\else % if luatex or xelatex
  \ifxetex
    \usepackage{mathspec}
    \usepackage{xltxtra,xunicode}
  \else
    \usepackage{fontspec}
  \fi
  \defaultfontfeatures{Mapping=tex-text,Scale=MatchLowercase}
  \newcommand{\euro}{€}
\fi
% use microtype if available
\IfFileExists{microtype.sty}{\usepackage{microtype}}{}
\ifxetex
  \usepackage[setpagesize=false, % page size defined by xetex
              unicode=false, % unicode breaks when used with xetex
              xetex]{hyperref}
\else
  \usepackage[unicode=true]{hyperref}
\fi
\hypersetup{breaklinks=true,
            bookmarks=true,
            pdfauthor={},
            pdftitle={Produit de convolution},
            colorlinks=true,
            citecolor=blue,
            urlcolor=blue,
            linkcolor=magenta,
            pdfborder={0 0 0}}
\urlstyle{same}  % don't use monospace font for urls
\setlength{\parindent}{0pt}
\setlength{\parskip}{6pt plus 2pt minus 1pt}
\setlength{\emergencystretch}{3em}  % prevent overfull lines
\setcounter{secnumdepth}{0}
 
/* start css.sty */
.cmr-5{font-size:50%;}
.cmr-7{font-size:70%;}
.cmmi-5{font-size:50%;font-style: italic;}
.cmmi-7{font-size:70%;font-style: italic;}
.cmmi-10{font-style: italic;}
.cmsy-5{font-size:50%;}
.cmsy-7{font-size:70%;}
.cmex-7{font-size:70%;}
.cmex-7x-x-71{font-size:49%;}
.msbm-7{font-size:70%;}
.cmtt-10{font-family: monospace;}
.cmti-10{ font-style: italic;}
.cmbx-10{ font-weight: bold;}
.cmr-17x-x-120{font-size:204%;}
.cmsl-10{font-style: oblique;}
.cmti-7x-x-71{font-size:49%; font-style: italic;}
.cmbxti-10{ font-weight: bold; font-style: italic;}
p.noindent { text-indent: 0em }
td p.noindent { text-indent: 0em; margin-top:0em; }
p.nopar { text-indent: 0em; }
p.indent{ text-indent: 1.5em }
@media print {div.crosslinks {visibility:hidden;}}
a img { border-top: 0; border-left: 0; border-right: 0; }
center { margin-top:1em; margin-bottom:1em; }
td center { margin-top:0em; margin-bottom:0em; }
.Canvas { position:relative; }
li p.indent { text-indent: 0em }
.enumerate1 {list-style-type:decimal;}
.enumerate2 {list-style-type:lower-alpha;}
.enumerate3 {list-style-type:lower-roman;}
.enumerate4 {list-style-type:upper-alpha;}
div.newtheorem { margin-bottom: 2em; margin-top: 2em;}
.obeylines-h,.obeylines-v {white-space: nowrap; }
div.obeylines-v p { margin-top:0; margin-bottom:0; }
.overline{ text-decoration:overline; }
.overline img{ border-top: 1px solid black; }
td.displaylines {text-align:center; white-space:nowrap;}
.centerline {text-align:center;}
.rightline {text-align:right;}
div.verbatim {font-family: monospace; white-space: nowrap; text-align:left; clear:both; }
.fbox {padding-left:3.0pt; padding-right:3.0pt; text-indent:0pt; border:solid black 0.4pt; }
div.fbox {display:table}
div.center div.fbox {text-align:center; clear:both; padding-left:3.0pt; padding-right:3.0pt; text-indent:0pt; border:solid black 0.4pt; }
div.minipage{width:100%;}
div.center, div.center div.center {text-align: center; margin-left:1em; margin-right:1em;}
div.center div {text-align: left;}
div.flushright, div.flushright div.flushright {text-align: right;}
div.flushright div {text-align: left;}
div.flushleft {text-align: left;}
.underline{ text-decoration:underline; }
.underline img{ border-bottom: 1px solid black; margin-bottom:1pt; }
.framebox-c, .framebox-l, .framebox-r { padding-left:3.0pt; padding-right:3.0pt; text-indent:0pt; border:solid black 0.4pt; }
.framebox-c {text-align:center;}
.framebox-l {text-align:left;}
.framebox-r {text-align:right;}
span.thank-mark{ vertical-align: super }
span.footnote-mark sup.textsuperscript, span.footnote-mark a sup.textsuperscript{ font-size:80%; }
div.tabular, div.center div.tabular {text-align: center; margin-top:0.5em; margin-bottom:0.5em; }
table.tabular td p{margin-top:0em;}
table.tabular {margin-left: auto; margin-right: auto;}
div.td00{ margin-left:0pt; margin-right:0pt; }
div.td01{ margin-left:0pt; margin-right:5pt; }
div.td10{ margin-left:5pt; margin-right:0pt; }
div.td11{ margin-left:5pt; margin-right:5pt; }
table[rules] {border-left:solid black 0.4pt; border-right:solid black 0.4pt; }
td.td00{ padding-left:0pt; padding-right:0pt; }
td.td01{ padding-left:0pt; padding-right:5pt; }
td.td10{ padding-left:5pt; padding-right:0pt; }
td.td11{ padding-left:5pt; padding-right:5pt; }
table[rules] {border-left:solid black 0.4pt; border-right:solid black 0.4pt; }
.hline hr, .cline hr{ height : 1px; margin:0px; }
.tabbing-right {text-align:right;}
span.TEX {letter-spacing: -0.125em; }
span.TEX span.E{ position:relative;top:0.5ex;left:-0.0417em;}
a span.TEX span.E {text-decoration: none; }
span.LATEX span.A{ position:relative; top:-0.5ex; left:-0.4em; font-size:85%;}
span.LATEX span.TEX{ position:relative; left: -0.4em; }
div.float img, div.float .caption {text-align:center;}
div.figure img, div.figure .caption {text-align:center;}
.marginpar {width:20%; float:right; text-align:left; margin-left:auto; margin-top:0.5em; font-size:85%; text-decoration:underline;}
.marginpar p{margin-top:0.4em; margin-bottom:0.4em;}
.equation td{text-align:center; vertical-align:middle; }
td.eq-no{ width:5%; }
table.equation { width:100%; } 
div.math-display, div.par-math-display{text-align:center;}
math .texttt { font-family: monospace; }
math .textit { font-style: italic; }
math .textsl { font-style: oblique; }
math .textsf { font-family: sans-serif; }
math .textbf { font-weight: bold; }
.partToc a, .partToc, .likepartToc a, .likepartToc {line-height: 200%; font-weight:bold; font-size:110%;}
.chapterToc a, .chapterToc, .likechapterToc a, .likechapterToc, .appendixToc a, .appendixToc {line-height: 200%; font-weight:bold;}
.index-item, .index-subitem, .index-subsubitem {display:block}
.caption td.id{font-weight: bold; white-space: nowrap; }
table.caption {text-align:center;}
h1.partHead{text-align: center}
p.bibitem { text-indent: -2em; margin-left: 2em; margin-top:0.6em; margin-bottom:0.6em; }
p.bibitem-p { text-indent: 0em; margin-left: 2em; margin-top:0.6em; margin-bottom:0.6em; }
.paragraphHead, .likeparagraphHead { margin-top:2em; font-weight: bold;}
.subparagraphHead, .likesubparagraphHead { font-weight: bold;}
.quote {margin-bottom:0.25em; margin-top:0.25em; margin-left:1em; margin-right:1em; text-align:justify;}
.verse{white-space:nowrap; margin-left:2em}
div.maketitle {text-align:center;}
h2.titleHead{text-align:center;}
div.maketitle{ margin-bottom: 2em; }
div.author, div.date {text-align:center;}
div.thanks{text-align:left; margin-left:10%; font-size:85%; font-style:italic; }
div.author{white-space: nowrap;}
.quotation {margin-bottom:0.25em; margin-top:0.25em; margin-left:1em; }
h1.partHead{text-align: center}
.sectionToc, .likesectionToc {margin-left:2em;}
.subsectionToc, .likesubsectionToc {margin-left:4em;}
.subsubsectionToc, .likesubsubsectionToc {margin-left:6em;}
.frenchb-nbsp{font-size:75%;}
.frenchb-thinspace{font-size:75%;}
.figure img.graphics {margin-left:10%;}
/* end css.sty */

\title{Produit de convolution}
\author{}
\date{}

\begin{document}
\maketitle

\textbf{Warning: 
requires JavaScript to process the mathematics on this page.\\ If your
browser supports JavaScript, be sure it is enabled.}

\begin{center}\rule{3in}{0.4pt}\end{center}

[
[
[]
[

\subsubsection{14.5 Produit de convolution}

\paragraph{14.5.1 Convolution de fonctions périodiques}

Définition~14.5.1 Soit f,g : \mathbb{R}~ \rightarrow~ \mathbb{C} continues par morceaux et périodiques
de période 2\pi~. On définit le produit de convolution de f et g comme la
fonction f ∗ g : \mathbb{R}~ \rightarrow~ \mathbb{C} définie par

\forall~~x \in \mathbb{R}~, f ∗ g(x) = 1 \over
2\pi~ \int  _0^2\pi~~f(t)g(x - t) dt

Théorème~14.5.1

\begin{itemize}
\itemsep1pt\parskip0pt\parsep0pt
\item
  (i) la fonction f ∗ g est continue et périodique de période 2\pi~
\item
  (ii) l'application (f,g)\mapsto~f ∗ g est
  bilinéaire
\item
  (iii) le produit de convolution est commutatif~: g ∗ f = f ∗ g
\item
  (iv) le produit de convolution est associatif~: (f ∗ g) ∗ h = f ∗ (g ∗
  h)
\end{itemize}

Démonstration (i) On a f ∗ g(x + 2\pi~) = 1 \over 2\pi~
\int  _0^2\pi~~f(t)g(x + 2\pi~ - t) dt
= 1 \over 2\pi~ \int ~
_0^2\pi~f(t)g(x - t) dt = f ∗ g(x) puisque g est périodique
de période 2\pi~. Montrons la continuité de f ∗ g. Supposons tout d'abord
que f est en escalier et soit a_0 = 0 \leq a_1
\leq\\ldots~ \leq
a_p = 2\pi~ une subdivision de [0,2\pi~] adaptée à f, si bien que
\forall~t \in]a_i-1,a_i~[, f(t) =
\lambda_i~; on a alors

\begin{align*} \int ~
_0^2\pi~f(t)g(x - t) dt& =& \\sum
_i=1^p\lambda_ i
\\int  ~
_a_i-1^a_i g(x - t) dt \%&
\\ & =& -\\sum
_i=1^p\lambda_ i
\\int  ~
_x-a_i-1^x-a_i g(u) du\%&
\\ \end{align*}

en faisant le changement de variable u = x - t. Comme une intégrale de
fonction réglée dépend de fa\ccon continue des bornes
d'intégration, l'application
x\mapsto~\int ~
_x-a_i-1^x-a_ig(u) du est continue et
donc f ∗ g est continue. Si maintenant f est continue par morceaux, soit
(f_n) une suite d'applications en escalier qui converge
uniformément vers f. On a alors

\begin{align*} f ∗ g(x) - f_n ∗
g(x)& =& \left  1 \over 2\pi~
\int  _0^2\pi~(f(t) - f_
n(t))g(x - t) dt\right \%&
\\ & \leq& 1 \over 2\pi~
\int  _0^2\pi~~f(t) -
f_ n(t)\,g(x - t)
dt \%& \\ & \leq&
\f -
f_n_\infty~\g_\infty~
\%& \\ \end{align*}

ce qui montre que la suite (f_n ∗ g) converge uniformément vers
f ∗ g. Comme ces applications sont continues, il en est de même de f ∗
g.

(ii) est évident

(iii) On a, en faisant le changement de variable u = x - t et en
remarquant que la fonction intégrée étant périodique de période 2\pi~, son
intégrale sur le segment de longueur 2\pi~, [x - 2\pi~,x] est égale à
l'intégrale sur [0,2\pi~]

\begin{align*} f ∗ g(x)& =& 1
\over 2\pi~ \int ~
_0^2\pi~f(t)g(x - t) dt \%& \\
& =& - 1 \over 2\pi~ \int ~
_x^x-2\pi~f(x - u)g(u) du \%&
\\ & =& 1 \over 2\pi~
\int  _x-2\pi~^x~f(x - u)g(u) du \%&
\\ & =& 1 \over 2\pi~
\int  _0^2\pi~~f(x - u)g(u) du = g ∗
f(x)\%& \\
\end{align*}

Ceci démontre la commutativité.

(iv) On a

\begin{align*} (f ∗ g) ∗ h(x)& =& 1
\over 2\pi~ \int ~
_0^2\pi~f ∗ g(t)h(x - t) dt \%&
\\ & =& 1 \over
4\pi~^2 \int  _0^2\pi~~h(x
- t)\left (\int ~
_0^2\pi~f(u)g(t - u) du\right ) dt\%&
\\ & =& 1 \over
4\pi~^2 \int ~
_0^2\pi~\left (\int ~
_0^2\pi~f(u)g(t - u)h(x - t) du\right ) dt\%&
\\ \end{align*}

Si f, g et h sont continues, le théorème de Fubini permet d'intervertir
les deux signes d'intégration et on obtient

\begin{align*} (f ∗ g) ∗ h(x)&& \%&
\\ & =& 1 \over
4\pi~^2 \int ~
_0^2\pi~\left (\int ~
_0^2\pi~f(u)g(t - u)h(x - t) dt\right ) du
\%& \\ & =& 1 \over
4\pi~^2 \int ~
_0^2\pi~f(u)\left (\\int
 _0^2\pi~g(t - u)h(x - t) dt\right ) du \%&
\\ & =& 1 \over
4\pi~^2 \int ~
_0^2\pi~f(u)\left (\\int
 _-u^2\pi~-ug(v)h(x - u - v) dv\right )
du\%& \\ & =& 1 \over
4\pi~^2 \int ~
_0^2\pi~f(u)\left (\\int
 _0^2\pi~g(v)h(x - u - v) dv\right ) du \%&
\\ & =& 1 \over 2\pi~
\int  _0^2\pi~~f(u) g ∗ h(x - u) du =
f ∗ (g ∗ h)(x) \%& \\
\end{align*}

en faisant le changement de variable v = t - u, soit t = v + u, dans
l'intégrale interne et en utilisant le fait que la fonction est
périodique de période 2\pi~. Si f et g sont seulement continues par
morceaux, il suffit d'utiliser des subdivisions adaptées et de découper
les intégrales suivant ces subdivisions.

Théorème~14.5.2 Soit f et g des applications de \mathbb{R}~ dans \mathbb{C} périodiques de
période 2\pi~. On suppose que f est continue par morceaux et que g est de
classe C^k. Alors f ∗ g est de classe C^k et (f
∗ g)^(k) = f ∗ (g^(k)).

Démonstration Une récurrence évidente permet d'obtenir le résultat pour
k quelconque à partir de k = 1. Quitte à utiliser une subdivision de
[0,2\pi~] et à découper l'intégrale, il suffit de montrer que
x\mapsto~\int ~
_a^bf(t)g(x - t) dt est de classe \mathcal{C}^1 lorsque f
est continue sur [a,b] et g de classe \mathcal{C}^1 sur \mathbb{R}~. Mais
l'application (x,t)\mapsto~f(t)g(x - t) admet une
dérivée partielle par rapport à x égale à  \partial~ \over \partial~x
(f(t)g(x - t)) = f(t)g'(x - t) qui est une fonction continue du couple
(x,t). Le théorème de dérivation des intégrales dépendant d'un paramètre
montre que x\mapsto~\int ~
_a^bf(t)g(x - t) dt est de classe \mathcal{C}^1 et que

(f ∗ g)'(x) =\int  _a^b~ \partial~
\over \partial~x (f(t)g(x - t)) dt =\\int
 _a^bf(t)g'(x - t) dt

Ceci montre que f ∗ g est de classe \mathcal{C}^1 et que (f ∗ g)' = f ∗
(g').

\paragraph{14.5.2 Produit de convolution et séries de Fourier}

Théorème~14.5.3 Soit f et g des applications de \mathbb{R}~ dans \mathbb{C} périodiques de
période 2\pi~, continues par morceaux. Alors \forall~~n \in
\mathbb{Z}, c_n(f ∗ g) = c_n(f)c_n(g).

Démonstration On a pour f continue par morceaux,

\begin{align*} f ∗ e_n(x)& =& 1
\over 2\pi~ \int ~
_0^2\pi~f(t)e^in(x-t) dt = 1
\over 2\pi~ e^inx\int ~
_0^2\pi~f(t)e^-int dt\%&
\\ & =& c_n(f)e^inx
\%& \\ \end{align*}

On en déduit que

\begin{align*} c_n(f ∗ g)& =& (f ∗ g) ∗
e_n(0) = f ∗ (g ∗ e_n)(0) \%&
\\ & =& f ∗
(c_n(g)e_n)(0) = c_n(g)(f ∗ e_n)(0)
= c_n(g)c_n(f)\%& \\
\end{align*}

Remarque~14.5.1 Pour une fonction g donnée, l'application
f\mapsto~f ∗ g se traduit donc comme un filtre sur
le signal f~: l'amplitude c_n(f) de l'harmonique de f
correspondant à la fréquence n est multipliée par le coefficient
c_n(g). Comme les c_n(g) tendent vers 0 quand
n tend vers + \infty~, on voit qu'il ne peut exister
d'élément neutre pour le produit de convolution, c'est-à-dire de
fonction \epsilon telle que \forall~~f \inC, f ∗ \epsilon = f.

[
[
[
[

\end{document}

\part{Calcul différentiel}
% \documentclass[]{article}
\usepackage[T1]{fontenc}
\usepackage{lmodern}
\usepackage{amssymb,amsmath}
\usepackage{ifxetex,ifluatex}
\usepackage{fixltx2e} % provides \textsubscript
% use upquote if available, for straight quotes in verbatim environments
\IfFileExists{upquote.sty}{\usepackage{upquote}}{}
\ifnum 0\ifxetex 1\fi\ifluatex 1\fi=0 % if pdftex
  \usepackage[utf8]{inputenc}
\else % if luatex or xelatex
  \ifxetex
    \usepackage{mathspec}
    \usepackage{xltxtra,xunicode}
  \else
    \usepackage{fontspec}
  \fi
  \defaultfontfeatures{Mapping=tex-text,Scale=MatchLowercase}
  \newcommand{\euro}{€}
\fi
% use microtype if available
\IfFileExists{microtype.sty}{\usepackage{microtype}}{}
\usepackage{graphicx}
% Redefine \includegraphics so that, unless explicit options are
% given, the image width will not exceed the width of the page.
% Images get their normal width if they fit onto the page, but
% are scaled down if they would overflow the margins.
\makeatletter
\def\ScaleIfNeeded{%
  \ifdim\Gin@nat@width>\linewidth
    \linewidth
  \else
    \Gin@nat@width
  \fi
}
\makeatother
\let\Oldincludegraphics\includegraphics
{%
 \catcode`\@=11\relax%
 \gdef\includegraphics{\@ifnextchar[{\Oldincludegraphics}{\Oldincludegraphics[width=\ScaleIfNeeded]}}%
}%
\ifxetex
  \usepackage[setpagesize=false, % page size defined by xetex
              unicode=false, % unicode breaks when used with xetex
              xetex]{hyperref}
\else
  \usepackage[unicode=true]{hyperref}
\fi
\hypersetup{breaklinks=true,
            bookmarks=true,
            pdfauthor={},
            pdftitle={Derivees partielles},
            colorlinks=true,
            citecolor=blue,
            urlcolor=blue,
            linkcolor=magenta,
            pdfborder={0 0 0}}
\urlstyle{same}  % don't use monospace font for urls
\setlength{\parindent}{0pt}
\setlength{\parskip}{6pt plus 2pt minus 1pt}
\setlength{\emergencystretch}{3em}  % prevent overfull lines
\setcounter{secnumdepth}{0}
 
/* start css.sty */
.cmr-5{font-size:50%;}
.cmr-7{font-size:70%;}
.cmmi-5{font-size:50%;font-style: italic;}
.cmmi-7{font-size:70%;font-style: italic;}
.cmmi-10{font-style: italic;}
.cmsy-5{font-size:50%;}
.cmsy-7{font-size:70%;}
.cmex-7{font-size:70%;}
.cmex-7x-x-71{font-size:49%;}
.msbm-7{font-size:70%;}
.cmtt-10{font-family: monospace;}
.cmti-10{ font-style: italic;}
.cmbx-10{ font-weight: bold;}
.cmr-17x-x-120{font-size:204%;}
.cmsl-10{font-style: oblique;}
.cmti-7x-x-71{font-size:49%; font-style: italic;}
.cmbxti-10{ font-weight: bold; font-style: italic;}
p.noindent { text-indent: 0em }
td p.noindent { text-indent: 0em; margin-top:0em; }
p.nopar { text-indent: 0em; }
p.indent{ text-indent: 1.5em }
@media print {div.crosslinks {visibility:hidden;}}
a img { border-top: 0; border-left: 0; border-right: 0; }
center { margin-top:1em; margin-bottom:1em; }
td center { margin-top:0em; margin-bottom:0em; }
.Canvas { position:relative; }
li p.indent { text-indent: 0em }
.enumerate1 {list-style-type:decimal;}
.enumerate2 {list-style-type:lower-alpha;}
.enumerate3 {list-style-type:lower-roman;}
.enumerate4 {list-style-type:upper-alpha;}
div.newtheorem { margin-bottom: 2em; margin-top: 2em;}
.obeylines-h,.obeylines-v {white-space: nowrap; }
div.obeylines-v p { margin-top:0; margin-bottom:0; }
.overline{ text-decoration:overline; }
.overline img{ border-top: 1px solid black; }
td.displaylines {text-align:center; white-space:nowrap;}
.centerline {text-align:center;}
.rightline {text-align:right;}
div.verbatim {font-family: monospace; white-space: nowrap; text-align:left; clear:both; }
.fbox {padding-left:3.0pt; padding-right:3.0pt; text-indent:0pt; border:solid black 0.4pt; }
div.fbox {display:table}
div.center div.fbox {text-align:center; clear:both; padding-left:3.0pt; padding-right:3.0pt; text-indent:0pt; border:solid black 0.4pt; }
div.minipage{width:100%;}
div.center, div.center div.center {text-align: center; margin-left:1em; margin-right:1em;}
div.center div {text-align: left;}
div.flushright, div.flushright div.flushright {text-align: right;}
div.flushright div {text-align: left;}
div.flushleft {text-align: left;}
.underline{ text-decoration:underline; }
.underline img{ border-bottom: 1px solid black; margin-bottom:1pt; }
.framebox-c, .framebox-l, .framebox-r { padding-left:3.0pt; padding-right:3.0pt; text-indent:0pt; border:solid black 0.4pt; }
.framebox-c {text-align:center;}
.framebox-l {text-align:left;}
.framebox-r {text-align:right;}
span.thank-mark{ vertical-align: super }
span.footnote-mark sup.textsuperscript, span.footnote-mark a sup.textsuperscript{ font-size:80%; }
div.tabular, div.center div.tabular {text-align: center; margin-top:0.5em; margin-bottom:0.5em; }
table.tabular td p{margin-top:0em;}
table.tabular {margin-left: auto; margin-right: auto;}
div.td00{ margin-left:0pt; margin-right:0pt; }
div.td01{ margin-left:0pt; margin-right:5pt; }
div.td10{ margin-left:5pt; margin-right:0pt; }
div.td11{ margin-left:5pt; margin-right:5pt; }
table[rules] {border-left:solid black 0.4pt; border-right:solid black 0.4pt; }
td.td00{ padding-left:0pt; padding-right:0pt; }
td.td01{ padding-left:0pt; padding-right:5pt; }
td.td10{ padding-left:5pt; padding-right:0pt; }
td.td11{ padding-left:5pt; padding-right:5pt; }
table[rules] {border-left:solid black 0.4pt; border-right:solid black 0.4pt; }
.hline hr, .cline hr{ height : 1px; margin:0px; }
.tabbing-right {text-align:right;}
span.TEX {letter-spacing: -0.125em; }
span.TEX span.E{ position:relative;top:0.5ex;left:-0.0417em;}
a span.TEX span.E {text-decoration: none; }
span.LATEX span.A{ position:relative; top:-0.5ex; left:-0.4em; font-size:85%;}
span.LATEX span.TEX{ position:relative; left: -0.4em; }
div.float img, div.float .caption {text-align:center;}
div.figure img, div.figure .caption {text-align:center;}
.marginpar {width:20%; float:right; text-align:left; margin-left:auto; margin-top:0.5em; font-size:85%; text-decoration:underline;}
.marginpar p{margin-top:0.4em; margin-bottom:0.4em;}
.equation td{text-align:center; vertical-align:middle; }
td.eq-no{ width:5%; }
table.equation { width:100%; } 
div.math-display, div.par-math-display{text-align:center;}
math .texttt { font-family: monospace; }
math .textit { font-style: italic; }
math .textsl { font-style: oblique; }
math .textsf { font-family: sans-serif; }
math .textbf { font-weight: bold; }
.partToc a, .partToc, .likepartToc a, .likepartToc {line-height: 200%; font-weight:bold; font-size:110%;}
.chapterToc a, .chapterToc, .likechapterToc a, .likechapterToc, .appendixToc a, .appendixToc {line-height: 200%; font-weight:bold;}
.index-item, .index-subitem, .index-subsubitem {display:block}
.caption td.id{font-weight: bold; white-space: nowrap; }
table.caption {text-align:center;}
h1.partHead{text-align: center}
p.bibitem { text-indent: -2em; margin-left: 2em; margin-top:0.6em; margin-bottom:0.6em; }
p.bibitem-p { text-indent: 0em; margin-left: 2em; margin-top:0.6em; margin-bottom:0.6em; }
.paragraphHead, .likeparagraphHead { margin-top:2em; font-weight: bold;}
.subparagraphHead, .likesubparagraphHead { font-weight: bold;}
.quote {margin-bottom:0.25em; margin-top:0.25em; margin-left:1em; margin-right:1em; text-align:justify;}
.verse{white-space:nowrap; margin-left:2em}
div.maketitle {text-align:center;}
h2.titleHead{text-align:center;}
div.maketitle{ margin-bottom: 2em; }
div.author, div.date {text-align:center;}
div.thanks{text-align:left; margin-left:10%; font-size:85%; font-style:italic; }
div.author{white-space: nowrap;}
.quotation {margin-bottom:0.25em; margin-top:0.25em; margin-left:1em; }
h1.partHead{text-align: center}
.sectionToc, .likesectionToc {margin-left:2em;}
.subsectionToc, .likesubsectionToc {margin-left:4em;}
.subsubsectionToc, .likesubsubsectionToc {margin-left:6em;}
.frenchb-nbsp{font-size:75%;}
.frenchb-thinspace{font-size:75%;}
.figure img.graphics {margin-left:10%;}
/* end css.sty */

\title{Derivees partielles}
\author{}
\date{}

\begin{document}
\maketitle

\textbf{Warning: 
requires JavaScript to process the mathematics on this page.\\ If your
browser supports JavaScript, be sure it is enabled.}

\begin{center}\rule{3in}{0.4pt}\end{center}

[
[]
[

\subsubsection{15.1 Dérivées partielles}

\paragraph{15.1.1 Notion de dérivée partielle}

Définition~15.1.1 Soit E et F deux espaces vectoriels normés. Soit U un
ouvert de E, f : U \rightarrow~ F, a \in U. Soit v \in E
\diagdown\0\. On dit que f admet au point a
une dérivée partielle suivant le vecteur v si l'application
t\mapsto~f(a + tv) (définie sur un voisinage de 0)
est dérivable au point 0.

Remarque~15.1.1 L'existence de la dérivée partielle en a suivant le
vecteur v est donc équivalente à l'existence de
lim_t\rightarrow~0~ f(a+tv)-f(a)
\over t = \partial_vf(a). Remarquons que si v' = \lambda~v,
\lambda~\neq~0, alors  f(a+tv')-f(a)
\over t = \lambda~ f(a+uv)-f(a) \over u
avec u = \lambda~t ce qui montre que f admet en a une dérivée partielle selon v
si et seulement si f admet une dérivée partielle suivant \lambda~v et qu'alors
\partial_\lambda~vf(a) = \lambda~\partial_vf(a).

Exemple~15.1.1 Soit f : \mathbb{R}~^2 \rightarrow~ \mathbb{R}~ définie par f(x,y) =
x^2 \over y si
y\neq~0 et f(x,0) = 0. Soit v =
(a,b)\neq~(0,0). On a  f((0,0)+tv)-f(0,0)
\over t = \left \
\cases 0 &si b = 0 \cr 
a^2 \over b &si
b\neq~0  \right .. On en déduit
que f admet une dérivée partielle suivant tout vecteur v et que
\partial_vf(0,0) = \left \
\cases 0 &si b = 0 \cr 
a^2 \over b &si
b\neq~0  \right .. Remarquons
que l'application v\mapsto~\partial_vf(0,0) n'est
pas linéaire. Remarquons également que f n'est pas continue en (0,0)
(puisque lim_t\rightarrow~0f(t,t^2~) =
1\neq~f(0,0)). L'existence de dérivée partielle
suivant tout vecteur n'implique donc pas la continuité.

Proposition~15.1.1 On a les propriétés évidentes de la dérivation de
t\mapsto~f(a + tv) à savoir (i) si f et g admettent
en a une dérivée partielle suivant le vecteur v, il en est de même de \alpha~f
+ \beta~g et \partial_v(\alpha~f + \beta~g)(a) = \alpha~\partial_vf(a) +
\beta~\partial_vg(a). (ii) si f et g (à valeurs scalaires) admettent en a
une dérivée partielle suivant le vecteur v, il en est de même de fg et
\partial_v(fg)(a) = g(a)\partial_vf(a) + f(a)\partial_vg(a).

Remarque~15.1.2 Par contre, on n'a pas de théorème général de
composition des dérivées partielles. En reprenant l'exemple ci dessus, f
: \mathbb{R}~^2 \rightarrow~ \mathbb{R}~ définie par f(x,y) = x^2
\over y si y\neq~0 et f(x,0) =
0, l'application f admet en (0,0) une dérivée partielle suivant tout
vecteur, l'application t\mapsto~(t,t^2)
est dérivable en 0 et pourtant
t\mapsto~f(t,t^2) n'est pas dérivable en
0 (elle n'y est même pas continue).

Définition~15.1.2 Soit E un espace vectoriel normé de dimension finie, \mathcal{E}
=
(e_1,\\ldots,e_n~)
une base de E, F un espace vectoriel normé. Soit U un ouvert de E, f : U
\rightarrow~ F, a \in U. On dit que f admet au point a une dérivée partielle d'indice
i (suivant la base \mathcal{E}) si elle admet une dérivée partielle suivant le
vecteur e_i. On note alors  \partial~f \over
\partial~x_i (a) = \partial_e_if(a).

Exemple~15.1.2 Si E = \mathbb{R}~^n et si \mathcal{E} est la base canonique de
\mathbb{R}~^n, l'existence d'une dérivée partielle d'indice i au point
a =
(a_1,\\ldots,a_n~)
équivaut à la dérivabilité au point a_i de l'application
partielle
x_i\mapsto~f(a_1,\\ldots,a_i-1,x_i,a_i+1,\\\ldots,a_n~).
On retrouve bien la notion habituelle de dérivée partielle~: dérivée
suivant la variable x_i, toutes les autres étant considérées
comme constantes.

\paragraph{15.1.2 Composition des dérivées partielles}

On a vu précédemment qu'on n'avait pas de théorème de composition des
dérivées partielles en toute généralité. On va introduire une notion
d'application de classe \mathcal{C}^1.

Définition~15.1.3 Soit U un ouvert de \mathbb{R}~^n, f : U \rightarrow~ F. On dit
que f est de classe \mathcal{C}^1 au point a si, sur un certain
voisinage V de a, f admet des dérivées partielles de tout indice i \in
[1,n] et si ces dérivées partielles x\mapsto~
\partial~f \over \partial~x_i (x) sont continues au point a.

Lemme~15.1.2 Soit F un espace vectoriel de dimension finie, V un ouvert
de \mathbb{R}~^n, f : V \rightarrow~ F. Soit I un intervalle de \mathbb{R}~, t \in I et \phi =
(\phi_1,\\ldots,\phi_n~)
: I \rightarrow~ V . On suppose que \phi est dérivable au point t et que f est de
classe \mathcal{C}^1 au point \phi(t). Alors f \cdot \phi est dérivable au point
t et (f \cdot \phi)'(t) =\ \\sum
 _i=1^n \partial~f \over \partial~x_i
(\phi(t))\phi_i'(t).

Démonstration Sans nuire à la généralité, en prenant une base (sur \mathbb{R}~) de
F et en travaillant composante par composante, on peut supposer que f
est à valeurs réelles. On écrit

\begin{align*} f(\phi(t + h)) - f(\phi(t))&& \%&
\\ & =& f(\phi_1(t +
h),\\ldots,\phi_n~(t
+ h)) -
f(\phi_1(t),\\ldots,\phi_n~(t))
\%& \\ & =& \\sum
_i=1^n(f(\ldots,\phi_
i-1(t),\phi_i(t + h),\phi_i+1(t +
h),\ldots~)\%&
\\ & & \qquad -
f(\\ldots,\phi_i-1(t),\phi_i(t),\phi_i+1~(t
+ h),\\ldots~)) \%&
\\ \end{align*}

Mais \phi est continue au point t et donc pour h assez petit, tous les
(\phi_1(t),\\ldots,\phi_i-1(t),\phi_i~(t
+ h),\phi_i+1(t +
h),\\ldots,\phi_n~(t
+ h)) se trouvent à l'intérieur d'une boule de centre a sur laquelle les
dérivées partielles de f de tout indice existent. En particulier,
l'application
x_i\mapsto~f(\phi_1(t),\\ldots,\phi_i-1(t),x_i,\phi_i+1~(t
+
h),\\ldots,\phi_n~(t
+ h)) est dérivable sur le segment [\phi_i(t),\phi_i(t +
h)] et on peut appliquer le théorème des accroissements finis. On
obtient l'existence d'un \xi_i \in
[\phi_i(t),\phi_i(t + h)] tel que

\begin{align*}
f(\phi_1(t),\\ldots,\phi_i-1(t),\phi_i~(t
+ h),\phi_i+1(t +
h),\\ldots,\phi_n~(t
+ h))&& \%& \\ & &
-f(\phi_1(t),\\ldots,\phi_i-1(t),\phi_i(t),\phi_i+1~(t
+
h),\\ldots,\phi_n~(t
+ h))\%& \\ & =& (\phi_i(t + h) -
\phi_i(t)) \%& \\ & &
\quad  \partial~f \over \partial~x_i
(\phi_1(t),\\ldots,\phi_i-1(t),\xi_i,\phi_i+1~(t
+
h),\\ldots,\phi_n~(t
+ h)) \%& \\
\end{align*}

Comme \phi est continue au point t et \xi_i \in
[\phi_i(t),\phi_i(t + h)], on a

\begin{align*}
lim_h\rightarrow~0(\phi_1(t),\\\ldots,\phi_i-1(t),\xi_i,\phi_i+1~(t
+
h),\\ldots,\phi_n~(t
+ h))&&\%& \\ & =&
(\phi_1(t),\\ldots,\phi_i-1(t),\phi_i(t),\phi_i+1(t),\\\ldots,\phi_n~(t))\%&
\\ & =& \phi(t) \%&
\\ \end{align*}

et comme  \partial~f \over \partial~x_i est continue au
point \phi(t), on a

\begin{align*} \partial~f \over
\partial~x_i (\phi(t))&& \%& \\ & =&
lim_h\rightarrow~0~ \partial~f \over
\partial~x_i
(\phi_1(t),\\ldots,\phi_i-1(t),\xi_i,\phi_i+1~(t
+
h),\\ldots,\phi_n~(t
+ h))\%& \\
\end{align*}

Il suffit alors de diviser par h et de faire tendre h vers 0 pour voir
que

lim_h\rightarrow~0~ f(\phi(t + h)) - f(\phi(t))
\over h = \\sum
_i=1^n \partial~f \over \partial~x_i
(\phi(t))\phi_i'(t)

ce qui achève la démonstration.

Appliquant ce lemme à \phi : t\mapsto~g(a + tv) au
point t = 0, on obtient le théorème suivant

Théorème~15.1.3 Soit F un espace vectoriel de dimension finie, U un
ouvert de \mathbb{R}~^n, f : U \rightarrow~ F. Soit E un espace vectoriel normé, V
un ouvert de E et g =
(g_1,\\ldots,g_n~)
: V \rightarrow~ U \subset~ \mathbb{R}~^n. Soit a \in V et v \in E
\diagdown\0\. Si g admet en a une dérivée
partielle suivant le vecteur v et si f est de classe \mathcal{C}^1 au
point a, alors f \cdot g admet en a une dérivée partielle suivant le vecteur
v et on a

\partial_v(f \cdot g)(a) = \\sum
_i=1^n \partial~f \over \partial~x_i
(g(a))\partial_vg_i(a)

Dans le cas particulier où E = \mathbb{R}~^p et où on prend pour v le
j-ième vecteur de la base canonique, on obtient la version suivante (on
a changé le nom des variables pour les appeler
y_1,\\ldots,y_n~
dans \mathbb{R}~^n).

Corollaire~15.1.4 Soit F un espace vectoriel de dimension finie, U un
ouvert de \mathbb{R}~^n, f : U \rightarrow~ F. Soit V un ouvert de \mathbb{R}~^p
et g =
(g_1,\\ldots,g_n~)
: V \rightarrow~ U \subset~ \mathbb{R}~^n. Soit a \in V et j \in [1,p]. Si g admet en a
une dérivée partielle d'indice j et si f est de classe \mathcal{C}^1 au
point a, alors f \cdot g admet en a une dérivée partielle d'indice j et on a

 \partial~(f \cdot g) \over \partial~x_j (a) =
\sum _i=1^n~ \partial~f
\over \partial~y_i (g(a)) \partial~g_i
\over \partial~x_j (a)

Remarque~15.1.3 On en déduit immédiatement que la composée de deux
applications de classe \mathcal{C}^1 est encore de classe
\mathcal{C}^1.

Citons aussi le corollaire suivant du lemme, où l'on prend \phi(t) = a + tv

Corollaire~15.1.5 Soit F un espace vectoriel de dimension finie, V un
ouvert de \mathbb{R}~^n, f : V \rightarrow~ F. Soit a \in V . Si f est de classe
\mathcal{C}^1 au point a, alors elle admet en a des dérivées partielles
suivant tout vecteur et on a

\partial_vf(a) = \\sum
_i=1^nv_ i \partial~f \over
\partial~x_i (a)\qquad \text si v =
(v_1,\ldots,v_n~)

Remarque~15.1.4 On voit que dans ce cas
v\mapsto~\partial_vf(a) est linéaire. Cette
remarque nous conduira à la définition de la différentielle dans la
section suivante.

\paragraph{15.1.3 Théorème des accroissements finis et applications}

Théorème~15.1.6 Soit U un ouvert de \mathbb{R}~^n, f : U \rightarrow~ \mathbb{R}~ de classe
\mathcal{C}^1. Soit a \in U et h \in \mathbb{R}~^n tel que [a,a + h] \subset~
U. Alors, il existe \theta \in]0,1[ tel que

f(a + h) - f(a) = \\sum
_i=1^nh_ i \partial~f \over
\partial~x_i (a + \thetah)

Démonstration Soit \psi : [0,1] \rightarrow~ \mathbb{R}~ définie par \psi(t) = f(a + th). Le
lemme du paragraphe précédent montre que \psi est dérivable sur [0,1]
et que \psi'(t) = \\sum ~
_i=1^nh_i \partial~f \over
\partial~x_i (a + th). Le théorème des accroissements finis assure
qu'il existe \theta \in]0,1[ tel que \psi(1) - \psi(0) = (1 - 0)\psi'(\theta), ce qui
n'est autre que la formule ci dessus.

Corollaire~15.1.7 Soit U un ouvert de \mathbb{R}~^n, F un espace
vectoriel de dimension finie, f : U \rightarrow~ F de classe \mathcal{C}^1. Alors
f est continue.

Démonstration En prenant une base (sur \mathbb{R}~) de F et en travaillant
composante par composante, on peut supposer que f est à valeurs réelles.
Puisque les dérivées partielles sont continues au point a, il existe \eta
> 0 tel que B(a,\eta) \subset~ U et \forall~~x \in
B(0,\eta),  \partial~f \over \partial~x_i (x) - \partial~f
\over \partial~x_i (a)\leq 1, d'où 
\partial~f \over \partial~x_i (x)\leq 1 + 
\partial~f \over \partial~x_i (a). Pour
\h\ < \eta, on
a alors [a,a + h] \subset~ B(0,\eta) et donc f(a + h) -
f(a)\leq\\sum ~
_i=1^nh_i\,\left
(\left  \partial~f \over
\partial~x_i (a)\right  +
1\right ), ce qui montre la continuité de f au point a.

Corollaire~15.1.8 Soit U un ouvert connexe de \mathbb{R}~^n, F un
espace vectoriel de dimension finie, f : U \rightarrow~ F. Alors f est constante si
et seulement si~elle est de classe \mathcal{C}^1 et toutes ses dérivées
partielles sont nulles.

Démonstration Si f est constante, il est clair qu'elle est de classe
\mathcal{C}^1 et que toutes ses dérivées partielles sont nulles. Pour
la réciproque, en prenant une base (sur \mathbb{R}~) de F et en travaillant
composante par composante, on peut supposer que f est à valeurs réelles.
Soit x_0 \in U et soit X = \x \in
U∣f(x) = f(x_0)\.
Puisque f est continue (d'après le corollaire précédent), X est un fermé
de U, évidemment non vide. Montrons que X est également ouvert dans U~;
soit en effet x_1 dans X et soit \eta > 0 tel que
B(x_1,\eta) \subset~ U. Pour
\h\ < \eta, on
a [x_1,x_1 + h] \subset~ B(x_1,\eta) \subset~ U et le
théorème des accroissements finis nous donne f(x_1 + h) =
f(x_1) = f(x_0), les dérivées partielles étant
supposées nulles. On a donc B(x_1,\eta) \subset~ X, et donc X est ouvert.
Comme X est à la fois ouvert et fermé, non vide dans U connexe, on a X =
U et donc f est constante.

\paragraph{15.1.4 Dérivées partielles successives}

On définit la notion de dérivées partielles successives de manière
récursive de la manière suivante

Définition~15.1.4 Soit U un ouvert de \mathbb{R}~^n, a \in U et f : U \rightarrow~
E. Soit
(i_1,\\ldots,i_k~)
\in [1,n]^k. On dit que  \partial~^kf
\over
\partial~x_i_1\\ldots\partial~x_i_k~
(a) existe s'il existe un ouvert V tel que a \in V \subset~ U et sur lequel 
\partial~^k-1f \over
\partial~x_i_2\\ldots\partial~x_i_k~
(x) existe et si l'application x\mapsto~
\partial~^k-1f \over
\partial~x_i_2\\ldots\partial~x_i_k~
(x) admet une dérivée partielle d'indice i_1. On pose alors

 \partial~^kf \over
\partial~x_i_1\\ldots\partial~x_i_k~
(a) = \partial~ \over \partial~x_i_1
\left ( \partial~^k-1f \over
\partial~x_i_2\\ldots\partial~x_i_k~
\right )(a)

Définition~15.1.5 Soit U un ouvert de \mathbb{R}~^n et f : U \rightarrow~ E. On
dit que f est de classe C^k sur U si,
\forall~(i_1,\\\ldots,i_k~)
\in [1,n]^k, l'application x\mapsto~
\partial~^kf \over
\partial~x_i_1\\ldots\partial~x_i_k~
(x) est définie et continue sur U.

Remarque~15.1.5 Comme on a vu que toute application de classe
\mathcal{C}^1 est continue, on en déduit immédiatement que toute
application de classe C^k est aussi de classe
C^k-1. On dira bien entendu que f est de classe
C^\infty~ si elle est de classe C^k pour tout k. Une
récurrence évidente sur k montre que la composée de deux applications de
classe C^k est encore de classe C^k et que donc la
composée de deux applications de classe C^\infty~ est encore de
classe C^\infty~.

Lemme~15.1.9 Soit U un ouvert de \mathbb{R}~^2, f : U \rightarrow~ \mathbb{R}~ de classe
C^2. Alors,  \partial~^2f \over
\partial~x_1\partial~x_2 = \partial~^2f \over
\partial~x_2\partial~x_1 .

Démonstration Soit (a_1,a_2) \in U et soit

\begin{align*} \phi(h_1,h_2)& =&
1 \over h_1h_2 (f(a_1 +
h_1,a_2 + h_2) - f(a_1 +
h_1,a_2)\%& \\ & &
\quad \quad \quad -
f(a_1,a_2 + h_2) +
f(a_1,a_2)) \%& \\
\end{align*}

définie pour h_1 et h_2 non nuls et assez petits. On a
\phi(h_1,h_2) = 1 \over
h_1h_2 \psi_1(a_1 + h_1) -
\psi_1(a_1) avec \psi_1(x_1) =
f(x_1,a_2 + h_2) -
f(x_1,a_2). Or \psi_1 est dérivable sur
[a_1,a_1 + h_1] avec
\psi_1'(x_1) = \partial~f \over \partial~x_1
(x_1,a_2 + h_2) - \partial~f \over
\partial~x_1 (x_1,a_2). On peut donc appliquer le
théorème des accroissements finis, et donc il existe \xi_1 \in
[a_1,a_1 + h_1] tel que

\begin{align*} \phi(h_1,h_2)& =&
1 \over h_2 \psi_1'(\xi_1) \%&
\\ & =& 1 \over
h_2 \left ( \partial~f \over
\partial~x_1 (\xi_1,a_2 + h_2) - \partial~f
\over \partial~x_1
(\xi_1,a_2)\right )\%&
\\ & =& \partial~^2f
\over \partial~x_2\partial~x_1
(\xi_1,\xi_2) \%& \\
\end{align*}

avec \xi_2 \in [a_2,a_2 + h_2] en
appliquant le théorème des accroissements finis à
x_2\mapsto~ \partial~f \over
\partial~x_1 (\xi_1,x_2) qui est dérivable sur
[a_2,a_2 + h_2], de dérivée 
\partial~^2f \over \partial~x_2\partial~x_1
(\xi_1,x_2). Quand h_1 et h_2 tendent
vers 0, \xi_1 et \xi_2 tendent respectivement vers
a_1 et a_2 et la continuité de  \partial~^2f
\over \partial~x_2\partial~x_1 montre que
lim_(h_1,h_2)\rightarrow~(0,0)\phi(h_1,h_2~)
= \partial~^2f \over \partial~x_2\partial~x_1
(a_1,a_2). Comme les deux variables jouent un rôle
symétrique dans la définition de \phi, en posant \psi_2(x_2)
= f(a_1 + h_1,x_2) -
f(a_1,x_2) et en appliquant deux fois le théorème des
accroissements finis, on obtient
lim_(h_1,h_2)\rightarrow~(0,0)\phi(h_1,h_2~)
= \partial~^2f \over \partial~x_1\partial~x_2
(a_1,a_2), ce qui démontre que  \partial~^2f
\over \partial~x_1\partial~x_2
(a_1,a_2) = \partial~^2f \over
\partial~x_2\partial~x_1 (a_1,a_2).

Théorème~15.1.10 (Schwarz). Soit U un ouvert de \mathbb{R}~^n et f : U
\rightarrow~ E (espace vectoriel normé de dimension finie) de classe
C^2. Alors \forall~~(i,j) \in
[1,n]^2,

 \partial~^2f \over \partial~x_i\partial~x_j =
\partial~^2f \over \partial~x_j\partial~x_i

Démonstration En prenant une base de E, on peut se contenter de montrer
le résultat lorsque E = \mathbb{R}~. Si i = j, le résultat est évident. Supposons
i < j et soit
(a_1,\\ldots,a_n~)
\in \mathbb{R}~^n. On applique le lemme précédent à l'application de
classe C^2, définie sur un ouvert contenant
(a_i,a_j),

g(x_i,x_j) =
f(a_1,\\ldots,a_i-1,x_i,a_i+1,\\\ldots,a_j-1,x_j,a_j+1,\\\ldots,a_n~)

qui est de classe C^2 (composée d'applications de classe
C^2). On a donc  \partial~^2g \over
\partial~x_i\partial~x_j (a_i,a_j) =
\partial~^2g \over \partial~x_j\partial~x_i
(a_i,a_j), soit encore

 \partial~^2f \over \partial~x_i\partial~x_j
(a_1,\\ldots,a_n~)
= \partial~^2f \over \partial~x_j\partial~x_i
(a_1,\\ldots,a_n~)

Corollaire~15.1.11 Soit U un ouvert de \mathbb{R}~^n et f : U \rightarrow~ E de
classe C^k. Soit
(i_1,\\ldots,i_k~)
\in [1,n]^k. Pour toute permutation \sigma de [1,k] on a

 \partial~^kf \over
\partial~x_i_\sigma(1)\\ldots\partial~x_i_\sigma(k)~
= \partial~^kf \over
\partial~x_i_1\\ldots\partial~x_i_k~

Démonstration D'après le théorème de Schwarz, le résultat est vrai
lorsque \sigma = \tau_j,j+1 est la transposition qui échange j et j +
1. Mais toute permutation de [1,k] est un produit de telles
transpositions (facile) ce qui démontre le corollaire.

Notation définitive Soit
(i_1,\\ldots,i_k~)
\in [1,n]^k. Pour j \in [1,n], soit k_j le
nombre de i_q qui sont égaux à j. On a donc à une permutation
près, la famille
(i_1,\\ldots,i_k~)
qui est égale à
(\overbrace1,\\ldots,1k_1~
fois,\\ldots,\overbracej,\\\ldots,j~
k_j
fois,\\ldots,\overbracen,\\\ldots,nk_n~
fois), chaque j étant compté k_j fois. En notant
\partial~x_j^k_j à la place de
\overbrace\partial~x_j\\ldots\partial~x_j~
k_j fois, on obtient

 \partial~^kf \over
\partial~x_i_1\\ldots\partial~x_i_k~
= \partial~^kf \over
\partial~x_1^k_1\\ldots\partial~x_n^k_n~

\paragraph{15.1.5 Formules de Taylor}

Lemme~15.1.12 Soit U un ouvert de \mathbb{R}~^n et f : U \rightarrow~ E de classe
C^k. Soit a \in U et h \in \mathbb{R}~^n tel que [a,a + h] \subset~
U. Posons \phi(t) = f(a + th), définie et de classe C^k sur
[0,1]. Alors, pour tout t \in [0,1],

\begin{align*} \phi^(k)(t) =
\\sum
_k_1+\ldots+k_n=k~
k! \over
k_1!\ldotsk_n!~
h_1^k_1
\ldotsh_n^k_n ~
\partial~^kf \over
\partial~x_1^k_1\ldots\partial~x_n^k_n~
(a + th)& & \%& \\
\end{align*}

Démonstration Par récurrence sur k. Pour k = 1, ce n'est qu'une autre
formulation du résultat

\begin{align*} \phi'(t)& =&
\sum _i=1^nh_ i~ \partial~f
\over \partial~x_i (a + th) \%&
\\ & =& \\sum
_k_1+\ldots+k_n=1h_1^k_1~
\ldotsh_n^k_n ~
\partial~f \over
\partial~x_1^k_1\ldots\partial~x_n^k_n~
(a + th)\%& \\
\end{align*}

en posant k_i = 1 et k_j = 0 pour
i\neq~j.

Supposons le résultat démontré pour k - 1. On a donc

\begin{align*} \phi^(k-1)(t) =&& \%&
\\ & & \\sum
_k_1+\ldots+k_n=k-1~
(k - 1)! \over
k_1!\ldotsk_n!~
h_1^k_1
\ldotsh_n^k_n ~
\partial~^k-1f \over
\partial~x_1^k_1\ldots\partial~x_n^k_n~
(a + th)\%& \\
\end{align*}

On en déduit que

\begin{align*} \phi^(k)(t) =&& \%&
\\ & & \\sum
_k_1+\ldots+k_n=k-1~
(k - 1)! \over
k_1!\ldotsk_n!~
h_1^k_1
\ldotsh_n^k_n ~
d \over dt \left ( \partial~^k-1f
\over
\partial~x_1^k_1\ldots\partial~x_n^k_n~
(a + th)\right )\%& \\
\end{align*}

soit encore

\begin{align*} \phi^(k)(t)& =&
\\sum
_k_1+\ldots+k_n=k-1~
(k - 1)! \over
k_1!\ldotsk_n!~
h_1^k_1
\ldotsh_n^k_n ~
\%& \\ & & \quad
\quad  \\sum
_i=1^nh_ i \partial~^kf
\over
\partial~x_1^k_1\ldots\partial~x_i^k_i+1\\ldots\partial~x_n^k_n~
(a + th)\%& \\
\end{align*}

En intervertissant les deux signes de somme on obtient

\begin{align*} \phi^(k)(t)& =&
\sum _i=1^n~
\\sum
_k_1+\ldots+k_n=k-1~
(k - 1)!(k_i + 1) \over
k_1!\ldots(k_i~ +
1)!\ldotsk_n!~ \%&
\\ & & \quad
\quad h_1^k_1
\\ldotsh_i^k_i+1\\\ldotsh_
n^k_n  \partial~^kf \over
\partial~x_1^k_1\\ldots\partial~x_i^k_i+1\\\ldots\partial~x_n^k_n~
(a + th)\%& \\
\end{align*}

et en faisant un changement d'indice

\begin{align*} \phi^(k)(t)& =&
\sum _i=1^n~
\sum _
k_1+\ldots+k_n~=k
\atop k_i≥1  (k - 1)!k_i
\over
k_1!\ldotsk_n!~ \%&
\\ & & \quad
\quad h_1^k_1
\\ldotsh_i^k_i~
\\ldotsh_n^k_n~
 \partial~^kf \over
\partial~x_1^k_1\\ldots\partial~x_i^k_i\\\ldots\partial~x_n^k_n~
(a + th)\%& \\
\end{align*}

Réintroduisons les termes pour k_i = 0 qui sont nuls puisqu'ils
contiennent le facteur (k - 1)!k_i, on obtient

\begin{align*} \phi^(k)(t)& =&
\sum _i=1^n~
\\sum
_k_1+\ldots+k_n=k~
(k - 1)!k_i \over
k_1!\ldotsk_n!~
h_1^k_1
\ldotsh_i^k_i~
\ldotsh_n^k_n~
\%& \\ & & \quad
\quad \quad  \partial~^kf
\over
\partial~x_1^k_1\\ldots\partial~x_i^k_i\\\ldots\partial~x_n^k_n~
(a + th) \%& \\
\end{align*}

Ceci nous permet de réintervertir les deux sommations, soit encore,
après mise en facteur

\begin{align*} \phi^(k)(t)& =&
\\sum
_k_1+\ldots+k_n=k~
(k - 1)!\sum _i=1^nk_i~
\over
k_1!\ldotsk_n!~
h_1^k_1
\ldotsh_i^k_i~
\ldotsh_n^k_n~
\%& \\ & & \quad
\quad \quad  \partial~^kf
\over
\partial~x_1^k_1\\ldots\partial~x_i^k_i\\\ldots\partial~x_n^k_n~
(a + th) \%& \\
\end{align*}

soit encore

\begin{align*} \phi^(k)(t)& =&
\\sum
_k_1+\ldots+k_n=k~
k! \over
k_1!\ldotsk_n!~
h_1^k_1
\ldotsh_n^k_n ~
\partial~^kf \over
\partial~x_1^k_1\ldots\partial~x_n^k_n~
(a + th)\%& \\
\end{align*}

ce qui achève la récurrence.

Remarque~15.1.6 Cette formule est tout à fait analogue à la formule du
binôme généralisée

(X_1 +
\\ldots~ +
X_n)^k = \\sum
_k_1+\ldots+k_n=k~
k! \over
k_1!\ldotsk_n!~
X_1^k_1
\ldotsX_n^k_n ~

Cette remarque nous conduira à une notation plus compacte. Introduisons
un produit symbolique sur les expressions du type
h_1^k_1\\ldotsh_n^k_n~
\partial~^k \over
\partial~x_1^k_1\\ldots\partial~x_n^k_n~
en posant

\begin{align*} \left
(h_1^k_1
\\ldotsh_n^k_n~
 \partial~^k \over
\partial~x_1^k_1\\ldots\partial~x_n^k_n~
\right ) ∗\left
(h_1^l_1
\\ldotsh_n^l_n~
 \partial~^l \over
\partial~x_1^l_1\\ldots\partial~x_n^l_n~
\right ) =&&\%& \\ & &
h_1^k_1+l_1
\\ldotsh_n^k_n+l_n~
 \partial~^k+l \over
\partial~x_1^k_1+l_1\\ldots\partial~x_n^k_n+l_n~
\quad \quad \quad \%&
\\ \end{align*}

Ce produit est commutatif, et

\begin{align*} \\sum
_k_1+\ldots+k_n=k~
k! \over
k_1!\ldotsk_n!~
h_1^k_1
\ldotsh_n^k_n ~
\partial~^k \over
\partial~x_1^k_1\ldots\partial~x_n^k_n~
&&\%& \\ & & = \left
(h_1 \partial~ \over \partial~x_1 +
\\ldots~ +
h_n \partial~ \over \partial~x_n
\right )^k∗\quad
\quad \quad \%&
\\ \end{align*}

où la notation ^k∗ désigne la puissance k-ième pour ce
produit commutatif. La formule s'écrit alors de manière plus agréable
sous la forme

\phi^(k)(t) = \left (h_ 1 \partial~
\over \partial~x_1 +
\\ldots~ +
h_n \partial~ \over \partial~x_n
\right )^k∗f(a + th)

Ces puissances se développent de la manière évidente en respectant la
règle de calcul pour le produit ∗.

Exemple~15.1.3 \phi'(t) = \left (h_1 \partial~
\over \partial~x_1 +
\\ldots~ +
h_n \partial~ \over \partial~x_n
\right )f(a + th)

\begin{align*} \phi''(t)& =& \left
(h_1 \partial~ \over \partial~x_1 +
\\ldots~ +
h_n \partial~ \over \partial~x_n
\right )^2∗f(a + th) \%&
\\ & =& \\sum
_i=1^nh_ i^2 \partial~^2f
\over \partial~x_i^2 (a + th) +
2\\sum
_i<jh_ih_j \partial~^2f
\over \partial~x_i\partial~x_j (a + th)\%&
\\ \end{align*}

et ainsi de suite.

Théorème~15.1.13 (formule de Taylor avec reste intégral). Soit U un
ouvert de \mathbb{R}~^n et f : U \rightarrow~ E de classe C^k+1. Soit a
\in U et h \in \mathbb{R}~^n tel que [a,a + h] \subset~ U. Alors

\begin{align*} f(a + h)& =& f(a) +
\sum _p=1^k~ 1
\over p! \left (h_1 \partial~
\over \partial~x_1 +
\ldots + h_n~ \partial~
\over \partial~x_n \right
)^p∗f(a)\%& \\
+\int  _0^1~ (1 -
t)^k \over k!  \left
(h_1 \partial~ \over \partial~x_1 +
\\ldots~ +
h_n \partial~ \over \partial~x_n
\right )^(k+1)∗f(a + th) dt&&\%&
\\ \end{align*}

Démonstration C'est simplement la formule de Taylor avec reste intégral
pour la fonction \phi~:

\phi(1) = \phi(0) + \sum _p=1^k~ 1
\over p! \phi^(p)(0) +
\\int  ~
_0^1 (1 - t)^k \over k!
\phi^(k+1)(t) dt

Remarque~15.1.7 On utilisera le plus souvent cette formule pour k = 1~;
dans cas d'une fonction définie sur un ouvert de \mathbb{R}~^2 on
obtiendra par exemple

\begin{align*} f(a + h)& =& f(a) + h_1
\partial~f \over \partial~x_1 (a) + h_2 \partial~f
\over \partial~x_2 (a) \%&
\\ & & \quad +
h_1^2\int  _0^1~(1
- t) \partial~^2f \over \partial~x_1^2 (a
+ th) dt \%& \\ & &
\quad + h_2^2\\int
 _0^1(1 - t) \partial~^2f \over
\partial~x_2^2 (a + th) dt \%&
\\ & & \quad +
2h_1h_2\int ~
_0^1(1 - t) \partial~^2f \over
\partial~x_1\partial~x_2 (a + th) dt\%&
\\ \end{align*}

Théorème~15.1.14 (formule de Taylor-Lagrange). Soit U un ouvert de
\mathbb{R}~^n et f : U \rightarrow~ \mathbb{R}~ de classe C^k+1. Soit a \in U et h
\in \mathbb{R}~^n tel que [a,a + h] \subset~ U. Alors, il existe \theta
\in]0,1[ tel que

\begin{align*} f(a + h)& =& f(a) +
\sum _p=1^k~ 1
\over p! \left (h_1 \partial~
\over \partial~x_1 +
\ldots + h_n~ \partial~
\over \partial~x_n \right
)^p∗f(a) \%& \\ & & + 1
\over (k + 1)! \left (h_1 \partial~
\over \partial~x_1 +
\\ldots~ +
h_n \partial~ \over \partial~x_n
\right )^(k+1)∗f(a + \thetah)\%&
\\ \end{align*}

Démonstration C'est simplement la formule de Taylor Lagrange pour la
fonction \phi~:

\phi(1) = \phi(0) + \sum _p=1^k~ 1
\over p! \phi^(p)(0) + 1 \over
(k + 1)! \phi^(k+1)(\theta)

Théorème~15.1.15 (formule de Taylor-Young). Soit U un ouvert de
\mathbb{R}~^n et f : U \rightarrow~ E (espace vectoriel normé de dimension finie)
de classe C^k. Soit a \in U. Alors, quand h tend vers 0 on a

f(a + h) = f(a) + \sum _p=1^k~ 1
\over p! \left (h_1 \partial~
\over \partial~x_1 +
\ldots + h_n~ \partial~
\over \partial~x_n \right
)^p∗f(a) +
o(\h\^k)

Démonstration Quitte à prendre une base de E et à travailler composante
par composante, on peut supposer que E = \mathbb{R}~~; toutes les normes sur
\mathbb{R}~^n étant équivalentes, on peut supposer que
\h\ =
h_1 +
\\ldots~ +
h_n. Soit \rho > 0 tel que B(a,\rho)
\subset~ U et soit h tel que
\h\ < \rho. On
a alors [a,a + h] \subset~ B(a,\rho) \subset~ U~; on peut donc appliquer la formule
de Taylor-Lagrange à l'ordre k - 1 qui nous donne

\begin{align*} f(a + h)& -& f(a)
-\sum _p=1^k~ 1
\over p! \left (h_1 \partial~
\over \partial~x_1 +
\ldots + h_n~ \partial~
\over \partial~x_n \right
)^p∗f(a)\%& \\ & =& 1
\over k! \left (h_1 \partial~
\over \partial~x_1 +
\\ldots~ +
h_n \partial~ \over \partial~x_n
\right )^k∗f(a + \thetah) \%&
\\ & -& 1 \over k!
\left (h_1 \partial~ \over
\partial~x_1 +
\\ldots~ +
h_n \partial~ \over \partial~x_n
\right )^k∗f(a) \%&
\\ \end{align*}

Mais les dérivées partielles de f sont continues. Soit \epsilon >
0~; il existe \eta > 0 tel que

\begin{align*}
\h\ < \eta&
\rigtharrow~&
\forall~(k_1,\\\ldots,k_n~)\text
tel que k_1 +
\\ldots~ +
k_n = k, \forall~~t \in [0,1] \%&
\\ & & \left 
\partial~^kf \over
\partial~x_1^k_1\\ldots\partial~x_n^k_n~
(a + th)\right . -\left .
\partial~^kf \over
\partial~x_1^k_1\\ldots\partial~x_n^k_n~
(a)\right  < \epsilon\%&
\\ \end{align*}

Pour \h\ <
\eta, on a alors (en développant les deux puissances symboliques)

\begin{align*} \big
\left (\\sum
h_i \partial~ \over \partial~x_i
\right )^k∗f(a + \thetah) -\left
(\sum h_i~ \partial~ \over
\partial~x_i \right
)^k∗f(a)\big &&\%&
\\ & <&
\epsilon\\sum
_k_1+\ldots+k_n=k~
k! \over
k_1!\ldotsk_n!~
h_1^k_1
\ldotsh_n^k_n~
\%& \\ & =&
\epsilon(h_1 +
\\ldots~ +
h_n)^k =
\epsilon\h\^k \%&
\\ \end{align*}

ce qui démontre le résultat.

\paragraph{15.1.6 Application aux extremums de fonctions de plusieurs
variables}

Soit U un ouvert de \mathbb{R}~^n et f : U \rightarrow~ \mathbb{R}~. Nous allons rechercher
les extremums de la fonction f à l'aide des résultats qui suivent.

Proposition~15.1.16 Soit U un ouvert de \mathbb{R}~^n et f : U \rightarrow~ \mathbb{R}~ de
classe \mathcal{C}^1. Soit a \in U. Si f admet en a un extremum local, on
a \forall~~i \in [1,n], \partial~f \over
\partial~x_i (a) = 0.

Démonstration Il suffit de remarquer que la fonction
t\mapsto~f(a + te_i) (définie sur un
voisinage de 0) admet en 0 un extremum local. On a donc

 \partial~f \over \partial~x_i (a) = d
\over dt \left (f(a +
te_i)\right )_t=0 = 0

Dans le cas des fonctions d'une variable, la condition ci dessus n'est
déjà pas suffisante (considérer
x\mapsto~x^3 au point 0). Il est clair
qu'il en est de même a fortiori pour une fonction de plusieurs
variables. Pour obtenir des résultats plus précis et en particulier des
conditions suffisantes d'extremums, nous allons introduire une forme
quadratique sur \mathbb{R}~^n

Définition~15.1.6 Soit U un ouvert de \mathbb{R}~^n et f : U \rightarrow~ \mathbb{R}~ de
classe C^2. Soit a \in U. On appelle différentielle seconde au
point a la forme quadratique sur \mathbb{R}~^n,

\begin{align*} h& =&
(h_1,\\ldots,h_n)\mathrel\mapsto~~\left
(h_1 \partial~ \over \partial~x_1 +
\\ldots~ +
h_n \partial~ \over \partial~x_n
\right )^2∗f(a) \%&
\\ & & = \\sum
_i=1^nh_ i^2 \partial~^2f
\over \partial~x_i^2 (a) +
2\\sum
_i<jh_ih_j \partial~^2f
\over \partial~x_i\partial~x_j (a)\%&
\\ \end{align*}

Théorème~15.1.17 Soit U un ouvert de \mathbb{R}~^n et f : U \rightarrow~ \mathbb{R}~ de
classe C^2. Soit a \in U tel que \forall~~i \in
[1,n], \partial~f \over \partial~x_i (a) = 0 et soit \Phi
la forme quadratique différentielle seconde au point a. Alors (i) si \Phi
est définie positive, c'est-à-dire si h\neq~0 \rigtharrow~
\Phi(h) > 0, alors f admet en a un minimum local strict (ii)
si \Phi est définie négative, c'est-à-dire si
h\neq~0 \rigtharrow~ \Phi(h) < 0, alors f admet en a
un maximum local strict (iii) si \Phi n'est ni positive ni négative, alors
f n'admet pas d'extremum en a (on dit dans ce cas que a est un point
selle ou point col de a, par analogie avec une selle de cheval ou un col
de montagne).

Démonstration (i). Utilisons la formule de Taylor Young à l'ordre 2. On
a donc, en tenant compte de  \partial~f \over \partial~x_i
(a) = 0, f(a + h) = f(a) + 1 \over 2 \Phi(h)
+\
h\^2\epsilon(h), avec
lim_h\rightarrow~0~\epsilon(h) = 0. Pour démontrer (i),
nous allons utiliser le lemme suivant

Lemme~15.1.18 Soit \Phi une forme quadratique définie positive sur
\mathbb{R}~^n (ou tout espace vectoriel normé de dimension finie).
Alors \exists~\alpha~ > 0,
\forall~h \in \mathbb{R}~^n~, \Phi(h) ≥
\alpha~\h\^2.

Démonstration Soit S la sphère unité de \mathbb{R}~^n. Comme \Phi est
continue sur S qui est compact, \Phi atteint sur S sa borne inférieure \alpha~.
Soit donc x_0 \in S tel que \Phi(x_0) = \alpha~
= inf _x\inS~\Phi(x). Comme
x_0\neq~0, on a \alpha~ > 0. De
plus, si h\neq~0, on a  h \over
\h\ \in S, soit \Phi( h
\over
\h\ ) ≥ \alpha~ soit 
\Phi(h) \over
\h\^2 ≥
\alpha~, soit encore \Phi(h) ≥
\alpha~\h\^2.

Puisque lim_h\rightarrow~0~\epsilon(h) = 0, il existe \eta
> 0 tel que
\h\ < \eta
\rigtharrow~\epsilon(h)\leq \alpha~ \over 4 . Pour
\h\ < \eta, on
a donc

\begin{align*} f(a + h) - f(a)& =& 1
\over 2 \Phi(h) +\
h\^2\epsilon(h) \%&
\\ & ≥& \alpha~ \over 2
\h\^2 - \alpha~
\over 4
\h\^2 = \alpha~
\over 4
\h\^2
> 0\%& \\
\end{align*}

pour h\neq~0. Donc f admet en a un minimum local
strict.

Pour démontrer (ii) à partir de (i), il suffit de changer f en - f.

(iii). Si \Phi n'est ni positive, ni négative, il existe v_1 \in
\mathbb{R}~^n tel que \Phi(v_1) < 0 et il existe
v_2 \in \mathbb{R}~^n tel que \Phi(v_2) > 0.
On a alors, d'après la même formule de Taylor, en posant h =
tv_i, f(a + tv_i) = f(a) + 1 \over
2 \Phi(tv_i) +
t^2\v_
i\^2\epsilon(tv_ i) = f(a) +
t^2 \over 2 \Phi(v_i) +
t^2\epsilon_ i(t) avec
lim_t\rightarrow~0\epsilon_i~(t) = 0. On en
déduit qu'il existe un \eta > 0 tel que t
< \eta \rigtharrow~ f(a + tv_1) <
f(a)\text et f(a + tv_2) >
f(a). Donc f n'a ni minimum, ni maximum en a.

Remarque~15.1.8 Dans le cas où \Phi est soit positive, soit négative, mais
non définie (c'est-à-dire que \Phi(h) peut être nul sans que h soit nul),
on ne peut pas conclure en général et il faut utiliser une formule de
Taylor à un ordre supérieur.

Exemple~15.1.4 n = 2~; soit U un ouvert de \mathbb{R}~^2 et f : U \rightarrow~ \mathbb{R}~,
(x,y)\mapsto~f(x,y). Soit (a,b) \in U. Une condition
nécessaire pour que f admette en (a,b) un extremum est que  \partial~f
\over \partial~x (a,b) = \partial~f \over \partial~y (a,b) =
0. Posons r = \partial~^2f \over \partial~x^2
(a,b), s = \partial~^2f \over \partial~x\partial~y (a,b), t =
\partial~^2f \over \partial~y^2 (a,b) (notations
de Monge). La forme quadratique \Phi est
(h,k)\mapsto~rh^2 + 2shk +
tk^2. Considérons suivant le cas le rapport  h
\over k ou le rapport  k \over h ,
on constate immédiatement à l'aide de l'étude du signe d'un trinome du
second degré que si (i) rt - s^2 > 0 et r
> 0, alors \Phi est définie positive et f a en a un minimum
local strict (ii) rt - s^2 > 0 et r <
0, alors \Phi est définie négative et f a en a un maximum local strict
(iii) rt - s^2 < 0, alors f a en a un point selle
(pas d'extremum local en a) (iv) rt - s^2 = 0, alors on ne
peut pas conclure.

Le lecteur comparera les surfaces z = f(x,y) ainsi que lignes de niveau
de ces surfaces dans les trois exemples ci dessous (correspondant
respectivement à un minimum local, un point selle et un point de type
(iv))

\includegraphics{cours8x.png}

[
[

\end{document}

% 
\subsubsection{15.2 Différentielle}

\subsubsection{Applications différentiables}
\label{sec:appl-diff}



\begin{de}
  Soit E et F deux espaces vectoriels normés, U un
ouvert de E, $a \in U$ et $f : U \rightarrow~ F$. On dit que f est différentiable au
point  a s'il  existe une  application linéaire  continue $L  : E  \rightarrow~ F$
telle
que, pour h voisin de 0,
\[
f(a + h) = f(a) + L(h) +
o(h)
\]

\end{de}
Dans ce cas, l'application L est unique et est appelée la différentielle
de f au point a, notée df(a) ou encore d_af.

Démonstration Supposons que$L_1$ et $L_2$ conviennent.
Par différence, on a $L_1(h) - L_2(h) =
o(\h\)$ . On a donc,
pour $ x \in E \diagdown\0\
\[
lim_{t\rightarrow 0},t\textgreater{}0L_1~(tx)
- L_2(tx)\over
\tx\ = 0
\]

Mais pour $t \textgreater{} 0$, on a
$L_1(tx)-L_2(tx)\over
\tx\ =
L_1(x)-L_2(x)\over
\x$; ceci montre
que $L_1(x) = L_2(x)$ et donc $L_1 =
L_2$.

Remarque~15.2.1 Pour alléger les notations, on écrira df(a).h à la place
de \big {[}df(a)\big {]}(h). On a donc par
définition f(a + h) = f(a) + df(a).h +
o(\h\) ou encore f(a +
h) = f(a) + df(a).h +\
h\\epsilon(h) avec
lim_h\rightarrow~0~\epsilon(h) = 0.

Remarque~15.2.2 Si E est de dimension finie, une application linéaire de
E dans F est automatiquement continue. Il est clair d'autre part que la
différentiabilité est une notion locale et que le changement des normes
sur E et F en normes équivalentes ne change ni la différentiabilité, ni
la différentielle.

Proposition~15.2.1 Si f est différentiable au point a, elle est continue
au point a.

Démonstration On a f(a + h) = f(a) + df(a).h +\
h\\epsilon(h) avec
lim_h\rightarrow~0~\epsilon(h) = 0. Comme df(a) est une
application linéaire continue, on a
$lim_h\rightarrow~0~df(a).h = df(a).0 = 0$ et donc
$lim_h\rightarrow~0~f(a + h) = f(a)$.

\paragraph{15.2.2 Exemples d'applications différentiables}

Proposition~15.2.2 Soit E et F deux espaces vectoriels normés, u une
application linéaire continue de E dans F. Alors u est différentiable en
tout point a de E et du(a) = u.

Démonstration On a en effet u(a + h) = u(a) + u(h) + 0.

Proposition~15.2.3 Soit E, F et G trois espaces vectoriels normés, u : E
\times F \rightarrow~ G une application bilinéaire continue. Alors f est différentiable
en tout point (a,b) de E \times F et du(a,b).(h,k) = u(a,k) + u(h,b).

Démonstration On a u((a,b) + (h,k)) = u(a + h,b + k) = u(a,b) +
\left (u(a,k) + u(h,b)\right ) + u(h,k).
Mais comme u est bilinéaire continue, il existe une constante A telle
que $\u(h,k)\ \leq Ah$\\k 
soit encore$ \u(h,k)\ \leq
Amax(\h\,\k\)^2~
=
A(h,k)^2.$
On a donc u((a,b) + (h,k)) = u(a,b) + \left (u(a,k) +
u(h,b)\right ) +
O((h,k)^2).
Comme (h,k)\mapsto~u(a,k) + u(h,b) est clairement
linéaire et continue, c'est la différentielle de u au point (a,b).

Exemple~15.2.1 Tous les produits usuels (produit dans K, produits
scalaires, produits vectoriels, produits matriciels) sont donc
différentiables en tout point.

\paragraph{15.2.3 Opérations sur les différentielles}

Proposition~15.2.4 Soit E et F deux espaces vectoriels normés, U un
ouvert de E, a \in U et f,g : U \rightarrow~ F. Si f et g sont différentiables en a,
il en est de même pour $\alpha~f + \beta~g$ et $d(\alpha~f + \beta~g)(a) = \alpha~df(a) + \beta~dg(a)$ .

Démonstration On a $ f(a + h) = f(a) + df(a).h +
o(\h\)$ et g(a + h) =
g(a) + dg(a).h +
o(\h\), d'où (\alpha~f +
\beta~g)(a + h) = (\alpha~f + \beta~g)(a) + \alpha~df(a).h + \beta~dg(a).h +
o(\h\) avec \alpha~df(a) +
\beta~dg(a) application linéaire continue.

Théorème~15.2.5 Soit E, F et G trois espaces vectoriels normés, U un
ouvert de E, V un ouvert de F, f : U \rightarrow~ F tel que f(U) \subset~ V et g : V \rightarrow~ G.
Soit a \in U. Si f est différentiable au point a et g différentiable au
point f(a), alors g \cdot f est différentiable au point a et d(g \cdot f)(a) =
dg\left (f(a)\right ) \cdot df(a).

Démonstration On a, en posant b = f(a), f(a + h) = f(a) + df(a).h
+\ h\\epsilon(h) avec
lim_h\rightarrow~0~\epsilon(h) = 0 et g(b + k) = g(b) +
dg(b).k +\ k\\eta(k) avec
lim_k\rightarrow~0~\eta(k) = 0. Prenons en
particulier

k = \phi(h) = f(a + h) - f(a) = df(a).h +\
h\\epsilon(h)

On a b + k = f(a + h), et donc

g(f(a + h)) = g(f(a)) + dg(b).\phi(h) +\
\phi(h)\\eta(\phi(h))

Mais on a

\begin{align*} dg(b).\phi(h)& =&
dg(b).\left (df(a).h +\
h\\epsilon(h)\right ) \%&
\\ & =& dg(b) \cdot df(a).h
+\ h\dg(b).\epsilon(h)\%&
\\ & =& dg(b) \cdot df(a).h +
o(\h\) \%&
\\ \end{align*}

puisque lim_h\rightarrow~0~dg(b).\epsilon(h) = dg(b).0 =
0. D'autre part,

\begin{align*}
\\phi(h)& \leq&
\df(a).h +\
h\\epsilon(h)\ \%&
\\ & \leq&
\df(a)\\,\h\
+\
\epsilon(h)\\,\h\
= O(\h\)\%&
\\ \end{align*}

donc \\phi(h)\\eta(\phi(h)) =
o(\h\) puisque
lim_h\rightarrow~0~\phi(h) = 0 (continuité de f au
point a) et donc lim_h\rightarrow~0~\eta(\phi(h)) = 0.
On a donc en définitive, g(f(a + h)) = g(f(a)) + dg(b) \cdot df(a).h +
o(\h\) ce qui termine
la démonstration.

Remarque~15.2.3 En particulier, si u est une application linéaire
continue et si f est différentiable, u \cdot f est différentiable et d(u \cdot
f)(a) = u \cdot df(a).

\paragraph{15.2.4 Différentielle et dérivées partielles}

Regardons tout d'abord le cas des fonctions d'une variable. On a un
résultat très simple qui montre que la notion de différentiabilité est
une généralisation de la notion de dérivabilité.

Théorème~15.2.6 Soit U un ouvert de \mathbb{R}~, a \in U et F un K-espace vectoriel
normé. Soit f : U \rightarrow~ F. Alors f est différentiable au point a si et
seulement si~elle est dérivable au point a et on a f'(a) = df(a).1 et
df(a).h = hf'(a).

Démonstration Si f est dérivable au point a, on a f(a + h) = f(a) +
hf'(a) + o(h), ce qui montre que f est différentiable en a et que
df(a).h = hf'(a). Inversement si f est différentiable au point a, on a
f(a + h) = f(a) + df(a).h + o(h) = f(a) + hdf(a).1 + o(h) (car h est un
réel), soit encore lim_h\rightarrow~0~
f(a+h)-f(a) \over h = df(a).1, ce qui montre que f est
dérivable au point 1 et que f'(a) = df(a).1.

Exemple~15.2.2 Soit U un ouvert de \mathbb{R}~, V un ouvert de E, soit \phi : U \rightarrow~ E
telle que \phi(U) \subset~ V et f : V \rightarrow~ F. Soit a \in U. Supposons que \phi est
dérivable (donc différentiable) au point a et que f est différentiable
au point \phi(a). Alors f \cdot \phi est différentiable (donc dérivable) au point
a et (f \cdot \phi)'(a) = d(f \cdot \phi)(a).1 = df(\phi(a)) \cdot d\phi(a).1 = df(\phi(a)).\phi'(a).
On retiendra donc la formule importante (f \cdot \phi)'(a) = df(\phi(a)).\phi'(a).

En ce qui concerne les fonctions de plusieurs variables, le lien entre
différentiabilité et dérivée partielle est plus complexe puisque l'on a
vu que l'existence de dérivées partielles n'impliquait même pas la
continuité, et donc certainement pas la différentiabilité. On a le
résultat suivant

Théorème~15.2.7 Soit E et F deux espaces vectoriels normés de dimensions
finies, U un ouvert de E et f : U \rightarrow~ F. (i) si f est différentiable au
point a \in U, alors f admet en a une dérivée partielle suivant tout
vecteur v et \partial_vf(a) = df(a).v (ii) inversement, si E =
\mathbb{R}~^n et si f est de classe \mathcal{C}^1 sur U, alors f est
différentiable en tout point a de U et df(a).h
= \\sum ~
_i=1^nh_i \partial~f \over
\partial~x_i (a).

Démonstration (i) On a f(a + h) = f(a) + df(a).h
+\ h\\epsilon(h), soit encore
f(a + tv) = f(a) + tdf(a).v +
t\,\v\\epsilon(tv),
c'est-à-dire lim_t\rightarrow~0~ f(a+tv)-f(a)
\over t = df(a).v, donc f admet en a une dérivée
partielle suivant v et \partial_vf(a) = df(a).v.

(ii) La formule de Taylor Young à l'ordre 1 montre en effet que f(a + h)
= f(a) + \\sum ~
_i=1^nh_i \partial~f \over
\partial~x_i (a) +
o(\h\) ce qui montre
que f est différentiable en a et que df(a).h =\
\sum  _i=1^nh_i~ \partial~f
\over \partial~x_i (a).

Remarque~15.2.4 Si E = \mathbb{R}~^n, et si f est différentiable au
point a, on a nécessairement

\begin{align*} df(a).h& =&
df(a).(\sum _i=1^nh_
ie_i) = \\sum
_i=1^nh_ idf(a).e_i\%&
\\ & =& \\sum
_i=1^nh_ i\partial_e_if(a) =
\sum _i=1^nh_ i~ \partial~f
\over \partial~x_i (a) \%&
\\ \end{align*}

Donc en fait montrer la différentiabilité de f en a, c'est montrer que
les  \partial~f \over \partial~x_i (a) existent et que f(a +
h) - f(a) -\\sum ~
_i=1^nh_i \partial~f \over
\partial~x_i (a) =
o(\h\).

\paragraph{15.2.5 Matrices \\jmathmathacobiennes, \\jmathmathacobiens}

Définition~15.2.2 Soit U un ouvert de \mathbb{R}~^n et f : U \rightarrow~
\mathbb{R}~^p. Soit a \in U tel que f soit différentiable au point a. On
appelle matrice \\jmathmathacobienne de f au point a la matrice J_f(a) de
l'application linéaire df(a) dans les bases canoniques de \mathbb{R}~^n
et \mathbb{R}~^p. Si
f(x_1,\\ldots,x_n~)
=
(f_1(x_1,\\ldots,x_n),\\\ldots,f_p(x_1,\\\ldots,x_n~)),
c'est la matrice

\begin{align*} J_f(a) =
\left ( \partial~f_i \over
\partial~x_\\jmathmath (a)\right )_ 1\leqi\leqp
\atop 1\leq\\jmathmath\leqn  = \left
(\matrix\, \partial~f_1
\over \partial~x_1
(a)&\\ldots~&
\partial~f_1 \over \partial~x_n (a)
\cr
\\ldots~
&\\ldots&\\\ldots~
\cr  \partial~f_p \over \partial~x_1
(a)&\\ldots~&
\partial~f_p \over \partial~x_n
(a)\right ) \in M_\mathbb{R}~(p,n)& & \%&
\\ \end{align*}

Démonstration Il faut en effet mettre dans la \\jmathmath-ième colonne de
J_f(a) les coordonnées du vecteur df(a).e_\\jmathmath =
\partial_e_\\jmathmathf(a) = \partial~f \over \partial~x_\\jmathmath
(a) = ( \partial~f_1 \over \partial~x_\\jmathmath
(a),\\ldots~,
\partial~f_p \over \partial~x_\\jmathmath (a)).

Le théorème de composition des applications différentiables va ainsi se
traduire de la manière suivante sur les matrices \\jmathmathacobiennes

Définition~15.2.3 Soit U un ouvert de \mathbb{R}~^n, V un ouvert de
\mathbb{R}~^p, f : U \rightarrow~ \mathbb{R}~^p telle que f(U) \subset~ V et g : V \rightarrow~
\mathbb{R}~^q. Soit a \in U tel que f soit différentiable au point a et g
différentiable au point f(a). Alors on a J_g\cdotf(a) =
J_g(f(a))J_f(a).

Démonstration La matrice de la composée de deux applications linéaires
est le produit des matrices de ces applications linéaires dans des bases
adéquates (ici les bases canoniques).

Remarque~15.2.5 Utilisons alors les formules donnant le produit de deux
matrices. On va ainsi obtenir

 \left ( \partial~g \cdot f \over \partial~x_\\jmathmath
(a)\right )_i = \partial~g_i \cdot f
\over \partial~x_\\jmathmath (a) = \\sum
_k=1^p \partial~g_i \over
\partial~y_k (f(a)) \partial~f_k \over
\partial~x_\\jmathmath (a)

ce qui n'est autre que la formule trouvée dans la première section pour
les dérivées partielles d'une fonction composée, mais avec des
hypothèses plus faibles (la condition g est \mathcal{C}^1 a été
remplacée par g est différentiable au point a).

Définition~15.2.4 Soit U un ouvert de \mathbb{R}~^n et f : U \rightarrow~
\mathbb{R}~^n. Soit a \in U tel que f soit différentiable au point a. On
appelle \\jmathmathacobien de f au point a le nombre réel

\\jmathmath_f(a) =\
\mathrm{det} J_f(a) = \left
\matrix\, \partial~f_1
\over \partial~x_1
(a)&\\ldots~&
\partial~f_1 \over \partial~x_n (a)
\cr \⋮~
&\\ldots&\\⋮~
\cr  \partial~f_n \over \partial~x_1
(a)&\\ldots~&
\partial~f_n \over \partial~x_n
(a)\right 

Remarque~15.2.6 Soit U un ouvert de \mathbb{R}~^n, V un ouvert de
\mathbb{R}~^n, f : U \rightarrow~ \mathbb{R}~^n telle que f(U) \subset~ V et g : V \rightarrow~
\mathbb{R}~^n. Soit a \in U tel que f soit différentiable au point a et g
différentiable au point f(a). Alors on a \\jmathmath_g\cdotf(a) =
\\jmathmath_g(f(a))\\jmathmath_f(a).

\paragraph{15.2.6 Inégalité des accroissements finis}

Théorème~15.2.8 Soit E et F deux espaces vectoriels normés, U un ouvert
de E et f : U \rightarrow~ F. Soit a,b \in U tels que {[}a,b{]} \subset~ U. On suppose que f
est différentiable en tout point x de {[}a,b{]} et que, pour tout x \in
{[}a,b{]}, la norme de l'application linéaire df(x) est ma\\jmathmathorée par M ≥
0. Alors

\f(b) - f(a)\ \leq
M\b - a\

Démonstration Considérons \phi : {[}0,1{]} \rightarrow~ F définie par \phi(t) = f((1 -
t)a + tb). On a \phi'(t) = df((1 - t)a + tb). d \over dt
((1 - t)a + tb) = df((1 - t)a + tb).(b - a). On en déduit que
\forall~~t \in {[}a,b{]},
\\phi'(t)\
\leq\ df((1 - t)a +
tb)\.\b -
a\ \leq M\b -
a\. L'inégalité des accroissements finis pour
les fonctions d'une variable donne alors \\phi(1)
- \phi(0)\ \leq M\b -
a\(1 - 0) = M\b -
a\, ce qu'il fallait démontrer.

Remarque~15.2.7 Cette inégalité des accroissements finis a des
conséquences similaires à celles de l'inégalité des accroissements finis
pour les fonctions d'une variable (en prenant soin de respecter la
condition restrictive~: {[}a,b{]} \subset~ U)~; parmi les plus importantes
citons celle là

Corollaire~15.2.9 Soit E et F deux espaces vectoriels normés, U un
ouvert convexe de E et f : U \rightarrow~ F une application différentiable telle
que \forall~~x \in U,
\df(x)\ \leq M. Alors f
est M-lipschitzienne.

% \documentclass[]{article}
\usepackage[T1]{fontenc}
\usepackage{lmodern}
\usepackage{amssymb,amsmath}
\usepackage{ifxetex,ifluatex}
\usepackage{fixltx2e} % provides \textsubscript
% use upquote if available, for straight quotes in verbatim environments
\IfFileExists{upquote.sty}{\usepackage{upquote}}{}
\ifnum 0\ifxetex 1\fi\ifluatex 1\fi=0 % if pdftex
  \usepackage[utf8]{inputenc}
\else % if luatex or xelatex
  \ifxetex
    \usepackage{mathspec}
    \usepackage{xltxtra,xunicode}
  \else
    \usepackage{fontspec}
  \fi
  \defaultfontfeatures{Mapping=tex-text,Scale=MatchLowercase}
  \newcommand{\euro}{€}
\fi
% use microtype if available
\IfFileExists{microtype.sty}{\usepackage{microtype}}{}
\ifxetex
  \usepackage[setpagesize=false, % page size defined by xetex
              unicode=false, % unicode breaks when used with xetex
              xetex]{hyperref}
\else
  \usepackage[unicode=true]{hyperref}
\fi
\hypersetup{breaklinks=true,
            bookmarks=true,
            pdfauthor={},
            pdftitle={Formes differentielles},
            colorlinks=true,
            citecolor=blue,
            urlcolor=blue,
            linkcolor=magenta,
            pdfborder={0 0 0}}
\urlstyle{same}  % don't use monospace font for urls
\setlength{\parindent}{0pt}
\setlength{\parskip}{6pt plus 2pt minus 1pt}
\setlength{\emergencystretch}{3em}  % prevent overfull lines
\setcounter{secnumdepth}{0}
 
/* start css.sty */
.cmr-5{font-size:50%;}
.cmr-7{font-size:70%;}
.cmmi-5{font-size:50%;font-style: italic;}
.cmmi-7{font-size:70%;font-style: italic;}
.cmmi-10{font-style: italic;}
.cmsy-5{font-size:50%;}
.cmsy-7{font-size:70%;}
.cmex-7{font-size:70%;}
.cmex-7x-x-71{font-size:49%;}
.msbm-7{font-size:70%;}
.cmtt-10{font-family: monospace;}
.cmti-10{ font-style: italic;}
.cmbx-10{ font-weight: bold;}
.cmr-17x-x-120{font-size:204%;}
.cmsl-10{font-style: oblique;}
.cmti-7x-x-71{font-size:49%; font-style: italic;}
.cmbxti-10{ font-weight: bold; font-style: italic;}
p.noindent { text-indent: 0em }
td p.noindent { text-indent: 0em; margin-top:0em; }
p.nopar { text-indent: 0em; }
p.indent{ text-indent: 1.5em }
@media print {div.crosslinks {visibility:hidden;}}
a img { border-top: 0; border-left: 0; border-right: 0; }
center { margin-top:1em; margin-bottom:1em; }
td center { margin-top:0em; margin-bottom:0em; }
.Canvas { position:relative; }
li p.indent { text-indent: 0em }
.enumerate1 {list-style-type:decimal;}
.enumerate2 {list-style-type:lower-alpha;}
.enumerate3 {list-style-type:lower-roman;}
.enumerate4 {list-style-type:upper-alpha;}
div.newtheorem { margin-bottom: 2em; margin-top: 2em;}
.obeylines-h,.obeylines-v {white-space: nowrap; }
div.obeylines-v p { margin-top:0; margin-bottom:0; }
.overline{ text-decoration:overline; }
.overline img{ border-top: 1px solid black; }
td.displaylines {text-align:center; white-space:nowrap;}
.centerline {text-align:center;}
.rightline {text-align:right;}
div.verbatim {font-family: monospace; white-space: nowrap; text-align:left; clear:both; }
.fbox {padding-left:3.0pt; padding-right:3.0pt; text-indent:0pt; border:solid black 0.4pt; }
div.fbox {display:table}
div.center div.fbox {text-align:center; clear:both; padding-left:3.0pt; padding-right:3.0pt; text-indent:0pt; border:solid black 0.4pt; }
div.minipage{width:100%;}
div.center, div.center div.center {text-align: center; margin-left:1em; margin-right:1em;}
div.center div {text-align: left;}
div.flushright, div.flushright div.flushright {text-align: right;}
div.flushright div {text-align: left;}
div.flushleft {text-align: left;}
.underline{ text-decoration:underline; }
.underline img{ border-bottom: 1px solid black; margin-bottom:1pt; }
.framebox-c, .framebox-l, .framebox-r { padding-left:3.0pt; padding-right:3.0pt; text-indent:0pt; border:solid black 0.4pt; }
.framebox-c {text-align:center;}
.framebox-l {text-align:left;}
.framebox-r {text-align:right;}
span.thank-mark{ vertical-align: super }
span.footnote-mark sup.textsuperscript, span.footnote-mark a sup.textsuperscript{ font-size:80%; }
div.tabular, div.center div.tabular {text-align: center; margin-top:0.5em; margin-bottom:0.5em; }
table.tabular td p{margin-top:0em;}
table.tabular {margin-left: auto; margin-right: auto;}
div.td00{ margin-left:0pt; margin-right:0pt; }
div.td01{ margin-left:0pt; margin-right:5pt; }
div.td10{ margin-left:5pt; margin-right:0pt; }
div.td11{ margin-left:5pt; margin-right:5pt; }
table[rules] {border-left:solid black 0.4pt; border-right:solid black 0.4pt; }
td.td00{ padding-left:0pt; padding-right:0pt; }
td.td01{ padding-left:0pt; padding-right:5pt; }
td.td10{ padding-left:5pt; padding-right:0pt; }
td.td11{ padding-left:5pt; padding-right:5pt; }
table[rules] {border-left:solid black 0.4pt; border-right:solid black 0.4pt; }
.hline hr, .cline hr{ height : 1px; margin:0px; }
.tabbing-right {text-align:right;}
span.TEX {letter-spacing: -0.125em; }
span.TEX span.E{ position:relative;top:0.5ex;left:-0.0417em;}
a span.TEX span.E {text-decoration: none; }
span.LATEX span.A{ position:relative; top:-0.5ex; left:-0.4em; font-size:85%;}
span.LATEX span.TEX{ position:relative; left: -0.4em; }
div.float img, div.float .caption {text-align:center;}
div.figure img, div.figure .caption {text-align:center;}
.marginpar {width:20%; float:right; text-align:left; margin-left:auto; margin-top:0.5em; font-size:85%; text-decoration:underline;}
.marginpar p{margin-top:0.4em; margin-bottom:0.4em;}
.equation td{text-align:center; vertical-align:middle; }
td.eq-no{ width:5%; }
table.equation { width:100%; } 
div.math-display, div.par-math-display{text-align:center;}
math .texttt { font-family: monospace; }
math .textit { font-style: italic; }
math .textsl { font-style: oblique; }
math .textsf { font-family: sans-serif; }
math .textbf { font-weight: bold; }
.partToc a, .partToc, .likepartToc a, .likepartToc {line-height: 200%; font-weight:bold; font-size:110%;}
.chapterToc a, .chapterToc, .likechapterToc a, .likechapterToc, .appendixToc a, .appendixToc {line-height: 200%; font-weight:bold;}
.index-item, .index-subitem, .index-subsubitem {display:block}
.caption td.id{font-weight: bold; white-space: nowrap; }
table.caption {text-align:center;}
h1.partHead{text-align: center}
p.bibitem { text-indent: -2em; margin-left: 2em; margin-top:0.6em; margin-bottom:0.6em; }
p.bibitem-p { text-indent: 0em; margin-left: 2em; margin-top:0.6em; margin-bottom:0.6em; }
.paragraphHead, .likeparagraphHead { margin-top:2em; font-weight: bold;}
.subparagraphHead, .likesubparagraphHead { font-weight: bold;}
.quote {margin-bottom:0.25em; margin-top:0.25em; margin-left:1em; margin-right:1em; text-align:\\jmathmathustify;}
.verse{white-space:nowrap; margin-left:2em}
div.maketitle {text-align:center;}
h2.titleHead{text-align:center;}
div.maketitle{ margin-bottom: 2em; }
div.author, div.date {text-align:center;}
div.thanks{text-align:left; margin-left:10%; font-size:85%; font-style:italic; }
div.author{white-space: nowrap;}
.quotation {margin-bottom:0.25em; margin-top:0.25em; margin-left:1em; }
h1.partHead{text-align: center}
.sectionToc, .likesectionToc {margin-left:2em;}
.subsectionToc, .likesubsectionToc {margin-left:4em;}
.subsubsectionToc, .likesubsubsectionToc {margin-left:6em;}
.frenchb-nbsp{font-size:75%;}
.frenchb-thinspace{font-size:75%;}
.figure img.graphics {margin-left:10%;}
/* end css.sty */

\title{Formes differentielles}
\author{}
\date{}

\begin{document}
\maketitle

\textbf{Warning: 
requires JavaScript to process the mathematics on this page.\\ If your
browser supports JavaScript, be sure it is enabled.}

\begin{center}\rule{3in}{0.4pt}\end{center}

{[}
{[}
{[}{]}
{[}

\subsubsection{15.3 Formes différentielles}

Remarque~15.3.1 En dehors de la notion de gradient, cette section ne
fait pas partie du programme des classes préparatoires. Cependant, les
formes différentielles de degré 1 sont un outil particulièrement commode
même à ce niveau.

\paragraph{15.3.1 Rappels sur les formes linéaires alternées}

Proposition~15.3.1 Soit E un \mathbb{R}~-espace vectoriel,
f_1,\\ldots,f_p~
\in E^∗. Alors f_1
∧\\ldots~ ∧
f_p : E^p \rightarrow~ K définie par
(x_1,\\ldots,x_p)\mapsto~\\mathrm{det}~
(f_i(x_\\jmathmath))_1\leqi\leqp,1\leq\\jmathmath\leqp est une forme p-
linéaire alternée sur E. L'application (E^∗)^p \rightarrow~
A_p(E),
(f_1,\\ldots,f_p)\mapsto~f_1~
∧\\ldots~ ∧
f_p est elle même p-linéaire et alternée.

Ceci permet d'exhiber une base de l'espace A_p(E) des formes
p-linéaires alternées sur E. Pour cela soit E un K-espace vectoriel de
dimension n et
(e_1,\\ldots,e_n~)
une base de E.

Théorème~15.3.2 La famille des
(e_i_1^∗∧\\ldots~
∧
e_i_p^∗)_1\leqi_1\textless{}i_2\textless{}\\ldots\textless{}i_p\leqn~
est une base de A_p(E) (qui est donc de dimension
C_n^p).

\paragraph{15.3.2 Notion de forme différentielle}

Définition~15.3.1 Soit U un ouvert de \mathbb{R}~^n. On appelle forme
différentielle de degré p sur U toute application de U dans
A_p(\mathbb{R}~^n) (en posant par convention
A_0(\mathbb{R}~^n) = \mathbb{R}~).

Remarque~15.3.2 Soit \omega : U \rightarrow~ A_p(\mathbb{R}~^n) une forme
différentielle de degré p. Soit
(e_1,\\ldots,e_n~)
la base canonique de \mathbb{R}~^n et
(e_i_1^∗∧\\ldots~
∧
e_i_p^∗)_1\leqi_1\textless{}i_2\textless{}\\ldots\textless{}i_p\leqn~
la base correspondante de A_p(\mathbb{R}~^n). On a alors, pour
x \in U, \omega(x) = \\sum ~
_1\leqi_1\textless{}i_2\textless{}\\ldots\textless{}i_p\leqna_i_1,\\\ldots,i_p(x)e_i_1^∗∧\\\ldots~
∧ e_i_p^∗. On dit que \omega est de classe
C^k si toutes les applications
a_i_1,\\ldots,i_p~
: U \rightarrow~ \mathbb{R}~ sont de classe C^k.

Remarque~15.3.3 Soit f : U \rightarrow~ \mathbb{R}~ de classe \mathcal{C}^1. Alors pour tout
x \in U, df(x) est une application linéaire de \mathbb{R}~^n dans \mathbb{R}~ donc
une forme linéaire sur \mathbb{R}~^n, donc un élément de
(\mathbb{R}~^n)^∗ = A_1(E). On en déduit que df :
x\mapsto~df(x) est une forme différentielle de degré
1 sur U. On sait que

df(x).h = \sum _i=1^n~ \partial~f
\over \partial~x_i (x)h_i =
\sum _i=1^n~ \partial~f
\over \partial~x_i (x)e_i^∗(h)

On en déduit que df(x) =\
\sum  _i=1^n~ \partial~f
\over \partial~x_i (x)e_i^∗. Prenons
par exemple f = e_i^∗. On a \forall~~x
\in U, df(x) = e_i^∗. Si on note x_i,
l'application i-ième coordonnée (c'est-à-dire encore
e_i^∗), on a donc dx_i = e_i^∗
si bien que l'on peut noter df(x) =\
\sum  _i=1^n~ \partial~f
\over \partial~x_i (x)dx_i. Plus
généralement, une forme différentielle de degré p sur U sera de la forme

\omega(x) = \\sum
_1\leqi_1\textless{}i_2\textless{}\ldots\textless{}i_p\leqna_i_1,\\ldots,i_p(x)dx_i_1~
∧\ldots ∧ dx_i_p~

C'est cette dernière forme que nous utiliserons par la suite, avec comme
seule propriété à connaître le fait que ∧ est multilinéaire et alternée.

Exemple~15.3.1 Dans le cas de la dimension 3 et de p = 2, on préfère
utiliser une base invariante par permutation circulaire, à savoir
dx_2 ∧ dx_3,dx_3 ∧
dx_1,dx_1 ∧ dx_2. On aura ainsi les
expressions générales de formes différentielles de degré p sur un ouvert
de \mathbb{R}~^n.

p = 0~: dans tous les cas, une forme différentielle de degré 0 est
simplement une fonction à valeurs réelles et une forme différentielle de
degré 1 s'écrit

\omega(x_1,\\ldots,x_n~))
=
a_1(x_1,\\ldots,x_n)dx_1~
+ \\ldots~ +
a_n(x_1,\\ldots,x_n)dx_n~

\begin{align*} n = 2,p = 2& :&
\omega(x_1,x_2) = a(x_1,x_2)dx_1
∧ dx_2 \%& \\ n = 3,p = 2& :&
\omega(x_1,x_2,x_3) =
a_1(x_1,x_2,x_3)dx_2 ∧
dx_3 \%& \\ & &
+a_2(x_1,x_2,x_3)dx_3 ∧
dx_1 +
a_3(x_1,x_2,x_3)dx_1 ∧
dx_2\%& \\ n = 3,p = 3& :&
\omega(x_1,x_2,x_3) =
a(x_1,x_2,x_3)dx_1 ∧ dx_2 ∧
dx_3 \%& \\
\end{align*}

\paragraph{15.3.3 Notion de gradient d'une fonction}

Soit E un espace euclidien, U un ouvert de E et f : U \rightarrow~ \mathbb{R}~ de classe
\mathcal{C}^1. Alors, pour x \in E, df(x) est une forme linéaire sur E~;
on sait qu'il existe un unique vecteur noté
grad~f(x) dans E tel que
\forall~~h \in E, df(x).h =
(gradf(x)\mathrel∣~h).

Définition~15.3.2 Le vecteur grad~f(x) défini
par \forall~~h \in E, df(x).h =
(gradf(x)\mathrel∣~h), est
appelé gradient de f au point x.

Remarque~15.3.4 Supposons que E = \mathbb{R}~^n muni de sa structure
euclidienne naturelle (celle qui rend la base canonique orthonormée).
Alors

df(x).h = \sum _i=1^nh_
i \partial~f \over \partial~x_i (x) =
(\sum _i=1^n~ \partial~f
\over \partial~x_i
(x)e_i∣h)

si bien que l'on retrouve l'expression classique du gradient de f

grad~f(x) = \\sum
_i=1^n \partial~f \over \partial~x_i
(x)e_i = ( \partial~f \over \partial~x_1
(x),\ldots~, \partial~f \over
\partial~x_n (x))

\paragraph{15.3.4 Invariance de la différentielle}

Soit U un ouvert de \mathbb{R}~^p et f : U \rightarrow~ \mathbb{R}~ de classe
\mathcal{C}^1. Soit V un ouvert de \mathbb{R}~^n et soit \phi =
(\phi_1,\\ldots,\phi_p~)
: V \rightarrow~ U. Posons y_1 =
\phi_1(x_1,\\ldots,x_n),\\\ldots,y_p~
=
\phi_p(x_1,\\ldots,x_n~).
On a donc dy_\\jmathmath =\
\sum  _i=1^n \partial~\phi_\\jmathmath~
\over \partial~x_i dx_i. De plus,
f(y_1,\\ldots,y_p~)
=
f(\phi_1(x_1,\\ldots,x_n),\\\ldots,\phi_p(x_1,\\\ldots,x_n~))
si bien que

\begin{align*}
d(f(y_1,\\ldots,y_p~))&&
\%& \\ & =& \\sum
_i=1^n \partial~ \over \partial~x_i
\left
(f(\phi_1(x_1,\ldots,x_n),\\ldots,\phi_p(x_1,\\ldots,x_n~))\right
)dx_i \%& \\ & =&
\sum _i=1^n~\left
(\sum _\\jmathmath=1^p~ \partial~f
\over \partial~y_\\jmathmath
(\phi_1(x_1,\ldots,x_n),\\ldots,\phi_p(x_1,\\ldots,x_n~))
\partial~\phi_\\jmathmath \over \partial~x_i \right
)dx_i\%& \\ & =&
\sum _\\jmathmath=1^p~ \partial~f
\over \partial~y_\\jmathmath
(y_1,\ldots,y_p~)\left
(\sum _i=1^n \partial~\phi_\\jmathmath~
\over \partial~x_i dx_i\right
) \%& \\ & =&
\sum _\\jmathmath=1^p~ \partial~f
\over \partial~y_\\jmathmath
(y_1,\ldots,y_p)dy_\\jmathmath~
\%& \\ \end{align*}

en utilisant la règle de dérivation partielle des fonctions composées et
en intervertissant les deux sommations.

On voit donc que la formule
d(f(y_1,\\ldots,y_p~))
= \\sum ~
_\\jmathmath=1^p \partial~f \over \partial~y_\\jmathmath
(y_1,\\ldots,y_p)dy_\\jmathmath~
est valable aussi bien quand
y_1,\\ldots,y_p~
désignent des variables libres (c'est-à-dire qui varient dans un ouvert
de \mathbb{R}~^p) que lorsque
y_1,\\ldots,y_p~
désignent des fonctions d'autres variables (ici
x_1,\\ldots,x_n~).
C'est une propriété essentielle de la différentielle qui fait tout
l'intérêt des formes différentielles (en particulier de degré 1)~: on
peut différentier une expression sans savoir quelles sont les variables
et quelles sont les fonctions.

On prendra simplement garde au fait suivant~: lorsque
y_1,\\ldots,y_p~
désignent des variables libres, qui varient dans des ouverts de
\mathbb{R}~^p, on a dy_\\jmathmath = e_\\jmathmath^∗, et donc les
formes différentielles
dy_1,\\ldots,dy_p~
forment une famille libre (ce qui permet en particulier des
identifications)~; il n'en est évidemment plus de même lorsque
y_1,\\ldots,y_p~
sont elles mêmes des fonctions d'autres variables
x_1,\\ldots,x_n~.

\paragraph{15.3.5 Différentielle extérieure}

Définition~15.3.3 Soit \omega(x) =\
\sum ~
_1\leqi_1\textless{}i_2\textless{}\\ldots\textless{}i_p\leqna_i_1,\\\ldots,i_p(x)dx_i_1~
∧\\ldots~ ∧
dx_i_p une forme différentielle de degré p, de classe
\mathcal{C}^1 sur l'ouvert U de \mathbb{R}~^n. On appelle
différentielle extérieure de \omega, la forme différentielle de degré p + 1
définie par

d\omega(x) = \\sum
_1\leqi_1\textless{}i_2\textless{}\ldots\textless{}i_p\leqnda_i_1,\\ldots,i_p~(x)
∧ dx_i_1 ∧\ldots~ ∧
dx_i_p

Remarque~15.3.5 Le calcul effectif se fait en utilisant la définition de
da_i_1,\\ldots,i_p~(x)
et les propriétés de l'opérateur ∧~: linéaire par rapport à chaque
terme, alterné, antisymétrique. On a donc

\begin{align*}
da_i_1,\\ldots,i_p~(x)
∧ dx_i_1
∧\\ldots~ ∧
dx_i_p&& \%& \\ &
=& \left (\\sum
_\\jmathmath=1^n
\partial~a_i_1,\ldots,i_p~
\over \partial~x_\\jmathmath
\,dx_\\jmathmath\right ) ∧
dx_i_1
∧\\ldots~ ∧
dx_i_p\%& \\ & =&
\sum _\\jmathmath=1^n~
\partial~a_i_1,\ldots,i_p~
\over \partial~x_\\jmathmath \,dx_\\jmathmath ∧
dx_i_1 ∧\ldots~ ∧
dx_i_p \%& \\
\end{align*}

avec dx_\\jmathmath ∧ dx_i_1
∧\\ldots~ ∧
dx_i_p = 0 si \\jmathmath
\in\i_1,\\ldots,i_p\~
et dx_\\jmathmath ∧ dx_i_1
∧\\ldots~ ∧
dx_i_p = ±dx_i_1
∧\\ldots~ ∧
dx_\\jmathmath
∧\\ldots~ ∧
dx_i_p où l'on met de signe + ou le signe - suivant la
parité du nombre de transpositions nécessaires pour intercaler \\jmathmath à la
bonne place dans la suite
\i_1,\\ldots,i_p\~.
Dans le cas d'une forme différentielle de degré 0 (une fonction), on
trouve bien entendu tout simplement la différentielle de la fonction.

Exemple~15.3.2 Calcul dans le cas n = 3. Si p = 0, on a \omega = f et d\omega =
\partial~f \over \partial~x_1 dx_1 + \partial~f
\over \partial~x_2 dx_2 + \partial~f
\over \partial~x_3 dx_3 et on retrouve
l'expression du gradient de la fonction f.

Si p = 1, on a \omega(x) = a_1(x)dx_1 +
a_2(x)dx_2 + a_3(x)dx_3, et donc

\begin{align*} d\omega(x)& =& da_1(x) ∧
dx_1 + da_2(x) ∧ dx_2 + da_3(x) ∧
dx_3 \%& \\ & =& (
\partial~a_1 \over \partial~x_1 (x)dx_1 +
\partial~a_1 \over \partial~x_2 (x)dx_2 +
\partial~a_1 \over \partial~x_3 (x)dx_3) ∧
dx_1 \%& \\ & & +(
\partial~a_2 \over \partial~x_1 (x)dx_1 +
\partial~a_2 \over \partial~x_2 (x)dx_2 +
\partial~a_2 \over \partial~x_3 (x)dx_3) ∧
dx_2\%& \\ & & +(
\partial~a_3 \over \partial~x_1 (x)dx_1 +
\partial~a_3 \over \partial~x_2 (x)dx_2 +
\partial~a_3 \over \partial~x_3 (x)dx_3) ∧
dx_3\%& \\ & =& (
\partial~a_3 \over \partial~x_2 (x) - \partial~a_2
\over \partial~x_3 (x))dx_2 ∧ dx_3
\%& \\ & & +( \partial~a_1
\over \partial~x_3 (x) - \partial~a_3
\over \partial~x_1 (x))dx_3 ∧ dx_1
\%& \\ & & +( \partial~a_2
\over \partial~x_1 (x) - \partial~a_1
\over \partial~x_2 (x))dx_1 ∧ dx_2
\%& \\ \end{align*}

en tenant compte de dx_i ∧ dx_i = 0 et de
dx_i ∧ dx_\\jmathmath = -dx_\\jmathmath ∧ dx_i. On
reconnaît là l'expression classique du rotationnel du champ de vecteurs
de composantes (a_1(x),a_2(x),a_3(x)).

Si p = 2, on a \omega(x) = a_1(x)dx_2 ∧ dx_3 +
a_2(x)dx_3 ∧ dx_1 +
a_3(x)dx_1 ∧ dx_2 et donc

\begin{align*} d\omega(x)& =& da_1(x) ∧
dx_2 ∧ dx_3 + da_2(x) ∧ dx_3 ∧
dx_1 \%& \\ & &
+da_3(x) ∧ dx_1 ∧ dx_2 \%&
\\ & =& ( \partial~a_1
\over \partial~x_1 (x)dx_1 + \partial~a_1
\over \partial~x_2 (x)dx_2 + \partial~a_1
\over \partial~x_3 (x)dx_3) ∧ dx_2 ∧
dx_3 \%& \\ & & +(
\partial~a_2 \over \partial~x_1 (x)dx_1 +
\partial~a_2 \over \partial~x_2 (x)dx_2 +
\partial~a_2 \over \partial~x_3 (x)dx_3) ∧
dx_3 ∧ dx_1\%& \\ & &
+( \partial~a_3 \over \partial~x_1 (x)dx_1
+ \partial~a_3 \over \partial~x_2 (x)dx_2
+ \partial~a_3 \over \partial~x_3 (x)dx_3)
∧ dx_1 ∧ dx_2\%& \\ &
=& \left ( \partial~a_1 \over
\partial~x_1 (x) + \partial~a_2 \over
\partial~x_2 (x) + \partial~a_3 \over
\partial~x_3 (x)\right )dx_1 ∧ dx_2
∧ dx_3 \%& \\
\end{align*}

en tenant compte de dx_i ∧ dx_\\jmathmath ∧ dx_k = 0 si
i,\\jmathmath et k ne sont pas distincts et de dx_\\jmathmath ∧ dx_k ∧
dx_i = dx_i ∧ dx_\\jmathmath ∧ dx_k si i,\\jmathmath,k
sont distincts (les permutations circulaires de trois éléments sont de
signature + 1). On reconnaît là l'expression classique de la divergence
du champ de vecteurs de composantes
(a_1(x),a_2(x),a_3(x)).

La différentielle extérieure des formes différentielles est donc une
généralisation (et une unification) des notions classiques de gradient
d'une fonction et de rotationnel ou divergence d'un champ de vecteurs.

\paragraph{15.3.6 Théorème de Poincaré}

Théorème~15.3.3 Soit U un ouvert de \mathbb{R}~^n et \omega une forme
différentielle de degré p de classe C^2 sur U. Alors d(d\omega) =
0.

Démonstration On a

\begin{align*} d\omega& =& \\sum
_1\leqi_1\textless{}i_2\textless{}\ldots\textless{}i_p\leqnda_i_1,\\ldots,i_p~
∧ dx_i_1 ∧\ldots~ ∧
dx_i_p \%& \\ & =&
\\sum
_1\leqi_1\textless{}i_2\textless{}\ldots\textless{}i_p\leqn~\left
(\sum _\\jmathmath=1^n~
\partial~a_i_1,\ldots,i_p~
\over \partial~x_\\jmathmath
\,dx_\\jmathmath\right ) ∧
dx_i_1 ∧\ldots~ ∧
dx_i_p\%& \\ & =&
\sum _\\jmathmath=1^n~
\\sum
_1\leqi_1\textless{}i_2\textless{}\ldots\textless{}i_p\leqn~
\partial~a_i_1,\ldots,i_p~
\over \partial~x_\\jmathmath dx_\\jmathmath ∧
dx_i_1 ∧\ldots~ ∧
dx_i_p \%& \\
\end{align*}

d'où

\begin{align*} d(d\omega)&& \%&
\\ & =& \\sum
_\\jmathmath=1^n \\sum
_1\leqi_1\textless{}i_2\textless{}\ldots\textless{}i_p\leqn~d\left
(
\partial~a_i_1,\ldots,i_p~
\over \partial~x_\\jmathmath \right ) ∧
dx_\\jmathmath ∧ dx_i_1
∧\ldots ∧ dx_i_p~ \%&
\\ & =& \\sum
_\\jmathmath=1^n \\sum
_1\leqi_1\textless{}i_2\textless{}\ldots\textless{}i_p\leqn~\left
(\sum _k=1^n~
\partial~^2a_
i_1,\ldots,i_p~
\over \partial~x_k\partial~x_\\jmathmath
dx_k\right ) ∧ dx_\\jmathmath ∧
dx_i_1 ∧\ldots~ ∧
dx_i_p \%& \\ & =&
\sum _k=1^n~
\sum _\\jmathmath=1^n~
\\sum
_1\leqi_1\textless{}i_2\textless{}\ldots\textless{}i_p\leqn~
\partial~^2a_i_1,\ldots,i_p~
\over \partial~x_k\partial~x_\\jmathmath dx_k ∧
dx_\\jmathmath ∧ dx_i_1
∧\ldots ∧ dx_i_p~ \%&
\\ & =& \\sum
_k\textless{}\\jmathmath \\sum
_1\leqi_1\textless{}i_2\textless{}\ldots\textless{}i_p\leqn~\left
(
\partial~^2a_i_1,\ldots,i_p~
\over \partial~x_k\partial~x_\\jmathmath -
\partial~^2a_i_1,\ldots,i_p~
\over \partial~x_\\jmathmath\partial~x_k \right
)dx_k ∧ dx_\\jmathmath ∧ dx_i_1
∧\ldots ∧ dx_i_p~\%&
\\ \end{align*}

en tenant compte de dx_\\jmathmath ∧ dx_k = 0 si \\jmathmath = k et
dx_\\jmathmath ∧ dx_k = -dx_k ∧ dx_\\jmathmath si
\\jmathmath\neq~k. Mais le théorème de Schwarz montre que
 \partial~^2a_
i_1,\\ldots,i_p~
\over \partial~x_k\partial~x_\\jmathmath =
\partial~^2a_
i_1,\\ldots,i_p~
\over \partial~x_\\jmathmath\partial~x_k et donc d(d\omega) = 0.

En tenant compte des expressions trouvées pour d\omega dans le cas n = 3, on
obtient donc le corollaire suivant

Corollaire~15.3.4 (i) Soit f une fonction de classe C^2 sur
un ouvert U de \mathbb{R}~^3. Alors
rot\grad~f = 0 (ii)
Soit V un champ de vecteurs de classe C^2 sur un ouvert U de
\mathbb{R}~^3. Alors
div\rot~V = 0

Nous allons maintenant nous intéresser à la réciproque du théorème
précédent

Théorème~15.3.5 (Poincaré). Soit U \subset~ \mathbb{R}~^n un ouvert étoilé en
a \in U (c'est-à-dire que \forall~~x \in U, {[}a,x{]} \subset~
U). Soit \omega une forme différentielle de degré p ≥ 1 de classe
\mathcal{C}^1 sur U. Alors les conditions suivantes sont équivalentes
(i) d\omega = 0 (ii) \omega est exacte~: il existe une forme différentielle \alpha~ de
degré p - 1 de classe C^2 sur U telle que \omega = d\alpha~.

Démonstration Le théorème précédent implique clairement que (ii) \rigtharrow~(i).
Nous nous contenterons de démontrer que (i) \rigtharrow~(ii) lorsque p = 1, en
admettant le cas général. Par une translation, sans nuire à la
généralité, on peut supposer que a = 0. Soit U \subset~ \mathbb{R}~^n un
ouvert étoilé en 0 \in U et soit \omega =\
\sum ~
_i=1^nc_i(x)dx_i. On a par un calcul
facile

d\omega = \\sum
_i\textless{}\\jmathmath\left ( \partial~c_\\jmathmath
\over \partial~x_i - \partial~c_i
\over \partial~x_\\jmathmath \right
)dx_i ∧ dx_\\jmathmath

Donc d\omega = 0 \Leftrightarrow
\forall~i,\\jmathmath, \partial~c_\\jmathmath~ \over
\partial~x_i = \partial~c_i \over \partial~x_\\jmathmath .

Définissons f : U \rightarrow~ \mathbb{R}~ par f(x) =\
\sum ~
_i=1^nx_i\int ~
_0^1c_i(tx) dt. Comme
(t,x_\\jmathmath)\mapsto~c_i(tx) =
c_i(tx_1,\\ldots,tx_n~)
admet une dérivée partielle par rapport à x_\\jmathmath,  \partial~
\over \partial~x_\\jmathmath
(c_i(tx_1,\\ldots,tx_n~))
= t \partial~c_i \over \partial~x_\\jmathmath
(tx_1,\\ldots,tx_n~)
qui est une fonction continue du couple (t,x_\\jmathmath), l'application
x_\\jmathmath\mapsto~\int ~
_0^1c_i(tx) dt est dérivable et  \partial~
\over \partial~x_\\jmathmath \int ~
_0^1c_i(tx) dt =\int ~
_0^1t \partial~c_i \over \partial~x_\\jmathmath
(tx) dt. On en déduit que

\begin{align*} \partial~f \over
\partial~x_\\jmathmath (x)& =& \\sum
_i=1^n \partial~x_i \over
\partial~x_\\jmathmath  \\int ~
 _0^1c_ i(tx) dt + \\sum
_i=1^nx_ i \partial~ \over
\partial~x_\\jmathmath  \\int ~
 _0^1c_ i(tx) dt\%&
\\ & =& \int ~
_0^1c_ \\jmathmath(tx) dt + \\sum
_i=1^nx_ i
\\int  ~
_0^1t \partial~c_i \over \partial~x_\\jmathmath
(tx) dt \%& \\ & =&
\int  _0^1~\left
(c_ \\jmathmath(tx) + \\sum
_i=1^ntx_ i \partial~c_i
\over \partial~x_\\jmathmath (tx)\right ) dt \%&
\\ \end{align*}

Utilisons alors  \partial~c_\\jmathmath \over \partial~x_i
= \partial~c_i \over \partial~x_\\jmathmath . On obtient

\begin{align*} \partial~f \over
\partial~x_\\jmathmath (x)& =& \int ~
_0^1\left (c_ \\jmathmath(tx) +
\sum _i=1^ntx_ i~
\partial~c_\\jmathmath \over \partial~x_i
(tx)\right ) dt\%& \\ &
=& \int  _0^1~ d
\over dt \left
(tc_\\jmathmath(tx_1,\\ldots,tx_n~)\right
) dt \%& \\ & =& \big
{[}tc_\\jmathmath(tx)\big {]}_0^1 =
c_ \\jmathmath(x) \%& \\
\end{align*}

Ceci montre à la fois que f est de classe C^2 et que \omega = df.

En réutilisant les calculs faits dans \mathbb{R}~^3, nous pouvons
traduire ce résultat sous la forme

Corollaire~15.3.6 Soit U \subset~ \mathbb{R}~^n un ouvert étoilé en a \in U
(c'est-à-dire que \forall~~x \in U, {[}a,x{]} \subset~ U), soit
V un champ de vecteurs de classe \mathcal{C}^1 sur U. Alors les
conditions suivantes sont équivalentes (i) il existe une fonction f de
classe C^2 telle que V = grad~f
(resp. il existe un champ de vecteurs W de classe C^2 tel que
V = rotW) (ii) \rot~V
= 0 (resp. div~ V = 0)

Remarque~15.3.6 Dans le premier cas, on dit que V dérive du potentiel
scalaire f, dans le deuxième cas qu'il dérive du potentiel vecteur W.

{[}
{[}
{[}
{[}

\end{document}

% \documentclass[]{article}
\usepackage[T1]{fontenc}
\usepackage{lmodern}
\usepackage{amssymb,amsmath}
\usepackage{ifxetex,ifluatex}
\usepackage{fixltx2e} % provides \textsubscript
% use upquote if available, for straight quotes in verbatim environments
\IfFileExists{upquote.sty}{\usepackage{upquote}}{}
\ifnum 0\ifxetex 1\fi\ifluatex 1\fi=0 % if pdftex
  \usepackage[utf8]{inputenc}
\else % if luatex or xelatex
  \ifxetex
    \usepackage{mathspec}
    \usepackage{xltxtra,xunicode}
  \else
    \usepackage{fontspec}
  \fi
  \defaultfontfeatures{Mapping=tex-text,Scale=MatchLowercase}
  \newcommand{\euro}{€}
\fi
% use microtype if available
\IfFileExists{microtype.sty}{\usepackage{microtype}}{}
\ifxetex
  \usepackage[setpagesize=false, % page size defined by xetex
              unicode=false, % unicode breaks when used with xetex
              xetex]{hyperref}
\else
  \usepackage[unicode=true]{hyperref}
\fi
\hypersetup{breaklinks=true,
            bookmarks=true,
            pdfauthor={},
            pdftitle={Fonctions implicites et inversion locale},
            colorlinks=true,
            citecolor=blue,
            urlcolor=blue,
            linkcolor=magenta,
            pdfborder={0 0 0}}
\urlstyle{same}  % don't use monospace font for urls
\setlength{\parindent}{0pt}
\setlength{\parskip}{6pt plus 2pt minus 1pt}
\setlength{\emergencystretch}{3em}  % prevent overfull lines
\setcounter{secnumdepth}{0}
 
/* start css.sty */
.cmr-5{font-size:50%;}
.cmr-7{font-size:70%;}
.cmmi-5{font-size:50%;font-style: italic;}
.cmmi-7{font-size:70%;font-style: italic;}
.cmmi-10{font-style: italic;}
.cmsy-5{font-size:50%;}
.cmsy-7{font-size:70%;}
.cmex-7{font-size:70%;}
.cmex-7x-x-71{font-size:49%;}
.msbm-7{font-size:70%;}
.cmtt-10{font-family: monospace;}
.cmti-10{ font-style: italic;}
.cmbx-10{ font-weight: bold;}
.cmr-17x-x-120{font-size:204%;}
.cmsl-10{font-style: oblique;}
.cmti-7x-x-71{font-size:49%; font-style: italic;}
.cmbxti-10{ font-weight: bold; font-style: italic;}
p.noindent { text-indent: 0em }
td p.noindent { text-indent: 0em; margin-top:0em; }
p.nopar { text-indent: 0em; }
p.indent{ text-indent: 1.5em }
@media print {div.crosslinks {visibility:hidden;}}
a img { border-top: 0; border-left: 0; border-right: 0; }
center { margin-top:1em; margin-bottom:1em; }
td center { margin-top:0em; margin-bottom:0em; }
.Canvas { position:relative; }
li p.indent { text-indent: 0em }
.enumerate1 {list-style-type:decimal;}
.enumerate2 {list-style-type:lower-alpha;}
.enumerate3 {list-style-type:lower-roman;}
.enumerate4 {list-style-type:upper-alpha;}
div.newtheorem { margin-bottom: 2em; margin-top: 2em;}
.obeylines-h,.obeylines-v {white-space: nowrap; }
div.obeylines-v p { margin-top:0; margin-bottom:0; }
.overline{ text-decoration:overline; }
.overline img{ border-top: 1px solid black; }
td.displaylines {text-align:center; white-space:nowrap;}
.centerline {text-align:center;}
.rightline {text-align:right;}
div.verbatim {font-family: monospace; white-space: nowrap; text-align:left; clear:both; }
.fbox {padding-left:3.0pt; padding-right:3.0pt; text-indent:0pt; border:solid black 0.4pt; }
div.fbox {display:table}
div.center div.fbox {text-align:center; clear:both; padding-left:3.0pt; padding-right:3.0pt; text-indent:0pt; border:solid black 0.4pt; }
div.minipage{width:100%;}
div.center, div.center div.center {text-align: center; margin-left:1em; margin-right:1em;}
div.center div {text-align: left;}
div.flushright, div.flushright div.flushright {text-align: right;}
div.flushright div {text-align: left;}
div.flushleft {text-align: left;}
.underline{ text-decoration:underline; }
.underline img{ border-bottom: 1px solid black; margin-bottom:1pt; }
.framebox-c, .framebox-l, .framebox-r { padding-left:3.0pt; padding-right:3.0pt; text-indent:0pt; border:solid black 0.4pt; }
.framebox-c {text-align:center;}
.framebox-l {text-align:left;}
.framebox-r {text-align:right;}
span.thank-mark{ vertical-align: super }
span.footnote-mark sup.textsuperscript, span.footnote-mark a sup.textsuperscript{ font-size:80%; }
div.tabular, div.center div.tabular {text-align: center; margin-top:0.5em; margin-bottom:0.5em; }
table.tabular td p{margin-top:0em;}
table.tabular {margin-left: auto; margin-right: auto;}
div.td00{ margin-left:0pt; margin-right:0pt; }
div.td01{ margin-left:0pt; margin-right:5pt; }
div.td10{ margin-left:5pt; margin-right:0pt; }
div.td11{ margin-left:5pt; margin-right:5pt; }
table[rules] {border-left:solid black 0.4pt; border-right:solid black 0.4pt; }
td.td00{ padding-left:0pt; padding-right:0pt; }
td.td01{ padding-left:0pt; padding-right:5pt; }
td.td10{ padding-left:5pt; padding-right:0pt; }
td.td11{ padding-left:5pt; padding-right:5pt; }
table[rules] {border-left:solid black 0.4pt; border-right:solid black 0.4pt; }
.hline hr, .cline hr{ height : 1px; margin:0px; }
.tabbing-right {text-align:right;}
span.TEX {letter-spacing: -0.125em; }
span.TEX span.E{ position:relative;top:0.5ex;left:-0.0417em;}
a span.TEX span.E {text-decoration: none; }
span.LATEX span.A{ position:relative; top:-0.5ex; left:-0.4em; font-size:85%;}
span.LATEX span.TEX{ position:relative; left: -0.4em; }
div.float img, div.float .caption {text-align:center;}
div.figure img, div.figure .caption {text-align:center;}
.marginpar {width:20%; float:right; text-align:left; margin-left:auto; margin-top:0.5em; font-size:85%; text-decoration:underline;}
.marginpar p{margin-top:0.4em; margin-bottom:0.4em;}
.equation td{text-align:center; vertical-align:middle; }
td.eq-no{ width:5%; }
table.equation { width:100%; } 
div.math-display, div.par-math-display{text-align:center;}
math .texttt { font-family: monospace; }
math .textit { font-style: italic; }
math .textsl { font-style: oblique; }
math .textsf { font-family: sans-serif; }
math .textbf { font-weight: bold; }
.partToc a, .partToc, .likepartToc a, .likepartToc {line-height: 200%; font-weight:bold; font-size:110%;}
.chapterToc a, .chapterToc, .likechapterToc a, .likechapterToc, .appendixToc a, .appendixToc {line-height: 200%; font-weight:bold;}
.index-item, .index-subitem, .index-subsubitem {display:block}
.caption td.id{font-weight: bold; white-space: nowrap; }
table.caption {text-align:center;}
h1.partHead{text-align: center}
p.bibitem { text-indent: -2em; margin-left: 2em; margin-top:0.6em; margin-bottom:0.6em; }
p.bibitem-p { text-indent: 0em; margin-left: 2em; margin-top:0.6em; margin-bottom:0.6em; }
.paragraphHead, .likeparagraphHead { margin-top:2em; font-weight: bold;}
.subparagraphHead, .likesubparagraphHead { font-weight: bold;}
.quote {margin-bottom:0.25em; margin-top:0.25em; margin-left:1em; margin-right:1em; text-align:\jmathustify;}
.verse{white-space:nowrap; margin-left:2em}
div.maketitle {text-align:center;}
h2.titleHead{text-align:center;}
div.maketitle{ margin-bottom: 2em; }
div.author, div.date {text-align:center;}
div.thanks{text-align:left; margin-left:10%; font-size:85%; font-style:italic; }
div.author{white-space: nowrap;}
.quotation {margin-bottom:0.25em; margin-top:0.25em; margin-left:1em; }
h1.partHead{text-align: center}
.sectionToc, .likesectionToc {margin-left:2em;}
.subsectionToc, .likesubsectionToc {margin-left:4em;}
.subsubsectionToc, .likesubsubsectionToc {margin-left:6em;}
.frenchb-nbsp{font-size:75%;}
.frenchb-thinspace{font-size:75%;}
.figure img.graphics {margin-left:10%;}
/* end css.sty */

\title{Fonctions implicites et inversion locale}
\author{}
\date{}

\begin{document}
\maketitle

\textbf{Warning: 
requires JavaScript to process the mathematics on this page.\\ If your
browser supports JavaScript, be sure it is enabled.}

\begin{center}\rule{3in}{0.4pt}\end{center}

{[}
{[}
{[}{]}
{[}

\subsubsection{15.4 Fonctions implicites et inversion locale}

\paragraph{15.4.1 Position du problème des fonctions implicites}

Soit E,F et G trois espaces vectoriels normés, W un ouvert de E \times F, f :
W \rightarrow~ G. On considère la courbe \Gamma = \(x,y) \in
W∣f(x,y) = 0\. On se pose la
question de savoir si \Gamma est le graphe d'une fonction \phi d'un ouvert U de
E dans F, autrement dit si f(x,y) = 0 \Leftrightarrow y =
\phi(x).

Cette question globale n'admet pas vraiment de réponse satisfaisante et
nous allons la transformer en une question locale. Soit (a,b) \in \Gamma. On se
pose la question de savoir si \Gamma, au voisinage de (a,b), est le graphe
d'une fonction \phi d'un ouvert U de E dans F, autrement dit si il existe U
ouvert contenant a et V ouvert contenant b tels que, pour (a,b) \in U \times V
, f(x,y) = 0 \Leftrightarrow y = \phi(x). Cela revient à
demander que \Gamma \bigcap (U \times V ) soit un graphe, autrement dit que

\forall~x \in U, \\exists~!y \in V,
f(x,y) = 0

Nous cherchons en plus des propriétés de la fonction \phi (lorsqu'elle
existe) à partir de propriétés de la fonction f.

Supposons que E = \mathbb{R}~^n, F = \mathbb{R}~^p et G =
\mathbb{R}~^q. On a f(x,y) =
(f\_1(x,y),\\ldots,f\_q~(x,y))
si bien que

f(x,y) = 0 \Leftrightarrow \left
\\matrix\,f\_1(x\_1,\\ldots,x\_n,y\_1,\\\ldots,y\_p~)
= 0 \cr \cr
f\_q(x\_1,\\ldots,x\_n,y\_1,\\\ldots,y\_p~)
= 0\right .

Pour
(x\_1,\\ldots,x\_n~)
fixé dans U \inV(a), ce système doit déterminer un unique
(y\_1,\\ldots,y\_p~)
dans V \inV(b). Ceci semble nécessiter qu'il y ait autant d'équations que
d'inconnues, c'est-à-dire que p = q.

Même, dans ce cas, l'exemple n = p = q = 1 et f(x,y) = x^2 +
y^2 - 1 montre que la réponse est positive en (a,b) \in
\mathbb{R}~^2 si b\neq~0, mais qu'elle est
négative aux points (1,0) et (-1,0) de \Gamma, points où l'on a  \partial~f
\over \partial~y (a,b) = 0.

Le théorème des fonctions implicites va nous donner une condition
suffisante pour que la réponse au problème local soit positive.

\paragraph{15.4.2 Théorème des fonctions implicites}

Théorème~15.4.1 Soit W un ouvert de \mathbb{R}~^n \times \mathbb{R}~^p et f
: W \rightarrow~ \mathbb{R}~^p de classe \mathcal{C}^1. On pose x =
(x\_1,\\ldots,x\_n~),
y =
(y\_1,\\ldots,y\_p~)
et f =
(f\_1,\\ldots,f\_p~).
Soit (a,b) \in W tel que f(a,b) = 0 et Q = \left (
\partial~f\_i \over \partial~y\_\jmath
(a,b)\right )\_1\leqi,\jmath\leqp est inversible. Alors, il
existe U \inV(a) et V \inV(b) (ouverts) tels que

\forall~x \in U \\exists~!y \in V
\text tel que f(x,y) = 0.

Si l'on pose y = \phi(x), \phi est continue sur U et de classe \mathcal{C}^1
sur un voisinage U\_0 de a.

Démonstration Soit \psi : W \rightarrow~ \mathbb{R}~^p,
(x,y)\mapsto~\psi(x,y) = \psi\_x(y) = y -
Q^-1(f(x,y)). On a de manière évidente

f(x,y) = 0 \Leftrightarrow \psi\_x(y) = y.

On va essayer d'appliquer le théorème du point fixe à l'équation
\psi\_x(y) = y. Notons Q(x,y) = \left (
\partial~f\_i \over \partial~y\_\jmath
(x,y)\right )\_1\leqi,\jmath\leqp, de sorte que Q = Q(a,b).
Puisque Q^-1 est une application linéaire, elle est sa propre
différentielle en tout point et la matrice de d\psi\_x(y) est donc
la matrice

J\_\psi\_x(y) = I\_p - Q^-1 \cdotQ(x,y)

Donc d\psi\_a(b) = 0 et l'application
(x,y)\mapsto~d\psi\_x(y) est continue. On en
déduit qu'il existe r \textgreater{} 0 tel que

\\textbar{}x - a\\textbar{} \leq
r\text et \\textbar{}y -
b\\textbar{} \leq r \rigtharrow~\\textbar{}
d\psi\_x(y)\\textbar{} \leq 1 \over
2 .

Soit x \in B'(a,r),y,y' \in B'(b,r). On a alors

\\textbar{}\psi\_x(y) -
\psi\_x(y')\\textbar{} \leq\\textbar{} y
-
y'\\textbar{}sup\_z\in{[}y,y'{]}\\textbar{}d\psi\_x~(z)\\textbar{}
\leq 1 \over 2 \\textbar{}y -
y'\\textbar{}

d'après l'inégalité des accroissements finis. Puisque \psi est continue en
(a,b), il existe U\_1 voisinage ouvert de a inclus dans B'(a,r)
tel que x \in U\_1 \rigtharrow~\\textbar{} \psi(x,b) -
\psi(a,b)\\textbar{} \leq r \over 2 , soit
encore, puisque \psi(a,b) = b, x \in U\_1 \rigtharrow~\\textbar{}
\psi(x,b) - b\\textbar{} \leq r \over 2 .
Pour x \in U\_1 et y \in B'(b,r), on a donc

\begin{align*}
\\textbar{}\psi\_x(y) -
b\\textbar{}& \leq&
\\textbar{}\psi\_x(y) -
\psi\_x(b)\\textbar{} +\\textbar{}
\psi(x,b) - \psi(a,b)\\textbar{}\%&
\\ & \leq& 1 \over 2
\\textbar{}y - b\\textbar{} + r
\over 2 \leq r. \%& \\
\end{align*}

Donc, si x \in U\_1, \psi\_x est une application de B'(b,r)
dans B'(b,r) qui est  1 \over 2
-\textcontractante~; mais B'(b,r) est un espace
métrique complet (fermé dans un complet). Donc pour x \in U\_1, il
existe un unique y \in V \_1 = B'(b,r) tel que \psi\_x(y) =
y, c'est-à-dire f(x,y) = 0.

Appelons \phi(x) cet unique y, on définit ainsi \phi : U\_1 \rightarrow~ V
\_1 telle que \psi\_x(\phi(x)) = \phi(x). Montrons que \phi est
continue. Soit x et x\_0 dans U\_1. On a

\begin{align*} \\textbar{}\phi(x) -
\phi(x\_0)\\textbar{} =\\textbar{}
\psi\_x(\phi(x)) -
\psi\_x\_0(\phi(x\_0))\\textbar{}&&
\%& \\ & & \%&
\\ & \leq&
\\textbar{}\psi\_x(\phi(x)) -
\psi\_x(\phi(x\_0))\\textbar{}
+\\textbar{} \psi(x,\phi(x\_0)) -
\psi(x\_0,\phi(x\_0))\\textbar{}\%&
\\ & \leq& 1 \over 2
\\textbar{}\phi(x) -
\phi(x\_0)\\textbar{} +\\textbar{}
\psi(x,\phi(x\_0)) -
\psi(x\_0,\phi(x\_0)\\textbar{} \%&
\\ \end{align*}

soit encore

\\textbar{}\phi(x) -
\phi(x\_0)\\textbar{} \leq
2\\textbar{}\psi(x,\phi(x\_0)) -
\psi(x\_0,\phi(x\_0)\\textbar{}.

Comme x\mapsto~\psi(x,\phi(x\_0)) est continue en
x\_0, il en est de même de x\mapsto~\phi(x).

Soit alors V = B(b,r) et U = U\_1 \bigcap \phi^-1(V ). V est
ouvert, et il en est de même de U comme intersection de l'ouvert
U\_1 et de l'image réciproque de l'ouvert V par l'application
continue \phi. Pour x \in U, il existe un unique y \in B'(b,r) tel que f(x,y) =
0, avec y = \phi(x). Mais comme x \in \phi^-1(V ), on a en fait y \in V
et en définitive

\forall~x \in U \\exists~!y \in V
\text tel que f(x,y) = 0.

Soit P = \left ( \partial~f\_i \over
\partial~x\_\jmath (a,b)\right )\_1\leqi\leqp,1\leq\jmath\leqn et soit
h \in \mathbb{R}~^n et k \in \mathbb{R}~^p. Les formules

\begin{align*} f(a + h,b + k) =
\sum \_i=1^n~ \partial~f
\over \partial~x\_i (a,b)h\_i +
\sum \_i=1^p~ \partial~f
\over \partial~y\_i (a,b)k\_i +
o(\\textbar{}(h,k)\\textbar{})& & \%&
\\ \end{align*}

se traduisent par

f(a + h,b + k) = Ph + Qk +
o(\\textbar{}h\\textbar{}
+\\textbar{} k\\textbar{}).

Prenons k = \theta(h) = \phi(a + h) - \phi(a) = \phi(a + h) - b. On a alors

\begin{align*} 0& =& f(a + h,\phi(a + h)) = f(a + h,b
+ \theta(h))\%& \\ & =& Ph + Q\theta(h) +
(\\textbar{}h\\textbar{}
+\\textbar{} \theta(h)\\textbar{})\epsilon(h) \%&
\\ \end{align*}

soit encore

\theta(h) = -Q^-1Ph +
(\\textbar{}h\\textbar{}
+\\textbar{} \theta(h)\\textbar{})\eta(h)

avec \eta(h) = -Q^-1(\epsilon(h)). Comme on a
lim\_h\rightarrow~0~\eta(h) = 0, soit \rho
\textgreater{} 0 tel que h \textless{} \rho \rigtharrow~\textbar{}\eta(h)\textbar{}
\textless{} 1 \over 2 . Alors pour h \textless{} \rho,
on a

\\textbar{}\theta(h)\\textbar{}
\leq\\textbar{}
Q^-1P\\textbar{}\\textbar{}h\\textbar{}
+ 1 \over 2
(\\textbar{}h\\textbar{}
+\\textbar{} \theta(h)\\textbar{}),

soit encore

\\textbar{}\theta(h)\\textbar{} \leq
(2\\textbar{}Q^-1P\\textbar{} +
1)\\textbar{}h\\textbar{}.

On a donc \\textbar{}\theta(h)\\textbar{} =
O(\\textbar{}h\\textbar{}), soit encore
(\\textbar{}h\\textbar{}
+\\textbar{} \theta(h)\\textbar{})\eta(h) =
o(\\textbar{}h\\textbar{}), ce qui montre
que

\phi(a + h) - \phi(a) = \theta(h) = -Q^-1Ph +
o(\\textbar{}h\\textbar{}).

Donc \phi est différentiable au point a et sa différentielle est -
Q^-1P. On montre de même que \phi est différentiable en tout
point x assez voisin de a pour que la matrice Q(x,\phi(x)) reste inversible
et que l'on a encore d\phi(x) = -Q(x,\phi(x))^-1P(x,\phi(x)), ce qui
montre que \phi est de classe \mathcal{C}^1 sur un tel voisinage.

\paragraph{15.4.3 Applications du théorème des fonctions implicites}

Nous nous intéresserons tout particulièrement au cas p = 1~; dans ce cas
Q = \left (\matrix\, \partial~f
\over \partial~y (a,b)\right ) et la matrice
est inversible si et seulement si  \partial~f \over \partial~y
(a,b)\neq~0. On obtient donc la formulation
suivante

Théorème~15.4.2 Soit W un ouvert de \mathbb{R}~^n \times \mathbb{R}~ et f : W \rightarrow~ \mathbb{R}~ de
classe \mathcal{C}^1,
(x\_1,\\ldots,x\_n,y)\mapsto~f(x\_1,\\\ldots,x\_n~,y).
Soit
(a\_1,\\ldots,a\_n~,b)
\in W tel que
f(a\_1,\\ldots,a\_n~,b)
= 0 et  \partial~f \over \partial~y
(a\_1,\\ldots,a\_n,b)\mathrel\neq~~0.
Alors, il existe U
\inV(a\_1,\\ldots,a\_n~)
et V \inV(b) (ouverts) tels que

\forall~(x\_1,\\\ldots,x\_n~)
\in U, \exists~!y \in V \text tel que
f(x\_1,\\ldots,x\_n~,y)
= 0.

Si l'on pose y =
\phi(x\_1,\\ldots,x\_n~),
\phi est continue sur U et de classe \mathcal{C}^1 sur un voisinage
U\_0 de a.

Remarque~15.4.1 Le calcul des dérivées partielles de \phi se fait très
facilement en utilisant les formes différentielles. Les variables
x\_1,\\ldots,x\_n~
et y étant liées par la relation
f(x\_1,\\ldots,x\_n~,y)
= 0, on obtient par différentiation

\sum \_i=1^n~ \partial~f
\over \partial~x\_i
(x\_1,\ldots,x\_n,y)dx\_i~
+ \partial~f \over \partial~y
(x\_1,\ldots,x\_n~,y)dy = 0

soit encore

dy = -\sum \_i=1^n~  \partial~f
\over \partial~x\_i
(x\_1,\ldots,x\_n~,y)
\over  \partial~f \over \partial~y
(x\_1,\ldots,x\_n,y)~
dx\_i

On en déduit que

 \partial~\phi \over \partial~x\_i
(x\_1,\\ldots,x\_n~)
= \partial~y \over \partial~x\_i
(x\_1,\\ldots,x\_n~)
= -  \partial~f \over \partial~x\_i
(x\_1,\\ldots,x\_n~,y)
\over  \partial~f \over \partial~y
(x\_1,\\ldots,x\_n,y)~

si y =
\phi(x\_1,\\ldots,x\_n~).

Théorème~15.4.3 Soit W un ouvert de \mathbb{R}~^2 et f : W \rightarrow~ \mathbb{R}~ de
classe \mathcal{C}^1. Soit \Gamma = \(x,y) \in
W∣f(x,y) = 0\. On suppose
que \forall~~(a,b) \in \Gamma, \left ( \partial~f
\over \partial~x (a,b), \partial~f \over \partial~y
(a,b)\right )\neq~(0,0). Alors, au
voisinage de chacun de ses points, \Gamma est soit le graphe d'une
application de classe \mathcal{C}^1 x\mapsto~y =
\phi(x), soit le graphe d'une application de classe \mathcal{C}^1
y\mapsto~x = \psi(y). La tangente à ce graphe au point
(a,b) est la droite d'équation

(x - a) \partial~f \over \partial~x (a,b) + (y - b) \partial~f
\over \partial~y (a,b) = 0

Démonstration Si par exemple  \partial~f \over \partial~y
(a,b)\neq~0, le théorème précédent s'applique et
permet de conclure, qu'au voisinage de (a,b), \Gamma est le graphe d'une
application de classe \mathcal{C}^1, x\mapsto~y =
\phi(x). La tangente à ce graphe est la droite d'équation y - b = \phi'(a)(x -
a) avec \phi'(a) = -  \partial~f \over \partial~x (a,b)
\over  \partial~f \over \partial~y (a,b) ce qui
donne l'équation ci dessus. Si  \partial~f \over \partial~x
(a,b)\neq~0, il suffit d'échanger les rôles \jmathoués
par x et y.

Nous avons un théorème similaire pour les surfaces de \mathbb{R}~^3

Théorème~15.4.4 Soit W un ouvert de \mathbb{R}~^3 et f : W \rightarrow~ \mathbb{R}~ de
classe \mathcal{C}^1. Soit \Sigma = \(x,y,z) \in
W∣f(x,y,z) = 0\. On suppose
que \forall~~(a,b,c) \in \Gamma, \left ( \partial~f
\over \partial~x (a,b,c), \partial~f \over \partial~y
(a,b,c), \partial~f \over \partial~z (a,b,c)\right
)\neq~(0,0,0). Alors, au voisinage de chacun de
ses points, \Sigma est soit le graphe d'une application de classe
\mathcal{C}^1, (x,y)\mapsto~z = \phi(x,y), soit le
graphe d'une application de classe \mathcal{C}^1,
(y,z)\mapsto~x = \psi(y,z), soit le graphe d'une
application de classe \mathcal{C}^1, (x,z)\mapsto~y
= \psi(x,z). Le plan tangent à ce graphe au point (a,b,c) est le plan
d'équation

(x - a) \partial~f \over \partial~x (a,b,c) + (y - b) \partial~f
\over \partial~y (a,b,c) + (z - c) \partial~f \over
\partial~z (a,b,c) = 0

Démonstration La même que précédemment sauf pour ce qui concerne
l'équation du plan tangent. Supposons que localement \Sigma est le graphe
d'une application de classe \mathcal{C}^1
(x,y)\mapsto~z = \phi(x,y). La surface est paramétrée
par (x,y)\mapsto~(x,y,\phi(x,y)) et les deux vecteurs
tangents dérivés partiels sont  \partial~ \over \partial~x
(x,y,\phi(x,y)) = (1,0, \partial~\phi \over \partial~x (x,y)) et  \partial~
\over \partial~y (x,y,\phi(x,y)) = (0,1, \partial~\phi \over
\partial~x (x,y)). Le plan tangent est le plan parallèle à ces deux vecteurs
(pour (x,y) = (a,b)) et contenant le point (a,b,c), c'est-à-dire le plan
d'équation

\left
\textbar{}\matrix\,x - a&1 &0
\cr y - b&0 &1 \cr z - c& \partial~\phi
\over \partial~x (a,b)& \partial~\phi \over \partial~y
(a,b)\right \textbar{} = 0

Mais on a

\begin{align*} \partial~\phi \over \partial~x (a,b)
= -  \partial~f \over \partial~x (a,b,c) \over  \partial~f
\over \partial~z (a,b,c) \text et  \partial~\phi
\over \partial~y (a,b) = -  \partial~f \over \partial~y
(a,b,c) \over  \partial~f \over \partial~z (a,b,c)
& & \%& \\
\end{align*}

Il suffit de reporter et de développer le déterminant suivant la
première colonne pour obtenir l'équation du plan tangent sous la forme
souhaitée.

\paragraph{15.4.4 Difféomorphismes et inversion locale}

Définition~15.4.1 Soit E et F deux K-espaces vectoriels normés, U un
ouvert de E et V un ouvert de F. On dit que f : U \rightarrow~ V est un
difféomorphisme de classe \mathcal{C}^1 si (i) f est bi\jmathective de U sur
V (ii) f et f^-1 sont de classe \mathcal{C}^1.

Remarque~15.4.2 Comme pour les fonctions d'une variable, le fait que f
soit bi\jmathective et de classe \mathcal{C}^1 n'implique évidemment pas que
f^-1 soit de classe \mathcal{C}^1.

Théorème~15.4.5 Soit E et F deux K-espaces vectoriels normés, U un
ouvert de E, V un ouvert de F, f : U \rightarrow~ V un difféomorphisme de classe
\mathcal{C}^1. Alors, pour tout x \in U, df(x) est un isomorphisme
d'espace vectoriel de E sur F et on a

\forall~y \in V, d(f^-1~)(y) =
\left (df(f^-1(y))\right
)^-1

Démonstration Soit g = f^-1 : V \rightarrow~ U. On a g \cdot f =
\mathrmId\_U. Comme f est différentiable en x
et g en f(x), on a \mathrmId\_E =
d(\mathrmId\_U)(x) = d(g \cdot f)(x) = dg(f(x)) \cdot
df(x). On montre de la même fa\ccon que
\mathrmId\_F =
d(\mathrmId\_V )(f(x)) = d(f \cdot g)(f(x)) =
df(x) \cdot dg(f(x)). On en déduit que df(x) est un isomorphisme de E sur F
d'isomorphisme réciproque dg(f(x)).

Remarque~15.4.3 On vérifie facilement à partir de la formule ci dessus
que si f est à la fois de classe C^k et un \mathcal{C}^1
difféomorphisme, alors f^-1 est aussi de classe
C^k. On dit alors que f est un
C^k-difféomorphisme.

Remarque~15.4.4 Si E et F sont de dimensions finies, l'existence d'un
\mathcal{C}^1 difféomorphisme d'un ouvert de E sur un ouvert de F
nécessite que E et F aient même dimension~; il ne peut y avoir de
difféomorphisme d'un ouvert de \mathbb{R}~^n sur un ouvert de
\mathbb{R}~^p pour n\neq~p. Si f : U \rightarrow~ V est un
difféomorphisme d'un ouvert U de \mathbb{R}~^n sur un ouvert V de
\mathbb{R}~^n, on peut calculer les dérivées partielles de
f^-1 à l'aide des matrices \jmathacobiennes grâce à la formule
J\_f^-1(y) = \left
(J\_f(f^-1(y))\right )^-1.

Nous allons maintenant nous intéresser à une réciproque partielle (en
fait locale) du théorème précédent.

Théorème~15.4.6 (inversion locale). Soit U un ouvert de \mathbb{R}~^n
et f : U \rightarrow~ \mathbb{R}~^n une application de classe \mathcal{C}^1. Soit
a \in U tel que df(a) soit un isomorphisme d'espace vectoriel de
\mathbb{R}~^n sur \mathbb{R}~^n (autrement dit la matrice
J\_f(a) est inversible). Alors il existe un ouvert U\_0
contenant a et un ouvert V \_0 contenant f(a) tel que f induise
un difféomorphisme de classe \mathcal{C}^1 de U\_0 sur V
\_0.

Démonstration Considérons l'application g : \mathbb{R}~^n \times U \rightarrow~
\mathbb{R}~^n, (y,x)\mapsto~f(x) - y. La matrice
\left ( \partial~g\_i \over
\partial~x\_\jmath (f(a),a)\right )\_1\leqi,\jmath\leqn n'est
autre que la matrice J\_f(a) qui est inversible. On peut donc
appliquer le théorème des fonctions implicites. On en déduit qu'il
existe V \_1 ouvert contenant f(a) et un ouvert U\_1
contenant a tel que \forall~y \in V \_1~,
\exists!x \in U\_1~, g(y,x) = 0, autrement dit

\forall~y \in V \_1~,
\exists!x \in U\_1~, f(x) = y

Si on pose x = g(y), on sait que quitte à restreindre V \_1, on
peut supposer que g est de classe \mathcal{C}^1. Par définition même,
on a f(g(y)) = y pour y \in V \_1. Par contre, il n'est pas vrai
en général que, pour x \in U\_1, on ait g(f(x)) = x car il n'y a
pas de raison que f(x) appartienne à V \_1. Mais comme f est
continue et V \_1 ouvert, f^-1(V \_1) est un
ouvert contenant a et donc il en est de même de U\_1 \bigcap
f^-1(V \_1) = U\_0. Pour x \in U\_0, on
a x \in U\_1 et f(x) \in V \_1. Comme g(f(x)) est l'unique
x' dans U\_1 tel que f(x') = f(x) et comme x convient bien
évidemment, on a g(f(x)) = x pour x \in U\_0. Comme on a aussi
f(g(y)) = y pour y \in V \_1, f est bi\jmathective de U\_0 sur
V \_1, et son inverse est g qui est encore de classe
\mathcal{C}^1.

Remarque~15.4.5 Le théorème précédent est uniquement local. A partir de
la dimension 2, il n'existe pas de moyen local simple qui permette de
garantir l'in\jmathectivité globale de f, comme le montre l'exemple de f
:{]}0,+\infty~{[}\times\mathbb{R}~ \rightarrow~ \mathbb{R}~^2
\diagdown\(0,0)\,
(\rho,\theta)\mapsto~(\rhocos~
\theta,\rhosin~ \theta). La matrice \jmathacobienne est
J\_f(\rho,\theta) = \left
(\matrix\,cos~
\theta&-\rhosin~ \theta \cr
sin \theta&\rho\cos~ \theta
\right ) qui est inversible (de déterminant
\rho\neq~0)~; l'application f est localement
in\jmathective (et même localement un difféomorphisme), mais elle ne l'est
pas globalement puisque f(\rho,\theta + 2\pi~) = f(\rho,\theta).

Corollaire~15.4.7 Soit U un ouvert de \mathbb{R}~^n et f : U \rightarrow~
\mathbb{R}~^n une application de classe \mathcal{C}^1. On suppose que
(i) f est in\jmathective (ii) \forall~~x \in U,
J\_f(x) est une matrice inversible. Alors f(U) = V est un ouvert
de \mathbb{R}~^n et f est un \mathcal{C}^1 difféomorphisme de U sur V
.

Démonstration Soit g = f^-1 : V \rightarrow~ U. Soit y \in V , y = f(x).
Comme J\_f(x) est une matrice inversible, le théorème
d'inversion locale assure l'existence d'un ouvert U\_0 contenant
x et d'un ouvert V \_0 contenant y = f(x) tel que f induise un
\mathcal{C}^1 difféomorphisme de U\_0 sur V \_0. Mais
alors V \_0 = f(U\_0) \subset~ V . Ceci nous garantit que V est
un voisinage de y. Donc V est un voisinage de chacun de ses points, et
donc il est ouvert. Mais d'autre part, le difféomorphisme réciproque de
f\_∣\_U\_ 0 :
U\_0 \rightarrow~ V \_0 ne peut être que
g\_∣\_V\_ 0. On en
déduit que g est de classe \mathcal{C}^1 au point y. Donc g est de
classe \mathcal{C}^1 sur V et f est un \mathcal{C}^1 difféomorphisme
de U sur V .

Remarque~15.4.6 Un difféomorphisme f d'un ouvert U de \mathbb{R}~^n sur
un ouvert V de \mathbb{R}~^n,
(\alpha~\_1,\\ldots,\alpha~\_n)\mapsto~(x\_1,\\\ldots,x\_n~)
=
(f\_1(\alpha~\_1,\\ldots,\alpha~\_n),\\\ldots,f\_n(\alpha~\_1,\\\ldots,\alpha~\_n~))
est souvent appelé un système de coordonnées curvilignes sur V . Un tel
système permet de repérer un point
(x\_1,\\ldots,x\_n~)
de V par ses coordonnées curvilignes
(\alpha~\_1,\\ldots\alpha~\_n~).
Les coordonnées polaires, cylindriques ou sphériques sont typiques de
coordonnées curvilignes locales.

{[}
{[}
{[}
{[}

\end{document}

\part{Equations différentielles}
% \documentclass[]{article}
\usepackage[T1]{fontenc}
\usepackage{lmodern}
\usepackage{amssymb,amsmath}
\usepackage{ifxetex,ifluatex}
\usepackage{fixltx2e} % provides \textsubscript
% use upquote if available, for straight quotes in verbatim environments
\IfFileExists{upquote.sty}{\usepackage{upquote}}{}
\ifnum 0\ifxetex 1\fi\ifluatex 1\fi=0 % if pdftex
  \usepackage[utf8]{inputenc}
\else % if luatex or xelatex
  \ifxetex
    \usepackage{mathspec}
    \usepackage{xltxtra,xunicode}
  \else
    \usepackage{fontspec}
  \fi
  \defaultfontfeatures{Mapping=tex-text,Scale=MatchLowercase}
  \newcommand{\euro}{€}
\fi
% use microtype if available
\IfFileExists{microtype.sty}{\usepackage{microtype}}{}
\ifxetex
  \usepackage[setpagesize=false, % page size defined by xetex
              unicode=false, % unicode breaks when used with xetex
              xetex]{hyperref}
\else
  \usepackage[unicode=true]{hyperref}
\fi
\hypersetup{breaklinks=true,
            bookmarks=true,
            pdfauthor={},
            pdftitle={Notions generales},
            colorlinks=true,
            citecolor=blue,
            urlcolor=blue,
            linkcolor=magenta,
            pdfborder={0 0 0}}
\urlstyle{same}  % don't use monospace font for urls
\setlength{\parindent}{0pt}
\setlength{\parskip}{6pt plus 2pt minus 1pt}
\setlength{\emergencystretch}{3em}  % prevent overfull lines
\setcounter{secnumdepth}{0}
 
/* start css.sty */
.cmr-5{font-size:50%;}
.cmr-7{font-size:70%;}
.cmmi-5{font-size:50%;font-style: italic;}
.cmmi-7{font-size:70%;font-style: italic;}
.cmmi-10{font-style: italic;}
.cmsy-5{font-size:50%;}
.cmsy-7{font-size:70%;}
.cmex-7{font-size:70%;}
.cmex-7x-x-71{font-size:49%;}
.msbm-7{font-size:70%;}
.cmtt-10{font-family: monospace;}
.cmti-10{ font-style: italic;}
.cmbx-10{ font-weight: bold;}
.cmr-17x-x-120{font-size:204%;}
.cmsl-10{font-style: oblique;}
.cmti-7x-x-71{font-size:49%; font-style: italic;}
.cmbxti-10{ font-weight: bold; font-style: italic;}
p.noindent { text-indent: 0em }
td p.noindent { text-indent: 0em; margin-top:0em; }
p.nopar { text-indent: 0em; }
p.indent{ text-indent: 1.5em }
@media print {div.crosslinks {visibility:hidden;}}
a img { border-top: 0; border-left: 0; border-right: 0; }
center { margin-top:1em; margin-bottom:1em; }
td center { margin-top:0em; margin-bottom:0em; }
.Canvas { position:relative; }
li p.indent { text-indent: 0em }
.enumerate1 {list-style-type:decimal;}
.enumerate2 {list-style-type:lower-alpha;}
.enumerate3 {list-style-type:lower-roman;}
.enumerate4 {list-style-type:upper-alpha;}
div.newtheorem { margin-bottom: 2em; margin-top: 2em;}
.obeylines-h,.obeylines-v {white-space: nowrap; }
div.obeylines-v p { margin-top:0; margin-bottom:0; }
.overline{ text-decoration:overline; }
.overline img{ border-top: 1px solid black; }
td.displaylines {text-align:center; white-space:nowrap;}
.centerline {text-align:center;}
.rightline {text-align:right;}
div.verbatim {font-family: monospace; white-space: nowrap; text-align:left; clear:both; }
.fbox {padding-left:3.0pt; padding-right:3.0pt; text-indent:0pt; border:solid black 0.4pt; }
div.fbox {display:table}
div.center div.fbox {text-align:center; clear:both; padding-left:3.0pt; padding-right:3.0pt; text-indent:0pt; border:solid black 0.4pt; }
div.minipage{width:100%;}
div.center, div.center div.center {text-align: center; margin-left:1em; margin-right:1em;}
div.center div {text-align: left;}
div.flushright, div.flushright div.flushright {text-align: right;}
div.flushright div {text-align: left;}
div.flushleft {text-align: left;}
.underline{ text-decoration:underline; }
.underline img{ border-bottom: 1px solid black; margin-bottom:1pt; }
.framebox-c, .framebox-l, .framebox-r { padding-left:3.0pt; padding-right:3.0pt; text-indent:0pt; border:solid black 0.4pt; }
.framebox-c {text-align:center;}
.framebox-l {text-align:left;}
.framebox-r {text-align:right;}
span.thank-mark{ vertical-align: super }
span.footnote-mark sup.textsuperscript, span.footnote-mark a sup.textsuperscript{ font-size:80%; }
div.tabular, div.center div.tabular {text-align: center; margin-top:0.5em; margin-bottom:0.5em; }
table.tabular td p{margin-top:0em;}
table.tabular {margin-left: auto; margin-right: auto;}
div.td00{ margin-left:0pt; margin-right:0pt; }
div.td01{ margin-left:0pt; margin-right:5pt; }
div.td10{ margin-left:5pt; margin-right:0pt; }
div.td11{ margin-left:5pt; margin-right:5pt; }
table[rules] {border-left:solid black 0.4pt; border-right:solid black 0.4pt; }
td.td00{ padding-left:0pt; padding-right:0pt; }
td.td01{ padding-left:0pt; padding-right:5pt; }
td.td10{ padding-left:5pt; padding-right:0pt; }
td.td11{ padding-left:5pt; padding-right:5pt; }
table[rules] {border-left:solid black 0.4pt; border-right:solid black 0.4pt; }
.hline hr, .cline hr{ height : 1px; margin:0px; }
.tabbing-right {text-align:right;}
span.TEX {letter-spacing: -0.125em; }
span.TEX span.E{ position:relative;top:0.5ex;left:-0.0417em;}
a span.TEX span.E {text-decoration: none; }
span.LATEX span.A{ position:relative; top:-0.5ex; left:-0.4em; font-size:85%;}
span.LATEX span.TEX{ position:relative; left: -0.4em; }
div.float img, div.float .caption {text-align:center;}
div.figure img, div.figure .caption {text-align:center;}
.marginpar {width:20%; float:right; text-align:left; margin-left:auto; margin-top:0.5em; font-size:85%; text-decoration:underline;}
.marginpar p{margin-top:0.4em; margin-bottom:0.4em;}
.equation td{text-align:center; vertical-align:middle; }
td.eq-no{ width:5%; }
table.equation { width:100%; } 
div.math-display, div.par-math-display{text-align:center;}
math .texttt { font-family: monospace; }
math .textit { font-style: italic; }
math .textsl { font-style: oblique; }
math .textsf { font-family: sans-serif; }
math .textbf { font-weight: bold; }
.partToc a, .partToc, .likepartToc a, .likepartToc {line-height: 200%; font-weight:bold; font-size:110%;}
.chapterToc a, .chapterToc, .likechapterToc a, .likechapterToc, .appendixToc a, .appendixToc {line-height: 200%; font-weight:bold;}
.index-item, .index-subitem, .index-subsubitem {display:block}
.caption td.id{font-weight: bold; white-space: nowrap; }
table.caption {text-align:center;}
h1.partHead{text-align: center}
p.bibitem { text-indent: -2em; margin-left: 2em; margin-top:0.6em; margin-bottom:0.6em; }
p.bibitem-p { text-indent: 0em; margin-left: 2em; margin-top:0.6em; margin-bottom:0.6em; }
.paragraphHead, .likeparagraphHead { margin-top:2em; font-weight: bold;}
.subparagraphHead, .likesubparagraphHead { font-weight: bold;}
.quote {margin-bottom:0.25em; margin-top:0.25em; margin-left:1em; margin-right:1em; text-align:justify;}
.verse{white-space:nowrap; margin-left:2em}
div.maketitle {text-align:center;}
h2.titleHead{text-align:center;}
div.maketitle{ margin-bottom: 2em; }
div.author, div.date {text-align:center;}
div.thanks{text-align:left; margin-left:10%; font-size:85%; font-style:italic; }
div.author{white-space: nowrap;}
.quotation {margin-bottom:0.25em; margin-top:0.25em; margin-left:1em; }
h1.partHead{text-align: center}
.sectionToc, .likesectionToc {margin-left:2em;}
.subsectionToc, .likesubsectionToc {margin-left:4em;}
.subsubsectionToc, .likesubsubsectionToc {margin-left:6em;}
.frenchb-nbsp{font-size:75%;}
.frenchb-thinspace{font-size:75%;}
.figure img.graphics {margin-left:10%;}
/* end css.sty */

\title{Notions generales}
\author{}
\date{}

\begin{document}
\maketitle

\textbf{Warning: \href{http://www.math.union.edu/locate/jsMath}{jsMath}
requires JavaScript to process the mathematics on this page.\\ If your
browser supports JavaScript, be sure it is enabled.}

\begin{center}\rule{3in}{0.4pt}\end{center}

{[}\href{coursse87.html}{next}{]}
{[}\hyperref[tailcoursse86.html]{tail}{]}
{[}\href{coursch17.html\#coursse86.html}{up}{]}

\subsubsection{16.1 Notions générales}

\paragraph{16.1.1 Solutions d'une équation différentielle}

Définition~16.1.1 Soit E un espace vectoriel normé de dimension finie, n
≥ 1, W un ouvert de ℝ × \{E\}\^{}\{n+1\} et G : W → E' une application.
On appelle solution de l'équation différentielle
G(t,y,y',\textbackslash{}mathop\{\textbackslash{}mathop\{\ldots{}\}\},\{y\}\^{}\{(n)\})
= 0 tout couple (I,φ) d'un intervalle I de ℝ et d'une application φ : I
→ E de classe \{C\}\^{}\{n\} telle que

\textbackslash{}begin\{eqnarray*\} \textbackslash{}mathop\{∀\}t ∈ I,\&
\&
(t,φ(t),φ'(t),\textbackslash{}mathop\{\textbackslash{}mathop\{\ldots{}\}\},\{φ\}\^{}\{(n)\}(t))
∈ W \%\& \textbackslash{}\textbackslash{} \& \& \textbackslash{}text\{
et
\}G(t,φ(t),φ'(t),\textbackslash{}mathop\{\textbackslash{}mathop\{\ldots{}\}\},\{φ\}\^{}\{(n)\}(t))
= 0\%\& \textbackslash{}\textbackslash{}
\textbackslash{}end\{eqnarray*\}

On dira alors que n est l'ordre de l'équation différentielle.

Remarque~16.1.1 Dans le cas particulier où E' = E et où
G(t,\{y\}\_\{0\},\textbackslash{}mathop\{\textbackslash{}mathop\{\ldots{}\}\},\{y\}\_\{n\})
= \{y\}\_\{n\} −
F(t,\{y\}\_\{0\},\textbackslash{}mathop\{\textbackslash{}mathop\{\ldots{}\}\},\{y\}\_\{n−1\})
avec F application d'un ouvert U de ℝ × \{E\}\^{}\{n\} dans E, on
obtient une équation différentielle de la forme \{y\}\^{}\{(n)\} =
F(t,y,y',\textbackslash{}mathop\{\textbackslash{}mathop\{\ldots{}\}\},\{y\}\^{}\{(n−1)\}).
On dira qu'une telle équation est sous forme normale. Une solution d'une
telle équation est donc un couple (I,φ) où φ : I → E est de classe
\{C\}\^{}\{n\} et vérifie

\textbackslash{}begin\{eqnarray*\} \textbackslash{}mathop\{∀\}t ∈ I,\&
\&
(t,φ(t),\textbackslash{}mathop\{\textbackslash{}mathop\{\ldots{}\}\},\{φ\}\^{}\{(n−1)\}(t))
∈ U \%\& \textbackslash{}\textbackslash{} \& \& \textbackslash{}text\{
et \}\{φ\}\^{}\{(n)\}(t) =
F(t,φ(t),φ'(t),\textbackslash{}mathop\{\textbackslash{}mathop\{\ldots{}\}\},\{φ\}\^{}\{(n−1)\}(t))\%\&
\textbackslash{}\textbackslash{} \textbackslash{}end\{eqnarray*\}

Par la suite on s'intéressera plus particulièrement aux équations
différentielles sous forme normale. Il est parfois possible de passer
d'une équation sous forme générale à une équation sous forme normale en
résolvant l'équation
G(t,\{y\}\_\{0\},\textbackslash{}mathop\{\textbackslash{}mathop\{\ldots{}\}\},\{y\}\_\{n\})
= 0 sous la forme \{y\}\_\{n\} =
F(t,\{y\}\_\{0\},\textbackslash{}mathop\{\textbackslash{}mathop\{\ldots{}\}\},\{y\}\_\{n−1\}).
A cet égard, le théorème des fonctions implicites peut rendre de grands
services.

\paragraph{16.1.2 Type de problèmes}

Nous distinguerons par la suite deux types de problèmes concernant les
équations différentielles. Le premier type de problème est appelé le
problème à condition initiale (ou problème de Cauchy-Lipschitz), le
second problème à conditions aux limites (ou conditions au bord).

Problème 1 (à conditions initiales). On considère une équation
différentielle sous forme normale \{y\}\^{}\{(n)\} =
F(t,y,y',\textbackslash{}mathop\{\textbackslash{}mathop\{\ldots{}\}\},\{y\}\^{}\{(n−1)\})
où F est une application d'un ouvert U de ℝ × \{E\}\^{}\{n\} dans E et
on se donne
(\{t\}\_\{0\},\{y\}\_\{0\},\textbackslash{}mathop\{\textbackslash{}mathop\{\ldots{}\}\},\{y\}\_\{n−1\})
∈ U. Peut-on trouver une solution (I,φ) de cette équation différentielle
telle que \{t\}\_\{0\} ∈ I, φ(\{t\}\_\{0\}) =
\{y\}\_\{0\},\textbackslash{}mathop\{\textbackslash{}mathop\{\ldots{}\}\},\{φ\}\^{}\{(n−1)\}(\{t\}\_\{0\})
= \{y\}\_\{n−1\}~? Si oui, a-t-on en un certain sens unicité d'une
solution~?

Problème 2 (à conditions aux limites). On considère une équation
différentielle sous forme normale \{y\}\^{}\{(n)\} =
F(t,y,y',\textbackslash{}mathop\{\textbackslash{}mathop\{\ldots{}\}\},\{y\}\^{}\{(n−1)\})
où F est une application d'un ouvert U de ℝ × \{E\}\^{}\{n\} dans E et
on se donne a et b ∈ ℝ. Peut-on trouver une solution (I,φ) de cette
équation différentielle vérifiant des équations
\{G\}\_\{1\}(a,φ(a),\textbackslash{}mathop\{\textbackslash{}mathop\{\ldots{}\}\},\{φ\}\^{}\{(n−1)\}(a))
= 0 et
\{G\}\_\{2\}(b,φ(b),\textbackslash{}mathop\{\textbackslash{}mathop\{\ldots{}\}\},\{φ\}\^{}\{(n−1)\}(b))
= 0, où \{G\}\_\{1\} et \{G\}\_\{2\} sont deux applications de ℝ ×
\{E\}\^{}\{n\} respectivement dans deux espaces vectoriels normés
\{E\}\_\{1\} et \{E\}\_\{2\}.

Le problème avec conditions aux limites est beaucoup plus difficile et a
des réponses beaucoup plus complexes que le problème avec conditions
initiales. Nous ne le traiterons donc pas en dehors d'exercices
particuliers et nous intéresserons presque exclusivement au problème à
conditions initiales.

\paragraph{16.1.3 Réduction à l'ordre 1}

Considérons une équation différentielle sous forme normale
\{y\}\^{}\{(n)\} =
f(t,y,y',\textbackslash{}mathop\{\textbackslash{}mathop\{\ldots{}\}\},\{y\}\^{}\{(n−1)\})
où f est une application d'un ouvert U de ℝ × \{E\}\^{}\{n\} dans E et
soit (I,φ) une solution de l'équation différentielle. Posons
\{φ\}\_\{1\}(t) =
φ(t),\textbackslash{}mathop\{\textbackslash{}mathop\{\ldots{}\}\},\{φ\}\_\{n\}(t)
= \{φ\}\^{}\{(n−1)\}(t) et Φ(t) =
(\{φ\}\_\{1\}(t),\textbackslash{}mathop\{\textbackslash{}mathop\{\ldots{}\}\},\{φ\}\_\{n\}(t))
=
(φ(t),φ'(t),\textbackslash{}mathop\{\textbackslash{}mathop\{\ldots{}\}\},\{φ\}\^{}\{(n−1)\}(t)).
Alors (I,Φ) est de classe \{C\}\^{}\{1\} et on a

\textbackslash{}begin\{eqnarray*\} Φ'(t)\& =\& \textbackslash{}left
(φ'(t),\textbackslash{}mathop\{\textbackslash{}mathop\{\ldots{}\}\},\{φ\}\^{}\{(n)\}(t)\textbackslash{}right
) \%\& \textbackslash{}\textbackslash{} \& =\& \textbackslash{}left
(φ'(t),\textbackslash{}mathop\{\textbackslash{}mathop\{\ldots{}\}\},\{φ\}\^{}\{(n−1)\}(t),f(t,φ(t),\textbackslash{}mathop\{\textbackslash{}mathop\{\ldots{}\}\},\{φ\}\^{}\{(n−1)\}(t))\textbackslash{}right
) \%\& \textbackslash{}\textbackslash{} \& =\& \textbackslash{}left
(\{φ\}\_\{2\}(t),\textbackslash{}mathop\{\textbackslash{}mathop\{\ldots{}\}\},\{φ\}\_\{n\}(t),f(t,\{φ\}\_\{1\}(t),\textbackslash{}mathop\{\textbackslash{}mathop\{\ldots{}\}\},\{φ\}\_\{n\}(t))\textbackslash{}right
) = F(t,Φ(t))\%\& \textbackslash{}\textbackslash{}
\textbackslash{}end\{eqnarray*\}

si l'on définit F : U → \{E\}\^{}\{n\} par
F(t,(\{y\}\_\{1\},\textbackslash{}mathop\{\textbackslash{}mathop\{\ldots{}\}\},\{y\}\_\{n\}))
=
(\{y\}\_\{2\},\textbackslash{}mathop\{\textbackslash{}mathop\{\ldots{}\}\},\{y\}\_\{n\},F(t,\{y\}\_\{1\},\textbackslash{}mathop\{\textbackslash{}mathop\{\ldots{}\}\},\{y\}\_\{n\})).
Donc (I,Φ) est solution de l'équation différentielle Y ' = F(t,Y ).

Inversement, donnons nous une solution (I,Φ) de l'équation
différentielle Y ' = F(t,Y ) où F : U → \{E\}\^{}\{n\} est définie par
F(t,(\{y\}\_\{1\},\textbackslash{}mathop\{\textbackslash{}mathop\{\ldots{}\}\},\{y\}\_\{n\}))
=
(\{y\}\_\{2\},\textbackslash{}mathop\{\textbackslash{}mathop\{\ldots{}\}\},\{y\}\_\{n\},f(t,\{y\}\_\{1\},\textbackslash{}mathop\{\textbackslash{}mathop\{\ldots{}\}\},\{y\}\_\{n\})).
Posons Φ(t) =
(\{φ\}\_\{1\}(t),\textbackslash{}mathop\{\textbackslash{}mathop\{\ldots{}\}\},\{φ\}\_\{n\}(t)).
On a donc

\textbackslash{}begin\{eqnarray*\} Φ'(t)\& =\&
(\{φ\}\_\{1\}'(t),\textbackslash{}mathop\{\textbackslash{}mathop\{\ldots{}\}\},\{φ\}\_\{n−1\}'(t),\{φ\}\_\{n\}'(t))
\%\& \textbackslash{}\textbackslash{} \& =\& F(t,Φ(t)) =
(\{φ\}\_\{2\}(t),\textbackslash{}mathop\{\textbackslash{}mathop\{\ldots{}\}\},\{φ\}\_\{n\}(t),f(t,\{φ\}\_\{1\}(t),\textbackslash{}mathop\{\textbackslash{}mathop\{\ldots{}\}\},\{φ\}\_\{n\}(t)))\%\&
\textbackslash{}\textbackslash{} \textbackslash{}end\{eqnarray*\}

On en déduit que pour i ∈ {[}1,n − 1{]} on a \{φ\}\_\{i\}'(t) =
\{φ\}\_\{i+1\}(t) et une récurrence évidente montre que pour i ∈
{[}2,n{]}, \{φ\}\_\{i\}(t) = \{φ\}\_\{1\}\^{}\{(i−1)\}(t). Mais alors la
dernière équation se traduit par \{φ\}\_\{1\}\^{}\{(n)\}(t) =
(\{φ\}\^{}\{(n−1)\})'(t) = \{φ\}\_\{n\}'(t) =
f(t,\{φ\}\_\{1\}(t),\textbackslash{}mathop\{\textbackslash{}mathop\{\ldots{}\}\},\{φ\}\_\{n\}(t))
=
f(t,\{φ\}\_\{1\}(t),\{φ\}\_\{1\}'(t),\textbackslash{}mathop\{\textbackslash{}mathop\{\ldots{}\}\},\{φ\}\_\{1\}\^{}\{(n−1)\}(t)),
si bien que (I,\{φ\}\_\{1\}) est de classe \{C\}\^{}\{n\} et solution de
l'équation différentielle \{y\}\^{}\{(n)\} =
f(t,y,y',\textbackslash{}mathop\{\textbackslash{}mathop\{\ldots{}\}\},\{y\}\^{}\{(n−1)\}).
On en déduit donc le théorème suivant

Théorème~16.1.1 Soit U un ouvert de U × \{E\}\^{}\{n\} et f : U → E.
Soit F : U → \{E\}\^{}\{n\} définie par
F(t,(\{y\}\_\{1\},\textbackslash{}mathop\{\textbackslash{}mathop\{\ldots{}\}\},\{y\}\_\{n\}))
=
(\{y\}\_\{2\},\textbackslash{}mathop\{\textbackslash{}mathop\{\ldots{}\}\},\{y\}\_\{n\},F(t,\{y\}\_\{1\},\textbackslash{}mathop\{\textbackslash{}mathop\{\ldots{}\}\},\{y\}\_\{n\})).
Alors l'application (I,φ)\textbackslash{}mathrel\{↦\}(I,Φ) définie par
Φ(t) =
(φ(t),\textbackslash{}mathop\{\textbackslash{}mathop\{\ldots{}\}\},\{φ\}\^{}\{(n−1)\}(t))
est une bijection de l'ensemble des solutions de l'équation
différentielle d'ordre n, \{y\}\^{}\{(n)\} =
f(t,y,y',\textbackslash{}mathop\{\textbackslash{}mathop\{\ldots{}\}\},\{y\}\^{}\{(n−1)\}),
sur l'ensemble des solutions de l'équation différentielle d'ordre un Y '
= F(t,Y ).

Remarque~16.1.2 En ce qui concerne le type de problème étudié, il est
clair que cette bijection préserve les problèmes à conditions initiales.
Autrement dit (I,φ) est une solution de \{y\}\^{}\{(n)\} =
f(t,y,y',\textbackslash{}mathop\{\textbackslash{}mathop\{\ldots{}\}\},\{y\}\^{}\{(n−1)\})
vérifiant les conditions initiales φ(\{t\}\_\{0\}) =
\{y\}\_\{0\},\textbackslash{}mathop\{\textbackslash{}mathop\{\ldots{}\}\},\{φ\}\^{}\{(n−1)\}(\{t\}\_\{0\})
= \{y\}\_\{n−1\} si et seulement si~(I,Φ) est une solution de Y ' =
F(t,Y ) vérifiant la condition initiale Φ(\{t\}\_\{0\}) = \{Y \}\_\{0\}
avec \{Y \}\_\{0\} =
(\{y\}\_\{0\},\textbackslash{}mathop\{\textbackslash{}mathop\{\ldots{}\}\},\{y\}\_\{n−1\}).
Nous pourrons donc par la suite, pour ce qui concerne les problèmes
d'existence et d'unicité du problème à conditions initiales, nous borner
à l'étude des équations différentielles d'ordre 1.

\paragraph{16.1.4 Equivalence avec une équation intégrale}

Théorème~16.1.2 Soit E un espace vectoriel normé de dimension finie, U
un ouvert de ℝ × E, F : U → E continue, (\{t\}\_\{0\},\{y\}\_\{0\}) ∈ U,
I un intervalle de ℝ contenant \{t\}\_\{0\} et φ une application
continue de I dans E. Alors on a équivalence de

\begin{itemize}
\itemsep1pt\parskip0pt\parsep0pt
\item
  (i) (I,φ) est une solution de l'équation différentielle y' = F(t,y)
  vérifiant φ(\{t\}\_\{0\}) = \{y\}\_\{0\}
\item
  (ii) \textbackslash{}mathop\{∀\}t ∈ I, φ(t) = \{y\}\_\{0\}
  +\{\textbackslash{}mathop\{∫ \} \}\_\{\{t\}\_\{0\}\}\^{}\{t\}F(u,φ(u))
  du.
\end{itemize}

Démonstration Supposons (i) vérifié. Alors, comme φ est de classe
\{C\}\^{}\{1\}, on a

φ(t) = φ(\{t\}\_\{0\}) +\{\textbackslash{}mathop\{∫ \}
\}\_\{\{t\}\_\{0\}\}\^{}\{t\}φ'(u) du = \{y\}\_\{ 0\}
+\{\textbackslash{}mathop\{∫ \} \}\_\{\{t\}\_\{0\}\}\^{}\{t\}F(u,φ(u))
du

ce qui montre que (i) ⇒(ii). Inversement supposons (ii) vérifié. Il est
clair que φ(\{t\}\_\{0\}) = \{y\}\_\{0\}. De plus comme
u\textbackslash{}mathrel\{↦\}F(u,φ(u)) est continue,
t\textbackslash{}mathrel\{↦\}\{\textbackslash{}mathop\{∫ \}
\}\_\{\{t\}\_\{0\}\}\^{}\{t\}F(u,φ(u)) du est de classe \{C\}\^{}\{1\}
et sa dérivée est F(t,φ(t))~; on en déduit que φ est de classe
\{C\}\^{}\{1\} et que φ'(t) = F(t,φ(t)), ce qui achève la démonstration.

\paragraph{16.1.5 Le lemme de Gronwall}

On utilisera par la suite le lemme suivant~:

Lemme~16.1.3 (Gronwall). Soit c un réel positif, g : {[}a,b{[}→ ℝ
continue positive. Soit u : I → ℝ continue telle que
\textbackslash{}mathop\{∀\}x ∈ {[}a,b{[}, \textbar{}u(x)\textbar{}≤ c
+\{\textbackslash{}mathop\{∫ \}
\}\_\{a\}\^{}\{x\}\textbar{}u(t)\textbar{}g(t) dt. Alors
\textbackslash{}mathop\{∀\}x ∈ {[}a,b{[}, \textbar{}u(x)\textbar{}≤
c\textbackslash{}mathop\{exp\} (\{\textbackslash{}mathop\{∫ \}
\}\_\{a\}\^{}\{x\}g(t) dt).

Démonstration Posons v(x) = c +\{\textbackslash{}mathop\{∫ \}
\}\_\{a\}\^{}\{x\}\textbar{}u(t)\textbar{}g(t) dt. Comme u et g sont
continues, v est de classe \{C\}\^{}\{1\} et on a v'(x) =
\textbar{}u(x)\textbar{}g(x) ≤ v(x)g(x) puisque
\textbar{}u(x)\textbar{}≤ v(x) et g(x) ≥ 0 par hypothèse. Soit w(x) =
v(x)\textbackslash{}mathop\{exp\} (−\{\textbackslash{}mathop\{∫ \}
\}\_\{a\}\^{}\{x\}g(t) dt). On a alors

\textbackslash{}begin\{eqnarray*\} w'(x)\& =\&
v'(x)\textbackslash{}mathop\{exp\} (−\{\textbackslash{}mathop\{∫ \}
\}\_\{a\}\^{}\{x\}g(t) dt) − v(x)g(x)\textbackslash{}mathop\{exp\}
(−\{\textbackslash{}mathop\{∫ \} \}\_\{a\}\^{}\{x\}g(t) dt)\%\&
\textbackslash{}\textbackslash{} \& =\& (v'(x) −
v(x)g(x))\textbackslash{}mathop\{exp\} (−\{\textbackslash{}mathop\{∫ \}
\}\_\{a\}\^{}\{x\}g(t) dt) ≤ 0 \%\& \textbackslash{}\textbackslash{}
\textbackslash{}end\{eqnarray*\}

On en déduit que w est décroissante, et donc
\textbackslash{}mathop\{∀\}x ∈ {[}a,b{[}, w(x) ≤ w(a). Mais w(a) = v(a)
= c. On a donc, pour x ∈ {[}a,b{[}, \textbar{}u(x)\textbar{}≤ v(x) ≤
c\textbackslash{}mathop\{exp\} (\{\textbackslash{}mathop\{∫ \}
\}\_\{a\}\^{}\{x\}g(t) dt), ce qu'on voulait démontrer.

Remarque~16.1.3 De la même fa\textbackslash{}c\{c\}on on montre que si
\textbackslash{}mathop\{∀\}x ∈{]}b,a{]}, \textbar{}u(x)\textbar{}≤ c
+\{\textbackslash{}mathop\{∫ \}
\}\_\{x\}\^{}\{a\}\textbar{}u(t)\textbar{}g(t) dt, alors
\textbackslash{}mathop\{∀\}x ∈{]}b,a{]}, \textbar{}u(x)\textbar{}≤
c\textbackslash{}mathop\{exp\} (\{\textbackslash{}mathop\{∫ \}
\}\_\{x\}\^{}\{a\}g(t) dt).

Remarque~16.1.4 Le cas c = 0 jouera un rôle crucial dans les
démonstrations d'unicité. On obtient alors la nullité de u sur
l'intervalle en question.

{[}\href{coursse87.html}{next}{]} {[}\href{coursse86.html}{front}{]}
{[}\href{coursch17.html\#coursse86.html}{up}{]}

\end{document}

% \documentclass[]{article}
\usepackage[T1]{fontenc}
\usepackage{lmodern}
\usepackage{amssymb,amsmath}
\usepackage{ifxetex,ifluatex}
\usepackage{fixltx2e} % provides \textsubscript
% use upquote if available, for straight quotes in verbatim environments
\IfFileExists{upquote.sty}{\usepackage{upquote}}{}
\ifnum 0\ifxetex 1\fi\ifluatex 1\fi=0 % if pdftex
  \usepackage[utf8]{inputenc}
\else % if luatex or xelatex
  \ifxetex
    \usepackage{mathspec}
    \usepackage{xltxtra,xunicode}
  \else
    \usepackage{fontspec}
  \fi
  \defaultfontfeatures{Mapping=tex-text,Scale=MatchLowercase}
  \newcommand{\euro}{€}
\fi
% use microtype if available
\IfFileExists{microtype.sty}{\usepackage{microtype}}{}
\ifxetex
  \usepackage[setpagesize=false, % page size defined by xetex
              unicode=false, % unicode breaks when used with xetex
              xetex]{hyperref}
\else
  \usepackage[unicode=true]{hyperref}
\fi
\hypersetup{breaklinks=true,
            bookmarks=true,
            pdfauthor={},
            pdftitle={Theorie de Cauchy-Lipschitz},
            colorlinks=true,
            citecolor=blue,
            urlcolor=blue,
            linkcolor=magenta,
            pdfborder={0 0 0}}
\urlstyle{same}  % don't use monospace font for urls
\setlength{\parindent}{0pt}
\setlength{\parskip}{6pt plus 2pt minus 1pt}
\setlength{\emergencystretch}{3em}  % prevent overfull lines
\setcounter{secnumdepth}{0}
 
/* start css.sty */
.cmr-5{font-size:50%;}
.cmr-7{font-size:70%;}
.cmmi-5{font-size:50%;font-style: italic;}
.cmmi-7{font-size:70%;font-style: italic;}
.cmmi-10{font-style: italic;}
.cmsy-5{font-size:50%;}
.cmsy-7{font-size:70%;}
.cmex-7{font-size:70%;}
.cmex-7x-x-71{font-size:49%;}
.msbm-7{font-size:70%;}
.cmtt-10{font-family: monospace;}
.cmti-10{ font-style: italic;}
.cmbx-10{ font-weight: bold;}
.cmr-17x-x-120{font-size:204%;}
.cmsl-10{font-style: oblique;}
.cmti-7x-x-71{font-size:49%; font-style: italic;}
.cmbxti-10{ font-weight: bold; font-style: italic;}
p.noindent { text-indent: 0em }
td p.noindent { text-indent: 0em; margin-top:0em; }
p.nopar { text-indent: 0em; }
p.indent{ text-indent: 1.5em }
@media print {div.crosslinks {visibility:hidden;}}
a img { border-top: 0; border-left: 0; border-right: 0; }
center { margin-top:1em; margin-bottom:1em; }
td center { margin-top:0em; margin-bottom:0em; }
.Canvas { position:relative; }
li p.indent { text-indent: 0em }
.enumerate1 {list-style-type:decimal;}
.enumerate2 {list-style-type:lower-alpha;}
.enumerate3 {list-style-type:lower-roman;}
.enumerate4 {list-style-type:upper-alpha;}
div.newtheorem { margin-bottom: 2em; margin-top: 2em;}
.obeylines-h,.obeylines-v {white-space: nowrap; }
div.obeylines-v p { margin-top:0; margin-bottom:0; }
.overline{ text-decoration:overline; }
.overline img{ border-top: 1px solid black; }
td.displaylines {text-align:center; white-space:nowrap;}
.centerline {text-align:center;}
.rightline {text-align:right;}
div.verbatim {font-family: monospace; white-space: nowrap; text-align:left; clear:both; }
.fbox {padding-left:3.0pt; padding-right:3.0pt; text-indent:0pt; border:solid black 0.4pt; }
div.fbox {display:table}
div.center div.fbox {text-align:center; clear:both; padding-left:3.0pt; padding-right:3.0pt; text-indent:0pt; border:solid black 0.4pt; }
div.minipage{width:100%;}
div.center, div.center div.center {text-align: center; margin-left:1em; margin-right:1em;}
div.center div {text-align: left;}
div.flushright, div.flushright div.flushright {text-align: right;}
div.flushright div {text-align: left;}
div.flushleft {text-align: left;}
.underline{ text-decoration:underline; }
.underline img{ border-bottom: 1px solid black; margin-bottom:1pt; }
.framebox-c, .framebox-l, .framebox-r { padding-left:3.0pt; padding-right:3.0pt; text-indent:0pt; border:solid black 0.4pt; }
.framebox-c {text-align:center;}
.framebox-l {text-align:left;}
.framebox-r {text-align:right;}
span.thank-mark{ vertical-align: super }
span.footnote-mark sup.textsuperscript, span.footnote-mark a sup.textsuperscript{ font-size:80%; }
div.tabular, div.center div.tabular {text-align: center; margin-top:0.5em; margin-bottom:0.5em; }
table.tabular td p{margin-top:0em;}
table.tabular {margin-left: auto; margin-right: auto;}
div.td00{ margin-left:0pt; margin-right:0pt; }
div.td01{ margin-left:0pt; margin-right:5pt; }
div.td10{ margin-left:5pt; margin-right:0pt; }
div.td11{ margin-left:5pt; margin-right:5pt; }
table[rules] {border-left:solid black 0.4pt; border-right:solid black 0.4pt; }
td.td00{ padding-left:0pt; padding-right:0pt; }
td.td01{ padding-left:0pt; padding-right:5pt; }
td.td10{ padding-left:5pt; padding-right:0pt; }
td.td11{ padding-left:5pt; padding-right:5pt; }
table[rules] {border-left:solid black 0.4pt; border-right:solid black 0.4pt; }
.hline hr, .cline hr{ height : 1px; margin:0px; }
.tabbing-right {text-align:right;}
span.TEX {letter-spacing: -0.125em; }
span.TEX span.E{ position:relative;top:0.5ex;left:-0.0417em;}
a span.TEX span.E {text-decoration: none; }
span.LATEX span.A{ position:relative; top:-0.5ex; left:-0.4em; font-size:85%;}
span.LATEX span.TEX{ position:relative; left: -0.4em; }
div.float img, div.float .caption {text-align:center;}
div.figure img, div.figure .caption {text-align:center;}
.marginpar {width:20%; float:right; text-align:left; margin-left:auto; margin-top:0.5em; font-size:85%; text-decoration:underline;}
.marginpar p{margin-top:0.4em; margin-bottom:0.4em;}
.equation td{text-align:center; vertical-align:middle; }
td.eq-no{ width:5%; }
table.equation { width:100%; } 
div.math-display, div.par-math-display{text-align:center;}
math .texttt { font-family: monospace; }
math .textit { font-style: italic; }
math .textsl { font-style: oblique; }
math .textsf { font-family: sans-serif; }
math .textbf { font-weight: bold; }
.partToc a, .partToc, .likepartToc a, .likepartToc {line-height: 200%; font-weight:bold; font-size:110%;}
.chapterToc a, .chapterToc, .likechapterToc a, .likechapterToc, .appendixToc a, .appendixToc {line-height: 200%; font-weight:bold;}
.index-item, .index-subitem, .index-subsubitem {display:block}
.caption td.id{font-weight: bold; white-space: nowrap; }
table.caption {text-align:center;}
h1.partHead{text-align: center}
p.bibitem { text-indent: -2em; margin-left: 2em; margin-top:0.6em; margin-bottom:0.6em; }
p.bibitem-p { text-indent: 0em; margin-left: 2em; margin-top:0.6em; margin-bottom:0.6em; }
.paragraphHead, .likeparagraphHead { margin-top:2em; font-weight: bold;}
.subparagraphHead, .likesubparagraphHead { font-weight: bold;}
.quote {margin-bottom:0.25em; margin-top:0.25em; margin-left:1em; margin-right:1em; text-align:justify;}
.verse{white-space:nowrap; margin-left:2em}
div.maketitle {text-align:center;}
h2.titleHead{text-align:center;}
div.maketitle{ margin-bottom: 2em; }
div.author, div.date {text-align:center;}
div.thanks{text-align:left; margin-left:10%; font-size:85%; font-style:italic; }
div.author{white-space: nowrap;}
.quotation {margin-bottom:0.25em; margin-top:0.25em; margin-left:1em; }
h1.partHead{text-align: center}
.sectionToc, .likesectionToc {margin-left:2em;}
.subsectionToc, .likesubsectionToc {margin-left:4em;}
.subsubsectionToc, .likesubsubsectionToc {margin-left:6em;}
.frenchb-nbsp{font-size:75%;}
.frenchb-thinspace{font-size:75%;}
.figure img.graphics {margin-left:10%;}
/* end css.sty */

\title{Theorie de Cauchy-Lipschitz}
\author{}
\date{}

\begin{document}
\maketitle

\textbf{Warning: 
requires JavaScript to process the mathematics on this page.\\ If your
browser supports JavaScript, be sure it is enabled.}

\begin{center}\rule{3in}{0.4pt}\end{center}

[
[
[]
[

\subsubsection{16.2 Théorie de Cauchy-Lipschitz}

\paragraph{16.2.1 Unicité de solutions, solutions maximales}

Définition~16.2.1 Soit E un espace vectoriel normé de dimension finie, U
un ouvert de \mathbb{R}~ \times E et F : U \rightarrow~ E. On dira que F vérifie la condition
d'unicité du problème de Cauchy Lipschitz si pour toutes solutions (I,\phi)
et (J,\psi) de l'équation différentielle y' = F(t,y) qui coïncident en un
point t_0 \in I \bigcap J (c'est-à-dire que \phi(t_0) =
\psi(t_0)), on a

\forall~~t \in I \bigcap J, \phi(t) = \psi(t)

Définition~16.2.2 Soit E un espace vectoriel normé de dimension finie, U
un ouvert de \mathbb{R}~ \times E et F : U \rightarrow~ E. On dira que F vérifie la condition
d'existence au problème de Cauchy-Lipschitz, si pour tout
(t_0,y_0) \in U, il existe \eta > 0 et une
solution (]t_0 - \eta,t_0 + \eta[,\phi) de l'équation
différentielle y' = F(t,y) vérifiant la condition \phi(t_0) =
y_0.

Définition~16.2.3 Soit (I,\phi) et (J,\psi) deux solutions de l'équation
différentielle y' = F(t,y). On dira que (J,\psi) est un prolongement de
(I,\phi), et on notera (I,\phi) \prec~ (J,\psi) si I \subset~ J et \phi est la restriction de \psi
à I.

Remarque~16.2.1 Il est clair qu'il s'agit d'une relation d'ordre partiel
sur l'ensemble des solutions de l'équation différentielle.

Définition~16.2.4 On appelle solution maximale de l'équation
différentielle y' = F(t,y) toute solution (I,\phi) qui est maximale pour la
relation d'ordre \prec~.

Remarque~16.2.2 Ceci signifie donc qu'il n'existe aucune solution
définie sur un intervalle I' contenant strictement I et qui prolonge \phi.

Théorème~16.2.1 (existence et unicité d'une solution maximale à
condition initiale donnée). On suppose que F vérifie les conditions
d'existence et d'unicité au problème de Cauchy Lipschitz. Soit
(t_0,y_0) \in U~; alors il existe une unique solution
maximale (I_0,\phi_0) de l'équation différentielle y' =
F(t,y) qui vérifie \phi_0(t_0) = y_0. Pour toute
solution (J,\psi) de l'équation différentielle vérifiant \psi(t_0) =
y_0, on a~:

\text\$J \subset~ I_0\$ et \$\psi\$ est la restriction
de \$\phi_0\$ à \$J\$.

Démonstration Unicité~: Soit (I_0,\phi_0) et
(I_1,\phi_1) deux solutions maximales vérifiant
\phi_0(t_0) = \phi_1(t_0) = y_0.
Définissons I_2 = I_0 \cup I_1 et soit
\phi_2 l'application de I_2 dans E définie par
\phi_2(t) = \left \
\cases \phi_0(t)&si t \in I_0
\cr \phi_1(t)&si t \in I_1\\ 
\right .. Comme \phi_0 et \phi_1 coïncident
sur I_0 \bigcap I_1, \phi_2 est bien définie. On
vérifie facilement qu'elle est de classe \mathcal{C}^1 et solution de
l'équation différentielle y' = F(t,y). La maximalité de
(I_0,\phi_0) et (I_1,\phi_1) exige alors
I_2 = I_0 = I_1 et \phi_2 =
\phi_0 = \phi_1, ce qui montre l'unicité de la solution
maximale.

Existence Soit \left
((I_j,\psi_j)\right )_j\inℱ la
famille de toutes les solutions de l'équation différentielle y' = F(t,y)
définies sur un intervalle I_j non réduit à un point contenant
t_0 et vérifiant \psi_j(t_0) = y_0~;
cette famille est non vide puisque la fonction F vérifie la condition
d'existence au problème de Cauchy-Lipschitz. Posons I_0
= \⋃ ~
_j\inℱI_j et définissons \phi_0 : I_0 \rightarrow~ E
par \phi_0(t) = \psi_j(t) si t \in I_j. Cette
définition est bien cohérente car si t \in I_j \bigcap I_k,
alors \psi_j et \psi_k coïncident sur I_j \bigcap
I_k, et en particulier \psi_j(t) = \psi_k(t). On
vérifie facilement que la fonction \phi_0 est de classe
\mathcal{C}^1 et si t \in I_j, on a \phi'_0(t) =
\psi_j'(t) = F(t,\psi_j(t)) = F(t,\phi_0(t)) ce qui
montre que (I_0,\phi_0) est bien une solution de
l'équation différentielle~; cette solution vérifie bien entendu
\phi_0(t_0) = y_0. De plus, si
(I_0,\phi_0) \prec~ (I_1,\phi_1), on a
\phi_1(t_0) = \phi_0(t_0) = y_0,
ce qui montre que (I_1,\phi_1) est l'une des
(I_j,\psi_j) et que donc I_1 \subset~ I_0~; on
a donc finalement I_0 = I_1 et \phi_0 =
\phi_1 ce qui montre que cette solution est maximale.

Si (J,\psi) est une solution de l'équation différentielle vérifiant
\psi(t_0) = y_0, alors (J,\psi) est l'une des
(I_j,\psi_j) ce qui montre que J \subset~ I_0 et que \psi
= \psi_j est la restriction de \phi_0 à J = I_j.
Ceci achève la démonstration.

Remarque~16.2.3 On constate que du point de vue de la relation \prec~, la
solution maximale vérifiant la condition \phi_0(t_0) =
y_0 est un plus grand élément de l'ensemble des solutions
vérifiant cette condition initiale, ce qui en explique d'ailleurs
l'unicité. Il est clair, d'après la condition d'existence, que
t_0 est un point intérieur à I_0, intervalle de
définition de la solution maximale~; nous allons d'ailleurs préciser ce
point dans la proposition suivante

Théorème~16.2.2 On suppose que F vérifie la condition d'existence et
d'unicité au problème de Cauchy Lipschitz. Alors toute solution maximale
de l'équation différentielle y' = F(t,y) est définie sur un intervalle
ouvert.

Démonstration Soit (I,\phi) une solution maximale et soit a
\in\overline\mathbb{R}~ une borne de I (par exemple la borne
supérieure). Supposons que a \in I si bien que (a,\phi(a)) \in U. D'après la
condition d'existence il existe \eta > 0 et une solution (]a
- \eta,a + \eta[,\psi) vérifiant la condition initiale \psi(a) = \phi(a). D'après la
condition d'unicité, \phi et \psi qui coïncident au point a, coïncident
également sur l'intersection de leurs intervalles de définition, ce qui
permet de définir I_1 = I\cup]a - \eta,a + \eta[ et \phi_1 :
I_1 \rightarrow~ E par \phi_1(t) = \left
\ \cases \phi_0(t)&si t \in I
\cr \psi(t) &si t \in]a - \eta,a + \eta[ 
\right .. Le couple (I_1,\phi_1) est une
solution de l'équation différentielle qui prolonge strictement (I,\phi) ce
qui contredit le caractère maximal de cette solution.

Remarque~16.2.4 On aurait pu aussi dire que si a \in I et si \phi(a) = b,
(I,\phi) est une solution maximale pour la condition initiale \phi(a) = b, ce
qui montre que a appartient à l'intérieur de I comme on l'a déjà
remarqué. Nous avons cependant pensé que la démonstration précédente
était plus constructive.

[
[
[
[

\end{document}

% \documentclass[]{article}
\usepackage[T1]{fontenc}
\usepackage{lmodern}
\usepackage{amssymb,amsmath}
\usepackage{ifxetex,ifluatex}
\usepackage{fixltx2e} % provides \textsubscript
% use upquote if available, for straight quotes in verbatim environments
\IfFileExists{upquote.sty}{\usepackage{upquote}}{}
\ifnum 0\ifxetex 1\fi\ifluatex 1\fi=0 % if pdftex
  \usepackage[utf8]{inputenc}
\else % if luatex or xelatex
  \ifxetex
    \usepackage{mathspec}
    \usepackage{xltxtra,xunicode}
  \else
    \usepackage{fontspec}
  \fi
  \defaultfontfeatures{Mapping=tex-text,Scale=MatchLowercase}
  \newcommand{\euro}{€}
\fi
% use microtype if available
\IfFileExists{microtype.sty}{\usepackage{microtype}}{}
\ifxetex
  \usepackage[setpagesize=false, % page size defined by xetex
              unicode=false, % unicode breaks when used with xetex
              xetex]{hyperref}
\else
  \usepackage[unicode=true]{hyperref}
\fi
\hypersetup{breaklinks=true,
            bookmarks=true,
            pdfauthor={},
            pdftitle={Equations differentielles lineaires d'ordre 1},
            colorlinks=true,
            citecolor=blue,
            urlcolor=blue,
            linkcolor=magenta,
            pdfborder={0 0 0}}
\urlstyle{same}  % don't use monospace font for urls
\setlength{\parindent}{0pt}
\setlength{\parskip}{6pt plus 2pt minus 1pt}
\setlength{\emergencystretch}{3em}  % prevent overfull lines
\setcounter{secnumdepth}{0}
 
/* start css.sty */
.cmr-5{font-size:50%;}
.cmr-7{font-size:70%;}
.cmmi-5{font-size:50%;font-style: italic;}
.cmmi-7{font-size:70%;font-style: italic;}
.cmmi-10{font-style: italic;}
.cmsy-5{font-size:50%;}
.cmsy-7{font-size:70%;}
.cmex-7{font-size:70%;}
.cmex-7x-x-71{font-size:49%;}
.msbm-7{font-size:70%;}
.cmtt-10{font-family: monospace;}
.cmti-10{ font-style: italic;}
.cmbx-10{ font-weight: bold;}
.cmr-17x-x-120{font-size:204%;}
.cmsl-10{font-style: oblique;}
.cmti-7x-x-71{font-size:49%; font-style: italic;}
.cmbxti-10{ font-weight: bold; font-style: italic;}
p.noindent { text-indent: 0em }
td p.noindent { text-indent: 0em; margin-top:0em; }
p.nopar { text-indent: 0em; }
p.indent{ text-indent: 1.5em }
@media print {div.crosslinks {visibility:hidden;}}
a img { border-top: 0; border-left: 0; border-right: 0; }
center { margin-top:1em; margin-bottom:1em; }
td center { margin-top:0em; margin-bottom:0em; }
.Canvas { position:relative; }
li p.indent { text-indent: 0em }
.enumerate1 {list-style-type:decimal;}
.enumerate2 {list-style-type:lower-alpha;}
.enumerate3 {list-style-type:lower-roman;}
.enumerate4 {list-style-type:upper-alpha;}
div.newtheorem { margin-bottom: 2em; margin-top: 2em;}
.obeylines-h,.obeylines-v {white-space: nowrap; }
div.obeylines-v p { margin-top:0; margin-bottom:0; }
.overline{ text-decoration:overline; }
.overline img{ border-top: 1px solid black; }
td.displaylines {text-align:center; white-space:nowrap;}
.centerline {text-align:center;}
.rightline {text-align:right;}
div.verbatim {font-family: monospace; white-space: nowrap; text-align:left; clear:both; }
.fbox {padding-left:3.0pt; padding-right:3.0pt; text-indent:0pt; border:solid black 0.4pt; }
div.fbox {display:table}
div.center div.fbox {text-align:center; clear:both; padding-left:3.0pt; padding-right:3.0pt; text-indent:0pt; border:solid black 0.4pt; }
div.minipage{width:100%;}
div.center, div.center div.center {text-align: center; margin-left:1em; margin-right:1em;}
div.center div {text-align: left;}
div.flushright, div.flushright div.flushright {text-align: right;}
div.flushright div {text-align: left;}
div.flushleft {text-align: left;}
.underline{ text-decoration:underline; }
.underline img{ border-bottom: 1px solid black; margin-bottom:1pt; }
.framebox-c, .framebox-l, .framebox-r { padding-left:3.0pt; padding-right:3.0pt; text-indent:0pt; border:solid black 0.4pt; }
.framebox-c {text-align:center;}
.framebox-l {text-align:left;}
.framebox-r {text-align:right;}
span.thank-mark{ vertical-align: super }
span.footnote-mark sup.textsuperscript, span.footnote-mark a sup.textsuperscript{ font-size:80%; }
div.tabular, div.center div.tabular {text-align: center; margin-top:0.5em; margin-bottom:0.5em; }
table.tabular td p{margin-top:0em;}
table.tabular {margin-left: auto; margin-right: auto;}
div.td00{ margin-left:0pt; margin-right:0pt; }
div.td01{ margin-left:0pt; margin-right:5pt; }
div.td10{ margin-left:5pt; margin-right:0pt; }
div.td11{ margin-left:5pt; margin-right:5pt; }
table[rules] {border-left:solid black 0.4pt; border-right:solid black 0.4pt; }
td.td00{ padding-left:0pt; padding-right:0pt; }
td.td01{ padding-left:0pt; padding-right:5pt; }
td.td10{ padding-left:5pt; padding-right:0pt; }
td.td11{ padding-left:5pt; padding-right:5pt; }
table[rules] {border-left:solid black 0.4pt; border-right:solid black 0.4pt; }
.hline hr, .cline hr{ height : 1px; margin:0px; }
.tabbing-right {text-align:right;}
span.TEX {letter-spacing: -0.125em; }
span.TEX span.E{ position:relative;top:0.5ex;left:-0.0417em;}
a span.TEX span.E {text-decoration: none; }
span.LATEX span.A{ position:relative; top:-0.5ex; left:-0.4em; font-size:85%;}
span.LATEX span.TEX{ position:relative; left: -0.4em; }
div.float img, div.float .caption {text-align:center;}
div.figure img, div.figure .caption {text-align:center;}
.marginpar {width:20%; float:right; text-align:left; margin-left:auto; margin-top:0.5em; font-size:85%; text-decoration:underline;}
.marginpar p{margin-top:0.4em; margin-bottom:0.4em;}
.equation td{text-align:center; vertical-align:middle; }
td.eq-no{ width:5%; }
table.equation { width:100%; } 
div.math-display, div.par-math-display{text-align:center;}
math .texttt { font-family: monospace; }
math .textit { font-style: italic; }
math .textsl { font-style: oblique; }
math .textsf { font-family: sans-serif; }
math .textbf { font-weight: bold; }
.partToc a, .partToc, .likepartToc a, .likepartToc {line-height: 200%; font-weight:bold; font-size:110%;}
.chapterToc a, .chapterToc, .likechapterToc a, .likechapterToc, .appendixToc a, .appendixToc {line-height: 200%; font-weight:bold;}
.index-item, .index-subitem, .index-subsubitem {display:block}
.caption td.id{font-weight: bold; white-space: nowrap; }
table.caption {text-align:center;}
h1.partHead{text-align: center}
p.bibitem { text-indent: -2em; margin-left: 2em; margin-top:0.6em; margin-bottom:0.6em; }
p.bibitem-p { text-indent: 0em; margin-left: 2em; margin-top:0.6em; margin-bottom:0.6em; }
.paragraphHead, .likeparagraphHead { margin-top:2em; font-weight: bold;}
.subparagraphHead, .likesubparagraphHead { font-weight: bold;}
.quote {margin-bottom:0.25em; margin-top:0.25em; margin-left:1em; margin-right:1em; text-align:justify;}
.verse{white-space:nowrap; margin-left:2em}
div.maketitle {text-align:center;}
h2.titleHead{text-align:center;}
div.maketitle{ margin-bottom: 2em; }
div.author, div.date {text-align:center;}
div.thanks{text-align:left; margin-left:10%; font-size:85%; font-style:italic; }
div.author{white-space: nowrap;}
.quotation {margin-bottom:0.25em; margin-top:0.25em; margin-left:1em; }
h1.partHead{text-align: center}
.sectionToc, .likesectionToc {margin-left:2em;}
.subsectionToc, .likesubsectionToc {margin-left:4em;}
.subsubsectionToc, .likesubsubsectionToc {margin-left:6em;}
.frenchb-nbsp{font-size:75%;}
.frenchb-thinspace{font-size:75%;}
.figure img.graphics {margin-left:10%;}
/* end css.sty */

\title{Equations differentielles lineaires d'ordre 1}
\author{}
\date{}

\begin{document}
\maketitle

\textbf{Warning: \href{http://www.math.union.edu/locate/jsMath}{jsMath}
requires JavaScript to process the mathematics on this page.\\ If your
browser supports JavaScript, be sure it is enabled.}

\begin{center}\rule{3in}{0.4pt}\end{center}

{[}\href{coursse89.html}{next}{]} {[}\href{coursse87.html}{prev}{]}
{[}\href{coursse87.html\#tailcoursse87.html}{prev-tail}{]}
{[}\hyperref[tailcoursse88.html]{tail}{]}
{[}\href{coursch17.html\#coursse88.html}{up}{]}

\subsubsection{16.3 Equations différentielles linéaires d'ordre 1}

\paragraph{16.3.1 Généralités}

Soit E un espace vectoriel normé de dimension finie, I un intervalle de
ℝ, ℓ : I → L(E) continue et g : I → E continue. On considère l'équation
différentielle linéaire d'ordre 1 vectorielle y' = ℓ(t).y + g(t)~; une
solution est un couple (J,φ) constitué d'un intervalle J de ℝ inclus
dans I et d'une application φ : J → E de classe \{C\}\^{}\{1\} telle que
\textbackslash{}mathop\{∀\}t ∈ J, φ'(t) = ℓ(t).φ(t) + g(t), où l'on note
ℓ(t).x à la place de \textbackslash{}left (ℓ(t)\textbackslash{}right
)(x) pour alléger l'écriture. A une telle équation différentielle
linéaire nous associerons l'équation différentielle homogène y' = ℓ(t).y

Théorème~16.3.1 L'ensemble \{S\}\_\{H\}(J) des solutions de l'équation
homogène y' = ℓ(t).y définies sur un intervalle J est un K espace
vectoriel. On obtient la solution générale sur J de l'équation linéaire
y' = ℓ(t).y + g(t) en ajoutant à une solution particulière de cette
équation la solution générale de l'équation homogène.

Démonstration La fonction nulle est bien évidemment solution de
l'équation homogène et si \{φ\}\_\{1\} et \{φ\}\_\{2\} sont deux
solutions définies sur J, il est clair que α\{φ\}\_\{1\} + β\{φ\}\_\{2\}
est encore une solution définie sur J~; donc \{S\}\_\{H\} est bien un K
espace vectoriel. Si maintenant \{φ\}\_\{0\} est une solution sur J de
l'équation linéaire et si φ est une fonction de classe \{C\}\^{}\{1\} de
J dans E, alors φ est solution de l'équation linéaire si et seulement
si~φ'(t) = ℓ(t).φ(t) + g(t) = ℓ(t).φ(t) + \{φ\}\_\{0\}'(t) −
ℓ(t).\{φ\}\_\{0\}(t), soit encore (φ − \{φ\}\_\{0\})'(t) = ℓ(t).(φ(t) −
\{φ\}\_\{0\}(t)), c'est-à-dire φ − \{φ\}\_\{0\} ∈ \{S\}\_\{H\} ce qui
montre le résultat.

On peut également voir ce problème sous forme matricielle. Pour cela
donnons nous ℰ une base de E, soit A(t) la matrice de ℓ(t) dans la base
ℰ, avec A(t) = \{(\{a\}\_\{i,j\}(t))\}\_\{1≤i,j≤n\}~; soit B(t) le
vecteur colonne des coordonnées de g(t) dans la base ℰ et Y (t) le
vecteur colonne des coordonnées de la fonction inconnue y(t) dans la
base ℰ. L'équation différentielle linéaire d'ordre 1 vectorielle s'écrit
encore sous la forme Y ' = A(t)Y + B(t), ou encore sous forme d'un
système différentiel linéaire

\textbackslash{}left
\textbackslash{}\{\textbackslash{}matrix\{\textbackslash{},\{y\}\_\{1\}'
= \{a\}\_\{1,1\}(t)\{y\}\_\{1\} +
\textbackslash{}mathop\{\textbackslash{}mathop\{\ldots{}\}\} +
\{a\}\_\{1,n\}(t)\{y\}\_\{n\} + \{b\}\_\{1\}(t) \textbackslash{}cr
\textbackslash{}mathop\{\textbackslash{}mathop\{\ldots{}\}\}\textbackslash{}mathop\{\textbackslash{}mathop\{\ldots{}\}\}\textbackslash{}mathop\{\textbackslash{}mathop\{\ldots{}\}\}
\textbackslash{}cr \{y\}\_\{n\}' = \{a\}\_\{n,1\}(t)\{y\}\_\{1\} +
\textbackslash{}mathop\{\textbackslash{}mathop\{\ldots{}\}\} +
\{a\}\_\{n,n\}(t)\{y\}\_\{n\} + \{b\}\_\{n\}(t)\}\textbackslash{}right .

les \{a\}\_\{i,j\} et les \{b\}\_\{i\} étant des fonctions continues de
I dans le corps de base K. Une solution de ce système est alors la
donnée d'un intervalle J de ℝ inclus dans I et de n fonctions
\{φ\}\_\{j\} : J → K de classe \{C\}\^{}\{1\} vérifiant

\textbackslash{}mathop\{∀\}t ∈ J,\textbackslash{}quad
\textbackslash{}left
\textbackslash{}\{\textbackslash{}matrix\{\textbackslash{},\{φ\}\_\{1\}'(t)
= \{a\}\_\{1,1\}(t)\{φ\}\_\{1\}(t) +
\textbackslash{}mathop\{\textbackslash{}mathop\{\ldots{}\}\} +
\{a\}\_\{1,n\}(t)\{φ\}\_\{n\}(t) + \{b\}\_\{1\}(t) \textbackslash{}cr
\textbackslash{}mathop\{\textbackslash{}mathop\{\ldots{}\}\}\textbackslash{}mathop\{\textbackslash{}mathop\{\ldots{}\}\}\textbackslash{}mathop\{\textbackslash{}mathop\{\ldots{}\}\}
\textbackslash{}cr \{φ\}\_\{n\}'(t) = \{a\}\_\{n,1\}(t)\{φ\}\_\{1\}(t) +
\textbackslash{}mathop\{\textbackslash{}mathop\{\ldots{}\}\} +
\{a\}\_\{n,n\}(t)\{φ\}\_\{n\}(t) +
\{b\}\_\{n\}(t)\}\textbackslash{}right .

Bien entendu le système homogène associé est alors le système Y ' =
A(t)Y ou encore

\textbackslash{}left
\textbackslash{}\{\textbackslash{}matrix\{\textbackslash{},\{y\}\_\{1\}'
= \{a\}\_\{1,1\}(t)\{y\}\_\{1\} +
\textbackslash{}mathop\{\textbackslash{}mathop\{\ldots{}\}\} +
\{a\}\_\{1,n\}(t)\{y\}\_\{n\} \textbackslash{}cr
\textbackslash{}mathop\{\textbackslash{}mathop\{\ldots{}\}\}\textbackslash{}mathop\{\textbackslash{}mathop\{\ldots{}\}\}\textbackslash{}mathop\{\textbackslash{}mathop\{\ldots{}\}\}
\textbackslash{}cr \{y\}\_\{n\}' = \{a\}\_\{n,1\}(t)\{y\}\_\{1\} +
\textbackslash{}mathop\{\textbackslash{}mathop\{\ldots{}\}\} +
\{a\}\_\{n,n\}(t)\{y\}\_\{n\}\}\textbackslash{}right .

\paragraph{16.3.2 Equation différentielle linéaire scalaire d'ordre 1}

Dans ce paragraphe, nous allons obtenir dans ce cas particulier, une
preuve élémentaire du théorème de Cauchy-Lipschitz.

Ici, on a n = 1 et donc l'équation différentielle linéaire s'écrit y' =
a(t)y + b(t), où a et b sont deux fonctions continues de I dans le corps
de base K (égal à ℝ ou ℂ). L'équation homogène associée est alors
l'équation y' = a(t)y. C'est cette équation que nous allons d'abord
résoudre. Soit J un intervalle inclus dans I et soit A une primitive de
a sur I. Soit φ : J → K de classe \{C\}\^{}\{1\}. On a alors

\textbackslash{}begin\{eqnarray*\} \textbackslash{}mathop\{∀\}t ∈ J,
φ'(t) − a(t)φ(t) = 0\&\& \%\& \textbackslash{}\textbackslash{} \&
\textbackslash{}mathrel\{⇔\} \& \textbackslash{}mathop\{∀\}t ∈ J, (φ'(t)
− a(t)φ(t))\{e\}\^{}\{−A(t)\} = 0 \%\& \textbackslash{}\textbackslash{}
\& \textbackslash{}mathrel\{⇔\} \& \textbackslash{}mathop\{∀\}t ∈ J,
\textbackslash{}left (φ\{e\}\^{}\{−A\}\textbackslash{}right )'(t) = 0
\textbackslash{}mathrel\{⇔\} φ\{e\}\^{}\{−A\}\textbackslash{}text\{ est
constante\}\%\& \textbackslash{}\textbackslash{}
\textbackslash{}end\{eqnarray*\}

On en déduit que les solution définies sur J sont les fonctions de la
forme t\textbackslash{}mathrel\{↦\}λ\{e\}\^{}\{A(t)\}. Les solutions
maximales sont donc définies sur I et ce sont les solutions
(I,t\textbackslash{}mathrel\{↦\}λ\{e\}\^{}\{A(t)\}). On constate
qu'elles forment un K espace vectoriel de dimension 1. La solution
maximale vérifiant y(\{t\}\_\{0\}) = \{y\}\_\{0\} est bien entendu
(I,t\textbackslash{}mathrel\{↦\}\{y\}\_\{0\}\{e\}\^{}\{\{\textbackslash{}mathop\{∫
\} \}\_\{\{t\}\_\{0\}\}\^{}\{t\}a(u) du \})~; elle est visiblement
unique. On obtient donc

Théorème~16.3.2 Soit a : I → K une application continue. Toute solution
maximale de l'équation homogène y' = a(t)y est définie sur I. L'ensemble
de ces solutions maximales est un K-espace vectoriel de dimension 1
engendré par la fonction \{e\}\^{}\{A\} où A est une primitive de a sur
I. Pour \{t\}\_\{0\} ∈ I et \{y\}\_\{0\} ∈ K, il existe une unique
solution maximale vérifiant la condition initiale y(\{t\}\_\{0\}) =
\{y\}\_\{0\} à savoir
(I,t\textbackslash{}mathrel\{↦\}\{y\}\_\{0\}\{e\}\^{}\{\{\textbackslash{}mathop\{∫
\} \}\_\{\{t\}\_\{0\}\}\^{}\{t\}a(u) du \}).

Pour résoudre l'équation linéaire, faisons le changement de fonction
inconnue z = y\{e\}\^{}\{−A\} soit encore y = z\{e\}\^{}\{A\}. On a
alors y' = z'\{e\}\^{}\{A\} + az\{e\}\^{}\{A\} = z'\{e\}\^{}\{A\} + ay
si bien que

y' = a(t)y + b(t) \textbackslash{}mathrel\{⇔\} z'\{e\}\^{}\{A(t)\} =
b(t) \textbackslash{}mathrel\{⇔\} z' = b(t)\{e\}\^{}\{−A(t)\}

ce qui conduit à un simple calcul de primitive de la fonction
b\{e\}\^{}\{−A\} pour déterminer la fonction inconnue z et donc la
fonction inconnue y.

Remarquons que la solution générale de l'équation homogène était écrite
sous la forme φ(t) = λ\{e\}\^{}\{A(t)\} où λ est une constante, et qu'à
un changement de notation près (celui de z en λ), la résolution de
l'équation linéaire se fait en posant φ(t) = λ(t)\{e\}\^{}\{A(t)\},
autrement dit en rempla\textbackslash{}c\{c\}ant la constante λ par une
fonction inconnue λ. Cette méthode porte le nom de méthode de variation
de la constante. On déduit immédiatement de l'étude précédente le
théorème suivant

Théorème~16.3.3 Soit a,b : I → K deux applications continues. Toute
solution maximale de l'équation linéaire y' = a(t)y + b(t) est définie
sur I. L'ensemble de ces solutions maximales est une droite affine ayant
pour direction la droite vectorielle des solutions de l'équation
homogène associée. Pour \{t\}\_\{0\} ∈ I et \{y\}\_\{0\} ∈ K, il existe
une unique solution maximale vérifiant la condition initiale
y(\{t\}\_\{0\}) = \{y\}\_\{0\}

Démonstration La méthode de variation de la constante montre que toute
solution maximale est définie sur I. La structure de droite affine
résulte immédiatement du théorème de structure de l'ensemble des
solutions d'une équation différentielle linéaire. Si \{φ\}\_\{0\} est
une solution particulière, toute solution est du type φ(t) =
\{φ\}\_\{0\}(t) + λ\{e\}\^{}\{A(t)\} et la condition φ(\{t\}\_\{0\}) =
\{y\}\_\{0\} fournit immédiatement λ =
\{e\}\^{}\{−A(\{t\}\_\{0\})\}(\{y\}\_\{0\} −
\{φ\}\_\{0\}(\{t\}\_\{0\})).

Remarque~16.3.1 La condition de continuité des applications a et b est
essentielle pour la validité du résultat. Si a et b ne sont pas
continues, les solutions maximales ne sont plus nécessairement définies
sur I, et, pour un problème à condition initiale y(\{t\}\_\{0\}) =
\{y\}\_\{0\}, soit l'existence soit l'unicité de la solution maximale
peut être prise en défaut. En particulier, les équations différentielles
linéaires se présentent souvent sous forme non normale α(t)y' + β(t)y =
γ(t) et la mise sous forme normale exige la division par α(t). Si la
fonction α peut s'annuler, les fonctions a(t) = −\{ β(t)
\textbackslash{}over α(t)\} et b(t) =\{ γ(t) \textbackslash{}over α(t)\}
ne sont pas nécessairement continues. Dans ce cas, on utilisera la
théorie précédente sur des intervalles maximaux sur lesquels la fonction
α ne s'annule pas, en essayant ensuite éventuellement de recoller les
solutions ainsi obtenues pour obtenir des solutions maximales.

Exemple~16.3.1 Considérons l'équation différentielle ty' − 2y = 1.
L'équation homogène associée s'écrit ty' − 2y = 0 soit encore sur {]}
−∞,0{[} ou {]}0,+∞{[}, y' =\{ 2 \textbackslash{}over t\} y qui admet
évidemment pour solution y(t) = λ\{t\}\^{}\{2\}. On voit alors que
toutes les fonctions \{φ\}\_\{λ,μ\} : ℝ → ℝ définies par
\{φ\}\_\{λ,μ\}(t) = \textbackslash{}left \textbackslash{}\{
\textbackslash{}cases\{ λ\{t\}\^{}\{2\}\&si t \textgreater{} 0
\textbackslash{}cr 0 \&si t = 0 \textbackslash{}cr μ\{t\}\^{}\{2\}\&si t
\textless{} 0 \} \textbackslash{}right . sont de classe \{C\}\^{}\{1\}
et solutions de l'équation homogène. En remarquant que
t\textbackslash{}mathrel\{↦\} −\{ 1 \textbackslash{}over 2\} est
solution particulière de l'équation linéaire, on obtient les solutions
maximales de l'équation linéaire sous la forme

\{ φ\}\_\{λ,μ\}(t) = \textbackslash{}left \textbackslash{}\{
\textbackslash{}cases\{ λ\{t\}\^{}\{2\} −\{ 1 \textbackslash{}over 2\}
\&si t \textgreater{} 0 \textbackslash{}cr −\{ 1 \textbackslash{}over
2\} \&si t = 0 \textbackslash{}cr μ\{t\}\^{}\{2\} −\{ 1
\textbackslash{}over 2\} \&si t \textless{} 0 \} \textbackslash{}right .

et elles forment un espace affine de dimension 2.

Il n'existe aucune solution vérifiant la condition y(0) = \{y\}\_\{0\}
pour \{y\}\_\{0\}\textbackslash{}mathrel\{≠\} −\{ 1 \textbackslash{}over
2\} par contre, il existe une infinité de solutions maximales vérifiant
y(0) = −\{ 1 \textbackslash{}over 2\} .

\paragraph{16.3.3 Théorie de Cauchy-Lipschitz pour les équations
linéaires}

Lemme~16.3.4 Soit E un espace vectoriel normé de dimension finie, I un
intervalle de ℝ, ℓ : I → L(E) continue et g : I → E continue. Alors
l'équation différentielle linéaire y' = ℓ(t).y + g(t) satisfait à la
condition d'unicité au problème de Cauchy-Lipschitz.

Démonstration Soit (\{I\}\_\{1\},\{φ\}\_\{1\}) et
(\{I\}\_\{2\},\{φ\}\_\{2\}) deux solutions de l'équation différentielle
linéaire vérifiant \{φ\}\_\{1\}(\{t\}\_\{0\}) =
\{φ\}\_\{2\}(\{t\}\_\{0\}) = \{y\}\_\{0\}. Alors, si t ∈ \{I\}\_\{1\} ∩
\{I\}\_\{2\}, on a à la fois \{φ\}\_\{1\}(t) = \{y\}\_\{0\}
+\{\textbackslash{}mathop\{∫ \}
\}\_\{\{t\}\_\{0\}\}\^{}\{t\}(ℓ(u).\{φ\}\_\{1\}(u) + g(u)) du et
\{φ\}\_\{2\}(t) = \{y\}\_\{0\} +\{\textbackslash{}mathop\{∫ \}
\}\_\{\{t\}\_\{0\}\}\^{}\{t\}(ℓ(u).\{φ\}\_\{2\}(u) + g(u)) du, d'où, si
φ = \{φ\}\_\{1\} − \{φ\}\_\{2\}, φ(t) =\{\textbackslash{}mathop\{∫ \}
\}\_\{\{t\}\_\{0\}\}\^{}\{t\}ℓ(u).φ(u) du. Pour t ≥ \{t\}\_\{0\}, on a
donc

\textbackslash{}begin\{eqnarray*\}
\textbackslash{}\textbar{}φ(t)\textbackslash{}\textbar{}\& ≤\&
\{\textbackslash{}mathop\{∫ \}
\}\_\{\{t\}\_\{0\}\}\^{}\{t\}\textbackslash{}\textbar{}ℓ(u).φ(u)\textbackslash{}\textbar{}
du\%\& \textbackslash{}\textbackslash{} \& ≤\&
\{\textbackslash{}mathop\{∫ \}
\}\_\{\{t\}\_\{0\}\}\^{}\{t\}\textbackslash{}\textbar{}ℓ(u)\textbackslash{}\textbar{}\textbackslash{}\textbar{}φ(u)\textbackslash{}\textbar{}
du \%\& \textbackslash{}\textbackslash{}
\textbackslash{}end\{eqnarray*\}

et le lemme de Gronwall (dans le cas où la constante c est nulle)
implique que φ est nulle sur \{I\}\_\{1\} ∩ \{I\}\_\{2\} ∩
{[}\{t\}\_\{0\},+∞{[}. Une démonstration similaire montre que φ est
nulle sur \{I\}\_\{1\} ∩ \{I\}\_\{2\}∩{]} −∞,\{t\}\_\{0\}{]}, et donc
\{φ\}\_\{1\} et \{φ\}\_\{2\} coïncident sur \{I\}\_\{1\} ∩ \{I\}\_\{2\}.

Lemme~16.3.5 Soit E un espace vectoriel normé de dimension finie, I un
intervalle de ℝ, ℓ : I → L(E) continue et g : I → E continue. Alors
l'équation différentielle linéaire y' = ℓ(t).y + g(t) satisfait à la
condition d'existence locale au problème de Cauchy-Lipschitz. De
fa\textbackslash{}c\{c\}on plus précise, pour tout \{t\}\_\{0\} ∈ I,
pour tout \{y\}\_\{0\} ∈ E et pour tout segment J tel que \{t\}\_\{0\} ∈
J ⊂ I, il existe une solution (J,φ) de l'équation différentielle
vérifiant φ(\{t\}\_\{0\}) = \{y\}\_\{0\}.

Démonstration Soit donc J un segment inclus dans I~; l'application ℓ est
continue sur le compact J, donc bornée et on peut poser L
=\{\textbackslash{}mathop\{
sup\}\}\_\{t∈J\}\textbackslash{}\textbar{}ℓ(t)\textbackslash{}\textbar{}.
On définit une suite (\{φ\}\_\{n\}) d'applications continues de J dans E
par \{φ\}\_\{0\}(t) = \{y\}\_\{0\} et pour n ≥ 0, \{φ\}\_\{n+1\}(t) =
\{y\}\_\{0\} +\{\textbackslash{}mathop\{∫ \}
\}\_\{\{t\}\_\{0\}\}\^{}\{t\}\textbackslash{}left (ℓ(u).\{φ\}\_\{n\}(u)
+ g(u)\textbackslash{}right ) du. En posant M
=\{\textbackslash{}mathop\{
sup\}\}\_\{t∈J\}\textbackslash{}\textbar{}ℓ(t).\{y\}\_\{0\} +
g(t)\textbackslash{}\textbar{}, on montre par récurrence sur n que

\textbackslash{}mathop\{∀\}n ∈ ℕ, \textbackslash{}mathop\{∀\}t ∈ J,
\textbackslash{}\textbar{}\{φ\}\_\{n+1\}(t) −
\{φ\}\_\{n\}(t)\textbackslash{}\textbar{} ≤\{ M \textbackslash{}over L\}
\{ \{L\}\^{}\{n+1\}\textbar{}t − \{t\}\_\{0\}\{\textbar{}\}\^{}\{n+1\}
\textbackslash{}over (n + 1)!\}

C'est vrai pour n = 1 d'après la définition même de M~:

\textbackslash{}begin\{eqnarray*\}
\textbackslash{}\textbar{}\{φ\}\_\{1\}(t) −
\{φ\}\_\{0\}(t)\textbackslash{}\textbar{}\& =\&
\textbackslash{}\textbar{}\{φ\}\_\{1\}(t) −
\{y\}\_\{0\}\textbackslash{}\textbar{} \%\&
\textbackslash{}\textbackslash{} \& =\&
\textbackslash{}\textbar{}\{\textbackslash{}mathop\{∫ \}
\}\_\{\{t\}\_\{0\}\}\^{}\{t\}(ℓ(u).\{y\}\_\{ 0\} + g(u))
du\textbackslash{}\textbar{} ≤ M\textbar{}t − \{t\}\_\{0\}\textbar{}\%\&
\textbackslash{}\textbackslash{} \textbackslash{}end\{eqnarray*\}

ce qui est bien la formule voulue. Si maintenant l'inégalité est
vérifiée pour n − 1, on a alors pour t ∈ {[}\{t\}\_\{0\},+∞{[}∩J

\textbackslash{}begin\{eqnarray*\}
\textbackslash{}\textbar{}\{φ\}\_\{n+1\}(t) −
\{φ\}\_\{n\}(t)\textbackslash{}\textbar{}\& =\&
\textbackslash{}\textbar{}\{\textbackslash{}mathop\{∫ \}
\}\_\{\{t\}\_\{0\}\}\^{}\{t\}\textbackslash{}left (ℓ(u).\{φ\}\_\{ n\}(u)
− ℓ(u).\{φ\}\_\{n−1\}(u))\textbackslash{}right )
du\textbackslash{}\textbar{}\%\& \textbackslash{}\textbackslash{} \& ≤\&
\{\textbackslash{}mathop\{∫ \}
\}\_\{\{t\}\_\{0\}\}\^{}\{t\}\textbackslash{}\textbar{}ℓ(u)\textbackslash{}\textbar{}.\textbackslash{}\textbar{}\{φ\}\_\{
n\}(u) − \{φ\}\_\{n−1\}(u)\textbackslash{}\textbar{} du \%\&
\textbackslash{}\textbackslash{} \& ≤\& \{\textbackslash{}mathop\{∫ \}
\}\_\{\{t\}\_\{0\}\}\^{}\{t\}L\textbackslash{}\textbar{}\{φ\}\_\{ n\}(u)
− \{φ\}\_\{n−1\}(u)\textbackslash{}\textbar{} du \%\&
\textbackslash{}\textbackslash{} \& ≤\& L\{\textbackslash{}mathop\{∫ \}
\}\_\{\{t\}\_\{0\}\}\^{}\{t\}\{ M \textbackslash{}over L\} \{
\{L\}\^{}\{n\}\textbar{}u − \{t\}\_\{0\}\{\textbar{}\}\^{}\{n\}
\textbackslash{}over n!\} du \%\& \textbackslash{}\textbackslash{} \&
=\&\{ M \textbackslash{}over L\} \{ \{L\}\^{}\{n+1\}\textbar{}t −
\{t\}\_\{0\}\{\textbar{}\}\^{}\{n+1\} \textbackslash{}over (n + 1)!\}
\%\& \textbackslash{}\textbackslash{} \textbackslash{}end\{eqnarray*\}

Un calcul similaire conduit à la même inégalité pour t ∈{]}
−∞,\{t\}\_\{0\}{]} ∩ J.

Désignons par η la longueur du segment J. On a donc

\textbackslash{}mathop\{∀\}n ∈ ℕ, \textbackslash{}mathop\{∀\}t ∈ J,
\textbackslash{}\textbar{}\{φ\}\_\{n+1\}(t) −
\{φ\}\_\{n\}(t)\textbackslash{}\textbar{} ≤\{ M \textbackslash{}over L\}
\{ \{L\}\^{}\{n+1\}\{η\}\^{}\{n+1\} \textbackslash{}over (n + 1)!\}

Alors pour q \textgreater{} p, on a, \textbackslash{}mathop\{∀\}t ∈ J

\textbackslash{}begin\{eqnarray*\}
\textbackslash{}\textbar{}\{φ\}\_\{q\}(t) −
\{φ\}\_\{p\}(t)\textbackslash{}\textbar{}\& ≤\&
\{\textbackslash{}mathop\{∑
\}\}\_\{n=p\}\^{}\{q−1\}\textbackslash{}\textbar{}\{φ\}\_\{ n+1\}(t) −
\{φ\}\_\{n\}(t)\textbackslash{}\textbar{}\%\&
\textbackslash{}\textbackslash{} \& ≤\&\{ M \textbackslash{}over L\}
\{\textbackslash{}mathop\{∑ \}\}\_\{n=p\}\^{}\{q−1\}\{
\{L\}\^{}\{n+1\}\{η\}\^{}\{n+1\} \textbackslash{}over (n + 1)!\} \%\&
\textbackslash{}\textbackslash{} \& ≤\&\{ M \textbackslash{}over L\}
\{\textbackslash{}mathop\{∑ \}\}\_\{n=p\}\^{}\{+∞\}\{
\{L\}\^{}\{n+1\}\{η\}\^{}\{n+1\} \textbackslash{}over (n + 1)!\} \%\&
\textbackslash{}\textbackslash{} \textbackslash{}end\{eqnarray*\}

Comme la série \{\textbackslash{}mathop\{\textbackslash{}mathop\{∑ \}\}
\}\_\{n\}\{ \{L\}\^{}\{n+1\}\{η\}\^{}\{n+1\} \textbackslash{}over
(n+1)!\} est une série convergente (exponentielle d'un nombre réel), son
reste tend vers 0~; étant donné ε \textgreater{} 0, il existe N ∈ ℕ tel
que p ≥ N ⇒\{ M \textbackslash{}over L\} \{\textbackslash{}mathop\{
\textbackslash{}mathop\{∑ \}\} \}\_\{n=p\}\^{}\{+∞\}\{
\{L\}\^{}\{n+1\}\{η\}\^{}\{n+1\} \textbackslash{}over (n+1)!\}
\textless{} ε. Alors

q \textgreater{} p ≥ N ⇒\textbackslash{}mathop\{∀\}t ∈ J,
\textbackslash{}\textbar{}\{φ\}\_\{q\}(t) −
\{φ\}\_\{p\}(t)\textbackslash{}\textbar{} \textless{} ε

La suite (\{φ\}\_\{n\}) vérifie donc le critère de Cauchy uniforme. En
conséquence, elle converge uniformément vers une fonction φ : J → E qui
est elle même continue.

L'inégalité \textbackslash{}\textbar{}(ℓ(u).φ(u) + g(u)) −
(ℓ(u).\{φ\}\_\{n\}(u) + g(u))\textbackslash{}\textbar{} ≤
L\textbackslash{}\textbar{}φ(u) −
\{φ\}\_\{n\}(u)\textbackslash{}\textbar{}, montre que la suite
\textbackslash{}left (ℓ(u).\{φ\}\_\{n\}(u) + g(u)\textbackslash{}right )
converge uniformément vers ℓ(u).φ(u) + g(u)~; ceci nous permet de passer
à la limite sous le signe d'intégration et d'obtenir

\textbackslash{}begin\{eqnarray*\}\{ y\}\_\{0\}
+\{\textbackslash{}mathop\{∫ \} \}\_\{\{t\}\_\{0\}\}\^{}\{t\}(ℓ(u).φ(u)
+ g(u)) du\&\& \%\& \textbackslash{}\textbackslash{} \& =\& \{y\}\_\{0\}
+\{\textbackslash{}mathop\{ lim\}\}\_\{n→+∞\}\{\textbackslash{}mathop\{∫
\} \}\_\{\{t\}\_\{0\}\}\^{}\{t\}(ℓ(u).\{φ\}\_\{ n\}(u) + g(u)) du\%\&
\textbackslash{}\textbackslash{} \& =\&
\{\textbackslash{}mathop\{lim\}\}\_\{n→+∞\}\{φ\}\_\{n+1\}(t) = φ(t) \%\&
\textbackslash{}\textbackslash{} \textbackslash{}end\{eqnarray*\}

Comme φ est continue, ceci montre que φ est la solution cherchée sur J
de l'équation y' = ℓ(t).y + g(t) vérifiant φ(\{t\}\_\{0\}) =
\{y\}\_\{0\}.

Théorème~16.3.6 Soit E un espace vectoriel normé de dimension finie, I
un intervalle de ℝ, ℓ : I → L(E) continue et g : I → E continue. Alors
toute solution maximale de l'équation différentielle linéaire y' =
ℓ(t).y + g(t) est définie sur I. Pour tout \{t\}\_\{0\} ∈ I et tout
\{y\}\_\{0\} ∈ E, il existe une et une seule solution (I,φ) de
l'équation différentielle linéaire y' = ℓ(t).y + g(t) vérifiant
φ(\{t\}\_\{0\}) = \{y\}\_\{0\}~; pour toute solution (J,ψ) de l'équation
différentielle vérifiant ψ(\{t\}\_\{0\}) = \{y\}\_\{0\}, on a~:

\textbackslash{}text\{\$J ⊂ \{I\}\_\{0\}\$ et \$ψ\$ est la restriction
de \$φ\$ à \$J\$.\}

Démonstration Puisque l'équation différentielle linéaire vérifie les
conditions d'unicité et d'existence locale au problème de
Cauchy-Lipschitz, on sait qu'il existe une unique solution maximale pour
une condition initiale donnée et que cette solution prolonge toutes les
autres solutions vérifiant cette même condition initiale. Mais le lemme
précédent, montre que l'intervalle de définition de cette solution doit
contenir tout segment J contenant \{t\}\_\{0\} et inclus dans I~; ce ne
peut donc être que I lui même. Le théorème en résulte.

\paragraph{16.3.4 Structure des solutions de l'équation homogène}

Théorème~16.3.7 Soit E un espace vectoriel normé de dimension finie, I
un intervalle de ℝ, ℓ : I → L(E) continue. L'ensemble \{S\}\_\{H\} des
solutions définies sur I de l'équation différentielle homogène y' =
ℓ(t).y est un espace vectoriel de dimension finie égale à
\textbackslash{}mathop\{dim\} E. Plus précisément, pour tout
\{t\}\_\{0\} ∈ I, l'application
φ\textbackslash{}mathrel\{↦\}φ(\{t\}\_\{0\}) est un isomorphisme
d'espaces vectoriels de \{S\}\_\{H\} sur E.

Démonstration On sait déjà que \{S\}\_\{H\} est un espace vectoriel.
L'application \{S\}\_\{H\} → E,
φ\textbackslash{}mathrel\{↦\}φ(\{t\}\_\{0\}) est visiblement linéaire et
le théorème de Cauchy Lipschitz assure que cette application est
bijective (puisque pour tout \{y\}\_\{0\} ∈ E, il existe une unique
solution définie sur I vérifiant φ(\{t\}\_\{0\}) = \{y\}\_\{0\}).
L'application en question est donc un isomorphisme d'espaces vectoriels
et les deux espaces vectoriels ont donc même dimension.

Corollaire~16.3.8 Soit E un espace vectoriel normé de dimension finie, I
un intervalle de ℝ, ℓ : I → L(E) continue. Soit
(\{φ\}\_\{1\},\textbackslash{}mathop\{\textbackslash{}mathop\{\ldots{}\}\},\{φ\}\_\{k\})
des solutions définies sur I de l'équation différentielle y' = ℓ(t).y.
Alors,

\textbackslash{}mathop\{∀\}\{t\}\_\{0\} ∈ I,
\textbackslash{}mathop\{\textbackslash{}mathrm\{rg\}\}(\{φ\}\_\{1\},\textbackslash{}mathop\{\textbackslash{}mathop\{\ldots{}\}\},\{φ\}\_\{k\})
=\textbackslash{}mathop\{
\textbackslash{}mathrm\{rg\}\}(\{φ\}\_\{1\}(\{t\}\_\{0\}),\textbackslash{}mathop\{\textbackslash{}mathop\{\ldots{}\}\},\{φ\}\_\{k\}(\{t\}\_\{0\}))

Démonstration En effet, un isomorphisme conserve le rang.

Remarque~16.3.2 En particulier, pour k = 1, une solution non
identiquement nulle de l'équation homogène y' = ℓ(t).y ne peut pas
s'annuler.

\paragraph{16.3.5 Méthode de variation des constantes}

Soit E un espace vectoriel normé de dimension finie n, I un intervalle
de ℝ, ℓ : I → L(E) continue et g : I → E continue. Soit ℰ une base de E,
A(t) la matrice de ℓ(t) dans la base ℰ, avec A(t) =
\{(\{a\}\_\{i,j\}(t))\}\_\{1≤i,j≤n\}~; soit B(t) le vecteur colonne des
coordonnées de g(t) dans la base ℰ et Y (t) le vecteur colonne des
coordonnées de la fonction inconnue y(t) dans la base ℰ. L'équation
différentielle linéaire d'ordre 1 vectorielle s'écrit encore sous la
forme Y ' = A(t)Y + B(t) et l'équation homogène associé s'écrit Y ' =
A(t)Y .

On sait que l'espace vectoriel \{S\}\_\{H\} des solutions de l'équation
homogène est de dimension n. Supposons connue une base
(\{φ\}\_\{1\},\textbackslash{}mathop\{\textbackslash{}mathop\{\ldots{}\}\},\{φ\}\_\{n\})
et soit \{Φ\}\_\{j\}(t) = \textbackslash{}left
(\textbackslash{}matrix\{\textbackslash{},\{φ\}\_\{1,j\}(t)
\textbackslash{}cr \textbackslash{}mathop\{\textbackslash{}mathop\{⋮\}\}
\textbackslash{}cr \{φ\}\_\{n,j\}(t)\}\textbackslash{}right ) le vecteur
colonne des coordonnées de \{φ\}\_\{j\}(t) dans la base E. On a alors
\{Φ\}\_\{j\}'(t) = A(t)\{Φ\}\_\{j\}(t) ce qui se traduit encore par

\textbackslash{}mathop\{∀\}i,j, \{φ\}\_\{i,j\}'(t) =\{
\textbackslash{}mathop\{∑ \}\}\_\{k=1\}\^{}\{n\}\{a\}\_\{
i,k\}(t)\{ψ\}\_\{k,j\}(t)

ou encore, en introduisant la matrice carrée R(t) =
\{(\{ψ\}\_\{i,j\}(t))\}\_\{1≤i,j≤n\}, par R'(t) = A(t)R(t).

Remarquons que cette matrice a pour vecteurs colonnes les vecteurs
\{Φ\}\_\{j\}(t)~; or on sait que \textbackslash{}mathop\{∀\}t ∈ J,
\textbackslash{}mathop\{\textbackslash{}mathrm\{rg\}\}(\{Φ\}\_\{1\}(t),\textbackslash{}mathop\{\textbackslash{}mathop\{\ldots{}\}\},\{Φ\}\_\{n\}(t))
=\textbackslash{}mathop\{
\textbackslash{}mathrm\{rg\}\}(\{Φ\}\_\{1\},\textbackslash{}mathop\{\textbackslash{}mathop\{\ldots{}\}\},\{Φ\}\_\{n\})
=\textbackslash{}mathop\{
\textbackslash{}mathrm\{rg\}\}(\{φ\}\_\{1\},\textbackslash{}mathop\{\textbackslash{}mathop\{\ldots{}\}\},\{φ\}\_\{n\})
= n. On en déduit que cette matrice est inversible. Faisons alors le
changement de fonction inconnue Z = R\{(t)\}\^{}\{−1\}Y , autrement dit
Y = R(t)Z. L'application t\textbackslash{}mathrel\{↦\}R(t) étant
visiblement de classe \{C\}\^{}\{1\}, il en est de même de
t\textbackslash{}mathrel\{↦\}R\{(t)\}\^{}\{−1\}, ce qui rend valide ce
changement de fonction inconnue. On a alors Y `= R'(t)Z + R(t)Z' =
A(t)R(t)Z + R(t)Z' = A(t)Y + R(t)Z', si bien que

\textbackslash{}begin\{eqnarray*\} Y `= A(t)Y + B(t)\&
\textbackslash{}mathrel\{⇔\} \& R(t)Z' = B(t) \%\&
\textbackslash{}\textbackslash{} \& \textbackslash{}mathrel\{⇔\} \& Z' =
R\{(t)\}\^{}\{−1\}B(t)\%\& \textbackslash{}\textbackslash{}
\textbackslash{}end\{eqnarray*\}

et la résolution de l'équation linéaire se ramène à un simple calcul de
primitive de la fonction vectorielle
t\textbackslash{}mathrel\{↦\}R\{(t)\}\^{}\{−1\}B(t) sur l'intervalle I.

Voyons une autre interprétation de cette méthode. Puisque
(\{Φ\}\_\{1\},\textbackslash{}mathop\{\textbackslash{}mathop\{\ldots{}\}\},\{Φ\}\_\{n\})
est une base de l'espace des solutions du système homogène Y ' = A(t)Y ,
toute solution s'écrit de manière unique sous la forme Y (t) =
\{λ\}\_\{1\}\{Φ\}\_\{1\}(t) +
\textbackslash{}mathop\{\textbackslash{}mathop\{\ldots{}\}\}.\{λ\}\_\{n\}\{Φ\}\_\{n\}(t),
où
\{λ\}\_\{1\},\textbackslash{}mathop\{\textbackslash{}mathop\{\ldots{}\}\},\{λ\}\_\{n\}
sont des constantes. Posons alors Z(t) = \textbackslash{}left
(\textbackslash{}matrix\{\textbackslash{},\{λ\}\_\{1\}(t)
\textbackslash{}cr
\textbackslash{}mathop\{\textbackslash{}mathop\{\ldots{}\}\}
\textbackslash{}cr \{λ\}\_\{n\}(t)\}\textbackslash{}right ). Le
changement de fonction inconnue Y = R(t)Z revient à poser
\{y\}\_\{i\}(t) =\{\textbackslash{}mathop\{ \textbackslash{}mathop\{∑
\}\} \}\_\{j=1\}\^{}\{n\}\{ψ\}\_\{i,j\}(t)\{λ\}\_\{j\}(t), soit encore Y
(t) =\{\textbackslash{}mathop\{ \textbackslash{}mathop\{∑ \}\}
\}\_\{j=1\}\^{}\{n\}\{λ\}\_\{j\}(t)\{Φ\}\_\{j\}(t) =
\{λ\}\_\{1\}(t)\{Φ\}\_\{1\}(t) +
\textbackslash{}mathop\{\textbackslash{}mathop\{\ldots{}\}\} +
\{λ\}\_\{n\}(t)\{Φ\}\_\{n\}(t)~; ce changement de fonction inconnue
consiste donc à substituer dans la solution générale de l'équation
homogène, aux constantes
\{λ\}\_\{1\},\textbackslash{}mathop\{\textbackslash{}mathop\{\ldots{}\}\},\{λ\}\_\{n\}
des fonctions de classe \{C\}\^{}\{1\}
\{λ\}\_\{1\}(t),\textbackslash{}mathop\{\textbackslash{}mathop\{\ldots{}\}\},\{λ\}\_\{n\}(t),
d'où le nom de méthode de variation des constantes. La résolution se
fait alors en écrivant

\textbackslash{}begin\{eqnarray*\} Y `(t)\& =\&
\{λ\}\_\{1\}'(t)\{Φ\}\_\{1\}(t) +
\textbackslash{}mathop\{\textbackslash{}mathop\{\ldots{}\}\} +
\{λ\}\_\{n\}'(t)\{Φ\}\_\{n\}(t) \%\& \textbackslash{}\textbackslash{} \&
\& \textbackslash{}quad + \{λ\}\_\{1\}(t)\{Φ\}\_\{1\}'(t) +
\textbackslash{}mathop\{\textbackslash{}mathop\{\ldots{}\}\} +
\{λ\}\_\{n\}(t)\{Φ\}\_\{n\}'(t) \%\& \textbackslash{}\textbackslash{} \&
=\& \{λ\}\_\{1\}'(t)\{Φ\}\_\{1\}(t) +
\textbackslash{}mathop\{\textbackslash{}mathop\{\ldots{}\}\} +
\{λ\}\_\{n\}'(t)\{Φ\}\_\{n\}(t) \%\& \textbackslash{}\textbackslash{} \&
\& \textbackslash{}quad + \{λ\}\_\{1\}(t)A(t)\{Φ\}\_\{1\}(t) +
\textbackslash{}mathop\{\textbackslash{}mathop\{\ldots{}\}\} +
\{λ\}\_\{n\}(t)A(t)\{Φ\}\_\{n\}(t)\%\& \textbackslash{}\textbackslash{}
\& =\& \{λ\}\_\{1\}'(t)\{Φ\}\_\{1\}(t) +
\textbackslash{}mathop\{\textbackslash{}mathop\{\ldots{}\}\} +
\{λ\}\_\{n\}'(t)\{Φ\}\_\{n\}(t) \%\& \textbackslash{}\textbackslash{} \&
\& \textbackslash{}quad + A(t)\textbackslash{}left
(\{λ\}\_\{1\}(t)\{Φ\}\_\{1\}(t) +
\textbackslash{}mathop\{\textbackslash{}mathop\{\ldots{}\}\} +
\{λ\}\_\{n\}(t)\{Φ\}\_\{n\}(t)\textbackslash{}right ) \%\&
\textbackslash{}\textbackslash{} \& =\& \{λ\}\_\{1\}'(t)\{Φ\}\_\{1\}(t)
+ \textbackslash{}mathop\{\textbackslash{}mathop\{\ldots{}\}\} +
\{λ\}\_\{n\}'(t)\{Φ\}\_\{n\}(t) + A(t)Y (t)\%\&
\textbackslash{}\textbackslash{} \textbackslash{}end\{eqnarray*\}

si bien que

\textbackslash{}begin\{eqnarray*\} Y `(t) = A(t)Y (t) + B(t)
\textbackslash{}mathrel\{⇔\}\&\& \%\& \textbackslash{}\textbackslash{}
\& \& \{λ\}\_\{1\}'(t)\{Φ\}\_\{1\}(t) +
\textbackslash{}mathop\{\textbackslash{}mathop\{\ldots{}\}\} +
\{λ\}\_\{n\}'(t)\{Φ\}\_\{n\}(t) = B(t)\%\&
\textbackslash{}\textbackslash{} \textbackslash{}end\{eqnarray*\}

C'est un système de Cramer en les inconnues
\{λ\}\_\{1\}'(t),\textbackslash{}mathop\{\textbackslash{}mathop\{\ldots{}\}\},\{λ\}\_\{n\}'(t)
, la matrice de ce système étant la matrice inversible R(t)~; la
résolution de ce système permet alors de déterminer les fonctions
\{λ\}\_\{j\}'~; on détermine ensuite les fonctions \{λ\}\_\{j\} par n
calculs de primitives de fonctions à valeurs dans le corps de base K.

\paragraph{16.3.6 Systèmes différentiels à coefficients constants}

C'est le cas où l'application t\textbackslash{}mathrel\{↦\}ℓ(t) est
constante. On préférera dans ce cas l'interprétation matricielle en
posant \textbackslash{}mathop\{∀\}t ∈ I, A(t) = A si bien que le système
linéaire s'écrit sous la forme Y ' = AY + B(t), le système homogène
associé étant le système Y ' = AY .

Rappelons à ce propos un théorème démontré dans le chapitre sur les
séries entières qui permet théoriquement de résoudre l'équation homogène

Théorème~16.3.9 Soit \{Y \}\_\{0\} ∈ \{M\}\_\{K\}(n,1). L'unique
solution du système homogène Y ' = AY vérifiant Y (0) = \{Y \}\_\{0\}
est l'application
t\textbackslash{}mathrel\{↦\}\textbackslash{}mathop\{exp\} (tA)\{Y
\}\_\{0\}.

Démonstration Cette application convient évidemment puisque \{ d
\textbackslash{}over dt\} (\textbackslash{}mathop\{exp\} (tA)\{Y
\}\_\{0\}) = A\textbackslash{}mathop\{exp\} (tA)\{Y \}\_\{0\}. Soit
t\textbackslash{}mathrel\{↦\}Y (t) une autre solution et soit Z(t)
=\textbackslash{}mathop\{ exp\} (−tA)Y (t). On a Z'(t) =
−\textbackslash{}mathop\{exp\} (−tA)AY (t) +\textbackslash{}mathop\{
exp\} (−tA)Y `(t) =\textbackslash{}mathop\{ exp\} (−tA)(Y'(t) − AY (t))
= 0. On en déduit que Z est constante égale à Z(0). Mais Z(0) = \{Y
\}\_\{0\}. On a donc Z(t) = \{Y \}\_\{0\} soit encore Y (t)
=\textbackslash{}mathop\{ exp\} (tA)\{Y \}\_\{0\}.

Remarque~16.3.3 Le même changement de fonction inconnue Y
=\textbackslash{}mathop\{ exp\} (tA)Z permet d'ailleurs de résoudre
théoriquement l'équation linéaire Y ' = AY + B(t), puisque l'on a alors
Y `= A\textbackslash{}mathop\{exp\} (tA)Z +\textbackslash{}mathop\{
exp\} (tA)Z' = AY +\textbackslash{}mathop\{ exp\} (tA)Z' et donc

\textbackslash{}begin\{eqnarray*\} Y `= AY + B(t)\&
\textbackslash{}mathrel\{⇔\} \& \textbackslash{}mathop\{exp\} (tA)Z'(t)
= B(t) \%\& \textbackslash{}\textbackslash{} \&
\textbackslash{}mathrel\{⇔\} \& Z'(t) =\textbackslash{}mathop\{ exp\}
(−tA)B(t)\%\& \textbackslash{}\textbackslash{}
\textbackslash{}end\{eqnarray*\}

ce qui ramène le problème de la résolution de l'équation linéaire à
celui d'un calcul de primitive de la fonction
t\textbackslash{}mathrel\{↦\}\textbackslash{}mathop\{exp\} (−tA)B(t).

En fait la méthode précédente bute sur le problème non évident du calcul
de l'exponentielle \textbackslash{}mathop\{exp\} (tA), si bien que, dans
la pratique, d'autres méthodes peuvent être préférées.

En premier lieu, supposons que la matrice A est diagonalisable et soit
(\{V
\}\_\{1\},\textbackslash{}mathop\{\textbackslash{}mathop\{\ldots{}\}\},\{V
\}\_\{n\}) une base de vecteurs propres de A associés aux valeurs
propres
\{λ\}\_\{1\},\textbackslash{}mathop\{\textbackslash{}mathop\{\ldots{}\}\},\{λ\}\_\{n\}.
Posons \{Φ\}\_\{i\} :
t\textbackslash{}mathrel\{↦\}\{e\}\^{}\{\{λ\}\_\{i\}t\}\{V \}\_\{i\}. On
a alors \{Φ\}\_\{i\}'(t) = \{λ\}\_\{i\}\{e\}\^{}\{\{λ\}\_\{i\}t\}\{V
\}\_\{i\} = \{e\}\^{}\{\{λ\}\_\{i\}t\}A\{V \}\_\{i\} = A\{Φ\}\_\{i\}(t)
si bien que
\{Φ\}\_\{1\},\textbackslash{}mathop\{\textbackslash{}mathop\{\ldots{}\}\},\{Φ\}\_\{n\}
sont solutions de l'équation homogène Y ' = AY . Mais, comme
(\{Φ\}\_\{1\}(0),\textbackslash{}mathop\{\textbackslash{}mathop\{\ldots{}\}\},\{Φ\}\_\{n\}(0))
= (\{V
\}\_\{1\},\textbackslash{}mathop\{\textbackslash{}mathop\{\ldots{}\}\},\{V
\}\_\{n\}) est une famille libre, la famille
(\{Φ\}\_\{1\},\textbackslash{}mathop\{\textbackslash{}mathop\{\ldots{}\}\},\{Φ\}\_\{n\})
est également une famille libre. Comme l'espace vectoriel des solutions
de l'équation homogène est de dimension n, cette famille en est une base
et donc la solution générale de l'équation homogène est de la forme

Y (t) = \{α\}\_\{1\}\{e\}\^{}\{\{λ\}\_\{1\}t\}\{V \}\_\{ 1\} +
\textbackslash{}mathop\{\textbackslash{}mathop\{\ldots{}\}\} +
\{α\}\_\{n\}\{e\}\^{}\{\{λ\}\_\{n\}t\}\{V \}\_\{ n\}

On peut ensuite résoudre l'équation linéaire en faisant varier les
constantes
\{α\}\_\{1\},\textbackslash{}mathop\{\textbackslash{}mathop\{\ldots{}\}\},\{α\}\_\{n\}.

Dans le cas général, soit P une matrice inversible et faisons le
changement de fonction inconnue Y = PZ. On a alors

\textbackslash{}begin\{eqnarray*\} Y `= AY + B(t)\&
\textbackslash{}mathrel\{⇔\} \& PZ' = APZ + B(t) \%\&
\textbackslash{}\textbackslash{} \& \textbackslash{}mathrel\{⇔\} \& Z' =
\{P\}\^{}\{−1\}AP + \{P\}\^{}\{−1\}B(t)\%\&
\textbackslash{}\textbackslash{} \textbackslash{}end\{eqnarray*\}

Quitte à passer sur le corps des complexes, on peut par exemple
s'arranger pour que la matrice \{P\}\^{}\{−1\}AP soit triangulaire
supérieure. En notant (\{α\}\_\{i,j\}) cette matrice et
\{P\}\^{}\{−1\}B(t) = \textbackslash{}left
(\textbackslash{}matrix\{\textbackslash{},\{β\}\_\{1\}(t)\textbackslash{}mathop\{\textbackslash{}mathop\{\ldots{}\}\}\{β\}\_\{n\}(t)\}\textbackslash{}right
), ceci conduit à un système différentiel

\textbackslash{}left \textbackslash{}\{\textbackslash{}array\{
\{z\}\_\{1\}' \& = \{α\}\_\{1,1\}\{z\}\_\{1\} +
\{α\}\_\{1,2\}\{z\}\_\{2\} + \textbackslash{}quad \textbackslash{}qquad
\textbackslash{}mathop\{\textbackslash{}mathop\{\ldots{}\}\}\textbackslash{}quad
\textbackslash{}qquad + \{α\}\_\{1,n\}\{z\}\_\{n\} +
\{β\}\_\{1\}(t)\textbackslash{}cr
\textbackslash{}mathop\{\textbackslash{}mathop\{\ldots{}\}\}
\textbackslash{}cr \{z\}\_\{n−1\}'\& = \{α\}\_\{n−1,n−1\}\{z\}\_\{n−1\}
+ \{α\}\_\{n−1,n\}\{z\}\_\{n\} + \{β\}\_\{n−1\}(t) \textbackslash{}cr
\{z\}\_\{n\}' \& = \{α\}\_\{n,n\}\{z\}\_\{n\} + \{β\}\_\{n\}(t) \}
\textbackslash{}right .

qui se résout en cascade à partir du bas en résolvant n équations
différentielles linéaires scalaires d'ordre 1~: lorsque
\{z\}\_\{n\},\textbackslash{}mathop\{\textbackslash{}mathop\{\ldots{}\}\},\{z\}\_\{i+1\}
sont connues, \{z\}\_\{i\} est solution de l'équation différentielle
linéaire scalaire d'ordre 1

\{z\}\_\{i\}' = \{α\}\_\{i,i\}\{z\}\_\{i\} +\{ \textbackslash{}mathop\{∑
\}\}\_\{k=i+1\}\^{}\{n\}\{α\}\_\{ i,k\}\{z\}\_\{k\}(t) + \{β\}\_\{i\}(t)

Ceci permet un calcul de Z et donc de Y .

{[}\href{coursse89.html}{next}{]} {[}\href{coursse87.html}{prev}{]}
{[}\href{coursse87.html\#tailcoursse87.html}{prev-tail}{]}
{[}\href{coursse88.html}{front}{]}
{[}\href{coursch17.html\#coursse88.html}{up}{]}

\end{document}

% \documentclass[]{article}
\usepackage[T1]{fontenc}
\usepackage{lmodern}
\usepackage{amssymb,amsmath}
\usepackage{ifxetex,ifluatex}
\usepackage{fixltx2e} % provides \textsubscript
% use upquote if available, for straight quotes in verbatim environments
\IfFileExists{upquote.sty}{\usepackage{upquote}}{}
\ifnum 0\ifxetex 1\fi\ifluatex 1\fi=0 % if pdftex
  \usepackage[utf8]{inputenc}
\else % if luatex or xelatex
  \ifxetex
    \usepackage{mathspec}
    \usepackage{xltxtra,xunicode}
  \else
    \usepackage{fontspec}
  \fi
  \defaultfontfeatures{Mapping=tex-text,Scale=MatchLowercase}
  \newcommand{\euro}{€}
\fi
% use microtype if available
\IfFileExists{microtype.sty}{\usepackage{microtype}}{}
\usepackage{graphicx}
% Redefine \includegraphics so that, unless explicit options are
% given, the image width will not exceed the width of the page.
% Images get their normal width if they fit onto the page, but
% are scaled down if they would overflow the margins.
\makeatletter
\def\ScaleIfNeeded{%
  \ifdim\Gin@nat@width>\linewidth
    \linewidth
  \else
    \Gin@nat@width
  \fi
}
\makeatother
\let\Oldincludegraphics\includegraphics
{%
 \catcode`\@=11\relax%
 \gdef\includegraphics{\@ifnextchar[{\Oldincludegraphics}{\Oldincludegraphics[width=\ScaleIfNeeded]}}%
}%
\ifxetex
  \usepackage[setpagesize=false, % page size defined by xetex
              unicode=false, % unicode breaks when used with xetex
              xetex]{hyperref}
\else
  \usepackage[unicode=true]{hyperref}
\fi
\hypersetup{breaklinks=true,
            bookmarks=true,
            pdfauthor={},
            pdftitle={Equation differentielle lineaire d'ordre n},
            colorlinks=true,
            citecolor=blue,
            urlcolor=blue,
            linkcolor=magenta,
            pdfborder={0 0 0}}
\urlstyle{same}  % don't use monospace font for urls
\setlength{\parindent}{0pt}
\setlength{\parskip}{6pt plus 2pt minus 1pt}
\setlength{\emergencystretch}{3em}  % prevent overfull lines
\setcounter{secnumdepth}{0}
 
/* start css.sty */
.cmr-5{font-size:50%;}
.cmr-7{font-size:70%;}
.cmmi-5{font-size:50%;font-style: italic;}
.cmmi-7{font-size:70%;font-style: italic;}
.cmmi-10{font-style: italic;}
.cmsy-5{font-size:50%;}
.cmsy-7{font-size:70%;}
.cmex-7{font-size:70%;}
.cmex-7x-x-71{font-size:49%;}
.msbm-7{font-size:70%;}
.cmtt-10{font-family: monospace;}
.cmti-10{ font-style: italic;}
.cmbx-10{ font-weight: bold;}
.cmr-17x-x-120{font-size:204%;}
.cmsl-10{font-style: oblique;}
.cmti-7x-x-71{font-size:49%; font-style: italic;}
.cmbxti-10{ font-weight: bold; font-style: italic;}
p.noindent { text-indent: 0em }
td p.noindent { text-indent: 0em; margin-top:0em; }
p.nopar { text-indent: 0em; }
p.indent{ text-indent: 1.5em }
@media print {div.crosslinks {visibility:hidden;}}
a img { border-top: 0; border-left: 0; border-right: 0; }
center { margin-top:1em; margin-bottom:1em; }
td center { margin-top:0em; margin-bottom:0em; }
.Canvas { position:relative; }
li p.indent { text-indent: 0em }
.enumerate1 {list-style-type:decimal;}
.enumerate2 {list-style-type:lower-alpha;}
.enumerate3 {list-style-type:lower-roman;}
.enumerate4 {list-style-type:upper-alpha;}
div.newtheorem { margin-bottom: 2em; margin-top: 2em;}
.obeylines-h,.obeylines-v {white-space: nowrap; }
div.obeylines-v p { margin-top:0; margin-bottom:0; }
.overline{ text-decoration:overline; }
.overline img{ border-top: 1px solid black; }
td.displaylines {text-align:center; white-space:nowrap;}
.centerline {text-align:center;}
.rightline {text-align:right;}
div.verbatim {font-family: monospace; white-space: nowrap; text-align:left; clear:both; }
.fbox {padding-left:3.0pt; padding-right:3.0pt; text-indent:0pt; border:solid black 0.4pt; }
div.fbox {display:table}
div.center div.fbox {text-align:center; clear:both; padding-left:3.0pt; padding-right:3.0pt; text-indent:0pt; border:solid black 0.4pt; }
div.minipage{width:100%;}
div.center, div.center div.center {text-align: center; margin-left:1em; margin-right:1em;}
div.center div {text-align: left;}
div.flushright, div.flushright div.flushright {text-align: right;}
div.flushright div {text-align: left;}
div.flushleft {text-align: left;}
.underline{ text-decoration:underline; }
.underline img{ border-bottom: 1px solid black; margin-bottom:1pt; }
.framebox-c, .framebox-l, .framebox-r { padding-left:3.0pt; padding-right:3.0pt; text-indent:0pt; border:solid black 0.4pt; }
.framebox-c {text-align:center;}
.framebox-l {text-align:left;}
.framebox-r {text-align:right;}
span.thank-mark{ vertical-align: super }
span.footnote-mark sup.textsuperscript, span.footnote-mark a sup.textsuperscript{ font-size:80%; }
div.tabular, div.center div.tabular {text-align: center; margin-top:0.5em; margin-bottom:0.5em; }
table.tabular td p{margin-top:0em;}
table.tabular {margin-left: auto; margin-right: auto;}
div.td00{ margin-left:0pt; margin-right:0pt; }
div.td01{ margin-left:0pt; margin-right:5pt; }
div.td10{ margin-left:5pt; margin-right:0pt; }
div.td11{ margin-left:5pt; margin-right:5pt; }
table[rules] {border-left:solid black 0.4pt; border-right:solid black 0.4pt; }
td.td00{ padding-left:0pt; padding-right:0pt; }
td.td01{ padding-left:0pt; padding-right:5pt; }
td.td10{ padding-left:5pt; padding-right:0pt; }
td.td11{ padding-left:5pt; padding-right:5pt; }
table[rules] {border-left:solid black 0.4pt; border-right:solid black 0.4pt; }
.hline hr, .cline hr{ height : 1px; margin:0px; }
.tabbing-right {text-align:right;}
span.TEX {letter-spacing: -0.125em; }
span.TEX span.E{ position:relative;top:0.5ex;left:-0.0417em;}
a span.TEX span.E {text-decoration: none; }
span.LATEX span.A{ position:relative; top:-0.5ex; left:-0.4em; font-size:85%;}
span.LATEX span.TEX{ position:relative; left: -0.4em; }
div.float img, div.float .caption {text-align:center;}
div.figure img, div.figure .caption {text-align:center;}
.marginpar {width:20%; float:right; text-align:left; margin-left:auto; margin-top:0.5em; font-size:85%; text-decoration:underline;}
.marginpar p{margin-top:0.4em; margin-bottom:0.4em;}
.equation td{text-align:center; vertical-align:middle; }
td.eq-no{ width:5%; }
table.equation { width:100%; } 
div.math-display, div.par-math-display{text-align:center;}
math .texttt { font-family: monospace; }
math .textit { font-style: italic; }
math .textsl { font-style: oblique; }
math .textsf { font-family: sans-serif; }
math .textbf { font-weight: bold; }
.partToc a, .partToc, .likepartToc a, .likepartToc {line-height: 200%; font-weight:bold; font-size:110%;}
.chapterToc a, .chapterToc, .likechapterToc a, .likechapterToc, .appendixToc a, .appendixToc {line-height: 200%; font-weight:bold;}
.index-item, .index-subitem, .index-subsubitem {display:block}
.caption td.id{font-weight: bold; white-space: nowrap; }
table.caption {text-align:center;}
h1.partHead{text-align: center}
p.bibitem { text-indent: -2em; margin-left: 2em; margin-top:0.6em; margin-bottom:0.6em; }
p.bibitem-p { text-indent: 0em; margin-left: 2em; margin-top:0.6em; margin-bottom:0.6em; }
.paragraphHead, .likeparagraphHead { margin-top:2em; font-weight: bold;}
.subparagraphHead, .likesubparagraphHead { font-weight: bold;}
.quote {margin-bottom:0.25em; margin-top:0.25em; margin-left:1em; margin-right:1em; text-align:\\jmathmathustify;}
.verse{white-space:nowrap; margin-left:2em}
div.maketitle {text-align:center;}
h2.titleHead{text-align:center;}
div.maketitle{ margin-bottom: 2em; }
div.author, div.date {text-align:center;}
div.thanks{text-align:left; margin-left:10%; font-size:85%; font-style:italic; }
div.author{white-space: nowrap;}
.quotation {margin-bottom:0.25em; margin-top:0.25em; margin-left:1em; }
h1.partHead{text-align: center}
.sectionToc, .likesectionToc {margin-left:2em;}
.subsectionToc, .likesubsectionToc {margin-left:4em;}
.subsubsectionToc, .likesubsubsectionToc {margin-left:6em;}
.frenchb-nbsp{font-size:75%;}
.frenchb-thinspace{font-size:75%;}
.figure img.graphics {margin-left:10%;}
/* end css.sty */

\title{Equation differentielle lineaire d'ordre n}
\author{}
\date{}

\begin{document}
\maketitle

\textbf{Warning: 
requires JavaScript to process the mathematics on this page.\\ If your
browser supports JavaScript, be sure it is enabled.}

\begin{center}\rule{3in}{0.4pt}\end{center}

{[}
{[}
{[}{]}
{[}

\subsubsection{16.4 Equation différentielle linéaire d'ordre n}

\paragraph{16.4.1 Généralités}

Soit E un K espace vectoriel normé de dimension finie et considérons
\ell_0,\\ldots,\ell_n-1~
des applications continues de I intervalle de \mathbb{R}~ dans L(E). Soit g : I \rightarrow~
E continue. On peut alors considérer l'équation différentielle linéaire
d'ordre n

y^(n) = \ell_ n-1(t).y^(n-1) +
\\ldots + \ell_
0(t).y + g(t)

En introduisant la fonction inconnue Y =
(y,y',\\ldots,y^(n-1)~),
on sait que cette équation est équivalente par réduction à l'ordre 1, à
l'équation Y ' = L(t).Y + G(t) où l'on a posé
L(t).(y_0,\\ldots,y_n-1~)
=
(y_1,\\ldots,y_n-1,\ell_n-1(t).y_n-1~
+ \\ldots~ +
\ell_0(t).y_0) est clairement une application linéaire de
E^n dans lui même et où G(t) =
(0,\\ldots~,0,g(t))
est une application continue de I dans E^n. Ceci va nous
permettre d'appliquer toute la théorie des équations différentielles
linéaires d'ordre 1 aux équations différentielles linéaires d'ordre n en
tenant compte de ce que l'application
\phi\mapsto~(\phi,\phi',\\ldots,\phi^(n-1)~)
est une bi\\jmathmathection de l'ensemble des solutions de l'équation d'ordre n,
y^(n) = \ell_n-1(t).y^(n-1) +
\\ldots~ +
\ell_0(t).y + g(t) sur l'ensemble des solutions de l'équation
linéaire d'ordre 1, Y ' = L(t).Y + G(t), cette bi\\jmathmathection étant
visiblement linéaire dans le cas où cela a un sens, c'est-à-dire lorsque
ces deux ensembles sont des espaces vectoriels, soit encore dans le cas
d'équations homogènes g = 0 (ce qui équivaut à G = 0).

Par la suite, nous nous intéresserons exclusivement au cas où E = K (le
corps de base). Le lecteur n'aura aucun mal à formuler les résultats
dans le cas d'un E quelconque, tout au moins lorsque cela aura un sens.
L'équation différentielle d'ordre n s'écrit alors sous la forme

y^(n) = a_ n-1(t)y^(n-1) +
\\ldots + a_
0(t)y + b(t)

où les fonctions
a_0,\\ldots,a_n-1~,b
sont des fonctions continues de I dans le corps de base K. L'application
\phi\mapsto~(\phi,\phi',\\ldots,\phi^(n-1)~)
est une bi\\jmathmathection de l'ensemble des solutions de l'équation d'ordre n
sur l'ensemble des solutions de l'équation d'ordre 1, Y ' = A(t).Y +
B(t) avec

A(t) = \left (\matrix\,0
&1&0&\\ldots~&0
\cr \⋮~
&\\ldots&\\\ldots&\\\ldots&\\⋮~
\cr 0
&\\ldots&\\\ldots~&0&1
\cr
a_0(t)&\\ldots&\\\ldots&\\\ldots&a_n-1(t)~\right
)

et

B(t) = \left (\matrix\,0
\cr \⋮~
& \cr 0 \cr b(t)\right )

cette bi\\jmathmathection étant un isomorphisme de l'espace des solutions de
l'équation homogène y^(n) =
a_n-1(t)y^(n-1) +
\\ldots~ +
a_0(t)y sur l'espace des solutions de l'équation homogène Y ' =
A(t).Y .

\paragraph{16.4.2 Théorie de Cauchy-Lipschitz}

Le théorème suivant se déduit immédiatement du théorème correspondant
pour l'équation Y ' = L(t).Y + B(t)

Théorème~16.4.1 Soit I un intervalle de \mathbb{R}~,
a_0,\\ldots,a_n-1~,b
: I \rightarrow~ K continues. Alors toute solution maximale de l'équation
différentielle linéaire y^(n) =
a_n-1(t)y^(n-1) +
\\ldots~ +
a_0(t)y + b(t) est définie sur I. Pour tout t_0 \in I et
tout
(y_0,\\ldotsy_n-1~)
\in K^n, il existe une et une seule solution (I,\phi) de
l'équation différentielle linéaire y^(n) =
a_n-1(t)y^(n-1) +
\\ldots~ +
a_0(t)y + b(t) vérifiant \phi(t_0) =
y_0,\\ldots,\phi^(n-1)(t_0~)
= y_n-1~; pour toute solution (J,\psi) de l'équation
différentielle vérifiant \psi(t_0) =
y_0,\\ldots,\psi^(n-1)(t_0~)
= y_n-1, on a~:

\text\$J \subset~ I\$ et \$\psi\$ est la restriction de \$\phi\$ à
\$J\$.

\paragraph{16.4.3 Structure des solutions de l'équation homogène.
Wronskien}

Le théorème suivant se déduit immédiatement du théorème correspondant
pour l'équation Y ' = L(t).Y

Théorème~16.4.2 Soit I un intervalle de \mathbb{R}~,
a_0,\\ldots,a_n-1~
: I \rightarrow~ K continues. L'ensemble S_H des solutions définies sur I
de l'équation différentielle homogène y^(n) =
a_n-1(t)y^(n-1) +
\\ldots~ +
a_0(t)y est un espace vectoriel de dimension finie égale à n.
Plus précisément, pour tout t_0 \in I, l'application
\epsilon_t_0 :
\phi\mapsto~(\phi(t_0),\phi'(t_0),\\ldots,\phi^(n-1)(t_0~))
est un isomorphisme d'espaces vectoriels de S_H sur
K^n.

Corollaire~16.4.3 Pour toute famille
(\phi_1,\\ldots,\phi_k~)
de solutions de l'équation homogène, on a

\forall~~t \in I,
\mathrmrg(\phi_1,\\\ldots,\phi_k~)
=\
\mathrmrg(\epsilon_t(\phi_1),\\ldots,\epsilon_t(\phi_k~))

Dans le cas k = n ceci amène à la définition suivante

Définition~16.4.1 Soit
(\phi_1,\\ldots,\phi_n~)
des fonctions de classe C^n de I dans K. On appelle wronskien
de la famille l'application
W_\phi_1,\\ldots,\phi_n~
: I \rightarrow~ K,

t\mapsto~\mathrm{det}~
(\epsilon_t(\phi_1),\\ldots,\epsilon_t(\phi_n~))
= \left
\matrix\,\phi_1(t)
&\\ldots&\phi_n~(t)
\cr \phi_1'(t)
&\\ldots&\phi_n~'(t)
\cr
\\ldots~
&\\ldots&\\\ldots~
\cr
\phi_1^(n-1)(t)&\\ldots&\phi_n^(n-1)(t)~\right


Théorème~16.4.4 Soit I un intervalle de \mathbb{R}~,
a_0,\\ldots,a_n-1~
: I \rightarrow~ K continues. Soit
\phi_1,\\ldots,\phi_n~
des solutions de l'équation différentielle homogène y^(n) =
a_n-1(t)y^(n-1) +
\\ldots~ +
a_0(t)y et W leur wronskien. Alors les conditions suivantes
sont équivalentes

\begin{itemize}
\itemsep1pt\parskip0pt\parsep0pt
\item
  (i)
  (\phi_1,\\ldots,\phi_n~)
  est une base de l'espace vectoriel S_H des solutions de
  l'équation homogène y^(n) =
  a_n-1(t)y^(n-1) +
  \\ldots~ +
  a_0(t)y
\item
  (ii) \exists~t \in I,
  W(t)\neq~0
\item
  (iii) \forall~~t \in I,
  W(t)\neq~0
\end{itemize}

Démonstration Puisque \epsilon_t est un isomorphisme de S_H
sur K^n,
(\phi_1,\\ldots,\phi_n~)
est une base de S_H si et seulement
si~(\epsilon_t(\phi_1),\\ldots,\epsilon_t(\phi_n~))
est une base de K^n, c'est-à-dire si et seulement
si~\mathrm{det}~
(\epsilon_t(\phi_1),\\ldots,\epsilon_t(\phi_n))\mathrel\neq~~0
ce qui montre bien l'équivalence des trois assertions. La proposition
suivante expliquera d'ailleurs complètement l'équivalence entre les
assertions (ii) et (iii).

Proposition~16.4.5 Soit I un intervalle de \mathbb{R}~,
a_0,\\ldots,a_n-1~
: I \rightarrow~ K continues. Soit
\phi_1,\\ldots,\phi_n~
des solutions de l'équation différentielle homogène y^(n) =
a_n-1(t)y^(n-1) +
\\ldots~ +
a_0(t)y et W leur wronskien. Alors pour tout y_0,t \in I
on a W(t) = W(t_0)exp~
\left (\int ~
_t_0^ta_n-1(u) du\right
).

Démonstration On sait que pour dériver W(t) on doit faire la somme de
tous les déterminants obtenus en dérivant la i-ème ligne et en laissant
toutes les autres inchangées. Mais comme dans le wronskien, la i + 1-ème
ligne est la dérivée de la i-ème, en dérivant la i-ème ligne et en
laissant inchangée la i + 1-ième (si elle existe), le déterminant obtenu
a deux lignes égales, donc il est nul. On en déduit que le seul
déterminant non trivialement nul est celui obtenu en dérivant la n-ème
ligne, soit encore

W'(t) = \left
\matrix\,\phi_1(t)
&\\ldots&\phi_n~(t)
\cr \phi_1'(t)
&\\ldots&\phi_n~'(t)
\cr
\\ldots~
&\\ldots&\\\ldots~
\cr
\phi_1^(n-2)(t)&\\ldots&\phi_n^(n-2)~(t)
\cr \phi_1^(n)(t)
&\\ldots&\phi_n^(n)~(t)
\right 

En soustrayant à la dernière ligne a_0(t) fois la première,
a_1(t) fois la
seconde,\\ldots~,
a_n-2(t) fois la dernière et tenant compte de ce que
\phi_\\jmathmath^(n)(t) - a_0(t)\phi_\\jmathmath(t) -
a_1(t)\phi_\\jmathmath'(t)
-\\ldots~ -
a_n-2(t)\phi_\\jmathmath^(n-2)(t) =
a_n-1(t)\phi_\\jmathmath^(n-1)(t), on obtient alors

\begin{align*} W'(t)& = \left
\matrix\,\phi_1(t)
&\\ldots&\phi_n~(t)
\cr \phi_1'(t)
&\\ldots&\phi_n~'(t)
\cr
\\ldots~
&\\ldots&\\\ldots~
\cr \phi_1^(n-2)(t)
&\\ldots&\phi_n^(n-2)~(t)
\cr
a_n-1(t)\phi_1^(n-1)(t)&\\ldots&a_n-1(t)\phi_n^(n-1)(t)~\right
 & \%& \\ & =
a_n-1(t)\left
\matrix\,\phi_1(t)
&\\ldots&\phi_n~(t)
\cr \phi_1'(t)
&\\ldots&\phi_n~'(t)
\cr
\\ldots~
&\\ldots&\\\ldots~
\cr
\phi_1^(n-2)(t)&\\ldots&\phi_n^(n-2)~(t)
\cr
\phi_1^(n-1)(t)&\\ldots&\phi_n^(n-1)(t)~\right
 = a_n-1(t)W(t)& \%&
\\ \end{align*}

en utilisant la linéarité du déterminant par rapport à sa dernière
ligne. Donc W est solution de l'équation différentielle linéaire
scalaire d'ordre 1, y' = a_n-1(t)y ce qui implique
immédiatement la formule voulue.

Remarque~16.4.1 Les fonctions \phi_1 :
t\mapsto~t et \phi_2 :
t\mapsto~sin~ t ont comme
wronskien W(t) = \left
\matrix\,t&sin~
t\cr 1 &cos~
t\right  = tcos~ t
- sin~ t. Ce wronskien n'est pas identiquement
nul et pourtant il s'annule en 0. Ceci montre que ces deux fonctions ne
peuvent pas être toutes deux solutions d'une même équation
différentielle homogène d'ordre 2 à coefficients continus sur \mathbb{R}~.

\paragraph{16.4.4 Méthode de variation des constantes}

Supposons connue une base
(\phi_1,\\ldots,\phi_n~)
de l'espace S_H de l'équation différentielle homogène
y^(n) = a_n-1(t)y^(n-1) +
\\ldots~ +
a_0(t)y et posons \Phi_\\jmathmath(t) = \left
(\matrix\,\phi_\\jmathmath(t)
\cr \phi_\\jmathmath'(t) \cr
\\ldots~
\cr \phi_\\jmathmath^(n-1)(t)\right
). Alors
(\Phi_1,\\ldots,\Phi_n~)
est une base de l'espace des solutions du système homogène Y ' = A(t)Y
où

A(t) = \left (\matrix\,0
&1&0&\\ldots~&0
\cr \⋮~
&\\ldots&\\\ldots&\\\ldots&\\⋮~
\cr 0
&\\ldots&\\\ldots~&0&1
\cr
a_0(t)&\\ldots&\\\ldots&\\\ldots&a_n-1(t)~\right
)

Si b : I \rightarrow~ \mathbb{R}~ est une fonction continue, la résolution de l'équation
linéaire y^(n) = a_n-1(t)y^(n-1) +
\\ldots~ +
a_0(t)y + b(t) est équivalente à la résolution du système
différentiel linéaire Y ' = A(t)Y + B(t) où B(t) = \left
(\matrix\,0 \cr
\⋮~ &
\cr 0 \cr b(t)\right ).
Comme nous connaissons une base de l'espace des solutions du système
homogène, nous pouvons résoudre ce système linéaire en posant Y (t) =
\lambda_1(t)\Phi_1(t) +
\\ldots~ +
\lambda_n(t)\Phi_n(t) où
\lambda_1,\\ldots,\lambda_n~
sont des fonctions de classe \mathcal{C}^1 de I dans K. Comme Y (t) =
\left (\matrix\,y(t)
\cr y'(t) \cr
\\ldots~
\cr y^(n-1)(t)\right ) ceci
revient à poser

\begin{align*} y(t)& =&
\lambda_1(t)\phi_1(t) +
\\ldots~ +
\lambda_n(t)\phi_n(t) \%& \\
y'(t)& =& \lambda_1(t)\phi_1'(t) +
\\ldots~ +
\lambda_n(t)\phi_n'(t) \%& \\
\\ldots~& & \%&
\\ y^(n-1)(t)& =& \lambda_
1(t)\phi_1^(n-1)(t) +
\\ldots + \lambda~_
n(t)\phi_n^(n-1)(t)\%& \\
\end{align*}

autrement dit, d'après la règle de dérivation des produits, cela revient
à imposer aux fonctions
\lambda_1,\\ldots,\lambda_n~
de vérifier les conditions

\begin{align*} \lambda_1'(t)\phi_1(t) +
\\ldots~ +
\lambda_n'(t)\phi_n(t)& =& 0\%&
\\ \lambda_1'(t)\phi_1'(t) +
\\ldots~ +
\lambda_n'(t)\phi_n'(t)& =& 0\%&
\\ &
\\ldots~& \%&
\\
\lambda_1'(t)\phi_1^(n-2)(t) +
\\ldots + \lambda~_
n'(t)\phi_n^(n-2)(t)& =& 0\%&
\\ \end{align*}

Dans ces conditions, par dérivation de y^(n-1)(t) =
\lambda_1(t)\phi_1^(n-1)(t) +
\\ldots~ +
\lambda_n(t)\phi_n^(n-1)(t), on obtient

\begin{align*} y^(n)(t)& =& \lambda_
1(t)\phi_1^(n)(t) +
\\ldots + \lambda~_
n(t)\phi_n^(n)(t) \%& \\
& & \quad +
\lambda_1'(t)\phi_1^(n-1)(t) +
\\ldots + \lambda~_
n'(t)\phi_n^(n-1)(t) \%&
\\ & =&
\lambda_1(t)\\sum
_k=0^n-1a_ k(t)\phi_1^(k)(t) +
\ldots + \lambda~_
n(t)\sum _k=0^n-1a_
k(t)\phi_n^(k)(t)\%& \\ &
& \quad + \lambda_1'(t)\phi_1^(n-1)(t) +
\\ldots + \lambda~_
n'(t)\phi_n^(n-1)(t) \%&
\\ & =& \\sum
_k=0^n-1a_ k(t)\left
(\lambda_1(t)\phi_1^(k)(t) +
\ldots + \lambda~_
n(t)\phi_n^(k)(t)\right ) \%&
\\ & & \quad +
\lambda_1'(t)\phi_1^(n-1)(t) +
\\ldots + \lambda~_
n'(t)\phi_n^(n-1)(t) \%&
\\ & =& \\sum
_k=0^n-1a_ k(t)y^(k)(t) \%&
\\ & & \quad +
\lambda_1'(t)\phi_1^(n-1)(t) +
\\ldots + \lambda~_
n'(t)\phi_n^(n-1)(t) \%&
\\ \end{align*}

si bien que

\begin{align*} y^(n)(t) = a_
n-1(t)y^(n-1) +
\\ldots + a_
0(t)y + b(t) \Leftrightarrow&& \%&
\\ & &
\lambda_1'(t)\phi_1^(n-1)(t) +
\\ldots + \lambda~_
n'(t)\phi_n^(n-1)(t) = b(t)\%&
\\ \end{align*}

En conclusion,
\lambda_1'(t),\\ldots,\lambda_n~'(t)
doivent être solutions du système d'équations linéaires

\left \\array
\lambda_1'(t)\phi_1(t) +
\\ldots~ +
\lambda_n'(t)\phi_n(t) & = 0 \cr
\lambda_1'(t)\phi_1'(t) +
\\ldots~ +
\lambda_n'(t)\phi_n'(t) & = 0\cr
\\ldots~
\cr \lambda_1'(t)\phi_1^(n-2)(t) +
\\ldots~ +
\lambda_n'(t)\phi_n^(n-2)(t)& = 0 \cr
\lambda_1'(t)\phi_1^(n-1)(t) +
\\ldots~ +
\lambda_n'(t)\phi_n^(n-1)(t)& = b(t) 
\right .

Or ce système est un système de Cramer (son déterminant est le wronskien
de
\phi_1,\\ldots,\phi_n~
qui par hypothèse ne s'annule pas)~; sa résolution conduit à la
détermination de
\lambda_1',\\ldots,\lambda_n~'
et il reste à faire n calculs de primitives de fonctions à valeurs dans
K pour terminer la résolution de l'équation linéaire.

Méthode. Supposons connue une base
(\phi_1,\\ldots,\phi_n~)
de l'espace S_H de l'équation différentielle homogène
y^(n) = a_n-1(t)y^(n-1) +
\\ldots~ +
a_0(t)y. On résout l'équation linéaire y^(n) =
a_n-1(t)y^(n-1) +
\\ldots~ +
a_0(t)y + b(t) en posant y(t) = \lambda_1(t)\phi_1(t)
+ \\ldots~ +
\lambda_n(t)\phi_n(t), où
\lambda_1,\\ldots,\lambda_n~
sont des fonctions de classe \mathcal{C}^1 auxquelles on impose les
conditions

\begin{align*} \lambda_1'(t)\phi_1(t) +
\\ldots~ +
\lambda_n'(t)\phi_n(t)& = 0& \%&
\\ \lambda_1'(t)\phi_1'(t) +
\\ldots~ +
\lambda_n'(t)\phi_n'(t)& = 0& \%&
\\
\\ldots~& & \%&
\\
\lambda_1'(t)\phi_1^(n-2)(t) +
\\ldots + \lambda~_
n'(t)\phi_n^(n-2)(t)& = 0& \%&
\\ \end{align*}

Alors les fonctions
\lambda_1',\\ldots,\lambda_n~'
sont solution du système de Cramer

\left \\array
\lambda_1'(t)\phi_1(t) +
\\ldots~ +
\lambda_n'(t)\phi_n(t) & = 0 \cr
\lambda_1'(t)\phi_1'(t) +
\\ldots~ +
\lambda_n'(t)\phi_n'(t) & = 0\cr
\\ldots~
\cr \lambda_1'(t)\phi_1^(n-2)(t) +
\\ldots~ +
\lambda_n'(t)\phi_n^(n-2)(t)& = 0 \cr
\lambda_1'(t)\phi_1^(n-1)(t) +
\\ldots~ +
\lambda_n'(t)\phi_n^(n-1)(t)& = b(t) 
\right .

On résout ce système, puis n calculs de primitives permettent de
déterminer les fonctions
\lambda_1,\\ldots,\lambda_n~.

\paragraph{16.4.5 Méthode d'abaissement du degré}

Cette méthode ne présente réellement d'intérêt que pour une équation
linéaire d'ordre 2, y'`= a(t)y' + b(t)y + c(t). Supposons connue une
solution \phi de l'équation homogène qui ne s'annule pas sur I et faisons
le changement de fonction inconnue z(t) = y(t) \over
\phi(t) autrement dit y(t) = z(t)\phi(t). On a alors y'(t) = z'(t)\phi(t) +
z(t)\phi'(t) et y'`(t) = z'`(t)\phi(t) + 2z'(t)\phi'(t) + z(t)\phi''(t), si bien que

\begin{align*} y'`(t) - a(t)y'(t) - c(t)y(t)&&
\%& \\ & =& z'`(t)\phi(t) + (2\phi'(t) -
\phi(t))z'(t) \%& \\ & &
\quad + (\phi'`(t) - a(t)\phi'(t) - b(t)\phi(t))z(t)\%&
\\ & =& z'`(t)\phi(t) + (2\phi'(t) -
\phi(t))z'(t) \%& \\
\end{align*}

compte tenu de ce que \phi'`(t) - a(t)\phi'(t) - b(t)\phi(t) = 0. On en déduit
que

\begin{align*} y'`(t) = a(t)y'(t) + b(t)y(t) + c(t)
\Leftrightarrow&& \%& \\
& & z'`(t)\phi(t) + (2\phi'(t) - \phi(t))z'(t) = c(t)\%&
\\ \end{align*}

qui est une équation différentielle d'ordre 1 en la fonction inconnue
z'(t), que l'on sait donc résoudre à l'aide de deux calculs de
primitives.

Remarque~16.4.2 Cette méthode n'est à utiliser qu'en dernier recours,
c'est-à-dire lorsqu'on ne dispose que d'une seule solution de l'équation
homogène, et on doit tou\\jmathmathours lui préférer (lorsque c'est possible) la
méthode de variation des constantes. En effet la méthode d'abaissement
du degré demande que soit réalisée une condition contraignante (une
solution ne s'annulant pas) et conduit à deux calculs de primitives pour
obtenir z'(t) et un troisième calcul de primitive pour obtenir z(t). De
plus ces trois calculs s'enchaînent (toute erreur dans l'un des calculs
oblige à recommencer l'ensemble des calculs). Par contre, la méthode de
variation des constantes n'impose aucune condition restrictive aux
solutions de l'équation homogène et ne nécessite que deux calculs de
primitives, qui plus est indépendants l'un de l'autre.

\paragraph{16.4.6 Equation homogène à coefficients constants}

Nous étudierons dans ce paragraphe l'équation y^(n) +
a_n-1y^(n-1) +
\\ldots~ +
a_0y = 0 où
a_0,\\ldots,a_n-1~
sont des éléments donnés du corps de base K (égal à \mathbb{R}~ ou \mathbb{C}). En cas de
besoin et pour unifier les notations, nous poserons a_n = 1.

Définition~16.4.2 Le polynôme \chi(X) = X^n +
a_n-1X^n-1 +
\\ldots~ +
a_0 \in K{[}X{]} est appelé le polynôme caractéristique de
l'équation homogène y^(n) + a_n-1y^(n-1)
+ \\ldots~ +
a_0y = 0.

Remarque~16.4.3 t\mapsto~e^\lambda~t est
solution de y^(n) + a_n-1y^(n-1) +
\\ldots~ +
a_0y = 0 si et seulement si~\chi(\lambda~) = 0.

Lemme~16.4.6 Soit P(X) \in K{[}X{]} et \lambda~ \in K. Soit f :
t\mapsto~P(t)e^\lambda~t de \mathbb{R}~ dans K. Alors

\forall~t \in \mathbb{R}~, f^(n)(t) + a_
n-1f^(n-1)(t) +
\\ldots + a_
0f(t) = e^\lambda~t \\sum
_p=0^n 1 \over p!
\chi^(p)(\lambda~)P^(p)(t)

Démonstration La formule de Leibnitz nous donne f^(k)(t)
= \\sum ~
_p=0^kC_k^pP^(p)(t)\lambda~^k-pe^\lambda~t
si bien que

\begin{align*} f^(n)(t) + a_
n-1f^(n-1)(t) +
\\ldots + a_
0f(t) = \sum _k=0^na_
kf^(k)(t)&&\%& \\ & =&
e^\lambda~t \\sum
_k=0^na_ k \\sum
_p=0^k k! \over p!(k - p)!
P^(p)(t)\lambda~^k-p \%&
\\ & =& e^\lambda~t
\sum _0\leqp\leqk\leqna_k~ k!
\over p!(k - p)! P^(p)(t)\lambda~^k-p
\%& \\ & =& e^\lambda~t
\sum _p=0^n~ 1
\over p! P^(p)(t)\\sum
_k=p^na_ k k! \over (k - p)!
\lambda~^k-p\%& \\ & =&
e^\lambda~t \sum _p=0^n~ 1
\over p! \chi^(p)(\lambda~)P^(p)(t) \%&
\\ \end{align*}

Supposons donc que \lambda~ est une racine de \chi de multiplicité m et que
deg~ P \leq m - 1. On a alors
\forall~p \leq m - 1, \chi^(p)~(\lambda~) = 0 et
\forall~p ≥ m, P^(p)~(t) = 0, si bien que
\forall~~p \in {[}0,n{]},
\chi^(p)(\lambda~)P^(p)(t) = 0. On en déduit donc que f :
t\mapsto~e^\lambda~tP(t) est solution de
l'équation différentielle y^(n) +
a_n-1y^(n-1) +
\\ldots~ +
a_0y = 0.

Lemme~16.4.7 Soit
\lambda_1,\\ldots,\lambda_k~
des éléments distincts de K et
m_1,\\ldots,m_k~
des entiers naturels. Alors la famille des applications
t\mapsto~t^\\jmathmathe^\lambda_it
avec 1 \leq i \leq k et 0 \leq \\jmathmath \leq m_i - 1 est libre.

Démonstration Supposons que cette famille est liée. Notons n =
m_1 + ⋯ + m_k,
f_1,\\ldots,f_n~
ces fonctions. Il existe donc
\alpha_1,\\ldots,\alpha_n~
\in K, non tous nuls, tels que \forall~~t \in \mathbb{R}~,
\alpha_1f_1(t) +
\\ldots~ +
\alpha_nf_n(t) = 0. Fixons t_0 \in \mathbb{R}~. Par dérivation
de l'identité précédente au point t_0, on a donc
\forall~~k \in \mathbb{N}~,
a_1f_1^(k)(t_0) +
\\ldots~ +
\alpha_nf_n^(k)(t_0) = 0 et en particulier
la matrice wronskienne \left
(\matrix\,f_1(t_0)
&\\ldots&f_n(t_0~)
\cr f_1'(t_0)
&\\ldots&f_n'(t_0~)
\cr
\\ldots~
&\\ldots&\\\ldots~
\cr
f_1^(n-1)(t_0)&\\ldots&f_n^(n-1)(t_0)~\right
) n'est pas inversible puisque ses vecteurs colonnes forment une famille
liée. On en déduit que ses vecteurs lignes forment une famille liée, et
que donc il existe
\beta_0,\\ldots,\beta_n-1~
non tous nuls (mais dépendant de t_0) tels que

\forall~~\\jmathmath \in {[}1,n{]},
\beta_0f_\\jmathmath(t_0) +
\beta_1f_\\jmathmath'(t_0) +
\\ldots~ +
\beta_n-1f_\\jmathmath^(n-1)(t_ 0) = 0

Posons alors \chi(X) = \beta_n-1X^n-1 +
⋯ + \beta_0. Le lemme précédent (où l'on
remplace n par n - 1) nous montre que si f(t) = e^\lambda~tP(t),
alors

\beta_0f(t_0) + \beta_1f'(t_0) +
\\ldots~ +
\beta_n-1f^(n-1)(t_ 0) =
e^\lambda~t_0  \\sum
_p=0^n-1 1 \over p!
\chi^(p)(\lambda~)P^(p)(t_ 0)

En particulier, pour P(X) = X^\\jmathmath et \lambda~ = \lambda_i, on
obtient

\begin{align*} \beta_0f(t_0) +
\beta_1f'(t_0) +
\\ldots~ +
\beta_n-1f^(n-1)(t_ 0)& & \%&
\\ = e^\lambda_it_0
 \sum _p=0^\\jmathmath~ 1
\over p! \chi^(p)(\lambda_ i) \\jmathmath!
\over (\\jmathmath - p)! t_0^\\jmathmath-p& & \%&
\\ \end{align*}

car les dérivées suivantes de X^\\jmathmath sont nulles. On en déduit
que

\forall~\\jmathmath \leq m_i~ - 1,
\sum _p=0^\\jmathmath~ 1
\over p! \chi^(p)(\lambda_ i) \\jmathmath!
\over (\\jmathmath - p)! t_0^\\jmathmath-p = 0

autrement dit (compte tenu de  \\jmathmath! \over p!(p-\\jmathmath)! =
C_\\jmathmath^p)

\left \\array
C_0^0\chi(\lambda_i) & = 0 \cr
C_1^0\chi(\lambda_i)t_0 +
C_1^1\chi'(\lambda_i) & = 0 \cr
C_2^0\chi(\lambda_i)t_0^2 +
C_2^1\chi'(\lambda_i)t +
C_2^2\chi''(\lambda_i) = 0\cr
&\\ldots~
\cr
C_m_i-1^0\chi(\lambda~)t_0^m_i-1
+ \\ldots~ +
C_m_
i-1^m_i-2\chi^(m_i-2)(\lambda~)t +
C_
m_i-1^m_i-1\chi^(m_i-1)(\lambda_
i)& = 0  \right .

ce qui implique évidemment que \chi(\lambda_i) =
\\ldots~ =
\chi^(m_i-1)(\lambda_i) = 0. Donc \lambda_i est
racine de \chi de multiplicité au moins égale à m_i. Mais alors la
somme des multiplicités des racines du polynôme non nul \chi(X) est au
moins égale à m_1 +
\\ldots~ +
m_k = n alors qu'il est de degré au plus n - 1. C'est absurde,
ce qui montre que la famille est libre.

Revenons à notre équation différentielle homogène y^(n) +
a_n-1y^(n-1) +
\\ldots~ +
a_0y = 0 et supposons que son polynôme caractéristique \chi(X) est
scindé sur K (ce qui est automatique si K = \mathbb{C}). Soit
\lambda_1,\\ldots,\lambda_k~
ses racines distinctes de multiplicités respectives
m_1,\\ldots,m_k~,
si bien que m_1 +
\\ldots~ +
m_k = n. Alors les n fonctions
t\mapsto~t^\\jmathmathe^\lambda_it
avec 1 \leq i \leq k et 0 \leq \\jmathmath \leq m_i - 1 sont solutions de l'équation
différentielle homogène et forment une famille libre. Comme l'espace des
solutions de l'équation homogène est de dimension n, ces fonctions
forment une base de l'espace des solutions. Autrement dit les solutions
de l'équation homogène sont exactement les fonctions qui s'écrivent sous
la forme

t\mapsto~\\sum
_i=1^k \\sum
_\\jmathmath=0^m_i-1\alpha_
i,\\jmathmatht^\\jmathmathe^\lambda_it =
\sum _i=1^kP_
i(t)e^\lambda_it

avec P_i(X) =\
\sum ~
_i=0^m_i-1\alpha_i,\\jmathmathX^\\jmathmath \in K{[}X{]}
et deg P_i \leq m_i~ - 1. On a
donc démontré le résultat suivant

Théorème~16.4.8 Soit
a_0,\\ldots,a_n-1~
\in K et l'équation différentielle homogène y^(n) +
a_n-1y^(n-1) +
\\ldots~ +
a_0y = 0. On suppose que le polynôme caractéristique \chi(X) =
X^n + a_n-1X^n-1 +
\\ldots~ +
a_0 est scindé sur K (ce qui est automatique si K = \mathbb{C}). Soit
\lambda_1,\\ldots,\lambda_k~
ses racines distinctes de multiplicités respectives
m_1,\\ldots,m_k~.
Alors les solutions de l'équation homogène sont exactement les fonctions

t\mapsto~\\sum
_i=1^kP_
i(t)e^\lambda_it\quad
\text avec \$P_ i(X) \in K{[}X{]}\$ et \$deg
P_i \leq m_i - 1\$.

\paragraph{16.4.7 Equation linéaire à coefficients constants}

Il s'agit ici de résoudre une équation différentielle linéaire à
coefficients constants y^(n) +
a_n-1y^(n-1) +
\\ldots~ +
a_0y = b(t) où b est une application continue de I dans K. Dans
le cas général, puisque nous savons résoudre l'équation homogène, la
méthode de variation des constantes permet d'aboutir au résultat au prix
du calcul de n primitives. Mais d'autre part, il suffit évidemment de
déterminer une solution particulière de l'équation différentielle
linéaire pour en avoir la solution générale en a\\jmathmathoutant à cette solution
particulière la solution générale de l'équation homogène.

Examinons le cas particulier où b(t) = Q(t)e^\mut avec Q(X) \in
K{[}X{]} et \mu \in K. Nous allons rechercher une solution particulière du
type f(t) = P(t)e^\mut. On sait alors que

f^(n)(t) + a_ n-1f^(n-1)(t) +
\\ldots + a_
0f(t) = e^\lambda~t \\sum
_p=0^n 1 \over p!
\chi^(p)(\lambda~)P^(p)(t)

Autrement dit, f sera solution de l'équation linéaire si et seulement
si~\\sum ~
_p=0^n 1 \over p!
\chi^(p)(\lambda~)P^(p)(X) = Q(X). Soit m la multiplicité de
\mu comme racine de \chi (nous poserons m = 0 si \mu n'est pas racine de \chi). On
a alors \chi(\mu) =
\\ldots~ =
\chi^(m-1)(\mu) = 0 et
\chi^(m)(\mu)\neq~0. On a donc à résoudre
l'équation

\sum _p=m^n~ 1
\over p! \chi^(p)(\mu)P^(p)(X) = Q(X)

Cela se fera par identification si nous connaissons un ma\\jmathmathorant du degré
de P. Mais pour cela il suffit d'appliquer le lemme suivant, où l'on
désigne par K_p{[}X{]} l'espace des polynômes de degré
inférieur ou égal à p,

Lemme~16.4.9 Soit d \in \mathbb{N}~, l'application \theta :
P\mapsto~\\\sum
 _p=m^n 1 \over p!
\chi^(p)(\mu)P^(p)(X) est une application linéaire
sur\\jmathmathective de K_d+m{[}X{]} dans K_d{[}X{]}.

Démonstration Si deg~ P \leq d + m, alors pour p ≥
m, on a deg P^(p)~(X) \leq d ce qui
montre que \theta(P) =\ \\sum
 _p=m^n 1 \over p!
\chi^(p)(\mu)P^(p)(X) \in K_ d{[}X{]}. Cette
application est visiblement linéaire. Cherchons le rang de cette
application linéaire et pour cela déterminons sa matrice. Si \\jmathmath \leq d + m,
on a

\theta(X^\\jmathmath) = \sum _p=m^\\jmathmath~
1 \over p! \chi^(p)(\mu) \\jmathmath!
\over (\\jmathmath - p)! X^\\jmathmath-p =
\sum _p=m^\\jmathmathC_
\\jmathmath^p\chi^(p)(\mu)X^\\jmathmath-p

si bien que la matrice de \theta dans les bases canoniques
(1,X,\\ldots,X^d+m~)
et
(1,X,\\ldots,X^d~)
est la matrice

\left ( \includegraphics{cours9x.png}
\,\right )

Or la matrice formée par les d + 1 dernières colonnes est visiblement
inversible, ce qui montre que le rang de \theta est égal à d + 1
= dim K_d~{[}X{]} et donc que \theta est
sur\\jmathmathective. On a donc la proposition suivante

Proposition~16.4.10 Soit
a_0,\\ldots,a_n-1~
\in K, Q \in K{[}X{]} et \mu \in K. L'équation différentielle linéaire
y^(n) + a_n-1y^(n-1) +
\\ldots~ +
a_0y = Q(t)e^\mut admet au moins une solution de la
forme P(t)e^\mut où P \in K{[}X{]} et
deg P \leq\ deg~ Q + m, m
désignant la multiplicité de \mu comme racine du polynôme caractéristique
de l'équation homogène.

Remarque~16.4.4 Le fait qu'il faille a\\jmathmathouter au degré de Q la
multiplicité m de \mu comme racine de \chi s'appelle le phénomène de
résonance. Il implique que même si Q est constante, il peut exister des
solutions du type P(t)e^\mut avec deg~
P ≥ 1 qui peuvent être non bornées et entraîner des catastrophes dans le
système contrôlé par l'équation différentielle.

Remarque~16.4.5 La méthode précédente par identification permet
également de résoudre des équations du type y^(n) +
a_n-1y^(n-1) +
\\ldots~ +
a_0y = \\sum ~
_i=1^qQ_i(t)e^\mu_it en
remarquant que si f_i est une particulière de l'équation
y^(n) + a_n-1y^(n-1) +
\\ldots~ +
a_0y = Q_i(t)e^\mu_it, alors
f_1 + ⋯ + f_q est solution
de y^(n) + a_n-1y^(n-1) +
\\ldots~ +
a_0y = \\sum ~
_i=1^qQ_i(t)e^\mu_it (ce que
l'on appelle le principe de superposition des solutions). Elle
s'applique en particulier à des équations du type y^(n) +
a_n-1y^(n-1) +
\\ldots~ +
a_0y = Q(t)cos~ (\omegat) ou
y^(n) + a_n-1y^(n-1) +
\\ldots~ +
a_0y = Q(t)sin~ (\omegat) pour lesquelles
il suffit de passer en exponentielle complexe~: poser
cos (\omegat) = 1 \over 2~
(e^i\omegat + e^-i\omegat) et sin~
(\omegat) = 1 \over 2i (e^i\omegat -
e^-i\omegat).

\paragraph{16.4.8 Equations d'Euler}

Considérons une équation différentielle homogène du type
t^ny^(n) +
a_n-1t^n-1y^(n-1) +
\\ldots~ +
a_1ty' + a_0y = 0 que nous étudierons sur {]}0,+\infty~{[}
(il suffit de changer t en - t pour faire une étude similaire sur {]}
-\infty~,0{[}). Faisons le changement de variable t = e^u, soit
encore u = log~ t. Nous poserons donc y(t) =
z(u) soit encore y(t) = z(log~ t). Une
récurrence facile montre que

\forall~k \in \mathbb{N}~, y^(k)~(t) = 1
\over t^k  \\sum
_p=0^k\lambda_ k,pz^(p)(log t)

C'est clair pour k = 0 et si c'est vrai pour k, on a

\begin{align*} y^(k+1)(t)& =& - k
\over t^k+1  \\sum
_p=0^k\lambda_ k,pz^(p)(log t) + 1
\over t^k  \\sum
_p=0^k\lambda_ k,p 1 \over t
z^(p+1)(log t) \%& \\ & =&
- k \over t^k+1 
\sum _p=0^k\lambda~_
k,pz^(p)(log t) + 1 \over
t^k+1  \\sum
_p=0^k\lambda_ k,pz^(p+1)(log t)\%&
\\ & =& 1 \over
t^k+1  \\sum
_p=0^k+1\lambda_ k+1,pz^(p)(log t) \%&
\\ \end{align*}

avec \lambda_k+1,p = -k\lambda_k,p + \lambda_k,p-1 si 1 \leq p \leq
k, \lambda_k+1,0 = -k\lambda_k,0 et \lambda_k+1,k+1 =
\lambda_k,k.

On a donc t^ky^(k)(t) =\
\sum ~
_p=0^k\lambda_k,pz^(p)(u) si bien que
l'équation devient une équation homogène d'ordre n à coefficients
constants en la fonction z(u). Ses solutions sont du type z(u)
= \\sum ~
_i=1^ke^\lambda_iuP_i(u) si bien
que les solutions de l'équation d'Euler sont de la forme

y(t) = \\sum
_i=1^kt^\lambda_i P_i(log t)

On retiendra

Proposition~16.4.11 Dans une équation d'Euler
t^ny^(n) +
a_n-1t^n-1y^(n-1) +
\\ldots~ +
a_1ty' + a_0y = 0, le changement de variable t =
e^u conduit à une équation différentielle homogène d'ordre n
à coefficients constants.

{[}
{[}
{[}
{[}

\end{document}

% \documentclass[]{article}
\usepackage[T1]{fontenc}
\usepackage{lmodern}
\usepackage{amssymb,amsmath}
\usepackage{ifxetex,ifluatex}
\usepackage{fixltx2e} % provides \textsubscript
% use upquote if available, for straight quotes in verbatim environments
\IfFileExists{upquote.sty}{\usepackage{upquote}}{}
\ifnum 0\ifxetex 1\fi\ifluatex 1\fi=0 % if pdftex
  \usepackage[utf8]{inputenc}
\else % if luatex or xelatex
  \ifxetex
    \usepackage{mathspec}
    \usepackage{xltxtra,xunicode}
  \else
    \usepackage{fontspec}
  \fi
  \defaultfontfeatures{Mapping=tex-text,Scale=MatchLowercase}
  \newcommand{\euro}{€}
\fi
% use microtype if available
\IfFileExists{microtype.sty}{\usepackage{microtype}}{}
\usepackage{graphicx}
% Redefine \includegraphics so that, unless explicit options are
% given, the image width will not exceed the width of the page.
% Images get their normal width if they fit onto the page, but
% are scaled down if they would overflow the margins.
\makeatletter
\def\ScaleIfNeeded{%
  \ifdim\Gin@nat@width>\linewidth
    \linewidth
  \else
    \Gin@nat@width
  \fi
}
\makeatother
\let\Oldincludegraphics\includegraphics
{%
 \catcode`\@=11\relax%
 \gdef\includegraphics{\@ifnextchar[{\Oldincludegraphics}{\Oldincludegraphics[width=\ScaleIfNeeded]}}%
}%
\ifxetex
  \usepackage[setpagesize=false, % page size defined by xetex
              unicode=false, % unicode breaks when used with xetex
              xetex]{hyperref}
\else
  \usepackage[unicode=true]{hyperref}
\fi
\hypersetup{breaklinks=true,
            bookmarks=true,
            pdfauthor={},
            pdftitle={Equations differentielles non lineaires},
            colorlinks=true,
            citecolor=blue,
            urlcolor=blue,
            linkcolor=magenta,
            pdfborder={0 0 0}}
\urlstyle{same}  % don't use monospace font for urls
\setlength{\parindent}{0pt}
\setlength{\parskip}{6pt plus 2pt minus 1pt}
\setlength{\emergencystretch}{3em}  % prevent overfull lines
\setcounter{secnumdepth}{0}
 
/* start css.sty */
.cmr-5{font-size:50%;}
.cmr-7{font-size:70%;}
.cmmi-5{font-size:50%;font-style: italic;}
.cmmi-7{font-size:70%;font-style: italic;}
.cmmi-10{font-style: italic;}
.cmsy-5{font-size:50%;}
.cmsy-7{font-size:70%;}
.cmex-7{font-size:70%;}
.cmex-7x-x-71{font-size:49%;}
.msbm-7{font-size:70%;}
.cmtt-10{font-family: monospace;}
.cmti-10{ font-style: italic;}
.cmbx-10{ font-weight: bold;}
.cmr-17x-x-120{font-size:204%;}
.cmsl-10{font-style: oblique;}
.cmti-7x-x-71{font-size:49%; font-style: italic;}
.cmbxti-10{ font-weight: bold; font-style: italic;}
p.noindent { text-indent: 0em }
td p.noindent { text-indent: 0em; margin-top:0em; }
p.nopar { text-indent: 0em; }
p.indent{ text-indent: 1.5em }
@media print {div.crosslinks {visibility:hidden;}}
a img { border-top: 0; border-left: 0; border-right: 0; }
center { margin-top:1em; margin-bottom:1em; }
td center { margin-top:0em; margin-bottom:0em; }
.Canvas { position:relative; }
li p.indent { text-indent: 0em }
.enumerate1 {list-style-type:decimal;}
.enumerate2 {list-style-type:lower-alpha;}
.enumerate3 {list-style-type:lower-roman;}
.enumerate4 {list-style-type:upper-alpha;}
div.newtheorem { margin-bottom: 2em; margin-top: 2em;}
.obeylines-h,.obeylines-v {white-space: nowrap; }
div.obeylines-v p { margin-top:0; margin-bottom:0; }
.overline{ text-decoration:overline; }
.overline img{ border-top: 1px solid black; }
td.displaylines {text-align:center; white-space:nowrap;}
.centerline {text-align:center;}
.rightline {text-align:right;}
div.verbatim {font-family: monospace; white-space: nowrap; text-align:left; clear:both; }
.fbox {padding-left:3.0pt; padding-right:3.0pt; text-indent:0pt; border:solid black 0.4pt; }
div.fbox {display:table}
div.center div.fbox {text-align:center; clear:both; padding-left:3.0pt; padding-right:3.0pt; text-indent:0pt; border:solid black 0.4pt; }
div.minipage{width:100%;}
div.center, div.center div.center {text-align: center; margin-left:1em; margin-right:1em;}
div.center div {text-align: left;}
div.flushright, div.flushright div.flushright {text-align: right;}
div.flushright div {text-align: left;}
div.flushleft {text-align: left;}
.underline{ text-decoration:underline; }
.underline img{ border-bottom: 1px solid black; margin-bottom:1pt; }
.framebox-c, .framebox-l, .framebox-r { padding-left:3.0pt; padding-right:3.0pt; text-indent:0pt; border:solid black 0.4pt; }
.framebox-c {text-align:center;}
.framebox-l {text-align:left;}
.framebox-r {text-align:right;}
span.thank-mark{ vertical-align: super }
span.footnote-mark sup.textsuperscript, span.footnote-mark a sup.textsuperscript{ font-size:80%; }
div.tabular, div.center div.tabular {text-align: center; margin-top:0.5em; margin-bottom:0.5em; }
table.tabular td p{margin-top:0em;}
table.tabular {margin-left: auto; margin-right: auto;}
div.td00{ margin-left:0pt; margin-right:0pt; }
div.td01{ margin-left:0pt; margin-right:5pt; }
div.td10{ margin-left:5pt; margin-right:0pt; }
div.td11{ margin-left:5pt; margin-right:5pt; }
table[rules] {border-left:solid black 0.4pt; border-right:solid black 0.4pt; }
td.td00{ padding-left:0pt; padding-right:0pt; }
td.td01{ padding-left:0pt; padding-right:5pt; }
td.td10{ padding-left:5pt; padding-right:0pt; }
td.td11{ padding-left:5pt; padding-right:5pt; }
table[rules] {border-left:solid black 0.4pt; border-right:solid black 0.4pt; }
.hline hr, .cline hr{ height : 1px; margin:0px; }
.tabbing-right {text-align:right;}
span.TEX {letter-spacing: -0.125em; }
span.TEX span.E{ position:relative;top:0.5ex;left:-0.0417em;}
a span.TEX span.E {text-decoration: none; }
span.LATEX span.A{ position:relative; top:-0.5ex; left:-0.4em; font-size:85%;}
span.LATEX span.TEX{ position:relative; left: -0.4em; }
div.float img, div.float .caption {text-align:center;}
div.figure img, div.figure .caption {text-align:center;}
.marginpar {width:20%; float:right; text-align:left; margin-left:auto; margin-top:0.5em; font-size:85%; text-decoration:underline;}
.marginpar p{margin-top:0.4em; margin-bottom:0.4em;}
.equation td{text-align:center; vertical-align:middle; }
td.eq-no{ width:5%; }
table.equation { width:100%; } 
div.math-display, div.par-math-display{text-align:center;}
math .texttt { font-family: monospace; }
math .textit { font-style: italic; }
math .textsl { font-style: oblique; }
math .textsf { font-family: sans-serif; }
math .textbf { font-weight: bold; }
.partToc a, .partToc, .likepartToc a, .likepartToc {line-height: 200%; font-weight:bold; font-size:110%;}
.chapterToc a, .chapterToc, .likechapterToc a, .likechapterToc, .appendixToc a, .appendixToc {line-height: 200%; font-weight:bold;}
.index-item, .index-subitem, .index-subsubitem {display:block}
.caption td.id{font-weight: bold; white-space: nowrap; }
table.caption {text-align:center;}
h1.partHead{text-align: center}
p.bibitem { text-indent: -2em; margin-left: 2em; margin-top:0.6em; margin-bottom:0.6em; }
p.bibitem-p { text-indent: 0em; margin-left: 2em; margin-top:0.6em; margin-bottom:0.6em; }
.paragraphHead, .likeparagraphHead { margin-top:2em; font-weight: bold;}
.subparagraphHead, .likesubparagraphHead { font-weight: bold;}
.quote {margin-bottom:0.25em; margin-top:0.25em; margin-left:1em; margin-right:1em; text-align:justify;}
.verse{white-space:nowrap; margin-left:2em}
div.maketitle {text-align:center;}
h2.titleHead{text-align:center;}
div.maketitle{ margin-bottom: 2em; }
div.author, div.date {text-align:center;}
div.thanks{text-align:left; margin-left:10%; font-size:85%; font-style:italic; }
div.author{white-space: nowrap;}
.quotation {margin-bottom:0.25em; margin-top:0.25em; margin-left:1em; }
h1.partHead{text-align: center}
.sectionToc, .likesectionToc {margin-left:2em;}
.subsectionToc, .likesubsectionToc {margin-left:4em;}
.subsubsectionToc, .likesubsubsectionToc {margin-left:6em;}
.frenchb-nbsp{font-size:75%;}
.frenchb-thinspace{font-size:75%;}
.figure img.graphics {margin-left:10%;}
/* end css.sty */

\title{Equations differentielles non lineaires}
\author{}
\date{}

\begin{document}
\maketitle

\textbf{Warning: \href{http://www.math.union.edu/locate/jsMath}{jsMath}
requires JavaScript to process the mathematics on this page.\\ If your
browser supports JavaScript, be sure it is enabled.}

\begin{center}\rule{3in}{0.4pt}\end{center}

{[}\href{coursse91.html}{next}{]} {[}\href{coursse89.html}{prev}{]}
{[}\href{coursse89.html\#tailcoursse89.html}{prev-tail}{]}
{[}\hyperref[tailcoursse90.html]{tail}{]}
{[}\href{coursch17.html\#coursse90.html}{up}{]}

\subsubsection{16.5 Equations différentielles non linéaires}

\paragraph{16.5.1 Théorie de Cauchy-Lipschitz}

Définition~16.5.1 Soit E un espace vectoriel normé de dimension finie, U
un ouvert de ℝ × E et F : U → E,
(t,y)\textbackslash{}mathrel\{↦\}F(t,y). On dira que F est localement
lipschitzienne par rapport à la variable y si, pour tout
(\{t\}\_\{0\},\{y\}\_\{0\}) ∈ U, il existe η \textgreater{} 0 et r
\textgreater{} 0 et une constante L ≥ 0 telle que

\textbackslash{}begin\{eqnarray*\} \textbackslash{}mathop\{∀\}t ∈
{[}\{t\}\_\{0\} − η,\{t\}\_\{0\} + η{]},
\textbackslash{}mathop\{∀\}\{y\}\_\{1\},\{y\}\_\{2\} ∈
B'(\{y\}\_\{0\},r),\& \& \%\& \textbackslash{}\textbackslash{}
\textbackslash{}\textbar{}F(t,\{y\}\_\{1\}) −
F(t,\{y\}\_\{2\})\textbackslash{}\textbar{} ≤
L\textbackslash{}\textbar{}\{y\}\_\{1\} −
\{y\}\_\{2\}\textbackslash{}\textbar{}\& \& \%\&
\textbackslash{}\textbackslash{} \textbackslash{}end\{eqnarray*\}

L'ensemble C = {[}\{t\}\_\{0\} − η,\{t\}\_\{0\} + η{]} ×
B'(\{y\}\_\{0\},r) sera appelé un cylindre de sécurité associé à la
constante L.

Théorème~16.5.1 Soit E un espace vectoriel normé de dimension finie, U
un ouvert de ℝ × E et f : U → E, (t,y)\textbackslash{}mathrel\{↦\}f(t,y)
de classe \{C\}\^{}\{1\}. Alors f est localement lipschitzienne par
rapport à la variable y.

Démonstration Notons \{∂\}\_\{y\}f(t,y) la différentielle au point y de
l'application z\textbackslash{}mathrel\{↦\}f(t,z) de E dans E. Cette
application est composée de z\textbackslash{}mathrel\{↦\}(t,z) dont la
différentielle en tout point est l'application j :
h\textbackslash{}mathrel\{↦\}(0,h) et de l'application f. Si bien que
\{∂\}\_\{y\}f(t,y).h = df(t,y).(0,h) = df(t,y) ∘ j(h). On en déduit que
l'application (t,y)\textbackslash{}mathrel\{↦\}\{∂\}\_\{y\}f(t,y) =
df(t,y) ∘ j est continue (comme l'application
(t,y)\textbackslash{}mathrel\{↦\}df(t,y)). Soit alors η \textgreater{} 0
et r \textgreater{} 0 tels que {[}\{t\}\_\{0\} − η,\{t\}\_\{0\} + η{]} ×
B'(0,r) ⊂ U. L'application
(t,y)\textbackslash{}mathrel\{↦\}\{∂\}\_\{y\}f(t,y) est continue sur le
compact {[}\{t\}\_\{0\} − η,\{t\}\_\{0\} + η{]} × B'(0,r), donc elle y
est bornée. Il existe donc L ≥ 0 tel que
\textbackslash{}mathop\{∀\}(t,y) ∈ {[}\{t\}\_\{0\} − η,\{t\}\_\{0\} +
η{]} × B'(0,r),
\textbackslash{}\textbar{}\{∂\}\_\{y\}f(t,y)\textbackslash{}\textbar{} ≤
L. Mais alors, soit t ∈ {[}\{t\}\_\{0\} − η,\{t\}\_\{0\} + η{]},
\{y\}\_\{1\},\{y\}\_\{2\} ∈ B'(\{y\}\_\{0\},r)~; l'inégalité des
accroissements finis appliquée à la fonction
y\textbackslash{}mathrel\{↦\}f(t,y) sur le segment
{[}\{y\}\_\{1\},\{y\}\_\{2\}{]} ⊂ B'(\{y\}\_\{0\},r) assure que

\textbackslash{}\textbar{}f(t,\{y\}\_\{1\}) −
f(t,\{y\}\_\{2\})\textbackslash{}\textbar{} ≤\textbackslash{}\textbar{}
\{y\}\_\{1\} −
\{y\}\_\{2\}\textbackslash{}\textbar{}\{\textbackslash{}mathop\{
sup\}\}\_\{y∈{[}\{y\}\_\{1\},\{y\}\_\{2\}{]}\}\textbackslash{}\textbar{}\{∂\}\_\{y\}f(t,y)\textbackslash{}\textbar{}
≤ L\textbackslash{}\textbar{}\{y\}\_\{1\} −
\{y\}\_\{2\}\textbackslash{}\textbar{}

ce qui démontre le résultat.

Théorème~16.5.2 (Cauchy Lipschitz, unicité). Soit E un espace vectoriel
normé de dimension finie, U un ouvert de ℝ × E et F : U → E,
(t,y)\textbackslash{}mathrel\{↦\}F(t,y) localement lipschitzienne par
rapport à la variable y. Alors F vérifie la condition d'unicité au
problème de Cauchy-Lipschitz~:

soit (I,φ) et (J,ψ) deux solutions de l'équation différentielle y' =
F(t,y) qui coïncident au point \{t\}\_\{0\} ∈ I ∩ J . Alors φ et ψ
coïncident sur I ∩ J.

Démonstration I ∩ J est un intervalle. En particulier I ∩ J est connexe.
Soit X = \textbackslash{}\{t ∈ I ∩ J\textbackslash{}mathrel\{∣\}φ(t) =
ψ(t)\textbackslash{}\}. Comme φ et ψ sont continues sur I ∩ J, X = \{(φ
− ψ)\}\^{}\{−1\}(\textbackslash{}\{0\textbackslash{}\}) est un fermé de
I ∩ J (image réciproque d'un fermé par une application continue). Soit
alors \{t\}\_\{1\} ∈ I ∩ J et \{y\}\_\{1\} = φ(\{t\}\_\{1\}) =
ψ(\{t\}\_\{1\}). On a donc d'après un théorème précédent
\textbackslash{}mathop\{∀\}t ∈ I ∩ J, φ(t) = \{y\}\_\{1\}
+\{\textbackslash{}mathop\{∫ \} \}\_\{\{t\}\_\{1\}\}\^{}\{t\}F(u,φ(u))
du et ψ(t) = \{y\}\_\{1\} +\{\textbackslash{}mathop\{∫ \}
\}\_\{\{t\}\_\{1\}\}\^{}\{t\}F(u,ψ(u)) du. On en déduit que φ(t) − ψ(t)
=\{\textbackslash{}mathop\{∫ \} \}\_\{\{t\}\_\{1\}\}\^{}\{t\}(F(u,φ(u))
− F(u,ψ(u))) du, et donc, pour t ≥ \{t\}\_\{1\},

\textbackslash{}\textbar{}φ(t) − ψ(t)\textbackslash{}\textbar{}
≤\{\textbackslash{}mathop\{∫ \}
\}\_\{\{t\}\_\{1\}\}\^{}\{t\}\textbackslash{}\textbar{}F(u,φ(u)) −
F(u,ψ(u))\textbackslash{}\textbar{} du

Puisque F est localement lipschitzienne, il existe η \textgreater{} 0 et
r \textgreater{} 0 et une constante L ≥ 0 telle que

\textbackslash{}begin\{eqnarray*\} \textbackslash{}mathop\{∀\}t ∈
{[}\{t\}\_\{1\} − η,\{t\}\_\{1\} + η{]},
\textbackslash{}mathop\{∀\}\{z\}\_\{1\},\{z\}\_\{2\} ∈
B'(\{y\}\_\{1\},r),\& \& \%\& \textbackslash{}\textbackslash{}
\textbackslash{}\textbar{}F(t,\{z\}\_\{1\}) −
F(t,\{z\}\_\{2\})\textbackslash{}\textbar{} ≤
L\textbackslash{}\textbar{}\{z\}\_\{1\} −
\{z\}\_\{2\}\textbackslash{}\textbar{}\& \& \%\&
\textbackslash{}\textbackslash{} \textbackslash{}end\{eqnarray*\}

Comme φ et ψ sont continues au point \{t\}\_\{1\}, il existe η'
\textgreater{} 0 tel que \textbar{}t − \{t\}\_\{1\}\textbar{}≤ η' ⇒
φ(t),ψ(t) ∈ B(\{y\}\_\{1\},r). Pour \textbar{}t − \{t\}\_\{1\}\textbar{}
\textless{} α =\textbackslash{}mathop\{ min\}(η,η'), on a donc
\textbackslash{}\textbar{}F(t,φ(t)) −
F(t,ψ(t))\textbackslash{}\textbar{} ≤ L\textbackslash{}\textbar{}φ(t) −
ψ(t)\textbackslash{}\textbar{}, et donc, en appliquant l'inégalité ci
dessus \textbackslash{}\textbar{}φ(t) − ψ(t)\textbackslash{}\textbar{}
≤\{\textbackslash{}mathop\{∫ \}
\}\_\{\{t\}\_\{1\}\}\^{}\{t\}L\textbackslash{}\textbar{}φ(u) −
ψ(u)\textbackslash{}\textbar{} du. On peut donc appliquer à la fonction
\textbackslash{}\textbar{}φ − ψ\textbackslash{}\textbar{} le lemme de
Gronwall avec c = 0, g(t) = L qui montre que \textbackslash{}\textbar{}φ
− ψ\textbackslash{}\textbar{} est nulle sur {[}\{t\}\_\{1\},\{t\}\_\{1\}
+ α{[}∩I ∩ J. On montre de manière similaire que
\textbackslash{}\textbar{}φ − ψ\textbackslash{}\textbar{} est nulle sur
{]}\{t\}\_\{1\} − α,\{t\}\_\{1\}{]}\textbackslash{}, ∩ I ∩ J. Donc, si X
contient \{t\}\_\{1\}, il contient {]}\{t\}\_\{1\} − α,\{t\}\_\{1\} +
α{[}\textbackslash{}, ∩ I ∩ J ce qui montre que X est un ouvert de I ∩
J. En résumé X est une partie ouverte et fermée, non vide (car
\{t\}\_\{0\} ∈ X) de I ∩ J qui est connexe. Par définition de la
connexité, on a X = I ∩ J, et donc φ et ψ coïncident sur I ∩ J.

Théorème~16.5.3 (Cauchy Lipschitz, existence locale). Soit E un espace
vectoriel normé de dimension finie, U un ouvert de ℝ × E et F : U → E,
(t,y)\textbackslash{}mathrel\{↦\}F(t,y) continue et localement
lipschitzienne par rapport à la variable y. Alors F vérifie la condition
d'existence au problème de Cauchy-Lipschitz. Soit
(\{t\}\_\{0\},\{y\}\_\{0\}) ∈ U, α \textgreater{} 0 et r \textgreater{}
0 tels que C = {[}\{t\}\_\{0\} − α,\{t\}\_\{0\} + α{]} ×
B'(\{y\}\_\{0\},r) ⊂ U soit un cylindre de sécurité pour F associé à la
constante L, M =\{\textbackslash{}mathop\{
sup\}\}\_\{(t,y)∈C\}\textbackslash{}\textbar{}F(t,y)\textbackslash{}\textbar{},
η =\textbackslash{}mathop\{ min\}(α,\{ r \textbackslash{}over M\} )~;
alors il existe une solution φ de l'équation différentielle y' = F(t,y)
définie sur {]}\{t\}\_\{0\} − η,\{t\}\_\{0\} + η{[} et vérifiant
φ(\{t\}\_\{0\}) = \{y\}\_\{0\}.

Démonstration Remarquons tout d'abord que l'existence de M résulte de la
compacité du cylindre de sécurité C et de la continuité de F. Nous
savons d'autre part que φ est une solution de l'équation différentielle
y' = F(t,y) vérifiant φ(\{t\}\_\{0\}) = \{y\}\_\{0\} si et seulement si
φ est une fonction continue vérifiant φ(t) = \{y\}\_\{0\}
+\{\textbackslash{}mathop\{∫ \} \}\_\{\{t\}\_\{0\}\}\^{}\{t\}F(u,φ(u))
du, autrement dit si φ est point fixe de l'application
ψ\textbackslash{}mathrel\{↦\}Γ(ψ), où l'on définit Γ(ψ)(t) =
\{y\}\_\{0\} +\{\textbackslash{}mathop\{∫ \}
\}\_\{\{t\}\_\{0\}\}\^{}\{t\}F(u,ψ(u)) du. Nous inspirant de la
démonstration du théorème du point fixe, nous allons rechercher φ par
une méthode d'approximation successive. Posons donc, pour t
∈{]}\{t\}\_\{0\} − η,\{t\}\_\{0\} + η{[}, \{φ\}\_\{0\}(t) =
\{y\}\_\{0\}~; supposons maintenant que \{φ\}\_\{n\} est définie de
telle sorte que \textbackslash{}mathop\{∀\}t ∈{]}\{t\}\_\{0\} −
η,\{t\}\_\{0\} + η{[}, \{φ\}\_\{n\}(t) ∈ B'(\{y\}\_\{0\},r) et posons
alors \{φ\}\_\{n+1\}(t) = \{y\}\_\{0\} +\{\textbackslash{}mathop\{∫ \}
\}\_\{\{t\}\_\{0\}\}\^{}\{t\}F(u,\{φ\}\_\{n\}(u)) du (ceci a bien un
sens car \textbackslash{}mathop\{∀\}u ∈ {[}\{t\}\_\{0\},t{]},
(u,\{φ\}\_\{n\}(u)) ∈ C ⊂ U). On a alors

\textbackslash{}begin\{eqnarray*\}
\textbackslash{}\textbar{}\{φ\}\_\{n+1\}(t) −
\{y\}\_\{0\}\textbackslash{}\textbar{}\& =\&
\textbackslash{}\textbar{}\{\textbackslash{}mathop\{∫ \}
\}\_\{\{t\}\_\{0\}\}\^{}\{t\}F(u,\{φ\}\_\{ n\}(u))
du\textbackslash{}\textbar{} \%\& \textbackslash{}\textbackslash{} \&
≤\& \textbar{}t −
\{t\}\_\{0\}\textbar{}\{\textbackslash{}mathop\{sup\}\}\_\{(u,y)∈C\}\textbackslash{}\textbar{}F(u,y)\textbackslash{}\textbar{}
≤ ηM ≤ r\%\& \textbackslash{}\textbackslash{}
\textbackslash{}end\{eqnarray*\}

Ceci montre qu'en posant \{φ\}\_\{0\}(t) = \{y\}\_\{0\} et pour n ≥ 0,
\{φ\}\_\{n+1\}(t) = \{y\}\_\{0\} +\{\textbackslash{}mathop\{∫ \}
\}\_\{\{t\}\_\{0\}\}\^{}\{t\}F(u,\{φ\}\_\{n\}(u)) du, on définit bien
une suite d'applications continues de {]}\{t\}\_\{0\} − η,\{t\}\_\{0\} +
η{[} dans B'(\{y\}\_\{0\},r).

Montrons maintenant par récurrence sur n que

\textbackslash{}begin\{eqnarray*\} \textbackslash{}mathop\{∀\}n ∈ ℕ,
\textbackslash{}mathop\{∀\}t ∈{]}\{t\}\_\{0\} − η,\{t\}\_\{0\} + η{[},
\textbackslash{}\textbar{}\{φ\}\_\{n+1\}(t) −
\{φ\}\_\{n\}(t)\textbackslash{}\textbar{} ≤\{ M \textbackslash{}over L\}
\{ \{L\}\^{}\{n+1\}\textbar{}t − \{t\}\_\{0\}\{\textbar{}\}\^{}\{n+1\}
\textbackslash{}over (n + 1)!\} \& \& \%\&
\textbackslash{}\textbackslash{} \textbackslash{}end\{eqnarray*\}

Pour n = 1, on a

\textbackslash{}begin\{eqnarray*\}
\textbackslash{}\textbar{}\{φ\}\_\{1\}(t) −
\{φ\}\_\{0\}(t)\textbackslash{}\textbar{} =\textbackslash{}\textbar{}
\{φ\}\_\{1\}(t) − \{y\}\_\{0\}\textbackslash{}\textbar{}
=\textbackslash{}\textbar{}\{\textbackslash{}mathop\{∫ \}
\}\_\{\{t\}\_\{0\}\}\^{}\{t\}F(u,\{y\}\_\{ 0\})
du\textbackslash{}\textbar{} ≤ M\textbar{}t − \{t\}\_\{0\}\textbar{}\&
\& \%\& \textbackslash{}\textbackslash{}
\textbackslash{}end\{eqnarray*\}

ce qui est bien la formule voulue. Si maintenant l'inégalité est
vérifiée pour n − 1, on a alors pour t ∈ {[}\{t\}\_\{0\},\{t\}\_\{0\} +
η{[}

\textbackslash{}begin\{eqnarray*\}
\textbackslash{}\textbar{}\{φ\}\_\{n+1\}(t) −
\{φ\}\_\{n\}(t)\textbackslash{}\textbar{}\& =\&
\textbackslash{}\textbar{}\{\textbackslash{}mathop\{∫ \}
\}\_\{\{t\}\_\{0\}\}\^{}\{t\}\textbackslash{}left (F(u,\{φ\}\_\{ n\}(u))
− F(u,\{φ\}\_\{n−1\}(u))\textbackslash{}right )
du\textbackslash{}\textbar{}\%\& \textbackslash{}\textbackslash{} \& ≤\&
\{\textbackslash{}mathop\{∫ \}
\}\_\{\{t\}\_\{0\}\}\^{}\{t\}\textbackslash{}\textbar{}F(u,\{φ\}\_\{
n\}(u)) − F(u,\{φ\}\_\{n−1\}(u))\textbackslash{}\textbar{} du \%\&
\textbackslash{}\textbackslash{} \& ≤\& \{\textbackslash{}mathop\{∫ \}
\}\_\{\{t\}\_\{0\}\}\^{}\{t\}L\textbackslash{}\textbar{}\{φ\}\_\{ n\}(u)
− \{φ\}\_\{n−1\}(u)\textbackslash{}\textbar{} du \%\&
\textbackslash{}\textbackslash{} \& ≤\& L\{\textbackslash{}mathop\{∫ \}
\}\_\{\{t\}\_\{0\}\}\^{}\{t\}\{ M \textbackslash{}over L\} \{
\{L\}\^{}\{n\}\textbar{}u − \{t\}\_\{0\}\{\textbar{}\}\^{}\{n\}
\textbackslash{}over n!\} du \%\& \textbackslash{}\textbackslash{} \&
=\&\{ M \textbackslash{}over L\} \{ \{L\}\^{}\{n+1\}\textbar{}t −
\{t\}\_\{0\}\{\textbar{}\}\^{}\{n+1\} \textbackslash{}over (n + 1)!\}
\%\& \textbackslash{}\textbackslash{} \textbackslash{}end\{eqnarray*\}

Un calcul similaire conduit à la même inégalité pour t ∈{]}\{t\}\_\{0\}
− η,\{t\}\_\{0\}{]}.

On en déduit donc que

\textbackslash{}mathop\{∀\}n ∈ ℕ, \textbackslash{}mathop\{∀\}t
∈{]}\{t\}\_\{0\} − η,\{t\}\_\{0\} + η{[},
\textbackslash{}\textbar{}\{φ\}\_\{n+1\}(t) −
\{φ\}\_\{n\}(t)\textbackslash{}\textbar{} ≤\{ M \textbackslash{}over L\}
\{ \{L\}\^{}\{n+1\}\{η\}\^{}\{n+1\} \textbackslash{}over (n + 1)!\}

Alors pour q \textgreater{} p, on a donc

\textbackslash{}begin\{eqnarray*\} \textbackslash{}mathop\{∀\}t
∈{]}\{t\}\_\{0\} − η,\{t\}\_\{0\} + η{[}\&\& \%\&
\textbackslash{}\textbackslash{}
\textbackslash{}\textbar{}\{φ\}\_\{q\}(t) −
\{φ\}\_\{p\}(t)\textbackslash{}\textbar{}\& ≤\&
\{\textbackslash{}mathop\{∑
\}\}\_\{n=p\}\^{}\{q−1\}\textbackslash{}\textbar{}\{φ\}\_\{ n+1\}(t) −
\{φ\}\_\{n\}(t)\textbackslash{}\textbar{}\%\&
\textbackslash{}\textbackslash{} \& ≤\&\{ M \textbackslash{}over L\}
\{\textbackslash{}mathop\{∑ \}\}\_\{n=p\}\^{}\{q−1\}\{
\{L\}\^{}\{n+1\}\{η\}\^{}\{n+1\} \textbackslash{}over (n + 1)!\} \%\&
\textbackslash{}\textbackslash{} \& ≤\&\{ M \textbackslash{}over L\}
\{\textbackslash{}mathop\{∑ \}\}\_\{n=p\}\^{}\{+∞\}\{
\{L\}\^{}\{n+1\}\{η\}\^{}\{n+1\} \textbackslash{}over (n + 1)!\} \%\&
\textbackslash{}\textbackslash{} \textbackslash{}end\{eqnarray*\}

Comme la série \{\textbackslash{}mathop\{\textbackslash{}mathop\{∑ \}\}
\}\_\{n\}\{ \{L\}\^{}\{n+1\}\{η\}\^{}\{n+1\} \textbackslash{}over
(n+1)!\} est une série convergente (exponentielle d'un nombre réel), son
reste tend vers 0~; étant donné ε \textgreater{} 0, il existe N ∈ ℕ tel
que p ≥ N ⇒\{ M \textbackslash{}over L\} \{\textbackslash{}mathop\{
\textbackslash{}mathop\{∑ \}\} \}\_\{n=p\}\^{}\{+∞\}\{
\{L\}\^{}\{n+1\}\{η\}\^{}\{n+1\} \textbackslash{}over (n+1)!\}
\textless{} ε. Alors

q \textgreater{} p ≥ N ⇒\textbackslash{}mathop\{∀\}t ∈{]}\{t\}\_\{0\} −
η,\{t\}\_\{0\} + η{[}, \textbackslash{}\textbar{}\{φ\}\_\{q\}(t) −
\{φ\}\_\{p\}(t)\textbackslash{}\textbar{} \textless{} ε

La suite (\{φ\}\_\{n\}) vérifie donc la critère de Cauchy uniforme. En
conséquence, elle converge uniformément vers une fonction φ
:{]}\{t\}\_\{0\} − η,\{t\}\_\{0\} + η{[}→ B'(\{y\}\_\{0\},r) qui est
elle même continue.

L'inégalité \textbackslash{}\textbar{}F(u,φ(u)) −
F(u,\{φ\}\_\{n\}(u))\textbackslash{}\textbar{} ≤
L\textbackslash{}\textbar{}φ(u) −
\{φ\}\_\{n\}(u)\textbackslash{}\textbar{}, montre que la suite
F(u,\{φ\}\_\{n\}(u)) converge uniformément vers F(u,φ(u))~; ceci nous
permet de passer à la limite sous le signe d'intégration et d'obtenir

\textbackslash{}begin\{eqnarray*\}\{ y\}\_\{0\}
+\{\textbackslash{}mathop\{∫ \} \}\_\{\{t\}\_\{0\}\}\^{}\{t\}F(u,φ(u))
du\&\& \%\& \textbackslash{}\textbackslash{} \& =\& \{y\}\_\{0\}
+\{\textbackslash{}mathop\{ lim\}\}\_\{n→+∞\}\{\textbackslash{}mathop\{∫
\} \}\_\{\{t\}\_\{0\}\}\^{}\{t\}F(u,\{φ\}\_\{ n\}(u)) du\%\&
\textbackslash{}\textbackslash{} \& =\&
\{\textbackslash{}mathop\{lim\}\}\_\{n→+∞\}\{φ\}\_\{n+1\}(t) = φ(t) \%\&
\textbackslash{}\textbackslash{} \textbackslash{}end\{eqnarray*\}

Comme φ est continue, ceci montre que φ est la solution cherchée de
l'équation y' = F(t,y) vérifiant φ(\{t\}\_\{0\}) = \{y\}\_\{0\}.

Théorème~16.5.4 (Cauchy Lipschitz~: existence et unicité d'une solution
maximale). Soit E un espace vectoriel normé de dimension finie, U un
ouvert de ℝ × E et F : U → E, (t,y)\textbackslash{}mathrel\{↦\}F(t,y)
continue et localement lipschitzienne par rapport à la variable y. Soit
(\{t\}\_\{0\},\{y\}\_\{0\}) ∈ U~; alors il existe une unique solution
maximale (\{I\}\_\{0\},\{φ\}\_\{0\}) de l'équation différentielle y' =
F(t,y) qui vérifie \{φ\}\_\{0\}(\{t\}\_\{0\}) = \{y\}\_\{0\}.
L'intervalle \{I\}\_\{0\} est ouvert. Pour toute solution (J,ψ) de
l'équation différentielle vérifiant ψ(\{t\}\_\{0\}) = \{y\}\_\{0\}, on
a~:

\textbackslash{}text\{\$J ⊂ \{I\}\_\{0\}\$ et \$ψ\$ est la restriction
de \$\{φ\}\_\{0\}\$ à \$J\$.\}

Démonstration La fonction F vérifie les conditions d'existence et
d'unicité au problème de Cauchy-Lipschitz et on sait que ces conditions
impliquent l'existence et l'unicité d'une solution maximale vérifiant
une condition initiale donnée. On sait d'autre part que cette solution
maximale est définie sur un intervalle ouvert et que c'est un plus grand
élément de l'ensemble des solutions vérifiant la condition initiale.

Remarque~16.5.1 L'intervalle \{I\}\_\{0\} de définition d'une solution
maximale est difficilement contrôlable a priori. Il dépend bien entendu
de l'équation différentielle, mais aussi de la condition initiale
imposée, quelle que soit la régularité de la fonction F. Considérons par
exemple l'équation différentielle scalaire (c'est-à-dire que les
solutions sont à valeurs réelles) du premier ordre y' = −\{y\}\^{}\{2\}.
La fonction F(t,y) = −\{y\}\^{}\{2\} est bien entendu de classe
\{C\}\^{}\{1\} donc localement lipschitzienne si bien que les résultats
précédents s'appliquent. Comme la fonction nulle définie sur ℝ est
solution, toute solution (I,φ) qui s'annule en un point doit coïncider
sur I ∩ ℝ = I avec la fonction nulle, donc être la fonction nulle. On en
déduit qu'une solution (I,φ) non identiquement nulle ne doit pas
s'annuler et doit donc vérifier \{ φ'(t) \textbackslash{}over
φ\{(t)\}\^{}\{2\}\} = −1 soit encore \{ d \textbackslash{}over dt\}
\textbackslash{}left (\{ 1 \textbackslash{}over φ(t)\}
\textbackslash{}right ) = 1, ou encore \{ 1 \textbackslash{}over φ(t)\}
= t + λ. On en déduit que toute solution est du type
(I,t\textbackslash{}mathrel\{↦\}\{ 1 \textbackslash{}over t+λ\} ).
Cherchons une solution vérifiant la condition initiale φ(\{t\}\_\{0\}) =
\{y\}\_\{0\} avec \{y\}\_\{0\}\textbackslash{}mathrel\{≠\}0 (puisque
seule la solution nulle peut s'annuler). On obtient \{y\}\_\{0\} =\{ 1
\textbackslash{}over \{t\}\_\{0\}+λ\} soit encore λ =\{ 1
\textbackslash{}over \{y\}\_\{0\}\} − \{t\}\_\{0\} si bien que toute
solution vérifiant la condition initiale en question est de la forme
(I,t\textbackslash{}mathrel\{↦\}\{ 1 \textbackslash{}over
t−\{t\}\_\{0\}+\{ 1 \textbackslash{}over \{y\}\_\{0\}\} \} ), la
continuité de la fonction imposant que \{t\}\_\{0\} −\{ 1
\textbackslash{}over \{y\}\_\{0\}\} \textbackslash{}mathrel\{∉\}I. A
partir de là, il est immédiat de déterminer les solutions maximales
vérifiant la condition initiale y(\{t\}\_\{0\}) = \{y\}\_\{0\}~: il
suffit de déterminer un intervalle maximal contenant \{t\}\_\{0\} et ne
contenant pas \{t\}\_\{0\} −\{ 1 \textbackslash{}over \{y\}\_\{0\}\} .
(i) si \{y\}\_\{0\} = 0, la solution maximale est ({]}
−∞,+∞{[},t\textbackslash{}mathrel\{↦\}0) (ii) si \{y\}\_\{0\}
\textgreater{} 0, la solution maximale est ({]}\{t\}\_\{0\} −\{ 1
\textbackslash{}over \{y\}\_\{0\}\}
,+∞{[},t\textbackslash{}mathrel\{↦\}\{ 1 \textbackslash{}over
t−\{t\}\_\{0\}+\{ 1 \textbackslash{}over \{y\}\_\{0\}\} \} ) (iii) si
\{y\}\_\{0\} \textless{} 0, la solution maximale est ({]}
−∞,\{t\}\_\{0\} −\{ 1 \textbackslash{}over \{y\}\_\{0\}\}
{[},t\textbackslash{}mathrel\{↦\}\{ 1 \textbackslash{}over
t−\{t\}\_\{0\}+\{ 1 \textbackslash{}over \{y\}\_\{0\}\} \} )

Bien que la fonction F(t,y) = −\{y\}\^{}\{2\} soit parfaitement
régulière sur \{ℝ\}\^{}\{2\} tout entier, les solutions maximales
présentent des asymptotes verticales et ne sont pas (sauf la solution
nulle), définies sur ℝ tout entier.

Remarque~16.5.2 La condition imposée à F d'être localement
lipschitzienne (par exemple de classe \{C\}\^{}\{1\}) est essentielle.
Si on considère par exemple l'équation différentielle y' =
2\textbackslash{}sqrt\{\textbar{}y\textbar{}\}, le lecteur vérifiera
sans difficulté que la fonction nulle et la fonction
t\textbackslash{}mathrel\{↦\}t\textbar{}t\textbar{} sont toutes deux des
solutions maximales définies sur ℝ et vérifiant la même condition
initiale y(0) = 0 si bien que l'unicité d'une solution maximale à
condition initiale donnée n'est pas vérifiée.

\paragraph{16.5.2 Application aux équations d'ordre n}

Par la technique de réduction à l'ordre 1, on aboutit aux résultats
suivants

Théorème~16.5.5 (Cauchy Lipschitz, unicité). Soit E un espace vectoriel
normé de dimension finie, U un ouvert de ℝ × \{E\}\^{}\{n\} et f : U →
E,
(t,\{y\}\_\{0\},\textbackslash{}mathop\{\textbackslash{}mathop\{\ldots{}\}\},\{y\}\_\{n−1\})\textbackslash{}mathrel\{↦\}f(t,\{y\}\_\{0\},\textbackslash{}mathop\{\textbackslash{}mathop\{\ldots{}\}\},\{y\}\_\{n−1\})
de classe \{C\}\^{}\{1\}. Alors f vérifie la condition d'unicité au
problème de Cauchy-Lipschitz~: soit (I,φ) et (J,ψ) deux solutions de
l'équation différentielle \{y\}\^{}\{(n)\} =
f(t,y,y',\textbackslash{}mathop\{\textbackslash{}mathop\{\ldots{}\}\},\{y\}\^{}\{(n−1)\})
qui vérifient φ(\{t\}\_\{0\}) =
ψ(\{t\}\_\{0\}),\textbackslash{}mathop\{\textbackslash{}mathop\{\ldots{}\}\},\{φ\}\^{}\{(n−1)\}(\{t\}\_\{0\})
= \{ψ\}\^{}\{(n−1)\}(\{t\}\_\{0\}). Alors φ et ψ coïncident sur I ∩ J.

Théorème~16.5.6 (Cauchy Lipschitz, existence locale). Soit E un espace
vectoriel normé de dimension finie, U un ouvert de ℝ × \{E\}\^{}\{n\} et
f : U → E de classe \{C\}\^{}\{1\}. Alors f vérifie la condition
d'existence au problème de Cauchy-Lipschitz. Soit
(\{t\}\_\{0\},\{y\}\_\{0\},\textbackslash{}mathop\{\textbackslash{}mathop\{\ldots{}\}\},\{y\}\_\{n−1\})
∈ U, alors il existe η \textgreater{} 0 et une solution φ de l'équation
différentielle \{y\}\^{}\{(n)\} =
f(t,y,y',\textbackslash{}mathop\{\textbackslash{}mathop\{\ldots{}\}\},\{y\}\^{}\{(n−1)\})
définie sur {]}\{t\}\_\{0\} − η,\{t\}\_\{0\} + η{[} et vérifiant
φ(\{t\}\_\{0\}) =
\{y\}\_\{0\},\textbackslash{}mathop\{\textbackslash{}mathop\{\ldots{}\}\},\{φ\}\^{}\{(n−1)\}(\{t\}\_\{0\})
= \{y\}\_\{n−1\}.

Théorème~16.5.7 (Cauchy Lipschitz, existence et unicité d'une solution
maximale). Soit E un espace vectoriel normé de dimension finie, U un
ouvert de ℝ × \{E\}\^{}\{n\} et f : U → E de classe \{C\}\^{}\{1\}. Soit
(\{t\}\_\{0\},\{y\}\_\{0\},\textbackslash{}mathop\{\textbackslash{}mathop\{\ldots{}\}\},\{y\}\_\{n−1\})
∈ U~; alors il existe une unique solution maximale
(\{I\}\_\{0\},\{φ\}\_\{0\}) de l'équation différentielle
\{y\}\^{}\{(n)\} =
f(t,y,y',\textbackslash{}mathop\{\textbackslash{}mathop\{\ldots{}\}\},\{y\}\^{}\{(n−1)\})
qui vérifie φ(\{t\}\_\{0\}) =
\{y\}\_\{0\},\textbackslash{}mathop\{\textbackslash{}mathop\{\ldots{}\}\},\{φ\}\^{}\{(n−1)\}(\{t\}\_\{0\})
= \{y\}\_\{n−1\}. L'intervalle \{I\}\_\{0\} est ouvert. Pour toute
solution (J,ψ) de l'équation différentielle vérifiant ψ(\{t\}\_\{0\}) =
\{y\}\_\{0\},\textbackslash{}mathop\{\textbackslash{}mathop\{\ldots{}\}\},\{ψ\}\^{}\{(n−1)\}(\{t\}\_\{0\})
= \{y\}\_\{n−1\}, on a~:

\textbackslash{}text\{\$J ⊂ \{I\}\_\{0\}\$ et \$ψ\$ est la restriction
de \$\{φ\}\_\{0\}\$ à \$J\$.\}

\paragraph{16.5.3 Systèmes différentiels autonomes d'ordre 1}

Définition~16.5.2 Soit E un espace vectoriel normé, U un ouvert de
\{E\}\^{}\{n\} et F : U → E. On dit que l'équation différentielle
d'ordre n, \{y\}\^{}\{(n)\} =
F(y,y',\textbackslash{}mathop\{\textbackslash{}mathop\{\ldots{}\}\},\{y\}\^{}\{(n−1)\})
(indépendante du temps t) est une équation différentielle autonome.

Proposition~16.5.8 (invariance par translation de l'ensemble des
solutions). Soit (I,φ) une solution de l'équation différentielle
autonome \{y\}\^{}\{(n)\} =
F(y,y',\textbackslash{}mathop\{\textbackslash{}mathop\{\ldots{}\}\},\{y\}\^{}\{(n−1)\}),
et soit T ∈ ℝ. Alors le couple (\{I\}\_\{T\},\{φ\}\_\{T\}), où
\{φ\}\_\{T\}(t) = φ(t + T) et \{I\}\_\{T\} est le translaté de I par le
nombre réel − T, est encore une solution de l'équation.

Démonstration En effet, pour t ∈ \{I\}\_\{T\}, on a t + T ∈ I et donc

\{φ\}\^{}\{(n)\}(t + T) = F(φ(t + T),φ'(t +
T),\textbackslash{}mathop\{\textbackslash{}mathop\{\ldots{}\}\},\{φ\}\^{}\{(n−1)\}(t
+ T))

c'est-à-dire \{φ\}\_\{T\}\^{}\{(n)\}(t) =
F(\{φ\}\_\{T\}(t),\{φ\}\_\{T\}'(t),\textbackslash{}mathop\{\textbackslash{}mathop\{\ldots{}\}\},\{φ\}\_\{T\}\^{}\{(n−1)\}(t))

Par la suite, nous nous intéresserons tout particulièrement au cas d'une
équation autonome d'ordre 1 à valeurs dans \{ℝ\}\^{}\{2\}. Dans ce cas,
en introduisant les deux composantes \{y\}\_\{1\} et \{y\}\_\{2\} de la
fonction inconnue y et les deux composantes f et g de la fonction F, on
obtient un système différentiel autonome d'ordre 1

\textbackslash{}left \textbackslash{}\{ \textbackslash{}cases\{
\{y\}\_\{1\}' = f(\{y\}\_\{1\},\{y\}\_\{2\})\& \textbackslash{}cr
\{y\}\_\{2\}' = g(\{y\}\_\{1\},\{y\}\_\{2\})\&\\ \}
\textbackslash{}right .

que l'on pourra encore écrire après un changement de notation

\textbackslash{}left \textbackslash{}\{ \textbackslash{}cases\{ x' =
f(x,y)\& \textbackslash{}cr y' = g(x,y)\&\\ \} \textbackslash{}right .

Le théorème de Cauchy-Lipschitz pour un tel système peut encore s'écrire
sous la forme

Théorème~16.5.9 Soit U un ouvert de \{ℝ\}\^{}\{2\}, f et g deux
applications de classe \{C\}\^{}\{1\} de U dans ℝ et S le système
différentiel autonome \textbackslash{}left \textbackslash{}\{
\textbackslash{}cases\{ x' = f(x,y)\& \textbackslash{}cr y' = g(x,y)\&\\
\} \textbackslash{}right .. Alors (i) (unicité) si deux solutions
(I,(\{φ\}\_\{1\},\{ψ\}\_\{1\})) et (J,(\{φ\}\_\{2\},\{ψ\}\_\{2\}))
coïncident en point \{t\}\_\{0\} ∈ I ∩ J, elles coïncident sur I ∩ J
(ii) (existence locale) pour tout \{t\}\_\{0\} ∈ ℝ et tout couple
(\{x\}\_\{0\},\{y\}\_\{0\}) ∈ U, il existe η \textgreater{} 0 et une
solution ({]}\{t\}\_\{0\} − η,\{t\}\_\{0\} + η{[},(φ,ψ)) du système
différentiel (S) vérifiant φ(\{t\}\_\{0\}) =
\{x\}\_\{0\},ψ(\{t\}\_\{0\}) = \{y\}\_\{0\} (iii) (solutions maximales)
pour tout \{t\}\_\{0\} ∈ ℝ et tout couple (\{x\}\_\{0\},\{y\}\_\{0\}) ∈
U, il existe une unique solution maximale (I,(φ,ψ)) du système
différentiel (S) vérifiant φ(\{t\}\_\{0\}) =
\{x\}\_\{0\},ψ(\{t\}\_\{0\}) = \{y\}\_\{0\}~; I est un intervalle
ouvert.

Comme nous l'avons vu ci dessus, si (I,(φ,ψ)) est une solution du
système différentiel autonome (S) et si T ∈ \{ℝ\}\^{}\{∗\}, alors
(\{I\}\_\{T\},(\{φ\}\_\{T\},\{ψ\}\_\{T\})) est encore une solution du
système différentiel autonome (S). Il est clair que l'une des deux
solutions est maximale si et seulement si l'autre l'est~; dans ce cas,
s'il existe \{t\}\_\{0\} ∈ I ∩ \{I\}\_\{T\} tel que φ(\{t\}\_\{0\} + T)
= φ(\{t\}\_\{0\}) et ψ(\{t\}\_\{0\} + T) = ψ(\{t\}\_\{0\}), on doit
avoir (I,(φ,ψ)) = (\{I\}\_\{T\},(\{φ\}\_\{T\},\{ψ\}\_\{T\})), ce qui
implique que I = ℝ et que φ et ψ sont périodiques de période T. On en
déduit~:

Théorème~16.5.10 Soit U un ouvert de \{ℝ\}\^{}\{2\}, f et g deux
applications de classe \{C\}\^{}\{1\} de U dans ℝ et (I,(φ,ψ)) une
solution maximale du système différentiel autonome \textbackslash{}left
\textbackslash{}\{ \textbackslash{}cases\{ x' = f(x,y)\&
\textbackslash{}cr y' = g(x,y)\& \} \textbackslash{}right .. S'il existe
\{t\}\_\{1\},\{t\}\_\{2\} ∈ I distincts tels que φ(\{t\}\_\{1\}) =
φ(\{t\}\_\{2\}) et ψ(\{t\}\_\{1\}) = ψ(\{t\}\_\{2\}), alors I = ℝ et φ
et ψ sont périodiques de période \{t\}\_\{2\} − \{t\}\_\{1\}.

Parmi les solutions d'un système différentiel autonome, on peut
rechercher les constantes t\textbackslash{}mathrel\{↦\}(a,b)~; il est
clair qu'une telle constante est solution si et seulement si f(a,b) =
g(a,b) = 0. Ceci conduit à la définition~:

Définition~16.5.3 On appelle position d'équilibre d'un système autonome
\textbackslash{}left \textbackslash{}\{ \textbackslash{}cases\{ x' =
f(x,y)\& \textbackslash{}cr y' = g(x,y)\& \} \textbackslash{}right .
tout couple (a,b) tel que f(a,b) = g(a,b) = 0.

Si (a,b) est une telle position d'équilibre,
(ℝ,(t\textbackslash{}mathrel\{↦\}a,t\textbackslash{}mathrel\{↦\}b)) est
clairement une solution maximale. Si f et g sont de classe
\{C\}\^{}\{1\}, on en déduit que toute solution (I,(φ,ψ)) qui passe par
cette position d'équilibre, c'est-à-dire pour laquelle il existe
\{t\}\_\{0\} ∈ I tel que φ(\{t\}\_\{0\}) = a,ψ(\{t\}\_\{0\}) = b, est
constante.

Interprétation géométrique des systèmes autonomes~: soit U un ouvert de
\{ℝ\}\^{}\{2\} et V un champ de vecteurs sur U, c'est-à-dire une
application de V dans \{ℝ\}\^{}\{2\}. Pour (x,y) ∈ U, on peut écrire V
(x,y) = (f(x,y),g(x,y)). Alors les solutions (I,(φ,ψ)) du système
autonome \textbackslash{}left \textbackslash{}\{ \textbackslash{}cases\{
x' = f(x,y)\& \textbackslash{}cr y' = g(x,y)\& \} \textbackslash{}right
.sont exactement les arcs paramétrés (I,Φ) qui admettent en chaque point
m = Φ(t) de l'image de l'arc un vecteur tangent Φ'(t) = V (Φ(t)), valeur
du champ de vecteurs V au point m. Un tel arc paramétré est appelé une
courbe intégrale du champ de vecteurs V .

\paragraph{16.5.4 Equations différentielles et formes différentielles}

Définition~16.5.4 Soit ω = a(t,y)dt + b(t,y)dy une forme différentielle
sur un ouvert U de \{ℝ\}\^{}\{2\}. On appelle solution de l'équation ω =
0 tout couple (I,(\{f\}\_\{1\},\{f\}\_\{2\})) d'un intervalle I de ℝ et
d'une application (\{f\}\_\{1\},\{f\}\_\{2\}) : I → \{ℝ\}\^{}\{2\} de
classe \{C\}\^{}\{1\} tel que (i) \textbackslash{}mathop\{∀\}u ∈ I,
(\{f\}\_\{1\}(u),\{f\}\_\{2\}(u)) ∈ U (ii) \textbackslash{}mathop\{∀\}u
∈ I, a(\{f\}\_\{1\}(u),\{f\}\_\{2\}(u))\{f\}\_\{1\}'(u) +
b(\{f\}\_\{1\}(u),\{f\}\_\{2\}(u))\{f\}\_\{2\}'(u) = 0

Formellement, (\{f\}\_\{1\},\{f\}\_\{2\}) est solution de l'équation ω =
0 si en rempla\textbackslash{}c\{c\}ant t par \{f\}\_\{1\}(u), dt par
\{f\}\_\{1\}'(u) du, y par \{f\}\_\{2\}(u) et dy par \{f\}\_\{2\}'(u)
du, on obtient la forme différentielle nulle. Nous allons à l'aide du
théorème suivant relier les solutions de l'équation a(t,y)dt + b(t,y)dy
= 0 aux solutions de l'équation différentielle a(t,y) + b(t,y)\{ dy
\textbackslash{}over dt\} = 0, obtenue formellement par division par dt
de la forme différentielle.

Théorème~16.5.11 Soit ω = a(t,y)dt + b(t,y)dy une forme différentielle
sur un ouvert U de \{ℝ\}\^{}\{2\}. (i) Pour toute solution (I,f) de
l'équation différentielle a(t,y) + b(t,y)\{ dy \textbackslash{}over dt\}
= 0, le couple (I,t\textbackslash{}mathrel\{↦\}(t,f(t))) est une
solution de l'équation ω = 0. (ii) Inversement si
(J,(\{f\}\_\{1\},\{f\}\_\{2\})) est une solution de l'équation ω = 0
telle que \textbackslash{}mathop\{∀\}u ∈
J,\{f\}\_\{1\}'(u)\textbackslash{}mathrel\{≠\}0, alors \{f\}\_\{1\} est
un difféomorphisme de classe \{C\}\^{}\{1\} de J sur un intervalle I =
\{f\}\_\{1\}(J) de ℝ et le couple (I,\{f\}\_\{2\} ∘
\{f\}\_\{1\}\^{}\{−1\}) est une solution de l'équation différentielle
a(t,y) + b(t,y)\{ dy \textbackslash{}over dt\} = 0.

Démonstration (i) On a en effet \{f\}\_\{1\}(t) = t et \{f\}\_\{2\}(t) =
f(t), si bien que

\textbackslash{}begin\{eqnarray*\}
a(\{f\}\_\{1\}(t),\{f\}\_\{2\}(t))\{f\}\_\{1\}'(t) +
b(\{f\}\_\{1\}(t),\{f\}\_\{2\}(t))\{f\}\_\{2\}'(t)\& \& \%\&
\textbackslash{}\textbackslash{} = a(t,f(t)) + b(t,f(t))f'(t) = 0\& \&
\%\& \textbackslash{}\textbackslash{} \textbackslash{}end\{eqnarray*\}

(ii) Comme \{f\}\_\{1\}' ne s'annule pas sur J et qu'elle est continue,
cette fonction dérivée est de signe constant. Donc \{f\}\_\{1\} est
strictement monotone et c'est donc un homéomorphisme de J sur un
intervalle I = \{f\}\_\{1\}(J). Comme \{f\}\_\{1\}' ne s'annule pas,
l'application \{f\}\_\{1\}\^{}\{−1\} est elle aussi de classe
\{C\}\^{}\{1\} et donc \{f\}\_\{1\} est un difféomorphisme. De plus si f
= \{f\}\_\{2\} ∘ \{f\}\_\{1\}\^{}\{−1\}, on a

f'(t) = \textbackslash{}left
(\{f\}\_\{1\}\^{}\{−1\}\textbackslash{}right )'(t)\{f\}\_\{
2\}'(\{f\}\_\{1\}\^{}\{−1\}(t)) =\{
\{f\}\_\{2\}'(\{f\}\_\{1\}\^{}\{−1\}(t)) \textbackslash{}over
\{f\}\_\{1\}'(\{f\}\_\{1\}\^{}\{−1\}(t))\}

si bien qu'en posant u = \{f\}\_\{1\}\^{}\{−1\}(t) et donc t =
\{f\}\_\{1\}(u), f(t) = \{f\}\_\{2\}(u), on a

\textbackslash{}begin\{eqnarray*\} a(t,f(t)) + b(t,f(t))f'(t)\&\& \%\&
\textbackslash{}\textbackslash{} \& =\&
a(\{f\}\_\{1\}(u),\{f\}\_\{2\}(u)) +
b(\{f\}\_\{1\}(u),\{f\}\_\{2\}(u))\{ \{f\}\_\{2\}'(u)
\textbackslash{}over \{f\}\_\{1\}'(u)\} \%\&
\textbackslash{}\textbackslash{} \& =\&\{
a(\{f\}\_\{1\}(u),\{f\}\_\{2\}(u))\{f\}\_\{1\}'(u) +
b(\{f\}\_\{1\}(u),\{f\}\_\{2\}(u))\{f\}\_\{2\}'(u) \textbackslash{}over
\{f\}\_\{1\}'(u)\} = 0\%\& \textbackslash{}\textbackslash{}
\textbackslash{}end\{eqnarray*\}

donc (I,f) est solution de l'équation différentielle.

Remarque~16.5.3 Le résultat précédent signifie qu'il est équivalent de
résoudre l'équation différentielle ou de rechercher les solutions de
l'équation ω = 0 tels que \{f\}\_\{1\}' ne s'annule pas~; autrement dit
les graphes des solutions de l'équation différentielle (c'est-à-dire les
courbes intégrales de l'équation différentielle) sont paramétrés par les
solutions de l'équation ω = 0. Ceci va nous permettre, plutôt que de
résoudre l'équation différentielle, de résoudre l'équation ω = 0 et de
rechercher, parmi les arcs paramétrés solutions, ceux qui paramètrent
des graphes d'applications de classe \{C\}\^{}\{1\}, c'est-à-dire ceux
tels que \{f\}\_\{1\}' ne s'annule pas.

\paragraph{16.5.5 Equations aux différentielles totales}

Théorème~16.5.12 Soit U un ouvert de \{ℝ\}\^{}\{2\} et F : U → ℝ de
classe \{C\}\^{}\{1\}, ω = dF =\{ ∂F \textbackslash{}over ∂t\} (t,y) dt
+\{ ∂F \textbackslash{}over ∂y\} (t,y) dy. Alors les solutions de
l'équation ω = 0 sont les applications (\{f\}\_\{1\},\{f\}\_\{2\}) de
classe \{C\}\^{}\{1\} d'un intervalle I de ℝ dans U telles que
l'application
u\textbackslash{}mathrel\{↦\}F(\{f\}\_\{1\}(u),\{f\}\_\{2\}(u)) soit
constante.

Démonstration On a en effet

\textbackslash{}begin\{eqnarray*\}\{ d \textbackslash{}over du\}
F(\{f\}\_\{1\}(u),\{f\}\_\{2\}(u))\&\& \%\&
\textbackslash{}\textbackslash{} \& =\&\{ ∂F \textbackslash{}over ∂t\}
(\{f\}\_\{1\}(u),\{f\}\_\{2\}(u)) \{f\}\_\{1\}'(u) +\{ ∂F
\textbackslash{}over ∂y\} (\{f\}\_\{1\}(u),\{f\}\_\{2\}(u))
\{f\}\_\{2\}'(u)\%\& \textbackslash{}\textbackslash{}
\textbackslash{}end\{eqnarray*\}

qui vaut 0 si et seulement si~f = (\{f\}\_\{1\},\{f\}\_\{2\}) est
solution de l'équation ω = 0.

On en déduit que si la forme différentielle a(t,y) dt + b(t,y) dy est la
différentielle d'une fonction F sur U, alors les solutions de l'équation
différentielle a(t,y) + b(t,y)y' = 0 sont les couples (I,f) d'un
intervalle I de ℝ et d'une application f : I → ℝ tel que l'application
t\textbackslash{}mathrel\{↦\}F(t,f(t)) soit constante. Autrement dit,
les solutions de l'équation différentielle sont définies implicitement
par l'équation F(t,y) = k où k est une constante. Une telle équation
sera dite équation aux différentielles totales~; une telle équation est
donc résoluble de manière implicite (ou même graphique~: il suffit de
tracer les courbes de niveau de la fonction F et d'en rechercher les
parties qui sont des graphes d'applications de classe \{C\}\^{}\{1\})~;
dans certains cas, ceci peut conduire à une résolution explicite.

Exemple~16.5.1 Soit à résoudre l'équation différentielle y + (t − y)y' =
0. Sur l'ouvert U défini par t − y\textbackslash{}mathrel\{≠\}0, on a y'
=\{ y \textbackslash{}over y−t\} si bien que la théorie de Cauchy
Lipschitz s'applique. La forme différentielle y dt + (t − y) dy est
exacte puisque \{ ∂y \textbackslash{}over ∂y\} =\{ ∂(t−y)
\textbackslash{}over ∂t\} . On constate sans difficulté qu'à un facteur
2 près, c'est la différentielle de la fonction F(t,y) = 2ty −
\{y\}\^{}\{2\}. Autrement dit, les solutions de l'équation
différentielle sont les couples (I,f) tels que f\{(t)\}\^{}\{2\} −
2tf(t) = k. Cherchons les solutions telles que f(\{t\}\_\{0\}) =
\{y\}\_\{0\}. Alors nécessairement f\{(t)\}\^{}\{2\} − 2tf(t) =
\{y\}\_\{0\}\^{}\{2\} − 2\{t\}\_\{0\}\{y\}\_\{0\}, si bien que nous
pouvons résoudre l'équation du second degré à l'inconnue f(t). Le
discriminant (réduit) est égal à Δ' = \{t\}\^{}\{2\} +
(\{y\}\_\{0\}\^{}\{2\} − 2\{t\}\_\{0\}\{y\}\_\{0\}) et on a donc

f(t) = t ±\textbackslash{}sqrt\{\{t\}\^{}\{2 \} + \{y\}\_\{0 \}\^{}\{2
\} − 2\{t\}\_\{0 \} \{y\}\_\{0\}\}

Cherchons donc les solutions dont le graphe est contenu dans U
c'est-à-dire telles que f(t) − t ne s'annule pas. Alors f(t) − t =
±\textbackslash{}sqrt\{\{t\}\^{}\{2 \} + \{y\}\_\{0 \}\^{}\{2 \} −
2\{t\}\_\{0 \} \{y\}\_\{0\}\} ne s'annule pas, et est donc de signe
constant. On en déduit que le signe ± ne dépend pas de t. De plus, on a
f(\{t\}\_\{0\}) = \{t\}\_\{0\} ±\textbackslash{}sqrt\{\{t\}\_\{0
\}\^{}\{2 \} + \{y\}\_\{0 \}\^{}\{2 \} − 2\{t\}\_\{0 \} \{y\}\_\{0\}\} =
\{t\}\_\{0\} ±\textbar{}\{y\}\_\{0\} − \{t\}\_\{0\}\textbar{} ce qui
montre que ce signe est égal à celui de \{y\}\_\{0\} − \{t\}\_\{0\}, que
nous noterons \{ε\}\_\{0\}. On a donc \textbackslash{}mathop\{∀\}t ∈ I,
f(t) = t + \{ε\}\_\{0\}\textbackslash{}sqrt\{\{t\}\^{}\{2 \} +
\{y\}\_\{0 \}\^{}\{2 \} − 2\{t\}\_\{0 \} \{y\}\_\{0\}\}. Un intervalle
maximal de définition de cette solution dépend évidemment du signe de
l'expression \{y\}\_\{0\}\^{}\{2\} − 2\{t\}\_\{0\}\{y\}\_\{0\} =
\{y\}\_\{0\}(\{y\}\_\{0\} − 2\{t\}\_\{0\}). Si cette expression est
positive, la solution est définie sur ℝ tout entier. Si cette expression
est négative, la solution est définie sur celui des intervalles {]}
−∞,−\textbackslash{}sqrt\{\{t\}\_\{0 \}\^{}\{2 \} + 2\{t\}\_\{0 \}
\{y\}\_\{0\}\}{[} ou {]}\textbackslash{}sqrt\{\{t\}\_\{0 \}\^{}\{2 \} +
2\{t\}\_\{0 \} \{y\}\_\{0\}\},+∞{[} qui contient \{t\}\_\{0\} (ce qui
dépend bien entendu du signe de \{t\}\_\{0\}). La figure ci dessous
représente un aper\textbackslash{}c\{c\}u des courbes intégrales de
l'équation différentielle avec les deux droites séparatrices
\{y\}\_\{0\} = 0 et \{y\}\_\{0\} = 2\{t\}\_\{0\}. On remarquera que par
les points vérifiant \{y\}\_\{0\} = \{t\}\_\{0\} passent deux courbes
intégrales (mais qui en fait ne sont pas dérivables en ce point)~: la
moitié supérieure de la branche d'hyperbole et la moitié inférieure~;
les autres courbes intégrales correspondent à des solutions définies sur
ℝ tout entier.

\includegraphics{cours10x.png}

\paragraph{16.5.6 Equations à variables séparables}

Définition~16.5.5 On appelle équation à variables séparées une équation
du type a(t) = b(y)y', où a : I → ℝ et b : J → ℝ sont continues.

La forme différentielle correspondante est la forme différentielle ω =
a(t) dt − b(y) dy. Si A désigne une primitive de a sur I et B une
primitive de b sur J, on a ω = dF avec F(t,y) = A(t) − B(y). Une
solution (\{I\}\_\{0\},f) vérifiant f(\{t\}\_\{0\}) = \{y\}\_\{0\} doit
donc vérifier A(t) − B(f(t)) = A(\{t\}\_\{0\}) − B(\{y\}\_\{0\}), où
encore en termes d'intégrales \{\textbackslash{}mathop\{∫ \}
\}\_\{\{t\}\_\{0\}\}\^{}\{t\}a(u) du =\{\textbackslash{}mathop\{∫ \}
\}\_\{\{y\}\_\{0\}\}\^{}\{f(t)\}b(v) dv. Si b = B' ne s'annule pas, B
est un difféomorphisme de J sur B(J) et la solution (\{I\}\_\{0\},f) est
donnée par f(t) = \{B\}\^{}\{−1\}\textbackslash{}left (A(t) −
A(\{t\}\_\{0\}) + B(\{y\}\_\{0\})\textbackslash{}right ).

Remarque~16.5.4 On appelle équation à variables séparables une équation
se ramenant par produits et quotients à une équation à variables
séparées. On prendra garde au fait que la séparation des variables est
une opération à risque. La multiplication par un facteur pouvant
s'annuler risque de faire apparaître de fausses solutions de l'équation
différentielle. Quant à la division par des facteurs pouvant s'annuler,
elle risque au contraire de faire disparaître des solutions annulant ces
facteurs.

\paragraph{16.5.7 Equations se ramenant à des équations à variables
séparables}

Certaines équations sont connues comme pouvant se ramener à des
équations à variables séparables, ce sont les équations incomplètes
(c'est-à-dire où l'un des termes t ou y n'apparaît pas) et les équations
homogènes (c'est à dire les équations ne dépendant que de y' et de \{ y
\textbackslash{}over t\} ).

Equations sous formes normales

Nous nous intéresserons tout d'abord aux équations sous forme normale.
Les équations incomplètes sont alors de deux types~: soit y' = F(t) soit
y' = F(y). La première se ramène à un simple calcul de primitive. La
seconde est une équation à variables séparables~: la division par F(y)
sépare les deux variables~; cependant cette division n'est possible que
pour des solutions t\textbackslash{}mathrel\{↦\}y(t) telles que F(y(t))
ne s'annule pas. Cherchons tout d'abord ces solutions~: l'équation
s'écrit alors \{ dy \textbackslash{}over F(y)\} = dt ce qui conduit pour
une solution vérifiant φ(\{t\}\_\{0\}) = \{y\}\_\{0\} à
\{\textbackslash{}mathop\{∫ \} \}\_\{\{y\}\_\{0\}\}\^{}\{φ(t)\}\{ du
\textbackslash{}over F(u)\} = t − \{t\}\_\{0\}~; on obtient donc ainsi t
en fonction de y = φ(t) et il ne reste plus qu'à inverser la fonction
obtenue. Inversement, si F(\{y\}\_\{0\}) = 0, il est clair que la
fonction constante t\textbackslash{}mathrel\{↦\}\{y\}\_\{0\} est
solution. Dans le cas où le théorème de Cauchy-Lipschitz s'applique (par
exemple si F est de classe \{C\}\^{}\{1\}), on est certain d'avoir
obtenu ainsi toutes les solutions.

Remarque~16.5.5 Les équations incomplètes du type y' = F(y) présentent
une certaine invariance par translation~: si φ en est une solution,
toutes les t\textbackslash{}mathrel\{↦\}φ(t + T) en sont également
solutions.

Une équation homogène sous forme normale se présente nécessairement sous
la forme y' = F(\{ y \textbackslash{}over t\} ), t variant soit dans {]}
−∞,0{[}, soit dans {]}0,+∞{[}. Faisons le changement de fonction
inconnue y = tz. On obtient alors tz' + z = F(z) soit encore tz' = F(z)
− z. Il s'agit d'une équation à variables séparables. La séparation des
variables conduisant à l'équation \{ dz \textbackslash{}over F(z)−z\}
=\{ dt \textbackslash{}over t\} nécessite une division par t qui de
toute fa\textbackslash{}c\{c\}on ne s'annule pas sur les intervalles
considérés, mais aussi une division par F(z) − z qui lui peut s'annuler.
Les solutions t\textbackslash{}mathrel\{↦\}z(t) telles que F(z(t)) −
z(t) ne s'annulent pas sont données par \{\textbackslash{}mathop\{∫ \}
\}\_\{\{z\}\_\{0\}\}\^{}\{z\}\{ du \textbackslash{}over F(u)−u\}
=\textbackslash{}mathop\{ log\} \textbackslash{}left \textbar{}\{ t
\textbackslash{}over \{t\}\_\{0\}\} \textbackslash{}right \textbar{},
avec z(t) =\{ y(t) \textbackslash{}over t\} et \{z\}\_\{0\} =\{
\{y\}\_\{0\} \textbackslash{}over \{t\}\_\{0\}\} . Par contre, si
F(\{z\}\_\{0\}) − \{z\}\_\{0\} = 0, il est clair que la fonction
constante t\textbackslash{}mathrel\{↦\}\{z\}\_\{0\} est solution de tz'
= F(z) − z et donc t\textbackslash{}mathrel\{↦\}t\{z\}\_\{0\} est
solution de y' = F(\{ y \textbackslash{}over t\} ). Dans le cas où le
théorème de Cauchy-Lipschitz s'applique (par exemple si F est de classe
\{C\}\^{}\{1\}), on est certain d'avoir obtenu ainsi toutes les
solutions.

Remarque~16.5.6 Les équations homogènes du type y' = F(\{ y
\textbackslash{}over t\} ) présentent une invariance par homothétie~: si
φ en est une solution, toutes les
t\textbackslash{}mathrel\{↦\}λφ\textbackslash{}left (\{ t
\textbackslash{}over λ\} \textbackslash{}right ) en sont également
solutions.

Equations sous formes non normales

Les équations incomplètes se présentent sous la forme F(t,y') = 0 ou
F(y,y') = 0, les équations homogènes sous la forme F(\{ y
\textbackslash{}over t\} ,y') = 0. Dans tous les cas, on supposera que
l'on sait paramétrer la courbe F(X,Y ) = 0 par X = φ(u) et Y = ψ(u). On
cherchera alors à paramétrer les courbes intégrales de l'équation
différentielle en fonction de u ce qui peut se faire simplement en
termes de formes différentielles.

\begin{itemize}
\itemsep1pt\parskip0pt\parsep0pt
\item
  F(t,y') = 0. On écrit t = φ(u), \{ dy \textbackslash{}over dt\} =
  ψ(u)~; ceci conduit à dy = ψ(u) dt = ψ(u)φ'(u) du. Un simple calcul de
  primitive conduit à y = Ψ(u) et les courbes intégrales sont
  paramétrées par t = φ(u), y = Ψ(u).
\item
  F(y,y') = 0. On écrit y = φ(u), \{ dy \textbackslash{}over dt\} =
  ψ(u)~; ceci conduit à dy = φ'(u) du = ψ(u) dt qui est une équation à
  variable séparable entre u et t~; si ψ(u) ne s'annule pas, un calcul
  de primitive conduit à t = Φ(u) et les courbes intégrales sont
  paramétrées par t = Φ(u), y = ψ(u). Si φ(\{u\}\_\{0\}) = 0, on obtient
  également comme solution la fonction constante
  t\textbackslash{}mathrel\{↦\}φ(\{u\}\_\{0\}).
\item
  F(\{ y \textbackslash{}over t\} ,y') = 0. On écrit \{ y
  \textbackslash{}over t\} = φ(u), \{ dy \textbackslash{}over dt\} =
  ψ(u)~; ceci conduit à y = tφ(u), d'où dy = φ(u) dt + tφ'(u) du = ψ(u)
  dt qui est une équation à variable séparable entre u et t que l'on
  peut écrire (ψ(u) − φ(u)) dt = tφ'(u) du. Comme t ne s'annule pas, si
  ψ − φ ne s'annule pas, un calcul de primitive conduit à t = Φ(u) et
  les courbes intégrales sont paramétrées par t = Φ(u), y = tψ(u) =
  Φ(u)ψ(u). Si ψ(\{u\}\_\{0\}) − φ(\{u\}\_\{0\}) = 0, on obtient
  également comme solution la fonction affine y = tφ(\{u\}\_\{0\}).
\end{itemize}

Remarque~16.5.7 Le lecteur vérifiera sans difficulté que l'ensemble des
courbes intégrales d'une équation incomplète est invariant par
translation parallèlement à l'un des axes et que l'ensemble des courbes
intégrales d'une équation homogène est invariant par homothétie de
centre O.

\paragraph{16.5.8 Equation de Riccati}

On considère une équation différentielle du type y' = a(t)\{y\}\^{}\{2\}
+ b(t)y + c(t), où b et c sont des applications continues de I dans ℝ et
a une application de classe \{C\}\^{}\{1\} de I dans ℝ ne s'annulant
pas. Faisons le changement de fonction inconnue z
=\textbackslash{}mathop\{ exp\} (−\textbackslash{}mathop\{∫ \} a(t)y(t)
dt) soit encore a(t)y = −\{ z' \textbackslash{}over z\} où z est une
fonction de classe \{C\}\^{}\{2\} ne s'annulant pas. On obtient alors y'
= −\{ 1 \textbackslash{}over a(t)\} \{ z'`\textbackslash{}over z\} +\{ 1
\textbackslash{}over a(t)\} \{ \{z'\}\^{}\{2\} \textbackslash{}over
\{z\}\^{}\{2\}\} +\{ a'(t) \textbackslash{}over a\{(t)\}\^{}\{2\}\} \{
z' \textbackslash{}over z\} si bien que l'équation devient

\textbackslash{}begin\{eqnarray*\} −\{ 1 \textbackslash{}over a(t)\} \{
z'`\textbackslash{}over z\} +\{ 1 \textbackslash{}over a(t)\} \{
\{z'\}\^{}\{2\} \textbackslash{}over \{z\}\^{}\{2\}\} +\{ a'(t)
\textbackslash{}over a\{(t)\}\^{}\{2\}\} \{ z' \textbackslash{}over z\}
=\{ 1 \textbackslash{}over a(t)\} \{ \{z'\}\^{}\{2\}
\textbackslash{}over \{z\}\^{}\{2\}\} −\{ b(t) \textbackslash{}over
a(t)\} \{ z' \textbackslash{}over z\} + c(t)\& \& \%\&
\textbackslash{}\textbackslash{} \textbackslash{}end\{eqnarray*\}

soit encore après multiplication par a(t)z,

z'`−\textbackslash{}left (\{ a'(t) \textbackslash{}over a(t)\} +
b(t)\textbackslash{}right )z' + a(t)c(t)z = 0

qui est une équation différentielle homogène d'ordre 2.

Inversement, étant donnée une équation homogène d'ordre 2, z'`+ p(t)z' +
q(t)z = 0, p et q étant des fonctions continues de I dans ℝ, si l'on
recherche les solutions ne s'annulant pas sous la forme z =
\{e\}\^{}\{u\}, on a z' = u'\{e\}\^{}\{u\} et z'`= (u'`+
\{u'\}\^{}\{2\})\{e\}\^{}\{u\} et donc, en posant y = u',

\textbackslash{}begin\{eqnarray*\} z'`+ p(t)z' + q(t)z = 0\&
\textbackslash{}mathrel\{⇔\} \& u'`+ \{u'\}\^{}\{2\} + p(t)u' + q(t) =
0\%\& \textbackslash{}\textbackslash{} \& \textbackslash{}mathrel\{⇔\}
\& y' = −\{y\}\^{}\{2\} + p(t)y + q(t) \%\&
\textbackslash{}\textbackslash{} \textbackslash{}end\{eqnarray*\}

qui est une équation de Riccati.

On voit donc en définitive que la résolution des équations de Riccati y'
= a(t)\{y\}\^{}\{2\} + b(t)y + c(t) est équivalente à la recherche des
solutions ne s'annulant pas des équations différentielles homogènes
d'ordre 2.

{[}\href{coursse91.html}{next}{]} {[}\href{coursse89.html}{prev}{]}
{[}\href{coursse89.html\#tailcoursse89.html}{prev-tail}{]}
{[}\href{coursse90.html}{front}{]}
{[}\href{coursch17.html\#coursse90.html}{up}{]}

\end{document}

% \documentclass[]{article}
\usepackage[T1]{fontenc}
\usepackage{lmodern}
\usepackage{amssymb,amsmath}
\usepackage{ifxetex,ifluatex}
\usepackage{fixltx2e} % provides \textsubscript
% use upquote if available, for straight quotes in verbatim environments
\IfFileExists{upquote.sty}{\usepackage{upquote}}{}
\ifnum 0\ifxetex 1\fi\ifluatex 1\fi=0 % if pdftex
  \usepackage[utf8]{inputenc}
\else % if luatex or xelatex
  \ifxetex
    \usepackage{mathspec}
    \usepackage{xltxtra,xunicode}
  \else
    \usepackage{fontspec}
  \fi
  \defaultfontfeatures{Mapping=tex-text,Scale=MatchLowercase}
  \newcommand{\euro}{€}
\fi
% use microtype if available
\IfFileExists{microtype.sty}{\usepackage{microtype}}{}
\ifxetex
  \usepackage[setpagesize=false, % page size defined by xetex
              unicode=false, % unicode breaks when used with xetex
              xetex]{hyperref}
\else
  \usepackage[unicode=true]{hyperref}
\fi
\hypersetup{breaklinks=true,
            bookmarks=true,
            pdfauthor={},
            pdftitle={Analyse numerique des equations differentielles},
            colorlinks=true,
            citecolor=blue,
            urlcolor=blue,
            linkcolor=magenta,
            pdfborder={0 0 0}}
\urlstyle{same}  % don't use monospace font for urls
\setlength{\parindent}{0pt}
\setlength{\parskip}{6pt plus 2pt minus 1pt}
\setlength{\emergencystretch}{3em}  % prevent overfull lines
\setcounter{secnumdepth}{0}
 
/* start css.sty */
.cmr-5{font-size:50%;}
.cmr-7{font-size:70%;}
.cmmi-5{font-size:50%;font-style: italic;}
.cmmi-7{font-size:70%;font-style: italic;}
.cmmi-10{font-style: italic;}
.cmsy-5{font-size:50%;}
.cmsy-7{font-size:70%;}
.cmex-7{font-size:70%;}
.cmex-7x-x-71{font-size:49%;}
.msbm-7{font-size:70%;}
.cmtt-10{font-family: monospace;}
.cmti-10{ font-style: italic;}
.cmbx-10{ font-weight: bold;}
.cmr-17x-x-120{font-size:204%;}
.cmsl-10{font-style: oblique;}
.cmti-7x-x-71{font-size:49%; font-style: italic;}
.cmbxti-10{ font-weight: bold; font-style: italic;}
p.noindent { text-indent: 0em }
td p.noindent { text-indent: 0em; margin-top:0em; }
p.nopar { text-indent: 0em; }
p.indent{ text-indent: 1.5em }
@media print {div.crosslinks {visibility:hidden;}}
a img { border-top: 0; border-left: 0; border-right: 0; }
center { margin-top:1em; margin-bottom:1em; }
td center { margin-top:0em; margin-bottom:0em; }
.Canvas { position:relative; }
li p.indent { text-indent: 0em }
.enumerate1 {list-style-type:decimal;}
.enumerate2 {list-style-type:lower-alpha;}
.enumerate3 {list-style-type:lower-roman;}
.enumerate4 {list-style-type:upper-alpha;}
div.newtheorem { margin-bottom: 2em; margin-top: 2em;}
.obeylines-h,.obeylines-v {white-space: nowrap; }
div.obeylines-v p { margin-top:0; margin-bottom:0; }
.overline{ text-decoration:overline; }
.overline img{ border-top: 1px solid black; }
td.displaylines {text-align:center; white-space:nowrap;}
.centerline {text-align:center;}
.rightline {text-align:right;}
div.verbatim {font-family: monospace; white-space: nowrap; text-align:left; clear:both; }
.fbox {padding-left:3.0pt; padding-right:3.0pt; text-indent:0pt; border:solid black 0.4pt; }
div.fbox {display:table}
div.center div.fbox {text-align:center; clear:both; padding-left:3.0pt; padding-right:3.0pt; text-indent:0pt; border:solid black 0.4pt; }
div.minipage{width:100%;}
div.center, div.center div.center {text-align: center; margin-left:1em; margin-right:1em;}
div.center div {text-align: left;}
div.flushright, div.flushright div.flushright {text-align: right;}
div.flushright div {text-align: left;}
div.flushleft {text-align: left;}
.underline{ text-decoration:underline; }
.underline img{ border-bottom: 1px solid black; margin-bottom:1pt; }
.framebox-c, .framebox-l, .framebox-r { padding-left:3.0pt; padding-right:3.0pt; text-indent:0pt; border:solid black 0.4pt; }
.framebox-c {text-align:center;}
.framebox-l {text-align:left;}
.framebox-r {text-align:right;}
span.thank-mark{ vertical-align: super }
span.footnote-mark sup.textsuperscript, span.footnote-mark a sup.textsuperscript{ font-size:80%; }
div.tabular, div.center div.tabular {text-align: center; margin-top:0.5em; margin-bottom:0.5em; }
table.tabular td p{margin-top:0em;}
table.tabular {margin-left: auto; margin-right: auto;}
div.td00{ margin-left:0pt; margin-right:0pt; }
div.td01{ margin-left:0pt; margin-right:5pt; }
div.td10{ margin-left:5pt; margin-right:0pt; }
div.td11{ margin-left:5pt; margin-right:5pt; }
table[rules] {border-left:solid black 0.4pt; border-right:solid black 0.4pt; }
td.td00{ padding-left:0pt; padding-right:0pt; }
td.td01{ padding-left:0pt; padding-right:5pt; }
td.td10{ padding-left:5pt; padding-right:0pt; }
td.td11{ padding-left:5pt; padding-right:5pt; }
table[rules] {border-left:solid black 0.4pt; border-right:solid black 0.4pt; }
.hline hr, .cline hr{ height : 1px; margin:0px; }
.tabbing-right {text-align:right;}
span.TEX {letter-spacing: -0.125em; }
span.TEX span.E{ position:relative;top:0.5ex;left:-0.0417em;}
a span.TEX span.E {text-decoration: none; }
span.LATEX span.A{ position:relative; top:-0.5ex; left:-0.4em; font-size:85%;}
span.LATEX span.TEX{ position:relative; left: -0.4em; }
div.float img, div.float .caption {text-align:center;}
div.figure img, div.figure .caption {text-align:center;}
.marginpar {width:20%; float:right; text-align:left; margin-left:auto; margin-top:0.5em; font-size:85%; text-decoration:underline;}
.marginpar p{margin-top:0.4em; margin-bottom:0.4em;}
.equation td{text-align:center; vertical-align:middle; }
td.eq-no{ width:5%; }
table.equation { width:100%; } 
div.math-display, div.par-math-display{text-align:center;}
math .texttt { font-family: monospace; }
math .textit { font-style: italic; }
math .textsl { font-style: oblique; }
math .textsf { font-family: sans-serif; }
math .textbf { font-weight: bold; }
.partToc a, .partToc, .likepartToc a, .likepartToc {line-height: 200%; font-weight:bold; font-size:110%;}
.chapterToc a, .chapterToc, .likechapterToc a, .likechapterToc, .appendixToc a, .appendixToc {line-height: 200%; font-weight:bold;}
.index-item, .index-subitem, .index-subsubitem {display:block}
.caption td.id{font-weight: bold; white-space: nowrap; }
table.caption {text-align:center;}
h1.partHead{text-align: center}
p.bibitem { text-indent: -2em; margin-left: 2em; margin-top:0.6em; margin-bottom:0.6em; }
p.bibitem-p { text-indent: 0em; margin-left: 2em; margin-top:0.6em; margin-bottom:0.6em; }
.paragraphHead, .likeparagraphHead { margin-top:2em; font-weight: bold;}
.subparagraphHead, .likesubparagraphHead { font-weight: bold;}
.quote {margin-bottom:0.25em; margin-top:0.25em; margin-left:1em; margin-right:1em; text-align:justify;}
.verse{white-space:nowrap; margin-left:2em}
div.maketitle {text-align:center;}
h2.titleHead{text-align:center;}
div.maketitle{ margin-bottom: 2em; }
div.author, div.date {text-align:center;}
div.thanks{text-align:left; margin-left:10%; font-size:85%; font-style:italic; }
div.author{white-space: nowrap;}
.quotation {margin-bottom:0.25em; margin-top:0.25em; margin-left:1em; }
h1.partHead{text-align: center}
.sectionToc, .likesectionToc {margin-left:2em;}
.subsectionToc, .likesubsectionToc {margin-left:4em;}
.subsubsectionToc, .likesubsubsectionToc {margin-left:6em;}
.frenchb-nbsp{font-size:75%;}
.frenchb-thinspace{font-size:75%;}
.figure img.graphics {margin-left:10%;}
/* end css.sty */

\title{Analyse numerique des equations differentielles}
\author{}
\date{}

\begin{document}
\maketitle

\textbf{Warning: 
requires JavaScript to process the mathematics on this page.\\ If your
browser supports JavaScript, be sure it is enabled.}

\begin{center}\rule{3in}{0.4pt}\end{center}

[
[
[]
[

\subsubsection{16.6 Analyse numérique des équations différentielles}

\paragraph{16.6.1 Méthode d'Euler}

Soit f : J \times E \rightarrow~ E de classe \mathcal{C}^1 , soit t_o dans J ,
y_o dans E et \phi : J_0 \rightarrow~ E la solution maximale
vérifiant la condition \phi(t_o) = y_o. J_0 est
un intervalle contenu dans J et contenant t_o, et dire que la
solution est maximale, c'est dire que \phi ne peut se prolonger en une
solution de l'équation différentielle sur un intervalle strictement plus
grand que J_0. Notre but est de trouver une approximation de la
fonction \phi .

Pour cela soit h un nombre réel suffisamment petit et t dans
J_0 tel que t + h appartienne encore à J_0. Alors \phi(t
+ h) est peu différent de \phi(t) + h\phi'(t). Mais \phi'(t) = f(t,\phi(t)). Donc
\phi(t + h) est peu différent de \phi(t) + hf(t,\phi(t)) = y + f(t,y) si y =
\phi(t). Ceci nous amène à définir pour un nombre réel h donné

\begin{itemize}
\itemsep1pt\parskip0pt\parsep0pt
\item
  (i) une suite (t_i)_i\in\mathbb{Z} par t_i =
  t_o + ih
\item
  (ii) une suite (y_i)_i\in\mathbb{Z} par y_i+1 =
  y_i + hf(t_i,y_i) pour i ≥ 0,
  y_i-1 = y_i - hf(t_i,y_i) pour i \leq
  0
\item
  (iii) une fonction \phi_h prenant aux points t_i la
  valeur y_i et affine sur chacun des intervalles
  [t_i,t_i+1]
\end{itemize}

Nous espérons bien entendu que la fonction \phi_h ainsi définie
sera une approximation de \phi pour h petit. Nous allons montrer que c'est
effectivement le cas, tout au moins sur un segment [a,b] contenu
dans J_0 et dans le cas simple où E = \mathbb{R}~ (bien que le résultat
reste valable si E est un espace vectoriel normé).

Remarquons tout d'abord que puisque f est de classe \mathcal{C}^1 et
que \phi'(t) = f(t,\phi(t)) , \phi' est dérivable et

\begin{align*} \phi'`(t)& =& \partial~f
\over \partial~t (t,\phi(t)) + \phi'(t) \partial~f \over
\partial~y (t,\phi(t)) \%& \\ & =& \partial~f
\over \partial~t (t,\phi(t)) + f(t,\phi(t)) \partial~f \over
\partial~y (t,\phi(t))\%& \\
\end{align*}

Soit \alpha~ > 0 et soit K = \(t,y) \in [a,b]
\times E∣\phi(t) - \alpha~ \leq y \leq \phi(t) +
\alpha~\.

K est un compact qui contient le graphe de \phi. La fonction
(t,y)\mapsto~ \partial~f \over \partial~t (t,y) +
f(t,y) \partial~f \over \partial~y (t,y) est continue sur ce compact,
donc bornée. Soit

M = sup_(t,y)\inK~ \partial~f
\over \partial~t (t,y) + f(t,y) \partial~f \over \partial~y
(t,y)

Alors on a pour tout t dans [a,b] \phi''(t)\leq M. La
formule de Taylor-Lagrange nous donne alors \phi(t + h) = \phi(t) + h\phi'(t) +
h^2 \over 2 \phi''(\xi), d'où

\left  \phi(t + h) - \phi(t) \over
h - f(t,\phi(t))\right \leq M
h \over 2

D'autre part la fonction (t,y)\mapsto~ \partial~f
\over \partial~y (t,y) est continue sur K donc bornée. Soit A
= sup_(t,y)\inK~ \partial~f
\over \partial~y (t,y). Alors on a si (t,y) \in K et
(t,y') \in K, f(t,y) - f(t,y') = (y - y') \partial~f \over \partial~y
(t,z) pour un certain z appartenant à [y,y'], donc f(t,y)
- f(t,y')\leq Ay - y' .

Définissons alors une suite (t_i)_i\in\mathbb{N}~ par t_i
= t_o + ih, une suite (y_i)_i\in\mathbb{N}~ par
y_i+1 = y_i + hf(t_i,y_i). Nous
allons mesurer l'erreur e_i = y_i -
\phi(t_i). Comme nous ne sommes malheureusement pas sûrs
que les couples (t_i,y_i) appartiennent à K nous
allons définir une fonction g : [a,b] \times \mathbb{R}~ \rightarrow~ \mathbb{R}~ par

g(x,y) = \left \ \cases
f(t,y) &si \phi(t) - \alpha~ \leq y \leq \phi(t) + \alpha~ \cr f(t,\phi(t) - \alpha~)&si
y < \phi(t) - \alpha~ \cr f(t,\phi(t) + \alpha~)&si y
> \phi(t) + \alpha~))  \right .

et une suite (z_i)_i\in\mathbb{N}~ par z_i+1 =
z_i + hg(t_i,z_i). Il est clair que
z_i = y_i tant que (t_i,y_i)
appartient à K. Posons \epsilon_i = z_i -
\phi(t_i). La fonction g est continue sur [a,b] \times \mathbb{R}~
et vérifie g(t,y) - g(t,y')\leq Ay -
y' pour tout t \in [a,b] et tous y,y' \in \mathbb{R}~ d'après (2). On a
alors

\begin{align*} \epsilon_i+1& =&
z_i+1 - \phi(t_i+1) =
z_i + hg(t_i,z_i) - \phi(t_i
+ h)\%& \\ & \leq&
z_i - \phi(t_i) +
hg(t_i,z_i) -
g(t_i,\phi(t_i)) \%&
\\ & & +\phi(t_i) +
hg(t_i,\phi(t_i)) - \phi(t_i + h) \%&
\\ & \leq& z_i -
\phi(t_i) + Ahz_i
- \phi(t_i) \%& \\ & &
+h\left  \phi(t_i +
h) - \phi(t_i) \over h -
f(t_i,\phi(t_i))\right  \%&
\\ \end{align*}

car g(t,\phi(t)) = f(t,\phi(t)). On obtient donc en utilisant (1)
\epsilon_i+1 \leq (1 + Ah)\epsilon_i + M
h^2 \over 2 . Comme
\epsilon_o = e_o = 0, on a donc par récurrence

\begin{align*} \epsilon_i& \leq& M
h^2 \over 2 (1 + (1 +
Ah) + \ldots + (1 +
Ah)^i-1)\%&
\\ & =& M
h^2 \over 2  (1 +
Ah)^i - 1 \over
hA \%& \\
\end{align*}

soit, puisque 1 + x \leq e^x,

\begin{align*} \epsilon_i& \leq M&
h \over 2 
e^Aih- 1 \over A
\leqh M \over 2A
(e^At-ti- 1)\%&
\\ & \leq & h M
\over 2A (e^A(b-a) - 1) \%&
\\ \end{align*}

On voit donc que pour h assez petit, on a pour tout i tel que
t_i appartienne à [a,b], \epsilon_i \leq \alpha~, donc
z_i = y_i, et donc \epsilon_i = e_i, avec
une erreur e_i \leqh M \over
2A (e^A(b-a) - 1).

Soit maintenant x \in]t_i,t_i+1[. Considérons la
fonction affine g qui vérifie g(t_i) = \phi(t_i) et
g(t_i+1) = \phi(t_i+1). Soit h(t) = \phi(t) - g(t) - \mu
(t-t_i)(t-t_i+1) \over 2 où \mu est
choisi de telle sorte que h(x) = 0. Deux applications du théorème de
Rolle à la fonction h qui s'annule en t_i, x et t_i+1
montrent qu'il existe \xi tel que h''(\xi) = 0. Or h'`(\xi) = \phi''(\xi) - \mu
puisque g'' = 0. On a donc en écrivant que h(x) = 0,

\phi(x) - g(x) = \phi''(\xi) (x - t_i)(x - t_i+1)
\over 2

soit

\phi(x) - g(x)\leq M (t_i+1 -
t_i)^2 \over 8 = M h^2
\over 8

puisque l'on a vu que pour tout t dans [a,b],
\phi''(t)\leq M. Or g et \phi_h sont affines sur
[t_i,t_i+1] et donc

\begin{align*} g(x) -
\phi_h(x)& \leq&
max(g(t_i~) -
\phi_h(t_i),g(t_i+1) -
\phi_h(t_i+1))\%&
\\ & =&
max(e_i,e_i+1~)
\leqh M \over 2A
(e^A(b-a) - 1) \%& \\
\end{align*}

On en déduit donc que

\forall~~x \in [a,b], \phi(x) -
\phi_h(x)\leqh M \over
2A (e^A(b-a) - 1) + M h^2 \over
8

(en fait on n'a vu cette majoration que sur [t_o,b], mais
il suffit de changer h en - h pour avoir le même résultat sur
[a,t_o], c'est pour cela que volontairement nous avons
laissé les valeurs absolues partout). Les fonctions \phi_h
convergent donc uniformément vers \phi sur [a,b] quand h tend vers 0,
avec une majoration du type

\forall~~x \in [a,b],\phi(x) -
\phi_h(x)\leq Bh

Dans la pratique il faut faire attention aux accumulations d'erreurs
d'arrondis. L'erreur sur le calcul de y_n est de l'ordre de n\epsilon,
où \epsilon est la précision de calcul de l'ordinateur. On en déduit que
l'erreur sur le calcul de \phi_h(x) est de l'ordre de grandeur de
 b-a \over h \epsilon. Soit une erreur totale du type Bh +
b-a \over h \epsilon. Une étude simple de cette fonction
montre que l'erreur totale est minimale pour des fonctions usuelles
quand h est de l'ordre de \sqrt\epsilon. On préférera
prendre une valeur de h un peu trop grande, plutôt que trop petite. Il
faut se méfier également du temps de calcul qui croit rapidement si l'on
prend des valeurs de h trop petites.

\paragraph{16.6.2 Méthode de Runge et Kutta}

La méthode est inspirée de la même idée que celle de la méthode d'Euler,
mais on améliore l'approximation faite. Dans la méthode d'Euler nous
prenions pour approximation de \phi(t + h) l'expression \phi(t) + h\phi'(t) =
\phi(t) + hf(t,\phi(t)). Ici on prendra comme approximation de \phi(t + h)
l'expression \phi(t) + hk(t,h) où k(t,h) est défini en posant

\begin{align*} k_1(t,h)& =& f(t,\phi(t)),
\%& \\ k_2(t,h)& =& f(t + h
\over 2 ,\phi(t) + h \over 2
k_1(t,h)),\%& \\
k_3(t,h)& =& f(t + h \over 2 ,\phi(t) + h
\over 2 k_2(t,h)),\%&
\\ k_4(t,h)& =& f(t + h,\phi(t) +
hk_3(t,h)) \%& \\
\end{align*}

et enfin

k(t,h) = 1 \over 6 (k_1(t,h) +
2k_2(t,h) + 2k_3(t,h) + k_4(t,h)).

On définit donc notre suite y_i par la relation de récurrence
y_i+1 = y_i + hk(i) où l'on a posé

\begin{align*} k_1(i)& =&
f(t_i,y_i), \%& \\
k_2(i)& =& f(t_i + h \over 2
,y_i + h \over 2 k_1(i)), \%&
\\ k_3(i)& =& f(t_i
+ h \over 2 ,y_i + h \over
2 k_2(i)), \%& \\
k_4(i)& =& f(t_i + h,y_i +
hk_3(i)), \%& \\ k(i)& =& 1
\over 6 (k_1(i) + 2k_2(i) +
2k_3(i) + k_4(i))\%& \\
\end{align*}

pour i ≥ 0 , pour i < 0 on change h en - h.

On montre alors par un calcul pénible (à base de formule de Taylor) que

\left  \phi(t+h)-\phi(t) \over h -
k(t,h)\right \leq M
h^4 \over 2 et la même
démonstration que dans la méthode d'Euler montre qu'il existe une
constante C telle que

\forall~~x \in [a,b], \phi(x) -
\phi_h(x)\leq Ch^4 +
Dh^2

Le terme en h^2 provient en fait de l'interpolation linéaire.
Aux points t_i l'erreur est en fait en
Ch^4. On obtient ainsi une convergence
beaucoup plus rapide que dans la méthode d'Euler. L'étude de
l'accumulation des erreurs montre que le meilleur h possible (pour les
points t_i) est de l'ordre de
\root5\of\epsilon où \epsilon est la précision
de l'ordinateur. On pourra par exemple prendre un h de l'ordre de
10^-2 ou 10^-3.

\paragraph{16.6.3 Equations différentielles d'ordre supérieur}

Il suffit de rappeler qu'une équation différentielle d'ordre p du type
y^(p) = f(t,y,y',\ldots,y^(p-1)) se ramène à un
système différentiel

\left \\array
y_1' & = y_2 \cr
&\\ldots~\cr
y_ p-1'& = y_p \cr y_p' & =
f(t,y_1,y_2,\ldots,y_p) 
\right .

en posant y_1 = y,y_2 = y',\ldots,y_p =
y^(p-1). On appliquera donc l'une des deux méthodes
précédentes à ce système.

[
[
[
[

\end{document}

\part{Espaces affines}
% \documentclass[]{article}
\usepackage[T1]{fontenc}
\usepackage{lmodern}
\usepackage{amssymb,amsmath}
\usepackage{ifxetex,ifluatex}
\usepackage{fixltx2e} % provides \textsubscript
% use upquote if available, for straight quotes in verbatim environments
\IfFileExists{upquote.sty}{\usepackage{upquote}}{}
\ifnum 0\ifxetex 1\fi\ifluatex 1\fi=0 % if pdftex
  \usepackage[utf8]{inputenc}
\else % if luatex or xelatex
  \ifxetex
    \usepackage{mathspec}
    \usepackage{xltxtra,xunicode}
  \else
    \usepackage{fontspec}
  \fi
  \defaultfontfeatures{Mapping=tex-text,Scale=MatchLowercase}
  \newcommand{\euro}{€}
\fi
% use microtype if available
\IfFileExists{microtype.sty}{\usepackage{microtype}}{}
\ifxetex
  \usepackage[setpagesize=false, % page size defined by xetex
              unicode=false, % unicode breaks when used with xetex
              xetex]{hyperref}
\else
  \usepackage[unicode=true]{hyperref}
\fi
\hypersetup{breaklinks=true,
            bookmarks=true,
            pdfauthor={},
            pdftitle={Generalites sur les espaces affines},
            colorlinks=true,
            citecolor=blue,
            urlcolor=blue,
            linkcolor=magenta,
            pdfborder={0 0 0}}
\urlstyle{same}  % don't use monospace font for urls
\setlength{\parindent}{0pt}
\setlength{\parskip}{6pt plus 2pt minus 1pt}
\setlength{\emergencystretch}{3em}  % prevent overfull lines
\setcounter{secnumdepth}{0}
 
/* start css.sty */
.cmr-5{font-size:50%;}
.cmr-7{font-size:70%;}
.cmmi-5{font-size:50%;font-style: italic;}
.cmmi-7{font-size:70%;font-style: italic;}
.cmmi-10{font-style: italic;}
.cmsy-5{font-size:50%;}
.cmsy-7{font-size:70%;}
.cmex-7{font-size:70%;}
.cmex-7x-x-71{font-size:49%;}
.msbm-7{font-size:70%;}
.cmtt-10{font-family: monospace;}
.cmti-10{ font-style: italic;}
.cmbx-10{ font-weight: bold;}
.cmr-17x-x-120{font-size:204%;}
.cmsl-10{font-style: oblique;}
.cmti-7x-x-71{font-size:49%; font-style: italic;}
.cmbxti-10{ font-weight: bold; font-style: italic;}
p.noindent { text-indent: 0em }
td p.noindent { text-indent: 0em; margin-top:0em; }
p.nopar { text-indent: 0em; }
p.indent{ text-indent: 1.5em }
@media print {div.crosslinks {visibility:hidden;}}
a img { border-top: 0; border-left: 0; border-right: 0; }
center { margin-top:1em; margin-bottom:1em; }
td center { margin-top:0em; margin-bottom:0em; }
.Canvas { position:relative; }
li p.indent { text-indent: 0em }
.enumerate1 {list-style-type:decimal;}
.enumerate2 {list-style-type:lower-alpha;}
.enumerate3 {list-style-type:lower-roman;}
.enumerate4 {list-style-type:upper-alpha;}
div.newtheorem { margin-bottom: 2em; margin-top: 2em;}
.obeylines-h,.obeylines-v {white-space: nowrap; }
div.obeylines-v p { margin-top:0; margin-bottom:0; }
.overline{ text-decoration:overline; }
.overline img{ border-top: 1px solid black; }
td.displaylines {text-align:center; white-space:nowrap;}
.centerline {text-align:center;}
.rightline {text-align:right;}
div.verbatim {font-family: monospace; white-space: nowrap; text-align:left; clear:both; }
.fbox {padding-left:3.0pt; padding-right:3.0pt; text-indent:0pt; border:solid black 0.4pt; }
div.fbox {display:table}
div.center div.fbox {text-align:center; clear:both; padding-left:3.0pt; padding-right:3.0pt; text-indent:0pt; border:solid black 0.4pt; }
div.minipage{width:100%;}
div.center, div.center div.center {text-align: center; margin-left:1em; margin-right:1em;}
div.center div {text-align: left;}
div.flushright, div.flushright div.flushright {text-align: right;}
div.flushright div {text-align: left;}
div.flushleft {text-align: left;}
.underline{ text-decoration:underline; }
.underline img{ border-bottom: 1px solid black; margin-bottom:1pt; }
.framebox-c, .framebox-l, .framebox-r { padding-left:3.0pt; padding-right:3.0pt; text-indent:0pt; border:solid black 0.4pt; }
.framebox-c {text-align:center;}
.framebox-l {text-align:left;}
.framebox-r {text-align:right;}
span.thank-mark{ vertical-align: super }
span.footnote-mark sup.textsuperscript, span.footnote-mark a sup.textsuperscript{ font-size:80%; }
div.tabular, div.center div.tabular {text-align: center; margin-top:0.5em; margin-bottom:0.5em; }
table.tabular td p{margin-top:0em;}
table.tabular {margin-left: auto; margin-right: auto;}
div.td00{ margin-left:0pt; margin-right:0pt; }
div.td01{ margin-left:0pt; margin-right:5pt; }
div.td10{ margin-left:5pt; margin-right:0pt; }
div.td11{ margin-left:5pt; margin-right:5pt; }
table[rules] {border-left:solid black 0.4pt; border-right:solid black 0.4pt; }
td.td00{ padding-left:0pt; padding-right:0pt; }
td.td01{ padding-left:0pt; padding-right:5pt; }
td.td10{ padding-left:5pt; padding-right:0pt; }
td.td11{ padding-left:5pt; padding-right:5pt; }
table[rules] {border-left:solid black 0.4pt; border-right:solid black 0.4pt; }
.hline hr, .cline hr{ height : 1px; margin:0px; }
.tabbing-right {text-align:right;}
span.TEX {letter-spacing: -0.125em; }
span.TEX span.E{ position:relative;top:0.5ex;left:-0.0417em;}
a span.TEX span.E {text-decoration: none; }
span.LATEX span.A{ position:relative; top:-0.5ex; left:-0.4em; font-size:85%;}
span.LATEX span.TEX{ position:relative; left: -0.4em; }
div.float img, div.float .caption {text-align:center;}
div.figure img, div.figure .caption {text-align:center;}
.marginpar {width:20%; float:right; text-align:left; margin-left:auto; margin-top:0.5em; font-size:85%; text-decoration:underline;}
.marginpar p{margin-top:0.4em; margin-bottom:0.4em;}
.equation td{text-align:center; vertical-align:middle; }
td.eq-no{ width:5%; }
table.equation { width:100%; } 
div.math-display, div.par-math-display{text-align:center;}
math .texttt { font-family: monospace; }
math .textit { font-style: italic; }
math .textsl { font-style: oblique; }
math .textsf { font-family: sans-serif; }
math .textbf { font-weight: bold; }
.partToc a, .partToc, .likepartToc a, .likepartToc {line-height: 200%; font-weight:bold; font-size:110%;}
.chapterToc a, .chapterToc, .likechapterToc a, .likechapterToc, .appendixToc a, .appendixToc {line-height: 200%; font-weight:bold;}
.index-item, .index-subitem, .index-subsubitem {display:block}
.caption td.id{font-weight: bold; white-space: nowrap; }
table.caption {text-align:center;}
h1.partHead{text-align: center}
p.bibitem { text-indent: -2em; margin-left: 2em; margin-top:0.6em; margin-bottom:0.6em; }
p.bibitem-p { text-indent: 0em; margin-left: 2em; margin-top:0.6em; margin-bottom:0.6em; }
.paragraphHead, .likeparagraphHead { margin-top:2em; font-weight: bold;}
.subparagraphHead, .likesubparagraphHead { font-weight: bold;}
.quote {margin-bottom:0.25em; margin-top:0.25em; margin-left:1em; margin-right:1em; text-align:\\jmathmathustify;}
.verse{white-space:nowrap; margin-left:2em}
div.maketitle {text-align:center;}
h2.titleHead{text-align:center;}
div.maketitle{ margin-bottom: 2em; }
div.author, div.date {text-align:center;}
div.thanks{text-align:left; margin-left:10%; font-size:85%; font-style:italic; }
div.author{white-space: nowrap;}
.quotation {margin-bottom:0.25em; margin-top:0.25em; margin-left:1em; }
h1.partHead{text-align: center}
.sectionToc, .likesectionToc {margin-left:2em;}
.subsectionToc, .likesubsectionToc {margin-left:4em;}
.subsubsectionToc, .likesubsubsectionToc {margin-left:6em;}
.frenchb-nbsp{font-size:75%;}
.frenchb-thinspace{font-size:75%;}
.figure img.graphics {margin-left:10%;}
/* end css.sty */

\title{Generalites sur les espaces affines}
\author{}
\date{}

\begin{document}
\maketitle

\textbf{Warning: 
requires JavaScript to process the mathematics on this page.\\ If your
browser supports JavaScript, be sure it is enabled.}

\begin{center}\rule{3in}{0.4pt}\end{center}

{[}
{[}{]}
{[}

\subsubsection{17.1 Généralités sur les espaces affines}

\paragraph{17.1.1 Notion d'espace affine}

Définition~17.1.1 On appelle espace affine un triplet
(E,\overrightarrowE,+) d'un ensemble E (l'ensemble
des points), d'un espace vectoriel \overrightarrowE
(l'espace des vecteurs) et d'une application + : E
\times\overrightarrow E \rightarrow~ E,
(x,\overrightarrow\xi)\mapsto~x
+\overrightarrow \xi vérifiant les propriétés

\begin{itemize}
\itemsep1pt\parskip0pt\parsep0pt
\item
  (i)
  \forall~(x,\overrightarrow\xi,\overrightarrow\eta~)
  \in E \times\overrightarrow E
  \times\overrightarrow E, (x
  +\overrightarrow \xi)
  +\overrightarrow \eta = x +
  (\overrightarrow\xi +\overrightarrow
  \eta)
\item
  (ii)
  \forall~(x,\overrightarrow\xi~) \in E
  \times\overrightarrow E, \left (x
  +\overrightarrow \xi = x \Leftrightarrow
  \overrightarrow\xi =\overrightarrow
  0\right )
\item
  (iii) \forall~~x,y \in E,
  \exists\overrightarrow\xi~
  \in\overrightarrow E, x
  +\overrightarrow \xi = y
\end{itemize}

Remarque~17.1.1 L'assertion (i) et la moitié '' ⇐'' de l'assertion (iii)
traduisent que l'application
(x,\overrightarrow\xi)\mapsto~x
+\overrightarrow \xi est une loi de groupe opérant sur
un ensemble du groupe additif de l'espace vectoriel
\overrightarrowE sur l'ensemble E. La propriété (iii)
traduit le fait que cette opération est transitive, c'est-à-dire qu'il y
a une seule orbite pour cette opération. La moitié '' \rigtharrow~'' de la
propriété (ii) est appelée la fidélité de l'opération. Un espace affine
est donc une opération transitive et fidèle du groupe additif d'un
espace vectoriel sur un ensemble. L'espace vectoriel
\overrightarrowE est appelé la direction de l'espace
affine. Par la suite on confondra abusivement l'espace affine
(E,\overrightarrowE,+) avec l'ensemble E de ses
points.

Proposition~17.1.1 Etant donné x,y \in E, il existe un unique vecteur noté
\overrightarrowxy \in\overrightarrow
E vérifiant x +\overrightarrow xy = y. On a
\overrightarrowxy = 0 \Leftrightarrow x
= y et on a la relation de Chasles

\forall~~x,y,z \in E,
\overrightarrowxz =\overrightarrow
xy +\overrightarrow yz

Démonstration L'existence est garantie par la propriété (iii). Pour
l'unicité, si on a à la fois y = x +\overrightarrow \xi
= x +\overrightarrow \eta, on a x = x
+\overrightarrow 0 = (x
+\overrightarrow \xi) +
(-\overrightarrow\xi) = (x
+\overrightarrow \eta) +
(-\overrightarrow\xi) = x +
(\overrightarrow\eta -\overrightarrow
\xi) d'où \overrightarrow\xi
-\overrightarrow \eta =\overrightarrow
0 et donc \overrightarrow\xi
=\overrightarrow \eta. La propriété
\overrightarrowxy = 0 \Leftrightarrow x
= y est une conséquence évidente de (ii). On a z = y
+\overrightarrow yz = (x
+\overrightarrow xy)
+\overrightarrow yz = x +
(\overrightarrowxy +\overrightarrow
yz) d'où \overrightarrowxz
=\overrightarrow xy +\overrightarrow
yz.

Proposition~17.1.2 Soit a \in E. L'application \phi_a :
x\mapsto~\overrightarrowax est
une bi\\jmathmathection de l'espace affine E sur l'espace vectoriel
\overrightarrowE.

Démonstration L'application réciproque est bien entendu l'application
\psi_a :\overrightarrow
\xi\mapsto~a +\overrightarrow \xi.
On vérifie immédiatement que \psi_a \cdot \phi_a =
\mathrmId_E et que \phi_a \cdot \psi_a
=
\mathrmId_\overrightarrowE.

Remarque~17.1.2 Par transport des opérations algébriques de
\overrightarrowE sur E, on munit ainsi E d'une
structure d'espace vectoriel isomorphe à
\overrightarrowE. On dira que l'espace vectoriel
ainsi obtenu est le vectorialisé de E en l'origine a et on le notera par
la suite E_a. On retiendra donc que le choix d'une origine dans
l'espace affine transforme cet espace affine en un espace vectoriel.

Définition~17.1.2 On appelle dimension de l'espace affine E la dimension
de sa direction \overrightarrowE comme espace
vectoriel.

\paragraph{17.1.2 Repères affines, bases affines}

Définition~17.1.3 On appelle repère affine de E tout couple (a,\mathcal{E}) d'un
point a de E (l'origine du repère) et d'une base \mathcal{E} de
\overrightarrowE. Si \mathcal{E} =
(\vece_i)_i\inI tout point x \in E
s'écrit de manière unique sous la forme x = a
+ \\sum ~
_i\inIx_i\vece_i~; on dit que
les x_i sont les coordonnées du point x dans le repère affine~;
ce sont également les coordonnées du vecteur
\overrightarrowax dans la base \mathcal{E}.

Soit (a,\mathcal{E}) et (b,ℱ) deux repères affines de E. Ecrivons alors (avec des
notations évidentes) b = a +\
\sum ~
_i\inI\beta_i\vece_i et
\vecf_\\jmathmath =\
\sum ~
_i\inIu_i,\\jmathmath\vece_i. On a
alors, si les x_i désignent les coordonnées de x dans le repère
(a,\mathcal{E}) et les y_\\jmathmath celles de x dans le repère (b,ℱ)

\begin{align*} x& =& b +
\\sum
_\\jmathmath\inJy_\\jmathmath\vecf_\\jmathmath = a +
\\sum
_i\inI\beta_i\vece_i +
\sum _\\jmathmath\inJy_\\jmathmath~
\\sum
_i\inIu_i,\\jmathmath\vece_i\%&
\\ & =& a + \\sum
_i\inI\beta_i\vece_i +
\sum _i\inI~\left
(\\sum
_\\jmathmath\inJu_i,\\jmathmathy_\\jmathmath\right
)e_i \%& \\ & =& a +
\sum _i\inI~\left
(\beta_i + \\sum
_\\jmathmath\inJu_i,\\jmathmathy_\\jmathmath\right
)e_i \%& \\
\end{align*}

si bien que \forall~i \in I,x_i = \beta_i~
+ \\sum ~
_\\jmathmath\inJu_i,\\jmathmathy_\\jmathmath ce qui fournit les formules de
changement de repère.

Proposition~17.1.3 Soit (a_i)_i\inI une famille non vide
de points de E. Alors les propriétés suivantes sont équivalentes

\begin{itemize}
\itemsep1pt\parskip0pt\parsep0pt
\item
  (i) il existe i_0 \in I tel que la famille
  (\overrightarrowa_i_0a_i)_i\inI\diagdown\i_0\
  soit libre (resp. génératrice, resp. une base) dans
  \overrightarrowE
\item
  (ii) pour tout i_0 \in I, la famille
  (\overrightarrowa_i_0a_i)_i\inI\diagdown\i_0\
  est libre (resp. génératrice, resp. une base) dans
  \overrightarrowE
\end{itemize}

On dit dans ce cas que la famille (a_i)_i\inI est
affinement libre (resp. affinement génératrice, resp. une base affine)
de E.

Démonstration Libre~: Il est clair que (ii) \rigtharrow~(i). Supposons (i) vérifiée
et soit i_1 \in I. Soit
(\lambda_i)_i\inI\diagdown\i_1\
une famille de scalaires (avec seulement un nombre fini de scalaires non
nuls) tels que \\sum ~
_i\inI\diagdown\i_1\\lambda_i\overrightarrowa_i_1a_i
=\overrightarrow 0. On a alors en notant que
\overrightarrowa_i_0a_i_0
=\overrightarrow 0)

\begin{align*} \overrightarrow0&
=& \\sum
_i\inI\diagdown\i_1\\lambda_i(\overrightarrowa_i_1a_i_0
+\overrightarrow a_i_0a_i)
\%& \\ & =& \\sum
_i\inI\diagdown\i_0,i_1\\lambda_i\overrightarrowa_i_0a_i
-\left (\\sum
_i\inI\diagdown\i_1\\lambda_i\right
)\overrightarrowa_i_0a_i_1\%&
\\ \end{align*}

Comme la famille
(\overrightarrowa_i_0a_i)_i\inI\diagdown\i_0\
est libre, on doit donc avoir \forall~~i \in I
\diagdown\i_0,i_1\,
\lambda_i = 0 et également
\\sum ~
_i\inI\diagdown\i_1\\lambda_i
= 0, ce qui nous donne évidemment que \lambda_i_0 est
également nul. D'où \forall~~i \in I
\diagdown\i_1\, \lambda_i = 0.

Génératrice Il est clair que (ii) \rigtharrow~(i). Supposons (i) vérifiée et soit
i_1 \in I. On peut alors écrire, si
\overrightarrow\xi \in\overrightarrow
E,

\begin{align*} \overrightarrow\xi&
=& \\sum
_i\inI\diagdown\i_0\\lambda_i\overrightarrowa_i_0a_i
= \\sum
_i\inI\diagdown\i_0\\lambda_i(\overrightarrowa_i_0a_i_1
+\overrightarrow
a_i_1a_i)\%&
\\ & =& \\sum
_i\inI\diagdown\i_0,i_1\\lambda_i\overrightarrowa_i_1a_i
-\left (\\sum
_i\inI\diagdown\i_0\\lambda_i\right
)\overrightarrowa_i_1a_i_0
\%& \\ & =& \\sum
_i\inI\diagdown\i_1\\mu_i\overrightarrowa_i_1a_i
\%& \\ \end{align*}

avec \mu_i = \lambda_i pour i \in I
\diagdown\i_0,i_1\ et
\mu_i_0 =\
\sum ~
_i\inI\diagdown\i_0\\lambda_i
ce qui montre que la famille
(\overrightarrowa_i_1a_i)_i\inI\diagdown\i_1\
est encore génératrice.

Le résultat pour les bases se déduit immédiatement de la combinaison des
deux résultats précédents.

\paragraph{17.1.3 Sous-espace affine}

Définition~17.1.4 Soit E un espace affine et F une partie de E. On dit
que F est un sous espace affine de E si on a les propriétés équivalentes

\begin{itemize}
\itemsep1pt\parskip0pt\parsep0pt
\item
  (i) il existe a \in F tel que
  \\overrightarrowax∣x
  \in F\ soit un sous-espace vectoriel de
  \overrightarrowE
\item
  (ii) F\neq~\varnothing~ et pour tout a \in F,
  \\overrightarrowax∣x
  \in F\ est un sous espace vectoriel de
  \overrightarrowE
\item
  (iii) il existe a \in F et un sous-espace vectoriel
  \overrightarrowF_a de
  \overrightarrowE tel que F = a
  +\overrightarrow F_a
\item
  (iv) F\neq~\varnothing~ et pour tout a \in F, il existe un
  sous-espace vectoriel \overrightarrowF_a
  de \overrightarrowE tel que F = a
  +\overrightarrow F_a
\end{itemize}

Le sous-espace vectoriel \overrightarrowF_a
est alors indépendant du choix de a \in F~; on l'appelle la direction de F
et on le note \overrightarrowF

Démonstration Il est clair que (ii) \rigtharrow~(i) et que (iv) \rigtharrow~(iii). D'autre
part

\begin{align*} F = a
+\overrightarrow F_a&
\Leftrightarrow & F = \a
+\overrightarrow
\xi∣\overrightarrow\xi
\in\overrightarrow F_a\\%&
\\ & \Leftrightarrow & F =
a +
\\overrightarrow\xi∣\overrightarrow\xi
\in\overrightarrow F_a\\%&
\\ & \Leftrightarrow &
\overrightarrowF_a =
\\overrightarrowax∣x
\in F\ \%& \\
\end{align*}

ce qui assure que (i) \Leftrightarrow(iii) et que (ii) \Leftrightarrow(iv). Il ne nous reste plus à
montrer que (i) \rigtharrow~(iii) pour avoir les équivalences. Mais si b \in E, on a

\\overrightarrowbx∣x
\in F\ =
\\overrightarrowba
+\overrightarrow ax∣x \in
F\ = -\overrightarrowab +
\\overrightarrowax∣x
\in F\

avec \overrightarrowab
\in\\overrightarrowax∣x
\in F\~; or un sous-espace vectoriel est stable par ses
translations et donc -\overrightarrow ab +
\\overrightarrowax∣x
\in F\ =
\\overrightarrowax∣x
\in F\ ce qui montre à la fois que
\\overrightarrowbx∣x
\in F\ =
\\overrightarrowax∣x
\in F\ et que
\\overrightarrowbx∣x
\in F\ est un sous-espace vectoriel.

Comme on l'a vu, on a alors

\overrightarrowF_b =
\\overrightarrowbx∣x
\in F\ =
\\overrightarrowax∣x
\in F\ =\overrightarrow F_a

ce qui montre que \overrightarrowF_a est
indépendant du choix de a \in F.

Remarque~17.1.3 On peut également traduire les propriétés (i) ou (ii)
sous la forme~: F est un sous-espace vectoriel de E_a
(vectorialisé de E en l'origine a). Remarquons qu'un sous-espace affine
est nécessairement non vide.

Définition~17.1.5 On appelle dimension du sous-espace affine F de E la
dimension de \overrightarrowF.

Proposition~17.1.4 Soit (F_i)_i\inI une famille de
sous-espaces affines de E. Alors
\⋂ ~
_i\inIF_i est soit l'ensemble vide, soit un sous-espace
affine de direction \\⋂
 _i\inI\overrightarrowF_i.

Démonstration Supposons l'intersection non vide et soit a
\in\⋂ ~
_i\inIF_i. Alors

x \in⋂ _i\inIF_i~
\Leftrightarrow \forall~~i \in I,
\overrightarrowax \in\overrightarrow
F_i \Leftrightarrow
\overrightarrowax \in\⋂
_i\inI\overrightarrowF_i

ce dernier étant un sous-espace vectoriel de
\overrightarrowE. Ceci montre que
\⋂ ~
_i\inIF_i est un sous-espace affine et que sa direction
est \⋂ ~
_i\inI\overrightarrowF_i.

Remarque~17.1.4 Ceci permet ensuite de parler de sous-espace affine
engendré par une partie non vide de E~: c'est l'intersection de tous les
sous-espaces affines contenant A. Cette intersection est un sous-espace
affine contenant A et il est contenu dans tout espace affine contenant
A. Comme dans les espaces vectoriels, ceci permet de définir le
sous-espace affine Aff(a_i~,i \in I)
engendrée par une famille (a_i)_i\inI que l'on doit ici
supposer non vide. Si i_0 \in I, on vérifie immédiatement que

\begin{align*}
Aff(a_i~,i \in I)& =&
a_i_0 +\
\mathrmVect(\overrightarrowa_i_0a_i,
i \in I \diagdown\i_0\)\%&
\\ & =& a_i_0 +
\\\sum
_i\inI\diagdown\i_0\\lambda_i\overrightarrowa_i_0a_i\
\%& \\ \end{align*}

On en déduit que dim~
Aff(a_i~,i \in I) est inférieur ou égal
au cardinal de I moins 1. Ceci permet également de parler de rang d'une
famille de points.

\paragraph{17.1.4 Parallélisme, intersection}

Définition~17.1.6 Soit deux sous-espaces affines F et G. On dit que

\begin{itemize}
\itemsep1pt\parskip0pt\parsep0pt
\item
  (i) F et G sont parallèles si \overrightarrowF
  =\overrightarrow G
\item
  (ii) F est faiblement parallèle à G si
  \overrightarrowF \subset~\overrightarrow
  G
\end{itemize}

Théorème~17.1.5

\begin{itemize}
\itemsep1pt\parskip0pt\parsep0pt
\item
  (i) Si a \in E, il existe un unique sous-espace affine passant par a et
  parallèle à un sous-espace affine donné F
\item
  (ii) Si F et G sont parallèles, on a soit F = G, soit F \bigcap G = \varnothing~
\item
  (iii) Si F est faiblement parallèle à G , on a soit F \subset~ G, soit F \bigcap G
  = \varnothing~
\end{itemize}

Démonstration (i) G = a +\overrightarrow F est bien
entendu le seul qui convient. (ii) et (iii) sont évidents.

Le théorème suivant va \\jmathmathouer un rôle important pour garantir que deux
sous-espaces affines ont une intersection non vide

Théorème~17.1.6 Soit F et G deux sous-espaces affines de E, a \in F et b \in
G. On a équivalence de

\begin{itemize}
\itemsep1pt\parskip0pt\parsep0pt
\item
  (i) F \bigcap G\neq~\varnothing~
\item
  (ii) \overrightarrowab
  \in\overrightarrow F
  +\overrightarrow G
\end{itemize}

Démonstration (i) \rigtharrow~(ii) Soit x \in F \bigcap G. On a donc
\overrightarrowax \in\overrightarrow
F et \overrightarrowxb
\in\overrightarrow G, d'où
\overrightarrowab =\overrightarrow
ax +\overrightarrow bx
\in\overrightarrow F +\overrightarrow
G.

(ii) \rigtharrow~(i) Ecrivons \overrightarrowab
=\overrightarrow \xi +\overrightarrow
\eta avec \overrightarrow\xi
\in\overrightarrow F et
\overrightarrow\eta \in\overrightarrow
G. On a alors b -\overrightarrow \eta = a
+\overrightarrow ab -\overrightarrow
\eta = a +\overrightarrow \xi. Alors le point x = b
-\overrightarrow \eta = a
+\overrightarrow \xi appartient à la fois à F et à G.

Remarque~17.1.5 En dimension 3 et pour deux droites non parallèles
D_1 = a + K\vecu et D_2 = b +
K\vecv, on en déduit que

D_1 \bigcap
D_2\neq~\varnothing~\mathrel\Leftrightarrow
\overrightarrowab \in K\vecu +
K\vecv \Leftrightarrow
(\overrightarrowab,\vecu,\vecv)\text
est liée 

Corollaire~17.1.7 Soit F et G deux sous-espaces affines de E tels que
\overrightarrowF et
\overrightarrowG soient deux sous-espaces
supplémentaires de \overrightarrowE. Alors F \bigcap G est
un point.

Démonstration Soit a \in F et b \in G~; on a
\overrightarrowab \in\overrightarrow
E =\overrightarrow F
+\overrightarrow G donc F \bigcap G n'est pas vide~; mais
alors \overrightarrowF \bigcap G
=\overrightarrow F \bigcap\overrightarrow
G =
\\overrightarrow0\
et donc l'intersection est un point.

\paragraph{17.1.5 Applications affines}

Définition~17.1.7 Soit E et F deux espaces affines et f : E \rightarrow~ F. On dit
que f est une application affine si elle vérifie

\begin{itemize}
\itemsep1pt\parskip0pt\parsep0pt
\item
  (i) il existe a \in E et une application linéaire
  \vecf de \overrightarrowE dans
  \overrightarrowF tels que
  \forall~~x \in E, f(x) = f(a) +\vec
  f(\overrightarrowax)
\item
  (ii) il existe une application linéaire \vecf de
  \overrightarrowE dans
  \overrightarrowF telle que
  \forall~~x \in E,
  \forall~\overrightarrow\xi~
  \in\overrightarrow E, f(x
  +\overrightarrow \xi) = f(x) +\vec
  f(\overrightarrow\xi)
\end{itemize}

L'application linéaire \vecf est unique, on l'appelle
l'application linéaire tangente à l'application affine f.

Démonstration Il est clair que (ii) \rigtharrow~(i). Montrons donc que (i) \rigtharrow~(ii).
On a en effet

\begin{align*} f(x +\overrightarrow
\xi)& =& f(a +\overrightarrow ax
+\overrightarrow \xi) = f(a) +\vec
f(\overrightarrowax
+\overrightarrow \xi) \%&
\\ & =& f(a) +\vec
f(\overrightarrowax) +\vec
f(\overrightarrow\xi) = f(x) +\vec
f(\overrightarrow\xi)\%&
\\ \end{align*}

La propriété (ii) montre que
\vecf(\overrightarrow\xi)
=\overrightarrow f(x)f(x
+\overrightarrow \xi) ce qui assure l'unicité de
\vecf.

Proposition~17.1.8

\begin{itemize}
\itemsep1pt\parskip0pt\parsep0pt
\item
  (i) Si f : E \rightarrow~ F et g : F \rightarrow~ G sont deux applications affines, alors g
  \cdot f est affine et on a \overrightarrowg \cdot f
  =\vec g \cdot\vec f
\item
  (ii) f est in\\jmathmathective (resp. sur\\jmathmathective, resp. bi\\jmathmathective) si et
  seulement si~\vecf est in\\jmathmathective (resp. sur\\jmathmathective,
  resp. bi\\jmathmathective)
\end{itemize}

Démonstration Elémentaire.

Remarque~17.1.6 Soit f une application affine de E dans E~; supposons
que f a un point fixe a. On a alors \forall~~x \in E,
\overrightarrowaf(x)
=\overrightarrow f(a)f(x) =\vec
f(\overrightarrowax) si bien qu'en identifiant E et
\overrightarrowE par le choix de l'origine a
(c'est-à-dire en identifiant x et \overrightarrowax),
f s'identifie à \vecf. Une application affine ayant
un point fixe s'identifie, par le choix d'un tel point fixe comme
origine, à son application linéaire tangente. Il est donc
particulièrement important de savoir qu'une application affine a un
point fixe.

Théorème~17.1.9 Soit E un espace affine de dimension finie, f : E \rightarrow~ E
une application affine. Si 1 n'est pas valeur propre de
\vecf, alors f a un unique point fixe.

Démonstration Soit a \in E. On a

\begin{align*}
\overrightarrowxf(x)& =&
\overrightarrowxa +\overrightarrow
af(a) +\overrightarrow f(a)f(x) =
-\overrightarrowax +\overrightarrow
af(a) +\vec
f(\overrightarrowax)\%&
\\ & =& (\vecf
-\mathrmId)(\overrightarrowax)
+\overrightarrow af(a) \%&
\\ \end{align*}

On en déduit que

\begin{align*} x = f(x)&
\Leftrightarrow & (\vecf
-\mathrmId)(\overrightarrowax)
+\overrightarrow af(a)
=\overrightarrow 0\%&
\\ & \Leftrightarrow &
(\vecf
-\mathrmId)(\overrightarrowax)
=\overrightarrow f(a)a \%&
\\ \end{align*}

Mais comme 1 n'est pas valeur propre de \vecf,
\vecf -\mathrmId est bi\\jmathmathective et
donc l'équation précédente définit un unique vecteur
\overrightarrowax et donc un unique point x.

Ceci nous amène à étudier certaines applications affines particulières.

Définition~17.1.8 Soit \overrightarrow\xi
\in\overrightarrow E. On appelle translation de vecteur
\overrightarrow\xi l'application
t_\overrightarrow\xi :
x\mapsto~x +\overrightarrow \xi.

Définition~17.1.9 Soit a \in E et \lambda~ \in K^∗. On appelle
homothétie de centre a de rapport \lambda~ l'application h_a,\lambda~ :
x\mapsto~a + \lambda~ \overrightarrowax.

Proposition~17.1.10 Soit f une application affine de E dans E. Alors

\begin{itemize}
\itemsep1pt\parskip0pt\parsep0pt
\item
  (i) f est une translation si et seulement si~\vecf
  = \mathrmId
\item
  (ii) f est une homothétie de rapport \lambda~\neq~1 si
  et seulement si~\vecf =
  \lambda~\mathrmId.
\end{itemize}

Démonstration Si f est une translation de vecteur
\overrightarrow\xi, on a f(x
+\overrightarrow \eta) = x
+\overrightarrow \eta +\overrightarrow
\xi = f(x) +\overrightarrow \eta ce qui montre que
\vecf = \mathrmId. Inversement, si
\vecf = \mathrmId, soit a \in E et
\overrightarrow\xi =\overrightarrow
af(a)~; alors f(x) = f(a) +\vec
f(\overrightarrowax) = f(a)
+\overrightarrow ax = a
+\overrightarrow \xi +\overrightarrow
ax = x +\overrightarrow \xi ce qui montre que f est
la translation t_\overrightarrow\xi.

Si f = h_a,\lambda~, la définition même montre que
\vecf = \lambda~\mathrmId. Inversement,
si \vecf = \lambda~\mathrmId avec
\lambda~\neq~1, \vecf
-\mathrmId est bi\\jmathmathective et le même raisonnement que
ci dessus montre que f a un unique point fixe a. Mais alors f(x) = f(a)
+\vec f(\overrightarrowax) = a + \lambda~
\overrightarrowax, donc f = h_a,\lambda~.

Remarque~17.1.7 Considérons le groupe GA(E) des applications affines
bi\\jmathmathectives de E dans E~; on dispose de l'application
f\mapsto~\vecf de GA(E) dans
GL(E) qui est visiblement un morphisme de groupes. L'ensemble constitué
des homothéties et des translations est l'image réciproque par cette
application du sous-groupe K^∗\mathrmId de
GL(E). Il s'agit donc d'un sous-groupe de GA(E). On peut préciser ce
résultat en utilisant \overrightarrowg \cdot f
=\vec g \cdot\vec f~:la composée de
deux translations est une translation, la composée d'une homothétie et
d'une translation (dans n'importe quel ordre) est une homothétie, la
composée de deux homothéties est en général une homothétie, à moins que
le produit des deux rapports soit égal à 1 auquel cas la composée est
une translation.

\paragraph{17.1.6 Utilisation de repères affines}

Théorème~17.1.11 Soit
(a,(\vece_i)_i\inI) un repère affine
de l'espace affine E. Soit b un point de l'espace affine F et
(\vecu_i)_i\inI une famille de
vecteurs de \overrightarrowF. Alors il existe une
unique application affine f : E \rightarrow~ F vérifiant f(a) = b et
\forall~~i \in I,
\vecf(\vece_i)
=\vec u_i.

Démonstration On a en effet nécessairement f(x) = f(a
+ \\sum ~
x_i\vece_i) = f(a)
+ \\sum ~
x_i\vecf(\vece_i)
= b + \\sum ~
_i\inIx_i\vecu_i. Inversement,
il est clair que f ainsi définie convient.

Corollaire~17.1.12 Soit (a_i)_i\inI une base affine de E
et (b_i)_i\inI une famille de points de F. Il existe une
unique application affine f : E \rightarrow~ F vérifiant
\forall~i \in I,f(a_i) = b_i~.

Démonstration En effet, si i_0 \in I,
(a_i_0,(\overrightarrowa_i_0a_i)_i\inI\diagdown\i_0\)
est un repère de E et

\begin{align*} \forall~~i \in
I,f(a_i) = b_i&& \%&
\\ & \Leftrightarrow &
f(a_i_0) =
b_i_0\text et
\forall~~i \in I
\diagdown\i_0\,
\overrightarrowf(a_i_0)f(a_i)
=\overrightarrow
b_i_0b_i\%&
\\ & \Leftrightarrow &
f(a_i_0) =
b_i_0\text et
\forall~~i \in I
\diagdown\i_0\,
\vecf(\overrightarrowa_i_0a_i)
=\overrightarrow b_i_0b_i
\%& \\ \end{align*}

ce qui ramène au problème précédent.

Supposons E et F de dimensions finies. Soit
(a,(\vece_\\jmathmath)_1\leqi\leqn) un repère
affine de l'espace affine E et
(b,(\vecf_i)_1\leqi\leqp) un repère
affine de F, soit u : E \rightarrow~ F une application affine. Considérons A =
(a_i,\\jmathmath)_1\leqi\leqp,1\leq\\jmathmath\leqn =\
\mathrmMat
(\vecu,(\vece_\\jmathmath)_1\leq\\jmathmath\leqn,(\vecf_i)_1\leqi\leqp)
la matrice de l' application linéaire \vecu dans les
bases respectives de \overrightarrowE et
\overrightarrowF. On a donc
\vecu(\vece_\\jmathmath)
= \\sum ~
_i=1^pa_i,\\jmathmath\vecf_i.
Posons u(a) = b +\ \\sum

_i=1^p\alpha_i\vecf_i.
Soit x = a + \\sum ~
_\\jmathmath=1^nx_\\jmathmath\vece_\\jmathmath \in E.
On a alors

\begin{align*} u(x)& =& u(a) +
\sum _\\jmathmath=1^nx_
\\jmathmath\vecu(\vece_\\jmathmath) = b +
\sum _i=1^p\alpha~_
i\vecf_i + \\sum
_\\jmathmath=1^nx_ \\jmathmath \\sum
_i=1^pa_
i,\\jmathmath\vecf_i\%&
\\ & =& b + \\sum
_i=1^p\left (\alpha_ i +
\sum _\\jmathmath=1^na_
i,\\jmathmathx_\\jmathmath\right
)\vecf_i \%&
\\ \end{align*}

Donc les coordonnées
y_1,\\ldots,y_p~
de u(x) dans le repère
(b,(\vecf_i)_1\leqi\leqp) de F sont
données par

\forall~i \in {[}1,p{]}, y_i~ =
\sum _\\jmathmath=1^na_
i,\\jmathmathx_\\jmathmath + \alpha_i

Introduisons les vecteurs colonnes X = \left
(\matrix\,x_1
\cr \⋮~
\cr x_n \cr 1
\right ), Y = \left
(\matrix\,y_1
\cr \⋮~
\cr y_p \cr 1
\right ) et A' = \left
(\matrix\,A
&\matrix\,\alpha_1
\cr \⋮~
\cr \alpha_p \cr
\matrix\,0&\\ldots&0~&1
\right ). On a alors

y = u(x) \Leftrightarrow Y = A'X

Définition~17.1.10 On dira que X = \left
(\matrix\,x_1
\cr \⋮~
\cr x_n \cr 1
\right ) est le vecteur colonne des coordonnées du
point x = a + \\sum ~
_\\jmathmath=1^nx_\\jmathmath\vece_\\jmathmath dans
le repère affine
(a,(\vece_\\jmathmath)_1\leq\\jmathmath\leqn) de E. La
matrice A' = \left
(\matrix\,A
&\matrix\,\alpha_1
\cr \⋮~
\cr \alpha_p \cr
\matrix\,0&\\ldots&0~&1
\right ) (où A = (a_i,\\jmathmath)_1\leqi\leqp,1\leq\\jmathmath\leqn
= \mathrmMat~
(\vecu,(\vece_\\jmathmath)_1\leq\\jmathmath\leqn,(\vecf_i)_1\leqi\leqp)
est la matrice de l' application linéaire \vecu dans
les bases respectives de \overrightarrowE et
\overrightarrowF et u(a) = b
+ \\sum ~
_i=1^p\alpha_i\vecf_i)
sera appelée la matrice de l'application affine u dans les repères
(a,(\vece_\\jmathmath)_1\leq\\jmathmath\leqn) et
(b,(\vecf_i)_1\leqi\leqp).

\paragraph{17.1.7 Formes affines et sous-espaces affines}

Définition~17.1.11 On appelle forme affine sur E toute application
affine f de E dans K.

Remarque~17.1.8 Si E est de dimension finie n, soit
(a,(\vece_\\jmathmath)_1\leq\\jmathmath\leqn) un repère
affine de E. L'étude précédente sur la matrice d'une application affine
montre que f : E \rightarrow~ K est une forme affine si et seulement si~elle est de
la forme x = a +\ \\sum

_\\jmathmath=1^nx_\\jmathmath\vece_\\jmathmath\mapsto~\\\sum
 _\\jmathmath=1^na_\\jmathmathx_\\jmathmath + \alpha~.

Théorème~17.1.13 Soit E un espace affine de dimension n et F une partie
de E. Alors on a équivalence de

\begin{itemize}
\itemsep1pt\parskip0pt\parsep0pt
\item
  (i) F est un sous-espace affine de E de dimension p
\item
  (ii) il existe des formes affines
  f_1,\\ldots,f_n-p~
  telles que la famille
  (\vecf_1,\\ldots,\vecf_n-p~)
  soit libre et F = \x \in
  E∣f_1(x) =
  \\ldots~ =
  f_n-p(x) = 0\.
\end{itemize}

Démonstration Soit tout d'abord F un sous-espace affine de dimension p
de direction \overrightarrowF. Soit
(\vecf_1,\\ldots,\vecf_n-p~)
une base de l'orthogonal de \overrightarrowF dans le
dual \overrightarrowE^∗ de l'espace
vectoriel \overrightarrowE. On a alors

\overrightarrow\xi \in\overrightarrow
F \Leftrightarrow \forall~~i \in {[}1,n -
p{]},
\vecf_i(\overrightarrow\xi)
= 0

Soit a \in F et soit f_i la forme affine qui vaut 0 au point a et
d'application linéaire tangente \vecf_i,
c'est-à-dire définie par f_i(x) =\vec
f_i(\overrightarrowax). Alors

\begin{align*} x \in F& \Leftrightarrow
& \overrightarrowax
\in\overrightarrow F \Leftrightarrow
\forall~~i \in {[}1,n - p{]},
\vecf_i(\overrightarrowax)
=\overrightarrow 0\%&
\\ & \Leftrightarrow &
\forall~i \in {[}1,n - p{]}, f_i~(x) = 0 \%&
\\ \end{align*}

ce qui montre que (i) \rigtharrow~(ii).

Inversement, soit a \in E. Nous allons tout d'abord démontrer que si (ii)
est vérifiée alors F n'est pas vide. Pour cela, complétons
(\vecf_1,\\ldots,\vecf_n-p~)
en une base
(\vecf_1,\\ldots,\vecf_n~)
de \overrightarrowE^∗. Soit
(\vece_1,\\ldots,\vece_n~)
une base dont c'est la duale, si bien que
\vecf_i(\vece_\\jmathmath)
= \delta_i^\\jmathmath. Posons b = a
-\\sum ~
_i=1^n-pf_i(a)\vece_i.
On a alors f_\\jmathmath(b) = f_\\jmathmath(a)
-\\sum ~
_i=1^n-pf_i(a)\vecf_\\jmathmath(\vece_i)
= f_\\jmathmath(a) -\\\sum
 _i=1^pf_i(a)\delta_i^\\jmathmath =
f_\\jmathmath(a) - f_\\jmathmath(a) = 0 si bien que b appartient à F. On a
alors

\begin{align*} x \in F& \Leftrightarrow
& \forall~i \in {[}1,n - p{]}, f_i~(x) = 0
\%& \\ & \Leftrightarrow &
\forall~i \in {[}1,n - p{]}, f_i~(b)
+\vec
f_i(\overrightarrowbx) = 0 \%&
\\ & \Leftrightarrow &
\forall~~i \in {[}1,n - p{]},
\vecf_i(\overrightarrowbx)
= 0 \%& \\ &
\Leftrightarrow & \forall~~i \in {[}1,n -
p{]}, \overrightarrowbx \in
(\vecf_1,\\ldots,\vecf_n-p)^\bot~\%&
\\ \end{align*}

On a donc F = b +
(\vecf_1,\\ldots,\vecf_n-p)^\bot~
qui est un sous-espace affine de dimension p.

De fa\ccon plus pratique, dans un repère de E, on en
déduit que F est un sous-espace affine de dimension p si et seulement
si~F est l'ensemble des solutions d'un système de n - p équations
linéaires de rang n - p

x \in F\quad \Leftrightarrow
\quad \left
\\array \alpha_1,1x_1 +
\\ldots~ +
\alpha_1,nx_n & = \beta_1 \cr
&\\ldots~\cr
\alpha_ n-p,1x_1 +
\\ldots~ +
\alpha_n-p,nx_n& = \beta_n-p  \right
.

en posant f_i(x) = \alpha_i,1x_1 +
\\ldots~ +
\alpha_i,nx_n - \beta_i (et donc
\vecf_i(\overrightarrow\xi)
= \alpha_i,1x_1 +
\\ldots~ +
\alpha_i,nx_n).

Exemple~17.1.1 Un hyperplan affine (sous-espace affine de dimension n -
1) est tou\\jmathmathours défini par F = \x \in
E∣f(x) = 0\ où f est une
forme affine telle que
\vecf\neq~0 (soit f non
constante). Autrement dit, avec des coordonnées dans un repère, un
hyperplan est défini par une équation u_1x_1 +
\\ldots~ +
u_nx_n + h = 0 avec
(u_1,\\ldots,u_n)\neq~(0,\\\ldots~,0).

De la même fa\ccon, dans un espace de dimension 3,
une droite est définie par deux équations

\left \\array ax + by +
cz + h& = 0\cr a'x + b'y + c'z + h' & = 0 
\right .

où la matrice \left
(\matrix\,a&b&c\cr a'
&b' &c'\right ) est de rang 2.

Théorème~17.1.14 (faisceaux d'hyperplans). Soit E un espace affine de
dimension finie, H_1 et H_2 deux hyperplans non
parallèles d'équations f_1(x) = 0 et f_2(x) = 0. Alors
les hyperplans contenant H_1 \bigcap H_2 sont exactement les
hyperplans d'équations \lambda_1f_1(x) +
\lambda_2f_2(x) = 0 avec
(\lambda_1,\lambda_2)\neq~(0,0).

Démonstration Les hyperplans vectoriels
\overrightarrowH_1 et
\overrightarrowH_2 sont définis par les
équations
\vecf_1(\overrightarrow\xi)
= 0 et
\vecf_2(\overrightarrow\xi)
= 0. Comme ils sont distincts, ces formes linéaires
\vecf_1 et
\vecf_2 ne sont pas proportionnelles, et
donc
(\vecf_1,\vecf_2)
est libre. On en déduit que F = H_1 \bigcap H_2 =
\x \in E∣f_1(x) =
f_2(x) = 0\ est un sous-espace affine de
dimension n - 2 de E, en particulier F\neq~\varnothing~.
Soit a \in F. Alors

x \in F \Leftrightarrow
\vecf_1(\overrightarrowax)
=\vec
f_2(\overrightarrowax) = 0
\Leftrightarrow \overrightarrowax
\in\mathrmVect(\vecf_1,\vecf_2)^\bot~

si bien que \overrightarrowF
=\
\mathrmVect(\vecf_1,\vecf_2)^\bot
Soit H un hyperplan et f une équation de H. Alors

\begin{align*} F \subset~ H& \Leftrightarrow
& f(a) = 0\text et
\forall~\overrightarrow\xi~ \in F,
\vecf(\overrightarrow\xi) = 0 \%&
\\ & \Leftrightarrow & f(a)
= 0\text et \vecf
\in\left
(\mathrmVect(\vecf_1,\vecf_2)^\bot~\right
)^\bot\%& \\ &
\Leftrightarrow & f(a) = 0\text et
\vecf
\in\mathrmVect(\vecf_1,\vecf_2~)
\%& \\ \end{align*}

Mais comme f_1(a) = f_2(a) = 0, la condition f =
\lambda_1f_1 + \lambda_2f_2 est manifestement
équivalente à f(a) = 0 et \vecf =
\lambda_1\vecf_1 +
\lambda_2\vecf_2 ce qui nous donne le
résultat souhaité.

Exemple~17.1.2 Dans un espace de dimension 3, soit D la droite
d'équations

\left \\array ax + by +
cz + h& = 0\cr a'x + b'y + c'z + h' & = 0 
\right .

où la matrice \left
(\matrix\,a&b&c\cr a'
&b' &c'\right ) est de rang 2. Alors les plans
contenant D sont exactement les plans d'équations \lambda_1(ax + by +
cz + h) + \lambda_2(a'x + b'y + c'z + h') = 0 avec
(\lambda_1,\lambda_2)\neq~(0,0).

De même en dimension 2, si un point A est défini comme intersection de
deux droites non parallèles d'équations ax + by + c = 0 et a'x + b'y +
c' = 0, les droites contenant A sont exactement les droites admettant
comme équations \lambda_1(ax + by + c) + \lambda_2(a'x + b'y + c')
= 0 avec (\lambda_1,\lambda_2)\neq~(0,0).

{[}
{[}

\end{document}

% \documentclass[]{article}
\usepackage[T1]{fontenc}
\usepackage{lmodern}
\usepackage{amssymb,amsmath}
\usepackage{ifxetex,ifluatex}
\usepackage{fixltx2e} % provides \textsubscript
% use upquote if available, for straight quotes in verbatim environments
\IfFileExists{upquote.sty}{\usepackage{upquote}}{}
\ifnum 0\ifxetex 1\fi\ifluatex 1\fi=0 % if pdftex
  \usepackage[utf8]{inputenc}
\else % if luatex or xelatex
  \ifxetex
    \usepackage{mathspec}
    \usepackage{xltxtra,xunicode}
  \else
    \usepackage{fontspec}
  \fi
  \defaultfontfeatures{Mapping=tex-text,Scale=MatchLowercase}
  \newcommand{\euro}{€}
\fi
% use microtype if available
\IfFileExists{microtype.sty}{\usepackage{microtype}}{}
\usepackage{graphicx}
% Redefine \includegraphics so that, unless explicit options are
% given, the image width will not exceed the width of the page.
% Images get their normal width if they fit onto the page, but
% are scaled down if they would overflow the margins.
\makeatletter
\def\ScaleIfNeeded{%
  \ifdim\Gin@nat@width>\linewidth
    \linewidth
  \else
    \Gin@nat@width
  \fi
}
\makeatother
\let\Oldincludegraphics\includegraphics
{%
 \catcode`\@=11\relax%
 \gdef\includegraphics{\@ifnextchar[{\Oldincludegraphics}{\Oldincludegraphics[width=\ScaleIfNeeded]}}%
}%
\ifxetex
  \usepackage[setpagesize=false, % page size defined by xetex
              unicode=false, % unicode breaks when used with xetex
              xetex]{hyperref}
\else
  \usepackage[unicode=true]{hyperref}
\fi
\hypersetup{breaklinks=true,
            bookmarks=true,
            pdfauthor={},
            pdftitle={Barycentres},
            colorlinks=true,
            citecolor=blue,
            urlcolor=blue,
            linkcolor=magenta,
            pdfborder={0 0 0}}
\urlstyle{same}  % don't use monospace font for urls
\setlength{\parindent}{0pt}
\setlength{\parskip}{6pt plus 2pt minus 1pt}
\setlength{\emergencystretch}{3em}  % prevent overfull lines
\setcounter{secnumdepth}{0}
 
/* start css.sty */
.cmr-5{font-size:50%;}
.cmr-7{font-size:70%;}
.cmmi-5{font-size:50%;font-style: italic;}
.cmmi-7{font-size:70%;font-style: italic;}
.cmmi-10{font-style: italic;}
.cmsy-5{font-size:50%;}
.cmsy-7{font-size:70%;}
.cmex-7{font-size:70%;}
.cmex-7x-x-71{font-size:49%;}
.msbm-7{font-size:70%;}
.cmtt-10{font-family: monospace;}
.cmti-10{ font-style: italic;}
.cmbx-10{ font-weight: bold;}
.cmr-17x-x-120{font-size:204%;}
.cmsl-10{font-style: oblique;}
.cmti-7x-x-71{font-size:49%; font-style: italic;}
.cmbxti-10{ font-weight: bold; font-style: italic;}
p.noindent { text-indent: 0em }
td p.noindent { text-indent: 0em; margin-top:0em; }
p.nopar { text-indent: 0em; }
p.indent{ text-indent: 1.5em }
@media print {div.crosslinks {visibility:hidden;}}
a img { border-top: 0; border-left: 0; border-right: 0; }
center { margin-top:1em; margin-bottom:1em; }
td center { margin-top:0em; margin-bottom:0em; }
.Canvas { position:relative; }
li p.indent { text-indent: 0em }
.enumerate1 {list-style-type:decimal;}
.enumerate2 {list-style-type:lower-alpha;}
.enumerate3 {list-style-type:lower-roman;}
.enumerate4 {list-style-type:upper-alpha;}
div.newtheorem { margin-bottom: 2em; margin-top: 2em;}
.obeylines-h,.obeylines-v {white-space: nowrap; }
div.obeylines-v p { margin-top:0; margin-bottom:0; }
.overline{ text-decoration:overline; }
.overline img{ border-top: 1px solid black; }
td.displaylines {text-align:center; white-space:nowrap;}
.centerline {text-align:center;}
.rightline {text-align:right;}
div.verbatim {font-family: monospace; white-space: nowrap; text-align:left; clear:both; }
.fbox {padding-left:3.0pt; padding-right:3.0pt; text-indent:0pt; border:solid black 0.4pt; }
div.fbox {display:table}
div.center div.fbox {text-align:center; clear:both; padding-left:3.0pt; padding-right:3.0pt; text-indent:0pt; border:solid black 0.4pt; }
div.minipage{width:100%;}
div.center, div.center div.center {text-align: center; margin-left:1em; margin-right:1em;}
div.center div {text-align: left;}
div.flushright, div.flushright div.flushright {text-align: right;}
div.flushright div {text-align: left;}
div.flushleft {text-align: left;}
.underline{ text-decoration:underline; }
.underline img{ border-bottom: 1px solid black; margin-bottom:1pt; }
.framebox-c, .framebox-l, .framebox-r { padding-left:3.0pt; padding-right:3.0pt; text-indent:0pt; border:solid black 0.4pt; }
.framebox-c {text-align:center;}
.framebox-l {text-align:left;}
.framebox-r {text-align:right;}
span.thank-mark{ vertical-align: super }
span.footnote-mark sup.textsuperscript, span.footnote-mark a sup.textsuperscript{ font-size:80%; }
div.tabular, div.center div.tabular {text-align: center; margin-top:0.5em; margin-bottom:0.5em; }
table.tabular td p{margin-top:0em;}
table.tabular {margin-left: auto; margin-right: auto;}
div.td00{ margin-left:0pt; margin-right:0pt; }
div.td01{ margin-left:0pt; margin-right:5pt; }
div.td10{ margin-left:5pt; margin-right:0pt; }
div.td11{ margin-left:5pt; margin-right:5pt; }
table[rules] {border-left:solid black 0.4pt; border-right:solid black 0.4pt; }
td.td00{ padding-left:0pt; padding-right:0pt; }
td.td01{ padding-left:0pt; padding-right:5pt; }
td.td10{ padding-left:5pt; padding-right:0pt; }
td.td11{ padding-left:5pt; padding-right:5pt; }
table[rules] {border-left:solid black 0.4pt; border-right:solid black 0.4pt; }
.hline hr, .cline hr{ height : 1px; margin:0px; }
.tabbing-right {text-align:right;}
span.TEX {letter-spacing: -0.125em; }
span.TEX span.E{ position:relative;top:0.5ex;left:-0.0417em;}
a span.TEX span.E {text-decoration: none; }
span.LATEX span.A{ position:relative; top:-0.5ex; left:-0.4em; font-size:85%;}
span.LATEX span.TEX{ position:relative; left: -0.4em; }
div.float img, div.float .caption {text-align:center;}
div.figure img, div.figure .caption {text-align:center;}
.marginpar {width:20%; float:right; text-align:left; margin-left:auto; margin-top:0.5em; font-size:85%; text-decoration:underline;}
.marginpar p{margin-top:0.4em; margin-bottom:0.4em;}
.equation td{text-align:center; vertical-align:middle; }
td.eq-no{ width:5%; }
table.equation { width:100%; } 
div.math-display, div.par-math-display{text-align:center;}
math .texttt { font-family: monospace; }
math .textit { font-style: italic; }
math .textsl { font-style: oblique; }
math .textsf { font-family: sans-serif; }
math .textbf { font-weight: bold; }
.partToc a, .partToc, .likepartToc a, .likepartToc {line-height: 200%; font-weight:bold; font-size:110%;}
.chapterToc a, .chapterToc, .likechapterToc a, .likechapterToc, .appendixToc a, .appendixToc {line-height: 200%; font-weight:bold;}
.index-item, .index-subitem, .index-subsubitem {display:block}
.caption td.id{font-weight: bold; white-space: nowrap; }
table.caption {text-align:center;}
h1.partHead{text-align: center}
p.bibitem { text-indent: -2em; margin-left: 2em; margin-top:0.6em; margin-bottom:0.6em; }
p.bibitem-p { text-indent: 0em; margin-left: 2em; margin-top:0.6em; margin-bottom:0.6em; }
.subsectionHead, .likesubsectionHead { margin-top:2em; font-weight: bold;}
.sectionHead, .likesectionHead { font-weight: bold;}
.quote {margin-bottom:0.25em; margin-top:0.25em; margin-left:1em; margin-right:1em; text-align:justify;}
.verse{white-space:nowrap; margin-left:2em}
div.maketitle {text-align:center;}
h2.titleHead{text-align:center;}
div.maketitle{ margin-bottom: 2em; }
div.author, div.date {text-align:center;}
div.thanks{text-align:left; margin-left:10%; font-size:85%; font-style:italic; }
div.author{white-space: nowrap;}
.quotation {margin-bottom:0.25em; margin-top:0.25em; margin-left:1em; }
h1.partHead{text-align: center}
.sectionToc, .likesectionToc {margin-left:2em;}
.subsectionToc, .likesubsectionToc {margin-left:4em;}
.sectionToc, .likesectionToc {margin-left:6em;}
.frenchb-nbsp{font-size:75%;}
.frenchb-thinspace{font-size:75%;}
.figure img.graphics {margin-left:10%;}
/* end css.sty */

\title{Barycentres}
\author{}
\date{}

\begin{document}
\maketitle

\textbf{Warning: 
requires JavaScript to process the mathematics on this page.\\ If your
browser supports JavaScript, be sure it is enabled.}

\begin{center}\rule{3in}{0.4pt}\end{center}

[
[
[]
[

\section{17.2 Barycentres}

\subsection{17.2.1 Notion de barycentres}

Théorème~17.2.1 Soit E un espace affine, (a_i)_i\inI une
famille finie de points de E et (\lambda_i)_i\inI une famille
de scalaires telle que
\\sum ~
_i\inI\lambda_i\neq~0. Alors il existe un
unique point g \in E vérifiant les conditions équivalentes

\begin{itemize}
\itemsep1pt\parskip0pt\parsep0pt
\item
  (i) \\sum ~
  _i\inI\lambda_i\overrightarrowga_i
  =\overrightarrow 0
\item
  (ii) \forall~~m \in E,
  \\sum ~
  _i\inI\lambda_i\overrightarrowma_i
  = (\\sum ~
  _i\inI\lambda_i)\overrightarrowmg
\item
  (iii) il existe a \in E tel que \overrightarrowag =
  1 \over
  \\sum ~
  _i\inI\lambda_i \
  \sum ~
  _i\inI\lambda_i\overrightarrowaa_i
\end{itemize}

On dit alors que g est le barycentre des points a_i affectés
des coefficients \lambda_i.

Démonstration Il est clair que (ii) \rigtharrow~(i) (prendre m = g) et que (ii)
\rigtharrow~(iii) (prendre m = a et diviser par
\\sum ~
_i\inI\lambda_i). Si maintenant (i) est vérifié, on a, pour m \in
E,

\begin{align*} \\sum
_i\inI\lambda_i\overrightarrowma_i&
=& \\sum
_i\inI\lambda_i(\overrightarrowmg
+\overrightarrow ga_i) =
(\\sum
_i\inI\lambda_i)\overrightarrowmg +
\\sum
_i\inI\lambda_i\overrightarrowga_i\%&
\\ & =& (\\sum
_i\inI\lambda_i)\overrightarrowmg \%&
\\ \end{align*}

et donc (ii) est vérifiée. De la même fa\ccon, si
(iii) est vérifiée, on a
\\sum ~
_i\inI\lambda_i\overrightarrowaa_i =
(\\sum ~
_i\inI\lambda_i)\overrightarrowag et donc,
pour m \in E,

\begin{align*} \\sum
_i\inI\lambda_i\overrightarrowma_i&
=& \\sum
_i\inI\lambda_i(\overrightarrowma
+\overrightarrow aa_i) =
(\\sum
_i\inI\lambda_i)\overrightarrowma +
\\sum
_i\inI\lambda_i\overrightarrowaa_i\%&
\\ & =& (\\sum
_i\inI\lambda_i)\overrightarrowma +
(\\sum
_i\inI\lambda_i)\overrightarrowag =
(\\sum
_i\inI\lambda_i)\overrightarrowmg \%&
\\ \end{align*}

et donc (ii) est vérifiée, ce qui achève la démonstration.

Remarque~17.2.1 Si \\\sum
 _i\inI\lambda_i = 0, on vérifie facilement que
\\sum ~
_i\inI\lambda_i\overrightarrowma_i
est un vecteur \vecu indépendant de m~; c'est parfois
ce vecteur qu'on appelle barycentre de la famille lorsque
\\sum ~
_i\inI\lambda_i = 0. Il s'agit alors d'un vecteur et non plus
d'un point.

Définition~17.2.1 On appelle point massique de E tout couple (a,\lambda~) d'un
point a \in E et d'un scalaire \lambda~ \in K.

Remarque~17.2.2 On dira indifféremment que g est est le barycentre des
points a_i affectés des coefficients \lambda_i ou que g est
le barycentre de la famille de points massiques \left
((a_i,\lambda_i)\right )_i\inI.

\subsection{17.2.2 Associativité des barycentres}

Théorème~17.2.2 Soit \left
((a_i,\lambda_i)\right )_i\inI une
famille de points massiques telle que
\\sum ~
_i\inI\lambda_i\neq~0 et g son
barycentre. Soit I = I_1
\cup\\ldots~ \cup
I_k une partition de I telle que \forall~~j \in
[1,k], \mu_j =\
\sum ~
_i\inI_j\lambda_i\neq~0. Soit
g_j le barycentre de la famille de points massiques
\left ((a_i,\lambda_i)\right
)_i\inI_j. Alors g est le barycentre des points
g_1,\\ldots,g_k~
affectés des coefficients
\mu_1,\\ldots,\mu_k~.

Démonstration On a

\overrightarrow0 = \\sum
_i\inI\lambda_i\overrightarrowga_i
= \sum _j=1^k~
\\sum
_i\inI_j\lambda_i\overrightarrowga_i

Mais d'après la définition de g_j, on a
\\sum ~
_i\inI_j\lambda_i\overrightarrowga_i
= \left
(\\sum ~
_i\inI_j\lambda_i\right
)\overrightarrowgg_j =
\mu_j\overrightarrowgg_j. On a donc
\overrightarrow0 =\
\sum ~
_j=1^k\mu_j\overrightarrowgg_j
ce qui démontre que g est le barycentre des points
g_1,\\ldots,g_k~
affectés des coefficients
\mu_1,\\ldots,\mu_k~.

Exemple~17.2.1 Si
(a_1,\\ldots,a_n~)
est une famille de points de E et si la caractéristique p de K (le corps
de base) ne divise pas n, on peut définir le barycentre des points
a_1,\\ldots,a_n~
tous affectés du coefficient 1 (puisque
n1_K\neq~0)~; on appelle ce point
l'isobarycentre des points
a_1,\\ldots,a_n~.
Notons le g. Soit [1,n] = I_1 \cup I_2 une partition
de [1,n] avec k = I_1 et n - k =
I_2. Supposons que p ne divise ni k ni n -
k et soit m_1 l'isobarycentre des a_i,i \in
I_1, m_2 l'isobarycentre des a_i,i \in
I_2. Alors le théorème d'associativité des barycentres assure
que g est aussi le barycentre de (m_1,k) et (m_2,n -
k), et en particulier g appartient à la droite m_1m_2.
Dans le cas où n = 3, on montre ainsi que sur un corps de
caractéristique différente de 2 ou 3, les droites qui joignent un sommet
du triangle au milieu du coté opposé qui n'est autre que l'isobarycentre
de ces deux points (ces droites sont les médianes du triangle)
contiennent toutes l'isobarycentre des sommets du triangle (le centre de
gravité du triangle), autrement dit ces trois médianes sont
concourantes. De même, en dimension 3 et pour n = 4, les quatre droites
joignant un sommet au centre de gravité de la face opposée et les trois
droites joignant les milieux de deux arêtes opposées passent toutes par
le centre de gravité du tétraèdre.

\includegraphics{cours11x.png}

\subsection{17.2.3 Barycentres, sous-espaces affines, applications
affines}

Les deux théorèmes suivants montrent que le barycentrage est l'opération
algébrique fondamentale dans les espaces affines.

Théorème~17.2.3 Soit f : E \rightarrow~ F une application d'un espace affine dans
un autre espace affine. Alors f est affine si et seulement si~elle
conserve la notion de barycentre, c'est-à-dire si et seulement si~pour
toute famille \left
((a_i,\lambda_i)\right )_i\inI telle que
\\sum ~
_i\inI\lambda_i\neq~0, de barycentre g,
le point f(g) est le barycentre de la famille de points massiques
\left
((f(a_i),\lambda_i)\right )_i\inI.

Démonstration Si f est affine, on a en effet

\overrightarrow0 =\vec
f(\\sum
_i\inI\lambda_i\overrightarrowga_i)
= \\sum
_i\inI\lambda_i\vecf(\overrightarrowga_i)
= \\sum
_i\inI\lambda_i\overrightarrowf(g)f(a_i)

ce qui montre que le point f(g) est le barycentre de la famille de
points massiques \left
((f(a_i),\lambda_i)\right )_i\inI.
Inversement, supposons que f vérifie cette propriété et montrons que f
est affine. Pour cela, soit a \in E et posons
\vecf(\overrightarrow\xi)
=\overrightarrow f(a)f(a
+\overrightarrow \xi). Il nous suffit donc de montrer
que \vecf est linéaire.

Pour cela soit tout d'abord \overrightarrow\xi
\in\overrightarrow E et \lambda~ \in K. Posons b = a
+\overrightarrow \xi et c = a +
\lambda~\overrightarrow\xi. On a donc
\overrightarrowac -
\lambda~\overrightarrowab =\overrightarrow
0, soit encore (1 - \lambda~)\overrightarrowac +
\lambda~\overrightarrowbc =\overrightarrow
0 (en écrivant \overrightarrowab
=\overrightarrow ac +\overrightarrow
cb). Donc c est le barycentre de (a,1 - \lambda~) et (b,\lambda~). On en déduit que
f(c) est le barycentre de (f(a),1 - \lambda~) et de (f(b),\lambda~), soit encore que
(1 - \lambda~)\overrightarrowf(a)f(c) +
\lambda~\overrightarrowf(b)f(c)
=\overrightarrow 0, soit encore
\overrightarrowf(a)f(c) -
\lambda~\overrightarrowf(a)f(b)
=\overrightarrow 0, ce qui se traduit par
\vecf(\lambda~\overrightarrow\xi) =
\lambda~\vecf(\overrightarrow\xi).

Soit maintenant, \overrightarrow\xi et
\overrightarrow\eta dans
\overrightarrowE. Posons b = a
+\overrightarrow \xi, c = a
+\overrightarrow \eta et d = a +
(\overrightarrow\xi +\overrightarrow
\eta). On a alors -\overrightarrow ad
+\overrightarrow ab +\overrightarrow
ac =\overrightarrow 0 si bien que a est barycentre
de (d,-1), (b,1) et (c,1). On en déduit que f(a) est barycentre de
(f(d),-1), (f(b),1) et (f(c),1), si bien que

-\overrightarrowf(a)f(d)
+\overrightarrow f(a)f(b)
+\overrightarrow f(a)f(c)
=\overrightarrow 0

ce qui se traduit par -\vec
f(\overrightarrow\xi
+\overrightarrow \eta) +\vec
f(\overrightarrow\xi) +\vec
f(\overrightarrow\eta)
=\overrightarrow 0. Donc \vecf est
bien linéaire.

Théorème~17.2.4 Soit F une partie non vide d'un espace affine E. Alors F
est un sous-espace affine si et seulement si~il est stable par
barycentrage, c'est-à-dire si et seulement si~pour toute famille
\left ((a_i,\lambda_i)\right
)_i\inI telle que a_i \in F et
\\sum ~
_i\inI\lambda_i\neq~0 de barycentre g, le
point g est encore dans F.

Démonstration Supposons que F est un sous-espace affine et soit a \in F.
Alors \overrightarrowag = 1 \over
\\sum ~
_i\inI\lambda_i \
\sum ~
_i\inI\lambda_i\overrightarrowaa_i
\in\overrightarrow F puisque chacun des
\overrightarrowaa_i est dans l'espace
vectoriel \overrightarrowF. Donc g \in F.

Inversement, supposons que F est stable par barycentrage et soit a \in F,
\overrightarrowF =
\\overrightarrow\xi
\in\overrightarrow E∣a
+\overrightarrow \xi \in F\. Il suffit
de montrer que \overrightarrowF est un sous-espace
vectoriel de \overrightarrowE.

Soit tout d'abord \overrightarrow\xi
\in\overrightarrow F et \lambda~ \in K. Posons b = a
+\overrightarrow \xi \in F et c = a +
\lambda~\overrightarrow\xi. On a donc
\overrightarrowac -
\lambda~\overrightarrowab =\overrightarrow
0, soit encore (1 - \lambda~)\overrightarrowac +
\lambda~\overrightarrowbc =\overrightarrow
0 (en écrivant \overrightarrowab
=\overrightarrow ac +\overrightarrow
cb). Donc c est le barycentre de (a,1 - \lambda~) et (b,\lambda~). Donc c \in F, soit
\lambda~\overrightarrow\xi \in\overrightarrow
F.

Soit maintenant, \overrightarrow\xi et
\overrightarrow\eta dans
\overrightarrowF. Posons b = a
+\overrightarrow \xi \in F, c = a
+\overrightarrow \eta \in F et d = a +
(\overrightarrow\xi +\overrightarrow
\eta). On a alors -\overrightarrow ad
+\overrightarrow ab +\overrightarrow
ac =\overrightarrow 0, ou encore
\overrightarrowad -\overrightarrow
bd -\overrightarrow cd
=\overrightarrow 0 (en utilisant la relation de
Chasles). Donc d est le barycentre de (a,1), (b,-1) et (c,-1). On a donc
d \in F, soit encore \overrightarrow\xi
+\overrightarrow \eta \in\overrightarrow
F, ce qui achève de montrer que \overrightarrowF
est un sous-espace vectoriel de \overrightarrowE, et
donc F un sous-espace affine de E.

\subsection{17.2.4 Barycentres et convexité}

On supposera ici que le corps de base est \mathbb{R}~.

Définition~17.2.2 Soit E un espace affine sur \mathbb{R}~, a et b deux points de
E. On appelle segment [a,b] l'ensemble des barycentres des points a
et b affectés des coefficients t et 1 - t pour t \in [0,1].

Remarque~17.2.3 Autrement dit [a,b] = \a +
t\overrightarrowab∣t \in
[0,1]\.

Définition~17.2.3 Soit E un espace affine sur \mathbb{R}~ et A une partie de E. On
dit que A est convexe si \forall~~a,b \in A, [a,b] \subset~
A.

Théorème~17.2.5 Une partie A de E est convexe si et seulement si tout
barycentre à coefficients positifs d'une famille finie de points de A
est encore dans A.

Démonstration La condition est évidemment suffisante puisque tout point
du segment [a,b] est barycentre à coefficients positifs de a et b.
Montrons qu'elle est nécessaire en montrant par récurrence sur
I que si \left
((a_i,\lambda_i)\right )_i\inI est une
famille finie de points massiques tels que \forall~~i,
a_i \in A et \lambda_i ≥ 0 avec
\\sum ~
_i\inI\lambda_i\neq~0, alors le
barycentre g de la famille est encore dans A. Si I =
2, g est encore le barycentre de (a, \lambda_1 \over
\lambda_1+\lambda_2 ) et de (a, \lambda_2
\over \lambda_1+\lambda_2 ), soit encore de (a,t)
et (b,1 - t) pour t = \lambda_1 \over
\lambda_1+\lambda_2 \in [0,1]~; donc on a g \in [a,b] \subset~ A.
Supposons maintenant le résultat démontré pour toute famille de cardinal
n - 1 et soit I = n. Soit i_0 \in I et
supposons que \\sum ~
_i\inI\diagdown\i_0\\lambda_i\neq~0~;
soit g' le barycentre de la famille \left
((a_i,\lambda_i)\right
)_i\inI\diagdown\i_0\~; d'après
l'hypothèse de récurrence, g' \in A. Mais le théorème d'associativité des
barycentres assure que g est le barycentre de
(g',\\sum ~
_i\inI\diagdown\i_0\\lambda_i)
et de (a_i_0,\lambda_i_0)~; donc, d'après
le cas n = 2, g \in [g',a_i_0] \subset~ A. Si par contre
\\sum ~
_i\inI\diagdown\i_0\\lambda_i
= 0, comme les \lambda_i sont positifs on a
\forall~~i \in I
\diagdown\i_0\, \lambda_i = 0 et g
n'est autre que a_i_0 \in A. Cela achève la
démonstration.

Remarque~17.2.4 Il est clair que toute intersection de parties convexes
est encore convexe. On en déduit qu'étant donnée une partie A de E,
l'intersection de tous les convexes contenant A est encore une partie
convexe contenant A, et qu'elle est contenue dans toute partie convexe
contenant A. Nous l'appellerons l'enveloppe convexe de A et la noterons
\hatA.

Théorème~17.2.6 L'enveloppe convexe \hatA de A est
l'ensemble des barycentres à coefficients positifs de points de A.

Démonstration Soit B l'ensemble des barycentres à coefficients positifs
de points de A. Comme \hatA est convexe, elle doit
contenir d'après le théorème précédent, tout barycentre à coefficients
positifs de points de \hatA et en particulier de
points de A, soit B \subset~\hat A. Mais B est convexe car
tout barycentre à coefficients positifs de points de B qui sont eux
mêmes des barycentres à coefficients positifs de points de A est,
d'après le théorème d'associativité des barycentres, un barycentre à
coefficients positifs de points de A, donc est dans B~; comme B contient
évidemment A et que \hatA est contenue dans tout
convexe contenant A, on a \hatA \subset~ B et en définitive
B =\hat A.

Le théorème suivant précise ce résultat en dimension finie

Théorème~17.2.7 (Carathéodory). Soit E un \mathbb{R}~- espace affine de dimension
n. Alors l'enveloppe convexe \hatA de A est
l'ensemble des barycentres à coefficients positifs des familles de
points de A de cardinal n + 1.

Démonstration Soit \left
((a_i,\lambda_i)\right )_1\leqi\leqp une
famille de points massiques avec \forall~~i,
a_i \in A et \lambda_i ≥ 0. Si p \leq n + 1, il est évidemment
toujours possible de compléter la famille en une famille de cardinal n +
1 avec des poids nuls. Si p > n + 1, nous allons montrer
que le barycentre g de la famille est aussi le barycentre d'une famille
\left ((a_i,\mu_i)\right
)_1\leqi\leqp, i\neq~i_0 avec les
\mu_i ≥ 0. Pour cela remarquons que la famille
(\overrightarrowa_pa_i)_1\leqi\leqp-1
est une famille de p - 1 > n vecteurs dans un espace de
dimension n~; elle est donc liée. En conséquence, il existe
\alpha_1,\\ldots,\alpha_p-1~
non tous nuls tels que
\\sum ~
_i=1^p-1\alpha_i\overrightarrowa_pa_i
= 0~; en posant \alpha_p =
-\\sum ~
_i=1^p-1\alpha_i et en écrivant
\overrightarrowa_pa_i
=\overrightarrow ga_i
-\overrightarrow ga_p, on obtient
\\sum ~
_i=1^p\alpha_i\overrightarrowga_i
=\overrightarrow 0 avec
\\sum ~
_i\alpha_i = 0, et les \alpha_i non tous nuls~; en
particulier l'un au moins des \alpha_i est strictement positif. Par
définition, on a \\\sum

_i=1^p\lambda_i\overrightarrowga_i
=\overrightarrow 0. On en déduit que pour tout réel
t, on a

\sum _i=1^p(\lambda_ i~ -
t\alpha_i)\overrightarrowga_i
=\overrightarrow 0

Prenons alors t = min~\
\lambda_i \over \alpha_i
∣\alpha_i >
0\ et soit \mu_i = \lambda_i - t\alpha_i.
On a t ≥ 0~; comme \lambda_i ≥ 0 deux cas sont possibles~:

\begin{itemize}
\itemsep1pt\parskip0pt\parsep0pt
\item
  si \alpha_i \leq 0, alors - t\alpha_i ≥ 0 et donc \mu_i =
  \lambda_i - t\alpha_i ≥ 0
\item
  si par contre \alpha_i > 0, alors t \leq \lambda_i
  \over \alpha_i soit encore \mu_i =
  \lambda_i - t\alpha_i ≥ 0.
\end{itemize}

On a donc \forall~i \in [1,p], \mu_i~ ≥ 0 et
\\sum ~
_i=1^p\mu_i\overrightarrowga_i
=\overrightarrow 0. Mais de plus t
= min\ \lambda_i~
\over \alpha_i
∣\alpha_i >
0\ = \lambda_i_0 \over
\alpha_i_0 , ce qui nous donne \mu_i_0 =
0. Donc g est encore le barycentre de la famille \left
((a_i,\mu_i)\right )_1\leqi\leqp,
i\neq~i_0 avec les \mu_i ≥ 0~;
ce qui achève la démonstration~: tant que le cardinal de la famille est
supérieur ou égal à n + 2 on peut retirer un point de la famille, donc
on finit par aboutir à une famille de cardinal n + 1.

[
[
[
[

\end{document}

% \documentclass[]{article}
\usepackage[T1]{fontenc}
\usepackage{lmodern}
\usepackage{amssymb,amsmath}
\usepackage{ifxetex,ifluatex}
\usepackage{fixltx2e} % provides \textsubscript
% use upquote if available, for straight quotes in verbatim environments
\IfFileExists{upquote.sty}{\usepackage{upquote}}{}
\ifnum 0\ifxetex 1\fi\ifluatex 1\fi=0 % if pdftex
  \usepackage[utf8]{inputenc}
\else % if luatex or xelatex
  \ifxetex
    \usepackage{mathspec}
    \usepackage{xltxtra,xunicode}
  \else
    \usepackage{fontspec}
  \fi
  \defaultfontfeatures{Mapping=tex-text,Scale=MatchLowercase}
  \newcommand{\euro}{€}
\fi
% use microtype if available
\IfFileExists{microtype.sty}{\usepackage{microtype}}{}
\ifxetex
  \usepackage[setpagesize=false, % page size defined by xetex
              unicode=false, % unicode breaks when used with xetex
              xetex]{hyperref}
\else
  \usepackage[unicode=true]{hyperref}
\fi
\hypersetup{breaklinks=true,
            bookmarks=true,
            pdfauthor={},
            pdftitle={Espaces affines euclidiens},
            colorlinks=true,
            citecolor=blue,
            urlcolor=blue,
            linkcolor=magenta,
            pdfborder={0 0 0}}
\urlstyle{same}  % don't use monospace font for urls
\setlength{\parindent}{0pt}
\setlength{\parskip}{6pt plus 2pt minus 1pt}
\setlength{\emergencystretch}{3em}  % prevent overfull lines
\setcounter{secnumdepth}{0}
 
/* start css.sty */
.cmr-5{font-size:50%;}
.cmr-7{font-size:70%;}
.cmmi-5{font-size:50%;font-style: italic;}
.cmmi-7{font-size:70%;font-style: italic;}
.cmmi-10{font-style: italic;}
.cmsy-5{font-size:50%;}
.cmsy-7{font-size:70%;}
.cmex-7{font-size:70%;}
.cmex-7x-x-71{font-size:49%;}
.msbm-7{font-size:70%;}
.cmtt-10{font-family: monospace;}
.cmti-10{ font-style: italic;}
.cmbx-10{ font-weight: bold;}
.cmr-17x-x-120{font-size:204%;}
.cmsl-10{font-style: oblique;}
.cmti-7x-x-71{font-size:49%; font-style: italic;}
.cmbxti-10{ font-weight: bold; font-style: italic;}
p.noindent { text-indent: 0em }
td p.noindent { text-indent: 0em; margin-top:0em; }
p.nopar { text-indent: 0em; }
p.indent{ text-indent: 1.5em }
@media print {div.crosslinks {visibility:hidden;}}
a img { border-top: 0; border-left: 0; border-right: 0; }
center { margin-top:1em; margin-bottom:1em; }
td center { margin-top:0em; margin-bottom:0em; }
.Canvas { position:relative; }
li p.indent { text-indent: 0em }
.enumerate1 {list-style-type:decimal;}
.enumerate2 {list-style-type:lower-alpha;}
.enumerate3 {list-style-type:lower-roman;}
.enumerate4 {list-style-type:upper-alpha;}
div.newtheorem { margin-bottom: 2em; margin-top: 2em;}
.obeylines-h,.obeylines-v {white-space: nowrap; }
div.obeylines-v p { margin-top:0; margin-bottom:0; }
.overline{ text-decoration:overline; }
.overline img{ border-top: 1px solid black; }
td.displaylines {text-align:center; white-space:nowrap;}
.centerline {text-align:center;}
.rightline {text-align:right;}
div.verbatim {font-family: monospace; white-space: nowrap; text-align:left; clear:both; }
.fbox {padding-left:3.0pt; padding-right:3.0pt; text-indent:0pt; border:solid black 0.4pt; }
div.fbox {display:table}
div.center div.fbox {text-align:center; clear:both; padding-left:3.0pt; padding-right:3.0pt; text-indent:0pt; border:solid black 0.4pt; }
div.minipage{width:100%;}
div.center, div.center div.center {text-align: center; margin-left:1em; margin-right:1em;}
div.center div {text-align: left;}
div.flushright, div.flushright div.flushright {text-align: right;}
div.flushright div {text-align: left;}
div.flushleft {text-align: left;}
.underline{ text-decoration:underline; }
.underline img{ border-bottom: 1px solid black; margin-bottom:1pt; }
.framebox-c, .framebox-l, .framebox-r { padding-left:3.0pt; padding-right:3.0pt; text-indent:0pt; border:solid black 0.4pt; }
.framebox-c {text-align:center;}
.framebox-l {text-align:left;}
.framebox-r {text-align:right;}
span.thank-mark{ vertical-align: super }
span.footnote-mark sup.textsuperscript, span.footnote-mark a sup.textsuperscript{ font-size:80%; }
div.tabular, div.center div.tabular {text-align: center; margin-top:0.5em; margin-bottom:0.5em; }
table.tabular td p{margin-top:0em;}
table.tabular {margin-left: auto; margin-right: auto;}
div.td00{ margin-left:0pt; margin-right:0pt; }
div.td01{ margin-left:0pt; margin-right:5pt; }
div.td10{ margin-left:5pt; margin-right:0pt; }
div.td11{ margin-left:5pt; margin-right:5pt; }
table[rules] {border-left:solid black 0.4pt; border-right:solid black 0.4pt; }
td.td00{ padding-left:0pt; padding-right:0pt; }
td.td01{ padding-left:0pt; padding-right:5pt; }
td.td10{ padding-left:5pt; padding-right:0pt; }
td.td11{ padding-left:5pt; padding-right:5pt; }
table[rules] {border-left:solid black 0.4pt; border-right:solid black 0.4pt; }
.hline hr, .cline hr{ height : 1px; margin:0px; }
.tabbing-right {text-align:right;}
span.TEX {letter-spacing: -0.125em; }
span.TEX span.E{ position:relative;top:0.5ex;left:-0.0417em;}
a span.TEX span.E {text-decoration: none; }
span.LATEX span.A{ position:relative; top:-0.5ex; left:-0.4em; font-size:85%;}
span.LATEX span.TEX{ position:relative; left: -0.4em; }
div.float img, div.float .caption {text-align:center;}
div.figure img, div.figure .caption {text-align:center;}
.marginpar {width:20%; float:right; text-align:left; margin-left:auto; margin-top:0.5em; font-size:85%; text-decoration:underline;}
.marginpar p{margin-top:0.4em; margin-bottom:0.4em;}
.equation td{text-align:center; vertical-align:middle; }
td.eq-no{ width:5%; }
table.equation { width:100%; } 
div.math-display, div.par-math-display{text-align:center;}
math .texttt { font-family: monospace; }
math .textit { font-style: italic; }
math .textsl { font-style: oblique; }
math .textsf { font-family: sans-serif; }
math .textbf { font-weight: bold; }
.partToc a, .partToc, .likepartToc a, .likepartToc {line-height: 200%; font-weight:bold; font-size:110%;}
.chapterToc a, .chapterToc, .likechapterToc a, .likechapterToc, .appendixToc a, .appendixToc {line-height: 200%; font-weight:bold;}
.index-item, .index-subitem, .index-subsubitem {display:block}
.caption td.id{font-weight: bold; white-space: nowrap; }
table.caption {text-align:center;}
h1.partHead{text-align: center}
p.bibitem { text-indent: -2em; margin-left: 2em; margin-top:0.6em; margin-bottom:0.6em; }
p.bibitem-p { text-indent: 0em; margin-left: 2em; margin-top:0.6em; margin-bottom:0.6em; }
.paragraphHead, .likeparagraphHead { margin-top:2em; font-weight: bold;}
.subparagraphHead, .likesubparagraphHead { font-weight: bold;}
.quote {margin-bottom:0.25em; margin-top:0.25em; margin-left:1em; margin-right:1em; text-align:justify;}
.verse{white-space:nowrap; margin-left:2em}
div.maketitle {text-align:center;}
h2.titleHead{text-align:center;}
div.maketitle{ margin-bottom: 2em; }
div.author, div.date {text-align:center;}
div.thanks{text-align:left; margin-left:10%; font-size:85%; font-style:italic; }
div.author{white-space: nowrap;}
.quotation {margin-bottom:0.25em; margin-top:0.25em; margin-left:1em; }
h1.partHead{text-align: center}
.sectionToc, .likesectionToc {margin-left:2em;}
.subsectionToc, .likesubsectionToc {margin-left:4em;}
.subsubsectionToc, .likesubsubsectionToc {margin-left:6em;}
.frenchb-nbsp{font-size:75%;}
.frenchb-thinspace{font-size:75%;}
.figure img.graphics {margin-left:10%;}
/* end css.sty */

\title{Espaces affines euclidiens}
\author{}
\date{}

\begin{document}
\maketitle

\textbf{Warning: \href{http://www.math.union.edu/locate/jsMath}{jsMath}
requires JavaScript to process the mathematics on this page.\\ If your
browser supports JavaScript, be sure it is enabled.}

\begin{center}\rule{3in}{0.4pt}\end{center}

{[}\href{coursse95.html}{next}{]} {[}\href{coursse93.html}{prev}{]}
{[}\href{coursse93.html\#tailcoursse93.html}{prev-tail}{]}
{[}\hyperref[tailcoursse94.html]{tail}{]}
{[}\href{coursch18.html\#coursse94.html}{up}{]}

\subsubsection{17.3 Espaces affines euclidiens}

\paragraph{17.3.1 Notion d'espace affine euclidien}

Définition~17.3.1 On appelle espace affine euclidien un couple (E,Φ)
d'un espace affine de dimension finie sur le corps ℝ des nombres réels
et d'une forme quadratique définie positive Φ sur
\textbackslash{}overrightarrow\{E\}.

Remarque~17.3.1 Comme d'habitude on notera
Φ(\textbackslash{}overrightarrow\{ξ\})
=\textbackslash{}\textbar{}\textbackslash{}overrightarrow\{
\{ξ\}\textbackslash{}\textbar{}\}\^{}\{2\} et on notera la forme polaire
de Φ sous la forme
(\textbackslash{}overrightarrow\{ξ\},\textbackslash{}overrightarrow\{η\})\textbackslash{}mathrel\{↦\}(\textbackslash{}overrightarrow\{ξ\}\textbackslash{}mathrel\{∣\}\textbackslash{}overrightarrow\{η\})
(le produit scalaire associé).

Proposition~17.3.1 Soit E un espace affine euclidien. L'application d :
E × E → ℝ,
(x,y)\textbackslash{}mathrel\{↦\}\textbackslash{}\textbar{}\textbackslash{}overrightarrow\{xy\}\textbackslash{}\textbar{}
est une distance sur E (appelée la distance euclidienne).

Démonstration Vérification laissée au lecteur.

\paragraph{17.3.2 Formule de Leibnitz et applications}

Théorème~17.3.2 Soit E un espace affine euclidien,
\{(\{a\}\_\{i\})\}\_\{i∈I\} une famille finie de points de E,
\{(\{λ\}\_\{i\})\}\_\{i∈I\} une famille de scalaires telle que
\{\textbackslash{}mathop\{\textbackslash{}mathop\{∑ \}\}
\}\_\{i∈I\}\{λ\}\_\{i\}\textbackslash{}mathrel\{≠\}0. Soit g le
barycentre de la famille de points massiques \{\textbackslash{}left
((\{a\}\_\{i\},\{λ\}\_\{i\})\textbackslash{}right )\}\_\{i∈I\}. Alors

\textbackslash{}mathop\{∀\}m ∈ E, \{\textbackslash{}mathop\{∑
\}\}\_\{i∈I\}\{λ\}\_\{i\}\textbackslash{}\textbar{}\textbackslash{}overrightarrow\{\{m\{a\}\_\{i\}\}\textbackslash{}\textbar{}\}\^{}\{2\}
= \textbackslash{}left (\{\textbackslash{}mathop\{∑
\}\}\_\{i∈I\}\{λ\}\_\{i\}\textbackslash{}right )
\textbackslash{}\textbar{}\textbackslash{}overrightarrow\{\{mg\}\textbackslash{}\textbar{}\}\^{}\{2\}
+\{ \textbackslash{}mathop\{∑
\}\}\_\{i∈I\}\{λ\}\_\{i\}\textbackslash{}\textbar{}\textbackslash{}overrightarrow\{\{g\{a\}\_\{i\}\}\textbackslash{}\textbar{}\}\^{}\{2\}

Démonstration On a

\textbackslash{}begin\{eqnarray*\} \{\textbackslash{}mathop\{∑
\}\}\_\{i∈I\}\{λ\}\_\{i\}\textbackslash{}\textbar{}\textbackslash{}overrightarrow\{\{m\{a\}\_\{i\}\}\textbackslash{}\textbar{}\}\^{}\{2\}
=\{ \textbackslash{}mathop\{∑
\}\}\_\{i∈I\}\{λ\}\_\{i\}\textbackslash{}\textbar{}\textbackslash{}overrightarrow\{mg\}
+\textbackslash{}overrightarrow\{\{
g\{a\}\_\{i\}\}\textbackslash{}\textbar{}\}\^{}\{2\}\&\& \%\&
\textbackslash{}\textbackslash{} \& =\& \textbackslash{}left
(\{\textbackslash{}mathop\{∑
\}\}\_\{i∈I\}\{λ\}\_\{i\}\textbackslash{}right )
\textbackslash{}\textbar{}\textbackslash{}overrightarrow\{\{mg\}\textbackslash{}\textbar{}\}\^{}\{2\}
+ 2\{\textbackslash{}mathop\{∑
\}\}\_\{i∈I\}\{λ\}\_\{i\}(\textbackslash{}overrightarrow\{mg\}\textbackslash{}mathrel\{∣\}\textbackslash{}overrightarrow\{g\{a\}\_\{i\}\})
+\{ \textbackslash{}mathop\{∑
\}\}\_\{i∈I\}\{λ\}\_\{i\}\textbackslash{}\textbar{}\textbackslash{}overrightarrow\{\{g\{a\}\_\{i\}\}\textbackslash{}\textbar{}\}\^{}\{2\}\%\&
\textbackslash{}\textbackslash{} \& =\& \textbackslash{}left
(\{\textbackslash{}mathop\{∑
\}\}\_\{i∈I\}\{λ\}\_\{i\}\textbackslash{}right )
\textbackslash{}\textbar{}\textbackslash{}overrightarrow\{\{mg\}\textbackslash{}\textbar{}\}\^{}\{2\}
+ 2\textbackslash{}left
(\textbackslash{}overrightarrow\{mg\}\textbackslash{}mathrel\{∣\}\{\textbackslash{}mathop\{∑
\}\}\_\{i∈I\}\{λ\}\_\{i\}\textbackslash{}overrightarrow\{g\{a\}\_\{i\}\}\textbackslash{}right
) +\{ \textbackslash{}mathop\{∑
\}\}\_\{i∈I\}\{λ\}\_\{i\}\textbackslash{}\textbar{}\textbackslash{}overrightarrow\{\{g\{a\}\_\{i\}\}\textbackslash{}\textbar{}\}\^{}\{2\}\%\&
\textbackslash{}\textbackslash{} \& =\& \textbackslash{}left
(\{\textbackslash{}mathop\{∑
\}\}\_\{i∈I\}\{λ\}\_\{i\}\textbackslash{}right )
\textbackslash{}\textbar{}\textbackslash{}overrightarrow\{\{mg\}\textbackslash{}\textbar{}\}\^{}\{2\}
+\{ \textbackslash{}mathop\{∑
\}\}\_\{i∈I\}\{λ\}\_\{i\}\textbackslash{}\textbar{}\textbackslash{}overrightarrow\{\{g\{a\}\_\{i\}\}\textbackslash{}\textbar{}\}\^{}\{2\}
\%\& \textbackslash{}\textbackslash{} \textbackslash{}end\{eqnarray*\}

en utilisant l'identité de polarisation et la formule
\{\textbackslash{}mathop\{\textbackslash{}mathop\{∑ \}\}
\}\_\{i∈I\}\{λ\}\_\{i\}\textbackslash{}overrightarrow\{g\{a\}\_\{i\}\}
=\textbackslash{}overrightarrow\{ 0\}.

Corollaire~17.3.3 Soit E un espace affine euclidien,
\{(\{a\}\_\{i\})\}\_\{i∈I\} une famille finie de points de E,
\{(\{λ\}\_\{i\})\}\_\{i∈I\} une famille de scalaires telle que
\{\textbackslash{}mathop\{\textbackslash{}mathop\{∑ \}\}
\}\_\{i∈I\}\{λ\}\_\{i\}\textbackslash{}mathrel\{≠\}0. Soit g le
barycentre de la famille de points massiques \{\textbackslash{}left
((\{a\}\_\{i\},\{λ\}\_\{i\})\textbackslash{}right )\}\_\{i∈I\}. Soit k ∈
ℝ. Alors \textbackslash{}\{m ∈
E\textbackslash{}mathrel\{∣\}\{\textbackslash{}mathop\{\textbackslash{}mathop\{∑
\}\}
\}\_\{i∈I\}\{λ\}\_\{i\}\textbackslash{}\textbar{}\textbackslash{}overrightarrow\{\{m\{a\}\_\{i\}\}\textbackslash{}\textbar{}\}\^{}\{2\}
= k\textbackslash{}\} est soit l'ensemble vide, soit le singleton
\textbackslash{}\{g\textbackslash{}\} soit une sphère de centre g.

Démonstration D'après la formule de Leibnitz, on a

\textbackslash{}begin\{eqnarray*\} \{\textbackslash{}mathop\{∑
\}\}\_\{i∈I\}\{λ\}\_\{i\}\textbackslash{}\textbar{}\textbackslash{}overrightarrow\{\{m\{a\}\_\{i\}\}\textbackslash{}\textbar{}\}\^{}\{2\}
= k\& \textbackslash{}mathrel\{⇔\} \& \textbackslash{}left
(\{\textbackslash{}mathop\{∑
\}\}\_\{i∈I\}\{λ\}\_\{i\}\textbackslash{}right )
\textbackslash{}\textbar{}\textbackslash{}overrightarrow\{\{mg\}\textbackslash{}\textbar{}\}\^{}\{2\}
= k −\{\textbackslash{}mathop\{∑
\}\}\_\{i∈I\}\{λ\}\_\{i\}\textbackslash{}\textbar{}\textbackslash{}overrightarrow\{\{g\{a\}\_\{i\}\}\textbackslash{}\textbar{}\}\^{}\{2\}\%\&
\textbackslash{}\textbackslash{} \& \textbackslash{}mathrel\{⇔\} \&
\textbackslash{}\textbar{}\textbackslash{}overrightarrow\{\{gm\}\textbackslash{}\textbar{}\}\^{}\{2\}
=\{ 1 \textbackslash{}over \{\textbackslash{}mathop\{∑
\}\}\_\{i∈I\}\{λ\}\_\{i\}\} \textbackslash{}left (k
−\{\textbackslash{}mathop\{∑
\}\}\_\{i∈I\}\{λ\}\_\{i\}\textbackslash{}\textbar{}\textbackslash{}overrightarrow\{\{g\{a\}\_\{i\}\}\textbackslash{}\textbar{}\}\^{}\{2\}\textbackslash{}right
) \%\& \textbackslash{}\textbackslash{} \textbackslash{}end\{eqnarray*\}

Donc, suivant que \{ 1 \textbackslash{}over
\{\textbackslash{}mathop\{\textbackslash{}mathop\{∑ \}\}
\}\_\{i∈I\}\{λ\}\_\{i\}\} \textbackslash{}left (k
−\{\textbackslash{}mathop\{\textbackslash{}mathop\{∑ \}\}
\}\_\{i∈I\}\{λ\}\_\{i\}\textbackslash{}\textbar{}\textbackslash{}overrightarrow\{\{g\{a\}\_\{i\}\}\textbackslash{}\textbar{}\}\^{}\{2\}\textbackslash{}right
) est strictement négatif, nul ou strictement positif, on trouve ∅,
\textbackslash{}\{g\textbackslash{}\} ou la sphère de centre g de rayon
\textbackslash{}sqrt\{\{ 1 \textbackslash{}over
\{\textbackslash{}mathop\{\textbackslash{}mathop\{∑ \}\}
\}\_\{i∈I\}\{λ\}\_\{i\}\} \textbackslash{}left (k
−\{\textbackslash{}mathop\{\textbackslash{}mathop\{∑ \}\}
\}\_\{i∈I\}\{λ\}\_\{i\}\textbackslash{}\textbar{}\textbackslash{}overrightarrow\{\{g\{a\}\_\{i\}\}\textbackslash{}\textbar{}\}\^{}\{2\}\textbackslash{}right
)\}.

Remarque~17.3.2 Soit E un espace affine euclidien,
\{(\{a\}\_\{i\})\}\_\{i∈I\} une famille finie de points de E,
\{(\{λ\}\_\{i\})\}\_\{i∈I\} une famille de scalaires telle que
\{\textbackslash{}mathop\{\textbackslash{}mathop\{∑ \}\}
\}\_\{i∈I\}\{λ\}\_\{i\} = 0. On sait qu'il existe un vecteur
\textbackslash{}overrightarrow\{u\} tel que \textbackslash{}mathop\{∀\}m
∈ E, \textbackslash{}mathop\{\textbackslash{}mathop\{∑ \}\}
\{λ\}\_\{i\}\textbackslash{}overrightarrow\{m\{a\}\_\{i\}\}
=\textbackslash{}overrightarrow\{ u\}. Soit a ∈ E. On a alors

\textbackslash{}begin\{eqnarray*\} \{\textbackslash{}mathop\{∑
\}\}\_\{i∈I\}\{λ\}\_\{i\}\textbackslash{}\textbar{}\textbackslash{}overrightarrow\{\{m\{a\}\_\{i\}\}\textbackslash{}\textbar{}\}\^{}\{2\}
=\{ \textbackslash{}mathop\{∑
\}\}\_\{i∈I\}\{λ\}\_\{i\}\textbackslash{}\textbar{}\textbackslash{}overrightarrow\{ma\}
+\textbackslash{}overrightarrow\{\{
a\{a\}\_\{i\}\}\textbackslash{}\textbar{}\}\^{}\{2\}\&\& \%\&
\textbackslash{}\textbackslash{} \& =\& \textbackslash{}left
(\{\textbackslash{}mathop\{∑
\}\}\_\{i∈I\}\{λ\}\_\{i\}\textbackslash{}right )
\textbackslash{}\textbar{}\textbackslash{}overrightarrow\{\{ma\}\textbackslash{}\textbar{}\}\^{}\{2\}
+ 2\{\textbackslash{}mathop\{∑
\}\}\_\{i∈I\}\{λ\}\_\{i\}(\textbackslash{}overrightarrow\{ma\}\textbackslash{}mathrel\{∣\}\textbackslash{}overrightarrow\{a\{a\}\_\{i\}\})
+\{ \textbackslash{}mathop\{∑
\}\}\_\{i∈I\}\{λ\}\_\{i\}\textbackslash{}\textbar{}\textbackslash{}overrightarrow\{\{a\{a\}\_\{i\}\}\textbackslash{}\textbar{}\}\^{}\{2\}\%\&
\textbackslash{}\textbackslash{} \& =\& 2\textbackslash{}left
(\textbackslash{}overrightarrow\{ma\}\textbackslash{}mathrel\{∣\}\{\textbackslash{}mathop\{∑
\}\}\_\{i∈I\}\{λ\}\_\{i\}\textbackslash{}overrightarrow\{a\{a\}\_\{i\}\}\textbackslash{}right
) +\{ \textbackslash{}mathop\{∑
\}\}\_\{i∈I\}\{λ\}\_\{i\}\textbackslash{}\textbar{}\textbackslash{}overrightarrow\{\{a\{a\}\_\{i\}\}\textbackslash{}\textbar{}\}\^{}\{2\}
\%\& \textbackslash{}\textbackslash{} \& =\&
2(\textbackslash{}overrightarrow\{ma\}\textbackslash{}mathrel\{∣\}\textbackslash{}overrightarrow\{u\})
+\{ \textbackslash{}mathop\{∑
\}\}\_\{i∈I\}\{λ\}\_\{i\}\textbackslash{}\textbar{}\textbackslash{}overrightarrow\{\{a\{a\}\_\{i\}\}\textbackslash{}\textbar{}\}\^{}\{2\}
\%\& \textbackslash{}\textbackslash{} \textbackslash{}end\{eqnarray*\}

On en déduit que

\{\textbackslash{}mathop\{∑
\}\}\_\{i∈I\}\{λ\}\_\{i\}\textbackslash{}\textbar{}\textbackslash{}overrightarrow\{\{m\{a\}\_\{i\}\}\textbackslash{}\textbar{}\}\^{}\{2\}
= k \textbackslash{}mathrel\{⇔\}
(\textbackslash{}overrightarrow\{ma\}\textbackslash{}mathrel\{∣\}\textbackslash{}overrightarrow\{u\})
=\{ 1 \textbackslash{}over 2\} \textbackslash{}left (k
−\{\textbackslash{}mathop\{∑
\}\}\_\{i∈I\}\{λ\}\_\{i\}\textbackslash{}\textbar{}\textbackslash{}overrightarrow\{\{a\{a\}\_\{i\}\}\textbackslash{}\textbar{}\}\^{}\{2\}\textbackslash{}right
)

qui est soit l'ensemble vide (si \textbackslash{}overrightarrow\{u\}
=\textbackslash{}overrightarrow\{ 0\} et k
−\{\textbackslash{}mathop\{\textbackslash{}mathop\{∑ \}\}
\}\_\{i∈I\}\{λ\}\_\{i\}\textbackslash{}\textbar{}\textbackslash{}overrightarrow\{\{a\{a\}\_\{i\}\}\textbackslash{}\textbar{}\}\^{}\{2\}\textbackslash{}mathrel\{≠\}0),
soit un hyperplan orthogonal à \textbackslash{}overrightarrow\{u\}
(prendre par exemple un repère orthonormé), soit encore l'espace tout
entier.

Corollaire~17.3.4 Soit k \textgreater{} 0, a,b ∈ E distincts. Alors
\textbackslash{}\{m ∈ E\textbackslash{}mathrel\{∣\}d(m,a) =
kd(m,b)\textbackslash{}\} est

\begin{itemize}
\itemsep1pt\parskip0pt\parsep0pt
\item
  si k\textbackslash{}mathrel\{≠\}1, une sphère de centre g barycentre
  de (a,1) et de (b,−\{k\}\^{}\{2\})
\item
  si k = 1, l'hyperplan médiateur de a et b.
\end{itemize}

Démonstration On a en effet d(m,a) = kd(m,b)
\textbackslash{}mathrel\{⇔\}
\textbackslash{}\textbar{}\textbackslash{}overrightarrow\{\{ma\}\textbackslash{}\textbar{}\}\^{}\{2\}
−
\{k\}\^{}\{2\}\textbackslash{}\textbar{}\textbackslash{}overrightarrow\{\{mb\}\textbackslash{}\textbar{}\}\^{}\{2\}
= 0. Si k\textbackslash{}mathrel\{≠\}1, alors 1 −
\{k\}\^{}\{2\}\textbackslash{}mathrel\{≠\}0 et l'ensemble qui ne peut
être ni l'ensemble vide, ni un singleton (car il y a deux solutions
évidentes sur la droite ab à savoir le barycentre de (a,1) et (b,k) et
le barycentre de (a,1) et (b,−k)) est une sphère de centre g. Si par
contre k = 1, on a (avec les notations de la remarque)
\textbackslash{}overrightarrow\{u\} =\textbackslash{}overrightarrow\{
ma\} −\textbackslash{}overrightarrow\{ mb\}
=\textbackslash{}overrightarrow\{ ba\}, si bien que l'ensemble qui n'est
pas l'ensemble vide (car le milieu de a et b convient) est un hyperplan
orthogonal à \textbackslash{}overrightarrow\{ba\} et contenant le milieu
de a et b, donc c'est l'hyperplan médiateur de a et b.

\paragraph{17.3.3 Isométries affines}

Définition~17.3.2 Soit E un espace affine euclidien et f : E → E une
application affine. On dit que f est une isométrie affine si elle
vérifie les conditions équivalentes

\begin{itemize}
\itemsep1pt\parskip0pt\parsep0pt
\item
  (i) \textbackslash{}mathop\{∀\}x,y ∈ E, d(f(x),f(y)) = d(x,y)
\item
  (ii) \textbackslash{}vec\{f\} est un endomorphisme orthogonal de
  \textbackslash{}overrightarrow\{E\}.
\end{itemize}

Démonstration On a en effet d(f(x),f(y))
=\textbackslash{}\textbar{}\textbackslash{}overrightarrow\{
f(x)f(y)\}\textbackslash{}\textbar{}
=\textbackslash{}\textbar{}\textbackslash{}vec\{
f\}(\textbackslash{}overrightarrow\{xy\})\textbackslash{}\textbar{} si
bien que la condition (i) est équivalente à
\textbackslash{}mathop\{∀\}\textbackslash{}overrightarrow\{ξ\}
∈\textbackslash{}overrightarrow\{ E\},
\textbackslash{}\textbar{}\textbackslash{}vec\{f\}(\textbackslash{}overrightarrow\{ξ\})\textbackslash{}\textbar{}
=\textbackslash{}\textbar{}\textbackslash{}overrightarrow\{
ξ\}\textbackslash{}\textbar{} ce qui caractérise les endomorphismes
orthogonaux.

On peut définir une isométrie affine à l'aide de repères orthonormés par
le théorème suivant

Théorème~17.3.5 Soit
(a,\textbackslash{}vec\{\{e\}\}\_\{1\},\textbackslash{}mathop\{\textbackslash{}mathop\{\ldots{}\}\},\textbackslash{}vec\{\{e\}\}\_\{n\})
et
(a',\textbackslash{}vec\{\{e\}\}\_\{1\}',\textbackslash{}mathop\{\textbackslash{}mathop\{\ldots{}\}\},\textbackslash{}vec\{\{e\}\}\_\{n\}')
deux repères orthonormés de E~; alors il existe une unique isométrie
affine f de E vérifiant f(a) = a' et \textbackslash{}mathop\{∀\}i ∈
{[}1,n{]}, \textbackslash{}vec\{f\}(\textbackslash{}vec\{\{e\}\}\_\{i\})
=\textbackslash{}vec\{ \{e\}\}\_\{i\}'.

Démonstration On sait qu'il existe une unique application affine
vérifiant f(a) = a' et \textbackslash{}mathop\{∀\}i ∈ {[}1,n{]},
\textbackslash{}vec\{f\}(\textbackslash{}vec\{\{e\}\}\_\{i\})
=\textbackslash{}vec\{ \{e\}\}\_\{i\}'~; comme \textbackslash{}vec\{f\}
envoie une base orthonormée sur une base orthonormée, c'est un
endomorphisme orthogonal.

\paragraph{17.3.4 Forme réduite d'une isométrie affine}

Théorème~17.3.6 Soit f : E → E une isométrie affine. Alors il existe un
unique couple (g,\textbackslash{}overrightarrow\{ξ\}) d'une isométrie
affine ayant un point fixe et d'un vecteur
\textbackslash{}overrightarrow\{ξ\} ∈\textbackslash{}overrightarrow\{
E\} vérifiant les propriétés équivalentes suivantes

\begin{itemize}
\itemsep1pt\parskip0pt\parsep0pt
\item
  (i) f = g ∘ \{t\}\_\{\textbackslash{}overrightarrow\{ξ\}\} =
  \{t\}\_\{\textbackslash{}overrightarrow\{ξ\}\} ∘ g
\item
  (ii) f = \{t\}\_\{\textbackslash{}overrightarrow\{ξ\}\} ∘ g et
  \textbackslash{}overrightarrow\{ξ\} est parallèle à l'ensemble des
  points fixes de g
\item
  (iii) f = \{t\}\_\{\textbackslash{}overrightarrow\{ξ\}\} ∘ g et
  \textbackslash{}vec\{f\}(\textbackslash{}overrightarrow\{ξ\})
  =\textbackslash{}overrightarrow\{ ξ\}
\end{itemize}

Démonstration Vérifions tout d'abord l'équivalence de (i), (ii) et
(iii). Tout d'abord soit a un point fixe de g. On a g(a) = a et donc
g(x) = x \textbackslash{}mathrel\{⇔\}
\textbackslash{}vec\{g\}(\textbackslash{}overrightarrow\{ax\})
=\textbackslash{}overrightarrow\{ ax\} ce qui montre que la direction
\textbackslash{}overrightarrow\{F\} de l'ensemble F des points fixes de
g n'est autre que l'ensemble des vecteurs
\textbackslash{}overrightarrow\{ξ\} tels que
\textbackslash{}vec\{g\}(\textbackslash{}overrightarrow\{ξ\})
=\textbackslash{}overrightarrow\{ ξ\}. Mais d'autre part f =
\{t\}\_\{\textbackslash{}overrightarrow\{ξ\}\} ∘ g
⇒\textbackslash{}vec\{ f\} =\textbackslash{}vec\{ g\}~; on a donc
immédiatement l'équivalence de (ii) et (iii). Il nous reste donc à
montrer l'équivalence de (i) et (iii). Mais g ∘
\{t\}\_\{\textbackslash{}overrightarrow\{ξ\}\}(x) = g(x
+\textbackslash{}overrightarrow\{ ξ\}) = g(x) +\textbackslash{}vec\{
g\}(\textbackslash{}overrightarrow\{ξ\}) = g(x) +\textbackslash{}vec\{
f\}(\textbackslash{}overrightarrow\{ξ\}) et
\{t\}\_\{\textbackslash{}overrightarrow\{ξ\}\} ∘ g(x) = g(x)
+\textbackslash{}overrightarrow\{ ξ\} ce qui montre bien l'équivalence
de (i) et (iii).

Montrons donc l'existence et l'unicité d'un couple
(g,\textbackslash{}overrightarrow\{ξ\}) d'une isométrie affine ayant un
point fixe et d'un vecteur \textbackslash{}overrightarrow\{ξ\}
∈\textbackslash{}overrightarrow\{ E\} vérifiant (iii). On doit donc
rechercher un vecteur \textbackslash{}overrightarrow\{ξ\}
∈\textbackslash{}overrightarrow\{ E\} tel que
\textbackslash{}vec\{f\}(\textbackslash{}overrightarrow\{ξ\})
=\textbackslash{}overrightarrow\{ ξ\} et tel que g =
\{t\}\_\{−\textbackslash{}overrightarrow\{ξ\}\} ∘ f ait un point fixe.
Soit a ∈ E et cherchons à résoudre l'équation
\{t\}\_\{−\textbackslash{}overrightarrow\{ξ\}\} ∘ f(x) = x, soit encore
f(x) = x +\textbackslash{}overrightarrow\{ ξ\}, c'est-à-dire f(a)
+\textbackslash{}vec\{ f\}(\textbackslash{}overrightarrow\{ax\}) = a
+\textbackslash{}overrightarrow\{ ax\} +\textbackslash{}overrightarrow\{
ξ\}, soit \textbackslash{}vec\{f\}(\textbackslash{}overrightarrow\{ax\})
−\textbackslash{}overrightarrow\{ ax\} =\textbackslash{}overrightarrow\{
f(a)a\} +\textbackslash{}overrightarrow\{ ξ\}.

Soit \textbackslash{}overrightarrow\{F\} =
\textbackslash{}\{\textbackslash{}overrightarrow\{u\}\textbackslash{}mathrel\{∣\}\textbackslash{}vec\{f\}(\textbackslash{}overrightarrow\{u\})
=\textbackslash{}overrightarrow\{ u\}\textbackslash{}\}. On a
\textbackslash{}overrightarrow\{E\} =\textbackslash{}overrightarrow\{
F\} ⊕\textbackslash{}overrightarrow\{ \{F\}\}\^{}\{⊥\} et comme
\textbackslash{}overrightarrow\{F\} est stable par l'endomorphisme
orthogonal \textbackslash{}vec\{f\}, il en est de même de
\textbackslash{}overrightarrow\{\{F\}\}\^{}\{⊥\}. Si on pose
\textbackslash{}overrightarrow\{ax\} =\textbackslash{}overrightarrow\{
\{x\}\_\{1\}\} +\textbackslash{}overrightarrow\{ \{x\}\_\{2\}\},
\textbackslash{}overrightarrow\{f(a)a\}
=\textbackslash{}overrightarrow\{ \{a\}\_\{1\}\}
+\textbackslash{}overrightarrow\{ \{a\}\_\{2\}\},
\textbackslash{}overrightarrow\{ξ\} =\textbackslash{}overrightarrow\{
ξ\} +\textbackslash{}overrightarrow\{ 0\} les décompositions des
différents vecteurs dans la somme directe
\textbackslash{}overrightarrow\{E\} =\textbackslash{}overrightarrow\{
F\} ⊕\textbackslash{}overrightarrow\{ \{F\}\}\^{}\{⊥\}, on a

\textbackslash{}begin\{eqnarray*\}
\textbackslash{}vec\{f\}(\textbackslash{}overrightarrow\{ax\})
−\textbackslash{}overrightarrow\{ ax\} =\textbackslash{}overrightarrow\{
f(a)a\} +\textbackslash{}overrightarrow\{ ξ\}\&
\textbackslash{}mathrel\{⇔\} \& \textbackslash{}left
\textbackslash{}\{\textbackslash{}array\{
\textbackslash{}vec\{f\}(\textbackslash{}overrightarrow\{\{x\}\_\{1\}\})
−\textbackslash{}overrightarrow\{ \{x\}\_\{1\}\}\&
=\textbackslash{}overrightarrow\{ \{a\}\_\{1\}\}
+\textbackslash{}overrightarrow\{ ξ\} \textbackslash{}cr
\textbackslash{}vec\{f\}(\textbackslash{}overrightarrow\{\{x\}\_\{2\}\})
−\textbackslash{}overrightarrow\{ \{x\}\_\{2\}\}\&
=\textbackslash{}overrightarrow\{ \{a\}\_\{2\}\}
+\textbackslash{}overrightarrow\{ 0\} \} \textbackslash{}right .\%\&
\textbackslash{}\textbackslash{} \& \textbackslash{}mathrel\{⇔\} \&
\textbackslash{}left \textbackslash{}\{\textbackslash{}array\{
\textbackslash{}overrightarrow\{0\} \& =\textbackslash{}overrightarrow\{
\{a\}\_\{1\}\} +\textbackslash{}overrightarrow\{ ξ\} \textbackslash{}cr
\textbackslash{}vec\{f\}(\textbackslash{}overrightarrow\{\{x\}\_\{2\}\})
−\textbackslash{}overrightarrow\{ \{x\}\_\{2\}\}\&
=\textbackslash{}overrightarrow\{ \{a\}\_\{2\}\} \}
\textbackslash{}right .\%\&\textbackslash{}\textbackslash{}
\textbackslash{}end\{eqnarray*\}

puisque
\textbackslash{}vec\{f\}(\textbackslash{}overrightarrow\{\{x\}\_\{1\}\})
−\textbackslash{}overrightarrow\{ \{x\}\_\{1\}\}
=\textbackslash{}overrightarrow\{ 0\}. Ceci montre déjà que
\textbackslash{}overrightarrow\{ξ\} =
−\textbackslash{}overrightarrow\{\{a\}\_\{1\}\} d'où l'unicité de
\textbackslash{}overrightarrow\{ξ\} et donc de g.

Il nous suffit donc de montrer que la deuxième équation a une solution.
Mais \textbackslash{}vec\{f\} −\textbackslash{}mathrm\{Id\} laisse
stable \textbackslash{}overrightarrow\{\{F\}\}\^{}\{⊥\} et définit un
endomorphisme injectif de cet espace car si
\textbackslash{}overrightarrow\{u\} ∈\textbackslash{}overrightarrow\{
\{F\}\}\^{}\{⊥\}, on a

f(\textbackslash{}overrightarrow\{u\}) −\textbackslash{}overrightarrow\{
u\} =\textbackslash{}overrightarrow\{ 0\}
⇒\textbackslash{}overrightarrow\{ u\} ∈\textbackslash{}overrightarrow\{
F\} ∩\textbackslash{}overrightarrow\{ \{F\}\}\^{}\{⊥\} =
\textbackslash{}\{\textbackslash{}overrightarrow\{0\}\textbackslash{}\}

Comme cet espace est de dimension finie, cet endomorphisme de
\textbackslash{}overrightarrow\{\{F\}\}\^{}\{⊥\} est aussi surjectif, et
donc la deuxième équation a bien une solution.

Remarque~17.3.3 Puisque g a un point fixe, en vectorialisant l'espace en
un tel point, l'isométrie affine g s'identifie à l'endomorphisme
orthogonal \textbackslash{}vec\{g\} =\textbackslash{}vec\{ f\}. Une
isométrie affine s'identifie au produit commutatif d'une isométrie
vectorielle (que l'on connait déjà) et d'une translation parallèlement à
l'ensemble des points fixes de cette isométrie vectorielle. Ceci va nous
permettre de décrire complètement les isométries affines en dimension 2
ou 3 en distinguant suivant que f est un déplacement (c'est-à-dire que
\textbackslash{}vec\{f\} est une rotation) ou un antidéplacement
(c'est-à-dire que \textbackslash{}vec\{f\} ∈
\{O\}\^{}\{−\}(\textbackslash{}overrightarrow\{E\})).

Théorème~17.3.7 Soit E un espace affine euclidien de dimension 2. Alors

\begin{itemize}
\itemsep1pt\parskip0pt\parsep0pt
\item
  (i) les déplacements de E sont d'une part les translations et d'autre
  part les rotations ayant pour centre un point de E et pour angle un
  élément non nul de ℝ∕2πℤ.
\item
  (ii) les antidéplacements de E sont les produits (commutatifs) d'une
  symétrie orthogonale par rapport à une droite et d'une translation
  parallèlement à cette droite.
\end{itemize}

Théorème~17.3.8 Soit E un espace affine euclidien de dimension 3. Alors

\begin{itemize}
\itemsep1pt\parskip0pt\parsep0pt
\item
  (i) les déplacements de E sont d'une part les translations et d'autre
  part les vissages~: produits commutatifs d'une rotation autour d'un
  axe D et d'une translation parallèlement à D
\item
  (ii) les antidéplacements de E sont d'une part les produits
  (commutatifs) d'une symétrie orthogonale par rapport à un plan et
  d'une translation parallèlement à ce plan et d'autre part les produits
  commutatifs d'une rotation autour d'un axe D et d'une symétrie
  orthogonale par rapport à un plan orthogonal à D.
\end{itemize}

\paragraph{17.3.5 Distance à un sous-espace affine}

Théorème~17.3.9 Soit F un sous-espace affine de direction
\textbackslash{}overrightarrow\{F\} et x ∈ E. Il existe un unique point
p ∈ F tel que d(x,F) = d(x,p). Le point p est l'unique point
d'intersection de F et de x +\textbackslash{}overrightarrow\{
\{F\}\}\^{}\{⊥\}~; on l'appelle la projection orthogonale de x sur F.
Soit a ∈ F et
(\textbackslash{}vec\{\{e\}\}\_\{1\},\textbackslash{}mathop\{\textbackslash{}mathop\{\ldots{}\}\},\textbackslash{}vec\{\{e\}\}\_\{k\})
une base de \textbackslash{}overrightarrow\{F\}. On a

d\{(x,F)\}\^{}\{2\} =\{
\textbackslash{}mathop\{\textbackslash{}mathrm\{det\}\}
\textbackslash{}mathop\{Gram\}(\textbackslash{}overrightarrow\{ax\},\textbackslash{}vec\{\{e\}\}\_\{1\},\textbackslash{}mathop\{\textbackslash{}mathop\{\ldots{}\}\},\textbackslash{}vec\{\{e\}\}\_\{k\})
\textbackslash{}over
\textbackslash{}mathop\{\textbackslash{}mathrm\{det\}\}
\textbackslash{}mathop\{Gram\}(\textbackslash{}vec\{\{e\}\}\_\{1\},\textbackslash{}mathop\{\textbackslash{}mathop\{\ldots{}\}\},\textbackslash{}vec\{\{e\}\}\_\{k\})\}

Démonstration Comme \textbackslash{}overrightarrow\{F\} et
\textbackslash{}overrightarrow\{\{F\}\}\^{}\{⊥\} sont supplémentaires,
un résultat précédent montre que F ∩\textbackslash{}left (x
+\textbackslash{}overrightarrow\{ \{F\}\}\^{}\{⊥\}\textbackslash{}right
) est un singleton \textbackslash{}\{p\textbackslash{}\}. On a donc p ∈
F et \textbackslash{}overrightarrow\{xp\}
∈\textbackslash{}overrightarrow\{ \{F\}\}\^{}\{⊥\} Soit alors y ∈ F. On
a

d\{(x,y)\}\^{}\{2\}
=\textbackslash{}\textbar{}\textbackslash{}overrightarrow\{\{
xy\}\textbackslash{}\textbar{}\}\^{}\{2\}
=\textbackslash{}\textbar{}\textbackslash{}overrightarrow\{ xp\}
+\textbackslash{}overrightarrow\{\{
py\}\textbackslash{}\textbar{}\}\^{}\{2\}
=\textbackslash{}\textbar{}\textbackslash{}overrightarrow\{\{
xp\}\textbackslash{}\textbar{}\}\^{}\{2\}
+\textbackslash{}\textbar{}\textbackslash{}overrightarrow\{\{
py\}\textbackslash{}\textbar{}\}\^{}\{2\}

car \textbackslash{}overrightarrow\{xp\}
∈\textbackslash{}overrightarrow\{ \{F\}\}\^{}\{⊥\} et
\textbackslash{}overrightarrow\{py\} ∈ F. On en déduit que d(x,y) ≥
d(x,p) avec égalité si et seulement
si~\textbackslash{}\textbar{}\textbackslash{}overrightarrow\{py\}\textbackslash{}\textbar{}
= 0 soit p = y, ce qui démontre que p est l'unique point de F tel que
d(x,F) = d(x,p).

De plus, si a ∈ F, soit y ∈ F. On a d(x,y)
=\textbackslash{}\textbar{}\textbackslash{}overrightarrow\{
xy\}\textbackslash{}\textbar{}
=\textbackslash{}\textbar{}\textbackslash{}overrightarrow\{ ax\}
−\textbackslash{}overrightarrow\{ ay\}\textbackslash{}\textbar{}. Mais
quand y décrit F, \textbackslash{}overrightarrow\{ay\} décrit
\textbackslash{}overrightarrow\{F\}, si bien que d(x,F) =
d(\textbackslash{}overrightarrow\{ax\},\textbackslash{}overrightarrow\{F\}).
Mais on a vu dans le chapitre sur les formes quadratiques que
d\{(\textbackslash{}overrightarrow\{ξ\},\textbackslash{}overrightarrow\{F\})\}\^{}\{2\}
=\{ \textbackslash{}mathop\{\textbackslash{}mathrm\{det\}\}
\textbackslash{}mathop\{
Gram\}(\textbackslash{}overrightarrow\{ξ\},\textbackslash{}vec\{\{e\}\}\_\{1\},\textbackslash{}mathop\{\textbackslash{}mathop\{\ldots{}\}\},\textbackslash{}vec\{\{e\}\}\_\{k\})
\textbackslash{}over
\textbackslash{}mathop\{\textbackslash{}mathrm\{det\}\}
\textbackslash{}mathop\{
Gram\}(\textbackslash{}vec\{\{e\}\}\_\{1\},\textbackslash{}mathop\{\textbackslash{}mathop\{\ldots{}\}\},\textbackslash{}vec\{\{e\}\}\_\{k\})\}
, ce qui donne la formule voulue.

En dimension 3, en interprétant le déterminant de Gram de trois vecteurs
comme le carré du produit mixte et le déterminant de Gram de deux
vecteurs comme le carré de la norme de leur produit vectoriel, on
obtient les formules suivantes pour la distance à une droite ou à un
plan

Corollaire~17.3.10 Soit E un espace euclidien de dimension 3, a,x ∈ E,
\textbackslash{}overrightarrow\{u\} et
\textbackslash{}overrightarrow\{v\} deux vecteurs non colinéaires de
\textbackslash{}overrightarrow\{E\}. Alors

d(x,a + ℝ\textbackslash{}overrightarrow\{u\}) =\{
\textbackslash{}\textbar{}\textbackslash{}overrightarrow\{ax\}
∧\textbackslash{}overrightarrow\{ u\}\textbackslash{}\textbar{}
\textbackslash{}over
\textbackslash{}\textbar{}\textbackslash{}overrightarrow\{u\}\textbackslash{}\textbar{}\}

d(x,a + ℝ\textbackslash{}overrightarrow\{u\} +
ℝ\textbackslash{}overrightarrow\{v\}) =\{ \textbackslash{}Big
\textbar{}{[}\textbackslash{}overrightarrow\{ax\},\textbackslash{}overrightarrow\{u\},\textbackslash{}overrightarrow\{v\}{]}\textbackslash{}Big
\textbar{} \textbackslash{}over
\textbackslash{}\textbar{}\textbackslash{}overrightarrow\{u\}
∧\textbackslash{}overrightarrow\{ v\}\textbackslash{}\textbar{}\}

En ce qui concerne les hyperplans, on a le résultat suivant

Théorème~17.3.11 Soit
(a,\textbackslash{}vec\{\{e\}\}\_\{1\},\textbackslash{}mathop\{\textbackslash{}mathop\{\ldots{}\}\},\textbackslash{}vec\{\{e\}\}\_\{n\})
un repère orthonormé de E, H un hyperplan affine de E d'équation
\{u\}\_\{1\}\{x\}\_\{1\} +
\textbackslash{}mathop\{\textbackslash{}mathop\{\ldots{}\}\} +
\{u\}\_\{n\}\{x\}\_\{n\} + h = 0. Alors le vecteur
\textbackslash{}overrightarrow\{n\} =
\{u\}\_\{1\}\textbackslash{}vec\{\{e\}\}\_\{1\} +
\textbackslash{}mathop\{\textbackslash{}mathop\{\ldots{}\}\} +
\{u\}\_\{n\}\textbackslash{}vec\{\{e\}\}\_\{n\} est un vecteur normal au
plan et pour tout x ∈ E de coordonnées
\{x\}\_\{1\},\textbackslash{}mathop\{\textbackslash{}mathop\{\ldots{}\}\},\{x\}\_\{n\},
on a

d(x,H) =\{ \textbar{}\{u\}\_\{1\}\{x\}\_\{1\} +
\textbackslash{}mathop\{\textbackslash{}mathop\{\ldots{}\}\} +
\{u\}\_\{n\}\{x\}\_\{n\} + h\textbar{} \textbackslash{}over
\textbackslash{}sqrt\{\{u\}\_\{1 \}\^{}\{2 \} +
\textbackslash{}mathop\{\textbackslash{}mathop\{\ldots{}\}\} +
\{u\}\_\{n \}\^{}\{2\}\}\}

Démonstration La direction \textbackslash{}overrightarrow\{H\} de H
admet pour équation \{u\}\_\{1\}\{x\}\_\{1\} +
\textbackslash{}mathop\{\textbackslash{}mathop\{\ldots{}\}\} +
\{u\}\_\{n\}\{x\}\_\{n\} = 0, c'est donc visiblement
\textbackslash{}overrightarrow\{\{n\}\}\^{}\{⊥\} ce qui montre que
\textbackslash{}overrightarrow\{n\} =
\{u\}\_\{1\}\textbackslash{}vec\{\{e\}\}\_\{1\} +
\textbackslash{}mathop\{\textbackslash{}mathop\{\ldots{}\}\} +
\{u\}\_\{n\}\textbackslash{}vec\{\{e\}\}\_\{n\} est un vecteur normal au
plan. Soit f la forme affine sur E définie par f(x) =
\{u\}\_\{1\}\{x\}\_\{1\} +
\textbackslash{}mathop\{\textbackslash{}mathop\{\ldots{}\}\} +
\{u\}\_\{n\}\{x\}\_\{n\} + h. On a donc
\textbackslash{}vec\{f\}(\textbackslash{}overrightarrow\{ξ\}) =
\{u\}\_\{1\}\{x\}\_\{1\} +
\textbackslash{}mathop\{\textbackslash{}mathop\{\ldots{}\}\} +
\{u\}\_\{n\}\{x\}\_\{n\} =
(\textbackslash{}overrightarrow\{ξ\}\textbackslash{}mathrel\{∣\}\textbackslash{}overrightarrow\{n\}).
Soit x ∈ E et p sa projection orthogonale sur H. On a donc f(p) = 0,
soit f(x) = f(x) − f(p) =\textbackslash{}vec\{
f\}(\textbackslash{}overrightarrow\{px\}) =
(\textbackslash{}overrightarrow\{px\}\textbackslash{}mathrel\{∣\}\textbackslash{}overrightarrow\{n\}).
Mais, les deux vecteurs \textbackslash{}overrightarrow\{px\} et
\textbackslash{}overrightarrow\{n\} qui sont tous deux orthogonaux à
\textbackslash{}overrightarrow\{H\} sont colinéaires, si bien que

\textbar{}f(x)\textbar{} =
\textbar{}(\textbackslash{}overrightarrow\{px\}\textbackslash{}mathrel\{∣\}\textbackslash{}overrightarrow\{n\})\textbar{}
=\textbackslash{}\textbar{}\textbackslash{}overrightarrow\{
px\}\textbackslash{}\textbar{}
\textbackslash{}\textbar{}\textbackslash{}overrightarrow\{n\}\textbackslash{}\textbar{}
=
d(x,H)\textbackslash{}\textbar{}\textbackslash{}overrightarrow\{n\}\textbackslash{}\textbar{}

On en déduit que d(x,H) =\{ \textbar{}f(x)\textbar{}
\textbackslash{}over
\textbackslash{}\textbar{}\textbackslash{}overrightarrow\{n\}\textbackslash{}\textbar{}\}
, ce qui n'est autre que la formule cherchée.

Corollaire~17.3.12 Soit \{H\}\_\{1\} et \{H\}\_\{2\} deux hyperplans non
parallèles de E. Alors l'ensemble des points x de E tels que
d(x,\{H\}\_\{1\}) = d(x,\{H\}\_\{2\}) est la réunion de deux hyperplans
orthogonaux, appelés les deux hyperplans bissecteurs de \{H\}\_\{1\} et
\{H\}\_\{2\}.

Démonstration Sans nuire à la généralité on peut supposer que
\{H\}\_\{1\} est d'équation \{u\}\_\{1\}\{x\}\_\{1\} +
\textbackslash{}mathop\{\textbackslash{}mathop\{\ldots{}\}\} +
\{u\}\_\{n\}\{x\}\_\{n\} + h = 0 et \{H\}\_\{2\} d'équation
\{v\}\_\{1\}\{x\}\_\{1\} +
\textbackslash{}mathop\{\textbackslash{}mathop\{\ldots{}\}\} +
\{v\}\_\{n\}\{x\}\_\{n\} + k = 0 avec \{u\}\_\{1\}\^{}\{2\} +
\textbackslash{}mathop\{\textbackslash{}mathop\{\ldots{}\}\} +
\{u\}\_\{n\}\^{}\{2\} = \{v\}\_\{1\}\^{}\{2\} +
\textbackslash{}mathop\{\textbackslash{}mathop\{\ldots{}\}\} +
\{v\}\_\{n\}\^{}\{2\} = 1. Alors les vecteurs normaux
\textbackslash{}overrightarrow\{\{n\}\_\{1\}\} et
\textbackslash{}overrightarrow\{\{n\}\_\{2\}\} à ces deux hyperplans
sont unitaires et

\textbackslash{}begin\{eqnarray*\} d(x,\{H\}\_\{1\}) =
d(x,\{H\}\_\{2\})\&\& \%\& \textbackslash{}\textbackslash{} \&
\textbackslash{}mathrel\{⇔\} \& \textbar{}\{u\}\_\{1\}\{x\}\_\{1\} +
\textbackslash{}mathop\{\textbackslash{}mathop\{\ldots{}\}\} +
\{u\}\_\{n\}\{x\}\_\{n\} + h\textbar{} =
\textbar{}\{v\}\_\{1\}\{x\}\_\{1\} +
\textbackslash{}mathop\{\textbackslash{}mathop\{\ldots{}\}\} +
\{v\}\_\{n\}\{x\}\_\{n\} + k\textbar{}\%\&
\textbackslash{}\textbackslash{} \& \textbackslash{}mathrel\{⇔\} \&
(\{u\}\_\{1\} + ε\{v\}\_\{1\})\{x\}\_\{1\} +
\textbackslash{}mathop\{\textbackslash{}mathop\{\ldots{}\}\} +
(\{u\}\_\{n\} + ε\{v\}\_\{n\})\{x\}\_\{n\} + h + εk = 0 \%\&
\textbackslash{}\textbackslash{} \textbackslash{}end\{eqnarray*\}

avec ε = ±1. Il s'agit visiblement de deux équations d'hyperplans de
vecteurs normaux \textbackslash{}overrightarrow\{\{n\}\_\{1\}\}
+\textbackslash{}overrightarrow\{ \{n\}\_\{2\}\} et
\textbackslash{}overrightarrow\{\{n\}\_\{1\}\}
−\textbackslash{}overrightarrow\{ \{n\}\_\{2\}\}. Or ces deux vecteurs
normaux sont orthogonaux puisque
(\textbackslash{}overrightarrow\{\{n\}\_\{1\}\}
+\textbackslash{}overrightarrow\{
\{n\}\_\{2\}\}\textbackslash{}mathrel\{∣\}\textbackslash{}overrightarrow\{\{n\}\_\{1\}\}
−\textbackslash{}overrightarrow\{ \{n\}\_\{2\}\})
=\textbackslash{}\textbar{}\textbackslash{}overrightarrow\{\{
\{n\}\_\{1\}\}\textbackslash{}\textbar{}\}\^{}\{2\}
−\textbackslash{}\textbar{}\textbackslash{}overrightarrow\{\{
\{n\}\_\{2\}\}\textbackslash{}\textbar{}\}\^{}\{2\} = 0. Ceci achève la
démonstration.

\paragraph{17.3.6 Distance de deux sous-espaces affines}

Théorème~17.3.13 Soit E un espace affine euclidien, F et G deux
sous-espaces affines de E. Alors il existe a ∈ F et b ∈ G tel que le
vecteur \textbackslash{}overrightarrow\{ab\} soit orthogonal à la fois à
F et à G. Ces points sont uniques si \textbackslash{}overrightarrow\{F\}
∩\textbackslash{}overrightarrow\{ G\} =
\textbackslash{}\{\textbackslash{}overrightarrow\{0\}\textbackslash{}\}.
On a d(F,G) = d(a,b).

Démonstration Soit H = G +\textbackslash{}overrightarrow\{ F\} le
sous-espace affine contenant G et auquel F est faiblement parallèle . On
doit avoir b ∈ H et \textbackslash{}overrightarrow\{ab\} ⊥ H ce qui
montre que b est nécessairement la projection orthogonale de a sur H.
Soit donc F' la projection orthogonale de F sur H. Comme F est
faiblement parallèle à H, F' est parallèle à F. Donc
\textbackslash{}overrightarrow\{F'\} +\textbackslash{}overrightarrow\{
G\} =\textbackslash{}overrightarrow\{ F\}
+\textbackslash{}overrightarrow\{ G\} =\textbackslash{}overrightarrow\{
H\}~; le théorème d'intersection des sous-espaces affines appliqués aux
sous-espaces affines F' et G de H, montre que F' ∩
G\textbackslash{}mathrel\{≠\}∅. Soit donc b ∈ F' ∩ G et a ∈ F dont la
projection orthogonale sur H est b. On a a ∈ F, b ∈ G et
\textbackslash{}overrightarrow\{ab\} ⊥ H, donc
\textbackslash{}overrightarrow\{ab\} est orthogonal à la fois à F et à
G.

Si x ∈ F et y ∈ G, on a

\textbackslash{}\textbar{}\textbackslash{}overrightarrow\{\{xy\}\textbackslash{}\textbar{}\}\^{}\{2\}
=\textbackslash{}\textbar{}\textbackslash{}overrightarrow\{ ab\} +
\{(\textbackslash{}overrightarrow\{xa\}
+\textbackslash{}overrightarrow\{
by\})\textbackslash{}\textbar{}\}\^{}\{2\}
=\textbackslash{}\textbar{}\textbackslash{}overrightarrow\{\{
ab\}\textbackslash{}\textbar{}\}\^{}\{2\}
+\textbackslash{}\textbar{}\textbackslash{}overrightarrow\{ xa\}
+\textbackslash{}overrightarrow\{\{
by\}\textbackslash{}\textbar{}\}\^{}\{2\}

d'après le théorème de Pythagore, puisque
\textbackslash{}overrightarrow\{ab\} ⊥\textbackslash{}overrightarrow\{
H\} et \textbackslash{}overrightarrow\{xa\}
+\textbackslash{}overrightarrow\{ by\} ∈\textbackslash{}overrightarrow\{
F\} +\textbackslash{}overrightarrow\{ G\}
=\textbackslash{}overrightarrow\{ H\}. On en déduit que
\textbackslash{}\textbar{}\textbackslash{}overrightarrow\{\{xy\}\textbackslash{}\textbar{}\}\^{}\{2\}
≥\textbackslash{}\textbar{}\textbackslash{}overrightarrow\{\{
ab\}\textbackslash{}\textbar{}\}\^{}\{2\}, soit encore d(x,y) ≥ d(a,b).
Ceci montre que d(F,G) = d(a,b). De plus l'égalité nécessite que
\textbackslash{}overrightarrow\{xa\} +\textbackslash{}overrightarrow\{
by\} =\textbackslash{}overrightarrow\{ 0\}. Si
\textbackslash{}overrightarrow\{F\} et
\textbackslash{}overrightarrow\{G\} sont en somme directe, ceci ne peut
se produire que si x = a et y = b, ce qui assure dans ce cas l'unicité
de a et b.

Définition~17.3.3 Soit E un espace affine euclidien, F et G deux
sous-espaces affines de E. On appelle perpendiculaire commune à F et à G
toute droite joignant un point de F à un point de G et orthogonale à ces
deux sous-espaces affines.

Remarque~17.3.4 D'après le théorème ci dessus, une telle droite est
unique si \textbackslash{}overrightarrow\{F\}
∩\textbackslash{}overrightarrow\{ G\} =
\textbackslash{}\{\textbackslash{}overrightarrow\{0\}\textbackslash{}\}
et si a et b sont distincts, soit F ∩ G = ∅. En particulier, en
dimension 3, deux droites non parallèles et non sécantes ont une unique
perpendiculaire commune (le résultat subsistant d'ailleurs évidemment
pour deux droites sécantes non confondues)~: si ces deux droites
\{D\}\_\{1\} et \{D\}\_\{2\} ont pour vecteurs directeurs
\textbackslash{}overrightarrow\{\{u\}\_\{1\}\} et
\textbackslash{}overrightarrow\{\{u\}\_\{2\}\}, cette perpendiculaire
commune est l'intersection du plan contenant \{D\}\_\{1\} et parallèle à
\textbackslash{}overrightarrow\{\{u\}\_\{1\}\}
∧\textbackslash{}overrightarrow\{ \{u\}\_\{2\}\} et du plan contenant
\{D\}\_\{2\} et parallèle à
\textbackslash{}overrightarrow\{\{u\}\_\{1\}\}
∧\textbackslash{}overrightarrow\{ \{u\}\_\{2\}\}.

Pour calculer la distance de F à G à savoir d(a,b), on peut également
remarquer que c'est la distance de a à H = G
+\textbackslash{}overrightarrow\{ F\} (puisque b est la projection
orthogonale de a sur H). Mais comme F est faiblement parallèle à H, on a
\textbackslash{}mathop\{∀\}x ∈ F, d(x,H) = d(a,H). On en déduit que la
distance de F à G est la distance de n'importe quel point x de F à H = G
+\textbackslash{}overrightarrow\{ F\}, que l'on sait calculer à l'aide
de déterminants de Gram moyennant la connaissance d'une base de
\textbackslash{}overrightarrow\{F\} +\textbackslash{}overrightarrow\{
G\}. En particulier, si E est de dimension 3, on a

Proposition~17.3.14 Soit E un espace euclidien de dimension 3,
\{D\}\_\{1\} = \{a\}\_\{1\} +
ℝ\textbackslash{}overrightarrow\{\{u\}\_\{1\}\} et \{D\}\_\{2\} =
\{a\}\_\{2\} + ℝ\textbackslash{}overrightarrow\{\{u\}\_\{2\}\} deux
droites non parallèles. Alors

d(\{D\}\_\{1\},\{D\}\_\{2\}) =\{ \textbackslash{}Big
\textbar{}{[}\textbackslash{}overrightarrow\{\{a\}\_\{1\}\{a\}\_\{2\}\},\textbackslash{}overrightarrow\{\{u\}\_\{1\}\},\textbackslash{}overrightarrow\{\{u\}\_\{2\}\}{]}\textbackslash{}Big
\textbar{} \textbackslash{}over
\textbackslash{}\textbar{}\textbackslash{}overrightarrow\{\{u\}\_\{1\}\}
∧\textbackslash{}overrightarrow\{
\{u\}\_\{2\}\}\textbackslash{}\textbar{}\}

Démonstration On a d(\{D\}\_\{1\},\{D\}\_\{2\}) =
d(\{a\}\_\{1\},\{a\}\_\{2\} +
ℝ\textbackslash{}overrightarrow\{\{u\}\_\{1\}\} +
ℝ\textbackslash{}overrightarrow\{\{u\}\_\{2\}\}) et il suffit
d'appliquer la formule donnant la distance d'un point à un plan.

{[}\href{coursse95.html}{next}{]} {[}\href{coursse93.html}{prev}{]}
{[}\href{coursse93.html\#tailcoursse93.html}{prev-tail}{]}
{[}\href{coursse94.html}{front}{]}
{[}\href{coursch18.html\#coursse94.html}{up}{]}

\end{document}

% \documentclass[]{article}
\usepackage[T1]{fontenc}
\usepackage{lmodern}
\usepackage{amssymb,amsmath}
\usepackage{ifxetex,ifluatex}
\usepackage{fixltx2e} % provides \textsubscript
% use upquote if available, for straight quotes in verbatim environments
\IfFileExists{upquote.sty}{\usepackage{upquote}}{}
\ifnum 0\ifxetex 1\fi\ifluatex 1\fi=0 % if pdftex
  \usepackage[utf8]{inputenc}
\else % if luatex or xelatex
  \ifxetex
    \usepackage{mathspec}
    \usepackage{xltxtra,xunicode}
  \else
    \usepackage{fontspec}
  \fi
  \defaultfontfeatures{Mapping=tex-text,Scale=MatchLowercase}
  \newcommand{\euro}{€}
\fi
% use microtype if available
\IfFileExists{microtype.sty}{\usepackage{microtype}}{}
\ifxetex
  \usepackage[setpagesize=false, % page size defined by xetex
              unicode=false, % unicode breaks when used with xetex
              xetex]{hyperref}
\else
  \usepackage[unicode=true]{hyperref}
\fi
\hypersetup{breaklinks=true,
            bookmarks=true,
            pdfauthor={},
            pdftitle={Cercles, sph`eres, triangle},
            colorlinks=true,
            citecolor=blue,
            urlcolor=blue,
            linkcolor=magenta,
            pdfborder={0 0 0}}
\urlstyle{same}  % don't use monospace font for urls
\setlength{\parindent}{0pt}
\setlength{\parskip}{6pt plus 2pt minus 1pt}
\setlength{\emergencystretch}{3em}  % prevent overfull lines
\setcounter{secnumdepth}{0}
 
/* start css.sty */
.cmr-5{font-size:50%;}
.cmr-7{font-size:70%;}
.cmmi-5{font-size:50%;font-style: italic;}
.cmmi-7{font-size:70%;font-style: italic;}
.cmmi-10{font-style: italic;}
.cmsy-5{font-size:50%;}
.cmsy-7{font-size:70%;}
.cmex-7{font-size:70%;}
.cmex-7x-x-71{font-size:49%;}
.msbm-7{font-size:70%;}
.cmtt-10{font-family: monospace;}
.cmti-10{ font-style: italic;}
.cmbx-10{ font-weight: bold;}
.cmr-17x-x-120{font-size:204%;}
.cmsl-10{font-style: oblique;}
.cmti-7x-x-71{font-size:49%; font-style: italic;}
.cmbxti-10{ font-weight: bold; font-style: italic;}
p.noindent { text-indent: 0em }
td p.noindent { text-indent: 0em; margin-top:0em; }
p.nopar { text-indent: 0em; }
p.indent{ text-indent: 1.5em }
@media print {div.crosslinks {visibility:hidden;}}
a img { border-top: 0; border-left: 0; border-right: 0; }
center { margin-top:1em; margin-bottom:1em; }
td center { margin-top:0em; margin-bottom:0em; }
.Canvas { position:relative; }
li p.indent { text-indent: 0em }
.enumerate1 {list-style-type:decimal;}
.enumerate2 {list-style-type:lower-alpha;}
.enumerate3 {list-style-type:lower-roman;}
.enumerate4 {list-style-type:upper-alpha;}
div.newtheorem { margin-bottom: 2em; margin-top: 2em;}
.obeylines-h,.obeylines-v {white-space: nowrap; }
div.obeylines-v p { margin-top:0; margin-bottom:0; }
.overline{ text-decoration:overline; }
.overline img{ border-top: 1px solid black; }
td.displaylines {text-align:center; white-space:nowrap;}
.centerline {text-align:center;}
.rightline {text-align:right;}
div.verbatim {font-family: monospace; white-space: nowrap; text-align:left; clear:both; }
.fbox {padding-left:3.0pt; padding-right:3.0pt; text-indent:0pt; border:solid black 0.4pt; }
div.fbox {display:table}
div.center div.fbox {text-align:center; clear:both; padding-left:3.0pt; padding-right:3.0pt; text-indent:0pt; border:solid black 0.4pt; }
div.minipage{width:100%;}
div.center, div.center div.center {text-align: center; margin-left:1em; margin-right:1em;}
div.center div {text-align: left;}
div.flushright, div.flushright div.flushright {text-align: right;}
div.flushright div {text-align: left;}
div.flushleft {text-align: left;}
.underline{ text-decoration:underline; }
.underline img{ border-bottom: 1px solid black; margin-bottom:1pt; }
.framebox-c, .framebox-l, .framebox-r { padding-left:3.0pt; padding-right:3.0pt; text-indent:0pt; border:solid black 0.4pt; }
.framebox-c {text-align:center;}
.framebox-l {text-align:left;}
.framebox-r {text-align:right;}
span.thank-mark{ vertical-align: super }
span.footnote-mark sup.textsuperscript, span.footnote-mark a sup.textsuperscript{ font-size:80%; }
div.tabular, div.center div.tabular {text-align: center; margin-top:0.5em; margin-bottom:0.5em; }
table.tabular td p{margin-top:0em;}
table.tabular {margin-left: auto; margin-right: auto;}
div.td00{ margin-left:0pt; margin-right:0pt; }
div.td01{ margin-left:0pt; margin-right:5pt; }
div.td10{ margin-left:5pt; margin-right:0pt; }
div.td11{ margin-left:5pt; margin-right:5pt; }
table[rules] {border-left:solid black 0.4pt; border-right:solid black 0.4pt; }
td.td00{ padding-left:0pt; padding-right:0pt; }
td.td01{ padding-left:0pt; padding-right:5pt; }
td.td10{ padding-left:5pt; padding-right:0pt; }
td.td11{ padding-left:5pt; padding-right:5pt; }
table[rules] {border-left:solid black 0.4pt; border-right:solid black 0.4pt; }
.hline hr, .cline hr{ height : 1px; margin:0px; }
.tabbing-right {text-align:right;}
span.TEX {letter-spacing: -0.125em; }
span.TEX span.E{ position:relative;top:0.5ex;left:-0.0417em;}
a span.TEX span.E {text-decoration: none; }
span.LATEX span.A{ position:relative; top:-0.5ex; left:-0.4em; font-size:85%;}
span.LATEX span.TEX{ position:relative; left: -0.4em; }
div.float img, div.float .caption {text-align:center;}
div.figure img, div.figure .caption {text-align:center;}
.marginpar {width:20%; float:right; text-align:left; margin-left:auto; margin-top:0.5em; font-size:85%; text-decoration:underline;}
.marginpar p{margin-top:0.4em; margin-bottom:0.4em;}
.equation td{text-align:center; vertical-align:middle; }
td.eq-no{ width:5%; }
table.equation { width:100%; } 
div.math-display, div.par-math-display{text-align:center;}
math .texttt { font-family: monospace; }
math .textit { font-style: italic; }
math .textsl { font-style: oblique; }
math .textsf { font-family: sans-serif; }
math .textbf { font-weight: bold; }
.partToc a, .partToc, .likepartToc a, .likepartToc {line-height: 200%; font-weight:bold; font-size:110%;}
.chapterToc a, .chapterToc, .likechapterToc a, .likechapterToc, .appendixToc a, .appendixToc {line-height: 200%; font-weight:bold;}
.index-item, .index-subitem, .index-subsubitem {display:block}
.caption td.id{font-weight: bold; white-space: nowrap; }
table.caption {text-align:center;}
h1.partHead{text-align: center}
p.bibitem { text-indent: -2em; margin-left: 2em; margin-top:0.6em; margin-bottom:0.6em; }
p.bibitem-p { text-indent: 0em; margin-left: 2em; margin-top:0.6em; margin-bottom:0.6em; }
.paragraphHead, .likeparagraphHead { margin-top:2em; font-weight: bold;}
.subparagraphHead, .likesubparagraphHead { font-weight: bold;}
.quote {margin-bottom:0.25em; margin-top:0.25em; margin-left:1em; margin-right:1em; text-align:justify;}
.verse{white-space:nowrap; margin-left:2em}
div.maketitle {text-align:center;}
h2.titleHead{text-align:center;}
div.maketitle{ margin-bottom: 2em; }
div.author, div.date {text-align:center;}
div.thanks{text-align:left; margin-left:10%; font-size:85%; font-style:italic; }
div.author{white-space: nowrap;}
.quotation {margin-bottom:0.25em; margin-top:0.25em; margin-left:1em; }
h1.partHead{text-align: center}
.sectionToc, .likesectionToc {margin-left:2em;}
.subsectionToc, .likesubsectionToc {margin-left:4em;}
.subsubsectionToc, .likesubsubsectionToc {margin-left:6em;}
.frenchb-nbsp{font-size:75%;}
.frenchb-thinspace{font-size:75%;}
.figure img.graphics {margin-left:10%;}
/* end css.sty */

\title{Cercles, sph`eres, triangle}
\author{}
\date{}

\begin{document}
\maketitle

\textbf{Warning: 
requires JavaScript to process the mathematics on this page.\\ If your
browser supports JavaScript, be sure it is enabled.}

\begin{center}\rule{3in}{0.4pt}\end{center}

[
[
[]
[

\subsubsection{17.4 Cercles, sphères, triangle}

\paragraph{17.4.1 Généralités sur les sphères}

Définition~17.4.1 Soit E un espace euclidien, a \in E, r > 0.
On appelle sphère de centre a de rayon r l'ensemble S(a,r) =
\x \in E∣d(a,x) =
r\.

Equation de sphères

Soit
(O,\vece_1,\\ldots,\vece_n~)
un repère orthonormé de E,
a_1,\\ldots,a_n~
les coordonnées de a et
x_1,\\ldots,x_n~
les coordonnées de x. Alors

\begin{align*} d(a,x) = r&
\Leftrightarrow &
\\overrightarrowax\^2
= r^2 \%& \\ &
\Leftrightarrow & (x_1 -
a_1)^2 +
\\ldots~ +
(x_ n - a_n)^2 = r^2 \%&
\\ & \Leftrightarrow &
x_1^2 +
\\ldots + x_
n^2 - 2a_ 1x_1
-\\ldots~ -
2a_nx_n + c = 0\%& \\
\end{align*}

avec c = a_1^2 +
\\ldots~ +
a_n^2 - r^2.

Inversement, si on se donne
a_1,\\ldots,a_n~,c
\in \mathbb{R}~, on a

\begin{align*} x_1^2 +
\\ldots + x_
n^2 - 2a_ 1x_1
-\\ldots~ -
2a_nx_n + c = 0&&\%&
\\ & \Leftrightarrow &
\\overrightarrowax\^2
= a_ 1^2 +
\\ldots + a_
n^2 - c\%& \\
\end{align*}

On en déduit que l'ensemble en question est soit l'ensemble vide si
a_1^2 +
\\ldots~ +
a_n^2 - c < 0, soit le singleton
\a\ avec a = O +
a_1\vece_1 +
\\ldots~ +
a_n\vece_n si a_1^2
+ \\ldots~ +
a_n^2 - c = 0, soit la sphère de centre a et de rayon
\sqrta_1 ^2  +
\\ldots~ +
a_n ^2  - c si a_1^2 +
\\ldots~ +
a_n^2 - c > 0.

Intersection d'une sphère et d'un sous-espace affine

Soit S(a,r) une sphère de E et F un sous-espace affine de E. Appelons b
la projection orthogonale de a sur F. Pour x \in F, on a
d(a,x)^2 = d(a,b)^2 + d(b,x)^2 =
d(a,F)^2 + d(b,x)^2 si bien que

x \in S(a,r) \bigcap F \Leftrightarrow d(b,x)^2 =
r^2 - d(a,F)^2

On en déduit que

\begin{itemize}
\itemsep1pt\parskip0pt\parsep0pt
\item
  (i) si d(a,F) > r, alors S(a,r) \bigcap F = \varnothing~
\item
  (ii) si d(a,F) = r, alors S(a,r) \bigcap F =
  \b\ où b désigne la projection
  orthogonale de a sur F~; on dit dans ce cas que F est tangent à la
  sphère
\item
  (iii) si d(a,F) < r, alors S(a,r) \bigcap F est la sphère de F de
  centre b, projection orthogonale de a sur F, et de rayon
  \sqrtr^2  - d(a, F)^2.
\end{itemize}

Intersection de deux sphères

Considérons deux sphères S(a,r) et S(a',r') de centres distincts. On a
alors

\begin{align*} x \in S(a,r) \bigcap S(a',r')&& \%&
\\ & \Leftrightarrow &
\left \\array
\\overrightarrowax\^2&
= r^2 \cr
\\overrightarrowa'x\^2&
= r'^2  \right .\%&
\\ & \Leftrightarrow &
\left \\array
\\overrightarrowax\^2
& = r^2 \cr
\\overrightarrowa'x\^2
-\\overrightarrow
ax\^2& = r'^2 -
r^2  \right .\%&
\\ & \Leftrightarrow &
\left \\array
\\overrightarrowax\^2
& = r^2 \cr
(\overrightarrowa'x +\overrightarrow
ax∣\overrightarrowa'x
-\overrightarrow ax)& = r'^2 -
r^2  \right .\%&
\\ & \Leftrightarrow &
\left \\array
\\overrightarrowax\^2
& = r^2 \cr
2(\overrightarrowbx∣\overrightarrowaa')&
= r'^2 - r^2  \right
.\%&\\ \end{align*}

si b désigne le milieu de aa'. Soit donc c le point de la droite aa' tel
que
2(\overrightarrowbc∣\overrightarrowaa')
= r'^2 - r^2, soit encore
\overlinebc.\overlineaa' =
r'^2-r^2 \over 2 . On obtient

\begin{align*} x \in S(a,r) \bigcap S(a',r')&
\Leftrightarrow & \left
\\array
\\overrightarrowax\^2
& = r^2 \cr
2(\overrightarrowbx∣\overrightarrowaa')&
=
2(\overrightarrowbc∣\overrightarrowaa')
 \right .\%& \\ &
\Leftrightarrow & \left
\\array
\\overrightarrowax\^2
& = r^2 \cr
2(\overrightarrowcx∣\overrightarrowaa')&
= 0  \right .\%&\\
\end{align*}

Or cette dernière équation est celle de l'hyperplan H orthogonal à aa'
passant par c. On obtient donc que S(a,r) \bigcap S(a',r') = S(a,r) \bigcap H.
L'intersection est donc soit \varnothing~, soit un singleton, soit une sphère de
l'hyperplan H suivant que d(a,H) > r, d(a,H) = r ou d(a,H)
< r. Mais comme la droite ac est aussi la droite aa' qui est
orthogonale à H, la distance de a à H n'est autre que la distance de a à
c. On a

 r'^2 - r^2 \over 2 =
\overlinebc.\overlineaa' =
(\overlineac
-\overlineab).\overlineaa' =
(\overlineac - 1 \over 2
\overlineaa').\overlineaa'

d'où l'on déduit que
\overlineac.\overlineaa' =
r^2-r'^2+d(a,a')^2 \over
2 , soit encore d(a,c)^2 =
(r^2-r'^2+d(a,a')^2)^2
\over 4d(a,a')^2 . On a donc

\begin{align*} d(a,H)^2 - r^2
= d(a,c)^2 - r^2&& \%&
\\ & =& (r^2 -
r'^2 + d(a,a')^2)^2 -
4r^2d(a,a')^2 \over
4d(a,a')^2 \%& \\ & =& 1
\over 4d(a,a')^2 (r^2 -
r'^2 + d(a,a')^2 + 2rd(a,a')) \%&
\\ & & \quad
(r^2 - r'^2 + d(a,a')^2 - 2rd(a,a'))
\%& \\ & =& 1 \over
4d(a,a')^2 ((r + d(a,a'))^2 -
r'^2)((r - d(a,a'))^2 - r'^2)\%&
\\ & =& 1 \over
4d(a,a')^2 (r + d(a,a') + r')(r + d(a,a') - r') \%&
\\ & & \quad (r - d(a,a')
+ r')(r - d(a,a') - r') \%& \\
\end{align*}

qui est du signe de \delta = (r + d(a,a') - r')(r - d(a,a') + r')(r - d(a,a')
- r').

\begin{itemize}
\itemsep1pt\parskip0pt\parsep0pt
\item
  (i) si r - r' < d(a,a') < r +
  r', on a \delta < 0 et donc S(a,r) \bigcap S(a',r') est une sphère de
  centre c de l'hyperplan H
\item
  (ii) si r - r' = d(a,a') ou r + r' = d(a,a'), on a
  \delta = 0 et donc S(a,r) \bigcap S(a',r') =
  \c\~; les deux sphères sont
  tangentes (intérieurement si r - r' = d(a,a') et
  extérieurement si r + r' = d(a,a'))
\item
  (iii) si d(a,a') < r - r' ou d(a,a')
  > r + r', alors \delta > 0 et S(a,r) \bigcap S(a',r') =
  \varnothing~.
\end{itemize}

\paragraph{17.4.2 Cercles et angles}

Théorème~17.4.1 Dans le plan euclidien orienté, soit \Gamma = S(\omega,r) le
cercle de centre \omega et de rayon r et soit trois points distincts m,a et b
de \Gamma. Alors
\widehat(\overrightarrow\omegaa,\overrightarrow\omegab)
= 2\widehat(ma,mb) (où
\widehat(\overrightarrow\omegaa,\overrightarrow\omegab)
désigne l'angle des vecteurs \overrightarrow\omegaa et
\overrightarrow\omegab et
\widehat(ma,mb) l'angle des droites ma et mb).

Démonstration Ecrivons
\widehat(\overrightarrow\omegaa,\overrightarrowm\omega)
=\widehat
(\overrightarrow\omegaa,\overrightarrowma)
+\widehat
(\overrightarrowma,\overrightarrowm\omega).
Soit s la symétrie par rapport à la médiatrice du couple (a,m). On a s :
\left
\\matrix\,a\mapsto~m
\cr m\mapsto~a \cr
\omega\mapsto~\omega\right . puisque le
centre du cercle appartient à la médiatrice de (a,m). De plus s change
un angle de vecteurs en son opposé, d'où

\begin{align*}
\widehat(\overrightarrow\omegaa,\overrightarrowma)&
=&
-\widehat(\overrightarrows(\omega)s(a),\overrightarrows(m)s(a))
\%& \\ & =&
-\widehat(\overrightarrow\omegam,\overrightarrowam)
\%& \\ & =&
-\widehat(\overrightarrow\omegam,\overrightarrowm\omega)
-\widehat
(\overrightarrowm\omega,\overrightarrowma)
-\widehat
(\overrightarrowma,\overrightarrowam)\%&
\\ & =& -\pi~ -\widehat
(\overrightarrowm\omega,\overrightarrowma)
- \pi~ =
-\widehat(\overrightarrowm\omega,\overrightarrowma)
\%& \\ & =&
\widehat(\overrightarrowma,\overrightarrowm\omega)
\%& \\ \end{align*}

On en déduit donc que
\widehat(\overrightarrow\omegaa,\overrightarrowm\omega)
=
2\widehat(\overrightarrowma,\overrightarrowm\omega)
= 2\widehat(ma,m\omega) (puisque la mesure de
\widehat(ma,m\omega) est un élément de \mathbb{R}~\diagup\pi~\mathbb{Z}, la mesure de
2\widehat(ma,m\omega) est un élément de \mathbb{R}~\diagup2\pi~\mathbb{Z}, donc la
mesure d'un angle de vecteurs).

On a de même
\widehat(\overrightarrow\omegab,\overrightarrowm\omega)
= 2\widehat(mb,m\omega), puis par soustraction
\widehat(\overrightarrow\omegaa,\overrightarrow\omegab)
= 2\widehat(ma,mb).

Remarque~17.4.1 Si m vient se confondre avec a, la droite ma vient se
confondre avec la tangente D_a à \Gamma en a~; le lecteur montrera
sans difficulté que le raisonnement précédent est encore valide dans ce
cas limite et que l'on a donc
\widehat(\overrightarrow\omegaa,\overrightarrow\omegab)
= 2\widehat(D_a,ab)

Corollaire~17.4.2 Dans le plan euclidien orienté, soit a,b,c et d quatre
points distincts. Alors ces quatre points sont cocycliques ou alignés si
et seulement si~\widehat(ca,cb)
=\widehat (da,db).

Démonstration La condition est bien entendu nécessaire, car si les
quatre points sont alignés, on a \widehat(ca,cb)
=\widehat (da,db) = 0 et s'ils appartiennent à un
même cercle \Gamma = S(\omega,r), on a 2\widehat(ca,cb) =
2\widehat(da,db) =\widehat
(\overrightarrow\omegaa,\overrightarrow\omegab)~;
mais l'application x\mapsto~2x, \mathbb{R}~\diagup\pi~\mathbb{Z} \rightarrow~ \mathbb{R}~\diagup2\pi~\mathbb{Z} est
injective (car si 2\alpha~ = 2\beta~ + 2k\pi~, on a \alpha~ = \beta~ + k\pi~) et donc on a
\widehat(ca,cb) =\widehat (da,db).

Inversement, supposons que \alpha~ =\widehat (ca,cb)
=\widehat (da,db). Si \alpha~ = 0, il est clair que a,b,c
et d sont alignés. Supposons donc \alpha~\neq~0. Alors
a,b et c ne sont pas alignés et il existe un unique cercle \Gamma contenant
a,b et c (voir le paragraphe suivant). Soit d' le point d'intersection
de la droite ad avec \Gamma différent de a. Comme a,b,c et d' sont sur \Gamma on a
\widehat(ca,cb) =\widehat
(d'a,d'b) =\widehat (da,d'b) puisque la droite da
est confondue avec la droite d'a. On a donc
\widehat(da,d'b) =\widehat
(da,db), soit encore \widehat(db,d'b) = 0. Ceci
signifie que d,d' et b sont alignés, de même que d,d' et a. Comme a,b et
d ne sont pas alignés, ceci nécessite que d = d' soit d \in \Gamma. Donc a,b,c
et d sont cocycliques.

Corollaire~17.4.3 Soit \alpha~ un angle de droites non nul, a et b deux points
distincts du plan euclidien orienté E. Alors l'ensemble des points m
tels que \widehat(ma,mb) = \alpha~ est un cercle passant
par les points a et b, privé de a et b.

Démonstration Soit D_a la droite passant par a telle que
\widehat(D_a,ab) = \alpha~ et soit \Gamma le cercle
tangent à D_a passant par b. Le centre \omega de ce cercle est le
point d'intersection de la perpendiculaire à D_a passant par a
avec la médiatrice de (a,b). Pour tout point c de ce cercle, on a
(d'après la remarque ci dessus) 2\widehat(ca,cb)
=\widehat
(\overrightarrow\omegaa,\overrightarrow\omegab)
= 2\widehat(D_a,ab) = 2\alpha~, soit de nouveau
\widehat(ca,cb) = \alpha~. Donc \m \in
E∣\widehat(ma,mb) =
\alpha~\ \subset~ \Gamma. Soit c \in \Gamma
\diagdown\a,b\~; pour tout point m tel que
\widehat(ma,mb) = \alpha~, on a
\widehat(ma,mb) =\widehat (ca,cb),
donc a,b,c et m sont cocycliques, ce qui montre que m \in \Gamma. Donc
\m \in
E∣\widehat(ma,mb) =
\alpha~\ = \Gamma \diagdown\a,b\.

\paragraph{17.4.3 Eléments de géométrie du triangle}

Soit E un plan affine euclidien orienté, A,B et C trois points non
alignés de E qui forment le triangle ABC. Nous utiliserons les notations
suivantes

\alpha~ =\widehat
(\overrightarrowAB,\overrightarrowAC),\beta~
=\widehat
(\overrightarrowBC,\overrightarrowBA),\gamma
=\widehat
(\overrightarrowCA,\overrightarrowCB)

a = BC,b = CA,c = AB

Proposition~17.4.4

\begin{itemize}
\itemsep1pt\parskip0pt\parsep0pt
\item
  (i) \alpha~ + \beta~ + \gamma = \pi~
\item
  (ii) cos~ \alpha~ =
  b^2+c^2-a^2 \over
  2bc ,\\ldots~
\end{itemize}

Démonstration (i) On a

\begin{align*} \alpha~ + \beta~ + \gamma& =&
\widehat(\overrightarrowAB,\overrightarrowAC)
+\widehat
(\overrightarrowBC,\overrightarrowBA)
+\widehat
(\overrightarrowCA,\overrightarrowCB)
\%& \\ & =&
(\widehat(\overrightarrowAB,\overrightarrowCA)
+ \pi~) + (\pi~ +\widehat
(\overrightarrowCB,\overrightarrowBA))
+\widehat
(\overrightarrowCA,\overrightarrowCB)\%&
\\ & =& 2\pi~ +\widehat
(\overrightarrowAB,\overrightarrowBA)
= 3\pi~ = \pi~ \%& \\
\end{align*}

(ii) On a

\begin{align*} a^2& =&
\\overrightarrowBC\^2
=\\overrightarrow AC
-\overrightarrow
AB\^2 \%&
\\ & =&
\\overrightarrowAC\^2
+\\overrightarrow
AB\^2 -
2(\overrightarrowAC∣\overrightarrowAB)
= b^2 + c^2 - 2bccos~
\alpha~\%& \\ \end{align*}

d'où cos~ \alpha~ =
b^2+c^2-a^2 \over 2bc
et les deux autres formules analogues.

Définition~17.4.2 On appelle médianes du triangle ABC les droites
joignant un sommet au milieu du côté opposé. On appelle hauteurs du
triangle ABC les droites passant par un sommet et orthogonales au coté
opposé. On appelle médiatrices du triangle ABC les trois médiatrices des
couples de sommets du triangle. On appelle bissectrices intérieures du
triangle les droites bissectrices des couples de vecteurs
(\overrightarrowAB,\overrightarrowAC),
(\overrightarrowBC,\overrightarrowBA)
et
(\overrightarrowCA,\overrightarrowCB).

Théorème~17.4.5 (i) Les trois médianes du triangle sont concourantes en
l'isobarycentre des trois points A,B et C (le centre de gravité du
triangle) (ii) Les trois médiatrices du triangle sont concourantes en le
centre de l'unique cercle circonscrit au triangle ABC (c'est-à-dire
passant par les points A,B et C). (iii) Les trois bissectrices
intérieures du triangle sont concourantes en le centre de l'unique
cercle inscrit dans le triangle. (iv) Les trois hauteurs du triangle
sont concourantes en un point appelé l'orthocentre du triangle.

Démonstration (i) Le théorème d'associativité des barycentres montre que
l'isobarycentre G du triangle ABC est aussi le barycentre de A affecté
du coefficient 1 et du milieu A' de (B,C) affecté du coefficient 2~;
donc G appartient à la médiane AA' et à chacune des deux autres
médianes.

(ii) Soit O le point d'intersection de la médiatrice de (A,B) avec la
médiatrice de (A,C). On a donc d(O,A) = d(O,B) et d(O,A) = d(O,C). On en
déduit que d(O,B) = d(O,C) et donc O est également sur la médiatrice de
(B,C).

(iii) Soit \Omega le point d'intersection de la bissectrice de
(\overrightarrowAB,\overrightarrowAC)
avec la bissectrice de
(\overrightarrowBC,\overrightarrowBA).
On a donc d(\Omega,AB) = d(\Omega,AC) et d(\Omega,BC) = d(\Omega,BA). On en déduit que
d(\Omega,CB) = d(\Omega,CA) et donc \Omega est sur une des deux bissectrices des
droites CB et CA. Comme \Omega est visiblement à l'intérieur du triangle, il
est également sur la bissectrice de
(\overrightarrowCA,\overrightarrowCB).

(iv) Soit A'B'C' le triangle défini par~: la droite B'C' passe par A et
est parallèle à BC, la droite A'C' passe par B et est parallèle à AC, la
droite A'B' passe par C et est parallèle à AB. Le quadrilatère AB'CB est
visiblement un parallélogramme (cotés deux à deux parallèles) donc AB' =
BC. De même le quadrilatère AC'BC est un parallélogramme, donc AC' = BC.
On en déduit que A est le milieu de (B',C'). Mais la hauteur issue de A
est orthogonale à BC donc B'C'. Il s'agit donc de la médiatrice de B'C'.
Les hauteurs du triangle ABC sont les médiatrices du triangle A'B'C',
elles sont donc concourantes.

Théorème~17.4.6 Soit ABC un triangle, \alpha~ =\widehat
(\overrightarrowAB,\overrightarrowAC),
\beta~ =\widehat
(\overrightarrowBC,\overrightarrowBA),
\gamma =\widehat
(\overrightarrowCA,\overrightarrowCB),
a = BC, b = CA, c = AB, R le rayon du cercle circonscrit au triangle, r
le rayon du cercle inscrit dans le triangle, S l'aire du triangle,
h_a = d(A,BC) la longueur de la hauteur issue de A. On a les
formules suivantes

\begin{align*} & a \over
sin \alpha~~ = b \over
sin \beta~~ = c \over
sin \gamma~ = 2R & (1)\%&
\\ & S = 1 \over 2
ah_a = 1 \over 2
bcsin \alpha~ = 1 \over 2~ (a + b
+ c)r& (2)\%& \\
\end{align*}

Démonstration Soit O le centre du cercle circonscrit au triangle. Le
triangle OBC est isocèle en O d'angle au sommet
2\widehat(ab,ac) = 2\alpha~ et de côté R. On en déduit que
a = BC = 2Rsin  2\alpha~ \over 2~
= 2Rsin~ \alpha~. On a donc  a \over
sin \alpha~~ = 2R et de même pour les deux autres
sommets, d'où la formule (1).

On sait que l'aire du triangle est égale à la moitié de l'aire du
rectangle correspondant, donc S = 1 \over 2
ah_a.

D'autre part l'aire du triangle est égale à la moitié de l'aire d'un
parallélogramme construit sur ce triangle, soit

S = 1 \over 2
[\overrightarrowAB,\overrightarrowAC]
= 1 \over 2
\\overrightarrowAB\
\\overrightarrowAC\sin~
\widehat(\overrightarrowAB,\overrightarrowAC)
= 1 \over 2 cbsin~ \alpha~

Soit \Omega le centre du cercle inscrit dans le triangle. L'aire du triangle
ABC est la somme des aires des triangles \OmegaAB, \OmegaBC et \OmegaCA. Or l'aire du
triangle \OmegaBC est égale à la moitié du produit de la longueur de la base
BC par la distance de \Omega à BC qui vaut justement r. Donc S = 1
\over 2 ar + 1 \over 2 br + 1
\over 2 cr.

Remarque~17.4.2 En combinant toutes ces formules, il n'est guère
difficile de calculer tous les éléments remarquables d'un triangle.

[
[
[
[

\end{document}

\part{Courbes}
% \documentclass[]{article}
\usepackage[T1]{fontenc}
\usepackage{lmodern}
\usepackage{amssymb,amsmath}
\usepackage{ifxetex,ifluatex}
\usepackage{fixltx2e} % provides \textsubscript
% use upquote if available, for straight quotes in verbatim environments
\IfFileExists{upquote.sty}{\usepackage{upquote}}{}
\ifnum 0\ifxetex 1\fi\ifluatex 1\fi=0 % if pdftex
  \usepackage[utf8]{inputenc}
\else % if luatex or xelatex
  \ifxetex
    \usepackage{mathspec}
    \usepackage{xltxtra,xunicode}
  \else
    \usepackage{fontspec}
  \fi
  \defaultfontfeatures{Mapping=tex-text,Scale=MatchLowercase}
  \newcommand{\euro}{€}
\fi
% use microtype if available
\IfFileExists{microtype.sty}{\usepackage{microtype}}{}
\usepackage{graphicx}
% Redefine \includegraphics so that, unless explicit options are
% given, the image width will not exceed the width of the page.
% Images get their normal width if they fit onto the page, but
% are scaled down if they would overflow the margins.
\makeatletter
\def\ScaleIfNeeded{%
  \ifdim\Gin@nat@width>\linewidth
    \linewidth
  \else
    \Gin@nat@width
  \fi
}
\makeatother
\let\Oldincludegraphics\includegraphics
{%
 \catcode`\@=11\relax%
 \gdef\includegraphics{\@ifnextchar[{\Oldincludegraphics}{\Oldincludegraphics[width=\ScaleIfNeeded]}}%
}%
\ifxetex
  \usepackage[setpagesize=false, % page size defined by xetex
              unicode=false, % unicode breaks when used with xetex
              xetex]{hyperref}
\else
  \usepackage[unicode=true]{hyperref}
\fi
\hypersetup{breaklinks=true,
            bookmarks=true,
            pdfauthor={},
            pdftitle={Arcs parametres},
            colorlinks=true,
            citecolor=blue,
            urlcolor=blue,
            linkcolor=magenta,
            pdfborder={0 0 0}}
\urlstyle{same}  % don't use monospace font for urls
\setlength{\parindent}{0pt}
\setlength{\parskip}{6pt plus 2pt minus 1pt}
\setlength{\emergencystretch}{3em}  % prevent overfull lines
\setcounter{secnumdepth}{0}
 
/* start css.sty */
.cmr-5{font-size:50%;}
.cmr-7{font-size:70%;}
.cmmi-5{font-size:50%;font-style: italic;}
.cmmi-7{font-size:70%;font-style: italic;}
.cmmi-10{font-style: italic;}
.cmsy-5{font-size:50%;}
.cmsy-7{font-size:70%;}
.cmex-7{font-size:70%;}
.cmex-7x-x-71{font-size:49%;}
.msbm-7{font-size:70%;}
.cmtt-10{font-family: monospace;}
.cmti-10{ font-style: italic;}
.cmbx-10{ font-weight: bold;}
.cmr-17x-x-120{font-size:204%;}
.cmsl-10{font-style: oblique;}
.cmti-7x-x-71{font-size:49%; font-style: italic;}
.cmbxti-10{ font-weight: bold; font-style: italic;}
p.noindent { text-indent: 0em }
td p.noindent { text-indent: 0em; margin-top:0em; }
p.nopar { text-indent: 0em; }
p.indent{ text-indent: 1.5em }
@media print {div.crosslinks {visibility:hidden;}}
a img { border-top: 0; border-left: 0; border-right: 0; }
center { margin-top:1em; margin-bottom:1em; }
td center { margin-top:0em; margin-bottom:0em; }
.Canvas { position:relative; }
li p.indent { text-indent: 0em }
.enumerate1 {list-style-type:decimal;}
.enumerate2 {list-style-type:lower-alpha;}
.enumerate3 {list-style-type:lower-roman;}
.enumerate4 {list-style-type:upper-alpha;}
div.newtheorem { margin-bottom: 2em; margin-top: 2em;}
.obeylines-h,.obeylines-v {white-space: nowrap; }
div.obeylines-v p { margin-top:0; margin-bottom:0; }
.overline{ text-decoration:overline; }
.overline img{ border-top: 1px solid black; }
td.displaylines {text-align:center; white-space:nowrap;}
.centerline {text-align:center;}
.rightline {text-align:right;}
div.verbatim {font-family: monospace; white-space: nowrap; text-align:left; clear:both; }
.fbox {padding-left:3.0pt; padding-right:3.0pt; text-indent:0pt; border:solid black 0.4pt; }
div.fbox {display:table}
div.center div.fbox {text-align:center; clear:both; padding-left:3.0pt; padding-right:3.0pt; text-indent:0pt; border:solid black 0.4pt; }
div.minipage{width:100%;}
div.center, div.center div.center {text-align: center; margin-left:1em; margin-right:1em;}
div.center div {text-align: left;}
div.flushright, div.flushright div.flushright {text-align: right;}
div.flushright div {text-align: left;}
div.flushleft {text-align: left;}
.underline{ text-decoration:underline; }
.underline img{ border-bottom: 1px solid black; margin-bottom:1pt; }
.framebox-c, .framebox-l, .framebox-r { padding-left:3.0pt; padding-right:3.0pt; text-indent:0pt; border:solid black 0.4pt; }
.framebox-c {text-align:center;}
.framebox-l {text-align:left;}
.framebox-r {text-align:right;}
span.thank-mark{ vertical-align: super }
span.footnote-mark sup.textsuperscript, span.footnote-mark a sup.textsuperscript{ font-size:80%; }
div.tabular, div.center div.tabular {text-align: center; margin-top:0.5em; margin-bottom:0.5em; }
table.tabular td p{margin-top:0em;}
table.tabular {margin-left: auto; margin-right: auto;}
div.td00{ margin-left:0pt; margin-right:0pt; }
div.td01{ margin-left:0pt; margin-right:5pt; }
div.td10{ margin-left:5pt; margin-right:0pt; }
div.td11{ margin-left:5pt; margin-right:5pt; }
table[rules] {border-left:solid black 0.4pt; border-right:solid black 0.4pt; }
td.td00{ padding-left:0pt; padding-right:0pt; }
td.td01{ padding-left:0pt; padding-right:5pt; }
td.td10{ padding-left:5pt; padding-right:0pt; }
td.td11{ padding-left:5pt; padding-right:5pt; }
table[rules] {border-left:solid black 0.4pt; border-right:solid black 0.4pt; }
.hline hr, .cline hr{ height : 1px; margin:0px; }
.tabbing-right {text-align:right;}
span.TEX {letter-spacing: -0.125em; }
span.TEX span.E{ position:relative;top:0.5ex;left:-0.0417em;}
a span.TEX span.E {text-decoration: none; }
span.LATEX span.A{ position:relative; top:-0.5ex; left:-0.4em; font-size:85%;}
span.LATEX span.TEX{ position:relative; left: -0.4em; }
div.float img, div.float .caption {text-align:center;}
div.figure img, div.figure .caption {text-align:center;}
.marginpar {width:20%; float:right; text-align:left; margin-left:auto; margin-top:0.5em; font-size:85%; text-decoration:underline;}
.marginpar p{margin-top:0.4em; margin-bottom:0.4em;}
.equation td{text-align:center; vertical-align:middle; }
td.eq-no{ width:5%; }
table.equation { width:100%; } 
div.math-display, div.par-math-display{text-align:center;}
math .texttt { font-family: monospace; }
math .textit { font-style: italic; }
math .textsl { font-style: oblique; }
math .textsf { font-family: sans-serif; }
math .textbf { font-weight: bold; }
.partToc a, .partToc, .likepartToc a, .likepartToc {line-height: 200%; font-weight:bold; font-size:110%;}
.chapterToc a, .chapterToc, .likechapterToc a, .likechapterToc, .appendixToc a, .appendixToc {line-height: 200%; font-weight:bold;}
.index-item, .index-subitem, .index-subsubitem {display:block}
.caption td.id{font-weight: bold; white-space: nowrap; }
table.caption {text-align:center;}
h1.partHead{text-align: center}
p.bibitem { text-indent: -2em; margin-left: 2em; margin-top:0.6em; margin-bottom:0.6em; }
p.bibitem-p { text-indent: 0em; margin-left: 2em; margin-top:0.6em; margin-bottom:0.6em; }
.paragraphHead, .likeparagraphHead { margin-top:2em; font-weight: bold;}
.subparagraphHead, .likesubparagraphHead { font-weight: bold;}
.quote {margin-bottom:0.25em; margin-top:0.25em; margin-left:1em; margin-right:1em; text-align:\\jmathmathustify;}
.verse{white-space:nowrap; margin-left:2em}
div.maketitle {text-align:center;}
h2.titleHead{text-align:center;}
div.maketitle{ margin-bottom: 2em; }
div.author, div.date {text-align:center;}
div.thanks{text-align:left; margin-left:10%; font-size:85%; font-style:italic; }
div.author{white-space: nowrap;}
.quotation {margin-bottom:0.25em; margin-top:0.25em; margin-left:1em; }
h1.partHead{text-align: center}
.sectionToc, .likesectionToc {margin-left:2em;}
.subsectionToc, .likesubsectionToc {margin-left:4em;}
.subsubsectionToc, .likesubsubsectionToc {margin-left:6em;}
.frenchb-nbsp{font-size:75%;}
.frenchb-thinspace{font-size:75%;}
.figure img.graphics {margin-left:10%;}
/* end css.sty */

\title{Arcs parametres}
\author{}
\date{}

\begin{document}
\maketitle

\textbf{Warning: 
requires JavaScript to process the mathematics on this page.\\ If your
browser supports JavaScript, be sure it is enabled.}

\begin{center}\rule{3in}{0.4pt}\end{center}

{[}
{[}{]}
{[}

\subsubsection{18.1 Arcs paramétrés}

\paragraph{18.1.1 Vocabulaire}

Définition~18.1.1 Soit E un espace vectoriel normé sur \mathbb{R}~. On appelle arc
paramétré de classe C^k de E tout couple (I,F) d'un
intervalle I de \mathbb{R}~ et d'une application F : I \rightarrow~ E de classe
C^k.

Remarque~18.1.1 Vocabulaire associé. Soit \Gamma = (I,F) un arc paramétré de
classe C^k de E.

\begin{itemize}
\itemsep1pt\parskip0pt\parsep0pt
\item
  (i) On appelle point de l'arc \Gamma un élément t_0 \in I
\item
  (ii) On appelle image ou support de \Gamma la partie F(I) de E
\item
  (iii) On appelle multiplicité d'un point m de l'image de \Gamma le cardinal
  de l'ensemble F^-1(\m\)
  (éventuellement infinie)~; on dit qu'un point de l'image est simple
  s'il est de multiplicité 1, sinon on dit qu'il est multiple~; on dit
  que l'arc est simple si tout point de l'image est de multiplicité 1
\item
  (iv) On dit qu'un point t_0 \in I de l'arc \Gamma est régulier si
  F'(t_0)\neq~0~; on dit que l'arc est
  régulier si tout point de l'arc est régulier~; un point non régulier
  est dit singulier.
\end{itemize}

\paragraph{18.1.2 Equivalence des arcs paramétrés}

Définition~18.1.2 Soit E un \mathbb{R}~ espace vectoriel normé, (I,F) et (J,G)
deux arcs paramétrés de classe C^k. On dit que ces deux arcs
sont C^k-équivalents s'il existe un difféomorphisme \theta : I \rightarrow~ J
de classe C^k tel que F = G \cdot \theta.

Remarque~18.1.2 On dira qu'un tel difféomorphisme est un changement de
paramétrage admissible. L'étude des arcs paramétrés concerne
essentiellement l'étude des propriétés des arcs qui sont invariantes par
équivalence. L'application \theta étant bi\\jmathmathective on voit immédiatement que

Proposition~18.1.1 Soit (I,F) et (J,G) deux arcs paramétrés de classe
C^k qui sont C^k-équivalents. Alors les deux arcs
ont la même image. Tous les points de l'image ont la même multiplicité
pour les deux arcs. En particulier un point de l'image est simple pour
l'un si et seulement si~il est simple pour l'autre.

Remarque~18.1.3 Si on a F = G \cdot \theta, on a F'(t_0) =
\theta'(t_0)G'(\theta(t_0)). Si \theta est un difféomorphisme, on a
\theta'(t_0)\neq~0 et donc
F'(t_0)\neq~0
\Leftrightarrow
G'(\theta(t_0))\neq~0. D'où

Proposition~18.1.2 Soit (I,F) et (J,G) deux arcs paramétrés de classe
C^k qui sont C^k-équivalents, \theta : I \rightarrow~ J un
difféomorphisme de classe C^k tel que F = G \cdot \theta. Alors
t_0 est un point régulier de (I,F) si et seulement
si~\theta(t_0) est un point régulier de (J,G). En particulier, (I,F)
est régulier si et seulement si~(J,G) l'est.

\paragraph{18.1.3 Orientation}

Un difféomorphisme d'un intervalle de \mathbb{R}~ sur un autre intervalle de \mathbb{R}~ est
soit croissant soit décroissant. On peut donc définir

Définition~18.1.3 Soit (I,F) et (J,G) deux arcs paramétrés de classe
C^k qui sont C^k-équivalents, \theta : I \rightarrow~ J un
difféomorphisme de classe C^k tel que F = G \cdot \theta. On dit que
(I,F) et (J,G) sont de même sens si \theta est croissant, de sens contraire
si \theta est décroissant.

Remarque~18.1.4 Dans certains cas, il peut arriver (de manière assez
exceptionnelle) qu'il existe deux difféomorphismes \theta_1 et
\theta_2 tels que F = G \cdot \theta_1 et F = G \cdot \theta_2, l'un
étant croissant et l'autre décroissant. Autrement dit deux arcs
paramétrés peuvent être à la fois de même sens et de sens contraire. On
dit alors que les arcs paramétrés ne sont pas orientables. Un exemple
typique est l'arc de \mathbb{R}~^2~: \Gamma =
(\mathbb{R}~,t\mapsto~(t^2,t^2)) où
l'arc est de sens contraire à lui même puisque F \cdot \theta = F pour \theta(t) = -t.
Le lecteur montrera facilement que cette situation ne peut pas se
produire dès que l'arc est régulier ou bien dès que l'arc a au moins
deux points simples.

\paragraph{18.1.4 Tangente à un arc paramétré}

Définition~18.1.4 Soit \Gamma = (I,F) un arc paramétré de classe
C^k, k \in \mathbb{N}~^∗\cup\ +
\infty~\. On dit que t_0 \in I est un point non
totalement singulier de \Gamma s'il existe n \leq k tel que
F^(n)(t_0)\neq~0.

Remarque~18.1.5 Soit \Gamma = (I,F) un arc paramétré de classe C^k
et t_0 \in I un point non totalement singulier de \Gamma. Considérons
p = min~\n \leq
k∣F^(n)(t_0)\mathrel\neq~0\.
La formule de Taylor montre alors que

F(t) = F(t_0) + (t - t_0)^p
\over p! F^(p)(t_ 0) + (t -
t_0)^p\epsilon(t - t_ 0)

avec lim_u\rightarrow~0~\epsilon(u) = 0. On en déduit
que lim_t\rightarrow~t_0~
F(t)-F(t_0) \over
(t-t_0)^p = 1 \over p!
F^(p)(t_ 0). Ceci montre tout d'abord que pour
t\neq~t_0 mais assez proche de
t_0, on a F(t)\neq~F(t_0) et
que d'autre part le vecteur  F(t)-F(t_0) \over
(t-t_0)^p qui est un vecteur directeur de la
droite affine qui passe par les points F(t) et F(t_0), admet
pour limite le vecteur  1 \over p!
F^(p)(t_ 0). On peut encore dire que la corde
\\jmathmathoignant les points F(t_0) et F(t) admet pour limite la droite
affine F(t_0) + \mathbb{R}~F^(p)(t_0) ce qui \\jmathmathustifie
la définition suivante~:

Définition~18.1.5 Soit \Gamma = (I,F) un arc paramétré de classe
C^k et t_0 \in I un point non totalement singulier de
\Gamma. Soit p = min~\n \leq
k∣F^(n)(t_0)\mathrel\neq~0\.
On appelle tangente à \Gamma au point t_0 la droite affine
F(t_0) + \mathbb{R}~F^(p)(t_0).

Proposition~18.1.3 La notion de tangente est invariante par changement
de paramétrage admissible. Soit (I,F) et (J,G) deux arcs paramétrés de
classe C^k qui sont C^k-équivalents, \theta : I \rightarrow~ J un
difféomorphisme de classe C^k tel que F = G \cdot \theta. Alors
t_0 est un point non totalement singulier de (I,F) si et
seulement si~\theta(t_0) est un point non totalement singulier de
(J,G), et dans ce cas la tangente à (I,F) au point t_0 est
égale à la tangente à (J,G) au point \theta(t_0).

Démonstration Nous allons montrer par récurrence sur n ≥ 1 que

F^(n)(t) = \theta'(t)^nG^(n)(\theta(t)) +
\sum _\\jmathmath=1^n-1a_
\\jmathmath,n(t)G^(\\jmathmath)(\theta(t))

où les applications a_\\jmathmath,n sont des applications de I dans \mathbb{R}~ de
classe C^k-n. C'est clair pour n = 1 puisque F'(t) =
\theta'(t)G'(\theta(t)). Supposons donc le résultat vrai pour n \leq k - 1. Alors
toutes les applications considérées étant de classe \mathcal{C}^1, on a

\begin{align*} F^(n+1)(t)& =&
\theta'(t)^n+1G^(n+1)(\theta(t)) +
n\theta'`(t)^n\theta'(t)^n-1G^(n)(\theta(t)) \%&
\\ & & +\\sum
_\\jmathmath=1^n-1a_ \\jmathmath,n(t)\theta'(t)G^(\\jmathmath+1)(\theta(t))
+ \sum _\\jmathmath=1^n-1a_
\\jmathmath,n'(t)G^(\\jmathmath)(\theta(t))\%& \\ &
=& \theta'(t)^n+1G^(n+1)(\theta(t)) +
\sum _\\jmathmath=1^na_
\\jmathmath,n+1(t)G^(\\jmathmath)(\theta(t)) \%& \\
\end{align*}

avec a_n,n+1(t) = a_n-1,n\theta'(t) +
n\theta'`(t)\theta'(t)^n-1 et pour \\jmathmath \leq n - 1, a_\\jmathmath,n+1(t) =
a_\\jmathmath,n(t)\theta'(t) + a_\\jmathmath-1,n'(t), ce qui achève la
récurrence.

Si on a alors G'(\theta(t_0)) =
\\ldots~ =
G^(p-1)(\theta(t_0)) = 0 et
G^(p)(\theta(t_0))\neq~0, on voit
immédiatement que F'(t_0) =
\\ldots~ =
F^(p-1)(t_0) = 0 et que
F^(p)(t_0) =
\theta'(t_0)^pG^(p)(\theta(t_0))\neq~0
car \theta'(t_0)\neq~0. On en déduit que si
\theta(t_0) est un point non totalement singulier de (J,G), alors
t_0 est un point non totalement singulier de (I,F) avec le même
indice p. L'équivalence en résulte vu le rôle symétrique des deux arcs.
De plus on a G(\theta(t_0)) + \mathbb{R}~G^(p)(\theta(t_0)) =
F(t_0) + \mathbb{R}~F^(p)(t_0) (puisque les deux
vecteurs tangents sont colinéaires) ce qui montre que les tangentes sont
les mêmes.

Remarque~18.1.6 Considérons F : \mathbb{R}~ \rightarrow~ \mathbb{R}~^2 définie par F(t) =
(e^-1\diagupt^2 ,0) si t \textgreater{} 0, F(0) = (0,0)
et F(t) = (0,e^-1\diagupt^2 ) si t \textless{} 0. On
constate facilement que F est de classe C^\infty~ en remarquant que
lim_t\rightarrow~0,t\neq~0F^(k)~(t)
= 0 et en appliquant un théorème du cours sur les fonctions d'une
variable. Pourtant, l'image de F est la réunion de deux segments faisant
en F(0) un angle droit. Il n'est donc pas question de pouvoir définir de
manière raisonnable une tangente à un arc (fût-il C^\infty~) en un
point totalement singulier.

\paragraph{18.1.5 Plan osculateur, concavité}

Définition~18.1.6 Soit \Gamma = (I,F) un arc paramétré de classe
C^k, k ≥ 2. On dit que t_0 \in I est un point
birégulier de \Gamma si la famille (F'(t_0),F''(t_0)) est
libre. Un arc est dit birégulier si tous ses points sont biréguliers.

Remarque~18.1.7 Un point birégulier est nécessairement régulier puisque
F'(t_0)\neq~0.

Définition~18.1.7 Soit \Gamma = (I,F) un arc paramétré de classe
C^k, k ≥ 2. Soit t_0 \in I un point birégulier de \Gamma.
On appelle plan osculateur au point t_0 le plan affine
F(t_0) +\
\mathrmVect(F'(t_0),F''(t_0)).

Théorème~18.1.4 Les notions de point birégulier et de plan osculateur
sont invariantes par changement de paramétrage admissible. Soit (I,F) et
(J,G) deux arcs paramétrés de classe C^k qui sont
C^k-équivalents, \theta : I \rightarrow~ J un difféomorphisme de classe
C^k tel que F = G \cdot \theta. Alors t_0 est un point
birégulier de (I,F) si et seulement si~\theta(t_0) est un point
birégulier de (J,G), et dans ce cas le plan osculateur à (I,F) au point
t_0 est égal au plan osculateur à (J,G) au point
\theta(t_0).

Démonstration On a F'(t_0) =
\theta'(t_0)G'(\theta(t_0)) et F'`(t_0) =
\theta'`(t_0)G'(\theta(t_0)) +
\theta'(t_0)^2G''(\theta(t_0)). Donc la matrice des
coordonnées de la famille (F'(t_0),F''(t_0)) par
rapport à la famille (G'(\theta(t_0)),G''(\theta(t_0))) est la
matrice inversible \left
(\matrix\,\theta'(t_0)&\theta'`(t_0)
\cr 0
&\theta'(t_0)^2\right ), ce qui montre
que (F'(t_0),F''(t_0)) libre équivaut à
(G'(\theta(t_0)),G''(\theta(t_0))) libre. Dans ce cas on voit
que
\mathrmVect(G'(\theta(t_0)),G'`(\theta(t_0~)))
=\
\mathrmVect(F'(t_0),F''(t_0)), ce
qui montre que les plans osculateurs qui passent tous deux par le point
image F(t_0) = G(\theta(t_0)) coïncident.

Remarque~18.1.8 De plus la coordonnée de F''(t_0) par rapport à
G''(\theta(t_0)) est égale à \theta'(t_0)^2
\textgreater{} 0 ce qui montre que les demi-plans F(t_0) +
\mathbb{R}~F'(t_0) + \mathbb{R}~^+∗F''(t_0) et
G(\theta(t_0)) + \mathbb{R}~G'(\theta(t_0)) +
\mathbb{R}~^+∗G''(\theta(t_0)) coïncident. Ce qui amène à
introduire

Définition~18.1.8 Soit \Gamma = (I,F) un arc paramétré de classe
C^k, k ≥ 2. Soit t_0 \in I un point birégulier de \Gamma.
On appelle demi-plan de concavité au point t_0 le demi plan
affine (inclus dans le plan osculateur et délimité par la tangente)
F(t_0) + \mathbb{R}~F'(t_0) + \mathbb{R}~^+∗F''(t_0).
Il est invariant par changement de paramétrage admissible.

\paragraph{18.1.6 Etude locale des arcs plans}

Nous supposerons dans ce paragraphe et dans les suivants que
dim~ E = 2. Soit \Gamma = (I,F) un arc paramétré de
classe C^k, k ≥ 2 et t_0 un point non totalement
singulier de l'arc. Soit p =\
min\n \leq
k∣F^(n)(t_0)\mathrel\neq~0\
et nous supposerons que l'on peut trouver n \in {[}p + 1,k{]} tel que la
famille (F^(p)(t_0),F^(n)(t_0))
soit une famille libre. Nous considérerons le plus petit entier n ayant
cette propriété, que nous noterons q. La famille
(F^(p)(t_0),F^(q)(t_0)) est donc
une base de E, et donc nous pouvons écrire F(t) - F(t_0) =
x(t)F^(p)(t_0) + y(t)F^(q)(t_0).
Les nombres réels x(t) et y(t) apparaissent donc comme les coordonnées
du point F(t) dans le repère affine
(F(t_0),(F^(p)(t_0),F^(q)(t_0))).

Des renseignements essentiels sur l'allure de la courbe au voisinage du
point t_0 sont fournis par l'étude des signes de x(t) et de
y(t) au voisinage de t_0, sachant que la courbe passe par le
point F(t_0) et est tangente au vecteur
F^(p)(t_0). La formule de Taylor Young donne

\begin{align*} F(t)& =& F(t_0) + (t -
t_0)^p \over p!
F^(p)(t_ 0) + \\sum
_n=p+1^q-1 (t - t_0)^n
\over n! F^(n)(t_ 0)\%&
\\ & & + (t -
t_0)^q \over q!
F^(q)(t_ 0) + (t - t_0)^q\epsilon(t -
t_ 0) \%& \\
\end{align*}

avec lim_u\rightarrow~0~\epsilon(u) = 0. En posant
F^(n)(t_0) =
\lambda_nF^(p)(t_0) pour p + 1 \leq n \leq q - 1 et
\epsilon(u) = \alpha~(u)F^(p)(t_0) +
\beta~(u)F^(q)(t_0), on a
lim_u\rightarrow~0~\alpha~(u)
= lim_u\rightarrow~0~\beta~(u) = 0 et

\begin{align*} F(t)& =& F(t_0) \%&
\\ & +& (t -
t_0)^p\left ( 1 \over
p! + \sum _n=p+1^q-1~ (t -
t_0)^n-p \over n! \lambda_n + (t
- t_0)^q-p\alpha~(t - t_ 0)\right
)F^(p)(t_ 0)\%& \\ &
+& (t - t_0)^q( 1 \over q! + \beta~(t
- t_0))F^(q)(t_ 0) \%&
\\ \end{align*}

ce qui montre que

\begin{align*} x(t)& =& (t -
t_0)^p\left ( 1 \over
p! + \sum _n=p+1^q-1~ (t -
t_0)^n-p \over n! \lambda_n + (t
- t_0)^q-p\alpha~(t - t_ 0)\right
)\%& \\ & ∼& (t -
t_0)^p \over p! \%&
\\ \end{align*}

et que

y(t) = (t - t_0)^q( 1 \over q! +
\beta~(t - t_0)) ∼ (t - t_0)^q
\over q!

Mais quand deux fonctions sont équivalentes au voisinage de
t_0, elles ont même signe au voisinage de t_0 ce qui
montre qu'il existe \eta \textgreater{} 0 tel que pour t \in{]}t_0 -
\eta,t_0 + \eta{[} on ait
\mathrmsgn~ (x(t))
= \mathrmsgn~ ((t -
t_0)^p) et
\mathrmsgn~ (y(t))
= \mathrmsgn~ ((t -
t_0))^q en posant
\mathrmsgn~ (u) =
\left \ \cases 1 &si u
\textgreater{} 0 \cr 0 &si u = 0 \cr
-1&si u \textless{} 0  \right .. Ceci conduit à la
discussion suivante

Premier cas~: p impair, q pair (c'est par exemple le cas d'un point
birégulier pour lequel p = 1 et q = 2) Pour t \in{]}t_0 -
\eta,t_0{[}, on a x(t) \textless{} 0 et y(t) \textgreater{} 0~;
pour t \in{]}t_0,t_0 + \eta{[}, on a x(t) \textgreater{} 0
et y(t) \textgreater{} 0. La courbe reste d'un même coté de sa tangente
F(t_0) + \mathbb{R}~F^(p)(t_0) tout en traversant la
droite F(t_0) + \mathbb{R}~F^(q)(t_0)~; on dit alors
que t_0 est un point banal de \Gamma.

\text\includegraphics{cours12x.png}

Deuxième cas~: p impair, q impair (c'est le cas le plus courant de point
non birégulier avec p = 1 et q = 3) Pour t \in{]}t_0 -
\eta,t_0{[}, on a x(t) \textless{} 0 et y(t) \textless{} 0~; pour
t \in{]}t_0,t_0 + \eta{[}, on a x(t) \textgreater{} 0 et
y(t) \textgreater{} 0. La courbe traverse sa tangente F(t_0) +
\mathbb{R}~F^(p)(t_0) tout en traversant la droite
F(t_0) + \mathbb{R}~F^(q)(t_0)~; on dit alors que
t_0 est un point d'inflexion de \Gamma.

\text\includegraphics{cours13x.png}

Troisième cas~: p pair, q impair (c'est le cas le plus courant de point
singulier avec p = 2 et q = 3) Pour t \in{]}t_0 -
\eta,t_0{[}, on a x(t) \textgreater{} 0 et y(t) \textless{} 0~;
pour t \in{]}t_0,t_0 + \eta{[}, on a x(t) \textgreater{} 0
et y(t) \textgreater{} 0. La courbe traverse sa tangente F(t_0)
+ \mathbb{R}~F^(p)(t_0) tout en restant d'un même coté de la
droite F(t_0) + \mathbb{R}~F^(q)(t_0)~; on dit alors
que t_0 est un point de rebroussement de première espèce de \Gamma.

\text\includegraphics{cours14x.png}

Quatrième cas~: p pair, q pair Pour t \in{]}t_0 -
\eta,t_0{[}, on a x(t) \textgreater{} 0 et y(t) \textgreater{} 0
et pour t \in{]}t_0,t_0 + \eta{[}, on a x(t) \textgreater{}
0 et y(t) \textgreater{} 0. La courbe reste d'un même coté de sa
tangente F(t_0) + \mathbb{R}~F^(p)(t_0) tout en
restant d'un même coté de la droite F(t_0) +
\mathbb{R}~F^(q)(t_0)~; on dit alors que t_0 est un
point de rebroussement de deuxième espèce de \Gamma.

\text\includegraphics{cours15x.png}

On a en particulier au passage démontré le résultat suivant

Théorème~18.1.5 (convexité locale). Soit \Gamma = (I,F) un arc paramétré de
classe C^k, k ≥ 2. Soit t_0 \in I un point birégulier
de \Gamma. Alors il existe un \eta \textgreater{} 0 tel que pour t
\in{]}t_0 - \eta,t_0 +
\eta{[}\diagdown\t_0\, F(t) soit dans le
demi plan ouvert de concavité.

Remarque~18.1.9 On vérifie facilement que les entiers p et q définis
ci-dessus sont invariants par changement de paramétrage admissible. Il
en est donc de même des notions de point banal, point d'inflexion, point
de rebroussement de première et deuxième espèce.

\paragraph{18.1.7 Branches infinies}

Définition~18.1.9 Soit \Gamma = (I,F) un arc paramétré de classe
C^k et \alpha~ \in \mathbb{R}~ \cup\-\infty~,+\infty~\ une
extrémité de I. On dit que \Gamma admet en \alpha~ une branche infinie si
lim_t\rightarrow~\alpha~~\F(t)\
= +\infty~.

Définition~18.1.10 Soit \Gamma = (I,F) un arc paramétré de classe
C^k et \alpha~ \in \mathbb{R}~ \cup\-\infty~,+\infty~\ une
extrémité de I où \Gamma admet une branche infinie. Si
lim_t\rightarrow~\alpha~~ F(t) \over
\F(t)\
=\vec u (vecteur nécessairement unitaire), on dit que
\Gamma admet en \alpha~ la droite \mathbb{R}~\vecu comme direction
asymptotique.

Définition~18.1.11 Soit \Gamma = (I,F) un arc paramétré de classe
C^k et \alpha~ \in \mathbb{R}~ \cup\-\infty~,+\infty~\ une
extrémité de I où \Gamma admet une branche infinie. Soit D une droite affine
de E. On dit que \Gamma admet en \alpha~ l'asymptote D si
lim_t\rightarrow~\alpha~~d(F(t),D) = 0 (où d(F(t),D)
désigne la distance du point F(t) au point D) (distance associée à une
norme quelconque sur E).

Proposition~18.1.6 Soit \Gamma = (I,F) un arc paramétré de classe
C^k et \alpha~ \in \mathbb{R}~ \cup\-\infty~,+\infty~\ une
extrémité de I où \Gamma admet une branche infinie. Si \Gamma admet en \alpha~ une
droite D comme asymptote, il admet sa direction \vecD
comme direction asymptotique.

Démonstration Soit a un point de D. Utilisons une distance associée à
une norme euclidienne sur E et soit P(t) la pro\\jmathmathection orthogonale de
F(t) sur la droite D. On a alors d(F(t),D) =\
F(t) - P(t)\. On a
\P(t)\
≥\ F(t)\
-\ P(t) - F(t)\ ce qui
montre que
lim~\P(t)\
= +\infty~. On en déduit que pour t assez proche de \alpha~, le point P(t) reste
d'un même côté de a et que donc le vecteur 
\overrightarrowaP(t) \over
\\overrightarrowaP(t)\
est constant égal à un vecteur unitaire \vecu de
\vecD. On a alors

\begin{align*} F(t) \over
\F(t)\ & =& F(t)
- P(t) + a \over
\F(t)\ +
\\overrightarrowaP(t)\
\over
\F(t)\ 
\overrightarrowaP(t) \over
\\overrightarrowaP(t)\
\%& \\ & =& F(t) - P(t)
\over
\F(t)\ +
\\overrightarrowaP(t)\
\over
\F(t)\
\vecu \%& \\
\end{align*}

avec lim~ F(t)-P(t)+a \over
\F(t)\ = 0 puisque
le numérateur tend vers a et le dénominateur vers + \infty~, et
lim~
\\overrightarrowaP(t)\
\over
\F(t)\ = 1 comme le
montre l'encadrement

\begin{align*}
\F(t)\
-\ P(t) - F(t)\
-\ a\ \leq&& \%&
\\
\\overrightarrowaP(t)&
=& \P(t) - a\
\leq\ F(t)\
+\ P(t) - F(t)\
+\ a\\%&
\\ \end{align*}

qui implique

1 - \P(t) - F(t)\
\over
\F(t)\ -
\a\
\over
\F(t)\ \leq
\\overrightarrowaP(t)\
\over
\F(t)\ \leq 1 +
\P(t) - F(t)\
\over
\F(t)\ +
\a\
\over
\F(t)\

On a donc lim~ F(t) \over
\F(t)\
=\vec u ce qui achève la démonstration.

Proposition~18.1.7 Soit \Gamma = (I,F) un arc paramétré de classe
C^k et \alpha~ \in \mathbb{R}~ \cup\-\infty~,+\infty~\ une
extrémité de I où \Gamma admet une branche infinie. Soit
\vecD une direction de droite, \Pi un hyperplan affine
non parallèle à \vecD, m un point de \Pi. Pour t \in I,
soit m_t le point d'intersection de la parallèle à
\vecD mené par F(t) avec l'hyperplan \Pi. On a
équivalence de (i) \Gamma admet en \alpha~ la droite D = m +\vec
D comme asymptote (ii)
lim_t\rightarrow~\alpha~m_t~ = m.

Démonstration Mettons sur E une structure euclidienne telle que
l'hyperplan \Pi soit orthogonal à \vecD et utilisons la
distance d correspondante. Alors la distance de F(t) à D est égale à la
distance de m_t à m ce qui montre le résultat.

Etude des branches infinies Si \Gamma admet en \alpha~ une branche infinie, on
commence par regarder si  F(t) \over
\F(t)\ admet une
limite. Si c'est le cas, soit \vecu cette limite et
\vecD = \mathbb{R}~\vecu. On choisit pour \Pi
un hyperplan affine non parallèle à \vecD et on
recherche le point m_t d'intersection de la droite F(t) +
\mathbb{R}~\vecu avec \Pi~; si le point m_t admet en \alpha~
une limite m, la droite m + \mathbb{R}~\vecu est asymptote à la
courbe, sinon la courbe n'admet pas d'asymptote en \alpha~. Pour la simplicité
des calculs, on est souvent amené à choisir pour \Pi l'un des hyperplans
de coordonnées.

\paragraph{18.1.8 Plan d'étude d'un arc plan en paramétriques}

Ici, E est un \mathbb{R}~ espace vectoriel normé de dimension 2, soit
(\vec\imath,\vecȷ) une base de E. Soit
F une fonction de \mathbb{R}~ vers E. On notera (lorsque cela a un sens) F(t) =
\phi(t)\vec\imath + \psi(t)\vecȷ.

Première étape~: domaine de définition On détermine le domaine de
définition D = Def~ (\phi)
\bigcap Def~ (\psi) de F~; c'est en général une réunion
finie d'intervalles deux à deux dis\\jmathmathoints, si bien que la courbe étudiée
sera une réunion finie d'arcs paramétrés.

Deuxième étape~: réduction du domaine d'étude Supposons qu'il existe une
application \theta : D\rightarrow~D tel que pour t \inD, F(\theta(t)) se déduise par une
transformation géométrique simple S de F(t), et soit \Delta un domaine
fondamental de \theta dans D c'est-à-dire une partie de D telle que D
= \⋃ ~
_n\in\mathbb{N}~\theta^n(\Delta) (ou éventuellement si \theta est bi\\jmathmathective, D
= \⋃ ~
_n\in\mathbb{Z}\theta^n(\Delta)). En se fondant sur la relation
F(\theta^n(t)) = S^n(F(t)), on voit qu'il suffit
d'étudier la courbe pour t \in \Delta pour en déduire simplement l'étude sur
\theta^n(\Delta) et donc sur D.

On recherchera principalement des transformations \theta du type

\begin{itemize}
\itemsep1pt\parskip0pt\parsep0pt
\item
  (i) \theta(t) = t + T avec alors \Delta = {[}a,a + T{]} \bigcapD
\item
  (ii) \theta(t) = \omega - t avec alors \Delta = {[} \omega \over 2
  ,+\infty~{[}\bigcapD
\item
  (iii) éventuellement \theta(t) = 1 \over t avec par
  exemple \Delta = {[}-1,1{]} \bigcapD
\end{itemize}

En ce qui concerne les transformations géométriques simples, on repérera
principalement

\begin{itemize}
\itemsep1pt\parskip0pt\parsep0pt
\item
  (i) l'identité \phi(\theta(t)) = \phi(t), \psi(\theta(t)) = \psi(t)
\item
  (ii) des symétries

  \begin{itemize}
  \itemsep1pt\parskip0pt\parsep0pt
  \item
    a) \phi(\theta(t)) = -\phi(t), \psi(\theta(t)) = \psi(t)~: symétrie par rapport à l'axe Oy
  \item
    b) \phi(\theta(t)) = \phi(t), \psi(\theta(t)) = -\psi(t)~: symétrie par rapport à l'axe Ox
  \item
    c) \phi(\theta(t)) = -\phi(t), \psi(\theta(t)) = -\psi(t)~: symétrie par rapport à
    l'origine O.
  \item
    d) \phi(\theta(t)) = \psi(t), \psi(\theta(t)) = \phi(t)~: symétrie par rapport à la
    première diagonale
  \end{itemize}
\item
  (iii) des translations \phi(\theta(t)) = \phi(t) + \alpha~, \psi(\theta(t)) = \psi(t) + \beta~~:
  translation de vecteur \alpha~\vec\imath +
  \beta~\vecȷ
\end{itemize}

Troisième étape~: variations de \phi et \psi On étudie les signes des dérivées
\phi' et \psi'. Au passage on repère un certain nombre de points remarquables

\begin{itemize}
\itemsep1pt\parskip0pt\parsep0pt
\item
  (i) \phi'(t_0) = 0,
  \psi'(t_0)\neq~0~: point régulier où la
  tangente est verticale
\item
  (ii) \phi'(t_0)\neq~0, \psi'(t_0) =
  0~: point régulier où la tangente est horizontale
\item
  (iii) \phi'(t_0) = 0, \psi'(t_0) = 0~: point singulier
\end{itemize}

Quatrième étape~: étude des points singuliers Une étude locale soit à
l'aide de dérivées successives, soit de préférence avec un développement
limité permet de déterminer les entiers p et q introduits ci dessus et
de déterminer en conséquence le type de point singulier~: point de
rebroussement de première espèce, point de rebroussement de deuxième
espèce, point banal ou point d'inflexion (dans l'ordre décroissant des
fréquences)~; au passage on détermine la tangente en un point singulier.

Cinquième étape~: étude des branches infinies L'arc admet en \alpha~
\in\overlineD une branche infinie si et seulement si
l'une au moins des deux coordonnées \phi(t) et \psi(t) admet une limite \infty~.

\begin{itemize}
\itemsep1pt\parskip0pt\parsep0pt
\item
  (i) Si lim~\phi(t) = +\infty~ et
  \psi(t) est bornée, l'arc admet une direction asymptotique horizontale~;
  il admet la droite y = y_0 comme asymptote si et seulement
  si~lim_t\rightarrow~\alpha~\psi(t) = y_0~.
\item
  (ii) Si lim~\psi(t) = +\infty~ et
  \phi(t) est bornée, l'arc admet une direction asymptotique verticale~; il
  admet la droite x = x_0 comme asymptote si et seulement
  si~lim_t\rightarrow~\alpha~\phi(t) = x_0~.
\item
  (iii) Si lim~\phi(t) = +\infty~ et
  lim~\psi(t) = +\infty~, on étudie
  le rapport  \psi(t) \over \phi(t)

  \begin{itemize}
  \itemsep1pt\parskip0pt\parsep0pt
  \item
    a) s'il admet une limite \infty~, l'arc admet la verticale comme direction
    asymptotique et il n'y a pas d'asymptote
  \item
    b) s'il admet la limite 0, l'arc admet l'horizontale comme direction
    asymptotique et il n'y a pas d'asymptote
  \item
    c) s'il admet la limite a\neq~0, l'arc admet
    la droite y = ax comme direction asymptotique et la droite
    d'équation y = ax + b est asymptote si et seulement
    si~lim_t\rightarrow~\alpha~~(\psi(t) - a\phi(t)) = b
  \item
    d) dans tous les autres cas, il n'y a pas de direction asymptotique
  \end{itemize}
\end{itemize}

Une étude complémentaire de signe peut parfois préciser la position du
point F(t) par rapport à une asymptote, ce qui peut permettre de
préciser un tracé.

Sixième étape~: ébauche de tracé En s'aidant d'une calculatrice ou d'un
ordinateur, on peut calculer un certain nombre de points supplémentaires
en plus des points remarquables~; ceci, en plus des points et des
tangentes remarquables, des variations de \phi et \psi et de l'étude des
branches infinies permet en général une ébauche convaincante du tracé.

Etapes facultatives

Si la question est posée ou si l'ébauche du tracé suggère la nécessité
de certaines précisions, on peut procéder à quelques étapes
supplémentaires

Septième étape~: détermination des points non biréguliers Il suffit
d'écrire \mathrm{det}~
_(\vec\imath,\vecȷ)(F'(t),F''(t))
= 0, c'est-à-dire \left
\matrix\,\phi'(t)&\psi'(t)
\cr \phi'`(t)&\psi''(t)\right ,
soit encore, si \psi' ne s'annule pas  d \over dt
\left ( \phi'(t) \over \psi'(t)
\right ) = 0~; ces points, une fois éliminés les points
singuliers, sont souvent des points d'inflexion.

Huitième étape~: détermination des points multiples Il s'agit de
résoudre l'équation F(t) = F(t') ou encore le système \phi(t) = \phi(t'), \psi(t)
= \psi(t') pour t\neq~t'.

\paragraph{18.1.9 Notion de contact}

Soit \Gamma = (I,F) un arc paramétré de classe C^k de
\mathbb{R}~^n~; on pose F(t) =
(f_1(t),\\ldots,f_n~(t)).
Soit U un ouvert contenant l'image de \Gamma, G une application de classe
C^k de U dans \mathbb{R}~, \Sigma =
\(x_1,\\ldots,x_n~)
\in
U∣G(x_1,\\ldots,x_n~)
= 0\ l'hypersurface correspondante de \mathbb{R}~^n.
On pose \phi(t) =
G(f_1(t),\\ldots,f_n~(t))~;
donc \phi est une application de classe C^k de I dans \mathbb{R}~.

Définition~18.1.12 On dit que \Gamma et \Sigma ont au point t_0 \in I de \Gamma
un contact d'ordre au moins p si \phi(t_0) = \phi'(t_0) =
\\ldots~ =
\phi^(p-1)(t_0) = 0. On dit en particulier que \Gamma et \Sigma
sont sécantes (resp. tangentes, resp. osculatrices, resp.
surosculatrices) en t_0 si elles ont en t_0 un contact
d'ordre au moins 1 (resp. 2, resp. 3, resp. 4).

Remarque~18.1.10 On voit que \Gamma et \Sigma sont sécantes en t_0 si et
seulement si~F(t_0) \in \Sigma ce qui est bien naturel. Pour que \Gamma et
\Sigma soient tangentes en t_0, il faut de plus que \phi'(t_0)
= 0. Mais \phi'(t_0) =\
\sum ~
_i=1^nf_i'(t_0) \partial~G
\over \partial~x_i (F(t_0)), c'est-à-dire que
\Gamma et \Sigma sont tangentes en t_0, si et seulement si
F(t_0) \in \Sigma et le vecteur F'(t_0) =
(f_1'(t_0),\\ldots,f_n'(t_0~))
appartient à l'hyperplan \Pi d'équation
\\sum ~
_i=1^nx_i \partial~G \over
\partial~x_i (F(t_0)) = 0 (si
\mathrmgrad~
G(F(t_0))\neq~0)~; or le vecteur
F'(t_0) détermine en général la tangente à \Gamma en t_0 et
l'hyperplan \Pi est en général (comme on l'a vu à l'occasion du théorème
des fonctions implicites) l'hyperplan tangent à \Sigma au point
F(t_0) ce qui \\jmathmathustifie le vocabulaire employé.

Remarque~18.1.11 Soit F(t) = x(t)\vec\imath +
y(t)\vecȷ un arc paramétré plan~; soit t_0
un point régulier de l'arc et recherchons le contact en t_0 de
\Gamma avec une droite D d'équation ax + by + c = 0~; la direction
\vecD de la droite D a pour équation ax + by = 0. On
a alors \phi(t) = ax(t) + by(t) + c. On voit alors que, si on appelle p
l'ordre du contact, on a (i) p ≥ 1 \Leftrightarrow
F(t_0) \in D (ii) p ≥ 2 \Leftrightarrow
F(t_0) \in D,F'(t_0) \in\vec D~: seule
la tangente en t_0 a un contact d'ordre au moins 2 (iii) p ≥ 3
\Leftrightarrow F(t_0) \in D,F'(t_0)
\in\vec D,F''(t_0) \in\vec
D~: en général aucune droite ne répond à ces exigences, à moins que le
point ne soit pas birégulier.

On obtient donc que les points réguliers non biréguliers sont les points
de \Gamma où existent une droite osculatrice ce qui peut fournir un autre
moyen de recherche des points non biréguliers (en général des points
d'inflexion) en étudiant la multiplicité d'intersection d'une droite D
avec l'arc, dans la mesure où cela a un sens, et en particulier lorsque
x(t) et y(t) sont des fractions rationnelles en t.

Sur le même modèle, le lecteur montrera facilement qu'en un point
birégulier d'un arc de \mathbb{R}~^3, il existe un seul plan qui soit
osculateur à l'arc, à savoir le plan osculateur défini précédemment.

\paragraph{18.1.10 Enveloppes}

Ce paragraphe ne fait pas partie du programme des classes préparatoires.

Soit (D_t)_t\inI une famille de droites de
\mathbb{R}~^2 indexée par un intervalle I de \mathbb{R}~. Intuitivement, on
appellera enveloppe de la famille de droites un arc \Gamma = (I,F) telle que
la tangente à \Gamma au point t soit la droite D_t. Nous allons
préciser cette définition de la matière suivante

Définition~18.1.13 Soit I un intervalle de \mathbb{R}~, a,b,c trois applications
de classe C^2 de I dans \mathbb{R}~ telles que
\forall~~t \in I,
(a(t),b(t))\neq~(0,0). Pour t \in I soit
D_t la droite de \mathbb{R}~^2 d'équation a(t)x + b(t)y + c(t)
= 0. On appelle enveloppe de la famille de droite D_t tout arc
paramétré (I,F) de classe \mathcal{C}^1 tel que

\forall~~t \in I, F(t) \in
D_t\text et F'(t)
\in\overrightarrow D_t

Remarque~18.1.12 En un point singulier de l'arc, on a F'(t) = 0 et la
condition F'(t) \in\overrightarrow D_t est
automatiquement vérifiée. En un tel point, la droite D_t n'a
donc aucune raison d'être la tangente à l'arc (I,F). Nous allons voir
cependant que c'est souvent le cas, sous des hypothèses raisonnables.

Théorème~18.1.8 On suppose que \forall~~t \in I,
a(t)b'(t) - a'(t)b(t)\neq~0. Alors la famille de
droites D_t d'équations a(t)x + b(t)y + c(t) = 0 admet une
unique enveloppe (I,F)~; pour tout t \in I, le point F(t) est le point
d'intersection de la droite D_t avec la droite D_t'
d'équation a'(t)x + b'(t)y + c'(t) = 0~; c'est aussi la limite quand h
tend vers 0 du point d'intersection de la droite D_t+h avec la
droite D_t. Pour tout point non totalement singulier t de l'arc
(I,F), la droite D_t est la tangente à l'arc au point t.

Démonstration Posons F(t) = (x(t),y(t)). Les conditions de la définition
de l'enveloppe se traduisent par \forall~~t \in I,
a(t)x(t) + b(t)y(t) + c(t) = 0, a(t)x'(t) + b(t)y'(t) = 0. Mais si la
première condition est vérifiée, par dérivation on a
\forall~~t \in I,a'(t)x(t) + b'(t)y(t) + c'(t) +
a(t)x'(t) + b(t)y'(t) = 0 si bien que la seconde condition est
équivalente à \forall~~t \in I, a'(t)x + b'(t)y + c'(t)
= 0. On obtient donc

\forall~~t \in I, \left
\ \cases a(t)x(t) + b(t)y(t) + c(t) =
0 \cr a'(t)x(t) + b'(t)y(y) + c'(t) = 0 
\right .

système aux inconnues x(t) et y(t) qui est un système de Cramer puisque
a(t)b'(t) - a'(t)b(t)\neq~0. Ceci montre dé\\jmathmathà
l'unicité de l'enveloppe et le fait que D_t \bigcap D_t' =
\F(t)\. Inversement, si F(t) est ainsi
défini, comme a,b et c sont de classe C^2, x(t) et y(t) sont
de classe \mathcal{C}^1 et le même calcul que ci dessus en sens inverse
montre que \forall~~t \in I, a(t)x(t) + b(t)y(t) + c(t)
= 0, a(t)x'(t) + b(t)y'(t) = 0, donc (I,F) est bien enveloppe de la
famille D_t.

Soit maintenant h\neq~0. L'équation aux
coordonnées des points d'intersection des droites D_t et
D_t+h est donnée par

\begin{align*} \left
\ \cases a(t)x + b(t)y + c(t) = 0
\cr a(t + h)x + b(t + h)y + c(t + h) = 0 
\right . \Leftrightarrow&& \%&
\\ & \left
\ \cases a(t)x + b(t)y + c(t) = 0
\cr  a(t+h)-a(t) \over h x +
b(t+h)-b(t) \over h y + c(t+h)-c(t)
\over h = 0  \right .&
\%&\\ \end{align*}

La limite du déterminant de ce système est a(t)b'(t) -
a'(t)b(t)\neq~0, donc pour h assez petit ce
déterminant est non nul et définit un unique point F_h(t) de
coordonnées (x_h(t),y_h(t)). Les formules de Cramer
montrent aussitôt que
lim_h\rightarrow~0F_h~(t) = F(t).

Supposons maintenant que a,b et c soient de classe C^k+1~;
alors F est de classe C^k. Soit t_0 un point non
totalement singulier de (I,F), et soit p tel que F'(t_0) =
\\ldots~ =
F^(p-1)(t_0) = 0,
F^(p)(t_0)\neq~0. On a
\forall~~t \in I,a(t)x'(t) + b(t)y'(t) = 0. En
appliquant la formule de Leibnitz, on a

\begin{align*} 0& =& d^p-1
\over dt^p-1 (a(t)x'(t) +
b(t)y'(t))_t=t_0 \%& \\
& =& \sum _n=0^p-1C_
p-1^n\left (a^(n)(t_
0)x^(p-n)(t_ 0) + b^(n)(t_
0)y^(p-n)(t_ 0)\right )\%&
\\ & =& a(t)x^(p)(t_
0) + b(t)y^(p)(t_ 0) \%&
\\ \end{align*}

puisque toutes les dérivées précédentes de x et y sont nulles au point
t_0. On en déduit que F^(p)(t_0)
\in\overrightarrow D_t_0. La droite
D_t_0 contient le point F(t_0) et est
parallèle au vecteur tangent F^(p)(t_0)~; c'est donc
la tangente au point t_0. Ceci achève la démonstration du
théorème.

{[}
{[}

\end{document}

% \documentclass[]{article}
\usepackage[T1]{fontenc}
\usepackage{lmodern}
\usepackage{amssymb,amsmath}
\usepackage{ifxetex,ifluatex}
\usepackage{fixltx2e} % provides \textsubscript
% use upquote if available, for straight quotes in verbatim environments
\IfFileExists{upquote.sty}{\usepackage{upquote}}{}
\ifnum 0\ifxetex 1\fi\ifluatex 1\fi=0 % if pdftex
  \usepackage[utf8]{inputenc}
\else % if luatex or xelatex
  \ifxetex
    \usepackage{mathspec}
    \usepackage{xltxtra,xunicode}
  \else
    \usepackage{fontspec}
  \fi
  \defaultfontfeatures{Mapping=tex-text,Scale=MatchLowercase}
  \newcommand{\euro}{€}
\fi
% use microtype if available
\IfFileExists{microtype.sty}{\usepackage{microtype}}{}
\ifxetex
  \usepackage[setpagesize=false, % page size defined by xetex
              unicode=false, % unicode breaks when used with xetex
              xetex]{hyperref}
\else
  \usepackage[unicode=true]{hyperref}
\fi
\hypersetup{breaklinks=true,
            bookmarks=true,
            pdfauthor={},
            pdftitle={Arcs en polaires},
            colorlinks=true,
            citecolor=blue,
            urlcolor=blue,
            linkcolor=magenta,
            pdfborder={0 0 0}}
\urlstyle{same}  % don't use monospace font for urls
\setlength{\parindent}{0pt}
\setlength{\parskip}{6pt plus 2pt minus 1pt}
\setlength{\emergencystretch}{3em}  % prevent overfull lines
\setcounter{secnumdepth}{0}
 
/* start css.sty */
.cmr-5{font-size:50%;}
.cmr-7{font-size:70%;}
.cmmi-5{font-size:50%;font-style: italic;}
.cmmi-7{font-size:70%;font-style: italic;}
.cmmi-10{font-style: italic;}
.cmsy-5{font-size:50%;}
.cmsy-7{font-size:70%;}
.cmex-7{font-size:70%;}
.cmex-7x-x-71{font-size:49%;}
.msbm-7{font-size:70%;}
.cmtt-10{font-family: monospace;}
.cmti-10{ font-style: italic;}
.cmbx-10{ font-weight: bold;}
.cmr-17x-x-120{font-size:204%;}
.cmsl-10{font-style: oblique;}
.cmti-7x-x-71{font-size:49%; font-style: italic;}
.cmbxti-10{ font-weight: bold; font-style: italic;}
p.noindent { text-indent: 0em }
td p.noindent { text-indent: 0em; margin-top:0em; }
p.nopar { text-indent: 0em; }
p.indent{ text-indent: 1.5em }
@media print {div.crosslinks {visibility:hidden;}}
a img { border-top: 0; border-left: 0; border-right: 0; }
center { margin-top:1em; margin-bottom:1em; }
td center { margin-top:0em; margin-bottom:0em; }
.Canvas { position:relative; }
li p.indent { text-indent: 0em }
.enumerate1 {list-style-type:decimal;}
.enumerate2 {list-style-type:lower-alpha;}
.enumerate3 {list-style-type:lower-roman;}
.enumerate4 {list-style-type:upper-alpha;}
div.newtheorem { margin-bottom: 2em; margin-top: 2em;}
.obeylines-h,.obeylines-v {white-space: nowrap; }
div.obeylines-v p { margin-top:0; margin-bottom:0; }
.overline{ text-decoration:overline; }
.overline img{ border-top: 1px solid black; }
td.displaylines {text-align:center; white-space:nowrap;}
.centerline {text-align:center;}
.rightline {text-align:right;}
div.verbatim {font-family: monospace; white-space: nowrap; text-align:left; clear:both; }
.fbox {padding-left:3.0pt; padding-right:3.0pt; text-indent:0pt; border:solid black 0.4pt; }
div.fbox {display:table}
div.center div.fbox {text-align:center; clear:both; padding-left:3.0pt; padding-right:3.0pt; text-indent:0pt; border:solid black 0.4pt; }
div.minipage{width:100%;}
div.center, div.center div.center {text-align: center; margin-left:1em; margin-right:1em;}
div.center div {text-align: left;}
div.flushright, div.flushright div.flushright {text-align: right;}
div.flushright div {text-align: left;}
div.flushleft {text-align: left;}
.underline{ text-decoration:underline; }
.underline img{ border-bottom: 1px solid black; margin-bottom:1pt; }
.framebox-c, .framebox-l, .framebox-r { padding-left:3.0pt; padding-right:3.0pt; text-indent:0pt; border:solid black 0.4pt; }
.framebox-c {text-align:center;}
.framebox-l {text-align:left;}
.framebox-r {text-align:right;}
span.thank-mark{ vertical-align: super }
span.footnote-mark sup.textsuperscript, span.footnote-mark a sup.textsuperscript{ font-size:80%; }
div.tabular, div.center div.tabular {text-align: center; margin-top:0.5em; margin-bottom:0.5em; }
table.tabular td p{margin-top:0em;}
table.tabular {margin-left: auto; margin-right: auto;}
div.td00{ margin-left:0pt; margin-right:0pt; }
div.td01{ margin-left:0pt; margin-right:5pt; }
div.td10{ margin-left:5pt; margin-right:0pt; }
div.td11{ margin-left:5pt; margin-right:5pt; }
table[rules] {border-left:solid black 0.4pt; border-right:solid black 0.4pt; }
td.td00{ padding-left:0pt; padding-right:0pt; }
td.td01{ padding-left:0pt; padding-right:5pt; }
td.td10{ padding-left:5pt; padding-right:0pt; }
td.td11{ padding-left:5pt; padding-right:5pt; }
table[rules] {border-left:solid black 0.4pt; border-right:solid black 0.4pt; }
.hline hr, .cline hr{ height : 1px; margin:0px; }
.tabbing-right {text-align:right;}
span.TEX {letter-spacing: -0.125em; }
span.TEX span.E{ position:relative;top:0.5ex;left:-0.0417em;}
a span.TEX span.E {text-decoration: none; }
span.LATEX span.A{ position:relative; top:-0.5ex; left:-0.4em; font-size:85%;}
span.LATEX span.TEX{ position:relative; left: -0.4em; }
div.float img, div.float .caption {text-align:center;}
div.figure img, div.figure .caption {text-align:center;}
.marginpar {width:20%; float:right; text-align:left; margin-left:auto; margin-top:0.5em; font-size:85%; text-decoration:underline;}
.marginpar p{margin-top:0.4em; margin-bottom:0.4em;}
.equation td{text-align:center; vertical-align:middle; }
td.eq-no{ width:5%; }
table.equation { width:100%; } 
div.math-display, div.par-math-display{text-align:center;}
math .texttt { font-family: monospace; }
math .textit { font-style: italic; }
math .textsl { font-style: oblique; }
math .textsf { font-family: sans-serif; }
math .textbf { font-weight: bold; }
.partToc a, .partToc, .likepartToc a, .likepartToc {line-height: 200%; font-weight:bold; font-size:110%;}
.chapterToc a, .chapterToc, .likechapterToc a, .likechapterToc, .appendixToc a, .appendixToc {line-height: 200%; font-weight:bold;}
.index-item, .index-subitem, .index-subsubitem {display:block}
.caption td.id{font-weight: bold; white-space: nowrap; }
table.caption {text-align:center;}
h1.partHead{text-align: center}
p.bibitem { text-indent: -2em; margin-left: 2em; margin-top:0.6em; margin-bottom:0.6em; }
p.bibitem-p { text-indent: 0em; margin-left: 2em; margin-top:0.6em; margin-bottom:0.6em; }
.paragraphHead, .likeparagraphHead { margin-top:2em; font-weight: bold;}
.subparagraphHead, .likesubparagraphHead { font-weight: bold;}
.quote {margin-bottom:0.25em; margin-top:0.25em; margin-left:1em; margin-right:1em; text-align:\jmathustify;}
.verse{white-space:nowrap; margin-left:2em}
div.maketitle {text-align:center;}
h2.titleHead{text-align:center;}
div.maketitle{ margin-bottom: 2em; }
div.author, div.date {text-align:center;}
div.thanks{text-align:left; margin-left:10%; font-size:85%; font-style:italic; }
div.author{white-space: nowrap;}
.quotation {margin-bottom:0.25em; margin-top:0.25em; margin-left:1em; }
h1.partHead{text-align: center}
.sectionToc, .likesectionToc {margin-left:2em;}
.subsectionToc, .likesubsectionToc {margin-left:4em;}
.subsubsectionToc, .likesubsubsectionToc {margin-left:6em;}
.frenchb-nbsp{font-size:75%;}
.frenchb-thinspace{font-size:75%;}
.figure img.graphics {margin-left:10%;}
/* end css.sty */

\title{Arcs en polaires}
\author{}
\date{}

\begin{document}
\maketitle

\textbf{Warning: 
requires JavaScript to process the mathematics on this page.\\ If your
browser supports JavaScript, be sure it is enabled.}

\begin{center}\rule{3in}{0.4pt}\end{center}

{[}
{[}
{[}{]}
{[}

\subsubsection{18.2 Arcs en polaires}

\paragraph{18.2.1 Coordonnées polaires}

Soit E un plan euclidien rapporté à un repère orthonormé
(O,\vec\imath,\vecȷ). On notera
\vecu(\theta) le vecteur cos~
(\theta)\vec\imath + sin~
(\theta)\vecȷ. On vérifie immédiatement le résultat
suivant

Proposition~18.2.1 Pour tout \theta \in \mathbb{R}~,
(\vecu(\theta),\vecu'(\theta)) est une base
orthonormée de E de même sens que
(\vec\imath,\vecȷ). On a 
d^n \over d\theta^n
\vecu(\theta) =\vec u(\theta + n \pi~
\over 2 ).

On dispose ainsi d'une application P : \mathbb{R}~^2 \rightarrow~ \mathbb{R}~ définie par
P(\rho,\theta) = \rho\vecu(\theta). Le résultat suivant est tout à
fait élémentaire

Lemme~18.2.2 On a

\begin{align*} P(\rho,\theta)& =& P(\rho',\theta')
\Leftrightarrow \left (\rho = \rho' =
0\right ) \%& \\ & &
\text ou \left (\rho =
\rho'\text et \theta = \theta' + 2k\pi~\right ) \%&
\\ & & \text ou
\left (\rho = -\rho'\text et \theta = \theta' + (2k
+ 1)\pi~\right )\%& \\
\end{align*}

pour un k \in ℤ.

Définition~18.2.1 On dit que (\rho,\theta) est un système de coordonnées
polaires de M \in E si M = P(\rho,\theta).

\paragraph{18.2.2 Arcs en coordonnées polaires~: étude locale}

Définition~18.2.2 Soit I un intervalle de \mathbb{R}~ et f : I \rightarrow~ \mathbb{R}~ de classe
C^k. On appelle arc en coordonnées polaires défini par
l'équation \rho = f(\theta), \theta \in I l'arc paramétré (I,F) où F(\theta) = O +
f(\theta)\vecu(\theta) = O + f(\theta)cos~
(\theta)\vec\imath + f(\theta)sin~
(\theta)\vecȷ.

Remarque~18.2.1 Par abus d'écriture on s'autorisera parfois la notation
\rho^(n) à la place de f^(n)(\theta).

La formule de Leibnitz nous fournit immédiatement

F^(n)(\theta) = \\sum
\_\jmath=0^nC\_
n^\jmathf^(\jmath)(\theta)\vecu^(n-\jmath)(\theta)
= \sum \_\jmath=0^nC~\_
n^\jmathf^(\jmath)(\theta)\vecu(\theta + (n - \jmath) \pi~
\over 2 )

En particulier F'(\theta) = f'(\theta)\vecu(\theta) +
f(\theta)\vecu'(\theta) et F'`(\theta) = (f'`(\theta) -
f(\theta))\vecu(\theta) + 2f'(\theta)\vecu'(\theta).

Etude en un point dont l'image n'est pas l'origine

Si f(\theta)\neq~0, on a
F'(\theta)\neq~0 et donc le point est régulier~; le
point est birégulier si et seulement si le produit mixte
{[}F'(\theta),F'`(\theta){]} = \left
\textbar{}\matrix\,f'(\theta)&f'`(\theta) - f(\theta)
\cr f(\theta)&2f'(\theta) \right \textbar{} =
f(\theta)^2 + 2f'(\theta)^2 - f(\theta)f''(\theta) est différent de 0.

Remarquons que le vecteur F'(\theta) n'est pas colinéaire au vecteur
\overrightarrowOF(\theta) ce qui montre que la tangente ne
passe \jmathamais par l'origine. On en déduit que l'origine appartient soit
au demi plan de concavité, soit au demi plan opposé. Elle appartient au
demi plan de concavité si et seulement si~le vecteur
\overrightarrowF(\theta)O a une coordonnée positive
suivant F''(\theta) dans la base (F'(\theta),F''(\theta))~; mais si on écrit - F(\theta)
=\overrightarrow F(\theta)O = \lambda~F'(\theta) + \muF''(\theta), on a
immédiatement {[}F(\theta),F'(\theta){]} = \mu{[}F'(\theta),F''(\theta){]} soit encore
f(\theta)^2 = \mu(f(\theta)^2 + 2f'(\theta)^2 -
f(\theta)f''(\theta)). On en déduit que \mu est du même signe que f(\theta)^2
+ 2f'(\theta)^2 - f(\theta)f''(\theta).

Si l'on appelle \alpha~ l'angle du vecteur F'(\theta) avec
\vecu(\theta), on a
\mathrmtg~ \alpha~ = f(\theta)
\over f'(\theta) . On peut donc résumer en

Théorème~18.2.3 Soit \Gamma l'arc en polaire défini par l'équation \rho = f(\theta),
\theta \in I. Soit \theta un point de I dont l'image n'est pas l'origine
(c'est-à-dire que f(\theta)\neq~0). Alors (i) le point
est régulier, c'est donc soit un point banal, soit un point d'inflexion
(ii) le point est birégulier si et seulement si~\rho^2 +
2\rho'^2 - \rho\rho''\neq~0 (iii) la tangente
ne passe pas par l'origine du repère~; l'origine appartient au demi plan
de concavité si et seulement si~\rho^2 + 2\rho'^2 - \rho\rho''
\textgreater{} 0~; la tangente fait un angle \alpha~ avec le rayon vecteur, où
\alpha~ est donné par

\mathrmtg~ \alpha~ = \rho
\over \rho'

Remarque~18.2.2 On vérifie facilement que lorsque \rho^2 +
2\rho'^2 - \rho\rho'' s'annule en changeant de signe on a un point
d'inflexion. On pourra également remarquer que si l'on pose \phi = 1
\over \rho , l'expression \rho^2 +
2\rho'^2 - \rho\rho'' est du même signe que \phi(\phi + \phi'') ce qui permet
dans certains cas d'alléger les calculs.

Etude en un point dont l'image est l'origine

Si f(\theta) = 0, on voit immédiatement que le point est non totalement
singulier si et seulement si~il existe n tel que
f^(n)(\theta)\neq~0. Supposons donc que
f(\theta) = f'(\theta) =
\\ldots~ =
f^(p-1)(\theta) = 0 et que
f^(p)(\theta)\neq~0. La formule de Leibnitz
écrite ci dessus montre que F'(\theta) =
\\ldots~ =
F^(p-1)(\theta) = 0, que F^(p)(\theta) =
f^(p)(\theta)\vecu(\theta)\neq~0
et que F^(p+1)(\theta) =
f^(p+1)(\theta)\vecu(\theta) + (p +
1)f^(p)(\theta)\vecu'(\theta). On voit donc tout
d'abord que le vecteur \vecu(\theta) est un vecteur
directeur de la tangente qui fait donc un angle \theta avec l'axe Ox = O +
\mathbb{R}~\vec\imath et que d'autre part les vecteurs
F^(p)(\theta) et F^(p+1)(\theta) forment une famille libre.
L'entier q qui intervient dans la classification locale des points est
donc tou\jmathours égal à p + 1, ce qui montre que le point est un point
banal si p est impair et un point de rebroussement de première espèce si
p est pair. En résumé

Théorème~18.2.4 Soit \Gamma l'arc en polaire défini par l'équation \rho = f(\theta),
\theta \in I. Soit \theta un point de I dont l'image est l'origine (c'est-à-dire que
f(\theta) = 0). Soit p tel que f(\theta) = f'(\theta) =
\\ldots~ =
f^(p-1)(\theta) = 0 et
f^(p)(\theta)\neq~0~; alors (i) la tangente
au point \theta est la droite passant par l'origine et faisant l'angle \theta avec
l'axe Ox (ii) le point est un point banal si p est impair et un point de
rebroussement de première espèce si p est pair.

\paragraph{18.2.3 Branches infinies et phénomènes asymptotiques}

Soit \Gamma l'arc en polaire défini par l'équation \rho = f(\theta), \theta \in I. Soit \alpha~ \in
\mathbb{R}~ \cup\-\infty~,+\infty~\ une extrémité de I. On a
\\textbar{}F(\theta)\\textbar{} =
\textbar{}f(\theta)\textbar{}, on a donc une branche infinie si et seulement
si~lim\_\theta\rightarrow~\alpha~~\textbar{}f(\theta)\textbar{} =
+\infty~. Dans ce cas, on a  F(\theta) \over
\\textbar{}F(\theta)\\textbar{}
= sgn(f(\theta))\vecu~(\theta) qui
admet une limite en \alpha~ si et seulement si \alpha~ est fini.

Si \alpha~ = ±\infty~, il n'y a pas de direction asymptotique~; le point F(\theta)
s'éloigne indéfiniment en tournant autour de l'origine~; on dit que la
courbe présente une branche spirale.

Si \alpha~ \in \mathbb{R}~, la droite \mathbb{R}~\vecu(\alpha~) est direction
asymptotique. Le point d'intersection de la droite passant par F(\theta) et
parallèle à la droite \mathbb{R}~\vecu(\alpha~) avec la droite affine
O + \mathbb{R}~\vecu'(\alpha~) a pour ordonnée dans le repère
(\vecu(\alpha~),\vecu'(\alpha~)) le nombre
f(\theta)sin~ (\theta - \alpha~). On en déduit que l'arc admet
une asymptote si et seulement si~f(\theta)sin~ (\theta -
\alpha~) admet une limite \ell quand \theta tend vers \alpha~ dans I~; dans ce cas
l'asymptote est la droite d'équation Y = \ell dans le repère
(O,\vecu(\alpha~),\vecu'(\alpha~)).

Remarque~18.2.3 Dans le cas où \alpha~ \in \pi~ \over 2 ℤ, il y
a de gros risques de confusions entre le repère mobile
(O,\vecu(\alpha~),\vecu'(\alpha~)) et le
repère fixe (0,\vec\imath,\vecȷ). Il
est de beaucoup préférable (i) si \alpha~ \in \pi~ℤ de regarder si y(\theta) =
\rhosin~ \theta admet une limite \ell~; dans ce cas la
droite d'équation y = \ell dans le repère fixe est asymptote (ii) si \alpha~ \in
\pi~ \over 2 + \pi~ℤ de regarder si x(\theta) =
\rhocos~ \theta admet une limite \ell~; dans ce cas la
droite d'équation x = \ell dans le repère fixe est asymptote.

Autres phénomènes asymptotiques On peut également remarquer que si \alpha~ =
±\infty~ est une borne de \alpha~ et si \rho = f(\theta) a une limite r, quand \theta tend vers \alpha~

\begin{itemize}
\itemsep1pt\parskip0pt\parsep0pt
\item
  (i) si r = 0, le point F(\theta) s'enroule autour de l'origine~; on dit que
  l'origine est un point asymptote de l'arc
\item
  (ii) si r\neq~0, le point F(\theta) s'enroule autour
  du cercle de centre O de rayon \textbar{}r\textbar{}~; on dit que ce
  cercle est un cercle asymptote.
\end{itemize}

\paragraph{18.2.4 Plan d'étude d'un arc plan en polaires}

Soit f une fonction de \mathbb{R}~ vers \mathbb{R}~. On considère l'arc défini par
l'équation \rho = f(\theta), \theta \in \mathbb{R}~.

Première étape~: domaine de définition. On détermine le domaine de
définition de f~; c'est en général une réunion finie d'intervalles deux
à deux dis\jmathoints, si bien que la courbe étudiée sera une réunion finie
d'arcs paramétrés.

Deuxième étape~: réduction du domaine d'étude. On recherche les
applications \sigma : D\rightarrow~D tel que pour \theta \inD, F(\sigma(\theta)) =
f(\sigma(\theta))\vecu(\sigma(\theta)) se déduise par une transformation
géométrique simple S de F(\theta) = f(\theta)\vecu(\theta). Soit \Delta
un domaine fondamental pour \sigma.

On recherchera principalement des transformations \theta du type

\begin{itemize}
\itemsep1pt\parskip0pt\parsep0pt
\item
  (i) \sigma(\theta) = \theta + T avec alors \Delta = {[}a,a + T{]} \bigcapD
\item
  (ii) \sigma(\theta) = \omega - \theta avec alors \Delta = {[} \omega \over 2
  ,+\infty~{[}\bigcapD
\end{itemize}

On aura alors

\begin{itemize}
\itemsep1pt\parskip0pt\parsep0pt
\item
  (i) Si f(\theta + T) = f(\theta), alors le point F(\theta + T) se déduit du point
  F(\theta) par la rotation de centre O et d'angle T~; trois cas sont alors à
  examiner

  \begin{itemize}
  \itemsep1pt\parskip0pt\parsep0pt
  \item
    a) si T \in 2\pi~ℤ, on étudie sur \Delta = {[}a,a + T{]} \bigcapD et c'est terminé
  \item
    b) si T \in \pi~ℚ, T = 2\pi~ p \over q , on étudie sur \Delta
    = {[}a,a + T{]} \bigcapD et on complète par q - 1 rotations d'angle T,
    2T,\\ldots~,(q -
    1)T
  \item
    c) si T∉\pi~ℚ, on étudie sur \Delta = {[}a,a +
    T{]} \bigcapD et on complète par une infinité de rotations d'angle kT, k \in
    ℤ
  \end{itemize}
\item
  (ii) Si f(\theta + T \over 2 ) = -f(\theta), alors le point
  F(\theta + T \over 2 ) se déduit du point F(\theta) par la
  rotation de centre O et d'angle \pi~ + T \over 2 ~; on
  étudie sur \Delta = {[}a,a + T \over 2 {]} \bigcapD, la
  discussion est ensuite similaire (on a bien entendu dans ce cas
  également f(\theta + T) = f(\theta))~; dans le cas particulier ou f(\theta + \pi~) =
  -f(\theta), on obtient toute la courbe en faisant varier \theta dans {[}a,a +
  \pi~{]}.
\item
  (iii) Si f(\omega - \theta) = f(\theta), alors le point F(\omega - \theta) se déduit du point
  F(\theta) par la symétrie orthogonale par rapport à la droite
  D\_\omega\diagup2 qui fait l'angle \omega\diagup2 avec l'axe Ox~; on étudie sur \Delta =
  {[} \omega \over 2 ,+\infty~{[}\bigcapD, et on complète par la
  symétrie par rapport à la droite D\_\omega\diagup2 (remarquer que \omega = 0
  correspond à une symétrie par rapport à Ox et \omega = \pi~ à une symétrie par
  rapport à Oy).
\item
  (iv) Si f(\omega - \theta) = -f(\theta), alors le point F(\omega - \theta) se déduit du point
  F(\theta) par la symétrie orthogonale par rapport à la droite
  D\_(\omega+\pi~)\diagup2 qui fait l'angle (\omega + \pi~)\diagup2 avec l'axe Ox~; on
  étudie sur \Delta = {[} \omega \over 2 ,+\infty~{[}\bigcapD, et on
  complète par la symétrie par rapport à la droite D\_(\omega+\pi~)\diagup2
  (remarquer que \omega = 0 correspond à une symétrie par rapport à Oy et \omega =
  \pi~ à une symétrie par rapport à Ox).
\end{itemize}

Troisième étape~: signe de \rho. On étudie le signe de \rho = f(\theta)~; au
passage on repère les points dont l'image est l'origine~; on peut
immédiatement placer les tangentes en ces points~: elles font l'angle
correspondant avec l'axe Ox

Quatrième étape~: étude des branches infinies. L'arc admet en \alpha~
\in\overlineD une branche infinie si et seulement si
lim\_\theta\rightarrow~\alpha~~\textbar{}f(\theta)\textbar{} = +\infty~.
On reproduit la discussion dé\jmathà faite. On peut également détecter
d'autres phénomènes asymptotiques comme point ou cercle asymptotes.

Une étude complémentaire de signe peut parfois préciser la position du
point F(\theta) par rapport à une asymptote, ce qui peut permettre de
préciser un tracé.

Cinquième étape~: ébauche de tracé. En s'aidant d'une calculatrice ou
d'un ordinateur, on peut calculer un certain nombre de points
supplémentaires en plus des points remarquables~; ceci, en plus des
points et des tangentes remarquables, du signe de \rho et de l'étude des
branches infinies permet en général une ébauche convaincante du tracé.

Etapes facultatives

Si la question est posée ou si l'ébauche du tracé suggère la nécessité
de certaines précisions, on peut procéder à quelques étapes
supplémentaires

Sixième étape~: variations de \rho = f(\theta). On étudie le signe de la dérivée
\rho' = f'(\theta). Au passage on repère des points remarquables où
f'(\theta\_0) = 0,f(\theta\_0)\neq~0~:
points où la tangente est orthogonale au rayon vecteur F(\theta\_0)

Septième étape~: détermination des points non biréguliers et étude de la
concavité. On étudie le signe de \rho^2 + 2\rho'^2 -
\rho\rho'' ou encore de \phi(\phi + \phi'') avec \phi = 1 \over \rho ~;
l'annulation correspond aux points non biréguliers, la positivité aux
points où l'origine appartient au demi plan de concavité

Huitième étape~: détermination des points multiples. Il s'agit de
résoudre l'équation F(\theta) = F(\theta') c'est-à-dire encore les systèmes f(\theta) =
f(\theta + 2k\pi~), k \in ℤ^∗, et f(\theta) = -f(\theta + (2k + 1)\pi~), k \in ℤ.

\paragraph{18.2.5 Equations polaires remarquables}

Equation polaire d'une droite ne passant pas par l'origine

Une telle droite a dans un repère orthonormé d'origine O une équation du
type xcos \theta\_0~ +
ysin \theta\_0~ - h = 0 avec
h\neq~0 (où \vecn
= cos \theta\_0\vec\imath~
+ sin \theta\_0\vecȷ~ est
un vecteur normal à la droite et où h désigne la distance de O à la
droite, orientée par le choix de \vecn). En reportant
x = \rhocos~ \theta et y =
\rhosin~ \theta, on obtient l'équation

\rho = h \over cos~ (\theta -
\theta\_0)

Inversement il est clair qu'une telle équation définit une droite ne
passant pas par O.

Remarque~18.2.4 Il suffit de faire varier \theta dans un intervalle de
longueur \pi~ pour avoir toute la droite.

Equation polaire d'un cercle passant par l'origine

Un tel cercle a dans un repère orthonormé d'origine O une équation du
type x^2 + y^2 - 2\alpha~x - 2\beta~y = 0 où le point \Omega de
coordonnées (\alpha~,\beta~) est le centre du cercle. Posons \alpha~ =
Rcos \theta\_0~ et \beta~ =
Rsin \theta\_0~ (où R est bien évidemment le
rayon du cercle puisque celui ci passe par O). En reportant x =
\rhocos \theta et y = \rho\sin~
\theta, on obtient l'équation

\rho = 2Rcos (\theta - \theta\_0~)

Inversement il est clair qu'une telle équation définit un cercle passant
par O.

Remarque~18.2.5 Il suffit de faire varier \theta dans un intervalle de
longueur \pi~ pour avoir tout le cercle.

Equation polaire d'une conique ayant l'origine pour foyer

Soit e son excentricité et D : xcos~
\theta\_0 + ysin \theta\_0~ - h = 0 la
directrice correspondant au foyer O. La conique est alors
\m \in E∣d(m,F) =
ed(m,D)\. Elle est donc définie par l'équation
x^2 + y^2 = e(xcos~
\theta\_0 + ysin \theta\_0~ -
h)^2. En portant x = \rhocos~ \theta et y =
\rhosin \theta, on obtient l'équation \rho^2~ =
e^2(\rhocos (\theta - \theta\_0~) - h)
soit encore \rho = ±e(\rhocos (\theta - \theta\_0~) - h
ou encore \rho = ±eh \over 1±e\
cos (\theta-\theta\_0) .Remarquons alors que les deux courbes
d'équations polaires \rho = eh \over
1+e cos (\theta-\theta\_0)~ et \rho = - eh
\over 1-e cos~
(\theta-\theta\_0) se déduisent l'une de l'autre par le changement de
(\rho,\theta) en (-\rho,\theta + \pi~) ce qui redonne le même point géométrique. Elles ont
donc la même image. Donc la conique est entièrement définie par l'une
des deux équations soit par exemple

\rho = p \over 1 + ecos~ (\theta -
\theta\_0)

où \theta décrit un intervalle de longueur 2\pi~ et p = eh. En remontant les
calculs on vérifie immédiatement qu'inversement une telle équation
polaire définit une conique de foyer O.

{[}
{[}
{[}
{[}

\end{document}

% \documentclass[]{article}
\usepackage[T1]{fontenc}
\usepackage{lmodern}
\usepackage{amssymb,amsmath}
\usepackage{ifxetex,ifluatex}
\usepackage{fixltx2e} % provides \textsubscript
% use upquote if available, for straight quotes in verbatim environments
\IfFileExists{upquote.sty}{\usepackage{upquote}}{}
\ifnum 0\ifxetex 1\fi\ifluatex 1\fi=0 % if pdftex
  \usepackage[utf8]{inputenc}
\else % if luatex or xelatex
  \ifxetex
    \usepackage{mathspec}
    \usepackage{xltxtra,xunicode}
  \else
    \usepackage{fontspec}
  \fi
  \defaultfontfeatures{Mapping=tex-text,Scale=MatchLowercase}
  \newcommand{\euro}{€}
\fi
% use microtype if available
\IfFileExists{microtype.sty}{\usepackage{microtype}}{}
\ifxetex
  \usepackage[setpagesize=false, % page size defined by xetex
              unicode=false, % unicode breaks when used with xetex
              xetex]{hyperref}
\else
  \usepackage[unicode=true]{hyperref}
\fi
\hypersetup{breaklinks=true,
            bookmarks=true,
            pdfauthor={},
            pdftitle={Probl`emes classiques sur les courbes},
            colorlinks=true,
            citecolor=blue,
            urlcolor=blue,
            linkcolor=magenta,
            pdfborder={0 0 0}}
\urlstyle{same}  % don't use monospace font for urls
\setlength{\parindent}{0pt}
\setlength{\parskip}{6pt plus 2pt minus 1pt}
\setlength{\emergencystretch}{3em}  % prevent overfull lines
\setcounter{secnumdepth}{0}
 
/* start css.sty */
.cmr-5{font-size:50%;}
.cmr-7{font-size:70%;}
.cmmi-5{font-size:50%;font-style: italic;}
.cmmi-7{font-size:70%;font-style: italic;}
.cmmi-10{font-style: italic;}
.cmsy-5{font-size:50%;}
.cmsy-7{font-size:70%;}
.cmex-7{font-size:70%;}
.cmex-7x-x-71{font-size:49%;}
.msbm-7{font-size:70%;}
.cmtt-10{font-family: monospace;}
.cmti-10{ font-style: italic;}
.cmbx-10{ font-weight: bold;}
.cmr-17x-x-120{font-size:204%;}
.cmsl-10{font-style: oblique;}
.cmti-7x-x-71{font-size:49%; font-style: italic;}
.cmbxti-10{ font-weight: bold; font-style: italic;}
p.noindent { text-indent: 0em }
td p.noindent { text-indent: 0em; margin-top:0em; }
p.nopar { text-indent: 0em; }
p.indent{ text-indent: 1.5em }
@media print {div.crosslinks {visibility:hidden;}}
a img { border-top: 0; border-left: 0; border-right: 0; }
center { margin-top:1em; margin-bottom:1em; }
td center { margin-top:0em; margin-bottom:0em; }
.Canvas { position:relative; }
li p.indent { text-indent: 0em }
.enumerate1 {list-style-type:decimal;}
.enumerate2 {list-style-type:lower-alpha;}
.enumerate3 {list-style-type:lower-roman;}
.enumerate4 {list-style-type:upper-alpha;}
div.newtheorem { margin-bottom: 2em; margin-top: 2em;}
.obeylines-h,.obeylines-v {white-space: nowrap; }
div.obeylines-v p { margin-top:0; margin-bottom:0; }
.overline{ text-decoration:overline; }
.overline img{ border-top: 1px solid black; }
td.displaylines {text-align:center; white-space:nowrap;}
.centerline {text-align:center;}
.rightline {text-align:right;}
div.verbatim {font-family: monospace; white-space: nowrap; text-align:left; clear:both; }
.fbox {padding-left:3.0pt; padding-right:3.0pt; text-indent:0pt; border:solid black 0.4pt; }
div.fbox {display:table}
div.center div.fbox {text-align:center; clear:both; padding-left:3.0pt; padding-right:3.0pt; text-indent:0pt; border:solid black 0.4pt; }
div.minipage{width:100%;}
div.center, div.center div.center {text-align: center; margin-left:1em; margin-right:1em;}
div.center div {text-align: left;}
div.flushright, div.flushright div.flushright {text-align: right;}
div.flushright div {text-align: left;}
div.flushleft {text-align: left;}
.underline{ text-decoration:underline; }
.underline img{ border-bottom: 1px solid black; margin-bottom:1pt; }
.framebox-c, .framebox-l, .framebox-r { padding-left:3.0pt; padding-right:3.0pt; text-indent:0pt; border:solid black 0.4pt; }
.framebox-c {text-align:center;}
.framebox-l {text-align:left;}
.framebox-r {text-align:right;}
span.thank-mark{ vertical-align: super }
span.footnote-mark sup.textsuperscript, span.footnote-mark a sup.textsuperscript{ font-size:80%; }
div.tabular, div.center div.tabular {text-align: center; margin-top:0.5em; margin-bottom:0.5em; }
table.tabular td p{margin-top:0em;}
table.tabular {margin-left: auto; margin-right: auto;}
div.td00{ margin-left:0pt; margin-right:0pt; }
div.td01{ margin-left:0pt; margin-right:5pt; }
div.td10{ margin-left:5pt; margin-right:0pt; }
div.td11{ margin-left:5pt; margin-right:5pt; }
table[rules] {border-left:solid black 0.4pt; border-right:solid black 0.4pt; }
td.td00{ padding-left:0pt; padding-right:0pt; }
td.td01{ padding-left:0pt; padding-right:5pt; }
td.td10{ padding-left:5pt; padding-right:0pt; }
td.td11{ padding-left:5pt; padding-right:5pt; }
table[rules] {border-left:solid black 0.4pt; border-right:solid black 0.4pt; }
.hline hr, .cline hr{ height : 1px; margin:0px; }
.tabbing-right {text-align:right;}
span.TEX {letter-spacing: -0.125em; }
span.TEX span.E{ position:relative;top:0.5ex;left:-0.0417em;}
a span.TEX span.E {text-decoration: none; }
span.LATEX span.A{ position:relative; top:-0.5ex; left:-0.4em; font-size:85%;}
span.LATEX span.TEX{ position:relative; left: -0.4em; }
div.float img, div.float .caption {text-align:center;}
div.figure img, div.figure .caption {text-align:center;}
.marginpar {width:20%; float:right; text-align:left; margin-left:auto; margin-top:0.5em; font-size:85%; text-decoration:underline;}
.marginpar p{margin-top:0.4em; margin-bottom:0.4em;}
.equation td{text-align:center; vertical-align:middle; }
td.eq-no{ width:5%; }
table.equation { width:100%; } 
div.math-display, div.par-math-display{text-align:center;}
math .texttt { font-family: monospace; }
math .textit { font-style: italic; }
math .textsl { font-style: oblique; }
math .textsf { font-family: sans-serif; }
math .textbf { font-weight: bold; }
.partToc a, .partToc, .likepartToc a, .likepartToc {line-height: 200%; font-weight:bold; font-size:110%;}
.chapterToc a, .chapterToc, .likechapterToc a, .likechapterToc, .appendixToc a, .appendixToc {line-height: 200%; font-weight:bold;}
.index-item, .index-subitem, .index-subsubitem {display:block}
.caption td.id{font-weight: bold; white-space: nowrap; }
table.caption {text-align:center;}
h1.partHead{text-align: center}
p.bibitem { text-indent: -2em; margin-left: 2em; margin-top:0.6em; margin-bottom:0.6em; }
p.bibitem-p { text-indent: 0em; margin-left: 2em; margin-top:0.6em; margin-bottom:0.6em; }
.paragraphHead, .likeparagraphHead { margin-top:2em; font-weight: bold;}
.subparagraphHead, .likesubparagraphHead { font-weight: bold;}
.quote {margin-bottom:0.25em; margin-top:0.25em; margin-left:1em; margin-right:1em; text-align:justify;}
.verse{white-space:nowrap; margin-left:2em}
div.maketitle {text-align:center;}
h2.titleHead{text-align:center;}
div.maketitle{ margin-bottom: 2em; }
div.author, div.date {text-align:center;}
div.thanks{text-align:left; margin-left:10%; font-size:85%; font-style:italic; }
div.author{white-space: nowrap;}
.quotation {margin-bottom:0.25em; margin-top:0.25em; margin-left:1em; }
h1.partHead{text-align: center}
.sectionToc, .likesectionToc {margin-left:2em;}
.subsectionToc, .likesubsectionToc {margin-left:4em;}
.subsubsectionToc, .likesubsubsectionToc {margin-left:6em;}
.frenchb-nbsp{font-size:75%;}
.frenchb-thinspace{font-size:75%;}
.figure img.graphics {margin-left:10%;}
/* end css.sty */

\title{Probl`emes classiques sur les courbes}
\author{}
\date{}

\begin{document}
\maketitle

\textbf{Warning: 
requires JavaScript to process the mathematics on this page.\\ If your
browser supports JavaScript, be sure it is enabled.}

\begin{center}\rule{3in}{0.4pt}\end{center}

[
[
[]
[

\subsubsection{18.3 Problèmes classiques sur les courbes}

\paragraph{18.3.1 Trajectoires orthogonales}

Ce paragraphe ne fait pas partie du programme des classes préparatoires.

Soit E un espace euclidien.

Soit (\Gamma_\lambda~)_\lambda~\inJ une famille d'arcs paramétrés indexée
par un intervalle J de \mathbb{R}~ où \Gamma_\lambda~ = (I,F_\lambda~). On posera
encore F_\lambda~(t) = F(t,\lambda~) et on supposera que F : I \times J \rightarrow~ E est de
classe \mathcal{C}^1.

Donnons nous un arc paramétré (K,G) qui rencontre tous les \Gamma_\lambda~.
Le point u \in K de G est donc le point t(u) de l'arc \Gamma_\lambda~(u) si
bien que G(u) = F_\lambda~(u)(t(u)) = F(t(u),\lambda~(u)). On obtient ainsi
un arc paramétré u\mapsto~F(t(u),\lambda~(u)). On dira que
c'est une trajectoire orthogonale de la famille (\Gamma_\lambda~) si cet
arc est orthogonal à l'arc \Gamma_\lambda~(u) au point t(u). On obtient
donc la définition suivante~:

Définition~18.3.1 On appelle trajectoire orthogonale des arcs
\Gamma_\lambda~ tout arc paramétré (K,G) de la forme G(u) = F(t(u),\lambda~(u)),
où u\mapsto~t(u) est une application de classe
\mathcal{C}^1 de K dans I et u\mapsto~\lambda~(u) une
application de classe \mathcal{C}^1 de K dans J telles que
\forall~u \in K, F_\lambda~(u)~'(t(u)) \bot G'(u)

La recherche s'effectue en remarquant que F_\lambda~'(t) = \partial~F
\over \partial~t (t,\lambda~), soit F_\lambda~(u)'(t(u)) = \partial~F
\over \partial~t (t(u),\lambda~(u)) et que G'(u) = dt
\over du (u) \partial~F \over \partial~t (t(u),\lambda~(u))
+ d\lambda~ \over du (u) \partial~F \over \partial~\lambda~
(t(u),\lambda~(u)). La condition F_\lambda~(u)'(t(u),\lambda~(u)) \bot G'(u) s'écrit
donc

\left ( \partial~F \over \partial~t
(t(u),\lambda~(u))∣ dt \over du
(u) \partial~F \over \partial~t (t(u),\lambda~(u)) + d\lambda~
\over du (u) \partial~F \over \partial~\lambda~
(t(u),\lambda~(u))\right ) = 0

soit encore, en termes de formes différentielles

\left ( \partial~F \over \partial~t
(t,\lambda~)∣ \partial~F \over \partial~t (t,\lambda~)
dt + \partial~F \over \partial~\lambda~ (t,\lambda~) d\lambda~\right ) = 0

qui conduit à une équation différentielle reliant t et \lambda~.

Exemples~: prenons la famille de paraboles y = x^2 + \lambda~, \lambda~ \in
\mathbb{R}~. On peut les paramétrer par F(t,\lambda~) = (t,t^2 + \lambda~). On a
alors  \partial~F \over \partial~t (t,\lambda~) = (1,2t) et  \partial~F
\over \partial~\lambda~ (t,\lambda~) = (0,1). L'équation ci dessus s'écrit
encore \ \partial~F \over \partial~t
(t,\lambda~)\^2 dt + \left
( \partial~F \over \partial~t (t,\lambda~)∣ \partial~F
\over \partial~\lambda~ (t,\lambda~)\right ) d\lambda~ = 0, soit ici
(1 + 4t^2) dt + 2t d\lambda~ = 0. Il s'agit d'une équation à
variable séparable. Elle s'écrit encore (2t + 1 \over
2t )dt = -d\lambda~. On trouve donc \lambda~ = -t^2 - 1
\over 2  log~
t + \lambda_0, soit (t,t^2 + \lambda~) =
(t,- 1 \over 2  log~
t + \lambda_0) et donc les trajectoires
orthogonales sont équivalentes aux arcs
t\mapsto~(t,- 1 \over 2
 log t + \lambda_0~). En fait la division
par t nous a fait perdre une solution évidente t = 0 correspondant à
l'axe Oy.

\paragraph{18.3.2 Inverse d'une courbe}

Ce paragraphe ne fait pas partie du programme des classes préparatoires.

Rappelons que si E est un espace affine euclidien, on appelle inversion
de pôle O \in E l'application de E \diagdown\O\
dans lui même qui à M \in E \diagdown\O\ associe
l'unique point M' défini par

\begin{itemize}
\itemsep1pt\parskip0pt\parsep0pt
\item
  (i) O,M et M' sont alignés
\item
  (ii) \overlineOM.\overlineOM' =
  1
\end{itemize}

On vérifie immédiatement que M' = O +
\overrightarrowOM \over
\\overrightarrowOM\^2
.

Etant donné un arc (I,F) de E dont l'image est contenue dans E
\diagdown\O\, on peut alors définir son
inverse de pôle O~; c'est l'arc (I,G) tel que, pour tout t \in I, G(t)
soit l'inverse de F(t).

Supposons que E soit un plan euclidien rapporté à un repère orthonormé
(O,\vec\imath,\vecȷ) et que
\overrightarrowOF(t) = x(t)\vec\imath +
y(t)\vecȷ. On a alors
\overrightarrowOG(t) = X(t)\vec\imath +
Y (t)\vecȷ avec

X(t) = x(t) \over x(t)^2 +
y(t)^2 ,\quad Y (t) = y(t)
\over x(t)^2 + y(t)^2

Si \Gamma = (I,F) est donné en polaires par l'équation \rho = f(\theta), on a
\overrightarrowOF(\theta) =
f(\theta)\vecu(\theta) et alors
\overrightarrowOG(\theta) = 1 \over
f(\theta) \vecu(\theta), si bien que l'inverse est donnée par
l'équation polaire \rho = 1 \over f(\theta) .

Exemple~18.3.1 Inverse des coniques de foyer O. On a vu qu'une telle
conique admettait une équation polaire \rho = p \over
1+e cos (\theta-\theta_0)~ . L'inverse d'une
telle conique est donc une courbe d'équation polaire \rho =
1+e cos (\theta-\theta_0~) \over
p , soit encore (à une rotation près autour de l'origine) \rho = a(1 +
ecos~ \theta). On obtient la famille des
lima\ccons de Pascal (avec le cas particulier de la
cardioïde, pour e = 1, qui est l'inverse d'une parabole par rapport à
son foyer).

\paragraph{18.3.3 Podaire d'une courbe}

Ce paragraphe ne fait pas partie du programme des classes préparatoires.

Soit E un espace affine euclidien, (I,F) un arc paramétré régulier de E
et A un point de E.

Définition~18.3.2 On appelle podaire de l'arc (I,F) par rapport au point
A l'arc (I,G) où pour chaque t \in I, G(t) est la projection orthogonale
de A sur la tangente au point t à l'arc (I,F).

Comme cette tangente est définie comme F(t) + \mathbb{R}~F'(t), il suffit donc
d'exprimer que la famille
(F'(t),\overrightarrowF(t)G(t)) est liée et que
\overrightarrowAG(t) \bot F'(t).

Supposons que E soit un plan euclidien rapporté à un repère orthonormé
(O,\vec\imath,\vecȷ) et que
\overrightarrowOF(t) = x(t)\vec\imath +
y(t)\vecȷ. Posons
\overrightarrowOA = a\vec\imath +
b\vecȷ et \overrightarrowOG(t) =
X(t)\vec\imath + Y (t)\vecȷ. On doit
donc écrire

\left
\matrix\,X(t) - x(t)&x'(t)
\cr Y (t) - y(t)&y'(t)\right 
= 0\text et (X(t) - a)x'(t) + (Y (t) - b)y'(t) = 0

ce qui conduit à un système de Cramer aux inconnues X(t) et Y (t)

\left
\\matrix\,y'(t)X(t) -
x'(t)Y (t) = y'(t)x(t) - x'(t)y(t) \cr X(t)x'(t) + Y
(t)y'(t) = ax'(t) + by'(t)\right .

Exemple~18.3.2 Recherchons la podaire d'un cercle par rapport à un point
du plan. On choisissant convenablement le repère, on peut supposer que
le cercle est paramétré par t\mapsto~(a +
Rcos t,R\sin~ t) avec
a ≥ 0 et R > 0 et que le point A a pour coordonnées (0,0).
Le système ci dessus devient alors (après simplification par R)

\left
\\matrix\,cos~
tX(t) + sin~ tY (t) =
acos~ t + R \cr
-sin tX(t) +\ cos~ tY
(t) = 0\right .

d'où l'on déduit X(t) = (acos~ t +
R)cos~ t et Y (t) =
(acos t + R)\sin~ t.
On en déduit que la podaire est la courbe d'équation polaire \rho =
acos~ \theta + R. Il s'agit d'un
lima\ccon de Pascal (évidemment dégénéré en un cercle
si a = 0 c'est-à-dire si le cercle de départ est centré en A). On trouve
une cardioïde lorsque a = R, c'est-à-dire lorsque le cercle passe par A.

\paragraph{18.3.4 Conchoïdes d'une courbe}

Ce paragraphe ne fait pas partie du programme des classes préparatoires.

Soit \Gamma = (I,F) un arc paramétré d'un plan euclidien E, O un point de E
n'appartenant pas à l'image de \Gamma et a > 0. On associe à \Gamma
les arcs \Gamma_1 = (I,F_1) et \Gamma_2 =
(I,F_2), appelés conchoïdes de centre O pour la longueur a, où
F_i(t) est défini pour i
\in\1,2\ par

\begin{itemize}
\itemsep1pt\parskip0pt\parsep0pt
\item
  (i) O,F(t) et F_i(t) sont alignés
\item
  (ii) la distance de F(t) à F_i(t) est égale à a.
\end{itemize}

Supposons que E soit un plan euclidien rapporté à un repère orthonormé
(O,\vec\imath,\vecȷ) et que
\overrightarrowOF(t) = x(t)\vec\imath +
y(t)\vecȷ. On a alors
\overrightarrowOF_i(t) =
X_i(t)\vec\imath + Y
_i(t)\vecȷ avec

X(t) = x(t) ± a x(t) \over
\sqrtx(t)^2  + y(t)^2
,\quad Y (t) = y(t) ± a y(t) \over
\sqrtx(t)^2  + y(t)^2

Si \Gamma = (I,F) est donné en polaires par l'équation \rho = f(\theta), on a
\overrightarrowOF(\theta) =
f(\theta)\vecu(\theta) et alors
\overrightarrowOF_i(\theta) = (f(\theta) ±
a)\vecu(\theta), si bien que les conchoïdes sont donnés
par l'équation polaire \rho = f(\theta) ± a.

Exemple~18.3.3 Conchoïdes d'un cercle par rapport à l'un de ses points.
En choisissant convenablement le repère d'origine 0, le cercle a pour
équation polaire \rho = 2Rcos~ \theta si bien que les
deux conchoïdes ont pour équation polaire \rho =
2Rcos~ \theta ± a. Remarquons que les deux courbes
d'équations polaires \rho = 2Rcos~ \theta + a et \rho =
2Rcos~ \theta - a se déduisent l'une de l'autre par
le changement de (\rho,\theta) en (-\rho,\theta + \pi~) ce qui redonne le même point
géométrique. Elles ont donc la même image. Ce sont des
lima\ccons de Pascal, le cas de la cardioïde
correspondant à a = 2R (la longueur a est égale au diamètre du cercle).

[
[
[
[

\end{document}

% \documentclass[]{article}
\usepackage[T1]{fontenc}
\usepackage{lmodern}
\usepackage{amssymb,amsmath}
\usepackage{ifxetex,ifluatex}
\usepackage{fixltx2e} % provides \textsubscript
% use upquote if available, for straight quotes in verbatim environments
\IfFileExists{upquote.sty}{\usepackage{upquote}}{}
\ifnum 0\ifxetex 1\fi\ifluatex 1\fi=0 % if pdftex
  \usepackage[utf8]{inputenc}
\else % if luatex or xelatex
  \ifxetex
    \usepackage{mathspec}
    \usepackage{xltxtra,xunicode}
  \else
    \usepackage{fontspec}
  \fi
  \defaultfontfeatures{Mapping=tex-text,Scale=MatchLowercase}
  \newcommand{\euro}{€}
\fi
% use microtype if available
\IfFileExists{microtype.sty}{\usepackage{microtype}}{}
\ifxetex
  \usepackage[setpagesize=false, % page size defined by xetex
              unicode=false, % unicode breaks when used with xetex
              xetex]{hyperref}
\else
  \usepackage[unicode=true]{hyperref}
\fi
\hypersetup{breaklinks=true,
            bookmarks=true,
            pdfauthor={},
            pdftitle={Etude metrique des arcs},
            colorlinks=true,
            citecolor=blue,
            urlcolor=blue,
            linkcolor=magenta,
            pdfborder={0 0 0}}
\urlstyle{same}  % don't use monospace font for urls
\setlength{\parindent}{0pt}
\setlength{\parskip}{6pt plus 2pt minus 1pt}
\setlength{\emergencystretch}{3em}  % prevent overfull lines
\setcounter{secnumdepth}{0}
 
/* start css.sty */
.cmr-5{font-size:50%;}
.cmr-7{font-size:70%;}
.cmmi-5{font-size:50%;font-style: italic;}
.cmmi-7{font-size:70%;font-style: italic;}
.cmmi-10{font-style: italic;}
.cmsy-5{font-size:50%;}
.cmsy-7{font-size:70%;}
.cmex-7{font-size:70%;}
.cmex-7x-x-71{font-size:49%;}
.msbm-7{font-size:70%;}
.cmtt-10{font-family: monospace;}
.cmti-10{ font-style: italic;}
.cmbx-10{ font-weight: bold;}
.cmr-17x-x-120{font-size:204%;}
.cmsl-10{font-style: oblique;}
.cmti-7x-x-71{font-size:49%; font-style: italic;}
.cmbxti-10{ font-weight: bold; font-style: italic;}
p.noindent { text-indent: 0em }
td p.noindent { text-indent: 0em; margin-top:0em; }
p.nopar { text-indent: 0em; }
p.indent{ text-indent: 1.5em }
@media print {div.crosslinks {visibility:hidden;}}
a img { border-top: 0; border-left: 0; border-right: 0; }
center { margin-top:1em; margin-bottom:1em; }
td center { margin-top:0em; margin-bottom:0em; }
.Canvas { position:relative; }
li p.indent { text-indent: 0em }
.enumerate1 {list-style-type:decimal;}
.enumerate2 {list-style-type:lower-alpha;}
.enumerate3 {list-style-type:lower-roman;}
.enumerate4 {list-style-type:upper-alpha;}
div.newtheorem { margin-bottom: 2em; margin-top: 2em;}
.obeylines-h,.obeylines-v {white-space: nowrap; }
div.obeylines-v p { margin-top:0; margin-bottom:0; }
.overline{ text-decoration:overline; }
.overline img{ border-top: 1px solid black; }
td.displaylines {text-align:center; white-space:nowrap;}
.centerline {text-align:center;}
.rightline {text-align:right;}
div.verbatim {font-family: monospace; white-space: nowrap; text-align:left; clear:both; }
.fbox {padding-left:3.0pt; padding-right:3.0pt; text-indent:0pt; border:solid black 0.4pt; }
div.fbox {display:table}
div.center div.fbox {text-align:center; clear:both; padding-left:3.0pt; padding-right:3.0pt; text-indent:0pt; border:solid black 0.4pt; }
div.minipage{width:100%;}
div.center, div.center div.center {text-align: center; margin-left:1em; margin-right:1em;}
div.center div {text-align: left;}
div.flushright, div.flushright div.flushright {text-align: right;}
div.flushright div {text-align: left;}
div.flushleft {text-align: left;}
.underline{ text-decoration:underline; }
.underline img{ border-bottom: 1px solid black; margin-bottom:1pt; }
.framebox-c, .framebox-l, .framebox-r { padding-left:3.0pt; padding-right:3.0pt; text-indent:0pt; border:solid black 0.4pt; }
.framebox-c {text-align:center;}
.framebox-l {text-align:left;}
.framebox-r {text-align:right;}
span.thank-mark{ vertical-align: super }
span.footnote-mark sup.textsuperscript, span.footnote-mark a sup.textsuperscript{ font-size:80%; }
div.tabular, div.center div.tabular {text-align: center; margin-top:0.5em; margin-bottom:0.5em; }
table.tabular td p{margin-top:0em;}
table.tabular {margin-left: auto; margin-right: auto;}
div.td00{ margin-left:0pt; margin-right:0pt; }
div.td01{ margin-left:0pt; margin-right:5pt; }
div.td10{ margin-left:5pt; margin-right:0pt; }
div.td11{ margin-left:5pt; margin-right:5pt; }
table[rules] {border-left:solid black 0.4pt; border-right:solid black 0.4pt; }
td.td00{ padding-left:0pt; padding-right:0pt; }
td.td01{ padding-left:0pt; padding-right:5pt; }
td.td10{ padding-left:5pt; padding-right:0pt; }
td.td11{ padding-left:5pt; padding-right:5pt; }
table[rules] {border-left:solid black 0.4pt; border-right:solid black 0.4pt; }
.hline hr, .cline hr{ height : 1px; margin:0px; }
.tabbing-right {text-align:right;}
span.TEX {letter-spacing: -0.125em; }
span.TEX span.E{ position:relative;top:0.5ex;left:-0.0417em;}
a span.TEX span.E {text-decoration: none; }
span.LATEX span.A{ position:relative; top:-0.5ex; left:-0.4em; font-size:85%;}
span.LATEX span.TEX{ position:relative; left: -0.4em; }
div.float img, div.float .caption {text-align:center;}
div.figure img, div.figure .caption {text-align:center;}
.marginpar {width:20%; float:right; text-align:left; margin-left:auto; margin-top:0.5em; font-size:85%; text-decoration:underline;}
.marginpar p{margin-top:0.4em; margin-bottom:0.4em;}
.equation td{text-align:center; vertical-align:middle; }
td.eq-no{ width:5%; }
table.equation { width:100%; } 
div.math-display, div.par-math-display{text-align:center;}
math .texttt { font-family: monospace; }
math .textit { font-style: italic; }
math .textsl { font-style: oblique; }
math .textsf { font-family: sans-serif; }
math .textbf { font-weight: bold; }
.partToc a, .partToc, .likepartToc a, .likepartToc {line-height: 200%; font-weight:bold; font-size:110%;}
.chapterToc a, .chapterToc, .likechapterToc a, .likechapterToc, .appendixToc a, .appendixToc {line-height: 200%; font-weight:bold;}
.index-item, .index-subitem, .index-subsubitem {display:block}
.caption td.id{font-weight: bold; white-space: nowrap; }
table.caption {text-align:center;}
h1.partHead{text-align: center}
p.bibitem { text-indent: -2em; margin-left: 2em; margin-top:0.6em; margin-bottom:0.6em; }
p.bibitem-p { text-indent: 0em; margin-left: 2em; margin-top:0.6em; margin-bottom:0.6em; }
.subsectionHead, .likesubsectionHead { margin-top:2em; font-weight: bold;}
.sectionHead, .likesectionHead { font-weight: bold;}
.quote {margin-bottom:0.25em; margin-top:0.25em; margin-left:1em; margin-right:1em; text-align:justify;}
.verse{white-space:nowrap; margin-left:2em}
div.maketitle {text-align:center;}
h2.titleHead{text-align:center;}
div.maketitle{ margin-bottom: 2em; }
div.author, div.date {text-align:center;}
div.thanks{text-align:left; margin-left:10%; font-size:85%; font-style:italic; }
div.author{white-space: nowrap;}
.quotation {margin-bottom:0.25em; margin-top:0.25em; margin-left:1em; }
h1.partHead{text-align: center}
.sectionToc, .likesectionToc {margin-left:2em;}
.subsectionToc, .likesubsectionToc {margin-left:4em;}
.sectionToc, .likesectionToc {margin-left:6em;}
.frenchb-nbsp{font-size:75%;}
.frenchb-thinspace{font-size:75%;}
.figure img.graphics {margin-left:10%;}
/* end css.sty */

\title{Etude metrique des arcs}
\author{}
\date{}

\begin{document}
\maketitle

\textbf{Warning: 
requires JavaScript to process the mathematics on this page.\\ If your
browser supports JavaScript, be sure it is enabled.}

\begin{center}\rule{3in}{0.4pt}\end{center}

[
[
[]
[

\section{18.4 Etude métrique des arcs}

\subsection{18.4.1 Arcs rectifiables}

Soit E un espace vectoriel normé de dimension finie et \Gamma = ([a,b],F)
un arc paramétré dont l'intervalle de définition est un segment de \mathbb{R}~. A
toute subdivision \sigma = (a_i)_0\leqi\leqn de [a,b] on
associe la longueur

L(\Gamma,\sigma)

de la ligne polygonale inscrite dans \Gamma dont les sommets sont les
F(a_i), c'est-à-dire

L(\Gamma,\sigma) = \sum _i=1^nd(F(a_
i-1,F(a_i)) = \\sum
_i=1^n\F(a_ i) -
F(a_i-1)\

Lemme~18.4.1 Soit \sigma et \sigma' deux subdivisions de [a,b]. Si \sigma' est plus
fine que \sigma, on a L(\Gamma,\sigma) \leq L(\Gamma,\sigma').

Démonstration Par une récurrence évidente sur le nombre de points
ajoutés à \sigma pour obtenir \sigma', il suffit de montrer le résultat lorsque \sigma
est composée de a_0 = a < a_1 <
\\ldots~ <
a_n = b et \sigma' est composée de a_0 = a <
\\ldots~ <
a_k-1 < c < a_k <
\\ldots~ <
a_n = b. Dans ce cas on a

\begin{align*} L(\Gamma,\sigma)& =&
\\sum
_i=1^n\F(a_ i) -
F(a_i-1)\ =
\\sum
_i=1^k-1\F(a_ i) -
F(a_i-1)\\%&
\\ & &
+\F(a_k) -
F(a_k-1)\ +
\\sum
_i=k+1^n\F(a_ i) -
F(a_i-1)\\ \%&
\\ \end{align*}

alors que

\begin{align*} L(\Gamma,\sigma')& =&
\\sum
_i=1^k-1\F(a_ i) -
F(a_i-1)\ +\
F(c) - F(a_k-1)\\%&
\\ & &
+\F(a_k) -
F(c)\ + \\sum
_i=k+1^n\F(a_ i) -
F(a_i-1)\\%&
\\ \end{align*}

et le résultat découle immédiatement de l'inégalité triangulaire.

Intuitivement, si l'on peut donner un sens à la longueur d'un arc
paramétré, suivant le principe la ligne droite est le plus court chemin
d'un point à un autre, la longueur de cet arc doit être plus grande que
la longueur de toute ligne polygonale inscrite dans l'arc. Ceci justifie
l'introduction de la définition suivante.

Définition~18.4.1 Soit E un espace vectoriel normé de dimension finie et
\Gamma = ([a,b],F) un arc paramétré dont l'intervalle de définition est
un segment de \mathbb{R}~. On dit que \Gamma est rectifiable si l'ensemble des L(\Gamma,\sigma)
est majoré, \sigma décrivant l'ensemble des subdivisions de [a,b]. On
appelle alors longueur de l'arc \Gamma le nombre l(\Gamma)
=\
sup\L(\Gamma,\sigma)∣\sigma\text
subdivision de [a,b]\.

Remarque~18.4.1 En général les arcs continus ne sont pas rectifiables~;
les fractales donnent de bons exemples d'arcs paramétrés dont tout sous
arc est de longueur infini. Sans aller jusque là, nous pouvons
construire facilement un graphe de fonction continue qui n'est pas
rectifiable. Prenons une fonction continue sur [0,1], de classe
\mathcal{C}^1 sur ]0,1] mais telle que la fonction
t\mapsto~\sqrt1 +
f'(t)^2 ne soit pas intégrable sur ]0,1] (par exemple
f(t) = \sqrttsin~
1\over  t^2 si
t\neq~0 et f(0) = 0). Comme nous le verrons par
la suite, la longueur du graphe de x à 1 est égale à
\int  _x^1~\sqrt1
+ f'(t)^2 dt et elle tend vers + \infty~ quand x tend vers 0,
bien que f soit continue.

Proposition~18.4.2 Soit \Gamma_1 et \Gamma_2 deux arcs définis
sur des segments et C^k-équivalents. Alors \Gamma_1 est
rectifiable si et seulement si~\Gamma_2 est rectifiable et dans ce
cas ils ont même longueur.

Démonstration Soit \Gamma_1 = ([a,b],F) et \Gamma_2 =
([c,d],G). Soit \theta un difféomorphisme de [a,b] sur [c,d] tel
que F = G \cdot \theta~; \theta est donc strictement monotone. A toute subdivision \sigma
de [a,b], on peut associer une subdivision \theta^∗(\sigma) de
[c,d] de la manière suivante~: si \sigma est donnée par a_0 = a
< a_1 <
\\ldots~ <
a_n = b, \theta^∗(\sigma) est la subdivision \theta(a_0) =
c < \theta(a_1) <
\\ldots~ <
\theta(a_n) = d si \theta est croissant et la subdivision \theta(a_n)
= c < \theta(a_n-1) <
\\ldots~ <
\theta(a_1) < \theta(a_0) = d si \theta est décroissant~;
on obtient ainsi une bijection de l'ensemble des subdivisions de
[a,b] sur l'ensemble des subdivisions de [c,d]. La ligne
polygonale joignant les points F(a_i) est encore la ligne
polygonale joignant les points G(\theta(a_i)), et donc
L(\Gamma_1,\sigma) = L(\Gamma_2,\theta^∗(\sigma)). Comme
\theta^∗ est bijective, \theta^∗(\sigma) parcourt toutes les
subdivisions de [c,d], et donc (en prenant des bornes supérieures
dans \overline\mathbb{R}~)

\begin{align*}
sup\L(\Gamma_1,\sigma)\mathrel∣~\sigma\text
subdivision de [a,b]&& \%&
\\ & =&
sup\L(\Gamma_2,\theta^∗(\sigma))\mathrel∣~\sigma\text
subdivision de [a,b]\\%&
\\ & =&
sup\L(\Gamma_2,\sigma')\mathrel∣~\sigma'\text
subdivision de [c,d]\ \%&
\\ \end{align*}

ce qui montre la proposition.

Proposition~18.4.3 Soit E un espace vectoriel normé de dimension finie
et \Gamma = ([a,b],F) un arc paramétré dont l'intervalle de définition
est un segment de \mathbb{R}~. Soit c \in]a,b[. Alors \Gamma est rectifiable si et
seulement si~les deux sous arcs \Gamma_1 =
([a,c],F__[a,c]) et \Gamma_2 =
([c,b],F__[c,b]) sont rectifiables.
Dans ce cas on a l(\Gamma) = l(\Gamma_1) + l(\Gamma_2).

Démonstration Supposons tout d'abord que \Gamma est rectifiable et soit
\sigma_1 une subdivision de [a,c], a_0 = a <
a_1 <
\\ldots~ <
a_p = c. En ajoutant le point b, on obtient une subdivision \sigma
de [a,b] et on a L(\Gamma,\sigma) = L(\Gamma_1,\sigma_1)
+\ f(b) - f(c)\. On en
déduit que L(\Gamma_1,\sigma_1) \leq l(\Gamma)
-\ f(b) - f(c)\ ce qui
montre que \Gamma_1 est rectifiable. On montre de même que
\Gamma_2 est rectifiable.

Inversement, supposons \Gamma_1 et \Gamma_2 rectifiables et soit
\sigma une subdivision de [a,b]. En ajoutant éventuellement à \sigma le point
c on obtient une subdivision \sigma' de [a,b], plus fine que \sigma et qui est
la juxtaposition d'une subdivision \sigma_1 de [a,c] et d'une
subdivision \sigma_2 de [c,b]. La longueur de la ligne
polygonale correspondant à \sigma' est donc la somme des longueurs des lignes
polygonales correspondant à \sigma_1 et \sigma_2. On a donc

L(\Gamma,\sigma) \leq L(\Gamma,\sigma') = L(\Gamma_1,\sigma_1) +
L(\Gamma_2,\sigma_2) \leq l(\Gamma_1) + l(\Gamma_2)

Ceci montre que \Gamma est rectifiable et que l(\Gamma) \leq l(\Gamma_1) +
l(\Gamma_2).

Inversement, soit \epsilon > 0. Par définition de la borne
supérieure, il existe \sigma_1 subdivision de [a,c] telle que
L(\Gamma_1,\sigma_1) ≥ l(\Gamma_1) - \epsilon
\over 2 . De même, il existe \sigma_2 subdivision
de [c,b] telle que L(\Gamma_2,\sigma_2) ≥ l(\Gamma_2)
- \epsilon \over 2 . La juxtaposition \sigma de \sigma_1 et
\sigma_2 est une subdivision de [a,b] et on a

l(\Gamma) ≥ L(\Gamma,\sigma) = L(\Gamma_1,\sigma_1) +
L(\Gamma_2,\sigma_2) ≥ l(\Gamma_1) + l(\Gamma_2) - \epsilon

Donc \forall~~\epsilon > 0, l(\Gamma) ≥
l(\Gamma_1) + l(\Gamma_2) - \epsilon et donc l(\Gamma) ≥ l(\Gamma_1) +
l(\Gamma_2). Comme l'inégalité en sens inverse était déjà connue, on
a l'égalité.

On déduit immédiatement de ce résultat que si ([a,b],F) est
rectifiable et si [c,d] est un segment contenu dans [a,b], alors
([c,d],F__[c,d]) est encore
rectifiable, autrement dit que tout sous arc d'un arc rectifiable est
rectifiable. Ceci justifie la définition suivante

Définition~18.4.2 Soit \Gamma = (I,F) un arc paramétré. On dit que \Gamma est
rectifiable si pour tout segment [a,b] \subset~ I, le sous arc
([a,b],F__[a,b]) est rectifiable.

Dans ce cas, pour tout couple a,b de I tel que a < b, on peut
définir \ell_\Gamma(a,b) =
l([a,b],F__[a,b]). La proposition
précédente montre clairement que si a < b < c, on a
\ell_\Gamma(a,c) = \ell_\Gamma(a,b) + \ell_\Gamma(b,c). On prolonge
la définition de \ell_\Gamma en posant \ell_\Gamma(a,b) = 0 si a = b
et \ell_\Gamma(a,b) = -\ell_\Gamma(b,a) si b < a (convention
de Chasles). On obtient alors facilement

Proposition~18.4.4 (relation de Chasles). Soit E un espace vectoriel
normé, \Gamma = (I,F) un arc paramétré rectifiable de E. Alors

\forall~a,b,c \in I, \ell_\Gamma~(a,c) =
\ell_\Gamma(a,b) + \ell_\Gamma(b,c)

Remarque~18.4.2 Il découle immédiatement des résultats précédents que si
deux arcs sont équivalents, ils sont simultanément rectifiables ou non
rectifiables~; de plus si \Gamma_1 est équivalent à \Gamma_2 et
de même sens (de manière à conserver la convention de Chasles), et si \theta
est le changement de paramétrage croissant qui permet de passer de l'un
à l'autre, on a

\ell_\Gamma_1(a,b) = \ell_\Gamma_2(\theta(a),\theta(b))

Par contre si \theta était décroissant, on aurait a < b \rigtharrow~ \theta(a)
> \theta(b) et la convention de Chasles donnerait
\ell_\Gamma_1(a,b) = -\ell_\Gamma_2(\theta(a),\theta(b))

\subsection{18.4.2 Arcs de classe \mathcal{C}^1}

Théorème~18.4.5 Tout arc de classe \mathcal{C}^1 est rectifiable. Plus
précisément, si \Gamma = (I,F) est un arc de classe \mathcal{C}^1 de E,
alors \Gamma est rectifiable et

\forall~a,b \in I, \ell_\Gamma~(a,b)
=\int ~
_a^b\F'(t)\
dt

Démonstration Soit [a,b] un segment inclus dans I, \Gamma_0 le
sous arc correspondant et \sigma = (a_i)_0\leqi\leqn une
subdivision de [a,b]. Comme F est de classe \mathcal{C}^1, on a

\begin{align*} L(\Gamma_0,\sigma)& =&
\\sum
_i=1^n\F(a_ i) -
F(a_i-1)\ =
\\sum
_i=1^n\
\\int  ~
_a_i-1^a_i F'(t)
dt\\%& \\ & \leq&
\sum _i=1^n~
\\int  ~
_a_i-1^a_i
\F'(t)\ dt =
\\int  ~
_a^b\F'(t)\
dt \%& \\
\end{align*}

Ceci montre que \Gamma_0 est rectifiable et que \ell_\Gamma(a,b) =
l(\Gamma_0) \leq\int ~
_a^b\F'(t)\
dt.

Fixons maintenant a \in I~; nous allons montrer que
\forall~t \in I, \ell_\Gamma~(a,t)
=\int ~
_a^t\F'(u)\
du. Comme \ell_\Gamma(a,a) = 0, il suffit de montrer que
t\mapsto~\ell_\Gamma(a,t) est dérivable et que sa
dérivée est \F'(t)\~;
on en déduira que t\mapsto~\ell_\Gamma(a,t) est de
classe \mathcal{C}^1 et que donc \ell_\Gamma(a,t) = \ell_\Gamma(a,a)
+\int  _a^t~ d
\over du \ell_\Gamma(a,u) du
=\int ~
_a^t\F'(u)\
du. Montrons tout d'abord la dérivabilité à droite. Soit h
> 0. On a alors \ell_\Gamma(a,t + h) - \ell_\Gamma(a,t) =
\ell_\Gamma(t,t + h). Mais en utilisant d'une part la ligne polygonale
triviale qui joint par un seul segment les points F(t) et F(t + h), et
d'autre part la majoration de la longueur par l'intégrale de
\F'\ déjà démontrée,
on a

\F(t + h) - F(t)\ \leq
\ell_\Gamma(t,t + h) \leq\int ~
_t^t+h\F'(u)\
du

soit encore, en divisant par h

\begin{align*} \ F(t + h)
- F(t) \over h & \leq&
\ell_\Gamma(a,t + h) - \ell_\Gamma(a,t) \over h \%&
\\ & \leq& 1 \over h
\int ~
_t^t+h\F'(u)\du
\%& \\ & =& 1 \over
h \left (\int ~
_a^t+h\F'(u)\
du -\int ~
_a^t\F'(u)\
du\right )\%& \\
\end{align*}

Quand h tend vers 0 le terme de gauche (par définition de la dérivée) et
le terme de droite (dérivée d'une intégrale par rapport à sa borne
supérieure) tendent tous les deux vers
\F'(t)\. On en déduit
que lim_h\rightarrow~0^+~
\ell_\Gamma(a,t+h)-\ell_\Gamma(a,t) \over h
=\ F'(t)\. Donc
t\mapsto~\ell_\Gamma(a,t) est dérivable à droite,
de dérivée \F'\. Le
raisonnement est similaire à gauche.

Si maintenant b et c sont dans I, on a \ell_\Gamma(b,c) =
\ell_\Gamma(a,c) - \ell_\Gamma(a,b) =\int ~
_a^c\F'(t)\
dt -\int ~
_a^b\F'(t)\
dt =\int ~
_b^c\F'(t)\
dt ce qui achève la démonstration.

Exemple~18.4.1 (i) Pour un arc paramétré plan
t\mapsto~x(t)\vec\imath +
y(t)\vecȷ dans un repère orthonormé, on a
\F'(t)\ =
\sqrtx'(t)^2  + y'(t)^2 et donc
\ell_\Gamma(a,b) =\int ~
_a^b\sqrtx'(t)^2  +
y'(t)^2 dt. En particulier le graphe d'une fonction f de
classe \mathcal{C}^1 sur [a,b] est rectifiable et sa longueur est
\int  _a^b~\sqrt1
+ f'(t)^2 dt.

(ii) Pour un arc plan donné par une équation polaire \rho = f(\theta), on a
F'(\theta) = f'(\theta)\vecu(\theta) +
f(\theta)\vecu'(\theta), soit
\F'(\theta)\ =
\sqrtf(\theta)^2  + f'(\theta)^2 et donc
\ell_\Gamma(a,b) =\int ~
_a^b\sqrtf(\theta)^2  +
f'(\theta)^2 d\theta

(iii) Pour un arc paramétré de l'espace \mathbb{R}~^3,
t\mapsto~x(t)\vec\imath +
y(t)\vecȷ + z(t)\veck dans un
repère orthonormé, on a
\F'(t)\ =
\sqrtx'(t)^2  + y'(t)^2  +
z'(t)^2 et donc \ell_\Gamma(a,b)
=\int ~
_a^b\sqrtx'(t)^2  +
y'(t)^2  + z'(t)^2 dt

Le lecteur adaptera ces formules pour un arc donné en coordonnées
cylindriques ou sphériques.

\subsection{18.4.3 Abscisses curvilignes}

Définition~18.4.3 Soit \Gamma = (I,F) un arc rectifiable. On appelle abscisse
curviligne sur \Gamma toute application s : I \rightarrow~ \mathbb{R}~ telle que
\forall~a,b \in I, \ell_\Gamma~(a,b) = s(b) - s(a). On
dit que \Gamma est paramétré par abscisse curviligne si s(t) = t est une
abscisse curviligne, autrement dit si \forall~~a,b \in
I, \ell_\Gamma(a,b) = b - a.

Remarque~18.4.3 Choisissons sur \Gamma une origine a_0. La relation
de Chasles montre de manière évidente que s(t) =
\ell_\Gamma(a_0,t) est une abscisse curviligne sur \Gamma.

Proposition~18.4.6 Sur un arc rectifiable, deux abscisses curvilignes
diffèrent d'une constante.

Démonstration Soit a_0 \in I. On a \ell_\Gamma(a_0,t) =
s_1(t) - s_1(a_0) = s_2(t) -
s_2(a_0), si bien que s_2(t) =
s_1(t) + K avec K = s_2(a_0) -
s_1(a_0).

Théorème~18.4.7 Un arc \Gamma = (I,F) de classe \mathcal{C}^1 est paramétré
par abscisse curviligne si et seulement si~\forall~~t
\in I, \F'(t)\ = 1.

Démonstration Supposons tout d'abord que \forall~~t \in
I, \F'(t)\ = 1. Alors
on a \forall~a,b \in \Gamma, \ell_\Gamma~(a,b)
=\int ~
_a^b\F'(t)\
dt =\int  _a^b~ dt = b - a.
Inversement, supposons l'arc paramétré par abscisse curviligne, alors si
a \in I, on a \forall~t \in I, t - a = \ell_\Gamma~(a,t)
=\int ~
_a^t\F'(u)\
du et en dérivant par rapport à t, on obtient 1
=\ F'(t)\.

Théorème~18.4.8 Un arc de classe C^k (k ≥ 1) est
C^k-équivalent à un arc paramétré par abscisse curviligne si
et seulement si~il est régulier.

Démonstration Le théorème précédent implique que tout arc paramétré par
abscisse curviligne est régulier et comme tout arc équivalent à un arc
régulier est lui même régulier, la condition est évidemment nécessaire.
Inversement, soit \Gamma = (I,F) un arc paramétré régulier. Soit s(t) une
abscisse curviligne sur \Gamma. On sait que, si a \in I, s(t) = s(a)
+\int ~
_a^t\F'(u)\
du et donc s est de classe C^k (car
u\mapsto~\F'(u)\
est de classe C^k-1 si F' ne s'annule pas) et s'(t)
=\ F'(t)\
> 0. On en déduit que s est un C^k
difféomorphisme de I sur J = s(I). Soit G = F \cdot s^-1 : J \rightarrow~ E.
L'arc (J,G) est C^k équivalent à (I,F) et on a (puisque
s^-1 est un difféomorphisme croissant)

\begin{align*}
\ell_J,G(u_1,u_2)& =&
\ell_(I,F)(s^-1(u_
1),s^-1(u_ 2)) \%&
\\ & =& s(s^-1(u_
1)) - s(s^-1(u_ 2)) = u_1 -
u_2\%& \\
\end{align*}

donc (J,G) est paramétré par abscisse curviligne.

Remarque~18.4.4 Soit (I,F) et (J,G) deux arcs paramétrés équivalents et
de même sens et \theta un difféomorphisme croissant tel que F = G \cdot \theta.
Supposons que (J,G) est paramétré par abscisse curviligne. On a alors
F'(t) = \theta'(t)G'(\theta(t)) et comme G'(\theta(t)) est de norme 1 et \theta'(t)
> 0, on a \theta'(t) =\
F'(t)\. On voit que \theta est déterminé à une
constante près. Si l'on pose s = \theta(t), on aura donc  ds
\over dt =\
F'(t)\ ce qui peut encore s'écrire de la
manière suivante

\begin{itemize}
\itemsep1pt\parskip0pt\parsep0pt
\item
  (i) Pour un arc paramétré plan
  t\mapsto~x(t)\vec\imath +
  y(t)\vecȷ dans un repère orthonormé, on a
  ds^2 = dx^2 + dy^2 (c'est-à-dire que
  \left ( ds \over dt
  \right )^2 = \left ( dx
  \over dt \right )^2 +
  \left ( dy \over dt
  \right )^2)
\item
  (ii) Pour un arc plan donné par une équation polaire \rho = f(\theta), on a
  ds^2 = d\rho^2 + \rho^2 d\theta^2
  (c'est-à-dire que \left ( ds \over
  d\theta \right )^2 = \left (
  d\rho \over d\theta \right )^2 +
  \rho^2)
\item
  (iii) Pour un arc paramétré de l'espace \mathbb{R}~^3,
  t\mapsto~x(t)\vec\imath +
  y(t)\vecȷ + z(t)\veck dans un
  repère orthonormé, on a ds^2 = dx^2 +
  dy^2 + dz^2 (c'est-à-dire que
  \left ( ds \over dt
  \right )^2 = \left ( dx
  \over dt \right )^2 +
  \left ( dy \over dt
  \right )^2 + \left ( dz
  \over dt \right )^2)
\end{itemize}

\subsection{18.4.4 Introduction à la méthode du repère mobile}

Soit E un espace affine euclidien de direction \vecE.
On désigne par \mathcal{R} l'ensemble des repères orthonormés de E.

Définition~18.4.4 On appelle repère mobile tout couple (I,R) d'un
intervalle I de \mathbb{R}~ et d'une application t\mapsto~R(t)
de I dans \mathcal{R}, de classe \mathcal{C}^1.

On notera a(t) \in E l'origine du repère R(t) et \mathcal{E}(t) =
(\overrightarrowe_1(t),\\ldots,\overrightarrowe_n~(t))
la base orthonormée définissant R(t). Le résultat essentiel est le
suivant

Théorème~18.4.9 Soit t\mapsto~R(t) =
(a(t),\overrightarrowe_1(t),\\ldots,\overrightarrowe_n~(t))
un repère mobile. Alors, pour tout t \in I, la matrice des coordonnées des
vecteurs  d\overrightarrowe_1
\over dt
(t),\\ldots~,
d\overrightarrowe_n \over
dt (t) dans la base
(\overrightarrowe_1(t),\\ldots,\overrightarrowe_n~(t))
est antisymétrique.

Démonstration Posons  d\overrightarrowe_j
\over dt (t) =\
\sum ~
_i=1^na_i,j(t)\overrightarrowe_i(t).
Par dérivation, la relation \forall~~t \in I,
(\overrightarrowe_i(t)∣\overrightarrowe_i(t))
= 1 donne \forall~~t \in I, 2(
d\overrightarrowe_i \over
dt
(t)∣\overrightarrowe_i(t))
= 0 soit encore a_i,i(t) = 0. De même la relation
\forall~~t \in I,
(\overrightarrowe_i(t)∣\overrightarrowe_j(t))
= 0 donne par dérivation \forall~~t \in I, (
d\overrightarrowe_i \over
dt
(t)∣\overrightarrowe_j(t))
+ ( d\overrightarrowe_j
\over dt
(t)∣\overrightarrowe_i(t))
= 0 soit encore a_i,j(t) + a_j,i(t) = 0. Ceci montre
bien que la matrice est antisymétrique.

\subsection{18.4.5 Repère de Frénet et courbure des arcs d'un plan
euclidien orienté}

On désigne par E un plan euclidien orienté.

Définition~18.4.5 Soit \Gamma = (I,F) un arc paramétré par abscisse
curviligne de classe \mathcal{C}^1. On appelle repère de Frénet au
point s \in I le repère orthonormé direct
(F(s),\vect(s),\vecn(s)) dont
l'origine est le point F(s) et tel que \vect(s) =
F'(s).

Justification~: on sait que
\F'(s)\ = 1. De plus
la connaissance de \vect(s) détermine parfaitement
\vecn(s) qui doit être l'image de
\vect(s) par la rotation d'angle + \pi~
\over 2 .

Supposons que (I,F) est un arc paramétré par abscisse curviligne de
classe C^2. Alors l'application
s\mapsto~\vect(s) = F'(s) est de
classe \mathcal{C}^1 et il en est de même de
s\mapsto~\vecn(s) =
r_\pi~\diagup2(\vect(s)). Le théorème sur le repère
mobile nous dit que la matrice des coordonnées des vecteurs 
d\vect \over ds (s),
d\vecn \over ds (s) dans la base
(\vect(s),\vecn(s)) est
antisymétrique, donc de la forme \left
(\matrix\,0 &-c(s) \cr
c(s)&0 \right ).

Définition~18.4.6 Soit \Gamma = (I,F) un arc paramétré par abscisse
curviligne de classe C^2. On appelle courbure de \Gamma au point s
de I le réel c(s) défini par la relation F''(s) =
d\vect \over ds (s) =
c(s)\vecn(s).

Remarque~18.4.5 On a donc les formules (dites formules de Frénet)

 d\vect \over ds (s) =
c(s)\vecn(s), d\vecn
\over ds (s) = -c(s)\vect(s)

Définition~18.4.7 Soit \Gamma = (I,F) un arc paramétré régulier de classe
C^2. Soit (J,G) un arc paramétré par abscisse curviligne
équivalent et de même sens, \theta : I \rightarrow~ J un difféomorphisme croissant tel
que F = G \cdot \theta. On appelle repère de Frénet (resp. courbure) à \Gamma au point
t \in I le repère de Frénet (resp. la courbure) au point \theta(t) à l'arc
(J,G).

Remarque~18.4.6 Il faut voir bien entendu que cette définition ne dépend
pas du choix de (J,G). Cela découle de la quasi-unicité de \theta que nous
avons vue précédemment, ou bien plus simplement du théorème suivant

Théorème~18.4.10 Soit \Gamma = (I,F) un arc paramétré régulier de classe
C^2. Alors le repère de Frénet et la courbure au point t à \Gamma
sont donnés respectivement par

\vect_\Gamma(t) = F'(t) \over
\F'(t)\
,\quad \vecn_\Gamma(t) =
r_\pi~\diagup2(\vect_\Gamma(t)),\quad
c_\Gamma(t) = [F'(t),F'`(t)] \over
\F'(t)\^3

où [F'(t),F''(t)] désigne le produit mixte des vecteurs F'(t) et
F''(t).

Démonstration Par définition, si (J,G) désigne un arc paramétré par
abscisse curviligne équivalent et de même sens, \theta : I \rightarrow~ J un
difféomorphisme tel que F = G \cdot \theta, on a
\vect_\Gamma(t) = G'(\theta(t)) et on a G''(\theta(t)) =
c_\Gamma(t)\vecn_\Gamma(t). On a alors

F'(t) = (G \cdot \theta)'(t) = \theta'(t)G'(\theta(t)) =
\theta'(t)\vect_\Gamma(t)

Comme \theta'(t) > 0, on a \theta'(t) =\
F'(t)\ et donc
\vect_\Gamma(t) = F'(t) \over
\F'(t)\ . De plus on
a

\begin{align*} F'`(t)& =& \theta'`(t)G'(\theta(t)) +
\theta'(t)^2G'`(\theta(t)) \%& \\ & =&
\theta'`(t)\vect_\Gamma(t) +\
F'(t)\^2c_
\Gamma(t)\vecn_\Gamma(t)\%&
\\ \end{align*}

d'où l'on déduit que

\begin{align*} [F'(t),F'`(t)]& =&
[\F'(t)\\vect_\Gamma(t),\theta'`(t)\vect_\Gamma(t)
+\
F'(t)\^2c_
\Gamma(t)\vecn_\Gamma(t)]\%&
\\ & =&
c_\Gamma(t)\F'(t)\^3[\vect_
\Gamma(t),\vecn_\Gamma(t)] =
c_\Gamma(t)\F'(t)\^3
\%& \\ \end{align*}

ce qui démontre la dernière formule.

Corollaire~18.4.11 Soit \Gamma = (I,F) un arc paramétré régulier de classe
C^2 et t \in I. On a équivalence de (i) t est un point
birégulier de \Gamma (ii) c_\Gamma(t)\neq~0

Démonstration En effet t est birégulier si et seulement si~la famille
(F'(t),F''(t)) est libre c'est-à-dire si et seulement si~le produit
mixte [F'(t),F''(t)] est non nul.

Corollaire~18.4.12 Soit \Gamma l'arc défini en polaire par \rho = f(\theta). Alors la
courbure au point \theta est donnée par

c = \rho^2 + 2\rho'^2 - \rho\rho'`\over
(\rho^2 + \rho'^2)^3\diagup2

(avec \rho' = f'(\theta) et \rho'`= f''(\theta)).

Démonstration On a F'(\theta) = f'(\theta)\vecu(\theta) +
f(\theta)\vecu'(\theta) et F'`(\theta) = (f'`(\theta) -
f(\theta))\vecu(\theta) + 2f'(\theta)\vecu'(\theta).
On en déduit que [F'(\theta),F'`(\theta)] = \rho^2 + 2\rho'^2
- \rho\rho'' et que \F'(\theta)\
= \sqrt\rho^2  + \rho'^2, d'où la
formule.

Nous allons maintenant donner une autre interprétation de la courbure à
l'aide d'une détermination de l'angle \phi du vecteur tangent unitaire
\vect avec un vecteur fixe \vec\imath.
Pour pouvoir dériver une telle détermination, nous utiliserons le lemme
suivant

Lemme~18.4.13 (lemme de relèvement \mathcal{C}^1). Soit I un intervalle
de \mathbb{R}~ et f : I \rightarrow~ U = \z \in
\mathbb{C}∣z = 1\
une application de classe \mathcal{C}^1. Alors il existe g : I \rightarrow~ \mathbb{R}~ de
classe \mathcal{C}^1 telle que \forall~~t \in I, f(t) =
e^ig(t). Si g_1 et g_2 sont deux telles
applications, il existe k \in \mathbb{Z} tel que \forall~~t \in I,
g_2(t) = g_1(t) + 2k\pi~.

Démonstration Traitons tout d'abord la question de l'unicité. Soit
g_1 et g_2 qui conviennent. Alors
\forall~~t \in I,
e^i(g_2(t)-g_1(t)) = 1 soit
\forall~t \in I, g_2(t) - g_1~(t) \in
2\pi~\mathbb{Z}. Or, d'après le théorème des valeurs intermédiaires, (g_2 -
g_1)(I) est un intervalle~; cet intervalle devant être contenu
dans 2\pi~\mathbb{Z}, c'est forcément un singleton et donc il existe k \in \mathbb{Z} tel que
\forall~t \in I, g_2(t) = g_1~(t) +
2k\pi~. En ce qui concerne l'existence, remarquons que si g existe, on doit
avoir f'(t) = ig'(t)e^ig(t) = ig'(t)f(t) et donc g'(t) =
f'(t) \over if(t) .

Puisque \forall~~t \in
I,f(t)\overlinef(t) = 1, par dérivation on obtient
\forall~t \in I,f(t)\overlinef'(t)~ +
f'(t)\overlinef(t) = 0, donc  1
\over i f'(t)\overlinef(t) =
f'(t)\over if(t) \in \mathbb{R}~. Soit donc a \in I~; choisissons \alpha~
\in \mathbb{R}~ tel que e^i\alpha~ = f(a) et soit g(t) = \alpha~
+\int  _a^t~ f'(u)
\over if(u) du = \alpha~ - i\int ~
_a^tf'(u)\overlinef(u) du. La fonction
g est de classe \mathcal{C}^1 de I dans \mathbb{R}~ et g'(t) = f'(t)
\over if(t) ~; posons h(t) = f(t)e^-ig(t)~;
h est de classe \mathcal{C}^1 et on a h'(t) = f'(t)e^-ig(t)
- ig'(t)f(t)e^-ig(t) = \left (f'(t) -
ig'(t)f(t)\right )e^-ig(t) = 0. Donc h est
constante. Comme h(a) = f(a)e^-ig(a) = f(a)e^-i\alpha~ =
1, h est la constante 1 et donc \forall~~t \in I, f(t) =
e^ig(t).

Lemme~18.4.14 Soit E un plan vectoriel euclidien,
(\vec\imath,\vecȷ) une base orthonormée
de E, S = \x \in
E∣\x\
= 1\. Soit I un intervalle de \mathbb{R}~ et f : I \rightarrow~ S de classe
\mathcal{C}^1. Alors il existe g : I \rightarrow~ \mathbb{R}~ de classe \mathcal{C}^1 telle
que \forall~~t \in I, f(t) =\
cos g(t)\,\vec\imath
+ sin~
g(t)\,\vecȷ. Si g_1 et
g_2 sont deux telles applications, il existe k \in \mathbb{Z} tel que
\forall~t \in I, g_2(t) = g_1~(t) +
2k\pi~.

Démonstration Il suffit d'utiliser l'isométrie L de E dans \mathbb{C} (considérés
tous deux comme espaces euclidiens) définie par
L(a\vec\imath + b\vecȷ) = a + ib et
d'appliquer le lemme précédent à L \cdot f. On a alors L \cdot f(t) =
e^ig(t), soit f(t) = L^-1(e^ig(t))
= cos~
g(t)\,\vec\imath +\
sin g(t)\,\vecȷ.

Théorème~18.4.15 Soit \Gamma = (I,F) un arc paramétré régulier de classe
C^2 du plan euclidien E. Soit
(\vec\imath,\vecȷ) une base orthonormée
de E. Pour chaque t \in I soit \vect_\Gamma(t) le
vecteur tangent unitaire au point t, soit \phi : I \rightarrow~ \mathbb{R}~ une application de
classe \mathcal{C}^1 telle que \forall~~t \in I,
\vect_\Gamma(t) = cos~
\phi(t)\,\vec\imath +\
sin \phi(t)\,\vecȷ et
t\mapsto~s(t) une abscisse curviligne sur \Gamma. Alors

\forall~t \in I, \phi'(t) = c_\Gamma~(t)s'(t)

Démonstration Si (J,G) désigne un arc paramétré par abscisse curviligne
équivalent et de même sens, \theta : I \rightarrow~ J un difféomorphisme tel que F = G \cdot
\theta, on a \vect_\Gamma(t) = G'(\theta(t))~; on sait
aussi que \theta'(t) =\
F'(t)\ = s'(t). On a donc F'(t) =
\theta'(t)\vect_\Gamma(t) =
s'(t)(cos~
\phi(t)\,\vec\imath +\
sin \phi(t)\,\vecȷ) d'où F'`(t) =
s'`(t)(cos~
\phi(t)\,\vec\imath +\
sin \phi(t)\,\vecȷ) +
s'(t)\phi'(t)(-sin~
\phi(t)\,\vec\imath +\
cos \phi(t)\,\vecȷ) ce qui nous donne
[F'(t),F'`(t)] = s'(t)^2\phi'(t) alors que
\F'(t)\ = s'(t). La
formule s'en déduit immédiatement.

Remarque~18.4.7 Sur un arc paramétré par abscisse curviligne
s\mapsto~F(s), on peut prendre le paramètre comme
abscisse curviligne et on a donc \forall~~s \in I,
c_\Gamma(s) = d\phi \over ds (s). La courbure mesure
donc la vitesse avec laquelle tourne le vecteur tangent en fonction de
l'abscisse curviligne.

Formulaire

Pour un arc paramétré
t\mapsto~x(t)\vec\imath +
y(t)\vecȷ, on a vu que \left ( ds
\over dt \right )^2 =
\left ( dx \over dt
\right )^2 + \left ( dy
\over dt \right )^2~; de plus
F'(t) =\
F'(t)\(cos~
\phi(t)\,\vec\imath +\
sin \phi(t)\,\vecȷ) = ds
\over dt (cos~
\phi(t)\,\vec\imath +\
sin \phi(t)\,\vecȷ) ce qui nous donne
 dx \over dt = ds \over dt
 cos \phi et  dy \over dt~ =
ds \over dt  sin~ \phi, avec
enfin  d\phi \over dt = c_\Gamma\,
ds \over dt . On retiendra ces formules en termes de
formes différentielles sous la forme

dx = ds\,cos~
\phi,\quad dy =
ds\,sin~ \phi,\quad
d\phi = c\,ds

d'où l'on retrouve que ds^2 = dx^2 +
dy^2.

Pour un arc en polaires donné par \rho = f(\theta), on a F'(\theta) =
f'(\theta)\vecu(\theta) + f(\theta)\vecu'(\theta). Par
le théorème de relèvement, il existe une application
\theta\mapsto~\alpha~(\theta) de classe C^1 telle que
\vect(\theta) = cos~
\alpha~(\theta)\,\vecu(\theta)
+ sin~
\alpha~(\theta)\,\vecu'(\theta) et on a alors

\begin{align*} F'(\theta)& =&
f'(\theta)\vecu(\theta) + f(\theta)\vecu'(\theta) \%&
\\ & =&
\F'(\theta)\(cos~
\alpha~(\theta)\,\vecu(\theta)
+ sin~
\alpha~(\theta)\,\vecu'(\theta))\%&
\\ & =& ds \over d\theta
(cos~
\alpha~(\theta)\,\vecu(\theta)
+ sin~
\alpha~(\theta)\,\vecu'(\theta)) \%&
\\ \end{align*}

On en déduit que \rho = f(\theta) = ds \over d\theta
 sin~ \alpha~(\theta) et que  d\rho \over
d\theta = ds \over d\theta  sin~
\alpha~(\theta). D'autre part on peut évidemment prendre \phi(\theta) = \theta + \alpha~(\theta) ce qui
nous donne les formules suivantes en termes de formes différentielles

d\rho = ds\,cos~
\alpha~,\quad \rho\,d\theta =
ds\,sin~ \alpha~, \phi = \theta +
\alpha~,\quad d\phi = c\,ds

Exemple~18.4.2 Considérons la cardioïde d'équation \rho = a(1
+ cos~ \theta), \theta \in] - \pi~,\pi~[. On a alors
ds\,cos~ \alpha~ = d\rho =
-asin~ \theta\,d\theta =
-2acos  \theta \over 2~
 sin  \theta \over 2~
\,d\theta et ds\,sin~
\alpha~ = \rho\,d\theta = a(1 + cos~
\theta)\,d\theta = 2acos ^2~
\theta \over 2 \,d\theta. Comme  ds
\over d\theta > 0, ceci nécessite ds =
2acos  \theta \over 2~ d\theta. On a
alors cos \alpha~ = -\sin~
 \theta \over 2 , sin~ \alpha~
= cos  \theta \over 2~ , ce qui
détermine \alpha~ à une constante près par \alpha~ = \theta \over 2
+ \pi~ \over 2 . On en déduit que \phi = \alpha~ + \theta = 3\theta
\over 2 + \pi~ \over 2 et donc d\phi = 3
\over 2 \,d\theta = c\,ds =
2accos  \theta \over 2~ d\theta. Par
conséquent c = 3 \over 4a cos~
 \theta \over 2  .

\subsection{18.4.6 Centre de courbure, cercle osculateur}

Définition~18.4.8 Soit \Gamma = (I,F) un arc paramétré régulier de classe
C^2 et t \in I un point birégulier de \Gamma. On appelle rayon de
courbure au point t le nombre réel R_\Gamma(t) = 1
\over c_\Gamma(t) .

Remarque~18.4.8 Dans un arc paramétré par abscisse curviligne
s\mapsto~F(s), les formules de Frénet peuvent alors
s'écrire

 d\vect \over ds (s) = 1
\over R(s) \vecn(s),
d\vecn \over ds (s) = - 1
\over R(s) \vect(s)

Définition~18.4.9 Soit \Gamma = (I,F) un arc paramétré régulier de classe
C^2 et t \in I un point birégulier de \Gamma. Soit
(F(t),\vect_\Gamma(t),\vecn_\Gamma(t))
le repère de Frénet au point t et R_\Gamma(t) le rayon de courbure
au point t. On appelle centre de courbure au point t le point
C_\Gamma(t) = F(t) +
R_\Gamma(t)\vecn_\Gamma(t), on appelle cercle
de courbure au point t le cercle ayant pour centre le centre de courbure
et pour rayon le rayon de courbure (et qui passe donc par le point
F(t)).

Remarque~18.4.9 Par définition même, il s'agit de notions invariantes
par changement de paramétrage admissible~: si \Gamma = (I,F) et \Gamma' = (J,G)
sont deux arcs équivalents et de même sens et si \theta est un
difféomorphisme croissant tel que F = G \cdot \theta, alors le rayon de courbure
(resp. le centre de courbure, resp. le cercle de courbure) au point \theta(t)
à \Gamma' est le rayon de courbure (resp. le centre de courbure, resp. le
cercle de courbure) au point t à \Gamma.

Théorème~18.4.16 Soit \Gamma = (I,F) un arc paramétré régulier de classe
C^2 et t_0 \in I un point birégulier de \Gamma. Alors le
cercle de courbure au point t_0 est l'unique cercle osculateur
à \Gamma au point t_0.

Démonstration La notion de contact étant invariante par changement de
paramétrage admissible, on peut supposer que l'arc est paramétré par
abscisse curviligne, s\mapsto~F(s) et nous noterons
s_0 à la place de t_0. L'étude du contact pouvant se
faire dans n'importe quel repère, nous pouvons à cet effet utiliser le
repère de Frénet au point s_0. Si on note alors
\overrightarrowF(s_0)F(s) =
x(s)\vect(s_0) +
y(s)\vecn(s_0), les formules
F'(s_0) =\vec t(s_0) et
F''(s_0) =
c(s_0)\vecn(s_0) se traduisent par
x'(s_0) = 1,y'(s_0) = 0,x'`(s_0) =
0,y''(s_0) = c(s_0) avec bien entendu x(s_0)
= y(s_0) = 0 puisque F(s_0) est l'origine du repère.
Considérons un cercle passant par F(s_0)~; dans ce repère de
Frénet il admet une équation du type X^2 + Y ^2 -
2\alpha~X - 2\beta~Y = 0, son centre ayant pour coordonnées (\alpha~,\beta~). Si on introduit
l'équation aux points d'intersection du cercle et de \Gamma, on pose \phi(s) =
x(s)^2 + y(s)^2 - 2\alpha~x(s) - 2\beta~y(s) et la condition
d'osculation se traduit par \phi(s_0) = \phi'(s_0) =
\phi''(s_0) = 0. Mais compte tenu des formules x(s_0) =
y(s_0) = 0,x'(s_0) = 1,y'(s_0) =
0,x'`(s_0) = 0,y''(s_0) = c(s_0), on a
\phi(s_0) = 0, \phi'(s_0) = 2x(s_0)x'(s_0)
+ 2y(s_0)y'(s_0) - 2\alpha~x'(s_0) -
2\beta~y'(s_0) = -2\alpha~ et \phi'`(s_0) =
2x'(s_0)^2 + 2x(s_0)x'`(s_0) +
2y'(s_0)^2 + 2y(s_0)y'`(s_0) -
2\alpha~x'`(s_0) - 2\beta~y''(s_0) = 2 - 2\beta~c(s_0). On
voit donc que le cercle est tangent si et seulement si~\alpha~ = 0 (autrement
dit le cercle est centré sur la normale, c'était prévisible) et qu'il
est osculateur si et seulement si~\alpha~ = 0,\beta~ = 1 \over
c(s_0) = R(s_0), c'est-à-dire si et seulement si~son
centre est le point F(s_0) +
R(s_0)\vecn(s_0), soit le centre de
courbure.

\subsection{18.4.7 Développée, développantes}

Ces notions ne sont pas au programme des classes préparatoires.

Définition~18.4.10 Soit \Gamma = (I,F) un arc paramétré birégulier de classe
C^3. On appelle développée de \Gamma l'arc (I,C) tel que pour tout
t \in I, C(t) soit le centre de courbure au point t à \Gamma.

Théorème~18.4.17 Soit \Gamma = (I,F) un arc paramétré birégulier de classe
C^k (k ≥ 3) de développée (I,C) et soit t_0 \in I un
point non totalement singulier de (I,C). Alors la tangente au point
t_0 à la développée est la normale au point t_0 à \Gamma,
c'est-à-dire la droite F(t_0) +
\mathbb{R}~\vecn_\Gamma(t_0)~: la développée de
l'arc \Gamma est l'enveloppe des normales à \Gamma.

Démonstration Par définition même cette normale passe par
C(t_0). Il nous suffit donc de montrer qu'elle est parallèle à
un vecteur tangent à (I,C). Toutes les notions étant invariantes par
changement de paramétrage admissible, on peut supposer que l'arc est
paramétré par abscisse curviligne, s\mapsto~F(s) et
nous noterons s_0 à la place de t_0. On a alors C(s) =
F(s) + R(s)\vecn(s). Les formules de calcul de la
courbure (et donc du rayon de courbure) montrent que
s\mapsto~R(s) est de classe C^k-2~;
d'autre part s\mapsto~\vecn(s)
est (comme s\mapsto~\vect(s) dont
elle se déduit par rotation) de classe C^k-1. On en déduit
que s\mapsto~C(s) est de classe C^k-2. On
a alors

\begin{align*} C'(s)& =& F'(s) +
R'(s)\vecn(s) + R(s) d\vecn
\over dt (s) \%& \\ &
=& \vect(s) + R'(s)\vecn(s) -
R(s) \vect(s) \over R(s) =
R'(s)\vecn(s)\%& \\
\end{align*}

On en déduit que s_0 est un point régulier de la développée si
et seulement si~R'(s_0)\neq~0 et que
dans ce cas le vecteur \vecn(s_0) est un
vecteur tangent à la développée, ce qui démontre dans ce cas le
résultat. La formule de Leibnitz appliquée à C'(s) =
R'(s)\vecn(s) nous montre clairement que s_0
est un point non totalement singulier de la développée si et seulement
si~il existe n tel que
R^(n)(s_0)\neq~0. Si l'on
suppose alors que R'(s_0) =
\\ldots~ =
R^(p-1)(s_0) = 0 et
R^(p)(s_0)\neq~0, cette même
formule de Leibnitz montre que C'(s_0) =
\\ldots~ =
C^(p-1)(s_0) = 0 et C^(p)(s_0) =
R^(p)(s_0)\vecn(s_0)\neq~0
(tous les autres termes faisant intervenir des dérivées d'ordre
inférieur de R qui sont nulles au point s_0). Le vecteur
\vecn(s_0) est donc un vecteur tangent à la
développée, ce qui démontre dans le résultat.

Remarque~18.4.10 Il est parfois beaucoup plus rapide de rechercher la
développée comme enveloppe des normales que de calculer repère de Frénet
et rayon de courbure. On pourra également, lorsque l'arc est paramétré
par t\mapsto~x(t)\vec\imath +
y(t)\vecȷ paramétrer la développée par
t\mapsto~\xi(t)\vec\imath +
\eta(t)\vecȷ avec

\begin{align*} \xi(t)\vec\imath +
\eta(t)\vecȷ& =& F(t) +
R(t)\vecn(t) \%& \\ &
=& x(t)\vec\imath + y(t)\vecȷ +
s'(t) \over \phi'(t) (-sin~
\phi(t)\vec\imath + cos~
\phi(t)\vecȷ)\%& \\
\end{align*}

et en tenant compte de x'(t) = s'(t)cos~ \phi(t)
et de y'(t) = s'(t)sin~ \phi(t) (qui traduit
simplement F'(t) =\
F'(t)\\vect(t), on obtient
\xi(t) = x(t) - y'(t) \over \phi'(t) , \eta(t) = y(t) +
x'(t) \over \phi'(t) qu'on retiendra en terme de formes
différentielles

\xi = x - dy \over d\phi ,\quad \eta = y +
dx \over d\phi

Nous avons vu également que s_0 est un point régulier de la
développée si et seulement
si~R'(s_0)\neq~0. On en déduit que les
points qui annulent R', et en particulier les points qui réalisent des
extremums locaux de R (qui sont appelés les sommets de \Gamma), donnent des
points singuliers de la développée (le plus souvent des points de
rebroussement de première espèce).

Considérons maintenant le problème inverse~: étant donné un arc \Gamma =
(I,F) (que nous supposerons régulier), peut-on trouver un arc (I,G) dont
\Gamma soit la développée. On peut, sans nuire à la généralité, supposer que
(I,F) est paramétré par abscisse curviligne. La tangente F(s) +
\mathbb{R}~\vect(s) à l'arc \Gamma doit être la normale à l'arc
(I,G) et en particulier le point G(s) doit appartenir à cette tangente.
On doit donc avoir G(s) = F(s) + \lambda~(s)\vect(s) avec
\lambda~(s) =
(\overrightarrowG(s)F(s)∣\vect(s))
qui doit donc être de classe \mathcal{C}^1. On en déduit que

\begin{align*} G'(s)& =& F'(s) +
\lambda~'(s)\vect(s) +
\lambda~(s)c(s)\vecn(s)\%& \\
& =& (1 + \lambda~'(s))\vect(s) +
\lambda~(s)c(s)\vecn(s) \%&
\\ \end{align*}

Comme la tangente à l'arc \Gamma doit être la normale à l'arc (I,G), il est
nécessaire que G'(s) soit orthogonal à \vect(s),
c'est-à-dire que 1 + \lambda~'(s) = 0. Ceci exige que \lambda~(s) = s_0 - s.
Inversement, si cette condition est vérifiée, on a G(s) = F(s) +
(s_0 - s)\vect(s) et G'(s) = (s_0 -
s)c(s)\vecn(s). Si l'arc \Gamma est birégulier et si
s_0∉I, alors l'arc (I,G) est
régulier et \Gamma est l'enveloppe des normales à (I,G), donc la développée
de (I,G). Pour un arc quelconque, ceci amène à introduire la définition
suivante

Définition~18.4.11 Soit \Gamma = (I,F) un arc paramétré régulier de classe
\mathcal{C}^1, t\mapsto~s(t) une abscisse
curviligne sur \Gamma et \vect(t) le vecteur tangent
unitaire au point t. Les arcs paramétrés
t\mapsto~F(t) + (s_0 -
s(t))\vect(t) où s_0 \in \mathbb{R}~, sont appelés les
développantes de l'arc \Gamma.

Remarque~18.4.11 L'arc n'est la développée d'une de ses développantes
t\mapsto~F(t) + (s_0 -
s(t))\vect(t) que s'il est suffisamment dérivable,
birégulier et si s_0 - s(t) ne s'annule pas sur I.

\subsection{18.4.8 Equations intrinsèques}

Cette notion n'est pas au programme des classes préparatoires.

Une question supplémentaire se pose~: étant donnée une fonction continue
\gamma : I \rightarrow~ \mathbb{R}~, peut-on trouver un arc paramétré par abscisse curviligne \Gamma
tel que la courbure au point s \in I soit égale à \gamma(s)~? Le théorème
suivant répond à cette question

Théorème~18.4.18 Soit \gamma : I \rightarrow~ \mathbb{R}~ une fonction continue, E un plan
euclidien orienté, a \in E et \vecu
\in\overrightarrow E tel que
\u\ = 1. Soit
s_0 \in I. Alors il existe un unique arc paramétré par abscisse
curviligne (I,F) de classe C^2 tel que F(s_0) = a et
F'(s_0) =\vec u.

Démonstration Soit (\vec\imath,\vecȷ)
une base orthonormée de \overrightarrowE et
s\mapsto~\phi(s) une détermination de classe
\mathcal{C}^1 de l'angle de F'(s) =\vec t(s) avec
\vec\imath. Alors nécessairement F'(s)
= cos \phi(s)\vec\imath~
+ sin \phi(s)\vecȷ~. Mais on
sait de plus que \phi'(s) = c_\Gamma(s) = \gamma(s). Alors nécessairement
\phi(s) = \alpha~ +\int ~
_s_0^s\gamma(u) du (où \alpha~ désigne une mesure de l'angle
de \vec\imath avec \vecu)~; ceci
détermine \phi(s) à un élément de 2\pi~\mathbb{Z} près et donc détermine parfaitement
\vect(s). Mais alors nécessairement

\begin{align*} F(s)& =& F(s_0)
+\int  _s_0^s~F'(u) du =
a +\int ~
_s_0^s\vect(u) du\%&
\\ & =& a +\\int
 _s_0^s(cos~
\phi(u)\vec\imath + sin~
\phi(u)\vecȷ) du \%& \\
\end{align*}

ce qui montre l'unicité de F. Mais en faisant les calculs en sens
inverse, on constate que si on définit F(s) par cette formule, l'arc
(I,F) convient~: il est paramétré par abscisse curviligne car
\F'(s)\
=\ cos~
\phi(s)\vec\imath + sin~
\phi(s)\vecȷ\ = 1, \phi(s) est une
détermination de classe \mathcal{C}^1 de l'angle de F'(s)
=\vec t(s) avec \vec\imath et comme
\phi'(s) = \gamma(s), la courbure au point s est bien \gamma(s).

Corollaire~18.4.19 Soit \Gamma_1 = (I,F_1) et \Gamma_2
= (I,F_2) deux arcs paramétrés par abscisse curviligne tels que
\forall~s \in I, c_\Gamma_1~(s) =
c_\Gamma_2(s). Alors les deux arcs sont isométriques.

Démonstration Soit s_0 \in I et soit D l'unique déplacement de E
qui vérifie D(F_1(s_0)) = F_2(s_0)
et \overrightarrowD(F_1'(s_0)) =
F_2'(s_0). Les deux arcs \Gamma_2 et
D(\Gamma_1) ont la même fonction de courbure en fonction de
l'abscisse curviligne (car la courbure, notion métrique, est bien
entendu invariante par déplacement), coïncident en s_0 ainsi
que leurs vecteurs tangents. D'après l'unicité du théorème précédent,
ces deux arcs sont égaux.

Remarque~18.4.12 Autrement dit, la fonction
s\mapsto~c_\Gamma(s) définit l'arc \Gamma à un
déplacement près du plan affine euclidien. On dira encore que l'équation
c_\Gamma(s) = \gamma(s) est une équation intrinsèque de l'arc \Gamma.

\subsection{18.4.9 Courbure des arcs gauches}

Nous supposerons maintenant que E est un espace euclidien de dimension
3, orienté. Soit \Gamma = (I,F) un arc paramétré par abscisse curviligne de
classe C^2. On a donc \forall~~s \in I,
\F'(s)\ = 1 et donc
\forall~~s \in I,
(F'(s)∣F'(s)) = 1. En dérivant, on obtient
\forall~~s \in I,(F'(s),F''(s)) = 0 ce qui montre que le
vecteur F''(s) est orthogonal au vecteur F'(s). Si le point s est
birégulier, on a  F'`(s) \over
\F''(s)\ qui est un
vecteur unitaire orthogonal au vecteur unitaire F'(s). Ceci définit de
manière unique une base orthonormée directe dont les deux premiers
vecteurs sont F'(s) et  F'`(s) \over
\F''(s)\ .

Définition~18.4.12 Soit E un espace euclidien de dimension 3 orienté.
Soit \Gamma = (I,F) un arc paramétré par abscisse curviligne de classe
C^2 et s un point birégulier de \Gamma. On appelle repère de
Frénet au point s le repère orthonormé direct
(F(s),\vect(s),\vecn(s),\vecb(s))
dont l'origine est le point image F(s) et tel que
\vect(s) = F'(s), \vecn(s) =
F'`(s) \over
\F''(s)\\
, \vecb(s) =\vec t(s)
∧\vec n(s).

Si l'arc est birégulier de classe C^3, on dispose ainsi d'un
repère mobile
s\mapsto~(F(s),\vect(s),\vecn(s),\vecb(s)).
La théorie du repère mobile assure que la matrice des coordonnées des
dérivées des vecteurs
\vect(s),\vecn(s),\vecb(s)
dans la base
(\vect(s),\vecn(s),\vecb(s))
est antisymétrique. En tenant compte de ce que 
d\vect \over ds (s) = F''(s) est
colinéaire à \vecn, cette matrice est nécessairement
de la forme

\left (\matrix\,0
&-c(s)&0 \cr c(s)&0 &-\tau(s) \cr 0
&\tau(s) &0 \right )

Définition~18.4.13 Soit E un espace euclidien de dimension 3 orienté.
Soit \Gamma = (I,F) un arc paramétré par abscisse curviligne birégulier de
classe C^3. Pour s \in I, on appelle courbure et torsion au
point s les nombres réels définis par les formules

 d\vect \over ds (s) =
c(s)\vecn(s),\quad 
d\vecb \over ds (s) =
-\tau(s)\vecn(s)

On a également  d\vecn \over ds
(s) = -c(s)\vect(s) +
\tau(s)\vecb(s).

Remarque~18.4.13 On a  d\vect \over
ds (s) = F''(s) = c(s)\vecn(s), d'où nécessairement
c(s) =\ F''(s)\
> 0. Contrairement aux arcs plans, la courbure d'un arc
gauche est nécessairement strictement positive.

Comme dans le cas d'un arc plan, on étend ces définitions aux arcs
biréguliers par équivalence d'arcs.

Définition~18.4.14 Soit \Gamma = (I,F) un arc paramétré birégulier de classe
C^3 de E espace euclidien de dimension 3 orienté. Soit (J,G)
un arc paramétré par abscisse curviligne équivalent et de même sens, \theta :
I \rightarrow~ J un difféomorphisme croissant tel que F = G \cdot \theta. On appelle repère
de Frénet (resp. courbure, resp. torsion) à \Gamma au point t \in I le repère
de Frénet (resp. la courbure, resp. la torsion) au point \theta(t) à l'arc
(J,G).

Méthodes de calcul

Par définition, si s_0 = \theta(t_0), on a
\vect(t_0) = G'(s_0) et
\vecn(t_0) = G'`(s_0)
\over
\G''(s_0)\
. Mais on a G = F \cdot \theta ce qui nous donne F'(t_0) =
\theta'(t_0)G'(\theta(t_0)) =
\theta'(t_0)\vect(t_0) et
F'`(t_0) = \theta'`(t_0)G'(\theta(t_0)) +
\theta'(t_0)^2G''(\theta(t_0)). On en déduit que
\theta'(t_0) =\
F'(t_0)\ (comme d'habitude) et que

\begin{align*} F'(t_0) ∧
F'`(t_0)& =& \theta'(t_0)^3G'(s_ 0) ∧
G'`(s_0) \%& \\ & =&
\theta'(t_0)^3c(t_
0)\vect(t_0) ∧\vec
n(t_0) = \theta'(t_0)^3c(t_
0)\vecb(t_0)\%&
\\ \end{align*}

Comme \theta'(t_0)^3c(t_0) > 0,
c'est que \vecb(t_0) =
F'(t_0)∧F'`(t_0) \over
\F'(t_0)∧F''(t_0)\
. On a ensuite \vecn(t_0)
=\vec b(t_0) ∧\vec
t(b_0).

Reprenons alors le calcul ci dessus. On a vu que F'(t_0) ∧
F'`(t_0) =
\theta'(t_0)^3c(t_0)\vecb(t_0).
On en déduit, puisque la courbure est positive, que c(t_0) =
\F'(t_0)∧F'`(t_0)\
\over \theta'(t_0)^3 =
\F'(t_0)∧F'`(t_0)\
\over
\F'(t_0)\^3
. On en déduit la proposition

Proposition~18.4.20 Soit \Gamma = (I,F) un arc paramétré birégulier de classe
C^3 de E espace euclidien de dimension 3 orienté et
t_0 \in I. Alors le repère de Frénet au point t_0 est
donné par les formules

\vect(t_0) = F'(t_0)
\over
\F'(t_0)\
,\quad \vecb(t_0) =
F'(t_0) ∧ F'`(t_0) \over
\F'(t_0) ∧
F''(t_0)\

\vecn(t_0) =\vec
b(t_0) ∧\vec t(b_0)

La courbure au point t_0 est donnée par

c(t_0) = \F'(t_0) ∧
F'`(t_0)\ \over
\F'(t_0)\^3

Remarque~18.4.14 Avec la même technique, le lecteur montrera sans
difficulté que la torsion est donnée par \tau(t_0) =
[F'(t_0),F'`(t_0),F'`'(t_0)]
\over
\F'(t_0)∧F''(t_0)\^2
.

[
[
[
[

\end{document}

\part{Surfaces}
% \documentclass[]{article}
\usepackage[T1]{fontenc}
\usepackage{lmodern}
\usepackage{amssymb,amsmath}
\usepackage{ifxetex,ifluatex}
\usepackage{fixltx2e} % provides \textsubscript
% use upquote if available, for straight quotes in verbatim environments
\IfFileExists{upquote.sty}{\usepackage{upquote}}{}
\ifnum 0\ifxetex 1\fi\ifluatex 1\fi=0 % if pdftex
  \usepackage[utf8]{inputenc}
\else % if luatex or xelatex
  \ifxetex
    \usepackage{mathspec}
    \usepackage{xltxtra,xunicode}
  \else
    \usepackage{fontspec}
  \fi
  \defaultfontfeatures{Mapping=tex-text,Scale=MatchLowercase}
  \newcommand{\euro}{€}
\fi
% use microtype if available
\IfFileExists{microtype.sty}{\usepackage{microtype}}{}
\ifxetex
  \usepackage[setpagesize=false, % page size defined by xetex
              unicode=false, % unicode breaks when used with xetex
              xetex]{hyperref}
\else
  \usepackage[unicode=true]{hyperref}
\fi
\hypersetup{breaklinks=true,
            bookmarks=true,
            pdfauthor={},
            pdftitle={Nappes parametrees},
            colorlinks=true,
            citecolor=blue,
            urlcolor=blue,
            linkcolor=magenta,
            pdfborder={0 0 0}}
\urlstyle{same}  % don't use monospace font for urls
\setlength{\parindent}{0pt}
\setlength{\parskip}{6pt plus 2pt minus 1pt}
\setlength{\emergencystretch}{3em}  % prevent overfull lines
\setcounter{secnumdepth}{0}
 
/* start css.sty */
.cmr-5{font-size:50%;}
.cmr-7{font-size:70%;}
.cmmi-5{font-size:50%;font-style: italic;}
.cmmi-7{font-size:70%;font-style: italic;}
.cmmi-10{font-style: italic;}
.cmsy-5{font-size:50%;}
.cmsy-7{font-size:70%;}
.cmex-7{font-size:70%;}
.cmex-7x-x-71{font-size:49%;}
.msbm-7{font-size:70%;}
.cmtt-10{font-family: monospace;}
.cmti-10{ font-style: italic;}
.cmbx-10{ font-weight: bold;}
.cmr-17x-x-120{font-size:204%;}
.cmsl-10{font-style: oblique;}
.cmti-7x-x-71{font-size:49%; font-style: italic;}
.cmbxti-10{ font-weight: bold; font-style: italic;}
p.noindent { text-indent: 0em }
td p.noindent { text-indent: 0em; margin-top:0em; }
p.nopar { text-indent: 0em; }
p.indent{ text-indent: 1.5em }
@media print {div.crosslinks {visibility:hidden;}}
a img { border-top: 0; border-left: 0; border-right: 0; }
center { margin-top:1em; margin-bottom:1em; }
td center { margin-top:0em; margin-bottom:0em; }
.Canvas { position:relative; }
li p.indent { text-indent: 0em }
.enumerate1 {list-style-type:decimal;}
.enumerate2 {list-style-type:lower-alpha;}
.enumerate3 {list-style-type:lower-roman;}
.enumerate4 {list-style-type:upper-alpha;}
div.newtheorem { margin-bottom: 2em; margin-top: 2em;}
.obeylines-h,.obeylines-v {white-space: nowrap; }
div.obeylines-v p { margin-top:0; margin-bottom:0; }
.overline{ text-decoration:overline; }
.overline img{ border-top: 1px solid black; }
td.displaylines {text-align:center; white-space:nowrap;}
.centerline {text-align:center;}
.rightline {text-align:right;}
div.verbatim {font-family: monospace; white-space: nowrap; text-align:left; clear:both; }
.fbox {padding-left:3.0pt; padding-right:3.0pt; text-indent:0pt; border:solid black 0.4pt; }
div.fbox {display:table}
div.center div.fbox {text-align:center; clear:both; padding-left:3.0pt; padding-right:3.0pt; text-indent:0pt; border:solid black 0.4pt; }
div.minipage{width:100%;}
div.center, div.center div.center {text-align: center; margin-left:1em; margin-right:1em;}
div.center div {text-align: left;}
div.flushright, div.flushright div.flushright {text-align: right;}
div.flushright div {text-align: left;}
div.flushleft {text-align: left;}
.underline{ text-decoration:underline; }
.underline img{ border-bottom: 1px solid black; margin-bottom:1pt; }
.framebox-c, .framebox-l, .framebox-r { padding-left:3.0pt; padding-right:3.0pt; text-indent:0pt; border:solid black 0.4pt; }
.framebox-c {text-align:center;}
.framebox-l {text-align:left;}
.framebox-r {text-align:right;}
span.thank-mark{ vertical-align: super }
span.footnote-mark sup.textsuperscript, span.footnote-mark a sup.textsuperscript{ font-size:80%; }
div.tabular, div.center div.tabular {text-align: center; margin-top:0.5em; margin-bottom:0.5em; }
table.tabular td p{margin-top:0em;}
table.tabular {margin-left: auto; margin-right: auto;}
div.td00{ margin-left:0pt; margin-right:0pt; }
div.td01{ margin-left:0pt; margin-right:5pt; }
div.td10{ margin-left:5pt; margin-right:0pt; }
div.td11{ margin-left:5pt; margin-right:5pt; }
table[rules] {border-left:solid black 0.4pt; border-right:solid black 0.4pt; }
td.td00{ padding-left:0pt; padding-right:0pt; }
td.td01{ padding-left:0pt; padding-right:5pt; }
td.td10{ padding-left:5pt; padding-right:0pt; }
td.td11{ padding-left:5pt; padding-right:5pt; }
table[rules] {border-left:solid black 0.4pt; border-right:solid black 0.4pt; }
.hline hr, .cline hr{ height : 1px; margin:0px; }
.tabbing-right {text-align:right;}
span.TEX {letter-spacing: -0.125em; }
span.TEX span.E{ position:relative;top:0.5ex;left:-0.0417em;}
a span.TEX span.E {text-decoration: none; }
span.LATEX span.A{ position:relative; top:-0.5ex; left:-0.4em; font-size:85%;}
span.LATEX span.TEX{ position:relative; left: -0.4em; }
div.float img, div.float .caption {text-align:center;}
div.figure img, div.figure .caption {text-align:center;}
.marginpar {width:20%; float:right; text-align:left; margin-left:auto; margin-top:0.5em; font-size:85%; text-decoration:underline;}
.marginpar p{margin-top:0.4em; margin-bottom:0.4em;}
.equation td{text-align:center; vertical-align:middle; }
td.eq-no{ width:5%; }
table.equation { width:100%; } 
div.math-display, div.par-math-display{text-align:center;}
math .texttt { font-family: monospace; }
math .textit { font-style: italic; }
math .textsl { font-style: oblique; }
math .textsf { font-family: sans-serif; }
math .textbf { font-weight: bold; }
.partToc a, .partToc, .likepartToc a, .likepartToc {line-height: 200%; font-weight:bold; font-size:110%;}
.chapterToc a, .chapterToc, .likechapterToc a, .likechapterToc, .appendixToc a, .appendixToc {line-height: 200%; font-weight:bold;}
.index-item, .index-subitem, .index-subsubitem {display:block}
.caption td.id{font-weight: bold; white-space: nowrap; }
table.caption {text-align:center;}
h1.partHead{text-align: center}
p.bibitem { text-indent: -2em; margin-left: 2em; margin-top:0.6em; margin-bottom:0.6em; }
p.bibitem-p { text-indent: 0em; margin-left: 2em; margin-top:0.6em; margin-bottom:0.6em; }
.paragraphHead, .likeparagraphHead { margin-top:2em; font-weight: bold;}
.subparagraphHead, .likesubparagraphHead { font-weight: bold;}
.quote {margin-bottom:0.25em; margin-top:0.25em; margin-left:1em; margin-right:1em; text-align:justify;}
.verse{white-space:nowrap; margin-left:2em}
div.maketitle {text-align:center;}
h2.titleHead{text-align:center;}
div.maketitle{ margin-bottom: 2em; }
div.author, div.date {text-align:center;}
div.thanks{text-align:left; margin-left:10%; font-size:85%; font-style:italic; }
div.author{white-space: nowrap;}
.quotation {margin-bottom:0.25em; margin-top:0.25em; margin-left:1em; }
h1.partHead{text-align: center}
.sectionToc, .likesectionToc {margin-left:2em;}
.subsectionToc, .likesubsectionToc {margin-left:4em;}
.subsubsectionToc, .likesubsubsectionToc {margin-left:6em;}
.frenchb-nbsp{font-size:75%;}
.frenchb-thinspace{font-size:75%;}
.figure img.graphics {margin-left:10%;}
/* end css.sty */

\title{Nappes parametrees}
\author{}
\date{}

\begin{document}
\maketitle

\textbf{Warning: \href{http://www.math.union.edu/locate/jsMath}{jsMath}
requires JavaScript to process the mathematics on this page.\\ If your
browser supports JavaScript, be sure it is enabled.}

\begin{center}\rule{3in}{0.4pt}\end{center}

{[}\href{coursse101.html}{next}{]}
{[}\hyperref[tailcoursse100.html]{tail}{]}
{[}\href{coursch20.html\#coursse100.html}{up}{]}

\subsubsection{19.1 Nappes paramétrées}

\paragraph{19.1.1 Notion de nappe paramétrée. Equivalence}

Définition~19.1.1 On appelle nappe paramétrée de classe \{C\}\^{}\{k\}
de E tout couple (D,F) d'un ouvert D de \{ℝ\}\^{}\{2\} et d'une
application F : D → E de classe \{C\}\^{}\{k\}, notée
(u,v)\textbackslash{}mathrel\{↦\}F(u,v)

Remarque~19.1.1 Vocabulaire associé. Soit Σ = (D,F) une nappe paramétrée
de classe \{C\}\^{}\{k\} de E

\begin{itemize}
\itemsep1pt\parskip0pt\parsep0pt
\item
  (i) On appelle point de la nappe Σ un couple
  (\{u\}\_\{0\},\{v\}\_\{0\}) ∈ D
\item
  (ii) On appelle image ou support de Σ la partie F(D) de E
\item
  (iii) On appelle multiplicité d'un point m de l'image de Σ le cardinal
  de l'ensemble \{F\}\^{}\{−1\}(\textbackslash{}\{m\textbackslash{}\})
  (éventuellement infinie)~; on dit qu'un point de l'image est simple
  s'il est de multiplicité 1, sinon on dit qu'il est multiple~; on dit
  que la nappe est simple si tout point de l'image est de multiplicité
  1, c'est-à-dire si F est injective.
\item
  (iv) On dit qu'un point (\{u\}\_\{0\},\{v\}\_\{0\}) de la nappe Σ est
  régulier si la famille (\{ ∂F \textbackslash{}over ∂u\}
  (\{u\}\_\{0\},\{v\}\_\{0\}),\{ ∂F \textbackslash{}over ∂v\}
  (\{u\}\_\{0\},\{v\}\_\{0\})) est libre~; on dit que la nappe est
  régulière si tout point de la nappe est régulier~; un point non
  régulier est dit singulier.
\end{itemize}

Exemple~19.1.1 Soit
(O,\textbackslash{}vec\{ı\},\textbackslash{}vec\{ȷ\},\textbackslash{}vec\{k\})
un repère affine, D un ouvert de \{ℝ\}\^{}\{2\} et f une application de
D dans ℝ. La nappe paramétrée F : (x,y)\textbackslash{}mathrel\{↦\}O +
x\textbackslash{}vec\{ı\} + y\textbackslash{}vec\{ȷ\} +
f(x,y)\textbackslash{}vec\{k\} sera appelée une nappe cartésienne. Pour
une telle nappe on a \{ ∂F \textbackslash{}over ∂x\}
(\{x\}\_\{0\},\{y\}\_\{0\}) = \textbackslash{}vec\{ı\} +\{ ∂f
\textbackslash{}over ∂x\}
(\{x\}\_\{0\},\{y\}\_\{0\})\textbackslash{}vec\{k\} et \{ ∂F
\textbackslash{}over ∂y\} (\{x\}\_\{0\},\{y\}\_\{0\}) =
\textbackslash{}vec\{ȷ\} +\{ ∂f \textbackslash{}over ∂x\}
(\{x\}\_\{0\},\{y\}\_\{0\})\textbackslash{}vec\{k\} qui sont évidemment
linéairement indépendants. Une nappe cartésienne est donc régulière.
Nous verrons un peu plus loin une réciproque partielle à ce résultat.

Définition~19.1.2 (D,F) et (Δ,G) deux nappes paramétrées de classe
\{C\}\^{}\{k\}. On dit que ces deux nappes sont
\{C\}\^{}\{k\}-équivalentes s'il existe un difféomorphisme θ : D → Δ de
classe \{C\}\^{}\{k\} tel que F = G ∘ θ.

Remarque~19.1.2 On dira qu'un tel difféomorphisme est un changement de
paramétrage admissible. L'étude des nappes paramétrées concerne
essentiellement l'étude des propriétés des arcs qui sont invariantes par
équivalence. L'application θ étant bijective on voit immédiatement que

Proposition~19.1.1 Soit (D,F) et (Δ,G) deux nappes paramétrées de classe
\{C\}\^{}\{k\} qui sont \{C\}\^{}\{k\}-équivalentes. Alors les deux
nappes ont la même image. Tous les points de l'image ont la même
multiplicité pour les deux nappes. En particulier un point de l'image
est simple pour l'un si et seulement si~il est simple pour l'autre.

Proposition~19.1.2 Soit (D,F) et (Δ,G) deux nappes paramétrées de classe
\{C\}\^{}\{k\} qui sont \{C\}\^{}\{k\}-équivalentes, θ : D → Δ un
difféomorphisme de classe \{C\}\^{}\{k\} tel que F = G ∘ θ. Alors
(\{u\}\_\{0\},\{v\}\_\{0\}) est un point régulier de (D,F) si et
seulement si~θ(\{u\}\_\{0\},\{v\}\_\{0\}) est un point régulier de
(J,G). En particulier, (D,F) est régulière si et seulement si~(J,G)
l'est.

Démonstration Supposons que F = G ∘ θ. Notons
(u,v)\textbackslash{}mathrel\{↦\}F(u,v) et
(λ,μ)\textbackslash{}mathrel\{↦\}G(λ,μ) les deux nappes paramétrées
équivalentes, et θ(u,v) = (\{θ\}\_\{1\}(u,v),\{θ\}\_\{2\}(u,v)) le
changement de paramétrage admissible. On a donc F(u,v) =
G(\{θ\}\_\{1\}(u,v),\{θ\}\_\{2\}(u,v)) d'où l'on déduit

\textbackslash{}begin\{eqnarray*\}\{ ∂F \textbackslash{}over ∂u\}
(\{u\}\_\{0\},\{v\}\_\{0\})\& =\&\{ ∂\{θ\}\_\{1\} \textbackslash{}over
∂u\} (\{u\}\_\{0\},\{v\}\_\{0\})\{ ∂G \textbackslash{}over ∂λ\}
(\{θ\}\_\{1\}(\{u\}\_\{0\},\{v\}\_\{0\}),\{θ\}\_\{2\}(\{u\}\_\{0\},\{v\}\_\{0\}))
\%\& \textbackslash{}\textbackslash{} \& \& +\{ ∂\{θ\}\_\{2\}
\textbackslash{}over ∂u\} (\{u\}\_\{0\},\{v\}\_\{0\})\{ ∂G
\textbackslash{}over ∂μ\}
(\{θ\}\_\{1\}(\{u\}\_\{0\},\{v\}\_\{0\}),\{θ\}\_\{2\}(\{u\}\_\{0\},\{v\}\_\{0\})),\%\&
\textbackslash{}\textbackslash{} \{ ∂F \textbackslash{}over ∂v\}
(\{u\}\_\{0\},\{v\}\_\{0\})\& =\&\{ ∂\{θ\}\_\{1\} \textbackslash{}over
∂v\} (\{u\}\_\{0\},\{v\}\_\{0\})\{ ∂G \textbackslash{}over ∂λ\}
(\{θ\}\_\{1\}(\{u\}\_\{0\},\{v\}\_\{0\}),\{θ\}\_\{2\}(\{u\}\_\{0\},\{v\}\_\{0\}))
\%\& \textbackslash{}\textbackslash{} \& \& +\{ ∂\{θ\}\_\{2\}
\textbackslash{}over ∂v\} (\{u\}\_\{0\},\{v\}\_\{0\})\{ ∂G
\textbackslash{}over ∂μ\}
(\{θ\}\_\{1\}(\{u\}\_\{0\},\{v\}\_\{0\}),\{θ\}\_\{2\}(\{u\}\_\{0\},\{v\}\_\{0\}))
\%\& \textbackslash{}\textbackslash{} \textbackslash{}end\{eqnarray*\}

Faisons le produit vectoriel en notant (\{λ\}\_\{0\},\{μ\}\_\{0\}) =
(\{θ\}\_\{1\}(\{u\}\_\{0\},\{v\}\_\{0\}),\{θ\}\_\{2\}(\{u\}\_\{0\},\{v\}\_\{0\}))~;
on a

\textbackslash{}begin\{eqnarray*\}\{ ∂F \textbackslash{}over ∂u\}
(\{u\}\_\{0\},\{v\}\_\{0\}) ∧\{ ∂F \textbackslash{}over ∂v\}
(\{u\}\_\{0\},\{v\}\_\{0\})\&\& \%\& \textbackslash{}\textbackslash{} \&
=\&\{ ∂\{θ\}\_\{1\} \textbackslash{}over ∂u\}
(\{u\}\_\{0\},\{v\}\_\{0\})\{ ∂\{θ\}\_\{2\} \textbackslash{}over ∂v\}
(\{u\}\_\{0\},\{v\}\_\{0\})\{ ∂G \textbackslash{}over ∂λ\}
(\{λ\}\_\{0\},\{μ\}\_\{0\}) ∧\{ ∂G \textbackslash{}over ∂μ\}
(\{λ\}\_\{0\},\{μ\}\_\{0\}) \%\& \textbackslash{}\textbackslash{} \& \&
+\{ ∂\{θ\}\_\{2\} \textbackslash{}over ∂u\}
(\{u\}\_\{0\},\{v\}\_\{0\})\{ ∂\{θ\}\_\{1\} \textbackslash{}over ∂v\}
(\{u\}\_\{0\},\{v\}\_\{0\})\{ ∂G \textbackslash{}over ∂μ\}
(\{λ\}\_\{0\},\{μ\}\_\{0\}) ∧\{ ∂G \textbackslash{}over ∂λ\}
(\{λ\}\_\{0\},\{μ\}\_\{0\}) \%\& \textbackslash{}\textbackslash{} \& =\&
\textbackslash{}left (\{ ∂\{θ\}\_\{1\} \textbackslash{}over ∂u\}
(\{u\}\_\{0\},\{v\}\_\{0\})\{ ∂\{θ\}\_\{2\} \textbackslash{}over ∂v\}
(\{u\}\_\{0\},\{v\}\_\{0\}) −\{ ∂\{θ\}\_\{2\} \textbackslash{}over ∂u\}
(\{u\}\_\{0\},\{v\}\_\{0\})\{ ∂\{θ\}\_\{1\} \textbackslash{}over ∂v\}
(\{u\}\_\{0\},\{v\}\_\{0\})\textbackslash{}right )\%\&
\textbackslash{}\textbackslash{} \& \&\{ ∂G \textbackslash{}over ∂λ\}
(\{λ\}\_\{0\},\{μ\}\_\{0\}) ∧\{ ∂G \textbackslash{}over ∂μ\}
(\{λ\}\_\{0\},\{μ\}\_\{0\}) \%\& \textbackslash{}\textbackslash{} \& =\&
\{j\}\_\{θ\}(\{u\}\_\{0\},\{v\}\_\{0\})\{ ∂G \textbackslash{}over ∂λ\}
(\{λ\}\_\{0\},\{μ\}\_\{0\}) ∧\{ ∂G \textbackslash{}over ∂μ\}
(\{λ\}\_\{0\},\{μ\}\_\{0\}) \%\& \textbackslash{}\textbackslash{}
\textbackslash{}end\{eqnarray*\}

où l'on désigne par \{j\}\_\{θ\}(u,v) le jacobien de θ au point (u,v).
Ce jacobien est non nul puisque θ est un difféomorphisme, et donc on a

\{ ∂F \textbackslash{}over ∂u\} (\{u\}\_\{0\},\{v\}\_\{0\}) ∧\{ ∂F
\textbackslash{}over ∂v\}
(\{u\}\_\{0\},\{v\}\_\{0\})\textbackslash{}mathrel\{≠\}0
\textbackslash{}mathrel\{⇔\}\{ ∂G \textbackslash{}over ∂λ\}
(\{λ\}\_\{0\},\{μ\}\_\{0\}) ∧\{ ∂G \textbackslash{}over ∂μ\}
(\{λ\}\_\{0\},\{μ\}\_\{0\})\textbackslash{}mathrel\{≠\}0

ce qui achève la démonstration.

\paragraph{19.1.2 Orientation}

Supposons que D est connexe. Le jacobien d'un difféomorphisme ne
s'annulant pas, il doit être de signe constant sur un connexe. Ceci
conduit à la définition suivante

Définition~19.1.3 Soit (D,F) et (Δ,G) deux nappes paramétrées de classe
\{C\}\^{}\{k\} qui sont \{C\}\^{}\{k\}-équivalentes, définies sur des
connexes, θ : D → Δ un difféomorphisme de classe \{C\}\^{}\{k\} tel que
F = G ∘ θ. On dit que (D,F) et (Δ,G) sont de même sens si θ possède un
jacobien positif, de sens contraire si θ est à jacobien négatif.

Remarque~19.1.3 Il peut se produire qu'il existe deux difféomorphismes
\{θ\}\_\{1\} et \{θ\}\_\{2\} tels que F = G ∘ \{θ\}\_\{1\} et F = G ∘
\{θ\}\_\{2\}, l'un étant à jacobien positif et l'autre à jacobien
négatif. Autrement dit deux nappes paramétrées peuvent être à la fois de
même sens et de sens contraire. On dit alors qu'une telle nappe
paramétrée n'est pas orientable. Un exemple typique est le ruban de
Moebius.

\paragraph{19.1.3 Plan tangent à une nappe paramétrée, vecteur normal}

Définition~19.1.4 Soit Γ = (D,F) une nappe paramétrée de classe
\{C\}\^{}\{k\} et (\{u\}\_\{0\},\{v\}\_\{0\}) ∈ D un point régulier de
Σ. On appelle plan tangent à Σ au point (\{u\}\_\{0\},\{v\}\_\{0\}) le
plan affine F(\{u\}\_\{0\},\{v\}\_\{0\}) +\textbackslash{}mathop\{
\textbackslash{}mathrm\{Vect\}\}(\{ ∂F \textbackslash{}over ∂u\}
(\{u\}\_\{0\},\{v\}\_\{0\}),\{ ∂F \textbackslash{}over ∂v\}
(\{u\}\_\{0\},\{v\}\_\{0\})).

Remarque~19.1.4 Le plan tangent en un point singulier n'est pas défini.

Définition~19.1.5 Soit Σ = (D,F) une nappe paramétrée de classe
\{C\}\^{}\{k\} et (\{u\}\_\{0\},\{v\}\_\{0\}) ∈ D un point régulier de
Σ. On appelle vecteur normal à Σ au point (\{u\}\_\{0\},\{v\}\_\{0\}) le
vecteur \{ ∂F \textbackslash{}over ∂u\} (\{u\}\_\{0\},\{v\}\_\{0\}) ∧\{
∂F \textbackslash{}over ∂v\} (\{u\}\_\{0\},\{v\}\_\{0\}). On appelle
normale à Σ au point (\{u\}\_\{0\},\{v\}\_\{0\}) la droite affine
F(\{u\}\_\{0\},\{v\}\_\{0\}) + ℝ\{ ∂F \textbackslash{}over ∂u\}
(\{u\}\_\{0\},\{v\}\_\{0\}) ∧\{ ∂F \textbackslash{}over ∂v\}
(\{u\}\_\{0\},\{v\}\_\{0\})

Remarque~19.1.5 Le plan tangent est donc le plan affine passant par le
point F(\{u\}\_\{0\},\{v\}\_\{0\}) et orthogonal au vecteur normal. On a
vu précédemment que si (D,F) et (Δ,G) sont deux nappes paramétrées de
classe \{C\}\^{}\{k\} qui sont \{C\}\^{}\{k\}-équivalentes et si θ : D →
Δ, (u,v)\textbackslash{}mathrel\{↦\}(λ,μ) = θ(u,v) est un
difféomorphisme de classe \{C\}\^{}\{k\} tel que F = G ∘ θ alors \{ ∂F
\textbackslash{}over ∂u\} (\{u\}\_\{0\},\{v\}\_\{0\}) ∧\{ ∂F
\textbackslash{}over ∂v\} (\{u\}\_\{0\},\{v\}\_\{0\}) =
\{j\}\_\{θ\}(\{u\}\_\{0\},\{v\}\_\{0\})\{ ∂G \textbackslash{}over ∂λ\}
(\{λ\}\_\{0\},\{μ\}\_\{0\}) ∧\{ ∂G \textbackslash{}over ∂μ\}
(\{λ\}\_\{0\},\{μ\}\_\{0\}). On en déduit que la direction du vecteur
normal est invariante par changement de paramétrage admissible, donc il
en est de même de la direction du plan tangent. Comme en plus les deux
plans tangents ont en commun le point F(\{u\}\_\{0\},\{v\}\_\{0\}) =
G(\{λ\}\_\{0\},\{μ\}\_\{0\}), ils sont nécessairement confondus. Il en
est bien entendu de même de la normale. D'où la proposition suivante

Proposition~19.1.3 La notion de plan tangent et de normale à une nappe
paramétrée est invariante par changement de paramétrage admissible. Soit
(D,F) et (Δ,G) deux nappes paramétrées de classe \{C\}\^{}\{k\} qui sont
\{C\}\^{}\{k\}-équivalentes, θ : D → Δ un difféomorphisme de classe
\{C\}\^{}\{k\} tel que F = G ∘ θ. Alors (\{u\}\_\{0\},\{v\}\_\{0\}) est
un point régulier de (D,F) si et seulement
si~θ(\{u\}\_\{0\},\{v\}\_\{0\}) est un point régulier de (Δ,G), et dans
ce cas le plan tangent (resp. la normale) à (D,f) au point
(\{u\}\_\{0\},\{v\}\_\{0\}) est égal au plan tangent (resp. à la
normale) à (Δ,G) au point θ(\{u\}\_\{0\},\{v\}\_\{0\}).

Remarque~19.1.6 Soit (D,F) une nappe paramétrée et
t\textbackslash{}mathrel\{↦\}(φ(t),ψ(t)) une application d'un intervalle
I de ℝ dans D. L'arc paramétré t\textbackslash{}mathrel\{↦\}F(φ(t),ψ(t))
est alors un arc dont l'image est contenue dans l'image de la nappe. On
dira qu'un tel arc est tracé sur la nappe. En un point régulier (sur
l'arc et sur la nappe), le vecteur tangent est le vecteur \{ d
\textbackslash{}over dt\} (F(φ(t),ψ(t))) = φ'(t)\{ ∂F
\textbackslash{}over ∂u\} (φ(t),ψ(t)) + ψ'(t)\{ ∂F \textbackslash{}over
∂v\} (φ(t),ψ(t)) et il est donc contenu dans le plan tangent. On en
déduit que la tangente à un arc tracé sur la nappe est contenue dans le
plan tangent à la nappe au point correspondant.

\paragraph{19.1.4 Points réguliers et nappes cartésiennes}

Le théorème suivant permet de ramener l'étude locale d'une nappe
paramétrée régulière à celle d'une nappe cartésienne.

Théorème~19.1.4 Soit Σ = (D,F) une nappe paramétrée de classe
\{C\}\^{}\{k\}, (\{u\}\_\{0\},\{v\}\_\{0\}) un point régulier de Σ et
(\textbackslash{}vec\{ı\},\textbackslash{}vec\{ȷ\},\textbackslash{}vec\{k\})
une base de \textbackslash{}vec\{E\} telle que \textbackslash{}vec\{k\}
ne soit pas tangent à la surface. Alors il existe un ouvert U ⊂ D et
contenant (\{u\}\_\{0\},\{v\}\_\{0\}) tel que la sous nappe \{Σ\}\_\{0\}
= (\{U\}\_\{0\},F) soit équivalente à une nappe cartésienne
(x,y)\textbackslash{}mathrel\{↦\}O + x\textbackslash{}vec\{ı\} +
y\textbackslash{}vec\{ȷ\} + f(x,y)\textbackslash{}vec\{k\}.

Démonstration Posons F(u,v) = O + φ(u,v)\textbackslash{}vec\{ı\} +
ψ(u,v)\textbackslash{}vec\{ȷ\} + ω(u,v)\textbackslash{}vec\{k\}. D'après
les hypothèses, la famille (\{ ∂F \textbackslash{}over ∂u\}
(\{u\}\_\{0\},\{v\}\_\{0\}),\{ ∂F \textbackslash{}over ∂v\}
(\{u\}\_\{0\},\{v\}\_\{0\}),\textbackslash{}vec\{k\}) est libre et donc
le déterminant

\textbackslash{}left
\textbar{}\textbackslash{}matrix\{\textbackslash{},\{ ∂φ
\textbackslash{}over ∂u\} (\{u\}\_\{0\},\{v\}\_\{0\})\&\{ ∂φ
\textbackslash{}over ∂v\} (\{u\}\_\{0\},\{v\}\_\{0\})\&0
\textbackslash{}cr \{ ∂ψ \textbackslash{}over ∂u\}
(\{u\}\_\{0\},\{v\}\_\{0\})\&\{ ∂ψ \textbackslash{}over ∂v\}
(\{u\}\_\{0\},\{v\}\_\{0\})\&0 \textbackslash{}cr \{ ∂ω
\textbackslash{}over ∂u\} (\{u\}\_\{0\},\{v\}\_\{0\})\&\{ ∂ω
\textbackslash{}over ∂v\}
(\{u\}\_\{0\},\{v\}\_\{0\})\&1\}\textbackslash{}right \textbar{}

est non nul, ce qui montre que \{ ∂φ \textbackslash{}over ∂u\}
(\{u\}\_\{0\},\{v\}\_\{0\})\{ ∂ψ \textbackslash{}over ∂v\}
(\{u\}\_\{0\},\{v\}\_\{0\}) −\{ ∂φ \textbackslash{}over ∂v\}
(\{u\}\_\{0\},\{v\}\_\{0\})\{ ∂ψ \textbackslash{}over ∂u\}
(\{u\}\_\{0\},\{v\}\_\{0\})\textbackslash{}mathrel\{≠\}0. Mais ceci
n'est autre que le jacobien au point (\{u\}\_\{0\},\{v\}\_\{0\}) de
l'application θ : D → \{ℝ\}\^{}\{2\},
(u,v)\textbackslash{}mathrel\{↦\}(φ(u,v),ψ(u,v)). En posant
(\{x\}\_\{0\},\{y\}\_\{0\}) = θ(\{u\}\_\{0\},\{v\}\_\{0\}) (ce sont
respectivement l'abscisse et l'ordonnée de
F(\{u\}\_\{0\},\{v\}\_\{0\})), le théorème d'inversion locale assure
qu'il existe \{U\}\_\{0\} ouvert contenant (\{x\}\_\{0\},\{y\}\_\{0\})
(que l'on peut bien entendu supposer inclus dans D) et \{V \}\_\{0\}
ouvert contenant (\{x\}\_\{0\},\{y\}\_\{0\}) tel que θ soit un
\{C\}\^{}\{k\} difféomorphisme de \{U\}\_\{0\} sur \{V \}\_\{0\}. On a
bien entendu θ(\{θ\}\^{}\{−1\}(x,y)) = (x,y), soit encore
φ(\{θ\}\^{}\{−1\}(x,y)) = x et ψ(\{θ\}\^{}\{−1\}(x,y)) = y. Posons alors
f(x,y) = ω(\{θ\}\^{}\{−1\}(x,y)). La nappe
(\{U\}\_\{0\},F\{\textbar{}\}\_\{\{U\}\_\{0\}\}) est équivalente à la
nappe (\{V \}\_\{0\},F ∘ \{θ\}\^{}\{−1\}\{\textbar{}\}\_\{\{V
\}\_\{0\}\}) avec

\textbackslash{}begin\{eqnarray*\} F ∘ \{θ\}\^{}\{−1\}(x,y)\& =\& O +
φ(\{θ\}\^{}\{−1\}(x,y))\textbackslash{}vec\{ı\} +
ψ(\{θ\}\^{}\{−1\}(x,y))\textbackslash{}vec\{ȷ\} +
ω(\{θ\}\^{}\{−1\}(x,y))\textbackslash{}vec\{k\}\%\&
\textbackslash{}\textbackslash{} \& =\& O + x\textbackslash{}vec\{ı\} +
y\textbackslash{}vec\{ȷ\} + f(x,y)\textbackslash{}vec\{k\} \%\&
\textbackslash{}\textbackslash{} \textbackslash{}end\{eqnarray*\}

ce qui démontre le résultat.

\paragraph{19.1.5 Intersection de nappes paramétrées}

Le théorème suivant assure que lorsque deux images de nappes paramétrées
se coupent franchement (c'est à dire de manière non tangentielle),
l'intersection des deux est localement l'image d'un arc paramétré
régulier.

Théorème~19.1.5 Soit \{Σ\}\_\{1\} = (\{D\}\_\{1\},\{F\}\_\{1\}) et
\{Σ\}\_\{2\} = (\{D\}\_\{2\},\{F\}\_\{2\}) deux nappes paramétrées
régulières dont les images ont un point en commun \{m\}\_\{0\} =
\{F\}\_\{1\}(\{u\}\_\{1\},\{v\}\_\{1\}) =
\{F\}\_\{2\}(\{u\}\_\{2\},\{v\}\_\{2\}). On suppose que les deux nappes
ne sont pas tangentes en ce point commun (c'est-à-dire que le plan
tangent à \{Σ\}\_\{1\} en (\{u\}\_\{1\},\{v\}\_\{1\}) est distinct du
plan tangent à \{Σ\}\_\{2\} en (\{u\}\_\{2\},\{v\}\_\{2\})). Alors
l'intersection des deux nappes est localement l'image d'un arc
paramétré, c'est-à-dire qu'il existe \{U\}\_\{1\} ouvert contenant
(\{u\}\_\{1\},\{v\}\_\{1\}) et \{U\}\_\{2\} ouvert contenant
(\{u\}\_\{2\},\{v\}\_\{2\}) tels que \{F\}\_\{1\}(\{U\}\_\{1\}) ∩
\{F\}\_\{2\}(\{U\}\_\{2\}) soit l'image d'un arc paramétré régulier.

Démonstration Choisissons un repère
(O,\textbackslash{}vec\{ı\},\textbackslash{}vec\{ȷ\},\textbackslash{}vec\{k\})
tel que \textbackslash{}vec\{k\} n'appartienne pas à la réunion des
directions des deux plans tangents en \{m\}\_\{0\} à \{Σ\}\_\{1\} et
\{Σ\}\_\{2\}. La propriété à démontrer ne concernant que les images,
elle est invariante par changement de paramétrage~; de plus c'est une
propriété locale puisqu'on a le choix des ouverts \{U\}\_\{1\} et
\{U\}\_\{2\}. Le théorème précédent nous permet de supposer que les deux
nappes sont des nappes cartésiennes \{Σ\}\_\{1\} :
(x,y)\textbackslash{}mathrel\{↦\}O + x\textbackslash{}vec\{ı\} +
y\textbackslash{}vec\{ȷ\} + \{f\}\_\{1\}(x,y)\textbackslash{}vec\{k\} et
\{Σ\}\_\{2\} : (x,y)\textbackslash{}mathrel\{↦\}O +
x\textbackslash{}vec\{ı\} + y\textbackslash{}vec\{ȷ\} +
\{f\}\_\{2\}(x,y)\textbackslash{}vec\{k\} (avec le même (x,y) qui
représente l'abscisse et l'ordonnée du point). On a alors \{m\}\_\{0\} =
O + \{x\}\_\{0\}\textbackslash{}vec\{ı\} +
\{y\}\_\{0\}\textbackslash{}vec\{ȷ\} +
\{f\}\_\{1\}(\{x\}\_\{0\},\{y\}\_\{0\})\textbackslash{}vec\{k\} = O +
\{x\}\_\{0\}\textbackslash{}vec\{ı\} +
\{y\}\_\{0\}\textbackslash{}vec\{ȷ\} +
\{f\}\_\{2\}(\{x\}\_\{0\},\{y\}\_\{0\})\textbackslash{}vec\{k\}, si bien
que \{f\}\_\{1\}(\{x\}\_\{0\},\{y\}\_\{0\}) =
\{f\}\_\{2\}(\{x\}\_\{0\},\{y\}\_\{0\}). L'intersection des images des
deux nappes est alors \textbackslash{}\{O + x\textbackslash{}vec\{ı\} +
y\textbackslash{}vec\{ȷ\} +
z\textbackslash{}vec\{k\}\textbackslash{}mathrel\{∣\}z =
\{f\}\_\{1\}(x,y) = \{f\}\_\{2\}(x,y)\textbackslash{}\}. En introduisant
une structure euclidienne qui rende la base
(\textbackslash{}vec\{ı\},\textbackslash{}vec\{ȷ\},\textbackslash{}vec\{k\})
orthonormée, le vecteur normal en \{m\}\_\{0\} à \{Σ\}\_\{1\} est le
vecteur

(\textbackslash{}vec\{i\} +\{ ∂\{f\}\_\{1\} \textbackslash{}over ∂x\}
(\{x\}\_\{0\},\{y\}\_\{0\})\textbackslash{}vec\{k\}) ∧
(\textbackslash{}vec\{j\} +\{ ∂\{f\}\_\{1\} \textbackslash{}over ∂y\}
(\{x\}\_\{0\},\{y\}\_\{0\})\textbackslash{}vec\{k\})

soit encore

−\{ ∂\{f\}\_\{1\} \textbackslash{}over ∂x\}
(\{x\}\_\{0\},\{y\}\_\{0\})\textbackslash{}vec\{ı\} −\{ ∂\{f\}\_\{1\}
\textbackslash{}over ∂y\}
(\{x\}\_\{0\},\{y\}\_\{0\})\textbackslash{}vec\{ȷ\} +
\textbackslash{}vec\{k\}

et de même le vecteur normal \{m\}\_\{0\} à \{Σ\}\_\{1\} est le vecteur
−\{ ∂\{f\}\_\{2\} \textbackslash{}over ∂x\}
(\{x\}\_\{0\},\{y\}\_\{0\})\textbackslash{}vec\{ı\} −\{ ∂\{f\}\_\{2\}
\textbackslash{}over ∂y\}
(\{x\}\_\{0\},\{y\}\_\{0\})\textbackslash{}vec\{ȷ\} +
\textbackslash{}vec\{k\}. Comme ces deux vecteurs doivent être
distincts, on peut supposer par exemple que \{ ∂\{f\}\_\{1\}
\textbackslash{}over ∂y\}
(\{x\}\_\{0\},\{y\}\_\{0\})\textbackslash{}mathrel\{≠\}\{ ∂\{f\}\_\{2\}
\textbackslash{}over ∂y\} (\{x\}\_\{0\},\{y\}\_\{0\}). Posons alors
g(x,y) = \{f\}\_\{1\}(x,y) − \{f\}\_\{2\}(x,y). On a donc
g(\{x\}\_\{0\},\{y\}\_\{0\}) = 0 et \{ ∂g \textbackslash{}over ∂y\}
(\{x\}\_\{0\},\{y\}\_\{0\})\textbackslash{}mathrel\{≠\}0. Le théorème
des fonctions implicites garantit qu'il existe \{I\}\_\{0\} intervalle
ouvert contenant \{x\}\_\{0\} et \{J\}\_\{0\} intervalle ouvert
contenant \{y\}\_\{0\} tel que, pour tout x ∈ \{I\}\_\{0\}, il existe un
unique y ∈ \{J\}\_\{0\} vérifiant g(x,y) = 0, autrement dit
\{f\}\_\{1\}(x,y) = \{f\}\_\{2\}(x,y). Si on note y = φ(x), alors
(quitte à restreindre \{I\}\_\{0\}), φ est de classe \{C\}\^{}\{k\}. On
en déduit que

\textbackslash{}begin\{eqnarray*\}\{ F\}\_\{1\}(\{I\}\_\{0\} ×
\{J\}\_\{0\}) ∩ \{F\}\_\{1\}(\{I\}\_\{0\} × \{J\}\_\{0\})\&\& \%\&
\textbackslash{}\textbackslash{} \& =\& \textbackslash{}\{O +
x\textbackslash{}vec\{ı\} + y\textbackslash{}vec\{ȷ\} +
z\textbackslash{}vec\{k\}\textbackslash{}mathrel\{∣\}x ∈ \{I\}\_\{0\}, y
∈ \{J\}\_\{0\}, z = \{f\}\_\{1\}(x,y) =
\{f\}\_\{2\}(x,y)\textbackslash{}\} \%\&
\textbackslash{}\textbackslash{} \& =\& \textbackslash{}\{O +
x\textbackslash{}vec\{ı\} + y\textbackslash{}vec\{ȷ\} +
z\textbackslash{}vec\{k\}\textbackslash{}mathrel\{∣\}x ∈ \{I\}\_\{0\}, y
∈ \{J\}\_\{0\}, g(x,y) = 0, z = \{f\}\_\{1\}(x,y)\textbackslash{}\}\%\&
\textbackslash{}\textbackslash{} \& =\& \textbackslash{}\{O +
x\textbackslash{}vec\{ı\} + y\textbackslash{}vec\{ȷ\} +
z\textbackslash{}vec\{k\}\textbackslash{}mathrel\{∣\}x ∈ \{I\}\_\{0\}, y
∈ \{J\}\_\{0\}, y = φ(x), z = \{f\}\_\{1\}(x,y)\textbackslash{}\} \%\&
\textbackslash{}\textbackslash{} \& =\& \textbackslash{}\{O +
x\textbackslash{}vec\{ı\} + φ(x)\textbackslash{}vec\{ȷ\} +
\{f\}\_\{1\}(x,φ(x))\textbackslash{}vec\{k\}\textbackslash{}mathrel\{∣\}x
∈ \{I\}\_\{0\}\textbackslash{}\} \%\& \textbackslash{}\textbackslash{}
\textbackslash{}end\{eqnarray*\}

ce qui montre que l'intersection est l'image de l'arc paramétré
x\textbackslash{}mathrel\{↦\}O + x\textbackslash{}vec\{ı\} +
φ(x)\textbackslash{}vec\{ȷ\} +
\{f\}\_\{1\}(x,φ(x))\textbackslash{}vec\{k\} (qui est bien entendu
régulier car le vecteur dérivée a 1 pour abscisse).

\paragraph{19.1.6 Intersection d'une nappe et de son plan tangent}

Le théorème d'intersection de deux nappes paramétrées régulières non
tangentes s'applique en particulier à une nappe et à un plan. Soit Σ =
(D,F) une nappe régulière de classe \{C\}\^{}\{k\}, \{m\}\_\{0\} =
F(\{u\}\_\{0\},\{v\}\_\{0\}) un point de l'image et Π un plan affine
contenant \{m\}\_\{0\}. Si Π n'est pas tangent à Σ en \{m\}\_\{0\},
l'intersection de Π et de la nappe est localement l'image d'un arc
paramétré régulier. Nous allons voir qu'il n'en est plus du tout de même
dans le cas où le plan Π est le plan tangent à la surface.

Soit
(0,\textbackslash{}vec\{ı\},\textbackslash{}vec\{ȷ\},\textbackslash{}vec\{k\})
un repère de E. Posons F(u,v) = O + φ(u,v)\textbackslash{}vec\{ı\} +
ψ(u,v)\textbackslash{}vec\{ȷ\} + ω(u,v)\textbackslash{}vec\{k\} et soit
(\{u\}\_\{0\},\{v\}\_\{0\}) ∈ D. Le plan tangent Π en
(\{u\}\_\{0\},\{v\}\_\{0\}) a pour équation

\textbackslash{}left
\textbar{}\textbackslash{}matrix\{\textbackslash{},x −
φ(\{u\}\_\{0\},\{v\}\_\{0\})\&\{ ∂φ \textbackslash{}over ∂u\}
(\{u\}\_\{0\},\{v\}\_\{0\})\&\{ ∂φ \textbackslash{}over ∂v\}
(\{u\}\_\{0\},\{v\}\_\{0\}) \textbackslash{}cr y −
ψ(\{u\}\_\{0\},\{v\}\_\{0\})\&\{ ∂ψ \textbackslash{}over ∂u\}
(\{u\}\_\{0\},\{v\}\_\{0\})\&\{ ∂ψ \textbackslash{}over ∂v\}
(\{u\}\_\{0\},\{v\}\_\{0\}) \textbackslash{}cr z −
ω(\{x\}\_\{0\},\{y\}\_\{0\})\&\{ ∂ω \textbackslash{}over ∂u\}
(\{u\}\_\{0\},\{v\}\_\{0\})\&\{ ∂ω \textbackslash{}over ∂v\}
(\{u\}\_\{0\},\{v\}\_\{0\})\}\textbackslash{}right \textbar{} = 0

et la position de F(u,v) par rapport à Π est donnée par le signe de la
fonction

Δ(u,v) = \textbackslash{}left
\textbar{}\textbackslash{}matrix\{\textbackslash{},φ(u,v) −
φ(\{u\}\_\{0\},\{v\}\_\{0\})\&\{ ∂φ \textbackslash{}over ∂u\}
(\{u\}\_\{0\},\{v\}\_\{0\})\&\{ ∂φ \textbackslash{}over ∂v\}
(\{u\}\_\{0\},\{v\}\_\{0\}) \textbackslash{}cr ψ(u,v) −
ψ(\{u\}\_\{0\},\{v\}\_\{0\})\&\{ ∂ψ \textbackslash{}over ∂u\}
(\{u\}\_\{0\},\{v\}\_\{0\})\&\{ ∂ψ \textbackslash{}over ∂v\}
(\{u\}\_\{0\},\{v\}\_\{0\}) \textbackslash{}cr ω(u,v) −
ω(\{x\}\_\{0\},\{y\}\_\{0\})\&\{ ∂ω \textbackslash{}over ∂u\}
(\{u\}\_\{0\},\{v\}\_\{0\})\&\{ ∂ω \textbackslash{}over ∂v\}
(\{u\}\_\{0\},\{v\}\_\{0\})\}\textbackslash{}right \textbar{}

Supposons que la nappe est de classe \{C\}\^{}\{2\}~; alors Δ est aussi
de classe \{C\}\^{}\{2\} et on a Δ(\{u\}\_\{0\},\{v\}\_\{0\}) = 0 (la
première colonne du déterminant est nulle), \{ ∂Δ \textbackslash{}over
∂u\} (\{u\}\_\{0\},\{v\}\_\{0\}) = 0 (la première et la deuxième colonne
du déterminant sont égales) et \{ ∂Δ \textbackslash{}over ∂v\}
(\{u\}\_\{0\},\{v\}\_\{0\}) = 0 (la première et la troisième colonne du
déterminant sont égales).

On peut donc appliquer à Δ la théorie des extremums de fonctions de deux
variables. En utilisant les notations de Monge \{r\}\_\{0\} =\{
\{∂\}\^{}\{2\}Δ \textbackslash{}over ∂\{u\}\^{}\{2\}\}
(\{u\}\_\{0\},\{v\}\_\{0\}), \{s\}\_\{0\} =\{ \{∂\}\^{}\{2\}Δ
\textbackslash{}over ∂u∂v\} (\{u\}\_\{0\},\{v\}\_\{0\}) et \{t\}\_\{0\}
=\{ \{∂\}\^{}\{2\}Δ \textbackslash{}over ∂\{v\}\^{}\{2\}\}
(\{u\}\_\{0\},\{v\}\_\{0\}), on a trois cas possibles

Premier cas~: \{s\}\_\{0\}\^{}\{2\} − \{r\}\_\{0\}\{t\}\_\{0\}
\textless{} 0~; alors on sait que la fonction Δ présente au point
(\{u\}\_\{0\},\{v\}\_\{0\}) un extremum local strict~; en particulier,
il existe \{U\}\_\{0\} ouvert contenant (\{u\}\_\{0\},\{v\}\_\{0\}) tel
que Δ soit de signe constant sur \{U\}\_\{0\}
∖\textbackslash{}\{(\{u\}\_\{0\},\{v\}\_\{0\})\textbackslash{}\}~; donc
localement, la nappe reste d'un même coté de son plan tangent et
l'intersection des deux est réduite au point \{m\}\_\{0\}~; on dit alors
que le point (\{u\}\_\{0\},\{v\}\_\{0\}) est un point elliptique de la
nappe.

Deuxième cas~: \{s\}\_\{0\}\^{}\{2\} − \{r\}\_\{0\}\{t\}\_\{0\}
\textgreater{} 0~; alors on sait que la fonction Δ ne présente pas au
point (\{u\}\_\{0\},\{v\}\_\{0\}) d'extremum local~; pour tout
\{U\}\_\{0\} ouvert contenant (\{u\}\_\{0\},\{v\}\_\{0\}), il existe des
points de \{U\}\_\{0\} où Δ est strictement positive et des points de
\{U\}\_\{0\} où Δ est strictement négative~; donc au voisinage de
\{m\}\_\{0\} la nappe a des points de part et d'autre de son plan
tangent~; on dit alors que le point (\{u\}\_\{0\},\{v\}\_\{0\}) est un
point hyperbolique de la nappe.

Troisième cas~: \{s\}\_\{0\}\^{}\{2\} − \{r\}\_\{0\}\{t\}\_\{0\} = 0~;
alors on ne sait pas étudier ainsi le signe de Δ~; on dit alors que le
point (\{u\}\_\{0\},\{v\}\_\{0\}) est un point parabolique de la nappe.

Remarque~19.1.7 La suite de cette section ne fait pas partie du
programme des classes préparatoires.

Une étude plus fine de la situation peut consister à étudier les lignes
de niveau de la nappe paramétrée dans la direction du plan tangent,
c'est-à-dire l'intersection de la nappe avec des plans parallèles au
plan tangent. Pour faire cette étude, on peut utiliser un repère
(O,\textbackslash{}vec\{ı\},\textbackslash{}vec\{ȷ\},\textbackslash{}vec\{k\})
tel que O = \{m\}\_\{0\}, \textbackslash{}vec\{ı\} et
\textbackslash{}vec\{ȷ\} sont dans le plan tangent et
\textbackslash{}vec\{k\} n'appartient pas au plan tangent. Alors
localement, la nappe est équivalente à une nappe cartésienne
(x,y)\textbackslash{}mathrel\{↦\}O + x\textbackslash{}vec\{ı\} +
y\textbackslash{}vec\{ȷ\} + f(x,y)\textbackslash{}vec\{k\}. Le fait que
\textbackslash{}vec\{ı\} et \textbackslash{}vec\{ȷ\} sont dans le plan
tangent va se traduire par \{ ∂f \textbackslash{}over ∂x\}
(\{x\}\_\{0\},\{y\}\_\{0\}) =\{ ∂f \textbackslash{}over ∂y\}
(\{x\}\_\{0\},\{y\}\_\{0\}) = 0. En utilisant les notations de Monge
\{r\}\_\{0\} =\{ \{∂\}\^{}\{2\}f \textbackslash{}over ∂\{x\}\^{}\{2\}\}
(\{x\}\_\{0\},\{y\}\_\{0\}), \{s\}\_\{0\} =\{ \{∂\}\^{}\{2\}f
\textbackslash{}over ∂x∂y\} (\{x\}\_\{0\},\{y\}\_\{0\}) et \{t\}\_\{0\}
=\{ \{∂\}\^{}\{2\}f \textbackslash{}over ∂\{y\}\^{}\{2\}\}
(\{x\}\_\{0\},\{y\}\_\{0\}), quitte à prendre une base de Sylvester dans
le plan
\textbackslash{}mathop\{\textbackslash{}mathrm\{Vect\}\}(\textbackslash{}vec\{ı\},\textbackslash{}vec\{ȷ\})
pour la forme quadratique différentielle seconde
\{r\}\_\{0\}\{x\}\^{}\{2\} + 2\{s\}\_\{0\}xy +
\{t\}\_\{0\}\{y\}\^{}\{2\}, on peut même supposer que \{s\}\_\{0\} = 0
et que \{r\}\_\{0\},\{t\}\_\{0\} ∈\textbackslash{}\{−
1,0,1\textbackslash{}\}. Le discriminant de cette forme quadratique
différentielle seconde est alors − \{r\}\_\{0\}\{t\}\_\{0\}. Supposons
que le point n'est pas parabolique. On a donc
\{r\}\_\{0\}\{t\}\_\{0\}\textbackslash{}mathrel\{≠\}0. Appelons
\{ε\}\_\{1\} le signe de \{r\}\_\{0\} et \{ε\}\_\{2\} le signe de
\{t\}\_\{0\}. On a~:

Lemme~19.1.6 (Morse). Sous ces hypothèses, il existe un ouvert
\{U\}\_\{0\} contenant (\{x\}\_\{0\},\{y\}\_\{0\}), un ouvert \{V
\}\_\{0\} contenant (0,0) et un difféomorphisme θ : \{V \}\_\{0\} →
\{U\}\_\{0\} tel que θ(0,0) = (\{x\}\_\{0\},\{y\}\_\{0\}) et

\textbackslash{}mathop\{∀\}(X,Y ) ∈ \{V \}\_\{0\}, f(θ(X,Y )) =
\{ε\}\_\{1\}\{X\}\^{}\{2\} + \{ε\}\_\{ 2\}\{Y \}\^{}\{2\}

Démonstration A une translation près, nous pouvons supposer que
\{x\}\_\{0\} = \{y\}\_\{0\} = 0. Appliquons alors la formule de Taylor
avec reste intégral à l'ordre 2. On a donc

\textbackslash{}begin\{eqnarray*\} f(x,y) = f(0,0) + x\{ ∂f
\textbackslash{}over ∂x\} (0,0) + y\{ ∂f \textbackslash{}over ∂y\}
(0,0)\&\& \%\& \textbackslash{}\textbackslash{} \& +\&
\{\textbackslash{}mathop\{∫ \} \}\_\{0\}\^{}\{1\}(1 −
t)\textbackslash{}left (\{x\}\^{}\{2\}\{ \{∂\}\^{}\{2\}f
\textbackslash{}over ∂\{x\}\^{}\{2\}\} (tx,ty) + 2xy\{ \{∂\}\^{}\{2\}f
\textbackslash{}over ∂x∂y\} (tx,ty) + \{y\}\^{}\{2\}\{ \{∂\}\^{}\{2\}f
\textbackslash{}over ∂\{y\}\^{}\{2\}\} (tx,ty)\textbackslash{}right )
dt\%\& \textbackslash{}\textbackslash{} \& =\& \{x\}\^{}\{2\}u(x,y) +
2xyv(x,y) + \{y\}\^{}\{2\}w(x,y) \%\& \textbackslash{}\textbackslash{}
\textbackslash{}end\{eqnarray*\}

compte tenu de f(0,0) =\{ ∂f \textbackslash{}over ∂x\}
(\{x\}\_\{0\},\{y\}\_\{0\}) =\{ ∂f \textbackslash{}over ∂y\}
(\{x\}\_\{0\},\{y\}\_\{0\}) = 0, avec des fonctions continues (théorème
sur les intégrales dépendant d'un paramètre)

\textbackslash{}begin\{eqnarray*\} u(x,y)\& =\&
\{\textbackslash{}mathop\{∫ \} \}\_\{0\}\^{}\{1\}(1 − t)\{
\{∂\}\^{}\{2\}f \textbackslash{}over ∂\{x\}\^{}\{2\}\} (tx,ty) dt \%\&
\textbackslash{}\textbackslash{} v(x,y)\& =\&
\{\textbackslash{}mathop\{∫ \} \}\_\{0\}\^{}\{1\}(1 − t)\{
\{∂\}\^{}\{2\}f \textbackslash{}over ∂x∂y\} (tx,ty) dt\%\&
\textbackslash{}\textbackslash{} w(x,y)\& =\&
\{\textbackslash{}mathop\{∫ \} \}\_\{0\}\^{}\{1\}(1 − t)\{
\{∂\}\^{}\{2\}f \textbackslash{}over ∂\{y\}\^{}\{2\}\} (tx,ty) dt \%\&
\textbackslash{}\textbackslash{} \textbackslash{}end\{eqnarray*\}

On a en particulier u(0,0) = \{r\}\_\{0\}, v(0,0) = \{s\}\_\{0\} = 0,
w(0,0) = \{t\}\_\{0\}. Posons D = \textbackslash{}left
(\textbackslash{}matrix\{\textbackslash{},\{r\}\_\{0\}\&\{s\}\_\{0\}
\textbackslash{}cr \{s\}\_\{0\}\&\{t\}\_\{0\}\}\textbackslash{}right ) =
\textbackslash{}left
(\textbackslash{}matrix\{\textbackslash{},\{r\}\_\{0\}\&0
\textbackslash{}cr 0 \&\{t\}\_\{0\}\}\textbackslash{}right ).
L'application T qui à une matrice triangulaire Y associe la matrice
\{\}\^{}\{t\}Y DY a pour différentielle au point Y l'application
H\{\textbackslash{}mathrel\{↦\}\}\^{}\{t\}HDY \{+ \}\^{}\{t\}Y DH et
donc dT(\textbackslash{}mathrm\{Id\}).H \{= \}\^{}\{t\}HD + DH. On
vérifie immédiatement que cette application linéaire
dT(\textbackslash{}mathrm\{Id\}) est bijective de l'espace vectoriel des
matrices triangulaires supérieures sur l'espace vectoriel des matrices
symétriques. On en déduit par le théorème d'inversion locale que T est
un difféomorphisme local d'un voisinage de l'identité (dans l'espace
vectoriel des matrices triangulaires) sur un voisinage W de
T(\textbackslash{}mathrm\{Id\}) = D dans l'espace vectoriel des matrices
symétriques. Or u,v et w sont continues en (0,0). Donc il existe
\{U\}\_\{1\} ouvert contenant (0,0) tel que pour (x,y) ∈ \{U\}\_\{1\} on
ait \textbackslash{}left
(\textbackslash{}matrix\{\textbackslash{},u(x,y)\&v(x,y)
\textbackslash{}cr v(x,y)\&w(x,y)\}\textbackslash{}right ) ∈ W. Posons
alors C(x,y) = \{T\}\^{}\{−1\}(\textbackslash{}left
(\textbackslash{}matrix\{\textbackslash{},u(x,y)\&v(x,y)
\textbackslash{}cr v(x,y)\&w(x,y)\}\textbackslash{}right ). Alors C(x,y)
est une matrice triangulaire, qui dépend de fa\textbackslash{}c\{c\}on
\{C\}\^{}\{1\} de (x,y) telle que

\textbackslash{}left
(\textbackslash{}matrix\{\textbackslash{},u(x,y)\&v(x,y)
\textbackslash{}cr v(x,y)\&w(x,y)\}\textbackslash{}right ) \{=
\}\^{}\{t\}C(x,y)DC(x,y)

et C(0,0) = \{T\}\^{}\{−1\}(D) = \textbackslash{}mathrm\{Id\}. On a
alors

\textbackslash{}begin\{eqnarray*\} f(x,y)\& =\& \{x\}\^{}\{2\}u(x,y) +
2xyv(x,y) + \{y\}\^{}\{2\}w(x,y)\%\& \textbackslash{}\textbackslash{} \&
=\& \textbackslash{}left
(\textbackslash{}matrix\{\textbackslash{},x\&y\}\textbackslash{}right
)\textbackslash{}left
(\textbackslash{}matrix\{\textbackslash{},u(x,y)\&v(x,y)
\textbackslash{}cr v(x,y)\&w(x,y)\}\textbackslash{}right
)\textbackslash{}left (\textbackslash{}matrix\{\textbackslash{},x
\textbackslash{}cr y\}\textbackslash{}right ) \%\&
\textbackslash{}\textbackslash{} \& =\&\{ \textbackslash{}left
(\textbackslash{}matrix\{\textbackslash{},x\&y \textbackslash{}cr
\}\textbackslash{}right )\}\^{}\{t\}C(x,y)DC(x,y)\textbackslash{}left
(\textbackslash{}matrix\{\textbackslash{},x \textbackslash{}cr
y\}\textbackslash{}right ) \%\& \textbackslash{}\textbackslash{} \{ \&
=\& \}\^{}\{t\}F(x,y)DF(x,y) \%\& \textbackslash{}\textbackslash{}
\textbackslash{}end\{eqnarray*\}

avec F(x,y) = C(x,y)\textbackslash{}left
(\textbackslash{}matrix\{\textbackslash{},x \textbackslash{}cr
y\}\textbackslash{}right ). On a alors \{ ∂F \textbackslash{}over ∂x\}
(x,y) =\{ ∂C \textbackslash{}over ∂x\} (x,y)\textbackslash{}left
(\textbackslash{}matrix\{\textbackslash{},x \textbackslash{}cr
y\}\textbackslash{}right ) + C(x,y)\textbackslash{}left
(\textbackslash{}matrix\{\textbackslash{},1 \textbackslash{}cr
0\}\textbackslash{}right ) et donc \{ ∂F \textbackslash{}over ∂x\} (0,0)
= C(0,0)\textbackslash{}left (\textbackslash{}matrix\{\textbackslash{},1
\textbackslash{}cr 0\}\textbackslash{}right ) = \textbackslash{}left
(\textbackslash{}matrix\{\textbackslash{},1 \textbackslash{}cr
0\}\textbackslash{}right ). De même \{ ∂F \textbackslash{}over ∂x\}
(0,0) = C(0,0)\textbackslash{}left
(\textbackslash{}matrix\{\textbackslash{},0 \textbackslash{}cr
1\}\textbackslash{}right ) = \textbackslash{}left
(\textbackslash{}matrix\{\textbackslash{},0 \textbackslash{}cr
1\}\textbackslash{}right ). Donc la différentielle de F en (0,0) est
l'identité de \{ℝ\}\^{}\{2\}. Une nouvelle application du théorème
d'inversion locale assure que F est un difféomorphisme d'un ouvert
\{U\}\_\{0\} contenant (0,0) sur un ouvert \{V \}\_\{0\} contenant
(0,0). Appelons θ le difféomorphisme réciproque. On a alors

\textbackslash{}begin\{eqnarray*\} f ∘ θ(X,Y )\{\& =\&
\}\^{}\{t\}F(θ(X,Y ))DF(θ(X,Y )) = \textbackslash{}left
(\textbackslash{}matrix\{\textbackslash{},X\&Y \textbackslash{}cr
\}\textbackslash{}right )D\textbackslash{}left
(\textbackslash{}matrix\{\textbackslash{},X\textbackslash{}cr
Y\}\textbackslash{}right )\%\& \textbackslash{}\textbackslash{} \& =\&
\{r\}\_\{0\}\{X\}\^{}\{2\} + \{t\}\_\{ 0\}\{Y \}\^{}\{2\} \%\&
\textbackslash{}\textbackslash{} \textbackslash{}end\{eqnarray*\}

Il suffit ensuite de changer \textbackslash{}sqrt\{\textbar{}\{r\}\_\{0
\} \textbar{}\}X en X' et \textbackslash{}sqrt\{
\textbar{}\{t\}\_\{0\}\textbar{}\}Y en Y ' pour obtenir le résultat
souhaité.

Les lignes de niveau de la nappe dans la direction du plan tangent sont
donc les courbes f(x,y) = k et elles sont localement difféomorphes aux
courbes \{ε\}\_\{1\}\{X\}\^{}\{2\} + \{ε\}\_\{2\}\{Y \}\^{}\{2\} au
voisinage de (0,0). Si le point (\{x\}\_\{0\},\{y\}\_\{0\}) est un point
elliptique, \{ε\}\_\{1\} et \{ε\}\_\{2\} sont de même signe (que l'on
peut supposer par exemple positif). On voit que l'intersection est vide
pour k \textless{} 0, réduite à un point pour k = 0 (c'est le plan
tangent lui même), difféomorphe à une ellipse pour k \textgreater{} 0
(et suffisamment petit). Par contre, si (\{x\}\_\{0\},\{y\}\_\{0\}) est
un point hyperbolique, \{ε\}\_\{1\} et \{ε\}\_\{2\} sont de signe
contraire et l'intersection est difféomorphe à une hyperbole pour
k\textbackslash{}mathrel\{≠\}0 (suffisamment petit) et à la réunion de
deux droites pour k = 0~; l'intersection avec le plan tangent est donc
localement la réunion de deux courbes passant par le point de contact~;
ces deux courbes séparent les lignes de niveau correspondant aux k
\textgreater{} 0 des lignes de niveau correspondant aux k \textless{} 0.
Voici des exemples de lignes de niveau au voisinage de points
elliptiques, hyperboliques ou paraboliques.

{[}\href{coursse101.html}{next}{]} {[}\href{coursse100.html}{front}{]}
{[}\href{coursch20.html\#coursse100.html}{up}{]}

\end{document}

% \documentclass[]{article}
\usepackage[T1]{fontenc}
\usepackage{lmodern}
\usepackage{amssymb,amsmath}
\usepackage{ifxetex,ifluatex}
\usepackage{fixltx2e} % provides \textsubscript
% use upquote if available, for straight quotes in verbatim environments
\IfFileExists{upquote.sty}{\usepackage{upquote}}{}
\ifnum 0\ifxetex 1\fi\ifluatex 1\fi=0 % if pdftex
  \usepackage[utf8]{inputenc}
\else % if luatex or xelatex
  \ifxetex
    \usepackage{mathspec}
    \usepackage{xltxtra,xunicode}
  \else
    \usepackage{fontspec}
  \fi
  \defaultfontfeatures{Mapping=tex-text,Scale=MatchLowercase}
  \newcommand{\euro}{€}
\fi
% use microtype if available
\IfFileExists{microtype.sty}{\usepackage{microtype}}{}
\ifxetex
  \usepackage[setpagesize=false, % page size defined by xetex
              unicode=false, % unicode breaks when used with xetex
              xetex]{hyperref}
\else
  \usepackage[unicode=true]{hyperref}
\fi
\hypersetup{breaklinks=true,
            bookmarks=true,
            pdfauthor={},
            pdftitle={Nappes reglees},
            colorlinks=true,
            citecolor=blue,
            urlcolor=blue,
            linkcolor=magenta,
            pdfborder={0 0 0}}
\urlstyle{same}  % don't use monospace font for urls
\setlength{\parindent}{0pt}
\setlength{\parskip}{6pt plus 2pt minus 1pt}
\setlength{\emergencystretch}{3em}  % prevent overfull lines
\setcounter{secnumdepth}{0}
 
/* start css.sty */
.cmr-5{font-size:50%;}
.cmr-7{font-size:70%;}
.cmmi-5{font-size:50%;font-style: italic;}
.cmmi-7{font-size:70%;font-style: italic;}
.cmmi-10{font-style: italic;}
.cmsy-5{font-size:50%;}
.cmsy-7{font-size:70%;}
.cmex-7{font-size:70%;}
.cmex-7x-x-71{font-size:49%;}
.msbm-7{font-size:70%;}
.cmtt-10{font-family: monospace;}
.cmti-10{ font-style: italic;}
.cmbx-10{ font-weight: bold;}
.cmr-17x-x-120{font-size:204%;}
.cmsl-10{font-style: oblique;}
.cmti-7x-x-71{font-size:49%; font-style: italic;}
.cmbxti-10{ font-weight: bold; font-style: italic;}
p.noindent { text-indent: 0em }
td p.noindent { text-indent: 0em; margin-top:0em; }
p.nopar { text-indent: 0em; }
p.indent{ text-indent: 1.5em }
@media print {div.crosslinks {visibility:hidden;}}
a img { border-top: 0; border-left: 0; border-right: 0; }
center { margin-top:1em; margin-bottom:1em; }
td center { margin-top:0em; margin-bottom:0em; }
.Canvas { position:relative; }
li p.indent { text-indent: 0em }
.enumerate1 {list-style-type:decimal;}
.enumerate2 {list-style-type:lower-alpha;}
.enumerate3 {list-style-type:lower-roman;}
.enumerate4 {list-style-type:upper-alpha;}
div.newtheorem { margin-bottom: 2em; margin-top: 2em;}
.obeylines-h,.obeylines-v {white-space: nowrap; }
div.obeylines-v p { margin-top:0; margin-bottom:0; }
.overline{ text-decoration:overline; }
.overline img{ border-top: 1px solid black; }
td.displaylines {text-align:center; white-space:nowrap;}
.centerline {text-align:center;}
.rightline {text-align:right;}
div.verbatim {font-family: monospace; white-space: nowrap; text-align:left; clear:both; }
.fbox {padding-left:3.0pt; padding-right:3.0pt; text-indent:0pt; border:solid black 0.4pt; }
div.fbox {display:table}
div.center div.fbox {text-align:center; clear:both; padding-left:3.0pt; padding-right:3.0pt; text-indent:0pt; border:solid black 0.4pt; }
div.minipage{width:100%;}
div.center, div.center div.center {text-align: center; margin-left:1em; margin-right:1em;}
div.center div {text-align: left;}
div.flushright, div.flushright div.flushright {text-align: right;}
div.flushright div {text-align: left;}
div.flushleft {text-align: left;}
.underline{ text-decoration:underline; }
.underline img{ border-bottom: 1px solid black; margin-bottom:1pt; }
.framebox-c, .framebox-l, .framebox-r { padding-left:3.0pt; padding-right:3.0pt; text-indent:0pt; border:solid black 0.4pt; }
.framebox-c {text-align:center;}
.framebox-l {text-align:left;}
.framebox-r {text-align:right;}
span.thank-mark{ vertical-align: super }
span.footnote-mark sup.textsuperscript, span.footnote-mark a sup.textsuperscript{ font-size:80%; }
div.tabular, div.center div.tabular {text-align: center; margin-top:0.5em; margin-bottom:0.5em; }
table.tabular td p{margin-top:0em;}
table.tabular {margin-left: auto; margin-right: auto;}
div.td00{ margin-left:0pt; margin-right:0pt; }
div.td01{ margin-left:0pt; margin-right:5pt; }
div.td10{ margin-left:5pt; margin-right:0pt; }
div.td11{ margin-left:5pt; margin-right:5pt; }
table[rules] {border-left:solid black 0.4pt; border-right:solid black 0.4pt; }
td.td00{ padding-left:0pt; padding-right:0pt; }
td.td01{ padding-left:0pt; padding-right:5pt; }
td.td10{ padding-left:5pt; padding-right:0pt; }
td.td11{ padding-left:5pt; padding-right:5pt; }
table[rules] {border-left:solid black 0.4pt; border-right:solid black 0.4pt; }
.hline hr, .cline hr{ height : 1px; margin:0px; }
.tabbing-right {text-align:right;}
span.TEX {letter-spacing: -0.125em; }
span.TEX span.E{ position:relative;top:0.5ex;left:-0.0417em;}
a span.TEX span.E {text-decoration: none; }
span.LATEX span.A{ position:relative; top:-0.5ex; left:-0.4em; font-size:85%;}
span.LATEX span.TEX{ position:relative; left: -0.4em; }
div.float img, div.float .caption {text-align:center;}
div.figure img, div.figure .caption {text-align:center;}
.marginpar {width:20%; float:right; text-align:left; margin-left:auto; margin-top:0.5em; font-size:85%; text-decoration:underline;}
.marginpar p{margin-top:0.4em; margin-bottom:0.4em;}
.equation td{text-align:center; vertical-align:middle; }
td.eq-no{ width:5%; }
table.equation { width:100%; } 
div.math-display, div.par-math-display{text-align:center;}
math .texttt { font-family: monospace; }
math .textit { font-style: italic; }
math .textsl { font-style: oblique; }
math .textsf { font-family: sans-serif; }
math .textbf { font-weight: bold; }
.partToc a, .partToc, .likepartToc a, .likepartToc {line-height: 200%; font-weight:bold; font-size:110%;}
.chapterToc a, .chapterToc, .likechapterToc a, .likechapterToc, .appendixToc a, .appendixToc {line-height: 200%; font-weight:bold;}
.index-item, .index-subitem, .index-subsubitem {display:block}
.caption td.id{font-weight: bold; white-space: nowrap; }
table.caption {text-align:center;}
h1.partHead{text-align: center}
p.bibitem { text-indent: -2em; margin-left: 2em; margin-top:0.6em; margin-bottom:0.6em; }
p.bibitem-p { text-indent: 0em; margin-left: 2em; margin-top:0.6em; margin-bottom:0.6em; }
.subsectionHead, .likesubsectionHead { margin-top:2em; font-weight: bold;}
.sectionHead, .likesectionHead { font-weight: bold;}
.quote {margin-bottom:0.25em; margin-top:0.25em; margin-left:1em; margin-right:1em; text-align:justify;}
.verse{white-space:nowrap; margin-left:2em}
div.maketitle {text-align:center;}
h2.titleHead{text-align:center;}
div.maketitle{ margin-bottom: 2em; }
div.author, div.date {text-align:center;}
div.thanks{text-align:left; margin-left:10%; font-size:85%; font-style:italic; }
div.author{white-space: nowrap;}
.quotation {margin-bottom:0.25em; margin-top:0.25em; margin-left:1em; }
h1.partHead{text-align: center}
.sectionToc, .likesectionToc {margin-left:2em;}
.subsectionToc, .likesubsectionToc {margin-left:4em;}
.sectionToc, .likesectionToc {margin-left:6em;}
.frenchb-nbsp{font-size:75%;}
.frenchb-thinspace{font-size:75%;}
.figure img.graphics {margin-left:10%;}
/* end css.sty */

\title{Nappes reglees}
\author{}
\date{}

\begin{document}
\maketitle

\textbf{Warning: 
requires JavaScript to process the mathematics on this page.\\ If your
browser supports JavaScript, be sure it is enabled.}

\begin{center}\rule{3in}{0.4pt}\end{center}

[
[
[]
[

\section{19.2 Nappes réglées}

Remarque~19.2.1 Cette notion n'est pas au programme des classes
préparatoires.

\subsection{19.2.1 Notion de nappe réglée}

Soit I un intervalle de \mathbb{R}~ et soit (D_u)_u\inI une
famille de droites indexée par u. Donnons nous pour chaque u \in I un
point f(u) de D_u et un vecteur directeur
\vecg(u) \in\overrightarrow
D_u \diagdown\0\ et supposons que
(I,f) et (I,\vecg) soient de classe \mathcal{C}^1.
La réunion des droites D_u est alors paramétrée par F : I \times \mathbb{R}~ \rightarrow~
E, (u,v)\mapsto~f(u) + v\vecg(u).
Nous allons montrer qu'à équivalence près, cette nappe paramétrée ne
dépend pas du choix de (I,f) et de (I,\vecg).

Supposons tout d'abord que nous changeons l'arc paramétré (I,f) en
(I,f_1)~; on a alors f_1(u) = f(u) +
\phi(u)\vecg(u) et on vérifie facilement (par exemple à
l'aide d'une structure euclidienne) que \phi est de classe \mathcal{C}^1.
Mais alors l'application \theta : (u,v)\mapsto~(u,v +
\phi(u)) est de classe \mathcal{C}^1 et sa réciproque
(u,w)\mapsto~(u,w - \phi(u)) est aussi de classe
\mathcal{C}^1. Donc \theta est un difféomorphisme de I \times \mathbb{R}~ sur lui même et
on a F \cdot \theta(u,v) = F(u,v + \phi(u)) = f(u) + (v +
\alpha~(u))\vecg(u) = f_1(u) +
v\vecg(u) = F_1(u,v) ce qui montre bien que
les deux nappes sont effectivement équivalentes.

Supposons maintenant que nous changeons (I,\vecg) en
(I,\vecg_1)~; on a alors
\vecg_1(u) = \psi(u)\vecg(u)
avec une application \psi de classe \mathcal{C}^1 qui ne s'annule pas.
Mais alors l'application \theta : (u,v)\mapsto~(u,\psi(u)v)
est de classe \mathcal{C}^1 et sa réciproque
(u,w)\mapsto~(u, w \over \psi(u) )
est aussi de classe \mathcal{C}^1. Donc \theta est un difféomorphisme de I \times
\mathbb{R}~ sur lui même et on a F \cdot \theta(u,v) = F(u,\psi(u)v) = f(u) +
\beta~(u)v\vecg(u) = f(u) +
v\vecg_1(u) = F_1(u,v) ce qui
montre que les deux nappes sont bien équivalentes.

Définition~19.2.1 Une telle nappe sera appelée une nappe réglée de
classe \mathcal{C}^1. Les droites D_u sont appelées les
génératrices de la nappe. Un arc (I,f) de classe \mathcal{C}^1 tel que
\forall~u \in I, f(u) \in D_u~ sera appelé une
directrice de la nappe~; une directrice plane est appelée une base de la
nappe.

\subsection{19.2.2 Plan tangent à une nappe réglée}

Donnons nous une nappe réglée de classe \mathcal{C}^1,
(D_u)_u\inI et soit F(u,v) = f(u) +
v\vecg(u) un paramétrage admissible de cette nappe.
Soit u_0 \in I. On a alors  \partial~F \over \partial~u
(u_0,v) = f'(u_0) +
v\vecg'(u_0) et  \partial~F \over
\partial~v (u_0,v) =\vec g(u_0). Un
vecteur normal à la nappe est alors le vecteur  \partial~F
\over \partial~u (u_0,v) ∧ \partial~F \over
\partial~v (u_0,v) = f'(u_0) ∧\vec
g(u_0) + v\vecg'(u_0)
∧\vec g(u_0).

Trois cas sont alors possibles~:

\begin{itemize}
\itemsep1pt\parskip0pt\parsep0pt
\item
  Premier cas La famille
  (f'(u_0),\vecg(u_0),\vecg'(u_0))
  est libre. Alors, pout tout v \in \mathbb{R}~, la famille ( \partial~F
  \over \partial~u (u_0,v), \partial~F \over
  \partial~v (u_0,v)) est libre. Lorsque v varie, le vecteur normal
  tourne dans le plan
  \mathrmVect(f'(u_0~)
  ∧\vec
  g(u_0),\vecg'(u_0)
  ∧\vec g(u_0)) en occupant toutes les
  directions sauf une, \mathbb{R}~g'(u_0) ∧\vec
  g(u_0), qui est la direction limite lorsque v tend vers
  ±\infty~~; autrement dit, lorsque le point se déplace sur la génératrice, le
  plan tangent tourne autour de celle-ci (il doit forcément la contenir
  puisque c'est une courbe tracée sur la surface et qu'elle est sa
  propre tangente) en occupant toutes les positions sauf une, la
  position limite à l'infini.
\item
  Deuxième cas
  \mathrmrg(f'(u_0),\vecg(u_0),\vecg'(u_0~))
  = 2. Alors le plan tangent, lorsqu'il existe, doit contenir la
  génératrice et être parallèle au plan
  \mathrmVect(f'(u_0),\vecg(u_0),\vecg'(u_0~))~;
  il doit être constant le long de la génératrice
  D_u_0. D'autre part, les deux vecteurs
  f'(u_0) ∧\vec g(u_0) et
  \vecg'(u_0) ∧\vec
  g(u_0) ne peuvent pas être tous deux nuls, et donc il
  existe au plus un v tel que  \partial~F \over \partial~u
  (u_0,v) ∧ \partial~F \over \partial~v (u_0,v) =
  f'(u_0) ∧\vec g(u_0) +
  v\vecg'(u_0) ∧\vec
  g(u_0) = 0, autrement dit il y a au plus un point singulier
  sur la génératrice~; en ce point le plan tangent n'existe pas.
\item
  Troisième cas
  \mathrmrg(f'(u_0),\vecg(u_0),\vecg'(u_0~))
  = 1. Alors, pour tout v \in \mathbb{R}~, on a  \partial~F \over \partial~u
  (u_0,v) ∧ \partial~F \over \partial~v (u_0,v) =
  f'(u_0) ∧\vec g(u_0) +
  v\vecg'(u_0) ∧\vec
  g(u_0) = 0. Tout point de la génératrice est singulier et
  il n'y a de plan tangent en aucun point de la génératrice. On dit que
  la génératrice D_u_0 est une génératrice singulière.
\end{itemize}

\subsection{19.2.3 Nappes cylindriques. Nappes coniques}

Définition~19.2.2 Soit \vecD une direction de droite
dans E. On appelle nappe cylindrique de direction
\vecD toute nappe réglée de classe \mathcal{C}^1,
(D_u)_u\inI telle que toutes les droites D_u
soient parallèles à \vecD.

Soit \veck un vecteur directeur de
\vecD. On peut donc choisir un paramétrage F(u,v) =
f(u) + v\vecg(u) avec \forall~~u \in
I, \vecg(u) =\vec k. On a alors
\vecg constante et donc \vecg'(u)
= 0. On voit donc que le premier cas de l'étude du plan tangent est
exclu et qu'il n'y a que deux possibilités pour un u_0 \in I.

Premier cas
f'(u_0)∉\vecD.
Alors  \partial~F \over \partial~u (u_0,v) ∧ \partial~F
\over \partial~v (u_0,v) = f'(u_0)
∧\vec k. Tout point de la génératrice est régulier et
le plan tangent est constant le long de la génératrice.

Deuxième cas f'(u_0) \in\vec D. Alors la
génératrice est singulière et le plan tangent n'existe en aucun point de
la génératrice.

Définition~19.2.3 Soit S un point de E. On appelle nappe conique de
sommet S toute nappe réglée de classe \mathcal{C}^1,
(D_u)_u\inI telle que toutes les droites D_u
passent par le point S.

On peut alors choisir un paramétrage F(u,v) = f(u) +
v\vecg(u) avec \forall~~u \in I, f(u)
= S. Donc f est constante et f'(u) = 0. On voit donc que le premier cas
de l'étude du plan tangent est exclu et qu'il n'y a que deux
possibilités pour un u_0 \in I.

Premier cas
(\vecg(u_0),\vecg'(u_0))
est libre. Alors  \partial~F \over \partial~u (u_0,v) ∧ \partial~F
\over \partial~v (u_0,v) =
v\vecg'(u_0),\vecg(u_0).
Tout point de la génératrice différent du sommet est régulier et le plan
tangent est constant le long de la génératrice.

Deuxième cas
(\vecg(u_0),\vecg'(u_0))
est liée. Alors la génératrice est singulière et le plan tangent
n'existe en aucun point de la génératrice.

Remarque~19.2.2 Les deux types de nappes réglées que nous venons
d'étudier vérifient la propriété remarquable que le plan tangent est
constant le long de chaque génératrice~; on appelle de telles nappes
réglées des nappes développables. Un autre type de nappes développables
peut être construit en prenant l'ensemble des tangentes à une courbe
gauche régulière. Soit (I,f) un tel arc paramétré régulier. On peut
paramétrer la tangente D_u par
v\mapsto~f(u) + vf'(u)~; on peut donc prendre
\vecg(u) = f'(u). La famille
(f'(u_0),\vecg(u_0),\vecg'(u_0))
est donc la famille (f'(u_0),f''(u_0)). Elle est donc
de rang au plus 2. On a deux possibilités~: soit u_0 est un
point birégulier de (I,f), alors la famille est de rang 2 et le plan
tangent est donc constant le long de la génératrice (dont le seul point
singulier est d'ailleurs v = 0, c'est-à-dire le point de contact de la
tangente), soit u_0 n'est pas birégulier et la génératrice est
singulière. En fait on peut montrer que ces trois types de nappes
(nappes cylindriques, nappes coniques et ensemble des tangentes à une
courbe) épuisent, au moins localement, les nappes développables.

[
[
[
[

\end{document}

% \documentclass[]{article}
\usepackage[T1]{fontenc}
\usepackage{lmodern}
\usepackage{amssymb,amsmath}
\usepackage{ifxetex,ifluatex}
\usepackage{fixltx2e} % provides \textsubscript
% use upquote if available, for straight quotes in verbatim environments
\IfFileExists{upquote.sty}{\usepackage{upquote}}{}
\ifnum 0\ifxetex 1\fi\ifluatex 1\fi=0 % if pdftex
  \usepackage[utf8]{inputenc}
\else % if luatex or xelatex
  \ifxetex
    \usepackage{mathspec}
    \usepackage{xltxtra,xunicode}
  \else
    \usepackage{fontspec}
  \fi
  \defaultfontfeatures{Mapping=tex-text,Scale=MatchLowercase}
  \newcommand{\euro}{€}
\fi
% use microtype if available
\IfFileExists{microtype.sty}{\usepackage{microtype}}{}
\ifxetex
  \usepackage[setpagesize=false, % page size defined by xetex
              unicode=false, % unicode breaks when used with xetex
              xetex]{hyperref}
\else
  \usepackage[unicode=true]{hyperref}
\fi
\hypersetup{breaklinks=true,
            bookmarks=true,
            pdfauthor={},
            pdftitle={Equations de surfaces},
            colorlinks=true,
            citecolor=blue,
            urlcolor=blue,
            linkcolor=magenta,
            pdfborder={0 0 0}}
\urlstyle{same}  % don't use monospace font for urls
\setlength{\parindent}{0pt}
\setlength{\parskip}{6pt plus 2pt minus 1pt}
\setlength{\emergencystretch}{3em}  % prevent overfull lines
\setcounter{secnumdepth}{0}
 
/* start css.sty */
.cmr-5{font-size:50%;}
.cmr-7{font-size:70%;}
.cmmi-5{font-size:50%;font-style: italic;}
.cmmi-7{font-size:70%;font-style: italic;}
.cmmi-10{font-style: italic;}
.cmsy-5{font-size:50%;}
.cmsy-7{font-size:70%;}
.cmex-7{font-size:70%;}
.cmex-7x-x-71{font-size:49%;}
.msbm-7{font-size:70%;}
.cmtt-10{font-family: monospace;}
.cmti-10{ font-style: italic;}
.cmbx-10{ font-weight: bold;}
.cmr-17x-x-120{font-size:204%;}
.cmsl-10{font-style: oblique;}
.cmti-7x-x-71{font-size:49%; font-style: italic;}
.cmbxti-10{ font-weight: bold; font-style: italic;}
p.noindent { text-indent: 0em }
td p.noindent { text-indent: 0em; margin-top:0em; }
p.nopar { text-indent: 0em; }
p.indent{ text-indent: 1.5em }
@media print {div.crosslinks {visibility:hidden;}}
a img { border-top: 0; border-left: 0; border-right: 0; }
center { margin-top:1em; margin-bottom:1em; }
td center { margin-top:0em; margin-bottom:0em; }
.Canvas { position:relative; }
li p.indent { text-indent: 0em }
.enumerate1 {list-style-type:decimal;}
.enumerate2 {list-style-type:lower-alpha;}
.enumerate3 {list-style-type:lower-roman;}
.enumerate4 {list-style-type:upper-alpha;}
div.newtheorem { margin-bottom: 2em; margin-top: 2em;}
.obeylines-h,.obeylines-v {white-space: nowrap; }
div.obeylines-v p { margin-top:0; margin-bottom:0; }
.overline{ text-decoration:overline; }
.overline img{ border-top: 1px solid black; }
td.displaylines {text-align:center; white-space:nowrap;}
.centerline {text-align:center;}
.rightline {text-align:right;}
div.verbatim {font-family: monospace; white-space: nowrap; text-align:left; clear:both; }
.fbox {padding-left:3.0pt; padding-right:3.0pt; text-indent:0pt; border:solid black 0.4pt; }
div.fbox {display:table}
div.center div.fbox {text-align:center; clear:both; padding-left:3.0pt; padding-right:3.0pt; text-indent:0pt; border:solid black 0.4pt; }
div.minipage{width:100%;}
div.center, div.center div.center {text-align: center; margin-left:1em; margin-right:1em;}
div.center div {text-align: left;}
div.flushright, div.flushright div.flushright {text-align: right;}
div.flushright div {text-align: left;}
div.flushleft {text-align: left;}
.underline{ text-decoration:underline; }
.underline img{ border-bottom: 1px solid black; margin-bottom:1pt; }
.framebox-c, .framebox-l, .framebox-r { padding-left:3.0pt; padding-right:3.0pt; text-indent:0pt; border:solid black 0.4pt; }
.framebox-c {text-align:center;}
.framebox-l {text-align:left;}
.framebox-r {text-align:right;}
span.thank-mark{ vertical-align: super }
span.footnote-mark sup.textsuperscript, span.footnote-mark a sup.textsuperscript{ font-size:80%; }
div.tabular, div.center div.tabular {text-align: center; margin-top:0.5em; margin-bottom:0.5em; }
table.tabular td p{margin-top:0em;}
table.tabular {margin-left: auto; margin-right: auto;}
div.td00{ margin-left:0pt; margin-right:0pt; }
div.td01{ margin-left:0pt; margin-right:5pt; }
div.td10{ margin-left:5pt; margin-right:0pt; }
div.td11{ margin-left:5pt; margin-right:5pt; }
table[rules] {border-left:solid black 0.4pt; border-right:solid black 0.4pt; }
td.td00{ padding-left:0pt; padding-right:0pt; }
td.td01{ padding-left:0pt; padding-right:5pt; }
td.td10{ padding-left:5pt; padding-right:0pt; }
td.td11{ padding-left:5pt; padding-right:5pt; }
table[rules] {border-left:solid black 0.4pt; border-right:solid black 0.4pt; }
.hline hr, .cline hr{ height : 1px; margin:0px; }
.tabbing-right {text-align:right;}
span.TEX {letter-spacing: -0.125em; }
span.TEX span.E{ position:relative;top:0.5ex;left:-0.0417em;}
a span.TEX span.E {text-decoration: none; }
span.LATEX span.A{ position:relative; top:-0.5ex; left:-0.4em; font-size:85%;}
span.LATEX span.TEX{ position:relative; left: -0.4em; }
div.float img, div.float .caption {text-align:center;}
div.figure img, div.figure .caption {text-align:center;}
.marginpar {width:20%; float:right; text-align:left; margin-left:auto; margin-top:0.5em; font-size:85%; text-decoration:underline;}
.marginpar p{margin-top:0.4em; margin-bottom:0.4em;}
.equation td{text-align:center; vertical-align:middle; }
td.eq-no{ width:5%; }
table.equation { width:100%; } 
div.math-display, div.par-math-display{text-align:center;}
math .texttt { font-family: monospace; }
math .textit { font-style: italic; }
math .textsl { font-style: oblique; }
math .textsf { font-family: sans-serif; }
math .textbf { font-weight: bold; }
.partToc a, .partToc, .likepartToc a, .likepartToc {line-height: 200%; font-weight:bold; font-size:110%;}
.chapterToc a, .chapterToc, .likechapterToc a, .likechapterToc, .appendixToc a, .appendixToc {line-height: 200%; font-weight:bold;}
.index-item, .index-subitem, .index-subsubitem {display:block}
.caption td.id{font-weight: bold; white-space: nowrap; }
table.caption {text-align:center;}
h1.partHead{text-align: center}
p.bibitem { text-indent: -2em; margin-left: 2em; margin-top:0.6em; margin-bottom:0.6em; }
p.bibitem-p { text-indent: 0em; margin-left: 2em; margin-top:0.6em; margin-bottom:0.6em; }
.subsectionHead, .likesubsectionHead { margin-top:2em; font-weight: bold;}
.sectionHead, .likesectionHead { font-weight: bold;}
.quote {margin-bottom:0.25em; margin-top:0.25em; margin-left:1em; margin-right:1em; text-align:justify;}
.verse{white-space:nowrap; margin-left:2em}
div.maketitle {text-align:center;}
h2.titleHead{text-align:center;}
div.maketitle{ margin-bottom: 2em; }
div.author, div.date {text-align:center;}
div.thanks{text-align:left; margin-left:10%; font-size:85%; font-style:italic; }
div.author{white-space: nowrap;}
.quotation {margin-bottom:0.25em; margin-top:0.25em; margin-left:1em; }
h1.partHead{text-align: center}
.sectionToc, .likesectionToc {margin-left:2em;}
.subsectionToc, .likesubsectionToc {margin-left:4em;}
.sectionToc, .likesectionToc {margin-left:6em;}
.frenchb-nbsp{font-size:75%;}
.frenchb-thinspace{font-size:75%;}
.figure img.graphics {margin-left:10%;}
/* end css.sty */

\title{Equations de surfaces}
\author{}
\date{}

\begin{document}
\maketitle

\textbf{Warning: 
requires JavaScript to process the mathematics on this page.\\ If your
browser supports JavaScript, be sure it is enabled.}

\begin{center}\rule{3in}{0.4pt}\end{center}

[
[
[]
[

\section{19.3 Equations de surfaces}

\subsection{19.3.1 Surfaces cartésiennes et nappes paramétrées}

Nous avons vu précédemment que, au voisinage d'un point régulier, une
nappe paramétrée était équivalente à une nappe cartésienne, donc définie
par une équation du type z = f(x,y) dans un repère convenablement
choisi. Inversement toute nappe cartésienne est bien évidemment une
nappe paramétrée par x = u,y = v,z = f(u,v).

Pla\ccons nous maintenant du point de vue d'un
sous-ensemble de \mathbb{R}~^3 défini par une équation du type f(x,y,z)
= 0 où f est une fonction de classe C^k d'un ouvert U de
\mathbb{R}~^3 dans \mathbb{R}~. Soit donc \Sigma = \(x,y,z) \in
U∣f(x,y,z) = 0\. Supposons
qu'en un point (a,b,c) de \Sigma on ait ( \partial~f \over \partial~x
(a,b,c), \partial~f \over \partial~y (a,b,c), \partial~f
\over \partial~z (a,b,c))\neq~(0,0,0).
Quitte à permuter les noms des coordonnées, on peut supposer par exemple
 \partial~f \over \partial~z (a,b,c)\neq~0. Le
théorème des fonctions implicites nous garantit qu'il existe
U_0 ouvert contenant (a,b), V _0 ouvert contenant c et
\phi : U_0 \rightarrow~ V _0 de classe C^k telle que

\forall~(x,y) \in U_0~,
\forall~z \in V _0~, f(x,y,z) = 0
\Leftrightarrow z = \phi(x,y)

Autrement dit \Sigma, au voisinage de (a,b,c) est l'image d'une nappe
cartésienne. Inversement, il est clair que l'image d'une nappe
cartésienne z = \phi(x,y) est définie par l'équation f(x,y,z) = 0 où
f(x,y,z) = z - \phi(x,y) (avec d'ailleurs  \partial~f \over \partial~z =
1). On dira que \Sigma est une surface cartésienne quand elle vérifie
\forall~(a,b,c) \in \Sigma, ( \partial~f \over \partial~x~
(a,b,c), \partial~f \over \partial~y (a,b,c), \partial~f
\over \partial~z (a,b,c))\neq~(0,0,0)
(la véritable dénomination étant en fait sous variété de dimension 2 de
\mathbb{R}~^3).

Ceci nous montre donc, qu'au moins localement les trois points de vue
(nappe cartésienne, nappe paramétrée et surfaces cartésiennes) sont
équivalents avec certaines hypothèses de régularité. On retiendra en
particulier sous ces trois formes les expressions du plan tangent et du
vecteur normal.

Nappes paramétrées (u,v)\mapsto~F(u,v) =
(\phi(u,v),\psi(u,v),\omega(u,v))

Le plan tangent en (u_0,v_0) est le plan
F(u_0,v_0) +\
\mathrmVect( \partial~F \over \partial~u
(u_0,v_0), \partial~F \over \partial~v
(u_0,v_0)) d'équation

\left
\matrix\,x -
\phi(u_0,v_0)& \partial~\phi \over \partial~u
(u_0,v_0)& \partial~\phi \over \partial~v
(u_0,v_0) \cr y -
\psi(u_0,v_0)& \partial~\psi \over \partial~u
(u_0,v_0)& \partial~\psi \over \partial~v
(u_0,v_0) \cr z -
\omega(u_0,v_0)& \partial~\omega \over \partial~u
(u_0,v_0)& \partial~\omega \over \partial~v
(u_0,v_0)\right  = 0

avec comme vecteur normal  \partial~F \over \partial~u
(u_0,v_0) ∧ \partial~F \over \partial~v
(u_0,v_0) = \left
(\matrix\, \partial~\phi \over
\partial~u (u_0,v_0) \cr  \partial~\psi
\over \partial~u (u_0,v_0) \cr
 \partial~\omega \over \partial~u
(u_0,v_0)\right )
∧\left (\matrix\, \partial~\phi
\over \partial~v (u_0,v_0) \cr
 \partial~\psi \over \partial~v (u_0,v_0)
\cr  \partial~\omega \over \partial~v
(u_0,v_0)\right )

Nappes cartésiennes z = f(x,y)

En utilisant le paramétrage
(x,y)\mapsto~(x,y,f(x,y)) et les formules
précédentes, on obtient les formules suivantes.

Le plan tangent en (x_0,y_0) est le plan d'équation
(en utilisant les notations de Monge p = \partial~f \over \partial~x
(x_0,y_0), q = \partial~f \over \partial~y
(x_0,y_0))

\left
\matrix\,x -
x_0&1&0 \cr y - y_0&0&1
\cr z -
f(x_0,y_0)&p&q\right  = 0

avec comme vecteur normal \left
(\matrix\,1 \cr 0
\cr p\right ) ∧\left
(\matrix\,0 \cr 1
\cr q\right ) = \left
(\matrix\,-p \cr -q
\cr 1 \right )

Surfaces cartésiennes f(x,y,z) = 0

A l'aide du théorème des fonctions implicites, on a obtenu les résultats
suivants.

Le plan tangent en (x_0,y_0,z_0) est le plan
d'équation

(x - x_0) \partial~f \over \partial~x
(x_0,y_0,z_0) + (y - y_0) \partial~f
\over \partial~y (x_0,y_0,z_0) + (z
- z_0) \partial~f \over \partial~z
(x_0,y_0,z_0) = 0

avec le vecteur normal
\overrightarrowgradf(x_0,y_0,z_0~)
= \left (\matrix\, \partial~f
\over \partial~x (x_0,y_0,z_0)
\cr  \partial~f \over \partial~y
(x_0,y_0,z_0) \cr  \partial~f
\over \partial~z
(x_0,y_0,z_0)\right )

\subsection{19.3.2 Cylindres}

Définition~19.3.1 Soit E un espace affine euclidien de dimension 3 et
\vecD une direction de droite. On dit qu'une partie \Sigma
de E est un cylindre de direction \vecD si, pour tout
m \in \Sigma, la droite m +\vec D est contenue dans \Sigma.

Définition~19.3.2 Les droites m +\vec D contenues
dans \Sigma sont appelées les génératrices du cylindre. Un sous-ensemble qui
rencontre toutes les génératrices est appelé un sous-ensemble directeur
du cylindre. Un sous-ensemble directeur plan est appelé une base du
cylindre.

On est parfois amené à rechercher un cylindre connaissant un
sous-ensemble directeur \Gamma et la direction du cylindre
\vecD = \mathbb{R}~\vecu. Nous supposerons
choisi un repère de E ce qui nous permet de supposer que E =
\mathbb{R}~^3. On posera donc \vecu = (\alpha~,\beta~,\gamma).

Premier cas \Gamma est l'image d'un arc paramétré
u\mapsto~(\phi(u),\psi(u),\omega(u)). On obtient immédiatement
une paramétrisation du cylindre par x = \phi(u) + \alpha~v,y = \psi(u) + \beta~v,z = \omega(u)
+ \gammav.

Deuxième cas \Gamma est donnée par deux équations f(x,y,z) = 0,g(x,y,z) = 0.
On écrit alors que

\begin{align*} m(x,y,z) \in \Sigma&
\Leftrightarrow & \exists~t \in \mathbb{R}~, m +
t\vecu \in \Gamma \%& \\ &
\Leftrightarrow & \exists~t \in \mathbb{R}~,
\left
\\matrix\,f(x + t\alpha~,y +
t\beta~,z + t\gamma) = 0 \cr g(x + t\alpha~,y + t\beta~,z + t\gamma) =
0\right .\%& \\
\end{align*}

et on élimine t entre ces équations.

Exemple~19.3.1 Cylindre de direction \vecu = (1,1,1)
de sous-ensemble directeur la parabole y^2 = 2px,z = 0. On
écrit

\begin{align*} m(x,y,z) \in \Sigma&
\Leftrightarrow & \exists~t \in \mathbb{R}~, m +
t\vecu \in \Gamma \%& \\ &
\Leftrightarrow & \exists~t \in \mathbb{R}~,
\left
\\matrix\,(y +
t)^2 = 2p(x + t) \cr z + t =
0\right .\%& \\ &
\Leftrightarrow & (y - z)^2 = 2p(x - z) \%&
\\ \end{align*}

et nous avons obtenu une équation du cylindre.

\subsection{19.3.3 Cônes}

Définition~19.3.3 Soit E un espace affine euclidien de dimension 3 et S
un point de E. On dit qu'une partie \Sigma de E est un cône de sommet S si,
pour tout m \in \Sigma \diagdown\S\, la droite Sm est
contenue dans \Sigma.

Définition~19.3.4 Les droites Sm contenues dans \Sigma sont appelées les
génératrices du cône. Un sous-ensemble ne contenant pas le sommet qui
rencontre toutes les génératrices est appelé un sous ensemble directeur
du cône. Un sous-espace directeur plan est appelé une base du cône.

On est parfois amené à rechercher un cône connaissant un sous-ensemble
directeur \Gamma et le sommet S du cône. Nous supposerons choisi un repère de
E ce qui nous permet de supposer que E = \mathbb{R}~^3. On posera donc
S = (\alpha~,\beta~,\gamma).

Premier cas \Gamma est l'image d'un arc paramétré
u\mapsto~(\phi(u),\psi(u),\omega(u)). On obtient immédiatement
une paramétrisation du cône par x = v\phi(u) + (1 - v)\alpha~,y = v\psi(u) + (1 -
v)\beta~,z = v\omega(u) + (1 - v)\gamma.

Deuxième cas \Gamma est donnée par deux équations f(x,y,z) = 0,g(x,y,z) = 0.
On écrit alors que

\begin{align*} m(x,y,z) \in \Sigma
\diagdown\S& \Leftrightarrow &
\exists~t \in \mathbb{R}~, S +
t\overrightarrowSm \in \Gamma\%&
\\ \end{align*}

soit encore

\exists~t \in \mathbb{R}~, \left
\\matrix\,f(tx + (1 -
t)\alpha~,ty + (1 - t)\beta~,tz + (1 - t)\gamma) = 0 \cr g(tx + (1 -
t)\alpha~,ty + (1 - t)\beta~,tz + (1 - t)\gamma) = 0\right .

et on élimine t entre ces équations.

Exemple~19.3.2 Cône de sommet S = (0,0,1) de sous-ensemble directeur la
parabole y^2 = 2px,z = 0. On écrit

\begin{align*} m(x,y,z) \in \Sigma
\diagdown\S& \Leftrightarrow &
\exists~t \in \mathbb{R}~, S +
t\overrightarrowSm \in \Gamma \Leftrightarrow
\exists~t \in \mathbb{R}~, \left
\\matrix\,(ty)^2
= 2ptx \cr tz + (1 - t) = 0\right . \%&
\\ & \Leftrightarrow &
z\neq~1\text et
\left ( y \over 1 - z
\right )^2 = 2px \over 1 - z
\Leftrightarrow
z\neq~1\text et y^2
= 2px(1 - z)\%& \\
\end{align*}

et nous avons obtenu une équation du cône (privé de son sommet).

\subsection{19.3.4 Surfaces de révolution}

Définition~19.3.5 Soit E un espace affine euclidien de dimension 3 et D
une droite de E. On dit qu'une partie \Sigma de E est une surface de
révolution d'axe D si, pour tout m \in \Sigma, le cercle C_m d'axe D
passant par m est contenu dans \Sigma.

Remarque~19.3.1 Les cercles C_m contenus dans \Sigma sont appelés
les parallèles de la surface de révolution. Un sous-ensemble qui
rencontre tous les parallèles est appelé un sous-ensemble directeur de
la surface de révolution. Un sous-ensemble directeur situé dans un plan
contenant l'axe D est appelé un méridien.

Remarque~19.3.2 Supposons que \Sigma soit défini par l'équation f(x,y,z) = 0
et supposons les axes choisis de telle sorte que D soit l'axe 0z. Posons
alors g(\rho,\theta,z) = f(\rhocos~
\theta,\rhosin~ \theta,z) en coordonnées cylindriques. Si g
ne dépend pas de \theta, on voit immédiatement que la surface est de
révolution, admettant pour méridiens les sous-ensembles g(\rho,z) = 0 dans
les plans \rhoOz. On retiendra en particulier que toute équation du type
F(x^2 + y^2,z) = 0 définit une surface de
révolution.

On est parfois amené à rechercher une surface de révolution connaissant
un sous-ensemble directeur \Gamma et l'axe de révolution. Nous supposerons
choisi un repère de E ce qui nous permet de supposer que E =
\mathbb{R}~^3. On posera alors D = (a,b,c) + \mathbb{R}~(\alpha~,\beta~,\gamma).

Premier cas \Gamma est l'image d'un arc paramétré
u\mapsto~(\phi(u),\psi(u),\omega(u)). On obtient immédiatement
une paramétrisation de la surface de révolution par (x,y,z) =
R_D(v)(\phi(u),\psi(u),\omega(u)) où R_D(\theta) désigne la rotation
d'axe D et d'angle \theta. Si le repère est bien choisi, on peut supposer que
D est l'axe 0z. Alors R_D(\theta) est l'application
(x,y,z)\mapsto~(xcos~ \theta -
ysin \theta,x\sin~ \theta +
ycos~ \theta,z) si bien que l'on a la
paramétrisation de la nappe par

x = \phi(u)cos~ v -
\psi(u)sin~ v, y =
\phi(u)sin v + \psi(u)\cos~
v,z = \omega(u)

Deuxième cas \Gamma est donné par deux équations f(x,y,z) = 0,g(x,y,z) = 0.
Remarquons alors que C_m est l'intersection de la sphère de
centre A passant par m avec le plan orthogonal à
\vecu passant par m. Il admet donc pour équations (si
m a pour coordonnées (x_0,y_0,z_0))

\left
\\matrix\,(x -
a)^2 + (y - b)^2 + (z - c)^2 =
(x_0 - a)^2 + (y_0 - b)^2 +
(z_0 - c)^2 \cr \alpha~x + \beta~y + \gammaz =
\alpha~x_0 + \beta~y_0 + \gammaz_0\right .

On écrit alors que

\begin{align*}
m(x_0,y_0,z_0) \in \Sigma&
\Leftrightarrow & C_m \bigcap
\Gamma\neq~\varnothing~\%& \\
\end{align*}

soit encore \exists~x,y,z \in \mathbb{R}~,

\begin{align*} \left
\\matrix\,(x -
a)^2 + (y - b)^2 + (z - c)^2 =
(x_0 - a)^2 + (y_0 - b)^2 +
(z_0 - c)^2 \cr \alpha~x + \beta~y + \gammaz =
\alpha~x_0 + \beta~y_0 + \gammaz_0 \cr
f(x,y,z) = 0 \cr g(x,y,z) = 0\right .&
& \%& \\
\end{align*}

et on élimine x,y et z entre ces équations.

Dans le cas où D est l'axe Oz, on remplacera avantageusement la sphère
par un cylindre d'axe Oz et on obtiendra comme équation de C_m
\left
\\matrix\,x^2
+ y^2 = x_0^2 + y_0^2
\cr z = z_0\right . et donc

\begin{align*}
m(x_0,y_0,z_0) \in \Sigma&
\Leftrightarrow & C_m \bigcap
\Gamma\neq~\varnothing~ \%& \\ &
\Leftrightarrow & \exists~x,y,z \in \mathbb{R}~,
\left
\\matrix\,x^2
+ y^2 = x_0^2 + y_0^2
\cr z = z_0 \cr f(x,y,z) = 0
\cr g(x,y,z) = 0\right .\%&
\\ \end{align*}

et on élimine x,y et z entre ces équations.

Exemple~19.3.3 Surface de révolution engendrée par la rotation de la
droite \Delta : x = 1, y = z autour de l'axe 0z. On a donc

\begin{align*}
m(x_0,y_0,z_0) \in \Sigma&
\Leftrightarrow & C_m \bigcap
\Delta\neq~\varnothing~ \%& \\ &
\Leftrightarrow & \exists~x,y,z \in \mathbb{R}~,
\left
\\matrix\,x^2
+ y^2 = x_0^2 + y_0^2
\cr z = z_0 \cr x = 1
\cr y = z\right .\%&
\\ & \Leftrightarrow &
x_0^2 + y_ 0^2 = 1 + z_
0^2 \%& \\
\end{align*}

si bien que la surface a pour équation cartésienne x^2 +
y^2 - z^2 = 1, comme équation cylindrique
\rho^2 - z^2 = 1 et que ses méridiennes sont des
hyperboles équilatères. Il s'agit là d'un hyperboloïde (à une nappe).

[
[
[
[

\end{document}

% \documentclass[]{article}
\usepackage[T1]{fontenc}
\usepackage{lmodern}
\usepackage{amssymb,amsmath}
\usepackage{ifxetex,ifluatex}
\usepackage{fixltx2e} % provides \textsubscript
% use upquote if available, for straight quotes in verbatim environments
\IfFileExists{upquote.sty}{\usepackage{upquote}}{}
\ifnum 0\ifxetex 1\fi\ifluatex 1\fi=0 % if pdftex
  \usepackage[utf8]{inputenc}
\else % if luatex or xelatex
  \ifxetex
    \usepackage{mathspec}
    \usepackage{xltxtra,xunicode}
  \else
    \usepackage{fontspec}
  \fi
  \defaultfontfeatures{Mapping=tex-text,Scale=MatchLowercase}
  \newcommand{\euro}{€}
\fi
% use microtype if available
\IfFileExists{microtype.sty}{\usepackage{microtype}}{}
\ifxetex
  \usepackage[setpagesize=false, % page size defined by xetex
              unicode=false, % unicode breaks when used with xetex
              xetex]{hyperref}
\else
  \usepackage[unicode=true]{hyperref}
\fi
\hypersetup{breaklinks=true,
            bookmarks=true,
            pdfauthor={},
            pdftitle={Quadriques},
            colorlinks=true,
            citecolor=blue,
            urlcolor=blue,
            linkcolor=magenta,
            pdfborder={0 0 0}}
\urlstyle{same}  % don't use monospace font for urls
\setlength{\parindent}{0pt}
\setlength{\parskip}{6pt plus 2pt minus 1pt}
\setlength{\emergencystretch}{3em}  % prevent overfull lines
\setcounter{secnumdepth}{0}
 
/* start css.sty */
.cmr-5{font-size:50%;}
.cmr-7{font-size:70%;}
.cmmi-5{font-size:50%;font-style: italic;}
.cmmi-7{font-size:70%;font-style: italic;}
.cmmi-10{font-style: italic;}
.cmsy-5{font-size:50%;}
.cmsy-7{font-size:70%;}
.cmex-7{font-size:70%;}
.cmex-7x-x-71{font-size:49%;}
.msbm-7{font-size:70%;}
.cmtt-10{font-family: monospace;}
.cmti-10{ font-style: italic;}
.cmbx-10{ font-weight: bold;}
.cmr-17x-x-120{font-size:204%;}
.cmsl-10{font-style: oblique;}
.cmti-7x-x-71{font-size:49%; font-style: italic;}
.cmbxti-10{ font-weight: bold; font-style: italic;}
p.noindent { text-indent: 0em }
td p.noindent { text-indent: 0em; margin-top:0em; }
p.nopar { text-indent: 0em; }
p.indent{ text-indent: 1.5em }
@media print {div.crosslinks {visibility:hidden;}}
a img { border-top: 0; border-left: 0; border-right: 0; }
center { margin-top:1em; margin-bottom:1em; }
td center { margin-top:0em; margin-bottom:0em; }
.Canvas { position:relative; }
li p.indent { text-indent: 0em }
.enumerate1 {list-style-type:decimal;}
.enumerate2 {list-style-type:lower-alpha;}
.enumerate3 {list-style-type:lower-roman;}
.enumerate4 {list-style-type:upper-alpha;}
div.newtheorem { margin-bottom: 2em; margin-top: 2em;}
.obeylines-h,.obeylines-v {white-space: nowrap; }
div.obeylines-v p { margin-top:0; margin-bottom:0; }
.overline{ text-decoration:overline; }
.overline img{ border-top: 1px solid black; }
td.displaylines {text-align:center; white-space:nowrap;}
.centerline {text-align:center;}
.rightline {text-align:right;}
div.verbatim {font-family: monospace; white-space: nowrap; text-align:left; clear:both; }
.fbox {padding-left:3.0pt; padding-right:3.0pt; text-indent:0pt; border:solid black 0.4pt; }
div.fbox {display:table}
div.center div.fbox {text-align:center; clear:both; padding-left:3.0pt; padding-right:3.0pt; text-indent:0pt; border:solid black 0.4pt; }
div.minipage{width:100%;}
div.center, div.center div.center {text-align: center; margin-left:1em; margin-right:1em;}
div.center div {text-align: left;}
div.flushright, div.flushright div.flushright {text-align: right;}
div.flushright div {text-align: left;}
div.flushleft {text-align: left;}
.underline{ text-decoration:underline; }
.underline img{ border-bottom: 1px solid black; margin-bottom:1pt; }
.framebox-c, .framebox-l, .framebox-r { padding-left:3.0pt; padding-right:3.0pt; text-indent:0pt; border:solid black 0.4pt; }
.framebox-c {text-align:center;}
.framebox-l {text-align:left;}
.framebox-r {text-align:right;}
span.thank-mark{ vertical-align: super }
span.footnote-mark sup.textsuperscript, span.footnote-mark a sup.textsuperscript{ font-size:80%; }
div.tabular, div.center div.tabular {text-align: center; margin-top:0.5em; margin-bottom:0.5em; }
table.tabular td p{margin-top:0em;}
table.tabular {margin-left: auto; margin-right: auto;}
div.td00{ margin-left:0pt; margin-right:0pt; }
div.td01{ margin-left:0pt; margin-right:5pt; }
div.td10{ margin-left:5pt; margin-right:0pt; }
div.td11{ margin-left:5pt; margin-right:5pt; }
table[rules] {border-left:solid black 0.4pt; border-right:solid black 0.4pt; }
td.td00{ padding-left:0pt; padding-right:0pt; }
td.td01{ padding-left:0pt; padding-right:5pt; }
td.td10{ padding-left:5pt; padding-right:0pt; }
td.td11{ padding-left:5pt; padding-right:5pt; }
table[rules] {border-left:solid black 0.4pt; border-right:solid black 0.4pt; }
.hline hr, .cline hr{ height : 1px; margin:0px; }
.tabbing-right {text-align:right;}
span.TEX {letter-spacing: -0.125em; }
span.TEX span.E{ position:relative;top:0.5ex;left:-0.0417em;}
a span.TEX span.E {text-decoration: none; }
span.LATEX span.A{ position:relative; top:-0.5ex; left:-0.4em; font-size:85%;}
span.LATEX span.TEX{ position:relative; left: -0.4em; }
div.float img, div.float .caption {text-align:center;}
div.figure img, div.figure .caption {text-align:center;}
.marginpar {width:20%; float:right; text-align:left; margin-left:auto; margin-top:0.5em; font-size:85%; text-decoration:underline;}
.marginpar p{margin-top:0.4em; margin-bottom:0.4em;}
.equation td{text-align:center; vertical-align:middle; }
td.eq-no{ width:5%; }
table.equation { width:100%; } 
div.math-display, div.par-math-display{text-align:center;}
math .texttt { font-family: monospace; }
math .textit { font-style: italic; }
math .textsl { font-style: oblique; }
math .textsf { font-family: sans-serif; }
math .textbf { font-weight: bold; }
.partToc a, .partToc, .likepartToc a, .likepartToc {line-height: 200%; font-weight:bold; font-size:110%;}
.chapterToc a, .chapterToc, .likechapterToc a, .likechapterToc, .appendixToc a, .appendixToc {line-height: 200%; font-weight:bold;}
.index-item, .index-subitem, .index-subsubitem {display:block}
.caption td.id{font-weight: bold; white-space: nowrap; }
table.caption {text-align:center;}
h1.partHead{text-align: center}
p.bibitem { text-indent: -2em; margin-left: 2em; margin-top:0.6em; margin-bottom:0.6em; }
p.bibitem-p { text-indent: 0em; margin-left: 2em; margin-top:0.6em; margin-bottom:0.6em; }
.paragraphHead, .likeparagraphHead { margin-top:2em; font-weight: bold;}
.subparagraphHead, .likesubparagraphHead { font-weight: bold;}
.quote {margin-bottom:0.25em; margin-top:0.25em; margin-left:1em; margin-right:1em; text-align:\jmathustify;}
.verse{white-space:nowrap; margin-left:2em}
div.maketitle {text-align:center;}
h2.titleHead{text-align:center;}
div.maketitle{ margin-bottom: 2em; }
div.author, div.date {text-align:center;}
div.thanks{text-align:left; margin-left:10%; font-size:85%; font-style:italic; }
div.author{white-space: nowrap;}
.quotation {margin-bottom:0.25em; margin-top:0.25em; margin-left:1em; }
h1.partHead{text-align: center}
.sectionToc, .likesectionToc {margin-left:2em;}
.subsectionToc, .likesubsectionToc {margin-left:4em;}
.subsubsectionToc, .likesubsubsectionToc {margin-left:6em;}
.frenchb-nbsp{font-size:75%;}
.frenchb-thinspace{font-size:75%;}
.figure img.graphics {margin-left:10%;}
/* end css.sty */

\title{Quadriques}
\author{}
\date{}

\begin{document}
\maketitle

\textbf{Warning: 
requires JavaScript to process the mathematics on this page.\\ If your
browser supports JavaScript, be sure it is enabled.}

\begin{center}\rule{3in}{0.4pt}\end{center}

{[}
{[}
{[}{]}
{[}

\subsubsection{19.4 Quadriques}

\paragraph{19.4.1 Notion de quadrique}

Définition~19.4.1 Soit E un espace affine de dimension finie de
direction \vecE et F : E \rightarrow~ \mathbb{R}~. On dit que F est une
forme quadratique affine si elle vérifie les conditions équivalentes (i)
il existe a \in E, une forme quadratique \Phi\_a sur
\vecE et une forme linéaire f\_a sur
\vecE telles que \forall~~x \in E,
F(x) = \Phi\_a(\overrightarrowax) +
f\_a(\overrightarrowax) + F(a) (ii) pour tout
a \in E, il existe une forme quadratique \Phi\_a sur
\vecE et une forme linéaire f\_a sur
\vecE telles que \forall~~x \in E,
F(x) = \Phi\_a(\overrightarrowax) +
f\_a(\overrightarrowax) + F(a) (iii) pour
tout repère affine
(a,\overrightarrowe\_1,\\ldots,\overrightarrowe\_n~),
il existe un polynôme P \in
\mathbb{R}~{[}X\_1,\\ldots,X\_n~{]}
de degré inférieur ou égal à 2 tel que F(x) =
P(x\_1,\\ldots,x\_n~)
si
x\_1,\\ldots,x\_n~
sont les coordonnées de x dans ce repère. La forme quadratique
\Phi\_a est en fait indépendante de a \in E~; on l'appelle la forme
quadratique principale de F.

Démonstration Il est clair que (ii) \rigtharrow~(i). Supposons (i) vérifié et soit
b \in E. On a alors par l'identité de polarisation

\begin{align*} F(x)& =&
\Phi\_a(\overrightarrowab
+\overrightarrow bx) +
f\_a(\overrightarrowab
+\overrightarrow bx) + F(a) \%&
\\ & =&
\Phi\_a(\overrightarrowbx) +
2\phi\_a(\overrightarrowab,\overrightarrowbx)
+ f\_a(\overrightarrowbx) +
\Phi\_a(\overrightarrowab) +
f\_a(\overrightarrowab) + F(a)\%&
\\ & =&
\Phi\_b(\overrightarrowbx) +
f\_b(\overrightarrowbx) + F(b) \%&
\\ \end{align*}

en posant \Phi\_b = \Phi\_a et
f\_b(\overrightarrow\xi) =
2\phi\_a(\overrightarrowab,\overrightarrow\xi)
+ f\_a(\overrightarrow\xi). Ceci montre à la
fois que (i) \rigtharrow~(ii) et que \Phi\_a ne dépend pas de a.

L'équivalence entre (i) et (iii) résulte immédiatement des isomorphismes
dé\jmathà connus entre formes quadratiques et polynômes homogènes de degré 2,
formes linéaires et polynômes homogènes de degré 1, constantes et
polynômes homogènes de degré 0. La décomposition F(x) =
\Phi\_a(\overrightarrowax) +
f\_a(\overrightarrowax) + F(a) correspond
exactement à la décomposition P = P\_2 + P\_1 +
P\_0 d'un polynôme de degré au plus 2 en un polynôme homogène de
degré 2, un polynôme homogène de degré 1 et une constante.

Remarque~19.4.1 Contrairement à la forme quadratique principale \Phi qui ne
dépend pas de a, la forme linéaire f\_a dépend de a. Supposons
que \Phi est non dégénérée~; on sait alors que l'on peut trouver un vecteur
\vecv \in\vec E tel que
\forall~\vec\xi~
\in\vec E, f\_a(\vec\xi) =
\phi(\vecv,\vec\xi). Prenons alors b =
a - 1 \over 2 \vecv. On a alors
f\_b(\vec\xi) =
2\phi(\overrightarrowab,\vec\xi) +
f\_a(\vec\xi) =
\phi(2\overrightarrowab +\vec
v,\vec\xi) = \phi(0,\vec\xi) = 0, si
bien que f\_b = 0.

Définition~19.4.2 On dit que a \in E est un centre de la forme quadratique
affine si f\_a = 0.

Remarque~19.4.2 On a donc montré que si \Phi est non dégénérée, F admet un
centre.

Définition~19.4.3 On dit qu'un sous-ensemble \Sigma de E est une quadrique
(ou une conique en dimension 2) s'il existe une forme quadratique affine
de forme quadratique principale non nulle telle que \Sigma =
\x \in E∣F(x) =
0\.

\paragraph{19.4.2 Réduction des quadriques}

Supposons que E est un espace affine euclidien. Soit \Sigma une quadrique
d'équation F(x) = 0 et soit \Phi la forme quadratique principale de F. On
sait qu'il existe une base orthonormée
(\overrightarrowe\_1,\\ldots,\overrightarrowe\_n~)
de \vecE qui est orthogonale pour \Phi. La matrice de \Phi
dans cette base est alors
diag(\lambda~\_1,\\\ldots,\lambda~\_n~)
et quitte à permuter la base on peut supposer que
\lambda~\_1\neq~0,\\ldots,\lambda~\_r\mathrel\neq~0,\lambda~\_r+1~
= \\ldots~ =
\lambda~\_n = 0 pour un r \in {[}1,n{]} (car on a supposé
\Phi\neq~0). Dans tout repère
(a,\overrightarrowe\_1,\\ldots,\overrightarrowe\_n~)
l'équation de \Sigma est donc de la forme \lambda~\_1x\_1^2 +
\\ldots~ +
\lambda~\_rx\_r^2 +\
\sum ~
\_i=1^n\alpha~\_ix\_i + k = 0 soit encore
\lambda~\_1(x\_1 + \alpha~\_1 \over
2\lambda~\_1 )^2 +
\\ldots~ +
\lambda~\_r(x\_r + \alpha~\_r \over
2\lambda~\_r )^2 +\
\sum ~
\_i=r+1^n\alpha~\_ix\_i + k' = 0 avec k' = k
-\\sum ~
\_i=1^r \alpha~\_i^2 \over
4\lambda~\_i^2 . En posant x\_1' = x\_1 +
\alpha~\_1 \over 2\lambda~\_1 ,x\_r' =
x\_r + \alpha~\_r \over 2\lambda~\_r ,
c'est-à-dire en faisant un changement d'origine du repère, on obtient un
nouveau repère
(a',\overrightarrowe\_1,\\ldots,\overrightarrowe\_n~)
dans lequel l'équation devient \lambda~\_1x\_1^2 +
\\ldots~ +
\lambda~\_rx\_r^2 +\
\sum ~
\_i=r+1^n\alpha~\_ix'\_i + k' = 0.

S'il existe i ≥ r + 1 tel que \alpha~\_i\neq~0,
posons e\_r+1' =
\\sum ~
\_i=r+1^n\alpha~\_ ie\_i \over
\sqrt\\\sum
 \_i=r+1^n\alpha~\_i^2 . Alors
(e\_1,\\ldots,e\_r,e\_r+1~')
est une famille orthonormée, que nous pouvons compléter en une base
orthonormée
(e\_1,\\ldots,e\_r,e\_r+1',\\\ldots,e\_n~').
Dans le repère
(a',e\_1,\\ldots,e\_r,e\_r+1',\\\ldots,e\_n~')
les nouvelles coordonnées sont x'`\_1 =
x\_1,\\ldots,x'`\_r~
= x\_r,x'`\_r+1 =
\\sum ~
\_i=r+1^n\alpha~\_ ix'\_i \over
\sqrt\\\sum
 \_i=r+1^n\alpha~\_i^2 si bien que
l'équation devient dans ce repère \lambda~\_1x\_1^2 +
\\ldots~ +
\lambda~\_rx\_r^2 + \beta~x'`\_r+1 + k' = 0 avec
\beta~\neq~0. On écrit alors l'équation sous la forme
\lambda~\_1x\_1^2 +
\\ldots~ +
\lambda~\_rx\_r^2 + \beta~(x'`\_r+1 + k'
\over \beta~ ) = 0 et un nouveau changement d'origine ramène
à une équation \lambda~\_1y\_1^2 +
\\ldots~ +
\lambda~\_ry\_r^2 + \beta~y\_r+1 = 0.

Si par contre tous les \alpha~\_i sont nuls pour i ≥ r + 1 ou si r =
n, alors l'équation est dé\jmathà réduite à la forme
\lambda~\_1x\_1^2 +
\\ldots~ +
\lambda~\_rx\_r^2 + k' = 0. On a donc démontré le
théorème suivant

Théorème~19.4.1 Soit \Sigma une quadrique. Alors il existe un repère
orthonormé
(a,\overrightarrowe\_1,\\ldots,\overrightarrowe\_n~)
tel que l'équation de \Sigma dans ce repère soit de l'une des deux formes
suivantes

\begin{align*} \lambda~\_1x\_1^2 +
\\ldots + \lambda~~\_
rx\_r^2 + k& =& 0, r \leq n \%&
\\ \text(quadrique à
centre)& & \%& \\
\lambda~\_1x\_1^2 +
\\ldots + \lambda~~\_
rx\_r^2 + \beta~x\_ r+1& =& 0, r \leq n - 1\%&
\\ \text(quadrique sans
centre)& & \%& \\
\end{align*}

avec
\lambda~\_1,\\ldots,\lambda~\_r~
non nuls. L'entier r est le rang de la forme quadratique principale et
\lambda~\_1,\\ldots,\lambda~\_r~
les valeurs propres non nulles (comptées avec leurs multiplicités) de la
matrice de la forme quadratique principale \Phi dans n'importe quelle base
orthonormée.

\paragraph{19.4.3 Classification des quadriques en dimension 2 et 3}

Dimension 2

Premier cas r = 2, \lambda~\_1\lambda~\_2 \textgreater{} 0~: on
obtient à partir de l'équation \lambda~\_1x^2 +
\lambda~\_2y^2 = -k que la conique est soit l'ensemble vide,
soit un point (si k = 0), soit une ellipse.

Deuxième cas r = 2, \lambda~\_1\lambda~\_2 \textless{} 0~: on obtient
à partir de l'équation \lambda~\_1x^2 +
\lambda~\_2y^2 = -k que la conique est soit la réunion de
deux droites sécantes (si k = 0), soit une hyperbole (si
k\neq~0).

Troisième cas r = 1 et conique sans centre~: on obtient à partir de
l'équation \lambda~\_1x^2 + \beta~y = 0 que la conique est une
parabole

Quatrième cas r = 1 et conique avec centre~: on obtient à partir de
l'équation \lambda~\_1x^2 + k = 0 que la conique est soit
l'ensemble vide, soit une droite soit la réunion de deux droites
parallèles.

Dimension 3

Premier cas r = 3, \lambda~\_1,\lambda~\_2 et \lambda~\_3 de même
signe (par exemple positifs)~; l'équation peut s'écrire sous la forme
\lambda~\_1x^2 + \lambda~\_2y^2 +
\lambda~\_3z^2 = k~; si k \textless{} 0, on obtient
l'ensemble vide~; si k = 0, la quadrique est réduite à un point~; si k
\textgreater{} 0, la quadrique se déduit par l'affinité
(x,y,z)\mapsto~(\sqrt\lambda~\_1x,\sqrt\lambda~\_2y,\sqrt\lambda~\_3z)
de la sphère x^2 + y^2 + z^2 = k, il
s'agit donc d'un ellipsoïde.

Deuxième cas r = 3, \lambda~\_1,\lambda~\_2 et \lambda~\_3 de signes
distincts. On peut par exemple supposer que \lambda~\_1 \textgreater{}
0,\lambda~\_2 \textgreater{} 0 et \lambda~\_3 \textless{} 0.
L'équation peut s'écrire sous la forme \lambda~\_1x\_2 +
\lambda~\_2y^2 + \lambda~\_3z^2 = k~; la
quadrique se déduit par l'affinité
(x,y,z)\mapsto~(\sqrt\lambda~\_1x,\sqrt\lambda~\_2y,\sqrt-\lambda~\_3z)
de la quadrique x^2 + y^2 - z^2 = k
autrement dit de la surface de révolution d'axe Oz dont une équation
cylindrique est \rho^2 - z^2 = k~; si k = 0, la
méridienne est la réunion de deux droites et la quadrique est un cône du
second degré~; si k\neq~0, la méridienne est une
hyperbole d'axe focal O\rho si k \textgreater{} 0, d'axe focal Oz si k
\textless{} 0~; dans le premier cas, la quadrique est un hyperboloïde à
une nappe obtenu par affinité à partir de la rotation d'une hyperbole
autour de son axe non focal, dans le second cas un hyperboloïde à deux
nappes, obtenu par affinité à partir de la rotation d'une hyperbole
autour de son axe focal

Troisième cas r = 2, quadrique à centre, \lambda~\_1\lambda~\_2
\textgreater{} 0. On peut écrire l'équation sous la forme
\lambda~\_1x^2 + \lambda~\_2y^2 = k~; il s'agit
soit de l'ensemble vide, soit d'une droite (si k = 0), soit d'un
cylindre d'axe Oz dont la base est une ellipse, c'est-à-dire d'un
cylindre elliptique.

Quatrième cas r = 2, quadrique à centre, \lambda~\_1\lambda~\_2
\textless{} 0. On peut écrire l'équation sous la forme
\lambda~\_1x^2 + \lambda~\_2y^2 = k~; il s'agit
soit de la réunion de deux plans sécants (si k = 0), soit d'un cylindre
d'axe Oz dont la base est une hyperbole, c'est-à-dire d'un cylindre
hyperbolique

Cinquième cas r = 2, quadrique sans centre, \lambda~\_1\lambda~\_2
\textgreater{} 0. On peut écrire l'équation sous la forme
\lambda~\_1x^2 + \lambda~\_2y^2 = \beta~z~; la
quadrique se déduit par l'affinité
(x,y,z)\mapsto~(\sqrt\lambda~\_1x,\sqrt\lambda~\_2y,\beta~z)
de la quadrique x^2 + y^2 = z autrement dit de la
surface de révolution d'axe Oz dont une équation cylindrique est
\rho^2 = z obtenue par rotation d'une parabole autour de son
axe~; il s'agit d'un paraboloïde elliptique.

Sixième cas r = 2, quadrique sans centre, \lambda~\_1\lambda~\_2
\textless{} 0. On peut écrire l'équation sous la forme
\lambda~\_1x^2 + \lambda~\_2y^2 = \beta~z~; la
quadrique se déduit par l'affinité
(x,y,z)\mapsto~(\sqrt\lambda~\_1x,\sqrt-\lambda~\_2y,\beta~z)
de la quadrique x^2 - y^2 = z~; il s'agit d'un
paraboloïde hyperbolique.

Septième cas r = 1, quadrique à centre. L'équation
\lambda~\_1x^2 + k = 0 définit soit l'ensemble vide, soit un
plan, soit la réunion de deux plans parallèles.

Huitième cas r = 1, quadrique sans centre. L'équation
\lambda~\_1x^2 = \beta~y définit un cylindre d'axe Oz dont la
base est une parabole. Il s'agit d'un cylindre parabolique.

\paragraph{19.4.4 Quadriques réglées, quadriques de révolution}

Parmi les neuf types de vraies quadriques, quatre sont des cylindres ou
des cônes qui sont évidemment des surfaces réglées. Il est clair qu'un
ellipsoïde qui est borné ne peut pas contenir de droites, dont ne peut
pas être réglé. Pour des raisons évidentes de non connexité, un
hyperboloïde à deux nappes ne peut pas contenir de droite (une telle
droite serait forcément horizontale car contenue dans un demi-espace
horizontal, or les sections horizontales de l'hyperboloïde sont des
cercles). Un paraboloïde elliptique étant situé dans un demi espace ne
peut évidemment pas contenir de droites (une telle droite serait
forcément horizontale car contenue dans un demi-espace horizontal, or
les sections horizontales du paraboloïde sont des cercles). Reste donc
le cas de l'hyperboloïde à une nappe et du paraboloïde hyperbolique.

En ce qui concerne l'hyperboloïde à une nappe, une équation réduite peut
s'écrire sous la forme  x^2 \over
a^2 + y^2 \over
b^2 - z^2 \over
c^2 = 1 soit encore

\left ( x \over a + z
\over c \right )\left (
x \over a - z \over c
\right ) = \left (1 + y
\over b \right )\left (1
- y \over b \right )

Cet hyperboloïde contient donc les deux familles de droites
(D\_\lambda~,\mu) et (\Delta\_\lambda~,\mu) définies pour
(\lambda~,\mu)\neq~(0,0) par

D\_\lambda~,\mu \left
\\matrix\,\lambda~\left
( x \over a + z \over c
\right ) = \mu\left (1 + y
\over b \right ) \cr
\cr \mu\left ( x \over a
- z \over c \right ) =
\lambda~\left (1 - y \over b
\right )\right .\quad
\text et \quad \Delta\_\lambda~,\mu
\left
\\matrix\,\lambda~\left
( x \over a + z \over c
\right ) = \mu\left (1 - y
\over b \right ) \cr
\cr \mu\left ( x \over a
- z \over c \right ) =
\lambda~\left (1 + y \over b
\right )\right .

Par tout point de l'hyperboloïde passe une et une seule droite de chaque
famille.

En ce qui concerne le paraboloïde hyperbolique, une équation réduite
peut s'écrire sous la forme  x^2 \over
a^2 - y^2 \over
b^2 = z soit encore

\left ( x \over a + y
\over b \right )\left (
x \over a - y \over b
\right ) = z

Cet hyperboloïde contient donc les deux familles de droites
(D\_\lambda~,\mu) et (\Delta\_\lambda~,\mu) définies pour
(\lambda~,\mu)\neq~(0,0) par

D\_\lambda~,\mu \left
\\matrix\,\lambda~\left
( x \over a + y \over b
\right ) = \muz \cr \cr
\mu\left ( x \over a - y
\over b \right ) =
\lambda~\right .\quad \text et
\quad \Delta\_\lambda~,\mu \left
\\matrix\,\lambda~\left
( x \over a + y \over b
\right ) = \mu \cr \cr
\mu\left ( x \over a - y
\over b \right ) =
\lambda~z\right .

Par tout point du paraboloïde hyperbolique passe une et une seule droite
de chaque famille.

En ce qui concerne la possibilité pour des quadriques d'être de
révolution, on constate immédiatement sur l'équation réduite que la
quadrique est de révolution si deux des \lambda~\_i sont égaux
(c'est-à-dire si la matrice de la forme quadratique principale \Phi dans
n'importe quelle base orthonormée admet une valeur propre double). Ceci
permet de compléter le tableau~:

\begin{center}\rule{3in}{0.4pt}\end{center}

\begin{center}\rule{3in}{0.4pt}\end{center}

\begin{center}\rule{3in}{0.4pt}\end{center}

\begin{center}\rule{3in}{0.4pt}\end{center}

Equation

Type

Réglé

De révolution

\begin{center}\rule{3in}{0.4pt}\end{center}

\begin{center}\rule{3in}{0.4pt}\end{center}

\begin{center}\rule{3in}{0.4pt}\end{center}

\begin{center}\rule{3in}{0.4pt}\end{center}

 x^2 \over a^2 +
y^2 \over b^2 + z^2
\over c^2 = 1

ellipsoïde

Non

si a = b ou b = c ou c = a

 x^2 \over a^2 +
y^2 \over b^2 - z^2
\over c^2 = 1

hyperboloïde à une nappe

Doublement

si a = b

 x^2 \over a^2 +
y^2 \over b^2 - z^2
\over c^2 = 0

cône du second degré

Oui

si a = b

 x^2 \over a^2 +
y^2 \over b^2 - z^2
\over c^2 = -1

hyperboloïde à deux nappes

Non

si a = b

 x^2 \over a^2 +
y^2 \over b^2 = 1

cylindre elliptique

Oui

si a = b

 x^2 \over a^2 -
y^2 \over b^2 = 1

cylindre hyperbolique

Oui

Non

z = x^2 \over a^2 +
y^2 \over b^2

paraboloïde elliptique

Non

si a = b

z = x^2 \over a^2 -
y^2 \over b^2

paraboloïde hyperbolique

Doublement

Non

2py = x^2

cylindre parabolique

Oui

Non

\begin{center}\rule{3in}{0.4pt}\end{center}

\begin{center}\rule{3in}{0.4pt}\end{center}

\begin{center}\rule{3in}{0.4pt}\end{center}

\begin{center}\rule{3in}{0.4pt}\end{center}

{[}
{[}
{[}
{[}

\end{document}

\part{Intégrales curvilignes, intégrales multiples}
% \documentclass[]{article}
\usepackage[T1]{fontenc}
\usepackage{lmodern}
\usepackage{amssymb,amsmath}
\usepackage{ifxetex,ifluatex}
\usepackage{fixltx2e} % provides \textsubscript
% use upquote if available, for straight quotes in verbatim environments
\IfFileExists{upquote.sty}{\usepackage{upquote}}{}
\ifnum 0\ifxetex 1\fi\ifluatex 1\fi=0 % if pdftex
  \usepackage[utf8]{inputenc}
\else % if luatex or xelatex
  \ifxetex
    \usepackage{mathspec}
    \usepackage{xltxtra,xunicode}
  \else
    \usepackage{fontspec}
  \fi
  \defaultfontfeatures{Mapping=tex-text,Scale=MatchLowercase}
  \newcommand{\euro}{€}
\fi
% use microtype if available
\IfFileExists{microtype.sty}{\usepackage{microtype}}{}
\ifxetex
  \usepackage[setpagesize=false, % page size defined by xetex
              unicode=false, % unicode breaks when used with xetex
              xetex]{hyperref}
\else
  \usepackage[unicode=true]{hyperref}
\fi
\hypersetup{breaklinks=true,
            bookmarks=true,
            pdfauthor={},
            pdftitle={Integrales curvilignes},
            colorlinks=true,
            citecolor=blue,
            urlcolor=blue,
            linkcolor=magenta,
            pdfborder={0 0 0}}
\urlstyle{same}  % don't use monospace font for urls
\setlength{\parindent}{0pt}
\setlength{\parskip}{6pt plus 2pt minus 1pt}
\setlength{\emergencystretch}{3em}  % prevent overfull lines
\setcounter{secnumdepth}{0}
 
/* start css.sty */
.cmr-5{font-size:50%;}
.cmr-7{font-size:70%;}
.cmmi-5{font-size:50%;font-style: italic;}
.cmmi-7{font-size:70%;font-style: italic;}
.cmmi-10{font-style: italic;}
.cmsy-5{font-size:50%;}
.cmsy-7{font-size:70%;}
.cmex-7{font-size:70%;}
.cmex-7x-x-71{font-size:49%;}
.msbm-7{font-size:70%;}
.cmtt-10{font-family: monospace;}
.cmti-10{ font-style: italic;}
.cmbx-10{ font-weight: bold;}
.cmr-17x-x-120{font-size:204%;}
.cmsl-10{font-style: oblique;}
.cmti-7x-x-71{font-size:49%; font-style: italic;}
.cmbxti-10{ font-weight: bold; font-style: italic;}
p.noindent { text-indent: 0em }
td p.noindent { text-indent: 0em; margin-top:0em; }
p.nopar { text-indent: 0em; }
p.indent{ text-indent: 1.5em }
@media print {div.crosslinks {visibility:hidden;}}
a img { border-top: 0; border-left: 0; border-right: 0; }
center { margin-top:1em; margin-bottom:1em; }
td center { margin-top:0em; margin-bottom:0em; }
.Canvas { position:relative; }
li p.indent { text-indent: 0em }
.enumerate1 {list-style-type:decimal;}
.enumerate2 {list-style-type:lower-alpha;}
.enumerate3 {list-style-type:lower-roman;}
.enumerate4 {list-style-type:upper-alpha;}
div.newtheorem { margin-bottom: 2em; margin-top: 2em;}
.obeylines-h,.obeylines-v {white-space: nowrap; }
div.obeylines-v p { margin-top:0; margin-bottom:0; }
.overline{ text-decoration:overline; }
.overline img{ border-top: 1px solid black; }
td.displaylines {text-align:center; white-space:nowrap;}
.centerline {text-align:center;}
.rightline {text-align:right;}
div.verbatim {font-family: monospace; white-space: nowrap; text-align:left; clear:both; }
.fbox {padding-left:3.0pt; padding-right:3.0pt; text-indent:0pt; border:solid black 0.4pt; }
div.fbox {display:table}
div.center div.fbox {text-align:center; clear:both; padding-left:3.0pt; padding-right:3.0pt; text-indent:0pt; border:solid black 0.4pt; }
div.minipage{width:100%;}
div.center, div.center div.center {text-align: center; margin-left:1em; margin-right:1em;}
div.center div {text-align: left;}
div.flushright, div.flushright div.flushright {text-align: right;}
div.flushright div {text-align: left;}
div.flushleft {text-align: left;}
.underline{ text-decoration:underline; }
.underline img{ border-bottom: 1px solid black; margin-bottom:1pt; }
.framebox-c, .framebox-l, .framebox-r { padding-left:3.0pt; padding-right:3.0pt; text-indent:0pt; border:solid black 0.4pt; }
.framebox-c {text-align:center;}
.framebox-l {text-align:left;}
.framebox-r {text-align:right;}
span.thank-mark{ vertical-align: super }
span.footnote-mark sup.textsuperscript, span.footnote-mark a sup.textsuperscript{ font-size:80%; }
div.tabular, div.center div.tabular {text-align: center; margin-top:0.5em; margin-bottom:0.5em; }
table.tabular td p{margin-top:0em;}
table.tabular {margin-left: auto; margin-right: auto;}
div.td00{ margin-left:0pt; margin-right:0pt; }
div.td01{ margin-left:0pt; margin-right:5pt; }
div.td10{ margin-left:5pt; margin-right:0pt; }
div.td11{ margin-left:5pt; margin-right:5pt; }
table[rules] {border-left:solid black 0.4pt; border-right:solid black 0.4pt; }
td.td00{ padding-left:0pt; padding-right:0pt; }
td.td01{ padding-left:0pt; padding-right:5pt; }
td.td10{ padding-left:5pt; padding-right:0pt; }
td.td11{ padding-left:5pt; padding-right:5pt; }
table[rules] {border-left:solid black 0.4pt; border-right:solid black 0.4pt; }
.hline hr, .cline hr{ height : 1px; margin:0px; }
.tabbing-right {text-align:right;}
span.TEX {letter-spacing: -0.125em; }
span.TEX span.E{ position:relative;top:0.5ex;left:-0.0417em;}
a span.TEX span.E {text-decoration: none; }
span.LATEX span.A{ position:relative; top:-0.5ex; left:-0.4em; font-size:85%;}
span.LATEX span.TEX{ position:relative; left: -0.4em; }
div.float img, div.float .caption {text-align:center;}
div.figure img, div.figure .caption {text-align:center;}
.marginpar {width:20%; float:right; text-align:left; margin-left:auto; margin-top:0.5em; font-size:85%; text-decoration:underline;}
.marginpar p{margin-top:0.4em; margin-bottom:0.4em;}
.equation td{text-align:center; vertical-align:middle; }
td.eq-no{ width:5%; }
table.equation { width:100%; } 
div.math-display, div.par-math-display{text-align:center;}
math .texttt { font-family: monospace; }
math .textit { font-style: italic; }
math .textsl { font-style: oblique; }
math .textsf { font-family: sans-serif; }
math .textbf { font-weight: bold; }
.partToc a, .partToc, .likepartToc a, .likepartToc {line-height: 200%; font-weight:bold; font-size:110%;}
.chapterToc a, .chapterToc, .likechapterToc a, .likechapterToc, .appendixToc a, .appendixToc {line-height: 200%; font-weight:bold;}
.index-item, .index-subitem, .index-subsubitem {display:block}
.caption td.id{font-weight: bold; white-space: nowrap; }
table.caption {text-align:center;}
h1.partHead{text-align: center}
p.bibitem { text-indent: -2em; margin-left: 2em; margin-top:0.6em; margin-bottom:0.6em; }
p.bibitem-p { text-indent: 0em; margin-left: 2em; margin-top:0.6em; margin-bottom:0.6em; }
.paragraphHead, .likeparagraphHead { margin-top:2em; font-weight: bold;}
.subparagraphHead, .likesubparagraphHead { font-weight: bold;}
.quote {margin-bottom:0.25em; margin-top:0.25em; margin-left:1em; margin-right:1em; text-align:\\jmathmathustify;}
.verse{white-space:nowrap; margin-left:2em}
div.maketitle {text-align:center;}
h2.titleHead{text-align:center;}
div.maketitle{ margin-bottom: 2em; }
div.author, div.date {text-align:center;}
div.thanks{text-align:left; margin-left:10%; font-size:85%; font-style:italic; }
div.author{white-space: nowrap;}
.quotation {margin-bottom:0.25em; margin-top:0.25em; margin-left:1em; }
h1.partHead{text-align: center}
.sectionToc, .likesectionToc {margin-left:2em;}
.subsectionToc, .likesubsectionToc {margin-left:4em;}
.subsubsectionToc, .likesubsubsectionToc {margin-left:6em;}
.frenchb-nbsp{font-size:75%;}
.frenchb-thinspace{font-size:75%;}
.figure img.graphics {margin-left:10%;}
/* end css.sty */

\title{Integrales curvilignes}
\author{}
\date{}

\begin{document}
\maketitle

\textbf{Warning: 
requires JavaScript to process the mathematics on this page.\\ If your
browser supports JavaScript, be sure it is enabled.}

\begin{center}\rule{3in}{0.4pt}\end{center}

{[}
{[}{]}
{[}

\subsubsection{20.1 Intégrales curvilignes}

\paragraph{20.1.1 Formes différentielles sur un arc paramétré}

Définition~20.1.1 Soit E un espace vectoriel normé, \Gamma = (I,f) un arc
paramétré de E de classe \mathcal{C}^1. On appelle forme différentielle
sur \Gamma toute forme différentielle \alpha~ = a(t) dt sur l'intervalle I.

Exemple~20.1.1 Soit \Gamma = (I,f) un arc paramétré de E

\begin{itemize}
\itemsep1pt\parskip0pt\parsep0pt
\item
  (i) soit h une fonction définie sur l'image de \Gamma et à valeurs dans K~;
  on peut associer à h la forme différentielle h(m) ds définie par h(m)
  ds = h(f(t)) \f'(t)\
  dt (obtenue en rempla\ccant m par f(t) et ds par
  \f'(t)\ dt, la
  différentielle de l'abscisse curviligne).
\item
  (ii) supposons que E est un espace euclidien et soit V un champ de
  vecteurs défini et continu sur l'image de \Gamma (c'est-à-dire une
  application de l'image de \Gamma dans E)~; on peut associer à V la forme
  différentielle (V (m)∣dm) définie par (V
  (m)∣dm) = \left (V
  (f(t))∣f'(t)\right ) dt
  (obtenue en rempla\ccant m par f(t) et donc dm par
  f'(t) dt).
\item
  (iii) supposons que E = \mathbb{R}~^n, que U est un ouvert de E
  contenant l'image de \Gamma et \omega = a_1(x) dx_1 +
  \\ldots~ +
  a_n(x) dx_n une forme différentielle de degré 1
  continue sur U~; posons f(t) =
  (f_1(t),\\ldots,f_n~(t))~;
  on peut considérer la forme différentielle restriction de \omega à \Gamma
  définie par \omega__\Gamma = \left
  (a_1(f(t))f_1'(t) +
  \\ldots~ +
  a_n(f(t))f_n'(t)\right ) dt (obtenue
  en rempla\ccant dans \omega, x par f(t), x_i
  par f_i(t) et donc dx_i par f_i'(t) dt).
\end{itemize}

Remarque~20.1.1 En fait les cas (ii) et (iii) sont étroitement liés. En
effet, munissons \mathbb{R}~^n de sa structure euclidienne canonique et
soit U un ouvert contenant l'image de \Gamma. A tout champ de vecteurs V
défini sur U défini par V (x) = (V
_1(x),\\ldots~,V
_n(x)), on peut associer la forme différentielle \omega = V
_1(x) dx_1 +
\\ldots~ + V
_n(x) dx_n. Cette application V
\mapsto~\omega est clairement bi\\jmathmathective. On a alors dans
ce cadre

\begin{align*} (V
(m)∣dm)& =& \left (V
(f(t))∣f'(t)\right ) dt \%&
\\ & =& \left
(a_1(f(t))f_1'(t) +
\\ldots~ +
a_n(f(t))f_n'(t)\right ) dt\%&
\\ & =& \omega__\Gamma
\%& \\ \end{align*}

Théorème~20.1.1 Les trois exemples fondamentaux sont invariants par
changement de paramétrage de sens direct.

Soit (I,f) et (J,g) deux arcs paramétrés équivalents et de même sens.
Soit \theta : I \rightarrow~ J un difféomorphisme croissant de classe \mathcal{C}^1 tel
que f = g \cdot \theta. Si \alpha~ = a(t) dt et \beta~ = b(u) du sont les formes
différentielles obtenues respectivement sur (I,f) et (J,g) par l'une des
trois constructions ci dessus, on a a(t) dt = b(u) du pour u = \theta(t).

Démonstration (i) Sur (I,f), on a h(m) ds = h(f(t))
\f'(t)\ dt~; mais
comme f = g \cdot \theta, on a

\begin{align*} h(m) ds& =& h(g(\theta(t)))
\\theta'(t)g'(\theta(t))\ dt\%&
\\ & =& h(g(\theta(t)))
\g'(\theta(t))\\theta'(t) dt\%&
\\ \end{align*}

car \theta'(t) \textgreater{} 0~; d'où encore, en posant u = \theta(t) et donc du
= \theta'(t) dt, h(m) ds = h(g(u))
\g'(u)\ du ce qu'on
voulait démontrer.

(ii) Sur (I,f), on a

\begin{align*} (V
(m)∣dm)& =& \left (V
(f(t))∣f'(t)\right ) dt \%&
\\ & =& \left (V
(g(\theta(t))∣\theta'(t)g'(\theta(t))\right )
dt\%& \\ & =& \left (V
(g(\theta(t))∣g'(\theta(t))\right )\theta'(t)
dt\%& \\ & =& \left (V
(g(u))∣g'(u)\right ) du \%&
\\ \end{align*}

ce qu'on voulait démontrer.

(iii) On peut faire un calcul similaire ou utiliser le lien entre formes
différentielles et champ de vecteurs décrit ci dessus.

\paragraph{20.1.2 Intégrale d'une forme différentielle sur un arc}

Définition~20.1.2 Soit \Gamma = ({[}a,b{]},f) un arc paramétré et \alpha~ = A(t) dt
une forme différentielle continue par morceaux sur \Gamma. On appelle
intégrale (curviligne) de la forme différentielle \alpha~ sur \Gamma le scalaire

\int  _\Gamma~\alpha~ =\\int
 _a^bA(t) dt

Remarque~20.1.2 Soit \Gamma = ({[}a,b{]},f) un arc paramétré et c \in
{[}a,b{]}. On peut alors considérer les deux arcs paramétrés
\Gamma_1 = ({[}a,c{]},f__{[}a,c{]}) et
\Gamma_2 = ({[}c,b{]},f__{[}c,b{]}). On
dira alors que \Gamma est la \\jmathmathuxtaposition de \Gamma_1 et \Gamma_2 et
on écrira \Gamma = \Gamma_1 ⊔ \Gamma_2.

Proposition~20.1.2 (i) L'application
\alpha~\mapsto~\int  _\Gamma~\alpha~
est linéaire (ii) On a \int ~
_\Gamma_1⊔\Gamma_2\alpha~ =\int ~
_\Gamma_1\alpha~ +\int ~
_\Gamma_2\alpha~

Démonstration Résulte immédiatement des propriétés de l'intégrale.

Théorème~20.1.3 (invariance de l'intégrale curviligne). Soit
\Gamma_1 = ({[}a,b{]},f) un arc paramétré, \Gamma_2 =
({[}c,d{]},g) un arc paramétré équivalent et de même sens. Soit \theta un
difféomorphisme croissant de {[}a,b{]} sur {[}c,d{]} tel que f = g \cdot \theta.
Soit \alpha~ = A(t) dt une forme différentielle sur \Gamma_1 et \beta~ = B(u)
du la forme différentielle qui s'en déduit en posant t = \theta(u). Alors

\int  _\Gamma_1~\alpha~
=\int  _\Gamma_2~\beta~

Démonstration Comme \theta est croissant, on a nécessairement c = \theta(a) et d =
\theta(b). De plus la relation A(t) dt = B(u) du pour u = \theta(t), montre que
A(t) = B(\theta(t))\theta'(t). On a donc

\begin{align*} \int ~
_\Gamma_1\alpha~& =& \int ~
_a^bA(t) dt =\int ~
_a^bB(\theta(t))\theta'(t) dt\%& \\
& =& \int  _\theta(a)^\theta(b)~B(u) du
=\int  _\Gamma_2~\beta~ \%&
\\ \end{align*}

en utilisant le théorème de changement de variable dans les intégrales.

Exemple~20.1.2 Les résultats précédents, en liaison avec les définitions
du paragraphe précédent nous permettent d'associer à un arc paramétré \Gamma
= ({[}a,b{]},f) les trois types suivants d'intégrales, tous trois
invariants par changement de paramétrage admissible et croissant

\begin{itemize}
\item
  (i) soit h une fonction définie et continue sur l'image de \Gamma et à
  valeurs dans K~; on peut associer à h l'intégrale curviligne

  \int  _\Gamma~h(m) ds
  =\int  _a^b~h(f(t))
  \f'(t)\ dt

  (obtenue en rempla\ccant m par f(t) et ds par
  \f'(t)\ dt,
  différentielle de l'abscisse curviligne).
\item
  (ii) supposons que E est un espace euclidien et soit V un champ de
  vecteurs défini sur l'image de \Gamma (c'est-à-dire une application de
  l'image de \Gamma dans E)~; on peut associer à V l'intégrale curviligne
  (appelée circulation du champ de vecteurs V le long de \Gamma)

  \int  _\Gamma~(V
  (m)∣dm) =\int ~
  _a^b\left (V
  (f(t))∣f'(t)\right ) dt

  (obtenue en rempla\ccant m par f(t) et donc dm par
  f'(t) dt).
\item
  (iii) supposons que E = \mathbb{R}~^n, que U est un ouvert de E
  contenant l'image de \Gamma et \omega = a_1(x) dx_1 +
  \\ldots~ +
  a_n(x) dx_n une forme différentielle de degré 1
  continue sur U~; posons f(t) =
  (f_1(t),\\ldots,f_n~(t))~;
  on peut considérer l'intégrale curviligne

  \begin{align*} \int ~
  _\Gammaa_1(x) dx_1 +
  \\ldots~ +
  a_n(x) dx_n& & \%&
  \\ & =& \int ~
  _a^b\left (a_
  1(f(t))f_1'(t) +
  \\ldots~ +
  a_n(f(t))f_n'(t)\right ) dt\%&
  \\ \end{align*}

  (obtenue en rempla\ccant dans \omega, x par f(t),
  x_i par f_i(t) et donc dx_i par
  f_i'(t) dt).
\end{itemize}

Remarque~20.1.3 Le lecteur vérifiera facilement que le premier type
d'intégrale est également invariant par changement d'orientation de \Gamma,
c'est-à-dire par un changement de paramétrage décroissant (car la forme
différentielle est changée en son opposée mais dans le même temps les
bornes de l'intégrale sont interverties)~; par contre les intégrales des
deux autres types sont changées en leurs opposées par changement
d'orientation.

Proposition~20.1.4 Soit \Gamma = ({[}a,b{]},f) un arc paramétré de classe
\mathcal{C}^1 de longueur l(\Gamma).

\begin{itemize}
\item
  (i) soit h une fonction continue bornée définie sur l'image de \Gamma et à
  valeurs dans K~; alors

  \left \int ~
  _\Gammah(m) ds\right  \leq
  l(\Gamma)sup_m\in\\mathrmIm~
  \Gammah(m)
\item
  (ii) supposons que E est un espace euclidien et soit V un champ de
  vecteurs continu défini sur l'image de \Gamma et borné~; alors

  \left \int ~
  _\Gamma(V (m)∣dm)\right
   \leq
  l(\Gamma)sup_m\in\\mathrmIm~
  \Gamma\V (m)\
\end{itemize}

Démonstration (i) On a

\begin{align*} \left
\int  _\Gamma~h(m)
ds\right & =& \left
\int  _a^b~h(f(t))
\f'(t)\
dt\right  \%& \\
& \leq& \int ~
_a^bh(f(t))\f'(t)\
dt \%& \\ & \leq& \left
(sup_m\in\\mathrmIm~
\Gammah(m)\right
)\int ~
_a^b\f'(t)\
dt\%& \\ & =&
l(\Gamma)sup_m\in\\mathrmIm~
\Gammah(m) \%& \\
\end{align*}

(ii) On a grâce à l'inégalité de Schwarz, \left
\left (V
(f(t))∣f'(t)\right
)\right  \leq\ V
(f(t))\
\f'(t)\ d'où

\begin{align*} \left
\int  _\Gamma~(V
(m)∣dm)\right & =&
\left \int ~
_a^b\left (V
(f(t))∣f'(t)\right )
dt\right  \%& \\
& \leq& \int ~
_a^b\V
(f(t))\
\f'(t)\ dt \%&
\\ & \leq& \left
(sup_m\in\\mathrmIm~
\Gamma\V
(m)\\right
)\int ~
_a^b\f'(t)\
dt\%& \\ & =&
l(\Gamma)sup_m\in\\mathrmIm~
\Gamma\V (m)\ \%&
\\ \end{align*}

\paragraph{20.1.3 Formes différentielles exactes et champs de gradients}

Théorème~20.1.5

\begin{itemize}
\item
  (i) Soit \Gamma = ({[}a,b{]},f) un arc paramétré de classe \mathcal{C}^1
  et soit V un champ de vecteurs défini et continu sur un ouvert U
  contenant l'image de \Gamma~; si V est le champ des gradients d'une
  fonction F : U \rightarrow~ \mathbb{R}~, alors

  \int  _\Gamma~(V
  (m)∣dm) = F(f(b)) - F(f(a))
\item
  (ii) Soit \Gamma = ({[}a,b{]},f) un arc paramétré de classe \mathcal{C}^1
  et soit \omega une forme différentielle définie et continue sur un ouvert U
  contenant l'image de \Gamma~; si \omega est la différentielle d'une fonction F :
  U \rightarrow~ \mathbb{R}~, alors

  \int  _\Gamma~\omega = F(f(b)) - F(f(a))
\end{itemize}

Démonstration (i) On a en effet

\begin{align*} d \over dt
(F(f(t)))& =& dF(f(t)).f'(t) = \left
((\mathrmgrad~
F)(f(t))∣f'(t)\right )\%&
\\ & =& (V
(f(t))∣f'(t)) \%&
\\ \end{align*}

d'où l'on déduit

\begin{align*} \int ~
_\Gamma(V (m)∣dm)& =&
\int  _a^b~(V
(f(t))∣f'(t)) =\int ~
_a^b(F \cdot f)'(t) dt\%& \\ &
=& F(f(b)) - F(f(a)) \%& \\
\end{align*}

(ii) Si \omega = dF, on a donc \omega = \partial~F \over \partial~x_1
(x) dx_1 +
\\ldots~ + \partial~F
\over \partial~x_n (x) dx_n, si bien que

\begin{align*} \int ~
_\Gamma\omega& =& \int  _a^b~( \partial~F
\over \partial~x_1 (f(t))f_1'(t) +
\\ldots~ + \partial~F
\over \partial~x_n (f(t))f_n'(t)) dt \%&
\\ & =& \int ~
_a^b d \over dt
(F(f_1(t),\\ldots,f_n~(t)))
dt = F(f(b)) - F(f(a))\%& \\
\end{align*}

Corollaire~20.1.6 Soit \Gamma = ({[}a,b{]},f) un arc paramétré de classe
\mathcal{C}^1, fermé (c'est-à-dire que f(b) = f(a)).

\begin{itemize}
\itemsep1pt\parskip0pt\parsep0pt
\item
  (i) Pour tout champ de gradients V sur un ouvert U de E contenant
  \mathrmIm~\Gamma, on a
  \int  _\Gamma~(V
  (m)∣dm) = 0
\item
  (ii) Pour toute forme différentielle exacte \omega sur un ouvert U de E
  contenant \mathrmIm~\Gamma, on
  a \int  _\Gamma~\omega = 0
\end{itemize}

Démonstration Conséquence évidente du résultat précédent.

Exemple~20.1.3 Considérons sur \mathbb{R}~^2
\diagdown\(0,0\ la forme différentielle de
classe C^\infty~, \omega = x dy-y dx \over
x^2+y^2 ~; on vérifie facilement que d\omega = 0
puisque  \partial~ \over \partial~y \left ( -y
\over x^2+y^2
\right ) = \partial~ \over \partial~x
\left ( x \over
x^2+y^2 \right ). Pourtant \omega
n'est pas exacte. En effet calculons l'intégrale de \omega le long du cercle
\Gamma de centre (0,0) de rayon 1. On a en posant x =\
cos \theta et y = sin~ \theta,

x dy - y dx = cos~ \theta \times
(cos \theta d\theta) -\ sin~ \theta \times
(-sin~ \theta d\theta) = d\theta

si bien que

\int  _\Gamma~\omega =\\int
 _0^2\pi~d\theta = 2\pi~\neq~0

ce qui montre que \omega ne peut pas être la différentielle d'une fonction.
L'hypothèse que l'ouvert est étoilé est donc essentielle pour la
validité du théorème de Poincaré qui dit que (sur un ouvert étoilé) une
forme différentielle est exacte si et seulement si~elle vérifie d\omega = 0.

{[}
{[}

\end{document}

% \documentclass[]{article}
\usepackage[T1]{fontenc}
\usepackage{lmodern}
\usepackage{amssymb,amsmath}
\usepackage{ifxetex,ifluatex}
\usepackage{fixltx2e} % provides \textsubscript
% use upquote if available, for straight quotes in verbatim environments
\IfFileExists{upquote.sty}{\usepackage{upquote}}{}
\ifnum 0\ifxetex 1\fi\ifluatex 1\fi=0 % if pdftex
  \usepackage[utf8]{inputenc}
\else % if luatex or xelatex
  \ifxetex
    \usepackage{mathspec}
    \usepackage{xltxtra,xunicode}
  \else
    \usepackage{fontspec}
  \fi
  \defaultfontfeatures{Mapping=tex-text,Scale=MatchLowercase}
  \newcommand{\euro}{€}
\fi
% use microtype if available
\IfFileExists{microtype.sty}{\usepackage{microtype}}{}
\ifxetex
  \usepackage[setpagesize=false, % page size defined by xetex
              unicode=false, % unicode breaks when used with xetex
              xetex]{hyperref}
\else
  \usepackage[unicode=true]{hyperref}
\fi
\hypersetup{breaklinks=true,
            bookmarks=true,
            pdfauthor={},
            pdftitle={Integrales multiples},
            colorlinks=true,
            citecolor=blue,
            urlcolor=blue,
            linkcolor=magenta,
            pdfborder={0 0 0}}
\urlstyle{same}  % don't use monospace font for urls
\setlength{\parindent}{0pt}
\setlength{\parskip}{6pt plus 2pt minus 1pt}
\setlength{\emergencystretch}{3em}  % prevent overfull lines
\setcounter{secnumdepth}{0}
 
/* start css.sty */
.cmr-5{font-size:50%;}
.cmr-7{font-size:70%;}
.cmmi-5{font-size:50%;font-style: italic;}
.cmmi-7{font-size:70%;font-style: italic;}
.cmmi-10{font-style: italic;}
.cmsy-5{font-size:50%;}
.cmsy-7{font-size:70%;}
.cmex-7{font-size:70%;}
.cmex-7x-x-71{font-size:49%;}
.msbm-7{font-size:70%;}
.cmtt-10{font-family: monospace;}
.cmti-10{ font-style: italic;}
.cmbx-10{ font-weight: bold;}
.cmr-17x-x-120{font-size:204%;}
.cmsl-10{font-style: oblique;}
.cmti-7x-x-71{font-size:49%; font-style: italic;}
.cmbxti-10{ font-weight: bold; font-style: italic;}
p.noindent { text-indent: 0em }
td p.noindent { text-indent: 0em; margin-top:0em; }
p.nopar { text-indent: 0em; }
p.indent{ text-indent: 1.5em }
@media print {div.crosslinks {visibility:hidden;}}
a img { border-top: 0; border-left: 0; border-right: 0; }
center { margin-top:1em; margin-bottom:1em; }
td center { margin-top:0em; margin-bottom:0em; }
.Canvas { position:relative; }
li p.indent { text-indent: 0em }
.enumerate1 {list-style-type:decimal;}
.enumerate2 {list-style-type:lower-alpha;}
.enumerate3 {list-style-type:lower-roman;}
.enumerate4 {list-style-type:upper-alpha;}
div.newtheorem { margin-bottom: 2em; margin-top: 2em;}
.obeylines-h,.obeylines-v {white-space: nowrap; }
div.obeylines-v p { margin-top:0; margin-bottom:0; }
.overline{ text-decoration:overline; }
.overline img{ border-top: 1px solid black; }
td.displaylines {text-align:center; white-space:nowrap;}
.centerline {text-align:center;}
.rightline {text-align:right;}
div.verbatim {font-family: monospace; white-space: nowrap; text-align:left; clear:both; }
.fbox {padding-left:3.0pt; padding-right:3.0pt; text-indent:0pt; border:solid black 0.4pt; }
div.fbox {display:table}
div.center div.fbox {text-align:center; clear:both; padding-left:3.0pt; padding-right:3.0pt; text-indent:0pt; border:solid black 0.4pt; }
div.minipage{width:100%;}
div.center, div.center div.center {text-align: center; margin-left:1em; margin-right:1em;}
div.center div {text-align: left;}
div.flushright, div.flushright div.flushright {text-align: right;}
div.flushright div {text-align: left;}
div.flushleft {text-align: left;}
.underline{ text-decoration:underline; }
.underline img{ border-bottom: 1px solid black; margin-bottom:1pt; }
.framebox-c, .framebox-l, .framebox-r { padding-left:3.0pt; padding-right:3.0pt; text-indent:0pt; border:solid black 0.4pt; }
.framebox-c {text-align:center;}
.framebox-l {text-align:left;}
.framebox-r {text-align:right;}
span.thank-mark{ vertical-align: super }
span.footnote-mark sup.textsuperscript, span.footnote-mark a sup.textsuperscript{ font-size:80%; }
div.tabular, div.center div.tabular {text-align: center; margin-top:0.5em; margin-bottom:0.5em; }
table.tabular td p{margin-top:0em;}
table.tabular {margin-left: auto; margin-right: auto;}
div.td00{ margin-left:0pt; margin-right:0pt; }
div.td01{ margin-left:0pt; margin-right:5pt; }
div.td10{ margin-left:5pt; margin-right:0pt; }
div.td11{ margin-left:5pt; margin-right:5pt; }
table[rules] {border-left:solid black 0.4pt; border-right:solid black 0.4pt; }
td.td00{ padding-left:0pt; padding-right:0pt; }
td.td01{ padding-left:0pt; padding-right:5pt; }
td.td10{ padding-left:5pt; padding-right:0pt; }
td.td11{ padding-left:5pt; padding-right:5pt; }
table[rules] {border-left:solid black 0.4pt; border-right:solid black 0.4pt; }
.hline hr, .cline hr{ height : 1px; margin:0px; }
.tabbing-right {text-align:right;}
span.TEX {letter-spacing: -0.125em; }
span.TEX span.E{ position:relative;top:0.5ex;left:-0.0417em;}
a span.TEX span.E {text-decoration: none; }
span.LATEX span.A{ position:relative; top:-0.5ex; left:-0.4em; font-size:85%;}
span.LATEX span.TEX{ position:relative; left: -0.4em; }
div.float img, div.float .caption {text-align:center;}
div.figure img, div.figure .caption {text-align:center;}
.marginpar {width:20%; float:right; text-align:left; margin-left:auto; margin-top:0.5em; font-size:85%; text-decoration:underline;}
.marginpar p{margin-top:0.4em; margin-bottom:0.4em;}
.equation td{text-align:center; vertical-align:middle; }
td.eq-no{ width:5%; }
table.equation { width:100%; } 
div.math-display, div.par-math-display{text-align:center;}
math .texttt { font-family: monospace; }
math .textit { font-style: italic; }
math .textsl { font-style: oblique; }
math .textsf { font-family: sans-serif; }
math .textbf { font-weight: bold; }
.partToc a, .partToc, .likepartToc a, .likepartToc {line-height: 200%; font-weight:bold; font-size:110%;}
.chapterToc a, .chapterToc, .likechapterToc a, .likechapterToc, .appendixToc a, .appendixToc {line-height: 200%; font-weight:bold;}
.index-item, .index-subitem, .index-subsubitem {display:block}
.caption td.id{font-weight: bold; white-space: nowrap; }
table.caption {text-align:center;}
h1.partHead{text-align: center}
p.bibitem { text-indent: -2em; margin-left: 2em; margin-top:0.6em; margin-bottom:0.6em; }
p.bibitem-p { text-indent: 0em; margin-left: 2em; margin-top:0.6em; margin-bottom:0.6em; }
.paragraphHead, .likeparagraphHead { margin-top:2em; font-weight: bold;}
.subparagraphHead, .likesubparagraphHead { font-weight: bold;}
.quote {margin-bottom:0.25em; margin-top:0.25em; margin-left:1em; margin-right:1em; text-align:\\jmathmathustify;}
.verse{white-space:nowrap; margin-left:2em}
div.maketitle {text-align:center;}
h2.titleHead{text-align:center;}
div.maketitle{ margin-bottom: 2em; }
div.author, div.date {text-align:center;}
div.thanks{text-align:left; margin-left:10%; font-size:85%; font-style:italic; }
div.author{white-space: nowrap;}
.quotation {margin-bottom:0.25em; margin-top:0.25em; margin-left:1em; }
h1.partHead{text-align: center}
.sectionToc, .likesectionToc {margin-left:2em;}
.subsectionToc, .likesubsectionToc {margin-left:4em;}
.subsubsectionToc, .likesubsubsectionToc {margin-left:6em;}
.frenchb-nbsp{font-size:75%;}
.frenchb-thinspace{font-size:75%;}
.figure img.graphics {margin-left:10%;}
/* end css.sty */

\title{Integrales multiples}
\author{}
\date{}

\begin{document}
\maketitle

\textbf{Warning: 
requires JavaScript to process the mathematics on this page.\\ If your
browser supports JavaScript, be sure it is enabled.}

\begin{center}\rule{3in}{0.4pt}\end{center}

{[}
{[}
{[}{]}
{[}

\subsubsection{20.2 Intégrales multiples}

\paragraph{20.2.1 Pavés et subdivisions. Ensembles négligeables}

Définition~20.2.1 On appelle pavé de \mathbb{R}~^n tout ensemble P de
la forme P = {[}a_1,b_1{]}
\times⋯ \times {[}a_n,b_n{]}. On
notera mesure du pavé P le nombre réel positif m(P)
= \∏ ~
_i=1^n(b_i - a_i).

Remarque~20.2.1 Un pavé est clairement compact.

Définition~20.2.2 Soit P = {[}a_1,b_1{]}
\times⋯ \times {[}a_n,b_n{]} un pavé
de \mathbb{R}~^n. On appelle subdivision de P toute famille \sigma =
(\sigma_1,\\ldots,\sigma_n~)
où chaque \sigma_i est une subdivision de
{[}a_i,b_i{]}. Si \sigma_i =
(a_i,\\jmathmath)_1\leq\\jmathmath\leqn_i, les sous pavés
P_\\jmathmath_1,\\ldots,\\jmathmath_n~
= {[}a_1,\\jmathmath_1-1,a_1,\\jmathmath_1{]}
\times⋯ \times
{[}a_n,\\jmathmath_n-1,a_n,\\jmathmath_n{]} sont appelés
les sous pavés de la subdivision. On appelle pas de la subdivision \sigma =
(\sigma_1,\\ldots,\sigma_n~)
le plus grand des diamètres des sous pavés de \sigma.

Définition~20.2.3 Un sous-ensemble A de \mathbb{R}~^n est dit
négligeable (au sens de Riemann) si quelque soit \epsilon \textgreater{} 0, il
existe une famille finie de pavés (P_i)_1\leqi\leqN
vérifiant

\begin{itemize}
\itemsep1pt\parskip0pt\parsep0pt
\item
  (i) A \subset~\⋃ ~
  _i=1^NP_i
\item
  (ii) \\sum ~
  _i=1^Nm(P_i) \leq \epsilon.
\end{itemize}

Proposition~20.2.1

\begin{itemize}
\itemsep1pt\parskip0pt\parsep0pt
\item
  (i) tout ensemble négligeable est borné
\item
  (ii) si A est négligeable et B \subset~ A, alors B est aussi négligeable
\item
  (iii) une partie A est négligeable si et seulement
  si~\overlineA est négligeable
\item
  (iv) toute réunion finie de parties négligeables est négligeable
\end{itemize}

Démonstration Tout est à peu près évident. L'affirmation (iii) résulte
de ce que, si A \subset~\\⋃
 _i=1^NP_i, alors on a aussi
\overlineA
\subset~\⋃ ~
_i=1^NP_i puisque
\⋃ ~
_i=1^NP_i est fermé.

Théorème~20.2.2 Soit Q un pavé de \mathbb{R}~^n-1 et f une application
continue de Q dans \mathbb{R}~. Alors le graphe de f est une partie négligeable de
\mathbb{R}~^n.

Démonstration Puisque f est continue sur le compact Q, elle est
uniformément continue. Soit donc \epsilon \textgreater{} 0. Il existe \eta
\textgreater{} 0 tel que \forall~~x,x' \in Q,
\x - x'\ \textless{} \eta
\rigtharrow~f(x) - f(x') \textless{} \epsilon \over
2m(Q) . Soit alors \sigma une subdivision de Q de pas inférieur strictement
à \eta et (Q_i)_1\leqi\leqN les sous pavés de la subdivision.
Choisissons un point x_i dans chaque Q_i et posons
P_i = Q_i \times {[}f(x_i) - \epsilon
\over 2m(Q) ,f(x_i) + \epsilon \over
2m(Q) {]}. Chaque P_i est un pavé de \mathbb{R}~^n et
m(P_i) = \epsilon \over m(Q) m(Q_i) si
bien que \\sum ~
m(P_i) = \epsilon \over m(Q)
 \\sum  m(Q_i~)
= \epsilon \over m(Q) m(Q) = \epsilon. Mais soit (x,f(x)) un point
du graphe de f avec x \in Q~; soit i tel que x \in Q_i~; on a alors
\x - x_i\ \leq
\delta(\sigma) \textless{} \eta et donc f(x) - f(x_i)\leq
\epsilon \over 2m(Q) , soit encore f(x) \in {[}f(x_i)
- \epsilon \over 2m(Q) ,f(x_i) + \epsilon
\over 2m(Q) {]}, si bien que (x,f(x)) \in P_i.
On en déduit que le graphe de f est contenu dans
\⋃ ~
_i=1^NP_i avec
\\sum  m(P_i~) =
\epsilon. Donc le graphe de f est négligeable.

Corollaire~20.2.3 Toute partie de \mathbb{R}~^n qui est une réunion
finie de graphes d'applications continues sur des pavés

\begin{align*}
(x_1,\\ldots,x_i-1,x_i+1,\\\ldots,x_n~)&&
\%& \\ & \mapsto~&
x_i =
f(x_1,\\ldots,x_i-1,x_i+1,\\\ldots,x_n~)\%&
\\ \end{align*}

est négligeable.

\paragraph{20.2.2 Intégrales multiples sur un pavé de \mathbb{R}~^n}

Remarque~20.2.2 On appelle point de discontinuité de f tout point où f
n'est pas continue. Si f est une application de l'espace métrique X dans
l'espace métrique E, on notera
\mathrmDisc~ (f) l'ensemble
des points de discontinuité de f.

Proposition~20.2.4 Soit E un espace vectoriel normé de dimension finie
et P un pavé de \mathbb{R}~^n. L'ensemble \mathcal{E} des fonctions f : P \rightarrow~ E
bornées et dont l'ensemble des points de discontinuité est négligeable
est un sous-espace vectoriel de l'espace des applications de P dans E.

Démonstration Cet ensemble est évidemment non vide (il contient par
exemple toutes les fonctions continues sur P)~; si f et g sont dans \mathcal{E},
si \alpha~ et \beta~ sont des scalaires, on a évidemment \alpha~f + \beta~g qui est bornée et
de plus \mathrmDisc~ (\alpha~f +
\beta~g) \subset~\mathrmDisc~ (f)
\cup\mathrmDisc~ (g) (puisque
là où f et g sont toutes deux continues, \alpha~f + \beta~g l'est également), donc
\mathrmDisc~ (\alpha~f + \beta~g) est
négligeable.

On admettra le théorème suivant

Théorème~20.2.5 Il existe une application qui à toute fonction f bornée
de P dans E dont l'ensemble des points de discontinuité est négligeable
associe un élément de E noté \int  _P~f
vérifiant les propriétés suivantes

\begin{itemize}
\itemsep1pt\parskip0pt\parsep0pt
\item
  (i) l'application
  f\mapsto~\int  _P~f
  est linéaire (\int  _P~(\alpha~f + \beta~g) =
  \alpha~\int  _P~f +
  \beta~\int  _P~g)
\item
  (ii) \\int ~
  _Pf\ \leq\int ~
  _P\f\
\item
  (iii) \int  _P~1 = m(P)
\item
  (iv) si P est la réunion de deux pavés P_1 et P_2
  dont l'intersection est contenue dans l'intersection des frontières,
  alors \int  _P~f
  =\int  _P_1~f
  +\int  _P_2~f
\item
  (v) si \x \in
  P∣f(x)\mathrel\neq~0\
  est négligeable, alors \int  _P~f = 0.
\end{itemize}

Proposition~20.2.6

\begin{itemize}
\itemsep1pt\parskip0pt\parsep0pt
\item
  (i) Si f : P \rightarrow~ \mathbb{R}~ est une fonction bornée dont l'ensemble des points de
  discontinuité est négligeable et si f est positive, alors
  \int  _p~f ≥ 0
\item
  (ii) Si f,g : P \rightarrow~ \mathbb{R}~ sont deux fonctions bornées dont l'ensemble des
  points de discontinuité est négligeable et si f \leq g alors
  \int  _P~f \leq\\int
   _Pg
\item
  (iii) Si f : P \rightarrow~ \mathbb{R}~ est une fonction bornée dont l'ensemble des points
  de discontinuité est négligeable, alors
  \\int ~
  _Pf\ \leq
  m(P)sup_x\inP~\f(x)\.
\end{itemize}

Démonstration (i) On a \int  _P~f
=\int ~
_Pf≥\left
\int  _P~f\right
 d'où \int  _P~f ≥ 0

(ii) On a \int  _P~g
-\int  _P~f =\\int
 _P(g - f) ≥ 0 puisque g - f ≥ 0

(iii) Si M =\
sup_x\inP\f(x)\,
on a \\int ~
_Pf\ \leq\int ~
_P\f\
\leq\int  _P~M =
M\int  _P~1 = Mm(P)

Définition~20.2.4 Soit f : P \rightarrow~ E une application, soit \sigma une subdivision
de P, (P_i)_1\leqi\leqN les sous pavés de la subdivision et
pour chaque i \in {[}1,N{]}, x_i un point de P_i~; la
somme S(f,\sigma,x) =\ \\sum
 _i=1^Nm(P_i)f(x_i) sera appelée une
somme de Riemann associée à la subdivision \sigma et à la famille x =
(x_i)_1\leqi\leqN.

On admettra également le résultat suivant

Théorème~20.2.7 Soit f une fonction bornée de P dans E dont l'ensemble
des points de discontinuité est négligeable. Alors, pour tout \epsilon
\textgreater{} 0, il existe un réel \eta \textgreater{} 0 tel que pour
toute subdivision \sigma de P de pas plus petit que \eta et pour toute famille x
= (x_i) associée, on a
\\int  _P~f -
S(f,\sigma,x)\ \textless{} \epsilon.

Remarque~20.2.3 Comme dans le cas des fonctions d'une variable, on peut
aussi définir, lorsque f est à valeurs réelles des sommes de Darboux
supérieure et inférieure D(f,\sigma) =\
\sum ~
_i=1^Nm(P_i)M_i et
\\sum ~
_i=1^Nm(P_i)m_i où M_i
= sup_x\inP_i~f(t) et
m_i = inf~
_x\inP_if(t). La même démonstration que pour les fonctions
d'une variable montre alors que ces sommes de Darboux inférieure et
supérieure tendent vers l'intégrale de f sur P lorsque le pas de la
subdivision tend vers 0.

\paragraph{20.2.3 Intégrales multiples sur une partie de \mathbb{R}~^n}

Soit A une partie de \mathbb{R}~^n bornée de frontière négligeable et f
: A \rightarrow~ E continue et bornée. Soit P un pavé de \mathbb{R}~^n contenant A
et f^∗ l'application de P dans E définie par

 f^∗(x) = \left \
\cases f(x)&si x \in A \cr 0 &si x \in A
 \right .

L'ensemble des points de discontinuité de f^∗ est contenu
dans la frontière de A car si x est dans l'intérieur de A, la fonction
f^∗ coïncide avec la fonction continue f sur tout un
voisinage de x, donc est continue au point x et si x appartient à
l'intérieur de P \diagdown A, alors f^∗ coïncide avec la fonction
nulle sur tout un voisinage de x, donc est continue au point x. On peut
donc définir \int  _Pf^∗~. De
plus, si P_1 et P_2 sont deux pavés contenant A, ''la
fonction f^∗'' est nulle sur P_2 \diagdown P_1 et
sur P_1 \diagdown P_2, si bien que \\int
 _P_1f^∗ =\int ~
_P_2f^∗, ce qui montre que
\int  _Pf^∗~ ne dépend pas du
choix du pavé P contenant A.

Définition~20.2.5 Soit A une partie de \mathbb{R}~^n bornée de
frontière négligeable et f : A \rightarrow~ E continue et bornée~; on posera
\int  _A~f =\\int
 _Pf^∗.

Théorème~20.2.8

\begin{itemize}
\itemsep1pt\parskip0pt\parsep0pt
\item
  (i) L'application
  f\mapsto~\int  _A~f
  est linéaire (\int  _A~(\alpha~f + \beta~g) =
  \alpha~\int  _A~f +
  \beta~\int  _A~g)
\item
  (ii) \\int ~
  _Af\ \leq\int ~
  _A\f\
\item
  (iii) si A_1 \bigcap A_2 est négligeable, alors
  \int  _A_1\cupA_2~f
  =\int  _A_1~f
  +\int  _A_2~f
\item
  (iv) si A est négligeable, alors \int ~
  _Af = 0.
\end{itemize}

Démonstration (i) et (ii) résultent de (\alpha~f + \beta~g)^∗ =
\alpha~f^∗ + \beta~g^∗ et de
\f^∗\
=\ f\^∗ qui
sont évidents. Pour (iii) si on considère P un pavé contenant
A_1 \cup A_2, f^∗ l'extension de f de
A_1 \cup A_2 à P, f_1^∗ et
f_2^∗ les extensions de f depuis respectivement
A_1 et A_2 à P, on a f^∗ =
f_1^∗ + f_2^∗ sauf sur A_1 \bigcap
A_2. On a donc f^∗ = f_1^∗ +
f_2^∗ + g où g est une fonction nulle sauf sur
l'ensemble négligeable A_1 \bigcap A_2~; on en déduit que

\begin{align*} \int ~
_A_1\cupA_2f& =& \int ~
_Pf^∗ =\int ~
_P(f_1^∗ + f_ 2^∗ + g) \%&
\\ & =& \int ~
_Pf_1^∗ +\int ~
_Pf_2^∗ +\int ~
_Pg =\int ~
_Pf_1^∗ +\int ~
_Pf_2^∗\%& \\ &
=& \int  _A_1~f
+\int  _A_2~f \%&
\\ \end{align*}

Quand à (iv), il est évident puisque \x \in
P∣f^∗(x)\mathrel\neq~0\
\subset~ A

\paragraph{20.2.4 Mesure d'un sous-ensemble borné de \mathbb{R}~^n}

Définition~20.2.6 On dit qu'une partie A de \mathbb{R}~^n est quarrable
si elle est bornée et de frontière négligeable.

Définition~20.2.7 Soit A une partie quarrable de \mathbb{R}~^n~; on
appelle mesure de A le nombre réel positif m(A)
=\int  _A~1.

Proposition~20.2.9 Soit A une partie quarrable de \mathbb{R}~^n~; alors
l'intérieur et l'adhérence de A sont aussi quarrables et ont la même
mesure.

Démonstration En effet, la frontière de l'intérieur et de l'adhérence de
A sont contenues dans la frontière de A qui est négligeable. De plus
\int  _\overlineA~1
=\int  _\overlineA\diagdownA~1
+\int  _A~1 =\\int
 _A1 puisque \overlineA \diagdown A est contenu dans
la frontière de A et donc est négligeable~; la démonstration est
similaire pour l'intérieur.

Proposition~20.2.10 Si f : A \rightarrow~ E est une fonction continue bornée sur
l'ensemble quarrable A, alors
\\int ~
_Af\ \leq
m(A)sup_x\inA~\f(x)\.

Démonstration Si M =\
sup_x\inA\f(x)\,
on a \\int ~
_Af\ \leq\int ~
_A\f\
\leq\int  _A~M =
M\int  _A~1 = Mm(A)

{[}
{[}
{[}
{[}

\end{document}

% \documentclass[]{article}
\usepackage[T1]{fontenc}
\usepackage{lmodern}
\usepackage{amssymb,amsmath}
\usepackage{ifxetex,ifluatex}
\usepackage{fixltx2e} % provides \textsubscript
% use upquote if available, for straight quotes in verbatim environments
\IfFileExists{upquote.sty}{\usepackage{upquote}}{}
\ifnum 0\ifxetex 1\fi\ifluatex 1\fi=0 % if pdftex
  \usepackage[utf8]{inputenc}
\else % if luatex or xelatex
  \ifxetex
    \usepackage{mathspec}
    \usepackage{xltxtra,xunicode}
  \else
    \usepackage{fontspec}
  \fi
  \defaultfontfeatures{Mapping=tex-text,Scale=MatchLowercase}
  \newcommand{\euro}{€}
\fi
% use microtype if available
\IfFileExists{microtype.sty}{\usepackage{microtype}}{}
\usepackage{graphicx}
% Redefine \includegraphics so that, unless explicit options are
% given, the image width will not exceed the width of the page.
% Images get their normal width if they fit onto the page, but
% are scaled down if they would overflow the margins.
\makeatletter
\def\ScaleIfNeeded{%
  \ifdim\Gin@nat@width>\linewidth
    \linewidth
  \else
    \Gin@nat@width
  \fi
}
\makeatother
\let\Oldincludegraphics\includegraphics
{%
 \catcode`\@=11\relax%
 \gdef\includegraphics{\@ifnextchar[{\Oldincludegraphics}{\Oldincludegraphics[width=\ScaleIfNeeded]}}%
}%
\ifxetex
  \usepackage[setpagesize=false, % page size defined by xetex
              unicode=false, % unicode breaks when used with xetex
              xetex]{hyperref}
\else
  \usepackage[unicode=true]{hyperref}
\fi
\hypersetup{breaklinks=true,
            bookmarks=true,
            pdfauthor={},
            pdftitle={Calcul des integrales doubles et triples},
            colorlinks=true,
            citecolor=blue,
            urlcolor=blue,
            linkcolor=magenta,
            pdfborder={0 0 0}}
\urlstyle{same}  % don't use monospace font for urls
\setlength{\parindent}{0pt}
\setlength{\parskip}{6pt plus 2pt minus 1pt}
\setlength{\emergencystretch}{3em}  % prevent overfull lines
\setcounter{secnumdepth}{0}
 
/* start css.sty */
.cmr-5{font-size:50%;}
.cmr-7{font-size:70%;}
.cmmi-5{font-size:50%;font-style: italic;}
.cmmi-7{font-size:70%;font-style: italic;}
.cmmi-10{font-style: italic;}
.cmsy-5{font-size:50%;}
.cmsy-7{font-size:70%;}
.cmex-7{font-size:70%;}
.cmex-7x-x-71{font-size:49%;}
.msbm-7{font-size:70%;}
.cmtt-10{font-family: monospace;}
.cmti-10{ font-style: italic;}
.cmbx-10{ font-weight: bold;}
.cmr-17x-x-120{font-size:204%;}
.cmsl-10{font-style: oblique;}
.cmti-7x-x-71{font-size:49%; font-style: italic;}
.cmbxti-10{ font-weight: bold; font-style: italic;}
p.noindent { text-indent: 0em }
td p.noindent { text-indent: 0em; margin-top:0em; }
p.nopar { text-indent: 0em; }
p.indent{ text-indent: 1.5em }
@media print {div.crosslinks {visibility:hidden;}}
a img { border-top: 0; border-left: 0; border-right: 0; }
center { margin-top:1em; margin-bottom:1em; }
td center { margin-top:0em; margin-bottom:0em; }
.Canvas { position:relative; }
li p.indent { text-indent: 0em }
.enumerate1 {list-style-type:decimal;}
.enumerate2 {list-style-type:lower-alpha;}
.enumerate3 {list-style-type:lower-roman;}
.enumerate4 {list-style-type:upper-alpha;}
div.newtheorem { margin-bottom: 2em; margin-top: 2em;}
.obeylines-h,.obeylines-v {white-space: nowrap; }
div.obeylines-v p { margin-top:0; margin-bottom:0; }
.overline{ text-decoration:overline; }
.overline img{ border-top: 1px solid black; }
td.displaylines {text-align:center; white-space:nowrap;}
.centerline {text-align:center;}
.rightline {text-align:right;}
div.verbatim {font-family: monospace; white-space: nowrap; text-align:left; clear:both; }
.fbox {padding-left:3.0pt; padding-right:3.0pt; text-indent:0pt; border:solid black 0.4pt; }
div.fbox {display:table}
div.center div.fbox {text-align:center; clear:both; padding-left:3.0pt; padding-right:3.0pt; text-indent:0pt; border:solid black 0.4pt; }
div.minipage{width:100%;}
div.center, div.center div.center {text-align: center; margin-left:1em; margin-right:1em;}
div.center div {text-align: left;}
div.flushright, div.flushright div.flushright {text-align: right;}
div.flushright div {text-align: left;}
div.flushleft {text-align: left;}
.underline{ text-decoration:underline; }
.underline img{ border-bottom: 1px solid black; margin-bottom:1pt; }
.framebox-c, .framebox-l, .framebox-r { padding-left:3.0pt; padding-right:3.0pt; text-indent:0pt; border:solid black 0.4pt; }
.framebox-c {text-align:center;}
.framebox-l {text-align:left;}
.framebox-r {text-align:right;}
span.thank-mark{ vertical-align: super }
span.footnote-mark sup.textsuperscript, span.footnote-mark a sup.textsuperscript{ font-size:80%; }
div.tabular, div.center div.tabular {text-align: center; margin-top:0.5em; margin-bottom:0.5em; }
table.tabular td p{margin-top:0em;}
table.tabular {margin-left: auto; margin-right: auto;}
div.td00{ margin-left:0pt; margin-right:0pt; }
div.td01{ margin-left:0pt; margin-right:5pt; }
div.td10{ margin-left:5pt; margin-right:0pt; }
div.td11{ margin-left:5pt; margin-right:5pt; }
table[rules] {border-left:solid black 0.4pt; border-right:solid black 0.4pt; }
td.td00{ padding-left:0pt; padding-right:0pt; }
td.td01{ padding-left:0pt; padding-right:5pt; }
td.td10{ padding-left:5pt; padding-right:0pt; }
td.td11{ padding-left:5pt; padding-right:5pt; }
table[rules] {border-left:solid black 0.4pt; border-right:solid black 0.4pt; }
.hline hr, .cline hr{ height : 1px; margin:0px; }
.tabbing-right {text-align:right;}
span.TEX {letter-spacing: -0.125em; }
span.TEX span.E{ position:relative;top:0.5ex;left:-0.0417em;}
a span.TEX span.E {text-decoration: none; }
span.LATEX span.A{ position:relative; top:-0.5ex; left:-0.4em; font-size:85%;}
span.LATEX span.TEX{ position:relative; left: -0.4em; }
div.float img, div.float .caption {text-align:center;}
div.figure img, div.figure .caption {text-align:center;}
.marginpar {width:20%; float:right; text-align:left; margin-left:auto; margin-top:0.5em; font-size:85%; text-decoration:underline;}
.marginpar p{margin-top:0.4em; margin-bottom:0.4em;}
.equation td{text-align:center; vertical-align:middle; }
td.eq-no{ width:5%; }
table.equation { width:100%; } 
div.math-display, div.par-math-display{text-align:center;}
math .texttt { font-family: monospace; }
math .textit { font-style: italic; }
math .textsl { font-style: oblique; }
math .textsf { font-family: sans-serif; }
math .textbf { font-weight: bold; }
.partToc a, .partToc, .likepartToc a, .likepartToc {line-height: 200%; font-weight:bold; font-size:110%;}
.chapterToc a, .chapterToc, .likechapterToc a, .likechapterToc, .appendixToc a, .appendixToc {line-height: 200%; font-weight:bold;}
.index-item, .index-subitem, .index-subsubitem {display:block}
.caption td.id{font-weight: bold; white-space: nowrap; }
table.caption {text-align:center;}
h1.partHead{text-align: center}
p.bibitem { text-indent: -2em; margin-left: 2em; margin-top:0.6em; margin-bottom:0.6em; }
p.bibitem-p { text-indent: 0em; margin-left: 2em; margin-top:0.6em; margin-bottom:0.6em; }
.subsectionHead, .likesubsectionHead { margin-top:2em; font-weight: bold;}
.sectionHead, .likesectionHead { font-weight: bold;}
.quote {margin-bottom:0.25em; margin-top:0.25em; margin-left:1em; margin-right:1em; text-align:justify;}
.verse{white-space:nowrap; margin-left:2em}
div.maketitle {text-align:center;}
h2.titleHead{text-align:center;}
div.maketitle{ margin-bottom: 2em; }
div.author, div.date {text-align:center;}
div.thanks{text-align:left; margin-left:10%; font-size:85%; font-style:italic; }
div.author{white-space: nowrap;}
.quotation {margin-bottom:0.25em; margin-top:0.25em; margin-left:1em; }
h1.partHead{text-align: center}
.sectionToc, .likesectionToc {margin-left:2em;}
.subsectionToc, .likesubsectionToc {margin-left:4em;}
.sectionToc, .likesectionToc {margin-left:6em;}
.frenchb-nbsp{font-size:75%;}
.frenchb-thinspace{font-size:75%;}
.figure img.graphics {margin-left:10%;}
/* end css.sty */

\title{Calcul des integrales doubles et triples}
\author{}
\date{}

\begin{document}
\maketitle

\textbf{Warning: 
requires JavaScript to process the mathematics on this page.\\ If your
browser supports JavaScript, be sure it is enabled.}

\begin{center}\rule{3in}{0.4pt}\end{center}

[
[
[]
[

\section{20.3 Calcul des intégrales doubles et triples}

\subsection{20.3.1 Théorème de Fubini sur une partie de \mathbb{R}~^2}

Théorème~20.3.1 (de Fubini pour un pavé de \mathbb{R}~^2). Soit P =
[a,b] \times [c,d] un pavé de \mathbb{R}~^2, E un espace vectoriel
normé de dimension finie, f : P \rightarrow~ E une fonction bornée dont l'ensemble
des points de discontinuité est négligeable vérifiant (i) pour chaque x
\in [a,b], l'application y\mapsto~f(x,y) est
réglée de [c,d] dans E (ii) l'application \phi : [a,b] \rightarrow~ E,
x\mapsto~\int ~
_c^df(x,y) dy est réglée. Alors

\int  _P~f =\\int
 _a^b\phi(x) dx =\int ~
_a^b\left (\int ~
_c^df(x,y) dy\right ) dx

Démonstration Quitte à prendre une base de E, il suffit de montrer le
résultat lorsque E = \mathbb{R}~. Soit n \in \mathbb{N}~, a_i = a + i b-a
\over n et c_j = c + j d-c
\over n . On définit ainsi une subdivision \sigma_n
de P dont le pas tend vers 0 quand n tend vers + \infty~. Soit P_i,j
= [a_i-1,a_i] \times [c_j-1,c_j]
les pavés de cette subdivision. Posons alors, pour (i,j) \in
[1,n]^2,

\mu_i,j = 1 \over c_j -
c_j-1 \int ~
_c_j-1^c_j f(a_i,y) dy = n
\over d - c \int ~
_c_j-1^c_j f(a_i,y) dy

et soit S_n =\
\sum  _i,j\in[1,n]~ (b-a)(d-c)
\over n^2 \mu_i,j. Si l'on note
M_i,j =\
sup_x\inP_i,jf(x) et m_i,j
= inf _x\inP_i,j~f(x), on a
clairement

\begin{align*} d(f,\sigma_n)& =&
\sum _i,j\in[1,n]~ (b - a)(d - c)
\over n^2 m_i,j \%&
\\ & \leq& \\sum
_i,j\in[1,n] (b - a)(d - c) \over
n^2 \mu_i,j \%& \\ &
\leq& \sum _i,j\in[1,n]~ (b - a)(d - c)
\over n^2 M_i,j =
D(f,\sigma_n)\%& \\
\end{align*}

où d(f,\sigma_n) et D(f,\sigma_n) désignent respectivement les
sommes de Darboux inférieure et supérieure de f associées à la
subdivision \sigma_n. Quand n tend vers + \infty~, ces sommes de Darboux
admettent toutes deux pour limite \int ~
_Pf, donc lim_n\rightarrow~+\infty~S_n~
=\int  _P~f. Mais d'autre part,

\begin{align*} S_n& =&
\sum _i=1^n~ b - a
\over n \\sum
_j=1^n
\\int  ~
_c_j-1^c_j f(a_i,y) dy\%&
\\ & =& \\sum
_i=1^n b - a \over n
\\int  ~
_c^df(a_ i,y) dy \%&
\\ & =& \\sum
_i=1^n b - a \over n \phi(a_i)
\%& \\ \end{align*}

Il s'agit donc d'une somme de Riemann pour la fonction réglée \phi :
[a,b] \rightarrow~ \mathbb{R}~ associée à la subdivision régulière
(a_i)_0\leqi\leqn. On en déduit que
lim_n\rightarrow~+\infty~S_n~
=\int  _a^b~\phi(x) dx, d'où
l'égalité recherchée.

Remarque~20.3.1 De même si on suppose que

\begin{itemize}
\itemsep1pt\parskip0pt\parsep0pt
\item
  (i) pour chaque y \in [c,d], l'application
  x\mapsto~f(x,y) est réglée de [a,b] dans E
\item
  (ii) l'application \psi : [c,d] \rightarrow~ E,
  y\mapsto~\int ~
  _a^bf(x,y) dx est réglée.
\end{itemize}

Alors

\int  _P~f =\\int
 _c^d\psi(y) dy =\int ~
_c^d\left (\int ~
_a^bf(x,y) dx\right ) dy

Tout ceci nous amène tout naturellement à introduire la notation
\int  \\int ~
_Pf(x,y) dx dy pour l'intégrale d'une fonction f sur un pavé P de
\mathbb{R}~^2 puis à généraliser cette notation à toute partie
quarrable de \mathbb{R}~^2.

Corollaire~20.3.2 Soit P = [a,b] \times [c,d] un pavé de
\mathbb{R}~^2, f : [a,b] \rightarrow~ K, g : [c,d] \rightarrow~ K réglées. Alors

\int  \\int ~
_[a,b]\times[c,d]f(x)g(y) dx dy = \left
(\int  _a^b~f(x)
dx\right )\left
(\int  _c^d~g(y)
dy\right )

Démonstration On a

\int  _c^d~f(x)g(y) dy =
f(x)\int  _c^d~g(y) dy = \lambda~f(x)

avec \lambda~ =\int  _c^d~g(y) dy. On en
déduit que

\begin{align*} \int ~
\int  _[a,b]\times[c,d]~f(x)g(y) dx
dy& =& \int  _a^b~\lambda~f(x) dx =
\lambda~\int  _a^b~f(x) dx\%&
\\ & =& \left
(\int  _a^b~f(x)
dx\right )\left
(\int  _c^d~g(y)
dy\right ) \%& \\
\end{align*}

Théorème~20.3.3 (théorème de Fubini pour une partie de \mathbb{R}~^2).
Soit \phi_1 et \phi_2 deux applications continues de
[a,b] dans \mathbb{R}~ vérifiant \forall~~t \in [a,b],
\phi_1(t) \leq \phi_2(t) et soit A = \(x,y) \in
\mathbb{R}~^2∣x \in
[a,b]\text et \phi_1(x) \leq y \leq
\phi_2(x)\. Soit f : A \rightarrow~ E continue. Alors A est
quarrable et

\int  \\int ~
_Af(x,y) dx dy =\int ~
_a^b\left (\int ~
_\phi_1(x)^\phi_2(x)f(x,y)
dy\right ) dx

Démonstration A est quarrable car il est borné (\phi_1 et
\phi_2 continues sur des compacts sont bornées) et sa frontière
est formée de la réunion de quatre graphes de fonctions continues
x\mapsto~\phi_1(x),
y\mapsto~b,
x\mapsto~\phi_2(x) et
y\mapsto~a. Soit M =\
sup\phi_2(x) et m = inf~
\phi_1(x) si bien que A \subset~ [a,b] \times [m,M] et soit
f^∗ le prolongement par 0 de f de A à P = [a,b] \times
[m,M]. On sait déjà que f^∗ est bornée et que
\mathrmDisc~ (f) est
négligeable (il est contenu dans la frontière de A). Pour x \in [a,b]
fixé, l'application y\mapsto~f^∗(x,y) est
continue par morceaux car f^∗(x,y) = \left
\ \cases f(x,y)&si \phi_1(x) \leq
y \leq \phi_2(x) \cr 0 &sinon 
\right .. De plus \int ~
_m^Mf^∗(x,y) dy =\int ~
_\phi_1(x)^\phi_2(x)f(x,y) dy est une fonction
continue de x comme on l'a vu dans le chapitre sur les intégrales de
fonctions d'une variable. On peut donc appliquer le théorème précédent
et on obtient

\begin{align*} \int ~
\int  _A~f(x,y) dx dy& =&
\int  \\int ~
_Pf^∗(x,y) dx dy \%& \\
& =& \int ~
_a^b\left (\int ~
_m^Mf^∗(x,y) dy\right ) dx \%&
\\ & =& \int ~
_a^b\left (\int ~
_\phi_1(x)^\phi_2(x)f(x,y)
dy\right ) dx\%& \\
\end{align*}

\includegraphics{cours16x.png}

Remarque~20.3.2 Ceci permet de ramener le calcul d'une intégrale double
sur une partie de \mathbb{R}~^2 délimitée par deux graphes au calcul de
deux intégrales de fonctions d'une variable. Pour un sous ensemble A
plus général, on cherchera à écrire A = A_1
\cup\\ldots~ \cup
A_k où chaque partie A_i est limitée par deux graphes
de fonctions continues, soit de la forme \(x,y) \in
\mathbb{R}~^2∣x \in
[a,b]\text et \phi_1(x) \leq y \leq
\phi_2(x)\ ou de la forme \(x,y)
\in \mathbb{R}~^2∣y \in
[c,d]\text et \psi_1(y) \leq x \leq
\psi_2(y)\~; on écrira alors, à condition que les
intersections deux à deux des A_i soient négligeables,
\int  \\int ~
_Af(x,y) dx dy =\
\sum ~
_i=1^k\int ~
\int  _A_i~f(x,y) dx dy puis on
ramènera le calcul de chaque intégrale double au calcul de deux
intégrales simples.

Exemple~20.3.1 Soit K = \(x,y) \in
\mathbb{R}~^2∣y ≥ 0, y \leq x,0 \leq x + 2y \leq
2\ et on cherche à calculer I
=\int  \\int ~
_Kx^2 dx dy (moment d'inertie par rapport à l'axe Oy).

\includegraphics{cours17x.png}

Le théorème de Fubini nous permet de calculer cette intégrale de deux
manières différentes suivant que l'on commence à intégrer suivant x ou
suivant y. Dans une première méthode on peut écrire

\begin{align*} K& =& \(x,y) \in
\mathbb{R}~^2∣x \in [0, 2
\over 3 ]\text et 0 \leq y \leq
x\ \%& \\ & \cup&
\(x,y) \in \mathbb{R}~^2∣x \in
[ 2 \over 3 ,2]\text et 0 \leq y
\leq 1 - x \over 2 \\%&
\\ \end{align*}

d'où (en faisant sortir de l'intégrale par rapport à y le terme en
x^2 qui ne dépend pas de y)

\begin{align*} I& =& \\int
 _0^2\diagup3x^2\left
(\int ~
_0^xdy\right ) dx
+\int ~
_2\diagup3^2x^2\left
(\int  _0~^1- x
\over 2 dy\right ) dx\%&
\\ & =& \int ~
_0^2\diagup3x^3 dx +\int ~
_2\diagup3^2x^2\left (1 - x
\over 2 \right ) dx = 52
\over 81 \%& \\
\end{align*}

On peut aussi la calculer en considérant que

K = \(x,y) \in
\mathbb{R}~^2∣y \in [0, 2
\over 3 ]\text et y \leq x \leq 2 -
2y\

d'où

I =\int ~
_0^2\diagup3\left (\int ~
_y^2-2yx^2 dx\right ) dy
=\int  _0^2\diagup3~\left
( (2 - 2y)^3 \over 3 - y^3
\over 3 \right ) dy = 52
\over 81

\subsection{20.3.2 Théorème de Fubini sur une partie de \mathbb{R}~^3}

On démontre d'une fa\ccon similaire le résultat
suivant pour les intégrales sur une partie de \mathbb{R}~^3

Théorème~20.3.4 (théorème de Fubini pour un pavé de \mathbb{R}~^3).
Soit P = [a_1,b_1] \times
[a_2,b_2] \times [a_3,b_3] un
pavé de \mathbb{R}~^3, E un espace vectoriel normé de dimension finie,
f : P \rightarrow~ E une fonction bornée dont l'ensemble des points de
discontinuité est négligeable vérifiant

\begin{itemize}
\item
  (i) pour chaque (x,y) \in [a_1,b_1] \times
  [a_2,b_2], l'application
  z\mapsto~f(x,y,z) est réglée de [c,d] dans E
\item
  (ii) pour chaque x \in [a_1,b_1], l'application
  y\mapsto~\int ~
  _a_3^b_3f(x,y,z) dz est réglée de
  [a_2,b_2] dans E
\item
  (iii) l'application
  x\mapsto~\int ~
  _a_2^b_2\left
  (\int ~
  _a_3^b_3f(x,y,z)
  dz\right ) dy est réglée de
  [a_1,b_1] dans E
\item
  Alors

  \begin{align*} \int ~
  \int  \\int ~
  _Pf(x,y,z) dx dy dz =& & \%&
  \\ & & \int ~
  _a_1^b_1 \left
  (\int  _a_2^b_2~
  \left (\int ~
  _a_3^b_3 f(x,y,z)
  dz\right ) dy\right ) dx\%&
  \\ \end{align*}
\end{itemize}

Bien entendu, dans la limite de validité (c'est-à-dire si l'on est
certain que toutes les intégrales écrites ont bien un sens), on peut
regrouper deux des intégrales simples en une intégrale double en
utilisant le théorème de Fubini pour les intégrales doubles, et ainsi
obtenir les formules (à permutation près sur les noms des variables)

\begin{align*} \int ~
\int  \\int ~
_Pf(x,y,z) dx dy dz&& \%& \\ &
=& \int  \\int ~
_[a_1,b_1]\times[a_2,b_2]\left
(\int  _a_3^b_3~
f(x,y,z) dz\right ) dx dy\%&
\\ & =& \int ~
_a_3^b_3 \left
(\int  \\int ~
_[a_1,b_1]\times[a_2,b_2]f(x,y,z)
dx dy\right ) dz\%& \\
\end{align*}

La première méthode d'intégration porte en général le nom d'intégration
par piles (on somme d'abord verticalement, puis ensuite
horizontalement), la deuxième méthode portant le nom d'intégration par
tranches (on somme d'abord horizontalement, puis ensuite verticalement).

De la même manière que pour les intégrales doubles et par prolongement
par 0 à un pavé, on montre alors les deux résultats suivants

Théorème~20.3.5 (théorème de Fubini pour une partie de \mathbb{R}~^3,
intégration par piles). Soit A un sous-ensemble quarrable de
\mathbb{R}~^2, \phi_1 et \phi_2 deux applications continues
de A dans \mathbb{R}~ vérifiant \forall~~(x,y) \in A,
\phi_1(x,y) \leq \phi_2(x,y) et soit K =
\(x,y,z) \in
\mathbb{R}~^2∣(x,y) \in
A\text et \phi_1(x,y) \leq z \leq
\phi_2(x,y)\. Soit f : K \rightarrow~ E continue. Alors

\int  \\int ~
\int  _K~f(x,y,z) dx dy dz
=\int  \\int ~
_A\left (\int ~
_\phi_1(x,y)^\phi_2(x,y)f(x,y,z)
dz\right ) dx dy

Théorème~20.3.6 (théorème de Fubini pour une partie de \mathbb{R}~^3,
intégration par tranches). Soit K une partie de \mathbb{R}~^3 compacte
et quarrable. Soit f une application continue de K dans E. On suppose
vérifiées les conditions suivantes (en posant m
= inf _(x,y,z)\inK~z et M
= sup_(x,y,z)\inK~z) (i) pour tout z \in
[m,M], K_z = \(x,y) \in
\mathbb{R}~^2∣(x,y,z) \in K\
(section de K par le plan horizontal de cote z) est un sous-ensemble
quarrable de \mathbb{R}~^2 (ii) l'application
z\mapsto~\int ~
\int  _K_z~f(x,y,z) dx dy est
réglée sur [m,M]. Alors

\int  \\int ~
\int  _K~f(x,y,z) dx dy dz
=\int  _m^M~\left
(\int  \\int ~
_K_zf(x,y,z) dx dy\right ) dz

Exemple~20.3.2 Supposons donnée une courbe dans un plan méridien donnée
en coordonnées cylindriques par r = \phi(z) où \phi : [a,b] \rightarrow~ \mathbb{R}~ est
continue positive. Considérons le volume K de révolution délimité par la
rotation de la courbe autour de l'axe Oz. Les sous-ensembles
K_z sont des disques de centre (0,0) de rayon \phi(z), donc de
mesure \pi~\phi(z)^2. On en déduit que la mesure de K est donnée
par

\begin{align*} m(K)& =& \\int
 \int  \\int  _K~
dx dy dz =\int ~
_a^b\left (\int ~
\int  _K_z~ dx
dy\right ) dz =\int ~
_a^bm(K_ z) dz\%&
\\ & =& \pi~\int ~
_a^b\phi(z)^2 dz \%&
\\ \end{align*}

Par exemple, pour une boule de rayon R, on peut prendre a = -R, b = R et
\phi(z) = \sqrtR^2  - z^2, d'où la
mesure de la boule

m(K) = \pi~\int ~
_-R^R(R^2 - z^2) dz =
\pi~\left [R^2z - z^3
\over 3 \right ]_-R^R
= 4 \over 3 \pi~R^3

\subsection{20.3.3 Théorème de changement de variables dans les
intégrales multiples}

On admettra le théorème suivant de démonstration difficile

Théorème~20.3.7 (théorème de changement de variables). Soit K_1
et K_2 deux parties compactes de \mathbb{R}~^n de frontières
négligeables, \phi : K_1 \rightarrow~ K_2 continue, E un espace
vectoriel normé de dimension finie. On suppose que \phi réalise un
\mathcal{C}^1 difféomorphisme de l'intérieur de K_1 sur
l'intérieur de K_2. Soit f : K_2 \rightarrow~ E continue. Alors
(si j_\phi(x) désigne le jacobien de \phi au point x \in
K_1^o)

\int  _K_2~f
=\int  _K_1~f \cdot \phi
j_\phi

En particulier on a les formules suivantes pour les intégrales doubles
et triples

\int  \\int ~
_K_2f(x,y) dx dy =\int ~
\int  _K_1~f(\phi(u,v))
j_\phi(u,v) du dv

\begin{align*} \int ~
\int  \\int ~
_K_2f(x,y,z) dx dy dz&& \%&
\\ & =& \int ~
\int  \\int ~
_K_1f(\phi(u,v,w)) j_\phi(u,v,w)
du dv dw\%& \\
\end{align*}

Remarque~20.3.3 Le lecteur comparera ce théorème de changement de
variables avec le théorème de changement de variable pour les fonctions
d'une variable. Le jacobien joue ici le rôle du terme \phi'(u). On prendra
garde qu'ici il est assorti d'une valeur absolue. Ceci est dû à
l'absence de convention de Chasles à partir de la dimension 2. En
dimension 1 et lorsque \phi est décroissante (donc \phi' \leq 0), les bornes se
retrouvent en sens contraire de l'ordre naturel et un rétablissement de
cet ordre transforme alors \phi' en - \phi' = \phi'.

Corollaire~20.3.8 En supposant vérifiées les hypothèses ci dessus pour
le changement de variable, on a les formules suivantes pour les passages
en coordonnées polaires, cylindriques ou sphériques

\begin{align*} \int ~
\int  _K_2~f(x,y) dx dy& =&
\int  \\int ~
_K_1f(rcos~
\theta,rsin~ \theta) r dr d\theta\%&
\\ & & \%&
\\ \end{align*}

\begin{align*} \int ~
\int  \\int ~
_K_2f(x,y,z) dx dy dz&& \%&
\\ & =& \int ~
\int  \\int ~
_K_1f(rcos~
\theta,rsin~ \theta,z) r dr d\theta dz
\%& \\ \int ~
\int  \\int ~
_K_2f(x,y,z) dx dy dz&& \%&
\\ & =& \int ~
\int  \\int ~
_K_1f(rcos~
\thetacos \phi,r\sin~
\thetacos \phi,r\sin~ \phi)
r^2 cos~ \phi dr d\theta
d\phi\%& \\ \end{align*}

Remarque~20.3.4 Le lecteur devra se persuader que le principal obstacle
au calcul explicite d'une intégrale multiple provient du domaine
d'intégration et non de la fonction à intégrer (penser par exemple
qu'une aire ou un volume peuvent être difficiles à calculer alors que la
fonction à intégrer est la constante 1). Ceci veut dire que lorsque l'on
recherche un changement de variable, on doit accorder une priorité
absolue à la simplification du domaine d'intégration, l'idéal étant de
transformer ce domaine en un pavé.

Exemple~20.3.3 Soit
(v_1,\\ldots,v_n~)
une famille libre de \mathbb{R}~^n et soit V le polytope construit sur
cette base, c'est-à-dire V = \t_1v_1
+ \\ldots~ +
t_nv_n∣\forall~~i
\in [1,n], t_i \in [0,1]\ (en dimension 2,
il s'agit d'un parallélogramme et en dimension 3 d'un parallélépipède).
Alors l'application \phi : [0,1]^n \rightarrow~ V ,
(t_1,\\ldots,t_n)\mapsto~t_1v_1~
+ \\ldots~ +
t_nv_n vérifie évidemment les conditions du théorème
de changement de variable. De plus son jacobien est égal au produit
mixte des n vecteurs, soit
[v_1,\\ldots,v_n~].
On en déduit que la mesure de V est donnée par

\begin{align*} m(V )& =&
\int  _V ~1 =\\int

_[0,1]^n[v_1,\\ldots,v_n~]
\%& \\ & =&
[v_1,\\ldots,v_n]m([0,1]^n~)
= [v_
1,\\ldots,v_n~]\%&
\\ \end{align*}

\includegraphics{cours18x.png}

Exemple~20.3.4 On considère deux paraboles d'axe Ox tangentes en O à
l'axe Oy et deux paraboles d'axe Oy tangentes en O à l'axe Ox. On
cherche à calculer l'aire du domaine K compris entre les paraboles
(hachuré sur le dessin ci dessous)

\includegraphics{cours19x.png}

Les paraboles auront pour équations x^2 = 2p_1y et
x^2 = 2p_2y pour les paraboles verticales,
y^2 = 2q_1x et y^2 = 2q_2x pour
les paraboles horizontales. Il n'est pas raisonnable de tenter un calcul
par le théorème de Fubini. Nous allons donc faire un changement de
variable en paramétrant un point de la zone hachuré~; pour cela nous
considérerons que tout point de la zone hachurée est l'intersection
d'une parabole x^2 = 2py et d'une parabole y^2 =
2qx avec p \in [p_1,p_2] et q \in
[q_1,q_2], autrement dit nous considérerons
l'application \phi : K \rightarrow~ [p_1,p_2] \times
[q_1,q_2] définie par \phi(x,y) = ( x^2
\over 2y , y^2 \over 2x
). Il est visible que \phi est bijective, ce que confirmerait un calcul
simple. De plus

j_\phi(x,y) = \left
\matrix\, x
\over y &- x^2 \over
2y^2 \cr - y^2
\over 2x^2 & y \over x
\right  = 3 \over 4

On en déduit par le théorème d'inversion locale que \phi est un
\mathcal{C}^1 difféomorphisme de K^o sur
]p_1,p_2[\times]q_1,q_2[ et que
j_\phi^-1(p,q) = 1 \over
j_\phi(\phi^-1(p,q)) = 4 \over 3 . On
a donc

\begin{align*} m(K)& =& \\int
 \int  _K~1 dx dy \%&
\\ & =& \int ~
\int ~
_[p_1,p_2]\times[q_1,q_2]1
\cdot \phi^-1(p,q)\left j_
\phi^-1(p,q)\right  dp dq\%&
\\ & =& 4 \over 3
m([p_1,p_2] \times [q_1,q_2]) =
4 \over 3 (p_2 - p_1)(q_2 -
q_1) \%& \\
\end{align*}

\subsection{20.3.4 Théorème de Green-Riemann}

Soit \phi_1 et \phi_2 deux applications continues de
[a,b] dans \mathbb{R}~ vérifiant \forall~~x \in [a,b],
\phi_1(x) \leq \phi_2(x) et soit A = \(x,y) \in
\mathbb{R}~^2∣x \in
[a,b]\text et \phi_1(x) \leq y \leq
\phi_2(x)\. On considère la frontière \partial~A de A en
tant qu'arc paramétré orientée comme l'indique la figure ci dessous~: on
parcourt la frontière en laissant A à sa main gauche. Cette frontière
est la réunion de quatre arcs paramétrés, deux étant des graphes de
fonctions x\mapsto~y = \phi_i(x) (l'un
parcouru dans le sens direct, l'autre dans le sens indirect), deux étant
des graphes de fonctions
y\mapsto~\textconstante.

\includegraphics{cours20x.png}

Soit U un ouvert contenant A et P une fonction de classe \mathcal{C}^1
sur U. On cherche à calculer l'intégrale I =\int ~
\int  _A \partial~P \over \partial~y~
(x,y) dx dy. D'après le théorème de Fubini, on a

\begin{align*} \int ~
\int  _A \partial~P \over \partial~y~
(x,y) dx dy &=&\int ~
_a^b\left (\int ~
_\phi_1(x)^\phi_2(x) \partial~P \over
\partial~y (x,y) dy\right ) dx&&\%&
\\ & =& \int ~
_a^b\Big [P(x,y)\Big
]_ y=\phi_1(x)^y=\phi_2(x) dx \%&
\\ & =& \int ~
_a^bP(x,\phi_ 2(x)) dx -\\int
 _a^bP(x,\phi_ 1(x)) dx\%&
\\ \end{align*}

La première intégrale \int ~
_a^bP(x,\phi_2(x)) dx n'est autre que l'intégrale
curviligne de la forme différentielle P(x,y) dx le long du graphe y =
\phi_2(x), c'est-à-dire l'opposée de l'intégrale curviligne de la
forme différentielle P(x,y) dx le long du quart supérieur de la
frontière \partial~A (un changement d'orientation changeant l'intégrale
curviligne en son opposée). La deuxième intégrale
-\int  _a^bP(x,\phi_1~(x))
dx n'est autre que l'opposée de l'intégrale curviligne de la forme
différentielle P(x,y) dx le long du graphe y = \phi_1(x),
c'est-à-dire l'opposée de l'intégrale curviligne de la forme
différentielle P(x,y) dx le long du quart inférieur de la frontière \partial~A .
Mais d'autre part les intégrales curvilignes de la forme différentielle
P(x,y) dx le long des quarts gauche et droite de la frontière sont
nulles, car sur ces arcs, x est une constante et donc dx = 0. On en
déduit donc que \int ~ \\int
 _A \partial~P \over \partial~y (x,y) dx dy =
-\int  _\partial~A~P(x,y) dx.

Soit \psi_1 et \psi_2 deux applications continues de
[c,d] dans \mathbb{R}~ vérifiant \forall~~y \in [c,d],
\psi_1(y) \leq \psi_2(y) et soit A = \(x,y) \in
\mathbb{R}~^2∣y \in
[c,d]\text et \psi_1(y) \leq x \leq
\psi_2(y)\. On considère la frontière \partial~A de A en
tant qu'arc paramétré orientée comme l'indique la figure ci dessous~: on
parcourt la frontière en laissant A à sa main gauche. Cette frontière
est la réunion de quatre arcs paramétrés, deux étant des graphes de
fonctions y\mapsto~x = \psi_i(y) (l'un
parcouru dans le sens direct, l'autre dans le sens indirect), deux étant
des graphes de fonctions
x\mapsto~\textconstante.

\includegraphics{cours21x.png}

Soit U un ouvert contenant A et Q une fonction de classe \mathcal{C}^1
sur U. La même méthode va nous fournir

\begin{align*} \int ~
\int  _A \partial~Q \over \partial~y~
(x,y) dx dy &=&\int ~
_c^d\left (\int ~
_\psi_1(y)^\psi_2(y) \partial~Q \over
\partial~x (x,y) dx\right ) dy&&\%&
\\ & =& \int ~
_c^d\Big[ Q(x,y)\Big
]_ x=\psi_1(y)^x=\psi_2(y) dx \%&
\\ & =& \int ~
_c^dQ(\psi_ 2(y),y) dy -\\int
 _c^dQ(\psi_ 1(y),y) dy\%&
\\ \end{align*}

La première intégrale \int ~
_c^dQ(\psi_2(y),y) dy est l'intégrale de la forme
différentielle Q(x,y) dy le long du quart droit de la frontière \partial~A, la
seconde -\int ~
_c^dQ(\psi_1(y),y) dy est l'intégrale de la forme
différentielle Q(x,y) dy le long du quart gauche de la frontière \partial~A (à
cause du changement d'orientation). Mais d'autre part les intégrales
curvilignes de la forme différentielle Q(x,y) dy le long des quarts
supérieur et inférieur de la frontière sont nulles, car sur ces arcs, y
est une constante et donc dy = 0. On en déduit donc que
\int  \\int  _A~
\partial~Q \over \partial~x (x,y) dx dy =\int ~
_\partial~AQ(x,y) dy.

Si A est à la fois des deux formes en question, on pourra additionner
les deux résultats obtenus ce qui nous conduira à la formule

\begin{align*} \int ~
_\partial~A(P(x,y) dx + Q(x,y) dy)&& \%&
\\ & =& \int ~
\int  _A~\left ( \partial~Q
\over \partial~x (x,y) - \partial~P \over \partial~y
(x,y)\right ) dx dy\%& \\
\end{align*}

Définition~20.3.1 On dit qu'une partie A de \mathbb{R}~^2 est un
compact élémentaire s'il existe a,b,c,d \in \mathbb{R}~, deux fonctions
\phi_1,\phi_2 : [a,b] \rightarrow~ \mathbb{R}~ continues telles que
\forall~x \in [a,b], \phi_1~(x) \leq
\phi_2(x) et deux fonctions \psi_1 et \psi_2 continues
de [c,d] dans \mathbb{R}~ vérifiant \forall~~y \in [c,d],
\psi_1(y) \leq \psi_2(y) telles que

\begin{align*} A& =& \(x,y) \in
\mathbb{R}~^2∣x \in
[a,b]\text et \phi_ 1(x) \leq y \leq
\phi_2(x)\\%& \\
& =& \(x,y) \in
\mathbb{R}~^2∣y \in
[c,d]\text et \psi_ 1(y) \leq x \leq
\psi_2(y)\\%& \\
\end{align*}

Définition~20.3.2 On dit qu'une partie A de \mathbb{R}~^2 est un
compact simple s'il existe un pavé P de \mathbb{R}~^2 contenant A et
une subdivision \sigma de P tels que pour tous les pavés P_i de la
subdivision, P_i \bigcap A soit un compact élémentaire.

On oriente la frontière d'un tel compact simple par la même règle que ci
dessus~: on parcourt la frontière en laissant le compact à sa main
gauche.

Exemple~20.3.5 Une couronne est un compact simple comme le montre le
dessin ci-dessous où on a exhibé une subdivision adaptée, ainsi que
l'orientation de la frontière de chaque compact élémentaire~:

\includegraphics{cours22x.png}

Posons alors A_i = P_i \bigcap A. On peut alors écrire

\begin{align*} \int ~
\int  _A~\left ( \partial~Q
\over \partial~x (x,y) - \partial~P \over \partial~y
(x,y)\right ) dx dy&& \%&
\\ & =& \\sum
_i \\int  ~ 
\\int  ~
_A_i\left ( \partial~Q \over
\partial~x (x,y) - \partial~P \over \partial~y (x,y)\right )
dx dy\%& \\ & =&
\sum _i~
\\int  ~
_\partial~A_i(P(x,y) dx + Q(x,y) dy) \%&
\\ \end{align*}

Mais la réunion des frontières des A_i est constituée de deux
types d'arcs paramétrés~: des arcs faisant partie de la frontière de A
plus des segments horizontaux et verticaux provenant de la subdivision
du pavé. Or (sans vouloir formaliser complètement ce raisonnement) ces
segments sont parcourus deux fois pour un A_i et un
A_j adjacents, une fois dans un sens et une fois dans l'autre
(voir le dessin ci dessus), si bien que les intégrales curvilignes le
long de ces segments horizontaux ou verticaux n'appartenant pas à la
frontière de A s'annulent deux à deux. On obtient donc

\int  \\int ~
_A\left ( \partial~Q \over \partial~x (x,y) -
\partial~P \over \partial~y (x,y)\right ) dx dy
=\int  _\partial~A~(P(x,y) dx + Q(x,y) dy)

Théorème~20.3.9 (Green-Riemann). Soit A un compact simple de
\mathbb{R}~^2 de frontière orientée \partial~A, U un ouvert de \mathbb{R}~^2
contenant A, P,Q : U \rightarrow~ \mathbb{R}~ de classe \mathcal{C}^1. Alors

\int  \\int ~
_A\left ( \partial~Q \over \partial~x (x,y) -
\partial~P \over \partial~y (x,y)\right ) dx dy
=\int  _\partial~A~(P(x,y) dx + Q(x,y) dy)

Remarque~20.3.5 Le théorème de Green-Riemann permet (entre autres
choses) de ramener le calcul d'une intégrale du type
\int  \\int ~
_Af(x,y) dx dy à celui d'une intégrale curviligne
\int  _\partial~A~(P(x,y) dx + Q(x,y) dy)
(c'est-à-dire d'une intégrale simple) à condition de connaître deux
fonctions P et Q telles que f(x,y) = \partial~Q \over \partial~x
(x,y) - \partial~P \over \partial~y (x,y).

Corollaire~20.3.10 Soit A un compact simple de \mathbb{R}~^2 de
frontière orientée \partial~A. Alors l'aire de A est donnée par

m(A) =\int  _\partial~A~x dy =
-\int  _\partial~A~y dx = 1
\over 2 \int  _\partial~A~(x dy
- y dx)

Démonstration Il suffit de prendre successivement P(x,y) = 0,Q(x,y) = x,
P(x,y) = -y,Q(x,y) = 0 et enfin P(x,y) = - y \over 2
,Q(x,y) = x \over 2 , couples pour lesquels  \partial~Q
\over \partial~x (x,y) - \partial~P \over \partial~y (x,y) =
1.

Corollaire~20.3.11 Soit A un compact simple de \mathbb{R}~^2 de
frontière orientée \partial~A. Alors l'aire de A est donnée en polaires par

m(A) = 1 \over 2 \int ~
_\partial~A\rho^2 d\theta

Démonstration En effet x dy - y dx = \rho^2 d\theta.

[
[
[
[

\end{document}

% \documentclass[]{article}
\usepackage[T1]{fontenc}
\usepackage{lmodern}
\usepackage{amssymb,amsmath}
\usepackage{ifxetex,ifluatex}
\usepackage{fixltx2e} % provides \textsubscript
% use upquote if available, for straight quotes in verbatim environments
\IfFileExists{upquote.sty}{\usepackage{upquote}}{}
\ifnum 0\ifxetex 1\fi\ifluatex 1\fi=0 % if pdftex
  \usepackage[utf8]{inputenc}
\else % if luatex or xelatex
  \ifxetex
    \usepackage{mathspec}
    \usepackage{xltxtra,xunicode}
  \else
    \usepackage{fontspec}
  \fi
  \defaultfontfeatures{Mapping=tex-text,Scale=MatchLowercase}
  \newcommand{\euro}{€}
\fi
% use microtype if available
\IfFileExists{microtype.sty}{\usepackage{microtype}}{}
\ifxetex
  \usepackage[setpagesize=false, % page size defined by xetex
              unicode=false, % unicode breaks when used with xetex
              xetex]{hyperref}
\else
  \usepackage[unicode=true]{hyperref}
\fi
\hypersetup{breaklinks=true,
            bookmarks=true,
            pdfauthor={},
            pdftitle={Introduction aux integrales de surface},
            colorlinks=true,
            citecolor=blue,
            urlcolor=blue,
            linkcolor=magenta,
            pdfborder={0 0 0}}
\urlstyle{same}  % don't use monospace font for urls
\setlength{\parindent}{0pt}
\setlength{\parskip}{6pt plus 2pt minus 1pt}
\setlength{\emergencystretch}{3em}  % prevent overfull lines
\setcounter{secnumdepth}{0}
 
/* start css.sty */
.cmr-5{font-size:50%;}
.cmr-7{font-size:70%;}
.cmmi-5{font-size:50%;font-style: italic;}
.cmmi-7{font-size:70%;font-style: italic;}
.cmmi-10{font-style: italic;}
.cmsy-5{font-size:50%;}
.cmsy-7{font-size:70%;}
.cmex-7{font-size:70%;}
.cmex-7x-x-71{font-size:49%;}
.msbm-7{font-size:70%;}
.cmtt-10{font-family: monospace;}
.cmti-10{ font-style: italic;}
.cmbx-10{ font-weight: bold;}
.cmr-17x-x-120{font-size:204%;}
.cmsl-10{font-style: oblique;}
.cmti-7x-x-71{font-size:49%; font-style: italic;}
.cmbxti-10{ font-weight: bold; font-style: italic;}
p.noindent { text-indent: 0em }
td p.noindent { text-indent: 0em; margin-top:0em; }
p.nopar { text-indent: 0em; }
p.indent{ text-indent: 1.5em }
@media print {div.crosslinks {visibility:hidden;}}
a img { border-top: 0; border-left: 0; border-right: 0; }
center { margin-top:1em; margin-bottom:1em; }
td center { margin-top:0em; margin-bottom:0em; }
.Canvas { position:relative; }
li p.indent { text-indent: 0em }
.enumerate1 {list-style-type:decimal;}
.enumerate2 {list-style-type:lower-alpha;}
.enumerate3 {list-style-type:lower-roman;}
.enumerate4 {list-style-type:upper-alpha;}
div.newtheorem { margin-bottom: 2em; margin-top: 2em;}
.obeylines-h,.obeylines-v {white-space: nowrap; }
div.obeylines-v p { margin-top:0; margin-bottom:0; }
.overline{ text-decoration:overline; }
.overline img{ border-top: 1px solid black; }
td.displaylines {text-align:center; white-space:nowrap;}
.centerline {text-align:center;}
.rightline {text-align:right;}
div.verbatim {font-family: monospace; white-space: nowrap; text-align:left; clear:both; }
.fbox {padding-left:3.0pt; padding-right:3.0pt; text-indent:0pt; border:solid black 0.4pt; }
div.fbox {display:table}
div.center div.fbox {text-align:center; clear:both; padding-left:3.0pt; padding-right:3.0pt; text-indent:0pt; border:solid black 0.4pt; }
div.minipage{width:100%;}
div.center, div.center div.center {text-align: center; margin-left:1em; margin-right:1em;}
div.center div {text-align: left;}
div.flushright, div.flushright div.flushright {text-align: right;}
div.flushright div {text-align: left;}
div.flushleft {text-align: left;}
.underline{ text-decoration:underline; }
.underline img{ border-bottom: 1px solid black; margin-bottom:1pt; }
.framebox-c, .framebox-l, .framebox-r { padding-left:3.0pt; padding-right:3.0pt; text-indent:0pt; border:solid black 0.4pt; }
.framebox-c {text-align:center;}
.framebox-l {text-align:left;}
.framebox-r {text-align:right;}
span.thank-mark{ vertical-align: super }
span.footnote-mark sup.textsuperscript, span.footnote-mark a sup.textsuperscript{ font-size:80%; }
div.tabular, div.center div.tabular {text-align: center; margin-top:0.5em; margin-bottom:0.5em; }
table.tabular td p{margin-top:0em;}
table.tabular {margin-left: auto; margin-right: auto;}
div.td00{ margin-left:0pt; margin-right:0pt; }
div.td01{ margin-left:0pt; margin-right:5pt; }
div.td10{ margin-left:5pt; margin-right:0pt; }
div.td11{ margin-left:5pt; margin-right:5pt; }
table[rules] {border-left:solid black 0.4pt; border-right:solid black 0.4pt; }
td.td00{ padding-left:0pt; padding-right:0pt; }
td.td01{ padding-left:0pt; padding-right:5pt; }
td.td10{ padding-left:5pt; padding-right:0pt; }
td.td11{ padding-left:5pt; padding-right:5pt; }
table[rules] {border-left:solid black 0.4pt; border-right:solid black 0.4pt; }
.hline hr, .cline hr{ height : 1px; margin:0px; }
.tabbing-right {text-align:right;}
span.TEX {letter-spacing: -0.125em; }
span.TEX span.E{ position:relative;top:0.5ex;left:-0.0417em;}
a span.TEX span.E {text-decoration: none; }
span.LATEX span.A{ position:relative; top:-0.5ex; left:-0.4em; font-size:85%;}
span.LATEX span.TEX{ position:relative; left: -0.4em; }
div.float img, div.float .caption {text-align:center;}
div.figure img, div.figure .caption {text-align:center;}
.marginpar {width:20%; float:right; text-align:left; margin-left:auto; margin-top:0.5em; font-size:85%; text-decoration:underline;}
.marginpar p{margin-top:0.4em; margin-bottom:0.4em;}
.equation td{text-align:center; vertical-align:middle; }
td.eq-no{ width:5%; }
table.equation { width:100%; } 
div.math-display, div.par-math-display{text-align:center;}
math .texttt { font-family: monospace; }
math .textit { font-style: italic; }
math .textsl { font-style: oblique; }
math .textsf { font-family: sans-serif; }
math .textbf { font-weight: bold; }
.partToc a, .partToc, .likepartToc a, .likepartToc {line-height: 200%; font-weight:bold; font-size:110%;}
.chapterToc a, .chapterToc, .likechapterToc a, .likechapterToc, .appendixToc a, .appendixToc {line-height: 200%; font-weight:bold;}
.index-item, .index-subitem, .index-subsubitem {display:block}
.caption td.id{font-weight: bold; white-space: nowrap; }
table.caption {text-align:center;}
h1.partHead{text-align: center}
p.bibitem { text-indent: -2em; margin-left: 2em; margin-top:0.6em; margin-bottom:0.6em; }
p.bibitem-p { text-indent: 0em; margin-left: 2em; margin-top:0.6em; margin-bottom:0.6em; }
.paragraphHead, .likeparagraphHead { margin-top:2em; font-weight: bold;}
.subparagraphHead, .likesubparagraphHead { font-weight: bold;}
.quote {margin-bottom:0.25em; margin-top:0.25em; margin-left:1em; margin-right:1em; text-align:justify;}
.verse{white-space:nowrap; margin-left:2em}
div.maketitle {text-align:center;}
h2.titleHead{text-align:center;}
div.maketitle{ margin-bottom: 2em; }
div.author, div.date {text-align:center;}
div.thanks{text-align:left; margin-left:10%; font-size:85%; font-style:italic; }
div.author{white-space: nowrap;}
.quotation {margin-bottom:0.25em; margin-top:0.25em; margin-left:1em; }
h1.partHead{text-align: center}
.sectionToc, .likesectionToc {margin-left:2em;}
.subsectionToc, .likesubsectionToc {margin-left:4em;}
.subsubsectionToc, .likesubsubsectionToc {margin-left:6em;}
.frenchb-nbsp{font-size:75%;}
.frenchb-thinspace{font-size:75%;}
.figure img.graphics {margin-left:10%;}
/* end css.sty */

\title{Introduction aux integrales de surface}
\author{}
\date{}

\begin{document}
\maketitle

\textbf{Warning: 
requires JavaScript to process the mathematics on this page.\\ If your
browser supports JavaScript, be sure it is enabled.}

\begin{center}\rule{3in}{0.4pt}\end{center}

[
[
[]
[

\subsubsection{20.4 Introduction aux intégrales de surface}

Définition~20.4.1 Soit \Sigma = (D,F) une nappe paramétrée de classe
\mathcal{C}^1de \mathbb{R}~^3, où D est un compact de \mathbb{R}~^2
de frontière négligeable. Soit f une fonction définie et continue sur
l'image de \Sigma et à valeurs dans l'espace vectoriel normé E. On appelle
intégrale de f le long de \Sigma et on note \int ~
\int  _\Sigma~f(m) d\sigma l'élément de E

\int  \\int ~
_\Sigmaf(m) d\sigma =\int ~
\int  _D~f(F(u,v))
\ \partial~F \over \partial~u (u,v) ∧ \partial~F
\over \partial~v (u,v)\ du dv

En particulier, on appelle aire de \Sigma le nombre réel positif

m(\Sigma) =\int  \\int ~
_\Sigma d\sigma =\int ~ \\int
 _D\ \partial~F \over \partial~u
(u,v) ∧ \partial~F \over \partial~v (u,v)\
du dv

Le principal résultat sur ces intégrales de surface est l'invariance par
changement de paramétrage admissible

Théorème~20.4.1 Soit \Sigma_1 = (D_1,F_1) et
\Sigma_2 = (D_2,F_2) deux nappes paramétrées de
classe \mathcal{C}^1 équivalentes, où D_1 et D_2 sont
des compacts de \mathbb{R}~^2 de frontières négligeables. Soit f une
fonction définie sur l'image de \Sigma_1 et \Sigma_2, à valeurs
dans l'espace vectoriel normé E. Alors

\int  \\int ~
_\Sigma_1f(m) d\sigma =\int ~
\int  _\Sigma_2~f(m) d\sigma

En particulier, l'aire de la nappe est invariante par changement de
paramétrage.

Démonstration Soit \theta : D_1 \rightarrow~ D_2 un difféomorphisme de
l'intérieur de D_1 sur l'intérieur de D_2 vérifiant
F_1 = F_2 \cdot \theta. Un calcul fait dans le chapitre sur les
nappes paramétrées montre que (si on note
(u,v)\mapsto~F_1(u,v) et
(\lambda~,\mu)\mapsto~F_2(\lambda~,\mu))

 \partial~F_1 \over \partial~u (u,v) ∧ \partial~F_1
\over \partial~v (u,v) = j_\theta(u,v) \partial~F_2
\over \partial~\lambda~ (\theta(u,v)) ∧ \partial~F_2
\over \partial~\mu (\theta(u,v))

On en déduit que

\begin{align*} \int ~
\int  _\Sigma_1~f(m) d\sigma&& \%&
\\ & =& \int ~
\int  _D_1f(F_1~(u,v))
\ \partial~F \over \partial~u (u,v) ∧ \partial~F
\over \partial~v (u,v)\ du dv \%&
\\ & =& \int ~
\int  _D_1f(F_2~ \cdot
\theta(u,v)) \ \partial~F_2 \over
\partial~\lambda~ (\theta(u,v)) ∧ \partial~F_2 \over \partial~\mu
(\theta(u,v))\ j_\theta(u,v)
du dv\%& \\ & =&
\int  \\int ~
_D_2f(F_2(\lambda~,\mu)) \
\partial~F_2 \over \partial~\lambda~ (\lambda~,\mu) ∧ \partial~F_2
\over \partial~\mu (\lambda~,\mu)\ d\lambda~ d\mu \%&
\\ \end{align*}

par le théorème de changement de variables dans les intégrales doubles.

Remarque~20.4.1 Le lecteur attentif aura remarqué que nous avons modifié
légèrement les définitions d'une nappe paramétrée et de l'équivalence de
deux nappes paramétrées, de fa\ccon à ce que cela
nous arrange. Nous réclamons toute son indulgence pour ces modifications
de détail.

Comme cas particulier, cherchons l'aire d'une nappe de révolution d'axe
Oz. Soit \Gamma une méridienne de cette nappe, paramétrée en coordonnées
cylindriques par r = \phi(t) et z = \psi(t), t \in [a,b]. Un paramétrage de
la nappe est alors F(t,\theta) = O + \phi(t)\vecu(\theta) +
\psi(t)\veck, (t,\theta) \in [a,b] \times [0,2\pi~] si bien que
 \partial~F \over \partial~t (t,\theta) = \phi'(t)\vecu(\theta)
+ \psi'(t)\veck et  \partial~F \over \partial~\theta (t,\theta)
= \phi(t)\vecu'(\theta). On a donc

 \partial~F \over \partial~t (t,\theta) ∧ \partial~F \over \partial~\theta
(t,\theta) = \phi(t)\left (\phi'(t)\veck -
\psi(t)\vecu(\theta)\right )

et donc

\ \partial~F \over \partial~t (t,\theta) ∧ \partial~F
\over \partial~\theta (t,\theta)\ =
\phi(t)\sqrt\phi'(t)^2  +
\psi'(t)^2

On en déduit que

\begin{align*} m(\Sigma)& =& \\int
 \int ~
_[a,b]\times[0,2\pi~]\phi(t)\sqrt\phi'(t)^2
 + \psi'(t)^2 dt d\theta\%& \\ &
=& 2\pi~\int ~
_a^b\phi(t)\sqrt\phi'(t)^2
 + \psi'(t)^2 dt \%& \\ &
=& 2\pi~\int  _\Gamma~r ds
\%& \\ \end{align*}

en notant r = \phi(t) et ds = \sqrt\phi'(t)^2  +
\psi'(t)^2 dt la différentielle de l'abscisse curviligne sur
\Gamma. On obtient donc

Proposition~20.4.2 Soit \Sigma la nappe de révolution engendrée par la
rotation de la méridienne \Gamma autour de la droite D. Soit ds la
différentielle de l'abscisse curviligne de \Gamma et r la distance d'un point
de \Gamma à la droite D. Alors l'aire de la nappe est égale à
2\pi~\int  _\Gamma~r ds.

[
[
[
[

\end{document}


% Index
\printindex
\end{document}
