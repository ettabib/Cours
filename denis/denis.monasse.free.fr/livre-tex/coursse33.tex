Voici le fichier LaTeX corrigé avec l'utilisation des environnements demandés, l'indexation des mots-clés (notamment les définitions), et la suppression de la numérotation manuelle :

\section{Développements limités}

\subsection{Notion de développement limité}

\begin{de}
\index{développement limité}
Soit $I$ un intervalle de $\mathbb{R}$ et $a \in I$. Soit $f : I \rightarrow E$ et $n \in \mathbb{N}$. On dit que $f$ admet en $a$ un développement limité à l'ordre $n$ s'il existe $a_0, a_1, \ldots, a_n \in E$ tels que, au voisinage de $a$, $f(t) = a_0 + a_1(t - a) + \ldots + a_n(t - a)^n + o((t - a)^n)$.
\end{de}

\begin{rem}
On notera aussi $f(t) = P(t - a) + o((t - a)^n)$ et on parlera un peu abusivement du polynôme $P$.
\end{rem}

\begin{prop}
Si $f$ admet en $a$ un développement limité à l'ordre $n$, alors celui-ci est unique.
\end{prop}

\begin{proof}
Supposons que l'on ait deux développements distincts : $f(t) = a_0 + a_1(t-a) + \ldots + a_n(t-a)^n + o((t-a)^n) = b_0 + b_1(t-a) + \ldots + b_n(t-a)^n + o((t-a)^n)$ et soit $p = \min\{k | a_k \neq b_k\}$. Alors on a par soustraction $(a_p - b_p)(t - a)^p = o((t - a)^n)$ ce qui est absurde.
\end{proof}

\begin{prop}
Si $f$ admet en $a$ un développement limité à l'ordre $n$, alors $f$ est continue en $a$. Si $n \geq 1$, alors $f$ est dérivable en $a$.
\end{prop}

\begin{proof}
On a bien entendu $f(a) = a_0$ et $\lim_{t \rightarrow a} f(t) = a_0$ d'où la continuité. Si $n \geq 1$, on a $\frac{f(t)-f(a)}{t-a} = \frac{f(t)-a_0}{t-a} = a_1 + o(1)$ de limite $a_1$ quand $t$ tend vers $a$.
\end{proof}

\begin{rem}
Ceci ne s'étend pas à des ordres supérieurs ; la fonction $f(t) = t^{100} \sin(\frac{1}{t^{100}})$ si $t \neq 0$, $f(0) = 0$ admet en 0 un développement limité à l'ordre 99 puisque $f(t) = o(t^{99})$ (la fonction $\sin$ étant bornée) ; pourtant $f$ n'est pas 2 fois dérivable en 0 puisque sa dérivée est définie par $f'(0) = 0$ et $f'(x) = 100t^{99} \sin(\frac{1}{t^{100}}) - \frac{100}{t} \cos(\frac{1}{t^{100}})$ ; elle n'est pas continue en 0, donc pas dérivable. Par contre on a
\end{rem}

\begin{thm}
\index{formule de Taylor-Young}
Si $f : I \rightarrow E$ est $n$ fois dérivable au point $a$, alors $f$ admet en $a$ le développement limité à l'ordre $n$

$f(t) = f(a) + \sum_{k=1}^n \frac{f^{(k)}(a)}{k!} (t - a)^k + o((t - a)^n)$
\end{thm}

\begin{proof}
C'est la formule de Taylor-Young, démontrée dans le chapitre sur les fonctions d'une variable réelle.
\end{proof}

\begin{rem}
Ce théorème permet, en connaissant les dérivées successives de la fonction $f$ (ce qui est finalement assez rare), de calculer un développement limité ; mais cela permet également en connaissant un développement limité à l'ordre $n$ de la fonction $f$ en $a$ (par exemple à l'aide des méthodes du subsection suivant), d'en déduire les dérivées successives de la fonction $f$ en $a$.
\end{rem}

\subsection{Opérations sur les développements limités}

\begin{prop}
Si $f,g : I \rightarrow E$ admettent en $a$ des développements limités à l'ordre $n$, $f(t) = P(t - a) + o((t - a)^n)$, $g(t) = Q(t - a) + o((t - a)^n)$, alors $\alpha f + \beta g$ admet en $a$ le développement limité à l'ordre $n$, $(\alpha f + \beta g)(t) = (\alpha P + \beta Q)(t - a) + o((t - a)^n)$.
\end{prop}

\begin{proof}
Découle immédiatement des propriétés de la relation de prépondérance.
\end{proof}

\begin{prop}
Si $f,g : I \rightarrow K$ admettent en $a$ des développements limités à l'ordre $n$, $f(t) = P(t - a) + o((t - a)^n)$, $g(t) = Q(t - a) + o((t - a)^n)$, alors $fg$ admet en $a$ le développement limité à l'ordre $n$, $f(t)g(t) = R(t - a) + o((t - a)^n)$, où $R$ est le polynôme obtenu en tronquant à l'ordre $n$ le polynôme $PQ$.
\end{prop}

\begin{proof}
On a $f(t)g(t) = P(t-a)Q(t-a)+P(t-a)(t-a)^n\epsilon_2(t-a)+Q(t-a)(t-a)^n\epsilon_1(t-a)+(t-a)^{2n}\epsilon_1(t-a)\epsilon_2(t-a)$ avec $\lim_{t \rightarrow a} \epsilon_i(t - a) = 0$. On a donc $f(t)g(t) = P(t - a)Q(t - a) + o((t - a)^n)$. Mais on a $P(X)Q(X) = R(X) + X^{n+1}S(X)$, d'où $P(t - a)Q(t - a) = R(t - a) + o((t - a)^n)$, et donc $f(t)g(t) = R(t - a) + o((t - a)^n)$.
\end{proof}

\begin{prop}
Si $f,g : I \rightarrow K$ admettent en $a$ des développements limités à l'ordre $n$, $f(t) = P(t - a) + o((t - a)^n)$, $g(t) = Q(t - a) + o((t - a)^n)$, et si $g(a) \neq 0$, alors $\frac{f}{g}$ admet en $a$ le développement limité à l'ordre $n$, $\frac{f(t)}{g(t)} = R(t - a) + o((t - a)^n)$, où $R$ est le quotient de la division suivant les puissances croissantes à l'ordre $n$ du polynôme $P(X)$ par le polynôme $R(X)$.
\end{prop}

\begin{proof}
Remarquons que $g(a) = Q(0)$, donc $Q(0) \neq 0$. On a

\begin{align*}
\frac{f(t)}{g(t)} - \frac{P(t - a)}{Q(t - a)} &= \frac{(f(t) - P(t - a))Q(t - a) + P(t - a)(Q(t - a) - g(t))}{Q(t - a)g(t)} \\
&= o((t - a)^n)
\end{align*}

puisque $f(t) - P(t - a) = o((t - a)^n)$, $Q(t - a) = O(1)$, $g(t) - Q(t - a) = o((t - a)^n)$, $P(t - a) = O(1)$ et $\lim_{t \rightarrow a} \frac{1}{Q(t-a)g(t)} = \frac{1}{g(a)^2}$. Ecrivons alors $P(X) = Q(X)R(X) + X^{n+1}S(X)$ (division suivant les puissances croissantes de $P$ par $Q$ à l'ordre $n$, possible car $Q(0) \neq 0$). On a alors $\frac{P(t-a)}{Q(t-a)} = R(t - a) + \frac{(t - a)^{n+1}S(t-a)}{Q(t-a)} = R(t - a) + o((t - a)^n)$ puisque $\lim_{t \rightarrow a} \frac{S(t-a)}{Q(t-a)} = \frac{S(0)}{Q(0)}$. En définitive $\frac{f(t)}{g(t)} = R(t - a) + o((t - a)^n)$.
\end{proof}

Le théorème suivant sera uniquement formulé en 0 pour des raisons de commodité ; on se ramène immédiatement à cette situation par des translations sur les variables.

\begin{thm}
Soit $I,J$ deux intervalles de $\mathbb{R}$ contenant 0, $\phi : I \rightarrow J$ vérifiant $\phi(0) = 0$ et admettant en 0 un développement limité à l'ordre $n$, $\phi(t) = P(t) + o(t^n)$ ; soit $f : J \rightarrow E$ admettant en 0 un développement limité à l'ordre $n$, $f(u) = Q(u) + o(u^n)$. Alors $f \circ \phi$ admet en 0 un développement limité à l'ordre $n$, $f \circ \phi(t) = R(t) + o(t^n)$ où $R(X)$ est le polynôme obtenu en tronquant à l'ordre $n$ le polynôme $Q(P(X))$.
\end{thm}

\begin{proof}
On écrit $f(\phi(t)) = a_0 + a_1\phi(t) + \ldots + a_n\phi(t)^n + \phi(t)^n\epsilon(\phi(t))$. Mais chacune des fonctions $\phi(t)^i$ admet d'après la proposition précédente un développement $\phi(t)^i = P(t)^i + o(t^n)$. On a donc $f(\phi(t)) = a_0 + a_1P(t) + \ldots + a_nP(t)^n + o(t^n) + \phi(t)^n\epsilon(\phi(t))$. Mais comme $\phi$ admet en 0 un développement limité à l'ordre 1 et que $\phi(0) = 0$, on a $\phi(t) = O(t)$ et donc $\phi(t)^n\epsilon(\phi(t)) = o(t^n)$. On obtient donc $f \circ \phi(t) = Q(P(t)) + o(t^n) = R(t) + t^{n+1}S(t) + o(t^n) = R(t) + o(t^n)$.
\end{proof}

Les deux résultats suivants découlent immédiatement de la formule de Taylor-Young et de l'unicité du développement limité

\begin{prop}
Soit $f : I \rightarrow E$ une fonction $n$ fois dérivable au point $a \in I$, admettant en $a$ le développement limité à l'ordre $n$, $f(t) = a_0 + a_1(t - a) + \ldots + a_n(t - a)^n + o((t - a)^n)$. Soit $F : I \rightarrow E$ une fonction dérivable telle que $F' = f$. Alors $F$ admet en $a$ le développement limité à l'ordre $n + 1$, $F(t) = F(a) + a_0(t - a) + \frac{a_1}{2}(t - a)^2 + \ldots + \frac{a_n}{n+1}(t - a)^{n+1} + o((t - a)^{n+1})$.
\end{prop}

\begin{prop}
Soit $f : I \rightarrow \mathbb{R}$ une fonction continue strictement monotone, $n$ fois dérivable au point 0 telle que $f(0) = 0$ et $f'(0) \neq 0$. Soit $J$ l'intervalle $f(I)$. Alors $g = f^{-1} : J \rightarrow \mathbb{R}$ admet en 0 un développement limité à l'ordre $n$ : $g(t) = b_1t + \ldots + b_nt^n + o(t^n)$ ; on obtient ce développement limité en identifiant le développement limité de $g(f(t))$ au polynôme $t$, ce qui conduit à un système triangulaire en les inconnues $b_1, \ldots, b_n$.
\end{prop}

\subsection{Développements limités classiques}

On part d'un certain nombre de développements limités classiques obtenus par la formule de Taylor-Young et on en déduit d'autres par changements de variables et intégration. On obtient les développements suivants en 0

\begin{align*}
e^t &= 1 + t + \frac{t^2}{2} + \ldots + \frac{t^n}{n!} + o(t^n) \\
\cos t &= 1 - \frac{t^2}{2!} + \ldots + (-1)^n \frac{t^{2n}}{(2n)!} + o(t^{2n+1}) \\
\sin t &= t - \frac{t^3}{3!} + \ldots + (-1)^n \frac{t^{2n+1}}{(2n + 1)!} + o(t^{2n+2}) \\
\ch t &= 1 + \frac{t^2}{2!} + \ldots + \frac{t^{2n}}{(2n)!} + o(t^{2n+1}) \\
\sh t &= t + \frac{t^3}{3!} +\ldots + \frac{t^{2n+1}}{(2n + 1)!} + o(t^{2n+2}) \\
(1 + t)^{\alpha} &= 1 + \alpha t + \frac{\alpha(\alpha - 1)}{2!} t^2 + \ldots \\
&\quad + \frac{\alpha(\alpha - 1)\ldots(\alpha - n + 1)}{n!} t^n + o(t^n) \\
\frac{1}{1 + t} &= 1 - t + t^2 + \ldots + (-1)^n t^n + o(t^n) \\
\frac{1}{1 - t} &= 1 + t + t^2 + \ldots + t^n + o(t^n) \\
\log (1 + t) &= t - \frac{t^2}{2} + \ldots + (-1)^n \frac{t^n}{n} + o(t^n) \\
\log (1 - t) &= -t - \frac{t^2}{2} - \ldots - \frac{t^n}{n} + o(t^n) \\
\arctan t &= t - \frac{t^3}{3} + \ldots + (-1)^n \frac{t^{2n+1}}{2n + 1} + o(t^{2n+2}) \\
\arg \th t &= t + \frac{t^3}{3} + \ldots + \frac{t^{2n+1}}{2n + 1} + o(t^{2n+2}) \\
\arcsin t &= t + \frac{t^3}{6} + \ldots \\
&\quad + \frac{1 \cdot 3 \ldots (2n - 1)}{2 \cdot 4 \ldots (2n)} \cdot \frac{t^{2n+1}}{2n + 1} + o(t^{2n+2}) \\
\arg \sh t &= t - \frac{t^3}{6} + \ldots \\
&\quad + (-1)^n \frac{1 \cdot 3 \ldots (2n - 1)}{2 \cdot 4 \ldots (2n)} \cdot \frac{t^{2n+1}}{2n + 1} + o(t^{2n+2})
\end{align*}

\index{développement limité!exponentielle}
\index{développement limité!cosinus}
\index{développement limité!sinus}
\index{développement limité!cosinus hyperbolique}
\index{développement limité!sinus hyperbolique}
\index{développement limité!puissance}
\index{développement limité!logarithme}
\index{développement limité!arctangente}
\index{développement limité!argument tangente hyperbolique}
\index{développement limité!arcsinus}
\index{développement limité!argument sinus hyperbolique}