\documentclass[]{article}
\usepackage[T1]{fontenc}
\usepackage{lmodern}
\usepackage{amssymb,amsmath}
\usepackage{ifxetex,ifluatex}
\usepackage{fixltx2e} % provides \textsubscript
% use upquote if available, for straight quotes in verbatim environments
\IfFileExists{upquote.sty}{\usepackage{upquote}}{}
\ifnum 0\ifxetex 1\fi\ifluatex 1\fi=0 % if pdftex
  \usepackage[utf8]{inputenc}
\else % if luatex or xelatex
  \ifxetex
    \usepackage{mathspec}
    \usepackage{xltxtra,xunicode}
  \else
    \usepackage{fontspec}
  \fi
  \defaultfontfeatures{Mapping=tex-text,Scale=MatchLowercase}
  \newcommand{\euro}{€}
\fi
% use microtype if available
\IfFileExists{microtype.sty}{\usepackage{microtype}}{}
\ifxetex
  \usepackage[setpagesize=false, % page size defined by xetex
              unicode=false, % unicode breaks when used with xetex
              xetex]{hyperref}
\else
  \usepackage[unicode=true]{hyperref}
\fi
\hypersetup{breaklinks=true,
            bookmarks=true,
            pdfauthor={},
            pdftitle={Developpements limites},
            colorlinks=true,
            citecolor=blue,
            urlcolor=blue,
            linkcolor=magenta,
            pdfborder={0 0 0}}
\urlstyle{same}  % don't use monospace font for urls
\setlength{\parindent}{0pt}
\setlength{\parskip}{6pt plus 2pt minus 1pt}
\setlength{\emergencystretch}{3em}  % prevent overfull lines
\setcounter{secnumdepth}{0}
 
/* start css.sty */
.cmr-5{font-size:50%;}
.cmr-7{font-size:70%;}
.cmmi-5{font-size:50%;font-style: italic;}
.cmmi-7{font-size:70%;font-style: italic;}
.cmmi-10{font-style: italic;}
.cmsy-5{font-size:50%;}
.cmsy-7{font-size:70%;}
.cmex-7{font-size:70%;}
.cmex-7x-x-71{font-size:49%;}
.msbm-7{font-size:70%;}
.cmtt-10{font-family: monospace;}
.cmti-10{ font-style: italic;}
.cmbx-10{ font-weight: bold;}
.cmr-17x-x-120{font-size:204%;}
.cmsl-10{font-style: oblique;}
.cmti-7x-x-71{font-size:49%; font-style: italic;}
.cmbxti-10{ font-weight: bold; font-style: italic;}
p.noindent { text-indent: 0em }
td p.noindent { text-indent: 0em; margin-top:0em; }
p.nopar { text-indent: 0em; }
p.indent{ text-indent: 1.5em }
@media print {div.crosslinks {visibility:hidden;}}
a img { border-top: 0; border-left: 0; border-right: 0; }
center { margin-top:1em; margin-bottom:1em; }
td center { margin-top:0em; margin-bottom:0em; }
.Canvas { position:relative; }
li p.indent { text-indent: 0em }
.enumerate1 {list-style-type:decimal;}
.enumerate2 {list-style-type:lower-alpha;}
.enumerate3 {list-style-type:lower-roman;}
.enumerate4 {list-style-type:upper-alpha;}
div.newtheorem { margin-bottom: 2em; margin-top: 2em;}
.obeylines-h,.obeylines-v {white-space: nowrap; }
div.obeylines-v p { margin-top:0; margin-bottom:0; }
.overline{ text-decoration:overline; }
.overline img{ border-top: 1px solid black; }
td.displaylines {text-align:center; white-space:nowrap;}
.centerline {text-align:center;}
.rightline {text-align:right;}
div.verbatim {font-family: monospace; white-space: nowrap; text-align:left; clear:both; }
.fbox {padding-left:3.0pt; padding-right:3.0pt; text-indent:0pt; border:solid black 0.4pt; }
div.fbox {display:table}
div.center div.fbox {text-align:center; clear:both; padding-left:3.0pt; padding-right:3.0pt; text-indent:0pt; border:solid black 0.4pt; }
div.minipage{width:100%;}
div.center, div.center div.center {text-align: center; margin-left:1em; margin-right:1em;}
div.center div {text-align: left;}
div.flushright, div.flushright div.flushright {text-align: right;}
div.flushright div {text-align: left;}
div.flushleft {text-align: left;}
.underline{ text-decoration:underline; }
.underline img{ border-bottom: 1px solid black; margin-bottom:1pt; }
.framebox-c, .framebox-l, .framebox-r { padding-left:3.0pt; padding-right:3.0pt; text-indent:0pt; border:solid black 0.4pt; }
.framebox-c {text-align:center;}
.framebox-l {text-align:left;}
.framebox-r {text-align:right;}
span.thank-mark{ vertical-align: super }
span.footnote-mark sup.textsuperscript, span.footnote-mark a sup.textsuperscript{ font-size:80%; }
div.tabular, div.center div.tabular {text-align: center; margin-top:0.5em; margin-bottom:0.5em; }
table.tabular td p{margin-top:0em;}
table.tabular {margin-left: auto; margin-right: auto;}
div.td00{ margin-left:0pt; margin-right:0pt; }
div.td01{ margin-left:0pt; margin-right:5pt; }
div.td10{ margin-left:5pt; margin-right:0pt; }
div.td11{ margin-left:5pt; margin-right:5pt; }
table[rules] {border-left:solid black 0.4pt; border-right:solid black 0.4pt; }
td.td00{ padding-left:0pt; padding-right:0pt; }
td.td01{ padding-left:0pt; padding-right:5pt; }
td.td10{ padding-left:5pt; padding-right:0pt; }
td.td11{ padding-left:5pt; padding-right:5pt; }
table[rules] {border-left:solid black 0.4pt; border-right:solid black 0.4pt; }
.hline hr, .cline hr{ height : 1px; margin:0px; }
.tabbing-right {text-align:right;}
span.TEX {letter-spacing: -0.125em; }
span.TEX span.E{ position:relative;top:0.5ex;left:-0.0417em;}
a span.TEX span.E {text-decoration: none; }
span.LATEX span.A{ position:relative; top:-0.5ex; left:-0.4em; font-size:85%;}
span.LATEX span.TEX{ position:relative; left: -0.4em; }
div.float img, div.float .caption {text-align:center;}
div.figure img, div.figure .caption {text-align:center;}
.marginpar {width:20%; float:right; text-align:left; margin-left:auto; margin-top:0.5em; font-size:85%; text-decoration:underline;}
.marginpar p{margin-top:0.4em; margin-bottom:0.4em;}
.equation td{text-align:center; vertical-align:middle; }
td.eq-no{ width:5%; }
table.equation { width:100%; } 
div.math-display, div.par-math-display{text-align:center;}
math .texttt { font-family: monospace; }
math .textit { font-style: italic; }
math .textsl { font-style: oblique; }
math .textsf { font-family: sans-serif; }
math .textbf { font-weight: bold; }
.partToc a, .partToc, .likepartToc a, .likepartToc {line-height: 200%; font-weight:bold; font-size:110%;}
.chapterToc a, .chapterToc, .likechapterToc a, .likechapterToc, .appendixToc a, .appendixToc {line-height: 200%; font-weight:bold;}
.index-item, .index-subitem, .index-subsubitem {display:block}
.caption td.id{font-weight: bold; white-space: nowrap; }
table.caption {text-align:center;}
h1.partHead{text-align: center}
p.bibitem { text-indent: -2em; margin-left: 2em; margin-top:0.6em; margin-bottom:0.6em; }
p.bibitem-p { text-indent: 0em; margin-left: 2em; margin-top:0.6em; margin-bottom:0.6em; }
.paragraphHead, .likeparagraphHead { margin-top:2em; font-weight: bold;}
.subparagraphHead, .likesubparagraphHead { font-weight: bold;}
.quote {margin-bottom:0.25em; margin-top:0.25em; margin-left:1em; margin-right:1em; text-align:\jmathustify;}
.verse{white-space:nowrap; margin-left:2em}
div.maketitle {text-align:center;}
h2.titleHead{text-align:center;}
div.maketitle{ margin-bottom: 2em; }
div.author, div.date {text-align:center;}
div.thanks{text-align:left; margin-left:10%; font-size:85%; font-style:italic; }
div.author{white-space: nowrap;}
.quotation {margin-bottom:0.25em; margin-top:0.25em; margin-left:1em; }
h1.partHead{text-align: center}
.sectionToc, .likesectionToc {margin-left:2em;}
.subsectionToc, .likesubsectionToc {margin-left:4em;}
.subsubsectionToc, .likesubsubsectionToc {margin-left:6em;}
.frenchb-nbsp{font-size:75%;}
.frenchb-thinspace{font-size:75%;}
.figure img.graphics {margin-left:10%;}
/* end css.sty */

\title{Developpements limites}
\author{}
\date{}

\begin{document}
\maketitle

\textbf{Warning: 
requires JavaScript to process the mathematics on this page.\\ If your
browser supports JavaScript, be sure it is enabled.}

\begin{center}\rule{3in}{0.4pt}\end{center}

{[}
{[}
{[}{]}
{[}

\subsubsection{6.2 Développements limités}

\paragraph{6.2.1 Notion de développement limité}

Définition~6.2.1 Soit I un intervalle de \mathbb{R}~ et a \in I. Soit f : I \rightarrow~ E et n
\in \mathbb{N}~. On dit que f admet en a un développement limité à l'ordre n s'il
existe
a\_0,a\_1,\\ldots,a\_n~
\in E tels que, au voisinage de a, f(t) = a\_0 + a\_1(t -
a) + \\ldots~ +
a\_n(t - a)^n + o((t - a)^n).

Remarque~6.2.1 On notera aussi f(t) = P(t - a) + o((t - a)^n)
et on parlera un peu abusivement du polynôme P.

Proposition~6.2.1 Si f admet en a un développement limité à l'ordre n,
alors celui ci est unique.

Démonstration Supposons que l'on ait deux développements distincts~:
f(t) = a\_0 + a\_1(t-a) +
\\ldots~ +
a\_n(t-a)^n + o((t-a)^n) = b\_0 +
b\_1(t-a) +
\\ldots~ +
b\_n(t-a)^n + o((t-a)^n)et soit p
=\
min\k∣a\_k\mathrel\neq~b\_k\.
Alors on a par soustraction (a\_p - b\_p)(t -
a)^p = o((t - a)^n) ce qui est absurde.

Proposition~6.2.2 Si f admet en a un développement limité à l'ordre n,
alors f est continue en a. Si n ≥ 1, alors f est dérivable en a.

Démonstration On a bien entendu f(a) = a\_0 et
lim\_t\rightarrow~af(t) = a\_0~ d'où la
continuité. Si n ≥ 1, on a  f(t)-f(a) \over t-a =
f(t)-a\_0 \over t-a = a\_1 + o(1) de
limite a\_1 quand t tend vers a.

Remarque~6.2.2 Ceci ne s'étend pas à des ordres supérieurs~; la fonction
f(t) = t^100 sin~ ( 1
\over t^100 ) si
t\neq~0, f(0) = 0 admet en 0 un développement
limité à l'ordre 99 puisque f(t) = o(t^99) (la fonction
sin~ étant bornée)~; pourtant f n'est pas 2
fois dérivable en 0 puisque sa dérivée est définie par f'(0) = 0 et
f'(x) = 100t^99 sin~ ( 1
\over t^100 ) - 100 \over
t  cos~ ( 1 \over
t^100 )~; elle n'est pas continue en 0, donc pas dérivable.
Par contre on a

Théorème~6.2.3 Si f : I \rightarrow~ E est n fois dérivable au point a, alors f
admet en a le développement limité à l'ordre n

f(t) = f(a) + \sum \_k=1^n~
f^(k)(a) \over k! (t - a)^k +
o((t - a)^n)

Démonstration C'est la formule de Taylor Young, démontrée dans le
chapitre sur les fonctions d'une variable réelle.

Remarque~6.2.3 Ce théorème permet, en connaissant les dérivées
successives de la fonction f (ce qui est finalement assez rare), de
calculer un développement limité~; mais cela permet également en
connaissant un développement limité à l'ordre n de la fonction f en a
(par exemple à l'aide des méthodes du paragraphe suivant), d'en déduire
les dérivées successives de la fonction f en a.

\paragraph{6.2.2 Opérations sur les développements limités}

Proposition~6.2.4 Si f,g : I \rightarrow~ E admettent en a des développements
limités à l'ordre n, f(t) = P(t - a) + o((t - a)^n),g(t) =
Q(t - a) + o((t - a)^n), alors \alpha~f + \beta~g admet en a le
développement limité à l'ordre n, (\alpha~f + \beta~g)(t) = (\alpha~P + \beta~Q)(t - a) +
o((t - a)^n).

Démonstration Découle immédiatement des propriétés de la relation de
prépondérance.

Proposition~6.2.5 Si f,g : I \rightarrow~ K admettent en a des développements
limités à l'ordre n, f(t) = P(t - a) + o((t - a)^n),g(t) =
Q(t - a) + o((t - a)^n), alors fg admet en a le développement
limité à l'ordre n, f(t)g(t) = R(t - a) + o((t - a)^n), où R
est le polynôme obtenu en tronquant à l'ordre n le polynôme PQ.

Démonstration On a f(t)g(t) =
P(t-a)Q(t-a)+P(t-a)(t-a)^n\epsilon\_2(t-a)+Q(t-a)(t-a)^n\epsilon\_1(t-a)+(t-a)^2n\epsilon\_1(t-a)\epsilon\_2(t-a)
avec lim\_t\rightarrow~a\epsilon\_i~(t - a) = 0.
On a donc f(t)g(t) = P(t - a)Q(t - a) + o((t - a)^n). Mais on
a P(X)Q(X) = R(X) + X^n+1S(X), d'où P(t - a)Q(t - a) = R(t -
a) + o((t - a)^n), et donc f(t)g(t) = R(t - a) + o((t -
a)^n).

Proposition~6.2.6 Si f,g : I \rightarrow~ K admettent en a des développements
limités à l'ordre n, f(t) = P(t - a) + o((t - a)^n),g(t) =
Q(t - a) + o((t - a)^n), et si
g(a)\neq~0, alors  f \over g
admet en a le développement limité à l'ordre n,  f(t)
\over g(t) = R(t - a) + o((t - a)^n), où R
est le quotient de la division suivant les puissances croissantes à
l'ordre n du polynôme P(X) par le polynôme R(X).

Démonstration Remarquons que g(a) = Q(0), donc
Q(0)\neq~0. On a

\begin{align*} f(t) \over g(t)
- P(t - a) \over Q(t - a) && \%&
\\ & =& (f(t) - P(t - a))Q(t - a) +
P(t - a)(Q(t - a) - g(t)) \over Q(t - a)g(t) \%&
\\ & =& o((t - a)^n) \%&
\\ \end{align*}

puisque f(t) - P(t - a) = o((t - a)^n), Q(t - a) = O(1), g(t)
- Q(t - a) = o((t - a)^n), P(t - a) = O(1) et
lim\_t\rightarrow~a~ 1 \over
Q(t-a)g(t) = 1 \over g(a)^2 . Ecrivons
alors P(X) = Q(X)R(X) + X^n+1S(X) (division suivant les
puissances croissantes de P par Q à l'ordre n, possible car
Q(0)\neq~0). On a alors  P(t-a)
\over Q(t-a) = R(t - a) + (t - a)^n+1
S(t-a) \over Q(t-a) = R(t - a) + o((t -
a)^n) puisque lim\_t\rightarrow~a~
S(t-a) \over Q(t-a) = S(0) \over
Q(0) . En définitive  f(t) \over g(t) = R(t - a) +
o((t - a)^n).

Le théorème suivant sera uniquement formulé en 0 pour des raisons de
commodité~; on se ramène immédiatement à cette situation par des
translations sur les variables.

Théorème~6.2.7 Soit I,J deux intervalles de \mathbb{R}~ contenant 0, \phi : I \rightarrow~ J
vérifiant \phi(0) = 0 et admettant en 0 un développement limité à l'ordre
n, \phi(t) = P(t) + o(t^n)~; soit f : J \rightarrow~ E admettant en 0 un
développement limité à l'ordre n, f(u) = Q(u) + o(u^n). Alors
f \cdot \phi admet en 0 un développement limité à l'ordre n, f \cdot \phi(t) = R(t) +
o(t^n) où R(X) est le polynôme obtenu en tronquant à l'ordre
n le polynôme Q(P(X)).

Démonstration On écrit f(\phi(t)) = a\_0 + a\_1\phi(t) +
\\ldots~ +
a\_n\phi(t)^n + \phi(t)^n\epsilon(\phi(t)). Mais chacune
des fonctions \phi(t)^i admet d'après la proposition précédente
un développement \phi(t)^i = P(t)^i +
o(t^n). On a donc f(\phi(t)) = a\_0 + a\_1P(t) +
\\ldots~ +
a\_nP(t)^n + o(t^n) +
\phi(t)^n\epsilon(\phi(t)). Mais comme \phi admet en 0 un développement
limité à l'ordre 1 et que \phi(0) = 0, on a \phi(t) = O(t) et donc
\phi(t)^n\epsilon(\phi(t)) = o(t^n). On obtient donc f \cdot \phi(t) =
Q(P(t)) + o(t^n) = R(t) + t^n+1S(t) +
o(t^n) = R(t) + o(t^n).

Les deux résultats suivants découlent immédiatement de la formule de
Taylor-Young et de l'unicité du développement limité

Proposition~6.2.8 Soit f : I \rightarrow~ E une fonction n fois dérivable au point
a \in I, admettant en a le développement limité à l'ordre n, f(t) =
a\_0 + a\_1(t - a) +
\\ldots~ +
a\_n(t - a)^n + o((t - a)^n). Soit F : I \rightarrow~
E une fonction dérivable telle que F' = f. Alors F admet en a le
développement limité à l'ordre n + 1, F(t) = F(a) + a\_0(t - a)
+ a\_1 \over 2 (t - a)^2 +
\\ldots~ +
a\_n \over n+1 (t - a)^n+1 + o((t
- a)^n+1).

Proposition~6.2.9 Soit f : I \rightarrow~ \mathbb{R}~ une fonction continue strictement
monotone, n fois dérivable au point 0 telle que f(0) = 0 et
f'(0)\neq~0. Soit J l'intervalle f(I). Alors g =
f^-1 : J \rightarrow~ \mathbb{R}~ admet en 0 un développement limité à l'ordre n~:
g(t) = b\_1t +
\\ldots~ +
b\_nt^n + o(t^n)~; on obtient ce
développement limité en identifiant le développement limité de g(f(t))
au polynôme t, ce qui conduit à un système triangulaire en les inconnues
b\_1,\\ldots,b\_n~.

\paragraph{6.2.3 Développements limités classiques}

On part d'un certain nombre de développements limités classiques obtenus
par la formule de Taylor-Young et on en déduit d'autres par changements
de variables et intégration. On obtient les développements suivants en 0

\begin{align*} e^t& =& 1 + t +
t^2 \over 2 +
\\ldots~ +
t^n \over n! + o(t^n) \%&
\\ cos~ t& =& 1
- t^2 \over 2! +
\\ldots~ +
(-1)^n t^2n \over (2n)! +
o(t^2n+1) \%& \\
sin t& =& t - t^3~
\over 3! +
\\ldots~ +
(-1)^n t^2n+1 \over (2n + 1)! +
o(t^2n+2)\%& \\
\mathrmch~ t& =& 1 +
t^2 \over 2! +
\\ldots~ +
t^2n \over (2n)! + o(t^2n+1) \%&
\\
\mathrmsh~ t& =& t +
t^3 \over 3! +
\\ldots~ +
t^2n+1 \over (2n + 1)! +
o(t^2n+2) \%& \\ (1 +
t)^\alpha~& =& 1 + \alpha~t + \alpha~(\alpha~ - 1) \over 2!
t^2 +
\\ldots~ \%&
\\ & \text & + \alpha~(\alpha~
- 1)\\ldots~(\alpha~ - n +
1) \over n! t^n + o(t^n) \%&
\\  1 \over 1 + t &
=& 1 - t + t^2 +
\\ldots~ +
(-1)^nt^n + o(t^n) \%&
\\  1 \over 1 - t &
=& 1 + t + t^2 +
\\ldots~ +
t^n + o(t^n) \%& \\
log (1 + t)& =& t - t^2~
\over 2 +
\\ldots~ +
(-1)^n t^n \over n +
o(t^n) \%& \\
log (1 - t)& =& -t - t^2~
\over 2
-\\ldots~ -
t^n \over n + o(t^n) \%&
\\
\mathrmarctg~ t& =& t -
t^3 \over 3 +
\\ldots~ +
(-1)^n t^2n+1 \over 2n + 1 +
o(t^2n+2) \%& \\
arg~
\mathrmth~ t& =& t +
t^3 \over 3 +
\\ldots~ +
t^2n+1 \over 2n + 1 + o(t^2n+2)
\%& \\ arcsin~
t& =& t + t^3 \over 6 +
\\ldots~ \%&
\\ & \text & +
1.3\\ldots~(2n - 1)
\over
2.4\\ldots(2n)~ 
t^2n+1 \over 2n + 1 + o(t^2n+2)
\%& \\ arg~
\mathrmsh~ t& =& t -
t^3 \over 6 +
\\ldots~ \%&
\\ & \text &
+(-1)^n
1.3\\ldots~(2n - 1)
\over
2.4\\ldots(2n)~ 
t^2n+1 \over 2n + 1 + o(t^2n+2)
\%& \\ \end{align*}

{[}
{[}
{[}
{[}

\end{document}
