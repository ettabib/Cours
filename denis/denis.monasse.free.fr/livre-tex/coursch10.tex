\documentclass[]{article}
\usepackage[T1]{fontenc}
\usepackage{lmodern}
\usepackage{amssymb,amsmath}
\usepackage{ifxetex,ifluatex}
\usepackage{fixltx2e} % provides \textsubscript
% use upquote if available, for straight quotes in verbatim environments
\IfFileExists{upquote.sty}{\usepackage{upquote}}{}
\ifnum 0\ifxetex 1\fi\ifluatex 1\fi=0 % if pdftex
  \usepackage[utf8]{inputenc}
\else % if luatex or xelatex
  \ifxetex
    \usepackage{mathspec}
    \usepackage{xltxtra,xunicode}
  \else
    \usepackage{fontspec}
  \fi
  \defaultfontfeatures{Mapping=tex-text,Scale=MatchLowercase}
  \newcommand{\euro}{€}
\fi
% use microtype if available
\IfFileExists{microtype.sty}{\usepackage{microtype}}{}
\ifxetex
  \usepackage[setpagesize=false, % page size defined by xetex
              unicode=false, % unicode breaks when used with xetex
              xetex]{hyperref}
\else
  \usepackage[unicode=true]{hyperref}
\fi
\hypersetup{breaklinks=true,
            bookmarks=true,
            pdfauthor={},
            pdftitle={9 Integration},
            colorlinks=true,
            citecolor=blue,
            urlcolor=blue,
            linkcolor=magenta,
            pdfborder={0 0 0}}
\urlstyle{same}  % don't use monospace font for urls
\setlength{\parindent}{0pt}
\setlength{\parskip}{6pt plus 2pt minus 1pt}
\setlength{\emergencystretch}{3em}  % prevent overfull lines
\setcounter{secnumdepth}{0}
 
/* start css.sty */
.cmr-5{font-size:50%;}
.cmr-7{font-size:70%;}
.cmmi-5{font-size:50%;font-style: italic;}
.cmmi-7{font-size:70%;font-style: italic;}
.cmmi-10{font-style: italic;}
.cmsy-5{font-size:50%;}
.cmsy-7{font-size:70%;}
.cmex-7{font-size:70%;}
.cmex-7x-x-71{font-size:49%;}
.msbm-7{font-size:70%;}
.cmtt-10{font-family: monospace;}
.cmti-10{ font-style: italic;}
.cmbx-10{ font-weight: bold;}
.cmr-17x-x-120{font-size:204%;}
.cmsl-10{font-style: oblique;}
.cmti-7x-x-71{font-size:49%; font-style: italic;}
.cmbxti-10{ font-weight: bold; font-style: italic;}
p.noindent { text-indent: 0em }
td p.noindent { text-indent: 0em; margin-top:0em; }
p.nopar { text-indent: 0em; }
p.indent{ text-indent: 1.5em }
@media print {div.crosslinks {visibility:hidden;}}
a img { border-top: 0; border-left: 0; border-right: 0; }
center { margin-top:1em; margin-bottom:1em; }
td center { margin-top:0em; margin-bottom:0em; }
.Canvas { position:relative; }
li p.indent { text-indent: 0em }
.enumerate1 {list-style-type:decimal;}
.enumerate2 {list-style-type:lower-alpha;}
.enumerate3 {list-style-type:lower-roman;}
.enumerate4 {list-style-type:upper-alpha;}
div.newtheorem { margin-bottom: 2em; margin-top: 2em;}
.obeylines-h,.obeylines-v {white-space: nowrap; }
div.obeylines-v p { margin-top:0; margin-bottom:0; }
.overline{ text-decoration:overline; }
.overline img{ border-top: 1px solid black; }
td.displaylines {text-align:center; white-space:nowrap;}
.centerline {text-align:center;}
.rightline {text-align:right;}
div.verbatim {font-family: monospace; white-space: nowrap; text-align:left; clear:both; }
.fbox {padding-left:3.0pt; padding-right:3.0pt; text-indent:0pt; border:solid black 0.4pt; }
div.fbox {display:table}
div.center div.fbox {text-align:center; clear:both; padding-left:3.0pt; padding-right:3.0pt; text-indent:0pt; border:solid black 0.4pt; }
div.minipage{width:100%;}
div.center, div.center div.center {text-align: center; margin-left:1em; margin-right:1em;}
div.center div {text-align: left;}
div.flushright, div.flushright div.flushright {text-align: right;}
div.flushright div {text-align: left;}
div.flushleft {text-align: left;}
.underline{ text-decoration:underline; }
.underline img{ border-bottom: 1px solid black; margin-bottom:1pt; }
.framebox-c, .framebox-l, .framebox-r { padding-left:3.0pt; padding-right:3.0pt; text-indent:0pt; border:solid black 0.4pt; }
.framebox-c {text-align:center;}
.framebox-l {text-align:left;}
.framebox-r {text-align:right;}
span.thank-mark{ vertical-align: super }
span.footnote-mark sup.textsuperscript, span.footnote-mark a sup.textsuperscript{ font-size:80%; }
div.tabular, div.center div.tabular {text-align: center; margin-top:0.5em; margin-bottom:0.5em; }
table.tabular td p{margin-top:0em;}
table.tabular {margin-left: auto; margin-right: auto;}
div.td00{ margin-left:0pt; margin-right:0pt; }
div.td01{ margin-left:0pt; margin-right:5pt; }
div.td10{ margin-left:5pt; margin-right:0pt; }
div.td11{ margin-left:5pt; margin-right:5pt; }
table[rules] {border-left:solid black 0.4pt; border-right:solid black 0.4pt; }
td.td00{ padding-left:0pt; padding-right:0pt; }
td.td01{ padding-left:0pt; padding-right:5pt; }
td.td10{ padding-left:5pt; padding-right:0pt; }
td.td11{ padding-left:5pt; padding-right:5pt; }
table[rules] {border-left:solid black 0.4pt; border-right:solid black 0.4pt; }
.hline hr, .cline hr{ height : 1px; margin:0px; }
.tabbing-right {text-align:right;}
span.TEX {letter-spacing: -0.125em; }
span.TEX span.E{ position:relative;top:0.5ex;left:-0.0417em;}
a span.TEX span.E {text-decoration: none; }
span.LATEX span.A{ position:relative; top:-0.5ex; left:-0.4em; font-size:85%;}
span.LATEX span.TEX{ position:relative; left: -0.4em; }
div.float img, div.float .caption {text-align:center;}
div.figure img, div.figure .caption {text-align:center;}
.marginpar {width:20%; float:right; text-align:left; margin-left:auto; margin-top:0.5em; font-size:85%; text-decoration:underline;}
.marginpar p{margin-top:0.4em; margin-bottom:0.4em;}
.equation td{text-align:center; vertical-align:middle; }
td.eq-no{ width:5%; }
table.equation { width:100%; } 
div.math-display, div.par-math-display{text-align:center;}
math .texttt { font-family: monospace; }
math .textit { font-style: italic; }
math .textsl { font-style: oblique; }
math .textsf { font-family: sans-serif; }
math .textbf { font-weight: bold; }
.partToc a, .partToc, .likepartToc a, .likepartToc {line-height: 200%; font-weight:bold; font-size:110%;}
.chapterToc a, .chapterToc, .likechapterToc a, .likechapterToc, .appendixToc a, .appendixToc {line-height: 200%; font-weight:bold;}
.index-item, .index-subitem, .index-subsubitem {display:block}
.caption td.id{font-weight: bold; white-space: nowrap; }
table.caption {text-align:center;}
h1.partHead{text-align: center}
p.bibitem { text-indent: -2em; margin-left: 2em; margin-top:0.6em; margin-bottom:0.6em; }
p.bibitem-p { text-indent: 0em; margin-left: 2em; margin-top:0.6em; margin-bottom:0.6em; }
.paragraphHead, .likeparagraphHead { margin-top:2em; font-weight: bold;}
.subparagraphHead, .likesubparagraphHead { font-weight: bold;}
.quote {margin-bottom:0.25em; margin-top:0.25em; margin-left:1em; margin-right:1em; text-align:justify;}
.verse{white-space:nowrap; margin-left:2em}
div.maketitle {text-align:center;}
h2.titleHead{text-align:center;}
div.maketitle{ margin-bottom: 2em; }
div.author, div.date {text-align:center;}
div.thanks{text-align:left; margin-left:10%; font-size:85%; font-style:italic; }
div.author{white-space: nowrap;}
.quotation {margin-bottom:0.25em; margin-top:0.25em; margin-left:1em; }
h1.partHead{text-align: center}
.sectionToc, .likesectionToc {margin-left:2em;}
.subsectionToc, .likesubsectionToc {margin-left:4em;}
.subsubsectionToc, .likesubsubsectionToc {margin-left:6em;}
.frenchb-nbsp{font-size:75%;}
.frenchb-thinspace{font-size:75%;}
.figure img.graphics {margin-left:10%;}
/* end css.sty */

\title{9 Integration}
\author{}
\date{}

\begin{document}
\maketitle

\textbf{Warning: \href{http://www.math.union.edu/locate/jsMath}{jsMath}
requires JavaScript to process the mathematics on this page.\\ If your
browser supports JavaScript, be sure it is enabled.}

\begin{center}\rule{3in}{0.4pt}\end{center}

{[}\href{coursch11.html}{next}{]} {[}\href{coursch9.html}{prev}{]}
{[}\href{coursch9.html\#tailcoursch9.html}{prev-tail}{]}
{[}\hyperref[tailcoursch10.html]{tail}{]}
{[}\href{cours.html\#coursch10.html}{up}{]}

\subsection{Chapitre~9\\Intégration}

~9.1 \href{coursse50.html\#x62-2640009.1}{Subdivisions, approximation
des fonctions} \\ ~~9.1.1
\href{coursse50.html\#x62-2650009.1.1}{Subdivisions d'un segment} \\
~~9.1.2 \href{coursse50.html\#x62-2660009.1.2}{Propriétés liées aux
subdivisions} \\ ~~9.1.3
\href{coursse50.html\#x62-2670009.1.3}{Approximation des fonctions} \\
~9.2 \href{coursse51.html\#x63-2680009.2}{Intégrale des fonctions
réglées sur un segment} \\ ~~9.2.1
\href{coursse51.html\#x63-2690009.2.1}{Intégrale des applications en
escalier} \\ ~~9.2.2 \href{coursse51.html\#x63-2700009.2.2}{Intégrale
des fonctions réglées} \\ ~~9.2.3
\href{coursse51.html\#x63-2710009.2.3}{Convention de Chasles} \\ ~~9.2.4
\href{coursse51.html\#x63-2720009.2.4}{Sommes de Riemann} \\ ~~9.2.5
\href{coursse51.html\#x63-2730009.2.5}{Sommes de Darboux} \\ ~9.3
\href{coursse52.html\#x64-2740009.3}{Primitives et intégrales} \\
~~9.3.1 \href{coursse52.html\#x64-2750009.3.1}{Continuité et
dérivabilité par rapport à une borne} \\ ~~9.3.2
\href{coursse52.html\#x64-2760009.3.2}{Primitives} \\ ~~9.3.3
\href{coursse52.html\#x64-2770009.3.3}{Changement de variable,
intégration par parties} \\ ~~9.3.4
\href{coursse52.html\#x64-2780009.3.4}{Deuxième formule de la moyenne}
\\ ~9.4 \href{coursse53.html\#x65-2790009.4}{Recherches de primitives}
\\ ~~9.4.1 \href{coursse53.html\#x65-2800009.4.1}{Position du problème}
\\ ~~9.4.2 \href{coursse53.html\#x65-2810009.4.2}{Techniques usuelles}
\\ ~~9.4.3 \href{coursse53.html\#x65-2820009.4.3}{Primitives usuelles}
\\ ~~9.4.4 \href{coursse53.html\#x65-2830009.4.4}{Fractions
rationnelles} \\ ~~9.4.5
\href{coursse53.html\#x65-2840009.4.5}{Fractions rationnelles en sinus
et cosinus} \\ ~~9.4.6 \href{coursse53.html\#x65-2850009.4.6}{Fractions
rationnelles en sinus et cosinus hyperboliques} \\ ~~9.4.7
\href{coursse53.html\#x65-2860009.4.7}{Intégrales abéliennes} \\ ~9.5
\href{coursse54.html\#x66-2870009.5}{Intégration sur un intervalle
quelconque: fonctions à valeurs réelles positives} \\ ~~9.5.1
\href{coursse54.html\#x66-2880009.5.1}{Fonctions intégrables à valeurs
réelles positives} \\ ~~9.5.2
\href{coursse54.html\#x66-2890009.5.2}{Règles de comparaison} \\ ~~9.5.3
\href{coursse54.html\#x66-2900009.5.3}{Exemples fondamentaux} \\ ~9.6
\href{coursse55.html\#x67-2910009.6}{Intégration sur un intervalle
quelconque: fonctions à valeurs complexes} \\ ~~9.6.1
\href{coursse55.html\#x67-2920009.6.1}{Fonctions à valeurs complexes
intégrables} \\ ~~9.6.2
\href{coursse55.html\#x67-2930009.6.2}{Décomposition des fonctions à
valeurs complexes} \\ ~~9.6.3
\href{coursse55.html\#x67-2940009.6.3}{Convention et relation de
Chasles} \\ ~~9.6.4 \href{coursse55.html\#x67-2950009.6.4}{Règles de
comparaison} \\ ~~9.6.5 \href{coursse55.html\#x67-2960009.6.5}{Espaces
de fonctions continues} \\ ~~9.6.6
\href{coursse55.html\#x67-2970009.6.6}{Notion d'intégrale impropre} \\
~9.7 \href{coursse56.html\#x68-2980009.7}{Développements asymptotiques
et analyse numérique} \\ ~~9.7.1
\href{coursse56.html\#x68-2990009.7.1}{La formule d'Euler-Mac Laurin} \\
~~9.7.2 \href{coursse56.html\#x68-3000009.7.2}{Calcul approché
d'intégrales} \\ ~~9.7.3 \href{coursse56.html\#x68-3010009.7.3}{La
méthode de Laplace} \\ ~9.8
\href{coursse57.html\#x69-3020009.8}{Généralités sur les intégrales
impropres} \\ ~~9.8.1 \href{coursse57.html\#x69-3030009.8.1}{Notion
d'intégrale impropre} \\ ~~9.8.2
\href{coursse57.html\#x69-3040009.8.2}{Intégrales plusieurs fois
impropres} \\ ~~9.8.3 \href{coursse57.html\#x69-3050009.8.3}{Opérations
sur les intégrales impropres} \\ ~~9.8.4
\href{coursse57.html\#x69-3060009.8.4}{Intégrales et séries: intégration
par paquets} \\ ~9.9 \href{coursse58.html\#x70-3070009.9}{Intégrale des
fonctions réelles positives} \\ ~~9.9.1
\href{coursse58.html\#x70-3080009.9.1}{Critère de convergence des
fonctions réelles positives} \\ ~~9.9.2
\href{coursse58.html\#x70-3090009.9.2}{Règles de comparaison} \\ ~~9.9.3
\href{coursse58.html\#x70-3100009.9.3}{Exemples fondamentaux} \\ ~9.10
\href{coursse59.html\#x71-3110009.10}{Convergence absolue,
semi-convergence} \\ ~~9.10.1
\href{coursse59.html\#x71-3120009.10.1}{Critère de Cauchy pour les
intégrales} \\ ~~9.10.2
\href{coursse59.html\#x71-3130009.10.2}{Convergence absolue} \\ ~~9.10.3
\href{coursse59.html\#x71-3140009.10.3}{Règles de convergence} \\
~~9.10.4 \href{coursse59.html\#x71-3150009.10.4}{Semi-convergence}

{[}\href{coursch11.html}{next}{]} {[}\href{coursch9.html}{prev}{]}
{[}\href{coursch9.html\#tailcoursch9.html}{prev-tail}{]}
{[}\href{coursch10.html}{front}{]}
{[}\href{cours.html\#coursch10.html}{up}{]}

\end{document}
