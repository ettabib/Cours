\documentclass[]{article}
\usepackage[T1]{fontenc}
\usepackage{lmodern}
\usepackage{amssymb,amsmath}
\usepackage{ifxetex,ifluatex}
\usepackage{fixltx2e} % provides \textsubscript
% use upquote if available, for straight quotes in verbatim environments
\IfFileExists{upquote.sty}{\usepackage{upquote}}{}
\ifnum 0\ifxetex 1\fi\ifluatex 1\fi=0 % if pdftex
  \usepackage[utf8]{inputenc}
\else % if luatex or xelatex
  \ifxetex
    \usepackage{mathspec}
    \usepackage{xltxtra,xunicode}
  \else
    \usepackage{fontspec}
  \fi
  \defaultfontfeatures{Mapping=tex-text,Scale=MatchLowercase}
  \newcommand{\euro}{€}
\fi
% use microtype if available
\IfFileExists{microtype.sty}{\usepackage{microtype}}{}
\ifxetex
  \usepackage[setpagesize=false, % page size defined by xetex
              unicode=false, % unicode breaks when used with xetex
              xetex]{hyperref}
\else
  \usepackage[unicode=true]{hyperref}
\fi
\hypersetup{breaklinks=true,
            bookmarks=true,
            pdfauthor={},
            pdftitle={Series `a termes reels positifs},
            colorlinks=true,
            citecolor=blue,
            urlcolor=blue,
            linkcolor=magenta,
            pdfborder={0 0 0}}
\urlstyle{same}  % don't use monospace font for urls
\setlength{\parindent}{0pt}
\setlength{\parskip}{6pt plus 2pt minus 1pt}
\setlength{\emergencystretch}{3em}  % prevent overfull lines
\setcounter{secnumdepth}{0}
 
/* start css.sty */
.cmr-5{font-size:50%;}
.cmr-7{font-size:70%;}
.cmmi-5{font-size:50%;font-style: italic;}
.cmmi-7{font-size:70%;font-style: italic;}
.cmmi-10{font-style: italic;}
.cmsy-5{font-size:50%;}
.cmsy-7{font-size:70%;}
.cmex-7{font-size:70%;}
.cmex-7x-x-71{font-size:49%;}
.msbm-7{font-size:70%;}
.cmtt-10{font-family: monospace;}
.cmti-10{ font-style: italic;}
.cmbx-10{ font-weight: bold;}
.cmr-17x-x-120{font-size:204%;}
.cmsl-10{font-style: oblique;}
.cmti-7x-x-71{font-size:49%; font-style: italic;}
.cmbxti-10{ font-weight: bold; font-style: italic;}
p.noindent { text-indent: 0em }
td p.noindent { text-indent: 0em; margin-top:0em; }
p.nopar { text-indent: 0em; }
p.indent{ text-indent: 1.5em }
@media print {div.crosslinks {visibility:hidden;}}
a img { border-top: 0; border-left: 0; border-right: 0; }
center { margin-top:1em; margin-bottom:1em; }
td center { margin-top:0em; margin-bottom:0em; }
.Canvas { position:relative; }
li p.indent { text-indent: 0em }
.enumerate1 {list-style-type:decimal;}
.enumerate2 {list-style-type:lower-alpha;}
.enumerate3 {list-style-type:lower-roman;}
.enumerate4 {list-style-type:upper-alpha;}
div.newtheorem { margin-bottom: 2em; margin-top: 2em;}
.obeylines-h,.obeylines-v {white-space: nowrap; }
div.obeylines-v p { margin-top:0; margin-bottom:0; }
.overline{ text-decoration:overline; }
.overline img{ border-top: 1px solid black; }
td.displaylines {text-align:center; white-space:nowrap;}
.centerline {text-align:center;}
.rightline {text-align:right;}
div.verbatim {font-family: monospace; white-space: nowrap; text-align:left; clear:both; }
.fbox {padding-left:3.0pt; padding-right:3.0pt; text-indent:0pt; border:solid black 0.4pt; }
div.fbox {display:table}
div.center div.fbox {text-align:center; clear:both; padding-left:3.0pt; padding-right:3.0pt; text-indent:0pt; border:solid black 0.4pt; }
div.minipage{width:100%;}
div.center, div.center div.center {text-align: center; margin-left:1em; margin-right:1em;}
div.center div {text-align: left;}
div.flushright, div.flushright div.flushright {text-align: right;}
div.flushright div {text-align: left;}
div.flushleft {text-align: left;}
.underline{ text-decoration:underline; }
.underline img{ border-bottom: 1px solid black; margin-bottom:1pt; }
.framebox-c, .framebox-l, .framebox-r { padding-left:3.0pt; padding-right:3.0pt; text-indent:0pt; border:solid black 0.4pt; }
.framebox-c {text-align:center;}
.framebox-l {text-align:left;}
.framebox-r {text-align:right;}
span.thank-mark{ vertical-align: super }
span.footnote-mark sup.textsuperscript, span.footnote-mark a sup.textsuperscript{ font-size:80%; }
div.tabular, div.center div.tabular {text-align: center; margin-top:0.5em; margin-bottom:0.5em; }
table.tabular td p{margin-top:0em;}
table.tabular {margin-left: auto; margin-right: auto;}
div.td00{ margin-left:0pt; margin-right:0pt; }
div.td01{ margin-left:0pt; margin-right:5pt; }
div.td10{ margin-left:5pt; margin-right:0pt; }
div.td11{ margin-left:5pt; margin-right:5pt; }
table[rules] {border-left:solid black 0.4pt; border-right:solid black 0.4pt; }
td.td00{ padding-left:0pt; padding-right:0pt; }
td.td01{ padding-left:0pt; padding-right:5pt; }
td.td10{ padding-left:5pt; padding-right:0pt; }
td.td11{ padding-left:5pt; padding-right:5pt; }
table[rules] {border-left:solid black 0.4pt; border-right:solid black 0.4pt; }
.hline hr, .cline hr{ height : 1px; margin:0px; }
.tabbing-right {text-align:right;}
span.TEX {letter-spacing: -0.125em; }
span.TEX span.E{ position:relative;top:0.5ex;left:-0.0417em;}
a span.TEX span.E {text-decoration: none; }
span.LATEX span.A{ position:relative; top:-0.5ex; left:-0.4em; font-size:85%;}
span.LATEX span.TEX{ position:relative; left: -0.4em; }
div.float img, div.float .caption {text-align:center;}
div.figure img, div.figure .caption {text-align:center;}
.marginpar {width:20%; float:right; text-align:left; margin-left:auto; margin-top:0.5em; font-size:85%; text-decoration:underline;}
.marginpar p{margin-top:0.4em; margin-bottom:0.4em;}
.equation td{text-align:center; vertical-align:middle; }
td.eq-no{ width:5%; }
table.equation { width:100%; } 
div.math-display, div.par-math-display{text-align:center;}
math .texttt { font-family: monospace; }
math .textit { font-style: italic; }
math .textsl { font-style: oblique; }
math .textsf { font-family: sans-serif; }
math .textbf { font-weight: bold; }
.partToc a, .partToc, .likepartToc a, .likepartToc {line-height: 200%; font-weight:bold; font-size:110%;}
.chapterToc a, .chapterToc, .likechapterToc a, .likechapterToc, .appendixToc a, .appendixToc {line-height: 200%; font-weight:bold;}
.index-item, .index-subitem, .index-subsubitem {display:block}
.caption td.id{font-weight: bold; white-space: nowrap; }
table.caption {text-align:center;}
h1.partHead{text-align: center}
p.bibitem { text-indent: -2em; margin-left: 2em; margin-top:0.6em; margin-bottom:0.6em; }
p.bibitem-p { text-indent: 0em; margin-left: 2em; margin-top:0.6em; margin-bottom:0.6em; }
.paragraphHead, .likeparagraphHead { margin-top:2em; font-weight: bold;}
.subparagraphHead, .likesubparagraphHead { font-weight: bold;}
.quote {margin-bottom:0.25em; margin-top:0.25em; margin-left:1em; margin-right:1em; text-align:\\jmathmathustify;}
.verse{white-space:nowrap; margin-left:2em}
div.maketitle {text-align:center;}
h2.titleHead{text-align:center;}
div.maketitle{ margin-bottom: 2em; }
div.author, div.date {text-align:center;}
div.thanks{text-align:left; margin-left:10%; font-size:85%; font-style:italic; }
div.author{white-space: nowrap;}
.quotation {margin-bottom:0.25em; margin-top:0.25em; margin-left:1em; }
h1.partHead{text-align: center}
.sectionToc, .likesectionToc {margin-left:2em;}
.subsectionToc, .likesubsectionToc {margin-left:4em;}
.subsubsectionToc, .likesubsubsectionToc {margin-left:6em;}
.frenchb-nbsp{font-size:75%;}
.frenchb-thinspace{font-size:75%;}
.figure img.graphics {margin-left:10%;}
/* end css.sty */

\title{Series `a termes reels positifs}
\author{}
\date{}

\begin{document}
\maketitle

\textbf{Warning: 
requires JavaScript to process the mathematics on this page.\\ If your
browser supports JavaScript, be sure it is enabled.}

\begin{center}\rule{3in}{0.4pt}\end{center}

{[}
{[}
{[}{]}
{[}

\subsubsection{7.3 Séries à termes réels positifs}

\paragraph{7.3.1 Convergence des séries à termes réels positifs}

Théorème~7.3.1 Soit \\\sum
 x_n une série à termes réels positifs. Alors la suite des
sommes partielles est une suite croissante~; la série converge si et
seulement si~ses sommes partielles sont ma\\jmathmathorées~:
\existsM \in \mathbb{R}~, \\forall~~n \in \mathbb{N}~,
S_n \leq M.

Démonstration On a S_n - S_n-1 = x_n ≥ 0 donc
la suite (S_n) est croissante~; par suite, elle converge si et
seulement si~elle est ma\\jmathmathorée.

Remarque~7.3.1 Si une série à termes positifs diverge, on a donc
nécessairement limS_n~ = +\infty~ (puisque
la suite (S_n) est croissante).

Corollaire~7.3.2 Soit \\\sum
 x_n et \\\sum
 y_n deux séries à termes réels telles que 0 \leq x_n
\leq y_n. Alors

\begin{itemize}
\itemsep1pt\parskip0pt\parsep0pt
\item
  (i) si la série \\sum ~
  y_n converge, la série
  \\sum  x_n~
  converge également
\item
  (ii) si la série \\\sum
   x_n diverge, la série
  \\sum  y_n~
  diverge
\end{itemize}

Démonstration On a S_n(x) \leq S_n(y) donc tout ma\\jmathmathorant
de la suite (S_n(y)) est aussi un ma\\jmathmathorant de la suite
(S_n(x)), d'où (i). L'énoncé (ii) n'en est que la contraposée.

Remarque~7.3.2 Pour que l'énoncé précédent soit valable, il suffit
évidemment qu'il existe k \textgreater{} 0 et N \in \mathbb{N}~ tels que n ≥ N \rigtharrow~ 0 \leq
x_n \leq ky_n, c'est-à-dire que x_n ≥ 0,
y_n ≥ 0 et x_n = O(y_n).

\paragraph{7.3.2 Comparaison des séries à termes réels positifs}

Théorème~7.3.3 Soit \\\sum
 x_n et \\\sum
 y_n deux séries à termes réels positifs telles que
x_n = O(y_n) (resp. x_n = o(y_n)).
Alors

\begin{itemize}
\itemsep1pt\parskip0pt\parsep0pt
\item
  (i) si la série \\sum ~
  y_n converge, la série
  \\sum  x_n~
  converge également et R_n(x) = O(R_n(y)) (resp.
  R_n(x) = o(R_n(y)))
\item
  (ii) si la série \\\sum
   x_n diverge, la série
  \\sum  y_n~
  diverge et S_n(x) = O(S_n(y)) (resp. S_n(x)
  = o(S_n(y)))
\end{itemize}

Démonstration Les convergences et divergences résultent immédiatement de
la remarque qui suit le corollaire précédent et du fait que x_n
= o(y_n) \rigtharrow~ x_n = O(y_n). Montrons par exemple
les énoncés sur les relations de comparaison dans le cas x_n =
o(y_n) (des modifications évidentes de \epsilon en k ou 2k permettent
de traiter le cas x_n = O(y_n)).

(i) Soit \epsilon \textgreater{} 0~; il existe N \in \mathbb{N}~ tel que n ≥ N \rigtharrow~ 0 \leq
x_n \leq \epsilony_n. Alors pour n ≥ N, on a 0
\leq\\sum ~
_p=n+1^+\infty~x_p \leq
\epsilon\\sum ~
_p=n+1^+\infty~y_p, soit 0 \leq R_n(x) \leq
\epsilonR_n(y). On a donc R_n(x) = o(R_n(y)).

(ii) Soit \epsilon \textgreater{} 0~; il existe N \in \mathbb{N}~ tel que n ≥ N \rigtharrow~ 0 \leq
x_n \leq \epsilon \over 2 y_n. Alors pour n
\textgreater{} N, on a 0
\leq\\sum ~
_p=N+1^nx_p \leq \epsilon \over 2
 \\sum ~
_p=N+1^ny_p, soit S_n(x) -
S_N(x) \leq \epsilon \over 2 (S_n(y) -
S_N(y)) ou encore 0 \leq S_n(x) \leq \epsilon
\over 2 S_n(y) + (S_N(x) - \epsilon
\over 2 S_N(y)). Mais comme la série
\\sum  y_n~ est
à termes positifs divergente, ses sommes partielles tendent vers + \infty~ et
donc il existe N' \in \mathbb{N}~ tel que n ≥ N' \rigtharrow~ \epsilon \over 2
S_n(y) ≥ S_N(x) - \epsilon \over 2
S_N(y). Alors pour n \textgreater{}\
max(N,N'), on a 0 \leq S_n(x) \leq \epsilon \over 2
S_n(y) + \epsilon \over 2 S_n(y) =
\epsilonS_n(y) et donc S_n(x) = o(S_n(y)).

Corollaire~7.3.4 Soit \\\sum
 x_n et \\\sum
 y_n deux séries à termes réels strictement positifs telles
que

\existsN \in \mathbb{N}~, n ≥ N \rigtharrow~ x_n+1~
\over x_n \leq y_n+1
\over y_n

Alors x_n = O(y_n) et en particulier

\begin{itemize}
\itemsep1pt\parskip0pt\parsep0pt
\item
  (i) si la série \\sum ~
  y_n converge, la série
  \\sum  x_n~
  converge
\item
  (ii) si la série \\\sum
   x_n diverge, la série
  \\sum  y_n~
  diverge
\end{itemize}

Démonstration On vérifie immédiatement par récurrence que pour n ≥ N on
a x_n \leq x_N \over y_N
y_n et donc x_n = O(y_n).

Théorème~7.3.5 Soit \\\sum
 x_n et \\\sum
 y_n deux séries à termes réels telles que y_n ≥ 0
et x_n ∼ y_n. Alors les deux séries sont de même
nature et

\begin{itemize}
\itemsep1pt\parskip0pt\parsep0pt
\item
  (i) si la série \\sum ~
  y_n converge, la série
  \\sum  x_n~
  converge également et R_n(x) ∼ R_n(y)
\item
  (ii) si la série \\\sum
   y_n diverge, la série
  \\sum  x_n~
  diverge et S_n(x) ∼ S_n(y)
\end{itemize}

Démonstration Soit \epsilon \textless{} 1. Il existe N \in \mathbb{N}~ tel que n ≥ N \rigtharrow~ (1 -
\epsilon)y_n \leq x_n \leq (1 + \epsilon)y_n et donc x_n
≥ 0 pour n ≥ N. On a à la fois x_n = O(y_n) et
y_n = O(x_n) ce qui d'après le théorème précédent
montre que les deux séries convergent ou divergent simultanément.
Supposons alors les séries convergentes. On a x_n -
y_n = o(y_n), on en déduit donc la
convergence de \\sum ~
x_n - y_n et que
R_n(x - y) = o(R_n(y)). Mais
R_n(x) - R_n(y)\leq
R_n(x - y) donc R_n(x) -
R_n(y) = o(R_n(y)) et donc R_n(x) ∼
R_n(y). Supposons maintenant les séries divergentes. Alors,
soit la série \\sum ~
x_n - y_n converge et comme
limS_n~(y) = +\infty~ on a
S_n(x - y) = o(S_n(y)), soit elle
diverge et le théorème précédent assure que S_n(x -
y) = o(S_n(y)). Mais alors S_n(x)
- S_n(y)\leq S_n(x - y) =
o(S_n(y)), soit S_n(x) ∼ S_n(y).

\paragraph{7.3.3 Séries de Riemann et de Bertrand}

Théorème~7.3.6 (séries de Riemann). Soit \alpha~ \in \mathbb{R}~. La série
\\sum ~  1
\over n^\alpha~ converge si et seulement si~\alpha~
\textgreater{} 1.

Si \alpha~ \textgreater{} 1, on a R_n ∼ 1 \over
\alpha~-1  1 \over n^\alpha~-1 ~; si \alpha~ \textless{}
1, on a S_n ∼ n^1-\alpha~ \over 1-\alpha~ ~;
si \alpha~ = 1, S_n ∼ log~ n.

Démonstration Soit \alpha~\neq~1. Posons x_n
= 1 \over n^\alpha~ et y_n = 1
\over n^\alpha~-1 - 1 \over
(n+1)^\alpha~-1 . On a

 y_n \over x_n = - (1 + 1
\over n )^1-\alpha~ - 1 \over  1
\over n 

qui admet pour limite l'opposé de la dérivée en 0 de
x\mapsto~(1 + x)^1-\alpha~ soit \alpha~ - 1. On a
donc x_n ∼ 1 \over \alpha~-1 y_n
\textgreater{} 0. Les deux séries sont donc de même nature. Or
S_n(y) = 1 - 1 \over (n+1)^\alpha~-1
admet une limite finie si et seulement si~\alpha~ \textgreater{} 1. Si \alpha~
\textgreater{} 1, on a R_n(x) ∼ 1 \over \alpha~-1
R_n(y) = 1 \over \alpha~-1  1
\over n^\alpha~-1 . Si \alpha~ \textless{} 1, on a
S_n(x) ∼ 1 \over \alpha~-1 S_n(y) = 1
\over 1-\alpha~ ((n + 1)^1-\alpha~ - 1) ∼
n^1-\alpha~ \over 1-\alpha~ . Enfin, si \alpha~ = 1, on
aboutit à une étude similaire avec y_n
= log (n + 1) -\ log~
n = log (1 + 1 \over n~ )
∼ 1 \over n .

Corollaire~7.3.7 (séries de Bertrand). Soit \alpha~,\beta~ \in \mathbb{R}~. La série
\\sum  _n≥2~ 1
\over n^\alpha~(log~
n)^\beta~ converge si et seulement si~\alpha~ \textgreater{} 1 ou \alpha~ =
1,\beta~ \textgreater{} 1.

Démonstration Soit x_n = 1 \over
n^\alpha~(log n)^\beta~~ . Si \alpha~
\textgreater{} 1, soit \gamma tel que \alpha~ \textgreater{} \gamma \textgreater{} 1 et
y_n = 1 \over n^\gamma . La série
\\sum  y_n~
converge et  x_n \over y_n = 1
\over n^\alpha~-\gamma(log~
n)^\beta~ tend vers 0 car \alpha~ - \gamma \textgreater{} 0. On a donc
x_n = o(y_n) et la série
\\sum  x_n~
converge. Si \alpha~ \textless{} 1, soit \gamma tel que \alpha~ \textless{} \gamma \textless{}
1 et y_n = 1 \over n^\gamma . La série
\\sum  y_n~
diverge et  y_n \over x_n =
(log n)^\beta~~ \over
n^\gamma-\alpha~ tend vers 0 car \gamma - \alpha~ \textgreater{} 0. On a donc
y_n = o(x_n) et la série
\\sum  x_n~
converge. Le cas \alpha~ = 1 résulte facilement du paragraphe suivant.

\paragraph{7.3.4 Comparaison à des intégrales}

Théorème~7.3.8 Soit f : {[}0,+\infty~{[}\rightarrow~ \mathbb{R}~ continue par morceaux,
décroissante, positive. Posons w_n =\\int
 _n-1^nf(t) dt - f(n). Alors la série
\\sum  w_n~ est
convergente.

Démonstration On a w_n =\int ~
_n-1^n(f(t) - f(n)) dt. Comme f est décroissante,
\forall~~t \in {[}n - 1,n{]}, f(t) ≥ f(n) et donc
w_n ≥ 0. Mais d'autre part

0 \leq w_n \leq\int  _n-1^n~f(n
- 1) dt - f(n) = f(n - 1) - f(n)

On a \\sum ~
_p=1^n(f(p - 1) - f(p)) = f(0) - f(n) qui admet une limite
quand p tend vers + \infty~ (car f admet une limite en + \infty~~: elle est
décroissante et positive). Ceci montre que la série
\\sum ~ (f(p - 1) - f(p))
converge. Il en est donc de même de la série
\\sum  w_n~.

Corollaire~7.3.9 Soit f : {[}0,+\infty~{[}\rightarrow~ \mathbb{R}~ continue décroissante positive.
Alors la série \\sum ~
f(n) converge si et seulement si f est intégrable sur {[}0,+\infty~{[}.

Démonstration En effet, on déduit du théorème précédent que les deux
séries \\sum ~ f(n) et
\\sum ~
\int  _n-1^n~f(t) dt convergent ou
divergent simultanément, car leur différence est une série convergente.
Mais on a \\sum ~
_p=1^n\int ~
_p-1^pf(t) dt =\int ~
_0^nf(t) dt =\int ~
_{[}0,n{]}f. Si f est intégrable, comme la suite
({[}0,n{]})_n\in\mathbb{N}~ est une suite croissante de segments dont la
réunion est {[}0,+\infty~{[}, la suite (\int ~
_{[}0,n{]}f) est convergente de limite
\int  _{[}0,+\infty~{[}~f, donc la série
\\sum ~
\int  _n-1^n~f(t) dt converge et
il en est de même de \\\sum
 f(n). Si \\sum ~
_f(n) converge, il en est de même de
\\sum ~
\int  _n-1^n~f(t) dt, et si
{[}a,b{]} est un segment contenu dans {[}0,+\infty~{[} les ma\\jmathmathorations

\int  _{[}a,b{]}~f
\leq\int  _0^{[}b{]}+1~f =
\sum _p=0^{[}b{]}+1~
\\int  ~
_p-1^pf(t) dt \leq\\sum
_p=0^+\infty~\\\int
  _p-1^pf(t) dt

et le fait que f soit positive, montrent que f est intégrable sur
{[}0,+\infty~{[}.

Remarque~7.3.3 Bien entendu, il suffit que la condition de décroissance
soit vérifiée sur un certain {[}t_0,+\infty~{[}.

Dans le cas d'une série divergente, l'encadrement

\int  _0^n+1~f(t) dt
\leq\sum _p=0^n~f(p) \leq f(0) +
\\int  ~
_0^nf(t) dt

permet souvent d'obtenir un équivalent de la somme partielle de la
série. Dans le cas d'une série convergente, on a de même

\int  _n+1^+\infty~~f(t) dt
\leq\sum _p=n+1^+\infty~~f(p)
\leq\\int  ~
_n^+\infty~f(t) dt

ce qui permet souvent d'obtenir une ma\\jmathmathoration ou un équivalent du reste
de la série.

Exemple~7.3.1 Dans le cas limite des séries de Bertrand,
\\sum ~  1
\over n(log n)^\beta~~ ,
la fonction f(t) = 1 \over
t(log t)^\beta~~ est continue
décroissante (pour t assez grand) de limite 0. Donc la série est de même
nature que l'intégrale \int ~
_3^+\infty~ dt \over
t(log t)^\beta~~ . Mais on a
\int  _3^x~ dt
\over t(log t)^\beta~~
=\int  _\log~
3^log x~ du \over
u^\beta~ (poser u = log~ t) qui admet
une limite finie quand x tend vers + \infty~ si et seulement si \beta~
\textgreater{} 1. Ceci achève la démonstration du critère de convergence
des séries de Bertrand.

{[}
{[}
{[}
{[}

\end{document}
