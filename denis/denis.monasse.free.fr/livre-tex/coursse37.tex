\documentclass[]{article}
\usepackage[T1]{fontenc}
\usepackage{lmodern}
\usepackage{amssymb,amsmath}
\usepackage{ifxetex,ifluatex}
\usepackage{fixltx2e} % provides \textsubscript
% use upquote if available, for straight quotes in verbatim environments
\IfFileExists{upquote.sty}{\usepackage{upquote}}{}
\ifnum 0\ifxetex 1\fi\ifluatex 1\fi=0 % if pdftex
  \usepackage[utf8]{inputenc}
\else % if luatex or xelatex
  \ifxetex
    \usepackage{mathspec}
    \usepackage{xltxtra,xunicode}
  \else
    \usepackage{fontspec}
  \fi
  \defaultfontfeatures{Mapping=tex-text,Scale=MatchLowercase}
  \newcommand{\euro}{€}
\fi
% use microtype if available
\IfFileExists{microtype.sty}{\usepackage{microtype}}{}
\ifxetex
  \usepackage[setpagesize=false, % page size defined by xetex
              unicode=false, % unicode breaks when used with xetex
              xetex]{hyperref}
\else
  \usepackage[unicode=true]{hyperref}
\fi
\hypersetup{breaklinks=true,
            bookmarks=true,
            pdfauthor={},
            pdftitle={Series `a termes reels positifs},
            colorlinks=true,
            citecolor=blue,
            urlcolor=blue,
            linkcolor=magenta,
            pdfborder={0 0 0}}
\urlstyle{same}  % don't use monospace font for urls
\setlength{\parindent}{0pt}
\setlength{\parskip}{6pt plus 2pt minus 1pt}
\setlength{\emergencystretch}{3em}  % prevent overfull lines
\setcounter{secnumdepth}{0}
 
/* start css.sty */
.cmr-5{font-size:50%;}
.cmr-7{font-size:70%;}
.cmmi-5{font-size:50%;font-style: italic;}
.cmmi-7{font-size:70%;font-style: italic;}
.cmmi-10{font-style: italic;}
.cmsy-5{font-size:50%;}
.cmsy-7{font-size:70%;}
.cmex-7{font-size:70%;}
.cmex-7x-x-71{font-size:49%;}
.msbm-7{font-size:70%;}
.cmtt-10{font-family: monospace;}
.cmti-10{ font-style: italic;}
.cmbx-10{ font-weight: bold;}
.cmr-17x-x-120{font-size:204%;}
.cmsl-10{font-style: oblique;}
.cmti-7x-x-71{font-size:49%; font-style: italic;}
.cmbxti-10{ font-weight: bold; font-style: italic;}
p.noindent { text-indent: 0em }
td p.noindent { text-indent: 0em; margin-top:0em; }
p.nopar { text-indent: 0em; }
p.indent{ text-indent: 1.5em }
@media print {div.crosslinks {visibility:hidden;}}
a img { border-top: 0; border-left: 0; border-right: 0; }
center { margin-top:1em; margin-bottom:1em; }
td center { margin-top:0em; margin-bottom:0em; }
.Canvas { position:relative; }
li p.indent { text-indent: 0em }
.enumerate1 {list-style-type:decimal;}
.enumerate2 {list-style-type:lower-alpha;}
.enumerate3 {list-style-type:lower-roman;}
.enumerate4 {list-style-type:upper-alpha;}
div.newtheorem { margin-bottom: 2em; margin-top: 2em;}
.obeylines-h,.obeylines-v {white-space: nowrap; }
div.obeylines-v p { margin-top:0; margin-bottom:0; }
.overline{ text-decoration:overline; }
.overline img{ border-top: 1px solid black; }
td.displaylines {text-align:center; white-space:nowrap;}
.centerline {text-align:center;}
.rightline {text-align:right;}
div.verbatim {font-family: monospace; white-space: nowrap; text-align:left; clear:both; }
.fbox {padding-left:3.0pt; padding-right:3.0pt; text-indent:0pt; border:solid black 0.4pt; }
div.fbox {display:table}
div.center div.fbox {text-align:center; clear:both; padding-left:3.0pt; padding-right:3.0pt; text-indent:0pt; border:solid black 0.4pt; }
div.minipage{width:100%;}
div.center, div.center div.center {text-align: center; margin-left:1em; margin-right:1em;}
div.center div {text-align: left;}
div.flushright, div.flushright div.flushright {text-align: right;}
div.flushright div {text-align: left;}
div.flushleft {text-align: left;}
.underline{ text-decoration:underline; }
.underline img{ border-bottom: 1px solid black; margin-bottom:1pt; }
.framebox-c, .framebox-l, .framebox-r { padding-left:3.0pt; padding-right:3.0pt; text-indent:0pt; border:solid black 0.4pt; }
.framebox-c {text-align:center;}
.framebox-l {text-align:left;}
.framebox-r {text-align:right;}
span.thank-mark{ vertical-align: super }
span.footnote-mark sup.textsuperscript, span.footnote-mark a sup.textsuperscript{ font-size:80%; }
div.tabular, div.center div.tabular {text-align: center; margin-top:0.5em; margin-bottom:0.5em; }
table.tabular td p{margin-top:0em;}
table.tabular {margin-left: auto; margin-right: auto;}
div.td00{ margin-left:0pt; margin-right:0pt; }
div.td01{ margin-left:0pt; margin-right:5pt; }
div.td10{ margin-left:5pt; margin-right:0pt; }
div.td11{ margin-left:5pt; margin-right:5pt; }
table[rules] {border-left:solid black 0.4pt; border-right:solid black 0.4pt; }
td.td00{ padding-left:0pt; padding-right:0pt; }
td.td01{ padding-left:0pt; padding-right:5pt; }
td.td10{ padding-left:5pt; padding-right:0pt; }
td.td11{ padding-left:5pt; padding-right:5pt; }
table[rules] {border-left:solid black 0.4pt; border-right:solid black 0.4pt; }
.hline hr, .cline hr{ height : 1px; margin:0px; }
.tabbing-right {text-align:right;}
span.TEX {letter-spacing: -0.125em; }
span.TEX span.E{ position:relative;top:0.5ex;left:-0.0417em;}
a span.TEX span.E {text-decoration: none; }
span.LATEX span.A{ position:relative; top:-0.5ex; left:-0.4em; font-size:85%;}
span.LATEX span.TEX{ position:relative; left: -0.4em; }
div.float img, div.float .caption {text-align:center;}
div.figure img, div.figure .caption {text-align:center;}
.marginpar {width:20%; float:right; text-align:left; margin-left:auto; margin-top:0.5em; font-size:85%; text-decoration:underline;}
.marginpar p{margin-top:0.4em; margin-bottom:0.4em;}
.equation td{text-align:center; vertical-align:middle; }
td.eq-no{ width:5%; }
table.equation { width:100%; } 
div.math-display, div.par-math-display{text-align:center;}
math .texttt { font-family: monospace; }
math .textit { font-style: italic; }
math .textsl { font-style: oblique; }
math .textsf { font-family: sans-serif; }
math .textbf { font-weight: bold; }
.partToc a, .partToc, .likepartToc a, .likepartToc {line-height: 200%; font-weight:bold; font-size:110%;}
.chapterToc a, .chapterToc, .likechapterToc a, .likechapterToc, .appendixToc a, .appendixToc {line-height: 200%; font-weight:bold;}
.index-item, .index-subitem, .index-subsubitem {display:block}
.caption td.id{font-weight: bold; white-space: nowrap; }
table.caption {text-align:center;}
h1.partHead{text-align: center}
p.bibitem { text-indent: -2em; margin-left: 2em; margin-top:0.6em; margin-bottom:0.6em; }
p.bibitem-p { text-indent: 0em; margin-left: 2em; margin-top:0.6em; margin-bottom:0.6em; }
.paragraphHead, .likeparagraphHead { margin-top:2em; font-weight: bold;}
.subparagraphHead, .likesubparagraphHead { font-weight: bold;}
.quote {margin-bottom:0.25em; margin-top:0.25em; margin-left:1em; margin-right:1em; text-align:justify;}
.verse{white-space:nowrap; margin-left:2em}
div.maketitle {text-align:center;}
h2.titleHead{text-align:center;}
div.maketitle{ margin-bottom: 2em; }
div.author, div.date {text-align:center;}
div.thanks{text-align:left; margin-left:10%; font-size:85%; font-style:italic; }
div.author{white-space: nowrap;}
.quotation {margin-bottom:0.25em; margin-top:0.25em; margin-left:1em; }
h1.partHead{text-align: center}
.sectionToc, .likesectionToc {margin-left:2em;}
.subsectionToc, .likesubsectionToc {margin-left:4em;}
.subsubsectionToc, .likesubsubsectionToc {margin-left:6em;}
.frenchb-nbsp{font-size:75%;}
.frenchb-thinspace{font-size:75%;}
.figure img.graphics {margin-left:10%;}
/* end css.sty */

\title{Series `a termes reels positifs}
\author{}
\date{}

\begin{document}
\maketitle

\textbf{Warning: \href{http://www.math.union.edu/locate/jsMath}{jsMath}
requires JavaScript to process the mathematics on this page.\\ If your
browser supports JavaScript, be sure it is enabled.}

\begin{center}\rule{3in}{0.4pt}\end{center}

{[}\href{coursse38.html}{next}{]} {[}\href{coursse36.html}{prev}{]}
{[}\href{coursse36.html\#tailcoursse36.html}{prev-tail}{]}
{[}\hyperref[tailcoursse37.html]{tail}{]}
{[}\href{coursch8.html\#coursse37.html}{up}{]}

\subsubsection{7.3 Séries à termes réels positifs}

\paragraph{7.3.1 Convergence des séries à termes réels positifs}

Théorème~7.3.1 Soit \textbackslash{}mathop\{\textbackslash{}mathop\{∑
\}\} \{x\}\_\{n\} une série à termes réels positifs. Alors la suite des
sommes partielles est une suite croissante~; la série converge si et
seulement si~ses sommes partielles sont majorées~:
\textbackslash{}mathop\{∃\}M ∈ ℝ, \textbackslash{}mathop\{∀\}n ∈ ℕ,
\{S\}\_\{n\} ≤ M.

Démonstration On a \{S\}\_\{n\} − \{S\}\_\{n−1\} = \{x\}\_\{n\} ≥ 0 donc
la suite (\{S\}\_\{n\}) est croissante~; par suite, elle converge si et
seulement si~elle est majorée.

Remarque~7.3.1 Si une série à termes positifs diverge, on a donc
nécessairement \textbackslash{}mathop\{lim\}\{S\}\_\{n\} = +∞ (puisque
la suite (\{S\}\_\{n\}) est croissante).

Corollaire~7.3.2 Soit \textbackslash{}mathop\{\textbackslash{}mathop\{∑
\}\} \{x\}\_\{n\} et \textbackslash{}mathop\{\textbackslash{}mathop\{∑
\}\} \{y\}\_\{n\} deux séries à termes réels telles que 0 ≤ \{x\}\_\{n\}
≤ \{y\}\_\{n\}. Alors

\begin{itemize}
\itemsep1pt\parskip0pt\parsep0pt
\item
  (i) si la série \textbackslash{}mathop\{\textbackslash{}mathop\{∑ \}\}
  \{y\}\_\{n\} converge, la série
  \textbackslash{}mathop\{\textbackslash{}mathop\{∑ \}\} \{x\}\_\{n\}
  converge également
\item
  (ii) si la série \textbackslash{}mathop\{\textbackslash{}mathop\{∑
  \}\} \{x\}\_\{n\} diverge, la série
  \textbackslash{}mathop\{\textbackslash{}mathop\{∑ \}\} \{y\}\_\{n\}
  diverge
\end{itemize}

Démonstration On a \{S\}\_\{n\}(x) ≤ \{S\}\_\{n\}(y) donc tout majorant
de la suite (\{S\}\_\{n\}(y)) est aussi un majorant de la suite
(\{S\}\_\{n\}(x)), d'où (i). L'énoncé (ii) n'en est que la contraposée.

Remarque~7.3.2 Pour que l'énoncé précédent soit valable, il suffit
évidemment qu'il existe k \textgreater{} 0 et N ∈ ℕ tels que n ≥ N ⇒ 0 ≤
\{x\}\_\{n\} ≤ k\{y\}\_\{n\}, c'est-à-dire que \{x\}\_\{n\} ≥ 0,
\{y\}\_\{n\} ≥ 0 et \{x\}\_\{n\} = O(\{y\}\_\{n\}).

\paragraph{7.3.2 Comparaison des séries à termes réels positifs}

Théorème~7.3.3 Soit \textbackslash{}mathop\{\textbackslash{}mathop\{∑
\}\} \{x\}\_\{n\} et \textbackslash{}mathop\{\textbackslash{}mathop\{∑
\}\} \{y\}\_\{n\} deux séries à termes réels positifs telles que
\{x\}\_\{n\} = O(\{y\}\_\{n\}) (resp. \{x\}\_\{n\} = o(\{y\}\_\{n\})).
Alors

\begin{itemize}
\itemsep1pt\parskip0pt\parsep0pt
\item
  (i) si la série \textbackslash{}mathop\{\textbackslash{}mathop\{∑ \}\}
  \{y\}\_\{n\} converge, la série
  \textbackslash{}mathop\{\textbackslash{}mathop\{∑ \}\} \{x\}\_\{n\}
  converge également et \{R\}\_\{n\}(x) = O(\{R\}\_\{n\}(y)) (resp.
  \{R\}\_\{n\}(x) = o(\{R\}\_\{n\}(y)))
\item
  (ii) si la série \textbackslash{}mathop\{\textbackslash{}mathop\{∑
  \}\} \{x\}\_\{n\} diverge, la série
  \textbackslash{}mathop\{\textbackslash{}mathop\{∑ \}\} \{y\}\_\{n\}
  diverge et \{S\}\_\{n\}(x) = O(\{S\}\_\{n\}(y)) (resp. \{S\}\_\{n\}(x)
  = o(\{S\}\_\{n\}(y)))
\end{itemize}

Démonstration Les convergences et divergences résultent immédiatement de
la remarque qui suit le corollaire précédent et du fait que \{x\}\_\{n\}
= o(\{y\}\_\{n\}) ⇒ \{x\}\_\{n\} = O(\{y\}\_\{n\}). Montrons par exemple
les énoncés sur les relations de comparaison dans le cas \{x\}\_\{n\} =
o(\{y\}\_\{n\}) (des modifications évidentes de ε en k ou 2k permettent
de traiter le cas \{x\}\_\{n\} = O(\{y\}\_\{n\})).

(i) Soit ε \textgreater{} 0~; il existe N ∈ ℕ tel que n ≥ N ⇒ 0 ≤
\{x\}\_\{n\} ≤ ε\{y\}\_\{n\}. Alors pour n ≥ N, on a 0
≤\{\textbackslash{}mathop\{\textbackslash{}mathop\{∑ \}\}
\}\_\{p=n+1\}\^{}\{+∞\}\{x\}\_\{p\} ≤
ε\{\textbackslash{}mathop\{\textbackslash{}mathop\{∑ \}\}
\}\_\{p=n+1\}\^{}\{+∞\}\{y\}\_\{p\}, soit 0 ≤ \{R\}\_\{n\}(x) ≤
ε\{R\}\_\{n\}(y). On a donc \{R\}\_\{n\}(x) = o(\{R\}\_\{n\}(y)).

(ii) Soit ε \textgreater{} 0~; il existe N ∈ ℕ tel que n ≥ N ⇒ 0 ≤
\{x\}\_\{n\} ≤\{ ε \textbackslash{}over 2\} \{y\}\_\{n\}. Alors pour n
\textgreater{} N, on a 0
≤\{\textbackslash{}mathop\{\textbackslash{}mathop\{∑ \}\}
\}\_\{p=N+1\}\^{}\{n\}\{x\}\_\{p\} ≤\{ ε \textbackslash{}over 2\}
\{\textbackslash{}mathop\{ \textbackslash{}mathop\{∑ \}\}
\}\_\{p=N+1\}\^{}\{n\}\{y\}\_\{p\}, soit \{S\}\_\{n\}(x) −
\{S\}\_\{N\}(x) ≤\{ ε \textbackslash{}over 2\} (\{S\}\_\{n\}(y) −
\{S\}\_\{N\}(y)) ou encore 0 ≤ \{S\}\_\{n\}(x) ≤\{ ε
\textbackslash{}over 2\} \{S\}\_\{n\}(y) + (\{S\}\_\{N\}(x) −\{ ε
\textbackslash{}over 2\} \{S\}\_\{N\}(y)). Mais comme la série
\textbackslash{}mathop\{\textbackslash{}mathop\{∑ \}\} \{y\}\_\{n\} est
à termes positifs divergente, ses sommes partielles tendent vers + ∞ et
donc il existe N' ∈ ℕ tel que n ≥ N' ⇒\{ ε \textbackslash{}over 2\}
\{S\}\_\{n\}(y) ≥ \{S\}\_\{N\}(x) −\{ ε \textbackslash{}over 2\}
\{S\}\_\{N\}(y). Alors pour n \textgreater{}\textbackslash{}mathop\{
max\}(N,N'), on a 0 ≤ \{S\}\_\{n\}(x) ≤\{ ε \textbackslash{}over 2\}
\{S\}\_\{n\}(y) +\{ ε \textbackslash{}over 2\} \{S\}\_\{n\}(y) =
ε\{S\}\_\{n\}(y) et donc \{S\}\_\{n\}(x) = o(\{S\}\_\{n\}(y)).

Corollaire~7.3.4 Soit \textbackslash{}mathop\{\textbackslash{}mathop\{∑
\}\} \{x\}\_\{n\} et \textbackslash{}mathop\{\textbackslash{}mathop\{∑
\}\} \{y\}\_\{n\} deux séries à termes réels strictement positifs telles
que

\textbackslash{}mathop\{∃\}N ∈ ℕ, n ≥ N ⇒\{ \{x\}\_\{n+1\}
\textbackslash{}over \{x\}\_\{n\}\} ≤\{ \{y\}\_\{n+1\}
\textbackslash{}over \{y\}\_\{n\}\}

Alors \{x\}\_\{n\} = O(\{y\}\_\{n\}) et en particulier

\begin{itemize}
\itemsep1pt\parskip0pt\parsep0pt
\item
  (i) si la série \textbackslash{}mathop\{\textbackslash{}mathop\{∑ \}\}
  \{y\}\_\{n\} converge, la série
  \textbackslash{}mathop\{\textbackslash{}mathop\{∑ \}\} \{x\}\_\{n\}
  converge
\item
  (ii) si la série \textbackslash{}mathop\{\textbackslash{}mathop\{∑
  \}\} \{x\}\_\{n\} diverge, la série
  \textbackslash{}mathop\{\textbackslash{}mathop\{∑ \}\} \{y\}\_\{n\}
  diverge
\end{itemize}

Démonstration On vérifie immédiatement par récurrence que pour n ≥ N on
a \{x\}\_\{n\} ≤\{ \{x\}\_\{N\} \textbackslash{}over \{y\}\_\{N\}\}
\{y\}\_\{n\} et donc \{x\}\_\{n\} = O(\{y\}\_\{n\}).

Théorème~7.3.5 Soit \textbackslash{}mathop\{\textbackslash{}mathop\{∑
\}\} \{x\}\_\{n\} et \textbackslash{}mathop\{\textbackslash{}mathop\{∑
\}\} \{y\}\_\{n\} deux séries à termes réels telles que \{y\}\_\{n\} ≥ 0
et \{x\}\_\{n\} ∼ \{y\}\_\{n\}. Alors les deux séries sont de même
nature et

\begin{itemize}
\itemsep1pt\parskip0pt\parsep0pt
\item
  (i) si la série \textbackslash{}mathop\{\textbackslash{}mathop\{∑ \}\}
  \{y\}\_\{n\} converge, la série
  \textbackslash{}mathop\{\textbackslash{}mathop\{∑ \}\} \{x\}\_\{n\}
  converge également et \{R\}\_\{n\}(x) ∼ \{R\}\_\{n\}(y)
\item
  (ii) si la série \textbackslash{}mathop\{\textbackslash{}mathop\{∑
  \}\} \{y\}\_\{n\} diverge, la série
  \textbackslash{}mathop\{\textbackslash{}mathop\{∑ \}\} \{x\}\_\{n\}
  diverge et \{S\}\_\{n\}(x) ∼ \{S\}\_\{n\}(y)
\end{itemize}

Démonstration Soit ε \textless{} 1. Il existe N ∈ ℕ tel que n ≥ N ⇒ (1 −
ε)\{y\}\_\{n\} ≤ \{x\}\_\{n\} ≤ (1 + ε)\{y\}\_\{n\} et donc \{x\}\_\{n\}
≥ 0 pour n ≥ N. On a à la fois \{x\}\_\{n\} = O(\{y\}\_\{n\}) et
\{y\}\_\{n\} = O(\{x\}\_\{n\}) ce qui d'après le théorème précédent
montre que les deux séries convergent ou divergent simultanément.
Supposons alors les séries convergentes. On a \textbar{}\{x\}\_\{n\} −
\{y\}\_\{n\}\textbar{} = o(\{y\}\_\{n\}), on en déduit donc la
convergence de \textbackslash{}mathop\{\textbackslash{}mathop\{∑ \}\}
\textbar{}\{x\}\_\{n\} − \{y\}\_\{n\}\textbar{} et que
\{R\}\_\{n\}(\textbar{}x − y\textbar{}) = o(\{R\}\_\{n\}(y)). Mais
\textbar{}\{R\}\_\{n\}(x) − \{R\}\_\{n\}(y)\textbar{}≤
\{R\}\_\{n\}(\textbar{}x − y\textbar{}) donc \textbar{}\{R\}\_\{n\}(x) −
\{R\}\_\{n\}(y)\textbar{} = o(\{R\}\_\{n\}(y)) et donc \{R\}\_\{n\}(x) ∼
\{R\}\_\{n\}(y). Supposons maintenant les séries divergentes. Alors,
soit la série \textbackslash{}mathop\{\textbackslash{}mathop\{∑ \}\}
\textbar{}\{x\}\_\{n\} − \{y\}\_\{n\}\textbar{} converge et comme
\textbackslash{}mathop\{lim\}\{S\}\_\{n\}(y) = +∞ on a
\{S\}\_\{n\}(\textbar{}x − y\textbar{}) = o(\{S\}\_\{n\}(y)), soit elle
diverge et le théorème précédent assure que \{S\}\_\{n\}(\textbar{}x −
y\textbar{}) = o(\{S\}\_\{n\}(y)). Mais alors \textbar{}\{S\}\_\{n\}(x)
− \{S\}\_\{n\}(y)\textbar{}≤ \{S\}\_\{n\}(\textbar{}x − y\textbar{}) =
o(\{S\}\_\{n\}(y)), soit \{S\}\_\{n\}(x) ∼ \{S\}\_\{n\}(y).

\paragraph{7.3.3 Séries de Riemann et de Bertrand}

Théorème~7.3.6 (séries de Riemann). Soit α ∈ ℝ. La série
\textbackslash{}mathop\{\textbackslash{}mathop\{∑ \}\} \{ 1
\textbackslash{}over \{n\}\^{}\{α\}\} converge si et seulement si~α
\textgreater{} 1.

Si α \textgreater{} 1, on a \{R\}\_\{n\} ∼\{ 1 \textbackslash{}over
α−1\} \{ 1 \textbackslash{}over \{n\}\^{}\{α−1\}\} ~; si α \textless{}
1, on a \{S\}\_\{n\} ∼\{ \{n\}\^{}\{1−α\} \textbackslash{}over 1−α\} ~;
si α = 1, \{S\}\_\{n\} ∼\textbackslash{}mathop\{ log\} n.

Démonstration Soit α\textbackslash{}mathrel\{≠\}1. Posons \{x\}\_\{n\}
=\{ 1 \textbackslash{}over \{n\}\^{}\{α\}\} et \{y\}\_\{n\} =\{ 1
\textbackslash{}over \{n\}\^{}\{α−1\}\} −\{ 1 \textbackslash{}over
\{(n+1)\}\^{}\{α−1\}\} . On a

\{ \{y\}\_\{n\} \textbackslash{}over \{x\}\_\{n\}\} = −\{ \{(1 +\{ 1
\textbackslash{}over n\} )\}\^{}\{1−α\} − 1 \textbackslash{}over \{ 1
\textbackslash{}over n\} \}

qui admet pour limite l'opposé de la dérivée en 0 de
x\textbackslash{}mathrel\{↦\}\{(1 + x)\}\^{}\{1−α\} soit α − 1. On a
donc \{x\}\_\{n\} ∼\{ 1 \textbackslash{}over α−1\} \{y\}\_\{n\}
\textgreater{} 0. Les deux séries sont donc de même nature. Or
\{S\}\_\{n\}(y) = 1 −\{ 1 \textbackslash{}over \{(n+1)\}\^{}\{α−1\}\}
admet une limite finie si et seulement si~α \textgreater{} 1. Si α
\textgreater{} 1, on a \{R\}\_\{n\}(x) ∼\{ 1 \textbackslash{}over α−1\}
\{R\}\_\{n\}(y) =\{ 1 \textbackslash{}over α−1\} \{ 1
\textbackslash{}over \{n\}\^{}\{α−1\}\} . Si α \textless{} 1, on a
\{S\}\_\{n\}(x) ∼\{ 1 \textbackslash{}over α−1\} \{S\}\_\{n\}(y) =\{ 1
\textbackslash{}over 1−α\} (\{(n + 1)\}\^{}\{1−α\} − 1) ∼\{
\{n\}\^{}\{1−α\} \textbackslash{}over 1−α\} . Enfin, si α = 1, on
aboutit à une étude similaire avec \{y\}\_\{n\}
=\textbackslash{}mathop\{ log\} (n + 1) −\textbackslash{}mathop\{ log\}
n =\textbackslash{}mathop\{ log\} (1 +\{ 1 \textbackslash{}over n\} )
∼\{ 1 \textbackslash{}over n\} .

Corollaire~7.3.7 (séries de Bertrand). Soit α,β ∈ ℝ. La série
\{\textbackslash{}mathop\{\textbackslash{}mathop\{∑ \}\} \}\_\{n≥2\}\{ 1
\textbackslash{}over \{n\}\^{}\{α\}\{(\textbackslash{}mathop\{log\}
n)\}\^{}\{β\}\} converge si et seulement si~α \textgreater{} 1 ou α =
1,β \textgreater{} 1.

Démonstration Soit \{x\}\_\{n\} =\{ 1 \textbackslash{}over
\{n\}\^{}\{α\}\{(\textbackslash{}mathop\{log\} n)\}\^{}\{β\}\} . Si α
\textgreater{} 1, soit γ tel que α \textgreater{} γ \textgreater{} 1 et
\{y\}\_\{n\} =\{ 1 \textbackslash{}over \{n\}\^{}\{γ\}\} . La série
\textbackslash{}mathop\{\textbackslash{}mathop\{∑ \}\} \{y\}\_\{n\}
converge et \{ \{x\}\_\{n\} \textbackslash{}over \{y\}\_\{n\}\} =\{ 1
\textbackslash{}over \{n\}\^{}\{α−γ\}\{(\textbackslash{}mathop\{log\}
n)\}\^{}\{β\}\} tend vers 0 car α − γ \textgreater{} 0. On a donc
\{x\}\_\{n\} = o(\{y\}\_\{n\}) et la série
\textbackslash{}mathop\{\textbackslash{}mathop\{∑ \}\} \{x\}\_\{n\}
converge. Si α \textless{} 1, soit γ tel que α \textless{} γ \textless{}
1 et \{y\}\_\{n\} =\{ 1 \textbackslash{}over \{n\}\^{}\{γ\}\} . La série
\textbackslash{}mathop\{\textbackslash{}mathop\{∑ \}\} \{y\}\_\{n\}
diverge et \{ \{y\}\_\{n\} \textbackslash{}over \{x\}\_\{n\}\} =\{
\{(\textbackslash{}mathop\{log\} n)\}\^{}\{β\} \textbackslash{}over
\{n\}\^{}\{γ−α\}\} tend vers 0 car γ − α \textgreater{} 0. On a donc
\{y\}\_\{n\} = o(\{x\}\_\{n\}) et la série
\textbackslash{}mathop\{\textbackslash{}mathop\{∑ \}\} \{x\}\_\{n\}
converge. Le cas α = 1 résulte facilement du paragraphe suivant.

\paragraph{7.3.4 Comparaison à des intégrales}

Théorème~7.3.8 Soit f : {[}0,+∞{[}→ ℝ continue par morceaux,
décroissante, positive. Posons \{w\}\_\{n\} =\{\textbackslash{}mathop\{∫
\} \}\_\{n−1\}\^{}\{n\}f(t) dt − f(n). Alors la série
\textbackslash{}mathop\{\textbackslash{}mathop\{∑ \}\} \{w\}\_\{n\} est
convergente.

Démonstration On a \{w\}\_\{n\} =\{\textbackslash{}mathop\{∫ \}
\}\_\{n−1\}\^{}\{n\}(f(t) − f(n)) dt. Comme f est décroissante,
\textbackslash{}mathop\{∀\}t ∈ {[}n − 1,n{]}, f(t) ≥ f(n) et donc
\{w\}\_\{n\} ≥ 0. Mais d'autre part

0 ≤ \{w\}\_\{n\} ≤\{\textbackslash{}mathop\{∫ \} \}\_\{n−1\}\^{}\{n\}f(n
− 1) dt − f(n) = f(n − 1) − f(n)

On a \{\textbackslash{}mathop\{\textbackslash{}mathop\{∑ \}\}
\}\_\{p=1\}\^{}\{n\}(f(p − 1) − f(p)) = f(0) − f(n) qui admet une limite
quand p tend vers + ∞ (car f admet une limite en + ∞~: elle est
décroissante et positive). Ceci montre que la série
\textbackslash{}mathop\{\textbackslash{}mathop\{∑ \}\} (f(p − 1) − f(p))
converge. Il en est donc de même de la série
\textbackslash{}mathop\{\textbackslash{}mathop\{∑ \}\} \{w\}\_\{n\}.

Corollaire~7.3.9 Soit f : {[}0,+∞{[}→ ℝ continue décroissante positive.
Alors la série \textbackslash{}mathop\{\textbackslash{}mathop\{∑ \}\}
f(n) converge si et seulement si f est intégrable sur {[}0,+∞{[}.

Démonstration En effet, on déduit du théorème précédent que les deux
séries \textbackslash{}mathop\{\textbackslash{}mathop\{∑ \}\} f(n) et
\textbackslash{}mathop\{\textbackslash{}mathop\{∑ \}\}
\{\textbackslash{}mathop\{∫ \} \}\_\{n−1\}\^{}\{n\}f(t) dt convergent ou
divergent simultanément, car leur différence est une série convergente.
Mais on a \{\textbackslash{}mathop\{\textbackslash{}mathop\{∑ \}\}
\}\_\{p=1\}\^{}\{n\}\{\textbackslash{}mathop\{∫ \}
\}\_\{p−1\}\^{}\{p\}f(t) dt =\{\textbackslash{}mathop\{∫ \}
\}\_\{0\}\^{}\{n\}f(t) dt =\{\textbackslash{}mathop\{∫ \}
\}\_\{{[}0,n{]}\}f. Si f est intégrable, comme la suite
\{({[}0,n{]})\}\_\{n∈ℕ\} est une suite croissante de segments dont la
réunion est {[}0,+∞{[}, la suite (\{\textbackslash{}mathop\{∫ \}
\}\_\{{[}0,n{]}\}f) est convergente de limite
\{\textbackslash{}mathop\{∫ \} \}\_\{{[}0,+∞{[}\}f, donc la série
\textbackslash{}mathop\{\textbackslash{}mathop\{∑ \}\}
\{\textbackslash{}mathop\{∫ \} \}\_\{n−1\}\^{}\{n\}f(t) dt converge et
il en est de même de \textbackslash{}mathop\{\textbackslash{}mathop\{∑
\}\} f(n). Si \{\textbackslash{}mathop\{\textbackslash{}mathop\{∑ \}\}
\}\_\{\}f(n) converge, il en est de même de
\textbackslash{}mathop\{\textbackslash{}mathop\{∑ \}\}
\{\textbackslash{}mathop\{∫ \} \}\_\{n−1\}\^{}\{n\}f(t) dt, et si
{[}a,b{]} est un segment contenu dans {[}0,+∞{[} les majorations

\{\textbackslash{}mathop\{∫ \} \}\_\{{[}a,b{]}\}f
≤\{\textbackslash{}mathop\{∫ \} \}\_\{0\}\^{}\{{[}b{]}+1\}f =\{
\textbackslash{}mathop\{∑ \}\}\_\{p=0\}\^{}\{{[}b{]}+1\}\{
\textbackslash{}mathop\{\textbackslash{}mathop\{∫ \} \}
\}\_\{p−1\}\^{}\{p\}f(t) dt ≤\{\textbackslash{}mathop\{∑
\}\}\_\{p=0\}\^{}\{+∞\}\{\textbackslash{}mathop\{\textbackslash{}mathop\{∫
\} \} \}\_\{p−1\}\^{}\{p\}f(t) dt

et le fait que f soit positive, montrent que f est intégrable sur
{[}0,+∞{[}.

Remarque~7.3.3 Bien entendu, il suffit que la condition de décroissance
soit vérifiée sur un certain {[}\{t\}\_\{0\},+∞{[}.

Dans le cas d'une série divergente, l'encadrement

\{\textbackslash{}mathop\{∫ \} \}\_\{0\}\^{}\{n+1\}f(t) dt
≤\{\textbackslash{}mathop\{∑ \}\}\_\{p=0\}\^{}\{n\}f(p) ≤ f(0) +\{
\textbackslash{}mathop\{\textbackslash{}mathop\{∫ \} \}
\}\_\{0\}\^{}\{n\}f(t) dt

permet souvent d'obtenir un équivalent de la somme partielle de la
série. Dans le cas d'une série convergente, on a de même

\{\textbackslash{}mathop\{∫ \} \}\_\{n+1\}\^{}\{+∞\}f(t) dt
≤\{\textbackslash{}mathop\{∑ \}\}\_\{p=n+1\}\^{}\{+∞\}f(p)
≤\{\textbackslash{}mathop\{\textbackslash{}mathop\{∫ \} \}
\}\_\{n\}\^{}\{+∞\}f(t) dt

ce qui permet souvent d'obtenir une majoration ou un équivalent du reste
de la série.

Exemple~7.3.1 Dans le cas limite des séries de Bertrand,
\textbackslash{}mathop\{\textbackslash{}mathop\{∑ \}\} \{ 1
\textbackslash{}over n\{(\textbackslash{}mathop\{log\} n)\}\^{}\{β\}\} ,
la fonction f(t) =\{ 1 \textbackslash{}over
t\{(\textbackslash{}mathop\{log\} t)\}\^{}\{β\}\} est continue
décroissante (pour t assez grand) de limite 0. Donc la série est de même
nature que l'intégrale \{\textbackslash{}mathop\{∫ \}
\}\_\{3\}\^{}\{+∞\}\{ dt \textbackslash{}over
t\{(\textbackslash{}mathop\{log\} t)\}\^{}\{β\}\} . Mais on a
\{\textbackslash{}mathop\{∫ \} \}\_\{3\}\^{}\{x\}\{ dt
\textbackslash{}over t\{(\textbackslash{}mathop\{log\} t)\}\^{}\{β\}\}
=\{\textbackslash{}mathop\{∫ \} \}\_\{\textbackslash{}mathop\{log\}
3\}\^{}\{\textbackslash{}mathop\{log\} x\}\{ du \textbackslash{}over
\{u\}\^{}\{β\}\} (poser u =\textbackslash{}mathop\{ log\} t) qui admet
une limite finie quand x tend vers + ∞ si et seulement si β
\textgreater{} 1. Ceci achève la démonstration du critère de convergence
des séries de Bertrand.

{[}\href{coursse38.html}{next}{]} {[}\href{coursse36.html}{prev}{]}
{[}\href{coursse36.html\#tailcoursse36.html}{prev-tail}{]}
{[}\href{coursse37.html}{front}{]}
{[}\href{coursch8.html\#coursse37.html}{up}{]}

\end{document}
