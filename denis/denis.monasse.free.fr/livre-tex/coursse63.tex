\documentclass[]{article}
\usepackage[T1]{fontenc}
\usepackage{lmodern}
\usepackage{amssymb,amsmath}
\usepackage{ifxetex,ifluatex}
\usepackage{fixltx2e} % provides \textsubscript
% use upquote if available, for straight quotes in verbatim environments
\IfFileExists{upquote.sty}{\usepackage{upquote}}{}
\ifnum 0\ifxetex 1\fi\ifluatex 1\fi=0 % if pdftex
  \usepackage[utf8]{inputenc}
\else % if luatex or xelatex
  \ifxetex
    \usepackage{mathspec}
    \usepackage{xltxtra,xunicode}
  \else
    \usepackage{fontspec}
  \fi
  \defaultfontfeatures{Mapping=tex-text,Scale=MatchLowercase}
  \newcommand{\euro}{€}
\fi
% use microtype if available
\IfFileExists{microtype.sty}{\usepackage{microtype}}{}
\ifxetex
  \usepackage[setpagesize=false, % page size defined by xetex
              unicode=false, % unicode breaks when used with xetex
              xetex]{hyperref}
\else
  \usepackage[unicode=true]{hyperref}
\fi
\hypersetup{breaklinks=true,
            bookmarks=true,
            pdfauthor={},
            pdftitle={Convergence des series enti`eres},
            colorlinks=true,
            citecolor=blue,
            urlcolor=blue,
            linkcolor=magenta,
            pdfborder={0 0 0}}
\urlstyle{same}  % don't use monospace font for urls
\setlength{\parindent}{0pt}
\setlength{\parskip}{6pt plus 2pt minus 1pt}
\setlength{\emergencystretch}{3em}  % prevent overfull lines
\setcounter{secnumdepth}{0}
 
/* start css.sty */
.cmr-5{font-size:50%;}
.cmr-7{font-size:70%;}
.cmmi-5{font-size:50%;font-style: italic;}
.cmmi-7{font-size:70%;font-style: italic;}
.cmmi-10{font-style: italic;}
.cmsy-5{font-size:50%;}
.cmsy-7{font-size:70%;}
.cmex-7{font-size:70%;}
.cmex-7x-x-71{font-size:49%;}
.msbm-7{font-size:70%;}
.cmtt-10{font-family: monospace;}
.cmti-10{ font-style: italic;}
.cmbx-10{ font-weight: bold;}
.cmr-17x-x-120{font-size:204%;}
.cmsl-10{font-style: oblique;}
.cmti-7x-x-71{font-size:49%; font-style: italic;}
.cmbxti-10{ font-weight: bold; font-style: italic;}
p.noindent { text-indent: 0em }
td p.noindent { text-indent: 0em; margin-top:0em; }
p.nopar { text-indent: 0em; }
p.indent{ text-indent: 1.5em }
@media print {div.crosslinks {visibility:hidden;}}
a img { border-top: 0; border-left: 0; border-right: 0; }
center { margin-top:1em; margin-bottom:1em; }
td center { margin-top:0em; margin-bottom:0em; }
.Canvas { position:relative; }
li p.indent { text-indent: 0em }
.enumerate1 {list-style-type:decimal;}
.enumerate2 {list-style-type:lower-alpha;}
.enumerate3 {list-style-type:lower-roman;}
.enumerate4 {list-style-type:upper-alpha;}
div.newtheorem { margin-bottom: 2em; margin-top: 2em;}
.obeylines-h,.obeylines-v {white-space: nowrap; }
div.obeylines-v p { margin-top:0; margin-bottom:0; }
.overline{ text-decoration:overline; }
.overline img{ border-top: 1px solid black; }
td.displaylines {text-align:center; white-space:nowrap;}
.centerline {text-align:center;}
.rightline {text-align:right;}
div.verbatim {font-family: monospace; white-space: nowrap; text-align:left; clear:both; }
.fbox {padding-left:3.0pt; padding-right:3.0pt; text-indent:0pt; border:solid black 0.4pt; }
div.fbox {display:table}
div.center div.fbox {text-align:center; clear:both; padding-left:3.0pt; padding-right:3.0pt; text-indent:0pt; border:solid black 0.4pt; }
div.minipage{width:100%;}
div.center, div.center div.center {text-align: center; margin-left:1em; margin-right:1em;}
div.center div {text-align: left;}
div.flushright, div.flushright div.flushright {text-align: right;}
div.flushright div {text-align: left;}
div.flushleft {text-align: left;}
.underline{ text-decoration:underline; }
.underline img{ border-bottom: 1px solid black; margin-bottom:1pt; }
.framebox-c, .framebox-l, .framebox-r { padding-left:3.0pt; padding-right:3.0pt; text-indent:0pt; border:solid black 0.4pt; }
.framebox-c {text-align:center;}
.framebox-l {text-align:left;}
.framebox-r {text-align:right;}
span.thank-mark{ vertical-align: super }
span.footnote-mark sup.textsuperscript, span.footnote-mark a sup.textsuperscript{ font-size:80%; }
div.tabular, div.center div.tabular {text-align: center; margin-top:0.5em; margin-bottom:0.5em; }
table.tabular td p{margin-top:0em;}
table.tabular {margin-left: auto; margin-right: auto;}
div.td00{ margin-left:0pt; margin-right:0pt; }
div.td01{ margin-left:0pt; margin-right:5pt; }
div.td10{ margin-left:5pt; margin-right:0pt; }
div.td11{ margin-left:5pt; margin-right:5pt; }
table[rules] {border-left:solid black 0.4pt; border-right:solid black 0.4pt; }
td.td00{ padding-left:0pt; padding-right:0pt; }
td.td01{ padding-left:0pt; padding-right:5pt; }
td.td10{ padding-left:5pt; padding-right:0pt; }
td.td11{ padding-left:5pt; padding-right:5pt; }
table[rules] {border-left:solid black 0.4pt; border-right:solid black 0.4pt; }
.hline hr, .cline hr{ height : 1px; margin:0px; }
.tabbing-right {text-align:right;}
span.TEX {letter-spacing: -0.125em; }
span.TEX span.E{ position:relative;top:0.5ex;left:-0.0417em;}
a span.TEX span.E {text-decoration: none; }
span.LATEX span.A{ position:relative; top:-0.5ex; left:-0.4em; font-size:85%;}
span.LATEX span.TEX{ position:relative; left: -0.4em; }
div.float img, div.float .caption {text-align:center;}
div.figure img, div.figure .caption {text-align:center;}
.marginpar {width:20%; float:right; text-align:left; margin-left:auto; margin-top:0.5em; font-size:85%; text-decoration:underline;}
.marginpar p{margin-top:0.4em; margin-bottom:0.4em;}
.equation td{text-align:center; vertical-align:middle; }
td.eq-no{ width:5%; }
table.equation { width:100%; } 
div.math-display, div.par-math-display{text-align:center;}
math .texttt { font-family: monospace; }
math .textit { font-style: italic; }
math .textsl { font-style: oblique; }
math .textsf { font-family: sans-serif; }
math .textbf { font-weight: bold; }
.partToc a, .partToc, .likepartToc a, .likepartToc {line-height: 200%; font-weight:bold; font-size:110%;}
.chapterToc a, .chapterToc, .likechapterToc a, .likechapterToc, .appendixToc a, .appendixToc {line-height: 200%; font-weight:bold;}
.index-item, .index-subitem, .index-subsubitem {display:block}
.caption td.id{font-weight: bold; white-space: nowrap; }
table.caption {text-align:center;}
h1.partHead{text-align: center}
p.bibitem { text-indent: -2em; margin-left: 2em; margin-top:0.6em; margin-bottom:0.6em; }
p.bibitem-p { text-indent: 0em; margin-left: 2em; margin-top:0.6em; margin-bottom:0.6em; }
.paragraphHead, .likeparagraphHead { margin-top:2em; font-weight: bold;}
.subparagraphHead, .likesubparagraphHead { font-weight: bold;}
.quote {margin-bottom:0.25em; margin-top:0.25em; margin-left:1em; margin-right:1em; text-align:justify;}
.verse{white-space:nowrap; margin-left:2em}
div.maketitle {text-align:center;}
h2.titleHead{text-align:center;}
div.maketitle{ margin-bottom: 2em; }
div.author, div.date {text-align:center;}
div.thanks{text-align:left; margin-left:10%; font-size:85%; font-style:italic; }
div.author{white-space: nowrap;}
.quotation {margin-bottom:0.25em; margin-top:0.25em; margin-left:1em; }
h1.partHead{text-align: center}
.sectionToc, .likesectionToc {margin-left:2em;}
.subsectionToc, .likesubsectionToc {margin-left:4em;}
.subsubsectionToc, .likesubsubsectionToc {margin-left:6em;}
.frenchb-nbsp{font-size:75%;}
.frenchb-thinspace{font-size:75%;}
.figure img.graphics {margin-left:10%;}
/* end css.sty */

\title{Convergence des series enti`eres}
\author{}
\date{}

\begin{document}
\maketitle

\textbf{Warning: 
requires JavaScript to process the mathematics on this page.\\ If your
browser supports JavaScript, be sure it is enabled.}

\begin{center}\rule{3in}{0.4pt}\end{center}

[
[]
[

\subsubsection{11.1 Convergence des séries entières}

\paragraph{11.1.1 Notion de série entière}

Définition~11.1.1 Soit (a_n)_n\in\mathbb{N}~ une suite de l'espace
vectoriel normé complet E. On appelle série entière associée à la suite
(a_n) la série de fonctions de \mathbb{C} (resp. \mathbb{R}~) dans E,
\\sum ~
_n≥0u_n, où l'on pose u_n(z) =
a_nz^n~; on notera simplement
\\sum ~
_n≥0a_nz^n cette série de fonctions de la
variable z.

Remarque~11.1.1 Dans le cas où E = \mathbb{R}~ ou E = \mathbb{C}, la série entière est
associée à une unique série formelle
\\sum ~
_n=0^+\infty~a_nX^n \in K[[X]].

\paragraph{11.1.2 Rayon de convergence}

Lemme~11.1.1 (Abel). Soit E un K-espace vectoriel normé complet,
\\sum ~
a_nz^n une série entière à coefficients dans E. Soit
z_0 \in K^∗ tel que la suite
(a_nz_0^n) soit bornée. Alors la série
\\sum ~
a_nz^n converge absolument pour tout z \in K tel que
z < z_0~; la
série entière converge même normalement dans tout disque fermé D'(0,r) =
\z \in
K∣z\leq r\
pour tout nombre réel r tel que r <
z_0.

Démonstration Soit M ≥ 0 tel que \forall~~n \in \mathbb{N}~,
\a_nz_0^n\
\leq M et soit z \in K tel que z <
z_0. On a alors
\a_nz^n\
=\
a_nz_0^n\
\left  z \over z_0
\right ^n \leq M\left
 z \over z_0 \right
^n. Comme \left  z
\over z_0 \right 
< 1, la série géométrique est convergente et donc la série
\\sum ~
a_nz^n converge absolument. Pour z \in D'(0,r), on a
de la même fa\ccon
\a_nz^n\
\leq M\left  r \over
z_0 \right ^n qui est une
série convergente indépendante de z, donc la série converge normalement
sur D'(0,r).

Théorème~11.1.2 Soit E un K-espace vectoriel normé complet,
\\sum ~
a_nz^n une série entière à coefficients dans E.
Posons R_1 =\
sup\z∣\\\sum
 a_nz^n\text converge
\ \in \mathbb{R}~^+ \cup\ +
\infty~\ et R_2 =\
sup\z∣(a_nz^n)\text
est bornée \ \in \mathbb{R}~^+ \cup\ +
\infty~\. On a R_1 = R_2. En notant R la
valeur commune, la série converge absolument dans D(0,R) =
\z \in K∣z
< R\ et converge normalement dans tout disque
fermé D'(0,r) = \z \in
K∣z\leq r\
tel que r < R.

Démonstration Soit r \in [0,R_1[~; d'après la propriété
caractéristique de la borne supérieure, il existe z \in K tel que la série
\\sum ~
a_nz^n converge avec r <
z\leq R_1. Mais alors
lima_nz^n~ = 0, donc la
suite (a_nz^n) est bornée et a fortiori la suite
(a_nr^n) est bornée~; donc r \in [0,R_2],
soit [0,R_1[\subset~ [0,R_2] et donc R_1 \leq
R_2. Soit r \in [0,R_2[~; d'après la propriété
caractéristique de la borne supérieure, il existe z \in K tel que la suite
(a_nz^n) soit bornée avec r <
z\leq R_2. Mais alors, d'après le lemme
d'Abel, la série \\sum ~
a_nr^n converge absolument, r \in
[0,R_1], soit [0,R_2[\subset~ [0,R_1] et
donc R_2 \leq R_1.

Soit alors z \in D(0,R)~; il existe z_0 \in K tel que la suite
(a_nz_0^n) soit bornée avec
z < z_0\leq R.
Mais alors, d'après le lemme d'Abel, la série
\\sum ~
a_nz^n converge absolument. De même, soit r
< R~; il existe z_0 \in K tel que la suite
(a_nz_0^n) soit bornée avec r <
z_0\leq R. Mais alors, d'après le lemme
d'Abel, la série \\sum ~
a_nz^n converge normalement sur D'(0,r).

Remarque~11.1.2 Le lecteur prendra garde au fait qu'en général la série
ne converge pas normalement sur D(0,R) ni même uniformément.

Définition~11.1.2 R est appelé le rayon de convergence de la série
entière, D(0,R) son disque ouvert de convergence, C(0,R) =
\z \in K∣z
= R\ son cercle de convergence.

Remarque~11.1.3 Par définition même au vu des résultats précédents, la
série converge absolument dans le disque ouvert de convergence (et même
uniformément dans tout disque fermé inclus dans le disque ouvert de
convergence)~; pour z > R la série
diverge et en fait, la suite (a_nz^n) n'est même pas
bornée. La nature de la série sur le disque fermé de convergence dépend
de la série et du point considéré.

Exemple~11.1.1 Soit \alpha~ \in \mathbb{R}~~; la série entière
\\sum   z^n~
\over n^\alpha~ a pour rayon de convergence 1~;
pour z = 1 la nature de la série dépend à la fois de
\alpha~ et de z.

\begin{itemize}
\itemsep1pt\parskip0pt\parsep0pt
\item
  (i) Pour \alpha~ > 1, la série converge pour tout z tel que
  z = 1
\item
  (ii) Pour \alpha~ \leq 0, la série diverge pour tout z tel que
  z = 1 (le terme général ne tend pas vers 0)
\item
  (iii) Pour 0 < \alpha~ \leq 1, la série diverge en z = 1 mais
  converge pout tout point z tel que z = 1 et
  z\neq~1 (appliquer le critère d'Abel).
\end{itemize}

\paragraph{11.1.3 Recherche du rayon de convergence}

Les deux remarques suivantes, qui découlent immédiatement des résultats
précédents peuvent rendre de grands services dans la détermination du
rayon de convergence

\begin{itemize}
\itemsep1pt\parskip0pt\parsep0pt
\item
  (i) si z \in K est tel que la série
  \\sum ~
  a_nz^n converge, alors z\leq R
\item
  (ii) si z \in K est tel que la série
  \\sum ~
  a_nz^n diverge, alors R \leqz
\end{itemize}

On pourra éventuellement, pour cette recherche, trouver refuge dans l'un
des théorèmes suivants

Théorème~11.1.3 (règle de d'Alembert). On suppose que
\forall~~n \in \mathbb{N}~,
a_n\neq~0~; si la suite 
\a_n+1\
\over
\a_n\ admet
une limite \ell \in \mathbb{R}~^+ \cup\ + \infty~\,
alors le rayon de convergence de la série entière
\\sum ~
a_nz^n est  1 \over \ell .

Démonstration Il suffit d'appliquer la règle de d'Alembert pour les
séries numériques en remarquant que 
\a_n+1z^n+1\
\over
\a_nz^n\
admet la limite \ellz et que donc la série converge
absolument pour \ellz < 1 et diverge pour
\ellz > 1.

Remarque~11.1.4 On prendra garde à la condition
\forall~~n \in \mathbb{N}~,
a_n\neq~0. En particulier, on ne tentera
pas d'appliquer cette règle à des séries comportant une infinité de
termes nuls comme les séries entières du type
\\sum ~
a_nz^2n ou
\\sum ~
a_nz^n^2 ~; c'est ainsi qu'une
application imprudente de la règle de d'Alembert à la série entière
\\sum ~
3^nz^2n pourrait faire croire que le rayon de
convergence est  1 \over 3 alors que l'écriture
3^nz^2n =
\left
(3z^2\right )^n
montre qu'il vaut  1 \over \sqrt3
.

Exemple~11.1.2 Pour \\\sum
  z^n \over n! , on a R = +\infty~~; pour
\\sum   z^n~
\over n^\alpha~ , on a R = 1~; pour
\\sum  n!z^n~,
on a R = 0 (la série diverge pour tout z\neq~0).

Théorème~11.1.4 (règle d'Hadamard). Le rayon de convergence de la série
entière \\sum ~
a_nz^n est égal à

 1 \over
limsup\rootn\of\a_n\~
\in \mathbb{R}~^+ \cup\ + \infty~\

Démonstration Posons \ell =\
limsup\rootn\of\a_n\
\in \mathbb{R}~^+ \cup\ + \infty~\. Soit z \in K
tel que z < 1 \over \ell .
On a alors \ell < 1 \over
z . Soit donc \rho tel que \ell < \rho
< 1 \over z . D'après
la propriété de la limite supérieure, il existe N \in \mathbb{N}~ tel que n ≥ N
\rigtharrow~\rootn\of\a_n\
\leq \rho soit encore
\a_n\ \leq
\rho^n et donc
\a_nz^n\
\leq (\rhoz)^n. Mais \rhoz
< 1 et donc la série
\\sum ~
(\rhoz)^n converge. On en déduit que la
série \\sum ~
a_nz^n converge absolument, soit R ≥ 1
\over \ell . De plus, si z
> 1 \over \ell , on a \ell > 1
\over z . Comme \ell est valeur
d'adhérence de la suite
(\rootn\of\a_n\),
il existe une infinité de n tels que
\rootn\of\a_n\
> 1 \over z soit
\a_nz^n\
> 1~; la suite (a_nz^n) ne tend pas
vers 0, donc la série diverge~; ceci montre que R \leq 1
\over \ell , ce qui achève la démonstration.

Remarque~11.1.5 En particulier, si la suite
(\rootn\of\a_n\)
converge vers \ell, on a R = 1 \over \ell .

\paragraph{11.1.4 Opérations sur les séries entières}

Proposition~11.1.5 Soit
\\sum ~
a_nz^n et
\\sum ~
b_nz^n deux séries entières à coefficients dans E de
rayons de convergence respectifs R_1 et R_2, \alpha~ et \beta~
des scalaires. Alors la série entière
\\sum  (\alpha~a_n~ +
\beta~b_n)z^n a un rayon de convergence supérieur ou égal
à min(R_1,R_2~) et

z <\
min(R_1,R_2) \rigtharrow~\\sum
_n=0^+\infty~(\alpha~a_ n + \beta~b_n)z^n =
\alpha~\sum _n=0^+\infty~a_
nz^n + \beta~\\sum
_n=0^+\infty~b_ nz^n

Démonstration En effet, si z
< min(R_1,R_2~),
les deux séries \\sum ~
a_nz^n et
\\sum ~
b_nz^n sont convergentes, et donc la série
\\sum  (\alpha~a_n~ +
\beta~b_n)z^n converge également, soit R
≥ min(R_1,R_2~). La formule
découle immédiatement du résultat similaire sur les séries numériques.

Remarque~11.1.6 L'exemple b_n = -a_n, \alpha~ = \beta~ = 1,
montre que l'on peut avoir R >\
min(R_1,R_2)

Proposition~11.1.6 Soit
\\sum ~
a_nz^n et
\\sum ~
b_nz^n deux séries entières à coefficients dans K de
rayons de convergence respectifs R_1 et R_2. Posons
c_n = \\sum ~
_k=0^na_kb_n-k
= \\sum ~
_p+q=na_pb_q (série entière produit). Alors la
série entière \\sum ~
c_nz^n a un rayon de convergence supérieur ou égal à
min(R_1,R_2~) et

z <\
min(R_1,R_2) \rigtharrow~\\sum
_n=0^+\infty~c_ nz^n =
\left (\\sum
_n=0^+\infty~a_ nz^n\right
)\left (\\sum
_n=0^+\infty~b_ nz^n\right
)

Démonstration En effet, si z
< min(R_1,R_2~),
les deux séries \\sum ~
a_nz^n et
\\sum ~
b_nz^n sont absolument convergentes, et donc la
série produit de Cauchy est également absolument convergente. Mais on a
\\sum ~
_p+q=n(a_pz^p)(b_qz^q) =
z^n \\sum ~
_p+q=na_pb_q = c_nz^n. On a
donc R ≥ min(R_1,R_2~). La
formule découle immédiatement du résultat similaire sur les séries
numériques (la somme du produit de Cauchy est le produit des sommes des
séries).

Proposition~11.1.7 Soit
\\sum ~
a_nz^n une série entière à coefficients dans K avec
a_0\neq~0. Il existe alors une unique
série entière \\sum ~
b_nz^n tel que le produit des deux séries entières
soit la constante 1. Si
\\sum ~
a_nz^n a un rayon de convergence non nul R, il en
est de même du rayon de convergence R' de la série entière
\\sum ~
b_nz^n. On a

z < min~(R,R')
\rigtharrow~\sum _n=0^+\infty~b_
nz^n = 1 \over \\sum
_n=0^+\infty~a_nz^n

Démonstration On doit avoir a_0b_0 = 1 et pour n ≥ 1,
\\sum ~
_k=0^na_n-kb_k = 0. La suite
(b_n) est donc définie par b_0 = 1
\over a_0 et pour n ≥ 1, b_n = - 1
\over a_0 \
\sum ~
_k=0^n-1a_n-kb_k ce qui définit
parfaitement la suite par récurrence. Remarquons que si l'on multiplie
tous les a_n par \lambda~\neq~0, tous les
b_n sont divisés par \lambda~, et les rayons de convergence ne sont
pas modifiés. Sans nuire à la généralité, on peut donc supposer que
a_0 = 1. On a alors b_0 = 1 et b_n =
-\\sum ~
_k=0^n-1a_n-kb_k. Supposons donc R
> 0 et soit r < R. La suite
(a_nr^n) est donc bornée. Soit M tel que
\forall~~n,
a_nr^n \leq M soit
a_n\leq Mr^-n. On va montrer que
\forall~n ≥ 1, b_n~\leq
M(M + 1)^n-1r^-n par récurrence sur n. Pour n = 1,
on a b_1 = -a_1, soit b_1
= a_1\leq Mr^-1 ce qui est bien
l'inégalité souhaitée. Supposons l'inégalité vérifiée de 1 à n - 1. On a
alors (compte tenu de b_0 = 1)

\begin{align*} b_n&
\leq& a_n + \\sum
_k=1^n-1a_
n-kb_k \%&
\\ & \leq& Mr^-n +
\sum _k=1^n-1Mr^k-n~M(M
+ 1)^k-1r^-k \%& \\
& =& Mr^-n\left (1 +
M\sum _k=1^n-1~(M +
1)^k-1\right ) \%&
\\ & =&
Mr^-n\left (1 + M (M + 1)^n-1 - 1
\over (M + 1) - 1 \right ) = M(M +
1)^n-1r^-n\%& \\
\end{align*}

ce qui achève la récurrence. Alors
b_nz^n\leq M \over
M+1 \left ( (M+1)z
\over r \right )^n ce qui
montre que la série \\\sum
 b_nz^n converge pour z
< r \over M+1 , soit R' ≥ r
\over M+1 > 0.

La formule découle immédiatement de la proposition précédente.

Remarque~11.1.7 L'exemple a_0 = 1, a_1 = -1,
a_n = 0 pour n ≥ 2 (c'est-à-dire de la série entière 1 - z),
pour laquelle la série inverse est la série
\\sum  z^n~
pour laquelle R' = 1, montre qu'on ne peut pas dire grand chose de la
valeur effective de R'. On peut avoir aussi bien R' \leq R que R' ≥ R
(échanger le rôle de \\\sum
 a_nz^n et
\\sum ~
b_nz^n).

[
[

\end{document}
