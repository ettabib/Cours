\documentclass[]{article}
\usepackage[T1]{fontenc}
\usepackage{lmodern}
\usepackage{amssymb,amsmath}
\usepackage{ifxetex,ifluatex}
\usepackage{fixltx2e} % provides \textsubscript
% use upquote if available, for straight quotes in verbatim environments
\IfFileExists{upquote.sty}{\usepackage{upquote}}{}
\ifnum 0\ifxetex 1\fi\ifluatex 1\fi=0 % if pdftex
  \usepackage[utf8]{inputenc}
\else % if luatex or xelatex
  \ifxetex
    \usepackage{mathspec}
    \usepackage{xltxtra,xunicode}
  \else
    \usepackage{fontspec}
  \fi
  \defaultfontfeatures{Mapping=tex-text,Scale=MatchLowercase}
  \newcommand{\euro}{€}
\fi
% use microtype if available
\IfFileExists{microtype.sty}{\usepackage{microtype}}{}
\ifxetex
  \usepackage[setpagesize=false, % page size defined by xetex
              unicode=false, % unicode breaks when used with xetex
              xetex]{hyperref}
\else
  \usepackage[unicode=true]{hyperref}
\fi
\hypersetup{breaklinks=true,
            bookmarks=true,
            pdfauthor={},
            pdftitle={Formes sesquilineaires},
            colorlinks=true,
            citecolor=blue,
            urlcolor=blue,
            linkcolor=magenta,
            pdfborder={0 0 0}}
\urlstyle{same}  % don't use monospace font for urls
\setlength{\parindent}{0pt}
\setlength{\parskip}{6pt plus 2pt minus 1pt}
\setlength{\emergencystretch}{3em}  % prevent overfull lines
\setcounter{secnumdepth}{0}
 
/* start css.sty */
.cmr-5{font-size:50%;}
.cmr-7{font-size:70%;}
.cmmi-5{font-size:50%;font-style: italic;}
.cmmi-7{font-size:70%;font-style: italic;}
.cmmi-10{font-style: italic;}
.cmsy-5{font-size:50%;}
.cmsy-7{font-size:70%;}
.cmex-7{font-size:70%;}
.cmex-7x-x-71{font-size:49%;}
.msbm-7{font-size:70%;}
.cmtt-10{font-family: monospace;}
.cmti-10{ font-style: italic;}
.cmbx-10{ font-weight: bold;}
.cmr-17x-x-120{font-size:204%;}
.cmsl-10{font-style: oblique;}
.cmti-7x-x-71{font-size:49%; font-style: italic;}
.cmbxti-10{ font-weight: bold; font-style: italic;}
p.noindent { text-indent: 0em }
td p.noindent { text-indent: 0em; margin-top:0em; }
p.nopar { text-indent: 0em; }
p.indent{ text-indent: 1.5em }
@media print {div.crosslinks {visibility:hidden;}}
a img { border-top: 0; border-left: 0; border-right: 0; }
center { margin-top:1em; margin-bottom:1em; }
td center { margin-top:0em; margin-bottom:0em; }
.Canvas { position:relative; }
li p.indent { text-indent: 0em }
.enumerate1 {list-style-type:decimal;}
.enumerate2 {list-style-type:lower-alpha;}
.enumerate3 {list-style-type:lower-roman;}
.enumerate4 {list-style-type:upper-alpha;}
div.newtheorem { margin-bottom: 2em; margin-top: 2em;}
.obeylines-h,.obeylines-v {white-space: nowrap; }
div.obeylines-v p { margin-top:0; margin-bottom:0; }
.overline{ text-decoration:overline; }
.overline img{ border-top: 1px solid black; }
td.displaylines {text-align:center; white-space:nowrap;}
.centerline {text-align:center;}
.rightline {text-align:right;}
div.verbatim {font-family: monospace; white-space: nowrap; text-align:left; clear:both; }
.fbox {padding-left:3.0pt; padding-right:3.0pt; text-indent:0pt; border:solid black 0.4pt; }
div.fbox {display:table}
div.center div.fbox {text-align:center; clear:both; padding-left:3.0pt; padding-right:3.0pt; text-indent:0pt; border:solid black 0.4pt; }
div.minipage{width:100%;}
div.center, div.center div.center {text-align: center; margin-left:1em; margin-right:1em;}
div.center div {text-align: left;}
div.flushright, div.flushright div.flushright {text-align: right;}
div.flushright div {text-align: left;}
div.flushleft {text-align: left;}
.underline{ text-decoration:underline; }
.underline img{ border-bottom: 1px solid black; margin-bottom:1pt; }
.framebox-c, .framebox-l, .framebox-r { padding-left:3.0pt; padding-right:3.0pt; text-indent:0pt; border:solid black 0.4pt; }
.framebox-c {text-align:center;}
.framebox-l {text-align:left;}
.framebox-r {text-align:right;}
span.thank-mark{ vertical-align: super }
span.footnote-mark sup.textsuperscript, span.footnote-mark a sup.textsuperscript{ font-size:80%; }
div.tabular, div.center div.tabular {text-align: center; margin-top:0.5em; margin-bottom:0.5em; }
table.tabular td p{margin-top:0em;}
table.tabular {margin-left: auto; margin-right: auto;}
div.td00{ margin-left:0pt; margin-right:0pt; }
div.td01{ margin-left:0pt; margin-right:5pt; }
div.td10{ margin-left:5pt; margin-right:0pt; }
div.td11{ margin-left:5pt; margin-right:5pt; }
table[rules] {border-left:solid black 0.4pt; border-right:solid black 0.4pt; }
td.td00{ padding-left:0pt; padding-right:0pt; }
td.td01{ padding-left:0pt; padding-right:5pt; }
td.td10{ padding-left:5pt; padding-right:0pt; }
td.td11{ padding-left:5pt; padding-right:5pt; }
table[rules] {border-left:solid black 0.4pt; border-right:solid black 0.4pt; }
.hline hr, .cline hr{ height : 1px; margin:0px; }
.tabbing-right {text-align:right;}
span.TEX {letter-spacing: -0.125em; }
span.TEX span.E{ position:relative;top:0.5ex;left:-0.0417em;}
a span.TEX span.E {text-decoration: none; }
span.LATEX span.A{ position:relative; top:-0.5ex; left:-0.4em; font-size:85%;}
span.LATEX span.TEX{ position:relative; left: -0.4em; }
div.float img, div.float .caption {text-align:center;}
div.figure img, div.figure .caption {text-align:center;}
.marginpar {width:20%; float:right; text-align:left; margin-left:auto; margin-top:0.5em; font-size:85%; text-decoration:underline;}
.marginpar p{margin-top:0.4em; margin-bottom:0.4em;}
.equation td{text-align:center; vertical-align:middle; }
td.eq-no{ width:5%; }
table.equation { width:100%; } 
div.math-display, div.par-math-display{text-align:center;}
math .texttt { font-family: monospace; }
math .textit { font-style: italic; }
math .textsl { font-style: oblique; }
math .textsf { font-family: sans-serif; }
math .textbf { font-weight: bold; }
.partToc a, .partToc, .likepartToc a, .likepartToc {line-height: 200%; font-weight:bold; font-size:110%;}
.chapterToc a, .chapterToc, .likechapterToc a, .likechapterToc, .appendixToc a, .appendixToc {line-height: 200%; font-weight:bold;}
.index-item, .index-subitem, .index-subsubitem {display:block}
.caption td.id{font-weight: bold; white-space: nowrap; }
table.caption {text-align:center;}
h1.partHead{text-align: center}
p.bibitem { text-indent: -2em; margin-left: 2em; margin-top:0.6em; margin-bottom:0.6em; }
p.bibitem-p { text-indent: 0em; margin-left: 2em; margin-top:0.6em; margin-bottom:0.6em; }
.paragraphHead, .likeparagraphHead { margin-top:2em; font-weight: bold;}
.subparagraphHead, .likesubparagraphHead { font-weight: bold;}
.quote {margin-bottom:0.25em; margin-top:0.25em; margin-left:1em; margin-right:1em; text-align:justify;}
.verse{white-space:nowrap; margin-left:2em}
div.maketitle {text-align:center;}
h2.titleHead{text-align:center;}
div.maketitle{ margin-bottom: 2em; }
div.author, div.date {text-align:center;}
div.thanks{text-align:left; margin-left:10%; font-size:85%; font-style:italic; }
div.author{white-space: nowrap;}
.quotation {margin-bottom:0.25em; margin-top:0.25em; margin-left:1em; }
h1.partHead{text-align: center}
.sectionToc, .likesectionToc {margin-left:2em;}
.subsectionToc, .likesubsectionToc {margin-left:4em;}
.subsubsectionToc, .likesubsubsectionToc {margin-left:6em;}
.frenchb-nbsp{font-size:75%;}
.frenchb-thinspace{font-size:75%;}
.figure img.graphics {margin-left:10%;}
/* end css.sty */

\title{Formes sesquilineaires}
\author{}
\date{}

\begin{document}
\maketitle

\textbf{Warning: 
requires JavaScript to process the mathematics on this page.\\ If your
browser supports JavaScript, be sure it is enabled.}

\begin{center}\rule{3in}{0.4pt}\end{center}

[
[
[]
[

\subsubsection{13.2 Formes sesquilinéaires}

\paragraph{13.2.1 Généralités}

Définition~13.2.1 Soit E un \mathbb{C}-espace vectoriel . On appelle forme
sesquilinéaire sur E toute application \phi : E \times E \rightarrow~ \mathbb{C} telle que

\begin{itemize}
\itemsep1pt\parskip0pt\parsep0pt
\item
  (i) \forall~~x \in E,
  y\mapsto~\phi(x,y) est linéaire
\item
  (ii) \forall~~y \in E,
  x\mapsto~\phi(x,y) est semilinéaire
\end{itemize}

Remarque~13.2.1 On a en particulier \forall~~y \in E,
\phi(y,0) = \phi(0,y) = 0~; de plus \phi(x,\lambda~y) = \lambda~\phi(x,y), \phi(\lambda~x,y) =
\overline\lambda~\phi(x,y). Plus généralement
\phi(\\sum ~
\lambda_ix_i,\\\sum
 \mu_jy_j) =\
\sum ~
_i,j\overline\lambda_i\mu_j\phi(x_i,y_j).

Il est clair que si \phi et \psi sont deux formes sesquilinéaires sur E, il en
est de même de \alpha~\phi + \beta~\psi, d'où la proposition

Proposition~13.2.1 L'ensemble L_3\diagup2(E) des formes
sesquilinéaires sur E est un sous-espace vectoriel de l'espace
\mathbb{C}^E\timesE des applications de E \times E dans \mathbb{C}.

Remarque~13.2.2 Soit \phi une forme sesquilinéaire sur E. Pour chaque x \in
E, l'application y\mapsto~\phi(x,y) est une forme
linéaire sur E donc un élément, noté g_\phi(x), du dual
E^∗ de E. De même, pour chaque y \in E, l'application
x\mapsto~\overline\phi(x,y) est une
forme linéaire sur E, donc un élément, noté d_\phi(y), de
E^∗. La relation

\begin{align*} \left
[g_\phi(\alpha~x + \beta~x')\right ](y)& =& \phi(\alpha~x +
\beta~x',y) = \overline\alpha~\phi(x,y) +
\overline\beta~\phi(x',y)\%&
\\ & =& \left
[\overline\alpha~g_\phi(x) +
\overline\beta~g_\phi(x')\right
](y) \%& \\
\end{align*}

montre clairement que g_\phi :
x\mapsto~g_\phi(x) est une application
semilinéaire de E dans E^∗. Il en est de même de d_\phi
: y\mapsto~d_\phi(y).

Définition~13.2.2 L'application g_\phi : E \rightarrow~ E^∗ (resp.
d_\phi) est appelée l'application semilinéaire gauche (resp.
droite) associée à la forme sesquilinéaire \phi.

\paragraph{13.2.2 Formes sesquilinéaires hermitiennes, antihermitiennes}

Définition~13.2.3 Soit \phi \in L_3\diagup2(E). On dit que \phi est
hermitienne (resp. antihermitienne) si \forall~~x,y \in
E, \phi(y,x) = \overline\phi(x,y) (resp. =
-\overline\phi(x,y)).

Proposition~13.2.2 \phi est hermitienne si et seulement si~i\phi est
antihermitienne.

Démonstration Evident

Remarque~13.2.3 Ceci nous permettra par la suite de ne considérer que le
cas des formes hermitiennes.

Proposition~13.2.3 Soit \phi \in L_3\diagup2(E). Alors \phi est hermitienne
si et seulement si~d_\phi = g_\phi.

Démonstration En effet \phi(x,y) =\big
[g_\phi(x)\big ](y) et
\overline\phi(y,x) =\big
[d_\phi(x)\big ](y). Alors

\begin{align*} \forall~~x,y \in E,
\overline\phi(y,x) = \phi(x,y)&& \%&
\\ & \Leftrightarrow &
\forall~~x,y \in E, \big
[g_\phi(x)\big ](y) = \epsilon\big
[d_\phi(x)\big ](y)\%&
\\ & \Leftrightarrow &
\forall~x \in E, g_\phi(x) = d_\phi~(x)
\Leftrightarrow g_\phi = d_\phi \%&
\\ \end{align*}

Proposition~13.2.4 L'ensemble H(E) des formes sesquilinéaires
hermitiennes est un \mathbb{R}~-sous-espace vectoriel de L_3\diagup2(E) (mais
pas un \mathbb{C} sous-espace vectoriel). On a L_3\diagup2(E) = H(E) \oplus~ iH(E).

Démonstration La première affirmation est laissée aux soins du lecteur.
On a clairement H(E) \bigcap iH(E) = \0\ et
la relation \phi = \psi + i\theta avec \psi(x,y) = 1 \over 2
(\phi(x,y) + \phi(y,x)), \theta(x,y) = 1 \over 2i (\phi(x,y) -
\phi(y,x)) qui sont toutes deux hermitiennes montre que L_3\diagup2(E) =
H(E) \oplus~ iH(E).

\paragraph{13.2.3 Matrice d'une forme sesquilinéaire}

Supposons que E est de dimension finie et soit \mathcal{E} =
(e_1,\\ldots,e_n~)
une base de E.

Définition~13.2.4 Soit \phi \in L_3\diagup2(E). On appelle matrice de \phi
dans la base \mathcal{E} la matrice

\mathrmMat~ (\phi,\mathcal{E}) =
(\phi(e_i,e_j))_1\leqi,j\leqn \in M_\mathbb{C}(n)

Proposition~13.2.5
\mathrmMat~ (\phi,\mathcal{E}) est
l'unique matrice \Omega \in M_\mathbb{C}(n) vérifiant

\forall~(x,y) \in E \times E, \phi(x,y) = X^∗~\OmegaY

où X (resp. Y ) désigne le vecteur colonne des coordonnées de x (resp.
y) dans la base \mathcal{E}.

Démonstration Si \Omega = (\omega_i,j), on a

X^∗\OmegaY = \\sum
_i=1^n\overlinex_ i(\OmegaY
)_i = \\sum
_i=1^n\overlinex_ i
\sum _j=1^n\omega_
i,jy_j = \\sum
_i,j\omega_i,j\overlinex_iy_j

Mais d'autre part \phi(x,y) =
\phi(\\sum ~
_i=1^nx_ie_i,\\\sum
 _j=1^ny_je_j)
= \\sum ~
_i,j\phi(e_i,e_j)\overlinex_iy_j
en utilisant la sesquilinéarité de \phi. Ceci montre que
\mathrmMat~ (\phi,\mathcal{E}) vérifie
bien la relation voulue. Inversement, si \Omega vérifie cette formule, on a
\phi(e_k,e_l) = E_k^∗\OmegaE_l
= \\sum ~
_i,j\omega_i,j\delta_i^k\delta_j^l =
\omega_k,l ce qui montre que \Omega =\
\mathrmMat (\phi,\mathcal{E}).

Théorème~13.2.6 L'application
\phi\mapsto~\mathrmMat~
(\phi,\mathcal{E}) est un isomorphisme d'espaces vectoriels de L_3\diagup2(E) sur
M_\mathbb{C}(n).

Démonstration Les détails sont laissés aux soins du lecteur.
L'application réciproque est bien entendu l'application qui à \Omega \in
M_\mathbb{C}(n) associe \phi : E \times E \rightarrow~ \mathbb{C} définie par \phi(x,y) =
X^∗\OmegaY qui est clairement sesquilinéaire.

Théorème~13.2.7 Soit E de dimension finie, \mathcal{E} =
(e_1,\\ldots,e_n~)
une base de E, \mathcal{E}^∗ =
(e_1^∗,\\ldots,e_n^∗~)
la base duale. Soit \phi \in L_3\diagup2(E). Alors

\mathrmMat~ (\phi,\mathcal{E}) =
\overline\mathrmMat~
(d_\phi,\mathcal{E},\mathcal{E}^∗) =
^t \mathrmMat~
(g_ \phi,\mathcal{E},\mathcal{E}^∗)

Démonstration Notons \Omega =\
\mathrmMat (\phi,\mathcal{E}), A =\
\mathrmMat (d_\phi,\mathcal{E},\mathcal{E}^∗) et B
= \mathrmMat~
(g_\phi,\mathcal{E},\mathcal{E}^∗). On a

\begin{align*}
\overline\omega_i,j& =&
\overline\phi(e_i,e_j) =
\left (d_\phi(e_j)\right
)(e_i) \%& \\ & =&
\left (\\sum
_k=1^na_
k,je_k^∗\right )(e_ i) =
a_i,j\%& \\
\end{align*}

compte tenu de e_k^∗(e_i) =
\delta_k^i~; de même

\begin{align*} \omega_i,j& =&
\phi(e_i,e_j) = \left
(g_\phi(e_i)\right )(e_j) \%&
\\ & =& \left
(\sum _k=1^nb_
k,ie_k^∗\right )(e_ j) =
b_j,i\%& \\
\end{align*}

ce qui démontre le résultat.

Corollaire~13.2.8 La forme sesquilinéaire \phi est hermitienne si et
seulement si~sa matrice dans la base \mathcal{E} est hermitienne.

Le rang de \mathrmMat~
(d_\phi,\mathcal{E},\mathcal{E}^∗) est indépendant du choix de la base \mathcal{E}~;
il en est donc de même du rang de
\mathrmMat~ (\phi,\mathcal{E}). Ceci
conduit à la définition suivante

Définition~13.2.5 Soit E de dimension finie et \phi \in L_3\diagup2(E). On
appelle rang de E le rang de sa matrice dans n'importe quelle base de E.
On a

\mathrmrg~\phi
= \mathrmrgd_\phi~
= \mathrmrgg_\phi~
=\
\mathrmrg\mathrmMat~
(\phi,\mathcal{E})

\paragraph{13.2.4 Changements de bases}

Théorème~13.2.9 Soit E un espace vectoriel de dimension finie, \mathcal{E} et \mathcal{E}'
deux bases de E, P = P_\mathcal{E}^\mathcal{E}' la matrice de passage de \mathcal{E} à
\mathcal{E}'. Soit \phi \in L_3\diagup2(E), \Omega =\
\mathrmMat (\phi,\mathcal{E}) et \Omega' =\
\mathrmMat (\phi,\mathcal{E}'). Alors

\Omega' = P^∗\OmegaP

Démonstration Si X (resp. Y ) désigne le vecteur colonne des coordonnées
de x (resp. y) dans la base \mathcal{E} et X' (resp. Y ') désigne le vecteur
colonne des coordonnées de x (resp. y) dans la base \mathcal{E}', on a X = PX', Y
= PY ', d'où

\phi(x,y) = (PX')^∗\Omega(PY `) = X'^∗(P^∗\OmegaP)Y
'

Comme \Omega' est l'unique matrice vérifiant \forall~~(x,y)
\in E \times E, \phi(x,y) = X'^∗\Omega'Y ', on a \Omega' = P^∗\OmegaP.

\paragraph{13.2.5 Orthogonalité}

Soit E un \mathbb{C}-espace vectoriel ~et \phi une forme sesquilinéaire hermitienne
sur E.

Définition~13.2.6 On dit que x est orthogonal à y (relativement à \phi), et
on pose x \bot y, si \phi(x,y) = 0.

Remarque~13.2.4 \phi étant supposée hermitienne, il s'agit visiblement
d'une relation symétrique

Définition~13.2.7 Soit A une partie de E. On pose A^\bot =
\x \in
E∣\forall~~a \in A, \phi(a,x) =
0\

Remarque~13.2.5 Notons A^\bot^∗  l'orthogonal de A
dans le dual E^∗ de E, c'est-à-dire l'espace vectoriel des
formes linéaires sur E qui sont nulles sur A. On a

\begin{align*} x \in A^\bot&
\Leftrightarrow & \forall~~a \in A, \phi(a,x)
= 0 \%& \\ &
\Leftrightarrow & \forall~~a \in A,
\big [d_\phi(x)\big ](a) = 0
\%& \\ & \Leftrightarrow &
d_\phi(x) \in A^\bot^∗ 
\Leftrightarrow x \in
d_\phi^-1(A^\bot^∗ )\%&
\\ \end{align*}

On en déduit que A^\bot =
d_\phi^-1(A^\bot^∗ ) =
g_\phi^-1(A^\bot^∗ ).

Proposition~13.2.10 Soit A une partie de E~; alors

\begin{itemize}
\itemsep1pt\parskip0pt\parsep0pt
\item
  (i)A^\bot est un sous-espace vectoriel de E
\item
  (ii)A^\bot =\
  \mathrmVect(A)^\bot
\item
  (iii) A \subset~ (A^\bot)^\bot
\item
  (iv) A \subset~ B \rigtharrow~ B^\bot\subset~ A^\bot.
\end{itemize}

Démonstration (i) découle immédiatement de la sesquilinéarité de \phi ou de
la remarque précédente. Il en est de même pour (ii) puisqu'un vecteur x
est orthogonal à tout vecteur de A si et seulement si il est orthogonal
à toute combinaison linéaire de vecteurs de A, c'est à dire à
\mathrmVect~(A). En ce qui
concerne (iii), il suffit de remarquer que tout vecteur a de A est
orthogonal à tout vecteur qui est orthogonal à tout vecteur de A. Pour
(iv), un vecteur x qui est orthogonal à tout vecteur de B est évidemment
orthogonal à tout vecteur de A.

\paragraph{13.2.6 Formes non dégénérées}

En règle générale on posera

Définition~13.2.8 Soit E un \mathbb{C}-espace vectoriel , \phi une forme
sesquilinéaire hermitienne sur E. On appelle noyau de \phi le sous-espace

\mathrmKer~\phi =
\x \in
E∣\forall~~y \in E, \phi(x,y) =
0\ = E^\bot =\
\mathrmKerd_ \phi

Définition~13.2.9 Soit E un \mathbb{C}-espace vectoriel , \phi une forme
sesquilinéaire hermitienne sur E. On dit que \phi est non dégénérée si elle
vérifie les conditions équivalentes

\begin{itemize}
\itemsep1pt\parskip0pt\parsep0pt
\item
  (i) \mathrmKer~\phi =
  E^\bot = \0\
\item
  (ii) pour x \in E on a \left
  (\forall~~y \in E, \phi(x,y) = 0\right ) \rigtharrow~
  x = 0
\item
  (iii) d_\phi (resp. g_\phi) est une application
  semilinéaire injective de E dans E^∗.
\end{itemize}

L'équivalence entre ces trois propriétés est évidente.

Si E est un espace vectoriel de dimension finie, on sait que
dim E^∗~ =\
dim E. Si g_\phi est injective, elle est nécessairement
bijective et on obtient

Théorème~13.2.11 Soit E un \mathbb{C}-espace vectoriel ~de dimension finie, \phi une
forme sesquilinéaire hermitienne non dégénérée sur E. Alors
l'application semilinéaire gauche g_\phi est un isomorphisme
d'espace vectoriel de E sur E^∗~; autrement dit, pour toute
forme linéaire f sur E, il existe un unique vecteur v_f \in E tel
que \forall~x \in E, f(x) = \phi(v_f~,x).

Corollaire~13.2.12 Soit E un \mathbb{C}-espace vectoriel ~de dimension finie, \phi
une forme sesquilinéaire hermitienne non dégénérée sur E. Soit A un
sous-espace vectoriel de E. Alors dim~ A
+ dim A^\bot~ =\
dim E et A = A^\bot\bot.

Démonstration On a en effet

dim A^\bot~ =\
dim g_ \phi^-1(A^\bot^∗ )
= dim A^\bot^∗ ~
= dim E -\ dim~ A

puisque g_\phi est un isomorphisme d'espaces vectoriels. On sait
d'autre part que A \subset~ A^\bot\bot et que
dim A^\bot\bot~ =\
dim E - dim A^\bot~
= dim~ A, d'où l'égalité.

Remarque~13.2.6 Il ne faudrait pas en déduire abusivement que A et
A^\bot sont supplémentaires~; en effet, en général A \bigcap
A^\bot\neq~\0\.
Nous nous intéresserons plus particulièrement à ce point dans le
paragraphe suivant.

Si \mathcal{E} est une base de E, alors \Omega =\
\mathrmMat (\phi,\mathcal{E}) =
^t \mathrmMat~
(g_\phi,\mathcal{E},\mathcal{E}^∗) et
\mathrmrg~\phi
= \mathrmrg~\Omega. On en déduit

Théorème~13.2.13 Soit E un \mathbb{C}-espace vectoriel ~de dimension finie n, \phi
une forme sesquilinéaire hermitienne sur E, \mathcal{E} une base de E et \Omega
= \mathrmMat~ (\phi,\mathcal{E}). Alors
les propositions suivantes sont équivalentes

\begin{itemize}
\itemsep1pt\parskip0pt\parsep0pt
\item
  (i) \phi est non dégénérée
\item
  (ii) \Omega est une matrice inversible
\item
  (iii) \mathrmrg~\phi = n.
\end{itemize}

Remarque~13.2.7 En général,
\mathrmKer~\phi
= \mathrmKerg_\phi~,
\mathrmrg~\phi
= \mathrmrgg_\phi~, si
bien que le théorème du rang devient

Proposition~13.2.14 Soit E un \mathbb{C}-espace vectoriel ~de dimension finie n,
\phi une forme sesquilinéaire hermitienne sur E, \mathcal{E} une base de E. Alors
dim~ E =\
\mathrmrg\phi + dim~
\mathrmKer~\phi.

[
[
[
[

\end{document}
