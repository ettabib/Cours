\documentclass[]{article}
\usepackage[T1]{fontenc}
\usepackage{lmodern}
\usepackage{amssymb,amsmath}
\usepackage{ifxetex,ifluatex}
\usepackage{fixltx2e} % provides \textsubscript
% use upquote if available, for straight quotes in verbatim environments
\IfFileExists{upquote.sty}{\usepackage{upquote}}{}
\ifnum 0\ifxetex 1\fi\ifluatex 1\fi=0 % if pdftex
  \usepackage[utf8]{inputenc}
\else % if luatex or xelatex
  \ifxetex
    \usepackage{mathspec}
    \usepackage{xltxtra,xunicode}
  \else
    \usepackage{fontspec}
  \fi
  \defaultfontfeatures{Mapping=tex-text,Scale=MatchLowercase}
  \newcommand{\euro}{€}
\fi
% use microtype if available
\IfFileExists{microtype.sty}{\usepackage{microtype}}{}
\ifxetex
  \usepackage[setpagesize=false, % page size defined by xetex
              unicode=false, % unicode breaks when used with xetex
              xetex]{hyperref}
\else
  \usepackage[unicode=true]{hyperref}
\fi
\hypersetup{breaklinks=true,
            bookmarks=true,
            pdfauthor={},
            pdftitle={Generalites sur les series},
            colorlinks=true,
            citecolor=blue,
            urlcolor=blue,
            linkcolor=magenta,
            pdfborder={0 0 0}}
\urlstyle{same}  % don't use monospace font for urls
\setlength{\parindent}{0pt}
\setlength{\parskip}{6pt plus 2pt minus 1pt}
\setlength{\emergencystretch}{3em}  % prevent overfull lines
\setcounter{secnumdepth}{0}
 
/* start css.sty */
.cmr-5{font-size:50%;}
.cmr-7{font-size:70%;}
.cmmi-5{font-size:50%;font-style: italic;}
.cmmi-7{font-size:70%;font-style: italic;}
.cmmi-10{font-style: italic;}
.cmsy-5{font-size:50%;}
.cmsy-7{font-size:70%;}
.cmex-7{font-size:70%;}
.cmex-7x-x-71{font-size:49%;}
.msbm-7{font-size:70%;}
.cmtt-10{font-family: monospace;}
.cmti-10{ font-style: italic;}
.cmbx-10{ font-weight: bold;}
.cmr-17x-x-120{font-size:204%;}
.cmsl-10{font-style: oblique;}
.cmti-7x-x-71{font-size:49%; font-style: italic;}
.cmbxti-10{ font-weight: bold; font-style: italic;}
p.noindent { text-indent: 0em }
td p.noindent { text-indent: 0em; margin-top:0em; }
p.nopar { text-indent: 0em; }
p.indent{ text-indent: 1.5em }
@media print {div.crosslinks {visibility:hidden;}}
a img { border-top: 0; border-left: 0; border-right: 0; }
center { margin-top:1em; margin-bottom:1em; }
td center { margin-top:0em; margin-bottom:0em; }
.Canvas { position:relative; }
li p.indent { text-indent: 0em }
.enumerate1 {list-style-type:decimal;}
.enumerate2 {list-style-type:lower-alpha;}
.enumerate3 {list-style-type:lower-roman;}
.enumerate4 {list-style-type:upper-alpha;}
div.newtheorem { margin-bottom: 2em; margin-top: 2em;}
.obeylines-h,.obeylines-v {white-space: nowrap; }
div.obeylines-v p { margin-top:0; margin-bottom:0; }
.overline{ text-decoration:overline; }
.overline img{ border-top: 1px solid black; }
td.displaylines {text-align:center; white-space:nowrap;}
.centerline {text-align:center;}
.rightline {text-align:right;}
div.verbatim {font-family: monospace; white-space: nowrap; text-align:left; clear:both; }
.fbox {padding-left:3.0pt; padding-right:3.0pt; text-indent:0pt; border:solid black 0.4pt; }
div.fbox {display:table}
div.center div.fbox {text-align:center; clear:both; padding-left:3.0pt; padding-right:3.0pt; text-indent:0pt; border:solid black 0.4pt; }
div.minipage{width:100%;}
div.center, div.center div.center {text-align: center; margin-left:1em; margin-right:1em;}
div.center div {text-align: left;}
div.flushright, div.flushright div.flushright {text-align: right;}
div.flushright div {text-align: left;}
div.flushleft {text-align: left;}
.underline{ text-decoration:underline; }
.underline img{ border-bottom: 1px solid black; margin-bottom:1pt; }
.framebox-c, .framebox-l, .framebox-r { padding-left:3.0pt; padding-right:3.0pt; text-indent:0pt; border:solid black 0.4pt; }
.framebox-c {text-align:center;}
.framebox-l {text-align:left;}
.framebox-r {text-align:right;}
span.thank-mark{ vertical-align: super }
span.footnote-mark sup.textsuperscript, span.footnote-mark a sup.textsuperscript{ font-size:80%; }
div.tabular, div.center div.tabular {text-align: center; margin-top:0.5em; margin-bottom:0.5em; }
table.tabular td p{margin-top:0em;}
table.tabular {margin-left: auto; margin-right: auto;}
div.td00{ margin-left:0pt; margin-right:0pt; }
div.td01{ margin-left:0pt; margin-right:5pt; }
div.td10{ margin-left:5pt; margin-right:0pt; }
div.td11{ margin-left:5pt; margin-right:5pt; }
table[rules] {border-left:solid black 0.4pt; border-right:solid black 0.4pt; }
td.td00{ padding-left:0pt; padding-right:0pt; }
td.td01{ padding-left:0pt; padding-right:5pt; }
td.td10{ padding-left:5pt; padding-right:0pt; }
td.td11{ padding-left:5pt; padding-right:5pt; }
table[rules] {border-left:solid black 0.4pt; border-right:solid black 0.4pt; }
.hline hr, .cline hr{ height : 1px; margin:0px; }
.tabbing-right {text-align:right;}
span.TEX {letter-spacing: -0.125em; }
span.TEX span.E{ position:relative;top:0.5ex;left:-0.0417em;}
a span.TEX span.E {text-decoration: none; }
span.LATEX span.A{ position:relative; top:-0.5ex; left:-0.4em; font-size:85%;}
span.LATEX span.TEX{ position:relative; left: -0.4em; }
div.float img, div.float .caption {text-align:center;}
div.figure img, div.figure .caption {text-align:center;}
.marginpar {width:20%; float:right; text-align:left; margin-left:auto; margin-top:0.5em; font-size:85%; text-decoration:underline;}
.marginpar p{margin-top:0.4em; margin-bottom:0.4em;}
.equation td{text-align:center; vertical-align:middle; }
td.eq-no{ width:5%; }
table.equation { width:100%; } 
div.math-display, div.par-math-display{text-align:center;}
math .texttt { font-family: monospace; }
math .textit { font-style: italic; }
math .textsl { font-style: oblique; }
math .textsf { font-family: sans-serif; }
math .textbf { font-weight: bold; }
.partToc a, .partToc, .likepartToc a, .likepartToc {line-height: 200%; font-weight:bold; font-size:110%;}
.chapterToc a, .chapterToc, .likechapterToc a, .likechapterToc, .appendixToc a, .appendixToc {line-height: 200%; font-weight:bold;}
.index-item, .index-subitem, .index-subsubitem {display:block}
.caption td.id{font-weight: bold; white-space: nowrap; }
table.caption {text-align:center;}
h1.partHead{text-align: center}
p.bibitem { text-indent: -2em; margin-left: 2em; margin-top:0.6em; margin-bottom:0.6em; }
p.bibitem-p { text-indent: 0em; margin-left: 2em; margin-top:0.6em; margin-bottom:0.6em; }
.paragraphHead, .likeparagraphHead { margin-top:2em; font-weight: bold;}
.subparagraphHead, .likesubparagraphHead { font-weight: bold;}
.quote {margin-bottom:0.25em; margin-top:0.25em; margin-left:1em; margin-right:1em; text-align:justify;}
.verse{white-space:nowrap; margin-left:2em}
div.maketitle {text-align:center;}
h2.titleHead{text-align:center;}
div.maketitle{ margin-bottom: 2em; }
div.author, div.date {text-align:center;}
div.thanks{text-align:left; margin-left:10%; font-size:85%; font-style:italic; }
div.author{white-space: nowrap;}
.quotation {margin-bottom:0.25em; margin-top:0.25em; margin-left:1em; }
h1.partHead{text-align: center}
.sectionToc, .likesectionToc {margin-left:2em;}
.subsectionToc, .likesubsectionToc {margin-left:4em;}
.subsubsectionToc, .likesubsubsectionToc {margin-left:6em;}
.frenchb-nbsp{font-size:75%;}
.frenchb-thinspace{font-size:75%;}
.figure img.graphics {margin-left:10%;}
/* end css.sty */

\title{Generalites sur les series}
\author{}
\date{}

\begin{document}
\maketitle

\textbf{Warning: \href{http://www.math.union.edu/locate/jsMath}{jsMath}
requires JavaScript to process the mathematics on this page.\\ If your
browser supports JavaScript, be sure it is enabled.}

\begin{center}\rule{3in}{0.4pt}\end{center}

{[}\href{coursse37.html}{next}{]} {[}\href{coursse35.html}{prev}{]}
{[}\href{coursse35.html\#tailcoursse35.html}{prev-tail}{]}
{[}\hyperref[tailcoursse36.html]{tail}{]}
{[}\href{coursch8.html\#coursse36.html}{up}{]}

\subsubsection{7.2 Généralités sur les séries}

\paragraph{7.2.1 Notion de série}

Définition~7.2.1 Soit E un espace vectoriel normé~et (\{x\}\_\{n\}) une
suite de E. On appelle sommes partielles de la série
\textbackslash{}mathop\{\textbackslash{}mathop\{∑ \}\} \{x\}\_\{n\} les
\{S\}\_\{n\} =\{\textbackslash{}mathop\{ \textbackslash{}mathop\{∑ \}\}
\}\_\{p=0\}\^{}\{n\}\{x\}\_\{p\} (notée \{S\}\_\{n\}(x) s'il y a risque
de confusion). On dit que la série converge si la suite des sommes
partielles converge dans E~; sa limite est alors appelée la somme de la
série et notée \{\textbackslash{}mathop\{\textbackslash{}mathop\{∑ \}\}
\}\_\{n=0\}\^{}\{+∞\}\{x\}\_\{n\} =\{\textbackslash{}mathop\{
lim\}\}\_\{n→+∞\}\{\textbackslash{}mathop\{\textbackslash{}mathop\{∑
\}\} \}\_\{p=0\}\^{}\{n\}\{x\}\_\{p\}. Une série non convergente est
dite divergente.

Remarque~7.2.1 Soit (\{a\}\_\{n\}) une suite de E. Définissons une suite
(\{x\}\_\{n\}) par \{x\}\_\{0\} = \{a\}\_\{0\} et pour n ≥ 1,
\{x\}\_\{n\} = \{a\}\_\{n\} − \{a\}\_\{n−1\}. On a immédiatement
\{S\}\_\{n\}(x) = \{a\}\_\{n\} et donc la série
\textbackslash{}mathop\{\textbackslash{}mathop\{∑ \}\} \{x\}\_\{n\}
converge si et seulement si~la suite (\{a\}\_\{n\}) converge~; dans ce
cas on a d'ailleurs \textbackslash{}mathop\{lim\}\{a\}\_\{n\}
=\{\textbackslash{}mathop\{ \textbackslash{}mathop\{∑ \}\}
\}\_\{n=0\}\^{}\{+∞\}\{x\}\_\{n\}. Ceci peut permettre dans certains cas
de ramener une étude de convergence de suite à une étude de convergence
de série.

Proposition~7.2.1 Soit E un espace vectoriel normé,
\textbackslash{}mathop\{\textbackslash{}mathop\{∑ \}\} \{x\}\_\{n\} et
\textbackslash{}mathop\{\textbackslash{}mathop\{∑ \}\} \{y\}\_\{n\} deux
séries d'éléments de E. On suppose qu'il existe N ∈ ℕ tel que n ≥ N ⇒
\{x\}\_\{n\} = \{y\}\_\{n\} (autrement dit les deux suites ne diffèrent
que par un nombre fini de termes). Alors les deux séries sont de même
nature (simultanément convergentes ou divergentes).

Démonstration Pour n ≥ N, on a \{S\}\_\{n\}(x) = \{S\}\_\{n\}(y) +
(\{S\}\_\{N\}(x) − \{S\}\_\{N\}(y)) donc l'une des suites \{S\}\_\{n\}
converge si et seulement si~l'autre converge.

Remarque~7.2.2 En faisant tendre n vers + ∞, on obtient S(x) = S(y) +
(\{S\}\_\{N\}(x) − \{S\}\_\{N\}(y)).

Définition~7.2.2 Soit E un espace vectoriel normé,
\textbackslash{}mathop\{\textbackslash{}mathop\{∑ \}\} \{x\}\_\{n\} une
série convergente et p ∈ ℕ. Alors la série
\{\textbackslash{}mathop\{\textbackslash{}mathop\{∑ \}\}
\}\_\{n≥p+1\}\{x\}\_\{n\} est convergente~; sa somme est notée
\{R\}\_\{p\} (ou \{R\}\_\{p\}(x)). On a par définition \{S\}\_\{n\} +
\{R\}\_\{n\} =\{\textbackslash{}mathop\{ \textbackslash{}mathop\{∑ \}\}
\}\_\{p=0\}\^{}\{+∞\}\{x\}\_\{p\} et
\textbackslash{}mathop\{lim\}\{R\}\_\{n\} = 0.

Proposition~7.2.2 Soit E un espace vectoriel normé. Alors l'ensemble des
suites (\{x\}\_\{n\}) telles que la série
\textbackslash{}mathop\{\textbackslash{}mathop\{∑ \}\} \{x\}\_\{n\}
convergent est un sous-espace vectoriel de \{E\}\^{}\{ℕ\}. L'application
\{(\{x\}\_\{n\})\}\_\{n∈ℕ\}\textbackslash{}mathrel\{↦\}\{\textbackslash{}mathop\{\textbackslash{}mathop\{∑
\}\} \}\_\{n=0\}\^{}\{+∞\}\{x\}\_\{n\} est linéaire de ce sous-espace
vectoriel dans E.

Démonstration Il suffit de remarquer que si α et β sont des scalaires,
\{S\}\_\{n\}(αx + βy) = α\{S\}\_\{n\}(x) + β\{S\}\_\{n\}(y).

\paragraph{7.2.2 Terme général, critère de Cauchy}

Théorème~7.2.3 Si la série
\textbackslash{}mathop\{\textbackslash{}mathop\{∑ \}\} \{x\}\_\{n\}
converge, alors la suite (\{x\}\_\{n\}) admet 0 pour limite.

Démonstration \{x\}\_\{n\} = \{S\}\_\{n\} − \{S\}\_\{n−1\} et les deux
suites ont la même limite S =\{\textbackslash{}mathop\{
\textbackslash{}mathop\{∑ \}\} \}\_\{n=0\}\^{}\{+∞\}\{x\}\_\{n\}.

Théorème~7.2.4 (critère de Cauchy pour les séries). Soit E un espace
vectoriel normé~complet et
\textbackslash{}mathop\{\textbackslash{}mathop\{∑ \}\} \{x\}\_\{n\} une
série à termes de E. La série
\textbackslash{}mathop\{\textbackslash{}mathop\{∑ \}\} \{x\}\_\{n\}
converge si et seulement si~elle vérifie

\textbackslash{}mathop\{∀\}ε \textgreater{} 0,
\textbackslash{}mathop\{∃\}N ∈ ℕ, q ≥ p ≥ N
⇒\textbackslash{}\textbar{}\{\textbackslash{}mathop\{∑
\}\}\_\{n=p\}\^{}\{q\}\{x\}\_\{ n\}\textbackslash{}\textbar{}
\textless{} ε

Démonstration C'est simplement le critère de Cauchy pour la suite
(\{S\}\_\{n\}) des sommes partielles puisque
\{\textbackslash{}mathop\{\textbackslash{}mathop\{∑ \}\}
\}\_\{n=p\}\^{}\{q\}\{x\}\_\{n\} = \{S\}\_\{q\} − \{S\}\_\{p−1\}.

Exemple~7.2.1 La série harmonique
\{\textbackslash{}mathop\{\textbackslash{}mathop\{∑ \}\} \}\_\{n≥1\}\{ 1
\textbackslash{}over n\} diverge puisque \{ 1 \textbackslash{}over n+1\}
+ \textbackslash{}mathop\{\textbackslash{}mathop\{\ldots{}\}\} +\{ 1
\textbackslash{}over 2n\} ≥ n ×\{ 1 \textbackslash{}over 2n\} =\{ 1
\textbackslash{}over 2\} . La série ne vérifie donc pas le critère de
Cauchy (bien que \textbackslash{}mathop\{lim\}\{ 1 \textbackslash{}over
n\} = 0), donc elle diverge.

{[}\href{coursse37.html}{next}{]} {[}\href{coursse35.html}{prev}{]}
{[}\href{coursse35.html\#tailcoursse35.html}{prev-tail}{]}
{[}\href{coursse36.html}{front}{]}
{[}\href{coursch8.html\#coursse36.html}{up}{]}

\end{document}
