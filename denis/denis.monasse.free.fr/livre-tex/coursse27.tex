\documentclass[]{article}
\usepackage[T1]{fontenc}
\usepackage{lmodern}
\usepackage{amssymb,amsmath}
\usepackage{ifxetex,ifluatex}
\usepackage{fixltx2e} % provides \textsubscript
% use upquote if available, for straight quotes in verbatim environments
\IfFileExists{upquote.sty}{\usepackage{upquote}}{}
\ifnum 0\ifxetex 1\fi\ifluatex 1\fi=0 % if pdftex
  \usepackage[utf8]{inputenc}
\else % if luatex or xelatex
  \ifxetex
    \usepackage{mathspec}
    \usepackage{xltxtra,xunicode}
  \else
    \usepackage{fontspec}
  \fi
  \defaultfontfeatures{Mapping=tex-text,Scale=MatchLowercase}
  \newcommand{\euro}{€}
\fi
% use microtype if available
\IfFileExists{microtype.sty}{\usepackage{microtype}}{}
\ifxetex
  \usepackage[setpagesize=false, % page size defined by xetex
              unicode=false, % unicode breaks when used with xetex
              xetex]{hyperref}
\else
  \usepackage[unicode=true]{hyperref}
\fi
\hypersetup{breaklinks=true,
            bookmarks=true,
            pdfauthor={},
            pdftitle={Notion d'espace vectoriel norme},
            colorlinks=true,
            citecolor=blue,
            urlcolor=blue,
            linkcolor=magenta,
            pdfborder={0 0 0}}
\urlstyle{same}  % don't use monospace font for urls
\setlength{\parindent}{0pt}
\setlength{\parskip}{6pt plus 2pt minus 1pt}
\setlength{\emergencystretch}{3em}  % prevent overfull lines
\setcounter{secnumdepth}{0}
 
/* start css.sty */
.cmr-5{font-size:50%;}
.cmr-7{font-size:70%;}
.cmmi-5{font-size:50%;font-style: italic;}
.cmmi-7{font-size:70%;font-style: italic;}
.cmmi-10{font-style: italic;}
.cmsy-5{font-size:50%;}
.cmsy-7{font-size:70%;}
.cmex-7{font-size:70%;}
.cmex-7x-x-71{font-size:49%;}
.msbm-7{font-size:70%;}
.cmtt-10{font-family: monospace;}
.cmti-10{ font-style: italic;}
.cmbx-10{ font-weight: bold;}
.cmr-17x-x-120{font-size:204%;}
.cmsl-10{font-style: oblique;}
.cmti-7x-x-71{font-size:49%; font-style: italic;}
.cmbxti-10{ font-weight: bold; font-style: italic;}
p.noindent { text-indent: 0em }
td p.noindent { text-indent: 0em; margin-top:0em; }
p.nopar { text-indent: 0em; }
p.indent{ text-indent: 1.5em }
@media print {div.crosslinks {visibility:hidden;}}
a img { border-top: 0; border-left: 0; border-right: 0; }
center { margin-top:1em; margin-bottom:1em; }
td center { margin-top:0em; margin-bottom:0em; }
.Canvas { position:relative; }
li p.indent { text-indent: 0em }
.enumerate1 {list-style-type:decimal;}
.enumerate2 {list-style-type:lower-alpha;}
.enumerate3 {list-style-type:lower-roman;}
.enumerate4 {list-style-type:upper-alpha;}
div.newtheorem { margin-bottom: 2em; margin-top: 2em;}
.obeylines-h,.obeylines-v {white-space: nowrap; }
div.obeylines-v p { margin-top:0; margin-bottom:0; }
.overline{ text-decoration:overline; }
.overline img{ border-top: 1px solid black; }
td.displaylines {text-align:center; white-space:nowrap;}
.centerline {text-align:center;}
.rightline {text-align:right;}
div.verbatim {font-family: monospace; white-space: nowrap; text-align:left; clear:both; }
.fbox {padding-left:3.0pt; padding-right:3.0pt; text-indent:0pt; border:solid black 0.4pt; }
div.fbox {display:table}
div.center div.fbox {text-align:center; clear:both; padding-left:3.0pt; padding-right:3.0pt; text-indent:0pt; border:solid black 0.4pt; }
div.minipage{width:100%;}
div.center, div.center div.center {text-align: center; margin-left:1em; margin-right:1em;}
div.center div {text-align: left;}
div.flushright, div.flushright div.flushright {text-align: right;}
div.flushright div {text-align: left;}
div.flushleft {text-align: left;}
.underline{ text-decoration:underline; }
.underline img{ border-bottom: 1px solid black; margin-bottom:1pt; }
.framebox-c, .framebox-l, .framebox-r { padding-left:3.0pt; padding-right:3.0pt; text-indent:0pt; border:solid black 0.4pt; }
.framebox-c {text-align:center;}
.framebox-l {text-align:left;}
.framebox-r {text-align:right;}
span.thank-mark{ vertical-align: super }
span.footnote-mark sup.textsuperscript, span.footnote-mark a sup.textsuperscript{ font-size:80%; }
div.tabular, div.center div.tabular {text-align: center; margin-top:0.5em; margin-bottom:0.5em; }
table.tabular td p{margin-top:0em;}
table.tabular {margin-left: auto; margin-right: auto;}
div.td00{ margin-left:0pt; margin-right:0pt; }
div.td01{ margin-left:0pt; margin-right:5pt; }
div.td10{ margin-left:5pt; margin-right:0pt; }
div.td11{ margin-left:5pt; margin-right:5pt; }
table[rules] {border-left:solid black 0.4pt; border-right:solid black 0.4pt; }
td.td00{ padding-left:0pt; padding-right:0pt; }
td.td01{ padding-left:0pt; padding-right:5pt; }
td.td10{ padding-left:5pt; padding-right:0pt; }
td.td11{ padding-left:5pt; padding-right:5pt; }
table[rules] {border-left:solid black 0.4pt; border-right:solid black 0.4pt; }
.hline hr, .cline hr{ height : 1px; margin:0px; }
.tabbing-right {text-align:right;}
span.TEX {letter-spacing: -0.125em; }
span.TEX span.E{ position:relative;top:0.5ex;left:-0.0417em;}
a span.TEX span.E {text-decoration: none; }
span.LATEX span.A{ position:relative; top:-0.5ex; left:-0.4em; font-size:85%;}
span.LATEX span.TEX{ position:relative; left: -0.4em; }
div.float img, div.float .caption {text-align:center;}
div.figure img, div.figure .caption {text-align:center;}
.marginpar {width:20%; float:right; text-align:left; margin-left:auto; margin-top:0.5em; font-size:85%; text-decoration:underline;}
.marginpar p{margin-top:0.4em; margin-bottom:0.4em;}
.equation td{text-align:center; vertical-align:middle; }
td.eq-no{ width:5%; }
table.equation { width:100%; } 
div.math-display, div.par-math-display{text-align:center;}
math .texttt { font-family: monospace; }
math .textit { font-style: italic; }
math .textsl { font-style: oblique; }
math .textsf { font-family: sans-serif; }
math .textbf { font-weight: bold; }
.partToc a, .partToc, .likepartToc a, .likepartToc {line-height: 200%; font-weight:bold; font-size:110%;}
.chapterToc a, .chapterToc, .likechapterToc a, .likechapterToc, .appendixToc a, .appendixToc {line-height: 200%; font-weight:bold;}
.index-item, .index-subitem, .index-subsubitem {display:block}
.caption td.id{font-weight: bold; white-space: nowrap; }
table.caption {text-align:center;}
h1.partHead{text-align: center}
p.bibitem { text-indent: -2em; margin-left: 2em; margin-top:0.6em; margin-bottom:0.6em; }
p.bibitem-p { text-indent: 0em; margin-left: 2em; margin-top:0.6em; margin-bottom:0.6em; }
.paragraphHead, .likeparagraphHead { margin-top:2em; font-weight: bold;}
.subparagraphHead, .likesubparagraphHead { font-weight: bold;}
.quote {margin-bottom:0.25em; margin-top:0.25em; margin-left:1em; margin-right:1em; text-align:justify;}
.verse{white-space:nowrap; margin-left:2em}
div.maketitle {text-align:center;}
h2.titleHead{text-align:center;}
div.maketitle{ margin-bottom: 2em; }
div.author, div.date {text-align:center;}
div.thanks{text-align:left; margin-left:10%; font-size:85%; font-style:italic; }
div.author{white-space: nowrap;}
.quotation {margin-bottom:0.25em; margin-top:0.25em; margin-left:1em; }
h1.partHead{text-align: center}
.sectionToc, .likesectionToc {margin-left:2em;}
.subsectionToc, .likesubsectionToc {margin-left:4em;}
.subsubsectionToc, .likesubsubsectionToc {margin-left:6em;}
.frenchb-nbsp{font-size:75%;}
.frenchb-thinspace{font-size:75%;}
.figure img.graphics {margin-left:10%;}
/* end css.sty */

\title{Notion d'espace vectoriel norme}
\author{}
\date{}

\begin{document}
\maketitle

\textbf{Warning: \href{http://www.math.union.edu/locate/jsMath}{jsMath}
requires JavaScript to process the mathematics on this page.\\ If your
browser supports JavaScript, be sure it is enabled.}

\begin{center}\rule{3in}{0.4pt}\end{center}

{[}\href{coursse28.html}{next}{]}
{[}\hyperref[tailcoursse27.html]{tail}{]}
{[}\href{coursch6.html\#coursse27.html}{up}{]}

\subsubsection{5.1 Notion d'espace vectoriel normé}

\paragraph{5.1.1 Norme et distance associée}

Définition~5.1.1 Soit E un K-espace vectoriel . On appelle norme sur E
toute application
x\textbackslash{}mathrel\{↦\}\textbackslash{}\textbar{}x\textbackslash{}\textbar{}
de E dans \{ℝ\}\^{}\{+\} vérifiant

\begin{itemize}
\itemsep1pt\parskip0pt\parsep0pt
\item
  (i) \textbackslash{}\textbar{}x\textbackslash{}\textbar{} = 0
  \textbackslash{}mathrel\{⇔\} x = 0 (séparation)
\item
  (ii) \textbackslash{}\textbar{}λx\textbackslash{}\textbar{} =
  \textbar{}λ\textbar{}\textbackslash{}\textbar{}x\textbackslash{}\textbar{}
  (homogénéité)
\item
  (iii) \textbackslash{}\textbar{}x + y\textbackslash{}\textbar{}
  ≤\textbackslash{}\textbar{} x\textbackslash{}\textbar{}
  +\textbackslash{}\textbar{} y\textbackslash{}\textbar{} (inégalité
  triangulaire)
\end{itemize}

On appelle espace vectoriel normé un couple
(E,\textbackslash{}\textbar{}.\textbackslash{}\textbar{}) d'un K-espace
vectoriel et d'une norme sur E.

Exemple~5.1.1 Sur \{K\}\^{}\{n\}, on définit trois normes usuelles,
\textbackslash{}\textbar{}\{x\textbackslash{}\textbar{}\}\_\{1\}
=\textbackslash{}mathop\{ \textbackslash{}mathop\{∑ \}\}
\textbar{}\{x\}\_\{i\}\textbar{},
\textbackslash{}\textbar{}\{x\textbackslash{}\textbar{}\}\_\{2\} =
\textbackslash{}sqrt\{\textbackslash{}mathop\{\textbackslash{}mathop\{∑
\}\} \textbar{}\{x\}\_\{i\}\{\textbar{}\}\^{}\{2\}\},
\textbackslash{}\textbar{}\{x\textbackslash{}\textbar{}\}\_\{∞\}
=\textbackslash{}mathop\{ sup\}\textbar{}\{x\}\_\{i\}\textbar{}. De la
même fa\textbackslash{}c\{c\}on, on définit sur l'espace vectoriel des
fonctions continues de {[}0,1{]} dans K, C({[}0,1{]},K), trois normes
usuelles,
\textbackslash{}\textbar{}\{f\textbackslash{}\textbar{}\}\_\{1\}
=\{\textbackslash{}mathop\{∫ \}
\}\_\{0\}\^{}\{1\}\textbar{}f(t)\textbar{} dt,
\textbackslash{}\textbar{}\{f\textbackslash{}\textbar{}\}\_\{2\} =
\textbackslash{}sqrt\{\{\textbackslash{}mathop\{∫ \}
\}\_\{0\}\^{}\{1\}\textbar{}f(t)\{\textbar{}\}\^{}\{2\} dt\},
\textbackslash{}\textbar{}\{f\textbackslash{}\textbar{}\}\_\{∞\}
=\{\textbackslash{}mathop\{
sup\}\}\_\{x∈{[}0,1{]}\}\textbar{}f(x)\textbar{}.

Proposition~5.1.1 Soit E un K-espace vectoriel normé. L'application d :
E × E → \{ℝ\}\^{}\{+\} définie par d(x,y) =\textbackslash{}\textbar{} x
− y\textbackslash{}\textbar{} est une distance sur E appelée distance
associée à la norme. La topologie associée à cette distance est appelée
topologie définie par la norme.

Remarque~5.1.1 Si E est un espace vectoriel normé, on dispose de deux
familles importantes d'homéomorphismes de E sur lui même~: les
translations \{t\}\_\{v\} : x\textbackslash{}mathrel\{↦\}x + v et les
homothéties \{h\}\_\{λ\} : x\textbackslash{}mathrel\{↦\}λx
(λ\textbackslash{}mathrel\{≠\}0). On constate que tous les points ont
les mêmes propriétés topologiques et que deux boules ouvertes sont
toujours homéomorphes.

Définition~5.1.2 On appelle espace de Banach un espace vectoriel normé
complet (pour la distance associée).

Définition~5.1.3 Soit
(\{E\}\_\{1\},\textbackslash{}\textbar{}\{.\textbackslash{}\textbar{}\}\_\{1\}),\textbackslash{}mathop\{\textbackslash{}mathop\{\ldots{}\}\},(\{E\}\_\{k\},\textbackslash{}\textbar{}\{.\textbackslash{}\textbar{}\}\_\{k\})
des espaces vectoriels normés. On définit une norme sur le produit E =
\{E\}\_\{1\} ×\textbackslash{}mathrel\{⋯\} × \{E\}\_\{k\} en posant
\textbackslash{}\textbar{}x\textbackslash{}\textbar{}
=\textbackslash{}mathop\{
max\}\textbackslash{}\textbar{}\{x\{\}\_\{i\}\textbackslash{}\textbar{}\}\_\{i\}.
L'espace vectoriel
normé~(E,\textbackslash{}\textbar{}.\textbackslash{}\textbar{}) est
appelé l'espace vectoriel normé produit. Il est complet si chacun des
\{E\}\_\{i\} est complet.

\paragraph{5.1.2 Convexes, connexes}

Définition~5.1.4 Soit E un espace vectoriel normé, a,b ∈ E. On pose
{[}a,b{]} = \textbackslash{}\{ta + (1 − t)b\textbackslash{}mathrel\{∣\}t
∈ {[}0,1{]}\textbackslash{}\}. On dit qu'une partie A de E est convexe
si \textbackslash{}mathop\{∀\}a,b ∈ A,\textbackslash{}quad {[}a,b{]} ⊂
A.

Remarque~5.1.2 Le théorème d'associativité des barycentres montre
immédiatement par récurrence que si A est convexe,
\{a\}\_\{1\},\textbackslash{}mathop\{\textbackslash{}mathop\{\ldots{}\}\},\{a\}\_\{n\}
∈ A et
\{λ\}\_\{1\},\textbackslash{}mathop\{\textbackslash{}mathop\{\ldots{}\}\},\{λ\}\_\{n\}
sont des réels positifs de somme 1, alors \{λ\}\_\{1\}\{a\}\_\{1\} +
\textbackslash{}mathop\{\textbackslash{}mathop\{\ldots{}\}\} +
\{λ\}\_\{n\}\{a\}\_\{n\} est encore dans A.

Proposition~5.1.2 Toute partie convexe est connexe par arcs (et donc
connexe).

Démonstration γ(t) = (1 − t)a + tb est un chemin continu (l'application
est \textbackslash{}\textbar{}b −
a\textbackslash{}\textbar{}-lipschitzienne) d'origine a et d'extrémité
b.

Proposition~5.1.3 Dans un espace vectoriel normé, les boules sont
convexes (et donc connexes).

Démonstration Montrons le par exemple pour une boule ouverte B(a,r).
Soit x,y ∈ B(a,r) et t ∈ {[}0,1{]}. On a alors

\textbackslash{}begin\{eqnarray*\} \textbackslash{}\textbar{}tx + (1 −
t)y − a\textbackslash{}\textbar{}\& =\& \textbackslash{}\textbar{}t(x −
a) + (1 − t)(y − a)\textbackslash{}\textbar{}\%\&
\textbackslash{}\textbackslash{} \& ≤\& t\textbackslash{}\textbar{}x −
a\textbackslash{}\textbar{} + (1 − t)\textbackslash{}\textbar{}y −
a\textbackslash{}\textbar{} \%\& \textbackslash{}\textbackslash{} \&
\textless{}\& tr + (1 − t)r = r \%\& \textbackslash{}\textbackslash{}
\textbackslash{}end\{eqnarray*\}

car t ≥ 0,1 − t ≥ 0 et soit t, soit 1 − t est non nul.

Théorème~5.1.4 Dans un espace vectoriel normé, tout ouvert connexe est
connexe par arcs.

Démonstration Soit U un ouvert connexe. Soit ℛ la relation d'équivalence
sur U~: aℛb s'il existe un chemin d'origine a et d'extrémité b. Soit
C(a) la classe d'équivalence de a et montrons que C(a) est ouverte. Pour
cela soit b ∈ C(a) ⊂ U. Il existe r \textgreater{} 0 tel que B(b,r) ⊂ U.
Mais la boule, étant convexe, est connexe par arcs et donc pour tout x
de B(b,r) on a x ∈ C(b) = C(a), soit B(b,r) ⊂ C(a). Les classes
d'équivalences sont donc ouvertes dans U. Mais on a alors
\{\}\^{}\{c\}C(a) =\{\textbackslash{}mathop\{ \textbackslash{}mathop\{⋃
\}\} \}\_\{x\textbackslash{}mathrel\{∉\}C(a)\}C(x) est ouvert dans U et
donc C(a) est fermé dans U. Comme U est connexe, les seules parties
ouvertes et fermées dans U sont ∅ et U, soit C(a) = U et donc U est
connexe par arcs.

\paragraph{5.1.3 Continuité des opérations algébriques}

Théorème~5.1.5 Soit E un espace vectoriel normé. L'application s : E × E
→ E, (x,y)\textbackslash{}mathrel\{↦\}x + y est uniformément continue,
et l'application p : K × E → E, (λ,x)\textbackslash{}mathrel\{↦\}λx est
continue.

Démonstration On a \textbackslash{}\textbar{}s(x,y)
−s(x',y')\textbackslash{}\textbar{} =\textbackslash{}\textbar{} (x−x') +
(y −y')\textbackslash{}\textbar{} ≤\textbackslash{}\textbar{}
x−x'\textbackslash{}\textbar{} +\textbackslash{}\textbar{} y
−y'\textbackslash{}\textbar{} ≤
2\textbackslash{}mathop\{max\}(\textbackslash{}\textbar{}x−x'\textbackslash{}\textbar{},\textbackslash{}\textbar{}y
−y'\textbackslash{}\textbar{}) = 2\textbackslash{}\textbar{}(x,y) −
(x',y')\textbackslash{}\textbar{}, ce qui montre que s est
2-lipschitzienne.

Soit (\{λ\}\_\{0\},\{x\}\_\{0\}) ∈ K × E, λ ∈ K et x ∈ E. On a

\textbackslash{}begin\{eqnarray*\} \textbackslash{}\textbar{}p(λ,x) −
p(\{λ\}\_\{0\},\{x\}\_\{0\})\textbackslash{}\textbar{}\& =\&
\textbackslash{}\textbar{}λx −
\{λ\}\_\{0\}\{x\}\_\{0\}\textbackslash{}\textbar{} \%\&
\textbackslash{}\textbackslash{} \& =\& \textbackslash{}\textbar{}λ(x −
\{x\}\_\{0\}) + (λ −
\{λ\}\_\{0\})\{x\}\_\{0\}\textbackslash{}\textbar{}\%\&
\textbackslash{}\textbackslash{} \& ≤\&
\textbar{}λ\textbar{}\textbackslash{},\textbackslash{}\textbar{}x −
\{x\}\_\{0\}\textbackslash{}\textbar{} + \textbar{}λ −
\{λ\}\_\{0\}\textbar{}\textbackslash{},\textbackslash{}\textbar{}\{x\}\_\{0\}\textbackslash{}\textbar{}
\%\& \textbackslash{}\textbackslash{} \textbackslash{}end\{eqnarray*\}

Pour η ≤ 1, on a \textbar{}λ − \{λ\}\_\{0\}\textbar{} \textless{} η
⇒\textbar{}λ\textbar{}≤\textbar{}\{λ\}\_\{0\}\textbar{} + 1. Soit donc ε
\textgreater{} 0 et η =\textbackslash{}mathop\{ max\}(1,\{ ε
\textbackslash{}over 2(1+\textbar{}\{λ\}\_\{0\}\textbar{})\} ,\{ ε
\textbackslash{}over
2(1+\textbackslash{}\textbar{}\{x\}\_\{0\}\textbackslash{}\textbar{})\}
). Alors

\textbackslash{}begin\{eqnarray*\} \textbackslash{}\textbar{}(λ,x) −
(\{λ\}\_\{0\},\{x\}\_\{0\})\textbackslash{}\textbar{}
=\textbackslash{}mathop\{ max\}(\textbar{}λ −
\{λ\}\_\{0\}\textbar{},\textbackslash{}\textbar{}x −
\{x\}\_\{0\}\textbackslash{}\textbar{}) \textless{} η\&\&\%\&
\textbackslash{}\textbackslash{} \& ⇒\& \textbackslash{}\textbar{}p(λ,x)
− p(\{λ\}\_\{0\},\{x\}\_\{0\})\textbackslash{}\textbar{} ≤\{ ε
\textbackslash{}over 2\} +\{ ε \textbackslash{}over 2\} = ε\%\&
\textbackslash{}\textbackslash{} \textbackslash{}end\{eqnarray*\}

ce qui montre la continuité de p au point (\{λ\}\_\{0\},\{x\}\_\{0\}).

Corollaire~5.1.6 Soit X un espace métrique, E un espace vectoriel normé,
A une partie de X et a ∈\textbackslash{}overline\{A\}. (i) Si f et g
sont deux fonctions de X vers E telles que A ⊂\textbackslash{}mathop\{
Def\} (f) ∩\textbackslash{}mathop\{ Def\} (g), si f et g ont toutes deux
des limites en a suivant A et si α,β ∈ K, alors αf + βg a une limite en
a suivant A et on a

\{\textbackslash{}mathop\{lim\}\}\_\{x→a,x∈A\}(αf(x) + βg(x)) =
α\{\textbackslash{}mathop\{lim\}\}\_\{x→a,x∈A\}f(x) +
β\{\textbackslash{}mathop\{lim\}\}\_\{x→a,x∈A\}g(x)

(ii) Si φ est une fonction de X vers K et f une fonction de X vers E
telles que A ⊂\textbackslash{}mathop\{ Def\} (f)
∩\textbackslash{}mathop\{ Def\} (φ), si f et φ ont toutes deux des
limites en a suivant A, alors φf a une limite en a suivant A et on a

\{\textbackslash{}mathop\{lim\}\}\_\{x→a,x∈A\}(φ(x)f(x))
=\{\textbackslash{}mathop\{
lim\}\}\_\{x→a,x∈A\}φ(x)\{\textbackslash{}mathop\{lim\}\}\_\{x→a,x∈A\}f(x)

Remarque~5.1.3 Dans le cas de E = ℝ, les opérations algébriques sur ℝ ×
ℝ ne peuvent pas s'étendre de manière continue à
\textbackslash{}overline\{ℝ\} ×\textbackslash{}overline\{ℝ\}, ce qui
fait que certaines opérations sur les limites ne sont pas valides en
général. On a cependant le théorème suivant qui permet d'étendre les
opérations sur les limites sauf dans les cas d'indéterminations ''∞−∞''
et ''0 ×∞''

Théorème~5.1.7 (i) L'application s : ℝ × ℝ → ℝ,
(x,y)\textbackslash{}mathrel\{↦\}x + y s'étend en une application
continue de \textbackslash{}overline\{ℝ\} ×\textbackslash{}overline\{ℝ\}
∖\textbackslash{}\{(−∞,+∞),(+∞,−∞)\textbackslash{}\} dans
\textbackslash{}overline\{ℝ\} en posant x + (+∞) = +∞ si
x\textbackslash{}mathrel\{≠\} −∞ et x + (−∞) = −∞ si
x\textbackslash{}mathrel\{≠\} + ∞. (ii) L'application p : ℝ × ℝ → ℝ,
(x,y)\textbackslash{}mathrel\{↦\}xy s'étend en une application continue
de

\textbackslash{}overline\{ℝ\} ×\textbackslash{}overline\{ℝ\}
∖\textbackslash{}\{(0,+∞),(+∞,0),(0,−∞),(−∞,0)\textbackslash{}\}

dans \textbackslash{}overline\{ℝ\} en posant x × (+∞)
=\textbackslash{}mathop\{ sgn\}(x)∞ si x\textbackslash{}mathrel\{≠\}0 et
x × (−∞) = −\textbackslash{}mathop\{sgn\}(x)∞ si
x\textbackslash{}mathrel\{≠\}0.

Démonstration La vérification de la continuité est tout à fait
élémentaire. Remarquons que puisque ℝ × ℝ est dense dans
\textbackslash{}overline\{ℝ\} ×\textbackslash{}overline\{ℝ\}, ces
prolongements sont uniques.

{[}\href{coursse28.html}{next}{]} {[}\href{coursse27.html}{front}{]}
{[}\href{coursch6.html\#coursse27.html}{up}{]}

\end{document}
