\documentclass[]{article}
\usepackage[T1]{fontenc}
\usepackage{lmodern}
\usepackage{amssymb,amsmath}
\usepackage{ifxetex,ifluatex}
\usepackage{fixltx2e} % provides \textsubscript
% use upquote if available, for straight quotes in verbatim environments
\IfFileExists{upquote.sty}{\usepackage{upquote}}{}
\ifnum 0\ifxetex 1\fi\ifluatex 1\fi=0 % if pdftex
  \usepackage[utf8]{inputenc}
\else % if luatex or xelatex
  \ifxetex
    \usepackage{mathspec}
    \usepackage{xltxtra,xunicode}
  \else
    \usepackage{fontspec}
  \fi
  \defaultfontfeatures{Mapping=tex-text,Scale=MatchLowercase}
  \newcommand{\euro}{€}
\fi
% use microtype if available
\IfFileExists{microtype.sty}{\usepackage{microtype}}{}
\ifxetex
  \usepackage[setpagesize=false, % page size defined by xetex
              unicode=false, % unicode breaks when used with xetex
              xetex]{hyperref}
\else
  \usepackage[unicode=true]{hyperref}
\fi
\hypersetup{breaklinks=true,
            bookmarks=true,
            pdfauthor={},
            pdftitle={Notion d'espace vectoriel norme},
            colorlinks=true,
            citecolor=blue,
            urlcolor=blue,
            linkcolor=magenta,
            pdfborder={0 0 0}}
\urlstyle{same}  % don't use monospace font for urls
\setlength{\parindent}{0pt}
\setlength{\parskip}{6pt plus 2pt minus 1pt}
\setlength{\emergencystretch}{3em}  % prevent overfull lines
\setcounter{secnumdepth}{0}
 
/* start css.sty */
.cmr-5{font-size:50%;}
.cmr-7{font-size:70%;}
.cmmi-5{font-size:50%;font-style: italic;}
.cmmi-7{font-size:70%;font-style: italic;}
.cmmi-10{font-style: italic;}
.cmsy-5{font-size:50%;}
.cmsy-7{font-size:70%;}
.cmex-7{font-size:70%;}
.cmex-7x-x-71{font-size:49%;}
.msbm-7{font-size:70%;}
.cmtt-10{font-family: monospace;}
.cmti-10{ font-style: italic;}
.cmbx-10{ font-weight: bold;}
.cmr-17x-x-120{font-size:204%;}
.cmsl-10{font-style: oblique;}
.cmti-7x-x-71{font-size:49%; font-style: italic;}
.cmbxti-10{ font-weight: bold; font-style: italic;}
p.noindent { text-indent: 0em }
td p.noindent { text-indent: 0em; margin-top:0em; }
p.nopar { text-indent: 0em; }
p.indent{ text-indent: 1.5em }
@media print {div.crosslinks {visibility:hidden;}}
a img { border-top: 0; border-left: 0; border-right: 0; }
center { margin-top:1em; margin-bottom:1em; }
td center { margin-top:0em; margin-bottom:0em; }
.Canvas { position:relative; }
li p.indent { text-indent: 0em }
.enumerate1 {list-style-type:decimal;}
.enumerate2 {list-style-type:lower-alpha;}
.enumerate3 {list-style-type:lower-roman;}
.enumerate4 {list-style-type:upper-alpha;}
div.newtheorem { margin-bottom: 2em; margin-top: 2em;}
.obeylines-h,.obeylines-v {white-space: nowrap; }
div.obeylines-v p { margin-top:0; margin-bottom:0; }
.overline{ text-decoration:overline; }
.overline img{ border-top: 1px solid black; }
td.displaylines {text-align:center; white-space:nowrap;}
.centerline {text-align:center;}
.rightline {text-align:right;}
div.verbatim {font-family: monospace; white-space: nowrap; text-align:left; clear:both; }
.fbox {padding-left:3.0pt; padding-right:3.0pt; text-indent:0pt; border:solid black 0.4pt; }
div.fbox {display:table}
div.center div.fbox {text-align:center; clear:both; padding-left:3.0pt; padding-right:3.0pt; text-indent:0pt; border:solid black 0.4pt; }
div.minipage{width:100%;}
div.center, div.center div.center {text-align: center; margin-left:1em; margin-right:1em;}
div.center div {text-align: left;}
div.flushright, div.flushright div.flushright {text-align: right;}
div.flushright div {text-align: left;}
div.flushleft {text-align: left;}
.underline{ text-decoration:underline; }
.underline img{ border-bottom: 1px solid black; margin-bottom:1pt; }
.framebox-c, .framebox-l, .framebox-r { padding-left:3.0pt; padding-right:3.0pt; text-indent:0pt; border:solid black 0.4pt; }
.framebox-c {text-align:center;}
.framebox-l {text-align:left;}
.framebox-r {text-align:right;}
span.thank-mark{ vertical-align: super }
span.footnote-mark sup.textsuperscript, span.footnote-mark a sup.textsuperscript{ font-size:80%; }
div.tabular, div.center div.tabular {text-align: center; margin-top:0.5em; margin-bottom:0.5em; }
table.tabular td p{margin-top:0em;}
table.tabular {margin-left: auto; margin-right: auto;}
div.td00{ margin-left:0pt; margin-right:0pt; }
div.td01{ margin-left:0pt; margin-right:5pt; }
div.td10{ margin-left:5pt; margin-right:0pt; }
div.td11{ margin-left:5pt; margin-right:5pt; }
table[rules] {border-left:solid black 0.4pt; border-right:solid black 0.4pt; }
td.td00{ padding-left:0pt; padding-right:0pt; }
td.td01{ padding-left:0pt; padding-right:5pt; }
td.td10{ padding-left:5pt; padding-right:0pt; }
td.td11{ padding-left:5pt; padding-right:5pt; }
table[rules] {border-left:solid black 0.4pt; border-right:solid black 0.4pt; }
.hline hr, .cline hr{ height : 1px; margin:0px; }
.tabbing-right {text-align:right;}
span.TEX {letter-spacing: -0.125em; }
span.TEX span.E{ position:relative;top:0.5ex;left:-0.0417em;}
a span.TEX span.E {text-decoration: none; }
span.LATEX span.A{ position:relative; top:-0.5ex; left:-0.4em; font-size:85%;}
span.LATEX span.TEX{ position:relative; left: -0.4em; }
div.float img, div.float .caption {text-align:center;}
div.figure img, div.figure .caption {text-align:center;}
.marginpar {width:20%; float:right; text-align:left; margin-left:auto; margin-top:0.5em; font-size:85%; text-decoration:underline;}
.marginpar p{margin-top:0.4em; margin-bottom:0.4em;}
.equation td{text-align:center; vertical-align:middle; }
td.eq-no{ width:5%; }
table.equation { width:100%; } 
div.math-display, div.par-math-display{text-align:center;}
math .texttt { font-family: monospace; }
math .textit { font-style: italic; }
math .textsl { font-style: oblique; }
math .textsf { font-family: sans-serif; }
math .textbf { font-weight: bold; }
.partToc a, .partToc, .likepartToc a, .likepartToc {line-height: 200%; font-weight:bold; font-size:110%;}
.chapterToc a, .chapterToc, .likechapterToc a, .likechapterToc, .appendixToc a, .appendixToc {line-height: 200%; font-weight:bold;}
.index-item, .index-subitem, .index-subsubitem {display:block}
.caption td.id{font-weight: bold; white-space: nowrap; }
table.caption {text-align:center;}
h1.partHead{text-align: center}
p.bibitem { text-indent: -2em; margin-left: 2em; margin-top:0.6em; margin-bottom:0.6em; }
p.bibitem-p { text-indent: 0em; margin-left: 2em; margin-top:0.6em; margin-bottom:0.6em; }
.paragraphHead, .likeparagraphHead { margin-top:2em; font-weight: bold;}
.subparagraphHead, .likesubparagraphHead { font-weight: bold;}
.quote {margin-bottom:0.25em; margin-top:0.25em; margin-left:1em; margin-right:1em; text-align:justify;}
.verse{white-space:nowrap; margin-left:2em}
div.maketitle {text-align:center;}
h2.titleHead{text-align:center;}
div.maketitle{ margin-bottom: 2em; }
div.author, div.date {text-align:center;}
div.thanks{text-align:left; margin-left:10%; font-size:85%; font-style:italic; }
div.author{white-space: nowrap;}
.quotation {margin-bottom:0.25em; margin-top:0.25em; margin-left:1em; }
h1.partHead{text-align: center}
.sectionToc, .likesectionToc {margin-left:2em;}
.subsectionToc, .likesubsectionToc {margin-left:4em;}
.subsubsectionToc, .likesubsubsectionToc {margin-left:6em;}
.frenchb-nbsp{font-size:75%;}
.frenchb-thinspace{font-size:75%;}
.figure img.graphics {margin-left:10%;}
/* end css.sty */

\title{Notion d'espace vectoriel norme}
\author{}
\date{}

\begin{document}
\maketitle

\textbf{Warning: 
requires JavaScript to process the mathematics on this page.\\ If your
browser supports JavaScript, be sure it is enabled.}

\begin{center}\rule{3in}{0.4pt}\end{center}

[
[]
[

\subsubsection{5.1 Notion d'espace vectoriel normé}

\paragraph{5.1.1 Norme et distance associée}

Définition~5.1.1 Soit E un K-espace vectoriel . On appelle norme sur E
toute application
x\mapsto~\x\
de E dans \mathbb{R}~^+ vérifiant

\begin{itemize}
\itemsep1pt\parskip0pt\parsep0pt
\item
  (i) \x\ = 0
  \Leftrightarrow x = 0 (séparation)
\item
  (ii) \\lambda~x\ =
  \lambda~\x\
  (homogénéité)
\item
  (iii) \x + y\
  \leq\ x\
  +\ y\ (inégalité
  triangulaire)
\end{itemize}

On appelle espace vectoriel normé un couple
(E,\.\) d'un K-espace
vectoriel et d'une norme sur E.

Exemple~5.1.1 Sur K^n, on définit trois normes usuelles,
\x_1
= \\sum ~
x_i,
\x_2 =
\sqrt\\\sum
 x_i^2,
\x_\infty~
= supx_i~. De la
même fa\ccon, on définit sur l'espace vectoriel des
fonctions continues de [0,1] dans K, C([0,1],K), trois normes
usuelles,
\f_1
=\int ~
_0^1f(t) dt,
\f_2 =
\sqrt\int ~
_0^1f(t)^2 dt,
\f_\infty~
=\
sup_x\in[0,1]f(x).

Proposition~5.1.1 Soit E un K-espace vectoriel normé. L'application d :
E \times E \rightarrow~ \mathbb{R}~^+ définie par d(x,y) =\ x
- y\ est une distance sur E appelée distance
associée à la norme. La topologie associée à cette distance est appelée
topologie définie par la norme.

Remarque~5.1.1 Si E est un espace vectoriel normé, on dispose de deux
familles importantes d'homéomorphismes de E sur lui même~: les
translations t_v : x\mapsto~x + v et les
homothéties h_\lambda~ : x\mapsto~\lambda~x
(\lambda~\neq~0). On constate que tous les points ont
les mêmes propriétés topologiques et que deux boules ouvertes sont
toujours homéomorphes.

Définition~5.1.2 On appelle espace de Banach un espace vectoriel normé
complet (pour la distance associée).

Définition~5.1.3 Soit
(E_1,\._1),\\ldots,(E_k,\._k~)
des espaces vectoriels normés. On définit une norme sur le produit E =
E_1 \times⋯ \times E_k en posant
\x\
=\
max\x_i_i.
L'espace vectoriel
normé~(E,\.\) est
appelé l'espace vectoriel normé produit. Il est complet si chacun des
E_i est complet.

\paragraph{5.1.2 Convexes, connexes}

Définition~5.1.4 Soit E un espace vectoriel normé, a,b \in E. On pose
[a,b] = \ta + (1 - t)b∣t
\in [0,1]\. On dit qu'une partie A de E est convexe
si \forall~~a,b \in A,\quad [a,b] \subset~
A.

Remarque~5.1.2 Le théorème d'associativité des barycentres montre
immédiatement par récurrence que si A est convexe,
a_1,\\ldots,a_n~
\in A et
\lambda_1,\\ldots,\lambda_n~
sont des réels positifs de somme 1, alors \lambda_1a_1 +
\\ldots~ +
\lambda_na_n est encore dans A.

Proposition~5.1.2 Toute partie convexe est connexe par arcs (et donc
connexe).

Démonstration \gamma(t) = (1 - t)a + tb est un chemin continu (l'application
est \b -
a\-lipschitzienne) d'origine a et d'extrémité
b.

Proposition~5.1.3 Dans un espace vectoriel normé, les boules sont
convexes (et donc connexes).

Démonstration Montrons le par exemple pour une boule ouverte B(a,r).
Soit x,y \in B(a,r) et t \in [0,1]. On a alors

\begin{align*} \tx + (1 -
t)y - a& =& \t(x -
a) + (1 - t)(y - a)\\%&
\\ & \leq& t\x -
a\ + (1 - t)\y -
a\ \%& \\ &
<& tr + (1 - t)r = r \%& \\
\end{align*}

car t ≥ 0,1 - t ≥ 0 et soit t, soit 1 - t est non nul.

Théorème~5.1.4 Dans un espace vectoriel normé, tout ouvert connexe est
connexe par arcs.

Démonstration Soit U un ouvert connexe. Soit \mathcal{R} la relation d'équivalence
sur U~: a\mathcal{R}b s'il existe un chemin d'origine a et d'extrémité b. Soit
C(a) la classe d'équivalence de a et montrons que C(a) est ouverte. Pour
cela soit b \in C(a) \subset~ U. Il existe r > 0 tel que B(b,r) \subset~ U.
Mais la boule, étant convexe, est connexe par arcs et donc pour tout x
de B(b,r) on a x \in C(b) = C(a), soit B(b,r) \subset~ C(a). Les classes
d'équivalences sont donc ouvertes dans U. Mais on a alors
^cC(a) =\ \⋃
 _x∉C(a)C(x) est ouvert dans U et
donc C(a) est fermé dans U. Comme U est connexe, les seules parties
ouvertes et fermées dans U sont \varnothing~ et U, soit C(a) = U et donc U est
connexe par arcs.

\paragraph{5.1.3 Continuité des opérations algébriques}

Théorème~5.1.5 Soit E un espace vectoriel normé. L'application s : E \times E
\rightarrow~ E, (x,y)\mapsto~x + y est uniformément continue,
et l'application p : K \times E \rightarrow~ E, (\lambda~,x)\mapsto~\lambda~x est
continue.

Démonstration On a \s(x,y)
-s(x',y')\ =\ (x-x') +
(y -y')\ \leq\
x-x'\ +\ y
-y'\ \leq
2max~(\x-x'\,\y
-y'\) = 2\(x,y) -
(x',y')\, ce qui montre que s est
2-lipschitzienne.

Soit (\lambda_0,x_0) \in K \times E, \lambda~ \in K et x \in E. On a

\begin{align*} \p(\lambda~,x) -
p(\lambda_0,x_0)& =&
\\lambda~x -
\lambda_0x_0\ \%&
\\ & =& \\lambda~(x -
x_0) + (\lambda~ -
\lambda_0)x_0\\%&
\\ & \leq&
\lambda~\,\x -
x_0\ + \lambda~ -
\lambda_0\,\x_0\
\%& \\ \end{align*}

Pour \eta \leq 1, on a \lambda~ - \lambda_0 < \eta
\rigtharrow~\lambda~\leq\lambda_0 + 1. Soit donc \epsilon
> 0 et \eta = max~(1, \epsilon
\over 2(1+\lambda_0) , \epsilon
\over
2(1+\x_0\)
). Alors

\begin{align*} \(\lambda~,x) -
(\lambda_0,x_0)\
= max~(\lambda~ -
\lambda_0,\x -
x_0\) < \eta&&\%&
\\ & \rigtharrow~& \p(\lambda~,x)
- p(\lambda_0,x_0)\ \leq \epsilon
\over 2 + \epsilon \over 2 = \epsilon\%&
\\ \end{align*}

ce qui montre la continuité de p au point (\lambda_0,x_0).

Corollaire~5.1.6 Soit X un espace métrique, E un espace vectoriel normé,
A une partie de X et a \in\overlineA. (i) Si f et g
sont deux fonctions de X vers E telles que A \subset~\
Def (f) \bigcap Def~ (g), si f et g ont toutes deux
des limites en a suivant A et si \alpha~,\beta~ \in K, alors \alpha~f + \beta~g a une limite en
a suivant A et on a

lim_x\rightarrow~a,x\inA~(\alpha~f(x) + \beta~g(x)) =
\alpha~lim_x\rightarrow~a,x\inA~f(x) +
\beta~lim_x\rightarrow~a,x\inA~g(x)

(ii) Si \phi est une fonction de X vers K et f une fonction de X vers E
telles que A \subset~ Def~ (f)
\bigcap Def~ (\phi), si f et \phi ont toutes deux des
limites en a suivant A, alors \phif a une limite en a suivant A et on a

lim_x\rightarrow~a,x\inA~(\phi(x)f(x))
=\
lim_x\rightarrow~a,x\inA\phi(x)lim_x\rightarrow~a,x\inA~f(x)

Remarque~5.1.3 Dans le cas de E = \mathbb{R}~, les opérations algébriques sur \mathbb{R}~ \times
\mathbb{R}~ ne peuvent pas s'étendre de manière continue à
\overline\mathbb{R}~ \times\overline\mathbb{R}~, ce qui
fait que certaines opérations sur les limites ne sont pas valides en
général. On a cependant le théorème suivant qui permet d'étendre les
opérations sur les limites sauf dans les cas d'indéterminations ''\infty~-\infty~''
et ''0 \times\infty~''

Théorème~5.1.7 (i) L'application s : \mathbb{R}~ \times \mathbb{R}~ \rightarrow~ \mathbb{R}~,
(x,y)\mapsto~x + y s'étend en une application
continue de \overline\mathbb{R}~ \times\overline\mathbb{R}~
\diagdown\(-\infty~,+\infty~),(+\infty~,-\infty~)\ dans
\overline\mathbb{R}~ en posant x + (+\infty~) = +\infty~ si
x\neq~ -\infty~ et x + (-\infty~) = -\infty~ si
x\neq~ + \infty~. (ii) L'application p : \mathbb{R}~ \times \mathbb{R}~ \rightarrow~ \mathbb{R}~,
(x,y)\mapsto~xy s'étend en une application continue
de

\overline\mathbb{R}~ \times\overline\mathbb{R}~
\diagdown\(0,+\infty~),(+\infty~,0),(0,-\infty~),(-\infty~,0)\

dans \overline\mathbb{R}~ en posant x \times (+\infty~)
= sgn(x)\infty~ si x\mathrel\neq~~0 et
x \times (-\infty~) = -sgn~(x)\infty~ si
x\neq~0.

Démonstration La vérification de la continuité est tout à fait
élémentaire. Remarquons que puisque \mathbb{R}~ \times \mathbb{R}~ est dense dans
\overline\mathbb{R}~ \times\overline\mathbb{R}~, ces
prolongements sont uniques.

[
[

\end{document}
