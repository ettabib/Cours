\documentclass[]{article}
\usepackage[T1]{fontenc}
\usepackage{lmodern}
\usepackage{amssymb,amsmath}
\usepackage{ifxetex,ifluatex}
\usepackage{fixltx2e} % provides \textsubscript
% use upquote if available, for straight quotes in verbatim environments
\IfFileExists{upquote.sty}{\usepackage{upquote}}{}
\ifnum 0\ifxetex 1\fi\ifluatex 1\fi=0 % if pdftex
  \usepackage[utf8]{inputenc}
\else % if luatex or xelatex
  \ifxetex
    \usepackage{mathspec}
    \usepackage{xltxtra,xunicode}
  \else
    \usepackage{fontspec}
  \fi
  \defaultfontfeatures{Mapping=tex-text,Scale=MatchLowercase}
  \newcommand{\euro}{€}
\fi
% use microtype if available
\IfFileExists{microtype.sty}{\usepackage{microtype}}{}
\ifxetex
  \usepackage[setpagesize=false, % page size defined by xetex
              unicode=false, % unicode breaks when used with xetex
              xetex]{hyperref}
\else
  \usepackage[unicode=true]{hyperref}
\fi
\hypersetup{breaklinks=true,
            bookmarks=true,
            pdfauthor={},
            pdftitle={Formes quadratiques hermitiennes},
            colorlinks=true,
            citecolor=blue,
            urlcolor=blue,
            linkcolor=magenta,
            pdfborder={0 0 0}}
\urlstyle{same}  % don't use monospace font for urls
\setlength{\parindent}{0pt}
\setlength{\parskip}{6pt plus 2pt minus 1pt}
\setlength{\emergencystretch}{3em}  % prevent overfull lines
\setcounter{secnumdepth}{0}
 
/* start css.sty */
.cmr-5{font-size:50%;}
.cmr-7{font-size:70%;}
.cmmi-5{font-size:50%;font-style: italic;}
.cmmi-7{font-size:70%;font-style: italic;}
.cmmi-10{font-style: italic;}
.cmsy-5{font-size:50%;}
.cmsy-7{font-size:70%;}
.cmex-7{font-size:70%;}
.cmex-7x-x-71{font-size:49%;}
.msbm-7{font-size:70%;}
.cmtt-10{font-family: monospace;}
.cmti-10{ font-style: italic;}
.cmbx-10{ font-weight: bold;}
.cmr-17x-x-120{font-size:204%;}
.cmsl-10{font-style: oblique;}
.cmti-7x-x-71{font-size:49%; font-style: italic;}
.cmbxti-10{ font-weight: bold; font-style: italic;}
p.noindent { text-indent: 0em }
td p.noindent { text-indent: 0em; margin-top:0em; }
p.nopar { text-indent: 0em; }
p.indent{ text-indent: 1.5em }
@media print {div.crosslinks {visibility:hidden;}}
a img { border-top: 0; border-left: 0; border-right: 0; }
center { margin-top:1em; margin-bottom:1em; }
td center { margin-top:0em; margin-bottom:0em; }
.Canvas { position:relative; }
li p.indent { text-indent: 0em }
.enumerate1 {list-style-type:decimal;}
.enumerate2 {list-style-type:lower-alpha;}
.enumerate3 {list-style-type:lower-roman;}
.enumerate4 {list-style-type:upper-alpha;}
div.newtheorem { margin-bottom: 2em; margin-top: 2em;}
.obeylines-h,.obeylines-v {white-space: nowrap; }
div.obeylines-v p { margin-top:0; margin-bottom:0; }
.overline{ text-decoration:overline; }
.overline img{ border-top: 1px solid black; }
td.displaylines {text-align:center; white-space:nowrap;}
.centerline {text-align:center;}
.rightline {text-align:right;}
div.verbatim {font-family: monospace; white-space: nowrap; text-align:left; clear:both; }
.fbox {padding-left:3.0pt; padding-right:3.0pt; text-indent:0pt; border:solid black 0.4pt; }
div.fbox {display:table}
div.center div.fbox {text-align:center; clear:both; padding-left:3.0pt; padding-right:3.0pt; text-indent:0pt; border:solid black 0.4pt; }
div.minipage{width:100%;}
div.center, div.center div.center {text-align: center; margin-left:1em; margin-right:1em;}
div.center div {text-align: left;}
div.flushright, div.flushright div.flushright {text-align: right;}
div.flushright div {text-align: left;}
div.flushleft {text-align: left;}
.underline{ text-decoration:underline; }
.underline img{ border-bottom: 1px solid black; margin-bottom:1pt; }
.framebox-c, .framebox-l, .framebox-r { padding-left:3.0pt; padding-right:3.0pt; text-indent:0pt; border:solid black 0.4pt; }
.framebox-c {text-align:center;}
.framebox-l {text-align:left;}
.framebox-r {text-align:right;}
span.thank-mark{ vertical-align: super }
span.footnote-mark sup.textsuperscript, span.footnote-mark a sup.textsuperscript{ font-size:80%; }
div.tabular, div.center div.tabular {text-align: center; margin-top:0.5em; margin-bottom:0.5em; }
table.tabular td p{margin-top:0em;}
table.tabular {margin-left: auto; margin-right: auto;}
div.td00{ margin-left:0pt; margin-right:0pt; }
div.td01{ margin-left:0pt; margin-right:5pt; }
div.td10{ margin-left:5pt; margin-right:0pt; }
div.td11{ margin-left:5pt; margin-right:5pt; }
table[rules] {border-left:solid black 0.4pt; border-right:solid black 0.4pt; }
td.td00{ padding-left:0pt; padding-right:0pt; }
td.td01{ padding-left:0pt; padding-right:5pt; }
td.td10{ padding-left:5pt; padding-right:0pt; }
td.td11{ padding-left:5pt; padding-right:5pt; }
table[rules] {border-left:solid black 0.4pt; border-right:solid black 0.4pt; }
.hline hr, .cline hr{ height : 1px; margin:0px; }
.tabbing-right {text-align:right;}
span.TEX {letter-spacing: -0.125em; }
span.TEX span.E{ position:relative;top:0.5ex;left:-0.0417em;}
a span.TEX span.E {text-decoration: none; }
span.LATEX span.A{ position:relative; top:-0.5ex; left:-0.4em; font-size:85%;}
span.LATEX span.TEX{ position:relative; left: -0.4em; }
div.float img, div.float .caption {text-align:center;}
div.figure img, div.figure .caption {text-align:center;}
.marginpar {width:20%; float:right; text-align:left; margin-left:auto; margin-top:0.5em; font-size:85%; text-decoration:underline;}
.marginpar p{margin-top:0.4em; margin-bottom:0.4em;}
.equation td{text-align:center; vertical-align:middle; }
td.eq-no{ width:5%; }
table.equation { width:100%; } 
div.math-display, div.par-math-display{text-align:center;}
math .texttt { font-family: monospace; }
math .textit { font-style: italic; }
math .textsl { font-style: oblique; }
math .textsf { font-family: sans-serif; }
math .textbf { font-weight: bold; }
.partToc a, .partToc, .likepartToc a, .likepartToc {line-height: 200%; font-weight:bold; font-size:110%;}
.chapterToc a, .chapterToc, .likechapterToc a, .likechapterToc, .appendixToc a, .appendixToc {line-height: 200%; font-weight:bold;}
.index-item, .index-subitem, .index-subsubitem {display:block}
.caption td.id{font-weight: bold; white-space: nowrap; }
table.caption {text-align:center;}
h1.partHead{text-align: center}
p.bibitem { text-indent: -2em; margin-left: 2em; margin-top:0.6em; margin-bottom:0.6em; }
p.bibitem-p { text-indent: 0em; margin-left: 2em; margin-top:0.6em; margin-bottom:0.6em; }
.paragraphHead, .likeparagraphHead { margin-top:2em; font-weight: bold;}
.subparagraphHead, .likesubparagraphHead { font-weight: bold;}
.quote {margin-bottom:0.25em; margin-top:0.25em; margin-left:1em; margin-right:1em; text-align:\\jmathmathustify;}
.verse{white-space:nowrap; margin-left:2em}
div.maketitle {text-align:center;}
h2.titleHead{text-align:center;}
div.maketitle{ margin-bottom: 2em; }
div.author, div.date {text-align:center;}
div.thanks{text-align:left; margin-left:10%; font-size:85%; font-style:italic; }
div.author{white-space: nowrap;}
.quotation {margin-bottom:0.25em; margin-top:0.25em; margin-left:1em; }
h1.partHead{text-align: center}
.sectionToc, .likesectionToc {margin-left:2em;}
.subsectionToc, .likesubsectionToc {margin-left:4em;}
.subsubsectionToc, .likesubsubsectionToc {margin-left:6em;}
.frenchb-nbsp{font-size:75%;}
.frenchb-thinspace{font-size:75%;}
.figure img.graphics {margin-left:10%;}
/* end css.sty */

\title{Formes quadratiques hermitiennes}
\author{}
\date{}

\begin{document}
\maketitle

\textbf{Warning: 
requires JavaScript to process the mathematics on this page.\\ If your
browser supports JavaScript, be sure it is enabled.}

\begin{center}\rule{3in}{0.4pt}\end{center}

{[}
{[}
{[}{]}
{[}

\subsubsection{13.3 Formes quadratiques hermitiennes}

\paragraph{13.3.1 Notion de forme quadratique hermitienne}

Soit E un \mathbb{C}-espace vectoriel et \phi une forme sesquilinéaire hermitienne
sur E. Soit \Phi l'application de E dans \mathbb{R}~ qui à x associe \Phi(x) = \phi(x,x)
(on a en effet \phi(x,x) = \overline\phi(x,x) donc \Phi(x) \in
\mathbb{R}~).

Proposition~13.3.1 On a les identités suivantes

\begin{itemize}
\itemsep1pt\parskip0pt\parsep0pt
\item
  (i) \Phi(\lambda~x) = \lambda~^2\Phi(x)
\item
  (ii) \Phi(x + y) = \Phi(x) +
  2\mathrmRe~(\phi(x,y)) + \Phi(y)
\item
  (ii)' \Phi(x + y) - \Phi(x - y) + i\Phi(x + iy) - i\Phi(x - iy) = 4\phi(y,x)
  (identité de polarisation)
\item
  (iii) \Phi(x + y) + \Phi(x - y) = 2(\Phi(x) + \Phi(y)) (identité de la médiane)
\end{itemize}

Démonstration (i) \Phi(\lambda~x) = \phi(\lambda~x,\lambda~x) =
\lambda~\overline\lambda~\phi(x,x) =
\lambda~^2\Phi(x)

(ii) \Phi(x + y) = \phi(x + y,x + y) = \Phi(x) + \phi(x,y) + \phi(y,x) + \Phi(y) = \Phi(x) +
2\mathrmRe~(\phi(x,y)) + \Phi(y)~;
(ii)' s'en déduit immédiatement par un petit calcul

(iii) changeant y en - y dans l'identité précédente, on a aussi \Phi(x - y)
= \Phi(x) - 2\phi(x,y) + \Phi(y), et en additionnant les deux on trouve \Phi(x + y)
+ \Phi(x - y) = 2(\Phi(x) + \Phi(y)).

Remarque~13.3.1 L'identité (ii)' montre que l'application
\phi\mapsto~\Phi est in\\jmathmathective de H(E) dans \mathbb{R}~^E
(espace vectoriel des applications de E dans \mathbb{R}~) puisque la connaissance
de \Phi permet de retrouver \phi. Ceci nous amène à poser

Définition~13.3.1 Soit E un \mathbb{C}-espace vectoriel . On appelle forme
quadratique hermitienne sur E toute application \Phi : E \rightarrow~ \mathbb{R}~ telle qu'il
existe une forme sesquilinéaire hermitienne \phi : E \times E \rightarrow~ \mathbb{C} vérifiant
\forall~~x \in E, \Phi(x) = \phi(x,x). Dans ce cas, \phi est
unique et est appelée la forme polaire de \Phi.

Exemple~13.3.1 Sur \mathbb{C}^n, \Phi(x) =\
\sum ~
_i=1^nx_i^2 est
une forme quadratique hermitienne dont la forme polaire associée est
\phi(x,y) = \\sum ~
_i=1^n\overlinex_iy_i.
Si E désigne l'espace vectoriel des fonctions continues de {[}a,b{]}
dans \mathbb{C}, \Phi(f) =\int ~
_a^bf(t)^2 dt est une forme
quadratique hermitienne dont la forme polaire est \phi(f,g)
=\int ~
_a^b\overlinef(t)g(t) dt.

Proposition~13.3.2 L'ensemble Q(E) des formes quadratiques sur E est un
\mathbb{R}~-sous-espace vectoriel de \mathbb{R}~^E~; l'application
\phi\mapsto~\Phi est un isomorphisme de \mathbb{R}~-espaces
vectoriels de H(E) sur Q(E).

Remarque~13.3.2 Par la suite on confondra toutes les notions relatives à
\phi et à \Phi~: orthogonalité, matrice, non dégénérescence, isotropie~; en
particulier on posera
\mathrmKer~\Phi
= \mathrmKer~\phi =
\x \in
E∣\forall~~y \in E, \phi(x,y) =
0\. On remarquera qu'en général,
\mathrmKer\Phi\mathrel\neq~~\x
\in E∣\Phi(x) = 0\.

Théorème~13.3.3 (Pythagore). Soit E un \mathbb{C}-espace vectoriel ~et \Phi \inQ(E), \phi
la forme polaire de \Phi. Alors

x \bot_\phiy \rigtharrow~ \Phi(x + y) = \Phi(x) + \Phi(y)

Démonstration C'est une conséquence évidente de l'identité \Phi(x + y) =
\Phi(x) + 2\mathrmRe~(\phi(x,y)) +
\Phi(y). Remarquons l'absence de réciproque, contrairement au cas des
formes quadratiques.

\paragraph{13.3.2 Formes quadratiques hermitiennes en dimension finie}

Soit E un \mathbb{C}-espace vectoriel ~de dimension finie, \Phi \inQ(E) de forme
polaire \phi.

Théorème~13.3.4 Soit \mathcal{E} une base de E. Alors
\mathrmMat~ (\phi,\mathcal{E}) est
l'unique matrice \Omega \in M_\mathbb{C}(n) qui est hermitienne et qui vérifie

\forall~x \in E, \Phi(x) = X^∗~\OmegaX

Démonstration Il est clair que \Omega =\
\mathrmMat (\Phi,\mathcal{E}) est hermitienne et vérifie \Phi(x) =
\phi(x,x) = X^∗\OmegaX. Inversement, soit \Omega une matrice hermitienne
vérifiant cette propriété. On a alors

\phi(y,x) = 1 \over 4 (\Phi(x + y) - \Phi(x - y) + i\Phi(x + iy)
- i\Phi(x - iy)) = Y ^∗\OmegaX

(après un calcul un peu pénible) ce qui montre que \Omega
= \mathrmMat~ (\phi,\mathcal{E}).

Posons \Omega = \mathrmMat~ (\phi,\mathcal{E})
= (\omega_i,\\jmathmath)_1\leqi,\\jmathmath\leqn. On a alors

\phi(x,y) = \\sum
_i,\\jmathmath\omega_i,\\jmathmath\overlinex_iy_\\jmathmath
= \\sum
_i\omega_i,i\overlinex_iy_i
+ \\sum
_i\textless{}\\jmathmath(\omega_i,\\jmathmath\overlinex_iy_\\jmathmath
+ \omega_\\jmathmath,i\overlinex_\\jmathmathy_i)

En tenant compte de \omega_i,\\jmathmath =
\overline\omega_\\jmathmath,i, on a donc

\Phi(x) = \phi(x,x) = \\sum
_i\omega_i,ix_i^2 +
2\mathrmRe(\\sum
_i\textless{}\\jmathmath\omega_i,\\jmathmath\overlinex_ix_\\jmathmath)
=
P_\Phi(x_1,\ldots,x_n~)

Inversement, soit P de la forme
P(x_1,\\ldots,x_n~)
= \\sum ~
_i=1^na_i,ix_i^2
+
2\mathrmRe~(\\\sum

_i\textless{}\\jmathmatha_i,\\jmathmath\overlinex_ix_\\jmathmath).
Définissons \phi sur E par

\phi(x,y) = \\sum
_ia_i,i\overlinex_iy_i
+ \\sum
_i\textless{}\\jmathmath(a_i,\\jmathmath\overlinex_iy_\\jmathmath
+
\overlinea_i,\\jmathmath\overlinex_\\jmathmathy_i)

si x = \\sum ~
x_ie_i et y =\
\sum  y_ie_i~. Alors \phi est
clairement une forme sesquilinéaire hermitienne sur E et la forme
quadratique associée vérifie \Phi(x) =
P(x_1,\\ldots,x_n~).
On obtient l'expression de \phi(x,y) à partir de l'expression polynomiale
de \Phi(x) en rempla\ccant partout les termes carrés
x_i^2 par
\overlinex_iy_i et les termes
rectangles
\mathrmRe(a_i,\\jmathmath\overlinex_ix_\\jmathmath~)
par  1 \over 2
(a_i,\\jmathmath\overlinex_ix_\\jmathmath +
\overlinea_i,\\jmathmath\overlinex_\\jmathmathy_i).

Théorème~13.3.5 Si \mathcal{E} est une base orthonormée de E (c'est à dire
\phi(e_i,e_\\jmathmath) = \delta_i^\\jmathmath), alors
\mathrmMat~ (\phi,\mathcal{E}) =
I_n, \phi(x,y) = X^∗Y =\
\sum ~
_i=1^n\overlinex_iy_i
et \Phi(x) = X^∗X =\
\sum ~
_i=1^nx_i^2.

Démonstration Evident.

\paragraph{13.3.3 Formes quadratiques hermitiennes définies positives}

Définition~13.3.2 Soit E un \mathbb{C} espace vectoriel et \Phi une forme
quadratique hermitienne sur E. On dit que \Phi est définie positive si
\forall~x \in E \diagdown\0\~,
\Phi(x) \textgreater{} 0.

Théorème~13.3.6 (inégalité de Schwarz). Soit E un \mathbb{C} espace vectoriel et
\Phi une forme quadratique hermitienne définie positive sur E de forme
polaire \phi. Alors

\forall~~x,y \in E,
\phi(x,y)^2 \leq \Phi(x)\Phi(y)

avec égalité si et seulement si~la famille (x,y) est liée.

Démonstration L'inégalité est évidente si y = 0~; supposons donc
y\neq~0. Soit \theta \in \mathbb{R}~. On écrit
\forall~t \in \mathbb{R}~, \Phi(x + te^i\theta~y) ≥ 0, soit
encore t^2\Phi(y) +
2t\mathrmRe(e^i\theta~\phi(x,y))
+ \Phi(x) ≥ 0. Choisissons \theta tel que \phi(x,y) =
e^-i\theta\phi(x,y) (autrement dit l'opposé d'un
argument de \phi(x,y)). On a donc t^2\Phi(y) +
2t\phi(x,y) + \Phi(x) ≥ 0. Ce trinome du second degré doit
donc avoir un discriminant réduit négatif, soit
\phi(x,y)^2 - \Phi(x)\Phi(y) \leq 0. Si on a
l'égalité, deux cas sont possibles. Soit y = 0 auquel cas la famille
(x,y) est liée, soit \Phi(y)\neq~0~; mais dans ce
cas ce trinome en t a une racine double t_0, et donc \Phi(x +
t_0e^i\thetay) = 0 d'où x + t_0e^i\thetay
= 0 et donc la famille est liée. Inversement, si la famille (x,y) est
liée, on a par exemple x = \lambda~y et dans ce cas
\phi(x,y)^2 =
\lambda~^2\Phi(y)^2 = \Phi(x)\Phi(y).

Théorème~13.3.7 (inégalité de Minkowski). Soit E un \mathbb{C} espace vectoriel
et \Phi une forme quadratique hermitienne définie positive sur E. Alors

\forall~x,y \in E, \sqrt\Phi(x + y)~
\leq\sqrt\Phi(x) + \sqrt\Phi(y)

avec égalité si et seulement si~la famille (x,y) est positivement liée.

Démonstration On a

\begin{align*} \Phi(x + y)& =& \Phi(x) +
2\mathrmRe~(\phi(x,y)) +
\Phi(y)\%& \\ & \leq& \Phi(x) +
2\phi(x,y) + \Phi(y) \%& \\
& \leq& \Phi(x) + 2\sqrt\Phi(x)\Phi(y) + \Phi(y)\%&
\\ & =& \left
(\sqrt\Phi(x) +
\sqrt\Phi(y)\right )^2 \%&
\\ \end{align*}

d'où \sqrt\Phi(x + y) \leq\sqrt\Phi(x) +
\sqrt\Phi(y). L'égalité nécessite à la fois que
\phi(x,y) = \sqrt\Phi(x)\Phi(y), donc que
(x,y) soit liée, et que
\mathrmRe~(\phi(x,y)) =
\phi(x,y)≥ 0, c'est-à-dire que le coefficient de
proportionnalité soit réel et positif.

Définition~13.3.3 On appelle espace préhilbertien complexe un couple
(E,\Phi) d'un \mathbb{C}-espace vectoriel ~E et d'une forme quadratique hermitienne
définie positive sur E. On appelle espace hermitien un espace
préhilbertien complexe de dimension finie.

Théorème~13.3.8 Soit (E,\Phi) un espace préhilbertien complexe. Alors
l'application x\mapsto~\sqrt\Phi(x)
est une norme sur E appelée norme hermitienne.

Démonstration La propriété de séparation provient du fait que \Phi est
définie. L'homogénéité provient de l'homogénéité de la forme
quadratique. Quant à l'inégalité triangulaire, ce n'est autre que
l'inégalité de Minkowski.

Définition~13.3.4 On notera (x∣y) = \phi(x,y) et
\x\^2 =
(x∣x) = \Phi(x)

\paragraph{13.3.4 Espaces hermitiens}

Une forme définie positive étant clairement non dégénérée, on a bien
évidemment

Théorème~13.3.9 Soit E un espace hermitien. Pour toute forme linéaire f
sur E, il existe un unique vecteur v_f \in E tel que
\forall~~x \in E, f(x) =
(v_f∣x)

D'autre part si \Phi est définie positive, et si A est un sous-espace
vectoriel de E on a

x \in A \bigcap A^\bot\rigtharrow~ x \bot x \rigtharrow~ (x∣x) = 0 \rigtharrow~ x
= 0

Comme de plus dim~ A +\
dim A^\bot = dim~ E, on obtient

Théorème~13.3.10 Soit E un espace hermitien.Pour tout sous-espace
vectoriel A de E, on a E = A \oplus~ A^\bot et
(A^\bot)^\bot = A.

Enfin l'existence de bases orthonormées nous est garanti par
l'algorithme de Gramm-Schmidt, dont la démonstration est strictement la
même que pour les formes quadratiques~:

Théorème~13.3.11 Soit E un espace hermitien. Soit \mathcal{E} =
(e_1,\\ldots,e_n~)
une base de E. Alors il existe une base orthonormée \mathcal{E}' =
(\epsilon_1,\\ldots,\epsilon_n~)
de E vérifiant les conditions équivalentes suivantes

\begin{itemize}
\itemsep1pt\parskip0pt\parsep0pt
\item
  (i) \forall~k \in {[}1,n{]}, \epsilon_k~
  \in\mathrmVect(e_1,\\\ldots,e_k~)
\item
  (ii) \forall~~k \in {[}1,n{]},
  \mathrmVect(\epsilon_1,\\\ldots,\epsilon_k~)
  =\
  \mathrmVect(e_1,\\ldots,e_k~)
\item
  (iii) la matrice de passage de \mathcal{E} à \mathcal{E}' est triangulaire supérieure
\end{itemize}

Si \mathcal{E}' =
(\epsilon_1,\\ldots,\epsilon_n~)
et \mathcal{E}'' =
(\eta_1,\\ldots,\eta_n~)
sont deux telles bases orthonormées, il existe des scalaires
\lambda_1,\\ldots,\lambda_n~
de module 1 tels que \forall~~i \in {[}1,n{]},
\eta_i = \lambda_i\epsilon_i.

{[}
{[}
{[}
{[}

\end{document}
