\documentclass[]{article}
\usepackage[T1]{fontenc}
\usepackage{lmodern}
\usepackage{amssymb,amsmath}
\usepackage{ifxetex,ifluatex}
\usepackage{fixltx2e} % provides \textsubscript
% use upquote if available, for straight quotes in verbatim environments
\IfFileExists{upquote.sty}{\usepackage{upquote}}{}
\ifnum 0\ifxetex 1\fi\ifluatex 1\fi=0 % if pdftex
  \usepackage[utf8]{inputenc}
\else % if luatex or xelatex
  \ifxetex
    \usepackage{mathspec}
    \usepackage{xltxtra,xunicode}
  \else
    \usepackage{fontspec}
  \fi
  \defaultfontfeatures{Mapping=tex-text,Scale=MatchLowercase}
  \newcommand{\euro}{€}
\fi
% use microtype if available
\IfFileExists{microtype.sty}{\usepackage{microtype}}{}
\ifxetex
  \usepackage[setpagesize=false, % page size defined by xetex
              unicode=false, % unicode breaks when used with xetex
              xetex]{hyperref}
\else
  \usepackage[unicode=true]{hyperref}
\fi
\hypersetup{breaklinks=true,
            bookmarks=true,
            pdfauthor={},
            pdftitle={Integrale des fonctions reelles positives},
            colorlinks=true,
            citecolor=blue,
            urlcolor=blue,
            linkcolor=magenta,
            pdfborder={0 0 0}}
\urlstyle{same}  % don't use monospace font for urls
\setlength{\parindent}{0pt}
\setlength{\parskip}{6pt plus 2pt minus 1pt}
\setlength{\emergencystretch}{3em}  % prevent overfull lines
\setcounter{secnumdepth}{0}
 
/* start css.sty */
.cmr-5{font-size:50%;}
.cmr-7{font-size:70%;}
.cmmi-5{font-size:50%;font-style: italic;}
.cmmi-7{font-size:70%;font-style: italic;}
.cmmi-10{font-style: italic;}
.cmsy-5{font-size:50%;}
.cmsy-7{font-size:70%;}
.cmex-7{font-size:70%;}
.cmex-7x-x-71{font-size:49%;}
.msbm-7{font-size:70%;}
.cmtt-10{font-family: monospace;}
.cmti-10{ font-style: italic;}
.cmbx-10{ font-weight: bold;}
.cmr-17x-x-120{font-size:204%;}
.cmsl-10{font-style: oblique;}
.cmti-7x-x-71{font-size:49%; font-style: italic;}
.cmbxti-10{ font-weight: bold; font-style: italic;}
p.noindent { text-indent: 0em }
td p.noindent { text-indent: 0em; margin-top:0em; }
p.nopar { text-indent: 0em; }
p.indent{ text-indent: 1.5em }
@media print {div.crosslinks {visibility:hidden;}}
a img { border-top: 0; border-left: 0; border-right: 0; }
center { margin-top:1em; margin-bottom:1em; }
td center { margin-top:0em; margin-bottom:0em; }
.Canvas { position:relative; }
li p.indent { text-indent: 0em }
.enumerate1 {list-style-type:decimal;}
.enumerate2 {list-style-type:lower-alpha;}
.enumerate3 {list-style-type:lower-roman;}
.enumerate4 {list-style-type:upper-alpha;}
div.newtheorem { margin-bottom: 2em; margin-top: 2em;}
.obeylines-h,.obeylines-v {white-space: nowrap; }
div.obeylines-v p { margin-top:0; margin-bottom:0; }
.overline{ text-decoration:overline; }
.overline img{ border-top: 1px solid black; }
td.displaylines {text-align:center; white-space:nowrap;}
.centerline {text-align:center;}
.rightline {text-align:right;}
div.verbatim {font-family: monospace; white-space: nowrap; text-align:left; clear:both; }
.fbox {padding-left:3.0pt; padding-right:3.0pt; text-indent:0pt; border:solid black 0.4pt; }
div.fbox {display:table}
div.center div.fbox {text-align:center; clear:both; padding-left:3.0pt; padding-right:3.0pt; text-indent:0pt; border:solid black 0.4pt; }
div.minipage{width:100%;}
div.center, div.center div.center {text-align: center; margin-left:1em; margin-right:1em;}
div.center div {text-align: left;}
div.flushright, div.flushright div.flushright {text-align: right;}
div.flushright div {text-align: left;}
div.flushleft {text-align: left;}
.underline{ text-decoration:underline; }
.underline img{ border-bottom: 1px solid black; margin-bottom:1pt; }
.framebox-c, .framebox-l, .framebox-r { padding-left:3.0pt; padding-right:3.0pt; text-indent:0pt; border:solid black 0.4pt; }
.framebox-c {text-align:center;}
.framebox-l {text-align:left;}
.framebox-r {text-align:right;}
span.thank-mark{ vertical-align: super }
span.footnote-mark sup.textsuperscript, span.footnote-mark a sup.textsuperscript{ font-size:80%; }
div.tabular, div.center div.tabular {text-align: center; margin-top:0.5em; margin-bottom:0.5em; }
table.tabular td p{margin-top:0em;}
table.tabular {margin-left: auto; margin-right: auto;}
div.td00{ margin-left:0pt; margin-right:0pt; }
div.td01{ margin-left:0pt; margin-right:5pt; }
div.td10{ margin-left:5pt; margin-right:0pt; }
div.td11{ margin-left:5pt; margin-right:5pt; }
table[rules] {border-left:solid black 0.4pt; border-right:solid black 0.4pt; }
td.td00{ padding-left:0pt; padding-right:0pt; }
td.td01{ padding-left:0pt; padding-right:5pt; }
td.td10{ padding-left:5pt; padding-right:0pt; }
td.td11{ padding-left:5pt; padding-right:5pt; }
table[rules] {border-left:solid black 0.4pt; border-right:solid black 0.4pt; }
.hline hr, .cline hr{ height : 1px; margin:0px; }
.tabbing-right {text-align:right;}
span.TEX {letter-spacing: -0.125em; }
span.TEX span.E{ position:relative;top:0.5ex;left:-0.0417em;}
a span.TEX span.E {text-decoration: none; }
span.LATEX span.A{ position:relative; top:-0.5ex; left:-0.4em; font-size:85%;}
span.LATEX span.TEX{ position:relative; left: -0.4em; }
div.float img, div.float .caption {text-align:center;}
div.figure img, div.figure .caption {text-align:center;}
.marginpar {width:20%; float:right; text-align:left; margin-left:auto; margin-top:0.5em; font-size:85%; text-decoration:underline;}
.marginpar p{margin-top:0.4em; margin-bottom:0.4em;}
.equation td{text-align:center; vertical-align:middle; }
td.eq-no{ width:5%; }
table.equation { width:100%; } 
div.math-display, div.par-math-display{text-align:center;}
math .texttt { font-family: monospace; }
math .textit { font-style: italic; }
math .textsl { font-style: oblique; }
math .textsf { font-family: sans-serif; }
math .textbf { font-weight: bold; }
.partToc a, .partToc, .likepartToc a, .likepartToc {line-height: 200%; font-weight:bold; font-size:110%;}
.chapterToc a, .chapterToc, .likechapterToc a, .likechapterToc, .appendixToc a, .appendixToc {line-height: 200%; font-weight:bold;}
.index-item, .index-subitem, .index-subsubitem {display:block}
.caption td.id{font-weight: bold; white-space: nowrap; }
table.caption {text-align:center;}
h1.partHead{text-align: center}
p.bibitem { text-indent: -2em; margin-left: 2em; margin-top:0.6em; margin-bottom:0.6em; }
p.bibitem-p { text-indent: 0em; margin-left: 2em; margin-top:0.6em; margin-bottom:0.6em; }
.paragraphHead, .likeparagraphHead { margin-top:2em; font-weight: bold;}
.subparagraphHead, .likesubparagraphHead { font-weight: bold;}
.quote {margin-bottom:0.25em; margin-top:0.25em; margin-left:1em; margin-right:1em; text-align:\jmathustify;}
.verse{white-space:nowrap; margin-left:2em}
div.maketitle {text-align:center;}
h2.titleHead{text-align:center;}
div.maketitle{ margin-bottom: 2em; }
div.author, div.date {text-align:center;}
div.thanks{text-align:left; margin-left:10%; font-size:85%; font-style:italic; }
div.author{white-space: nowrap;}
.quotation {margin-bottom:0.25em; margin-top:0.25em; margin-left:1em; }
h1.partHead{text-align: center}
.sectionToc, .likesectionToc {margin-left:2em;}
.subsectionToc, .likesubsectionToc {margin-left:4em;}
.subsubsectionToc, .likesubsubsectionToc {margin-left:6em;}
.frenchb-nbsp{font-size:75%;}
.frenchb-thinspace{font-size:75%;}
.figure img.graphics {margin-left:10%;}
/* end css.sty */

\title{Integrale des fonctions reelles positives}
\author{}
\date{}

\begin{document}
\maketitle

\textbf{Warning: 
requires JavaScript to process the mathematics on this page.\\ If your
browser supports JavaScript, be sure it is enabled.}

\begin{center}\rule{3in}{0.4pt}\end{center}

{[}
{[}
{[}{]}
{[}

\subsubsection{9.9 Intégrale des fonctions réelles positives}

\paragraph{9.9.1 Critère de convergence des fonctions réelles positives}

Remarque~9.9.1 Dans toute la suite, sauf précision contraire on
supposera que -\infty~ \textless{} a \textless{} b \leq +\infty~ si on considère
l'intervalle {[}a,b{[} et que -\infty~\leq a \textless{} b \textless{} +\infty~ si on
considère l'intervalle {]}a,b{]}.

Théorème~9.9.1 Soit f : {[}a,b{[}\rightarrow~ \mathbb{R}~ réglée positive. Alors
\int  \_a^b~f(t) dt converge si et
seulement si~les intégrales partielles \int ~
\_a^xf(t) dt sont ma\jmathorées~:

\existsM ≥ 0, \\forall~~x \in
{[}a,b{[}, \int  \_a^x~f(t) dt \leq M

Démonstration Si a \leq x \textless{} x' \textless{} b, on a
\int  \_a^x'~f(t) dt
-\int  \_a^x~f(t) dt
=\int  \_x^x'~f(t) dt ≥ 0, dont
l'application x\mapsto~\\int
 \_a^xf(t) dt est croissante. En conséquence, elle admet
une limite au point b si et seulement si~elle est ma\jmathorée.

Remarque~9.9.2 Dans le cas d'une fonction réglée f :{]}a,b{]} \rightarrow~ \mathbb{R}~
positive, l'application
x\mapsto~\int ~
\_x^bf(t) dt est cette fois-ci décroissante~; donc elle
admet une limite (à droite) au point a si et seulement si~elle est
ma\jmathorée. Dans tous les cas d'intégrales impropres à gauche ou à droite
d'une fonction réelle positive, la convergence est équivalente à la
ma\jmathoration des intégrales partielles~; dans ce cas, si l'intégrale
diverge, les intégrales partielles tendent vers + \infty~.

Théorème~9.9.2 Soit f,g : {[}a,b{[}\rightarrow~ \mathbb{R}~ réglées telles que
\forall~~t \in {[}a,b{[}, 0 \leq f(t) \leq g(t). Alors (i) si
l'intégrale \int  \_a^b~g(t) dt
converge, l'intégrale \int ~
\_a^bf(t) dt converge. (ii) si l'intégrale
\int  \_a^b~f(t) dt diverge,
l'intégrale \int  \_a^b~g(t) dt
diverge.

Démonstration Pour (i), il suffit de remarquer que
\int  \_a^x~f(t) dt
\leq\int  \_a^x~g(t) dt, donc que tout
ma\jmathorant des intégrales partielles de g ma\jmathore également les intégrales
partielles de f. Quant à (ii), ce n'est que la contraposée de (i).

Remarque~9.9.3 On a dé\jmathà vu que la convergence ou la divergence de
l'intégrale était une propriété locale en b. Pour appliquer le résultat
précédent, il suffit donc de supposer que sur un voisinage de b on a 0 \leq
f(t) \leq Kg(t), autrement dit que f(t) = O(g(t)) au voisinage de b.

\paragraph{9.9.2 Règles de comparaison}

Théorème~9.9.3 Soit f,g : {[}a,b{[}\rightarrow~ \mathbb{R}~ réglées positives. On suppose
qu'au voisinage de b on a f = 0(g) (resp. f = o(g)). Alors (i) si
\int  \_a^b~g(t) dt converge,
\int  \_a^b~f(t) dt converge
également et \int  \_x^b~f(t) dt =
0(\int  \_x^b~g(t) dt) (resp.
\int  \_x^b~f(t) dt =
o(\int  \_x^b~g(t) dt)) (ii) si
\int  \_a^b~f(t) dt converge,
\int  \_a^b~g(t) dt converge
également et \int  \_a^x~f(t) dt =
0(\int  \_a^x~g(t) dt) (resp.
\int  \_a^x~f(t) dt =
o(\int  \_a^x~g(t) dt))

Démonstration Les convergences et divergences découlent immédiatement de
la remarque qui suit le théorème ci-dessus et du fait que f = o(g) \rigtharrow~ f =
O(g). En ce qui concerne la comparaison des restes ou des intégrales
partielles, la démonstration est tout à fait similaire à celle du
théorème analogue sur les séries. Nous allons les faire dans le cas f =
o(g), la démonstration étant analogue pour f = O(g) en changeant \epsilon en K
ou en 2K.

(i) Supposons f = o(g) et \int ~
\_a^bg(t) dt convergente. Soit \epsilon \textgreater{} 0. Il
existe c \in {[}a,b{[} tel que t ≥ c \rigtharrow~ 0 \leq f(t) \leq \epsilong(t). Alors pour x ≥ c,
on a (en intégrant l'inégalité de c à b), 0 \leq\\int
 \_x^bf(t) dt \leq \epsilon\int ~
\_x^bg(t) dt et donc \int ~
\_x^bf(t) dt = o(\int ~
\_x^bg(t) dt).

(ii) Supposons f = o(g) et \int ~
\_a^bf(t) dt divergente. Soit \epsilon \textgreater{} 0. Il existe
c \in {[}a,b{[} tel que t ≥ c \rigtharrow~ 0 \leq f(t) \leq \epsilon \over 2
g(t). Alors pour x ≥ c, on a (en intégrant l'inégalité de c à x),
\int  \_c^x~f(t) dt \leq \epsilon
\over 2 \int ~
\_c^xg(t) dt, soit encore à l'aide de la relation de
Chasles

0 \leq\int  \_a^x~f(t) dt \leq \epsilon
\over 2 \int ~
\_a^xg(t) dt + \left
(\int  \_a^c~f(t) dt - \epsilon
\over 2 \int ~
\_a^cg(t) dt\right )

Mais comme on sait que l'intégrale \int ~
\_a^bg(t) dt diverge et que g ≥ 0, on a
lim\_x\rightarrow~b\\int ~
\_a^xg(t) dt = +\infty~. Donc il existe c' \in {[}a,b{[} tel que x
≥ c' \rigtharrow~ \epsilon \over 2 \int ~
\_a^xg(t) dt \textgreater{}\int ~
\_a^cf(t) dt - \epsilon \over 2
\int  \_a^c~g(t) dt. Alors, pour x
≥ max~(c,c'), on a

0 \leq\int  \_a^x~f(t) dt \leq \epsilon
\over 2 \int ~
\_a^xg(t) dt + \epsilon \over 2
\int  \_a^x~g(t) dt =
\epsilon\int  \_a^x~g(t) dt

et donc \int  \_a^x~f(t) dt =
o(\int  \_a^x~g(t) dt).

Remarque~9.9.4 Il suffit pour appliquer le théorème précédent que la
condition de positivité de f et g soit vérifiée dans un voisinage de b.

Théorème~9.9.4 Soit f,g : {[}a,b{[}\rightarrow~ \mathbb{R}~ réglées. On suppose que g est
positive et que au voisinage de b, on a f ∼ g. Alors les deux intégrales
\int  \_a^b~f(t) dt et
\int  \_a^b~g(t) dt sont de même
nature. Plus précisément (i) Si \int ~
\_a^bg(t) dt converge, alors \int ~
\_a^bf(t) dt converge également et
\int  \_x^b~f(t) dt
∼\int  \_x^b~g(t) dt (ii) Si
\int  \_a^b~g(t) dt diverge, alors
\int  \_a^b~f(t) dt diverge
également et \int  \_a^x~f(t) dt
∼\int  \_a^x~g(t) dt.

Démonstration Puisque f(t) ∼ g(t), il existe c \in {[}a,b{[} tel que x
\textgreater{} c \rigtharrow~ 1 \over 2 g(t) \leq f(t) \leq 3
\over 2 g(t) ce qui montre que f est positive au
voisinage de b et que l'on a à la fois f = O(g) et g = O(f). Le théorème
précédent assure alors que \int ~
\_a^bg(t) dt converge si et seulement
si~\int  \_a^b~f(t) dt converge.
Pla\ccons nous dans le cas de convergence. On a
\textbar{}f - g\textbar{} = o(g), on en déduit que l'intégrale
\int  \_a^b~\textbar{}f(t) -
g(t)\textbar{} dt converge et que \int ~
\_x^b\textbar{}f(t) - g(t)\textbar{} dt =
o(\int  \_x^b~g(t) dt). Mais bien
évidemment \left \textbar{}\int ~
\_x^bf(t) dt -\int ~
\_x^bg(t) dt\right
\textbar{}\leq\int ~
\_x^b\textbar{}f(t) - g(t)\textbar{} dt. On a donc
\int  \_x^b~f(t) dt
-\int  \_x^b~g(t) dt =
o(\int  \_x^b~g(t) dt) et donc
\int  \_x^b~f(t) dt
∼\int  \_x^b~g(t) dt. Dans le cas
de divergence, deux cas se présentent. Si l'intégrale
\int  \_a^b~\textbar{}f(t) -
g(t)\textbar{} dt diverge, le théorème précédent assure que
\int  \_a^x~\textbar{}f(t) -
g(t)\textbar{} dt = o(\int ~
\_a^xg(t) dt)~; si par contre elle converge,
\int  \_a^x~\textbar{}f(t) -
g(t)\textbar{} dt admet une limite finie en b alors que
\int  \_a^x~g(t) dt tend vers + \infty~
et on a donc encore \int ~
\_a^x\textbar{}f(t) - g(t)\textbar{} dt =
o(\int  \_a^x~g(t) dt). L'inégalité
\left \textbar{}\int ~
\_a^xf(t) dt -\int ~
\_a^xg(t) dt\right
\textbar{}\leq\int ~
\_a^x\textbar{}f(t) - g(t)\textbar{} dt donne alors
\int  \_a^x~f(t) dt
-\int  \_a^x~g(t) dt =
o(\int  \_a^x~g(t) dt) et donc
\int  \_a^x~f(t) dt
∼\int  \_a^x~g(t) dt.

\paragraph{9.9.3 Exemples fondamentaux}

L'idée générale est d'obtenir une famille de fonctions étalon.

Proposition~9.9.5 L'intégrale \int ~
\_1^+\infty~ dt \over t^\alpha~ converge
si et seulement si~\alpha~ \textgreater{} 1.

Démonstration \int  \_1^x~ dt
\over t^\alpha~ = \left
\ \cases  1 \over
\alpha~-1 (1 - x^1-\alpha~)&si \alpha~\neq~1
\cr log~ x &si \alpha~ = 1
\cr  \right . qui admet une limite finie
en + \infty~ si et seulement si~\alpha~ \textgreater{} 1.

Exemple~9.9.1 Intégrales de Bertrand \int ~
\_e^+\infty~ dt \over
t^\alpha~(log t)^\beta~~ . Si \alpha~
\textgreater{} 1, soit \gamma tel que 1 \textless{} \alpha~ \textless{} \gamma. On a
alors  1 \over
t^\alpha~(log t)^\beta~~ = o( 1
\over t^\gamma ) et donc l'intégrale converge. Si
\alpha~ \textless{} 1, soit \gamma tel que \alpha~ \textless{} \gamma \textless{} 1~; on a
alors  1 \over t^\gamma = o( 1
\over t^\alpha~(log~
t)^\beta~ ) et comme \int ~
\_e^+\infty~ dt \over t^\gamma diverge,
l'intégrale diverge. Si \alpha~ = 1, on a par le changement de variables u
= log~ t,

\int  \_e^x~ dt
\over t(log t)^\beta~~
=\int ~
\_1^log x~ du
\over u^\beta~ = \left
\ \cases  1 \over
\alpha~-1 (1 - (log x)^1-\alpha~~)&si
\alpha~\neq~1 \cr
log \log~ x &si \alpha~ = 1
 \right .

qui admet une limite en + \infty~ si et seulement si~\beta~ \textgreater{} 1. En
définitive l'intégrale de Bertrand \int ~
\_e^+\infty~ dt \over
t^\alpha~(log t)^\beta~~ converge
si et seulement si~\alpha~ \textgreater{} 1 ou (\alpha~ = 1 et \beta~ \textgreater{} 1).

Proposition~9.9.6 Soit -\infty~ \textless{} a \textless{} b \textless{} +\infty~.
L'intégrale \int  \_a^b~ dt
\over \textbar{}b-t\textbar{}^\alpha~ converge si
et seulement si~\alpha~ \textless{} 1.

Démonstration On a

\int  \_a^b~ dt
\over (b - t)^\alpha~ = \left
\ \cases  1 \over
1-\alpha~ ((b - a)^1-\alpha~ - (b - x)^1-\alpha~)&si
\alpha~\neq~1 \cr
log (b - a) -\ log~ (b
- x)&si \alpha~ = 1  \right .

qui admet une limite au point b si et seulement si~\alpha~ \textless{} 1.

Exemple~9.9.2 Intégrales de Bertrand \int ~
\_0^1\diagupet^\alpha~\textbar{}log~
t\textbar{}^\beta~ dt. Si \alpha~ \textgreater{} -1, soit \gamma tel que \alpha~
\textgreater{} \gamma \textgreater{} -1. On a alors en 0,
t^\alpha~\textbar{}log~
t\textbar{}^\beta~ = o(t^\gamma) (car 
t^\alpha~\textbar{} log~
t\textbar{}^\beta~ \over t^\gamma =
t^\alpha~-\gamma\textbar{}log~
t\textbar{}^\beta~ tend vers 0 quand t tend vers 0) et comme
\int  \_0^1\diagupet^\gamma~ dt
converge, l'intégrale converge. Si \alpha~ \textless{} -1, soit \gamma tel que \alpha~
\textless{} \gamma \textless{} -1. Alors t^\gamma =
o(t^\alpha~\textbar{}log~
t\textbar{}^\beta~) et comme \int ~
\_0^1\diagupet^\gamma dt diverge, l'intégrale diverge. Si \alpha~
= -1, le changement de variables u = -log~ t
conduit à

\int  \_x^1\diagupe~
\textbar{}log t\textbar{}^\beta~~
\over t dt =\int ~
\_1^- log xu^\beta~~ du

qui admet une limite quand x tend vers 0 si et seulement si~\beta~
\textless{} -1. En définitive, l'intégrale de Bertrand
\int ~
\_0^1\diagupet^\alpha~\textbar{}log~
t\textbar{}^\beta~ dt converge si et seulement si~\alpha~ \textgreater{}
-1 ou (\alpha~ = -1 et \beta~ \textless{} -1).

Remarque~9.9.5 Ce dernier exemple aurait pu également être traité à
partir des intégrales de Bertrand en + \infty~ à l'aide du changement de
variables u = 1 \over t qui est de classe
\mathcal{C}^1 et un homéomorphisme de {]}0,1\diagupe{]} sur {[}e,+\infty~{[} et
donc qui conserve la nature des intégrales. Ceci montre que l'intégrale
\int ~
\_0^1\diagupet^\alpha~\textbar{}log~
t\textbar{}^\beta~ dt est de même nature que l'intégrale
\int  \_e^+\infty~~
(log u)^\beta~~ \over
u^\alpha~  du \over u^2 qui
converge si et seulement si~2 + \alpha~ \textgreater{} 1 ou 2 + \alpha~ = 1 et \beta~
\textless{} -1.

{[}
{[}
{[}
{[}

\end{document}
