\documentclass[]{article}
\usepackage[T1]{fontenc}
\usepackage{lmodern}
\usepackage{amssymb,amsmath}
\usepackage{ifxetex,ifluatex}
\usepackage{fixltx2e} % provides \textsubscript
% use upquote if available, for straight quotes in verbatim environments
\IfFileExists{upquote.sty}{\usepackage{upquote}}{}
\ifnum 0\ifxetex 1\fi\ifluatex 1\fi=0 % if pdftex
  \usepackage[utf8]{inputenc}
\else % if luatex or xelatex
  \ifxetex
    \usepackage{mathspec}
    \usepackage{xltxtra,xunicode}
  \else
    \usepackage{fontspec}
  \fi
  \defaultfontfeatures{Mapping=tex-text,Scale=MatchLowercase}
  \newcommand{\euro}{€}
\fi
% use microtype if available
\IfFileExists{microtype.sty}{\usepackage{microtype}}{}
\usepackage{graphicx}
% Redefine \includegraphics so that, unless explicit options are
% given, the image width will not exceed the width of the page.
% Images get their normal width if they fit onto the page, but
% are scaled down if they would overflow the margins.
\makeatletter
\def\ScaleIfNeeded{%
  \ifdim\Gin@nat@width>\linewidth
    \linewidth
  \else
    \Gin@nat@width
  \fi
}
\makeatother
\let\Oldincludegraphics\includegraphics
{%
 \catcode`\@=11\relax%
 \gdef\includegraphics{\@ifnextchar[{\Oldincludegraphics}{\Oldincludegraphics[width=\ScaleIfNeeded]}}%
}%
\ifxetex
  \usepackage[setpagesize=false, % page size defined by xetex
              unicode=false, % unicode breaks when used with xetex
              xetex]{hyperref}
\else
  \usepackage[unicode=true]{hyperref}
\fi
\hypersetup{breaklinks=true,
            bookmarks=true,
            pdfauthor={},
            pdftitle={Barycentres},
            colorlinks=true,
            citecolor=blue,
            urlcolor=blue,
            linkcolor=magenta,
            pdfborder={0 0 0}}
\urlstyle{same}  % don't use monospace font for urls
\setlength{\parindent}{0pt}
\setlength{\parskip}{6pt plus 2pt minus 1pt}
\setlength{\emergencystretch}{3em}  % prevent overfull lines
\setcounter{secnumdepth}{0}
 
/* start css.sty */
.cmr-5{font-size:50%;}
.cmr-7{font-size:70%;}
.cmmi-5{font-size:50%;font-style: italic;}
.cmmi-7{font-size:70%;font-style: italic;}
.cmmi-10{font-style: italic;}
.cmsy-5{font-size:50%;}
.cmsy-7{font-size:70%;}
.cmex-7{font-size:70%;}
.cmex-7x-x-71{font-size:49%;}
.msbm-7{font-size:70%;}
.cmtt-10{font-family: monospace;}
.cmti-10{ font-style: italic;}
.cmbx-10{ font-weight: bold;}
.cmr-17x-x-120{font-size:204%;}
.cmsl-10{font-style: oblique;}
.cmti-7x-x-71{font-size:49%; font-style: italic;}
.cmbxti-10{ font-weight: bold; font-style: italic;}
p.noindent { text-indent: 0em }
td p.noindent { text-indent: 0em; margin-top:0em; }
p.nopar { text-indent: 0em; }
p.indent{ text-indent: 1.5em }
@media print {div.crosslinks {visibility:hidden;}}
a img { border-top: 0; border-left: 0; border-right: 0; }
center { margin-top:1em; margin-bottom:1em; }
td center { margin-top:0em; margin-bottom:0em; }
.Canvas { position:relative; }
li p.indent { text-indent: 0em }
.enumerate1 {list-style-type:decimal;}
.enumerate2 {list-style-type:lower-alpha;}
.enumerate3 {list-style-type:lower-roman;}
.enumerate4 {list-style-type:upper-alpha;}
div.newtheorem { margin-bottom: 2em; margin-top: 2em;}
.obeylines-h,.obeylines-v {white-space: nowrap; }
div.obeylines-v p { margin-top:0; margin-bottom:0; }
.overline{ text-decoration:overline; }
.overline img{ border-top: 1px solid black; }
td.displaylines {text-align:center; white-space:nowrap;}
.centerline {text-align:center;}
.rightline {text-align:right;}
div.verbatim {font-family: monospace; white-space: nowrap; text-align:left; clear:both; }
.fbox {padding-left:3.0pt; padding-right:3.0pt; text-indent:0pt; border:solid black 0.4pt; }
div.fbox {display:table}
div.center div.fbox {text-align:center; clear:both; padding-left:3.0pt; padding-right:3.0pt; text-indent:0pt; border:solid black 0.4pt; }
div.minipage{width:100%;}
div.center, div.center div.center {text-align: center; margin-left:1em; margin-right:1em;}
div.center div {text-align: left;}
div.flushright, div.flushright div.flushright {text-align: right;}
div.flushright div {text-align: left;}
div.flushleft {text-align: left;}
.underline{ text-decoration:underline; }
.underline img{ border-bottom: 1px solid black; margin-bottom:1pt; }
.framebox-c, .framebox-l, .framebox-r { padding-left:3.0pt; padding-right:3.0pt; text-indent:0pt; border:solid black 0.4pt; }
.framebox-c {text-align:center;}
.framebox-l {text-align:left;}
.framebox-r {text-align:right;}
span.thank-mark{ vertical-align: super }
span.footnote-mark sup.textsuperscript, span.footnote-mark a sup.textsuperscript{ font-size:80%; }
div.tabular, div.center div.tabular {text-align: center; margin-top:0.5em; margin-bottom:0.5em; }
table.tabular td p{margin-top:0em;}
table.tabular {margin-left: auto; margin-right: auto;}
div.td00{ margin-left:0pt; margin-right:0pt; }
div.td01{ margin-left:0pt; margin-right:5pt; }
div.td10{ margin-left:5pt; margin-right:0pt; }
div.td11{ margin-left:5pt; margin-right:5pt; }
table[rules] {border-left:solid black 0.4pt; border-right:solid black 0.4pt; }
td.td00{ padding-left:0pt; padding-right:0pt; }
td.td01{ padding-left:0pt; padding-right:5pt; }
td.td10{ padding-left:5pt; padding-right:0pt; }
td.td11{ padding-left:5pt; padding-right:5pt; }
table[rules] {border-left:solid black 0.4pt; border-right:solid black 0.4pt; }
.hline hr, .cline hr{ height : 1px; margin:0px; }
.tabbing-right {text-align:right;}
span.TEX {letter-spacing: -0.125em; }
span.TEX span.E{ position:relative;top:0.5ex;left:-0.0417em;}
a span.TEX span.E {text-decoration: none; }
span.LATEX span.A{ position:relative; top:-0.5ex; left:-0.4em; font-size:85%;}
span.LATEX span.TEX{ position:relative; left: -0.4em; }
div.float img, div.float .caption {text-align:center;}
div.figure img, div.figure .caption {text-align:center;}
.marginpar {width:20%; float:right; text-align:left; margin-left:auto; margin-top:0.5em; font-size:85%; text-decoration:underline;}
.marginpar p{margin-top:0.4em; margin-bottom:0.4em;}
.equation td{text-align:center; vertical-align:middle; }
td.eq-no{ width:5%; }
table.equation { width:100%; } 
div.math-display, div.par-math-display{text-align:center;}
math .texttt { font-family: monospace; }
math .textit { font-style: italic; }
math .textsl { font-style: oblique; }
math .textsf { font-family: sans-serif; }
math .textbf { font-weight: bold; }
.partToc a, .partToc, .likepartToc a, .likepartToc {line-height: 200%; font-weight:bold; font-size:110%;}
.chapterToc a, .chapterToc, .likechapterToc a, .likechapterToc, .appendixToc a, .appendixToc {line-height: 200%; font-weight:bold;}
.index-item, .index-subitem, .index-subsubitem {display:block}
.caption td.id{font-weight: bold; white-space: nowrap; }
table.caption {text-align:center;}
h1.partHead{text-align: center}
p.bibitem { text-indent: -2em; margin-left: 2em; margin-top:0.6em; margin-bottom:0.6em; }
p.bibitem-p { text-indent: 0em; margin-left: 2em; margin-top:0.6em; margin-bottom:0.6em; }
.paragraphHead, .likeparagraphHead { margin-top:2em; font-weight: bold;}
.subparagraphHead, .likesubparagraphHead { font-weight: bold;}
.quote {margin-bottom:0.25em; margin-top:0.25em; margin-left:1em; margin-right:1em; text-align:justify;}
.verse{white-space:nowrap; margin-left:2em}
div.maketitle {text-align:center;}
h2.titleHead{text-align:center;}
div.maketitle{ margin-bottom: 2em; }
div.author, div.date {text-align:center;}
div.thanks{text-align:left; margin-left:10%; font-size:85%; font-style:italic; }
div.author{white-space: nowrap;}
.quotation {margin-bottom:0.25em; margin-top:0.25em; margin-left:1em; }
h1.partHead{text-align: center}
.sectionToc, .likesectionToc {margin-left:2em;}
.subsectionToc, .likesubsectionToc {margin-left:4em;}
.subsubsectionToc, .likesubsubsectionToc {margin-left:6em;}
.frenchb-nbsp{font-size:75%;}
.frenchb-thinspace{font-size:75%;}
.figure img.graphics {margin-left:10%;}
/* end css.sty */

\title{Barycentres}
\author{}
\date{}

\begin{document}
\maketitle

\textbf{Warning: \href{http://www.math.union.edu/locate/jsMath}{jsMath}
requires JavaScript to process the mathematics on this page.\\ If your
browser supports JavaScript, be sure it is enabled.}

\begin{center}\rule{3in}{0.4pt}\end{center}

{[}\href{coursse94.html}{next}{]} {[}\href{coursse92.html}{prev}{]}
{[}\href{coursse92.html\#tailcoursse92.html}{prev-tail}{]}
{[}\hyperref[tailcoursse93.html]{tail}{]}
{[}\href{coursch18.html\#coursse93.html}{up}{]}

\subsubsection{17.2 Barycentres}

\paragraph{17.2.1 Notion de barycentres}

Théorème~17.2.1 Soit E un espace affine, \{(\{a\}\_\{i\})\}\_\{i∈I\} une
famille finie de points de E et \{(\{λ\}\_\{i\})\}\_\{i∈I\} une famille
de scalaires telle que
\{\textbackslash{}mathop\{\textbackslash{}mathop\{∑ \}\}
\}\_\{i∈I\}\{λ\}\_\{i\}\textbackslash{}mathrel\{≠\}0. Alors il existe un
unique point g ∈ E vérifiant les conditions équivalentes

\begin{itemize}
\itemsep1pt\parskip0pt\parsep0pt
\item
  (i) \{\textbackslash{}mathop\{\textbackslash{}mathop\{∑ \}\}
  \}\_\{i∈I\}\{λ\}\_\{i\}\textbackslash{}overrightarrow\{g\{a\}\_\{i\}\}
  =\textbackslash{}overrightarrow\{ 0\}
\item
  (ii) \textbackslash{}mathop\{∀\}m ∈ E,
  \{\textbackslash{}mathop\{\textbackslash{}mathop\{∑ \}\}
  \}\_\{i∈I\}\{λ\}\_\{i\}\textbackslash{}overrightarrow\{m\{a\}\_\{i\}\}
  = (\{\textbackslash{}mathop\{\textbackslash{}mathop\{∑ \}\}
  \}\_\{i∈I\}\{λ\}\_\{i\})\textbackslash{}overrightarrow\{mg\}
\item
  (iii) il existe a ∈ E tel que \textbackslash{}overrightarrow\{ag\} =\{
  1 \textbackslash{}over
  \{\textbackslash{}mathop\{\textbackslash{}mathop\{∑ \}\}
  \}\_\{i∈I\}\{λ\}\_\{i\}\} \{\textbackslash{}mathop\{
  \textbackslash{}mathop\{∑ \}\}
  \}\_\{i∈I\}\{λ\}\_\{i\}\textbackslash{}overrightarrow\{a\{a\}\_\{i\}\}
\end{itemize}

On dit alors que g est le barycentre des points \{a\}\_\{i\} affectés
des coefficients \{λ\}\_\{i\}.

Démonstration Il est clair que (ii) ⇒(i) (prendre m = g) et que (ii)
⇒(iii) (prendre m = a et diviser par
\{\textbackslash{}mathop\{\textbackslash{}mathop\{∑ \}\}
\}\_\{i∈I\}\{λ\}\_\{i\}). Si maintenant (i) est vérifié, on a, pour m ∈
E,

\textbackslash{}begin\{eqnarray*\} \{\textbackslash{}mathop\{∑
\}\}\_\{i∈I\}\{λ\}\_\{i\}\textbackslash{}overrightarrow\{m\{a\}\_\{i\}\}\&
=\& \{\textbackslash{}mathop\{∑
\}\}\_\{i∈I\}\{λ\}\_\{i\}(\textbackslash{}overrightarrow\{mg\}
+\textbackslash{}overrightarrow\{ g\{a\}\_\{i\}\}) =
(\{\textbackslash{}mathop\{∑
\}\}\_\{i∈I\}\{λ\}\_\{i\})\textbackslash{}overrightarrow\{mg\} +\{
\textbackslash{}mathop\{∑
\}\}\_\{i∈I\}\{λ\}\_\{i\}\textbackslash{}overrightarrow\{g\{a\}\_\{i\}\}\%\&
\textbackslash{}\textbackslash{} \& =\& (\{\textbackslash{}mathop\{∑
\}\}\_\{i∈I\}\{λ\}\_\{i\})\textbackslash{}overrightarrow\{mg\} \%\&
\textbackslash{}\textbackslash{} \textbackslash{}end\{eqnarray*\}

et donc (ii) est vérifiée. De la même fa\textbackslash{}c\{c\}on, si
(iii) est vérifiée, on a
\{\textbackslash{}mathop\{\textbackslash{}mathop\{∑ \}\}
\}\_\{i∈I\}\{λ\}\_\{i\}\textbackslash{}overrightarrow\{a\{a\}\_\{i\}\} =
(\{\textbackslash{}mathop\{\textbackslash{}mathop\{∑ \}\}
\}\_\{i∈I\}\{λ\}\_\{i\})\textbackslash{}overrightarrow\{ag\} et donc,
pour m ∈ E,

\textbackslash{}begin\{eqnarray*\} \{\textbackslash{}mathop\{∑
\}\}\_\{i∈I\}\{λ\}\_\{i\}\textbackslash{}overrightarrow\{m\{a\}\_\{i\}\}\&
=\& \{\textbackslash{}mathop\{∑
\}\}\_\{i∈I\}\{λ\}\_\{i\}(\textbackslash{}overrightarrow\{ma\}
+\textbackslash{}overrightarrow\{ a\{a\}\_\{i\}\}) =
(\{\textbackslash{}mathop\{∑
\}\}\_\{i∈I\}\{λ\}\_\{i\})\textbackslash{}overrightarrow\{ma\} +\{
\textbackslash{}mathop\{∑
\}\}\_\{i∈I\}\{λ\}\_\{i\}\textbackslash{}overrightarrow\{a\{a\}\_\{i\}\}\%\&
\textbackslash{}\textbackslash{} \& =\& (\{\textbackslash{}mathop\{∑
\}\}\_\{i∈I\}\{λ\}\_\{i\})\textbackslash{}overrightarrow\{ma\} +
(\{\textbackslash{}mathop\{∑
\}\}\_\{i∈I\}\{λ\}\_\{i\})\textbackslash{}overrightarrow\{ag\} =
(\{\textbackslash{}mathop\{∑
\}\}\_\{i∈I\}\{λ\}\_\{i\})\textbackslash{}overrightarrow\{mg\} \%\&
\textbackslash{}\textbackslash{} \textbackslash{}end\{eqnarray*\}

et donc (ii) est vérifiée, ce qui achève la démonstration.

Remarque~17.2.1 Si \{\textbackslash{}mathop\{\textbackslash{}mathop\{∑
\}\} \}\_\{i∈I\}\{λ\}\_\{i\} = 0, on vérifie facilement que
\{\textbackslash{}mathop\{\textbackslash{}mathop\{∑ \}\}
\}\_\{i∈I\}\{λ\}\_\{i\}\textbackslash{}overrightarrow\{m\{a\}\_\{i\}\}
est un vecteur \textbackslash{}vec\{u\} indépendant de m~; c'est parfois
ce vecteur qu'on appelle barycentre de la famille lorsque
\{\textbackslash{}mathop\{\textbackslash{}mathop\{∑ \}\}
\}\_\{i∈I\}\{λ\}\_\{i\} = 0. Il s'agit alors d'un vecteur et non plus
d'un point.

Définition~17.2.1 On appelle point massique de E tout couple (a,λ) d'un
point a ∈ E et d'un scalaire λ ∈ K.

Remarque~17.2.2 On dira indifféremment que g est est le barycentre des
points \{a\}\_\{i\} affectés des coefficients \{λ\}\_\{i\} ou que g est
le barycentre de la famille de points massiques \{\textbackslash{}left
((\{a\}\_\{i\},\{λ\}\_\{i\})\textbackslash{}right )\}\_\{i∈I\}.

\paragraph{17.2.2 Associativité des barycentres}

Théorème~17.2.2 Soit \{\textbackslash{}left
((\{a\}\_\{i\},\{λ\}\_\{i\})\textbackslash{}right )\}\_\{i∈I\} une
famille de points massiques telle que
\{\textbackslash{}mathop\{\textbackslash{}mathop\{∑ \}\}
\}\_\{i∈I\}\{λ\}\_\{i\}\textbackslash{}mathrel\{≠\}0 et g son
barycentre. Soit I = \{I\}\_\{1\}
∪\textbackslash{}mathop\{\textbackslash{}mathop\{\ldots{}\}\} ∪
\{I\}\_\{k\} une partition de I telle que \textbackslash{}mathop\{∀\}j ∈
{[}1,k{]}, \{μ\}\_\{j\} =\{\textbackslash{}mathop\{
\textbackslash{}mathop\{∑ \}\}
\}\_\{i∈\{I\}\_\{j\}\}\{λ\}\_\{i\}\textbackslash{}mathrel\{≠\}0. Soit
\{g\}\_\{j\} le barycentre de la famille de points massiques
\{\textbackslash{}left ((\{a\}\_\{i\},\{λ\}\_\{i\})\textbackslash{}right
)\}\_\{i∈\{I\}\_\{j\}\}. Alors g est le barycentre des points
\{g\}\_\{1\},\textbackslash{}mathop\{\textbackslash{}mathop\{\ldots{}\}\},\{g\}\_\{k\}
affectés des coefficients
\{μ\}\_\{1\},\textbackslash{}mathop\{\textbackslash{}mathop\{\ldots{}\}\},\{μ\}\_\{k\}.

Démonstration On a

\textbackslash{}overrightarrow\{0\} =\{ \textbackslash{}mathop\{∑
\}\}\_\{i∈I\}\{λ\}\_\{i\}\textbackslash{}overrightarrow\{g\{a\}\_\{i\}\}
=\{ \textbackslash{}mathop\{∑ \}\}\_\{j=1\}\^{}\{k\}\{
\textbackslash{}mathop\{∑
\}\}\_\{i∈\{I\}\_\{j\}\}\{λ\}\_\{i\}\textbackslash{}overrightarrow\{g\{a\}\_\{i\}\}

Mais d'après la définition de \{g\}\_\{j\}, on a
\{\textbackslash{}mathop\{\textbackslash{}mathop\{∑ \}\}
\}\_\{i∈\{I\}\_\{j\}\}\{λ\}\_\{i\}\textbackslash{}overrightarrow\{g\{a\}\_\{i\}\}
= \textbackslash{}left
(\{\textbackslash{}mathop\{\textbackslash{}mathop\{∑ \}\}
\}\_\{i∈\{I\}\_\{j\}\}\{λ\}\_\{i\}\textbackslash{}right
)\textbackslash{}overrightarrow\{g\{g\}\_\{j\}\} =
\{μ\}\_\{j\}\textbackslash{}overrightarrow\{g\{g\}\_\{j\}\}. On a donc
\textbackslash{}overrightarrow\{0\} =\{\textbackslash{}mathop\{
\textbackslash{}mathop\{∑ \}\}
\}\_\{j=1\}\^{}\{k\}\{μ\}\_\{j\}\textbackslash{}overrightarrow\{g\{g\}\_\{j\}\}
ce qui démontre que g est le barycentre des points
\{g\}\_\{1\},\textbackslash{}mathop\{\textbackslash{}mathop\{\ldots{}\}\},\{g\}\_\{k\}
affectés des coefficients
\{μ\}\_\{1\},\textbackslash{}mathop\{\textbackslash{}mathop\{\ldots{}\}\},\{μ\}\_\{k\}.

Exemple~17.2.1 Si
(\{a\}\_\{1\},\textbackslash{}mathop\{\textbackslash{}mathop\{\ldots{}\}\},\{a\}\_\{n\})
est une famille de points de E et si la caractéristique p de K (le corps
de base) ne divise pas n, on peut définir le barycentre des points
\{a\}\_\{1\},\textbackslash{}mathop\{\textbackslash{}mathop\{\ldots{}\}\},\{a\}\_\{n\}
tous affectés du coefficient 1 (puisque
n\{1\}\_\{K\}\textbackslash{}mathrel\{≠\}0)~; on appelle ce point
l'isobarycentre des points
\{a\}\_\{1\},\textbackslash{}mathop\{\textbackslash{}mathop\{\ldots{}\}\},\{a\}\_\{n\}.
Notons le g. Soit {[}1,n{]} = \{I\}\_\{1\} ∪ \{I\}\_\{2\} une partition
de {[}1,n{]} avec k = \textbar{}\{I\}\_\{1\}\textbar{} et n − k =
\textbar{}\{I\}\_\{2\}\textbar{}. Supposons que p ne divise ni k ni n −
k et soit \{m\}\_\{1\} l'isobarycentre des \{a\}\_\{i\},i ∈
\{I\}\_\{1\}, \{m\}\_\{2\} l'isobarycentre des \{a\}\_\{i\},i ∈
\{I\}\_\{2\}. Alors le théorème d'associativité des barycentres assure
que g est aussi le barycentre de (\{m\}\_\{1\},k) et (\{m\}\_\{2\},n −
k), et en particulier g appartient à la droite \{m\}\_\{1\}\{m\}\_\{2\}.
Dans le cas où n = 3, on montre ainsi que sur un corps de
caractéristique différente de 2 ou 3, les droites qui joignent un sommet
du triangle au milieu du coté opposé qui n'est autre que l'isobarycentre
de ces deux points (ces droites sont les médianes du triangle)
contiennent toutes l'isobarycentre des sommets du triangle (le centre de
gravité du triangle), autrement dit ces trois médianes sont
concourantes. De même, en dimension 3 et pour n = 4, les quatre droites
joignant un sommet au centre de gravité de la face opposée et les trois
droites joignant les milieux de deux arêtes opposées passent toutes par
le centre de gravité du tétraèdre.

\includegraphics{cours11x.png}

\paragraph{17.2.3 Barycentres, sous-espaces affines, applications
affines}

Les deux théorèmes suivants montrent que le barycentrage est l'opération
algébrique fondamentale dans les espaces affines.

Théorème~17.2.3 Soit f : E → F une application d'un espace affine dans
un autre espace affine. Alors f est affine si et seulement si~elle
conserve la notion de barycentre, c'est-à-dire si et seulement si~pour
toute famille \{\textbackslash{}left
((\{a\}\_\{i\},\{λ\}\_\{i\})\textbackslash{}right )\}\_\{i∈I\} telle que
\{\textbackslash{}mathop\{\textbackslash{}mathop\{∑ \}\}
\}\_\{i∈I\}\{λ\}\_\{i\}\textbackslash{}mathrel\{≠\}0, de barycentre g,
le point f(g) est le barycentre de la famille de points massiques
\{\textbackslash{}left
((f(\{a\}\_\{i\}),\{λ\}\_\{i\})\textbackslash{}right )\}\_\{i∈I\}.

Démonstration Si f est affine, on a en effet

\textbackslash{}overrightarrow\{0\} =\textbackslash{}vec\{
f\}(\{\textbackslash{}mathop\{∑
\}\}\_\{i∈I\}\{λ\}\_\{i\}\textbackslash{}overrightarrow\{g\{a\}\_\{i\}\})
=\{ \textbackslash{}mathop\{∑
\}\}\_\{i∈I\}\{λ\}\_\{i\}\textbackslash{}vec\{f\}(\textbackslash{}overrightarrow\{g\{a\}\_\{i\}\})
=\{ \textbackslash{}mathop\{∑
\}\}\_\{i∈I\}\{λ\}\_\{i\}\textbackslash{}overrightarrow\{f(g)f(\{a\}\_\{i\})\}

ce qui montre que le point f(g) est le barycentre de la famille de
points massiques \{\textbackslash{}left
((f(\{a\}\_\{i\}),\{λ\}\_\{i\})\textbackslash{}right )\}\_\{i∈I\}.
Inversement, supposons que f vérifie cette propriété et montrons que f
est affine. Pour cela, soit a ∈ E et posons
\textbackslash{}vec\{f\}(\textbackslash{}overrightarrow\{ξ\})
=\textbackslash{}overrightarrow\{ f(a)f(a
+\textbackslash{}overrightarrow\{ ξ\})\}. Il nous suffit donc de montrer
que \textbackslash{}vec\{f\} est linéaire.

Pour cela soit tout d'abord \textbackslash{}overrightarrow\{ξ\}
∈\textbackslash{}overrightarrow\{ E\} et λ ∈ K. Posons b = a
+\textbackslash{}overrightarrow\{ ξ\} et c = a +
λ\textbackslash{}overrightarrow\{ξ\}. On a donc
\textbackslash{}overrightarrow\{ac\} −
λ\textbackslash{}overrightarrow\{ab\} =\textbackslash{}overrightarrow\{
0\}, soit encore (1 − λ)\textbackslash{}overrightarrow\{ac\} +
λ\textbackslash{}overrightarrow\{bc\} =\textbackslash{}overrightarrow\{
0\} (en écrivant \textbackslash{}overrightarrow\{ab\}
=\textbackslash{}overrightarrow\{ ac\} +\textbackslash{}overrightarrow\{
cb\}). Donc c est le barycentre de (a,1 − λ) et (b,λ). On en déduit que
f(c) est le barycentre de (f(a),1 − λ) et de (f(b),λ), soit encore que
(1 − λ)\textbackslash{}overrightarrow\{f(a)f(c)\} +
λ\textbackslash{}overrightarrow\{f(b)f(c)\}
=\textbackslash{}overrightarrow\{ 0\}, soit encore
\textbackslash{}overrightarrow\{f(a)f(c)\} −
λ\textbackslash{}overrightarrow\{f(a)f(b)\}
=\textbackslash{}overrightarrow\{ 0\}, ce qui se traduit par
\textbackslash{}vec\{f\}(λ\textbackslash{}overrightarrow\{ξ\}) =
λ\textbackslash{}vec\{f\}(\textbackslash{}overrightarrow\{ξ\}).

Soit maintenant, \textbackslash{}overrightarrow\{ξ\} et
\textbackslash{}overrightarrow\{η\} dans
\textbackslash{}overrightarrow\{E\}. Posons b = a
+\textbackslash{}overrightarrow\{ ξ\}, c = a
+\textbackslash{}overrightarrow\{ η\} et d = a +
(\textbackslash{}overrightarrow\{ξ\} +\textbackslash{}overrightarrow\{
η\}). On a alors −\textbackslash{}overrightarrow\{ ad\}
+\textbackslash{}overrightarrow\{ ab\} +\textbackslash{}overrightarrow\{
ac\} =\textbackslash{}overrightarrow\{ 0\} si bien que a est barycentre
de (d,−1), (b,1) et (c,1). On en déduit que f(a) est barycentre de
(f(d),−1), (f(b),1) et (f(c),1), si bien que

−\textbackslash{}overrightarrow\{f(a)f(d)\}
+\textbackslash{}overrightarrow\{ f(a)f(b)\}
+\textbackslash{}overrightarrow\{ f(a)f(c)\}
=\textbackslash{}overrightarrow\{ 0\}

ce qui se traduit par −\textbackslash{}vec\{
f\}(\textbackslash{}overrightarrow\{ξ\}
+\textbackslash{}overrightarrow\{ η\}) +\textbackslash{}vec\{
f\}(\textbackslash{}overrightarrow\{ξ\}) +\textbackslash{}vec\{
f\}(\textbackslash{}overrightarrow\{η\})
=\textbackslash{}overrightarrow\{ 0\}. Donc \textbackslash{}vec\{f\} est
bien linéaire.

Théorème~17.2.4 Soit F une partie non vide d'un espace affine E. Alors F
est un sous-espace affine si et seulement si~il est stable par
barycentrage, c'est-à-dire si et seulement si~pour toute famille
\{\textbackslash{}left ((\{a\}\_\{i\},\{λ\}\_\{i\})\textbackslash{}right
)\}\_\{i∈I\} telle que \{a\}\_\{i\} ∈ F et
\{\textbackslash{}mathop\{\textbackslash{}mathop\{∑ \}\}
\}\_\{i∈I\}\{λ\}\_\{i\}\textbackslash{}mathrel\{≠\}0 de barycentre g, le
point g est encore dans F.

Démonstration Supposons que F est un sous-espace affine et soit a ∈ F.
Alors \textbackslash{}overrightarrow\{ag\} =\{ 1 \textbackslash{}over
\{\textbackslash{}mathop\{\textbackslash{}mathop\{∑ \}\}
\}\_\{i∈I\}\{λ\}\_\{i\}\} \{\textbackslash{}mathop\{
\textbackslash{}mathop\{∑ \}\}
\}\_\{i∈I\}\{λ\}\_\{i\}\textbackslash{}overrightarrow\{a\{a\}\_\{i\}\}
∈\textbackslash{}overrightarrow\{ F\} puisque chacun des
\textbackslash{}overrightarrow\{a\{a\}\_\{i\}\} est dans l'espace
vectoriel \textbackslash{}overrightarrow\{F\}. Donc g ∈ F.

Inversement, supposons que F est stable par barycentrage et soit a ∈ F,
\textbackslash{}overrightarrow\{F\} =
\textbackslash{}\{\textbackslash{}overrightarrow\{ξ\}
∈\textbackslash{}overrightarrow\{ E\}\textbackslash{}mathrel\{∣\}a
+\textbackslash{}overrightarrow\{ ξ\} ∈ F\textbackslash{}\}. Il suffit
de montrer que \textbackslash{}overrightarrow\{F\} est un sous-espace
vectoriel de \textbackslash{}overrightarrow\{E\}.

Soit tout d'abord \textbackslash{}overrightarrow\{ξ\}
∈\textbackslash{}overrightarrow\{ F\} et λ ∈ K. Posons b = a
+\textbackslash{}overrightarrow\{ ξ\} ∈ F et c = a +
λ\textbackslash{}overrightarrow\{ξ\}. On a donc
\textbackslash{}overrightarrow\{ac\} −
λ\textbackslash{}overrightarrow\{ab\} =\textbackslash{}overrightarrow\{
0\}, soit encore (1 − λ)\textbackslash{}overrightarrow\{ac\} +
λ\textbackslash{}overrightarrow\{bc\} =\textbackslash{}overrightarrow\{
0\} (en écrivant \textbackslash{}overrightarrow\{ab\}
=\textbackslash{}overrightarrow\{ ac\} +\textbackslash{}overrightarrow\{
cb\}). Donc c est le barycentre de (a,1 − λ) et (b,λ). Donc c ∈ F, soit
λ\textbackslash{}overrightarrow\{ξ\} ∈\textbackslash{}overrightarrow\{
F\}.

Soit maintenant, \textbackslash{}overrightarrow\{ξ\} et
\textbackslash{}overrightarrow\{η\} dans
\textbackslash{}overrightarrow\{F\}. Posons b = a
+\textbackslash{}overrightarrow\{ ξ\} ∈ F, c = a
+\textbackslash{}overrightarrow\{ η\} ∈ F et d = a +
(\textbackslash{}overrightarrow\{ξ\} +\textbackslash{}overrightarrow\{
η\}). On a alors −\textbackslash{}overrightarrow\{ ad\}
+\textbackslash{}overrightarrow\{ ab\} +\textbackslash{}overrightarrow\{
ac\} =\textbackslash{}overrightarrow\{ 0\}, ou encore
\textbackslash{}overrightarrow\{ad\} −\textbackslash{}overrightarrow\{
bd\} −\textbackslash{}overrightarrow\{ cd\}
=\textbackslash{}overrightarrow\{ 0\} (en utilisant la relation de
Chasles). Donc d est le barycentre de (a,1), (b,−1) et (c,−1). On a donc
d ∈ F, soit encore \textbackslash{}overrightarrow\{ξ\}
+\textbackslash{}overrightarrow\{ η\} ∈\textbackslash{}overrightarrow\{
F\}, ce qui achève de montrer que \textbackslash{}overrightarrow\{F\}
est un sous-espace vectoriel de \textbackslash{}overrightarrow\{E\}, et
donc F un sous-espace affine de E.

\paragraph{17.2.4 Barycentres et convexité}

On supposera ici que le corps de base est ℝ.

Définition~17.2.2 Soit E un espace affine sur ℝ, a et b deux points de
E. On appelle segment {[}a,b{]} l'ensemble des barycentres des points a
et b affectés des coefficients t et 1 − t pour t ∈ {[}0,1{]}.

Remarque~17.2.3 Autrement dit {[}a,b{]} = \textbackslash{}\{a +
t\textbackslash{}overrightarrow\{ab\}\textbackslash{}mathrel\{∣\}t ∈
{[}0,1{]}\textbackslash{}\}.

Définition~17.2.3 Soit E un espace affine sur ℝ et A une partie de E. On
dit que A est convexe si \textbackslash{}mathop\{∀\}a,b ∈ A, {[}a,b{]} ⊂
A.

Théorème~17.2.5 Une partie A de E est convexe si et seulement si tout
barycentre à coefficients positifs d'une famille finie de points de A
est encore dans A.

Démonstration La condition est évidemment suffisante puisque tout point
du segment {[}a,b{]} est barycentre à coefficients positifs de a et b.
Montrons qu'elle est nécessaire en montrant par récurrence sur
\textbar{}I\textbar{} que si \{\textbackslash{}left
((\{a\}\_\{i\},\{λ\}\_\{i\})\textbackslash{}right )\}\_\{i∈I\} est une
famille finie de points massiques tels que \textbackslash{}mathop\{∀\}i,
\{a\}\_\{i\} ∈ A et \{λ\}\_\{i\} ≥ 0 avec
\{\textbackslash{}mathop\{\textbackslash{}mathop\{∑ \}\}
\}\_\{i∈I\}\{λ\}\_\{i\}\textbackslash{}mathrel\{≠\}0, alors le
barycentre g de la famille est encore dans A. Si \textbar{}I\textbar{} =
2, g est encore le barycentre de (a,\{ \{λ\}\_\{1\} \textbackslash{}over
\{λ\}\_\{1\}+\{λ\}\_\{2\}\} ) et de (a,\{ \{λ\}\_\{2\}
\textbackslash{}over \{λ\}\_\{1\}+\{λ\}\_\{2\}\} ), soit encore de (a,t)
et (b,1 − t) pour t =\{ \{λ\}\_\{1\} \textbackslash{}over
\{λ\}\_\{1\}+\{λ\}\_\{2\}\} ∈ {[}0,1{]}~; donc on a g ∈ {[}a,b{]} ⊂ A.
Supposons maintenant le résultat démontré pour toute famille de cardinal
n − 1 et soit \textbar{}I\textbar{} = n. Soit \{i\}\_\{0\} ∈ I et
supposons que \{\textbackslash{}mathop\{\textbackslash{}mathop\{∑ \}\}
\}\_\{i∈I∖\textbackslash{}\{\{i\}\_\{0\}\textbackslash{}\}\}\{λ\}\_\{i\}\textbackslash{}mathrel\{≠\}0~;
soit g' le barycentre de la famille \{\textbackslash{}left
((\{a\}\_\{i\},\{λ\}\_\{i\})\textbackslash{}right
)\}\_\{i∈I∖\textbackslash{}\{\{i\}\_\{0\}\textbackslash{}\}\}~; d'après
l'hypothèse de récurrence, g' ∈ A. Mais le théorème d'associativité des
barycentres assure que g est le barycentre de
(g',\{\textbackslash{}mathop\{\textbackslash{}mathop\{∑ \}\}
\}\_\{i∈I∖\textbackslash{}\{\{i\}\_\{0\}\textbackslash{}\}\}\{λ\}\_\{i\})
et de (\{a\}\_\{\{i\}\_\{0\}\},\{λ\}\_\{\{i\}\_\{0\}\})~; donc, d'après
le cas n = 2, g ∈ {[}g',\{a\}\_\{\{i\}\_\{0\}\}{]} ⊂ A. Si par contre
\{\textbackslash{}mathop\{\textbackslash{}mathop\{∑ \}\}
\}\_\{i∈I∖\textbackslash{}\{\{i\}\_\{0\}\textbackslash{}\}\}\{λ\}\_\{i\}
= 0, comme les \{λ\}\_\{i\} sont positifs on a
\textbackslash{}mathop\{∀\}i ∈ I
∖\textbackslash{}\{\{i\}\_\{0\}\textbackslash{}\}, \{λ\}\_\{i\} = 0 et g
n'est autre que \{a\}\_\{\{i\}\_\{0\}\} ∈ A. Cela achève la
démonstration.

Remarque~17.2.4 Il est clair que toute intersection de parties convexes
est encore convexe. On en déduit qu'étant donnée une partie A de E,
l'intersection de tous les convexes contenant A est encore une partie
convexe contenant A, et qu'elle est contenue dans toute partie convexe
contenant A. Nous l'appellerons l'enveloppe convexe de A et la noterons
\textbackslash{}hat\{A\}.

Théorème~17.2.6 L'enveloppe convexe \textbackslash{}hat\{A\} de A est
l'ensemble des barycentres à coefficients positifs de points de A.

Démonstration Soit B l'ensemble des barycentres à coefficients positifs
de points de A. Comme \textbackslash{}hat\{A\} est convexe, elle doit
contenir d'après le théorème précédent, tout barycentre à coefficients
positifs de points de \textbackslash{}hat\{A\} et en particulier de
points de A, soit B ⊂\textbackslash{}hat\{ A\}. Mais B est convexe car
tout barycentre à coefficients positifs de points de B qui sont eux
mêmes des barycentres à coefficients positifs de points de A est,
d'après le théorème d'associativité des barycentres, un barycentre à
coefficients positifs de points de A, donc est dans B~; comme B contient
évidemment A et que \textbackslash{}hat\{A\} est contenue dans tout
convexe contenant A, on a \textbackslash{}hat\{A\} ⊂ B et en définitive
B =\textbackslash{}hat\{ A\}.

Le théorème suivant précise ce résultat en dimension finie

Théorème~17.2.7 (Carathéodory). Soit E un ℝ- espace affine de dimension
n. Alors l'enveloppe convexe \textbackslash{}hat\{A\} de A est
l'ensemble des barycentres à coefficients positifs des familles de
points de A de cardinal n + 1.

Démonstration Soit \{\textbackslash{}left
((\{a\}\_\{i\},\{λ\}\_\{i\})\textbackslash{}right )\}\_\{1≤i≤p\} une
famille de points massiques avec \textbackslash{}mathop\{∀\}i,
\{a\}\_\{i\} ∈ A et \{λ\}\_\{i\} ≥ 0. Si p ≤ n + 1, il est évidemment
toujours possible de compléter la famille en une famille de cardinal n +
1 avec des poids nuls. Si p \textgreater{} n + 1, nous allons montrer
que le barycentre g de la famille est aussi le barycentre d'une famille
\{\textbackslash{}left ((\{a\}\_\{i\},\{μ\}\_\{i\})\textbackslash{}right
)\}\_\{1≤i≤p, i\textbackslash{}mathrel\{≠\}\{i\}\_\{0\}\} avec les
\{μ\}\_\{i\} ≥ 0. Pour cela remarquons que la famille
\{(\textbackslash{}overrightarrow\{\{a\}\_\{p\}\{a\}\_\{i\}\})\}\_\{1≤i≤p−1\}
est une famille de p − 1 \textgreater{} n vecteurs dans un espace de
dimension n~; elle est donc liée. En conséquence, il existe
\{α\}\_\{1\},\textbackslash{}mathop\{\textbackslash{}mathop\{\ldots{}\}\},\{α\}\_\{p−1\}
non tous nuls tels que
\{\textbackslash{}mathop\{\textbackslash{}mathop\{∑ \}\}
\}\_\{i=1\}\^{}\{p−1\}\{α\}\_\{i\}\textbackslash{}overrightarrow\{\{a\}\_\{p\}\{a\}\_\{i\}\}
= 0~; en posant \{α\}\_\{p\} =
−\{\textbackslash{}mathop\{\textbackslash{}mathop\{∑ \}\}
\}\_\{i=1\}\^{}\{p−1\}\{α\}\_\{i\} et en écrivant
\textbackslash{}overrightarrow\{\{a\}\_\{p\}\{a\}\_\{i\}\}
=\textbackslash{}overrightarrow\{ g\{a\}\_\{i\}\}
−\textbackslash{}overrightarrow\{ g\{a\}\_\{p\}\}, on obtient
\{\textbackslash{}mathop\{\textbackslash{}mathop\{∑ \}\}
\}\_\{i=1\}\^{}\{p\}\{α\}\_\{i\}\textbackslash{}overrightarrow\{g\{a\}\_\{i\}\}
=\textbackslash{}overrightarrow\{ 0\} avec
\{\textbackslash{}mathop\{\textbackslash{}mathop\{∑ \}\}
\}\_\{i\}\{α\}\_\{i\} = 0, et les \{α\}\_\{i\} non tous nuls~; en
particulier l'un au moins des \{α\}\_\{i\} est strictement positif. Par
définition, on a \{\textbackslash{}mathop\{\textbackslash{}mathop\{∑
\}\}
\}\_\{i=1\}\^{}\{p\}\{λ\}\_\{i\}\textbackslash{}overrightarrow\{g\{a\}\_\{i\}\}
=\textbackslash{}overrightarrow\{ 0\}. On en déduit que pour tout réel
t, on a

\{\textbackslash{}mathop\{∑ \}\}\_\{i=1\}\^{}\{p\}(\{λ\}\_\{ i\} −
t\{α\}\_\{i\})\textbackslash{}overrightarrow\{g\{a\}\_\{i\}\}
=\textbackslash{}overrightarrow\{ 0\}

Prenons alors t =\textbackslash{}mathop\{ min\}\textbackslash{}\{\{
\{λ\}\_\{i\} \textbackslash{}over \{α\}\_\{i\}\}
\textbackslash{}mathrel\{∣\}\{α\}\_\{i\} \textgreater{}
0\textbackslash{}\} et soit \{μ\}\_\{i\} = \{λ\}\_\{i\} − t\{α\}\_\{i\}.
On a t ≥ 0~; comme \{λ\}\_\{i\} ≥ 0 deux cas sont possibles~:

\begin{itemize}
\itemsep1pt\parskip0pt\parsep0pt
\item
  si \{α\}\_\{i\} ≤ 0, alors − t\{α\}\_\{i\} ≥ 0 et donc \{μ\}\_\{i\} =
  \{λ\}\_\{i\} − t\{α\}\_\{i\} ≥ 0
\item
  si par contre \{α\}\_\{i\} \textgreater{} 0, alors t ≤\{ \{λ\}\_\{i\}
  \textbackslash{}over \{α\}\_\{i\}\} soit encore \{μ\}\_\{i\} =
  \{λ\}\_\{i\} − t\{α\}\_\{i\} ≥ 0.
\end{itemize}

On a donc \textbackslash{}mathop\{∀\}i ∈ {[}1,p{]}, \{μ\}\_\{i\} ≥ 0 et
\{\textbackslash{}mathop\{\textbackslash{}mathop\{∑ \}\}
\}\_\{i=1\}\^{}\{p\}\{μ\}\_\{i\}\textbackslash{}overrightarrow\{g\{a\}\_\{i\}\}
=\textbackslash{}overrightarrow\{ 0\}. Mais de plus t
=\textbackslash{}mathop\{ min\}\textbackslash{}\{\{ \{λ\}\_\{i\}
\textbackslash{}over \{α\}\_\{i\}\}
\textbackslash{}mathrel\{∣\}\{α\}\_\{i\} \textgreater{}
0\textbackslash{}\} =\{ \{λ\}\_\{\{i\}\_\{0\}\} \textbackslash{}over
\{α\}\_\{\{i\}\_\{0\}\}\} , ce qui nous donne \{μ\}\_\{\{i\}\_\{0\}\} =
0. Donc g est encore le barycentre de la famille \{\textbackslash{}left
((\{a\}\_\{i\},\{μ\}\_\{i\})\textbackslash{}right )\}\_\{1≤i≤p,
i\textbackslash{}mathrel\{≠\}\{i\}\_\{0\}\} avec les \{μ\}\_\{i\} ≥ 0~;
ce qui achève la démonstration~: tant que le cardinal de la famille est
supérieur ou égal à n + 2 on peut retirer un point de la famille, donc
on finit par aboutir à une famille de cardinal n + 1.

{[}\href{coursse94.html}{next}{]} {[}\href{coursse92.html}{prev}{]}
{[}\href{coursse92.html\#tailcoursse92.html}{prev-tail}{]}
{[}\href{coursse93.html}{front}{]}
{[}\href{coursch18.html\#coursse93.html}{up}{]}

\end{document}
