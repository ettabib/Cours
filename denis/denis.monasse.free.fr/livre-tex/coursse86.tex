\documentclass[]{article}
\usepackage[T1]{fontenc}
\usepackage{lmodern}
\usepackage{amssymb,amsmath}
\usepackage{ifxetex,ifluatex}
\usepackage{fixltx2e} % provides \textsubscript
% use upquote if available, for straight quotes in verbatim environments
\IfFileExists{upquote.sty}{\usepackage{upquote}}{}
\ifnum 0\ifxetex 1\fi\ifluatex 1\fi=0 % if pdftex
  \usepackage[utf8]{inputenc}
\else % if luatex or xelatex
  \ifxetex
    \usepackage{mathspec}
    \usepackage{xltxtra,xunicode}
  \else
    \usepackage{fontspec}
  \fi
  \defaultfontfeatures{Mapping=tex-text,Scale=MatchLowercase}
  \newcommand{\euro}{€}
\fi
% use microtype if available
\IfFileExists{microtype.sty}{\usepackage{microtype}}{}
\ifxetex
  \usepackage[setpagesize=false, % page size defined by xetex
              unicode=false, % unicode breaks when used with xetex
              xetex]{hyperref}
\else
  \usepackage[unicode=true]{hyperref}
\fi
\hypersetup{breaklinks=true,
            bookmarks=true,
            pdfauthor={},
            pdftitle={Notions generales},
            colorlinks=true,
            citecolor=blue,
            urlcolor=blue,
            linkcolor=magenta,
            pdfborder={0 0 0}}
\urlstyle{same}  % don't use monospace font for urls
\setlength{\parindent}{0pt}
\setlength{\parskip}{6pt plus 2pt minus 1pt}
\setlength{\emergencystretch}{3em}  % prevent overfull lines
\setcounter{secnumdepth}{0}
 
/* start css.sty */
.cmr-5{font-size:50%;}
.cmr-7{font-size:70%;}
.cmmi-5{font-size:50%;font-style: italic;}
.cmmi-7{font-size:70%;font-style: italic;}
.cmmi-10{font-style: italic;}
.cmsy-5{font-size:50%;}
.cmsy-7{font-size:70%;}
.cmex-7{font-size:70%;}
.cmex-7x-x-71{font-size:49%;}
.msbm-7{font-size:70%;}
.cmtt-10{font-family: monospace;}
.cmti-10{ font-style: italic;}
.cmbx-10{ font-weight: bold;}
.cmr-17x-x-120{font-size:204%;}
.cmsl-10{font-style: oblique;}
.cmti-7x-x-71{font-size:49%; font-style: italic;}
.cmbxti-10{ font-weight: bold; font-style: italic;}
p.noindent { text-indent: 0em }
td p.noindent { text-indent: 0em; margin-top:0em; }
p.nopar { text-indent: 0em; }
p.indent{ text-indent: 1.5em }
@media print {div.crosslinks {visibility:hidden;}}
a img { border-top: 0; border-left: 0; border-right: 0; }
center { margin-top:1em; margin-bottom:1em; }
td center { margin-top:0em; margin-bottom:0em; }
.Canvas { position:relative; }
li p.indent { text-indent: 0em }
.enumerate1 {list-style-type:decimal;}
.enumerate2 {list-style-type:lower-alpha;}
.enumerate3 {list-style-type:lower-roman;}
.enumerate4 {list-style-type:upper-alpha;}
div.newtheorem { margin-bottom: 2em; margin-top: 2em;}
.obeylines-h,.obeylines-v {white-space: nowrap; }
div.obeylines-v p { margin-top:0; margin-bottom:0; }
.overline{ text-decoration:overline; }
.overline img{ border-top: 1px solid black; }
td.displaylines {text-align:center; white-space:nowrap;}
.centerline {text-align:center;}
.rightline {text-align:right;}
div.verbatim {font-family: monospace; white-space: nowrap; text-align:left; clear:both; }
.fbox {padding-left:3.0pt; padding-right:3.0pt; text-indent:0pt; border:solid black 0.4pt; }
div.fbox {display:table}
div.center div.fbox {text-align:center; clear:both; padding-left:3.0pt; padding-right:3.0pt; text-indent:0pt; border:solid black 0.4pt; }
div.minipage{width:100%;}
div.center, div.center div.center {text-align: center; margin-left:1em; margin-right:1em;}
div.center div {text-align: left;}
div.flushright, div.flushright div.flushright {text-align: right;}
div.flushright div {text-align: left;}
div.flushleft {text-align: left;}
.underline{ text-decoration:underline; }
.underline img{ border-bottom: 1px solid black; margin-bottom:1pt; }
.framebox-c, .framebox-l, .framebox-r { padding-left:3.0pt; padding-right:3.0pt; text-indent:0pt; border:solid black 0.4pt; }
.framebox-c {text-align:center;}
.framebox-l {text-align:left;}
.framebox-r {text-align:right;}
span.thank-mark{ vertical-align: super }
span.footnote-mark sup.textsuperscript, span.footnote-mark a sup.textsuperscript{ font-size:80%; }
div.tabular, div.center div.tabular {text-align: center; margin-top:0.5em; margin-bottom:0.5em; }
table.tabular td p{margin-top:0em;}
table.tabular {margin-left: auto; margin-right: auto;}
div.td00{ margin-left:0pt; margin-right:0pt; }
div.td01{ margin-left:0pt; margin-right:5pt; }
div.td10{ margin-left:5pt; margin-right:0pt; }
div.td11{ margin-left:5pt; margin-right:5pt; }
table[rules] {border-left:solid black 0.4pt; border-right:solid black 0.4pt; }
td.td00{ padding-left:0pt; padding-right:0pt; }
td.td01{ padding-left:0pt; padding-right:5pt; }
td.td10{ padding-left:5pt; padding-right:0pt; }
td.td11{ padding-left:5pt; padding-right:5pt; }
table[rules] {border-left:solid black 0.4pt; border-right:solid black 0.4pt; }
.hline hr, .cline hr{ height : 1px; margin:0px; }
.tabbing-right {text-align:right;}
span.TEX {letter-spacing: -0.125em; }
span.TEX span.E{ position:relative;top:0.5ex;left:-0.0417em;}
a span.TEX span.E {text-decoration: none; }
span.LATEX span.A{ position:relative; top:-0.5ex; left:-0.4em; font-size:85%;}
span.LATEX span.TEX{ position:relative; left: -0.4em; }
div.float img, div.float .caption {text-align:center;}
div.figure img, div.figure .caption {text-align:center;}
.marginpar {width:20%; float:right; text-align:left; margin-left:auto; margin-top:0.5em; font-size:85%; text-decoration:underline;}
.marginpar p{margin-top:0.4em; margin-bottom:0.4em;}
.equation td{text-align:center; vertical-align:middle; }
td.eq-no{ width:5%; }
table.equation { width:100%; } 
div.math-display, div.par-math-display{text-align:center;}
math .texttt { font-family: monospace; }
math .textit { font-style: italic; }
math .textsl { font-style: oblique; }
math .textsf { font-family: sans-serif; }
math .textbf { font-weight: bold; }
.partToc a, .partToc, .likepartToc a, .likepartToc {line-height: 200%; font-weight:bold; font-size:110%;}
.chapterToc a, .chapterToc, .likechapterToc a, .likechapterToc, .appendixToc a, .appendixToc {line-height: 200%; font-weight:bold;}
.index-item, .index-subitem, .index-subsubitem {display:block}
.caption td.id{font-weight: bold; white-space: nowrap; }
table.caption {text-align:center;}
h1.partHead{text-align: center}
p.bibitem { text-indent: -2em; margin-left: 2em; margin-top:0.6em; margin-bottom:0.6em; }
p.bibitem-p { text-indent: 0em; margin-left: 2em; margin-top:0.6em; margin-bottom:0.6em; }
.paragraphHead, .likeparagraphHead { margin-top:2em; font-weight: bold;}
.subparagraphHead, .likesubparagraphHead { font-weight: bold;}
.quote {margin-bottom:0.25em; margin-top:0.25em; margin-left:1em; margin-right:1em; text-align:\\jmathmathustify;}
.verse{white-space:nowrap; margin-left:2em}
div.maketitle {text-align:center;}
h2.titleHead{text-align:center;}
div.maketitle{ margin-bottom: 2em; }
div.author, div.date {text-align:center;}
div.thanks{text-align:left; margin-left:10%; font-size:85%; font-style:italic; }
div.author{white-space: nowrap;}
.quotation {margin-bottom:0.25em; margin-top:0.25em; margin-left:1em; }
h1.partHead{text-align: center}
.sectionToc, .likesectionToc {margin-left:2em;}
.subsectionToc, .likesubsectionToc {margin-left:4em;}
.subsubsectionToc, .likesubsubsectionToc {margin-left:6em;}
.frenchb-nbsp{font-size:75%;}
.frenchb-thinspace{font-size:75%;}
.figure img.graphics {margin-left:10%;}
/* end css.sty */

\title{Notions generales}
\author{}
\date{}

\begin{document}
\maketitle

\textbf{Warning: 
requires JavaScript to process the mathematics on this page.\\ If your
browser supports JavaScript, be sure it is enabled.}

\begin{center}\rule{3in}{0.4pt}\end{center}

{[}
{[}{]}
{[}

\subsubsection{16.1 Notions générales}

\paragraph{16.1.1 Solutions d'une équation différentielle}

Définition~16.1.1 Soit E un espace vectoriel normé de dimension finie, n
≥ 1, W un ouvert de \mathbb{R}~ \times E^n+1 et G : W \rightarrow~ E' une application.
On appelle solution de l'équation différentielle
G(t,y,y',\\ldots,y^(n)~)
= 0 tout couple (I,\phi) d'un intervalle I de \mathbb{R}~ et d'une application \phi : I
\rightarrow~ E de classe C^n telle que

\begin{align*} \forall~~t \in I,&
&
(t,\phi(t),\phi'(t),\\ldots,\phi^(n)~(t))
\in W \%& \\ & & \text
et
G(t,\phi(t),\phi'(t),\\ldots,\phi^(n)~(t))
= 0\%& \\
\end{align*}

On dira alors que n est l'ordre de l'équation différentielle.

Remarque~16.1.1 Dans le cas particulier où E' = E et où
G(t,y_0,\\ldots,y_n~)
= y_n -
F(t,y_0,\\ldots,y_n-1~)
avec F application d'un ouvert U de \mathbb{R}~ \times E^n dans E, on
obtient une équation différentielle de la forme y^(n) =
F(t,y,y',\\ldots,y^(n-1)~).
On dira qu'une telle équation est sous forme normale. Une solution d'une
telle équation est donc un couple (I,\phi) où \phi : I \rightarrow~ E est de classe
C^n et vérifie

\begin{align*} \forall~~t \in I,&
&
(t,\phi(t),\\ldots,\phi^(n-1)~(t))
\in U \%& \\ & & \text
et \phi^(n)(t) =
F(t,\phi(t),\phi'(t),\\ldots,\phi^(n-1)~(t))\%&
\\ \end{align*}

Par la suite on s'intéressera plus particulièrement aux équations
différentielles sous forme normale. Il est parfois possible de passer
d'une équation sous forme générale à une équation sous forme normale en
résolvant l'équation
G(t,y_0,\\ldots,y_n~)
= 0 sous la forme y_n =
F(t,y_0,\\ldots,y_n-1~).
A cet égard, le théorème des fonctions implicites peut rendre de grands
services.

\paragraph{16.1.2 Type de problèmes}

Nous distinguerons par la suite deux types de problèmes concernant les
équations différentielles. Le premier type de problème est appelé le
problème à condition initiale (ou problème de Cauchy-Lipschitz), le
second problème à conditions aux limites (ou conditions au bord).

Problème 1 (à conditions initiales). On considère une équation
différentielle sous forme normale y^(n) =
F(t,y,y',\\ldots,y^(n-1)~)
où F est une application d'un ouvert U de \mathbb{R}~ \times E^n dans E et
on se donne
(t_0,y_0,\\ldots,y_n-1~)
\in U. Peut-on trouver une solution (I,\phi) de cette équation différentielle
telle que t_0 \in I, \phi(t_0) =
y_0,\\ldots,\phi^(n-1)(t_0~)
= y_n-1~? Si oui, a-t-on en un certain sens unicité d'une
solution~?

Problème 2 (à conditions aux limites). On considère une équation
différentielle sous forme normale y^(n) =
F(t,y,y',\\ldots,y^(n-1)~)
où F est une application d'un ouvert U de \mathbb{R}~ \times E^n dans E et
on se donne a et b \in \mathbb{R}~. Peut-on trouver une solution (I,\phi) de cette
équation différentielle vérifiant des équations
G_1(a,\phi(a),\\ldots,\phi^(n-1)~(a))
= 0 et
G_2(b,\phi(b),\\ldots,\phi^(n-1)~(b))
= 0, où G_1 et G_2 sont deux applications de \mathbb{R}~ \times
E^n respectivement dans deux espaces vectoriels normés
E_1 et E_2.

Le problème avec conditions aux limites est beaucoup plus difficile et a
des réponses beaucoup plus complexes que le problème avec conditions
initiales. Nous ne le traiterons donc pas en dehors d'exercices
particuliers et nous intéresserons presque exclusivement au problème à
conditions initiales.

\paragraph{16.1.3 Réduction à l'ordre 1}

Considérons une équation différentielle sous forme normale
y^(n) =
f(t,y,y',\\ldots,y^(n-1)~)
où f est une application d'un ouvert U de \mathbb{R}~ \times E^n dans E et
soit (I,\phi) une solution de l'équation différentielle. Posons
\phi_1(t) =
\phi(t),\\ldots,\phi_n~(t)
= \phi^(n-1)(t) et \Phi(t) =
(\phi_1(t),\\ldots,\phi_n~(t))
=
(\phi(t),\phi'(t),\\ldots,\phi^(n-1)~(t)).
Alors (I,\Phi) est de classe \mathcal{C}^1 et on a

\begin{align*} \Phi'(t)& =& \left
(\phi'(t),\\ldots,\phi^(n)~(t)\right
) \%& \\ & =& \left
(\phi'(t),\\ldots,\phi^(n-1)(t),f(t,\phi(t),\\\ldots,\phi^(n-1)~(t))\right
) \%& \\ & =& \left
(\phi_2(t),\\ldots,\phi_n(t),f(t,\phi_1(t),\\\ldots,\phi_n~(t))\right
) = F(t,\Phi(t))\%& \\
\end{align*}

si l'on définit F : U \rightarrow~ E^n par
F(t,(y_1,\\ldots,y_n~))
=
(y_2,\\ldots,y_n,F(t,y_1,\\\ldots,y_n~)).
Donc (I,\Phi) est solution de l'équation différentielle Y ' = F(t,Y ).

Inversement, donnons nous une solution (I,\Phi) de l'équation
différentielle Y ' = F(t,Y ) où F : U \rightarrow~ E^n est définie par
F(t,(y_1,\\ldots,y_n~))
=
(y_2,\\ldots,y_n,f(t,y_1,\\\ldots,y_n~)).
Posons \Phi(t) =
(\phi_1(t),\\ldots,\phi_n~(t)).
On a donc

\begin{align*} \Phi'(t)& =&
(\phi_1'(t),\\ldots,\phi_n-1'(t),\phi_n~'(t))
\%& \\ & =& F(t,\Phi(t)) =
(\phi_2(t),\\ldots,\phi_n(t),f(t,\phi_1(t),\\\ldots,\phi_n~(t)))\%&
\\ \end{align*}

On en déduit que pour i \in {[}1,n - 1{]} on a \phi_i'(t) =
\phi_i+1(t) et une récurrence évidente montre que pour i \in
{[}2,n{]}, \phi_i(t) = \phi_1^(i-1)(t). Mais alors la
dernière équation se traduit par \phi_1^(n)(t) =
(\phi^(n-1))'(t) = \phi_n'(t) =
f(t,\phi_1(t),\\ldots,\phi_n~(t))
=
f(t,\phi_1(t),\phi_1'(t),\\ldots,\phi_1^(n-1)~(t)),
si bien que (I,\phi_1) est de classe C^n et solution de
l'équation différentielle y^(n) =
f(t,y,y',\\ldots,y^(n-1)~).
On en déduit donc le théorème suivant

Théorème~16.1.1 Soit U un ouvert de U \times E^n et f : U \rightarrow~ E.
Soit F : U \rightarrow~ E^n définie par
F(t,(y_1,\\ldots,y_n~))
=
(y_2,\\ldots,y_n,F(t,y_1,\\\ldots,y_n~)).
Alors l'application (I,\phi)\mapsto~(I,\Phi) définie par
\Phi(t) =
(\phi(t),\\ldots,\phi^(n-1)~(t))
est une bi\\jmathmathection de l'ensemble des solutions de l'équation
différentielle d'ordre n, y^(n) =
f(t,y,y',\\ldots,y^(n-1)~),
sur l'ensemble des solutions de l'équation différentielle d'ordre un Y '
= F(t,Y ).

Remarque~16.1.2 En ce qui concerne le type de problème étudié, il est
clair que cette bi\\jmathmathection préserve les problèmes à conditions initiales.
Autrement dit (I,\phi) est une solution de y^(n) =
f(t,y,y',\\ldots,y^(n-1)~)
vérifiant les conditions initiales \phi(t_0) =
y_0,\\ldots,\phi^(n-1)(t_0~)
= y_n-1 si et seulement si~(I,\Phi) est une solution de Y ' =
F(t,Y ) vérifiant la condition initiale \Phi(t_0) = Y _0
avec Y _0 =
(y_0,\\ldots,y_n-1~).
Nous pourrons donc par la suite, pour ce qui concerne les problèmes
d'existence et d'unicité du problème à conditions initiales, nous borner
à l'étude des équations différentielles d'ordre 1.

\paragraph{16.1.4 Equivalence avec une équation intégrale}

Théorème~16.1.2 Soit E un espace vectoriel normé de dimension finie, U
un ouvert de \mathbb{R}~ \times E, F : U \rightarrow~ E continue, (t_0,y_0) \in U,
I un intervalle de \mathbb{R}~ contenant t_0 et \phi une application
continue de I dans E. Alors on a équivalence de

\begin{itemize}
\itemsep1pt\parskip0pt\parsep0pt
\item
  (i) (I,\phi) est une solution de l'équation différentielle y' = F(t,y)
  vérifiant \phi(t_0) = y_0
\item
  (ii) \forall~t \in I, \phi(t) = y_0~
  +\int  _t_0^t~F(u,\phi(u))
  du.
\end{itemize}

Démonstration Supposons (i) vérifié. Alors, comme \phi est de classe
\mathcal{C}^1, on a

\phi(t) = \phi(t_0) +\int ~
_t_0^t\phi'(u) du = y_ 0
+\int  _t_0^t~F(u,\phi(u))
du

ce qui montre que (i) \rigtharrow~(ii). Inversement supposons (ii) vérifié. Il est
clair que \phi(t_0) = y_0. De plus comme
u\mapsto~F(u,\phi(u)) est continue,
t\mapsto~\int ~
_t_0^tF(u,\phi(u)) du est de classe \mathcal{C}^1
et sa dérivée est F(t,\phi(t))~; on en déduit que \phi est de classe
\mathcal{C}^1 et que \phi'(t) = F(t,\phi(t)), ce qui achève la démonstration.

\paragraph{16.1.5 Le lemme de Gronwall}

On utilisera par la suite le lemme suivant~:

Lemme~16.1.3 (Gronwall). Soit c un réel positif, g : {[}a,b{[}\rightarrow~ \mathbb{R}~
continue positive. Soit u : I \rightarrow~ \mathbb{R}~ continue telle que
\forall~~x \in {[}a,b{[}, u(x)\leq c
+\int ~
_a^xu(t)g(t) dt. Alors
\forall~~x \in {[}a,b{[}, u(x)\leq
cexp (\\int ~
_a^xg(t) dt).

Démonstration Posons v(x) = c +\int ~
_a^xu(t)g(t) dt. Comme u et g sont
continues, v est de classe \mathcal{C}^1 et on a v'(x) =
u(x)g(x) \leq v(x)g(x) puisque
u(x)\leq v(x) et g(x) ≥ 0 par hypothèse. Soit w(x) =
v(x)exp (-\\int ~
_a^xg(t) dt). On a alors

\begin{align*} w'(x)& =&
v'(x)exp (-\\int ~
_a^xg(t) dt) - v(x)g(x)exp~
(-\int  _a^x~g(t) dt)\%&
\\ & =& (v'(x) -
v(x)g(x))exp (-\\int ~
_a^xg(t) dt) \leq 0 \%& \\
\end{align*}

On en déduit que w est décroissante, et donc
\forall~~x \in {[}a,b{[}, w(x) \leq w(a). Mais w(a) = v(a)
= c. On a donc, pour x \in {[}a,b{[}, u(x)\leq v(x) \leq
cexp (\\int ~
_a^xg(t) dt), ce qu'on voulait démontrer.

Remarque~16.1.3 De la même fa\ccon on montre que si
\forall~~x \in{]}b,a{]}, u(x)\leq c
+\int ~
_x^au(t)g(t) dt, alors
\forall~~x \in{]}b,a{]}, u(x)\leq
cexp (\\int ~
_x^ag(t) dt).

Remarque~16.1.4 Le cas c = 0 \\jmathmathouera un rôle crucial dans les
démonstrations d'unicité. On obtient alors la nullité de u sur
l'intervalle en question.

{[}
{[}

\end{document}
