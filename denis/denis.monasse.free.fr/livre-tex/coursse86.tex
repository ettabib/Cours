\section{Notions générales}

\subsection{Notions de solutions d'une équation différentielle}
\label{subsec:notions-de-solutions-d-une-equation-differentielle}

\begin{de}
\label{def:solution-equation-differentielle}Soit (E) un espace
vectoriel normé de dimension finie, (n \ge 1), (W) un ouvert de
(\mathbb{R} \times E^{n+1}) et (G : W \rightarrow E') une
application. On appelle \emph{solution} de l'équation
différentielle
[G(t,y,y',\dotsc,y^{(n)}) = 0]tout couple ((I,\phi)) d'un
intervalle (I) de (\mathbb{R}) et d'une application
(\phi : I \rightarrow E) de classe (\mathcal{C}^n) telle que
\begin{align*}
\forall t \in I, \qquad (t,\phi(t),\phi'(t),\dotsc,\phi^{(n)}(t)) &\in W \
\text{et} \quad G(t,\phi(t),\phi'(t),\dotsc,\phi^{(n)}(t)) &= 0.
\end{align*}
On dira alors que (n) est l'\emph{ordre} de l'équation
différentielle.
\end{de}

\begin{rem}
Dans le cas particulier où (E' = E) et où
[G(t,y_0,\dotsc,y_n) = y_n - F(t,y_0,\dotsc,y_{n-1})]avec (F)
application d'un ouvert (U) de (\mathbb{R} \times E^n) dans
(E), on obtient une équation différentielle de la forme
[y^{(n)} = F(t,y,y',\dotsc,y^{(n-1)}).]On dira qu'une telle
équation est \emph{sous forme normale}. Une solution d'une telle
équation est donc un couple ((I,\phi)) où (\phi : I \rightarrow
E) est de classe (\mathcal{C}^n) et vérifie
\begin{align*}
\forall t \in I, \qquad (t,\phi(t),\dotsc,\phi^{(n-1)}(t)) &\in U \
\text{et} \quad \phi^{(n)}(t) &= F(t,\phi(t),\phi'(t),\dotsc,\phi^{(n-1)}(t)).
\end{align*}
Par la suite on s'intéressera plus particulièrement aux équations
différentielles sous forme normale. Il est parfois possible de passer
d'une équation sous forme générale à une équation sous forme normale
en résolvant l'équation
[G(t,y_0,\dotsc,y_n) = 0]sous la forme
(y_n = F(t,y_0,\dotsc,y_{n-1})). À cet égard, le théorème des
fonctions implicites peut rendre de grands services.
\end{rem}

\subsection{Types de problèmes}
\label{subsec:types-de-problemes}

Nous distinguerons par la suite deux types de problèmes concernant les
équations différentielles. Le premier type de problème est appelé le
\emph{problème à condition initiale} (ou problème de Cauchy-Lipschitz),
le second \emph{problème à conditions aux limites} (ou conditions au
bord).

\begin{description}
\item[Problème 1 (à conditions initiales).] On considère une équation
différentielle sous forme normale
[y^{(n)} = F(t,y,y',\dotsc,y^{(n-1)})]où (F) est une application
d'un ouvert (U) de (\mathbb{R} \times E^n) dans (E) et on se
donne ((t_0,y_0,\dotsc,y_{n-1}) \in U). Peut-on trouver une
solution ((I,\phi)) de cette équation différentielle telle que
(t_0 \in I), (\phi(t_0) = y_0), \dots,
(\phi^{(n-1)}(t_0) = y_{n-1}) ? Si oui, a-t-on en un certain sens
unicité d'une solution ?
\item[Problème 2 (à conditions aux limites).] On considère une équation
différentielle sous forme normale
[y^{(n)} = F(t,y,y',\dotsc,y^{(n-1)})]où (F) est une application
d'un ouvert (U) de (\mathbb{R} \times E^n) dans (E) et on se
donne (a) et (b \in \mathbb{R}). Peut-on trouver une solution
((I,\phi)) de cette équation différentielle vérifiant des équations
\begin{align*}
G_1(a,\phi(a),\dotsc,\phi^{(n-1)}(a)) &= 0 \
\text{et} \quad G_2(b,\phi(b),\dotsc,\phi^{(n-1)}(b)) &= 0,
\end{align*}
où (G_1) et (G_2) sont deux applications de
(\mathbb{R} \times E^n) respectivement dans deux espaces vectoriels
normés (E_1) et (E_2).
\end{description}

Le problème avec conditions aux limites est beaucoup plus difficile et a
des réponses beaucoup plus complexes que le problème avec conditions
initiales. Nous ne le traiterons donc pas en dehors d'exercices
particuliers et nous intéresserons presque exclusivement au problème à
conditions initiales.

\subsection{Réduction à l'ordre 1}
\label{subsec:reduction-a-l-ordre-1}

Considérons une équation différentielle sous forme normale
[y^{(n)} = f(t,y,y',\dotsc,y^{(n-1)})]où (f) est une application
d'un ouvert (U) de (\mathbb{R} \times E^n) dans (E) et soit
((I,\phi)) une solution de l'équation différentielle. Posons
(\phi_1(t) = \phi(t)), \dots, (\phi_n(t) =
\phi^{(n-1)}(t)) et (\Phi(t) =
(\phi_1(t),\dotsc,\phi_n(t)) =
(\phi(t),\phi'(t),\dotsc,\phi^{(n-1)}(t))). Alors ((I,\Phi)) est
de classe (\mathcal{C}^1) et on a
\begin{align*}
\Phi'(t) &= (\phi'(t),\dotsc,\phi^{(n)}(t)) \
&= (\phi'(t),\dotsc,\phi^{(n-1)}(t),f(t,\phi(t),\dotsc,\phi^{(n-1)}(t))) \
&= (\phi_2(t),\dotsc,\phi_n(t),f(t,\phi_1(t),\dotsc,\phi_n(t))) = F(t,\Phi(t))
\end{align*}
si l'on définit (F : U \rightarrow E^n) par
[F(t,(y_1,\dotsc,y_n)) = (y_2,\dotsc,y_n,F(t,y_1,\dotsc,y_n)).]Donc
((I,\Phi)) est solution de l'équation différentielle (Y' =
F(t,Y)).

Inversement, donnons nous une solution ((I,\Phi)) de l'équation
différentielle (Y' = F(t,Y)) où (F : U \rightarrow E^n) est
définie par
[F(t,(y_1,\dotsc,y_n)) = (y_2,\dotsc,y_n,f(t,y_1,\dotsc,y_n)).]Posons
(\Phi(t) = (\phi_1(t),\dotsc,\phi_n(t))). On a donc
\begin{align*}
\Phi'(t) &= (\phi_1'(t),\dotsc,\phi_{n-1}'(t),\phi_n'(t)) \
&= F(t,\Phi(t)) = (\phi_2(t),\dotsc,\phi_n(t),f(t,\phi_1(t),\dotsc,\phi_n(t))).
\end{align*}
On en déduit que pour (i \in [1,n-1]) on a
(\phi_i'(t) = \phi_{i+1}(t)) et une récurrence évidente montre que
pour (i \in [2,n]), (\phi_i(t) = \phi_1^{(i-1)}(t)). Mais alors
la dernière équation se traduit par
[\phi_1^{(n)}(t) = (\phi^{(n-1)})'(t) = \phi_n'(t) = f(t,\phi_1(t),\dotsc,\phi_n(t)) = f(t,\phi_1(t),\phi_1'(t),\dotsc,\phi_1^{(n-1)}(t)),]si
bien que ((I,\phi_1)) est de classe (\mathcal{C}^n) et solution
de l'équation différentielle
[y^{(n)} = f(t,y,y',\dotsc,y^{(n-1)}).]On en déduit donc le théorème
suivant :

\begin{thm}
Soit (U) un ouvert de (U \times E^n) et (f : U \rightarrow
E). Soit (F : U \rightarrow E^n) définie par
[F(t,(y_1,\dotsc,y_n)) = (y_2,\dotsc,y_n,F(t,y_1,\dotsc,y_n)).]Alors
l'application ((I,\phi) \mapsto (I,\Phi)) définie par
[\Phi(t) = (\phi(t),\dotsc,\phi^{(n-1)}(t))]est une bijection de
l'ensemble des solutions de l'équation différentielle d'ordre (n),
[y^{(n)} = f(t,y,y',\dotsc,y^{(n-1)}),]sur l'ensemble des
solutions de l'équation différentielle d'ordre un (Y' = F(t,Y)).
\end{thm}

\begin{rem}
En ce qui concerne le type de problème étudié, il est clair que cette
bijection préserve les problèmes à conditions initiales. Autrement dit
((I,\phi)) est une solution de
[y^{(n)} = f(t,y,y',\dotsc,y^{(n-1)})]vérifiant les conditions
initiales (\phi(t_0) = y_0), \dots,
(\phi^{(n-1)}(t_0) = y_{n-1}) si et seulement si ((I,\Phi)) est
une solution de (Y' = F(t,Y)) vérifiant la condition initiale
(\Phi(t_0) = Y_0) avec (Y_0 = (y_0,\dotsc,y_{n-1})). Nous
pourrons donc par la suite, pour ce qui concerne les problèmes
d'existence et d'unicité du problème à conditions initiales, nous
borner à l'étude des équations différentielles d'ordre 1.
\end{rem}

\subsection{Équivalence avec une équation
intégrale}
\label{subsec:equivalence-avec-une-equation-integrale}

\begin{thm}
Soit (E) un espace vectoriel normé de dimension finie, (U) un
ouvert de (\mathbb{R} \times E), (F : U \rightarrow E)
continue, ((t_0,y_0) \in U), (I) un intervalle de
(\mathbb{R}) contenant (t_0) et (\phi) une application
continue de (I) dans (E). Alors on a équivalence de
\begin{enumerate}
\item ((I,\phi)) est une solution de l'équation différentielle
(y' = F(t,y)) vérifiant (\phi(t_0) = y_0)
\item (\forall t \in I),
(\phi(t) = y_0 + \int_{t_0}^t F(u,\phi(u)) , du).
\end{enumerate}
\end{thm}

\begin{proof}
Supposons (1) vérifié. Alors, comme (\phi) est de classe
(\mathcal{C}^1), on a
[\phi(t) = \phi(t_0) + \int_{t_0}^t \phi'(u) , du = y_0 + \int_{t_0}^t F(u,\phi(u)) , du]ce
qui montre que (1) (\Rightarrow) (2). Inversement supposons (2)
vérifié. Il est clair que (\phi(t_0) = y_0). De plus comme
(u \mapsto F(u,\phi(u))) est continue,
(t \mapsto \int_{t_0}^t F(u,\phi(u)) , du) est de classe
(\mathcal{C}^1) et sa dérivée est (F(t,\phi(t))) ; on en déduit
que (\phi) est de classe (\mathcal{C}^1) et que
(\phi'(t) = F(t,\phi(t))), ce qui achève la démonstration.
\end{proof}

\subsection{Le lemme de Gronwall}
\label{subsec:le-lemme-de-gronwall}

On utilisera par la suite le lemme suivant :

\begin{lem}[Gronwall]
\label{lem:gronwall}Soit (c) un réel positif, (g : [a,b[
\rightarrow \mathbb{R}) continue positive. Soit (u : I
\rightarrow \mathbb{R}) continue telle que
[\forall x \in [a,b[, \quad u(x) \le c + \int_a^x u(t)g(t) , dt.]Alors
[\forall x \in [a,b[, \quad u(x) \le c \exp\left(\int_a^x g(t) , dt\right).]
\end{lem}

\begin{proof}
Posons (v(x) = c + \int_a^x u(t)g(t) , dt). Comme (u) et (g)
sont continues, (v) est de classe (\mathcal{C}^1) et on a
(v'(x) = u(x)g(x) \le v(x)g(x)) puisque (u(x) \le v(x)) et
(g(x) \ge 0) par hypothèse. Soit
(w(x) = v(x) \exp\left(-\int_a^x g(t) , dt\right)). On a alors
\begin{align*}
w'(x) &= v'(x) \exp\left(-\int_a^x g(t) , dt\right) - v(x)g(x) \exp\left(-\int_a^x g(t) , dt\right) \
&= (v'(x) - v(x)g(x)) \exp\left(-\int_a^x g(t) , dt\right) \le 0.
\end{align*}
On en déduit que (w) est décroissante, et donc
[\forall x \in [a,b[, \quad w(x) \le w(a).]Mais (w(a) = v(a) =
c). On a donc, pour (x \in [a,b[),
(u(x) \le v(x) \le c \exp\left(\int_a^x g(t) , dt\right)), ce
qu'on voulait démontrer.
\end{proof}

\begin{rem}
De la même façon on montre que si
[\forall x \in ]b,a], \quad u(x) \le c + \int_x^a u(t)g(t) , dt,]alors
[\forall x \in ]b,a], \quad u(x) \le c \exp\left(\int_x^a g(t) , dt\right).]
\end{rem}

\begin{rem}
Le cas (c = 0) jouera un rôle crucial dans les démonstrations
d'unicité. On obtient alors la nullité de (u) sur l'intervalle en
question.
\end{rem}