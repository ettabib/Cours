\documentclass[]{article}
\usepackage[T1]{fontenc}
\usepackage{lmodern}
\usepackage{amssymb,amsmath}
\usepackage{ifxetex,ifluatex}
\usepackage{fixltx2e} % provides \textsubscript
% use upquote if available, for straight quotes in verbatim environments
\IfFileExists{upquote.sty}{\usepackage{upquote}}{}
\ifnum 0\ifxetex 1\fi\ifluatex 1\fi=0 % if pdftex
  \usepackage[utf8]{inputenc}
\else % if luatex or xelatex
  \ifxetex
    \usepackage{mathspec}
    \usepackage{xltxtra,xunicode}
  \else
    \usepackage{fontspec}
  \fi
  \defaultfontfeatures{Mapping=tex-text,Scale=MatchLowercase}
  \newcommand{\euro}{€}
\fi
% use microtype if available
\IfFileExists{microtype.sty}{\usepackage{microtype}}{}
\ifxetex
  \usepackage[setpagesize=false, % page size defined by xetex
              unicode=false, % unicode breaks when used with xetex
              xetex]{hyperref}
\else
  \usepackage[unicode=true]{hyperref}
\fi
\hypersetup{breaklinks=true,
            bookmarks=true,
            pdfauthor={},
            pdftitle={Integrales curvilignes},
            colorlinks=true,
            citecolor=blue,
            urlcolor=blue,
            linkcolor=magenta,
            pdfborder={0 0 0}}
\urlstyle{same}  % don't use monospace font for urls
\setlength{\parindent}{0pt}
\setlength{\parskip}{6pt plus 2pt minus 1pt}
\setlength{\emergencystretch}{3em}  % prevent overfull lines
\setcounter{secnumdepth}{0}
 
/* start css.sty */
.cmr-5{font-size:50%;}
.cmr-7{font-size:70%;}
.cmmi-5{font-size:50%;font-style: italic;}
.cmmi-7{font-size:70%;font-style: italic;}
.cmmi-10{font-style: italic;}
.cmsy-5{font-size:50%;}
.cmsy-7{font-size:70%;}
.cmex-7{font-size:70%;}
.cmex-7x-x-71{font-size:49%;}
.msbm-7{font-size:70%;}
.cmtt-10{font-family: monospace;}
.cmti-10{ font-style: italic;}
.cmbx-10{ font-weight: bold;}
.cmr-17x-x-120{font-size:204%;}
.cmsl-10{font-style: oblique;}
.cmti-7x-x-71{font-size:49%; font-style: italic;}
.cmbxti-10{ font-weight: bold; font-style: italic;}
p.noindent { text-indent: 0em }
td p.noindent { text-indent: 0em; margin-top:0em; }
p.nopar { text-indent: 0em; }
p.indent{ text-indent: 1.5em }
@media print {div.crosslinks {visibility:hidden;}}
a img { border-top: 0; border-left: 0; border-right: 0; }
center { margin-top:1em; margin-bottom:1em; }
td center { margin-top:0em; margin-bottom:0em; }
.Canvas { position:relative; }
li p.indent { text-indent: 0em }
.enumerate1 {list-style-type:decimal;}
.enumerate2 {list-style-type:lower-alpha;}
.enumerate3 {list-style-type:lower-roman;}
.enumerate4 {list-style-type:upper-alpha;}
div.newtheorem { margin-bottom: 2em; margin-top: 2em;}
.obeylines-h,.obeylines-v {white-space: nowrap; }
div.obeylines-v p { margin-top:0; margin-bottom:0; }
.overline{ text-decoration:overline; }
.overline img{ border-top: 1px solid black; }
td.displaylines {text-align:center; white-space:nowrap;}
.centerline {text-align:center;}
.rightline {text-align:right;}
div.verbatim {font-family: monospace; white-space: nowrap; text-align:left; clear:both; }
.fbox {padding-left:3.0pt; padding-right:3.0pt; text-indent:0pt; border:solid black 0.4pt; }
div.fbox {display:table}
div.center div.fbox {text-align:center; clear:both; padding-left:3.0pt; padding-right:3.0pt; text-indent:0pt; border:solid black 0.4pt; }
div.minipage{width:100%;}
div.center, div.center div.center {text-align: center; margin-left:1em; margin-right:1em;}
div.center div {text-align: left;}
div.flushright, div.flushright div.flushright {text-align: right;}
div.flushright div {text-align: left;}
div.flushleft {text-align: left;}
.underline{ text-decoration:underline; }
.underline img{ border-bottom: 1px solid black; margin-bottom:1pt; }
.framebox-c, .framebox-l, .framebox-r { padding-left:3.0pt; padding-right:3.0pt; text-indent:0pt; border:solid black 0.4pt; }
.framebox-c {text-align:center;}
.framebox-l {text-align:left;}
.framebox-r {text-align:right;}
span.thank-mark{ vertical-align: super }
span.footnote-mark sup.textsuperscript, span.footnote-mark a sup.textsuperscript{ font-size:80%; }
div.tabular, div.center div.tabular {text-align: center; margin-top:0.5em; margin-bottom:0.5em; }
table.tabular td p{margin-top:0em;}
table.tabular {margin-left: auto; margin-right: auto;}
div.td00{ margin-left:0pt; margin-right:0pt; }
div.td01{ margin-left:0pt; margin-right:5pt; }
div.td10{ margin-left:5pt; margin-right:0pt; }
div.td11{ margin-left:5pt; margin-right:5pt; }
table[rules] {border-left:solid black 0.4pt; border-right:solid black 0.4pt; }
td.td00{ padding-left:0pt; padding-right:0pt; }
td.td01{ padding-left:0pt; padding-right:5pt; }
td.td10{ padding-left:5pt; padding-right:0pt; }
td.td11{ padding-left:5pt; padding-right:5pt; }
table[rules] {border-left:solid black 0.4pt; border-right:solid black 0.4pt; }
.hline hr, .cline hr{ height : 1px; margin:0px; }
.tabbing-right {text-align:right;}
span.TEX {letter-spacing: -0.125em; }
span.TEX span.E{ position:relative;top:0.5ex;left:-0.0417em;}
a span.TEX span.E {text-decoration: none; }
span.LATEX span.A{ position:relative; top:-0.5ex; left:-0.4em; font-size:85%;}
span.LATEX span.TEX{ position:relative; left: -0.4em; }
div.float img, div.float .caption {text-align:center;}
div.figure img, div.figure .caption {text-align:center;}
.marginpar {width:20%; float:right; text-align:left; margin-left:auto; margin-top:0.5em; font-size:85%; text-decoration:underline;}
.marginpar p{margin-top:0.4em; margin-bottom:0.4em;}
.equation td{text-align:center; vertical-align:middle; }
td.eq-no{ width:5%; }
table.equation { width:100%; } 
div.math-display, div.par-math-display{text-align:center;}
math .texttt { font-family: monospace; }
math .textit { font-style: italic; }
math .textsl { font-style: oblique; }
math .textsf { font-family: sans-serif; }
math .textbf { font-weight: bold; }
.partToc a, .partToc, .likepartToc a, .likepartToc {line-height: 200%; font-weight:bold; font-size:110%;}
.chapterToc a, .chapterToc, .likechapterToc a, .likechapterToc, .appendixToc a, .appendixToc {line-height: 200%; font-weight:bold;}
.index-item, .index-subitem, .index-subsubitem {display:block}
.caption td.id{font-weight: bold; white-space: nowrap; }
table.caption {text-align:center;}
h1.partHead{text-align: center}
p.bibitem { text-indent: -2em; margin-left: 2em; margin-top:0.6em; margin-bottom:0.6em; }
p.bibitem-p { text-indent: 0em; margin-left: 2em; margin-top:0.6em; margin-bottom:0.6em; }
.paragraphHead, .likeparagraphHead { margin-top:2em; font-weight: bold;}
.subparagraphHead, .likesubparagraphHead { font-weight: bold;}
.quote {margin-bottom:0.25em; margin-top:0.25em; margin-left:1em; margin-right:1em; text-align:justify;}
.verse{white-space:nowrap; margin-left:2em}
div.maketitle {text-align:center;}
h2.titleHead{text-align:center;}
div.maketitle{ margin-bottom: 2em; }
div.author, div.date {text-align:center;}
div.thanks{text-align:left; margin-left:10%; font-size:85%; font-style:italic; }
div.author{white-space: nowrap;}
.quotation {margin-bottom:0.25em; margin-top:0.25em; margin-left:1em; }
h1.partHead{text-align: center}
.sectionToc, .likesectionToc {margin-left:2em;}
.subsectionToc, .likesubsectionToc {margin-left:4em;}
.subsubsectionToc, .likesubsubsectionToc {margin-left:6em;}
.frenchb-nbsp{font-size:75%;}
.frenchb-thinspace{font-size:75%;}
.figure img.graphics {margin-left:10%;}
/* end css.sty */

\title{Integrales curvilignes}
\author{}
\date{}

\begin{document}
\maketitle

\textbf{Warning: \href{http://www.math.union.edu/locate/jsMath}{jsMath}
requires JavaScript to process the mathematics on this page.\\ If your
browser supports JavaScript, be sure it is enabled.}

\begin{center}\rule{3in}{0.4pt}\end{center}

{[}\href{coursse105.html}{next}{]}
{[}\hyperref[tailcoursse104.html]{tail}{]}
{[}\href{coursch21.html\#coursse104.html}{up}{]}

\subsubsection{20.1 Intégrales curvilignes}

\paragraph{20.1.1 Formes différentielles sur un arc paramétré}

Définition~20.1.1 Soit E un espace vectoriel normé, Γ = (I,f) un arc
paramétré de E de classe \{C\}\^{}\{1\}. On appelle forme différentielle
sur Γ toute forme différentielle α = a(t) dt sur l'intervalle I.

Exemple~20.1.1 Soit Γ = (I,f) un arc paramétré de E

\begin{itemize}
\itemsep1pt\parskip0pt\parsep0pt
\item
  (i) soit h une fonction définie sur l'image de Γ et à valeurs dans K~;
  on peut associer à h la forme différentielle h(m) ds définie par h(m)
  ds = h(f(t)) \textbackslash{}\textbar{}f'(t)\textbackslash{}\textbar{}
  dt (obtenue en rempla\textbackslash{}c\{c\}ant m par f(t) et ds par
  \textbackslash{}\textbar{}f'(t)\textbackslash{}\textbar{} dt, la
  différentielle de l'abscisse curviligne).
\item
  (ii) supposons que E est un espace euclidien et soit V un champ de
  vecteurs défini et continu sur l'image de Γ (c'est-à-dire une
  application de l'image de Γ dans E)~; on peut associer à V la forme
  différentielle (V (m)\textbackslash{}mathrel\{∣\}dm) définie par (V
  (m)\textbackslash{}mathrel\{∣\}dm) = \textbackslash{}left (V
  (f(t))\textbackslash{}mathrel\{∣\}f'(t)\textbackslash{}right ) dt
  (obtenue en rempla\textbackslash{}c\{c\}ant m par f(t) et donc dm par
  f'(t) dt).
\item
  (iii) supposons que E = \{ℝ\}\^{}\{n\}, que U est un ouvert de E
  contenant l'image de Γ et ω = \{a\}\_\{1\}(x) d\{x\}\_\{1\} +
  \textbackslash{}mathop\{\textbackslash{}mathop\{\ldots{}\}\} +
  \{a\}\_\{n\}(x) d\{x\}\_\{n\} une forme différentielle de degré 1
  continue sur U~; posons f(t) =
  (\{f\}\_\{1\}(t),\textbackslash{}mathop\{\textbackslash{}mathop\{\ldots{}\}\},\{f\}\_\{n\}(t))~;
  on peut considérer la forme différentielle restriction de ω à Γ
  définie par \{ω\}\_\{\{\textbar{}\}\_\{Γ\}\} = \textbackslash{}left
  (\{a\}\_\{1\}(f(t))\{f\}\_\{1\}'(t) +
  \textbackslash{}mathop\{\textbackslash{}mathop\{\ldots{}\}\} +
  \{a\}\_\{n\}(f(t))\{f\}\_\{n\}'(t)\textbackslash{}right ) dt (obtenue
  en rempla\textbackslash{}c\{c\}ant dans ω, x par f(t), \{x\}\_\{i\}
  par \{f\}\_\{i\}(t) et donc d\{x\}\_\{i\} par \{f\}\_\{i\}'(t) dt).
\end{itemize}

Remarque~20.1.1 En fait les cas (ii) et (iii) sont étroitement liés. En
effet, munissons \{ℝ\}\^{}\{n\} de sa structure euclidienne canonique et
soit U un ouvert contenant l'image de Γ. A tout champ de vecteurs V
défini sur U défini par V (x) = (\{V
\}\_\{1\}(x),\textbackslash{}mathop\{\textbackslash{}mathop\{\ldots{}\}\},\{V
\}\_\{n\}(x)), on peut associer la forme différentielle ω = \{V
\}\_\{1\}(x) d\{x\}\_\{1\} +
\textbackslash{}mathop\{\textbackslash{}mathop\{\ldots{}\}\} + \{V
\}\_\{n\}(x) d\{x\}\_\{n\}. Cette application V
\textbackslash{}mathrel\{↦\}ω est clairement bijective. On a alors dans
ce cadre

\textbackslash{}begin\{eqnarray*\} (V
(m)\textbackslash{}mathrel\{∣\}dm)\& =\& \textbackslash{}left (V
(f(t))\textbackslash{}mathrel\{∣\}f'(t)\textbackslash{}right ) dt \%\&
\textbackslash{}\textbackslash{} \& =\& \textbackslash{}left
(\{a\}\_\{1\}(f(t))\{f\}\_\{1\}'(t) +
\textbackslash{}mathop\{\textbackslash{}mathop\{\ldots{}\}\} +
\{a\}\_\{n\}(f(t))\{f\}\_\{n\}'(t)\textbackslash{}right ) dt\%\&
\textbackslash{}\textbackslash{} \& =\& \{ω\}\_\{\{\textbar{}\}\_\{Γ\}\}
\%\& \textbackslash{}\textbackslash{} \textbackslash{}end\{eqnarray*\}

Théorème~20.1.1 Les trois exemples fondamentaux sont invariants par
changement de paramétrage de sens direct.

Soit (I,f) et (J,g) deux arcs paramétrés équivalents et de même sens.
Soit θ : I → J un difféomorphisme croissant de classe \{C\}\^{}\{1\} tel
que f = g ∘ θ. Si α = a(t) dt et β = b(u) du sont les formes
différentielles obtenues respectivement sur (I,f) et (J,g) par l'une des
trois constructions ci dessus, on a a(t) dt = b(u) du pour u = θ(t).

Démonstration (i) Sur (I,f), on a h(m) ds = h(f(t))
\textbackslash{}\textbar{}f'(t)\textbackslash{}\textbar{} dt~; mais
comme f = g ∘ θ, on a

\textbackslash{}begin\{eqnarray*\} h(m) ds\& =\& h(g(θ(t)))
\textbackslash{}\textbar{}θ'(t)g'(θ(t))\textbackslash{}\textbar{} dt\%\&
\textbackslash{}\textbackslash{} \& =\& h(g(θ(t)))
\textbackslash{}\textbar{}g'(θ(t))\textbackslash{}\textbar{}θ'(t) dt\%\&
\textbackslash{}\textbackslash{} \textbackslash{}end\{eqnarray*\}

car θ'(t) \textgreater{} 0~; d'où encore, en posant u = θ(t) et donc du
= θ'(t) dt, h(m) ds = h(g(u))
\textbackslash{}\textbar{}g'(u)\textbackslash{}\textbar{} du ce qu'on
voulait démontrer.

(ii) Sur (I,f), on a

\textbackslash{}begin\{eqnarray*\} (V
(m)\textbackslash{}mathrel\{∣\}dm)\& =\& \textbackslash{}left (V
(f(t))\textbackslash{}mathrel\{∣\}f'(t)\textbackslash{}right ) dt \%\&
\textbackslash{}\textbackslash{} \& =\& \textbackslash{}left (V
(g(θ(t))\textbackslash{}mathrel\{∣\}θ'(t)g'(θ(t))\textbackslash{}right )
dt\%\& \textbackslash{}\textbackslash{} \& =\& \textbackslash{}left (V
(g(θ(t))\textbackslash{}mathrel\{∣\}g'(θ(t))\textbackslash{}right )θ'(t)
dt\%\& \textbackslash{}\textbackslash{} \& =\& \textbackslash{}left (V
(g(u))\textbackslash{}mathrel\{∣\}g'(u)\textbackslash{}right ) du \%\&
\textbackslash{}\textbackslash{} \textbackslash{}end\{eqnarray*\}

ce qu'on voulait démontrer.

(iii) On peut faire un calcul similaire ou utiliser le lien entre formes
différentielles et champ de vecteurs décrit ci dessus.

\paragraph{20.1.2 Intégrale d'une forme différentielle sur un arc}

Définition~20.1.2 Soit Γ = ({[}a,b{]},f) un arc paramétré et α = A(t) dt
une forme différentielle continue par morceaux sur Γ. On appelle
intégrale (curviligne) de la forme différentielle α sur Γ le scalaire

\{\textbackslash{}mathop\{∫ \} \}\_\{Γ\}α =\{\textbackslash{}mathop\{∫
\} \}\_\{a\}\^{}\{b\}A(t) dt

Remarque~20.1.2 Soit Γ = ({[}a,b{]},f) un arc paramétré et c ∈
{[}a,b{]}. On peut alors considérer les deux arcs paramétrés
\{Γ\}\_\{1\} = ({[}a,c{]},\{f\}\_\{\{\textbar{}\}\_\{{[}a,c{]}\}\}) et
\{Γ\}\_\{2\} = ({[}c,b{]},\{f\}\_\{\{\textbar{}\}\_\{{[}c,b{]}\}\}). On
dira alors que Γ est la juxtaposition de \{Γ\}\_\{1\} et \{Γ\}\_\{2\} et
on écrira Γ = \{Γ\}\_\{1\} ⊔ \{Γ\}\_\{2\}.

Proposition~20.1.2 (i) L'application
α\textbackslash{}mathrel\{↦\}\{\textbackslash{}mathop\{∫ \} \}\_\{Γ\}α
est linéaire (ii) On a \{\textbackslash{}mathop\{∫ \}
\}\_\{\{Γ\}\_\{1\}⊔\{Γ\}\_\{2\}\}α =\{\textbackslash{}mathop\{∫ \}
\}\_\{\{Γ\}\_\{1\}\}α +\{\textbackslash{}mathop\{∫ \}
\}\_\{\{Γ\}\_\{2\}\}α

Démonstration Résulte immédiatement des propriétés de l'intégrale.

Théorème~20.1.3 (invariance de l'intégrale curviligne). Soit
\{Γ\}\_\{1\} = ({[}a,b{]},f) un arc paramétré, \{Γ\}\_\{2\} =
({[}c,d{]},g) un arc paramétré équivalent et de même sens. Soit θ un
difféomorphisme croissant de {[}a,b{]} sur {[}c,d{]} tel que f = g ∘ θ.
Soit α = A(t) dt une forme différentielle sur \{Γ\}\_\{1\} et β = B(u)
du la forme différentielle qui s'en déduit en posant t = θ(u). Alors

\{\textbackslash{}mathop\{∫ \} \}\_\{\{Γ\}\_\{1\}\}α
=\{\textbackslash{}mathop\{∫ \} \}\_\{\{Γ\}\_\{2\}\}β

Démonstration Comme θ est croissant, on a nécessairement c = θ(a) et d =
θ(b). De plus la relation A(t) dt = B(u) du pour u = θ(t), montre que
A(t) = B(θ(t))θ'(t). On a donc

\textbackslash{}begin\{eqnarray*\} \{\textbackslash{}mathop\{∫ \}
\}\_\{\{Γ\}\_\{1\}\}α\& =\& \{\textbackslash{}mathop\{∫ \}
\}\_\{a\}\^{}\{b\}A(t) dt =\{\textbackslash{}mathop\{∫ \}
\}\_\{a\}\^{}\{b\}B(θ(t))θ'(t) dt\%\& \textbackslash{}\textbackslash{}
\& =\& \{\textbackslash{}mathop\{∫ \} \}\_\{θ(a)\}\^{}\{θ(b)\}B(u) du
=\{\textbackslash{}mathop\{∫ \} \}\_\{\{Γ\}\_\{2\}\}β \%\&
\textbackslash{}\textbackslash{} \textbackslash{}end\{eqnarray*\}

en utilisant le théorème de changement de variable dans les intégrales.

Exemple~20.1.2 Les résultats précédents, en liaison avec les définitions
du paragraphe précédent nous permettent d'associer à un arc paramétré Γ
= ({[}a,b{]},f) les trois types suivants d'intégrales, tous trois
invariants par changement de paramétrage admissible et croissant

\begin{itemize}
\item
  (i) soit h une fonction définie et continue sur l'image de Γ et à
  valeurs dans K~; on peut associer à h l'intégrale curviligne

  \{\textbackslash{}mathop\{∫ \} \}\_\{Γ\}h(m) ds
  =\{\textbackslash{}mathop\{∫ \} \}\_\{a\}\^{}\{b\}h(f(t))
  \textbackslash{}\textbar{}f'(t)\textbackslash{}\textbar{} dt

  (obtenue en rempla\textbackslash{}c\{c\}ant m par f(t) et ds par
  \textbackslash{}\textbar{}f'(t)\textbackslash{}\textbar{} dt,
  différentielle de l'abscisse curviligne).
\item
  (ii) supposons que E est un espace euclidien et soit V un champ de
  vecteurs défini sur l'image de Γ (c'est-à-dire une application de
  l'image de Γ dans E)~; on peut associer à V l'intégrale curviligne
  (appelée circulation du champ de vecteurs V le long de Γ)

  \{\textbackslash{}mathop\{∫ \} \}\_\{Γ\}(V
  (m)\textbackslash{}mathrel\{∣\}dm) =\{\textbackslash{}mathop\{∫ \}
  \}\_\{a\}\^{}\{b\}\textbackslash{}left (V
  (f(t))\textbackslash{}mathrel\{∣\}f'(t)\textbackslash{}right ) dt

  (obtenue en rempla\textbackslash{}c\{c\}ant m par f(t) et donc dm par
  f'(t) dt).
\item
  (iii) supposons que E = \{ℝ\}\^{}\{n\}, que U est un ouvert de E
  contenant l'image de Γ et ω = \{a\}\_\{1\}(x) d\{x\}\_\{1\} +
  \textbackslash{}mathop\{\textbackslash{}mathop\{\ldots{}\}\} +
  \{a\}\_\{n\}(x) d\{x\}\_\{n\} une forme différentielle de degré 1
  continue sur U~; posons f(t) =
  (\{f\}\_\{1\}(t),\textbackslash{}mathop\{\textbackslash{}mathop\{\ldots{}\}\},\{f\}\_\{n\}(t))~;
  on peut considérer l'intégrale curviligne

  \textbackslash{}begin\{eqnarray*\} \{\textbackslash{}mathop\{∫ \}
  \}\_\{Γ\}\{a\}\_\{1\}(x) d\{x\}\_\{1\} +
  \textbackslash{}mathop\{\textbackslash{}mathop\{\ldots{}\}\} +
  \{a\}\_\{n\}(x) d\{x\}\_\{n\}\& \& \%\&
  \textbackslash{}\textbackslash{} \& =\& \{\textbackslash{}mathop\{∫ \}
  \}\_\{a\}\^{}\{b\}\textbackslash{}left (\{a\}\_\{
  1\}(f(t))\{f\}\_\{1\}'(t) +
  \textbackslash{}mathop\{\textbackslash{}mathop\{\ldots{}\}\} +
  \{a\}\_\{n\}(f(t))\{f\}\_\{n\}'(t)\textbackslash{}right ) dt\%\&
  \textbackslash{}\textbackslash{} \textbackslash{}end\{eqnarray*\}

  (obtenue en rempla\textbackslash{}c\{c\}ant dans ω, x par f(t),
  \{x\}\_\{i\} par \{f\}\_\{i\}(t) et donc d\{x\}\_\{i\} par
  \{f\}\_\{i\}'(t) dt).
\end{itemize}

Remarque~20.1.3 Le lecteur vérifiera facilement que le premier type
d'intégrale est également invariant par changement d'orientation de Γ,
c'est-à-dire par un changement de paramétrage décroissant (car la forme
différentielle est changée en son opposée mais dans le même temps les
bornes de l'intégrale sont interverties)~; par contre les intégrales des
deux autres types sont changées en leurs opposées par changement
d'orientation.

Proposition~20.1.4 Soit Γ = ({[}a,b{]},f) un arc paramétré de classe
\{C\}\^{}\{1\} de longueur l(Γ).

\begin{itemize}
\item
  (i) soit h une fonction continue bornée définie sur l'image de Γ et à
  valeurs dans K~; alors

  \textbackslash{}left \textbar{}\{\textbackslash{}mathop\{∫ \}
  \}\_\{Γ\}h(m) ds\textbackslash{}right \textbar{} ≤
  l(Γ)\{\textbackslash{}mathop\{sup\}\}\_\{m∈\textbackslash{}mathop\{\textbackslash{}mathrm\{Im\}\}
  Γ\}\textbar{}h(m)\textbar{}
\item
  (ii) supposons que E est un espace euclidien et soit V un champ de
  vecteurs continu défini sur l'image de Γ et borné~; alors

  \textbackslash{}left \textbar{}\{\textbackslash{}mathop\{∫ \}
  \}\_\{Γ\}(V (m)\textbackslash{}mathrel\{∣\}dm)\textbackslash{}right
  \textbar{} ≤
  l(Γ)\{\textbackslash{}mathop\{sup\}\}\_\{m∈\textbackslash{}mathop\{\textbackslash{}mathrm\{Im\}\}
  Γ\}\textbackslash{}\textbar{}V (m)\textbackslash{}\textbar{}
\end{itemize}

Démonstration (i) On a

\textbackslash{}begin\{eqnarray*\} \textbackslash{}left
\textbar{}\{\textbackslash{}mathop\{∫ \} \}\_\{Γ\}h(m)
ds\textbackslash{}right \textbar{}\& =\& \textbackslash{}left
\textbar{}\{\textbackslash{}mathop\{∫ \} \}\_\{a\}\^{}\{b\}h(f(t))
\textbackslash{}\textbar{}f'(t)\textbackslash{}\textbar{}
dt\textbackslash{}right \textbar{} \%\& \textbackslash{}\textbackslash{}
\& ≤\& \{\textbackslash{}mathop\{∫ \}
\}\_\{a\}\^{}\{b\}\textbar{}h(f(t))\textbar{}\textbackslash{}\textbar{}f'(t)\textbackslash{}\textbar{}
dt \%\& \textbackslash{}\textbackslash{} \& ≤\& \textbackslash{}left
(\{\textbackslash{}mathop\{sup\}\}\_\{m∈\textbackslash{}mathop\{\textbackslash{}mathrm\{Im\}\}
Γ\}\textbar{}h(m)\textbar{}\textbackslash{}right
)\{\textbackslash{}mathop\{∫ \}
\}\_\{a\}\^{}\{b\}\textbackslash{}\textbar{}f'(t)\textbackslash{}\textbar{}
dt\%\& \textbackslash{}\textbackslash{} \& =\&
l(Γ)\{\textbackslash{}mathop\{sup\}\}\_\{m∈\textbackslash{}mathop\{\textbackslash{}mathrm\{Im\}\}
Γ\}\textbar{}h(m)\textbar{} \%\& \textbackslash{}\textbackslash{}
\textbackslash{}end\{eqnarray*\}

(ii) On a grâce à l'inégalité de Schwarz, \textbackslash{}left
\textbar{}\textbackslash{}left (V
(f(t))\textbackslash{}mathrel\{∣\}f'(t)\textbackslash{}right
)\textbackslash{}right \textbar{} ≤\textbackslash{}\textbar{} V
(f(t))\textbackslash{}\textbar{}
\textbackslash{}\textbar{}f'(t)\textbackslash{}\textbar{} d'où

\textbackslash{}begin\{eqnarray*\} \textbackslash{}left
\textbar{}\{\textbackslash{}mathop\{∫ \} \}\_\{Γ\}(V
(m)\textbackslash{}mathrel\{∣\}dm)\textbackslash{}right \textbar{}\& =\&
\textbackslash{}left \textbar{}\{\textbackslash{}mathop\{∫ \}
\}\_\{a\}\^{}\{b\}\textbackslash{}left (V
(f(t))\textbackslash{}mathrel\{∣\}f'(t)\textbackslash{}right )
dt\textbackslash{}right \textbar{} \%\& \textbackslash{}\textbackslash{}
\& ≤\& \{\textbackslash{}mathop\{∫ \}
\}\_\{a\}\^{}\{b\}\textbackslash{}\textbar{}V
(f(t))\textbackslash{}\textbar{}
\textbackslash{}\textbar{}f'(t)\textbackslash{}\textbar{} dt \%\&
\textbackslash{}\textbackslash{} \& ≤\& \textbackslash{}left
(\{\textbackslash{}mathop\{sup\}\}\_\{m∈\textbackslash{}mathop\{\textbackslash{}mathrm\{Im\}\}
Γ\}\textbackslash{}\textbar{}V
(m)\textbackslash{}\textbar{}\textbackslash{}right
)\{\textbackslash{}mathop\{∫ \}
\}\_\{a\}\^{}\{b\}\textbackslash{}\textbar{}f'(t)\textbackslash{}\textbar{}
dt\%\& \textbackslash{}\textbackslash{} \& =\&
l(Γ)\{\textbackslash{}mathop\{sup\}\}\_\{m∈\textbackslash{}mathop\{\textbackslash{}mathrm\{Im\}\}
Γ\}\textbackslash{}\textbar{}V (m)\textbackslash{}\textbar{} \%\&
\textbackslash{}\textbackslash{} \textbackslash{}end\{eqnarray*\}

\paragraph{20.1.3 Formes différentielles exactes et champs de gradients}

Théorème~20.1.5

\begin{itemize}
\item
  (i) Soit Γ = ({[}a,b{]},f) un arc paramétré de classe \{C\}\^{}\{1\}
  et soit V un champ de vecteurs défini et continu sur un ouvert U
  contenant l'image de Γ~; si V est le champ des gradients d'une
  fonction F : U → ℝ, alors

  \{\textbackslash{}mathop\{∫ \} \}\_\{Γ\}(V
  (m)\textbackslash{}mathrel\{∣\}dm) = F(f(b)) − F(f(a))
\item
  (ii) Soit Γ = ({[}a,b{]},f) un arc paramétré de classe \{C\}\^{}\{1\}
  et soit ω une forme différentielle définie et continue sur un ouvert U
  contenant l'image de Γ~; si ω est la différentielle d'une fonction F :
  U → ℝ, alors

  \{\textbackslash{}mathop\{∫ \} \}\_\{Γ\}ω = F(f(b)) − F(f(a))
\end{itemize}

Démonstration (i) On a en effet

\textbackslash{}begin\{eqnarray*\}\{ d \textbackslash{}over dt\}
(F(f(t)))\& =\& dF(f(t)).f'(t) = \textbackslash{}left
((\textbackslash{}mathop\{\textbackslash{}mathrm\{grad\}\}
F)(f(t))\textbackslash{}mathrel\{∣\}f'(t)\textbackslash{}right )\%\&
\textbackslash{}\textbackslash{} \& =\& (V
(f(t))\textbackslash{}mathrel\{∣\}f'(t)) \%\&
\textbackslash{}\textbackslash{} \textbackslash{}end\{eqnarray*\}

d'où l'on déduit

\textbackslash{}begin\{eqnarray*\} \{\textbackslash{}mathop\{∫ \}
\}\_\{Γ\}(V (m)\textbackslash{}mathrel\{∣\}dm)\& =\&
\{\textbackslash{}mathop\{∫ \} \}\_\{a\}\^{}\{b\}(V
(f(t))\textbackslash{}mathrel\{∣\}f'(t)) =\{\textbackslash{}mathop\{∫ \}
\}\_\{a\}\^{}\{b\}(F ∘ f)'(t) dt\%\& \textbackslash{}\textbackslash{} \&
=\& F(f(b)) − F(f(a)) \%\& \textbackslash{}\textbackslash{}
\textbackslash{}end\{eqnarray*\}

(ii) Si ω = dF, on a donc ω =\{ ∂F \textbackslash{}over ∂\{x\}\_\{1\}\}
(x) d\{x\}\_\{1\} +
\textbackslash{}mathop\{\textbackslash{}mathop\{\ldots{}\}\} +\{ ∂F
\textbackslash{}over ∂\{x\}\_\{n\}\} (x) d\{x\}\_\{n\}, si bien que

\textbackslash{}begin\{eqnarray*\} \{\textbackslash{}mathop\{∫ \}
\}\_\{Γ\}ω\& =\& \{\textbackslash{}mathop\{∫ \} \}\_\{a\}\^{}\{b\}(\{ ∂F
\textbackslash{}over ∂\{x\}\_\{1\}\} (f(t))\{f\}\_\{1\}'(t) +
\textbackslash{}mathop\{\textbackslash{}mathop\{\ldots{}\}\} +\{ ∂F
\textbackslash{}over ∂\{x\}\_\{n\}\} (f(t))\{f\}\_\{n\}'(t)) dt \%\&
\textbackslash{}\textbackslash{} \& =\& \{\textbackslash{}mathop\{∫ \}
\}\_\{a\}\^{}\{b\}\{ d \textbackslash{}over dt\}
(F(\{f\}\_\{1\}(t),\textbackslash{}mathop\{\textbackslash{}mathop\{\ldots{}\}\},\{f\}\_\{n\}(t)))
dt = F(f(b)) − F(f(a))\%\& \textbackslash{}\textbackslash{}
\textbackslash{}end\{eqnarray*\}

Corollaire~20.1.6 Soit Γ = ({[}a,b{]},f) un arc paramétré de classe
\{C\}\^{}\{1\}, fermé (c'est-à-dire que f(b) = f(a)).

\begin{itemize}
\itemsep1pt\parskip0pt\parsep0pt
\item
  (i) Pour tout champ de gradients V sur un ouvert U de E contenant
  \textbackslash{}mathop\{\textbackslash{}mathrm\{Im\}\}Γ, on a
  \{\textbackslash{}mathop\{∫ \} \}\_\{Γ\}(V
  (m)\textbackslash{}mathrel\{∣\}dm) = 0
\item
  (ii) Pour toute forme différentielle exacte ω sur un ouvert U de E
  contenant \textbackslash{}mathop\{\textbackslash{}mathrm\{Im\}\}Γ, on
  a \{\textbackslash{}mathop\{∫ \} \}\_\{Γ\}ω = 0
\end{itemize}

Démonstration Conséquence évidente du résultat précédent.

Exemple~20.1.3 Considérons sur \{ℝ\}\^{}\{2\}
∖\textbackslash{}\{(0,0\textbackslash{}\} la forme différentielle de
classe \{C\}\^{}\{∞\}, ω =\{ x dy−y dx \textbackslash{}over
\{x\}\^{}\{2\}+\{y\}\^{}\{2\}\} ~; on vérifie facilement que dω = 0
puisque \{ ∂ \textbackslash{}over ∂y\} \textbackslash{}left (\{ −y
\textbackslash{}over \{x\}\^{}\{2\}+\{y\}\^{}\{2\}\}
\textbackslash{}right ) =\{ ∂ \textbackslash{}over ∂x\}
\textbackslash{}left (\{ x \textbackslash{}over
\{x\}\^{}\{2\}+\{y\}\^{}\{2\}\} \textbackslash{}right ). Pourtant ω
n'est pas exacte. En effet calculons l'intégrale de ω le long du cercle
Γ de centre (0,0) de rayon 1. On a en posant x =\textbackslash{}mathop\{
cos\} θ et y =\textbackslash{}mathop\{ sin\} θ,

x dy − y dx =\textbackslash{}mathop\{ cos\} θ ×
(\textbackslash{}mathop\{cos\} θ dθ) −\textbackslash{}mathop\{ sin\} θ ×
(−\textbackslash{}mathop\{sin\} θ dθ) = dθ

si bien que

\{\textbackslash{}mathop\{∫ \} \}\_\{Γ\}ω =\{\textbackslash{}mathop\{∫
\} \}\_\{0\}\^{}\{2π\}dθ = 2π\textbackslash{}mathrel\{≠\}0

ce qui montre que ω ne peut pas être la différentielle d'une fonction.
L'hypothèse que l'ouvert est étoilé est donc essentielle pour la
validité du théorème de Poincaré qui dit que (sur un ouvert étoilé) une
forme différentielle est exacte si et seulement si~elle vérifie dω = 0.

{[}\href{coursse105.html}{next}{]} {[}\href{coursse104.html}{front}{]}
{[}\href{coursch21.html\#coursse104.html}{up}{]}

\end{document}
