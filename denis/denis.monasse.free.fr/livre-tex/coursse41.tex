\documentclass[]{article}
\usepackage[T1]{fontenc}
\usepackage{lmodern}
\usepackage{amssymb,amsmath}
\usepackage{ifxetex,ifluatex}
\usepackage{fixltx2e} % provides \textsubscript
% use upquote if available, for straight quotes in verbatim environments
\IfFileExists{upquote.sty}{\usepackage{upquote}}{}
\ifnum 0\ifxetex 1\fi\ifluatex 1\fi=0 % if pdftex
  \usepackage[utf8]{inputenc}
\else % if luatex or xelatex
  \ifxetex
    \usepackage{mathspec}
    \usepackage{xltxtra,xunicode}
  \else
    \usepackage{fontspec}
  \fi
  \defaultfontfeatures{Mapping=tex-text,Scale=MatchLowercase}
  \newcommand{\euro}{€}
\fi
% use microtype if available
\IfFileExists{microtype.sty}{\usepackage{microtype}}{}
\ifxetex
  \usepackage[setpagesize=false, % page size defined by xetex
              unicode=false, % unicode breaks when used with xetex
              xetex]{hyperref}
\else
  \usepackage[unicode=true]{hyperref}
\fi
\hypersetup{breaklinks=true,
            bookmarks=true,
            pdfauthor={},
            pdftitle={Series doubles},
            colorlinks=true,
            citecolor=blue,
            urlcolor=blue,
            linkcolor=magenta,
            pdfborder={0 0 0}}
\urlstyle{same}  % don't use monospace font for urls
\setlength{\parindent}{0pt}
\setlength{\parskip}{6pt plus 2pt minus 1pt}
\setlength{\emergencystretch}{3em}  % prevent overfull lines
\setcounter{secnumdepth}{0}
 
/* start css.sty */
.cmr-5{font-size:50%;}
.cmr-7{font-size:70%;}
.cmmi-5{font-size:50%;font-style: italic;}
.cmmi-7{font-size:70%;font-style: italic;}
.cmmi-10{font-style: italic;}
.cmsy-5{font-size:50%;}
.cmsy-7{font-size:70%;}
.cmex-7{font-size:70%;}
.cmex-7x-x-71{font-size:49%;}
.msbm-7{font-size:70%;}
.cmtt-10{font-family: monospace;}
.cmti-10{ font-style: italic;}
.cmbx-10{ font-weight: bold;}
.cmr-17x-x-120{font-size:204%;}
.cmsl-10{font-style: oblique;}
.cmti-7x-x-71{font-size:49%; font-style: italic;}
.cmbxti-10{ font-weight: bold; font-style: italic;}
p.noindent { text-indent: 0em }
td p.noindent { text-indent: 0em; margin-top:0em; }
p.nopar { text-indent: 0em; }
p.indent{ text-indent: 1.5em }
@media print {div.crosslinks {visibility:hidden;}}
a img { border-top: 0; border-left: 0; border-right: 0; }
center { margin-top:1em; margin-bottom:1em; }
td center { margin-top:0em; margin-bottom:0em; }
.Canvas { position:relative; }
li p.indent { text-indent: 0em }
.enumerate1 {list-style-type:decimal;}
.enumerate2 {list-style-type:lower-alpha;}
.enumerate3 {list-style-type:lower-roman;}
.enumerate4 {list-style-type:upper-alpha;}
div.newtheorem { margin-bottom: 2em; margin-top: 2em;}
.obeylines-h,.obeylines-v {white-space: nowrap; }
div.obeylines-v p { margin-top:0; margin-bottom:0; }
.overline{ text-decoration:overline; }
.overline img{ border-top: 1px solid black; }
td.displaylines {text-align:center; white-space:nowrap;}
.centerline {text-align:center;}
.rightline {text-align:right;}
div.verbatim {font-family: monospace; white-space: nowrap; text-align:left; clear:both; }
.fbox {padding-left:3.0pt; padding-right:3.0pt; text-indent:0pt; border:solid black 0.4pt; }
div.fbox {display:table}
div.center div.fbox {text-align:center; clear:both; padding-left:3.0pt; padding-right:3.0pt; text-indent:0pt; border:solid black 0.4pt; }
div.minipage{width:100%;}
div.center, div.center div.center {text-align: center; margin-left:1em; margin-right:1em;}
div.center div {text-align: left;}
div.flushright, div.flushright div.flushright {text-align: right;}
div.flushright div {text-align: left;}
div.flushleft {text-align: left;}
.underline{ text-decoration:underline; }
.underline img{ border-bottom: 1px solid black; margin-bottom:1pt; }
.framebox-c, .framebox-l, .framebox-r { padding-left:3.0pt; padding-right:3.0pt; text-indent:0pt; border:solid black 0.4pt; }
.framebox-c {text-align:center;}
.framebox-l {text-align:left;}
.framebox-r {text-align:right;}
span.thank-mark{ vertical-align: super }
span.footnote-mark sup.textsuperscript, span.footnote-mark a sup.textsuperscript{ font-size:80%; }
div.tabular, div.center div.tabular {text-align: center; margin-top:0.5em; margin-bottom:0.5em; }
table.tabular td p{margin-top:0em;}
table.tabular {margin-left: auto; margin-right: auto;}
div.td00{ margin-left:0pt; margin-right:0pt; }
div.td01{ margin-left:0pt; margin-right:5pt; }
div.td10{ margin-left:5pt; margin-right:0pt; }
div.td11{ margin-left:5pt; margin-right:5pt; }
table[rules] {border-left:solid black 0.4pt; border-right:solid black 0.4pt; }
td.td00{ padding-left:0pt; padding-right:0pt; }
td.td01{ padding-left:0pt; padding-right:5pt; }
td.td10{ padding-left:5pt; padding-right:0pt; }
td.td11{ padding-left:5pt; padding-right:5pt; }
table[rules] {border-left:solid black 0.4pt; border-right:solid black 0.4pt; }
.hline hr, .cline hr{ height : 1px; margin:0px; }
.tabbing-right {text-align:right;}
span.TEX {letter-spacing: -0.125em; }
span.TEX span.E{ position:relative;top:0.5ex;left:-0.0417em;}
a span.TEX span.E {text-decoration: none; }
span.LATEX span.A{ position:relative; top:-0.5ex; left:-0.4em; font-size:85%;}
span.LATEX span.TEX{ position:relative; left: -0.4em; }
div.float img, div.float .caption {text-align:center;}
div.figure img, div.figure .caption {text-align:center;}
.marginpar {width:20%; float:right; text-align:left; margin-left:auto; margin-top:0.5em; font-size:85%; text-decoration:underline;}
.marginpar p{margin-top:0.4em; margin-bottom:0.4em;}
.equation td{text-align:center; vertical-align:middle; }
td.eq-no{ width:5%; }
table.equation { width:100%; } 
div.math-display, div.par-math-display{text-align:center;}
math .texttt { font-family: monospace; }
math .textit { font-style: italic; }
math .textsl { font-style: oblique; }
math .textsf { font-family: sans-serif; }
math .textbf { font-weight: bold; }
.partToc a, .partToc, .likepartToc a, .likepartToc {line-height: 200%; font-weight:bold; font-size:110%;}
.chapterToc a, .chapterToc, .likechapterToc a, .likechapterToc, .appendixToc a, .appendixToc {line-height: 200%; font-weight:bold;}
.index-item, .index-subitem, .index-subsubitem {display:block}
.caption td.id{font-weight: bold; white-space: nowrap; }
table.caption {text-align:center;}
h1.partHead{text-align: center}
p.bibitem { text-indent: -2em; margin-left: 2em; margin-top:0.6em; margin-bottom:0.6em; }
p.bibitem-p { text-indent: 0em; margin-left: 2em; margin-top:0.6em; margin-bottom:0.6em; }
.paragraphHead, .likeparagraphHead { margin-top:2em; font-weight: bold;}
.subparagraphHead, .likesubparagraphHead { font-weight: bold;}
.quote {margin-bottom:0.25em; margin-top:0.25em; margin-left:1em; margin-right:1em; text-align:justify;}
.verse{white-space:nowrap; margin-left:2em}
div.maketitle {text-align:center;}
h2.titleHead{text-align:center;}
div.maketitle{ margin-bottom: 2em; }
div.author, div.date {text-align:center;}
div.thanks{text-align:left; margin-left:10%; font-size:85%; font-style:italic; }
div.author{white-space: nowrap;}
.quotation {margin-bottom:0.25em; margin-top:0.25em; margin-left:1em; }
h1.partHead{text-align: center}
.sectionToc, .likesectionToc {margin-left:2em;}
.subsectionToc, .likesubsectionToc {margin-left:4em;}
.subsubsectionToc, .likesubsubsectionToc {margin-left:6em;}
.frenchb-nbsp{font-size:75%;}
.frenchb-thinspace{font-size:75%;}
.figure img.graphics {margin-left:10%;}
/* end css.sty */

\title{Series doubles}
\author{}
\date{}

\begin{document}
\maketitle

\textbf{Warning: \href{http://www.math.union.edu/locate/jsMath}{jsMath}
requires JavaScript to process the mathematics on this page.\\ If your
browser supports JavaScript, be sure it is enabled.}

\begin{center}\rule{3in}{0.4pt}\end{center}

{[}\href{coursse42.html}{next}{]} {[}\href{coursse40.html}{prev}{]}
{[}\href{coursse40.html\#tailcoursse40.html}{prev-tail}{]}
{[}\hyperref[tailcoursse41.html]{tail}{]}
{[}\href{coursch8.html\#coursse41.html}{up}{]}

\subsubsection{7.7 Séries doubles}

En anticipant un peu sur le chapitre concernant les séries de fonctions,
nous ferons appel au lemme suivant pour la démonstration du théorème
fondamental sur les séries doubles.

Lemme~7.7.1 (Weierstrass~: théorème de convergence dominée pour les
séries) Soit \{(\{x\}\_\{n,q\})\}\_\{(n,q)∈ℕ×ℕ\} une famille de nombres
réels ou complexes indexée qar ℕ × ℕ. On fait les hypothèses suivantes

\begin{itemize}
\itemsep1pt\parskip0pt\parsep0pt
\item
  il existe une séries à termes réels positifs
  \textbackslash{}mathop\{\textbackslash{}mathop\{∑ \}\} \{α\}\_\{n\}
  convergente telle que \textbackslash{}mathop\{∀\}q ∈
  ℕ,\textbar{}\{x\}\_\{n,q\}\textbar{}≤ \{α\}\_\{n\}
\item
  pour chaque n ∈ ℕ,
  \{\textbackslash{}mathop\{lim\}\}\_\{q→+∞\}\{x\}\_\{n,q\} existe (on
  appelle \{y\}\_\{n\} cette limite)
\end{itemize}

Alors, pour chaque q ∈ ℕ, la série
\{\textbackslash{}mathop\{\textbackslash{}mathop\{∑ \}\}
\}\_\{n\}\{x\}\_\{n,q\} est absolument convergente ainsi que la série
\{\textbackslash{}mathop\{\textbackslash{}mathop\{∑ \}\}
\}\_\{n\}\{y\}\_\{n\}, la suite \{\textbackslash{}left
(\{\textbackslash{}mathop\{\textbackslash{}mathop\{∑ \}\}
\}\_\{n=0\}\^{}\{+∞\}\{x\}\_\{n,q\}\textbackslash{}right )\}\_\{q∈ℕ\}
admet une limite quand q tend vers + ∞ et on a

\{\textbackslash{}mathop\{lim\}\}\_\{q→+∞\}\{\textbackslash{}mathop\{∑
\}\}\_\{n=0\}\^{}\{+∞\}\{x\}\_\{ n,q\} =\{ \textbackslash{}mathop\{∑
\}\}\_\{n=0\}\^{}\{+∞\}\{y\}\_\{ n\}

autrement dit

\{\textbackslash{}mathop\{lim\}\}\_\{q→+∞\}\{\textbackslash{}mathop\{∑
\}\}\_\{n=0\}\^{}\{+∞\}\{x\}\_\{ n,q\} =\{ \textbackslash{}mathop\{∑
\}\}\_\{n=0\}\^{}\{+∞\}\{lim\}\_\{ q→+∞\}\{x\}\_\{n,q\}

(interversion de la limite et du signe somme)

Démonstration L'inégalité \textbar{}\{x\}\_\{n,q\}\textbar{}≤
\{α\}\_\{n\}, celle qui s'en déduit par passage à la limite
\textbar{}\{y\}\_\{n\}\textbar{}≤ \{α\}\_\{n\} et la convergence de la
série \textbackslash{}mathop\{\textbackslash{}mathop\{∑ \}\}
\{α\}\_\{n\} montrent les convergences absolues des séries
\{\textbackslash{}mathop\{\textbackslash{}mathop\{∑ \}\}
\}\_\{n\}\{x\}\_\{n,q\} et
\{\textbackslash{}mathop\{\textbackslash{}mathop\{∑ \}\}
\}\_\{n\}\{y\}\_\{n\}. Prenons donc ε \textgreater{} 0 et choisissons M
tel que \{\textbackslash{}mathop\{\textbackslash{}mathop\{∑ \}\}
\}\_\{n=M+1\}\^{}\{+∞\}\{α\}\_\{n\} \textless{} \{ε\textbackslash{}over
4\}. On a alors

\textbackslash{}begin\{eqnarray*\} \textbackslash{}left
\textbar{}\{\textbackslash{}mathop\{∑ \}\}\_\{n=0\}\^{}\{+∞\}\{y\}\_\{
n\} −\{\textbackslash{}mathop\{∑ \}\}\_\{n=0\}\^{}\{+∞\}\{x\}\_\{
n,q\}\textbackslash{}right \textbar{}\& ≤\& \{\textbackslash{}mathop\{∑
\}\}\_\{n=0\}\^{}\{+∞\}\textbar{}\{y\}\_\{ n\} −
\{x\}\_\{n,q\}\textbar{}≤\{\textbackslash{}mathop\{∑
\}\}\_\{n=0\}\^{}\{M\}\textbar{}\{y\}\_\{ n\} − \{x\}\_\{n,q\}\textbar{}
+\{ \textbackslash{}mathop\{∑
\}\}\_\{n=M+1\}\^{}\{+∞\}(\textbar{}\{y\}\_\{ n\}\textbar{} +
\textbar{}\{x\}\_\{n,q\}\textbar{})\%\& \textbackslash{}\textbackslash{}
\& ≤\& \{\textbackslash{}mathop\{∑
\}\}\_\{n=0\}\^{}\{M\}\textbar{}\{y\}\_\{ n\} − \{x\}\_\{n,q\}\textbar{}
+ 2\{\textbackslash{}mathop\{∑ \}\}\_\{n=M+1\}\^{}\{+∞\}\{α\}\_\{ n\}
≤\{\textbackslash{}mathop\{∑ \}\}\_\{n=0\}\^{}\{M\}\textbar{}\{y\}\_\{
n\} − \{x\}\_\{n,q\}\textbar{} + \{ε\textbackslash{}over 2\}
\%\&\textbackslash{}\textbackslash{} \textbackslash{}end\{eqnarray*\}

Maintenant, on a
\{\textbackslash{}mathop\{lim\}\}\_\{q→+∞\}\{\textbackslash{}mathop\{\textbackslash{}mathop\{∑
\}\} \}\_\{n=0\}\^{}\{M\}\textbar{}\{y\}\_\{n\} −
\{x\}\_\{n,q\}\textbar{} = 0 (chacun des termes de cette somme admet 0
pour limite), et donc il existe N ∈ ℕ tel que q ≥ N
⇒\{\textbackslash{}mathop\{\textbackslash{}mathop\{∑ \}\}
\}\_\{n=0\}\^{}\{M\}\textbar{}\{y\}\_\{n\} − \{x\}\_\{n,q\}\textbar{}
\textless{} \{ε\textbackslash{}over 2\}. On a donc

q ≥ N ⇒\textbackslash{}left \textbar{}\{\textbackslash{}mathop\{∑
\}\}\_\{n=0\}\^{}\{+∞\}\{y\}\_\{ n\} −\{\textbackslash{}mathop\{∑
\}\}\_\{n=0\}\^{}\{+∞\}\{x\}\_\{ n,q\}\textbackslash{}right \textbar{}≤
\{ε\textbackslash{}over 2\} + \{ε\textbackslash{}over 2\} = ε

ce qui montre que la suite \{\textbackslash{}left
(\{\textbackslash{}mathop\{\textbackslash{}mathop\{∑ \}\}
\}\_\{n=0\}\^{}\{+∞\}\{x\}\_\{n,q\}\textbackslash{}right )\}\_\{q∈ℕ\}
admet la limite \{\textbackslash{}mathop\{\textbackslash{}mathop\{∑ \}\}
\}\_\{n=0\}\^{}\{+∞\}\{y\}\_\{n\} quand q tend vers + ∞.

Remarque~7.7.1 Le lecteur qui a déjà des connaissances sur les séries de
fonctions, remarquera qu'il s'agit là tout simplement du théorème
d'interversion des limites dans le cas de convergence normale (donc
uniforme) d'une série de fonctions.

Nous pouvons maintenant démontrer le théorème d'interversion des signes
somme dans les séries doubles.

Théorème~7.7.2 Soit u = \{(\{u\}\_\{n,p\})\}\_\{(n,p)∈ℕ×ℕ\} une famille
de nombres réels ou complexes indexée par ℕ × ℕ. On suppose que

\begin{itemize}
\itemsep1pt\parskip0pt\parsep0pt
\item
  pour tout entier n la série
  \{\textbackslash{}mathop\{\textbackslash{}mathop\{∑ \}\}
  \}\_\{p\}\{u\}\_\{n,p\} est absolument convergente
\item
  la série \{\textbackslash{}mathop\{\textbackslash{}mathop\{∑ \}\}
  \}\_\{n\}\{\textbackslash{}mathop\{ \textbackslash{}mathop\{∑ \}\}
  \}\_\{p=0\}\^{}\{+∞\}\textbar{}\{u\}\_\{n,p\}\textbar{} est
  convergente
\end{itemize}

Alors les séries \{\textbackslash{}mathop\{\textbackslash{}mathop\{∑
\}\} \}\_\{n\}\textbackslash{}left
(\{\textbackslash{}mathop\{\textbackslash{}mathop\{∑ \}\}
\}\_\{p=0\}\^{}\{+∞\}\{u\}\_\{n,p\}\textbackslash{}right ) et
\{\textbackslash{}mathop\{\textbackslash{}mathop\{∑ \}\}
\}\_\{p\}\textbackslash{}left
(\{\textbackslash{}mathop\{\textbackslash{}mathop\{∑ \}\}
\}\_\{n=0\}\^{}\{+∞\}\{u\}\_\{n,p\}\textbackslash{}right ) sont
convergentes et on a

\{\textbackslash{}mathop\{∑ \}\}\_\{n=0\}\^{}\{+∞\}\textbackslash{}left
(\{\textbackslash{}mathop\{∑ \}\}\_\{p=0\}\^{}\{+∞\}\{u\}\_\{
n,p\}\textbackslash{}right ) =\{ \textbackslash{}mathop\{∑
\}\}\_\{p=0\}\^{}\{+∞\}\textbackslash{}left (\{\textbackslash{}mathop\{∑
\}\}\_\{n=0\}\^{}\{+∞\}\{u\}\_\{ n,p\}\textbackslash{}right )

Démonstration Nous allons appliquer le lemme précédent en posant
\{x\}\_\{n,q\} =\{\textbackslash{}mathop\{ \textbackslash{}mathop\{∑
\}\} \}\_\{p=0\}\^{}\{q\}\{u\}\_\{n,p\} et \{α\}\_\{n\}
=\{\textbackslash{}mathop\{ \textbackslash{}mathop\{∑ \}\}
\}\_\{p=0\}\^{}\{+∞\}\textbar{}\{u\}\_\{n,p\}\textbar{} et bien entendu
\{y\}\_\{n\} =\{\textbackslash{}mathop\{ \textbackslash{}mathop\{∑ \}\}
\}\_\{p=0\}\^{}\{+∞\}\{u\}\_\{n,p\} =\{\textbackslash{}mathop\{
lim\}\}\_\{q→+∞\}\{x\}\_\{n,q\}. Les hypothèses du lemme étant
évidemment vérifiées, on sait que
\{\textbackslash{}mathop\{\textbackslash{}mathop\{∑ \}\}
\}\_\{n=0\}\^{}\{+∞\}\{x\}\_\{n,q\} admet la limite
\{\textbackslash{}mathop\{\textbackslash{}mathop\{∑ \}\}
\}\_\{n=0\}\^{}\{+∞\}\{y\}\_\{n\} quand q tend vers + ∞. Mais, puisque
l'on a l'égalité
\textbar{}\{u\}\_\{n,p\}\textbar{}≤\{\textbackslash{}mathop\{\textbackslash{}mathop\{∑
\}\} \}\_\{p=0\}\^{}\{+∞\}\textbar{}\{u\}\_\{n,p\}\textbar{}, la série
\{\textbackslash{}mathop\{\textbackslash{}mathop\{∑ \}\}
\}\_\{n\}\{u\}\_\{n,p\} est absolument convergente pour tout p ∈ ℕ et
donc, par linéarité de la somme,

\{\textbackslash{}mathop\{∑ \}\}\_\{n=0\}\^{}\{+∞\}\{x\}\_\{ n,q\} =\{
\textbackslash{}mathop\{∑
\}\}\_\{n=0\}\^{}\{+∞\}\{\textbackslash{}mathop\{∑
\}\}\_\{p=0\}\^{}\{q\}\{u\}\_\{ n,p\} =\{ \textbackslash{}mathop\{∑
\}\}\_\{p=0\}\^{}\{q\}\{ \textbackslash{}mathop\{∑
\}\}\_\{n=0\}\^{}\{+∞\}\{u\}\_\{ n,p\}

L'existence de
\{\textbackslash{}mathop\{lim\}\}\_\{q→+∞\}\{\textbackslash{}mathop\{\textbackslash{}mathop\{∑
\}\} \}\_\{n=0\}\^{}\{+∞\}\{x\}\_\{n,q\} montre donc que la série
\{\textbackslash{}mathop\{\textbackslash{}mathop\{∑ \}\}
\}\_\{p\}\{\textbackslash{}mathop\{ \textbackslash{}mathop\{∑ \}\}
\}\_\{n=0\}\^{}\{+∞\}\{u\}\_\{n,p\} est convergente et a pour somme
\{\textbackslash{}mathop\{\textbackslash{}mathop\{∑ \}\}
\}\_\{n=0\}\^{}\{+∞\}\{y\}\_\{n\} =\{\textbackslash{}mathop\{
\textbackslash{}mathop\{∑ \}\}
\}\_\{n=0\}\^{}\{+∞\}\{\textbackslash{}mathop\{\textbackslash{}mathop\{∑
\}\} \}\_\{p=0\}\^{}\{+∞\}\{u\}\_\{n,p\} autrement dit que

\{\textbackslash{}mathop\{∑ \}\}\_\{n=0\}\^{}\{+∞\}\textbackslash{}left
(\{\textbackslash{}mathop\{∑ \}\}\_\{p=0\}\^{}\{+∞\}\{u\}\_\{
n,p\}\textbackslash{}right ) =\{ \textbackslash{}mathop\{∑
\}\}\_\{p=0\}\^{}\{+∞\}\textbackslash{}left (\{\textbackslash{}mathop\{∑
\}\}\_\{n=0\}\^{}\{+∞\}\{u\}\_\{ n,p\}\textbackslash{}right )

Remarque~7.7.2 En appliquant le théorème à la suite u' =
\{(\textbar{}\{u\}\_\{n,p\}\textbar{})\}\_\{(n,p)∈ℕ×ℕ\}, on constate que
la série \{\textbackslash{}mathop\{\textbackslash{}mathop\{∑ \}\}
\}\_\{p\}\textbackslash{}left
(\{\textbackslash{}mathop\{\textbackslash{}mathop\{∑ \}\}
\}\_\{n=0\}\^{}\{+∞\}\textbar{}\{u\}\_\{n,p\}\textbar{}\textbackslash{}right
) est convergente, ce qui implique la convergence absolue de la série
\{\textbackslash{}mathop\{\textbackslash{}mathop\{∑ \}\}
\}\_\{p\}\{\textbackslash{}mathop\{ \textbackslash{}mathop\{∑ \}\}
\}\_\{n=0\}\^{}\{+∞\}\{u\}\_\{n,p\}.

Remarque~7.7.3 On pourra retenir le théorème précédent sous la forme
suivante

\{\textbackslash{}mathop\{∑ \}\}\_\{n=0\}\^{}\{+∞\}\textbackslash{}left
(\{\textbackslash{}mathop\{∑ \}\}\_\{p=0\}\^{}\{+∞\}\textbar{}\{u\}\_\{
n,p\}\textbar{}\textbackslash{}right ) \textless{}
+∞⇒\{\textbackslash{}mathop\{∑
\}\}\_\{n=0\}\^{}\{+∞\}\textbackslash{}left (\{\textbackslash{}mathop\{∑
\}\}\_\{p=0\}\^{}\{+∞\}\{u\}\_\{ n,p\}\textbackslash{}right ) =\{
\textbackslash{}mathop\{∑ \}\}\_\{p=0\}\^{}\{+∞\}\textbackslash{}left
(\{\textbackslash{}mathop\{∑ \}\}\_\{n=0\}\^{}\{+∞\}\{u\}\_\{
n,p\}\textbackslash{}right )

{[}\href{coursse42.html}{next}{]} {[}\href{coursse40.html}{prev}{]}
{[}\href{coursse40.html\#tailcoursse40.html}{prev-tail}{]}
{[}\href{coursse41.html}{front}{]}
{[}\href{coursch8.html\#coursse41.html}{up}{]}

\end{document}
