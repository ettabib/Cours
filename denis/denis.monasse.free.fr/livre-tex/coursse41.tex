\documentclass[]{article}
\usepackage[T1]{fontenc}
\usepackage{lmodern}
\usepackage{amssymb,amsmath}
\usepackage{ifxetex,ifluatex}
\usepackage{fixltx2e} % provides \textsubscript
% use upquote if available, for straight quotes in verbatim environments
\IfFileExists{upquote.sty}{\usepackage{upquote}}{}
\ifnum 0\ifxetex 1\fi\ifluatex 1\fi=0 % if pdftex
  \usepackage[utf8]{inputenc}
\else % if luatex or xelatex
  \ifxetex
    \usepackage{mathspec}
    \usepackage{xltxtra,xunicode}
  \else
    \usepackage{fontspec}
  \fi
  \defaultfontfeatures{Mapping=tex-text,Scale=MatchLowercase}
  \newcommand{\euro}{€}
\fi
% use microtype if available
\IfFileExists{microtype.sty}{\usepackage{microtype}}{}
\ifxetex
  \usepackage[setpagesize=false, % page size defined by xetex
              unicode=false, % unicode breaks when used with xetex
              xetex]{hyperref}
\else
  \usepackage[unicode=true]{hyperref}
\fi
\hypersetup{breaklinks=true,
            bookmarks=true,
            pdfauthor={},
            pdftitle={Series doubles},
            colorlinks=true,
            citecolor=blue,
            urlcolor=blue,
            linkcolor=magenta,
            pdfborder={0 0 0}}
\urlstyle{same}  % don't use monospace font for urls
\setlength{\parindent}{0pt}
\setlength{\parskip}{6pt plus 2pt minus 1pt}
\setlength{\emergencystretch}{3em}  % prevent overfull lines
\setcounter{secnumdepth}{0}
 
/* start css.sty */
.cmr-5{font-size:50%;}
.cmr-7{font-size:70%;}
.cmmi-5{font-size:50%;font-style: italic;}
.cmmi-7{font-size:70%;font-style: italic;}
.cmmi-10{font-style: italic;}
.cmsy-5{font-size:50%;}
.cmsy-7{font-size:70%;}
.cmex-7{font-size:70%;}
.cmex-7x-x-71{font-size:49%;}
.msbm-7{font-size:70%;}
.cmtt-10{font-family: monospace;}
.cmti-10{ font-style: italic;}
.cmbx-10{ font-weight: bold;}
.cmr-17x-x-120{font-size:204%;}
.cmsl-10{font-style: oblique;}
.cmti-7x-x-71{font-size:49%; font-style: italic;}
.cmbxti-10{ font-weight: bold; font-style: italic;}
p.noindent { text-indent: 0em }
td p.noindent { text-indent: 0em; margin-top:0em; }
p.nopar { text-indent: 0em; }
p.indent{ text-indent: 1.5em }
@media print {div.crosslinks {visibility:hidden;}}
a img { border-top: 0; border-left: 0; border-right: 0; }
center { margin-top:1em; margin-bottom:1em; }
td center { margin-top:0em; margin-bottom:0em; }
.Canvas { position:relative; }
li p.indent { text-indent: 0em }
.enumerate1 {list-style-type:decimal;}
.enumerate2 {list-style-type:lower-alpha;}
.enumerate3 {list-style-type:lower-roman;}
.enumerate4 {list-style-type:upper-alpha;}
div.newtheorem { margin-bottom: 2em; margin-top: 2em;}
.obeylines-h,.obeylines-v {white-space: nowrap; }
div.obeylines-v p { margin-top:0; margin-bottom:0; }
.overline{ text-decoration:overline; }
.overline img{ border-top: 1px solid black; }
td.displaylines {text-align:center; white-space:nowrap;}
.centerline {text-align:center;}
.rightline {text-align:right;}
div.verbatim {font-family: monospace; white-space: nowrap; text-align:left; clear:both; }
.fbox {padding-left:3.0pt; padding-right:3.0pt; text-indent:0pt; border:solid black 0.4pt; }
div.fbox {display:table}
div.center div.fbox {text-align:center; clear:both; padding-left:3.0pt; padding-right:3.0pt; text-indent:0pt; border:solid black 0.4pt; }
div.minipage{width:100%;}
div.center, div.center div.center {text-align: center; margin-left:1em; margin-right:1em;}
div.center div {text-align: left;}
div.flushright, div.flushright div.flushright {text-align: right;}
div.flushright div {text-align: left;}
div.flushleft {text-align: left;}
.underline{ text-decoration:underline; }
.underline img{ border-bottom: 1px solid black; margin-bottom:1pt; }
.framebox-c, .framebox-l, .framebox-r { padding-left:3.0pt; padding-right:3.0pt; text-indent:0pt; border:solid black 0.4pt; }
.framebox-c {text-align:center;}
.framebox-l {text-align:left;}
.framebox-r {text-align:right;}
span.thank-mark{ vertical-align: super }
span.footnote-mark sup.textsuperscript, span.footnote-mark a sup.textsuperscript{ font-size:80%; }
div.tabular, div.center div.tabular {text-align: center; margin-top:0.5em; margin-bottom:0.5em; }
table.tabular td p{margin-top:0em;}
table.tabular {margin-left: auto; margin-right: auto;}
div.td00{ margin-left:0pt; margin-right:0pt; }
div.td01{ margin-left:0pt; margin-right:5pt; }
div.td10{ margin-left:5pt; margin-right:0pt; }
div.td11{ margin-left:5pt; margin-right:5pt; }
table[rules] {border-left:solid black 0.4pt; border-right:solid black 0.4pt; }
td.td00{ padding-left:0pt; padding-right:0pt; }
td.td01{ padding-left:0pt; padding-right:5pt; }
td.td10{ padding-left:5pt; padding-right:0pt; }
td.td11{ padding-left:5pt; padding-right:5pt; }
table[rules] {border-left:solid black 0.4pt; border-right:solid black 0.4pt; }
.hline hr, .cline hr{ height : 1px; margin:0px; }
.tabbing-right {text-align:right;}
span.TEX {letter-spacing: -0.125em; }
span.TEX span.E{ position:relative;top:0.5ex;left:-0.0417em;}
a span.TEX span.E {text-decoration: none; }
span.LATEX span.A{ position:relative; top:-0.5ex; left:-0.4em; font-size:85%;}
span.LATEX span.TEX{ position:relative; left: -0.4em; }
div.float img, div.float .caption {text-align:center;}
div.figure img, div.figure .caption {text-align:center;}
.marginpar {width:20%; float:right; text-align:left; margin-left:auto; margin-top:0.5em; font-size:85%; text-decoration:underline;}
.marginpar p{margin-top:0.4em; margin-bottom:0.4em;}
.equation td{text-align:center; vertical-align:middle; }
td.eq-no{ width:5%; }
table.equation { width:100%; } 
div.math-display, div.par-math-display{text-align:center;}
math .texttt { font-family: monospace; }
math .textit { font-style: italic; }
math .textsl { font-style: oblique; }
math .textsf { font-family: sans-serif; }
math .textbf { font-weight: bold; }
.partToc a, .partToc, .likepartToc a, .likepartToc {line-height: 200%; font-weight:bold; font-size:110%;}
.chapterToc a, .chapterToc, .likechapterToc a, .likechapterToc, .appendixToc a, .appendixToc {line-height: 200%; font-weight:bold;}
.index-item, .index-subitem, .index-subsubitem {display:block}
.caption td.id{font-weight: bold; white-space: nowrap; }
table.caption {text-align:center;}
h1.partHead{text-align: center}
p.bibitem { text-indent: -2em; margin-left: 2em; margin-top:0.6em; margin-bottom:0.6em; }
p.bibitem-p { text-indent: 0em; margin-left: 2em; margin-top:0.6em; margin-bottom:0.6em; }
.paragraphHead, .likeparagraphHead { margin-top:2em; font-weight: bold;}
.subparagraphHead, .likesubparagraphHead { font-weight: bold;}
.quote {margin-bottom:0.25em; margin-top:0.25em; margin-left:1em; margin-right:1em; text-align:\\jmathmathustify;}
.verse{white-space:nowrap; margin-left:2em}
div.maketitle {text-align:center;}
h2.titleHead{text-align:center;}
div.maketitle{ margin-bottom: 2em; }
div.author, div.date {text-align:center;}
div.thanks{text-align:left; margin-left:10%; font-size:85%; font-style:italic; }
div.author{white-space: nowrap;}
.quotation {margin-bottom:0.25em; margin-top:0.25em; margin-left:1em; }
h1.partHead{text-align: center}
.sectionToc, .likesectionToc {margin-left:2em;}
.subsectionToc, .likesubsectionToc {margin-left:4em;}
.subsubsectionToc, .likesubsubsectionToc {margin-left:6em;}
.frenchb-nbsp{font-size:75%;}
.frenchb-thinspace{font-size:75%;}
.figure img.graphics {margin-left:10%;}
/* end css.sty */

\title{Series doubles}
\author{}
\date{}

\begin{document}
\maketitle

\textbf{Warning: 
requires JavaScript to process the mathematics on this page.\\ If your
browser supports JavaScript, be sure it is enabled.}

\begin{center}\rule{3in}{0.4pt}\end{center}

{[}
{[}
{[}{]}
{[}

\subsubsection{7.7 Séries doubles}

En anticipant un peu sur le chapitre concernant les séries de fonctions,
nous ferons appel au lemme suivant pour la démonstration du théorème
fondamental sur les séries doubles.

Lemme~7.7.1 (Weierstrass~: théorème de convergence dominée pour les
séries) Soit (x_n,q)_(n,q)\in\mathbb{N}~\times\mathbb{N}~ une famille de nombres
réels ou complexes indexée qar \mathbb{N}~ \times \mathbb{N}~. On fait les hypothèses suivantes

\begin{itemize}
\itemsep1pt\parskip0pt\parsep0pt
\item
  il existe une séries à termes réels positifs
  \\sum  \alpha_n~
  convergente telle que \forall~~q \in
  \mathbb{N}~,x_n,q\leq \alpha_n
\item
  pour chaque n \in \mathbb{N}~,
  lim_q\rightarrow~+\infty~x_n,q~ existe (on
  appelle y_n cette limite)
\end{itemize}

Alors, pour chaque q \in \mathbb{N}~, la série
\\sum ~
_nx_n,q est absolument convergente ainsi que la série
\\sum ~
_ny_n, la suite \left
(\\sum ~
_n=0^+\infty~x_n,q\right )_q\in\mathbb{N}~
admet une limite quand q tend vers + \infty~ et on a

lim_q\rightarrow~+\infty~~\\sum
_n=0^+\infty~x_ n,q = \\sum
_n=0^+\infty~y_ n

autrement dit

lim_q\rightarrow~+\infty~~\\sum
_n=0^+\infty~x_ n,q = \\sum
_n=0^+\infty~lim_ q\rightarrow~+\infty~x_n,q

(interversion de la limite et du signe somme)

Démonstration L'inégalité x_n,q\leq
\alpha_n, celle qui s'en déduit par passage à la limite
y_n\leq \alpha_n et la convergence de la
série \\sum ~
\alpha_n montrent les convergences absolues des séries
\\sum ~
_nx_n,q et
\\sum ~
_ny_n. Prenons donc \epsilon \textgreater{} 0 et choisissons M
tel que \\sum ~
_n=M+1^+\infty~\alpha_n \textless{} \epsilon\over
4. On a alors

\begin{align*} \left
\sum _n=0^+\infty~y_
n -\sum _n=0^+\infty~x_
n,q\right & \leq& \\sum
_n=0^+\infty~y_ n -
x_n,q\leq\\sum
_n=0^My_ n - x_n,q
+ \\sum
_n=M+1^+\infty~(y_ n +
x_n,q)\%& \\
& \leq& \\sum
_n=0^My_ n - x_n,q
+ 2\sum _n=M+1^+\infty~\alpha_ n~
\leq\sum _n=0^My_
n - x_n,q + \epsilon\over 2
\%&\\ \end{align*}

Maintenant, on a
lim_q\rightarrow~+\infty~~\\\sum
 _n=0^My_n -
x_n,q = 0 (chacun des termes de cette somme admet 0
pour limite), et donc il existe N \in \mathbb{N}~ tel que q ≥ N
\rigtharrow~\\sum ~
_n=0^My_n - x_n,q
\textless{} \epsilon\over 2. On a donc

q ≥ N \rigtharrow~\left \\sum
_n=0^+\infty~y_ n -\\sum
_n=0^+\infty~x_ n,q\right \leq
\epsilon\over 2 + \epsilon\over 2 = \epsilon

ce qui montre que la suite \left
(\\sum ~
_n=0^+\infty~x_n,q\right )_q\in\mathbb{N}~
admet la limite \\sum ~
_n=0^+\infty~y_n quand q tend vers + \infty~.

Remarque~7.7.1 Le lecteur qui a dé\\jmathmathà des connaissances sur les séries de
fonctions, remarquera qu'il s'agit là tout simplement du théorème
d'interversion des limites dans le cas de convergence normale (donc
uniforme) d'une série de fonctions.

Nous pouvons maintenant démontrer le théorème d'interversion des signes
somme dans les séries doubles.

Théorème~7.7.2 Soit u = (u_n,p)_(n,p)\in\mathbb{N}~\times\mathbb{N}~ une famille
de nombres réels ou complexes indexée par \mathbb{N}~ \times \mathbb{N}~. On suppose que

\begin{itemize}
\itemsep1pt\parskip0pt\parsep0pt
\item
  pour tout entier n la série
  \\sum ~
  _pu_n,p est absolument convergente
\item
  la série \\sum ~
  _n \\sum ~
  _p=0^+\infty~u_n,p est
  convergente
\end{itemize}

Alors les séries \\\sum
 _n\left
(\\sum ~
_p=0^+\infty~u_n,p\right ) et
\\sum ~
_p\left
(\\sum ~
_n=0^+\infty~u_n,p\right ) sont
convergentes et on a

\sum _n=0^+\infty~~\left
(\sum _p=0^+\infty~u_
n,p\right ) = \\sum
_p=0^+\infty~\left (\\sum
_n=0^+\infty~u_ n,p\right )

Démonstration Nous allons appliquer le lemme précédent en posant
x_n,q =\ \\sum
 _p=0^qu_n,p et \alpha_n
= \\sum ~
_p=0^+\infty~u_n,p et bien entendu
y_n = \\sum ~
_p=0^+\infty~u_n,p =\
lim_q\rightarrow~+\infty~x_n,q. Les hypothèses du lemme étant
évidemment vérifiées, on sait que
\\sum ~
_n=0^+\infty~x_n,q admet la limite
\\sum ~
_n=0^+\infty~y_n quand q tend vers + \infty~. Mais, puisque
l'on a l'égalité
u_n,p\leq\\\sum
 _p=0^+\infty~u_n,p, la série
\\sum ~
_nu_n,p est absolument convergente pour tout p \in \mathbb{N}~ et
donc, par linéarité de la somme,

\sum _n=0^+\infty~x_ n,q~ =
\\sum
_n=0^+\infty~\\sum
_p=0^qu_ n,p = \\sum
_p=0^q \\sum
_n=0^+\infty~u_ n,p

L'existence de
lim_q\rightarrow~+\infty~~\\\sum
 _n=0^+\infty~x_n,q montre donc que la série
\\sum ~
_p \\sum ~
_n=0^+\infty~u_n,p est convergente et a pour somme
\\sum ~
_n=0^+\infty~y_n =\
\sum ~
_n=0^+\infty~\\\sum
 _p=0^+\infty~u_n,p autrement dit que

\sum _n=0^+\infty~~\left
(\sum _p=0^+\infty~u_
n,p\right ) = \\sum
_p=0^+\infty~\left (\\sum
_n=0^+\infty~u_ n,p\right )

Remarque~7.7.2 En appliquant le théorème à la suite u' =
(u_n,p)_(n,p)\in\mathbb{N}~\times\mathbb{N}~, on constate que
la série \\sum ~
_p\left
(\\sum ~
_n=0^+\infty~u_n,p\right
) est convergente, ce qui implique la convergence absolue de la série
\\sum ~
_p \\sum ~
_n=0^+\infty~u_n,p.

Remarque~7.7.3 On pourra retenir le théorème précédent sous la forme
suivante

\sum _n=0^+\infty~~\left
(\sum _p=0^+\infty~u_
n,p\right ) \textless{}
+\infty~\rigtharrow~\\sum
_n=0^+\infty~\left (\\sum
_p=0^+\infty~u_ n,p\right ) =
\sum _p=0^+\infty~~\left
(\sum _n=0^+\infty~u_
n,p\right )

{[}
{[}
{[}
{[}

\end{document}
