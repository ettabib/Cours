\documentclass[]{article}
\usepackage[T1]{fontenc}
\usepackage{lmodern}
\usepackage{amssymb,amsmath}
\usepackage{ifxetex,ifluatex}
\usepackage{fixltx2e} % provides \textsubscript
% use upquote if available, for straight quotes in verbatim environments
\IfFileExists{upquote.sty}{\usepackage{upquote}}{}
\ifnum 0\ifxetex 1\fi\ifluatex 1\fi=0 % if pdftex
  \usepackage[utf8]{inputenc}
\else % if luatex or xelatex
  \ifxetex
    \usepackage{mathspec}
    \usepackage{xltxtra,xunicode}
  \else
    \usepackage{fontspec}
  \fi
  \defaultfontfeatures{Mapping=tex-text,Scale=MatchLowercase}
  \newcommand{\euro}{€}
\fi
% use microtype if available
\IfFileExists{microtype.sty}{\usepackage{microtype}}{}
\usepackage{graphicx}
% Redefine \includegraphics so that, unless explicit options are
% given, the image width will not exceed the width of the page.
% Images get their normal width if they fit onto the page, but
% are scaled down if they would overflow the margins.
\makeatletter
\def\ScaleIfNeeded{%
  \ifdim\Gin@nat@width>\linewidth
    \linewidth
  \else
    \Gin@nat@width
  \fi
}
\makeatother
\let\Oldincludegraphics\includegraphics
{%
 \catcode`\@=11\relax%
 \gdef\includegraphics{\@ifnextchar[{\Oldincludegraphics}{\Oldincludegraphics[width=\ScaleIfNeeded]}}%
}%
\ifxetex
  \usepackage[setpagesize=false, % page size defined by xetex
              unicode=false, % unicode breaks when used with xetex
              xetex]{hyperref}
\else
  \usepackage[unicode=true]{hyperref}
\fi
\hypersetup{breaklinks=true,
            bookmarks=true,
            pdfauthor={},
            pdftitle={Arcs parametres},
            colorlinks=true,
            citecolor=blue,
            urlcolor=blue,
            linkcolor=magenta,
            pdfborder={0 0 0}}
\urlstyle{same}  % don't use monospace font for urls
\setlength{\parindent}{0pt}
\setlength{\parskip}{6pt plus 2pt minus 1pt}
\setlength{\emergencystretch}{3em}  % prevent overfull lines
\setcounter{secnumdepth}{0}
 
/* start css.sty */
.cmr-5{font-size:50%;}
.cmr-7{font-size:70%;}
.cmmi-5{font-size:50%;font-style: italic;}
.cmmi-7{font-size:70%;font-style: italic;}
.cmmi-10{font-style: italic;}
.cmsy-5{font-size:50%;}
.cmsy-7{font-size:70%;}
.cmex-7{font-size:70%;}
.cmex-7x-x-71{font-size:49%;}
.msbm-7{font-size:70%;}
.cmtt-10{font-family: monospace;}
.cmti-10{ font-style: italic;}
.cmbx-10{ font-weight: bold;}
.cmr-17x-x-120{font-size:204%;}
.cmsl-10{font-style: oblique;}
.cmti-7x-x-71{font-size:49%; font-style: italic;}
.cmbxti-10{ font-weight: bold; font-style: italic;}
p.noindent { text-indent: 0em }
td p.noindent { text-indent: 0em; margin-top:0em; }
p.nopar { text-indent: 0em; }
p.indent{ text-indent: 1.5em }
@media print {div.crosslinks {visibility:hidden;}}
a img { border-top: 0; border-left: 0; border-right: 0; }
center { margin-top:1em; margin-bottom:1em; }
td center { margin-top:0em; margin-bottom:0em; }
.Canvas { position:relative; }
li p.indent { text-indent: 0em }
.enumerate1 {list-style-type:decimal;}
.enumerate2 {list-style-type:lower-alpha;}
.enumerate3 {list-style-type:lower-roman;}
.enumerate4 {list-style-type:upper-alpha;}
div.newtheorem { margin-bottom: 2em; margin-top: 2em;}
.obeylines-h,.obeylines-v {white-space: nowrap; }
div.obeylines-v p { margin-top:0; margin-bottom:0; }
.overline{ text-decoration:overline; }
.overline img{ border-top: 1px solid black; }
td.displaylines {text-align:center; white-space:nowrap;}
.centerline {text-align:center;}
.rightline {text-align:right;}
div.verbatim {font-family: monospace; white-space: nowrap; text-align:left; clear:both; }
.fbox {padding-left:3.0pt; padding-right:3.0pt; text-indent:0pt; border:solid black 0.4pt; }
div.fbox {display:table}
div.center div.fbox {text-align:center; clear:both; padding-left:3.0pt; padding-right:3.0pt; text-indent:0pt; border:solid black 0.4pt; }
div.minipage{width:100%;}
div.center, div.center div.center {text-align: center; margin-left:1em; margin-right:1em;}
div.center div {text-align: left;}
div.flushright, div.flushright div.flushright {text-align: right;}
div.flushright div {text-align: left;}
div.flushleft {text-align: left;}
.underline{ text-decoration:underline; }
.underline img{ border-bottom: 1px solid black; margin-bottom:1pt; }
.framebox-c, .framebox-l, .framebox-r { padding-left:3.0pt; padding-right:3.0pt; text-indent:0pt; border:solid black 0.4pt; }
.framebox-c {text-align:center;}
.framebox-l {text-align:left;}
.framebox-r {text-align:right;}
span.thank-mark{ vertical-align: super }
span.footnote-mark sup.textsuperscript, span.footnote-mark a sup.textsuperscript{ font-size:80%; }
div.tabular, div.center div.tabular {text-align: center; margin-top:0.5em; margin-bottom:0.5em; }
table.tabular td p{margin-top:0em;}
table.tabular {margin-left: auto; margin-right: auto;}
div.td00{ margin-left:0pt; margin-right:0pt; }
div.td01{ margin-left:0pt; margin-right:5pt; }
div.td10{ margin-left:5pt; margin-right:0pt; }
div.td11{ margin-left:5pt; margin-right:5pt; }
table[rules] {border-left:solid black 0.4pt; border-right:solid black 0.4pt; }
td.td00{ padding-left:0pt; padding-right:0pt; }
td.td01{ padding-left:0pt; padding-right:5pt; }
td.td10{ padding-left:5pt; padding-right:0pt; }
td.td11{ padding-left:5pt; padding-right:5pt; }
table[rules] {border-left:solid black 0.4pt; border-right:solid black 0.4pt; }
.hline hr, .cline hr{ height : 1px; margin:0px; }
.tabbing-right {text-align:right;}
span.TEX {letter-spacing: -0.125em; }
span.TEX span.E{ position:relative;top:0.5ex;left:-0.0417em;}
a span.TEX span.E {text-decoration: none; }
span.LATEX span.A{ position:relative; top:-0.5ex; left:-0.4em; font-size:85%;}
span.LATEX span.TEX{ position:relative; left: -0.4em; }
div.float img, div.float .caption {text-align:center;}
div.figure img, div.figure .caption {text-align:center;}
.marginpar {width:20%; float:right; text-align:left; margin-left:auto; margin-top:0.5em; font-size:85%; text-decoration:underline;}
.marginpar p{margin-top:0.4em; margin-bottom:0.4em;}
.equation td{text-align:center; vertical-align:middle; }
td.eq-no{ width:5%; }
table.equation { width:100%; } 
div.math-display, div.par-math-display{text-align:center;}
math .texttt { font-family: monospace; }
math .textit { font-style: italic; }
math .textsl { font-style: oblique; }
math .textsf { font-family: sans-serif; }
math .textbf { font-weight: bold; }
.partToc a, .partToc, .likepartToc a, .likepartToc {line-height: 200%; font-weight:bold; font-size:110%;}
.chapterToc a, .chapterToc, .likechapterToc a, .likechapterToc, .appendixToc a, .appendixToc {line-height: 200%; font-weight:bold;}
.index-item, .index-subitem, .index-subsubitem {display:block}
.caption td.id{font-weight: bold; white-space: nowrap; }
table.caption {text-align:center;}
h1.partHead{text-align: center}
p.bibitem { text-indent: -2em; margin-left: 2em; margin-top:0.6em; margin-bottom:0.6em; }
p.bibitem-p { text-indent: 0em; margin-left: 2em; margin-top:0.6em; margin-bottom:0.6em; }
.paragraphHead, .likeparagraphHead { margin-top:2em; font-weight: bold;}
.subparagraphHead, .likesubparagraphHead { font-weight: bold;}
.quote {margin-bottom:0.25em; margin-top:0.25em; margin-left:1em; margin-right:1em; text-align:justify;}
.verse{white-space:nowrap; margin-left:2em}
div.maketitle {text-align:center;}
h2.titleHead{text-align:center;}
div.maketitle{ margin-bottom: 2em; }
div.author, div.date {text-align:center;}
div.thanks{text-align:left; margin-left:10%; font-size:85%; font-style:italic; }
div.author{white-space: nowrap;}
.quotation {margin-bottom:0.25em; margin-top:0.25em; margin-left:1em; }
h1.partHead{text-align: center}
.sectionToc, .likesectionToc {margin-left:2em;}
.subsectionToc, .likesubsectionToc {margin-left:4em;}
.subsubsectionToc, .likesubsubsectionToc {margin-left:6em;}
.frenchb-nbsp{font-size:75%;}
.frenchb-thinspace{font-size:75%;}
.figure img.graphics {margin-left:10%;}
/* end css.sty */

\title{Arcs parametres}
\author{}
\date{}

\begin{document}
\maketitle

\textbf{Warning: \href{http://www.math.union.edu/locate/jsMath}{jsMath}
requires JavaScript to process the mathematics on this page.\\ If your
browser supports JavaScript, be sure it is enabled.}

\begin{center}\rule{3in}{0.4pt}\end{center}

{[}\href{coursse97.html}{next}{]}
{[}\hyperref[tailcoursse96.html]{tail}{]}
{[}\href{coursch19.html\#coursse96.html}{up}{]}

\subsubsection{18.1 Arcs paramétrés}

\paragraph{18.1.1 Vocabulaire}

Définition~18.1.1 Soit E un espace vectoriel normé sur ℝ. On appelle arc
paramétré de classe \{C\}\^{}\{k\} de E tout couple (I,F) d'un
intervalle I de ℝ et d'une application F : I → E de classe
\{C\}\^{}\{k\}.

Remarque~18.1.1 Vocabulaire associé. Soit Γ = (I,F) un arc paramétré de
classe \{C\}\^{}\{k\} de E.

\begin{itemize}
\itemsep1pt\parskip0pt\parsep0pt
\item
  (i) On appelle point de l'arc Γ un élément \{t\}\_\{0\} ∈ I
\item
  (ii) On appelle image ou support de Γ la partie F(I) de E
\item
  (iii) On appelle multiplicité d'un point m de l'image de Γ le cardinal
  de l'ensemble \{F\}\^{}\{−1\}(\textbackslash{}\{m\textbackslash{}\})
  (éventuellement infinie)~; on dit qu'un point de l'image est simple
  s'il est de multiplicité 1, sinon on dit qu'il est multiple~; on dit
  que l'arc est simple si tout point de l'image est de multiplicité 1
\item
  (iv) On dit qu'un point \{t\}\_\{0\} ∈ I de l'arc Γ est régulier si
  F'(\{t\}\_\{0\})\textbackslash{}mathrel\{≠\}0~; on dit que l'arc est
  régulier si tout point de l'arc est régulier~; un point non régulier
  est dit singulier.
\end{itemize}

\paragraph{18.1.2 Equivalence des arcs paramétrés}

Définition~18.1.2 Soit E un ℝ espace vectoriel normé, (I,F) et (J,G)
deux arcs paramétrés de classe \{C\}\^{}\{k\}. On dit que ces deux arcs
sont \{C\}\^{}\{k\}-équivalents s'il existe un difféomorphisme θ : I → J
de classe \{C\}\^{}\{k\} tel que F = G ∘ θ.

Remarque~18.1.2 On dira qu'un tel difféomorphisme est un changement de
paramétrage admissible. L'étude des arcs paramétrés concerne
essentiellement l'étude des propriétés des arcs qui sont invariantes par
équivalence. L'application θ étant bijective on voit immédiatement que

Proposition~18.1.1 Soit (I,F) et (J,G) deux arcs paramétrés de classe
\{C\}\^{}\{k\} qui sont \{C\}\^{}\{k\}-équivalents. Alors les deux arcs
ont la même image. Tous les points de l'image ont la même multiplicité
pour les deux arcs. En particulier un point de l'image est simple pour
l'un si et seulement si~il est simple pour l'autre.

Remarque~18.1.3 Si on a F = G ∘ θ, on a F'(\{t\}\_\{0\}) =
θ'(\{t\}\_\{0\})G'(θ(\{t\}\_\{0\})). Si θ est un difféomorphisme, on a
θ'(\{t\}\_\{0\})\textbackslash{}mathrel\{≠\}0 et donc
F'(\{t\}\_\{0\})\textbackslash{}mathrel\{≠\}0
\textbackslash{}mathrel\{⇔\}
G'(θ(\{t\}\_\{0\}))\textbackslash{}mathrel\{≠\}0. D'où

Proposition~18.1.2 Soit (I,F) et (J,G) deux arcs paramétrés de classe
\{C\}\^{}\{k\} qui sont \{C\}\^{}\{k\}-équivalents, θ : I → J un
difféomorphisme de classe \{C\}\^{}\{k\} tel que F = G ∘ θ. Alors
\{t\}\_\{0\} est un point régulier de (I,F) si et seulement
si~θ(\{t\}\_\{0\}) est un point régulier de (J,G). En particulier, (I,F)
est régulier si et seulement si~(J,G) l'est.

\paragraph{18.1.3 Orientation}

Un difféomorphisme d'un intervalle de ℝ sur un autre intervalle de ℝ est
soit croissant soit décroissant. On peut donc définir

Définition~18.1.3 Soit (I,F) et (J,G) deux arcs paramétrés de classe
\{C\}\^{}\{k\} qui sont \{C\}\^{}\{k\}-équivalents, θ : I → J un
difféomorphisme de classe \{C\}\^{}\{k\} tel que F = G ∘ θ. On dit que
(I,F) et (J,G) sont de même sens si θ est croissant, de sens contraire
si θ est décroissant.

Remarque~18.1.4 Dans certains cas, il peut arriver (de manière assez
exceptionnelle) qu'il existe deux difféomorphismes \{θ\}\_\{1\} et
\{θ\}\_\{2\} tels que F = G ∘ \{θ\}\_\{1\} et F = G ∘ \{θ\}\_\{2\}, l'un
étant croissant et l'autre décroissant. Autrement dit deux arcs
paramétrés peuvent être à la fois de même sens et de sens contraire. On
dit alors que les arcs paramétrés ne sont pas orientables. Un exemple
typique est l'arc de \{ℝ\}\^{}\{2\}~: Γ =
(ℝ,t\textbackslash{}mathrel\{↦\}(\{t\}\^{}\{2\},\{t\}\^{}\{2\})) où
l'arc est de sens contraire à lui même puisque F ∘ θ = F pour θ(t) = −t.
Le lecteur montrera facilement que cette situation ne peut pas se
produire dès que l'arc est régulier ou bien dès que l'arc a au moins
deux points simples.

\paragraph{18.1.4 Tangente à un arc paramétré}

Définition~18.1.4 Soit Γ = (I,F) un arc paramétré de classe
\{C\}\^{}\{k\}, k ∈ \{ℕ\}\^{}\{∗\}∪\textbackslash{}\{ +
∞\textbackslash{}\}. On dit que \{t\}\_\{0\} ∈ I est un point non
totalement singulier de Γ s'il existe n ≤ k tel que
\{F\}\^{}\{(n)\}(\{t\}\_\{0\})\textbackslash{}mathrel\{≠\}0.

Remarque~18.1.5 Soit Γ = (I,F) un arc paramétré de classe \{C\}\^{}\{k\}
et \{t\}\_\{0\} ∈ I un point non totalement singulier de Γ. Considérons
p =\textbackslash{}mathop\{ min\}\textbackslash{}\{n ≤
k\textbackslash{}mathrel\{∣\}\{F\}\^{}\{(n)\}(\{t\}\_\{0\})\textbackslash{}mathrel\{≠\}0\textbackslash{}\}.
La formule de Taylor montre alors que

F(t) = F(\{t\}\_\{0\}) +\{ \{(t − \{t\}\_\{0\})\}\^{}\{p\}
\textbackslash{}over p!\} \{F\}\^{}\{(p)\}(\{t\}\_\{ 0\}) + \{(t −
\{t\}\_\{0\})\}\^{}\{p\}ε(t − \{t\}\_\{ 0\})

avec \{\textbackslash{}mathop\{lim\}\}\_\{u→0\}ε(u) = 0. On en déduit
que \{\textbackslash{}mathop\{lim\}\}\_\{t→\{t\}\_\{0\}\}\{
F(t)−F(\{t\}\_\{0\}) \textbackslash{}over
\{(t−\{t\}\_\{0\})\}\^{}\{p\}\} =\{ 1 \textbackslash{}over p!\}
\{F\}\^{}\{(p)\}(\{t\}\_\{ 0\}). Ceci montre tout d'abord que pour
t\textbackslash{}mathrel\{≠\}\{t\}\_\{0\} mais assez proche de
\{t\}\_\{0\}, on a F(t)\textbackslash{}mathrel\{≠\}F(\{t\}\_\{0\}) et
que d'autre part le vecteur \{ F(t)−F(\{t\}\_\{0\}) \textbackslash{}over
\{(t−\{t\}\_\{0\})\}\^{}\{p\}\} qui est un vecteur directeur de la
droite affine qui passe par les points F(t) et F(\{t\}\_\{0\}), admet
pour limite le vecteur \{ 1 \textbackslash{}over p!\}
\{F\}\^{}\{(p)\}(\{t\}\_\{ 0\}). On peut encore dire que la corde
joignant les points F(\{t\}\_\{0\}) et F(t) admet pour limite la droite
affine F(\{t\}\_\{0\}) + ℝ\{F\}\^{}\{(p)\}(\{t\}\_\{0\}) ce qui justifie
la définition suivante~:

Définition~18.1.5 Soit Γ = (I,F) un arc paramétré de classe
\{C\}\^{}\{k\} et \{t\}\_\{0\} ∈ I un point non totalement singulier de
Γ. Soit p =\textbackslash{}mathop\{ min\}\textbackslash{}\{n ≤
k\textbackslash{}mathrel\{∣\}\{F\}\^{}\{(n)\}(\{t\}\_\{0\})\textbackslash{}mathrel\{≠\}0\textbackslash{}\}.
On appelle tangente à Γ au point \{t\}\_\{0\} la droite affine
F(\{t\}\_\{0\}) + ℝ\{F\}\^{}\{(p)\}(\{t\}\_\{0\}).

Proposition~18.1.3 La notion de tangente est invariante par changement
de paramétrage admissible. Soit (I,F) et (J,G) deux arcs paramétrés de
classe \{C\}\^{}\{k\} qui sont \{C\}\^{}\{k\}-équivalents, θ : I → J un
difféomorphisme de classe \{C\}\^{}\{k\} tel que F = G ∘ θ. Alors
\{t\}\_\{0\} est un point non totalement singulier de (I,F) si et
seulement si~θ(\{t\}\_\{0\}) est un point non totalement singulier de
(J,G), et dans ce cas la tangente à (I,F) au point \{t\}\_\{0\} est
égale à la tangente à (J,G) au point θ(\{t\}\_\{0\}).

Démonstration Nous allons montrer par récurrence sur n ≥ 1 que

\{F\}\^{}\{(n)\}(t) = θ'\{(t)\}\^{}\{n\}\{G\}\^{}\{(n)\}(θ(t)) +\{
\textbackslash{}mathop\{∑ \}\}\_\{j=1\}\^{}\{n−1\}\{a\}\_\{
j,n\}(t)\{G\}\^{}\{(j)\}(θ(t))

où les applications \{a\}\_\{j,n\} sont des applications de I dans ℝ de
classe \{C\}\^{}\{k−n\}. C'est clair pour n = 1 puisque F'(t) =
θ'(t)G'(θ(t)). Supposons donc le résultat vrai pour n ≤ k − 1. Alors
toutes les applications considérées étant de classe \{C\}\^{}\{1\}, on a

\textbackslash{}begin\{eqnarray*\}\{ F\}\^{}\{(n+1)\}(t)\& =\&
θ'\{(t)\}\^{}\{n+1\}\{G\}\^{}\{(n+1)\}(θ(t)) +
nθ'`\{(t)\}\^{}\{n\}θ'\{(t)\}\^{}\{n−1\}\{G\}\^{}\{(n)\}(θ(t)) \%\&
\textbackslash{}\textbackslash{} \& \& +\{\textbackslash{}mathop\{∑
\}\}\_\{j=1\}\^{}\{n−1\}\{a\}\_\{ j,n\}(t)θ'(t)\{G\}\^{}\{(j+1)\}(θ(t))
+\{ \textbackslash{}mathop\{∑ \}\}\_\{j=1\}\^{}\{n−1\}\{a\}\_\{
j,n\}'(t)\{G\}\^{}\{(j)\}(θ(t))\%\& \textbackslash{}\textbackslash{} \&
=\& θ'\{(t)\}\^{}\{n+1\}\{G\}\^{}\{(n+1)\}(θ(t)) +\{
\textbackslash{}mathop\{∑ \}\}\_\{j=1\}\^{}\{n\}\{a\}\_\{
j,n+1\}(t)\{G\}\^{}\{(j)\}(θ(t)) \%\& \textbackslash{}\textbackslash{}
\textbackslash{}end\{eqnarray*\}

avec \{a\}\_\{n,n+1\}(t) = \{a\}\_\{n−1,n\}θ'(t) +
nθ'`(t)θ'\{(t)\}\^{}\{n−1\} et pour j ≤ n − 1, \{a\}\_\{j,n+1\}(t) =
\{a\}\_\{j,n\}(t)θ'(t) + \{a\}\_\{j−1,n\}'(t), ce qui achève la
récurrence.

Si on a alors G'(θ(\{t\}\_\{0\})) =
\textbackslash{}mathop\{\textbackslash{}mathop\{\ldots{}\}\} =
\{G\}\^{}\{(p−1)\}(θ(\{t\}\_\{0\})) = 0 et
\{G\}\^{}\{(p)\}(θ(\{t\}\_\{0\}))\textbackslash{}mathrel\{≠\}0, on voit
immédiatement que F'(\{t\}\_\{0\}) =
\textbackslash{}mathop\{\textbackslash{}mathop\{\ldots{}\}\} =
\{F\}\^{}\{(p−1)\}(\{t\}\_\{0\}) = 0 et que
\{F\}\^{}\{(p)\}(\{t\}\_\{0\}) =
θ'\{(\{t\}\_\{0\})\}\^{}\{p\}\{G\}\^{}\{(p)\}(θ(\{t\}\_\{0\}))\textbackslash{}mathrel\{≠\}0
car θ'(\{t\}\_\{0\})\textbackslash{}mathrel\{≠\}0. On en déduit que si
θ(\{t\}\_\{0\}) est un point non totalement singulier de (J,G), alors
\{t\}\_\{0\} est un point non totalement singulier de (I,F) avec le même
indice p. L'équivalence en résulte vu le rôle symétrique des deux arcs.
De plus on a G(θ(\{t\}\_\{0\})) + ℝ\{G\}\^{}\{(p)\}(θ(\{t\}\_\{0\})) =
F(\{t\}\_\{0\}) + ℝ\{F\}\^{}\{(p)\}(\{t\}\_\{0\}) (puisque les deux
vecteurs tangents sont colinéaires) ce qui montre que les tangentes sont
les mêmes.

Remarque~18.1.6 Considérons F : ℝ → \{ℝ\}\^{}\{2\} définie par F(t) =
(\{e\}\^{}\{−1∕\{t\}\^{}\{2\} \},0) si t \textgreater{} 0, F(0) = (0,0)
et F(t) = (0,\{e\}\^{}\{−1∕\{t\}\^{}\{2\} \}) si t \textless{} 0. On
constate facilement que F est de classe \{C\}\^{}\{∞\} en remarquant que
\{\textbackslash{}mathop\{lim\}\}\_\{t→0,t\textbackslash{}mathrel\{≠\}0\}\{F\}\^{}\{(k)\}(t)
= 0 et en appliquant un théorème du cours sur les fonctions d'une
variable. Pourtant, l'image de F est la réunion de deux segments faisant
en F(0) un angle droit. Il n'est donc pas question de pouvoir définir de
manière raisonnable une tangente à un arc (fût-il \{C\}\^{}\{∞\}) en un
point totalement singulier.

\paragraph{18.1.5 Plan osculateur, concavité}

Définition~18.1.6 Soit Γ = (I,F) un arc paramétré de classe
\{C\}\^{}\{k\}, k ≥ 2. On dit que \{t\}\_\{0\} ∈ I est un point
birégulier de Γ si la famille (F'(\{t\}\_\{0\}),F''(\{t\}\_\{0\})) est
libre. Un arc est dit birégulier si tous ses points sont biréguliers.

Remarque~18.1.7 Un point birégulier est nécessairement régulier puisque
F'(\{t\}\_\{0\})\textbackslash{}mathrel\{≠\}0.

Définition~18.1.7 Soit Γ = (I,F) un arc paramétré de classe
\{C\}\^{}\{k\}, k ≥ 2. Soit \{t\}\_\{0\} ∈ I un point birégulier de Γ.
On appelle plan osculateur au point \{t\}\_\{0\} le plan affine
F(\{t\}\_\{0\}) +\textbackslash{}mathop\{
\textbackslash{}mathrm\{Vect\}\}(F'(\{t\}\_\{0\}),F''(\{t\}\_\{0\})).

Théorème~18.1.4 Les notions de point birégulier et de plan osculateur
sont invariantes par changement de paramétrage admissible. Soit (I,F) et
(J,G) deux arcs paramétrés de classe \{C\}\^{}\{k\} qui sont
\{C\}\^{}\{k\}-équivalents, θ : I → J un difféomorphisme de classe
\{C\}\^{}\{k\} tel que F = G ∘ θ. Alors \{t\}\_\{0\} est un point
birégulier de (I,F) si et seulement si~θ(\{t\}\_\{0\}) est un point
birégulier de (J,G), et dans ce cas le plan osculateur à (I,F) au point
\{t\}\_\{0\} est égal au plan osculateur à (J,G) au point
θ(\{t\}\_\{0\}).

Démonstration On a F'(\{t\}\_\{0\}) =
θ'(\{t\}\_\{0\})G'(θ(\{t\}\_\{0\})) et F'`(\{t\}\_\{0\}) =
θ'`(\{t\}\_\{0\})G'(θ(\{t\}\_\{0\})) +
θ'\{(\{t\}\_\{0\})\}\^{}\{2\}G''(θ(\{t\}\_\{0\})). Donc la matrice des
coordonnées de la famille (F'(\{t\}\_\{0\}),F''(\{t\}\_\{0\})) par
rapport à la famille (G'(θ(\{t\}\_\{0\})),G''(θ(\{t\}\_\{0\}))) est la
matrice inversible \textbackslash{}left
(\textbackslash{}matrix\{\textbackslash{},θ'(\{t\}\_\{0\})\&θ'`(\{t\}\_\{0\})
\textbackslash{}cr 0
\&θ'\{(\{t\}\_\{0\})\}\^{}\{2\}\}\textbackslash{}right ), ce qui montre
que (F'(\{t\}\_\{0\}),F''(\{t\}\_\{0\})) libre équivaut à
(G'(θ(\{t\}\_\{0\})),G''(θ(\{t\}\_\{0\}))) libre. Dans ce cas on voit
que
\textbackslash{}mathop\{\textbackslash{}mathrm\{Vect\}\}(G'(θ(\{t\}\_\{0\})),G'`(θ(\{t\}\_\{0\})))
=\textbackslash{}mathop\{
\textbackslash{}mathrm\{Vect\}\}(F'(\{t\}\_\{0\}),F''(\{t\}\_\{0\})), ce
qui montre que les plans osculateurs qui passent tous deux par le point
image F(\{t\}\_\{0\}) = G(θ(\{t\}\_\{0\})) coïncident.

Remarque~18.1.8 De plus la coordonnée de F''(\{t\}\_\{0\}) par rapport à
G''(θ(\{t\}\_\{0\})) est égale à θ'\{(\{t\}\_\{0\})\}\^{}\{2\}
\textgreater{} 0 ce qui montre que les demi-plans F(\{t\}\_\{0\}) +
ℝF'(\{t\}\_\{0\}) + \{ℝ\}\^{}\{+∗\}F''(\{t\}\_\{0\}) et
G(θ(\{t\}\_\{0\})) + ℝG'(θ(\{t\}\_\{0\})) +
\{ℝ\}\^{}\{+∗\}G''(θ(\{t\}\_\{0\})) coïncident. Ce qui amène à
introduire

Définition~18.1.8 Soit Γ = (I,F) un arc paramétré de classe
\{C\}\^{}\{k\}, k ≥ 2. Soit \{t\}\_\{0\} ∈ I un point birégulier de Γ.
On appelle demi-plan de concavité au point \{t\}\_\{0\} le demi plan
affine (inclus dans le plan osculateur et délimité par la tangente)
F(\{t\}\_\{0\}) + ℝF'(\{t\}\_\{0\}) + \{ℝ\}\^{}\{+∗\}F''(\{t\}\_\{0\}).
Il est invariant par changement de paramétrage admissible.

\paragraph{18.1.6 Etude locale des arcs plans}

Nous supposerons dans ce paragraphe et dans les suivants que
\textbackslash{}mathop\{dim\} E = 2. Soit Γ = (I,F) un arc paramétré de
classe \{C\}\^{}\{k\}, k ≥ 2 et \{t\}\_\{0\} un point non totalement
singulier de l'arc. Soit p =\textbackslash{}mathop\{
min\}\textbackslash{}\{n ≤
k\textbackslash{}mathrel\{∣\}\{F\}\^{}\{(n)\}(\{t\}\_\{0\})\textbackslash{}mathrel\{≠\}0\textbackslash{}\}
et nous supposerons que l'on peut trouver n ∈ {[}p + 1,k{]} tel que la
famille (\{F\}\^{}\{(p)\}(\{t\}\_\{0\}),\{F\}\^{}\{(n)\}(\{t\}\_\{0\}))
soit une famille libre. Nous considérerons le plus petit entier n ayant
cette propriété, que nous noterons q. La famille
(\{F\}\^{}\{(p)\}(\{t\}\_\{0\}),\{F\}\^{}\{(q)\}(\{t\}\_\{0\})) est donc
une base de E, et donc nous pouvons écrire F(t) − F(\{t\}\_\{0\}) =
x(t)\{F\}\^{}\{(p)\}(\{t\}\_\{0\}) + y(t)\{F\}\^{}\{(q)\}(\{t\}\_\{0\}).
Les nombres réels x(t) et y(t) apparaissent donc comme les coordonnées
du point F(t) dans le repère affine
(F(\{t\}\_\{0\}),(\{F\}\^{}\{(p)\}(\{t\}\_\{0\}),\{F\}\^{}\{(q)\}(\{t\}\_\{0\}))).

Des renseignements essentiels sur l'allure de la courbe au voisinage du
point \{t\}\_\{0\} sont fournis par l'étude des signes de x(t) et de
y(t) au voisinage de \{t\}\_\{0\}, sachant que la courbe passe par le
point F(\{t\}\_\{0\}) et est tangente au vecteur
\{F\}\^{}\{(p)\}(\{t\}\_\{0\}). La formule de Taylor Young donne

\textbackslash{}begin\{eqnarray*\} F(t)\& =\& F(\{t\}\_\{0\}) +\{ \{(t −
\{t\}\_\{0\})\}\^{}\{p\} \textbackslash{}over p!\}
\{F\}\^{}\{(p)\}(\{t\}\_\{ 0\}) +\{ \textbackslash{}mathop\{∑
\}\}\_\{n=p+1\}\^{}\{q−1\}\{ \{(t − \{t\}\_\{0\})\}\^{}\{n\}
\textbackslash{}over n!\} \{F\}\^{}\{(n)\}(\{t\}\_\{ 0\})\%\&
\textbackslash{}\textbackslash{} \& \& +\{ \{(t −
\{t\}\_\{0\})\}\^{}\{q\} \textbackslash{}over q!\}
\{F\}\^{}\{(q)\}(\{t\}\_\{ 0\}) + \{(t − \{t\}\_\{0\})\}\^{}\{q\}ε(t −
\{t\}\_\{ 0\}) \%\& \textbackslash{}\textbackslash{}
\textbackslash{}end\{eqnarray*\}

avec \{\textbackslash{}mathop\{lim\}\}\_\{u→0\}ε(u) = 0. En posant
\{F\}\^{}\{(n)\}(\{t\}\_\{0\}) =
\{λ\}\_\{n\}\{F\}\^{}\{(p)\}(\{t\}\_\{0\}) pour p + 1 ≤ n ≤ q − 1 et
ε(u) = α(u)\{F\}\^{}\{(p)\}(\{t\}\_\{0\}) +
β(u)\{F\}\^{}\{(q)\}(\{t\}\_\{0\}), on a
\{\textbackslash{}mathop\{lim\}\}\_\{u→0\}α(u)
=\{\textbackslash{}mathop\{ lim\}\}\_\{u→0\}β(u) = 0 et

\textbackslash{}begin\{eqnarray*\} F(t)\& =\& F(\{t\}\_\{0\}) \%\&
\textbackslash{}\textbackslash{} \& +\& \{(t −
\{t\}\_\{0\})\}\^{}\{p\}\textbackslash{}left (\{ 1 \textbackslash{}over
p!\} +\{ \textbackslash{}mathop\{∑ \}\}\_\{n=p+1\}\^{}\{q−1\}\{ \{(t −
\{t\}\_\{0\})\}\^{}\{n−p\} \textbackslash{}over n!\} \{λ\}\_\{n\} + \{(t
− \{t\}\_\{0\})\}\^{}\{q−p\}α(t − \{t\}\_\{ 0\})\textbackslash{}right
)\{F\}\^{}\{(p)\}(\{t\}\_\{ 0\})\%\& \textbackslash{}\textbackslash{} \&
+\& \{(t − \{t\}\_\{0\})\}\^{}\{q\}(\{ 1 \textbackslash{}over q!\} + β(t
− \{t\}\_\{0\}))\{F\}\^{}\{(q)\}(\{t\}\_\{ 0\}) \%\&
\textbackslash{}\textbackslash{} \textbackslash{}end\{eqnarray*\}

ce qui montre que

\textbackslash{}begin\{eqnarray*\} x(t)\& =\& \{(t −
\{t\}\_\{0\})\}\^{}\{p\}\textbackslash{}left (\{ 1 \textbackslash{}over
p!\} +\{ \textbackslash{}mathop\{∑ \}\}\_\{n=p+1\}\^{}\{q−1\}\{ \{(t −
\{t\}\_\{0\})\}\^{}\{n−p\} \textbackslash{}over n!\} \{λ\}\_\{n\} + \{(t
− \{t\}\_\{0\})\}\^{}\{q−p\}α(t − \{t\}\_\{ 0\})\textbackslash{}right
)\%\& \textbackslash{}\textbackslash{} \& ∼\&\{ \{(t −
\{t\}\_\{0\})\}\^{}\{p\} \textbackslash{}over p!\} \%\&
\textbackslash{}\textbackslash{} \textbackslash{}end\{eqnarray*\}

et que

y(t) = \{(t − \{t\}\_\{0\})\}\^{}\{q\}(\{ 1 \textbackslash{}over q!\} +
β(t − \{t\}\_\{0\})) ∼\{ \{(t − \{t\}\_\{0\})\}\^{}\{q\}
\textbackslash{}over q!\}

Mais quand deux fonctions sont équivalentes au voisinage de
\{t\}\_\{0\}, elles ont même signe au voisinage de \{t\}\_\{0\} ce qui
montre qu'il existe η \textgreater{} 0 tel que pour t ∈{]}\{t\}\_\{0\} −
η,\{t\}\_\{0\} + η{[} on ait
\textbackslash{}mathop\{\textbackslash{}mathrm\{sgn\}\} (x(t))
=\textbackslash{}mathop\{ \textbackslash{}mathrm\{sgn\}\} (\{(t −
\{t\}\_\{0\})\}\^{}\{p\}) et
\textbackslash{}mathop\{\textbackslash{}mathrm\{sgn\}\} (y(t))
=\textbackslash{}mathop\{ \textbackslash{}mathrm\{sgn\}\} \{((t −
\{t\}\_\{0\}))\}\^{}\{q\} en posant
\textbackslash{}mathop\{\textbackslash{}mathrm\{sgn\}\} (u) =
\textbackslash{}left \textbackslash{}\{ \textbackslash{}cases\{ 1 \&si u
\textgreater{} 0 \textbackslash{}cr 0 \&si u = 0 \textbackslash{}cr
−1\&si u \textless{} 0 \} \textbackslash{}right .. Ceci conduit à la
discussion suivante

Premier cas~: p impair, q pair (c'est par exemple le cas d'un point
birégulier pour lequel p = 1 et q = 2) Pour t ∈{]}\{t\}\_\{0\} −
η,\{t\}\_\{0\}{[}, on a x(t) \textless{} 0 et y(t) \textgreater{} 0~;
pour t ∈{]}\{t\}\_\{0\},\{t\}\_\{0\} + η{[}, on a x(t) \textgreater{} 0
et y(t) \textgreater{} 0. La courbe reste d'un même coté de sa tangente
F(\{t\}\_\{0\}) + ℝ\{F\}\^{}\{(p)\}(\{t\}\_\{0\}) tout en traversant la
droite F(\{t\}\_\{0\}) + ℝ\{F\}\^{}\{(q)\}(\{t\}\_\{0\})~; on dit alors
que \{t\}\_\{0\} est un point banal de Γ.

\textbackslash{}text\{\includegraphics{cours12x.png}\}

Deuxième cas~: p impair, q impair (c'est le cas le plus courant de point
non birégulier avec p = 1 et q = 3) Pour t ∈{]}\{t\}\_\{0\} −
η,\{t\}\_\{0\}{[}, on a x(t) \textless{} 0 et y(t) \textless{} 0~; pour
t ∈{]}\{t\}\_\{0\},\{t\}\_\{0\} + η{[}, on a x(t) \textgreater{} 0 et
y(t) \textgreater{} 0. La courbe traverse sa tangente F(\{t\}\_\{0\}) +
ℝ\{F\}\^{}\{(p)\}(\{t\}\_\{0\}) tout en traversant la droite
F(\{t\}\_\{0\}) + ℝ\{F\}\^{}\{(q)\}(\{t\}\_\{0\})~; on dit alors que
\{t\}\_\{0\} est un point d'inflexion de Γ.

\textbackslash{}text\{\includegraphics{cours13x.png}\}

Troisième cas~: p pair, q impair (c'est le cas le plus courant de point
singulier avec p = 2 et q = 3) Pour t ∈{]}\{t\}\_\{0\} −
η,\{t\}\_\{0\}{[}, on a x(t) \textgreater{} 0 et y(t) \textless{} 0~;
pour t ∈{]}\{t\}\_\{0\},\{t\}\_\{0\} + η{[}, on a x(t) \textgreater{} 0
et y(t) \textgreater{} 0. La courbe traverse sa tangente F(\{t\}\_\{0\})
+ ℝ\{F\}\^{}\{(p)\}(\{t\}\_\{0\}) tout en restant d'un même coté de la
droite F(\{t\}\_\{0\}) + ℝ\{F\}\^{}\{(q)\}(\{t\}\_\{0\})~; on dit alors
que \{t\}\_\{0\} est un point de rebroussement de première espèce de Γ.

\textbackslash{}text\{\includegraphics{cours14x.png}\}

Quatrième cas~: p pair, q pair Pour t ∈{]}\{t\}\_\{0\} −
η,\{t\}\_\{0\}{[}, on a x(t) \textgreater{} 0 et y(t) \textgreater{} 0
et pour t ∈{]}\{t\}\_\{0\},\{t\}\_\{0\} + η{[}, on a x(t) \textgreater{}
0 et y(t) \textgreater{} 0. La courbe reste d'un même coté de sa
tangente F(\{t\}\_\{0\}) + ℝ\{F\}\^{}\{(p)\}(\{t\}\_\{0\}) tout en
restant d'un même coté de la droite F(\{t\}\_\{0\}) +
ℝ\{F\}\^{}\{(q)\}(\{t\}\_\{0\})~; on dit alors que \{t\}\_\{0\} est un
point de rebroussement de deuxième espèce de Γ.

\textbackslash{}text\{\includegraphics{cours15x.png}\}

On a en particulier au passage démontré le résultat suivant

Théorème~18.1.5 (convexité locale). Soit Γ = (I,F) un arc paramétré de
classe \{C\}\^{}\{k\}, k ≥ 2. Soit \{t\}\_\{0\} ∈ I un point birégulier
de Γ. Alors il existe un η \textgreater{} 0 tel que pour t
∈{]}\{t\}\_\{0\} − η,\{t\}\_\{0\} +
η{[}∖\textbackslash{}\{\{t\}\_\{0\}\textbackslash{}\}, F(t) soit dans le
demi plan ouvert de concavité.

Remarque~18.1.9 On vérifie facilement que les entiers p et q définis
ci-dessus sont invariants par changement de paramétrage admissible. Il
en est donc de même des notions de point banal, point d'inflexion, point
de rebroussement de première et deuxième espèce.

\paragraph{18.1.7 Branches infinies}

Définition~18.1.9 Soit Γ = (I,F) un arc paramétré de classe
\{C\}\^{}\{k\} et α ∈ ℝ ∪\textbackslash{}\{−∞,+∞\textbackslash{}\} une
extrémité de I. On dit que Γ admet en α une branche infinie si
\{\textbackslash{}mathop\{lim\}\}\_\{t→α\}\textbackslash{}\textbar{}F(t)\textbackslash{}\textbar{}
= +∞.

Définition~18.1.10 Soit Γ = (I,F) un arc paramétré de classe
\{C\}\^{}\{k\} et α ∈ ℝ ∪\textbackslash{}\{−∞,+∞\textbackslash{}\} une
extrémité de I où Γ admet une branche infinie. Si
\{\textbackslash{}mathop\{lim\}\}\_\{t→α\}\{ F(t) \textbackslash{}over
\textbackslash{}\textbar{}F(t)\textbackslash{}\textbar{}\}
=\textbackslash{}vec\{ u\} (vecteur nécessairement unitaire), on dit que
Γ admet en α la droite ℝ\textbackslash{}vec\{u\} comme direction
asymptotique.

Définition~18.1.11 Soit Γ = (I,F) un arc paramétré de classe
\{C\}\^{}\{k\} et α ∈ ℝ ∪\textbackslash{}\{−∞,+∞\textbackslash{}\} une
extrémité de I où Γ admet une branche infinie. Soit D une droite affine
de E. On dit que Γ admet en α l'asymptote D si
\{\textbackslash{}mathop\{lim\}\}\_\{t→α\}d(F(t),D) = 0 (où d(F(t),D)
désigne la distance du point F(t) au point D) (distance associée à une
norme quelconque sur E).

Proposition~18.1.6 Soit Γ = (I,F) un arc paramétré de classe
\{C\}\^{}\{k\} et α ∈ ℝ ∪\textbackslash{}\{−∞,+∞\textbackslash{}\} une
extrémité de I où Γ admet une branche infinie. Si Γ admet en α une
droite D comme asymptote, il admet sa direction \textbackslash{}vec\{D\}
comme direction asymptotique.

Démonstration Soit a un point de D. Utilisons une distance associée à
une norme euclidienne sur E et soit P(t) la projection orthogonale de
F(t) sur la droite D. On a alors d(F(t),D) =\textbackslash{}\textbar{}
F(t) − P(t)\textbackslash{}\textbar{}. On a
\textbackslash{}\textbar{}P(t)\textbackslash{}\textbar{}
≥\textbackslash{}\textbar{} F(t)\textbackslash{}\textbar{}
−\textbackslash{}\textbar{} P(t) − F(t)\textbackslash{}\textbar{} ce qui
montre que
\textbackslash{}mathop\{lim\}\textbackslash{}\textbar{}P(t)\textbackslash{}\textbar{}
= +∞. On en déduit que pour t assez proche de α, le point P(t) reste
d'un même côté de a et que donc le vecteur \{
\textbackslash{}overrightarrow\{aP(t)\} \textbackslash{}over
\textbackslash{}\textbar{}\textbackslash{}overrightarrow\{aP(t)\}\textbackslash{}\textbar{}\}
est constant égal à un vecteur unitaire \textbackslash{}vec\{u\} de
\textbackslash{}vec\{D\}. On a alors

\textbackslash{}begin\{eqnarray*\}\{ F(t) \textbackslash{}over
\textbackslash{}\textbar{}F(t)\textbackslash{}\textbar{}\} \& =\&\{ F(t)
− P(t) + a \textbackslash{}over
\textbackslash{}\textbar{}F(t)\textbackslash{}\textbar{}\} +\{
\textbackslash{}\textbar{}\textbackslash{}overrightarrow\{aP(t)\}\textbackslash{}\textbar{}
\textbackslash{}over
\textbackslash{}\textbar{}F(t)\textbackslash{}\textbar{}\} \{
\textbackslash{}overrightarrow\{aP(t)\} \textbackslash{}over
\textbackslash{}\textbar{}\textbackslash{}overrightarrow\{aP(t)\}\textbackslash{}\textbar{}\}
\%\& \textbackslash{}\textbackslash{} \& =\&\{ F(t) − P(t)
\textbackslash{}over
\textbackslash{}\textbar{}F(t)\textbackslash{}\textbar{}\} +\{
\textbackslash{}\textbar{}\textbackslash{}overrightarrow\{aP(t)\}\textbackslash{}\textbar{}
\textbackslash{}over
\textbackslash{}\textbar{}F(t)\textbackslash{}\textbar{}\}
\textbackslash{}vec\{u\} \%\& \textbackslash{}\textbackslash{}
\textbackslash{}end\{eqnarray*\}

avec \textbackslash{}mathop\{lim\}\{ F(t)−P(t)+a \textbackslash{}over
\textbackslash{}\textbar{}F(t)\textbackslash{}\textbar{}\} = 0 puisque
le numérateur tend vers a et le dénominateur vers + ∞, et
\textbackslash{}mathop\{lim\}\{
\textbackslash{}\textbar{}\textbackslash{}overrightarrow\{aP(t)\}\textbackslash{}\textbar{}
\textbackslash{}over
\textbackslash{}\textbar{}F(t)\textbackslash{}\textbar{}\} = 1 comme le
montre l'encadrement

\textbackslash{}begin\{eqnarray*\}
\textbackslash{}\textbar{}F(t)\textbackslash{}\textbar{}
−\textbackslash{}\textbar{} P(t) − F(t)\textbackslash{}\textbar{}
−\textbackslash{}\textbar{} a\textbackslash{}\textbar{} ≤\&\& \%\&
\textbackslash{}\textbackslash{}
\textbackslash{}\textbar{}\textbackslash{}overrightarrow\{aP(t)\}\textbackslash{}\textbar{}\&
=\& \textbackslash{}\textbar{}P(t) − a\textbackslash{}\textbar{}
≤\textbackslash{}\textbar{} F(t)\textbackslash{}\textbar{}
+\textbackslash{}\textbar{} P(t) − F(t)\textbackslash{}\textbar{}
+\textbackslash{}\textbar{} a\textbackslash{}\textbar{}\%\&
\textbackslash{}\textbackslash{} \textbackslash{}end\{eqnarray*\}

qui implique

1 −\{ \textbackslash{}\textbar{}P(t) − F(t)\textbackslash{}\textbar{}
\textbackslash{}over
\textbackslash{}\textbar{}F(t)\textbackslash{}\textbar{}\} −\{
\textbackslash{}\textbar{}a\textbackslash{}\textbar{}
\textbackslash{}over
\textbackslash{}\textbar{}F(t)\textbackslash{}\textbar{}\} ≤\{
\textbackslash{}\textbar{}\textbackslash{}overrightarrow\{aP(t)\}\textbackslash{}\textbar{}
\textbackslash{}over
\textbackslash{}\textbar{}F(t)\textbackslash{}\textbar{}\} ≤ 1 +\{
\textbackslash{}\textbar{}P(t) − F(t)\textbackslash{}\textbar{}
\textbackslash{}over
\textbackslash{}\textbar{}F(t)\textbackslash{}\textbar{}\} +\{
\textbackslash{}\textbar{}a\textbackslash{}\textbar{}
\textbackslash{}over
\textbackslash{}\textbar{}F(t)\textbackslash{}\textbar{}\}

On a donc \textbackslash{}mathop\{lim\}\{ F(t) \textbackslash{}over
\textbackslash{}\textbar{}F(t)\textbackslash{}\textbar{}\}
=\textbackslash{}vec\{ u\} ce qui achève la démonstration.

Proposition~18.1.7 Soit Γ = (I,F) un arc paramétré de classe
\{C\}\^{}\{k\} et α ∈ ℝ ∪\textbackslash{}\{−∞,+∞\textbackslash{}\} une
extrémité de I où Γ admet une branche infinie. Soit
\textbackslash{}vec\{D\} une direction de droite, Π un hyperplan affine
non parallèle à \textbackslash{}vec\{D\}, m un point de Π. Pour t ∈ I,
soit \{m\}\_\{t\} le point d'intersection de la parallèle à
\textbackslash{}vec\{D\} mené par F(t) avec l'hyperplan Π. On a
équivalence de (i) Γ admet en α la droite D = m +\textbackslash{}vec\{
D\} comme asymptote (ii)
\{\textbackslash{}mathop\{lim\}\}\_\{t→α\}\{m\}\_\{t\} = m.

Démonstration Mettons sur E une structure euclidienne telle que
l'hyperplan Π soit orthogonal à \textbackslash{}vec\{D\} et utilisons la
distance d correspondante. Alors la distance de F(t) à D est égale à la
distance de \{m\}\_\{t\} à m ce qui montre le résultat.

Etude des branches infinies Si Γ admet en α une branche infinie, on
commence par regarder si \{ F(t) \textbackslash{}over
\textbackslash{}\textbar{}F(t)\textbackslash{}\textbar{}\} admet une
limite. Si c'est le cas, soit \textbackslash{}vec\{u\} cette limite et
\textbackslash{}vec\{D\} = ℝ\textbackslash{}vec\{u\}. On choisit pour Π
un hyperplan affine non parallèle à \textbackslash{}vec\{D\} et on
recherche le point \{m\}\_\{t\} d'intersection de la droite F(t) +
ℝ\textbackslash{}vec\{u\} avec Π~; si le point \{m\}\_\{t\} admet en α
une limite m, la droite m + ℝ\textbackslash{}vec\{u\} est asymptote à la
courbe, sinon la courbe n'admet pas d'asymptote en α. Pour la simplicité
des calculs, on est souvent amené à choisir pour Π l'un des hyperplans
de coordonnées.

\paragraph{18.1.8 Plan d'étude d'un arc plan en paramétriques}

Ici, E est un ℝ espace vectoriel normé de dimension 2, soit
(\textbackslash{}vec\{ı\},\textbackslash{}vec\{ȷ\}) une base de E. Soit
F une fonction de ℝ vers E. On notera (lorsque cela a un sens) F(t) =
φ(t)\textbackslash{}vec\{ı\} + ψ(t)\textbackslash{}vec\{ȷ\}.

Première étape~: domaine de définition On détermine le domaine de
définition D =\textbackslash{}mathop\{ Def\} (φ)
∩\textbackslash{}mathop\{ Def\} (ψ) de F~; c'est en général une réunion
finie d'intervalles deux à deux disjoints, si bien que la courbe étudiée
sera une réunion finie d'arcs paramétrés.

Deuxième étape~: réduction du domaine d'étude Supposons qu'il existe une
application θ : D→D tel que pour t ∈D, F(θ(t)) se déduise par une
transformation géométrique simple S de F(t), et soit Δ un domaine
fondamental de θ dans D c'est-à-dire une partie de D telle que D
=\{\textbackslash{}mathop\{ \textbackslash{}mathop\{⋃ \}\}
\}\_\{n∈ℕ\}\{θ\}\^{}\{n\}(Δ) (ou éventuellement si θ est bijective, D
=\{\textbackslash{}mathop\{ \textbackslash{}mathop\{⋃ \}\}
\}\_\{n∈ℤ\}\{θ\}\^{}\{n\}(Δ)). En se fondant sur la relation
F(\{θ\}\^{}\{n\}(t)) = \{S\}\^{}\{n\}(F(t)), on voit qu'il suffit
d'étudier la courbe pour t ∈ Δ pour en déduire simplement l'étude sur
\{θ\}\^{}\{n\}(Δ) et donc sur D.

On recherchera principalement des transformations θ du type

\begin{itemize}
\itemsep1pt\parskip0pt\parsep0pt
\item
  (i) θ(t) = t + T avec alors Δ = {[}a,a + T{]} ∩D
\item
  (ii) θ(t) = ω − t avec alors Δ = {[}\{ ω \textbackslash{}over 2\}
  ,+∞{[}∩D
\item
  (iii) éventuellement θ(t) =\{ 1 \textbackslash{}over t\} avec par
  exemple Δ = {[}−1,1{]} ∩D
\end{itemize}

En ce qui concerne les transformations géométriques simples, on repérera
principalement

\begin{itemize}
\itemsep1pt\parskip0pt\parsep0pt
\item
  (i) l'identité φ(θ(t)) = φ(t), ψ(θ(t)) = ψ(t)
\item
  (ii) des symétries

  \begin{itemize}
  \itemsep1pt\parskip0pt\parsep0pt
  \item
    a) φ(θ(t)) = −φ(t), ψ(θ(t)) = ψ(t)~: symétrie par rapport à l'axe Oy
  \item
    b) φ(θ(t)) = φ(t), ψ(θ(t)) = −ψ(t)~: symétrie par rapport à l'axe Ox
  \item
    c) φ(θ(t)) = −φ(t), ψ(θ(t)) = −ψ(t)~: symétrie par rapport à
    l'origine O.
  \item
    d) φ(θ(t)) = ψ(t), ψ(θ(t)) = φ(t)~: symétrie par rapport à la
    première diagonale
  \end{itemize}
\item
  (iii) des translations φ(θ(t)) = φ(t) + α, ψ(θ(t)) = ψ(t) + β~:
  translation de vecteur α\textbackslash{}vec\{ı\} +
  β\textbackslash{}vec\{ȷ\}
\end{itemize}

Troisième étape~: variations de φ et ψ On étudie les signes des dérivées
φ' et ψ'. Au passage on repère un certain nombre de points remarquables

\begin{itemize}
\itemsep1pt\parskip0pt\parsep0pt
\item
  (i) φ'(\{t\}\_\{0\}) = 0,
  ψ'(\{t\}\_\{0\})\textbackslash{}mathrel\{≠\}0~: point régulier où la
  tangente est verticale
\item
  (ii) φ'(\{t\}\_\{0\})\textbackslash{}mathrel\{≠\}0, ψ'(\{t\}\_\{0\}) =
  0~: point régulier où la tangente est horizontale
\item
  (iii) φ'(\{t\}\_\{0\}) = 0, ψ'(\{t\}\_\{0\}) = 0~: point singulier
\end{itemize}

Quatrième étape~: étude des points singuliers Une étude locale soit à
l'aide de dérivées successives, soit de préférence avec un développement
limité permet de déterminer les entiers p et q introduits ci dessus et
de déterminer en conséquence le type de point singulier~: point de
rebroussement de première espèce, point de rebroussement de deuxième
espèce, point banal ou point d'inflexion (dans l'ordre décroissant des
fréquences)~; au passage on détermine la tangente en un point singulier.

Cinquième étape~: étude des branches infinies L'arc admet en α
∈\textbackslash{}overline\{D\} une branche infinie si et seulement si
l'une au moins des deux coordonnées φ(t) et ψ(t) admet une limite ∞.

\begin{itemize}
\itemsep1pt\parskip0pt\parsep0pt
\item
  (i) Si \textbackslash{}mathop\{lim\}\textbar{}φ(t)\textbar{} = +∞ et
  ψ(t) est bornée, l'arc admet une direction asymptotique horizontale~;
  il admet la droite y = \{y\}\_\{0\} comme asymptote si et seulement
  si~\{\textbackslash{}mathop\{lim\}\}\_\{t→α\}ψ(t) = \{y\}\_\{0\}.
\item
  (ii) Si \textbackslash{}mathop\{lim\}\textbar{}ψ(t)\textbar{} = +∞ et
  φ(t) est bornée, l'arc admet une direction asymptotique verticale~; il
  admet la droite x = \{x\}\_\{0\} comme asymptote si et seulement
  si~\{\textbackslash{}mathop\{lim\}\}\_\{t→α\}φ(t) = \{x\}\_\{0\}.
\item
  (iii) Si \textbackslash{}mathop\{lim\}\textbar{}φ(t)\textbar{} = +∞ et
  \textbackslash{}mathop\{lim\}\textbar{}ψ(t)\textbar{} = +∞, on étudie
  le rapport \{ ψ(t) \textbackslash{}over φ(t)\}

  \begin{itemize}
  \itemsep1pt\parskip0pt\parsep0pt
  \item
    a) s'il admet une limite ∞, l'arc admet la verticale comme direction
    asymptotique et il n'y a pas d'asymptote
  \item
    b) s'il admet la limite 0, l'arc admet l'horizontale comme direction
    asymptotique et il n'y a pas d'asymptote
  \item
    c) s'il admet la limite a\textbackslash{}mathrel\{≠\}0, l'arc admet
    la droite y = ax comme direction asymptotique et la droite
    d'équation y = ax + b est asymptote si et seulement
    si~\{\textbackslash{}mathop\{lim\}\}\_\{t→α\}(ψ(t) − aφ(t)) = b
  \item
    d) dans tous les autres cas, il n'y a pas de direction asymptotique
  \end{itemize}
\end{itemize}

Une étude complémentaire de signe peut parfois préciser la position du
point F(t) par rapport à une asymptote, ce qui peut permettre de
préciser un tracé.

Sixième étape~: ébauche de tracé En s'aidant d'une calculatrice ou d'un
ordinateur, on peut calculer un certain nombre de points supplémentaires
en plus des points remarquables~; ceci, en plus des points et des
tangentes remarquables, des variations de φ et ψ et de l'étude des
branches infinies permet en général une ébauche convaincante du tracé.

Etapes facultatives

Si la question est posée ou si l'ébauche du tracé suggère la nécessité
de certaines précisions, on peut procéder à quelques étapes
supplémentaires

Septième étape~: détermination des points non biréguliers Il suffit
d'écrire \{\textbackslash{}mathop\{\textbackslash{}mathrm\{det\}\}
\}\_\{(\textbackslash{}vec\{ı\},\textbackslash{}vec\{ȷ\})\}(F'(t),F''(t))
= 0, c'est-à-dire \textbackslash{}left
\textbar{}\textbackslash{}matrix\{\textbackslash{},φ'(t)\&ψ'(t)
\textbackslash{}cr φ'`(t)\&ψ''(t)\}\textbackslash{}right \textbar{},
soit encore, si ψ' ne s'annule pas \{ d \textbackslash{}over dt\}
\textbackslash{}left (\{ φ'(t) \textbackslash{}over ψ'(t)\}
\textbackslash{}right ) = 0~; ces points, une fois éliminés les points
singuliers, sont souvent des points d'inflexion.

Huitième étape~: détermination des points multiples Il s'agit de
résoudre l'équation F(t) = F(t') ou encore le système φ(t) = φ(t'), ψ(t)
= ψ(t') pour t\textbackslash{}mathrel\{≠\}t'.

\paragraph{18.1.9 Notion de contact}

Soit Γ = (I,F) un arc paramétré de classe \{C\}\^{}\{k\} de
\{ℝ\}\^{}\{n\}~; on pose F(t) =
(\{f\}\_\{1\}(t),\textbackslash{}mathop\{\textbackslash{}mathop\{\ldots{}\}\},\{f\}\_\{n\}(t)).
Soit U un ouvert contenant l'image de Γ, G une application de classe
\{C\}\^{}\{k\} de U dans ℝ, Σ =
\textbackslash{}\{(\{x\}\_\{1\},\textbackslash{}mathop\{\textbackslash{}mathop\{\ldots{}\}\},\{x\}\_\{n\})
∈
U\textbackslash{}mathrel\{∣\}G(\{x\}\_\{1\},\textbackslash{}mathop\{\textbackslash{}mathop\{\ldots{}\}\},\{x\}\_\{n\})
= 0\textbackslash{}\} l'hypersurface correspondante de \{ℝ\}\^{}\{n\}.
On pose φ(t) =
G(\{f\}\_\{1\}(t),\textbackslash{}mathop\{\textbackslash{}mathop\{\ldots{}\}\},\{f\}\_\{n\}(t))~;
donc φ est une application de classe \{C\}\^{}\{k\} de I dans ℝ.

Définition~18.1.12 On dit que Γ et Σ ont au point \{t\}\_\{0\} ∈ I de Γ
un contact d'ordre au moins p si φ(\{t\}\_\{0\}) = φ'(\{t\}\_\{0\}) =
\textbackslash{}mathop\{\textbackslash{}mathop\{\ldots{}\}\} =
\{φ\}\^{}\{(p−1)\}(\{t\}\_\{0\}) = 0. On dit en particulier que Γ et Σ
sont sécantes (resp. tangentes, resp. osculatrices, resp.
surosculatrices) en \{t\}\_\{0\} si elles ont en \{t\}\_\{0\} un contact
d'ordre au moins 1 (resp. 2, resp. 3, resp. 4).

Remarque~18.1.10 On voit que Γ et Σ sont sécantes en \{t\}\_\{0\} si et
seulement si~F(\{t\}\_\{0\}) ∈ Σ ce qui est bien naturel. Pour que Γ et
Σ soient tangentes en \{t\}\_\{0\}, il faut de plus que φ'(\{t\}\_\{0\})
= 0. Mais φ'(\{t\}\_\{0\}) =\{\textbackslash{}mathop\{
\textbackslash{}mathop\{∑ \}\}
\}\_\{i=1\}\^{}\{n\}\{f\}\_\{i\}'(\{t\}\_\{0\})\{ ∂G
\textbackslash{}over ∂\{x\}\_\{i\}\} (F(\{t\}\_\{0\})), c'est-à-dire que
Γ et Σ sont tangentes en \{t\}\_\{0\}, si et seulement si
F(\{t\}\_\{0\}) ∈ Σ et le vecteur F'(\{t\}\_\{0\}) =
(\{f\}\_\{1\}'(\{t\}\_\{0\}),\textbackslash{}mathop\{\textbackslash{}mathop\{\ldots{}\}\},\{f\}\_\{n\}'(\{t\}\_\{0\}))
appartient à l'hyperplan Π d'équation
\{\textbackslash{}mathop\{\textbackslash{}mathop\{∑ \}\}
\}\_\{i=1\}\^{}\{n\}\{x\}\_\{i\}\{ ∂G \textbackslash{}over
∂\{x\}\_\{i\}\} (F(\{t\}\_\{0\})) = 0 (si
\textbackslash{}mathop\{\textbackslash{}mathrm\{grad\}\}
G(F(\{t\}\_\{0\}))\textbackslash{}mathrel\{≠\}0)~; or le vecteur
F'(\{t\}\_\{0\}) détermine en général la tangente à Γ en \{t\}\_\{0\} et
l'hyperplan Π est en général (comme on l'a vu à l'occasion du théorème
des fonctions implicites) l'hyperplan tangent à Σ au point
F(\{t\}\_\{0\}) ce qui justifie le vocabulaire employé.

Remarque~18.1.11 Soit F(t) = x(t)\textbackslash{}vec\{ı\} +
y(t)\textbackslash{}vec\{ȷ\} un arc paramétré plan~; soit \{t\}\_\{0\}
un point régulier de l'arc et recherchons le contact en \{t\}\_\{0\} de
Γ avec une droite D d'équation ax + by + c = 0~; la direction
\textbackslash{}vec\{D\} de la droite D a pour équation ax + by = 0. On
a alors φ(t) = ax(t) + by(t) + c. On voit alors que, si on appelle p
l'ordre du contact, on a (i) p ≥ 1 \textbackslash{}mathrel\{⇔\}
F(\{t\}\_\{0\}) ∈ D (ii) p ≥ 2 \textbackslash{}mathrel\{⇔\}
F(\{t\}\_\{0\}) ∈ D,F'(\{t\}\_\{0\}) ∈\textbackslash{}vec\{ D\}~: seule
la tangente en \{t\}\_\{0\} a un contact d'ordre au moins 2 (iii) p ≥ 3
\textbackslash{}mathrel\{⇔\} F(\{t\}\_\{0\}) ∈ D,F'(\{t\}\_\{0\})
∈\textbackslash{}vec\{ D\},F''(\{t\}\_\{0\}) ∈\textbackslash{}vec\{
D\}~: en général aucune droite ne répond à ces exigences, à moins que le
point ne soit pas birégulier.

On obtient donc que les points réguliers non biréguliers sont les points
de Γ où existent une droite osculatrice ce qui peut fournir un autre
moyen de recherche des points non biréguliers (en général des points
d'inflexion) en étudiant la multiplicité d'intersection d'une droite D
avec l'arc, dans la mesure où cela a un sens, et en particulier lorsque
x(t) et y(t) sont des fractions rationnelles en t.

Sur le même modèle, le lecteur montrera facilement qu'en un point
birégulier d'un arc de \{ℝ\}\^{}\{3\}, il existe un seul plan qui soit
osculateur à l'arc, à savoir le plan osculateur défini précédemment.

\paragraph{18.1.10 Enveloppes}

Ce paragraphe ne fait pas partie du programme des classes préparatoires.

Soit \{(\{D\}\_\{t\})\}\_\{t∈I\} une famille de droites de
\{ℝ\}\^{}\{2\} indexée par un intervalle I de ℝ. Intuitivement, on
appellera enveloppe de la famille de droites un arc Γ = (I,F) telle que
la tangente à Γ au point t soit la droite \{D\}\_\{t\}. Nous allons
préciser cette définition de la matière suivante

Définition~18.1.13 Soit I un intervalle de ℝ, a,b,c trois applications
de classe \{C\}\^{}\{2\} de I dans ℝ telles que
\textbackslash{}mathop\{∀\}t ∈ I,
(a(t),b(t))\textbackslash{}mathrel\{≠\}(0,0). Pour t ∈ I soit
\{D\}\_\{t\} la droite de \{ℝ\}\^{}\{2\} d'équation a(t)x + b(t)y + c(t)
= 0. On appelle enveloppe de la famille de droite \{D\}\_\{t\} tout arc
paramétré (I,F) de classe \{C\}\^{}\{1\} tel que

\textbackslash{}mathop\{∀\}t ∈ I, F(t) ∈
\{D\}\_\{t\}\textbackslash{}text\{ et \}F'(t)
∈\textbackslash{}overrightarrow\{ \{D\}\_\{t\}\}

Remarque~18.1.12 En un point singulier de l'arc, on a F'(t) = 0 et la
condition F'(t) ∈\textbackslash{}overrightarrow\{ \{D\}\_\{t\}\} est
automatiquement vérifiée. En un tel point, la droite \{D\}\_\{t\} n'a
donc aucune raison d'être la tangente à l'arc (I,F). Nous allons voir
cependant que c'est souvent le cas, sous des hypothèses raisonnables.

Théorème~18.1.8 On suppose que \textbackslash{}mathop\{∀\}t ∈ I,
a(t)b'(t) − a'(t)b(t)\textbackslash{}mathrel\{≠\}0. Alors la famille de
droites \{D\}\_\{t\} d'équations a(t)x + b(t)y + c(t) = 0 admet une
unique enveloppe (I,F)~; pour tout t ∈ I, le point F(t) est le point
d'intersection de la droite \{D\}\_\{t\} avec la droite \{D\}\_\{t\}'
d'équation a'(t)x + b'(t)y + c'(t) = 0~; c'est aussi la limite quand h
tend vers 0 du point d'intersection de la droite \{D\}\_\{t+h\} avec la
droite \{D\}\_\{t\}. Pour tout point non totalement singulier t de l'arc
(I,F), la droite \{D\}\_\{t\} est la tangente à l'arc au point t.

Démonstration Posons F(t) = (x(t),y(t)). Les conditions de la définition
de l'enveloppe se traduisent par \textbackslash{}mathop\{∀\}t ∈ I,
a(t)x(t) + b(t)y(t) + c(t) = 0, a(t)x'(t) + b(t)y'(t) = 0. Mais si la
première condition est vérifiée, par dérivation on a
\textbackslash{}mathop\{∀\}t ∈ I,a'(t)x(t) + b'(t)y(t) + c'(t) +
a(t)x'(t) + b(t)y'(t) = 0 si bien que la seconde condition est
équivalente à \textbackslash{}mathop\{∀\}t ∈ I, a'(t)x + b'(t)y + c'(t)
= 0. On obtient donc

\textbackslash{}mathop\{∀\}t ∈ I, \textbackslash{}left
\textbackslash{}\{ \textbackslash{}cases\{ a(t)x(t) + b(t)y(t) + c(t) =
0 \textbackslash{}cr a'(t)x(t) + b'(t)y(y) + c'(t) = 0 \}
\textbackslash{}right .

système aux inconnues x(t) et y(t) qui est un système de Cramer puisque
a(t)b'(t) − a'(t)b(t)\textbackslash{}mathrel\{≠\}0. Ceci montre déjà
l'unicité de l'enveloppe et le fait que \{D\}\_\{t\} ∩ \{D\}\_\{t\}' =
\textbackslash{}\{F(t)\textbackslash{}\}. Inversement, si F(t) est ainsi
défini, comme a,b et c sont de classe \{C\}\^{}\{2\}, x(t) et y(t) sont
de classe \{C\}\^{}\{1\} et le même calcul que ci dessus en sens inverse
montre que \textbackslash{}mathop\{∀\}t ∈ I, a(t)x(t) + b(t)y(t) + c(t)
= 0, a(t)x'(t) + b(t)y'(t) = 0, donc (I,F) est bien enveloppe de la
famille \{D\}\_\{t\}.

Soit maintenant h\textbackslash{}mathrel\{≠\}0. L'équation aux
coordonnées des points d'intersection des droites \{D\}\_\{t\} et
\{D\}\_\{t+h\} est donnée par

\textbackslash{}begin\{eqnarray*\} \textbackslash{}left
\textbackslash{}\{ \textbackslash{}cases\{ a(t)x + b(t)y + c(t) = 0
\textbackslash{}cr a(t + h)x + b(t + h)y + c(t + h) = 0 \}
\textbackslash{}right . \textbackslash{}mathrel\{⇔\}\&\& \%\&
\textbackslash{}\textbackslash{} \& \textbackslash{}left
\textbackslash{}\{ \textbackslash{}cases\{ a(t)x + b(t)y + c(t) = 0
\textbackslash{}cr \{ a(t+h)−a(t) \textbackslash{}over h\} x +\{
b(t+h)−b(t) \textbackslash{}over h\} y +\{ c(t+h)−c(t)
\textbackslash{}over h\} = 0 \} \textbackslash{}right .\&
\%\&\textbackslash{}\textbackslash{} \textbackslash{}end\{eqnarray*\}

La limite du déterminant de ce système est a(t)b'(t) −
a'(t)b(t)\textbackslash{}mathrel\{≠\}0, donc pour h assez petit ce
déterminant est non nul et définit un unique point \{F\}\_\{h\}(t) de
coordonnées (\{x\}\_\{h\}(t),\{y\}\_\{h\}(t)). Les formules de Cramer
montrent aussitôt que
\{\textbackslash{}mathop\{lim\}\}\_\{h→0\}\{F\}\_\{h\}(t) = F(t).

Supposons maintenant que a,b et c soient de classe \{C\}\^{}\{k+1\}~;
alors F est de classe \{C\}\^{}\{k\}. Soit \{t\}\_\{0\} un point non
totalement singulier de (I,F), et soit p tel que F'(\{t\}\_\{0\}) =
\textbackslash{}mathop\{\textbackslash{}mathop\{\ldots{}\}\} =
\{F\}\^{}\{(p−1)\}(\{t\}\_\{0\}) = 0,
\{F\}\^{}\{(p)\}(\{t\}\_\{0\})\textbackslash{}mathrel\{≠\}0. On a
\textbackslash{}mathop\{∀\}t ∈ I,a(t)x'(t) + b(t)y'(t) = 0. En
appliquant la formule de Leibnitz, on a

\textbackslash{}begin\{eqnarray*\} 0\& =\&\{ \{d\}\^{}\{p−1\}
\textbackslash{}over d\{t\}\^{}\{p−1\}\} \{(a(t)x'(t) +
b(t)y'(t))\}\_\{t=\{t\}\_\{0\}\} \%\& \textbackslash{}\textbackslash{}
\& =\& \{\textbackslash{}mathop\{∑ \}\}\_\{n=0\}\^{}\{p−1\}\{C\}\_\{
p−1\}\^{}\{n\}\textbackslash{}left (\{a\}\^{}\{(n)\}(\{t\}\_\{
0\})\{x\}\^{}\{(p−n)\}(\{t\}\_\{ 0\}) + \{b\}\^{}\{(n)\}(\{t\}\_\{
0\})\{y\}\^{}\{(p−n)\}(\{t\}\_\{ 0\})\textbackslash{}right )\%\&
\textbackslash{}\textbackslash{} \& =\& a(t)\{x\}\^{}\{(p)\}(\{t\}\_\{
0\}) + b(t)\{y\}\^{}\{(p)\}(\{t\}\_\{ 0\}) \%\&
\textbackslash{}\textbackslash{} \textbackslash{}end\{eqnarray*\}

puisque toutes les dérivées précédentes de x et y sont nulles au point
\{t\}\_\{0\}. On en déduit que \{F\}\^{}\{(p)\}(\{t\}\_\{0\})
∈\textbackslash{}overrightarrow\{ \{D\}\_\{\{t\}\_\{0\}\}\}. La droite
\{D\}\_\{\{t\}\_\{0\}\} contient le point F(\{t\}\_\{0\}) et est
parallèle au vecteur tangent \{F\}\^{}\{(p)\}(\{t\}\_\{0\})~; c'est donc
la tangente au point \{t\}\_\{0\}. Ceci achève la démonstration du
théorème.

{[}\href{coursse97.html}{next}{]} {[}\href{coursse96.html}{front}{]}
{[}\href{coursch19.html\#coursse96.html}{up}{]}

\end{document}
