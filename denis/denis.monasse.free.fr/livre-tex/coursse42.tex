\documentclass[]{article}
\usepackage[T1]{fontenc}
\usepackage{lmodern}
\usepackage{amssymb,amsmath}
\usepackage{ifxetex,ifluatex}
\usepackage{fixltx2e} % provides \textsubscript
% use upquote if available, for straight quotes in verbatim environments
\IfFileExists{upquote.sty}{\usepackage{upquote}}{}
\ifnum 0\ifxetex 1\fi\ifluatex 1\fi=0 % if pdftex
  \usepackage[utf8]{inputenc}
\else % if luatex or xelatex
  \ifxetex
    \usepackage{mathspec}
    \usepackage{xltxtra,xunicode}
  \else
    \usepackage{fontspec}
  \fi
  \defaultfontfeatures{Mapping=tex-text,Scale=MatchLowercase}
  \newcommand{\euro}{€}
\fi
% use microtype if available
\IfFileExists{microtype.sty}{\usepackage{microtype}}{}
\ifxetex
  \usepackage[setpagesize=false, % page size defined by xetex
              unicode=false, % unicode breaks when used with xetex
              xetex]{hyperref}
\else
  \usepackage[unicode=true]{hyperref}
\fi
\hypersetup{breaklinks=true,
            bookmarks=true,
            pdfauthor={},
            pdftitle={Espaces de suites},
            colorlinks=true,
            citecolor=blue,
            urlcolor=blue,
            linkcolor=magenta,
            pdfborder={0 0 0}}
\urlstyle{same}  % don't use monospace font for urls
\setlength{\parindent}{0pt}
\setlength{\parskip}{6pt plus 2pt minus 1pt}
\setlength{\emergencystretch}{3em}  % prevent overfull lines
\setcounter{secnumdepth}{0}
 
/* start css.sty */
.cmr-5{font-size:50%;}
.cmr-7{font-size:70%;}
.cmmi-5{font-size:50%;font-style: italic;}
.cmmi-7{font-size:70%;font-style: italic;}
.cmmi-10{font-style: italic;}
.cmsy-5{font-size:50%;}
.cmsy-7{font-size:70%;}
.cmex-7{font-size:70%;}
.cmex-7x-x-71{font-size:49%;}
.msbm-7{font-size:70%;}
.cmtt-10{font-family: monospace;}
.cmti-10{ font-style: italic;}
.cmbx-10{ font-weight: bold;}
.cmr-17x-x-120{font-size:204%;}
.cmsl-10{font-style: oblique;}
.cmti-7x-x-71{font-size:49%; font-style: italic;}
.cmbxti-10{ font-weight: bold; font-style: italic;}
p.noindent { text-indent: 0em }
td p.noindent { text-indent: 0em; margin-top:0em; }
p.nopar { text-indent: 0em; }
p.indent{ text-indent: 1.5em }
@media print {div.crosslinks {visibility:hidden;}}
a img { border-top: 0; border-left: 0; border-right: 0; }
center { margin-top:1em; margin-bottom:1em; }
td center { margin-top:0em; margin-bottom:0em; }
.Canvas { position:relative; }
li p.indent { text-indent: 0em }
.enumerate1 {list-style-type:decimal;}
.enumerate2 {list-style-type:lower-alpha;}
.enumerate3 {list-style-type:lower-roman;}
.enumerate4 {list-style-type:upper-alpha;}
div.newtheorem { margin-bottom: 2em; margin-top: 2em;}
.obeylines-h,.obeylines-v {white-space: nowrap; }
div.obeylines-v p { margin-top:0; margin-bottom:0; }
.overline{ text-decoration:overline; }
.overline img{ border-top: 1px solid black; }
td.displaylines {text-align:center; white-space:nowrap;}
.centerline {text-align:center;}
.rightline {text-align:right;}
div.verbatim {font-family: monospace; white-space: nowrap; text-align:left; clear:both; }
.fbox {padding-left:3.0pt; padding-right:3.0pt; text-indent:0pt; border:solid black 0.4pt; }
div.fbox {display:table}
div.center div.fbox {text-align:center; clear:both; padding-left:3.0pt; padding-right:3.0pt; text-indent:0pt; border:solid black 0.4pt; }
div.minipage{width:100%;}
div.center, div.center div.center {text-align: center; margin-left:1em; margin-right:1em;}
div.center div {text-align: left;}
div.flushright, div.flushright div.flushright {text-align: right;}
div.flushright div {text-align: left;}
div.flushleft {text-align: left;}
.underline{ text-decoration:underline; }
.underline img{ border-bottom: 1px solid black; margin-bottom:1pt; }
.framebox-c, .framebox-l, .framebox-r { padding-left:3.0pt; padding-right:3.0pt; text-indent:0pt; border:solid black 0.4pt; }
.framebox-c {text-align:center;}
.framebox-l {text-align:left;}
.framebox-r {text-align:right;}
span.thank-mark{ vertical-align: super }
span.footnote-mark sup.textsuperscript, span.footnote-mark a sup.textsuperscript{ font-size:80%; }
div.tabular, div.center div.tabular {text-align: center; margin-top:0.5em; margin-bottom:0.5em; }
table.tabular td p{margin-top:0em;}
table.tabular {margin-left: auto; margin-right: auto;}
div.td00{ margin-left:0pt; margin-right:0pt; }
div.td01{ margin-left:0pt; margin-right:5pt; }
div.td10{ margin-left:5pt; margin-right:0pt; }
div.td11{ margin-left:5pt; margin-right:5pt; }
table[rules] {border-left:solid black 0.4pt; border-right:solid black 0.4pt; }
td.td00{ padding-left:0pt; padding-right:0pt; }
td.td01{ padding-left:0pt; padding-right:5pt; }
td.td10{ padding-left:5pt; padding-right:0pt; }
td.td11{ padding-left:5pt; padding-right:5pt; }
table[rules] {border-left:solid black 0.4pt; border-right:solid black 0.4pt; }
.hline hr, .cline hr{ height : 1px; margin:0px; }
.tabbing-right {text-align:right;}
span.TEX {letter-spacing: -0.125em; }
span.TEX span.E{ position:relative;top:0.5ex;left:-0.0417em;}
a span.TEX span.E {text-decoration: none; }
span.LATEX span.A{ position:relative; top:-0.5ex; left:-0.4em; font-size:85%;}
span.LATEX span.TEX{ position:relative; left: -0.4em; }
div.float img, div.float .caption {text-align:center;}
div.figure img, div.figure .caption {text-align:center;}
.marginpar {width:20%; float:right; text-align:left; margin-left:auto; margin-top:0.5em; font-size:85%; text-decoration:underline;}
.marginpar p{margin-top:0.4em; margin-bottom:0.4em;}
.equation td{text-align:center; vertical-align:middle; }
td.eq-no{ width:5%; }
table.equation { width:100%; } 
div.math-display, div.par-math-display{text-align:center;}
math .texttt { font-family: monospace; }
math .textit { font-style: italic; }
math .textsl { font-style: oblique; }
math .textsf { font-family: sans-serif; }
math .textbf { font-weight: bold; }
.partToc a, .partToc, .likepartToc a, .likepartToc {line-height: 200%; font-weight:bold; font-size:110%;}
.chapterToc a, .chapterToc, .likechapterToc a, .likechapterToc, .appendixToc a, .appendixToc {line-height: 200%; font-weight:bold;}
.index-item, .index-subitem, .index-subsubitem {display:block}
.caption td.id{font-weight: bold; white-space: nowrap; }
table.caption {text-align:center;}
h1.partHead{text-align: center}
p.bibitem { text-indent: -2em; margin-left: 2em; margin-top:0.6em; margin-bottom:0.6em; }
p.bibitem-p { text-indent: 0em; margin-left: 2em; margin-top:0.6em; margin-bottom:0.6em; }
.paragraphHead, .likeparagraphHead { margin-top:2em; font-weight: bold;}
.subparagraphHead, .likesubparagraphHead { font-weight: bold;}
.quote {margin-bottom:0.25em; margin-top:0.25em; margin-left:1em; margin-right:1em; text-align:\jmathustify;}
.verse{white-space:nowrap; margin-left:2em}
div.maketitle {text-align:center;}
h2.titleHead{text-align:center;}
div.maketitle{ margin-bottom: 2em; }
div.author, div.date {text-align:center;}
div.thanks{text-align:left; margin-left:10%; font-size:85%; font-style:italic; }
div.author{white-space: nowrap;}
.quotation {margin-bottom:0.25em; margin-top:0.25em; margin-left:1em; }
h1.partHead{text-align: center}
.sectionToc, .likesectionToc {margin-left:2em;}
.subsectionToc, .likesubsectionToc {margin-left:4em;}
.subsubsectionToc, .likesubsubsectionToc {margin-left:6em;}
.frenchb-nbsp{font-size:75%;}
.frenchb-thinspace{font-size:75%;}
.figure img.graphics {margin-left:10%;}
/* end css.sty */

\title{Espaces de suites}
\author{}
\date{}

\begin{document}
\maketitle

\textbf{Warning: 
requires JavaScript to process the mathematics on this page.\\ If your
browser supports JavaScript, be sure it is enabled.}

\begin{center}\rule{3in}{0.4pt}\end{center}

{[}
{[}
{[}{]}
{[}

\subsubsection{7.8 Espaces de suites}

Définition~7.8.1 On dit qu'une suite (x\_n)\_n\in\mathbb{N}~ de
nombres réels ou complexes est sommable si la série
\\sum  x\_n~ est
absolument convergente.

Proposition~7.8.1 L'ensemble \ell^1(\mathbb{N}~) des suites sommables de
nombres complexes est un sous espace vectoriel de \mathbb{C}^\mathbb{N}~.
L'application u =
(u\_n)\_n\in\mathbb{N}~\mapsto~\\textbar{}u\\textbar{}\_1
= \\sum ~
\_n=0^+\infty~\textbar{}u\_n\textbar{} est une norme sur
cet espace vectoriel. L'application
u\mapsto~\\\sum
 \_n\in\mathbb{N}~u\_n est linéaire de \ell^1(\mathbb{N}~) dans \mathbb{C}.

Démonstration Si (u\_n) et (v\_n) sont deux suites
sommables et \alpha~,\beta~ \in \mathbb{C}, les suites (\textbar{}u\_n\textbar{}) et
(\textbar{}v\_n\textbar{}) sont sommables~; il en est donc de
même de la suite (\textbar{}\alpha~\textbar{}\textbar{}u\_n\textbar{}
+ \textbar{}\beta~\textbar{}\textbar{}v\_n\textbar{}) (résultat sur
les séries à réels positifs) et donc de la suite
(\textbar{}\alpha~u\_n + \beta~v\_n\textbar{}) puisque
\textbar{}\alpha~u\_n +
\beta~v\_n\textbar{}\leq\textbar{}\alpha~\textbar{}\textbar{}u\_n\textbar{}
+ \textbar{}\beta~\textbar{}\textbar{}v\_n\textbar{}. Donc la suite
(\alpha~u\_n + \beta~v\_n) est sommable. La suite nulle étant de
surcroît sommable, l'ensemble \ell^1(\mathbb{N}~) des suites sommables de
nombres complexes est un sous espace vectoriel de \mathbb{C}^\mathbb{N}~. La
vérification des propriétés d'une norme est élémentaire. On a alors

\begin{align*} \\sum
\_n=0^+\infty~(\alpha~u\_ n + \beta~v\_n)& =&
lim\_p\rightarrow~+\infty~~\\sum
\_n=0^p(\alpha~u\_ n + \beta~v\_n) \%&
\\ & =&
\alpha~lim\_p\rightarrow~+\infty~~\\sum
\_n=0^pu\_ n +
\beta~lim\_p\rightarrow~+\infty~\\sum
\_n=0^pv\_ n\%& \\
& =& \alpha~\sum \_n=0^+\infty~u\_ n~
+ \beta~\sum \_n=0^+\infty~v\_ n~ \%&
\\ \end{align*}

d'où la linéarité de
u\mapsto~\\\sum
 \_n=0^+\infty~u\_n.

Proposition~7.8.2 L'ensemble \ell^2(\mathbb{N}~) des suites de nombres
complexes dont les carrés forment une suite sommable est un sous-espace
vectoriel de \mathbb{C}^\mathbb{N}~. L'application (u,v) = \left
((u\_n)\_n\in\mathbb{N}~,(v\_n)\_n\in\mathbb{N}~\right
)\mapsto~(u\mathrel∣v)
= \\sum ~
\_n=0^+\infty~\overlineu\_nv\_n
est un produit scalaire hermitien sur cet espace~; en conséquence
l'application u =
(u\_n)\_n\in\mathbb{N}~\mapsto~\\textbar{}u\\textbar{}\_2
= \left
(\\sum ~
\_n=0^+\infty~\textbar{}u\_n\textbar{}^2\right
)^1\diagup2 est une norme sur cet espace vectoriel.

Démonstration Il est clair que si (u\_n) est de carré sommable,
il en est de même de \alpha~(u\_n) = (\alpha~u\_n). Si
(u\_n) et (v\_n) sont de carré sommable, l'inégalité
élémentaire \textbar{}u\_n + v\_n\textbar{}^2
\leq 2\textbar{}u\_n\textbar{}^2 +
2\textbar{}v\_n\textbar{}^2 montre que la suite
(u\_n + v\_n) est de carré sommable. La suite nulle
étant de surcroît de carré sommable, les suites de carrés sommables
forment donc bien un sous-espace vectoriel de \mathbb{C}^\mathbb{N}~. Si
(u\_n) et (v\_n) sont de carré sommable, l'inégalité
élémentaire
\textbar{}\overlineu\_nv\_n\textbar{}\leq
1 \over 2 \textbar{}u\_n\textbar{}^2
+ 1 \over 2
\textbar{}v\_n\textbar{}^2 montre que la suite
(\overlineu\_nv\_n) est sommable. On
peut donc poser (u∣v)
= \\sum ~
\_n=0^+\infty~\overlineu\_nv\_n.
Il est clair que
(u,v)\mapsto~(u\mathrel∣v) est
sesquilinéaire hermitienne. De plus, si u\neq~0,
(u∣u) \in \mathbb{R}~^+∗ ce qui montre que
cette forme sesquilinéaire est définie positive~; on a donc un produit
scalaire hermitien et la norme associée est
\\textbar{}u\\textbar{}\_2^2
= (u∣u).

Remarque~7.8.1 Le théorème ci dessus n'est plus valable pour des séries
convergentes~: posons a\_n = b\_n = (-1)^n
\over \sqrtn+1 . On a
\textbar{}c\_n\textbar{} =\
\sum  \_k=0^n~ 1
\over \sqrt(k+1)(n-k+1) . Mais pour
k \in {[}0,n{]}, (k + 1)(n - k + 1) \leq ( n \over 2 +
1)^2 (facile). Donc \textbar{}c\_n\textbar{}≥ n+1
\over  n \over 2 +1 qui tend vers
2~; donc la suite (c\_n) ne tend pas vers 0 et la série
\\sum  c\_n~
diverge.

{[}
{[}
{[}
{[}

\end{document}
