\documentclass[]{article}
\usepackage[T1]{fontenc}
\usepackage{lmodern}
\usepackage{amssymb,amsmath}
\usepackage{ifxetex,ifluatex}
\usepackage{fixltx2e} % provides \textsubscript
% use upquote if available, for straight quotes in verbatim environments
\IfFileExists{upquote.sty}{\usepackage{upquote}}{}
\ifnum 0\ifxetex 1\fi\ifluatex 1\fi=0 % if pdftex
  \usepackage[utf8]{inputenc}
\else % if luatex or xelatex
  \ifxetex
    \usepackage{mathspec}
    \usepackage{xltxtra,xunicode}
  \else
    \usepackage{fontspec}
  \fi
  \defaultfontfeatures{Mapping=tex-text,Scale=MatchLowercase}
  \newcommand{\euro}{€}
\fi
% use microtype if available
\IfFileExists{microtype.sty}{\usepackage{microtype}}{}
\ifxetex
  \usepackage[setpagesize=false, % page size defined by xetex
              unicode=false, % unicode breaks when used with xetex
              xetex]{hyperref}
\else
  \usepackage[unicode=true]{hyperref}
\fi
\hypersetup{breaklinks=true,
            bookmarks=true,
            pdfauthor={},
            pdftitle={Application aux endomorphismes continus et aux matrices},
            colorlinks=true,
            citecolor=blue,
            urlcolor=blue,
            linkcolor=magenta,
            pdfborder={0 0 0}}
\urlstyle{same}  % don't use monospace font for urls
\setlength{\parindent}{0pt}
\setlength{\parskip}{6pt plus 2pt minus 1pt}
\setlength{\emergencystretch}{3em}  % prevent overfull lines
\setcounter{secnumdepth}{0}
 
/* start css.sty */
.cmr-5{font-size:50%;}
.cmr-7{font-size:70%;}
.cmmi-5{font-size:50%;font-style: italic;}
.cmmi-7{font-size:70%;font-style: italic;}
.cmmi-10{font-style: italic;}
.cmsy-5{font-size:50%;}
.cmsy-7{font-size:70%;}
.cmex-7{font-size:70%;}
.cmex-7x-x-71{font-size:49%;}
.msbm-7{font-size:70%;}
.cmtt-10{font-family: monospace;}
.cmti-10{ font-style: italic;}
.cmbx-10{ font-weight: bold;}
.cmr-17x-x-120{font-size:204%;}
.cmsl-10{font-style: oblique;}
.cmti-7x-x-71{font-size:49%; font-style: italic;}
.cmbxti-10{ font-weight: bold; font-style: italic;}
p.noindent { text-indent: 0em }
td p.noindent { text-indent: 0em; margin-top:0em; }
p.nopar { text-indent: 0em; }
p.indent{ text-indent: 1.5em }
@media print {div.crosslinks {visibility:hidden;}}
a img { border-top: 0; border-left: 0; border-right: 0; }
center { margin-top:1em; margin-bottom:1em; }
td center { margin-top:0em; margin-bottom:0em; }
.Canvas { position:relative; }
li p.indent { text-indent: 0em }
.enumerate1 {list-style-type:decimal;}
.enumerate2 {list-style-type:lower-alpha;}
.enumerate3 {list-style-type:lower-roman;}
.enumerate4 {list-style-type:upper-alpha;}
div.newtheorem { margin-bottom: 2em; margin-top: 2em;}
.obeylines-h,.obeylines-v {white-space: nowrap; }
div.obeylines-v p { margin-top:0; margin-bottom:0; }
.overline{ text-decoration:overline; }
.overline img{ border-top: 1px solid black; }
td.displaylines {text-align:center; white-space:nowrap;}
.centerline {text-align:center;}
.rightline {text-align:right;}
div.verbatim {font-family: monospace; white-space: nowrap; text-align:left; clear:both; }
.fbox {padding-left:3.0pt; padding-right:3.0pt; text-indent:0pt; border:solid black 0.4pt; }
div.fbox {display:table}
div.center div.fbox {text-align:center; clear:both; padding-left:3.0pt; padding-right:3.0pt; text-indent:0pt; border:solid black 0.4pt; }
div.minipage{width:100%;}
div.center, div.center div.center {text-align: center; margin-left:1em; margin-right:1em;}
div.center div {text-align: left;}
div.flushright, div.flushright div.flushright {text-align: right;}
div.flushright div {text-align: left;}
div.flushleft {text-align: left;}
.underline{ text-decoration:underline; }
.underline img{ border-bottom: 1px solid black; margin-bottom:1pt; }
.framebox-c, .framebox-l, .framebox-r { padding-left:3.0pt; padding-right:3.0pt; text-indent:0pt; border:solid black 0.4pt; }
.framebox-c {text-align:center;}
.framebox-l {text-align:left;}
.framebox-r {text-align:right;}
span.thank-mark{ vertical-align: super }
span.footnote-mark sup.textsuperscript, span.footnote-mark a sup.textsuperscript{ font-size:80%; }
div.tabular, div.center div.tabular {text-align: center; margin-top:0.5em; margin-bottom:0.5em; }
table.tabular td p{margin-top:0em;}
table.tabular {margin-left: auto; margin-right: auto;}
div.td00{ margin-left:0pt; margin-right:0pt; }
div.td01{ margin-left:0pt; margin-right:5pt; }
div.td10{ margin-left:5pt; margin-right:0pt; }
div.td11{ margin-left:5pt; margin-right:5pt; }
table[rules] {border-left:solid black 0.4pt; border-right:solid black 0.4pt; }
td.td00{ padding-left:0pt; padding-right:0pt; }
td.td01{ padding-left:0pt; padding-right:5pt; }
td.td10{ padding-left:5pt; padding-right:0pt; }
td.td11{ padding-left:5pt; padding-right:5pt; }
table[rules] {border-left:solid black 0.4pt; border-right:solid black 0.4pt; }
.hline hr, .cline hr{ height : 1px; margin:0px; }
.tabbing-right {text-align:right;}
span.TEX {letter-spacing: -0.125em; }
span.TEX span.E{ position:relative;top:0.5ex;left:-0.0417em;}
a span.TEX span.E {text-decoration: none; }
span.LATEX span.A{ position:relative; top:-0.5ex; left:-0.4em; font-size:85%;}
span.LATEX span.TEX{ position:relative; left: -0.4em; }
div.float img, div.float .caption {text-align:center;}
div.figure img, div.figure .caption {text-align:center;}
.marginpar {width:20%; float:right; text-align:left; margin-left:auto; margin-top:0.5em; font-size:85%; text-decoration:underline;}
.marginpar p{margin-top:0.4em; margin-bottom:0.4em;}
.equation td{text-align:center; vertical-align:middle; }
td.eq-no{ width:5%; }
table.equation { width:100%; } 
div.math-display, div.par-math-display{text-align:center;}
math .texttt { font-family: monospace; }
math .textit { font-style: italic; }
math .textsl { font-style: oblique; }
math .textsf { font-family: sans-serif; }
math .textbf { font-weight: bold; }
.partToc a, .partToc, .likepartToc a, .likepartToc {line-height: 200%; font-weight:bold; font-size:110%;}
.chapterToc a, .chapterToc, .likechapterToc a, .likechapterToc, .appendixToc a, .appendixToc {line-height: 200%; font-weight:bold;}
.index-item, .index-subitem, .index-subsubitem {display:block}
.caption td.id{font-weight: bold; white-space: nowrap; }
table.caption {text-align:center;}
h1.partHead{text-align: center}
p.bibitem { text-indent: -2em; margin-left: 2em; margin-top:0.6em; margin-bottom:0.6em; }
p.bibitem-p { text-indent: 0em; margin-left: 2em; margin-top:0.6em; margin-bottom:0.6em; }
.paragraphHead, .likeparagraphHead { margin-top:2em; font-weight: bold;}
.subparagraphHead, .likesubparagraphHead { font-weight: bold;}
.quote {margin-bottom:0.25em; margin-top:0.25em; margin-left:1em; margin-right:1em; text-align:justify;}
.verse{white-space:nowrap; margin-left:2em}
div.maketitle {text-align:center;}
h2.titleHead{text-align:center;}
div.maketitle{ margin-bottom: 2em; }
div.author, div.date {text-align:center;}
div.thanks{text-align:left; margin-left:10%; font-size:85%; font-style:italic; }
div.author{white-space: nowrap;}
.quotation {margin-bottom:0.25em; margin-top:0.25em; margin-left:1em; }
h1.partHead{text-align: center}
.sectionToc, .likesectionToc {margin-left:2em;}
.subsectionToc, .likesubsectionToc {margin-left:4em;}
.subsubsectionToc, .likesubsubsectionToc {margin-left:6em;}
.frenchb-nbsp{font-size:75%;}
.frenchb-thinspace{font-size:75%;}
.figure img.graphics {margin-left:10%;}
/* end css.sty */

\title{Application aux endomorphismes continus et aux matrices}
\author{}
\date{}

\begin{document}
\maketitle

\textbf{Warning: 
requires JavaScript to process the mathematics on this page.\\ If your
browser supports JavaScript, be sure it is enabled.}

\begin{center}\rule{3in}{0.4pt}\end{center}

[
[
[]
[

\subsubsection{11.4 Application aux endomorphismes continus et aux
matrices}

\paragraph{11.4.1 Calcul fonctionnel et premières applications}

Soit E un K-espace vectoriel normé complet. Si u est un endomorphisme
continu de E, on pose
\u\
=\
sup_x\neq~0
\u(x)\
\over
\x\ . On sait que
\forall~~x \in E,
\u(x)\
\leq\
u\\,\x\.
Soit ℒ(E) l'algèbre des endomorphismes continus sur E. On sait que
(ℒ(E),\.\) est un
espace vectoriel normé complet et que \forall~~u,v
\inℒ(E), \v \cdot u\
\leq\
v\\,\u\.
En particulier, par une récurrence évidente sur n, on a

\forall~n \in \mathbb{N}~, \\forall~~u \inℒ(E),
\u^n\
\leq\ u\^n

Proposition~11.4.1 Soit
\\sum ~
a_nz^n une série entière à coefficients dans K de
rayon de convergence R > 0 et soit u \inℒ(E) tel que
\u\ < R.
Alors la série \\sum ~
a_nu^n est absolument convergente.

Démonstration On a
\a_nu^n\
\leqa_n\,\u\^n
et comme \u\
< R, la série
\\sum ~
a_n\,\u\^n
est convergente.

Remarque~11.4.1 Bien entendu, en introduisant la somme
\\sum ~
_n=0^+\infty~a_nu^n, on espère que bon
nombre des propriétés formelles de la somme S(z)
= \\sum ~
_n=0^+\infty~a_nz^n, valables pour z \in
D(0,R), se transmettront à
\\sum ~
_n=0^+\infty~a_nu^n.

Donnons une première application de ce calcul fonctionnel qui généralise
l'identité (1 - z)\\\sum
 _n=0^+\infty~z^n = 1~:

Proposition~11.4.2 L'ensemble des automorphismes continus de E est un
ouvert de ℒ(E).

Démonstration Soit u \inℒ(E) tel que
\u\ < 1.
D'après la proposition précédente, la série
\\sum ~
_n≥0u^n converge absolument. Soit s sa somme. On a

(\mathrmId_E - u) \cdot\left
(\\sum
_n=0^Nu^n\right ) =
\left (\\sum
_n=0^Nu^n\right ) \cdot
(\mathrmId_ E - u) =
\mathrmId_E - u^n+1

En faisant tendre n vers + \infty~, on a
(\mathrmId_E - u) \cdot s = s \cdot
(\mathrmId_E - u) =
\mathrmId_E, ce qui montre que
\mathrmId_E - u est un automorphisme continu
de E d'inverse s. Soit maintenant v un automorphisme continu de E et u
\inℒ(E). On écrit v + u = v \cdot (\mathrmId_E +
v^-1 \cdot u). D'après les préliminaires,
\mathrmId_E + v^-1 \cdot u (et donc v
+ u) est un automorphisme continu de E dès que
\v^-1 \cdot u\
< 1 et donc dès que
\u\ < 1
\over
\v^-1\ .
On en déduit que la boule B(v, 1 \over
\v^-1\ )
est contenue dans l'ensemble des automorphismes continus de E, qui est
donc ouvert.

\paragraph{11.4.2 Exponentielle d'un endomorphisme ou d'une matrice}

Définition~11.4.1 Si u \inℒ(E), on pose exp~ (u)
= \\sum ~
_n=0^+\infty~ u^n \over n! (série
absolument convergente)

Démonstration La série entière
\\sum  _n≥0~
z^n \over n! étant de rayon de convergence
infinie, la série \\\sum
 _n≥0 u^n \over n! est
absolument convergente quelle que soit la norme de u \inℒ(E).

Remarque~11.4.2 De même, si A \in M_p(K), on définit de la même
fa\ccon exp~ (A) =
e^A =\ \\sum
 _n=0^+\infty~ A^n \over n! .
On a bien entendu
Mat(\exp~ (u),\mathcal{E})
= exp (\Mat~(u,\mathcal{E})) si
\mathcal{E} est une base de E de dimension finie.

Proposition~11.4.3

\begin{itemize}
\item
  (i) Pour tout automorphisme continu v de E, on a
  exp (v^-1~ \cdot u \cdot v) =
  v^-1 \cdot exp~ (u) \cdot v
\item
  (ii) si u,v \inℒ(E) commutent, alors exp~ (u +
  v) = exp (u) \cdot\ exp~
  (v) = exp~ (v) \cdot\
  exp (u)~; en particulier, pour tout u \inℒ(E),
  exp~ (u) est un automorphisme continu de E et
  (exp (u))^-1~
  = exp~ (-u)
\item
  (iii) l'application \mathbb{R}~\mapsto~ℒ(E),
  t\mapsto~exp~ (tu) est de
  classe C^\infty~ et on a

  \forall~n \in \mathbb{N}~, d^n~
  \over dt^n  exp~
  (tu) = u^n \cdot exp~ (tu)
  = exp (tu) \cdot u^n~
\end{itemize}

Démonstration (i) On a
\\sum ~
_n=0^N (v^-1\cdotu\cdotv)^n
\over n! =\
\sum  _n=0^N~
v^-1\cdotu^n\cdotv \over n! =
v^-1 \cdot\left
(\\sum _
n=0^N u^n \over n!
\right ) \cdot v et en faisant tendre N vers + \infty~, on obtient
exp (v^-1~ \cdot u \cdot v) =
v^-1 \cdot exp~ (u) \cdot v.

(ii) Si u,v \inℒ(E) commutent, on pose a_n = u^n
\over n! et b_n = v^n
\over n! . Ces séries sont absolument convergentes. On
peut donc faire le produit de Cauchy de ces deux séries et on a alors
c_n = \\sum ~
_k=0^n 1 \over k!(n-k)!
u^kv^n-k = 1 \over n! (u +
v)^n d'après la formule du binôme (car u et v commutent). On a
donc

\sum _n=0^+\infty~~ (u +
v)^n \over n! = \left
(\sum _n=0^+\infty~ u^n~
\over n! \right ) \cdot\left
(\sum _n=0^+\infty~ v^n~
\over n! \right )

formule dans laquelle on peut également échanger u et v. On a alors bien
entendu exp~ (u) \cdot\
exp (-u) = exp~ (-u)
\cdot exp (u) =\ exp~ (u -
u) = exp~ (0) =
\mathrmId_E, ce qui montre que
exp~ (u) est un automorphisme continu de E et
que (exp (u))^-1~
= exp~ (-u)

(iii) On a exp~ (tu)
= \\sum ~
_k=0^+\infty~ u^k \over k!
t^k, série entière en t de rayon de convergence infini
puisqu'elle converge pour tout t. Sa somme est donc de classe
C^\infty~ et (en dérivant terme à terme cette série entière) on a

\begin{align*} d^n \over
dt^n  exp~ (tu)& =&
\sum _k=n^+\infty~ u^k~
\over (k - n)! t^k-n = u^n
\cdot\sum _k=n^+\infty~ u^k-n~
\over (k - n)! t^k-n\%&
\\ & =& u^n
\cdot exp~ (tu) \%&
\\ \end{align*}

Mais exp~ (tu) et u commutent évidemment, d'où
 d^n \over dt^n
 exp (tu) = u^n~
\cdot exp (tu) =\ exp~
(tu) \cdot u^n.

Bien entendu, ce théorème a sa traduction matricielle et on a

Théorème~11.4.4

\begin{itemize}
\item
  (i) \forall~A \in M_p~(K),
  \forall~P \in GL_p~(K),

  exp (P^-1~AP) =
  P^-1 exp~ (A)P
\item
  (ii) si A,B \in M_p(K) commutent, alors
  exp (A + B) =\ exp~
  (A)exp (B) =\ exp~
  (B)exp~ (A)~; en particulier, pour tout A \in
  M_p(K), exp~ (A) est dans
  GL_p(K) et (exp (A))^-1~
  = exp~ (-A)
\item
  (iii) l'application \mathbb{R}~\mapsto~M_p(K),
  t\mapsto~exp~ (tA) est de
  classe C^\infty~ et on a

  \forall~n \in \mathbb{N}~, d^n~
  \over dt^n  exp~
  (tA) = A^n exp~ (tA)
  = exp (tA)A^n~
\end{itemize}

La première propriété montre en particulier que si A est diagonalisable,
on a A =
P\mathrmdiag(\lambda_1,\\\ldots,\lambda_p)P^-1~,
et donc exp~ (A) =
P\mathrmdiag(e^\lambda_1,\\\ldots,e^\lambda_p)P^-1~.

Si A est nilpotente d'indice r, on a exp~ (A)
= \\sum ~
_n=0^r-1 A^n \over n! .

Si A \in M_p(\mathbb{C}) est quelconque, on a la décomposition de Jordan A
= D + N avec D diagonalisable, N nilpotente et DN = ND. On a donc
d'après la propriété (ii) ci dessus exp~ (A)
= exp (D)\exp~ (N) ce
qui permet le calcul de exp~ (A).

Une autre manière de voir, est d'introduire les sous-espaces
caractéristiques de u \in L(E). Soit
\lambda_1,\\ldots,\lambda_k~
les valeurs propres distinctes de u et E_i le sous-espace
caractéristique de u associé à \lambda_i. Soit u_i la
restriction de u à E_i, \pi_i la projection sur
E_i parallèlement à
\\oplus~ ~
_j\neq~iE_j. On a évidemment
exp (tu)__E_
i = exp (tu_i~) et donc
exp~ (tu) =\
\sum ~
_i=1^k exp (tu_i~) \cdot
\pi_i. Mais u_i =
\lambda_i\mathrmId + n_i avec
n_i nilpotent. On a donc exp~
(tu_i) = e^t\lambda_i\
\sum~
_k=0^r_i-1t^kn_i^k. On
en déduit que exp~ (tu)
= \\sum ~
_i=0^ke^t\lambda_i\
\sum~
_k=0^r_i-1t^kv_i,k, avec
v_i,k = n_i^k \cdot \pi_i ce qui donne la
forme générique de exp~ (tu) sous forme de
sommes de produits de fonctions exponentielles par des fonctions
polynomiales.

\paragraph{11.4.3 Application aux systèmes différentiels homogènes à
coefficients constants}

Soit A \in M_p(K) et le système différentiel à coefficients
constants

 dX \over dt = AX \Leftrightarrow
\left \ \cases 
dx_1 \over dt &= a_11x_1 +
\\ldots~ +
a_1px_p \cr
\\ldots~
\cr  dx_p \over dt &=
a_p1x_1 +
\\ldots~ +
a_ppx_p  \right .

Théorème~11.4.5 Soit X_0 \in M_p,1(K). L'unique solution
du système homogène  dX \over dt = AX vérifiant X(0)
= X_0 est l'application
t\mapsto~exp~
(tA)X_0.

Démonstration Cette application convient évidemment puisque  d
\over dt (exp~
(tA)X_0) = Aexp (tA)X_0~.
Soit t\mapsto~X(t) une autre solution et soit Y (t)
= exp~ (-tA)X(t). On a Y `(t) =
-exp~ (-tA)AX(t) +\
exp (-tA)X'(t) = exp~ (-tA)(X'(t) - AX(t)) =
0. On en déduit que Y est constante égale à Y (0). Mais Y (0) =
X_0. On a donc Y (t) = X_0 soit encore X(t)
= exp (tA)X_0~.

Remarque~11.4.3 En particulier, si K = \mathbb{C}, la discussion précédente
montre que les fonctions
x_1,\\ldots,x_p~
sont des exponentielles polynômes.

[
[
[
[

\end{document}
