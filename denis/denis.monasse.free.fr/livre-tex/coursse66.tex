\documentclass[]{article}
\usepackage[T1]{fontenc}
\usepackage{lmodern}
\usepackage{amssymb,amsmath}
\usepackage{ifxetex,ifluatex}
\usepackage{fixltx2e} % provides \textsubscript
% use upquote if available, for straight quotes in verbatim environments
\IfFileExists{upquote.sty}{\usepackage{upquote}}{}
\ifnum 0\ifxetex 1\fi\ifluatex 1\fi=0 % if pdftex
  \usepackage[utf8]{inputenc}
\else % if luatex or xelatex
  \ifxetex
    \usepackage{mathspec}
    \usepackage{xltxtra,xunicode}
  \else
    \usepackage{fontspec}
  \fi
  \defaultfontfeatures{Mapping=tex-text,Scale=MatchLowercase}
  \newcommand{\euro}{€}
\fi
% use microtype if available
\IfFileExists{microtype.sty}{\usepackage{microtype}}{}
\ifxetex
  \usepackage[setpagesize=false, % page size defined by xetex
              unicode=false, % unicode breaks when used with xetex
              xetex]{hyperref}
\else
  \usepackage[unicode=true]{hyperref}
\fi
\hypersetup{breaklinks=true,
            bookmarks=true,
            pdfauthor={},
            pdftitle={Application aux endomorphismes continus et aux matrices},
            colorlinks=true,
            citecolor=blue,
            urlcolor=blue,
            linkcolor=magenta,
            pdfborder={0 0 0}}
\urlstyle{same}  % don't use monospace font for urls
\setlength{\parindent}{0pt}
\setlength{\parskip}{6pt plus 2pt minus 1pt}
\setlength{\emergencystretch}{3em}  % prevent overfull lines
\setcounter{secnumdepth}{0}
 
/* start css.sty */
.cmr-5{font-size:50%;}
.cmr-7{font-size:70%;}
.cmmi-5{font-size:50%;font-style: italic;}
.cmmi-7{font-size:70%;font-style: italic;}
.cmmi-10{font-style: italic;}
.cmsy-5{font-size:50%;}
.cmsy-7{font-size:70%;}
.cmex-7{font-size:70%;}
.cmex-7x-x-71{font-size:49%;}
.msbm-7{font-size:70%;}
.cmtt-10{font-family: monospace;}
.cmti-10{ font-style: italic;}
.cmbx-10{ font-weight: bold;}
.cmr-17x-x-120{font-size:204%;}
.cmsl-10{font-style: oblique;}
.cmti-7x-x-71{font-size:49%; font-style: italic;}
.cmbxti-10{ font-weight: bold; font-style: italic;}
p.noindent { text-indent: 0em }
td p.noindent { text-indent: 0em; margin-top:0em; }
p.nopar { text-indent: 0em; }
p.indent{ text-indent: 1.5em }
@media print {div.crosslinks {visibility:hidden;}}
a img { border-top: 0; border-left: 0; border-right: 0; }
center { margin-top:1em; margin-bottom:1em; }
td center { margin-top:0em; margin-bottom:0em; }
.Canvas { position:relative; }
li p.indent { text-indent: 0em }
.enumerate1 {list-style-type:decimal;}
.enumerate2 {list-style-type:lower-alpha;}
.enumerate3 {list-style-type:lower-roman;}
.enumerate4 {list-style-type:upper-alpha;}
div.newtheorem { margin-bottom: 2em; margin-top: 2em;}
.obeylines-h,.obeylines-v {white-space: nowrap; }
div.obeylines-v p { margin-top:0; margin-bottom:0; }
.overline{ text-decoration:overline; }
.overline img{ border-top: 1px solid black; }
td.displaylines {text-align:center; white-space:nowrap;}
.centerline {text-align:center;}
.rightline {text-align:right;}
div.verbatim {font-family: monospace; white-space: nowrap; text-align:left; clear:both; }
.fbox {padding-left:3.0pt; padding-right:3.0pt; text-indent:0pt; border:solid black 0.4pt; }
div.fbox {display:table}
div.center div.fbox {text-align:center; clear:both; padding-left:3.0pt; padding-right:3.0pt; text-indent:0pt; border:solid black 0.4pt; }
div.minipage{width:100%;}
div.center, div.center div.center {text-align: center; margin-left:1em; margin-right:1em;}
div.center div {text-align: left;}
div.flushright, div.flushright div.flushright {text-align: right;}
div.flushright div {text-align: left;}
div.flushleft {text-align: left;}
.underline{ text-decoration:underline; }
.underline img{ border-bottom: 1px solid black; margin-bottom:1pt; }
.framebox-c, .framebox-l, .framebox-r { padding-left:3.0pt; padding-right:3.0pt; text-indent:0pt; border:solid black 0.4pt; }
.framebox-c {text-align:center;}
.framebox-l {text-align:left;}
.framebox-r {text-align:right;}
span.thank-mark{ vertical-align: super }
span.footnote-mark sup.textsuperscript, span.footnote-mark a sup.textsuperscript{ font-size:80%; }
div.tabular, div.center div.tabular {text-align: center; margin-top:0.5em; margin-bottom:0.5em; }
table.tabular td p{margin-top:0em;}
table.tabular {margin-left: auto; margin-right: auto;}
div.td00{ margin-left:0pt; margin-right:0pt; }
div.td01{ margin-left:0pt; margin-right:5pt; }
div.td10{ margin-left:5pt; margin-right:0pt; }
div.td11{ margin-left:5pt; margin-right:5pt; }
table[rules] {border-left:solid black 0.4pt; border-right:solid black 0.4pt; }
td.td00{ padding-left:0pt; padding-right:0pt; }
td.td01{ padding-left:0pt; padding-right:5pt; }
td.td10{ padding-left:5pt; padding-right:0pt; }
td.td11{ padding-left:5pt; padding-right:5pt; }
table[rules] {border-left:solid black 0.4pt; border-right:solid black 0.4pt; }
.hline hr, .cline hr{ height : 1px; margin:0px; }
.tabbing-right {text-align:right;}
span.TEX {letter-spacing: -0.125em; }
span.TEX span.E{ position:relative;top:0.5ex;left:-0.0417em;}
a span.TEX span.E {text-decoration: none; }
span.LATEX span.A{ position:relative; top:-0.5ex; left:-0.4em; font-size:85%;}
span.LATEX span.TEX{ position:relative; left: -0.4em; }
div.float img, div.float .caption {text-align:center;}
div.figure img, div.figure .caption {text-align:center;}
.marginpar {width:20%; float:right; text-align:left; margin-left:auto; margin-top:0.5em; font-size:85%; text-decoration:underline;}
.marginpar p{margin-top:0.4em; margin-bottom:0.4em;}
.equation td{text-align:center; vertical-align:middle; }
td.eq-no{ width:5%; }
table.equation { width:100%; } 
div.math-display, div.par-math-display{text-align:center;}
math .texttt { font-family: monospace; }
math .textit { font-style: italic; }
math .textsl { font-style: oblique; }
math .textsf { font-family: sans-serif; }
math .textbf { font-weight: bold; }
.partToc a, .partToc, .likepartToc a, .likepartToc {line-height: 200%; font-weight:bold; font-size:110%;}
.chapterToc a, .chapterToc, .likechapterToc a, .likechapterToc, .appendixToc a, .appendixToc {line-height: 200%; font-weight:bold;}
.index-item, .index-subitem, .index-subsubitem {display:block}
.caption td.id{font-weight: bold; white-space: nowrap; }
table.caption {text-align:center;}
h1.partHead{text-align: center}
p.bibitem { text-indent: -2em; margin-left: 2em; margin-top:0.6em; margin-bottom:0.6em; }
p.bibitem-p { text-indent: 0em; margin-left: 2em; margin-top:0.6em; margin-bottom:0.6em; }
.paragraphHead, .likeparagraphHead { margin-top:2em; font-weight: bold;}
.subparagraphHead, .likesubparagraphHead { font-weight: bold;}
.quote {margin-bottom:0.25em; margin-top:0.25em; margin-left:1em; margin-right:1em; text-align:justify;}
.verse{white-space:nowrap; margin-left:2em}
div.maketitle {text-align:center;}
h2.titleHead{text-align:center;}
div.maketitle{ margin-bottom: 2em; }
div.author, div.date {text-align:center;}
div.thanks{text-align:left; margin-left:10%; font-size:85%; font-style:italic; }
div.author{white-space: nowrap;}
.quotation {margin-bottom:0.25em; margin-top:0.25em; margin-left:1em; }
h1.partHead{text-align: center}
.sectionToc, .likesectionToc {margin-left:2em;}
.subsectionToc, .likesubsectionToc {margin-left:4em;}
.subsubsectionToc, .likesubsubsectionToc {margin-left:6em;}
.frenchb-nbsp{font-size:75%;}
.frenchb-thinspace{font-size:75%;}
.figure img.graphics {margin-left:10%;}
/* end css.sty */

\title{Application aux endomorphismes continus et aux matrices}
\author{}
\date{}

\begin{document}
\maketitle

\textbf{Warning: \href{http://www.math.union.edu/locate/jsMath}{jsMath}
requires JavaScript to process the mathematics on this page.\\ If your
browser supports JavaScript, be sure it is enabled.}

\begin{center}\rule{3in}{0.4pt}\end{center}

{[}\href{coursse65.html}{prev}{]}
{[}\href{coursse65.html\#tailcoursse65.html}{prev-tail}{]}
{[}\hyperref[tailcoursse66.html]{tail}{]}
{[}\href{coursch12.html\#coursse66.html}{up}{]}

\subsubsection{11.4 Application aux endomorphismes continus et aux
matrices}

\paragraph{11.4.1 Calcul fonctionnel et premières applications}

Soit E un K-espace vectoriel normé complet. Si u est un endomorphisme
continu de E, on pose
\textbackslash{}\textbar{}u\textbackslash{}\textbar{}
=\{\textbackslash{}mathop\{
sup\}\}\_\{x\textbackslash{}mathrel\{≠\}0\}\{
\textbackslash{}\textbar{}u(x)\textbackslash{}\textbar{}
\textbackslash{}over
\textbackslash{}\textbar{}x\textbackslash{}\textbar{}\} . On sait que
\textbackslash{}mathop\{∀\}x ∈ E,
\textbackslash{}\textbar{}u(x)\textbackslash{}\textbar{}
≤\textbackslash{}\textbar{}
u\textbackslash{}\textbar{}\textbackslash{},\textbackslash{}\textbar{}x\textbackslash{}\textbar{}.
Soit ℒ(E) l'algèbre des endomorphismes continus sur E. On sait que
(ℒ(E),\textbackslash{}\textbar{}.\textbackslash{}\textbar{}) est un
espace vectoriel normé complet et que \textbackslash{}mathop\{∀\}u,v
∈ℒ(E), \textbackslash{}\textbar{}v ∘ u\textbackslash{}\textbar{}
≤\textbackslash{}\textbar{}
v\textbackslash{}\textbar{}\textbackslash{},\textbackslash{}\textbar{}u\textbackslash{}\textbar{}.
En particulier, par une récurrence évidente sur n, on a

\textbackslash{}mathop\{∀\}n ∈ ℕ, \textbackslash{}mathop\{∀\}u ∈ℒ(E),
\textbackslash{}\textbar{}\{u\}\^{}\{n\}\textbackslash{}\textbar{}
≤\textbackslash{}\textbar{} \{u\textbackslash{}\textbar{}\}\^{}\{n\}

Proposition~11.4.1 Soit
\textbackslash{}mathop\{\textbackslash{}mathop\{∑ \}\}
\{a\}\_\{n\}\{z\}\^{}\{n\} une série entière à coefficients dans K de
rayon de convergence R \textgreater{} 0 et soit u ∈ℒ(E) tel que
\textbackslash{}\textbar{}u\textbackslash{}\textbar{} \textless{} R.
Alors la série \textbackslash{}mathop\{\textbackslash{}mathop\{∑ \}\}
\{a\}\_\{n\}\{u\}\^{}\{n\} est absolument convergente.

Démonstration On a
\textbackslash{}\textbar{}\{a\}\_\{n\}\{u\}\^{}\{n\}\textbackslash{}\textbar{}
≤\textbar{}\{a\}\_\{n\}\textbar{}\textbackslash{},\textbackslash{}\textbar{}\{u\textbackslash{}\textbar{}\}\^{}\{n\}
et comme \textbackslash{}\textbar{}u\textbackslash{}\textbar{}
\textless{} R, la série
\textbackslash{}mathop\{\textbackslash{}mathop\{∑ \}\}
\textbar{}\{a\}\_\{n\}\textbar{}\textbackslash{},\textbackslash{}\textbar{}\{u\textbackslash{}\textbar{}\}\^{}\{n\}
est convergente.

Remarque~11.4.1 Bien entendu, en introduisant la somme
\{\textbackslash{}mathop\{\textbackslash{}mathop\{∑ \}\}
\}\_\{n=0\}\^{}\{+∞\}\{a\}\_\{n\}\{u\}\^{}\{n\}, on espère que bon
nombre des propriétés formelles de la somme S(z)
=\{\textbackslash{}mathop\{ \textbackslash{}mathop\{∑ \}\}
\}\_\{n=0\}\^{}\{+∞\}\{a\}\_\{n\}\{z\}\^{}\{n\}, valables pour z ∈
D(0,R), se transmettront à
\{\textbackslash{}mathop\{\textbackslash{}mathop\{∑ \}\}
\}\_\{n=0\}\^{}\{+∞\}\{a\}\_\{n\}\{u\}\^{}\{n\}.

Donnons une première application de ce calcul fonctionnel qui généralise
l'identité (1 − z)\{\textbackslash{}mathop\{\textbackslash{}mathop\{∑
\}\} \}\_\{n=0\}\^{}\{+∞\}\{z\}\^{}\{n\} = 1~:

Proposition~11.4.2 L'ensemble des automorphismes continus de E est un
ouvert de ℒ(E).

Démonstration Soit u ∈ℒ(E) tel que
\textbackslash{}\textbar{}u\textbackslash{}\textbar{} \textless{} 1.
D'après la proposition précédente, la série
\{\textbackslash{}mathop\{\textbackslash{}mathop\{∑ \}\}
\}\_\{n≥0\}\{u\}\^{}\{n\} converge absolument. Soit s sa somme. On a

(\{\textbackslash{}mathrm\{Id\}\}\_\{E\} − u) ∘\textbackslash{}left
(\{\textbackslash{}mathop\{∑
\}\}\_\{n=0\}\^{}\{N\}\{u\}\^{}\{n\}\textbackslash{}right ) =
\textbackslash{}left (\{\textbackslash{}mathop\{∑
\}\}\_\{n=0\}\^{}\{N\}\{u\}\^{}\{n\}\textbackslash{}right ) ∘
(\{\textbackslash{}mathrm\{Id\}\}\_\{ E\} − u) =\{
\textbackslash{}mathrm\{Id\}\}\_\{E\} − \{u\}\^{}\{n+1\}

En faisant tendre n vers + ∞, on a
(\{\textbackslash{}mathrm\{Id\}\}\_\{E\} − u) ∘ s = s ∘
(\{\textbackslash{}mathrm\{Id\}\}\_\{E\} − u) =\{
\textbackslash{}mathrm\{Id\}\}\_\{E\}, ce qui montre que
\{\textbackslash{}mathrm\{Id\}\}\_\{E\} − u est un automorphisme continu
de E d'inverse s. Soit maintenant v un automorphisme continu de E et u
∈ℒ(E). On écrit v + u = v ∘ (\{\textbackslash{}mathrm\{Id\}\}\_\{E\} +
\{v\}\^{}\{−1\} ∘ u). D'après les préliminaires,
\{\textbackslash{}mathrm\{Id\}\}\_\{E\} + \{v\}\^{}\{−1\} ∘ u (et donc v
+ u) est un automorphisme continu de E dès que
\textbackslash{}\textbar{}\{v\}\^{}\{−1\} ∘ u\textbackslash{}\textbar{}
\textless{} 1 et donc dès que
\textbackslash{}\textbar{}u\textbackslash{}\textbar{} \textless{}\{ 1
\textbackslash{}over
\textbackslash{}\textbar{}\{v\}\^{}\{−1\}\textbackslash{}\textbar{}\} .
On en déduit que la boule B(v,\{ 1 \textbackslash{}over
\textbackslash{}\textbar{}\{v\}\^{}\{−1\}\textbackslash{}\textbar{}\} )
est contenue dans l'ensemble des automorphismes continus de E, qui est
donc ouvert.

\paragraph{11.4.2 Exponentielle d'un endomorphisme ou d'une matrice}

Définition~11.4.1 Si u ∈ℒ(E), on pose \textbackslash{}mathop\{exp\} (u)
=\{\textbackslash{}mathop\{ \textbackslash{}mathop\{∑ \}\}
\}\_\{n=0\}\^{}\{+∞\}\{ \{u\}\^{}\{n\} \textbackslash{}over n!\} (série
absolument convergente)

Démonstration La série entière
\{\textbackslash{}mathop\{\textbackslash{}mathop\{∑ \}\} \}\_\{n≥0\}\{
\{z\}\^{}\{n\} \textbackslash{}over n!\} étant de rayon de convergence
infinie, la série \{\textbackslash{}mathop\{\textbackslash{}mathop\{∑
\}\} \}\_\{n≥0\}\{ \{u\}\^{}\{n\} \textbackslash{}over n!\} est
absolument convergente quelle que soit la norme de u ∈ℒ(E).

Remarque~11.4.2 De même, si A ∈ \{M\}\_\{p\}(K), on définit de la même
fa\textbackslash{}c\{c\}on \textbackslash{}mathop\{exp\} (A) =
\{e\}\^{}\{A\} =\{\textbackslash{}mathop\{ \textbackslash{}mathop\{∑
\}\} \}\_\{n=0\}\^{}\{+∞\}\{ \{A\}\^{}\{n\} \textbackslash{}over n!\} .
On a bien entendu
\textbackslash{}mathop\{Mat\}(\textbackslash{}mathop\{exp\} (u),ℰ)
=\textbackslash{}mathop\{ exp\} (\textbackslash{}mathop\{Mat\}(u,ℰ)) si
ℰ est une base de E de dimension finie.

Proposition~11.4.3

\begin{itemize}
\item
  (i) Pour tout automorphisme continu v de E, on a
  \textbackslash{}mathop\{exp\} (\{v\}\^{}\{−1\} ∘ u ∘ v) =
  \{v\}\^{}\{−1\} ∘\textbackslash{}mathop\{ exp\} (u) ∘ v
\item
  (ii) si u,v ∈ℒ(E) commutent, alors \textbackslash{}mathop\{exp\} (u +
  v) =\textbackslash{}mathop\{ exp\} (u) ∘\textbackslash{}mathop\{ exp\}
  (v) =\textbackslash{}mathop\{ exp\} (v) ∘\textbackslash{}mathop\{
  exp\} (u)~; en particulier, pour tout u ∈ℒ(E),
  \textbackslash{}mathop\{exp\} (u) est un automorphisme continu de E et
  \{(\textbackslash{}mathop\{exp\} (u))\}\^{}\{−1\}
  =\textbackslash{}mathop\{ exp\} (−u)
\item
  (iii) l'application ℝ\textbackslash{}mathrel\{↦\}ℒ(E),
  t\textbackslash{}mathrel\{↦\}\textbackslash{}mathop\{exp\} (tu) est de
  classe \{C\}\^{}\{∞\} et on a

  \textbackslash{}mathop\{∀\}n ∈ ℕ,\{ \{d\}\^{}\{n\}
  \textbackslash{}over d\{t\}\^{}\{n\}\} \textbackslash{}mathop\{ exp\}
  (tu) = \{u\}\^{}\{n\} ∘\textbackslash{}mathop\{ exp\} (tu)
  =\textbackslash{}mathop\{ exp\} (tu) ∘ \{u\}\^{}\{n\}
\end{itemize}

Démonstration (i) On a
\{\textbackslash{}mathop\{\textbackslash{}mathop\{∑ \}\}
\}\_\{n=0\}\^{}\{N\}\{ \{(\{v\}\^{}\{−1\}∘u∘v)\}\^{}\{n\}
\textbackslash{}over n!\} =\{\textbackslash{}mathop\{
\textbackslash{}mathop\{∑ \}\} \}\_\{n=0\}\^{}\{N\}\{
\{v\}\^{}\{−1\}∘\{u\}\^{}\{n\}∘v \textbackslash{}over n!\} =
\{v\}\^{}\{−1\} ∘\textbackslash{}left
(\{\textbackslash{}mathop\{\textbackslash{}mathop\{∑\}\} \}\_\{
n=0\}\^{}\{N\}\{ \{u\}\^{}\{n\} \textbackslash{}over n!\}
\textbackslash{}right ) ∘ v et en faisant tendre N vers + ∞, on obtient
\textbackslash{}mathop\{exp\} (\{v\}\^{}\{−1\} ∘ u ∘ v) =
\{v\}\^{}\{−1\} ∘\textbackslash{}mathop\{ exp\} (u) ∘ v.

(ii) Si u,v ∈ℒ(E) commutent, on pose \{a\}\_\{n\} =\{ \{u\}\^{}\{n\}
\textbackslash{}over n!\} et \{b\}\_\{n\} =\{ \{v\}\^{}\{n\}
\textbackslash{}over n!\} . Ces séries sont absolument convergentes. On
peut donc faire le produit de Cauchy de ces deux séries et on a alors
\{c\}\_\{n\} =\{\textbackslash{}mathop\{ \textbackslash{}mathop\{∑ \}\}
\}\_\{k=0\}\^{}\{n\}\{ 1 \textbackslash{}over k!(n−k)!\}
\{u\}\^{}\{k\}\{v\}\^{}\{n−k\} =\{ 1 \textbackslash{}over n!\} \{(u +
v)\}\^{}\{n\} d'après la formule du binôme (car u et v commutent). On a
donc

\{\textbackslash{}mathop\{∑ \}\}\_\{n=0\}\^{}\{+∞\}\{ \{(u +
v)\}\^{}\{n\} \textbackslash{}over n!\} = \textbackslash{}left
(\{\textbackslash{}mathop\{∑ \}\}\_\{n=0\}\^{}\{+∞\}\{ \{u\}\^{}\{n\}
\textbackslash{}over n!\} \textbackslash{}right ) ∘\textbackslash{}left
(\{\textbackslash{}mathop\{∑ \}\}\_\{n=0\}\^{}\{+∞\}\{ \{v\}\^{}\{n\}
\textbackslash{}over n!\} \textbackslash{}right )

formule dans laquelle on peut également échanger u et v. On a alors bien
entendu \textbackslash{}mathop\{exp\} (u) ∘\textbackslash{}mathop\{
exp\} (−u) =\textbackslash{}mathop\{ exp\} (−u)
∘\textbackslash{}mathop\{ exp\} (u) =\textbackslash{}mathop\{ exp\} (u −
u) =\textbackslash{}mathop\{ exp\} (0) =\{
\textbackslash{}mathrm\{Id\}\}\_\{E\}, ce qui montre que
\textbackslash{}mathop\{exp\} (u) est un automorphisme continu de E et
que \{(\textbackslash{}mathop\{exp\} (u))\}\^{}\{−1\}
=\textbackslash{}mathop\{ exp\} (−u)

(iii) On a \textbackslash{}mathop\{exp\} (tu)
=\{\textbackslash{}mathop\{ \textbackslash{}mathop\{∑ \}\}
\}\_\{k=0\}\^{}\{+∞\}\{ \{u\}\^{}\{k\} \textbackslash{}over k!\}
\{t\}\^{}\{k\}, série entière en t de rayon de convergence infini
puisqu'elle converge pour tout t. Sa somme est donc de classe
\{C\}\^{}\{∞\} et (en dérivant terme à terme cette série entière) on a

\textbackslash{}begin\{eqnarray*\}\{ \{d\}\^{}\{n\} \textbackslash{}over
d\{t\}\^{}\{n\}\} \textbackslash{}mathop\{ exp\} (tu)\& =\&
\{\textbackslash{}mathop\{∑ \}\}\_\{k=n\}\^{}\{+∞\}\{ \{u\}\^{}\{k\}
\textbackslash{}over (k − n)!\} \{t\}\^{}\{k−n\} = \{u\}\^{}\{n\}
∘\{\textbackslash{}mathop\{∑ \}\}\_\{k=n\}\^{}\{+∞\}\{ \{u\}\^{}\{k−n\}
\textbackslash{}over (k − n)!\} \{t\}\^{}\{k−n\}\%\&
\textbackslash{}\textbackslash{} \& =\& \{u\}\^{}\{n\}
∘\textbackslash{}mathop\{ exp\} (tu) \%\&
\textbackslash{}\textbackslash{} \textbackslash{}end\{eqnarray*\}

Mais \textbackslash{}mathop\{exp\} (tu) et u commutent évidemment, d'où
\{ \{d\}\^{}\{n\} \textbackslash{}over d\{t\}\^{}\{n\}\}
\textbackslash{}mathop\{ exp\} (tu) = \{u\}\^{}\{n\}
∘\textbackslash{}mathop\{ exp\} (tu) =\textbackslash{}mathop\{ exp\}
(tu) ∘ \{u\}\^{}\{n\}.

Bien entendu, ce théorème a sa traduction matricielle et on a

Théorème~11.4.4

\begin{itemize}
\item
  (i) \textbackslash{}mathop\{∀\}A ∈ \{M\}\_\{p\}(K),
  \textbackslash{}mathop\{∀\}P ∈ G\{L\}\_\{p\}(K),

  \textbackslash{}mathop\{exp\} (\{P\}\^{}\{−1\}AP) =
  \{P\}\^{}\{−1\}\textbackslash{}mathop\{ exp\} (A)P
\item
  (ii) si A,B ∈ \{M\}\_\{p\}(K) commutent, alors
  \textbackslash{}mathop\{exp\} (A + B) =\textbackslash{}mathop\{ exp\}
  (A)\textbackslash{}mathop\{exp\} (B) =\textbackslash{}mathop\{ exp\}
  (B)\textbackslash{}mathop\{exp\} (A)~; en particulier, pour tout A ∈
  \{M\}\_\{p\}(K), \textbackslash{}mathop\{exp\} (A) est dans
  G\{L\}\_\{p\}(K) et \{(\textbackslash{}mathop\{exp\} (A))\}\^{}\{−1\}
  =\textbackslash{}mathop\{ exp\} (−A)
\item
  (iii) l'application ℝ\textbackslash{}mathrel\{↦\}\{M\}\_\{p\}(K),
  t\textbackslash{}mathrel\{↦\}\textbackslash{}mathop\{exp\} (tA) est de
  classe \{C\}\^{}\{∞\} et on a

  \textbackslash{}mathop\{∀\}n ∈ ℕ,\{ \{d\}\^{}\{n\}
  \textbackslash{}over d\{t\}\^{}\{n\}\} \textbackslash{}mathop\{ exp\}
  (tA) = \{A\}\^{}\{n\}\textbackslash{}mathop\{ exp\} (tA)
  =\textbackslash{}mathop\{ exp\} (tA)\{A\}\^{}\{n\}
\end{itemize}

La première propriété montre en particulier que si A est diagonalisable,
on a A =
P\textbackslash{}mathop\{\textbackslash{}mathrm\{diag\}\}(\{λ\}\_\{1\},\textbackslash{}mathop\{\textbackslash{}mathop\{\ldots{}\}\},\{λ\}\_\{p\})\{P\}\^{}\{−1\},
et donc \textbackslash{}mathop\{exp\} (A) =
P\textbackslash{}mathop\{\textbackslash{}mathrm\{diag\}\}(\{e\}\^{}\{\{λ\}\_\{1\}\},\textbackslash{}mathop\{\textbackslash{}mathop\{\ldots{}\}\},\{e\}\^{}\{\{λ\}\_\{p\}\})\{P\}\^{}\{−1\}.

Si A est nilpotente d'indice r, on a \textbackslash{}mathop\{exp\} (A)
=\{\textbackslash{}mathop\{ \textbackslash{}mathop\{∑ \}\}
\}\_\{n=0\}\^{}\{r−1\}\{ \{A\}\^{}\{n\} \textbackslash{}over n!\} .

Si A ∈ \{M\}\_\{p\}(ℂ) est quelconque, on a la décomposition de Jordan A
= D + N avec D diagonalisable, N nilpotente et DN = ND. On a donc
d'après la propriété (ii) ci dessus \textbackslash{}mathop\{exp\} (A)
=\textbackslash{}mathop\{ exp\} (D)\textbackslash{}mathop\{exp\} (N) ce
qui permet le calcul de \textbackslash{}mathop\{exp\} (A).

Une autre manière de voir, est d'introduire les sous-espaces
caractéristiques de u ∈ L(E). Soit
\{λ\}\_\{1\},\textbackslash{}mathop\{\textbackslash{}mathop\{\ldots{}\}\},\{λ\}\_\{k\}
les valeurs propres distinctes de u et \{E\}\_\{i\} le sous-espace
caractéristique de u associé à \{λ\}\_\{i\}. Soit \{u\}\_\{i\} la
restriction de u à \{E\}\_\{i\}, \{π\}\_\{i\} la projection sur
\{E\}\_\{i\} parallèlement à
\{\textbackslash{}mathop\{\textbackslash{}mathop\{⊕ \}\}
\}\_\{j\textbackslash{}mathrel\{≠\}i\}\{E\}\_\{j\}. On a évidemment
\textbackslash{}mathop\{exp\} \{(tu)\}\_\{\{\textbar{}\}\_\{\{E\}\_\{
i\}\}\} =\textbackslash{}mathop\{ exp\} (t\{u\}\_\{i\}) et donc
\textbackslash{}mathop\{exp\} (tu) =\{\textbackslash{}mathop\{
\textbackslash{}mathop\{∑ \}\}
\}\_\{i=1\}\^{}\{k\}\textbackslash{}mathop\{ exp\} (t\{u\}\_\{i\}) ∘
\{π\}\_\{i\}. Mais \{u\}\_\{i\} =
\{λ\}\_\{i\}\textbackslash{}mathrm\{Id\} + \{n\}\_\{i\} avec
\{n\}\_\{i\} nilpotent. On a donc \textbackslash{}mathop\{exp\}
(t\{u\}\_\{i\}) = \{e\}\^{}\{t\{λ\}\_\{i\}\}\{\textbackslash{}mathop\{
\textbackslash{}mathop\{∑\}\}
\}\_\{k=0\}\^{}\{\{r\}\_\{i\}−1\}\{t\}\^{}\{k\}\{n\}\_\{i\}\^{}\{k\}. On
en déduit que \textbackslash{}mathop\{exp\} (tu)
=\{\textbackslash{}mathop\{ \textbackslash{}mathop\{∑ \}\}
\}\_\{i=0\}\^{}\{k\}\{e\}\^{}\{t\{λ\}\_\{i\}\}\{\textbackslash{}mathop\{
\textbackslash{}mathop\{∑\}\}
\}\_\{k=0\}\^{}\{\{r\}\_\{i\}−1\}\{t\}\^{}\{k\}\{v\}\_\{i,k\}, avec
\{v\}\_\{i,k\} = \{n\}\_\{i\}\^{}\{k\} ∘ \{π\}\_\{i\} ce qui donne la
forme générique de \textbackslash{}mathop\{exp\} (tu) sous forme de
sommes de produits de fonctions exponentielles par des fonctions
polynomiales.

\paragraph{11.4.3 Application aux systèmes différentiels homogènes à
coefficients constants}

Soit A ∈ \{M\}\_\{p\}(K) et le système différentiel à coefficients
constants

\{ dX \textbackslash{}over dt\} = AX \textbackslash{}mathrel\{⇔\}
\textbackslash{}left \textbackslash{}\{ \textbackslash{}cases\{ \{
d\{x\}\_\{1\} \textbackslash{}over dt\} \&= \{a\}\_\{11\}\{x\}\_\{1\} +
\textbackslash{}mathop\{\textbackslash{}mathop\{\ldots{}\}\} +
\{a\}\_\{1p\}\{x\}\_\{p\} \textbackslash{}cr
\textbackslash{}mathop\{\textbackslash{}mathop\{\ldots{}\}\}
\textbackslash{}cr \{ d\{x\}\_\{p\} \textbackslash{}over dt\} \&=
\{a\}\_\{p1\}\{x\}\_\{1\} +
\textbackslash{}mathop\{\textbackslash{}mathop\{\ldots{}\}\} +
\{a\}\_\{pp\}\{x\}\_\{p\} \} \textbackslash{}right .

Théorème~11.4.5 Soit \{X\}\_\{0\} ∈ \{M\}\_\{p,1\}(K). L'unique solution
du système homogène \{ dX \textbackslash{}over dt\} = AX vérifiant X(0)
= \{X\}\_\{0\} est l'application
t\textbackslash{}mathrel\{↦\}\textbackslash{}mathop\{exp\}
(tA)\{X\}\_\{0\}.

Démonstration Cette application convient évidemment puisque \{ d
\textbackslash{}over dt\} (\textbackslash{}mathop\{exp\}
(tA)\{X\}\_\{0\}) = A\textbackslash{}mathop\{exp\} (tA)\{X\}\_\{0\}.
Soit t\textbackslash{}mathrel\{↦\}X(t) une autre solution et soit Y (t)
=\textbackslash{}mathop\{ exp\} (−tA)X(t). On a Y `(t) =
−\textbackslash{}mathop\{exp\} (−tA)AX(t) +\textbackslash{}mathop\{
exp\} (−tA)X'(t) =\textbackslash{}mathop\{ exp\} (−tA)(X'(t) − AX(t)) =
0. On en déduit que Y est constante égale à Y (0). Mais Y (0) =
\{X\}\_\{0\}. On a donc Y (t) = \{X\}\_\{0\} soit encore X(t)
=\textbackslash{}mathop\{ exp\} (tA)\{X\}\_\{0\}.

Remarque~11.4.3 En particulier, si K = ℂ, la discussion précédente
montre que les fonctions
\{x\}\_\{1\},\textbackslash{}mathop\{\textbackslash{}mathop\{\ldots{}\}\},\{x\}\_\{p\}
sont des exponentielles polynômes.

{[}\href{coursse65.html}{prev}{]}
{[}\href{coursse65.html\#tailcoursse65.html}{prev-tail}{]}
{[}\href{coursse66.html}{front}{]}
{[}\href{coursch12.html\#coursse66.html}{up}{]}

\end{document}
