\documentclass[]{article}
\usepackage[T1]{fontenc}
\usepackage{lmodern}
\usepackage{amssymb,amsmath}
\usepackage{ifxetex,ifluatex}
\usepackage{fixltx2e} % provides \textsubscript
% use upquote if available, for straight quotes in verbatim environments
\IfFileExists{upquote.sty}{\usepackage{upquote}}{}
\ifnum 0\ifxetex 1\fi\ifluatex 1\fi=0 % if pdftex
  \usepackage[utf8]{inputenc}
\else % if luatex or xelatex
  \ifxetex
    \usepackage{mathspec}
    \usepackage{xltxtra,xunicode}
  \else
    \usepackage{fontspec}
  \fi
  \defaultfontfeatures{Mapping=tex-text,Scale=MatchLowercase}
  \newcommand{\euro}{€}
\fi
% use microtype if available
\IfFileExists{microtype.sty}{\usepackage{microtype}}{}
\ifxetex
  \usepackage[setpagesize=false, % page size defined by xetex
              unicode=false, % unicode breaks when used with xetex
              xetex]{hyperref}
\else
  \usepackage[unicode=true]{hyperref}
\fi
\hypersetup{breaklinks=true,
            bookmarks=true,
            pdfauthor={},
            pdftitle={Formes quadratiques},
            colorlinks=true,
            citecolor=blue,
            urlcolor=blue,
            linkcolor=magenta,
            pdfborder={0 0 0}}
\urlstyle{same}  % don't use monospace font for urls
\setlength{\parindent}{0pt}
\setlength{\parskip}{6pt plus 2pt minus 1pt}
\setlength{\emergencystretch}{3em}  % prevent overfull lines
\setcounter{secnumdepth}{0}
 
/* start css.sty */
.cmr-5{font-size:50%;}
.cmr-7{font-size:70%;}
.cmmi-5{font-size:50%;font-style: italic;}
.cmmi-7{font-size:70%;font-style: italic;}
.cmmi-10{font-style: italic;}
.cmsy-5{font-size:50%;}
.cmsy-7{font-size:70%;}
.cmex-7{font-size:70%;}
.cmex-7x-x-71{font-size:49%;}
.msbm-7{font-size:70%;}
.cmtt-10{font-family: monospace;}
.cmti-10{ font-style: italic;}
.cmbx-10{ font-weight: bold;}
.cmr-17x-x-120{font-size:204%;}
.cmsl-10{font-style: oblique;}
.cmti-7x-x-71{font-size:49%; font-style: italic;}
.cmbxti-10{ font-weight: bold; font-style: italic;}
p.noindent { text-indent: 0em }
td p.noindent { text-indent: 0em; margin-top:0em; }
p.nopar { text-indent: 0em; }
p.indent{ text-indent: 1.5em }
@media print {div.crosslinks {visibility:hidden;}}
a img { border-top: 0; border-left: 0; border-right: 0; }
center { margin-top:1em; margin-bottom:1em; }
td center { margin-top:0em; margin-bottom:0em; }
.Canvas { position:relative; }
li p.indent { text-indent: 0em }
.enumerate1 {list-style-type:decimal;}
.enumerate2 {list-style-type:lower-alpha;}
.enumerate3 {list-style-type:lower-roman;}
.enumerate4 {list-style-type:upper-alpha;}
div.newtheorem { margin-bottom: 2em; margin-top: 2em;}
.obeylines-h,.obeylines-v {white-space: nowrap; }
div.obeylines-v p { margin-top:0; margin-bottom:0; }
.overline{ text-decoration:overline; }
.overline img{ border-top: 1px solid black; }
td.displaylines {text-align:center; white-space:nowrap;}
.centerline {text-align:center;}
.rightline {text-align:right;}
div.verbatim {font-family: monospace; white-space: nowrap; text-align:left; clear:both; }
.fbox {padding-left:3.0pt; padding-right:3.0pt; text-indent:0pt; border:solid black 0.4pt; }
div.fbox {display:table}
div.center div.fbox {text-align:center; clear:both; padding-left:3.0pt; padding-right:3.0pt; text-indent:0pt; border:solid black 0.4pt; }
div.minipage{width:100%;}
div.center, div.center div.center {text-align: center; margin-left:1em; margin-right:1em;}
div.center div {text-align: left;}
div.flushright, div.flushright div.flushright {text-align: right;}
div.flushright div {text-align: left;}
div.flushleft {text-align: left;}
.underline{ text-decoration:underline; }
.underline img{ border-bottom: 1px solid black; margin-bottom:1pt; }
.framebox-c, .framebox-l, .framebox-r { padding-left:3.0pt; padding-right:3.0pt; text-indent:0pt; border:solid black 0.4pt; }
.framebox-c {text-align:center;}
.framebox-l {text-align:left;}
.framebox-r {text-align:right;}
span.thank-mark{ vertical-align: super }
span.footnote-mark sup.textsuperscript, span.footnote-mark a sup.textsuperscript{ font-size:80%; }
div.tabular, div.center div.tabular {text-align: center; margin-top:0.5em; margin-bottom:0.5em; }
table.tabular td p{margin-top:0em;}
table.tabular {margin-left: auto; margin-right: auto;}
div.td00{ margin-left:0pt; margin-right:0pt; }
div.td01{ margin-left:0pt; margin-right:5pt; }
div.td10{ margin-left:5pt; margin-right:0pt; }
div.td11{ margin-left:5pt; margin-right:5pt; }
table[rules] {border-left:solid black 0.4pt; border-right:solid black 0.4pt; }
td.td00{ padding-left:0pt; padding-right:0pt; }
td.td01{ padding-left:0pt; padding-right:5pt; }
td.td10{ padding-left:5pt; padding-right:0pt; }
td.td11{ padding-left:5pt; padding-right:5pt; }
table[rules] {border-left:solid black 0.4pt; border-right:solid black 0.4pt; }
.hline hr, .cline hr{ height : 1px; margin:0px; }
.tabbing-right {text-align:right;}
span.TEX {letter-spacing: -0.125em; }
span.TEX span.E{ position:relative;top:0.5ex;left:-0.0417em;}
a span.TEX span.E {text-decoration: none; }
span.LATEX span.A{ position:relative; top:-0.5ex; left:-0.4em; font-size:85%;}
span.LATEX span.TEX{ position:relative; left: -0.4em; }
div.float img, div.float .caption {text-align:center;}
div.figure img, div.figure .caption {text-align:center;}
.marginpar {width:20%; float:right; text-align:left; margin-left:auto; margin-top:0.5em; font-size:85%; text-decoration:underline;}
.marginpar p{margin-top:0.4em; margin-bottom:0.4em;}
.equation td{text-align:center; vertical-align:middle; }
td.eq-no{ width:5%; }
table.equation { width:100%; } 
div.math-display, div.par-math-display{text-align:center;}
math .texttt { font-family: monospace; }
math .textit { font-style: italic; }
math .textsl { font-style: oblique; }
math .textsf { font-family: sans-serif; }
math .textbf { font-weight: bold; }
.partToc a, .partToc, .likepartToc a, .likepartToc {line-height: 200%; font-weight:bold; font-size:110%;}
.chapterToc a, .chapterToc, .likechapterToc a, .likechapterToc, .appendixToc a, .appendixToc {line-height: 200%; font-weight:bold;}
.index-item, .index-subitem, .index-subsubitem {display:block}
.caption td.id{font-weight: bold; white-space: nowrap; }
table.caption {text-align:center;}
h1.partHead{text-align: center}
p.bibitem { text-indent: -2em; margin-left: 2em; margin-top:0.6em; margin-bottom:0.6em; }
p.bibitem-p { text-indent: 0em; margin-left: 2em; margin-top:0.6em; margin-bottom:0.6em; }
.paragraphHead, .likeparagraphHead { margin-top:2em; font-weight: bold;}
.subparagraphHead, .likesubparagraphHead { font-weight: bold;}
.quote {margin-bottom:0.25em; margin-top:0.25em; margin-left:1em; margin-right:1em; text-align:justify;}
.verse{white-space:nowrap; margin-left:2em}
div.maketitle {text-align:center;}
h2.titleHead{text-align:center;}
div.maketitle{ margin-bottom: 2em; }
div.author, div.date {text-align:center;}
div.thanks{text-align:left; margin-left:10%; font-size:85%; font-style:italic; }
div.author{white-space: nowrap;}
.quotation {margin-bottom:0.25em; margin-top:0.25em; margin-left:1em; }
h1.partHead{text-align: center}
.sectionToc, .likesectionToc {margin-left:2em;}
.subsectionToc, .likesubsectionToc {margin-left:4em;}
.subsubsectionToc, .likesubsubsectionToc {margin-left:6em;}
.frenchb-nbsp{font-size:75%;}
.frenchb-thinspace{font-size:75%;}
.figure img.graphics {margin-left:10%;}
/* end css.sty */

\title{Formes quadratiques}
\author{}
\date{}

\begin{document}
\maketitle

\textbf{Warning: \href{http://www.math.union.edu/locate/jsMath}{jsMath}
requires JavaScript to process the mathematics on this page.\\ If your
browser supports JavaScript, be sure it is enabled.}

\begin{center}\rule{3in}{0.4pt}\end{center}

{[}\href{coursse69.html}{next}{]} {[}\href{coursse67.html}{prev}{]}
{[}\href{coursse67.html\#tailcoursse67.html}{prev-tail}{]}
{[}\hyperref[tailcoursse68.html]{tail}{]}
{[}\href{coursch13.html\#coursse68.html}{up}{]}

\subsubsection{12.2 Formes quadratiques}

\paragraph{12.2.1 Notion de forme quadratique}

Soit E un K-espace vectoriel et φ une forme bilinéaire symétrique sur E.
Soit Φ l'application de E dans K qui à x associe Φ(x) = φ(x,x).

Proposition~12.2.1 On a les identités suivantes

\begin{itemize}
\itemsep1pt\parskip0pt\parsep0pt
\item
  (i) Φ(λx) = \{λ\}\^{}\{2\}Φ(x)
\item
  (ii) Φ(x + y) = Φ(x) + 2φ(x,y) + Φ(y) (identité de polarisation)
\item
  (iii) Φ(x + y) + Φ(x − y) = 2(Φ(x) + Φ(y)) (identité de la médiane)
\end{itemize}

Démonstration (i) Φ(λx) = φ(λx,λx) = \{λ\}\^{}\{2\}φ(x,x) =
\{λ\}\^{}\{2\}Φ(x)

(ii) Φ(x + y) = φ(x + y,x + y) = Φ(x) + φ(x,y) + φ(y,x) + Φ(y) = Φ(x) +
2φ(x,y) + Φ(y)

(iii) changeant y en − y dans l'identité précédente, on a aussi Φ(x − y)
= Φ(x) − 2φ(x,y) + Φ(y), et en additionnant les deux on trouve Φ(x + y)
+ Φ(x − y) = 2(Φ(x) + Φ(y)).

Remarque~12.2.1 Si
\textbackslash{}mathop\{\textbackslash{}mathrm\{car\}\}K\textbackslash{}mathrel\{≠\}2,
l'identité (ii) montre que l'application φ\textbackslash{}mathrel\{↦\}Φ
est injective de \{S\}\_\{2\}(E) dans \{K\}\^{}\{E\} (espace vectoriel
des applications de E dans K) puisque la connaissance de Φ permet de
retrouver φ par

φ(x,y) =\{ 1 \textbackslash{}over 2\} (Φ(x + y) − Φ(x) − Φ(y))

Ceci nous amène à poser

Définition~12.2.1 Soit K un corps de caractéristique différente de 2 et
E un K-espace vectoriel . On appelle forme quadratique sur E toute
application Φ : E → K vérifiant les deux propriétés

\begin{itemize}
\itemsep1pt\parskip0pt\parsep0pt
\item
  (i) \textbackslash{}mathop\{∀\}λ ∈ K, \textbackslash{}mathop\{∀\}x ∈
  E, Φ(λx) = \{λ\}\^{}\{2\}Φ(x)
\item
  (ii) l'application φ : E × E → K, φ(x,y) =\{ 1 \textbackslash{}over
  2\} (Φ(x + y) − Φ(x) − Φ(y)) est une forme bilinéaire (évidemment
  symétrique) sur E.
\end{itemize}

Dans ce cas, on a \textbackslash{}mathop\{∀\}x ∈ E, Φ(x) = φ(x,x)~; φ
est appelée la forme polaire de Φ.

Démonstration On a φ(x,x) =\{ 1 \textbackslash{}over 2\} (Φ(2x) − 2Φ(x))
=\{ 1 \textbackslash{}over 2\} (4Φ(x) − 2Φ(x)) = Φ(x) en utilisant la
propriété (i).

Exemple~12.2.1 Sur \{K\}\^{}\{n\}, Φ(x) =\{\textbackslash{}mathop\{
\textbackslash{}mathop\{∑ \}\} \}\_\{i=1\}\^{}\{n\}\{x\}\_\{i\}\^{}\{2\}
est une forme quadratique dont la forme polaire associée est φ(x,y)
=\{\textbackslash{}mathop\{ \textbackslash{}mathop\{∑ \}\}
\}\_\{i=1\}\^{}\{n\}\{x\}\_\{i\}\{y\}\_\{i\}. Si K = ℝ ou K = ℂ, et si E
désigne l'espace vectoriel des fonctions continues de {[}a,b{]} dans K,
Φ(f) =\{\textbackslash{}mathop\{∫ \} \}\_\{a\}\^{}\{b\}f\{(t)\}\^{}\{2\}
dt est une forme quadratique dont la forme polaire est φ(f,g)
=\{\textbackslash{}mathop\{∫ \} \}\_\{a\}\^{}\{b\}f(t)g(t) dt.

Proposition~12.2.2 L'ensemble Q(E) des formes quadratiques sur E est un
sous-espace vectoriel de \{K\}\^{}\{E\}~; l'application
φ\textbackslash{}mathrel\{↦\}Φ est un isomorphisme d'espaces vectoriels
de \{S\}\_\{2\}(E) sur Q(E).

Remarque~12.2.2 Par la suite on confondra toutes les notions relatives à
φ et à Φ~: orthogonalité, matrice, non dégénérescence, isotropie~; en
particulier on posera
\textbackslash{}mathop\{\textbackslash{}mathrm\{Ker\}\}Φ
=\textbackslash{}mathop\{ \textbackslash{}mathrm\{Ker\}\}φ =
\textbackslash{}\{x ∈
E\textbackslash{}mathrel\{∣\}\textbackslash{}mathop\{∀\}y ∈ E, φ(x,y) =
0\textbackslash{}\}. On remarquera qu'en général,
\textbackslash{}mathop\{\textbackslash{}mathrm\{Ker\}\}Φ\textbackslash{}mathrel\{≠\}\textbackslash{}\{x
∈ E\textbackslash{}mathrel\{∣\}Φ(x) = 0\textbackslash{}\}.

Théorème~12.2.3 (Pythagore). Soit E un K-espace vectoriel ~et Φ ∈Q(E), φ
la forme polaire de Φ. Alors

x \{⊥\}\_\{φ\}y \textbackslash{}mathrel\{⇔\} Φ(x + y) = Φ(x) + Φ(y)

Démonstration C'est une conséquence évidente de l'identité de
polarisation.

\paragraph{12.2.2 Formes quadratiques en dimension finie}

Soit E un K-espace vectoriel ~de dimension finie, Φ ∈Q(E) de forme
polaire φ.

Théorème~12.2.4 Soit ℰ une base de E. Alors
\textbackslash{}mathop\{\textbackslash{}mathrm\{Mat\}\} (φ,ℰ) est
l'unique matrice Ω ∈ \{M\}\_\{K\}(n) qui est symétrique et qui vérifie

\textbackslash{}mathop\{∀\}x ∈ E, Φ(x) \{= \}\^{}\{t\}XΩX

Démonstration Il est clair que Ω =\textbackslash{}mathop\{
\textbackslash{}mathrm\{Mat\}\} (Φ,ℰ) est symétrique et vérifie Φ(x) =
φ(x,x) \{= \}\^{}\{t\}XΩX. Inversement, soit Ω une matrice symétrique
vérifiant cette propriété. On a alors

\textbackslash{}begin\{eqnarray*\} φ(x,y)\& =\&\{ 1 \textbackslash{}over
2\} (Φ(x + y) − Φ(x) − Φ(y)) \%\& \textbackslash{}\textbackslash{} \&
=\&\{ 1 \textbackslash{}over 2\} \{(\}\^{}\{t\}(X + Y )Ω(X + Y )
\{−\}\^{}\{t\}XΩX \{−\}\^{}\{t\}Y ΩY \%\&
\textbackslash{}\textbackslash{} \& =\&\{ 1 \textbackslash{}over 2\}
\{(\}\^{}\{t\}XΩY \{+ \}\^{}\{t\}Y ΩX) \%\&
\textbackslash{}\textbackslash{} \textbackslash{}end\{eqnarray*\}

Mais, une matrice 1 × 1 étant forcément symétrique \{\}\^{}\{t\}Y ΩX \{=
\}\^{}\{t\}\{(\}\^{}\{t\}Y ΩX) \{= \}\^{}\{t\}\{X\}\^{}\{t\}ΩY \{=
\}\^{}\{t\}XΩY puisque Ω est symétrique. On a donc φ(x,y) \{=
\}\^{}\{t\}XΩY ce qui montre que Ω =\textbackslash{}mathop\{
\textbackslash{}mathrm\{Mat\}\} (φ,ℰ).

Remarque~12.2.3 On prendra garde à la condition de symétrie de Ω. Il est
en effet clair que l'on peut remplacer, dans la condition Φ(x) \{=
\}\^{}\{t\}XΩX, la matrice Ω par une matrice Ω' = Ω + A où A est
antisymétrique, puisque dans ce cas \{\}\^{}\{t\}XAX = 0. On aura alors
Φ(x) \{= \}\^{}\{t\}XΩ'X bien que Ω' ne soit pas la matrice de Φ dans la
base ℰ.

Posons Ω =\textbackslash{}mathop\{ \textbackslash{}mathrm\{Mat\}\} (φ,ℰ)
= \{(\{ω\}\_\{i,j\})\}\_\{1≤i,j≤n\}. On a alors

φ(x,y) =\{ \textbackslash{}mathop\{∑
\}\}\_\{i,j\}\{ω\}\_\{i,j\}\{x\}\_\{i\}\{y\}\_\{j\} =\{
\textbackslash{}mathop\{∑
\}\}\_\{i\}\{ω\}\_\{i,i\}\{x\}\_\{i\}\{y\}\_\{i\} +\{
\textbackslash{}mathop\{∑
\}\}\_\{i\textless{}j\}\{ω\}\_\{i,j\}(\{x\}\_\{i\}\{y\}\_\{j\} +
\{x\}\_\{j\}\{y\}\_\{i\})

en tenant compte de \{ω\}\_\{i,j\} = \{ω\}\_\{j,i\}. On a donc

Φ(x) = φ(x,x) =\{ \textbackslash{}mathop\{∑
\}\}\_\{i\}\{ω\}\_\{i,i\}\{x\}\_\{i\}\^{}\{2\} +
2\{\textbackslash{}mathop\{∑
\}\}\_\{i\textless{}j\}\{ω\}\_\{i,j\}\{x\}\_\{i\}\{x\}\_\{j\} =
\{P\}\_\{Φ\}(\{x\}\_\{1\},\textbackslash{}mathop\{\ldots{}\},\{x\}\_\{n\})

où \{P\}\_\{Φ\} est le polynôme homogène de degré 2 à n variables
\{P\}\_\{Φ\}(\{X\}\_\{1\},\textbackslash{}mathop\{\textbackslash{}mathop\{\ldots{}\}\},\{X\}\_\{n\})
=\{\textbackslash{}mathop\{ \textbackslash{}mathop\{∑ \}\}
\}\_\{i\}\{ω\}\_\{i,i\}\{X\}\_\{i\}\^{}\{2\} +\{\textbackslash{}mathop\{
\textbackslash{}mathop\{∑ \}\}
\}\_\{i\textless{}j\}\{ω\}\_\{i,j\}\{X\}\_\{i\}\{X\}\_\{j\}.
Inversement, soit P un polynôme homogène de degré 2 à n variables,
P(\{X\}\_\{1\},\textbackslash{}mathop\{\textbackslash{}mathop\{\ldots{}\}\},\{X\}\_\{n\})
=\{\textbackslash{}mathop\{ \textbackslash{}mathop\{∑ \}\}
\}\_\{i=1\}\^{}\{n\}\{a\}\_\{i,i\}\{X\}\_\{i\}\^{}\{2\}
+\{\textbackslash{}mathop\{ \textbackslash{}mathop\{∑ \}\}
\}\_\{i\textless{}j\}\{a\}\_\{i,j\}\{X\}\_\{i\}\{X\}\_\{j\}. Définissons
φ sur E par

φ(x,y) =\{ \textbackslash{}mathop\{∑
\}\}\_\{i\}\{a\}\_\{i,i\}\{x\}\_\{i\}\{y\}\_\{i\} +\{
\textbackslash{}mathop\{∑ \}\}\_\{i\textless{}j\}\{ \{a\}\_\{i,j\}
\textbackslash{}over 2\} (\{x\}\_\{i\}\{y\}\_\{j\} +
\{x\}\_\{j\}\{y\}\_\{i\})

si x =\textbackslash{}mathop\{ \textbackslash{}mathop\{∑ \}\}
\{x\}\_\{i\}\{e\}\_\{i\} et y =\textbackslash{}mathop\{
\textbackslash{}mathop\{∑ \}\} \{y\}\_\{i\}\{e\}\_\{i\}. Alors φ est
clairement une forme bilinéaire symétrique sur E et la forme quadratique
associée vérifie Φ(x) =
P(\{x\}\_\{1\},\textbackslash{}mathop\{\textbackslash{}mathop\{\ldots{}\}\},\{x\}\_\{n\}).
On obtient donc

Théorème~12.2.5 Soit E un K-espace vectoriel ~de dimension finie n, ℰ
une base de E. L'application qui à une forme quadratique Φ sur E de
matrice Ω = \{(\{ω\}\_\{i,j\})\}\_\{1≤i,j≤n\} dans la base ℰ associe le
polynôme à n variables
\{P\}\_\{Φ\}(\{X\}\_\{1\},\textbackslash{}mathop\{\textbackslash{}mathop\{\ldots{}\}\},\{X\}\_\{n\})
=\{\textbackslash{}mathop\{ \textbackslash{}mathop\{∑ \}\}
\}\_\{i\}\{ω\}\_\{i,i\}\{X\}\_\{i\}\^{}\{2\} +
2\{\textbackslash{}mathop\{\textbackslash{}mathop\{∑ \}\}
\}\_\{i\textless{}j\}\{ω\}\_\{i,j\}\{X\}\_\{i\}\{X\}\_\{j\} est un
isomorphisme d'espaces vectoriels de Q(E) sur l'espace
\{H\}\_\{2\}(\{X\}\_\{1\},\textbackslash{}mathop\{\textbackslash{}mathop\{\ldots{}\}\},\{X\}\_\{n\})
des polynômes homogènes de degré 2 à n variables. Inversement, étant
donné
P(\{X\}\_\{1\},\textbackslash{}mathop\{\textbackslash{}mathop\{\ldots{}\}\},\{X\}\_\{n\})
=\{\textbackslash{}mathop\{ \textbackslash{}mathop\{∑ \}\}
\}\_\{i=1\}\^{}\{n\}\{a\}\_\{i,i\}\{X\}\_\{i\}\^{}\{2\}
+\{\textbackslash{}mathop\{ \textbackslash{}mathop\{∑ \}\}
\}\_\{i\textless{}j\}\{a\}\_\{i,j\}\{X\}\_\{i\}\{X\}\_\{j\} ∈
\{H\}\_\{2\}(\{X\}\_\{1\},\textbackslash{}mathop\{\textbackslash{}mathop\{\ldots{}\}\},\{X\}\_\{n\}),
la forme bilinéaire symétrique associée est donnée par φ(x,y)
=\{\textbackslash{}mathop\{ \textbackslash{}mathop\{∑ \}\}
\}\_\{i\}\{a\}\_\{i,i\}\{x\}\_\{i\}\{y\}\_\{i\}
+\{\textbackslash{}mathop\{ \textbackslash{}mathop\{∑ \}\}
\}\_\{i\textless{}j\}\{ \{a\}\_\{i,j\} \textbackslash{}over 2\}
(\{x\}\_\{i\}\{y\}\_\{j\} + \{x\}\_\{j\}\{y\}\_\{i\}) (règle du
dédoublement des termes).

Remarque~12.2.4 La règle du dédoublement des termes signifie donc que
l'on obtient l'expression de φ(x,y) à partir de l'expression polynomiale
de Φ(x) en rempla\textbackslash{}c\{c\}ant partout les termes carrés
\{x\}\_\{i\}\^{}\{2\} par \{x\}\_\{i\}\{y\}\_\{i\} et les termes
rectangles \{x\}\_\{i\}\{x\}\_\{j\} par \{ 1 \textbackslash{}over 2\}
(\{x\}\_\{i\}\{y\}\_\{j\} + \{x\}\_\{j\}\{y\}\_\{i\}). Le lecteur
vérifiera également sans difficulté que

φ(x,y) =\{ 1 \textbackslash{}over 2\} \{\textbackslash{}mathop\{∑
\}\}\_\{i=1\}\^{}\{n\}\{x\}\_\{ i\}\{ ∂P \textbackslash{}over
∂\{X\}\_\{i\}\}
(\{y\}\_\{1\},\textbackslash{}mathop\{\ldots{}\},\{y\}\_\{n\})

\paragraph{12.2.3 Matrices et déterminants de Gram}

Définition~12.2.2 Soit E un K-espace vectoriel , Φ ∈Q(E), φ la forme
polaire de Φ. Soit
(\{v\}\_\{1\},\textbackslash{}mathop\{\textbackslash{}mathop\{\ldots{}\}\},\{v\}\_\{n\})
une famille finie d'éléments de E. On appelle matrice de Gram de la
famille la matrice
\textbackslash{}mathop\{Gram\}(\{v\}\_\{1\},\textbackslash{}mathop\{\textbackslash{}mathop\{\ldots{}\}\},\{v\}\_\{n\})
= \{(φ(\{v\}\_\{i\},\{v\}\_\{j\}))\}\_\{1≤i,j≤n\} et déterminant de Gram
le scalaire
G(\{v\}\_\{1\},\textbackslash{}mathop\{\textbackslash{}mathop\{\ldots{}\}\},\{v\}\_\{n\})
=\textbackslash{}mathop\{ \textbackslash{}mathrm\{det\}\}
\textbackslash{}mathop\{Gram\}(\{v\}\_\{1\},\textbackslash{}mathop\{\textbackslash{}mathop\{\ldots{}\}\},\{v\}\_\{n\}).

Lemme~12.2.6 Soit V =\textbackslash{}mathop\{
\textbackslash{}mathrm\{Vect\}\}(\{v\}\_\{1\},\textbackslash{}mathop\{\textbackslash{}mathop\{\ldots{}\}\},\{v\}\_\{n\}).
Alors

\textbackslash{}mathop\{\textbackslash{}mathrm\{rg\}\}(\textbackslash{}mathop\{Gram\}(\{v\}\_\{1\},\textbackslash{}mathop\{\textbackslash{}mathop\{\ldots{}\}\},\{v\}\_\{n\}))
=\textbackslash{}mathop\{ dim\} V −\textbackslash{}mathop\{ dim\} (V ∩
\{V \}\^{}\{⊥\})

Démonstration Soit ℰ =
(\{e\}\_\{1\},\textbackslash{}mathop\{\textbackslash{}mathop\{\ldots{}\}\},\{e\}\_\{n\})
la base canonique de \{K\}\^{}\{n\} et u l'application linéaire de
\{K\}\^{}\{n\} dans E définie par u(\{e\}\_\{i\}) = \{v\}\_\{i\}. Alors
V = u(\{K\}\^{}\{n\}). Soit ψ la forme bilinéaire symétrique sur
\{K\}\^{}\{n\} définie par ψ(x,y) = φ(u(x),u(y)). On a donc, d'après le
théorème du rang

n =\textbackslash{}mathop\{ dim\} \{K\}\^{}\{n\}
=\textbackslash{}mathop\{ \textbackslash{}mathrm\{rg\}\}ψ
+\textbackslash{}mathop\{ dim\}
\textbackslash{}mathop\{\textbackslash{}mathrm\{Ker\}\}ψ

Mais \textbackslash{}mathop\{\textbackslash{}mathrm\{Mat\}\} (ψ,ℰ) =
\textbackslash{}left (ψ(\{e\}\_\{i\},\{e\}\_\{j\})\textbackslash{}right
) = \textbackslash{}left
(φ(u(\{e\}\_\{i\})),u(\{e\}\_\{j\}))\textbackslash{}right ) =
\textbackslash{}left (φ(\{v\}\_\{i\},\{v\}\_\{j\})\textbackslash{}right
) =\textbackslash{}mathop\{
Gram\}(\{v\}\_\{1\},\textbackslash{}mathop\{\textbackslash{}mathop\{\ldots{}\}\},\{v\}\_\{n\}),
si bien que \textbackslash{}mathop\{\textbackslash{}mathrm\{rg\}\}ψ
=\textbackslash{}mathop\{
\textbackslash{}mathrm\{rg\}\}\textbackslash{}mathop\{Gram\}(\{v\}\_\{1\},\textbackslash{}mathop\{\textbackslash{}mathop\{\ldots{}\}\},\{v\}\_\{n\}).
D'autre part

\textbackslash{}begin\{eqnarray*\} x
∈\textbackslash{}mathop\{\textbackslash{}mathrm\{Ker\}\}ψ\&
\textbackslash{}mathrel\{⇔\} \& \textbackslash{}mathop\{∀\}y ∈
\{K\}\^{}\{n\}, ψ(x,y) = 0 \%\& \textbackslash{}\textbackslash{} \&
\textbackslash{}mathrel\{⇔\} \& \textbackslash{}mathop\{∀\}y ∈
\{K\}\^{}\{n\},φ(u(x),u(y)) = 0 \%\& \textbackslash{}\textbackslash{} \&
\textbackslash{}mathrel\{⇔\} \& \textbackslash{}mathop\{∀\}v ∈ V =
u(\{K\}\^{}\{n\}), φ(u(x),v) = 0\%\& \textbackslash{}\textbackslash{} \&
\textbackslash{}mathrel\{⇔\} \& u(x) ∈ \{V \}\^{}\{⊥\}∩ V \%\&
\textbackslash{}\textbackslash{} \textbackslash{}end\{eqnarray*\}

On a donc \textbackslash{}mathop\{\textbackslash{}mathrm\{Ker\}\}ψ =
\{u\}\^{}\{−1\}(V ∩ \{V \}\^{}\{⊥\}), soit (puisque V ∩ \{V \}\^{}\{⊥\}⊂
V =\textbackslash{}mathop\{ \textbackslash{}mathrm\{Im\}\}u),

\textbackslash{}begin\{eqnarray*\} \textbackslash{}mathop\{dim\}
\textbackslash{}mathop\{\textbackslash{}mathrm\{Ker\}\}ψ\& =\&
\textbackslash{}mathop\{dim\} V ∩ \{V \}\^{}\{⊥\}
+\textbackslash{}mathop\{ dim\}
\textbackslash{}mathop\{\textbackslash{}mathrm\{Ker\}\}u \%\&
\textbackslash{}\textbackslash{} \& =\& \textbackslash{}mathop\{dim\} V
∩ \{V \}\^{}\{⊥\} + n −\textbackslash{}mathop\{ dim\}
\textbackslash{}mathop\{\textbackslash{}mathrm\{Im\}\}u\%\&
\textbackslash{}\textbackslash{} \& =\& \textbackslash{}mathop\{dim\} V
∩ \{V \}\^{}\{⊥\} + n −\textbackslash{}mathop\{ dim\} V \%\&
\textbackslash{}\textbackslash{} \textbackslash{}end\{eqnarray*\}

D'où finalement

\textbackslash{}begin\{eqnarray*\}
\textbackslash{}mathop\{\textbackslash{}mathrm\{rg\}\}(\textbackslash{}mathop\{Gram\}(\{v\}\_\{1\},\textbackslash{}mathop\{\textbackslash{}mathop\{\ldots{}\}\},\{v\}\_\{n\}))\&
=\& \textbackslash{}mathop\{\textbackslash{}mathrm\{rg\}\}ψ = n
−\textbackslash{}mathop\{ dim\}
\textbackslash{}mathop\{\textbackslash{}mathrm\{Ker\}\}ψ \%\&
\textbackslash{}\textbackslash{} \& =\& \textbackslash{}mathop\{dim\} V
−\textbackslash{}mathop\{ dim\} (V ∩ \{V \}\^{}\{⊥\})\%\&
\textbackslash{}\textbackslash{} \textbackslash{}end\{eqnarray*\}

Comme \textbackslash{}mathop\{dim\} V ≤ n, on a donc
\textbackslash{}mathop\{\textbackslash{}mathrm\{rg\}\}\textbackslash{}mathop\{Gram\}(\{v\}\_\{1\},\textbackslash{}mathop\{\textbackslash{}mathop\{\ldots{}\}\},\{v\}\_\{n\})
= n \textbackslash{}mathrel\{⇔\} \textbackslash{}mathop\{dim\} V = n et
V ∩ \{V \}\^{}\{⊥\} = \textbackslash{}\{0\textbackslash{}\}. On a donc

Proposition~12.2.7 Soit E un K-espace vectoriel , Φ ∈Q(E), φ la forme
polaire de Φ. Soit
(\{v\}\_\{1\},\textbackslash{}mathop\{\textbackslash{}mathop\{\ldots{}\}\},\{v\}\_\{n\})
une famille finie d'éléments de E. Alors on a équivalence de

\begin{itemize}
\itemsep1pt\parskip0pt\parsep0pt
\item
  (i)
  G(\{v\}\_\{1\},\textbackslash{}mathop\{\textbackslash{}mathop\{\ldots{}\}\},\{v\}\_\{n\})\textbackslash{}mathrel\{≠\}0
\item
  (ii) la famille
  (\{v\}\_\{1\},\textbackslash{}mathop\{\textbackslash{}mathop\{\ldots{}\}\},\{v\}\_\{n\})
  est libre et le sous-espace
  \textbackslash{}mathop\{\textbackslash{}mathrm\{Vect\}\}(\{v\}\_\{1\},\textbackslash{}mathop\{\textbackslash{}mathop\{\ldots{}\}\},\{v\}\_\{n\})
  est non isotrope.
\end{itemize}

Corollaire~12.2.8 Soit E un K-espace vectoriel , Φ ∈Q(E) une forme
quadratique sur E qui est définie. Soit
(\{v\}\_\{1\},\textbackslash{}mathop\{\textbackslash{}mathop\{\ldots{}\}\},\{v\}\_\{n\})
une famille finie d'éléments de E. Alors on a équivalence de

\begin{itemize}
\itemsep1pt\parskip0pt\parsep0pt
\item
  (i)
  G(\{v\}\_\{1\},\textbackslash{}mathop\{\textbackslash{}mathop\{\ldots{}\}\},\{v\}\_\{n\})\textbackslash{}mathrel\{≠\}0
\item
  (ii) la famille
  (\{v\}\_\{1\},\textbackslash{}mathop\{\textbackslash{}mathop\{\ldots{}\}\},\{v\}\_\{n\})
  est libre.
\end{itemize}

Remarque~12.2.5 Les déterminants de Gram permettent donc, moyennant la
connaissance d'une forme quadratique définie sur E (s'il en existe), de
tester la liberté d'une famille finie, quel que soit le cardinal de
cette famille et même si l'espace vectoriel E est de dimension infinie.
C'est ainsi que pour une famille
(\{f\}\_\{1\},\textbackslash{}mathop\{\textbackslash{}mathop\{\ldots{}\}\},\{f\}\_\{n\})
de fonctions continues de {[}0,1{]} dans ℝ, on a

(\{f\}\_\{1\},\textbackslash{}mathop\{\textbackslash{}mathop\{\ldots{}\}\},\{f\}\_\{n\})\textbackslash{}text\{
libre \} \textbackslash{}mathrel\{⇔\}
\textbackslash{}mathop\{\textbackslash{}mathrm\{det\}\}
\textbackslash{}left (\{\textbackslash{}mathop\{∫ \}
\}\_\{0\}\^{}\{1\}\{f\}\_\{ i\}\{f\}\_\{j\}\textbackslash{}right
)\textbackslash{}mathrel\{≠\}0

{[}\href{coursse69.html}{next}{]} {[}\href{coursse67.html}{prev}{]}
{[}\href{coursse67.html\#tailcoursse67.html}{prev-tail}{]}
{[}\href{coursse68.html}{front}{]}
{[}\href{coursch13.html\#coursse68.html}{up}{]}

\end{document}
