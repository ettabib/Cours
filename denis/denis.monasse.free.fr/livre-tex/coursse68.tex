\documentclass[]{article}
\usepackage[T1]{fontenc}
\usepackage{lmodern}
\usepackage{amssymb,amsmath}
\usepackage{ifxetex,ifluatex}
\usepackage{fixltx2e} % provides \textsubscript
% use upquote if available, for straight quotes in verbatim environments
\IfFileExists{upquote.sty}{\usepackage{upquote}}{}
\ifnum 0\ifxetex 1\fi\ifluatex 1\fi=0 % if pdftex
  \usepackage[utf8]{inputenc}
\else % if luatex or xelatex
  \ifxetex
    \usepackage{mathspec}
    \usepackage{xltxtra,xunicode}
  \else
    \usepackage{fontspec}
  \fi
  \defaultfontfeatures{Mapping=tex-text,Scale=MatchLowercase}
  \newcommand{\euro}{€}
\fi
% use microtype if available
\IfFileExists{microtype.sty}{\usepackage{microtype}}{}
\ifxetex
  \usepackage[setpagesize=false, % page size defined by xetex
              unicode=false, % unicode breaks when used with xetex
              xetex]{hyperref}
\else
  \usepackage[unicode=true]{hyperref}
\fi
\hypersetup{breaklinks=true,
            bookmarks=true,
            pdfauthor={},
            pdftitle={Formes quadratiques},
            colorlinks=true,
            citecolor=blue,
            urlcolor=blue,
            linkcolor=magenta,
            pdfborder={0 0 0}}
\urlstyle{same}  % don't use monospace font for urls
\setlength{\parindent}{0pt}
\setlength{\parskip}{6pt plus 2pt minus 1pt}
\setlength{\emergencystretch}{3em}  % prevent overfull lines
\setcounter{secnumdepth}{0}
 
/* start css.sty */
.cmr-5{font-size:50%;}
.cmr-7{font-size:70%;}
.cmmi-5{font-size:50%;font-style: italic;}
.cmmi-7{font-size:70%;font-style: italic;}
.cmmi-10{font-style: italic;}
.cmsy-5{font-size:50%;}
.cmsy-7{font-size:70%;}
.cmex-7{font-size:70%;}
.cmex-7x-x-71{font-size:49%;}
.msbm-7{font-size:70%;}
.cmtt-10{font-family: monospace;}
.cmti-10{ font-style: italic;}
.cmbx-10{ font-weight: bold;}
.cmr-17x-x-120{font-size:204%;}
.cmsl-10{font-style: oblique;}
.cmti-7x-x-71{font-size:49%; font-style: italic;}
.cmbxti-10{ font-weight: bold; font-style: italic;}
p.noindent { text-indent: 0em }
td p.noindent { text-indent: 0em; margin-top:0em; }
p.nopar { text-indent: 0em; }
p.indent{ text-indent: 1.5em }
@media print {div.crosslinks {visibility:hidden;}}
a img { border-top: 0; border-left: 0; border-right: 0; }
center { margin-top:1em; margin-bottom:1em; }
td center { margin-top:0em; margin-bottom:0em; }
.Canvas { position:relative; }
li p.indent { text-indent: 0em }
.enumerate1 {list-style-type:decimal;}
.enumerate2 {list-style-type:lower-alpha;}
.enumerate3 {list-style-type:lower-roman;}
.enumerate4 {list-style-type:upper-alpha;}
div.newtheorem { margin-bottom: 2em; margin-top: 2em;}
.obeylines-h,.obeylines-v {white-space: nowrap; }
div.obeylines-v p { margin-top:0; margin-bottom:0; }
.overline{ text-decoration:overline; }
.overline img{ border-top: 1px solid black; }
td.displaylines {text-align:center; white-space:nowrap;}
.centerline {text-align:center;}
.rightline {text-align:right;}
div.verbatim {font-family: monospace; white-space: nowrap; text-align:left; clear:both; }
.fbox {padding-left:3.0pt; padding-right:3.0pt; text-indent:0pt; border:solid black 0.4pt; }
div.fbox {display:table}
div.center div.fbox {text-align:center; clear:both; padding-left:3.0pt; padding-right:3.0pt; text-indent:0pt; border:solid black 0.4pt; }
div.minipage{width:100%;}
div.center, div.center div.center {text-align: center; margin-left:1em; margin-right:1em;}
div.center div {text-align: left;}
div.flushright, div.flushright div.flushright {text-align: right;}
div.flushright div {text-align: left;}
div.flushleft {text-align: left;}
.underline{ text-decoration:underline; }
.underline img{ border-bottom: 1px solid black; margin-bottom:1pt; }
.framebox-c, .framebox-l, .framebox-r { padding-left:3.0pt; padding-right:3.0pt; text-indent:0pt; border:solid black 0.4pt; }
.framebox-c {text-align:center;}
.framebox-l {text-align:left;}
.framebox-r {text-align:right;}
span.thank-mark{ vertical-align: super }
span.footnote-mark sup.textsuperscript, span.footnote-mark a sup.textsuperscript{ font-size:80%; }
div.tabular, div.center div.tabular {text-align: center; margin-top:0.5em; margin-bottom:0.5em; }
table.tabular td p{margin-top:0em;}
table.tabular {margin-left: auto; margin-right: auto;}
div.td00{ margin-left:0pt; margin-right:0pt; }
div.td01{ margin-left:0pt; margin-right:5pt; }
div.td10{ margin-left:5pt; margin-right:0pt; }
div.td11{ margin-left:5pt; margin-right:5pt; }
table[rules] {border-left:solid black 0.4pt; border-right:solid black 0.4pt; }
td.td00{ padding-left:0pt; padding-right:0pt; }
td.td01{ padding-left:0pt; padding-right:5pt; }
td.td10{ padding-left:5pt; padding-right:0pt; }
td.td11{ padding-left:5pt; padding-right:5pt; }
table[rules] {border-left:solid black 0.4pt; border-right:solid black 0.4pt; }
.hline hr, .cline hr{ height : 1px; margin:0px; }
.tabbing-right {text-align:right;}
span.TEX {letter-spacing: -0.125em; }
span.TEX span.E{ position:relative;top:0.5ex;left:-0.0417em;}
a span.TEX span.E {text-decoration: none; }
span.LATEX span.A{ position:relative; top:-0.5ex; left:-0.4em; font-size:85%;}
span.LATEX span.TEX{ position:relative; left: -0.4em; }
div.float img, div.float .caption {text-align:center;}
div.figure img, div.figure .caption {text-align:center;}
.marginpar {width:20%; float:right; text-align:left; margin-left:auto; margin-top:0.5em; font-size:85%; text-decoration:underline;}
.marginpar p{margin-top:0.4em; margin-bottom:0.4em;}
.equation td{text-align:center; vertical-align:middle; }
td.eq-no{ width:5%; }
table.equation { width:100%; } 
div.math-display, div.par-math-display{text-align:center;}
math .texttt { font-family: monospace; }
math .textit { font-style: italic; }
math .textsl { font-style: oblique; }
math .textsf { font-family: sans-serif; }
math .textbf { font-weight: bold; }
.partToc a, .partToc, .likepartToc a, .likepartToc {line-height: 200%; font-weight:bold; font-size:110%;}
.chapterToc a, .chapterToc, .likechapterToc a, .likechapterToc, .appendixToc a, .appendixToc {line-height: 200%; font-weight:bold;}
.index-item, .index-subitem, .index-subsubitem {display:block}
.caption td.id{font-weight: bold; white-space: nowrap; }
table.caption {text-align:center;}
h1.partHead{text-align: center}
p.bibitem { text-indent: -2em; margin-left: 2em; margin-top:0.6em; margin-bottom:0.6em; }
p.bibitem-p { text-indent: 0em; margin-left: 2em; margin-top:0.6em; margin-bottom:0.6em; }
.paragraphHead, .likeparagraphHead { margin-top:2em; font-weight: bold;}
.subparagraphHead, .likesubparagraphHead { font-weight: bold;}
.quote {margin-bottom:0.25em; margin-top:0.25em; margin-left:1em; margin-right:1em; text-align:\\jmathmathustify;}
.verse{white-space:nowrap; margin-left:2em}
div.maketitle {text-align:center;}
h2.titleHead{text-align:center;}
div.maketitle{ margin-bottom: 2em; }
div.author, div.date {text-align:center;}
div.thanks{text-align:left; margin-left:10%; font-size:85%; font-style:italic; }
div.author{white-space: nowrap;}
.quotation {margin-bottom:0.25em; margin-top:0.25em; margin-left:1em; }
h1.partHead{text-align: center}
.sectionToc, .likesectionToc {margin-left:2em;}
.subsectionToc, .likesubsectionToc {margin-left:4em;}
.subsubsectionToc, .likesubsubsectionToc {margin-left:6em;}
.frenchb-nbsp{font-size:75%;}
.frenchb-thinspace{font-size:75%;}
.figure img.graphics {margin-left:10%;}
/* end css.sty */

\title{Formes quadratiques}
\author{}
\date{}

\begin{document}
\maketitle

\textbf{Warning: 
requires JavaScript to process the mathematics on this page.\\ If your
browser supports JavaScript, be sure it is enabled.}

\begin{center}\rule{3in}{0.4pt}\end{center}

{[}
{[}
{[}{]}
{[}

\subsubsection{12.2 Formes quadratiques}

\paragraph{12.2.1 Notion de forme quadratique}

Soit E un K-espace vectoriel et \phi une forme bilinéaire symétrique sur E.
Soit \Phi l'application de E dans K qui à x associe \Phi(x) = \phi(x,x).

Proposition~12.2.1 On a les identités suivantes

\begin{itemize}
\itemsep1pt\parskip0pt\parsep0pt
\item
  (i) \Phi(\lambda~x) = \lambda~^2\Phi(x)
\item
  (ii) \Phi(x + y) = \Phi(x) + 2\phi(x,y) + \Phi(y) (identité de polarisation)
\item
  (iii) \Phi(x + y) + \Phi(x - y) = 2(\Phi(x) + \Phi(y)) (identité de la médiane)
\end{itemize}

Démonstration (i) \Phi(\lambda~x) = \phi(\lambda~x,\lambda~x) = \lambda~^2\phi(x,x) =
\lambda~^2\Phi(x)

(ii) \Phi(x + y) = \phi(x + y,x + y) = \Phi(x) + \phi(x,y) + \phi(y,x) + \Phi(y) = \Phi(x) +
2\phi(x,y) + \Phi(y)

(iii) changeant y en - y dans l'identité précédente, on a aussi \Phi(x - y)
= \Phi(x) - 2\phi(x,y) + \Phi(y), et en additionnant les deux on trouve \Phi(x + y)
+ \Phi(x - y) = 2(\Phi(x) + \Phi(y)).

Remarque~12.2.1 Si
\mathrmcarK\mathrel\neq~~2,
l'identité (ii) montre que l'application \phi\mapsto~\Phi
est in\\jmathmathective de S_2(E) dans K^E (espace vectoriel
des applications de E dans K) puisque la connaissance de \Phi permet de
retrouver \phi par

\phi(x,y) = 1 \over 2 (\Phi(x + y) - \Phi(x) - \Phi(y))

Ceci nous amène à poser

Définition~12.2.1 Soit K un corps de caractéristique différente de 2 et
E un K-espace vectoriel . On appelle forme quadratique sur E toute
application \Phi : E \rightarrow~ K vérifiant les deux propriétés

\begin{itemize}
\itemsep1pt\parskip0pt\parsep0pt
\item
  (i) \forall~\lambda~ \in K, \\forall~~x \in
  E, \Phi(\lambda~x) = \lambda~^2\Phi(x)
\item
  (ii) l'application \phi : E \times E \rightarrow~ K, \phi(x,y) = 1 \over
  2 (\Phi(x + y) - \Phi(x) - \Phi(y)) est une forme bilinéaire (évidemment
  symétrique) sur E.
\end{itemize}

Dans ce cas, on a \forall~~x \in E, \Phi(x) = \phi(x,x)~; \phi
est appelée la forme polaire de \Phi.

Démonstration On a \phi(x,x) = 1 \over 2 (\Phi(2x) - 2\Phi(x))
= 1 \over 2 (4\Phi(x) - 2\Phi(x)) = \Phi(x) en utilisant la
propriété (i).

Exemple~12.2.1 Sur K^n, \Phi(x) =\
\sum  _i=1^nx_i^2~
est une forme quadratique dont la forme polaire associée est \phi(x,y)
= \\sum ~
_i=1^nx_iy_i. Si K = \mathbb{R}~ ou K = \mathbb{C}, et si E
désigne l'espace vectoriel des fonctions continues de {[}a,b{]} dans K,
\Phi(f) =\int  _a^bf(t)^2~
dt est une forme quadratique dont la forme polaire est \phi(f,g)
=\int  _a^b~f(t)g(t) dt.

Proposition~12.2.2 L'ensemble Q(E) des formes quadratiques sur E est un
sous-espace vectoriel de K^E~; l'application
\phi\mapsto~\Phi est un isomorphisme d'espaces vectoriels
de S_2(E) sur Q(E).

Remarque~12.2.2 Par la suite on confondra toutes les notions relatives à
\phi et à \Phi~: orthogonalité, matrice, non dégénérescence, isotropie~; en
particulier on posera
\mathrmKer~\Phi
= \mathrmKer~\phi =
\x \in
E∣\forall~~y \in E, \phi(x,y) =
0\. On remarquera qu'en général,
\mathrmKer\Phi\mathrel\neq~~\x
\in E∣\Phi(x) = 0\.

Théorème~12.2.3 (Pythagore). Soit E un K-espace vectoriel ~et \Phi \inQ(E), \phi
la forme polaire de \Phi. Alors

x \bot_\phiy \Leftrightarrow \Phi(x + y) = \Phi(x) + \Phi(y)

Démonstration C'est une conséquence évidente de l'identité de
polarisation.

\paragraph{12.2.2 Formes quadratiques en dimension finie}

Soit E un K-espace vectoriel ~de dimension finie, \Phi \inQ(E) de forme
polaire \phi.

Théorème~12.2.4 Soit \mathcal{E} une base de E. Alors
\mathrmMat~ (\phi,\mathcal{E}) est
l'unique matrice \Omega \in M_K(n) qui est symétrique et qui vérifie

\forall~x \in E, \Phi(x) = ^t~X\OmegaX

Démonstration Il est clair que \Omega =\
\mathrmMat (\Phi,\mathcal{E}) est symétrique et vérifie \Phi(x) =
\phi(x,x) = ^tX\OmegaX. Inversement, soit \Omega une matrice symétrique
vérifiant cette propriété. On a alors

\begin{align*} \phi(x,y)& =& 1 \over
2 (\Phi(x + y) - \Phi(x) - \Phi(y)) \%& \\ &
=& 1 \over 2 (^t(X + Y )\Omega(X + Y )
-^tX\OmegaX -^tY \OmegaY \%&
\\ & =& 1 \over 2
(^tX\OmegaY + ^tY \OmegaX) \%&
\\ \end{align*}

Mais, une matrice 1 \times 1 étant forcément symétrique ^tY \OmegaX =
^t(^tY \OmegaX) = ^tX^t\OmegaY =
^tX\OmegaY puisque \Omega est symétrique. On a donc \phi(x,y) =
^tX\OmegaY ce qui montre que \Omega =\
\mathrmMat (\phi,\mathcal{E}).

Remarque~12.2.3 On prendra garde à la condition de symétrie de \Omega. Il est
en effet clair que l'on peut remplacer, dans la condition \Phi(x) =
^tX\OmegaX, la matrice \Omega par une matrice \Omega' = \Omega + A où A est
antisymétrique, puisque dans ce cas ^tXAX = 0. On aura alors
\Phi(x) = ^tX\Omega'X bien que \Omega' ne soit pas la matrice de \Phi dans la
base \mathcal{E}.

Posons \Omega = \mathrmMat~ (\phi,\mathcal{E})
= (\omega_i,\\jmathmath)_1\leqi,\\jmathmath\leqn. On a alors

\phi(x,y) = \\sum
_i,\\jmathmath\omega_i,\\jmathmathx_iy_\\jmathmath =
\\sum
_i\omega_i,ix_iy_i +
\\sum
_i\textless{}\\jmathmath\omega_i,\\jmathmath(x_iy_\\jmathmath +
x_\\jmathmathy_i)

en tenant compte de \omega_i,\\jmathmath = \omega_\\jmathmath,i. On a donc

\Phi(x) = \phi(x,x) = \\sum
_i\omega_i,ix_i^2 +
2\\sum
_i\textless{}\\jmathmath\omega_i,\\jmathmathx_ix_\\jmathmath =
P_\Phi(x_1,\ldots,x_n~)

où P_\Phi est le polynôme homogène de degré 2 à n variables
P_\Phi(X_1,\\ldots,X_n~)
= \\sum ~
_i\omega_i,iX_i^2 +\
\sum ~
_i\textless{}\\jmathmath\omega_i,\\jmathmathX_iX_\\jmathmath.
Inversement, soit P un polynôme homogène de degré 2 à n variables,
P(X_1,\\ldots,X_n~)
= \\sum ~
_i=1^na_i,iX_i^2
+ \\sum ~
_i\textless{}\\jmathmatha_i,\\jmathmathX_iX_\\jmathmath. Définissons
\phi sur E par

\phi(x,y) = \\sum
_ia_i,ix_iy_i +
\sum _i\textless{}\\jmathmath a_i,\\jmathmath~
\over 2 (x_iy_\\jmathmath +
x_\\jmathmathy_i)

si x = \\sum ~
x_ie_i et y =\
\sum  y_ie_i~. Alors \phi est
clairement une forme bilinéaire symétrique sur E et la forme quadratique
associée vérifie \Phi(x) =
P(x_1,\\ldots,x_n~).
On obtient donc

Théorème~12.2.5 Soit E un K-espace vectoriel ~de dimension finie n, \mathcal{E}
une base de E. L'application qui à une forme quadratique \Phi sur E de
matrice \Omega = (\omega_i,\\jmathmath)_1\leqi,\\jmathmath\leqn dans la base \mathcal{E} associe le
polynôme à n variables
P_\Phi(X_1,\\ldots,X_n~)
= \\sum ~
_i\omega_i,iX_i^2 +
2\\sum ~
_i\textless{}\\jmathmath\omega_i,\\jmathmathX_iX_\\jmathmath est un
isomorphisme d'espaces vectoriels de Q(E) sur l'espace
H_2(X_1,\\ldots,X_n~)
des polynômes homogènes de degré 2 à n variables. Inversement, étant
donné
P(X_1,\\ldots,X_n~)
= \\sum ~
_i=1^na_i,iX_i^2
+ \\sum ~
_i\textless{}\\jmathmatha_i,\\jmathmathX_iX_\\jmathmath \in
H_2(X_1,\\ldots,X_n~),
la forme bilinéaire symétrique associée est donnée par \phi(x,y)
= \\sum ~
_ia_i,ix_iy_i
+ \\sum ~
_i\textless{}\\jmathmath a_i,\\jmathmath \over 2
(x_iy_\\jmathmath + x_\\jmathmathy_i) (règle du
dédoublement des termes).

Remarque~12.2.4 La règle du dédoublement des termes signifie donc que
l'on obtient l'expression de \phi(x,y) à partir de l'expression polynomiale
de \Phi(x) en rempla\ccant partout les termes carrés
x_i^2 par x_iy_i et les termes
rectangles x_ix_\\jmathmath par  1 \over 2
(x_iy_\\jmathmath + x_\\jmathmathy_i). Le lecteur
vérifiera également sans difficulté que

\phi(x,y) = 1 \over 2 \\sum
_i=1^nx_ i \partial~P \over
\partial~X_i
(y_1,\ldots,y_n~)

\paragraph{12.2.3 Matrices et déterminants de Gram}

Définition~12.2.2 Soit E un K-espace vectoriel , \Phi \inQ(E), \phi la forme
polaire de \Phi. Soit
(v_1,\\ldots,v_n~)
une famille finie d'éléments de E. On appelle matrice de Gram de la
famille la matrice
Gram(v_1,\\\ldots,v_n~)
= (\phi(v_i,v_\\jmathmath))_1\leqi,\\jmathmath\leqn et déterminant de Gram
le scalaire
G(v_1,\\ldots,v_n~)
= \mathrm{det}~
Gram(v_1,\\\ldots,v_n~).

Lemme~12.2.6 Soit V =\
\mathrmVect(v_1,\\ldots,v_n~).
Alors

\mathrmrg(\Gram(v_1,\\\ldots,v_n~))
= dim V -\ dim~ (V \bigcap
V ^\bot)

Démonstration Soit \mathcal{E} =
(e_1,\\ldots,e_n~)
la base canonique de K^n et u l'application linéaire de
K^n dans E définie par u(e_i) = v_i. Alors
V = u(K^n). Soit \psi la forme bilinéaire symétrique sur
K^n définie par \psi(x,y) = \phi(u(x),u(y)). On a donc, d'après le
théorème du rang

n = dim K^n~
= \mathrmrg~\psi
+ dim~
\mathrmKer~\psi

Mais \mathrmMat~ (\psi,\mathcal{E}) =
\left (\psi(e_i,e_\\jmathmath)\right
) = \left
(\phi(u(e_i)),u(e_\\jmathmath))\right ) =
\left (\phi(v_i,v_\\jmathmath)\right
) =\
Gram(v_1,\\ldots,v_n~),
si bien que \mathrmrg~\psi
=\
\mathrmrgGram(v_1,\\\ldots,v_n~).
D'autre part

\begin{align*} x
\in\mathrmKer~\psi&
\Leftrightarrow & \forall~~y \in
K^n, \psi(x,y) = 0 \%& \\ &
\Leftrightarrow & \forall~~y \in
K^n,\phi(u(x),u(y)) = 0 \%& \\ &
\Leftrightarrow & \forall~~v \in V =
u(K^n), \phi(u(x),v) = 0\%& \\ &
\Leftrightarrow & u(x) \in V ^\bot\bigcap V \%&
\\ \end{align*}

On a donc \mathrmKer~\psi =
u^-1(V \bigcap V ^\bot), soit (puisque V \bigcap V ^\bot\subset~
V = \mathrmIm~u),

\begin{align*} dim~
\mathrmKer~\psi& =&
dim V \bigcap V ^\bot~
+ dim~
\mathrmKer~u \%&
\\ & =& dim~ V
\bigcap V ^\bot + n - dim~
\mathrmIm~u\%&
\\ & =& dim~ V
\bigcap V ^\bot + n - dim~ V \%&
\\ \end{align*}

D'où finalement

\begin{align*}
\mathrmrg(\Gram(v_1,\\\ldots,v_n~))&
=& \mathrmrg~\psi = n
- dim~
\mathrmKer~\psi \%&
\\ & =& dim~ V
- dim (V \bigcap V ^\bot~)\%&
\\ \end{align*}

Comme dim~ V \leq n, on a donc
\mathrmrg\Gram(v_1,\\\ldots,v_n~)
= n \Leftrightarrow dim~ V = n et
V \bigcap V ^\bot = \0\. On a donc

Proposition~12.2.7 Soit E un K-espace vectoriel , \Phi \inQ(E), \phi la forme
polaire de \Phi. Soit
(v_1,\\ldots,v_n~)
une famille finie d'éléments de E. Alors on a équivalence de

\begin{itemize}
\itemsep1pt\parskip0pt\parsep0pt
\item
  (i)
  G(v_1,\\ldots,v_n)\mathrel\neq~~0
\item
  (ii) la famille
  (v_1,\\ldots,v_n~)
  est libre et le sous-espace
  \mathrmVect(v_1,\\\ldots,v_n~)
  est non isotrope.
\end{itemize}

Corollaire~12.2.8 Soit E un K-espace vectoriel , \Phi \inQ(E) une forme
quadratique sur E qui est définie. Soit
(v_1,\\ldots,v_n~)
une famille finie d'éléments de E. Alors on a équivalence de

\begin{itemize}
\itemsep1pt\parskip0pt\parsep0pt
\item
  (i)
  G(v_1,\\ldots,v_n)\mathrel\neq~~0
\item
  (ii) la famille
  (v_1,\\ldots,v_n~)
  est libre.
\end{itemize}

Remarque~12.2.5 Les déterminants de Gram permettent donc, moyennant la
connaissance d'une forme quadratique définie sur E (s'il en existe), de
tester la liberté d'une famille finie, quel que soit le cardinal de
cette famille et même si l'espace vectoriel E est de dimension infinie.
C'est ainsi que pour une famille
(f_1,\\ldots,f_n~)
de fonctions continues de {[}0,1{]} dans \mathbb{R}~, on a

(f_1,\\ldots,f_n~)\text
libre  \Leftrightarrow
\mathrm{det}~
\left (\int ~
_0^1f_ if_\\jmathmath\right
)\neq~0

{[}
{[}
{[}
{[}

\end{document}
