\documentclass[]{article}
\usepackage[T1]{fontenc}
\usepackage{lmodern}
\usepackage{amssymb,amsmath}
\usepackage{ifxetex,ifluatex}
\usepackage{fixltx2e} % provides \textsubscript
% use upquote if available, for straight quotes in verbatim environments
\IfFileExists{upquote.sty}{\usepackage{upquote}}{}
\ifnum 0\ifxetex 1\fi\ifluatex 1\fi=0 % if pdftex
  \usepackage[utf8]{inputenc}
\else % if luatex or xelatex
  \ifxetex
    \usepackage{mathspec}
    \usepackage{xltxtra,xunicode}
  \else
    \usepackage{fontspec}
  \fi
  \defaultfontfeatures{Mapping=tex-text,Scale=MatchLowercase}
  \newcommand{\euro}{€}
\fi
% use microtype if available
\IfFileExists{microtype.sty}{\usepackage{microtype}}{}
\ifxetex
  \usepackage[setpagesize=false, % page size defined by xetex
              unicode=false, % unicode breaks when used with xetex
              xetex]{hyperref}
\else
  \usepackage[unicode=true]{hyperref}
\fi
\hypersetup{breaklinks=true,
            bookmarks=true,
            pdfauthor={},
            pdftitle={Cercles, sph`eres, triangle},
            colorlinks=true,
            citecolor=blue,
            urlcolor=blue,
            linkcolor=magenta,
            pdfborder={0 0 0}}
\urlstyle{same}  % don't use monospace font for urls
\setlength{\parindent}{0pt}
\setlength{\parskip}{6pt plus 2pt minus 1pt}
\setlength{\emergencystretch}{3em}  % prevent overfull lines
\setcounter{secnumdepth}{0}
 
/* start css.sty */
.cmr-5{font-size:50%;}
.cmr-7{font-size:70%;}
.cmmi-5{font-size:50%;font-style: italic;}
.cmmi-7{font-size:70%;font-style: italic;}
.cmmi-10{font-style: italic;}
.cmsy-5{font-size:50%;}
.cmsy-7{font-size:70%;}
.cmex-7{font-size:70%;}
.cmex-7x-x-71{font-size:49%;}
.msbm-7{font-size:70%;}
.cmtt-10{font-family: monospace;}
.cmti-10{ font-style: italic;}
.cmbx-10{ font-weight: bold;}
.cmr-17x-x-120{font-size:204%;}
.cmsl-10{font-style: oblique;}
.cmti-7x-x-71{font-size:49%; font-style: italic;}
.cmbxti-10{ font-weight: bold; font-style: italic;}
p.noindent { text-indent: 0em }
td p.noindent { text-indent: 0em; margin-top:0em; }
p.nopar { text-indent: 0em; }
p.indent{ text-indent: 1.5em }
@media print {div.crosslinks {visibility:hidden;}}
a img { border-top: 0; border-left: 0; border-right: 0; }
center { margin-top:1em; margin-bottom:1em; }
td center { margin-top:0em; margin-bottom:0em; }
.Canvas { position:relative; }
li p.indent { text-indent: 0em }
.enumerate1 {list-style-type:decimal;}
.enumerate2 {list-style-type:lower-alpha;}
.enumerate3 {list-style-type:lower-roman;}
.enumerate4 {list-style-type:upper-alpha;}
div.newtheorem { margin-bottom: 2em; margin-top: 2em;}
.obeylines-h,.obeylines-v {white-space: nowrap; }
div.obeylines-v p { margin-top:0; margin-bottom:0; }
.overline{ text-decoration:overline; }
.overline img{ border-top: 1px solid black; }
td.displaylines {text-align:center; white-space:nowrap;}
.centerline {text-align:center;}
.rightline {text-align:right;}
div.verbatim {font-family: monospace; white-space: nowrap; text-align:left; clear:both; }
.fbox {padding-left:3.0pt; padding-right:3.0pt; text-indent:0pt; border:solid black 0.4pt; }
div.fbox {display:table}
div.center div.fbox {text-align:center; clear:both; padding-left:3.0pt; padding-right:3.0pt; text-indent:0pt; border:solid black 0.4pt; }
div.minipage{width:100%;}
div.center, div.center div.center {text-align: center; margin-left:1em; margin-right:1em;}
div.center div {text-align: left;}
div.flushright, div.flushright div.flushright {text-align: right;}
div.flushright div {text-align: left;}
div.flushleft {text-align: left;}
.underline{ text-decoration:underline; }
.underline img{ border-bottom: 1px solid black; margin-bottom:1pt; }
.framebox-c, .framebox-l, .framebox-r { padding-left:3.0pt; padding-right:3.0pt; text-indent:0pt; border:solid black 0.4pt; }
.framebox-c {text-align:center;}
.framebox-l {text-align:left;}
.framebox-r {text-align:right;}
span.thank-mark{ vertical-align: super }
span.footnote-mark sup.textsuperscript, span.footnote-mark a sup.textsuperscript{ font-size:80%; }
div.tabular, div.center div.tabular {text-align: center; margin-top:0.5em; margin-bottom:0.5em; }
table.tabular td p{margin-top:0em;}
table.tabular {margin-left: auto; margin-right: auto;}
div.td00{ margin-left:0pt; margin-right:0pt; }
div.td01{ margin-left:0pt; margin-right:5pt; }
div.td10{ margin-left:5pt; margin-right:0pt; }
div.td11{ margin-left:5pt; margin-right:5pt; }
table[rules] {border-left:solid black 0.4pt; border-right:solid black 0.4pt; }
td.td00{ padding-left:0pt; padding-right:0pt; }
td.td01{ padding-left:0pt; padding-right:5pt; }
td.td10{ padding-left:5pt; padding-right:0pt; }
td.td11{ padding-left:5pt; padding-right:5pt; }
table[rules] {border-left:solid black 0.4pt; border-right:solid black 0.4pt; }
.hline hr, .cline hr{ height : 1px; margin:0px; }
.tabbing-right {text-align:right;}
span.TEX {letter-spacing: -0.125em; }
span.TEX span.E{ position:relative;top:0.5ex;left:-0.0417em;}
a span.TEX span.E {text-decoration: none; }
span.LATEX span.A{ position:relative; top:-0.5ex; left:-0.4em; font-size:85%;}
span.LATEX span.TEX{ position:relative; left: -0.4em; }
div.float img, div.float .caption {text-align:center;}
div.figure img, div.figure .caption {text-align:center;}
.marginpar {width:20%; float:right; text-align:left; margin-left:auto; margin-top:0.5em; font-size:85%; text-decoration:underline;}
.marginpar p{margin-top:0.4em; margin-bottom:0.4em;}
.equation td{text-align:center; vertical-align:middle; }
td.eq-no{ width:5%; }
table.equation { width:100%; } 
div.math-display, div.par-math-display{text-align:center;}
math .texttt { font-family: monospace; }
math .textit { font-style: italic; }
math .textsl { font-style: oblique; }
math .textsf { font-family: sans-serif; }
math .textbf { font-weight: bold; }
.partToc a, .partToc, .likepartToc a, .likepartToc {line-height: 200%; font-weight:bold; font-size:110%;}
.chapterToc a, .chapterToc, .likechapterToc a, .likechapterToc, .appendixToc a, .appendixToc {line-height: 200%; font-weight:bold;}
.index-item, .index-subitem, .index-subsubitem {display:block}
.caption td.id{font-weight: bold; white-space: nowrap; }
table.caption {text-align:center;}
h1.partHead{text-align: center}
p.bibitem { text-indent: -2em; margin-left: 2em; margin-top:0.6em; margin-bottom:0.6em; }
p.bibitem-p { text-indent: 0em; margin-left: 2em; margin-top:0.6em; margin-bottom:0.6em; }
.paragraphHead, .likeparagraphHead { margin-top:2em; font-weight: bold;}
.subparagraphHead, .likesubparagraphHead { font-weight: bold;}
.quote {margin-bottom:0.25em; margin-top:0.25em; margin-left:1em; margin-right:1em; text-align:justify;}
.verse{white-space:nowrap; margin-left:2em}
div.maketitle {text-align:center;}
h2.titleHead{text-align:center;}
div.maketitle{ margin-bottom: 2em; }
div.author, div.date {text-align:center;}
div.thanks{text-align:left; margin-left:10%; font-size:85%; font-style:italic; }
div.author{white-space: nowrap;}
.quotation {margin-bottom:0.25em; margin-top:0.25em; margin-left:1em; }
h1.partHead{text-align: center}
.sectionToc, .likesectionToc {margin-left:2em;}
.subsectionToc, .likesubsectionToc {margin-left:4em;}
.subsubsectionToc, .likesubsubsectionToc {margin-left:6em;}
.frenchb-nbsp{font-size:75%;}
.frenchb-thinspace{font-size:75%;}
.figure img.graphics {margin-left:10%;}
/* end css.sty */

\title{Cercles, sph`eres, triangle}
\author{}
\date{}

\begin{document}
\maketitle

\textbf{Warning: \href{http://www.math.union.edu/locate/jsMath}{jsMath}
requires JavaScript to process the mathematics on this page.\\ If your
browser supports JavaScript, be sure it is enabled.}

\begin{center}\rule{3in}{0.4pt}\end{center}

{[}\href{coursse94.html}{prev}{]}
{[}\href{coursse94.html\#tailcoursse94.html}{prev-tail}{]}
{[}\hyperref[tailcoursse95.html]{tail}{]}
{[}\href{coursch18.html\#coursse95.html}{up}{]}

\subsubsection{17.4 Cercles, sphères, triangle}

\paragraph{17.4.1 Généralités sur les sphères}

Définition~17.4.1 Soit E un espace euclidien, a ∈ E, r \textgreater{} 0.
On appelle sphère de centre a de rayon r l'ensemble S(a,r) =
\textbackslash{}\{x ∈ E\textbackslash{}mathrel\{∣\}d(a,x) =
r\textbackslash{}\}.

Equation de sphères

Soit
(O,\textbackslash{}vec\{\{e\}\}\_\{1\},\textbackslash{}mathop\{\textbackslash{}mathop\{\ldots{}\}\},\textbackslash{}vec\{\{e\}\}\_\{n\})
un repère orthonormé de E,
\{a\}\_\{1\},\textbackslash{}mathop\{\textbackslash{}mathop\{\ldots{}\}\},\{a\}\_\{n\}
les coordonnées de a et
\{x\}\_\{1\},\textbackslash{}mathop\{\textbackslash{}mathop\{\ldots{}\}\},\{x\}\_\{n\}
les coordonnées de x. Alors

\textbackslash{}begin\{eqnarray*\} d(a,x) = r\&
\textbackslash{}mathrel\{⇔\} \&
\textbackslash{}\textbar{}\textbackslash{}overrightarrow\{\{ax\}\textbackslash{}\textbar{}\}\^{}\{2\}
= \{r\}\^{}\{2\} \%\& \textbackslash{}\textbackslash{} \&
\textbackslash{}mathrel\{⇔\} \& \{(\{x\}\_\{1\} −
\{a\}\_\{1\})\}\^{}\{2\} +
\textbackslash{}mathop\{\textbackslash{}mathop\{\ldots{}\}\} +
\{(\{x\}\_\{ n\} − \{a\}\_\{n\})\}\^{}\{2\} = \{r\}\^{}\{2\} \%\&
\textbackslash{}\textbackslash{} \& \textbackslash{}mathrel\{⇔\} \&
\{x\}\_\{1\}\^{}\{2\} +
\textbackslash{}mathop\{\textbackslash{}mathop\{\ldots{}\}\} + \{x\}\_\{
n\}\^{}\{2\} − 2\{a\}\_\{ 1\}\{x\}\_\{1\}
−\textbackslash{}mathop\{\textbackslash{}mathop\{\ldots{}\}\} −
2\{a\}\_\{n\}\{x\}\_\{n\} + c = 0\%\& \textbackslash{}\textbackslash{}
\textbackslash{}end\{eqnarray*\}

avec c = \{a\}\_\{1\}\^{}\{2\} +
\textbackslash{}mathop\{\textbackslash{}mathop\{\ldots{}\}\} +
\{a\}\_\{n\}\^{}\{2\} − \{r\}\^{}\{2\}.

Inversement, si on se donne
\{a\}\_\{1\},\textbackslash{}mathop\{\textbackslash{}mathop\{\ldots{}\}\},\{a\}\_\{n\},c
∈ ℝ, on a

\textbackslash{}begin\{eqnarray*\}\{ x\}\_\{1\}\^{}\{2\} +
\textbackslash{}mathop\{\textbackslash{}mathop\{\ldots{}\}\} + \{x\}\_\{
n\}\^{}\{2\} − 2\{a\}\_\{ 1\}\{x\}\_\{1\}
−\textbackslash{}mathop\{\textbackslash{}mathop\{\ldots{}\}\} −
2\{a\}\_\{n\}\{x\}\_\{n\} + c = 0\&\&\%\&
\textbackslash{}\textbackslash{} \& \textbackslash{}mathrel\{⇔\} \&
\textbackslash{}\textbar{}\textbackslash{}overrightarrow\{\{ax\}\textbackslash{}\textbar{}\}\^{}\{2\}
= \{a\}\_\{ 1\}\^{}\{2\} +
\textbackslash{}mathop\{\textbackslash{}mathop\{\ldots{}\}\} + \{a\}\_\{
n\}\^{}\{2\} − c\%\& \textbackslash{}\textbackslash{}
\textbackslash{}end\{eqnarray*\}

On en déduit que l'ensemble en question est soit l'ensemble vide si
\{a\}\_\{1\}\^{}\{2\} +
\textbackslash{}mathop\{\textbackslash{}mathop\{\ldots{}\}\} +
\{a\}\_\{n\}\^{}\{2\} − c \textless{} 0, soit le singleton
\textbackslash{}\{a\textbackslash{}\} avec a = O +
\{a\}\_\{1\}\textbackslash{}vec\{\{e\}\}\_\{1\} +
\textbackslash{}mathop\{\textbackslash{}mathop\{\ldots{}\}\} +
\{a\}\_\{n\}\textbackslash{}vec\{\{e\}\}\_\{n\} si \{a\}\_\{1\}\^{}\{2\}
+ \textbackslash{}mathop\{\textbackslash{}mathop\{\ldots{}\}\} +
\{a\}\_\{n\}\^{}\{2\} − c = 0, soit la sphère de centre a et de rayon
\textbackslash{}sqrt\{\{a\}\_\{1 \}\^{}\{2 \} +
\textbackslash{}mathop\{\textbackslash{}mathop\{\ldots{}\}\} +
\{a\}\_\{n \}\^{}\{2 \} − c\} si \{a\}\_\{1\}\^{}\{2\} +
\textbackslash{}mathop\{\textbackslash{}mathop\{\ldots{}\}\} +
\{a\}\_\{n\}\^{}\{2\} − c \textgreater{} 0.

Intersection d'une sphère et d'un sous-espace affine

Soit S(a,r) une sphère de E et F un sous-espace affine de E. Appelons b
la projection orthogonale de a sur F. Pour x ∈ F, on a
d\{(a,x)\}\^{}\{2\} = d\{(a,b)\}\^{}\{2\} + d\{(b,x)\}\^{}\{2\} =
d\{(a,F)\}\^{}\{2\} + d\{(b,x)\}\^{}\{2\} si bien que

x ∈ S(a,r) ∩ F \textbackslash{}mathrel\{⇔\} d\{(b,x)\}\^{}\{2\} =
\{r\}\^{}\{2\} − d\{(a,F)\}\^{}\{2\}

On en déduit que

\begin{itemize}
\itemsep1pt\parskip0pt\parsep0pt
\item
  (i) si d(a,F) \textgreater{} r, alors S(a,r) ∩ F = ∅
\item
  (ii) si d(a,F) = r, alors S(a,r) ∩ F =
  \textbackslash{}\{b\textbackslash{}\} où b désigne la projection
  orthogonale de a sur F~; on dit dans ce cas que F est tangent à la
  sphère
\item
  (iii) si d(a,F) \textless{} r, alors S(a,r) ∩ F est la sphère de F de
  centre b, projection orthogonale de a sur F, et de rayon
  \textbackslash{}sqrt\{\{r\}\^{}\{2 \} − d\{(a, F)\}\^{}\{2\}\}.
\end{itemize}

Intersection de deux sphères

Considérons deux sphères S(a,r) et S(a',r') de centres distincts. On a
alors

\textbackslash{}begin\{eqnarray*\} x ∈ S(a,r) ∩ S(a',r')\&\& \%\&
\textbackslash{}\textbackslash{} \& \textbackslash{}mathrel\{⇔\} \&
\textbackslash{}left \textbackslash{}\{\textbackslash{}array\{
\textbackslash{}\textbar{}\textbackslash{}overrightarrow\{\{ax\}\textbackslash{}\textbar{}\}\^{}\{2\}\&
= \{r\}\^{}\{2\} \textbackslash{}cr
\textbackslash{}\textbar{}\textbackslash{}overrightarrow\{\{a'x\}\textbackslash{}\textbar{}\}\^{}\{2\}\&
= \{r'\}\^{}\{2\} \} \textbackslash{}right .\%\&
\textbackslash{}\textbackslash{} \& \textbackslash{}mathrel\{⇔\} \&
\textbackslash{}left \textbackslash{}\{\textbackslash{}array\{
\textbackslash{}\textbar{}\textbackslash{}overrightarrow\{\{ax\}\textbackslash{}\textbar{}\}\^{}\{2\}
\& = \{r\}\^{}\{2\} \textbackslash{}cr
\textbackslash{}\textbar{}\textbackslash{}overrightarrow\{\{a'x\}\textbackslash{}\textbar{}\}\^{}\{2\}
−\textbackslash{}\textbar{}\textbackslash{}overrightarrow\{\{
ax\}\textbackslash{}\textbar{}\}\^{}\{2\}\& = \{r'\}\^{}\{2\} −
\{r\}\^{}\{2\} \} \textbackslash{}right .\%\&
\textbackslash{}\textbackslash{} \& \textbackslash{}mathrel\{⇔\} \&
\textbackslash{}left \textbackslash{}\{\textbackslash{}array\{
\textbackslash{}\textbar{}\textbackslash{}overrightarrow\{\{ax\}\textbackslash{}\textbar{}\}\^{}\{2\}
\& = \{r\}\^{}\{2\} \textbackslash{}cr
(\textbackslash{}overrightarrow\{a'x\} +\textbackslash{}overrightarrow\{
ax\}\textbackslash{}mathrel\{∣\}\textbackslash{}overrightarrow\{a'x\}
−\textbackslash{}overrightarrow\{ ax\})\& = \{r'\}\^{}\{2\} −
\{r\}\^{}\{2\} \} \textbackslash{}right .\%\&
\textbackslash{}\textbackslash{} \& \textbackslash{}mathrel\{⇔\} \&
\textbackslash{}left \textbackslash{}\{\textbackslash{}array\{
\textbackslash{}\textbar{}\textbackslash{}overrightarrow\{\{ax\}\textbackslash{}\textbar{}\}\^{}\{2\}
\& = \{r\}\^{}\{2\} \textbackslash{}cr
2(\textbackslash{}overrightarrow\{bx\}\textbackslash{}mathrel\{∣\}\textbackslash{}overrightarrow\{aa'\})\&
= \{r'\}\^{}\{2\} − \{r\}\^{}\{2\} \} \textbackslash{}right
.\%\&\textbackslash{}\textbackslash{} \textbackslash{}end\{eqnarray*\}

si b désigne le milieu de aa'. Soit donc c le point de la droite aa' tel
que
2(\textbackslash{}overrightarrow\{bc\}\textbackslash{}mathrel\{∣\}\textbackslash{}overrightarrow\{aa'\})
= \{r'\}\^{}\{2\} − \{r\}\^{}\{2\}, soit encore
\textbackslash{}overline\{bc\}.\textbackslash{}overline\{aa'\} =\{
\{r'\}\^{}\{2\}−\{r\}\^{}\{2\} \textbackslash{}over 2\} . On obtient

\textbackslash{}begin\{eqnarray*\} x ∈ S(a,r) ∩ S(a',r')\&
\textbackslash{}mathrel\{⇔\} \& \textbackslash{}left
\textbackslash{}\{\textbackslash{}array\{
\textbackslash{}\textbar{}\textbackslash{}overrightarrow\{\{ax\}\textbackslash{}\textbar{}\}\^{}\{2\}
\& = \{r\}\^{}\{2\} \textbackslash{}cr
2(\textbackslash{}overrightarrow\{bx\}\textbackslash{}mathrel\{∣\}\textbackslash{}overrightarrow\{aa'\})\&
=
2(\textbackslash{}overrightarrow\{bc\}\textbackslash{}mathrel\{∣\}\textbackslash{}overrightarrow\{aa'\})
\} \textbackslash{}right .\%\& \textbackslash{}\textbackslash{} \&
\textbackslash{}mathrel\{⇔\} \& \textbackslash{}left
\textbackslash{}\{\textbackslash{}array\{
\textbackslash{}\textbar{}\textbackslash{}overrightarrow\{\{ax\}\textbackslash{}\textbar{}\}\^{}\{2\}
\& = \{r\}\^{}\{2\} \textbackslash{}cr
2(\textbackslash{}overrightarrow\{cx\}\textbackslash{}mathrel\{∣\}\textbackslash{}overrightarrow\{aa'\})\&
= 0 \} \textbackslash{}right .\%\&\textbackslash{}\textbackslash{}
\textbackslash{}end\{eqnarray*\}

Or cette dernière équation est celle de l'hyperplan H orthogonal à aa'
passant par c. On obtient donc que S(a,r) ∩ S(a',r') = S(a,r) ∩ H.
L'intersection est donc soit ∅, soit un singleton, soit une sphère de
l'hyperplan H suivant que d(a,H) \textgreater{} r, d(a,H) = r ou d(a,H)
\textless{} r. Mais comme la droite ac est aussi la droite aa' qui est
orthogonale à H, la distance de a à H n'est autre que la distance de a à
c. On a

\{ \{r'\}\^{}\{2\} − \{r\}\^{}\{2\} \textbackslash{}over 2\} =
\textbackslash{}overline\{bc\}.\textbackslash{}overline\{aa'\} =
(\textbackslash{}overline\{ac\}
−\textbackslash{}overline\{ab\}).\textbackslash{}overline\{aa'\} =
(\textbackslash{}overline\{ac\} −\{ 1 \textbackslash{}over 2\}
\textbackslash{}overline\{aa'\}).\textbackslash{}overline\{aa'\}

d'où l'on déduit que
\textbackslash{}overline\{ac\}.\textbackslash{}overline\{aa'\} =\{
\{r\}\^{}\{2\}−\{r'\}\^{}\{2\}+d\{(a,a')\}\^{}\{2\} \textbackslash{}over
2\} , soit encore d\{(a,c)\}\^{}\{2\} =\{
\{(\{r\}\^{}\{2\}−\{r'\}\^{}\{2\}+d\{(a,a')\}\^{}\{2\})\}\^{}\{2\}
\textbackslash{}over 4d\{(a,a')\}\^{}\{2\}\} . On a donc

\textbackslash{}begin\{eqnarray*\} d\{(a,H)\}\^{}\{2\} − \{r\}\^{}\{2\}
= d\{(a,c)\}\^{}\{2\} − \{r\}\^{}\{2\}\&\& \%\&
\textbackslash{}\textbackslash{} \& =\&\{ \{(\{r\}\^{}\{2\} −
\{r'\}\^{}\{2\} + d\{(a,a')\}\^{}\{2\})\}\^{}\{2\} −
4\{r\}\^{}\{2\}d\{(a,a')\}\^{}\{2\} \textbackslash{}over
4d\{(a,a')\}\^{}\{2\}\} \%\& \textbackslash{}\textbackslash{} \& =\&\{ 1
\textbackslash{}over 4d\{(a,a')\}\^{}\{2\}\} (\{r\}\^{}\{2\} −
\{r'\}\^{}\{2\} + d\{(a,a')\}\^{}\{2\} + 2rd(a,a')) \%\&
\textbackslash{}\textbackslash{} \& \& \textbackslash{}quad
(\{r\}\^{}\{2\} − \{r'\}\^{}\{2\} + d\{(a,a')\}\^{}\{2\} − 2rd(a,a'))
\%\& \textbackslash{}\textbackslash{} \& =\&\{ 1 \textbackslash{}over
4d\{(a,a')\}\^{}\{2\}\} (\{(r + d(a,a'))\}\^{}\{2\} −
\{r'\}\^{}\{2\})(\{(r − d(a,a'))\}\^{}\{2\} − \{r'\}\^{}\{2\})\%\&
\textbackslash{}\textbackslash{} \& =\&\{ 1 \textbackslash{}over
4d\{(a,a')\}\^{}\{2\}\} (r + d(a,a') + r')(r + d(a,a') − r') \%\&
\textbackslash{}\textbackslash{} \& \& \textbackslash{}quad (r − d(a,a')
+ r')(r − d(a,a') − r') \%\& \textbackslash{}\textbackslash{}
\textbackslash{}end\{eqnarray*\}

qui est du signe de δ = (r + d(a,a') − r')(r − d(a,a') + r')(r − d(a,a')
− r').

\begin{itemize}
\itemsep1pt\parskip0pt\parsep0pt
\item
  (i) si \textbar{}r − r'\textbar{} \textless{} d(a,a') \textless{} r +
  r', on a δ \textless{} 0 et donc S(a,r) ∩ S(a',r') est une sphère de
  centre c de l'hyperplan H
\item
  (ii) si \textbar{}r − r'\textbar{} = d(a,a') ou r + r' = d(a,a'), on a
  δ = 0 et donc S(a,r) ∩ S(a',r') =
  \textbackslash{}\{c\textbackslash{}\}~; les deux sphères sont
  tangentes (intérieurement si \textbar{}r − r'\textbar{} = d(a,a') et
  extérieurement si r + r' = d(a,a'))
\item
  (iii) si d(a,a') \textless{} \textbar{}r − r'\textbar{} ou d(a,a')
  \textgreater{} r + r', alors δ \textgreater{} 0 et S(a,r) ∩ S(a',r') =
  ∅.
\end{itemize}

\paragraph{17.4.2 Cercles et angles}

Théorème~17.4.1 Dans le plan euclidien orienté, soit Γ = S(ω,r) le
cercle de centre ω et de rayon r et soit trois points distincts m,a et b
de Γ. Alors
\textbackslash{}widehat\{(\textbackslash{}overrightarrow\{ωa\},\textbackslash{}overrightarrow\{ωb\})\}
= 2\textbackslash{}widehat\{(ma,mb)\} (où
\textbackslash{}widehat\{(\textbackslash{}overrightarrow\{ωa\},\textbackslash{}overrightarrow\{ωb\})\}
désigne l'angle des vecteurs \textbackslash{}overrightarrow\{ωa\} et
\textbackslash{}overrightarrow\{ωb\} et
\textbackslash{}widehat\{(ma,mb)\} l'angle des droites ma et mb).

Démonstration Ecrivons
\textbackslash{}widehat\{(\textbackslash{}overrightarrow\{ωa\},\textbackslash{}overrightarrow\{mω\})\}
=\textbackslash{}widehat\{
(\textbackslash{}overrightarrow\{ωa\},\textbackslash{}overrightarrow\{ma\})\}
+\textbackslash{}widehat\{
(\textbackslash{}overrightarrow\{ma\},\textbackslash{}overrightarrow\{mω\})\}.
Soit s la symétrie par rapport à la médiatrice du couple (a,m). On a s :
\textbackslash{}left
\textbackslash{}\{\textbackslash{}matrix\{\textbackslash{},a\textbackslash{}mathrel\{↦\}m
\textbackslash{}cr m\textbackslash{}mathrel\{↦\}a \textbackslash{}cr
ω\textbackslash{}mathrel\{↦\}ω\}\textbackslash{}right . puisque le
centre du cercle appartient à la médiatrice de (a,m). De plus s change
un angle de vecteurs en son opposé, d'où

\textbackslash{}begin\{eqnarray*\}
\textbackslash{}widehat\{(\textbackslash{}overrightarrow\{ωa\},\textbackslash{}overrightarrow\{ma\})\}\&
=\&
−\textbackslash{}widehat\{(\textbackslash{}overrightarrow\{s(ω)s(a)\},\textbackslash{}overrightarrow\{s(m)s(a)\})\}
\%\& \textbackslash{}\textbackslash{} \& =\&
−\textbackslash{}widehat\{(\textbackslash{}overrightarrow\{ωm\},\textbackslash{}overrightarrow\{am\})\}
\%\& \textbackslash{}\textbackslash{} \& =\&
−\textbackslash{}widehat\{(\textbackslash{}overrightarrow\{ωm\},\textbackslash{}overrightarrow\{mω\})\}
−\textbackslash{}widehat\{
(\textbackslash{}overrightarrow\{mω\},\textbackslash{}overrightarrow\{ma\})\}
−\textbackslash{}widehat\{
(\textbackslash{}overrightarrow\{ma\},\textbackslash{}overrightarrow\{am\})\}\%\&
\textbackslash{}\textbackslash{} \& =\& −π −\textbackslash{}widehat\{
(\textbackslash{}overrightarrow\{mω\},\textbackslash{}overrightarrow\{ma\})\}
− π =
−\textbackslash{}widehat\{(\textbackslash{}overrightarrow\{mω\},\textbackslash{}overrightarrow\{ma\})\}
\%\& \textbackslash{}\textbackslash{} \& =\&
\textbackslash{}widehat\{(\textbackslash{}overrightarrow\{ma\},\textbackslash{}overrightarrow\{mω\})\}
\%\& \textbackslash{}\textbackslash{} \textbackslash{}end\{eqnarray*\}

On en déduit donc que
\textbackslash{}widehat\{(\textbackslash{}overrightarrow\{ωa\},\textbackslash{}overrightarrow\{mω\})\}
=
2\textbackslash{}widehat\{(\textbackslash{}overrightarrow\{ma\},\textbackslash{}overrightarrow\{mω\})\}
= 2\textbackslash{}widehat\{(ma,mω)\} (puisque la mesure de
\textbackslash{}widehat\{(ma,mω)\} est un élément de ℝ∕πℤ, la mesure de
2\textbackslash{}widehat\{(ma,mω)\} est un élément de ℝ∕2πℤ, donc la
mesure d'un angle de vecteurs).

On a de même
\textbackslash{}widehat\{(\textbackslash{}overrightarrow\{ωb\},\textbackslash{}overrightarrow\{mω\})\}
= 2\textbackslash{}widehat\{(mb,mω)\}, puis par soustraction
\textbackslash{}widehat\{(\textbackslash{}overrightarrow\{ωa\},\textbackslash{}overrightarrow\{ωb\})\}
= 2\textbackslash{}widehat\{(ma,mb)\}.

Remarque~17.4.1 Si m vient se confondre avec a, la droite ma vient se
confondre avec la tangente \{D\}\_\{a\} à Γ en a~; le lecteur montrera
sans difficulté que le raisonnement précédent est encore valide dans ce
cas limite et que l'on a donc
\textbackslash{}widehat\{(\textbackslash{}overrightarrow\{ωa\},\textbackslash{}overrightarrow\{ωb\})\}
= 2\textbackslash{}widehat\{(\{D\}\_\{a\},ab)\}

Corollaire~17.4.2 Dans le plan euclidien orienté, soit a,b,c et d quatre
points distincts. Alors ces quatre points sont cocycliques ou alignés si
et seulement si~\textbackslash{}widehat\{(ca,cb)\}
=\textbackslash{}widehat\{ (da,db)\}.

Démonstration La condition est bien entendu nécessaire, car si les
quatre points sont alignés, on a \textbackslash{}widehat\{(ca,cb)\}
=\textbackslash{}widehat\{ (da,db)\} = 0 et s'ils appartiennent à un
même cercle Γ = S(ω,r), on a 2\textbackslash{}widehat\{(ca,cb)\} =
2\textbackslash{}widehat\{(da,db)\} =\textbackslash{}widehat\{
(\textbackslash{}overrightarrow\{ωa\},\textbackslash{}overrightarrow\{ωb\})\}~;
mais l'application x\textbackslash{}mathrel\{↦\}2x, ℝ∕πℤ → ℝ∕2πℤ est
injective (car si 2α = 2β + 2kπ, on a α = β + kπ) et donc on a
\textbackslash{}widehat\{(ca,cb)\} =\textbackslash{}widehat\{ (da,db)\}.

Inversement, supposons que α =\textbackslash{}widehat\{ (ca,cb)\}
=\textbackslash{}widehat\{ (da,db)\}. Si α = 0, il est clair que a,b,c
et d sont alignés. Supposons donc α\textbackslash{}mathrel\{≠\}0. Alors
a,b et c ne sont pas alignés et il existe un unique cercle Γ contenant
a,b et c (voir le paragraphe suivant). Soit d' le point d'intersection
de la droite ad avec Γ différent de a. Comme a,b,c et d' sont sur Γ on a
\textbackslash{}widehat\{(ca,cb)\} =\textbackslash{}widehat\{
(d'a,d'b)\} =\textbackslash{}widehat\{ (da,d'b)\} puisque la droite da
est confondue avec la droite d'a. On a donc
\textbackslash{}widehat\{(da,d'b)\} =\textbackslash{}widehat\{
(da,db)\}, soit encore \textbackslash{}widehat\{(db,d'b)\} = 0. Ceci
signifie que d,d' et b sont alignés, de même que d,d' et a. Comme a,b et
d ne sont pas alignés, ceci nécessite que d = d' soit d ∈ Γ. Donc a,b,c
et d sont cocycliques.

Corollaire~17.4.3 Soit α un angle de droites non nul, a et b deux points
distincts du plan euclidien orienté E. Alors l'ensemble des points m
tels que \textbackslash{}widehat\{(ma,mb)\} = α est un cercle passant
par les points a et b, privé de a et b.

Démonstration Soit \{D\}\_\{a\} la droite passant par a telle que
\textbackslash{}widehat\{(\{D\}\_\{a\},ab)\} = α et soit Γ le cercle
tangent à \{D\}\_\{a\} passant par b. Le centre ω de ce cercle est le
point d'intersection de la perpendiculaire à \{D\}\_\{a\} passant par a
avec la médiatrice de (a,b). Pour tout point c de ce cercle, on a
(d'après la remarque ci dessus) 2\textbackslash{}widehat\{(ca,cb)\}
=\textbackslash{}widehat\{
(\textbackslash{}overrightarrow\{ωa\},\textbackslash{}overrightarrow\{ωb\})\}
= 2\textbackslash{}widehat\{(\{D\}\_\{a\},ab)\} = 2α, soit de nouveau
\textbackslash{}widehat\{(ca,cb)\} = α. Donc \textbackslash{}\{m ∈
E\textbackslash{}mathrel\{∣\}\textbackslash{}widehat\{(ma,mb)\} =
α\textbackslash{}\} ⊂ Γ. Soit c ∈ Γ
∖\textbackslash{}\{a,b\textbackslash{}\}~; pour tout point m tel que
\textbackslash{}widehat\{(ma,mb)\} = α, on a
\textbackslash{}widehat\{(ma,mb)\} =\textbackslash{}widehat\{ (ca,cb)\},
donc a,b,c et m sont cocycliques, ce qui montre que m ∈ Γ. Donc
\textbackslash{}\{m ∈
E\textbackslash{}mathrel\{∣\}\textbackslash{}widehat\{(ma,mb)\} =
α\textbackslash{}\} = Γ ∖\textbackslash{}\{a,b\textbackslash{}\}.

\paragraph{17.4.3 Eléments de géométrie du triangle}

Soit E un plan affine euclidien orienté, A,B et C trois points non
alignés de E qui forment le triangle ABC. Nous utiliserons les notations
suivantes

α =\textbackslash{}widehat\{
(\textbackslash{}overrightarrow\{AB\},\textbackslash{}overrightarrow\{AC\})\},β
=\textbackslash{}widehat\{
(\textbackslash{}overrightarrow\{BC\},\textbackslash{}overrightarrow\{BA\})\},γ
=\textbackslash{}widehat\{
(\textbackslash{}overrightarrow\{CA\},\textbackslash{}overrightarrow\{CB\})\}

a = BC,b = CA,c = AB

Proposition~17.4.4

\begin{itemize}
\itemsep1pt\parskip0pt\parsep0pt
\item
  (i) α + β + γ = π
\item
  (ii) \textbackslash{}mathop\{cos\} α =\{
  \{b\}\^{}\{2\}+\{c\}\^{}\{2\}−\{a\}\^{}\{2\} \textbackslash{}over
  2bc\} ,\textbackslash{}mathop\{\textbackslash{}mathop\{\ldots{}\}\}
\end{itemize}

Démonstration (i) On a

\textbackslash{}begin\{eqnarray*\} α + β + γ\& =\&
\textbackslash{}widehat\{(\textbackslash{}overrightarrow\{AB\},\textbackslash{}overrightarrow\{AC\})\}
+\textbackslash{}widehat\{
(\textbackslash{}overrightarrow\{BC\},\textbackslash{}overrightarrow\{BA\})\}
+\textbackslash{}widehat\{
(\textbackslash{}overrightarrow\{CA\},\textbackslash{}overrightarrow\{CB\})\}
\%\& \textbackslash{}\textbackslash{} \& =\&
(\textbackslash{}widehat\{(\textbackslash{}overrightarrow\{AB\},\textbackslash{}overrightarrow\{CA\})\}
+ π) + (π +\textbackslash{}widehat\{
(\textbackslash{}overrightarrow\{CB\},\textbackslash{}overrightarrow\{BA\})\})
+\textbackslash{}widehat\{
(\textbackslash{}overrightarrow\{CA\},\textbackslash{}overrightarrow\{CB\})\}\%\&
\textbackslash{}\textbackslash{} \& =\& 2π +\textbackslash{}widehat\{
(\textbackslash{}overrightarrow\{AB\},\textbackslash{}overrightarrow\{BA\})\}
= 3π = π \%\& \textbackslash{}\textbackslash{}
\textbackslash{}end\{eqnarray*\}

(ii) On a

\textbackslash{}begin\{eqnarray*\}\{ a\}\^{}\{2\}\& =\&
\textbackslash{}\textbar{}\textbackslash{}overrightarrow\{\{BC\}\textbackslash{}\textbar{}\}\^{}\{2\}
=\textbackslash{}\textbar{}\textbackslash{}overrightarrow\{ AC\}
−\textbackslash{}overrightarrow\{\{
AB\}\textbackslash{}\textbar{}\}\^{}\{2\} \%\&
\textbackslash{}\textbackslash{} \& =\&
\textbackslash{}\textbar{}\textbackslash{}overrightarrow\{\{AC\}\textbackslash{}\textbar{}\}\^{}\{2\}
+\textbackslash{}\textbar{}\textbackslash{}overrightarrow\{\{
AB\}\textbackslash{}\textbar{}\}\^{}\{2\} −
2(\textbackslash{}overrightarrow\{AC\}\textbackslash{}mathrel\{∣\}\textbackslash{}overrightarrow\{AB\})
= \{b\}\^{}\{2\} + \{c\}\^{}\{2\} − 2bc\textbackslash{}mathop\{cos\}
α\%\& \textbackslash{}\textbackslash{} \textbackslash{}end\{eqnarray*\}

d'où \textbackslash{}mathop\{cos\} α =\{
\{b\}\^{}\{2\}+\{c\}\^{}\{2\}−\{a\}\^{}\{2\} \textbackslash{}over 2bc\}
et les deux autres formules analogues.

Définition~17.4.2 On appelle médianes du triangle ABC les droites
joignant un sommet au milieu du côté opposé. On appelle hauteurs du
triangle ABC les droites passant par un sommet et orthogonales au coté
opposé. On appelle médiatrices du triangle ABC les trois médiatrices des
couples de sommets du triangle. On appelle bissectrices intérieures du
triangle les droites bissectrices des couples de vecteurs
(\textbackslash{}overrightarrow\{AB\},\textbackslash{}overrightarrow\{AC\}),
(\textbackslash{}overrightarrow\{BC\},\textbackslash{}overrightarrow\{BA\})
et
(\textbackslash{}overrightarrow\{CA\},\textbackslash{}overrightarrow\{CB\}).

Théorème~17.4.5 (i) Les trois médianes du triangle sont concourantes en
l'isobarycentre des trois points A,B et C (le centre de gravité du
triangle) (ii) Les trois médiatrices du triangle sont concourantes en le
centre de l'unique cercle circonscrit au triangle ABC (c'est-à-dire
passant par les points A,B et C). (iii) Les trois bissectrices
intérieures du triangle sont concourantes en le centre de l'unique
cercle inscrit dans le triangle. (iv) Les trois hauteurs du triangle
sont concourantes en un point appelé l'orthocentre du triangle.

Démonstration (i) Le théorème d'associativité des barycentres montre que
l'isobarycentre G du triangle ABC est aussi le barycentre de A affecté
du coefficient 1 et du milieu A' de (B,C) affecté du coefficient 2~;
donc G appartient à la médiane AA' et à chacune des deux autres
médianes.

(ii) Soit O le point d'intersection de la médiatrice de (A,B) avec la
médiatrice de (A,C). On a donc d(O,A) = d(O,B) et d(O,A) = d(O,C). On en
déduit que d(O,B) = d(O,C) et donc O est également sur la médiatrice de
(B,C).

(iii) Soit Ω le point d'intersection de la bissectrice de
(\textbackslash{}overrightarrow\{AB\},\textbackslash{}overrightarrow\{AC\})
avec la bissectrice de
(\textbackslash{}overrightarrow\{BC\},\textbackslash{}overrightarrow\{BA\}).
On a donc d(Ω,AB) = d(Ω,AC) et d(Ω,BC) = d(Ω,BA). On en déduit que
d(Ω,CB) = d(Ω,CA) et donc Ω est sur une des deux bissectrices des
droites CB et CA. Comme Ω est visiblement à l'intérieur du triangle, il
est également sur la bissectrice de
(\textbackslash{}overrightarrow\{CA\},\textbackslash{}overrightarrow\{CB\}).

(iv) Soit A'B'C' le triangle défini par~: la droite B'C' passe par A et
est parallèle à BC, la droite A'C' passe par B et est parallèle à AC, la
droite A'B' passe par C et est parallèle à AB. Le quadrilatère AB'CB est
visiblement un parallélogramme (cotés deux à deux parallèles) donc AB' =
BC. De même le quadrilatère AC'BC est un parallélogramme, donc AC' = BC.
On en déduit que A est le milieu de (B',C'). Mais la hauteur issue de A
est orthogonale à BC donc B'C'. Il s'agit donc de la médiatrice de B'C'.
Les hauteurs du triangle ABC sont les médiatrices du triangle A'B'C',
elles sont donc concourantes.

Théorème~17.4.6 Soit ABC un triangle, α =\textbackslash{}widehat\{
(\textbackslash{}overrightarrow\{AB\},\textbackslash{}overrightarrow\{AC\})\},
β =\textbackslash{}widehat\{
(\textbackslash{}overrightarrow\{BC\},\textbackslash{}overrightarrow\{BA\})\},
γ =\textbackslash{}widehat\{
(\textbackslash{}overrightarrow\{CA\},\textbackslash{}overrightarrow\{CB\})\},
a = BC, b = CA, c = AB, R le rayon du cercle circonscrit au triangle, r
le rayon du cercle inscrit dans le triangle, S l'aire du triangle,
\{h\}\_\{a\} = d(A,BC) la longueur de la hauteur issue de A. On a les
formules suivantes

\textbackslash{}begin\{eqnarray*\} \&\{ a \textbackslash{}over
\textbackslash{}mathop\{sin\} α\} =\{ b \textbackslash{}over
\textbackslash{}mathop\{sin\} β\} =\{ c \textbackslash{}over
\textbackslash{}mathop\{sin\} γ\} = 2R \& (1)\%\&
\textbackslash{}\textbackslash{} \& S =\{ 1 \textbackslash{}over 2\}
a\{h\}\_\{a\} =\{ 1 \textbackslash{}over 2\}
bc\textbackslash{}mathop\{sin\} α =\{ 1 \textbackslash{}over 2\} (a + b
+ c)r\& (2)\%\& \textbackslash{}\textbackslash{}
\textbackslash{}end\{eqnarray*\}

Démonstration Soit O le centre du cercle circonscrit au triangle. Le
triangle OBC est isocèle en O d'angle au sommet
2\textbackslash{}widehat\{(ab,ac)\} = 2α et de côté R. On en déduit que
a = BC = 2R\textbackslash{}mathop\{sin\} \{ 2α \textbackslash{}over 2\}
= 2R\textbackslash{}mathop\{sin\} α. On a donc \{ a \textbackslash{}over
\textbackslash{}mathop\{sin\} α\} = 2R et de même pour les deux autres
sommets, d'où la formule (1).

On sait que l'aire du triangle est égale à la moitié de l'aire du
rectangle correspondant, donc S =\{ 1 \textbackslash{}over 2\}
a\{h\}\_\{a\}.

D'autre part l'aire du triangle est égale à la moitié de l'aire d'un
parallélogramme construit sur ce triangle, soit

S =\{ 1 \textbackslash{}over 2\}
\textbar{}{[}\textbackslash{}overrightarrow\{AB\},\textbackslash{}overrightarrow\{AC\}{]}\textbar{}
=\{ 1 \textbackslash{}over 2\}
\textbackslash{}\textbar{}\textbackslash{}overrightarrow\{AB\}\textbackslash{}\textbar{}
\textbackslash{}\textbar{}\textbackslash{}overrightarrow\{AC\}\textbackslash{}\textbar{}\textbackslash{}mathop\{sin\}
\textbackslash{}widehat\{(\textbackslash{}overrightarrow\{AB\},\textbackslash{}overrightarrow\{AC\})\}
=\{ 1 \textbackslash{}over 2\} cb\textbackslash{}mathop\{sin\} α

Soit Ω le centre du cercle inscrit dans le triangle. L'aire du triangle
ABC est la somme des aires des triangles ΩAB, ΩBC et ΩCA. Or l'aire du
triangle ΩBC est égale à la moitié du produit de la longueur de la base
BC par la distance de Ω à BC qui vaut justement r. Donc S =\{ 1
\textbackslash{}over 2\} ar +\{ 1 \textbackslash{}over 2\} br +\{ 1
\textbackslash{}over 2\} cr.

Remarque~17.4.2 En combinant toutes ces formules, il n'est guère
difficile de calculer tous les éléments remarquables d'un triangle.

{[}\href{coursse94.html}{prev}{]}
{[}\href{coursse94.html\#tailcoursse94.html}{prev-tail}{]}
{[}\href{coursse95.html}{front}{]}
{[}\href{coursch18.html\#coursse95.html}{up}{]}

\end{document}
