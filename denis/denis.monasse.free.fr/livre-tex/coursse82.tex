\documentclass[]{article}
\usepackage[T1]{fontenc}
\usepackage{lmodern}
\usepackage{amssymb,amsmath}
\usepackage{ifxetex,ifluatex}
\usepackage{fixltx2e} % provides \textsubscript
% use upquote if available, for straight quotes in verbatim environments
\IfFileExists{upquote.sty}{\usepackage{upquote}}{}
\ifnum 0\ifxetex 1\fi\ifluatex 1\fi=0 % if pdftex
  \usepackage[utf8]{inputenc}
\else % if luatex or xelatex
  \ifxetex
    \usepackage{mathspec}
    \usepackage{xltxtra,xunicode}
  \else
    \usepackage{fontspec}
  \fi
  \defaultfontfeatures{Mapping=tex-text,Scale=MatchLowercase}
  \newcommand{\euro}{€}
\fi
% use microtype if available
\IfFileExists{microtype.sty}{\usepackage{microtype}}{}
\usepackage{graphicx}
% Redefine \includegraphics so that, unless explicit options are
% given, the image width will not exceed the width of the page.
% Images get their normal width if they fit onto the page, but
% are scaled down if they would overflow the margins.
\makeatletter
\def\ScaleIfNeeded{%
  \ifdim\Gin@nat@width>\linewidth
    \linewidth
  \else
    \Gin@nat@width
  \fi
}
\makeatother
\let\Oldincludegraphics\includegraphics
{%
 \catcode`\@=11\relax%
 \gdef\includegraphics{\@ifnextchar[{\Oldincludegraphics}{\Oldincludegraphics[width=\ScaleIfNeeded]}}%
}%
\ifxetex
  \usepackage[setpagesize=false, % page size defined by xetex
              unicode=false, % unicode breaks when used with xetex
              xetex]{hyperref}
\else
  \usepackage[unicode=true]{hyperref}
\fi
\hypersetup{breaklinks=true,
            bookmarks=true,
            pdfauthor={},
            pdftitle={Derivees partielles},
            colorlinks=true,
            citecolor=blue,
            urlcolor=blue,
            linkcolor=magenta,
            pdfborder={0 0 0}}
\urlstyle{same}  % don't use monospace font for urls
\setlength{\parindent}{0pt}
\setlength{\parskip}{6pt plus 2pt minus 1pt}
\setlength{\emergencystretch}{3em}  % prevent overfull lines
\setcounter{secnumdepth}{0}
 
/* start css.sty */
.cmr-5{font-size:50%;}
.cmr-7{font-size:70%;}
.cmmi-5{font-size:50%;font-style: italic;}
.cmmi-7{font-size:70%;font-style: italic;}
.cmmi-10{font-style: italic;}
.cmsy-5{font-size:50%;}
.cmsy-7{font-size:70%;}
.cmex-7{font-size:70%;}
.cmex-7x-x-71{font-size:49%;}
.msbm-7{font-size:70%;}
.cmtt-10{font-family: monospace;}
.cmti-10{ font-style: italic;}
.cmbx-10{ font-weight: bold;}
.cmr-17x-x-120{font-size:204%;}
.cmsl-10{font-style: oblique;}
.cmti-7x-x-71{font-size:49%; font-style: italic;}
.cmbxti-10{ font-weight: bold; font-style: italic;}
p.noindent { text-indent: 0em }
td p.noindent { text-indent: 0em; margin-top:0em; }
p.nopar { text-indent: 0em; }
p.indent{ text-indent: 1.5em }
@media print {div.crosslinks {visibility:hidden;}}
a img { border-top: 0; border-left: 0; border-right: 0; }
center { margin-top:1em; margin-bottom:1em; }
td center { margin-top:0em; margin-bottom:0em; }
.Canvas { position:relative; }
li p.indent { text-indent: 0em }
.enumerate1 {list-style-type:decimal;}
.enumerate2 {list-style-type:lower-alpha;}
.enumerate3 {list-style-type:lower-roman;}
.enumerate4 {list-style-type:upper-alpha;}
div.newtheorem { margin-bottom: 2em; margin-top: 2em;}
.obeylines-h,.obeylines-v {white-space: nowrap; }
div.obeylines-v p { margin-top:0; margin-bottom:0; }
.overline{ text-decoration:overline; }
.overline img{ border-top: 1px solid black; }
td.displaylines {text-align:center; white-space:nowrap;}
.centerline {text-align:center;}
.rightline {text-align:right;}
div.verbatim {font-family: monospace; white-space: nowrap; text-align:left; clear:both; }
.fbox {padding-left:3.0pt; padding-right:3.0pt; text-indent:0pt; border:solid black 0.4pt; }
div.fbox {display:table}
div.center div.fbox {text-align:center; clear:both; padding-left:3.0pt; padding-right:3.0pt; text-indent:0pt; border:solid black 0.4pt; }
div.minipage{width:100%;}
div.center, div.center div.center {text-align: center; margin-left:1em; margin-right:1em;}
div.center div {text-align: left;}
div.flushright, div.flushright div.flushright {text-align: right;}
div.flushright div {text-align: left;}
div.flushleft {text-align: left;}
.underline{ text-decoration:underline; }
.underline img{ border-bottom: 1px solid black; margin-bottom:1pt; }
.framebox-c, .framebox-l, .framebox-r { padding-left:3.0pt; padding-right:3.0pt; text-indent:0pt; border:solid black 0.4pt; }
.framebox-c {text-align:center;}
.framebox-l {text-align:left;}
.framebox-r {text-align:right;}
span.thank-mark{ vertical-align: super }
span.footnote-mark sup.textsuperscript, span.footnote-mark a sup.textsuperscript{ font-size:80%; }
div.tabular, div.center div.tabular {text-align: center; margin-top:0.5em; margin-bottom:0.5em; }
table.tabular td p{margin-top:0em;}
table.tabular {margin-left: auto; margin-right: auto;}
div.td00{ margin-left:0pt; margin-right:0pt; }
div.td01{ margin-left:0pt; margin-right:5pt; }
div.td10{ margin-left:5pt; margin-right:0pt; }
div.td11{ margin-left:5pt; margin-right:5pt; }
table[rules] {border-left:solid black 0.4pt; border-right:solid black 0.4pt; }
td.td00{ padding-left:0pt; padding-right:0pt; }
td.td01{ padding-left:0pt; padding-right:5pt; }
td.td10{ padding-left:5pt; padding-right:0pt; }
td.td11{ padding-left:5pt; padding-right:5pt; }
table[rules] {border-left:solid black 0.4pt; border-right:solid black 0.4pt; }
.hline hr, .cline hr{ height : 1px; margin:0px; }
.tabbing-right {text-align:right;}
span.TEX {letter-spacing: -0.125em; }
span.TEX span.E{ position:relative;top:0.5ex;left:-0.0417em;}
a span.TEX span.E {text-decoration: none; }
span.LATEX span.A{ position:relative; top:-0.5ex; left:-0.4em; font-size:85%;}
span.LATEX span.TEX{ position:relative; left: -0.4em; }
div.float img, div.float .caption {text-align:center;}
div.figure img, div.figure .caption {text-align:center;}
.marginpar {width:20%; float:right; text-align:left; margin-left:auto; margin-top:0.5em; font-size:85%; text-decoration:underline;}
.marginpar p{margin-top:0.4em; margin-bottom:0.4em;}
.equation td{text-align:center; vertical-align:middle; }
td.eq-no{ width:5%; }
table.equation { width:100%; } 
div.math-display, div.par-math-display{text-align:center;}
math .texttt { font-family: monospace; }
math .textit { font-style: italic; }
math .textsl { font-style: oblique; }
math .textsf { font-family: sans-serif; }
math .textbf { font-weight: bold; }
.partToc a, .partToc, .likepartToc a, .likepartToc {line-height: 200%; font-weight:bold; font-size:110%;}
.chapterToc a, .chapterToc, .likechapterToc a, .likechapterToc, .appendixToc a, .appendixToc {line-height: 200%; font-weight:bold;}
.index-item, .index-subitem, .index-subsubitem {display:block}
.caption td.id{font-weight: bold; white-space: nowrap; }
table.caption {text-align:center;}
h1.partHead{text-align: center}
p.bibitem { text-indent: -2em; margin-left: 2em; margin-top:0.6em; margin-bottom:0.6em; }
p.bibitem-p { text-indent: 0em; margin-left: 2em; margin-top:0.6em; margin-bottom:0.6em; }
.paragraphHead, .likeparagraphHead { margin-top:2em; font-weight: bold;}
.subparagraphHead, .likesubparagraphHead { font-weight: bold;}
.quote {margin-bottom:0.25em; margin-top:0.25em; margin-left:1em; margin-right:1em; text-align:\\jmathmathustify;}
.verse{white-space:nowrap; margin-left:2em}
div.maketitle {text-align:center;}
h2.titleHead{text-align:center;}
div.maketitle{ margin-bottom: 2em; }
div.author, div.date {text-align:center;}
div.thanks{text-align:left; margin-left:10%; font-size:85%; font-style:italic; }
div.author{white-space: nowrap;}
.quotation {margin-bottom:0.25em; margin-top:0.25em; margin-left:1em; }
h1.partHead{text-align: center}
.sectionToc, .likesectionToc {margin-left:2em;}
.subsectionToc, .likesubsectionToc {margin-left:4em;}
.subsubsectionToc, .likesubsubsectionToc {margin-left:6em;}
.frenchb-nbsp{font-size:75%;}
.frenchb-thinspace{font-size:75%;}
.figure img.graphics {margin-left:10%;}
/* end css.sty */

\title{Derivees partielles}
\author{}
\date{}

\begin{document}
\maketitle

\textbf{Warning: 
requires JavaScript to process the mathematics on this page.\\ If your
browser supports JavaScript, be sure it is enabled.}

\begin{center}\rule{3in}{0.4pt}\end{center}

{[}
{[}{]}
{[}

\subsubsection{15.1 Dérivées partielles}

\paragraph{15.1.1 Notion de dérivée partielle}

Définition~15.1.1 Soit E et F deux espaces vectoriels normés. Soit U un
ouvert de E, f : U \rightarrow~ F, a \in U. Soit v \in E
\diagdown\0\. On dit que f admet au point a
une dérivée partielle suivant le vecteur v si l'application
t\mapsto~f(a + tv) (définie sur un voisinage de 0)
est dérivable au point 0.

Remarque~15.1.1 L'existence de la dérivée partielle en a suivant le
vecteur v est donc équivalente à l'existence de
lim_t\rightarrow~0~ f(a+tv)-f(a)
\over t = \partial_vf(a). Remarquons que si v' = \lambda~v,
\lambda~\neq~0, alors  f(a+tv')-f(a)
\over t = \lambda~ f(a+uv)-f(a) \over u
avec u = \lambda~t ce qui montre que f admet en a une dérivée partielle selon v
si et seulement si f admet une dérivée partielle suivant \lambda~v et qu'alors
\partial_\lambda~vf(a) = \lambda~\partial_vf(a).

Exemple~15.1.1 Soit f : \mathbb{R}~^2 \rightarrow~ \mathbb{R}~ définie par f(x,y) =
x^2 \over y si
y\neq~0 et f(x,0) = 0. Soit v =
(a,b)\neq~(0,0). On a  f((0,0)+tv)-f(0,0)
\over t = \left \
\cases 0 &si b = 0 \cr 
a^2 \over b &si
b\neq~0  \right .. On en déduit
que f admet une dérivée partielle suivant tout vecteur v et que
\partial_vf(0,0) = \left \
\cases 0 &si b = 0 \cr 
a^2 \over b &si
b\neq~0  \right .. Remarquons
que l'application v\mapsto~\partial_vf(0,0) n'est
pas linéaire. Remarquons également que f n'est pas continue en (0,0)
(puisque lim_t\rightarrow~0f(t,t^2~) =
1\neq~f(0,0)). L'existence de dérivée partielle
suivant tout vecteur n'implique donc pas la continuité.

Proposition~15.1.1 On a les propriétés évidentes de la dérivation de
t\mapsto~f(a + tv) à savoir (i) si f et g admettent
en a une dérivée partielle suivant le vecteur v, il en est de même de \alpha~f
+ \beta~g et \partial_v(\alpha~f + \beta~g)(a) = \alpha~\partial_vf(a) +
\beta~\partial_vg(a). (ii) si f et g (à valeurs scalaires) admettent en a
une dérivée partielle suivant le vecteur v, il en est de même de fg et
\partial_v(fg)(a) = g(a)\partial_vf(a) + f(a)\partial_vg(a).

Remarque~15.1.2 Par contre, on n'a pas de théorème général de
composition des dérivées partielles. En reprenant l'exemple ci dessus, f
: \mathbb{R}~^2 \rightarrow~ \mathbb{R}~ définie par f(x,y) = x^2
\over y si y\neq~0 et f(x,0) =
0, l'application f admet en (0,0) une dérivée partielle suivant tout
vecteur, l'application t\mapsto~(t,t^2)
est dérivable en 0 et pourtant
t\mapsto~f(t,t^2) n'est pas dérivable en
0 (elle n'y est même pas continue).

Définition~15.1.2 Soit E un espace vectoriel normé de dimension finie, \mathcal{E}
=
(e_1,\\ldots,e_n~)
une base de E, F un espace vectoriel normé. Soit U un ouvert de E, f : U
\rightarrow~ F, a \in U. On dit que f admet au point a une dérivée partielle d'indice
i (suivant la base \mathcal{E}) si elle admet une dérivée partielle suivant le
vecteur e_i. On note alors  \partial~f \over
\partial~x_i (a) = \partial_e_if(a).

Exemple~15.1.2 Si E = \mathbb{R}~^n et si \mathcal{E} est la base canonique de
\mathbb{R}~^n, l'existence d'une dérivée partielle d'indice i au point
a =
(a_1,\\ldots,a_n~)
équivaut à la dérivabilité au point a_i de l'application
partielle
x_i\mapsto~f(a_1,\\ldots,a_i-1,x_i,a_i+1,\\\ldots,a_n~).
On retrouve bien la notion habituelle de dérivée partielle~: dérivée
suivant la variable x_i, toutes les autres étant considérées
comme constantes.

\paragraph{15.1.2 Composition des dérivées partielles}

On a vu précédemment qu'on n'avait pas de théorème de composition des
dérivées partielles en toute généralité. On va introduire une notion
d'application de classe \mathcal{C}^1.

Définition~15.1.3 Soit U un ouvert de \mathbb{R}~^n, f : U \rightarrow~ F. On dit
que f est de classe \mathcal{C}^1 au point a si, sur un certain
voisinage V de a, f admet des dérivées partielles de tout indice i \in
{[}1,n{]} et si ces dérivées partielles x\mapsto~
\partial~f \over \partial~x_i (x) sont continues au point a.

Lemme~15.1.2 Soit F un espace vectoriel de dimension finie, V un ouvert
de \mathbb{R}~^n, f : V \rightarrow~ F. Soit I un intervalle de \mathbb{R}~, t \in I et \phi =
(\phi_1,\\ldots,\phi_n~)
: I \rightarrow~ V . On suppose que \phi est dérivable au point t et que f est de
classe \mathcal{C}^1 au point \phi(t). Alors f \cdot \phi est dérivable au point
t et (f \cdot \phi)'(t) =\ \\sum
 _i=1^n \partial~f \over \partial~x_i
(\phi(t))\phi_i'(t).

Démonstration Sans nuire à la généralité, en prenant une base (sur \mathbb{R}~) de
F et en travaillant composante par composante, on peut supposer que f
est à valeurs réelles. On écrit

\begin{align*} f(\phi(t + h)) - f(\phi(t))&& \%&
\\ & =& f(\phi_1(t +
h),\\ldots,\phi_n~(t
+ h)) -
f(\phi_1(t),\\ldots,\phi_n~(t))
\%& \\ & =& \\sum
_i=1^n(f(\ldots,\phi_
i-1(t),\phi_i(t + h),\phi_i+1(t +
h),\ldots~)\%&
\\ & & \qquad -
f(\\ldots,\phi_i-1(t),\phi_i(t),\phi_i+1~(t
+ h),\\ldots~)) \%&
\\ \end{align*}

Mais \phi est continue au point t et donc pour h assez petit, tous les
(\phi_1(t),\\ldots,\phi_i-1(t),\phi_i~(t
+ h),\phi_i+1(t +
h),\\ldots,\phi_n~(t
+ h)) se trouvent à l'intérieur d'une boule de centre a sur laquelle les
dérivées partielles de f de tout indice existent. En particulier,
l'application
x_i\mapsto~f(\phi_1(t),\\ldots,\phi_i-1(t),x_i,\phi_i+1~(t
+
h),\\ldots,\phi_n~(t
+ h)) est dérivable sur le segment {[}\phi_i(t),\phi_i(t +
h){]} et on peut appliquer le théorème des accroissements finis. On
obtient l'existence d'un \xi_i \in
{[}\phi_i(t),\phi_i(t + h){]} tel que

\begin{align*}
f(\phi_1(t),\\ldots,\phi_i-1(t),\phi_i~(t
+ h),\phi_i+1(t +
h),\\ldots,\phi_n~(t
+ h))&& \%& \\ & &
-f(\phi_1(t),\\ldots,\phi_i-1(t),\phi_i(t),\phi_i+1~(t
+
h),\\ldots,\phi_n~(t
+ h))\%& \\ & =& (\phi_i(t + h) -
\phi_i(t)) \%& \\ & &
\quad  \partial~f \over \partial~x_i
(\phi_1(t),\\ldots,\phi_i-1(t),\xi_i,\phi_i+1~(t
+
h),\\ldots,\phi_n~(t
+ h)) \%& \\
\end{align*}

Comme \phi est continue au point t et \xi_i \in
{[}\phi_i(t),\phi_i(t + h){]}, on a

\begin{align*}
lim_h\rightarrow~0(\phi_1(t),\\\ldots,\phi_i-1(t),\xi_i,\phi_i+1~(t
+
h),\\ldots,\phi_n~(t
+ h))&&\%& \\ & =&
(\phi_1(t),\\ldots,\phi_i-1(t),\phi_i(t),\phi_i+1(t),\\\ldots,\phi_n~(t))\%&
\\ & =& \phi(t) \%&
\\ \end{align*}

et comme  \partial~f \over \partial~x_i est continue au
point \phi(t), on a

\begin{align*} \partial~f \over
\partial~x_i (\phi(t))&& \%& \\ & =&
lim_h\rightarrow~0~ \partial~f \over
\partial~x_i
(\phi_1(t),\\ldots,\phi_i-1(t),\xi_i,\phi_i+1~(t
+
h),\\ldots,\phi_n~(t
+ h))\%& \\
\end{align*}

Il suffit alors de diviser par h et de faire tendre h vers 0 pour voir
que

lim_h\rightarrow~0~ f(\phi(t + h)) - f(\phi(t))
\over h = \\sum
_i=1^n \partial~f \over \partial~x_i
(\phi(t))\phi_i'(t)

ce qui achève la démonstration.

Appliquant ce lemme à \phi : t\mapsto~g(a + tv) au
point t = 0, on obtient le théorème suivant

Théorème~15.1.3 Soit F un espace vectoriel de dimension finie, U un
ouvert de \mathbb{R}~^n, f : U \rightarrow~ F. Soit E un espace vectoriel normé, V
un ouvert de E et g =
(g_1,\\ldots,g_n~)
: V \rightarrow~ U \subset~ \mathbb{R}~^n. Soit a \in V et v \in E
\diagdown\0\. Si g admet en a une dérivée
partielle suivant le vecteur v et si f est de classe \mathcal{C}^1 au
point a, alors f \cdot g admet en a une dérivée partielle suivant le vecteur
v et on a

\partial_v(f \cdot g)(a) = \\sum
_i=1^n \partial~f \over \partial~x_i
(g(a))\partial_vg_i(a)

Dans le cas particulier où E = \mathbb{R}~^p et où on prend pour v le
\\jmathmath-ième vecteur de la base canonique, on obtient la version suivante (on
a changé le nom des variables pour les appeler
y_1,\\ldots,y_n~
dans \mathbb{R}~^n).

Corollaire~15.1.4 Soit F un espace vectoriel de dimension finie, U un
ouvert de \mathbb{R}~^n, f : U \rightarrow~ F. Soit V un ouvert de \mathbb{R}~^p
et g =
(g_1,\\ldots,g_n~)
: V \rightarrow~ U \subset~ \mathbb{R}~^n. Soit a \in V et \\jmathmath \in {[}1,p{]}. Si g admet en a
une dérivée partielle d'indice \\jmathmath et si f est de classe \mathcal{C}^1 au
point a, alors f \cdot g admet en a une dérivée partielle d'indice \\jmathmath et on a

 \partial~(f \cdot g) \over \partial~x_\\jmathmath (a) =
\sum _i=1^n~ \partial~f
\over \partial~y_i (g(a)) \partial~g_i
\over \partial~x_\\jmathmath (a)

Remarque~15.1.3 On en déduit immédiatement que la composée de deux
applications de classe \mathcal{C}^1 est encore de classe
\mathcal{C}^1.

Citons aussi le corollaire suivant du lemme, où l'on prend \phi(t) = a + tv

Corollaire~15.1.5 Soit F un espace vectoriel de dimension finie, V un
ouvert de \mathbb{R}~^n, f : V \rightarrow~ F. Soit a \in V . Si f est de classe
\mathcal{C}^1 au point a, alors elle admet en a des dérivées partielles
suivant tout vecteur et on a

\partial_vf(a) = \\sum
_i=1^nv_ i \partial~f \over
\partial~x_i (a)\qquad \text si v =
(v_1,\ldots,v_n~)

Remarque~15.1.4 On voit que dans ce cas
v\mapsto~\partial_vf(a) est linéaire. Cette
remarque nous conduira à la définition de la différentielle dans la
section suivante.

\paragraph{15.1.3 Théorème des accroissements finis et applications}

Théorème~15.1.6 Soit U un ouvert de \mathbb{R}~^n, f : U \rightarrow~ \mathbb{R}~ de classe
\mathcal{C}^1. Soit a \in U et h \in \mathbb{R}~^n tel que {[}a,a + h{]} \subset~
U. Alors, il existe \theta \in{]}0,1{[} tel que

f(a + h) - f(a) = \\sum
_i=1^nh_ i \partial~f \over
\partial~x_i (a + \thetah)

Démonstration Soit \psi : {[}0,1{]} \rightarrow~ \mathbb{R}~ définie par \psi(t) = f(a + th). Le
lemme du paragraphe précédent montre que \psi est dérivable sur {[}0,1{]}
et que \psi'(t) = \\sum ~
_i=1^nh_i \partial~f \over
\partial~x_i (a + th). Le théorème des accroissements finis assure
qu'il existe \theta \in{]}0,1{[} tel que \psi(1) - \psi(0) = (1 - 0)\psi'(\theta), ce qui
n'est autre que la formule ci dessus.

Corollaire~15.1.7 Soit U un ouvert de \mathbb{R}~^n, F un espace
vectoriel de dimension finie, f : U \rightarrow~ F de classe \mathcal{C}^1. Alors
f est continue.

Démonstration En prenant une base (sur \mathbb{R}~) de F et en travaillant
composante par composante, on peut supposer que f est à valeurs réelles.
Puisque les dérivées partielles sont continues au point a, il existe \eta
\textgreater{} 0 tel que B(a,\eta) \subset~ U et \forall~~x \in
B(0,\eta),  \partial~f \over \partial~x_i (x) - \partial~f
\over \partial~x_i (a)\leq 1, d'où 
\partial~f \over \partial~x_i (x)\leq 1 + 
\partial~f \over \partial~x_i (a). Pour
\h\ \textless{} \eta, on
a alors {[}a,a + h{]} \subset~ B(0,\eta) et donc f(a + h) -
f(a)\leq\\sum ~
_i=1^nh_i\,\left
(\left  \partial~f \over
\partial~x_i (a)\right  +
1\right ), ce qui montre la continuité de f au point a.

Corollaire~15.1.8 Soit U un ouvert connexe de \mathbb{R}~^n, F un
espace vectoriel de dimension finie, f : U \rightarrow~ F. Alors f est constante si
et seulement si~elle est de classe \mathcal{C}^1 et toutes ses dérivées
partielles sont nulles.

Démonstration Si f est constante, il est clair qu'elle est de classe
\mathcal{C}^1 et que toutes ses dérivées partielles sont nulles. Pour
la réciproque, en prenant une base (sur \mathbb{R}~) de F et en travaillant
composante par composante, on peut supposer que f est à valeurs réelles.
Soit x_0 \in U et soit X = \x \in
U∣f(x) = f(x_0)\.
Puisque f est continue (d'après le corollaire précédent), X est un fermé
de U, évidemment non vide. Montrons que X est également ouvert dans U~;
soit en effet x_1 dans X et soit \eta \textgreater{} 0 tel que
B(x_1,\eta) \subset~ U. Pour
\h\ \textless{} \eta, on
a {[}x_1,x_1 + h{]} \subset~ B(x_1,\eta) \subset~ U et le
théorème des accroissements finis nous donne f(x_1 + h) =
f(x_1) = f(x_0), les dérivées partielles étant
supposées nulles. On a donc B(x_1,\eta) \subset~ X, et donc X est ouvert.
Comme X est à la fois ouvert et fermé, non vide dans U connexe, on a X =
U et donc f est constante.

\paragraph{15.1.4 Dérivées partielles successives}

On définit la notion de dérivées partielles successives de manière
récursive de la manière suivante

Définition~15.1.4 Soit U un ouvert de \mathbb{R}~^n, a \in U et f : U \rightarrow~
E. Soit
(i_1,\\ldots,i_k~)
\in {[}1,n{]}^k. On dit que  \partial~^kf
\over
\partial~x_i_1\\ldots\partial~x_i_k~
(a) existe s'il existe un ouvert V tel que a \in V \subset~ U et sur lequel 
\partial~^k-1f \over
\partial~x_i_2\\ldots\partial~x_i_k~
(x) existe et si l'application x\mapsto~
\partial~^k-1f \over
\partial~x_i_2\\ldots\partial~x_i_k~
(x) admet une dérivée partielle d'indice i_1. On pose alors

 \partial~^kf \over
\partial~x_i_1\\ldots\partial~x_i_k~
(a) = \partial~ \over \partial~x_i_1
\left ( \partial~^k-1f \over
\partial~x_i_2\\ldots\partial~x_i_k~
\right )(a)

Définition~15.1.5 Soit U un ouvert de \mathbb{R}~^n et f : U \rightarrow~ E. On
dit que f est de classe C^k sur U si,
\forall~(i_1,\\\ldots,i_k~)
\in {[}1,n{]}^k, l'application x\mapsto~
\partial~^kf \over
\partial~x_i_1\\ldots\partial~x_i_k~
(x) est définie et continue sur U.

Remarque~15.1.5 Comme on a vu que toute application de classe
\mathcal{C}^1 est continue, on en déduit immédiatement que toute
application de classe C^k est aussi de classe
C^k-1. On dira bien entendu que f est de classe
C^\infty~ si elle est de classe C^k pour tout k. Une
récurrence évidente sur k montre que la composée de deux applications de
classe C^k est encore de classe C^k et que donc la
composée de deux applications de classe C^\infty~ est encore de
classe C^\infty~.

Lemme~15.1.9 Soit U un ouvert de \mathbb{R}~^2, f : U \rightarrow~ \mathbb{R}~ de classe
C^2. Alors,  \partial~^2f \over
\partial~x_1\partial~x_2 = \partial~^2f \over
\partial~x_2\partial~x_1 .

Démonstration Soit (a_1,a_2) \in U et soit

\begin{align*} \phi(h_1,h_2)& =&
1 \over h_1h_2 (f(a_1 +
h_1,a_2 + h_2) - f(a_1 +
h_1,a_2)\%& \\ & &
\quad \quad \quad -
f(a_1,a_2 + h_2) +
f(a_1,a_2)) \%& \\
\end{align*}

définie pour h_1 et h_2 non nuls et assez petits. On a
\phi(h_1,h_2) = 1 \over
h_1h_2 \psi_1(a_1 + h_1) -
\psi_1(a_1) avec \psi_1(x_1) =
f(x_1,a_2 + h_2) -
f(x_1,a_2). Or \psi_1 est dérivable sur
{[}a_1,a_1 + h_1{]} avec
\psi_1'(x_1) = \partial~f \over \partial~x_1
(x_1,a_2 + h_2) - \partial~f \over
\partial~x_1 (x_1,a_2). On peut donc appliquer le
théorème des accroissements finis, et donc il existe \xi_1 \in
{[}a_1,a_1 + h_1{]} tel que

\begin{align*} \phi(h_1,h_2)& =&
1 \over h_2 \psi_1'(\xi_1) \%&
\\ & =& 1 \over
h_2 \left ( \partial~f \over
\partial~x_1 (\xi_1,a_2 + h_2) - \partial~f
\over \partial~x_1
(\xi_1,a_2)\right )\%&
\\ & =& \partial~^2f
\over \partial~x_2\partial~x_1
(\xi_1,\xi_2) \%& \\
\end{align*}

avec \xi_2 \in {[}a_2,a_2 + h_2{]} en
appliquant le théorème des accroissements finis à
x_2\mapsto~ \partial~f \over
\partial~x_1 (\xi_1,x_2) qui est dérivable sur
{[}a_2,a_2 + h_2{]}, de dérivée 
\partial~^2f \over \partial~x_2\partial~x_1
(\xi_1,x_2). Quand h_1 et h_2 tendent
vers 0, \xi_1 et \xi_2 tendent respectivement vers
a_1 et a_2 et la continuité de  \partial~^2f
\over \partial~x_2\partial~x_1 montre que
lim_(h_1,h_2)\rightarrow~(0,0)\phi(h_1,h_2~)
= \partial~^2f \over \partial~x_2\partial~x_1
(a_1,a_2). Comme les deux variables \\jmathmathouent un rôle
symétrique dans la définition de \phi, en posant \psi_2(x_2)
= f(a_1 + h_1,x_2) -
f(a_1,x_2) et en appliquant deux fois le théorème des
accroissements finis, on obtient
lim_(h_1,h_2)\rightarrow~(0,0)\phi(h_1,h_2~)
= \partial~^2f \over \partial~x_1\partial~x_2
(a_1,a_2), ce qui démontre que  \partial~^2f
\over \partial~x_1\partial~x_2
(a_1,a_2) = \partial~^2f \over
\partial~x_2\partial~x_1 (a_1,a_2).

Théorème~15.1.10 (Schwarz). Soit U un ouvert de \mathbb{R}~^n et f : U
\rightarrow~ E (espace vectoriel normé de dimension finie) de classe
C^2. Alors \forall~~(i,\\jmathmath) \in
{[}1,n{]}^2,

 \partial~^2f \over \partial~x_i\partial~x_\\jmathmath =
\partial~^2f \over \partial~x_\\jmathmath\partial~x_i

Démonstration En prenant une base de E, on peut se contenter de montrer
le résultat lorsque E = \mathbb{R}~. Si i = \\jmathmath, le résultat est évident. Supposons
i \textless{} \\jmathmath et soit
(a_1,\\ldots,a_n~)
\in \mathbb{R}~^n. On applique le lemme précédent à l'application de
classe C^2, définie sur un ouvert contenant
(a_i,a_\\jmathmath),

g(x_i,x_\\jmathmath) =
f(a_1,\\ldots,a_i-1,x_i,a_i+1,\\\ldots,a_\\jmathmath-1,x_\\jmathmath,a_\\jmathmath+1,\\\ldots,a_n~)

qui est de classe C^2 (composée d'applications de classe
C^2). On a donc  \partial~^2g \over
\partial~x_i\partial~x_\\jmathmath (a_i,a_\\jmathmath) =
\partial~^2g \over \partial~x_\\jmathmath\partial~x_i
(a_i,a_\\jmathmath), soit encore

 \partial~^2f \over \partial~x_i\partial~x_\\jmathmath
(a_1,\\ldots,a_n~)
= \partial~^2f \over \partial~x_\\jmathmath\partial~x_i
(a_1,\\ldots,a_n~)

Corollaire~15.1.11 Soit U un ouvert de \mathbb{R}~^n et f : U \rightarrow~ E de
classe C^k. Soit
(i_1,\\ldots,i_k~)
\in {[}1,n{]}^k. Pour toute permutation \sigma de {[}1,k{]} on a

 \partial~^kf \over
\partial~x_i_\sigma(1)\\ldots\partial~x_i_\sigma(k)~
= \partial~^kf \over
\partial~x_i_1\\ldots\partial~x_i_k~

Démonstration D'après le théorème de Schwarz, le résultat est vrai
lorsque \sigma = \tau_\\jmathmath,\\jmathmath+1 est la transposition qui échange \\jmathmath et \\jmathmath +
1. Mais toute permutation de {[}1,k{]} est un produit de telles
transpositions (facile) ce qui démontre le corollaire.

Notation définitive Soit
(i_1,\\ldots,i_k~)
\in {[}1,n{]}^k. Pour \\jmathmath \in {[}1,n{]}, soit k_\\jmathmath le
nombre de i_q qui sont égaux à \\jmathmath. On a donc à une permutation
près, la famille
(i_1,\\ldots,i_k~)
qui est égale à
(\overbrace1,\\ldots,1k_1~
fois,\\ldots,\overbrace\\jmathmath,\\\ldots,\\jmathmath~
k_\\jmathmath
fois,\\ldots,\overbracen,\\\ldots,nk_n~
fois), chaque \\jmathmath étant compté k_\\jmathmath fois. En notant
\partial~x_\\jmathmath^k_\\jmathmath à la place de
\overbrace\partial~x_\\jmathmath\\ldots\partial~x_\\jmathmath~
k_\\jmathmath fois, on obtient

 \partial~^kf \over
\partial~x_i_1\\ldots\partial~x_i_k~
= \partial~^kf \over
\partial~x_1^k_1\\ldots\partial~x_n^k_n~

\paragraph{15.1.5 Formules de Taylor}

Lemme~15.1.12 Soit U un ouvert de \mathbb{R}~^n et f : U \rightarrow~ E de classe
C^k. Soit a \in U et h \in \mathbb{R}~^n tel que {[}a,a + h{]} \subset~
U. Posons \phi(t) = f(a + th), définie et de classe C^k sur
{[}0,1{]}. Alors, pour tout t \in {[}0,1{]},

\begin{align*} \phi^(k)(t) =
\\sum
_k_1+\ldots+k_n=k~
k! \over
k_1!\ldotsk_n!~
h_1^k_1
\ldotsh_n^k_n ~
\partial~^kf \over
\partial~x_1^k_1\ldots\partial~x_n^k_n~
(a + th)& & \%& \\
\end{align*}

Démonstration Par récurrence sur k. Pour k = 1, ce n'est qu'une autre
formulation du résultat

\begin{align*} \phi'(t)& =&
\sum _i=1^nh_ i~ \partial~f
\over \partial~x_i (a + th) \%&
\\ & =& \\sum
_k_1+\ldots+k_n=1h_1^k_1~
\ldotsh_n^k_n ~
\partial~f \over
\partial~x_1^k_1\ldots\partial~x_n^k_n~
(a + th)\%& \\
\end{align*}

en posant k_i = 1 et k_\\jmathmath = 0 pour
i\neq~\\jmathmath.

Supposons le résultat démontré pour k - 1. On a donc

\begin{align*} \phi^(k-1)(t) =&& \%&
\\ & & \\sum
_k_1+\ldots+k_n=k-1~
(k - 1)! \over
k_1!\ldotsk_n!~
h_1^k_1
\ldotsh_n^k_n ~
\partial~^k-1f \over
\partial~x_1^k_1\ldots\partial~x_n^k_n~
(a + th)\%& \\
\end{align*}

On en déduit que

\begin{align*} \phi^(k)(t) =&& \%&
\\ & & \\sum
_k_1+\ldots+k_n=k-1~
(k - 1)! \over
k_1!\ldotsk_n!~
h_1^k_1
\ldotsh_n^k_n ~
d \over dt \left ( \partial~^k-1f
\over
\partial~x_1^k_1\ldots\partial~x_n^k_n~
(a + th)\right )\%& \\
\end{align*}

soit encore

\begin{align*} \phi^(k)(t)& =&
\\sum
_k_1+\ldots+k_n=k-1~
(k - 1)! \over
k_1!\ldotsk_n!~
h_1^k_1
\ldotsh_n^k_n ~
\%& \\ & & \quad
\quad  \\sum
_i=1^nh_ i \partial~^kf
\over
\partial~x_1^k_1\ldots\partial~x_i^k_i+1\\ldots\partial~x_n^k_n~
(a + th)\%& \\
\end{align*}

En intervertissant les deux signes de somme on obtient

\begin{align*} \phi^(k)(t)& =&
\sum _i=1^n~
\\sum
_k_1+\ldots+k_n=k-1~
(k - 1)!(k_i + 1) \over
k_1!\ldots(k_i~ +
1)!\ldotsk_n!~ \%&
\\ & & \quad
\quad h_1^k_1
\\ldotsh_i^k_i+1\\\ldotsh_
n^k_n  \partial~^kf \over
\partial~x_1^k_1\\ldots\partial~x_i^k_i+1\\\ldots\partial~x_n^k_n~
(a + th)\%& \\
\end{align*}

et en faisant un changement d'indice

\begin{align*} \phi^(k)(t)& =&
\sum _i=1^n~
\sum _
k_1+\ldots+k_n~=k
\atop k_i≥1  (k - 1)!k_i
\over
k_1!\ldotsk_n!~ \%&
\\ & & \quad
\quad h_1^k_1
\\ldotsh_i^k_i~
\\ldotsh_n^k_n~
 \partial~^kf \over
\partial~x_1^k_1\\ldots\partial~x_i^k_i\\\ldots\partial~x_n^k_n~
(a + th)\%& \\
\end{align*}

Réintroduisons les termes pour k_i = 0 qui sont nuls puisqu'ils
contiennent le facteur (k - 1)!k_i, on obtient

\begin{align*} \phi^(k)(t)& =&
\sum _i=1^n~
\\sum
_k_1+\ldots+k_n=k~
(k - 1)!k_i \over
k_1!\ldotsk_n!~
h_1^k_1
\ldotsh_i^k_i~
\ldotsh_n^k_n~
\%& \\ & & \quad
\quad \quad  \partial~^kf
\over
\partial~x_1^k_1\\ldots\partial~x_i^k_i\\\ldots\partial~x_n^k_n~
(a + th) \%& \\
\end{align*}

Ceci nous permet de réintervertir les deux sommations, soit encore,
après mise en facteur

\begin{align*} \phi^(k)(t)& =&
\\sum
_k_1+\ldots+k_n=k~
(k - 1)!\sum _i=1^nk_i~
\over
k_1!\ldotsk_n!~
h_1^k_1
\ldotsh_i^k_i~
\ldotsh_n^k_n~
\%& \\ & & \quad
\quad \quad  \partial~^kf
\over
\partial~x_1^k_1\\ldots\partial~x_i^k_i\\\ldots\partial~x_n^k_n~
(a + th) \%& \\
\end{align*}

soit encore

\begin{align*} \phi^(k)(t)& =&
\\sum
_k_1+\ldots+k_n=k~
k! \over
k_1!\ldotsk_n!~
h_1^k_1
\ldotsh_n^k_n ~
\partial~^kf \over
\partial~x_1^k_1\ldots\partial~x_n^k_n~
(a + th)\%& \\
\end{align*}

ce qui achève la récurrence.

Remarque~15.1.6 Cette formule est tout à fait analogue à la formule du
binôme généralisée

(X_1 +
\\ldots~ +
X_n)^k = \\sum
_k_1+\ldots+k_n=k~
k! \over
k_1!\ldotsk_n!~
X_1^k_1
\ldotsX_n^k_n ~

Cette remarque nous conduira à une notation plus compacte. Introduisons
un produit symbolique sur les expressions du type
h_1^k_1\\ldotsh_n^k_n~
\partial~^k \over
\partial~x_1^k_1\\ldots\partial~x_n^k_n~
en posant

\begin{align*} \left
(h_1^k_1
\\ldotsh_n^k_n~
 \partial~^k \over
\partial~x_1^k_1\\ldots\partial~x_n^k_n~
\right ) ∗\left
(h_1^l_1
\\ldotsh_n^l_n~
 \partial~^l \over
\partial~x_1^l_1\\ldots\partial~x_n^l_n~
\right ) =&&\%& \\ & &
h_1^k_1+l_1
\\ldotsh_n^k_n+l_n~
 \partial~^k+l \over
\partial~x_1^k_1+l_1\\ldots\partial~x_n^k_n+l_n~
\quad \quad \quad \%&
\\ \end{align*}

Ce produit est commutatif, et

\begin{align*} \\sum
_k_1+\ldots+k_n=k~
k! \over
k_1!\ldotsk_n!~
h_1^k_1
\ldotsh_n^k_n ~
\partial~^k \over
\partial~x_1^k_1\ldots\partial~x_n^k_n~
&&\%& \\ & & = \left
(h_1 \partial~ \over \partial~x_1 +
\\ldots~ +
h_n \partial~ \over \partial~x_n
\right )^k∗\quad
\quad \quad \%&
\\ \end{align*}

où la notation ^k∗ désigne la puissance k-ième pour ce
produit commutatif. La formule s'écrit alors de manière plus agréable
sous la forme

\phi^(k)(t) = \left (h_ 1 \partial~
\over \partial~x_1 +
\\ldots~ +
h_n \partial~ \over \partial~x_n
\right )^k∗f(a + th)

Ces puissances se développent de la manière évidente en respectant la
règle de calcul pour le produit ∗.

Exemple~15.1.3 \phi'(t) = \left (h_1 \partial~
\over \partial~x_1 +
\\ldots~ +
h_n \partial~ \over \partial~x_n
\right )f(a + th)

\begin{align*} \phi''(t)& =& \left
(h_1 \partial~ \over \partial~x_1 +
\\ldots~ +
h_n \partial~ \over \partial~x_n
\right )^2∗f(a + th) \%&
\\ & =& \\sum
_i=1^nh_ i^2 \partial~^2f
\over \partial~x_i^2 (a + th) +
2\\sum
_i\textless{}\\jmathmathh_ih_\\jmathmath \partial~^2f
\over \partial~x_i\partial~x_\\jmathmath (a + th)\%&
\\ \end{align*}

et ainsi de suite.

Théorème~15.1.13 (formule de Taylor avec reste intégral). Soit U un
ouvert de \mathbb{R}~^n et f : U \rightarrow~ E de classe C^k+1. Soit a
\in U et h \in \mathbb{R}~^n tel que {[}a,a + h{]} \subset~ U. Alors

\begin{align*} f(a + h)& =& f(a) +
\sum _p=1^k~ 1
\over p! \left (h_1 \partial~
\over \partial~x_1 +
\ldots + h_n~ \partial~
\over \partial~x_n \right
)^p∗f(a)\%& \\
+\int  _0^1~ (1 -
t)^k \over k!  \left
(h_1 \partial~ \over \partial~x_1 +
\\ldots~ +
h_n \partial~ \over \partial~x_n
\right )^(k+1)∗f(a + th) dt&&\%&
\\ \end{align*}

Démonstration C'est simplement la formule de Taylor avec reste intégral
pour la fonction \phi~:

\phi(1) = \phi(0) + \sum _p=1^k~ 1
\over p! \phi^(p)(0) +
\\int  ~
_0^1 (1 - t)^k \over k!
\phi^(k+1)(t) dt

Remarque~15.1.7 On utilisera le plus souvent cette formule pour k = 1~;
dans cas d'une fonction définie sur un ouvert de \mathbb{R}~^2 on
obtiendra par exemple

\begin{align*} f(a + h)& =& f(a) + h_1
\partial~f \over \partial~x_1 (a) + h_2 \partial~f
\over \partial~x_2 (a) \%&
\\ & & \quad +
h_1^2\int  _0^1~(1
- t) \partial~^2f \over \partial~x_1^2 (a
+ th) dt \%& \\ & &
\quad + h_2^2\\int
 _0^1(1 - t) \partial~^2f \over
\partial~x_2^2 (a + th) dt \%&
\\ & & \quad +
2h_1h_2\int ~
_0^1(1 - t) \partial~^2f \over
\partial~x_1\partial~x_2 (a + th) dt\%&
\\ \end{align*}

Théorème~15.1.14 (formule de Taylor-Lagrange). Soit U un ouvert de
\mathbb{R}~^n et f : U \rightarrow~ \mathbb{R}~ de classe C^k+1. Soit a \in U et h
\in \mathbb{R}~^n tel que {[}a,a + h{]} \subset~ U. Alors, il existe \theta
\in{]}0,1{[} tel que

\begin{align*} f(a + h)& =& f(a) +
\sum _p=1^k~ 1
\over p! \left (h_1 \partial~
\over \partial~x_1 +
\ldots + h_n~ \partial~
\over \partial~x_n \right
)^p∗f(a) \%& \\ & & + 1
\over (k + 1)! \left (h_1 \partial~
\over \partial~x_1 +
\\ldots~ +
h_n \partial~ \over \partial~x_n
\right )^(k+1)∗f(a + \thetah)\%&
\\ \end{align*}

Démonstration C'est simplement la formule de Taylor Lagrange pour la
fonction \phi~:

\phi(1) = \phi(0) + \sum _p=1^k~ 1
\over p! \phi^(p)(0) + 1 \over
(k + 1)! \phi^(k+1)(\theta)

Théorème~15.1.15 (formule de Taylor-Young). Soit U un ouvert de
\mathbb{R}~^n et f : U \rightarrow~ E (espace vectoriel normé de dimension finie)
de classe C^k. Soit a \in U. Alors, quand h tend vers 0 on a

f(a + h) = f(a) + \sum _p=1^k~ 1
\over p! \left (h_1 \partial~
\over \partial~x_1 +
\ldots + h_n~ \partial~
\over \partial~x_n \right
)^p∗f(a) +
o(\h\^k)

Démonstration Quitte à prendre une base de E et à travailler composante
par composante, on peut supposer que E = \mathbb{R}~~; toutes les normes sur
\mathbb{R}~^n étant équivalentes, on peut supposer que
\h\ =
h_1 +
\\ldots~ +
h_n. Soit \rho \textgreater{} 0 tel que B(a,\rho)
\subset~ U et soit h tel que
\h\ \textless{} \rho. On
a alors {[}a,a + h{]} \subset~ B(a,\rho) \subset~ U~; on peut donc appliquer la formule
de Taylor-Lagrange à l'ordre k - 1 qui nous donne

\begin{align*} f(a + h)& -& f(a)
-\sum _p=1^k~ 1
\over p! \left (h_1 \partial~
\over \partial~x_1 +
\ldots + h_n~ \partial~
\over \partial~x_n \right
)^p∗f(a)\%& \\ & =& 1
\over k! \left (h_1 \partial~
\over \partial~x_1 +
\\ldots~ +
h_n \partial~ \over \partial~x_n
\right )^k∗f(a + \thetah) \%&
\\ & -& 1 \over k!
\left (h_1 \partial~ \over
\partial~x_1 +
\\ldots~ +
h_n \partial~ \over \partial~x_n
\right )^k∗f(a) \%&
\\ \end{align*}

Mais les dérivées partielles de f sont continues. Soit \epsilon \textgreater{}
0~; il existe \eta \textgreater{} 0 tel que

\begin{align*}
\h\ \textless{} \eta&
\rigtharrow~&
\forall~(k_1,\\\ldots,k_n~)\text
tel que k_1 +
\\ldots~ +
k_n = k, \forall~~t \in {[}0,1{]} \%&
\\ & & \left 
\partial~^kf \over
\partial~x_1^k_1\\ldots\partial~x_n^k_n~
(a + th)\right . -\left .
\partial~^kf \over
\partial~x_1^k_1\\ldots\partial~x_n^k_n~
(a)\right  \textless{} \epsilon\%&
\\ \end{align*}

Pour \h\ \textless{}
\eta, on a alors (en développant les deux puissances symboliques)

\begin{align*} \big
\left (\\sum
h_i \partial~ \over \partial~x_i
\right )^k∗f(a + \thetah) -\left
(\sum h_i~ \partial~ \over
\partial~x_i \right
)^k∗f(a)\big &&\%&
\\ & \textless{}&
\epsilon\\sum
_k_1+\ldots+k_n=k~
k! \over
k_1!\ldotsk_n!~
h_1^k_1
\ldotsh_n^k_n~
\%& \\ & =&
\epsilon(h_1 +
\\ldots~ +
h_n)^k =
\epsilon\h\^k \%&
\\ \end{align*}

ce qui démontre le résultat.

\paragraph{15.1.6 Application aux extremums de fonctions de plusieurs
variables}

Soit U un ouvert de \mathbb{R}~^n et f : U \rightarrow~ \mathbb{R}~. Nous allons rechercher
les extremums de la fonction f à l'aide des résultats qui suivent.

Proposition~15.1.16 Soit U un ouvert de \mathbb{R}~^n et f : U \rightarrow~ \mathbb{R}~ de
classe \mathcal{C}^1. Soit a \in U. Si f admet en a un extremum local, on
a \forall~~i \in {[}1,n{]}, \partial~f \over
\partial~x_i (a) = 0.

Démonstration Il suffit de remarquer que la fonction
t\mapsto~f(a + te_i) (définie sur un
voisinage de 0) admet en 0 un extremum local. On a donc

 \partial~f \over \partial~x_i (a) = d
\over dt \left (f(a +
te_i)\right )_t=0 = 0

Dans le cas des fonctions d'une variable, la condition ci dessus n'est
dé\\jmathmathà pas suffisante (considérer
x\mapsto~x^3 au point 0). Il est clair
qu'il en est de même a fortiori pour une fonction de plusieurs
variables. Pour obtenir des résultats plus précis et en particulier des
conditions suffisantes d'extremums, nous allons introduire une forme
quadratique sur \mathbb{R}~^n

Définition~15.1.6 Soit U un ouvert de \mathbb{R}~^n et f : U \rightarrow~ \mathbb{R}~ de
classe C^2. Soit a \in U. On appelle différentielle seconde au
point a la forme quadratique sur \mathbb{R}~^n,

\begin{align*} h& =&
(h_1,\\ldots,h_n)\mathrel\mapsto~~\left
(h_1 \partial~ \over \partial~x_1 +
\\ldots~ +
h_n \partial~ \over \partial~x_n
\right )^2∗f(a) \%&
\\ & & = \\sum
_i=1^nh_ i^2 \partial~^2f
\over \partial~x_i^2 (a) +
2\\sum
_i\textless{}\\jmathmathh_ih_\\jmathmath \partial~^2f
\over \partial~x_i\partial~x_\\jmathmath (a)\%&
\\ \end{align*}

Théorème~15.1.17 Soit U un ouvert de \mathbb{R}~^n et f : U \rightarrow~ \mathbb{R}~ de
classe C^2. Soit a \in U tel que \forall~~i \in
{[}1,n{]}, \partial~f \over \partial~x_i (a) = 0 et soit \Phi
la forme quadratique différentielle seconde au point a. Alors (i) si \Phi
est définie positive, c'est-à-dire si h\neq~0 \rigtharrow~
\Phi(h) \textgreater{} 0, alors f admet en a un minimum local strict (ii)
si \Phi est définie négative, c'est-à-dire si
h\neq~0 \rigtharrow~ \Phi(h) \textless{} 0, alors f admet en a
un maximum local strict (iii) si \Phi n'est ni positive ni négative, alors
f n'admet pas d'extremum en a (on dit dans ce cas que a est un point
selle ou point col de a, par analogie avec une selle de cheval ou un col
de montagne).

Démonstration (i). Utilisons la formule de Taylor Young à l'ordre 2. On
a donc, en tenant compte de  \partial~f \over \partial~x_i
(a) = 0, f(a + h) = f(a) + 1 \over 2 \Phi(h)
+\
h\^2\epsilon(h), avec
lim_h\rightarrow~0~\epsilon(h) = 0. Pour démontrer (i),
nous allons utiliser le lemme suivant

Lemme~15.1.18 Soit \Phi une forme quadratique définie positive sur
\mathbb{R}~^n (ou tout espace vectoriel normé de dimension finie).
Alors \exists~\alpha~ \textgreater{} 0,
\forall~h \in \mathbb{R}~^n~, \Phi(h) ≥
\alpha~\h\^2.

Démonstration Soit S la sphère unité de \mathbb{R}~^n. Comme \Phi est
continue sur S qui est compact, \Phi atteint sur S sa borne inférieure \alpha~.
Soit donc x_0 \in S tel que \Phi(x_0) = \alpha~
= inf _x\inS~\Phi(x). Comme
x_0\neq~0, on a \alpha~ \textgreater{} 0. De
plus, si h\neq~0, on a  h \over
\h\ \in S, soit \Phi( h
\over
\h\ ) ≥ \alpha~ soit 
\Phi(h) \over
\h\^2 ≥
\alpha~, soit encore \Phi(h) ≥
\alpha~\h\^2.

Puisque lim_h\rightarrow~0~\epsilon(h) = 0, il existe \eta
\textgreater{} 0 tel que
\h\ \textless{} \eta
\rigtharrow~\epsilon(h)\leq \alpha~ \over 4 . Pour
\h\ \textless{} \eta, on
a donc

\begin{align*} f(a + h) - f(a)& =& 1
\over 2 \Phi(h) +\
h\^2\epsilon(h) \%&
\\ & ≥& \alpha~ \over 2
\h\^2 - \alpha~
\over 4
\h\^2 = \alpha~
\over 4
\h\^2
\textgreater{} 0\%& \\
\end{align*}

pour h\neq~0. Donc f admet en a un minimum local
strict.

Pour démontrer (ii) à partir de (i), il suffit de changer f en - f.

(iii). Si \Phi n'est ni positive, ni négative, il existe v_1 \in
\mathbb{R}~^n tel que \Phi(v_1) \textless{} 0 et il existe
v_2 \in \mathbb{R}~^n tel que \Phi(v_2) \textgreater{} 0.
On a alors, d'après la même formule de Taylor, en posant h =
tv_i, f(a + tv_i) = f(a) + 1 \over
2 \Phi(tv_i) +
t^2\v_
i\^2\epsilon(tv_ i) = f(a) +
t^2 \over 2 \Phi(v_i) +
t^2\epsilon_ i(t) avec
lim_t\rightarrow~0\epsilon_i~(t) = 0. On en
déduit qu'il existe un \eta \textgreater{} 0 tel que t
\textless{} \eta \rigtharrow~ f(a + tv_1) \textless{}
f(a)\text et f(a + tv_2) \textgreater{}
f(a). Donc f n'a ni minimum, ni maximum en a.

Remarque~15.1.8 Dans le cas où \Phi est soit positive, soit négative, mais
non définie (c'est-à-dire que \Phi(h) peut être nul sans que h soit nul),
on ne peut pas conclure en général et il faut utiliser une formule de
Taylor à un ordre supérieur.

Exemple~15.1.4 n = 2~; soit U un ouvert de \mathbb{R}~^2 et f : U \rightarrow~ \mathbb{R}~,
(x,y)\mapsto~f(x,y). Soit (a,b) \in U. Une condition
nécessaire pour que f admette en (a,b) un extremum est que  \partial~f
\over \partial~x (a,b) = \partial~f \over \partial~y (a,b) =
0. Posons r = \partial~^2f \over \partial~x^2
(a,b), s = \partial~^2f \over \partial~x\partial~y (a,b), t =
\partial~^2f \over \partial~y^2 (a,b) (notations
de Monge). La forme quadratique \Phi est
(h,k)\mapsto~rh^2 + 2shk +
tk^2. Considérons suivant le cas le rapport  h
\over k ou le rapport  k \over h ,
on constate immédiatement à l'aide de l'étude du signe d'un trinome du
second degré que si (i) rt - s^2 \textgreater{} 0 et r
\textgreater{} 0, alors \Phi est définie positive et f a en a un minimum
local strict (ii) rt - s^2 \textgreater{} 0 et r \textless{}
0, alors \Phi est définie négative et f a en a un maximum local strict
(iii) rt - s^2 \textless{} 0, alors f a en a un point selle
(pas d'extremum local en a) (iv) rt - s^2 = 0, alors on ne
peut pas conclure.

Le lecteur comparera les surfaces z = f(x,y) ainsi que lignes de niveau
de ces surfaces dans les trois exemples ci dessous (correspondant
respectivement à un minimum local, un point selle et un point de type
(iv))

\includegraphics{cours8x.png}

{[}
{[}

\end{document}
