\documentclass[]{article}
\usepackage[T1]{fontenc}
\usepackage{lmodern}
\usepackage{amssymb,amsmath}
\usepackage{ifxetex,ifluatex}
\usepackage{fixltx2e} % provides \textsubscript
% use upquote if available, for straight quotes in verbatim environments
\IfFileExists{upquote.sty}{\usepackage{upquote}}{}
\ifnum 0\ifxetex 1\fi\ifluatex 1\fi=0 % if pdftex
  \usepackage[utf8]{inputenc}
\else % if luatex or xelatex
  \ifxetex
    \usepackage{mathspec}
    \usepackage{xltxtra,xunicode}
  \else
    \usepackage{fontspec}
  \fi
  \defaultfontfeatures{Mapping=tex-text,Scale=MatchLowercase}
  \newcommand{\euro}{€}
\fi
% use microtype if available
\IfFileExists{microtype.sty}{\usepackage{microtype}}{}
\usepackage{graphicx}
% Redefine \includegraphics so that, unless explicit options are
% given, the image width will not exceed the width of the page.
% Images get their normal width if they fit onto the page, but
% are scaled down if they would overflow the margins.
\makeatletter
\def\ScaleIfNeeded{%
  \ifdim\Gin@nat@width>\linewidth
    \linewidth
  \else
    \Gin@nat@width
  \fi
}
\makeatother
\let\Oldincludegraphics\includegraphics
{%
 \catcode`\@=11\relax%
 \gdef\includegraphics{\@ifnextchar[{\Oldincludegraphics}{\Oldincludegraphics[width=\ScaleIfNeeded]}}%
}%
\ifxetex
  \usepackage[setpagesize=false, % page size defined by xetex
              unicode=false, % unicode breaks when used with xetex
              xetex]{hyperref}
\else
  \usepackage[unicode=true]{hyperref}
\fi
\hypersetup{breaklinks=true,
            bookmarks=true,
            pdfauthor={},
            pdftitle={Derivees partielles},
            colorlinks=true,
            citecolor=blue,
            urlcolor=blue,
            linkcolor=magenta,
            pdfborder={0 0 0}}
\urlstyle{same}  % don't use monospace font for urls
\setlength{\parindent}{0pt}
\setlength{\parskip}{6pt plus 2pt minus 1pt}
\setlength{\emergencystretch}{3em}  % prevent overfull lines
\setcounter{secnumdepth}{0}
 
/* start css.sty */
.cmr-5{font-size:50%;}
.cmr-7{font-size:70%;}
.cmmi-5{font-size:50%;font-style: italic;}
.cmmi-7{font-size:70%;font-style: italic;}
.cmmi-10{font-style: italic;}
.cmsy-5{font-size:50%;}
.cmsy-7{font-size:70%;}
.cmex-7{font-size:70%;}
.cmex-7x-x-71{font-size:49%;}
.msbm-7{font-size:70%;}
.cmtt-10{font-family: monospace;}
.cmti-10{ font-style: italic;}
.cmbx-10{ font-weight: bold;}
.cmr-17x-x-120{font-size:204%;}
.cmsl-10{font-style: oblique;}
.cmti-7x-x-71{font-size:49%; font-style: italic;}
.cmbxti-10{ font-weight: bold; font-style: italic;}
p.noindent { text-indent: 0em }
td p.noindent { text-indent: 0em; margin-top:0em; }
p.nopar { text-indent: 0em; }
p.indent{ text-indent: 1.5em }
@media print {div.crosslinks {visibility:hidden;}}
a img { border-top: 0; border-left: 0; border-right: 0; }
center { margin-top:1em; margin-bottom:1em; }
td center { margin-top:0em; margin-bottom:0em; }
.Canvas { position:relative; }
li p.indent { text-indent: 0em }
.enumerate1 {list-style-type:decimal;}
.enumerate2 {list-style-type:lower-alpha;}
.enumerate3 {list-style-type:lower-roman;}
.enumerate4 {list-style-type:upper-alpha;}
div.newtheorem { margin-bottom: 2em; margin-top: 2em;}
.obeylines-h,.obeylines-v {white-space: nowrap; }
div.obeylines-v p { margin-top:0; margin-bottom:0; }
.overline{ text-decoration:overline; }
.overline img{ border-top: 1px solid black; }
td.displaylines {text-align:center; white-space:nowrap;}
.centerline {text-align:center;}
.rightline {text-align:right;}
div.verbatim {font-family: monospace; white-space: nowrap; text-align:left; clear:both; }
.fbox {padding-left:3.0pt; padding-right:3.0pt; text-indent:0pt; border:solid black 0.4pt; }
div.fbox {display:table}
div.center div.fbox {text-align:center; clear:both; padding-left:3.0pt; padding-right:3.0pt; text-indent:0pt; border:solid black 0.4pt; }
div.minipage{width:100%;}
div.center, div.center div.center {text-align: center; margin-left:1em; margin-right:1em;}
div.center div {text-align: left;}
div.flushright, div.flushright div.flushright {text-align: right;}
div.flushright div {text-align: left;}
div.flushleft {text-align: left;}
.underline{ text-decoration:underline; }
.underline img{ border-bottom: 1px solid black; margin-bottom:1pt; }
.framebox-c, .framebox-l, .framebox-r { padding-left:3.0pt; padding-right:3.0pt; text-indent:0pt; border:solid black 0.4pt; }
.framebox-c {text-align:center;}
.framebox-l {text-align:left;}
.framebox-r {text-align:right;}
span.thank-mark{ vertical-align: super }
span.footnote-mark sup.textsuperscript, span.footnote-mark a sup.textsuperscript{ font-size:80%; }
div.tabular, div.center div.tabular {text-align: center; margin-top:0.5em; margin-bottom:0.5em; }
table.tabular td p{margin-top:0em;}
table.tabular {margin-left: auto; margin-right: auto;}
div.td00{ margin-left:0pt; margin-right:0pt; }
div.td01{ margin-left:0pt; margin-right:5pt; }
div.td10{ margin-left:5pt; margin-right:0pt; }
div.td11{ margin-left:5pt; margin-right:5pt; }
table[rules] {border-left:solid black 0.4pt; border-right:solid black 0.4pt; }
td.td00{ padding-left:0pt; padding-right:0pt; }
td.td01{ padding-left:0pt; padding-right:5pt; }
td.td10{ padding-left:5pt; padding-right:0pt; }
td.td11{ padding-left:5pt; padding-right:5pt; }
table[rules] {border-left:solid black 0.4pt; border-right:solid black 0.4pt; }
.hline hr, .cline hr{ height : 1px; margin:0px; }
.tabbing-right {text-align:right;}
span.TEX {letter-spacing: -0.125em; }
span.TEX span.E{ position:relative;top:0.5ex;left:-0.0417em;}
a span.TEX span.E {text-decoration: none; }
span.LATEX span.A{ position:relative; top:-0.5ex; left:-0.4em; font-size:85%;}
span.LATEX span.TEX{ position:relative; left: -0.4em; }
div.float img, div.float .caption {text-align:center;}
div.figure img, div.figure .caption {text-align:center;}
.marginpar {width:20%; float:right; text-align:left; margin-left:auto; margin-top:0.5em; font-size:85%; text-decoration:underline;}
.marginpar p{margin-top:0.4em; margin-bottom:0.4em;}
.equation td{text-align:center; vertical-align:middle; }
td.eq-no{ width:5%; }
table.equation { width:100%; } 
div.math-display, div.par-math-display{text-align:center;}
math .texttt { font-family: monospace; }
math .textit { font-style: italic; }
math .textsl { font-style: oblique; }
math .textsf { font-family: sans-serif; }
math .textbf { font-weight: bold; }
.partToc a, .partToc, .likepartToc a, .likepartToc {line-height: 200%; font-weight:bold; font-size:110%;}
.chapterToc a, .chapterToc, .likechapterToc a, .likechapterToc, .appendixToc a, .appendixToc {line-height: 200%; font-weight:bold;}
.index-item, .index-subitem, .index-subsubitem {display:block}
.caption td.id{font-weight: bold; white-space: nowrap; }
table.caption {text-align:center;}
h1.partHead{text-align: center}
p.bibitem { text-indent: -2em; margin-left: 2em; margin-top:0.6em; margin-bottom:0.6em; }
p.bibitem-p { text-indent: 0em; margin-left: 2em; margin-top:0.6em; margin-bottom:0.6em; }
.paragraphHead, .likeparagraphHead { margin-top:2em; font-weight: bold;}
.subparagraphHead, .likesubparagraphHead { font-weight: bold;}
.quote {margin-bottom:0.25em; margin-top:0.25em; margin-left:1em; margin-right:1em; text-align:justify;}
.verse{white-space:nowrap; margin-left:2em}
div.maketitle {text-align:center;}
h2.titleHead{text-align:center;}
div.maketitle{ margin-bottom: 2em; }
div.author, div.date {text-align:center;}
div.thanks{text-align:left; margin-left:10%; font-size:85%; font-style:italic; }
div.author{white-space: nowrap;}
.quotation {margin-bottom:0.25em; margin-top:0.25em; margin-left:1em; }
h1.partHead{text-align: center}
.sectionToc, .likesectionToc {margin-left:2em;}
.subsectionToc, .likesubsectionToc {margin-left:4em;}
.subsubsectionToc, .likesubsubsectionToc {margin-left:6em;}
.frenchb-nbsp{font-size:75%;}
.frenchb-thinspace{font-size:75%;}
.figure img.graphics {margin-left:10%;}
/* end css.sty */

\title{Derivees partielles}
\author{}
\date{}

\begin{document}
\maketitle

\textbf{Warning: \href{http://www.math.union.edu/locate/jsMath}{jsMath}
requires JavaScript to process the mathematics on this page.\\ If your
browser supports JavaScript, be sure it is enabled.}

\begin{center}\rule{3in}{0.4pt}\end{center}

{[}\href{coursse83.html}{next}{]}
{[}\hyperref[tailcoursse82.html]{tail}{]}
{[}\href{coursch16.html\#coursse82.html}{up}{]}

\subsubsection{15.1 Dérivées partielles}

\paragraph{15.1.1 Notion de dérivée partielle}

Définition~15.1.1 Soit E et F deux espaces vectoriels normés. Soit U un
ouvert de E, f : U → F, a ∈ U. Soit v ∈ E
∖\textbackslash{}\{0\textbackslash{}\}. On dit que f admet au point a
une dérivée partielle suivant le vecteur v si l'application
t\textbackslash{}mathrel\{↦\}f(a + tv) (définie sur un voisinage de 0)
est dérivable au point 0.

Remarque~15.1.1 L'existence de la dérivée partielle en a suivant le
vecteur v est donc équivalente à l'existence de
\{\textbackslash{}mathop\{lim\}\}\_\{t→0\}\{ f(a+tv)−f(a)
\textbackslash{}over t\} = \{∂\}\_\{v\}f(a). Remarquons que si v' = λv,
λ\textbackslash{}mathrel\{≠\}0, alors \{ f(a+tv')−f(a)
\textbackslash{}over t\} = λ\{ f(a+uv)−f(a) \textbackslash{}over u\}
avec u = λt ce qui montre que f admet en a une dérivée partielle selon v
si et seulement si f admet une dérivée partielle suivant λv et qu'alors
\{∂\}\_\{λv\}f(a) = λ\{∂\}\_\{v\}f(a).

Exemple~15.1.1 Soit f : \{ℝ\}\^{}\{2\} → ℝ définie par f(x,y) =\{
\{x\}\^{}\{2\} \textbackslash{}over y\} si
y\textbackslash{}mathrel\{≠\}0 et f(x,0) = 0. Soit v =
(a,b)\textbackslash{}mathrel\{≠\}(0,0). On a \{ f((0,0)+tv)−f(0,0)
\textbackslash{}over t\} = \textbackslash{}left \textbackslash{}\{
\textbackslash{}cases\{ 0 \&si b = 0 \textbackslash{}cr \{
\{a\}\^{}\{2\} \textbackslash{}over b\} \&si
b\textbackslash{}mathrel\{≠\}0 \} \textbackslash{}right .. On en déduit
que f admet une dérivée partielle suivant tout vecteur v et que
\{∂\}\_\{v\}f(0,0) = \textbackslash{}left \textbackslash{}\{
\textbackslash{}cases\{ 0 \&si b = 0 \textbackslash{}cr \{
\{a\}\^{}\{2\} \textbackslash{}over b\} \&si
b\textbackslash{}mathrel\{≠\}0 \} \textbackslash{}right .. Remarquons
que l'application v\textbackslash{}mathrel\{↦\}\{∂\}\_\{v\}f(0,0) n'est
pas linéaire. Remarquons également que f n'est pas continue en (0,0)
(puisque \{\textbackslash{}mathop\{lim\}\}\_\{t→0\}f(t,\{t\}\^{}\{2\}) =
1\textbackslash{}mathrel\{≠\}f(0,0)). L'existence de dérivée partielle
suivant tout vecteur n'implique donc pas la continuité.

Proposition~15.1.1 On a les propriétés évidentes de la dérivation de
t\textbackslash{}mathrel\{↦\}f(a + tv) à savoir (i) si f et g admettent
en a une dérivée partielle suivant le vecteur v, il en est de même de αf
+ βg et \{∂\}\_\{v\}(αf + βg)(a) = α\{∂\}\_\{v\}f(a) +
β\{∂\}\_\{v\}g(a). (ii) si f et g (à valeurs scalaires) admettent en a
une dérivée partielle suivant le vecteur v, il en est de même de fg et
\{∂\}\_\{v\}(fg)(a) = g(a)\{∂\}\_\{v\}f(a) + f(a)\{∂\}\_\{v\}g(a).

Remarque~15.1.2 Par contre, on n'a pas de théorème général de
composition des dérivées partielles. En reprenant l'exemple ci dessus, f
: \{ℝ\}\^{}\{2\} → ℝ définie par f(x,y) =\{ \{x\}\^{}\{2\}
\textbackslash{}over y\} si y\textbackslash{}mathrel\{≠\}0 et f(x,0) =
0, l'application f admet en (0,0) une dérivée partielle suivant tout
vecteur, l'application t\textbackslash{}mathrel\{↦\}(t,\{t\}\^{}\{2\})
est dérivable en 0 et pourtant
t\textbackslash{}mathrel\{↦\}f(t,\{t\}\^{}\{2\}) n'est pas dérivable en
0 (elle n'y est même pas continue).

Définition~15.1.2 Soit E un espace vectoriel normé de dimension finie, ℰ
=
(\{e\}\_\{1\},\textbackslash{}mathop\{\textbackslash{}mathop\{\ldots{}\}\},\{e\}\_\{n\})
une base de E, F un espace vectoriel normé. Soit U un ouvert de E, f : U
→ F, a ∈ U. On dit que f admet au point a une dérivée partielle d'indice
i (suivant la base ℰ) si elle admet une dérivée partielle suivant le
vecteur \{e\}\_\{i\}. On note alors \{ ∂f \textbackslash{}over
∂\{x\}\_\{i\}\} (a) = \{∂\}\_\{\{e\}\_\{i\}\}f(a).

Exemple~15.1.2 Si E = \{ℝ\}\^{}\{n\} et si ℰ est la base canonique de
\{ℝ\}\^{}\{n\}, l'existence d'une dérivée partielle d'indice i au point
a =
(\{a\}\_\{1\},\textbackslash{}mathop\{\textbackslash{}mathop\{\ldots{}\}\},\{a\}\_\{n\})
équivaut à la dérivabilité au point \{a\}\_\{i\} de l'application
partielle
\{x\}\_\{i\}\textbackslash{}mathrel\{↦\}f(\{a\}\_\{1\},\textbackslash{}mathop\{\textbackslash{}mathop\{\ldots{}\}\},\{a\}\_\{i−1\},\{x\}\_\{i\},\{a\}\_\{i+1\},\textbackslash{}mathop\{\textbackslash{}mathop\{\ldots{}\}\},\{a\}\_\{n\}).
On retrouve bien la notion habituelle de dérivée partielle~: dérivée
suivant la variable \{x\}\_\{i\}, toutes les autres étant considérées
comme constantes.

\paragraph{15.1.2 Composition des dérivées partielles}

On a vu précédemment qu'on n'avait pas de théorème de composition des
dérivées partielles en toute généralité. On va introduire une notion
d'application de classe \{C\}\^{}\{1\}.

Définition~15.1.3 Soit U un ouvert de \{ℝ\}\^{}\{n\}, f : U → F. On dit
que f est de classe \{C\}\^{}\{1\} au point a si, sur un certain
voisinage V de a, f admet des dérivées partielles de tout indice i ∈
{[}1,n{]} et si ces dérivées partielles x\textbackslash{}mathrel\{↦\}\{
∂f \textbackslash{}over ∂\{x\}\_\{i\}\} (x) sont continues au point a.

Lemme~15.1.2 Soit F un espace vectoriel de dimension finie, V un ouvert
de \{ℝ\}\^{}\{n\}, f : V → F. Soit I un intervalle de ℝ, t ∈ I et φ =
(\{φ\}\_\{1\},\textbackslash{}mathop\{\textbackslash{}mathop\{\ldots{}\}\},\{φ\}\_\{n\})
: I → V . On suppose que φ est dérivable au point t et que f est de
classe \{C\}\^{}\{1\} au point φ(t). Alors f ∘ φ est dérivable au point
t et (f ∘ φ)'(t) =\{\textbackslash{}mathop\{ \textbackslash{}mathop\{∑
\}\} \}\_\{i=1\}\^{}\{n\}\{ ∂f \textbackslash{}over ∂\{x\}\_\{i\}\}
(φ(t))\{φ\}\_\{i\}'(t).

Démonstration Sans nuire à la généralité, en prenant une base (sur ℝ) de
F et en travaillant composante par composante, on peut supposer que f
est à valeurs réelles. On écrit

\textbackslash{}begin\{eqnarray*\} f(φ(t + h)) − f(φ(t))\&\& \%\&
\textbackslash{}\textbackslash{} \& =\& f(\{φ\}\_\{1\}(t +
h),\textbackslash{}mathop\{\textbackslash{}mathop\{\ldots{}\}\},\{φ\}\_\{n\}(t
+ h)) −
f(\{φ\}\_\{1\}(t),\textbackslash{}mathop\{\textbackslash{}mathop\{\ldots{}\}\},\{φ\}\_\{n\}(t))
\%\& \textbackslash{}\textbackslash{} \& =\& \{\textbackslash{}mathop\{∑
\}\}\_\{i=1\}\^{}\{n\}(f(\textbackslash{}mathop\{\ldots{}\},\{φ\}\_\{
i−1\}(t),\{φ\}\_\{i\}(t + h),\{φ\}\_\{i+1\}(t +
h),\textbackslash{}mathop\{\ldots{}\})\%\&
\textbackslash{}\textbackslash{} \& \& \textbackslash{}qquad −
f(\textbackslash{}mathop\{\textbackslash{}mathop\{\ldots{}\}\},\{φ\}\_\{i−1\}(t),\{φ\}\_\{i\}(t),\{φ\}\_\{i+1\}(t
+ h),\textbackslash{}mathop\{\textbackslash{}mathop\{\ldots{}\}\})) \%\&
\textbackslash{}\textbackslash{} \textbackslash{}end\{eqnarray*\}

Mais φ est continue au point t et donc pour h assez petit, tous les
(\{φ\}\_\{1\}(t),\textbackslash{}mathop\{\textbackslash{}mathop\{\ldots{}\}\},\{φ\}\_\{i−1\}(t),\{φ\}\_\{i\}(t
+ h),\{φ\}\_\{i+1\}(t +
h),\textbackslash{}mathop\{\textbackslash{}mathop\{\ldots{}\}\},\{φ\}\_\{n\}(t
+ h)) se trouvent à l'intérieur d'une boule de centre a sur laquelle les
dérivées partielles de f de tout indice existent. En particulier,
l'application
\{x\}\_\{i\}\textbackslash{}mathrel\{↦\}f(\{φ\}\_\{1\}(t),\textbackslash{}mathop\{\textbackslash{}mathop\{\ldots{}\}\},\{φ\}\_\{i−1\}(t),\{x\}\_\{i\},\{φ\}\_\{i+1\}(t
+
h),\textbackslash{}mathop\{\textbackslash{}mathop\{\ldots{}\}\},\{φ\}\_\{n\}(t
+ h)) est dérivable sur le segment {[}\{φ\}\_\{i\}(t),\{φ\}\_\{i\}(t +
h){]} et on peut appliquer le théorème des accroissements finis. On
obtient l'existence d'un \{ξ\}\_\{i\} ∈
{[}\{φ\}\_\{i\}(t),\{φ\}\_\{i\}(t + h){]} tel que

\textbackslash{}begin\{eqnarray*\}
f(\{φ\}\_\{1\}(t),\textbackslash{}mathop\{\textbackslash{}mathop\{\ldots{}\}\},\{φ\}\_\{i−1\}(t),\{φ\}\_\{i\}(t
+ h),\{φ\}\_\{i+1\}(t +
h),\textbackslash{}mathop\{\textbackslash{}mathop\{\ldots{}\}\},\{φ\}\_\{n\}(t
+ h))\&\& \%\& \textbackslash{}\textbackslash{} \& \&
−f(\{φ\}\_\{1\}(t),\textbackslash{}mathop\{\textbackslash{}mathop\{\ldots{}\}\},\{φ\}\_\{i−1\}(t),\{φ\}\_\{i\}(t),\{φ\}\_\{i+1\}(t
+
h),\textbackslash{}mathop\{\textbackslash{}mathop\{\ldots{}\}\},\{φ\}\_\{n\}(t
+ h))\%\& \textbackslash{}\textbackslash{} \& =\& (\{φ\}\_\{i\}(t + h) −
\{φ\}\_\{i\}(t)) \%\& \textbackslash{}\textbackslash{} \& \&
\textbackslash{}quad \{ ∂f \textbackslash{}over ∂\{x\}\_\{i\}\}
(\{φ\}\_\{1\}(t),\textbackslash{}mathop\{\textbackslash{}mathop\{\ldots{}\}\},\{φ\}\_\{i−1\}(t),\{ξ\}\_\{i\},\{φ\}\_\{i+1\}(t
+
h),\textbackslash{}mathop\{\textbackslash{}mathop\{\ldots{}\}\},\{φ\}\_\{n\}(t
+ h)) \%\& \textbackslash{}\textbackslash{}
\textbackslash{}end\{eqnarray*\}

Comme φ est continue au point t et \{ξ\}\_\{i\} ∈
{[}\{φ\}\_\{i\}(t),\{φ\}\_\{i\}(t + h){]}, on a

\textbackslash{}begin\{eqnarray*\}
\{\textbackslash{}mathop\{lim\}\}\_\{h→0\}(\{φ\}\_\{1\}(t),\textbackslash{}mathop\{\textbackslash{}mathop\{\ldots{}\}\},\{φ\}\_\{i−1\}(t),\{ξ\}\_\{i\},\{φ\}\_\{i+1\}(t
+
h),\textbackslash{}mathop\{\textbackslash{}mathop\{\ldots{}\}\},\{φ\}\_\{n\}(t
+ h))\&\&\%\& \textbackslash{}\textbackslash{} \& =\&
(\{φ\}\_\{1\}(t),\textbackslash{}mathop\{\textbackslash{}mathop\{\ldots{}\}\},\{φ\}\_\{i−1\}(t),\{φ\}\_\{i\}(t),\{φ\}\_\{i+1\}(t),\textbackslash{}mathop\{\textbackslash{}mathop\{\ldots{}\}\},\{φ\}\_\{n\}(t))\%\&
\textbackslash{}\textbackslash{} \& =\& φ(t) \%\&
\textbackslash{}\textbackslash{} \textbackslash{}end\{eqnarray*\}

et comme \{ ∂f \textbackslash{}over ∂\{x\}\_\{i\}\} est continue au
point φ(t), on a

\textbackslash{}begin\{eqnarray*\}\{ ∂f \textbackslash{}over
∂\{x\}\_\{i\}\} (φ(t))\&\& \%\& \textbackslash{}\textbackslash{} \& =\&
\{\textbackslash{}mathop\{lim\}\}\_\{h→0\}\{ ∂f \textbackslash{}over
∂\{x\}\_\{i\}\}
(\{φ\}\_\{1\}(t),\textbackslash{}mathop\{\textbackslash{}mathop\{\ldots{}\}\},\{φ\}\_\{i−1\}(t),\{ξ\}\_\{i\},\{φ\}\_\{i+1\}(t
+
h),\textbackslash{}mathop\{\textbackslash{}mathop\{\ldots{}\}\},\{φ\}\_\{n\}(t
+ h))\%\& \textbackslash{}\textbackslash{}
\textbackslash{}end\{eqnarray*\}

Il suffit alors de diviser par h et de faire tendre h vers 0 pour voir
que

\{\textbackslash{}mathop\{lim\}\}\_\{h→0\}\{ f(φ(t + h)) − f(φ(t))
\textbackslash{}over h\} =\{ \textbackslash{}mathop\{∑
\}\}\_\{i=1\}\^{}\{n\}\{ ∂f \textbackslash{}over ∂\{x\}\_\{i\}\}
(φ(t))\{φ\}\_\{i\}'(t)

ce qui achève la démonstration.

Appliquant ce lemme à φ : t\textbackslash{}mathrel\{↦\}g(a + tv) au
point t = 0, on obtient le théorème suivant

Théorème~15.1.3 Soit F un espace vectoriel de dimension finie, U un
ouvert de \{ℝ\}\^{}\{n\}, f : U → F. Soit E un espace vectoriel normé, V
un ouvert de E et g =
(\{g\}\_\{1\},\textbackslash{}mathop\{\textbackslash{}mathop\{\ldots{}\}\},\{g\}\_\{n\})
: V → U ⊂ \{ℝ\}\^{}\{n\}. Soit a ∈ V et v ∈ E
∖\textbackslash{}\{0\textbackslash{}\}. Si g admet en a une dérivée
partielle suivant le vecteur v et si f est de classe \{C\}\^{}\{1\} au
point a, alors f ∘ g admet en a une dérivée partielle suivant le vecteur
v et on a

\{∂\}\_\{v\}(f ∘ g)(a) =\{ \textbackslash{}mathop\{∑
\}\}\_\{i=1\}\^{}\{n\}\{ ∂f \textbackslash{}over ∂\{x\}\_\{i\}\}
(g(a))\{∂\}\_\{v\}\{g\}\_\{i\}(a)

Dans le cas particulier où E = \{ℝ\}\^{}\{p\} et où on prend pour v le
j-ième vecteur de la base canonique, on obtient la version suivante (on
a changé le nom des variables pour les appeler
\{y\}\_\{1\},\textbackslash{}mathop\{\textbackslash{}mathop\{\ldots{}\}\},\{y\}\_\{n\}
dans \{ℝ\}\^{}\{n\}).

Corollaire~15.1.4 Soit F un espace vectoriel de dimension finie, U un
ouvert de \{ℝ\}\^{}\{n\}, f : U → F. Soit V un ouvert de \{ℝ\}\^{}\{p\}
et g =
(\{g\}\_\{1\},\textbackslash{}mathop\{\textbackslash{}mathop\{\ldots{}\}\},\{g\}\_\{n\})
: V → U ⊂ \{ℝ\}\^{}\{n\}. Soit a ∈ V et j ∈ {[}1,p{]}. Si g admet en a
une dérivée partielle d'indice j et si f est de classe \{C\}\^{}\{1\} au
point a, alors f ∘ g admet en a une dérivée partielle d'indice j et on a

\{ ∂(f ∘ g) \textbackslash{}over ∂\{x\}\_\{j\}\} (a) =\{
\textbackslash{}mathop\{∑ \}\}\_\{i=1\}\^{}\{n\}\{ ∂f
\textbackslash{}over ∂\{y\}\_\{i\}\} (g(a))\{ ∂\{g\}\_\{i\}
\textbackslash{}over ∂\{x\}\_\{j\}\} (a)

Remarque~15.1.3 On en déduit immédiatement que la composée de deux
applications de classe \{C\}\^{}\{1\} est encore de classe
\{C\}\^{}\{1\}.

Citons aussi le corollaire suivant du lemme, où l'on prend φ(t) = a + tv

Corollaire~15.1.5 Soit F un espace vectoriel de dimension finie, V un
ouvert de \{ℝ\}\^{}\{n\}, f : V → F. Soit a ∈ V . Si f est de classe
\{C\}\^{}\{1\} au point a, alors elle admet en a des dérivées partielles
suivant tout vecteur et on a

\{∂\}\_\{v\}f(a) =\{ \textbackslash{}mathop\{∑
\}\}\_\{i=1\}\^{}\{n\}\{v\}\_\{ i\}\{ ∂f \textbackslash{}over
∂\{x\}\_\{i\}\} (a)\textbackslash{}qquad \textbackslash{}text\{ si \}v =
(\{v\}\_\{1\},\textbackslash{}mathop\{\ldots{}\},\{v\}\_\{n\})

Remarque~15.1.4 On voit que dans ce cas
v\textbackslash{}mathrel\{↦\}\{∂\}\_\{v\}f(a) est linéaire. Cette
remarque nous conduira à la définition de la différentielle dans la
section suivante.

\paragraph{15.1.3 Théorème des accroissements finis et applications}

Théorème~15.1.6 Soit U un ouvert de \{ℝ\}\^{}\{n\}, f : U → ℝ de classe
\{C\}\^{}\{1\}. Soit a ∈ U et h ∈ \{ℝ\}\^{}\{n\} tel que {[}a,a + h{]} ⊂
U. Alors, il existe θ ∈{]}0,1{[} tel que

f(a + h) − f(a) =\{ \textbackslash{}mathop\{∑
\}\}\_\{i=1\}\^{}\{n\}\{h\}\_\{ i\}\{ ∂f \textbackslash{}over
∂\{x\}\_\{i\}\} (a + θh)

Démonstration Soit ψ : {[}0,1{]} → ℝ définie par ψ(t) = f(a + th). Le
lemme du paragraphe précédent montre que ψ est dérivable sur {[}0,1{]}
et que ψ'(t) =\{\textbackslash{}mathop\{ \textbackslash{}mathop\{∑ \}\}
\}\_\{i=1\}\^{}\{n\}\{h\}\_\{i\}\{ ∂f \textbackslash{}over
∂\{x\}\_\{i\}\} (a + th). Le théorème des accroissements finis assure
qu'il existe θ ∈{]}0,1{[} tel que ψ(1) − ψ(0) = (1 − 0)ψ'(θ), ce qui
n'est autre que la formule ci dessus.

Corollaire~15.1.7 Soit U un ouvert de \{ℝ\}\^{}\{n\}, F un espace
vectoriel de dimension finie, f : U → F de classe \{C\}\^{}\{1\}. Alors
f est continue.

Démonstration En prenant une base (sur ℝ) de F et en travaillant
composante par composante, on peut supposer que f est à valeurs réelles.
Puisque les dérivées partielles sont continues au point a, il existe η
\textgreater{} 0 tel que B(a,η) ⊂ U et \textbackslash{}mathop\{∀\}x ∈
B(0,η), \textbar{}\{ ∂f \textbackslash{}over ∂\{x\}\_\{i\}\} (x) −\{ ∂f
\textbackslash{}over ∂\{x\}\_\{i\}\} (a)\textbar{}≤ 1, d'où \textbar{}\{
∂f \textbackslash{}over ∂\{x\}\_\{i\}\} (x)\textbar{}≤ 1 + \textbar{}\{
∂f \textbackslash{}over ∂\{x\}\_\{i\}\} (a)\textbar{}. Pour
\textbackslash{}\textbar{}h\textbackslash{}\textbar{} \textless{} η, on
a alors {[}a,a + h{]} ⊂ B(0,η) et donc \textbar{}f(a + h) −
f(a)\textbar{}≤\{\textbackslash{}mathop\{\textbackslash{}mathop\{∑ \}\}
\}\_\{i=1\}\^{}\{n\}\textbar{}\{h\}\_\{i\}\textbar{}\textbackslash{},\textbackslash{}left
(\textbackslash{}left \textbar{}\{ ∂f \textbackslash{}over
∂\{x\}\_\{i\}\} (a)\textbackslash{}right \textbar{} +
1\textbackslash{}right ), ce qui montre la continuité de f au point a.

Corollaire~15.1.8 Soit U un ouvert connexe de \{ℝ\}\^{}\{n\}, F un
espace vectoriel de dimension finie, f : U → F. Alors f est constante si
et seulement si~elle est de classe \{C\}\^{}\{1\} et toutes ses dérivées
partielles sont nulles.

Démonstration Si f est constante, il est clair qu'elle est de classe
\{C\}\^{}\{1\} et que toutes ses dérivées partielles sont nulles. Pour
la réciproque, en prenant une base (sur ℝ) de F et en travaillant
composante par composante, on peut supposer que f est à valeurs réelles.
Soit \{x\}\_\{0\} ∈ U et soit X = \textbackslash{}\{x ∈
U\textbackslash{}mathrel\{∣\}f(x) = f(\{x\}\_\{0\})\textbackslash{}\}.
Puisque f est continue (d'après le corollaire précédent), X est un fermé
de U, évidemment non vide. Montrons que X est également ouvert dans U~;
soit en effet \{x\}\_\{1\} dans X et soit η \textgreater{} 0 tel que
B(\{x\}\_\{1\},η) ⊂ U. Pour
\textbackslash{}\textbar{}h\textbackslash{}\textbar{} \textless{} η, on
a {[}\{x\}\_\{1\},\{x\}\_\{1\} + h{]} ⊂ B(\{x\}\_\{1\},η) ⊂ U et le
théorème des accroissements finis nous donne f(\{x\}\_\{1\} + h) =
f(\{x\}\_\{1\}) = f(\{x\}\_\{0\}), les dérivées partielles étant
supposées nulles. On a donc B(\{x\}\_\{1\},η) ⊂ X, et donc X est ouvert.
Comme X est à la fois ouvert et fermé, non vide dans U connexe, on a X =
U et donc f est constante.

\paragraph{15.1.4 Dérivées partielles successives}

On définit la notion de dérivées partielles successives de manière
récursive de la manière suivante

Définition~15.1.4 Soit U un ouvert de \{ℝ\}\^{}\{n\}, a ∈ U et f : U →
E. Soit
(\{i\}\_\{1\},\textbackslash{}mathop\{\textbackslash{}mathop\{\ldots{}\}\},\{i\}\_\{k\})
∈ \{{[}1,n{]}\}\^{}\{k\}. On dit que \{ \{∂\}\^{}\{k\}f
\textbackslash{}over
∂\{x\}\_\{\{i\}\_\{1\}\}\textbackslash{}mathop\{\textbackslash{}mathop\{\ldots{}\}\}∂\{x\}\_\{\{i\}\_\{k\}\}\}
(a) existe s'il existe un ouvert V tel que a ∈ V ⊂ U et sur lequel \{
\{∂\}\^{}\{k−1\}f \textbackslash{}over
∂\{x\}\_\{\{i\}\_\{2\}\}\textbackslash{}mathop\{\textbackslash{}mathop\{\ldots{}\}\}∂\{x\}\_\{\{i\}\_\{k\}\}\}
(x) existe et si l'application x\textbackslash{}mathrel\{↦\}\{
\{∂\}\^{}\{k−1\}f \textbackslash{}over
∂\{x\}\_\{\{i\}\_\{2\}\}\textbackslash{}mathop\{\textbackslash{}mathop\{\ldots{}\}\}∂\{x\}\_\{\{i\}\_\{k\}\}\}
(x) admet une dérivée partielle d'indice \{i\}\_\{1\}. On pose alors

\{ \{∂\}\^{}\{k\}f \textbackslash{}over
∂\{x\}\_\{\{i\}\_\{1\}\}\textbackslash{}mathop\{\textbackslash{}mathop\{\ldots{}\}\}∂\{x\}\_\{\{i\}\_\{k\}\}\}
(a) =\{ ∂ \textbackslash{}over ∂\{x\}\_\{\{i\}\_\{1\}\}\}
\textbackslash{}left (\{ \{∂\}\^{}\{k−1\}f \textbackslash{}over
∂\{x\}\_\{\{i\}\_\{2\}\}\textbackslash{}mathop\{\textbackslash{}mathop\{\ldots{}\}\}∂\{x\}\_\{\{i\}\_\{k\}\}\}
\textbackslash{}right )(a)

Définition~15.1.5 Soit U un ouvert de \{ℝ\}\^{}\{n\} et f : U → E. On
dit que f est de classe \{C\}\^{}\{k\} sur U si,
\textbackslash{}mathop\{∀\}(\{i\}\_\{1\},\textbackslash{}mathop\{\textbackslash{}mathop\{\ldots{}\}\},\{i\}\_\{k\})
∈ \{{[}1,n{]}\}\^{}\{k\}, l'application x\textbackslash{}mathrel\{↦\}\{
\{∂\}\^{}\{k\}f \textbackslash{}over
∂\{x\}\_\{\{i\}\_\{1\}\}\textbackslash{}mathop\{\textbackslash{}mathop\{\ldots{}\}\}∂\{x\}\_\{\{i\}\_\{k\}\}\}
(x) est définie et continue sur U.

Remarque~15.1.5 Comme on a vu que toute application de classe
\{C\}\^{}\{1\} est continue, on en déduit immédiatement que toute
application de classe \{C\}\^{}\{k\} est aussi de classe
\{C\}\^{}\{k−1\}. On dira bien entendu que f est de classe
\{C\}\^{}\{∞\} si elle est de classe \{C\}\^{}\{k\} pour tout k. Une
récurrence évidente sur k montre que la composée de deux applications de
classe \{C\}\^{}\{k\} est encore de classe \{C\}\^{}\{k\} et que donc la
composée de deux applications de classe \{C\}\^{}\{∞\} est encore de
classe \{C\}\^{}\{∞\}.

Lemme~15.1.9 Soit U un ouvert de \{ℝ\}\^{}\{2\}, f : U → ℝ de classe
\{C\}\^{}\{2\}. Alors, \{ \{∂\}\^{}\{2\}f \textbackslash{}over
∂\{x\}\_\{1\}∂\{x\}\_\{2\}\} =\{ \{∂\}\^{}\{2\}f \textbackslash{}over
∂\{x\}\_\{2\}∂\{x\}\_\{1\}\} .

Démonstration Soit (\{a\}\_\{1\},\{a\}\_\{2\}) ∈ U et soit

\textbackslash{}begin\{eqnarray*\} φ(\{h\}\_\{1\},\{h\}\_\{2\})\& =\&\{
1 \textbackslash{}over \{h\}\_\{1\}\{h\}\_\{2\}\} (f(\{a\}\_\{1\} +
\{h\}\_\{1\},\{a\}\_\{2\} + \{h\}\_\{2\}) − f(\{a\}\_\{1\} +
\{h\}\_\{1\},\{a\}\_\{2\})\%\& \textbackslash{}\textbackslash{} \& \&
\textbackslash{}quad \textbackslash{}quad \textbackslash{}quad −
f(\{a\}\_\{1\},\{a\}\_\{2\} + \{h\}\_\{2\}) +
f(\{a\}\_\{1\},\{a\}\_\{2\})) \%\& \textbackslash{}\textbackslash{}
\textbackslash{}end\{eqnarray*\}

définie pour \{h\}\_\{1\} et \{h\}\_\{2\} non nuls et assez petits. On a
φ(\{h\}\_\{1\},\{h\}\_\{2\}) =\{ 1 \textbackslash{}over
\{h\}\_\{1\}\{h\}\_\{2\}\} \{ψ\}\_\{1\}(\{a\}\_\{1\} + \{h\}\_\{1\}) −
\{ψ\}\_\{1\}(\{a\}\_\{1\}) avec \{ψ\}\_\{1\}(\{x\}\_\{1\}) =
f(\{x\}\_\{1\},\{a\}\_\{2\} + \{h\}\_\{2\}) −
f(\{x\}\_\{1\},\{a\}\_\{2\}). Or \{ψ\}\_\{1\} est dérivable sur
{[}\{a\}\_\{1\},\{a\}\_\{1\} + \{h\}\_\{1\}{]} avec
\{ψ\}\_\{1\}'(\{x\}\_\{1\}) =\{ ∂f \textbackslash{}over ∂\{x\}\_\{1\}\}
(\{x\}\_\{1\},\{a\}\_\{2\} + \{h\}\_\{2\}) −\{ ∂f \textbackslash{}over
∂\{x\}\_\{1\}\} (\{x\}\_\{1\},\{a\}\_\{2\}). On peut donc appliquer le
théorème des accroissements finis, et donc il existe \{ξ\}\_\{1\} ∈
{[}\{a\}\_\{1\},\{a\}\_\{1\} + \{h\}\_\{1\}{]} tel que

\textbackslash{}begin\{eqnarray*\} φ(\{h\}\_\{1\},\{h\}\_\{2\})\& =\&\{
1 \textbackslash{}over \{h\}\_\{2\}\} \{ψ\}\_\{1\}'(\{ξ\}\_\{1\}) \%\&
\textbackslash{}\textbackslash{} \& =\&\{ 1 \textbackslash{}over
\{h\}\_\{2\}\} \textbackslash{}left (\{ ∂f \textbackslash{}over
∂\{x\}\_\{1\}\} (\{ξ\}\_\{1\},\{a\}\_\{2\} + \{h\}\_\{2\}) −\{ ∂f
\textbackslash{}over ∂\{x\}\_\{1\}\}
(\{ξ\}\_\{1\},\{a\}\_\{2\})\textbackslash{}right )\%\&
\textbackslash{}\textbackslash{} \& =\&\{ \{∂\}\^{}\{2\}f
\textbackslash{}over ∂\{x\}\_\{2\}∂\{x\}\_\{1\}\}
(\{ξ\}\_\{1\},\{ξ\}\_\{2\}) \%\& \textbackslash{}\textbackslash{}
\textbackslash{}end\{eqnarray*\}

avec \{ξ\}\_\{2\} ∈ {[}\{a\}\_\{2\},\{a\}\_\{2\} + \{h\}\_\{2\}{]} en
appliquant le théorème des accroissements finis à
\{x\}\_\{2\}\textbackslash{}mathrel\{↦\}\{ ∂f \textbackslash{}over
∂\{x\}\_\{1\}\} (\{ξ\}\_\{1\},\{x\}\_\{2\}) qui est dérivable sur
{[}\{a\}\_\{2\},\{a\}\_\{2\} + \{h\}\_\{2\}{]}, de dérivée \{
\{∂\}\^{}\{2\}f \textbackslash{}over ∂\{x\}\_\{2\}∂\{x\}\_\{1\}\}
(\{ξ\}\_\{1\},\{x\}\_\{2\}). Quand \{h\}\_\{1\} et \{h\}\_\{2\} tendent
vers 0, \{ξ\}\_\{1\} et \{ξ\}\_\{2\} tendent respectivement vers
\{a\}\_\{1\} et \{a\}\_\{2\} et la continuité de \{ \{∂\}\^{}\{2\}f
\textbackslash{}over ∂\{x\}\_\{2\}∂\{x\}\_\{1\}\} montre que
\{\textbackslash{}mathop\{lim\}\}\_\{(\{h\}\_\{1\},\{h\}\_\{2\})→(0,0)\}φ(\{h\}\_\{1\},\{h\}\_\{2\})
=\{ \{∂\}\^{}\{2\}f \textbackslash{}over ∂\{x\}\_\{2\}∂\{x\}\_\{1\}\}
(\{a\}\_\{1\},\{a\}\_\{2\}). Comme les deux variables jouent un rôle
symétrique dans la définition de φ, en posant \{ψ\}\_\{2\}(\{x\}\_\{2\})
= f(\{a\}\_\{1\} + \{h\}\_\{1\},\{x\}\_\{2\}) −
f(\{a\}\_\{1\},\{x\}\_\{2\}) et en appliquant deux fois le théorème des
accroissements finis, on obtient
\{\textbackslash{}mathop\{lim\}\}\_\{(\{h\}\_\{1\},\{h\}\_\{2\})→(0,0)\}φ(\{h\}\_\{1\},\{h\}\_\{2\})
=\{ \{∂\}\^{}\{2\}f \textbackslash{}over ∂\{x\}\_\{1\}∂\{x\}\_\{2\}\}
(\{a\}\_\{1\},\{a\}\_\{2\}), ce qui démontre que \{ \{∂\}\^{}\{2\}f
\textbackslash{}over ∂\{x\}\_\{1\}∂\{x\}\_\{2\}\}
(\{a\}\_\{1\},\{a\}\_\{2\}) =\{ \{∂\}\^{}\{2\}f \textbackslash{}over
∂\{x\}\_\{2\}∂\{x\}\_\{1\}\} (\{a\}\_\{1\},\{a\}\_\{2\}).

Théorème~15.1.10 (Schwarz). Soit U un ouvert de \{ℝ\}\^{}\{n\} et f : U
→ E (espace vectoriel normé de dimension finie) de classe
\{C\}\^{}\{2\}. Alors \textbackslash{}mathop\{∀\}(i,j) ∈
\{{[}1,n{]}\}\^{}\{2\},

\{ \{∂\}\^{}\{2\}f \textbackslash{}over ∂\{x\}\_\{i\}∂\{x\}\_\{j\}\} =\{
\{∂\}\^{}\{2\}f \textbackslash{}over ∂\{x\}\_\{j\}∂\{x\}\_\{i\}\}

Démonstration En prenant une base de E, on peut se contenter de montrer
le résultat lorsque E = ℝ. Si i = j, le résultat est évident. Supposons
i \textless{} j et soit
(\{a\}\_\{1\},\textbackslash{}mathop\{\textbackslash{}mathop\{\ldots{}\}\},\{a\}\_\{n\})
∈ \{ℝ\}\^{}\{n\}. On applique le lemme précédent à l'application de
classe \{C\}\^{}\{2\}, définie sur un ouvert contenant
(\{a\}\_\{i\},\{a\}\_\{j\}),

g(\{x\}\_\{i\},\{x\}\_\{j\}) =
f(\{a\}\_\{1\},\textbackslash{}mathop\{\textbackslash{}mathop\{\ldots{}\}\},\{a\}\_\{i−1\},\{x\}\_\{i\},\{a\}\_\{i+1\},\textbackslash{}mathop\{\textbackslash{}mathop\{\ldots{}\}\},\{a\}\_\{j−1\},\{x\}\_\{j\},\{a\}\_\{j+1\},\textbackslash{}mathop\{\textbackslash{}mathop\{\ldots{}\}\},\{a\}\_\{n\})

qui est de classe \{C\}\^{}\{2\} (composée d'applications de classe
\{C\}\^{}\{2\}). On a donc \{ \{∂\}\^{}\{2\}g \textbackslash{}over
∂\{x\}\_\{i\}∂\{x\}\_\{j\}\} (\{a\}\_\{i\},\{a\}\_\{j\}) =\{
\{∂\}\^{}\{2\}g \textbackslash{}over ∂\{x\}\_\{j\}∂\{x\}\_\{i\}\}
(\{a\}\_\{i\},\{a\}\_\{j\}), soit encore

\{ \{∂\}\^{}\{2\}f \textbackslash{}over ∂\{x\}\_\{i\}∂\{x\}\_\{j\}\}
(\{a\}\_\{1\},\textbackslash{}mathop\{\textbackslash{}mathop\{\ldots{}\}\},\{a\}\_\{n\})
=\{ \{∂\}\^{}\{2\}f \textbackslash{}over ∂\{x\}\_\{j\}∂\{x\}\_\{i\}\}
(\{a\}\_\{1\},\textbackslash{}mathop\{\textbackslash{}mathop\{\ldots{}\}\},\{a\}\_\{n\})

Corollaire~15.1.11 Soit U un ouvert de \{ℝ\}\^{}\{n\} et f : U → E de
classe \{C\}\^{}\{k\}. Soit
(\{i\}\_\{1\},\textbackslash{}mathop\{\textbackslash{}mathop\{\ldots{}\}\},\{i\}\_\{k\})
∈ \{{[}1,n{]}\}\^{}\{k\}. Pour toute permutation σ de {[}1,k{]} on a

\{ \{∂\}\^{}\{k\}f \textbackslash{}over
∂\{x\}\_\{\{i\}\_\{σ(1)\}\}\textbackslash{}mathop\{\textbackslash{}mathop\{\ldots{}\}\}∂\{x\}\_\{\{i\}\_\{σ(k)\}\}\}
=\{ \{∂\}\^{}\{k\}f \textbackslash{}over
∂\{x\}\_\{\{i\}\_\{1\}\}\textbackslash{}mathop\{\textbackslash{}mathop\{\ldots{}\}\}∂\{x\}\_\{\{i\}\_\{k\}\}\}

Démonstration D'après le théorème de Schwarz, le résultat est vrai
lorsque σ = \{τ\}\_\{j,j+1\} est la transposition qui échange j et j +
1. Mais toute permutation de {[}1,k{]} est un produit de telles
transpositions (facile) ce qui démontre le corollaire.

Notation définitive Soit
(\{i\}\_\{1\},\textbackslash{}mathop\{\textbackslash{}mathop\{\ldots{}\}\},\{i\}\_\{k\})
∈ \{{[}1,n{]}\}\^{}\{k\}. Pour j ∈ {[}1,n{]}, soit \{k\}\_\{j\} le
nombre de \{i\}\_\{q\} qui sont égaux à j. On a donc à une permutation
près, la famille
(\{i\}\_\{1\},\textbackslash{}mathop\{\textbackslash{}mathop\{\ldots{}\}\},\{i\}\_\{k\})
qui est égale à
(\textbackslash{}overbrace\{1,\textbackslash{}mathop\{\textbackslash{}mathop\{\ldots{}\}\},1\}\{k\}\_\{1\}
fois,\textbackslash{}mathop\{\textbackslash{}mathop\{\ldots{}\}\},\textbackslash{}overbrace\{j,\textbackslash{}mathop\{\textbackslash{}mathop\{\ldots{}\}\},j\}
\{k\}\_\{j\}
fois,\textbackslash{}mathop\{\textbackslash{}mathop\{\ldots{}\}\},\textbackslash{}overbrace\{n,\textbackslash{}mathop\{\textbackslash{}mathop\{\ldots{}\}\},n\}\{k\}\_\{n\}
fois), chaque j étant compté \{k\}\_\{j\} fois. En notant
∂\{x\}\_\{j\}\^{}\{\{k\}\_\{j\}\} à la place de
\textbackslash{}overbrace\{∂\{x\}\_\{j\}\textbackslash{}mathop\{\textbackslash{}mathop\{\ldots{}\}\}∂\{x\}\_\{j\}\}
\{k\}\_\{j\} fois, on obtient

\{ \{∂\}\^{}\{k\}f \textbackslash{}over
∂\{x\}\_\{\{i\}\_\{1\}\}\textbackslash{}mathop\{\textbackslash{}mathop\{\ldots{}\}\}∂\{x\}\_\{\{i\}\_\{k\}\}\}
=\{ \{∂\}\^{}\{k\}f \textbackslash{}over
∂\{x\}\_\{1\}\^{}\{\{k\}\_\{1\}\}\textbackslash{}mathop\{\textbackslash{}mathop\{\ldots{}\}\}∂\{x\}\_\{n\}\^{}\{\{k\}\_\{n\}\}\}

\paragraph{15.1.5 Formules de Taylor}

Lemme~15.1.12 Soit U un ouvert de \{ℝ\}\^{}\{n\} et f : U → E de classe
\{C\}\^{}\{k\}. Soit a ∈ U et h ∈ \{ℝ\}\^{}\{n\} tel que {[}a,a + h{]} ⊂
U. Posons φ(t) = f(a + th), définie et de classe \{C\}\^{}\{k\} sur
{[}0,1{]}. Alors, pour tout t ∈ {[}0,1{]},

\textbackslash{}begin\{eqnarray*\}\{ φ\}\^{}\{(k)\}(t) =\{
\textbackslash{}mathop\{∑
\}\}\_\{\{k\}\_\{1\}+\textbackslash{}mathop\{\ldots{}\}+\{k\}\_\{n\}=k\}\{
k! \textbackslash{}over
\{k\}\_\{1\}!\textbackslash{}mathop\{\ldots{}\}\{k\}\_\{n\}!\}
\{h\}\_\{1\}\^{}\{\{k\}\_\{1\}
\}\textbackslash{}mathop\{\ldots{}\}\{h\}\_\{n\}\^{}\{\{k\}\_\{n\} \}\{
\{∂\}\^{}\{k\}f \textbackslash{}over
∂\{x\}\_\{1\}\^{}\{\{k\}\_\{1\}\}\textbackslash{}mathop\{\ldots{}\}∂\{x\}\_\{n\}\^{}\{\{k\}\_\{n\}\}\}
(a + th)\& \& \%\& \textbackslash{}\textbackslash{}
\textbackslash{}end\{eqnarray*\}

Démonstration Par récurrence sur k. Pour k = 1, ce n'est qu'une autre
formulation du résultat

\textbackslash{}begin\{eqnarray*\} φ'(t)\& =\&
\{\textbackslash{}mathop\{∑ \}\}\_\{i=1\}\^{}\{n\}\{h\}\_\{ i\}\{ ∂f
\textbackslash{}over ∂\{x\}\_\{i\}\} (a + th) \%\&
\textbackslash{}\textbackslash{} \& =\& \{\textbackslash{}mathop\{∑
\}\}\_\{\{k\}\_\{1\}+\textbackslash{}mathop\{\ldots{}\}+\{k\}\_\{n\}=1\}\{h\}\_\{1\}\^{}\{\{k\}\_\{1\}
\}\textbackslash{}mathop\{\ldots{}\}\{h\}\_\{n\}\^{}\{\{k\}\_\{n\} \}\{
∂f \textbackslash{}over
∂\{x\}\_\{1\}\^{}\{\{k\}\_\{1\}\}\textbackslash{}mathop\{\ldots{}\}∂\{x\}\_\{n\}\^{}\{\{k\}\_\{n\}\}\}
(a + th)\%\& \textbackslash{}\textbackslash{}
\textbackslash{}end\{eqnarray*\}

en posant \{k\}\_\{i\} = 1 et \{k\}\_\{j\} = 0 pour
i\textbackslash{}mathrel\{≠\}j.

Supposons le résultat démontré pour k − 1. On a donc

\textbackslash{}begin\{eqnarray*\}\{ φ\}\^{}\{(k−1)\}(t) =\&\& \%\&
\textbackslash{}\textbackslash{} \& \& \{\textbackslash{}mathop\{∑
\}\}\_\{\{k\}\_\{1\}+\textbackslash{}mathop\{\ldots{}\}+\{k\}\_\{n\}=k−1\}\{
(k − 1)! \textbackslash{}over
\{k\}\_\{1\}!\textbackslash{}mathop\{\ldots{}\}\{k\}\_\{n\}!\}
\{h\}\_\{1\}\^{}\{\{k\}\_\{1\}
\}\textbackslash{}mathop\{\ldots{}\}\{h\}\_\{n\}\^{}\{\{k\}\_\{n\} \}\{
\{∂\}\^{}\{k−1\}f \textbackslash{}over
∂\{x\}\_\{1\}\^{}\{\{k\}\_\{1\}\}\textbackslash{}mathop\{\ldots{}\}∂\{x\}\_\{n\}\^{}\{\{k\}\_\{n\}\}\}
(a + th)\%\& \textbackslash{}\textbackslash{}
\textbackslash{}end\{eqnarray*\}

On en déduit que

\textbackslash{}begin\{eqnarray*\}\{ φ\}\^{}\{(k)\}(t) =\&\& \%\&
\textbackslash{}\textbackslash{} \& \& \{\textbackslash{}mathop\{∑
\}\}\_\{\{k\}\_\{1\}+\textbackslash{}mathop\{\ldots{}\}+\{k\}\_\{n\}=k−1\}\{
(k − 1)! \textbackslash{}over
\{k\}\_\{1\}!\textbackslash{}mathop\{\ldots{}\}\{k\}\_\{n\}!\}
\{h\}\_\{1\}\^{}\{\{k\}\_\{1\}
\}\textbackslash{}mathop\{\ldots{}\}\{h\}\_\{n\}\^{}\{\{k\}\_\{n\} \}\{
d \textbackslash{}over dt\} \textbackslash{}left (\{ \{∂\}\^{}\{k−1\}f
\textbackslash{}over
∂\{x\}\_\{1\}\^{}\{\{k\}\_\{1\}\}\textbackslash{}mathop\{\ldots{}\}∂\{x\}\_\{n\}\^{}\{\{k\}\_\{n\}\}\}
(a + th)\textbackslash{}right )\%\& \textbackslash{}\textbackslash{}
\textbackslash{}end\{eqnarray*\}

soit encore

\textbackslash{}begin\{eqnarray*\}\{ φ\}\^{}\{(k)\}(t)\& =\&
\{\textbackslash{}mathop\{∑
\}\}\_\{\{k\}\_\{1\}+\textbackslash{}mathop\{\ldots{}\}+\{k\}\_\{n\}=k−1\}\{
(k − 1)! \textbackslash{}over
\{k\}\_\{1\}!\textbackslash{}mathop\{\ldots{}\}\{k\}\_\{n\}!\}
\{h\}\_\{1\}\^{}\{\{k\}\_\{1\}
\}\textbackslash{}mathop\{\ldots{}\}\{h\}\_\{n\}\^{}\{\{k\}\_\{n\} \}
\%\& \textbackslash{}\textbackslash{} \& \& \textbackslash{}quad
\textbackslash{}quad \{ \textbackslash{}mathop\{∑
\}\}\_\{i=1\}\^{}\{n\}\{h\}\_\{ i\}\{ \{∂\}\^{}\{k\}f
\textbackslash{}over
∂\{x\}\_\{1\}\^{}\{\{k\}\_\{1\}\}\textbackslash{}mathop\{\ldots{}\}∂\{x\}\_\{i\}\^{}\{\{k\}\_\{i\}+1\}\textbackslash{}mathop\{\ldots{}\}∂\{x\}\_\{n\}\^{}\{\{k\}\_\{n\}\}\}
(a + th)\%\& \textbackslash{}\textbackslash{}
\textbackslash{}end\{eqnarray*\}

En intervertissant les deux signes de somme on obtient

\textbackslash{}begin\{eqnarray*\}\{ φ\}\^{}\{(k)\}(t)\& =\&
\{\textbackslash{}mathop\{∑ \}\}\_\{i=1\}\^{}\{n\}\{
\textbackslash{}mathop\{∑
\}\}\_\{\{k\}\_\{1\}+\textbackslash{}mathop\{\ldots{}\}+\{k\}\_\{n\}=k−1\}\{
(k − 1)!(\{k\}\_\{i\} + 1) \textbackslash{}over
\{k\}\_\{1\}!\textbackslash{}mathop\{\ldots{}\}(\{k\}\_\{i\} +
1)!\textbackslash{}mathop\{\ldots{}\}\{k\}\_\{n\}!\} \%\&
\textbackslash{}\textbackslash{} \& \& \textbackslash{}quad
\textbackslash{}quad \{h\}\_\{1\}\^{}\{\{k\}\_\{1\}
\}\textbackslash{}mathop\{\textbackslash{}mathop\{\ldots{}\}\}\{h\}\_\{i\}\^{}\{\{k\}\_\{i\}+1\}\textbackslash{}mathop\{\textbackslash{}mathop\{\ldots{}\}\}\{h\}\_\{
n\}\^{}\{\{k\}\_\{n\} \}\{ \{∂\}\^{}\{k\}f \textbackslash{}over
∂\{x\}\_\{1\}\^{}\{\{k\}\_\{1\}\}\textbackslash{}mathop\{\textbackslash{}mathop\{\ldots{}\}\}∂\{x\}\_\{i\}\^{}\{\{k\}\_\{i\}+1\}\textbackslash{}mathop\{\textbackslash{}mathop\{\ldots{}\}\}∂\{x\}\_\{n\}\^{}\{\{k\}\_\{n\}\}\}
(a + th)\%\& \textbackslash{}\textbackslash{}
\textbackslash{}end\{eqnarray*\}

et en faisant un changement d'indice

\textbackslash{}begin\{eqnarray*\}\{ φ\}\^{}\{(k)\}(t)\& =\&
\{\textbackslash{}mathop\{∑ \}\}\_\{i=1\}\^{}\{n\}\{
\textbackslash{}mathop\{∑ \}\}\_\{\{
\{k\}\_\{1\}+\textbackslash{}mathop\{\ldots{}\}+\{k\}\_\{n\}=k
\textbackslash{}atop \{k\}\_\{i\}≥1\} \}\{ (k − 1)!\{k\}\_\{i\}
\textbackslash{}over
\{k\}\_\{1\}!\textbackslash{}mathop\{\ldots{}\}\{k\}\_\{n\}!\} \%\&
\textbackslash{}\textbackslash{} \& \& \textbackslash{}quad
\textbackslash{}quad \{h\}\_\{1\}\^{}\{\{k\}\_\{1\}
\}\textbackslash{}mathop\{\textbackslash{}mathop\{\ldots{}\}\}\{h\}\_\{i\}\^{}\{\{k\}\_\{i\}
\}\textbackslash{}mathop\{\textbackslash{}mathop\{\ldots{}\}\}\{h\}\_\{n\}\^{}\{\{k\}\_\{n\}
\}\{ \{∂\}\^{}\{k\}f \textbackslash{}over
∂\{x\}\_\{1\}\^{}\{\{k\}\_\{1\}\}\textbackslash{}mathop\{\textbackslash{}mathop\{\ldots{}\}\}∂\{x\}\_\{i\}\^{}\{\{k\}\_\{i\}\}\textbackslash{}mathop\{\textbackslash{}mathop\{\ldots{}\}\}∂\{x\}\_\{n\}\^{}\{\{k\}\_\{n\}\}\}
(a + th)\%\& \textbackslash{}\textbackslash{}
\textbackslash{}end\{eqnarray*\}

Réintroduisons les termes pour \{k\}\_\{i\} = 0 qui sont nuls puisqu'ils
contiennent le facteur (k − 1)!\{k\}\_\{i\}, on obtient

\textbackslash{}begin\{eqnarray*\}\{ φ\}\^{}\{(k)\}(t)\& =\&
\{\textbackslash{}mathop\{∑ \}\}\_\{i=1\}\^{}\{n\}\{
\textbackslash{}mathop\{∑
\}\}\_\{\{k\}\_\{1\}+\textbackslash{}mathop\{\ldots{}\}+\{k\}\_\{n\}=k\}\{
(k − 1)!\{k\}\_\{i\} \textbackslash{}over
\{k\}\_\{1\}!\textbackslash{}mathop\{\ldots{}\}\{k\}\_\{n\}!\}
\{h\}\_\{1\}\^{}\{\{k\}\_\{1\}
\}\textbackslash{}mathop\{\ldots{}\}\{h\}\_\{i\}\^{}\{\{k\}\_\{i\}
\}\textbackslash{}mathop\{\ldots{}\}\{h\}\_\{n\}\^{}\{\{k\}\_\{n\}
\}\%\& \textbackslash{}\textbackslash{} \& \& \textbackslash{}quad
\textbackslash{}quad \textbackslash{}quad \{ \{∂\}\^{}\{k\}f
\textbackslash{}over
∂\{x\}\_\{1\}\^{}\{\{k\}\_\{1\}\}\textbackslash{}mathop\{\textbackslash{}mathop\{\ldots{}\}\}∂\{x\}\_\{i\}\^{}\{\{k\}\_\{i\}\}\textbackslash{}mathop\{\textbackslash{}mathop\{\ldots{}\}\}∂\{x\}\_\{n\}\^{}\{\{k\}\_\{n\}\}\}
(a + th) \%\& \textbackslash{}\textbackslash{}
\textbackslash{}end\{eqnarray*\}

Ceci nous permet de réintervertir les deux sommations, soit encore,
après mise en facteur

\textbackslash{}begin\{eqnarray*\}\{ φ\}\^{}\{(k)\}(t)\& =\&
\{\textbackslash{}mathop\{∑
\}\}\_\{\{k\}\_\{1\}+\textbackslash{}mathop\{\ldots{}\}+\{k\}\_\{n\}=k\}\{
(k − 1)!\{\textbackslash{}mathop\{∑ \}\}\_\{i=1\}\^{}\{n\}\{k\}\_\{i\}
\textbackslash{}over
\{k\}\_\{1\}!\textbackslash{}mathop\{\ldots{}\}\{k\}\_\{n\}!\}
\{h\}\_\{1\}\^{}\{\{k\}\_\{1\}
\}\textbackslash{}mathop\{\ldots{}\}\{h\}\_\{i\}\^{}\{\{k\}\_\{i\}
\}\textbackslash{}mathop\{\ldots{}\}\{h\}\_\{n\}\^{}\{\{k\}\_\{n\}
\}\%\& \textbackslash{}\textbackslash{} \& \& \textbackslash{}quad
\textbackslash{}quad \textbackslash{}quad \{ \{∂\}\^{}\{k\}f
\textbackslash{}over
∂\{x\}\_\{1\}\^{}\{\{k\}\_\{1\}\}\textbackslash{}mathop\{\textbackslash{}mathop\{\ldots{}\}\}∂\{x\}\_\{i\}\^{}\{\{k\}\_\{i\}\}\textbackslash{}mathop\{\textbackslash{}mathop\{\ldots{}\}\}∂\{x\}\_\{n\}\^{}\{\{k\}\_\{n\}\}\}
(a + th) \%\& \textbackslash{}\textbackslash{}
\textbackslash{}end\{eqnarray*\}

soit encore

\textbackslash{}begin\{eqnarray*\}\{ φ\}\^{}\{(k)\}(t)\& =\&
\{\textbackslash{}mathop\{∑
\}\}\_\{\{k\}\_\{1\}+\textbackslash{}mathop\{\ldots{}\}+\{k\}\_\{n\}=k\}\{
k! \textbackslash{}over
\{k\}\_\{1\}!\textbackslash{}mathop\{\ldots{}\}\{k\}\_\{n\}!\}
\{h\}\_\{1\}\^{}\{\{k\}\_\{1\}
\}\textbackslash{}mathop\{\ldots{}\}\{h\}\_\{n\}\^{}\{\{k\}\_\{n\} \}\{
\{∂\}\^{}\{k\}f \textbackslash{}over
∂\{x\}\_\{1\}\^{}\{\{k\}\_\{1\}\}\textbackslash{}mathop\{\ldots{}\}∂\{x\}\_\{n\}\^{}\{\{k\}\_\{n\}\}\}
(a + th)\%\& \textbackslash{}\textbackslash{}
\textbackslash{}end\{eqnarray*\}

ce qui achève la récurrence.

Remarque~15.1.6 Cette formule est tout à fait analogue à la formule du
binôme généralisée

\{(\{X\}\_\{1\} +
\textbackslash{}mathop\{\textbackslash{}mathop\{\ldots{}\}\} +
\{X\}\_\{n\})\}\^{}\{k\} =\{ \textbackslash{}mathop\{∑
\}\}\_\{\{k\}\_\{1\}+\textbackslash{}mathop\{\ldots{}\}+\{k\}\_\{n\}=k\}\{
k! \textbackslash{}over
\{k\}\_\{1\}!\textbackslash{}mathop\{\ldots{}\}\{k\}\_\{n\}!\}
\{X\}\_\{1\}\^{}\{\{k\}\_\{1\}
\}\textbackslash{}mathop\{\ldots{}\}\{X\}\_\{n\}\^{}\{\{k\}\_\{n\} \}

Cette remarque nous conduira à une notation plus compacte. Introduisons
un produit symbolique sur les expressions du type
\{h\}\_\{1\}\^{}\{\{k\}\_\{1\}\}\textbackslash{}mathop\{\textbackslash{}mathop\{\ldots{}\}\}\{h\}\_\{n\}\^{}\{\{k\}\_\{n\}\}\{
\{∂\}\^{}\{k\} \textbackslash{}over
∂\{x\}\_\{1\}\^{}\{\{k\}\_\{1\}\}\textbackslash{}mathop\{\textbackslash{}mathop\{\ldots{}\}\}∂\{x\}\_\{n\}\^{}\{\{k\}\_\{n\}\}\}
en posant

\textbackslash{}begin\{eqnarray*\} \textbackslash{}left
(\{h\}\_\{1\}\^{}\{\{k\}\_\{1\}
\}\textbackslash{}mathop\{\textbackslash{}mathop\{\ldots{}\}\}\{h\}\_\{n\}\^{}\{\{k\}\_\{n\}
\}\{ \{∂\}\^{}\{k\} \textbackslash{}over
∂\{x\}\_\{1\}\^{}\{\{k\}\_\{1\}\}\textbackslash{}mathop\{\textbackslash{}mathop\{\ldots{}\}\}∂\{x\}\_\{n\}\^{}\{\{k\}\_\{n\}\}\}
\textbackslash{}right ) ∗\textbackslash{}left
(\{h\}\_\{1\}\^{}\{\{l\}\_\{1\}
\}\textbackslash{}mathop\{\textbackslash{}mathop\{\ldots{}\}\}\{h\}\_\{n\}\^{}\{\{l\}\_\{n\}
\}\{ \{∂\}\^{}\{l\} \textbackslash{}over
∂\{x\}\_\{1\}\^{}\{\{l\}\_\{1\}\}\textbackslash{}mathop\{\textbackslash{}mathop\{\ldots{}\}\}∂\{x\}\_\{n\}\^{}\{\{l\}\_\{n\}\}\}
\textbackslash{}right ) =\&\&\%\& \textbackslash{}\textbackslash{} \& \&
\{h\}\_\{1\}\^{}\{\{k\}\_\{1\}+\{l\}\_\{1\}
\}\textbackslash{}mathop\{\textbackslash{}mathop\{\ldots{}\}\}\{h\}\_\{n\}\^{}\{\{k\}\_\{n\}+\{l\}\_\{n\}
\}\{ \{∂\}\^{}\{k+l\} \textbackslash{}over
∂\{x\}\_\{1\}\^{}\{\{k\}\_\{1\}+\{l\}\_\{1\}\}\textbackslash{}mathop\{\textbackslash{}mathop\{\ldots{}\}\}∂\{x\}\_\{n\}\^{}\{\{k\}\_\{n\}+\{l\}\_\{n\}\}\}
\textbackslash{}quad \textbackslash{}quad \textbackslash{}quad \%\&
\textbackslash{}\textbackslash{} \textbackslash{}end\{eqnarray*\}

Ce produit est commutatif, et

\textbackslash{}begin\{eqnarray*\} \{\textbackslash{}mathop\{∑
\}\}\_\{\{k\}\_\{1\}+\textbackslash{}mathop\{\ldots{}\}+\{k\}\_\{n\}=k\}\{
k! \textbackslash{}over
\{k\}\_\{1\}!\textbackslash{}mathop\{\ldots{}\}\{k\}\_\{n\}!\}
\{h\}\_\{1\}\^{}\{\{k\}\_\{1\}
\}\textbackslash{}mathop\{\ldots{}\}\{h\}\_\{n\}\^{}\{\{k\}\_\{n\} \}\{
\{∂\}\^{}\{k\} \textbackslash{}over
∂\{x\}\_\{1\}\^{}\{\{k\}\_\{1\}\}\textbackslash{}mathop\{\ldots{}\}∂\{x\}\_\{n\}\^{}\{\{k\}\_\{n\}\}\}
\&\&\%\& \textbackslash{}\textbackslash{} \& \& =\{ \textbackslash{}left
(\{h\}\_\{1\}\{ ∂ \textbackslash{}over ∂\{x\}\_\{1\}\} +
\textbackslash{}mathop\{\textbackslash{}mathop\{\ldots{}\}\} +
\{h\}\_\{n\}\{ ∂ \textbackslash{}over ∂\{x\}\_\{n\}\}
\textbackslash{}right )\}\^{}\{k∗\}\textbackslash{}quad
\textbackslash{}quad \textbackslash{}quad \%\&
\textbackslash{}\textbackslash{} \textbackslash{}end\{eqnarray*\}

où la notation \{\}\^{}\{k∗\} désigne la puissance k-ième pour ce
produit commutatif. La formule s'écrit alors de manière plus agréable
sous la forme

\{φ\}\^{}\{(k)\}(t) =\{ \textbackslash{}left (\{h\}\_\{ 1\}\{ ∂
\textbackslash{}over ∂\{x\}\_\{1\}\} +
\textbackslash{}mathop\{\textbackslash{}mathop\{\ldots{}\}\} +
\{h\}\_\{n\}\{ ∂ \textbackslash{}over ∂\{x\}\_\{n\}\}
\textbackslash{}right )\}\^{}\{k∗\}f(a + th)

Ces puissances se développent de la manière évidente en respectant la
règle de calcul pour le produit ∗.

Exemple~15.1.3 φ'(t) = \textbackslash{}left (\{h\}\_\{1\}\{ ∂
\textbackslash{}over ∂\{x\}\_\{1\}\} +
\textbackslash{}mathop\{\textbackslash{}mathop\{\ldots{}\}\} +
\{h\}\_\{n\}\{ ∂ \textbackslash{}over ∂\{x\}\_\{n\}\}
\textbackslash{}right )f(a + th)

\textbackslash{}begin\{eqnarray*\} φ''(t)\& =\&\{ \textbackslash{}left
(\{h\}\_\{1\}\{ ∂ \textbackslash{}over ∂\{x\}\_\{1\}\} +
\textbackslash{}mathop\{\textbackslash{}mathop\{\ldots{}\}\} +
\{h\}\_\{n\}\{ ∂ \textbackslash{}over ∂\{x\}\_\{n\}\}
\textbackslash{}right )\}\^{}\{2∗\}f(a + th) \%\&
\textbackslash{}\textbackslash{} \& =\& \{\textbackslash{}mathop\{∑
\}\}\_\{i=1\}\^{}\{n\}\{h\}\_\{ i\}\^{}\{2\}\{ \{∂\}\^{}\{2\}f
\textbackslash{}over ∂\{x\}\_\{i\}\^{}\{2\}\} (a + th) +
2\{\textbackslash{}mathop\{∑
\}\}\_\{i\textless{}j\}\{h\}\_\{i\}\{h\}\_\{j\}\{ \{∂\}\^{}\{2\}f
\textbackslash{}over ∂\{x\}\_\{i\}∂\{x\}\_\{j\}\} (a + th)\%\&
\textbackslash{}\textbackslash{} \textbackslash{}end\{eqnarray*\}

et ainsi de suite.

Théorème~15.1.13 (formule de Taylor avec reste intégral). Soit U un
ouvert de \{ℝ\}\^{}\{n\} et f : U → E de classe \{C\}\^{}\{k+1\}. Soit a
∈ U et h ∈ \{ℝ\}\^{}\{n\} tel que {[}a,a + h{]} ⊂ U. Alors

\textbackslash{}begin\{eqnarray*\} f(a + h)\& =\& f(a) +\{
\textbackslash{}mathop\{∑ \}\}\_\{p=1\}\^{}\{k\}\{ 1
\textbackslash{}over p!\} \{\textbackslash{}left (\{h\}\_\{1\}\{ ∂
\textbackslash{}over ∂\{x\}\_\{1\}\} +
\textbackslash{}mathop\{\ldots{}\} + \{h\}\_\{n\}\{ ∂
\textbackslash{}over ∂\{x\}\_\{n\}\} \textbackslash{}right
)\}\^{}\{p∗\}f(a)\%\& \textbackslash{}\textbackslash{}
+\{\textbackslash{}mathop\{∫ \} \}\_\{0\}\^{}\{1\}\{ \{(1 −
t)\}\^{}\{k\} \textbackslash{}over k!\} \{ \textbackslash{}left
(\{h\}\_\{1\}\{ ∂ \textbackslash{}over ∂\{x\}\_\{1\}\} +
\textbackslash{}mathop\{\textbackslash{}mathop\{\ldots{}\}\} +
\{h\}\_\{n\}\{ ∂ \textbackslash{}over ∂\{x\}\_\{n\}\}
\textbackslash{}right )\}\^{}\{(k+1)∗\}f(a + th) dt\&\&\%\&
\textbackslash{}\textbackslash{} \textbackslash{}end\{eqnarray*\}

Démonstration C'est simplement la formule de Taylor avec reste intégral
pour la fonction φ~:

φ(1) = φ(0) +\{ \textbackslash{}mathop\{∑ \}\}\_\{p=1\}\^{}\{k\}\{ 1
\textbackslash{}over p!\} \{φ\}\^{}\{(p)\}(0) +\{
\textbackslash{}mathop\{\textbackslash{}mathop\{∫ \} \}
\}\_\{0\}\^{}\{1\}\{ \{(1 − t)\}\^{}\{k\} \textbackslash{}over k!\}
\{φ\}\^{}\{(k+1)\}(t) dt

Remarque~15.1.7 On utilisera le plus souvent cette formule pour k = 1~;
dans cas d'une fonction définie sur un ouvert de \{ℝ\}\^{}\{2\} on
obtiendra par exemple

\textbackslash{}begin\{eqnarray*\} f(a + h)\& =\& f(a) + \{h\}\_\{1\}\{
∂f \textbackslash{}over ∂\{x\}\_\{1\}\} (a) + \{h\}\_\{2\}\{ ∂f
\textbackslash{}over ∂\{x\}\_\{2\}\} (a) \%\&
\textbackslash{}\textbackslash{} \& \& \textbackslash{}quad +
\{h\}\_\{1\}\^{}\{2\}\{\textbackslash{}mathop\{∫ \} \}\_\{0\}\^{}\{1\}(1
− t)\{ \{∂\}\^{}\{2\}f \textbackslash{}over ∂\{x\}\_\{1\}\^{}\{2\}\} (a
+ th) dt \%\& \textbackslash{}\textbackslash{} \& \&
\textbackslash{}quad + \{h\}\_\{2\}\^{}\{2\}\{\textbackslash{}mathop\{∫
\} \}\_\{0\}\^{}\{1\}(1 − t)\{ \{∂\}\^{}\{2\}f \textbackslash{}over
∂\{x\}\_\{2\}\^{}\{2\}\} (a + th) dt \%\&
\textbackslash{}\textbackslash{} \& \& \textbackslash{}quad +
2\{h\}\_\{1\}\{h\}\_\{2\}\{\textbackslash{}mathop\{∫ \}
\}\_\{0\}\^{}\{1\}(1 − t)\{ \{∂\}\^{}\{2\}f \textbackslash{}over
∂\{x\}\_\{1\}∂\{x\}\_\{2\}\} (a + th) dt\%\&
\textbackslash{}\textbackslash{} \textbackslash{}end\{eqnarray*\}

Théorème~15.1.14 (formule de Taylor-Lagrange). Soit U un ouvert de
\{ℝ\}\^{}\{n\} et f : U → ℝ de classe \{C\}\^{}\{k+1\}. Soit a ∈ U et h
∈ \{ℝ\}\^{}\{n\} tel que {[}a,a + h{]} ⊂ U. Alors, il existe θ
∈{]}0,1{[} tel que

\textbackslash{}begin\{eqnarray*\} f(a + h)\& =\& f(a) +\{
\textbackslash{}mathop\{∑ \}\}\_\{p=1\}\^{}\{k\}\{ 1
\textbackslash{}over p!\} \{\textbackslash{}left (\{h\}\_\{1\}\{ ∂
\textbackslash{}over ∂\{x\}\_\{1\}\} +
\textbackslash{}mathop\{\ldots{}\} + \{h\}\_\{n\}\{ ∂
\textbackslash{}over ∂\{x\}\_\{n\}\} \textbackslash{}right
)\}\^{}\{p∗\}f(a) \%\& \textbackslash{}\textbackslash{} \& \& +\{ 1
\textbackslash{}over (k + 1)!\} \{\textbackslash{}left (\{h\}\_\{1\}\{ ∂
\textbackslash{}over ∂\{x\}\_\{1\}\} +
\textbackslash{}mathop\{\textbackslash{}mathop\{\ldots{}\}\} +
\{h\}\_\{n\}\{ ∂ \textbackslash{}over ∂\{x\}\_\{n\}\}
\textbackslash{}right )\}\^{}\{(k+1)∗\}f(a + θh)\%\&
\textbackslash{}\textbackslash{} \textbackslash{}end\{eqnarray*\}

Démonstration C'est simplement la formule de Taylor Lagrange pour la
fonction φ~:

φ(1) = φ(0) +\{ \textbackslash{}mathop\{∑ \}\}\_\{p=1\}\^{}\{k\}\{ 1
\textbackslash{}over p!\} \{φ\}\^{}\{(p)\}(0) +\{ 1 \textbackslash{}over
(k + 1)!\} \{φ\}\^{}\{(k+1)\}(θ)

Théorème~15.1.15 (formule de Taylor-Young). Soit U un ouvert de
\{ℝ\}\^{}\{n\} et f : U → E (espace vectoriel normé de dimension finie)
de classe \{C\}\^{}\{k\}. Soit a ∈ U. Alors, quand h tend vers 0 on a

f(a + h) = f(a) +\{ \textbackslash{}mathop\{∑ \}\}\_\{p=1\}\^{}\{k\}\{ 1
\textbackslash{}over p!\} \{\textbackslash{}left (\{h\}\_\{1\}\{ ∂
\textbackslash{}over ∂\{x\}\_\{1\}\} +
\textbackslash{}mathop\{\ldots{}\} + \{h\}\_\{n\}\{ ∂
\textbackslash{}over ∂\{x\}\_\{n\}\} \textbackslash{}right
)\}\^{}\{p∗\}f(a) +
o(\textbackslash{}\textbar{}\{h\textbackslash{}\textbar{}\}\^{}\{k\})

Démonstration Quitte à prendre une base de E et à travailler composante
par composante, on peut supposer que E = ℝ~; toutes les normes sur
\{ℝ\}\^{}\{n\} étant équivalentes, on peut supposer que
\textbackslash{}\textbar{}h\textbackslash{}\textbar{} =
\textbar{}\{h\}\_\{1\}\textbar{} +
\textbackslash{}mathop\{\textbackslash{}mathop\{\ldots{}\}\} +
\textbar{}\{h\}\_\{n\}\textbar{}. Soit ρ \textgreater{} 0 tel que B(a,ρ)
⊂ U et soit h tel que
\textbackslash{}\textbar{}h\textbackslash{}\textbar{} \textless{} ρ. On
a alors {[}a,a + h{]} ⊂ B(a,ρ) ⊂ U~; on peut donc appliquer la formule
de Taylor-Lagrange à l'ordre k − 1 qui nous donne

\textbackslash{}begin\{eqnarray*\} f(a + h)\& −\& f(a)
−\{\textbackslash{}mathop\{∑ \}\}\_\{p=1\}\^{}\{k\}\{ 1
\textbackslash{}over p!\} \{\textbackslash{}left (\{h\}\_\{1\}\{ ∂
\textbackslash{}over ∂\{x\}\_\{1\}\} +
\textbackslash{}mathop\{\ldots{}\} + \{h\}\_\{n\}\{ ∂
\textbackslash{}over ∂\{x\}\_\{n\}\} \textbackslash{}right
)\}\^{}\{p∗\}f(a)\%\& \textbackslash{}\textbackslash{} \& =\&\{ 1
\textbackslash{}over k!\} \{\textbackslash{}left (\{h\}\_\{1\}\{ ∂
\textbackslash{}over ∂\{x\}\_\{1\}\} +
\textbackslash{}mathop\{\textbackslash{}mathop\{\ldots{}\}\} +
\{h\}\_\{n\}\{ ∂ \textbackslash{}over ∂\{x\}\_\{n\}\}
\textbackslash{}right )\}\^{}\{k∗\}f(a + θh) \%\&
\textbackslash{}\textbackslash{} \& −\&\{ 1 \textbackslash{}over k!\}
\{\textbackslash{}left (\{h\}\_\{1\}\{ ∂ \textbackslash{}over
∂\{x\}\_\{1\}\} +
\textbackslash{}mathop\{\textbackslash{}mathop\{\ldots{}\}\} +
\{h\}\_\{n\}\{ ∂ \textbackslash{}over ∂\{x\}\_\{n\}\}
\textbackslash{}right )\}\^{}\{k∗\}f(a) \%\&
\textbackslash{}\textbackslash{} \textbackslash{}end\{eqnarray*\}

Mais les dérivées partielles de f sont continues. Soit ε \textgreater{}
0~; il existe η \textgreater{} 0 tel que

\textbackslash{}begin\{eqnarray*\}
\textbackslash{}\textbar{}h\textbackslash{}\textbar{} \textless{} η\&
⇒\&
\textbackslash{}mathop\{∀\}(\{k\}\_\{1\},\textbackslash{}mathop\{\textbackslash{}mathop\{\ldots{}\}\},\{k\}\_\{n\})\textbackslash{}text\{
tel que \}\{k\}\_\{1\} +
\textbackslash{}mathop\{\textbackslash{}mathop\{\ldots{}\}\} +
\{k\}\_\{n\} = k, \textbackslash{}mathop\{∀\}t ∈ {[}0,1{]} \%\&
\textbackslash{}\textbackslash{} \& \& \textbackslash{}left \textbar{}\{
\{∂\}\^{}\{k\}f \textbackslash{}over
∂\{x\}\_\{1\}\^{}\{\{k\}\_\{1\}\}\textbackslash{}mathop\{\textbackslash{}mathop\{\ldots{}\}\}∂\{x\}\_\{n\}\^{}\{\{k\}\_\{n\}\}\}
(a + th)\textbackslash{}right . −\textbackslash{}left .\{
\{∂\}\^{}\{k\}f \textbackslash{}over
∂\{x\}\_\{1\}\^{}\{\{k\}\_\{1\}\}\textbackslash{}mathop\{\textbackslash{}mathop\{\ldots{}\}\}∂\{x\}\_\{n\}\^{}\{\{k\}\_\{n\}\}\}
(a)\textbackslash{}right \textbar{} \textless{} ε\%\&
\textbackslash{}\textbackslash{} \textbackslash{}end\{eqnarray*\}

Pour \textbackslash{}\textbar{}h\textbackslash{}\textbar{} \textless{}
η, on a alors (en développant les deux puissances symboliques)

\textbackslash{}begin\{eqnarray*\} \textbackslash{}big
\textbar{}\{\textbackslash{}left (\textbackslash{}mathop\{∑
\}\{h\}\_\{i\}\{ ∂ \textbackslash{}over ∂\{x\}\_\{i\}\}
\textbackslash{}right )\}\^{}\{k∗\}f(a + θh) −\{\textbackslash{}left
(\textbackslash{}mathop\{∑ \}\{h\}\_\{i\}\{ ∂ \textbackslash{}over
∂\{x\}\_\{i\}\} \textbackslash{}right
)\}\^{}\{k∗\}f(a)\textbackslash{}big \textbar{}\&\&\%\&
\textbackslash{}\textbackslash{} \& \textless{}\&
ε\{\textbackslash{}mathop\{∑
\}\}\_\{\{k\}\_\{1\}+\textbackslash{}mathop\{\ldots{}\}+\{k\}\_\{n\}=k\}\{
k! \textbackslash{}over
\{k\}\_\{1\}!\textbackslash{}mathop\{\ldots{}\}\{k\}\_\{n\}!\}
\textbar{}\{h\}\_\{1\}\{\textbar{}\}\^{}\{\{k\}\_\{1\}
\}\textbackslash{}mathop\{\ldots{}\}\textbar{}\{h\}\_\{n\}\{\textbar{}\}\^{}\{\{k\}\_\{n\}
\}\%\& \textbackslash{}\textbackslash{} \& =\&
ε\{(\textbar{}\{h\}\_\{1\}\textbar{} +
\textbackslash{}mathop\{\textbackslash{}mathop\{\ldots{}\}\} +
\textbar{}\{h\}\_\{n\}\textbar{})\}\^{}\{k\} =
ε\textbackslash{}\textbar{}\{h\textbackslash{}\textbar{}\}\^{}\{k\} \%\&
\textbackslash{}\textbackslash{} \textbackslash{}end\{eqnarray*\}

ce qui démontre le résultat.

\paragraph{15.1.6 Application aux extremums de fonctions de plusieurs
variables}

Soit U un ouvert de \{ℝ\}\^{}\{n\} et f : U → ℝ. Nous allons rechercher
les extremums de la fonction f à l'aide des résultats qui suivent.

Proposition~15.1.16 Soit U un ouvert de \{ℝ\}\^{}\{n\} et f : U → ℝ de
classe \{C\}\^{}\{1\}. Soit a ∈ U. Si f admet en a un extremum local, on
a \textbackslash{}mathop\{∀\}i ∈ {[}1,n{]},\{ ∂f \textbackslash{}over
∂\{x\}\_\{i\}\} (a) = 0.

Démonstration Il suffit de remarquer que la fonction
t\textbackslash{}mathrel\{↦\}f(a + t\{e\}\_\{i\}) (définie sur un
voisinage de 0) admet en 0 un extremum local. On a donc

\{ ∂f \textbackslash{}over ∂\{x\}\_\{i\}\} (a) =\{ d
\textbackslash{}over dt\} \{\textbackslash{}left (f(a +
t\{e\}\_\{i\})\textbackslash{}right )\}\_\{t=0\} = 0

Dans le cas des fonctions d'une variable, la condition ci dessus n'est
déjà pas suffisante (considérer
x\textbackslash{}mathrel\{↦\}\{x\}\^{}\{3\} au point 0). Il est clair
qu'il en est de même a fortiori pour une fonction de plusieurs
variables. Pour obtenir des résultats plus précis et en particulier des
conditions suffisantes d'extremums, nous allons introduire une forme
quadratique sur \{ℝ\}\^{}\{n\}

Définition~15.1.6 Soit U un ouvert de \{ℝ\}\^{}\{n\} et f : U → ℝ de
classe \{C\}\^{}\{2\}. Soit a ∈ U. On appelle différentielle seconde au
point a la forme quadratique sur \{ℝ\}\^{}\{n\},

\textbackslash{}begin\{eqnarray*\} h\& =\&
(\{h\}\_\{1\},\textbackslash{}mathop\{\textbackslash{}mathop\{\ldots{}\}\},\{h\}\_\{n\})\textbackslash{}mathrel\{↦\}\{\textbackslash{}left
(\{h\}\_\{1\}\{ ∂ \textbackslash{}over ∂\{x\}\_\{1\}\} +
\textbackslash{}mathop\{\textbackslash{}mathop\{\ldots{}\}\} +
\{h\}\_\{n\}\{ ∂ \textbackslash{}over ∂\{x\}\_\{n\}\}
\textbackslash{}right )\}\^{}\{2∗\}f(a) \%\&
\textbackslash{}\textbackslash{} \& \& =\{ \textbackslash{}mathop\{∑
\}\}\_\{i=1\}\^{}\{n\}\{h\}\_\{ i\}\^{}\{2\}\{ \{∂\}\^{}\{2\}f
\textbackslash{}over ∂\{x\}\_\{i\}\^{}\{2\}\} (a) +
2\{\textbackslash{}mathop\{∑
\}\}\_\{i\textless{}j\}\{h\}\_\{i\}\{h\}\_\{j\}\{ \{∂\}\^{}\{2\}f
\textbackslash{}over ∂\{x\}\_\{i\}∂\{x\}\_\{j\}\} (a)\%\&
\textbackslash{}\textbackslash{} \textbackslash{}end\{eqnarray*\}

Théorème~15.1.17 Soit U un ouvert de \{ℝ\}\^{}\{n\} et f : U → ℝ de
classe \{C\}\^{}\{2\}. Soit a ∈ U tel que \textbackslash{}mathop\{∀\}i ∈
{[}1,n{]},\{ ∂f \textbackslash{}over ∂\{x\}\_\{i\}\} (a) = 0 et soit Φ
la forme quadratique différentielle seconde au point a. Alors (i) si Φ
est définie positive, c'est-à-dire si h\textbackslash{}mathrel\{≠\}0 ⇒
Φ(h) \textgreater{} 0, alors f admet en a un minimum local strict (ii)
si Φ est définie négative, c'est-à-dire si
h\textbackslash{}mathrel\{≠\}0 ⇒ Φ(h) \textless{} 0, alors f admet en a
un maximum local strict (iii) si Φ n'est ni positive ni négative, alors
f n'admet pas d'extremum en a (on dit dans ce cas que a est un point
selle ou point col de a, par analogie avec une selle de cheval ou un col
de montagne).

Démonstration (i). Utilisons la formule de Taylor Young à l'ordre 2. On
a donc, en tenant compte de \{ ∂f \textbackslash{}over ∂\{x\}\_\{i\}\}
(a) = 0, f(a + h) = f(a) +\{ 1 \textbackslash{}over 2\} Φ(h)
+\textbackslash{}\textbar{}
\{h\textbackslash{}\textbar{}\}\^{}\{2\}ε(h), avec
\{\textbackslash{}mathop\{lim\}\}\_\{h→0\}ε(h) = 0. Pour démontrer (i),
nous allons utiliser le lemme suivant

Lemme~15.1.18 Soit Φ une forme quadratique définie positive sur
\{ℝ\}\^{}\{n\} (ou tout espace vectoriel normé de dimension finie).
Alors \textbackslash{}mathop\{∃\}α \textgreater{} 0,
\textbackslash{}mathop\{∀\}h ∈ \{ℝ\}\^{}\{n\}, Φ(h) ≥
α\textbackslash{}\textbar{}\{h\textbackslash{}\textbar{}\}\^{}\{2\}.

Démonstration Soit S la sphère unité de \{ℝ\}\^{}\{n\}. Comme Φ est
continue sur S qui est compact, Φ atteint sur S sa borne inférieure α.
Soit donc \{x\}\_\{0\} ∈ S tel que Φ(\{x\}\_\{0\}) = α
=\{\textbackslash{}mathop\{ inf\} \}\_\{x∈S\}Φ(x). Comme
\{x\}\_\{0\}\textbackslash{}mathrel\{≠\}0, on a α \textgreater{} 0. De
plus, si h\textbackslash{}mathrel\{≠\}0, on a \{ h \textbackslash{}over
\textbackslash{}\textbar{}h\textbackslash{}\textbar{}\} ∈ S, soit Φ(\{ h
\textbackslash{}over
\textbackslash{}\textbar{}h\textbackslash{}\textbar{}\} ) ≥ α soit \{
Φ(h) \textbackslash{}over
\textbackslash{}\textbar{}\{h\textbackslash{}\textbar{}\}\^{}\{2\}\} ≥
α, soit encore Φ(h) ≥
α\textbackslash{}\textbar{}\{h\textbackslash{}\textbar{}\}\^{}\{2\}.

Puisque \{\textbackslash{}mathop\{lim\}\}\_\{h→0\}ε(h) = 0, il existe η
\textgreater{} 0 tel que
\textbackslash{}\textbar{}h\textbackslash{}\textbar{} \textless{} η
⇒\textbar{}ε(h)\textbar{}≤\{ α \textbackslash{}over 4\} . Pour
\textbackslash{}\textbar{}h\textbackslash{}\textbar{} \textless{} η, on
a donc

\textbackslash{}begin\{eqnarray*\} f(a + h) − f(a)\& =\&\{ 1
\textbackslash{}over 2\} Φ(h) +\textbackslash{}\textbar{}
\{h\textbackslash{}\textbar{}\}\^{}\{2\}ε(h) \%\&
\textbackslash{}\textbackslash{} \& ≥\&\{ α \textbackslash{}over 2\}
\textbackslash{}\textbar{}\{h\textbackslash{}\textbar{}\}\^{}\{2\} −\{ α
\textbackslash{}over 4\}
\textbackslash{}\textbar{}\{h\textbackslash{}\textbar{}\}\^{}\{2\} =\{ α
\textbackslash{}over 4\}
\textbackslash{}\textbar{}\{h\textbackslash{}\textbar{}\}\^{}\{2\}
\textgreater{} 0\%\& \textbackslash{}\textbackslash{}
\textbackslash{}end\{eqnarray*\}

pour h\textbackslash{}mathrel\{≠\}0. Donc f admet en a un minimum local
strict.

Pour démontrer (ii) à partir de (i), il suffit de changer f en − f.

(iii). Si Φ n'est ni positive, ni négative, il existe \{v\}\_\{1\} ∈
\{ℝ\}\^{}\{n\} tel que Φ(\{v\}\_\{1\}) \textless{} 0 et il existe
\{v\}\_\{2\} ∈ \{ℝ\}\^{}\{n\} tel que Φ(\{v\}\_\{2\}) \textgreater{} 0.
On a alors, d'après la même formule de Taylor, en posant h =
t\{v\}\_\{i\}, f(a + t\{v\}\_\{i\}) = f(a) +\{ 1 \textbackslash{}over
2\} Φ(t\{v\}\_\{i\}) +
\{t\}\^{}\{2\}\textbackslash{}\textbar{}\{v\{\}\_\{
i\}\textbackslash{}\textbar{}\}\^{}\{2\}ε(t\{v\}\_\{ i\}) = f(a) +\{
\{t\}\^{}\{2\} \textbackslash{}over 2\} Φ(\{v\}\_\{i\}) +
\{t\}\^{}\{2\}\{ε\}\_\{ i\}(t) avec
\{\textbackslash{}mathop\{lim\}\}\_\{t→0\}\{ε\}\_\{i\}(t) = 0. On en
déduit qu'il existe un η \textgreater{} 0 tel que \textbar{}t\textbar{}
\textless{} η ⇒ f(a + t\{v\}\_\{1\}) \textless{}
f(a)\textbackslash{}text\{ et \}f(a + t\{v\}\_\{2\}) \textgreater{}
f(a). Donc f n'a ni minimum, ni maximum en a.

Remarque~15.1.8 Dans le cas où Φ est soit positive, soit négative, mais
non définie (c'est-à-dire que Φ(h) peut être nul sans que h soit nul),
on ne peut pas conclure en général et il faut utiliser une formule de
Taylor à un ordre supérieur.

Exemple~15.1.4 n = 2~; soit U un ouvert de \{ℝ\}\^{}\{2\} et f : U → ℝ,
(x,y)\textbackslash{}mathrel\{↦\}f(x,y). Soit (a,b) ∈ U. Une condition
nécessaire pour que f admette en (a,b) un extremum est que \{ ∂f
\textbackslash{}over ∂x\} (a,b) =\{ ∂f \textbackslash{}over ∂y\} (a,b) =
0. Posons r =\{ \{∂\}\^{}\{2\}f \textbackslash{}over ∂\{x\}\^{}\{2\}\}
(a,b), s =\{ \{∂\}\^{}\{2\}f \textbackslash{}over ∂x∂y\} (a,b), t =\{
\{∂\}\^{}\{2\}f \textbackslash{}over ∂\{y\}\^{}\{2\}\} (a,b) (notations
de Monge). La forme quadratique Φ est
(h,k)\textbackslash{}mathrel\{↦\}r\{h\}\^{}\{2\} + 2shk +
t\{k\}\^{}\{2\}. Considérons suivant le cas le rapport \{ h
\textbackslash{}over k\} ou le rapport \{ k \textbackslash{}over h\} ,
on constate immédiatement à l'aide de l'étude du signe d'un trinome du
second degré que si (i) rt − \{s\}\^{}\{2\} \textgreater{} 0 et r
\textgreater{} 0, alors Φ est définie positive et f a en a un minimum
local strict (ii) rt − \{s\}\^{}\{2\} \textgreater{} 0 et r \textless{}
0, alors Φ est définie négative et f a en a un maximum local strict
(iii) rt − \{s\}\^{}\{2\} \textless{} 0, alors f a en a un point selle
(pas d'extremum local en a) (iv) rt − \{s\}\^{}\{2\} = 0, alors on ne
peut pas conclure.

Le lecteur comparera les surfaces z = f(x,y) ainsi que lignes de niveau
de ces surfaces dans les trois exemples ci dessous (correspondant
respectivement à un minimum local, un point selle et un point de type
(iv))

\includegraphics{cours8x.png}

{[}\href{coursse83.html}{next}{]} {[}\href{coursse82.html}{front}{]}
{[}\href{coursch16.html\#coursse82.html}{up}{]}

\end{document}
