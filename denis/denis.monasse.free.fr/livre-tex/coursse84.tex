\documentclass[]{article}
\usepackage[T1]{fontenc}
\usepackage{lmodern}
\usepackage{amssymb,amsmath}
\usepackage{ifxetex,ifluatex}
\usepackage{fixltx2e} % provides \textsubscript
% use upquote if available, for straight quotes in verbatim environments
\IfFileExists{upquote.sty}{\usepackage{upquote}}{}
\ifnum 0\ifxetex 1\fi\ifluatex 1\fi=0 % if pdftex
  \usepackage[utf8]{inputenc}
\else % if luatex or xelatex
  \ifxetex
    \usepackage{mathspec}
    \usepackage{xltxtra,xunicode}
  \else
    \usepackage{fontspec}
  \fi
  \defaultfontfeatures{Mapping=tex-text,Scale=MatchLowercase}
  \newcommand{\euro}{€}
\fi
% use microtype if available
\IfFileExists{microtype.sty}{\usepackage{microtype}}{}
\ifxetex
  \usepackage[setpagesize=false, % page size defined by xetex
              unicode=false, % unicode breaks when used with xetex
              xetex]{hyperref}
\else
  \usepackage[unicode=true]{hyperref}
\fi
\hypersetup{breaklinks=true,
            bookmarks=true,
            pdfauthor={},
            pdftitle={Formes differentielles},
            colorlinks=true,
            citecolor=blue,
            urlcolor=blue,
            linkcolor=magenta,
            pdfborder={0 0 0}}
\urlstyle{same}  % don't use monospace font for urls
\setlength{\parindent}{0pt}
\setlength{\parskip}{6pt plus 2pt minus 1pt}
\setlength{\emergencystretch}{3em}  % prevent overfull lines
\setcounter{secnumdepth}{0}
 
/* start css.sty */
.cmr-5{font-size:50%;}
.cmr-7{font-size:70%;}
.cmmi-5{font-size:50%;font-style: italic;}
.cmmi-7{font-size:70%;font-style: italic;}
.cmmi-10{font-style: italic;}
.cmsy-5{font-size:50%;}
.cmsy-7{font-size:70%;}
.cmex-7{font-size:70%;}
.cmex-7x-x-71{font-size:49%;}
.msbm-7{font-size:70%;}
.cmtt-10{font-family: monospace;}
.cmti-10{ font-style: italic;}
.cmbx-10{ font-weight: bold;}
.cmr-17x-x-120{font-size:204%;}
.cmsl-10{font-style: oblique;}
.cmti-7x-x-71{font-size:49%; font-style: italic;}
.cmbxti-10{ font-weight: bold; font-style: italic;}
p.noindent { text-indent: 0em }
td p.noindent { text-indent: 0em; margin-top:0em; }
p.nopar { text-indent: 0em; }
p.indent{ text-indent: 1.5em }
@media print {div.crosslinks {visibility:hidden;}}
a img { border-top: 0; border-left: 0; border-right: 0; }
center { margin-top:1em; margin-bottom:1em; }
td center { margin-top:0em; margin-bottom:0em; }
.Canvas { position:relative; }
li p.indent { text-indent: 0em }
.enumerate1 {list-style-type:decimal;}
.enumerate2 {list-style-type:lower-alpha;}
.enumerate3 {list-style-type:lower-roman;}
.enumerate4 {list-style-type:upper-alpha;}
div.newtheorem { margin-bottom: 2em; margin-top: 2em;}
.obeylines-h,.obeylines-v {white-space: nowrap; }
div.obeylines-v p { margin-top:0; margin-bottom:0; }
.overline{ text-decoration:overline; }
.overline img{ border-top: 1px solid black; }
td.displaylines {text-align:center; white-space:nowrap;}
.centerline {text-align:center;}
.rightline {text-align:right;}
div.verbatim {font-family: monospace; white-space: nowrap; text-align:left; clear:both; }
.fbox {padding-left:3.0pt; padding-right:3.0pt; text-indent:0pt; border:solid black 0.4pt; }
div.fbox {display:table}
div.center div.fbox {text-align:center; clear:both; padding-left:3.0pt; padding-right:3.0pt; text-indent:0pt; border:solid black 0.4pt; }
div.minipage{width:100%;}
div.center, div.center div.center {text-align: center; margin-left:1em; margin-right:1em;}
div.center div {text-align: left;}
div.flushright, div.flushright div.flushright {text-align: right;}
div.flushright div {text-align: left;}
div.flushleft {text-align: left;}
.underline{ text-decoration:underline; }
.underline img{ border-bottom: 1px solid black; margin-bottom:1pt; }
.framebox-c, .framebox-l, .framebox-r { padding-left:3.0pt; padding-right:3.0pt; text-indent:0pt; border:solid black 0.4pt; }
.framebox-c {text-align:center;}
.framebox-l {text-align:left;}
.framebox-r {text-align:right;}
span.thank-mark{ vertical-align: super }
span.footnote-mark sup.textsuperscript, span.footnote-mark a sup.textsuperscript{ font-size:80%; }
div.tabular, div.center div.tabular {text-align: center; margin-top:0.5em; margin-bottom:0.5em; }
table.tabular td p{margin-top:0em;}
table.tabular {margin-left: auto; margin-right: auto;}
div.td00{ margin-left:0pt; margin-right:0pt; }
div.td01{ margin-left:0pt; margin-right:5pt; }
div.td10{ margin-left:5pt; margin-right:0pt; }
div.td11{ margin-left:5pt; margin-right:5pt; }
table[rules] {border-left:solid black 0.4pt; border-right:solid black 0.4pt; }
td.td00{ padding-left:0pt; padding-right:0pt; }
td.td01{ padding-left:0pt; padding-right:5pt; }
td.td10{ padding-left:5pt; padding-right:0pt; }
td.td11{ padding-left:5pt; padding-right:5pt; }
table[rules] {border-left:solid black 0.4pt; border-right:solid black 0.4pt; }
.hline hr, .cline hr{ height : 1px; margin:0px; }
.tabbing-right {text-align:right;}
span.TEX {letter-spacing: -0.125em; }
span.TEX span.E{ position:relative;top:0.5ex;left:-0.0417em;}
a span.TEX span.E {text-decoration: none; }
span.LATEX span.A{ position:relative; top:-0.5ex; left:-0.4em; font-size:85%;}
span.LATEX span.TEX{ position:relative; left: -0.4em; }
div.float img, div.float .caption {text-align:center;}
div.figure img, div.figure .caption {text-align:center;}
.marginpar {width:20%; float:right; text-align:left; margin-left:auto; margin-top:0.5em; font-size:85%; text-decoration:underline;}
.marginpar p{margin-top:0.4em; margin-bottom:0.4em;}
.equation td{text-align:center; vertical-align:middle; }
td.eq-no{ width:5%; }
table.equation { width:100%; } 
div.math-display, div.par-math-display{text-align:center;}
math .texttt { font-family: monospace; }
math .textit { font-style: italic; }
math .textsl { font-style: oblique; }
math .textsf { font-family: sans-serif; }
math .textbf { font-weight: bold; }
.partToc a, .partToc, .likepartToc a, .likepartToc {line-height: 200%; font-weight:bold; font-size:110%;}
.chapterToc a, .chapterToc, .likechapterToc a, .likechapterToc, .appendixToc a, .appendixToc {line-height: 200%; font-weight:bold;}
.index-item, .index-subitem, .index-subsubitem {display:block}
.caption td.id{font-weight: bold; white-space: nowrap; }
table.caption {text-align:center;}
h1.partHead{text-align: center}
p.bibitem { text-indent: -2em; margin-left: 2em; margin-top:0.6em; margin-bottom:0.6em; }
p.bibitem-p { text-indent: 0em; margin-left: 2em; margin-top:0.6em; margin-bottom:0.6em; }
.paragraphHead, .likeparagraphHead { margin-top:2em; font-weight: bold;}
.subparagraphHead, .likesubparagraphHead { font-weight: bold;}
.quote {margin-bottom:0.25em; margin-top:0.25em; margin-left:1em; margin-right:1em; text-align:justify;}
.verse{white-space:nowrap; margin-left:2em}
div.maketitle {text-align:center;}
h2.titleHead{text-align:center;}
div.maketitle{ margin-bottom: 2em; }
div.author, div.date {text-align:center;}
div.thanks{text-align:left; margin-left:10%; font-size:85%; font-style:italic; }
div.author{white-space: nowrap;}
.quotation {margin-bottom:0.25em; margin-top:0.25em; margin-left:1em; }
h1.partHead{text-align: center}
.sectionToc, .likesectionToc {margin-left:2em;}
.subsectionToc, .likesubsectionToc {margin-left:4em;}
.subsubsectionToc, .likesubsubsectionToc {margin-left:6em;}
.frenchb-nbsp{font-size:75%;}
.frenchb-thinspace{font-size:75%;}
.figure img.graphics {margin-left:10%;}
/* end css.sty */

\title{Formes differentielles}
\author{}
\date{}

\begin{document}
\maketitle

\textbf{Warning: \href{http://www.math.union.edu/locate/jsMath}{jsMath}
requires JavaScript to process the mathematics on this page.\\ If your
browser supports JavaScript, be sure it is enabled.}

\begin{center}\rule{3in}{0.4pt}\end{center}

{[}\href{coursse85.html}{next}{]} {[}\href{coursse83.html}{prev}{]}
{[}\href{coursse83.html\#tailcoursse83.html}{prev-tail}{]}
{[}\hyperref[tailcoursse84.html]{tail}{]}
{[}\href{coursch16.html\#coursse84.html}{up}{]}

\subsubsection{15.3 Formes différentielles}

Remarque~15.3.1 En dehors de la notion de gradient, cette section ne
fait pas partie du programme des classes préparatoires. Cependant, les
formes différentielles de degré 1 sont un outil particulièrement commode
même à ce niveau.

\paragraph{15.3.1 Rappels sur les formes linéaires alternées}

Proposition~15.3.1 Soit E un ℝ-espace vectoriel,
\{f\}\_\{1\},\textbackslash{}mathop\{\textbackslash{}mathop\{\ldots{}\}\},\{f\}\_\{p\}
∈ \{E\}\^{}\{∗\}. Alors \{f\}\_\{1\}
∧\textbackslash{}mathop\{\textbackslash{}mathop\{\ldots{}\}\} ∧
\{f\}\_\{p\} : \{E\}\^{}\{p\} → K définie par
(\{x\}\_\{1\},\textbackslash{}mathop\{\textbackslash{}mathop\{\ldots{}\}\},\{x\}\_\{p\})\textbackslash{}mathrel\{↦\}\textbackslash{}mathop\{\textbackslash{}mathrm\{det\}\}
\{(\{f\}\_\{i\}(\{x\}\_\{j\}))\}\_\{1≤i≤p,1≤j≤p\} est une forme p-
linéaire alternée sur E. L'application \{(\{E\}\^{}\{∗\})\}\^{}\{p\} →
\{A\}\_\{p\}(E),
(\{f\}\_\{1\},\textbackslash{}mathop\{\textbackslash{}mathop\{\ldots{}\}\},\{f\}\_\{p\})\textbackslash{}mathrel\{↦\}\{f\}\_\{1\}
∧\textbackslash{}mathop\{\textbackslash{}mathop\{\ldots{}\}\} ∧
\{f\}\_\{p\} est elle même p-linéaire et alternée.

Ceci permet d'exhiber une base de l'espace \{A\}\_\{p\}(E) des formes
p-linéaires alternées sur E. Pour cela soit E un K-espace vectoriel de
dimension n et
(\{e\}\_\{1\},\textbackslash{}mathop\{\textbackslash{}mathop\{\ldots{}\}\},\{e\}\_\{n\})
une base de E.

Théorème~15.3.2 La famille des
\{(\{e\}\_\{\{i\}\_\{1\}\}\^{}\{∗\}∧\textbackslash{}mathop\{\textbackslash{}mathop\{\ldots{}\}\}
∧
\{e\}\_\{\{i\}\_\{p\}\}\^{}\{∗\})\}\_\{1≤\{i\}\_\{1\}\textless{}\{i\}\_\{2\}\textless{}\textbackslash{}mathop\{\textbackslash{}mathop\{\ldots{}\}\}\textless{}\{i\}\_\{p\}≤n\}
est une base de \{A\}\_\{p\}(E) (qui est donc de dimension
\{C\}\_\{n\}\^{}\{p\}).

\paragraph{15.3.2 Notion de forme différentielle}

Définition~15.3.1 Soit U un ouvert de \{ℝ\}\^{}\{n\}. On appelle forme
différentielle de degré p sur U toute application de U dans
\{A\}\_\{p\}(\{ℝ\}\^{}\{n\}) (en posant par convention
\{A\}\_\{0\}(\{ℝ\}\^{}\{n\}) = ℝ).

Remarque~15.3.2 Soit ω : U → \{A\}\_\{p\}(\{ℝ\}\^{}\{n\}) une forme
différentielle de degré p. Soit
(\{e\}\_\{1\},\textbackslash{}mathop\{\textbackslash{}mathop\{\ldots{}\}\},\{e\}\_\{n\})
la base canonique de \{ℝ\}\^{}\{n\} et
\{(\{e\}\_\{\{i\}\_\{1\}\}\^{}\{∗\}∧\textbackslash{}mathop\{\textbackslash{}mathop\{\ldots{}\}\}
∧
\{e\}\_\{\{i\}\_\{p\}\}\^{}\{∗\})\}\_\{1≤\{i\}\_\{1\}\textless{}\{i\}\_\{2\}\textless{}\textbackslash{}mathop\{\textbackslash{}mathop\{\ldots{}\}\}\textless{}\{i\}\_\{p\}≤n\}
la base correspondante de \{A\}\_\{p\}(\{ℝ\}\^{}\{n\}). On a alors, pour
x ∈ U, ω(x) =\{\textbackslash{}mathop\{ \textbackslash{}mathop\{∑ \}\}
\}\_\{1≤\{i\}\_\{1\}\textless{}\{i\}\_\{2\}\textless{}\textbackslash{}mathop\{\textbackslash{}mathop\{\ldots{}\}\}\textless{}\{i\}\_\{p\}≤n\}\{a\}\_\{\{i\}\_\{1\},\textbackslash{}mathop\{\textbackslash{}mathop\{\ldots{}\}\},\{i\}\_\{p\}\}(x)\{e\}\_\{\{i\}\_\{1\}\}\^{}\{∗\}∧\textbackslash{}mathop\{\textbackslash{}mathop\{\ldots{}\}\}
∧ \{e\}\_\{\{i\}\_\{p\}\}\^{}\{∗\}. On dit que ω est de classe
\{C\}\^{}\{k\} si toutes les applications
\{a\}\_\{\{i\}\_\{1\},\textbackslash{}mathop\{\textbackslash{}mathop\{\ldots{}\}\},\{i\}\_\{p\}\}
: U → ℝ sont de classe \{C\}\^{}\{k\}.

Remarque~15.3.3 Soit f : U → ℝ de classe \{C\}\^{}\{1\}. Alors pour tout
x ∈ U, df(x) est une application linéaire de \{ℝ\}\^{}\{n\} dans ℝ donc
une forme linéaire sur \{ℝ\}\^{}\{n\}, donc un élément de
\{(\{ℝ\}\^{}\{n\})\}\^{}\{∗\} = \{A\}\_\{1\}(E). On en déduit que df :
x\textbackslash{}mathrel\{↦\}df(x) est une forme différentielle de degré
1 sur U. On sait que

df(x).h =\{ \textbackslash{}mathop\{∑ \}\}\_\{i=1\}\^{}\{n\}\{ ∂f
\textbackslash{}over ∂\{x\}\_\{i\}\} (x)\{h\}\_\{i\} =\{
\textbackslash{}mathop\{∑ \}\}\_\{i=1\}\^{}\{n\}\{ ∂f
\textbackslash{}over ∂\{x\}\_\{i\}\} (x)\{e\}\_\{i\}\^{}\{∗\}(h)

On en déduit que df(x) =\{\textbackslash{}mathop\{
\textbackslash{}mathop\{∑ \}\} \}\_\{i=1\}\^{}\{n\}\{ ∂f
\textbackslash{}over ∂\{x\}\_\{i\}\} (x)\{e\}\_\{i\}\^{}\{∗\}. Prenons
par exemple f = \{e\}\_\{i\}\^{}\{∗\}. On a \textbackslash{}mathop\{∀\}x
∈ U, df(x) = \{e\}\_\{i\}\^{}\{∗\}. Si on note \{x\}\_\{i\},
l'application i-ième coordonnée (c'est-à-dire encore
\{e\}\_\{i\}\^{}\{∗\}), on a donc d\{x\}\_\{i\} = \{e\}\_\{i\}\^{}\{∗\}
si bien que l'on peut noter df(x) =\{\textbackslash{}mathop\{
\textbackslash{}mathop\{∑ \}\} \}\_\{i=1\}\^{}\{n\}\{ ∂f
\textbackslash{}over ∂\{x\}\_\{i\}\} (x)d\{x\}\_\{i\}. Plus
généralement, une forme différentielle de degré p sur U sera de la forme

ω(x) =\{ \textbackslash{}mathop\{∑
\}\}\_\{1≤\{i\}\_\{1\}\textless{}\{i\}\_\{2\}\textless{}\textbackslash{}mathop\{\ldots{}\}\textless{}\{i\}\_\{p\}≤n\}\{a\}\_\{\{i\}\_\{1\},\textbackslash{}mathop\{\ldots{}\},\{i\}\_\{p\}\}(x)d\{x\}\_\{\{i\}\_\{1\}\}
∧\textbackslash{}mathop\{\ldots{}\} ∧ d\{x\}\_\{\{i\}\_\{p\}\}

C'est cette dernière forme que nous utiliserons par la suite, avec comme
seule propriété à connaître le fait que ∧ est multilinéaire et alternée.

Exemple~15.3.1 Dans le cas de la dimension 3 et de p = 2, on préfère
utiliser une base invariante par permutation circulaire, à savoir
d\{x\}\_\{2\} ∧ d\{x\}\_\{3\},d\{x\}\_\{3\} ∧
d\{x\}\_\{1\},d\{x\}\_\{1\} ∧ d\{x\}\_\{2\}. On aura ainsi les
expressions générales de formes différentielles de degré p sur un ouvert
de \{ℝ\}\^{}\{n\}.

p = 0~: dans tous les cas, une forme différentielle de degré 0 est
simplement une fonction à valeurs réelles et une forme différentielle de
degré 1 s'écrit

ω(\{x\}\_\{1\},\textbackslash{}mathop\{\textbackslash{}mathop\{\ldots{}\}\},\{x\}\_\{n\}))
=
\{a\}\_\{1\}(\{x\}\_\{1\},\textbackslash{}mathop\{\textbackslash{}mathop\{\ldots{}\}\},\{x\}\_\{n\})d\{x\}\_\{1\}
+ \textbackslash{}mathop\{\textbackslash{}mathop\{\ldots{}\}\} +
\{a\}\_\{n\}(\{x\}\_\{1\},\textbackslash{}mathop\{\textbackslash{}mathop\{\ldots{}\}\},\{x\}\_\{n\})d\{x\}\_\{n\}

\textbackslash{}begin\{eqnarray*\} n = 2,p = 2\& :\&
ω(\{x\}\_\{1\},\{x\}\_\{2\}) = a(\{x\}\_\{1\},\{x\}\_\{2\})d\{x\}\_\{1\}
∧ d\{x\}\_\{2\} \%\& \textbackslash{}\textbackslash{} n = 3,p = 2\& :\&
ω(\{x\}\_\{1\},\{x\}\_\{2\},\{x\}\_\{3\}) =
\{a\}\_\{1\}(\{x\}\_\{1\},\{x\}\_\{2\},\{x\}\_\{3\})d\{x\}\_\{2\} ∧
d\{x\}\_\{3\} \%\& \textbackslash{}\textbackslash{} \& \&
+\{a\}\_\{2\}(\{x\}\_\{1\},\{x\}\_\{2\},\{x\}\_\{3\})d\{x\}\_\{3\} ∧
d\{x\}\_\{1\} +
\{a\}\_\{3\}(\{x\}\_\{1\},\{x\}\_\{2\},\{x\}\_\{3\})d\{x\}\_\{1\} ∧
d\{x\}\_\{2\}\%\& \textbackslash{}\textbackslash{} n = 3,p = 3\& :\&
ω(\{x\}\_\{1\},\{x\}\_\{2\},\{x\}\_\{3\}) =
a(\{x\}\_\{1\},\{x\}\_\{2\},\{x\}\_\{3\})d\{x\}\_\{1\} ∧ d\{x\}\_\{2\} ∧
d\{x\}\_\{3\} \%\& \textbackslash{}\textbackslash{}
\textbackslash{}end\{eqnarray*\}

\paragraph{15.3.3 Notion de gradient d'une fonction}

Soit E un espace euclidien, U un ouvert de E et f : U → ℝ de classe
\{C\}\^{}\{1\}. Alors, pour x ∈ E, df(x) est une forme linéaire sur E~;
on sait qu'il existe un unique vecteur noté
\textbackslash{}mathop\{grad\}f(x) dans E tel que
\textbackslash{}mathop\{∀\}h ∈ E, df(x).h =
(\textbackslash{}mathop\{grad\}f(x)\textbackslash{}mathrel\{∣\}h).

Définition~15.3.2 Le vecteur \textbackslash{}mathop\{grad\}f(x) défini
par \textbackslash{}mathop\{∀\}h ∈ E, df(x).h =
(\textbackslash{}mathop\{grad\}f(x)\textbackslash{}mathrel\{∣\}h), est
appelé gradient de f au point x.

Remarque~15.3.4 Supposons que E = \{ℝ\}\^{}\{n\} muni de sa structure
euclidienne naturelle (celle qui rend la base canonique orthonormée).
Alors

df(x).h =\{ \textbackslash{}mathop\{∑ \}\}\_\{i=1\}\^{}\{n\}\{h\}\_\{
i\}\{ ∂f \textbackslash{}over ∂\{x\}\_\{i\}\} (x) =
(\{\textbackslash{}mathop\{∑ \}\}\_\{i=1\}\^{}\{n\}\{ ∂f
\textbackslash{}over ∂\{x\}\_\{i\}\}
(x)\{e\}\_\{i\}\textbackslash{}mathrel\{∣\}h)

si bien que l'on retrouve l'expression classique du gradient de f

\textbackslash{}mathop\{grad\}f(x) =\{ \textbackslash{}mathop\{∑
\}\}\_\{i=1\}\^{}\{n\}\{ ∂f \textbackslash{}over ∂\{x\}\_\{i\}\}
(x)\{e\}\_\{i\} = (\{ ∂f \textbackslash{}over ∂\{x\}\_\{1\}\}
(x),\textbackslash{}mathop\{\ldots{}\},\{ ∂f \textbackslash{}over
∂\{x\}\_\{n\}\} (x))

\paragraph{15.3.4 Invariance de la différentielle}

Soit U un ouvert de \{ℝ\}\^{}\{p\} et f : U → ℝ de classe
\{C\}\^{}\{1\}. Soit V un ouvert de \{ℝ\}\^{}\{n\} et soit φ =
(\{φ\}\_\{1\},\textbackslash{}mathop\{\textbackslash{}mathop\{\ldots{}\}\},\{φ\}\_\{p\})
: V → U. Posons \{y\}\_\{1\} =
\{φ\}\_\{1\}(\{x\}\_\{1\},\textbackslash{}mathop\{\textbackslash{}mathop\{\ldots{}\}\},\{x\}\_\{n\}),\textbackslash{}mathop\{\textbackslash{}mathop\{\ldots{}\}\},\{y\}\_\{p\}
=
\{φ\}\_\{p\}(\{x\}\_\{1\},\textbackslash{}mathop\{\textbackslash{}mathop\{\ldots{}\}\},\{x\}\_\{n\}).
On a donc d\{y\}\_\{j\} =\{\textbackslash{}mathop\{
\textbackslash{}mathop\{∑ \}\} \}\_\{i=1\}\^{}\{n\}\{ ∂\{φ\}\_\{j\}
\textbackslash{}over ∂\{x\}\_\{i\}\} d\{x\}\_\{i\}. De plus,
f(\{y\}\_\{1\},\textbackslash{}mathop\{\textbackslash{}mathop\{\ldots{}\}\},\{y\}\_\{p\})
=
f(\{φ\}\_\{1\}(\{x\}\_\{1\},\textbackslash{}mathop\{\textbackslash{}mathop\{\ldots{}\}\},\{x\}\_\{n\}),\textbackslash{}mathop\{\textbackslash{}mathop\{\ldots{}\}\},\{φ\}\_\{p\}(\{x\}\_\{1\},\textbackslash{}mathop\{\textbackslash{}mathop\{\ldots{}\}\},\{x\}\_\{n\}))
si bien que

\textbackslash{}begin\{eqnarray*\}
d(f(\{y\}\_\{1\},\textbackslash{}mathop\{\textbackslash{}mathop\{\ldots{}\}\},\{y\}\_\{p\}))\&\&
\%\& \textbackslash{}\textbackslash{} \& =\& \{\textbackslash{}mathop\{∑
\}\}\_\{i=1\}\^{}\{n\}\{ ∂ \textbackslash{}over ∂\{x\}\_\{i\}\}
\textbackslash{}left
(f(\{φ\}\_\{1\}(\{x\}\_\{1\},\textbackslash{}mathop\{\ldots{}\},\{x\}\_\{n\}),\textbackslash{}mathop\{\ldots{}\},\{φ\}\_\{p\}(\{x\}\_\{1\},\textbackslash{}mathop\{\ldots{}\},\{x\}\_\{n\}))\textbackslash{}right
)d\{x\}\_\{i\} \%\& \textbackslash{}\textbackslash{} \& =\&
\{\textbackslash{}mathop\{∑ \}\}\_\{i=1\}\^{}\{n\}\textbackslash{}left
(\{\textbackslash{}mathop\{∑ \}\}\_\{j=1\}\^{}\{p\}\{ ∂f
\textbackslash{}over ∂\{y\}\_\{j\}\}
(\{φ\}\_\{1\}(\{x\}\_\{1\},\textbackslash{}mathop\{\ldots{}\},\{x\}\_\{n\}),\textbackslash{}mathop\{\ldots{}\},\{φ\}\_\{p\}(\{x\}\_\{1\},\textbackslash{}mathop\{\ldots{}\},\{x\}\_\{n\}))\{
∂\{φ\}\_\{j\} \textbackslash{}over ∂\{x\}\_\{i\}\} \textbackslash{}right
)d\{x\}\_\{i\}\%\& \textbackslash{}\textbackslash{} \& =\&
\{\textbackslash{}mathop\{∑ \}\}\_\{j=1\}\^{}\{p\}\{ ∂f
\textbackslash{}over ∂\{y\}\_\{j\}\}
(\{y\}\_\{1\},\textbackslash{}mathop\{\ldots{}\},\{y\}\_\{p\})\textbackslash{}left
(\{\textbackslash{}mathop\{∑ \}\}\_\{i=1\}\^{}\{n\}\{ ∂\{φ\}\_\{j\}
\textbackslash{}over ∂\{x\}\_\{i\}\} d\{x\}\_\{i\}\textbackslash{}right
) \%\& \textbackslash{}\textbackslash{} \& =\&
\{\textbackslash{}mathop\{∑ \}\}\_\{j=1\}\^{}\{p\}\{ ∂f
\textbackslash{}over ∂\{y\}\_\{j\}\}
(\{y\}\_\{1\},\textbackslash{}mathop\{\ldots{}\},\{y\}\_\{p\})d\{y\}\_\{j\}
\%\& \textbackslash{}\textbackslash{} \textbackslash{}end\{eqnarray*\}

en utilisant la règle de dérivation partielle des fonctions composées et
en intervertissant les deux sommations.

On voit donc que la formule
d(f(\{y\}\_\{1\},\textbackslash{}mathop\{\textbackslash{}mathop\{\ldots{}\}\},\{y\}\_\{p\}))
=\{\textbackslash{}mathop\{ \textbackslash{}mathop\{∑ \}\}
\}\_\{j=1\}\^{}\{p\}\{ ∂f \textbackslash{}over ∂\{y\}\_\{j\}\}
(\{y\}\_\{1\},\textbackslash{}mathop\{\textbackslash{}mathop\{\ldots{}\}\},\{y\}\_\{p\})d\{y\}\_\{j\}
est valable aussi bien quand
\{y\}\_\{1\},\textbackslash{}mathop\{\textbackslash{}mathop\{\ldots{}\}\},\{y\}\_\{p\}
désignent des variables libres (c'est-à-dire qui varient dans un ouvert
de \{ℝ\}\^{}\{p\}) que lorsque
\{y\}\_\{1\},\textbackslash{}mathop\{\textbackslash{}mathop\{\ldots{}\}\},\{y\}\_\{p\}
désignent des fonctions d'autres variables (ici
\{x\}\_\{1\},\textbackslash{}mathop\{\textbackslash{}mathop\{\ldots{}\}\},\{x\}\_\{n\}).
C'est une propriété essentielle de la différentielle qui fait tout
l'intérêt des formes différentielles (en particulier de degré 1)~: on
peut différentier une expression sans savoir quelles sont les variables
et quelles sont les fonctions.

On prendra simplement garde au fait suivant~: lorsque
\{y\}\_\{1\},\textbackslash{}mathop\{\textbackslash{}mathop\{\ldots{}\}\},\{y\}\_\{p\}
désignent des variables libres, qui varient dans des ouverts de
\{ℝ\}\^{}\{p\}, on a d\{y\}\_\{j\} = \{e\}\_\{j\}\^{}\{∗\}, et donc les
formes différentielles
d\{y\}\_\{1\},\textbackslash{}mathop\{\textbackslash{}mathop\{\ldots{}\}\},d\{y\}\_\{p\}
forment une famille libre (ce qui permet en particulier des
identifications)~; il n'en est évidemment plus de même lorsque
\{y\}\_\{1\},\textbackslash{}mathop\{\textbackslash{}mathop\{\ldots{}\}\},\{y\}\_\{p\}
sont elles mêmes des fonctions d'autres variables
\{x\}\_\{1\},\textbackslash{}mathop\{\textbackslash{}mathop\{\ldots{}\}\},\{x\}\_\{n\}.

\paragraph{15.3.5 Différentielle extérieure}

Définition~15.3.3 Soit ω(x) =\{\textbackslash{}mathop\{
\textbackslash{}mathop\{∑ \}\}
\}\_\{1≤\{i\}\_\{1\}\textless{}\{i\}\_\{2\}\textless{}\textbackslash{}mathop\{\textbackslash{}mathop\{\ldots{}\}\}\textless{}\{i\}\_\{p\}≤n\}\{a\}\_\{\{i\}\_\{1\},\textbackslash{}mathop\{\textbackslash{}mathop\{\ldots{}\}\},\{i\}\_\{p\}\}(x)d\{x\}\_\{\{i\}\_\{1\}\}
∧\textbackslash{}mathop\{\textbackslash{}mathop\{\ldots{}\}\} ∧
d\{x\}\_\{\{i\}\_\{p\}\} une forme différentielle de degré p, de classe
\{C\}\^{}\{1\} sur l'ouvert U de \{ℝ\}\^{}\{n\}. On appelle
différentielle extérieure de ω, la forme différentielle de degré p + 1
définie par

dω(x) =\{ \textbackslash{}mathop\{∑
\}\}\_\{1≤\{i\}\_\{1\}\textless{}\{i\}\_\{2\}\textless{}\textbackslash{}mathop\{\ldots{}\}\textless{}\{i\}\_\{p\}≤n\}d\{a\}\_\{\{i\}\_\{1\},\textbackslash{}mathop\{\ldots{}\},\{i\}\_\{p\}\}(x)
∧ d\{x\}\_\{\{i\}\_\{1\}\} ∧\textbackslash{}mathop\{\ldots{}\} ∧
d\{x\}\_\{\{i\}\_\{p\}\}

Remarque~15.3.5 Le calcul effectif se fait en utilisant la définition de
d\{a\}\_\{\{i\}\_\{1\},\textbackslash{}mathop\{\textbackslash{}mathop\{\ldots{}\}\},\{i\}\_\{p\}\}(x)
et les propriétés de l'opérateur ∧~: linéaire par rapport à chaque
terme, alterné, antisymétrique. On a donc

\textbackslash{}begin\{eqnarray*\}
d\{a\}\_\{\{i\}\_\{1\},\textbackslash{}mathop\{\textbackslash{}mathop\{\ldots{}\}\},\{i\}\_\{p\}\}(x)
∧ d\{x\}\_\{\{i\}\_\{1\}\}
∧\textbackslash{}mathop\{\textbackslash{}mathop\{\ldots{}\}\} ∧
d\{x\}\_\{\{i\}\_\{p\}\}\&\& \%\& \textbackslash{}\textbackslash{} \&
=\& \textbackslash{}left (\{\textbackslash{}mathop\{∑
\}\}\_\{j=1\}\^{}\{n\}\{
∂\{a\}\_\{\{i\}\_\{1\},\textbackslash{}mathop\{\ldots{}\},\{i\}\_\{p\}\}
\textbackslash{}over ∂\{x\}\_\{j\}\}
\textbackslash{},d\{x\}\_\{j\}\textbackslash{}right ) ∧
d\{x\}\_\{\{i\}\_\{1\}\}
∧\textbackslash{}mathop\{\textbackslash{}mathop\{\ldots{}\}\} ∧
d\{x\}\_\{\{i\}\_\{p\}\}\%\& \textbackslash{}\textbackslash{} \& =\&
\{\textbackslash{}mathop\{∑ \}\}\_\{j=1\}\^{}\{n\}\{
∂\{a\}\_\{\{i\}\_\{1\},\textbackslash{}mathop\{\ldots{}\},\{i\}\_\{p\}\}
\textbackslash{}over ∂\{x\}\_\{j\}\} \textbackslash{},d\{x\}\_\{j\} ∧
d\{x\}\_\{\{i\}\_\{1\}\} ∧\textbackslash{}mathop\{\ldots{}\} ∧
d\{x\}\_\{\{i\}\_\{p\}\} \%\& \textbackslash{}\textbackslash{}
\textbackslash{}end\{eqnarray*\}

avec d\{x\}\_\{j\} ∧ d\{x\}\_\{\{i\}\_\{1\}\}
∧\textbackslash{}mathop\{\textbackslash{}mathop\{\ldots{}\}\} ∧
d\{x\}\_\{\{i\}\_\{p\}\} = 0 si j
∈\textbackslash{}\{\{i\}\_\{1\},\textbackslash{}mathop\{\textbackslash{}mathop\{\ldots{}\}\},\{i\}\_\{p\}\textbackslash{}\}
et d\{x\}\_\{j\} ∧ d\{x\}\_\{\{i\}\_\{1\}\}
∧\textbackslash{}mathop\{\textbackslash{}mathop\{\ldots{}\}\} ∧
d\{x\}\_\{\{i\}\_\{p\}\} = ±d\{x\}\_\{\{i\}\_\{1\}\}
∧\textbackslash{}mathop\{\textbackslash{}mathop\{\ldots{}\}\} ∧
d\{x\}\_\{j\}
∧\textbackslash{}mathop\{\textbackslash{}mathop\{\ldots{}\}\} ∧
d\{x\}\_\{\{i\}\_\{p\}\} où l'on met de signe + ou le signe − suivant la
parité du nombre de transpositions nécessaires pour intercaler j à la
bonne place dans la suite
\textbackslash{}\{\{i\}\_\{1\},\textbackslash{}mathop\{\textbackslash{}mathop\{\ldots{}\}\},\{i\}\_\{p\}\textbackslash{}\}.
Dans le cas d'une forme différentielle de degré 0 (une fonction), on
trouve bien entendu tout simplement la différentielle de la fonction.

Exemple~15.3.2 Calcul dans le cas n = 3. Si p = 0, on a ω = f et dω =\{
∂f \textbackslash{}over ∂\{x\}\_\{1\}\} d\{x\}\_\{1\} +\{ ∂f
\textbackslash{}over ∂\{x\}\_\{2\}\} d\{x\}\_\{2\} +\{ ∂f
\textbackslash{}over ∂\{x\}\_\{3\}\} d\{x\}\_\{3\} et on retrouve
l'expression du gradient de la fonction f.

Si p = 1, on a ω(x) = \{a\}\_\{1\}(x)d\{x\}\_\{1\} +
\{a\}\_\{2\}(x)d\{x\}\_\{2\} + \{a\}\_\{3\}(x)d\{x\}\_\{3\}, et donc

\textbackslash{}begin\{eqnarray*\} dω(x)\& =\& d\{a\}\_\{1\}(x) ∧
d\{x\}\_\{1\} + d\{a\}\_\{2\}(x) ∧ d\{x\}\_\{2\} + d\{a\}\_\{3\}(x) ∧
d\{x\}\_\{3\} \%\& \textbackslash{}\textbackslash{} \& =\& (\{
∂\{a\}\_\{1\} \textbackslash{}over ∂\{x\}\_\{1\}\} (x)d\{x\}\_\{1\} +\{
∂\{a\}\_\{1\} \textbackslash{}over ∂\{x\}\_\{2\}\} (x)d\{x\}\_\{2\} +\{
∂\{a\}\_\{1\} \textbackslash{}over ∂\{x\}\_\{3\}\} (x)d\{x\}\_\{3\}) ∧
d\{x\}\_\{1\} \%\& \textbackslash{}\textbackslash{} \& \& +(\{
∂\{a\}\_\{2\} \textbackslash{}over ∂\{x\}\_\{1\}\} (x)d\{x\}\_\{1\} +\{
∂\{a\}\_\{2\} \textbackslash{}over ∂\{x\}\_\{2\}\} (x)d\{x\}\_\{2\} +\{
∂\{a\}\_\{2\} \textbackslash{}over ∂\{x\}\_\{3\}\} (x)d\{x\}\_\{3\}) ∧
d\{x\}\_\{2\}\%\& \textbackslash{}\textbackslash{} \& \& +(\{
∂\{a\}\_\{3\} \textbackslash{}over ∂\{x\}\_\{1\}\} (x)d\{x\}\_\{1\} +\{
∂\{a\}\_\{3\} \textbackslash{}over ∂\{x\}\_\{2\}\} (x)d\{x\}\_\{2\} +\{
∂\{a\}\_\{3\} \textbackslash{}over ∂\{x\}\_\{3\}\} (x)d\{x\}\_\{3\}) ∧
d\{x\}\_\{3\}\%\& \textbackslash{}\textbackslash{} \& =\& (\{
∂\{a\}\_\{3\} \textbackslash{}over ∂\{x\}\_\{2\}\} (x) −\{ ∂\{a\}\_\{2\}
\textbackslash{}over ∂\{x\}\_\{3\}\} (x))d\{x\}\_\{2\} ∧ d\{x\}\_\{3\}
\%\& \textbackslash{}\textbackslash{} \& \& +(\{ ∂\{a\}\_\{1\}
\textbackslash{}over ∂\{x\}\_\{3\}\} (x) −\{ ∂\{a\}\_\{3\}
\textbackslash{}over ∂\{x\}\_\{1\}\} (x))d\{x\}\_\{3\} ∧ d\{x\}\_\{1\}
\%\& \textbackslash{}\textbackslash{} \& \& +(\{ ∂\{a\}\_\{2\}
\textbackslash{}over ∂\{x\}\_\{1\}\} (x) −\{ ∂\{a\}\_\{1\}
\textbackslash{}over ∂\{x\}\_\{2\}\} (x))d\{x\}\_\{1\} ∧ d\{x\}\_\{2\}
\%\& \textbackslash{}\textbackslash{} \textbackslash{}end\{eqnarray*\}

en tenant compte de d\{x\}\_\{i\} ∧ d\{x\}\_\{i\} = 0 et de
d\{x\}\_\{i\} ∧ d\{x\}\_\{j\} = −d\{x\}\_\{j\} ∧ d\{x\}\_\{i\}. On
reconnaît là l'expression classique du rotationnel du champ de vecteurs
de composantes (\{a\}\_\{1\}(x),\{a\}\_\{2\}(x),\{a\}\_\{3\}(x)).

Si p = 2, on a ω(x) = \{a\}\_\{1\}(x)d\{x\}\_\{2\} ∧ d\{x\}\_\{3\} +
\{a\}\_\{2\}(x)d\{x\}\_\{3\} ∧ d\{x\}\_\{1\} +
\{a\}\_\{3\}(x)d\{x\}\_\{1\} ∧ d\{x\}\_\{2\} et donc

\textbackslash{}begin\{eqnarray*\} dω(x)\& =\& d\{a\}\_\{1\}(x) ∧
d\{x\}\_\{2\} ∧ d\{x\}\_\{3\} + d\{a\}\_\{2\}(x) ∧ d\{x\}\_\{3\} ∧
d\{x\}\_\{1\} \%\& \textbackslash{}\textbackslash{} \& \&
+d\{a\}\_\{3\}(x) ∧ d\{x\}\_\{1\} ∧ d\{x\}\_\{2\} \%\&
\textbackslash{}\textbackslash{} \& =\& (\{ ∂\{a\}\_\{1\}
\textbackslash{}over ∂\{x\}\_\{1\}\} (x)d\{x\}\_\{1\} +\{ ∂\{a\}\_\{1\}
\textbackslash{}over ∂\{x\}\_\{2\}\} (x)d\{x\}\_\{2\} +\{ ∂\{a\}\_\{1\}
\textbackslash{}over ∂\{x\}\_\{3\}\} (x)d\{x\}\_\{3\}) ∧ d\{x\}\_\{2\} ∧
d\{x\}\_\{3\} \%\& \textbackslash{}\textbackslash{} \& \& +(\{
∂\{a\}\_\{2\} \textbackslash{}over ∂\{x\}\_\{1\}\} (x)d\{x\}\_\{1\} +\{
∂\{a\}\_\{2\} \textbackslash{}over ∂\{x\}\_\{2\}\} (x)d\{x\}\_\{2\} +\{
∂\{a\}\_\{2\} \textbackslash{}over ∂\{x\}\_\{3\}\} (x)d\{x\}\_\{3\}) ∧
d\{x\}\_\{3\} ∧ d\{x\}\_\{1\}\%\& \textbackslash{}\textbackslash{} \& \&
+(\{ ∂\{a\}\_\{3\} \textbackslash{}over ∂\{x\}\_\{1\}\} (x)d\{x\}\_\{1\}
+\{ ∂\{a\}\_\{3\} \textbackslash{}over ∂\{x\}\_\{2\}\} (x)d\{x\}\_\{2\}
+\{ ∂\{a\}\_\{3\} \textbackslash{}over ∂\{x\}\_\{3\}\} (x)d\{x\}\_\{3\})
∧ d\{x\}\_\{1\} ∧ d\{x\}\_\{2\}\%\& \textbackslash{}\textbackslash{} \&
=\& \textbackslash{}left (\{ ∂\{a\}\_\{1\} \textbackslash{}over
∂\{x\}\_\{1\}\} (x) +\{ ∂\{a\}\_\{2\} \textbackslash{}over
∂\{x\}\_\{2\}\} (x) +\{ ∂\{a\}\_\{3\} \textbackslash{}over
∂\{x\}\_\{3\}\} (x)\textbackslash{}right )d\{x\}\_\{1\} ∧ d\{x\}\_\{2\}
∧ d\{x\}\_\{3\} \%\& \textbackslash{}\textbackslash{}
\textbackslash{}end\{eqnarray*\}

en tenant compte de d\{x\}\_\{i\} ∧ d\{x\}\_\{j\} ∧ d\{x\}\_\{k\} = 0 si
i,j et k ne sont pas distincts et de d\{x\}\_\{j\} ∧ d\{x\}\_\{k\} ∧
d\{x\}\_\{i\} = d\{x\}\_\{i\} ∧ d\{x\}\_\{j\} ∧ d\{x\}\_\{k\} si i,j,k
sont distincts (les permutations circulaires de trois éléments sont de
signature + 1). On reconnaît là l'expression classique de la divergence
du champ de vecteurs de composantes
(\{a\}\_\{1\}(x),\{a\}\_\{2\}(x),\{a\}\_\{3\}(x)).

La différentielle extérieure des formes différentielles est donc une
généralisation (et une unification) des notions classiques de gradient
d'une fonction et de rotationnel ou divergence d'un champ de vecteurs.

\paragraph{15.3.6 Théorème de Poincaré}

Théorème~15.3.3 Soit U un ouvert de \{ℝ\}\^{}\{n\} et ω une forme
différentielle de degré p de classe \{C\}\^{}\{2\} sur U. Alors d(dω) =
0.

Démonstration On a

\textbackslash{}begin\{eqnarray*\} dω\& =\& \{\textbackslash{}mathop\{∑
\}\}\_\{1≤\{i\}\_\{1\}\textless{}\{i\}\_\{2\}\textless{}\textbackslash{}mathop\{\ldots{}\}\textless{}\{i\}\_\{p\}≤n\}d\{a\}\_\{\{i\}\_\{1\},\textbackslash{}mathop\{\ldots{}\},\{i\}\_\{p\}\}
∧ d\{x\}\_\{\{i\}\_\{1\}\} ∧\textbackslash{}mathop\{\ldots{}\} ∧
d\{x\}\_\{\{i\}\_\{p\}\} \%\& \textbackslash{}\textbackslash{} \& =\&
\{\textbackslash{}mathop\{∑
\}\}\_\{1≤\{i\}\_\{1\}\textless{}\{i\}\_\{2\}\textless{}\textbackslash{}mathop\{\ldots{}\}\textless{}\{i\}\_\{p\}≤n\}\textbackslash{}left
(\{\textbackslash{}mathop\{∑ \}\}\_\{j=1\}\^{}\{n\}\{
∂\{a\}\_\{\{i\}\_\{1\},\textbackslash{}mathop\{\ldots{}\},\{i\}\_\{p\}\}
\textbackslash{}over ∂\{x\}\_\{j\}\}
\textbackslash{},d\{x\}\_\{j\}\textbackslash{}right ) ∧
d\{x\}\_\{\{i\}\_\{1\}\} ∧\textbackslash{}mathop\{\ldots{}\} ∧
d\{x\}\_\{\{i\}\_\{p\}\}\%\& \textbackslash{}\textbackslash{} \& =\&
\{\textbackslash{}mathop\{∑ \}\}\_\{j=1\}\^{}\{n\}\{
\textbackslash{}mathop\{∑
\}\}\_\{1≤\{i\}\_\{1\}\textless{}\{i\}\_\{2\}\textless{}\textbackslash{}mathop\{\ldots{}\}\textless{}\{i\}\_\{p\}≤n\}\{
∂\{a\}\_\{\{i\}\_\{1\},\textbackslash{}mathop\{\ldots{}\},\{i\}\_\{p\}\}
\textbackslash{}over ∂\{x\}\_\{j\}\} d\{x\}\_\{j\} ∧
d\{x\}\_\{\{i\}\_\{1\}\} ∧\textbackslash{}mathop\{\ldots{}\} ∧
d\{x\}\_\{\{i\}\_\{p\}\} \%\& \textbackslash{}\textbackslash{}
\textbackslash{}end\{eqnarray*\}

d'où

\textbackslash{}begin\{eqnarray*\} d(dω)\&\& \%\&
\textbackslash{}\textbackslash{} \& =\& \{\textbackslash{}mathop\{∑
\}\}\_\{j=1\}\^{}\{n\}\{ \textbackslash{}mathop\{∑
\}\}\_\{1≤\{i\}\_\{1\}\textless{}\{i\}\_\{2\}\textless{}\textbackslash{}mathop\{\ldots{}\}\textless{}\{i\}\_\{p\}≤n\}d\textbackslash{}left
(\{
∂\{a\}\_\{\{i\}\_\{1\},\textbackslash{}mathop\{\ldots{}\},\{i\}\_\{p\}\}
\textbackslash{}over ∂\{x\}\_\{j\}\} \textbackslash{}right ) ∧
d\{x\}\_\{j\} ∧ d\{x\}\_\{\{i\}\_\{1\}\}
∧\textbackslash{}mathop\{\ldots{}\} ∧ d\{x\}\_\{\{i\}\_\{p\}\} \%\&
\textbackslash{}\textbackslash{} \& =\& \{\textbackslash{}mathop\{∑
\}\}\_\{j=1\}\^{}\{n\}\{ \textbackslash{}mathop\{∑
\}\}\_\{1≤\{i\}\_\{1\}\textless{}\{i\}\_\{2\}\textless{}\textbackslash{}mathop\{\ldots{}\}\textless{}\{i\}\_\{p\}≤n\}\textbackslash{}left
(\{\textbackslash{}mathop\{∑ \}\}\_\{k=1\}\^{}\{n\}\{
\{∂\}\^{}\{2\}\{a\}\_\{\{
i\}\_\{1\},\textbackslash{}mathop\{\ldots{}\},\{i\}\_\{p\}\}
\textbackslash{}over ∂\{x\}\_\{k\}∂\{x\}\_\{j\}\}
d\{x\}\_\{k\}\textbackslash{}right ) ∧ d\{x\}\_\{j\} ∧
d\{x\}\_\{\{i\}\_\{1\}\} ∧\textbackslash{}mathop\{\ldots{}\} ∧
d\{x\}\_\{\{i\}\_\{p\}\} \%\& \textbackslash{}\textbackslash{} \& =\&
\{\textbackslash{}mathop\{∑ \}\}\_\{k=1\}\^{}\{n\}\{
\textbackslash{}mathop\{∑ \}\}\_\{j=1\}\^{}\{n\}\{
\textbackslash{}mathop\{∑
\}\}\_\{1≤\{i\}\_\{1\}\textless{}\{i\}\_\{2\}\textless{}\textbackslash{}mathop\{\ldots{}\}\textless{}\{i\}\_\{p\}≤n\}\{
\{∂\}\^{}\{2\}\{a\}\_\{\{i\}\_\{1\},\textbackslash{}mathop\{\ldots{}\},\{i\}\_\{p\}\}
\textbackslash{}over ∂\{x\}\_\{k\}∂\{x\}\_\{j\}\} d\{x\}\_\{k\} ∧
d\{x\}\_\{j\} ∧ d\{x\}\_\{\{i\}\_\{1\}\}
∧\textbackslash{}mathop\{\ldots{}\} ∧ d\{x\}\_\{\{i\}\_\{p\}\} \%\&
\textbackslash{}\textbackslash{} \& =\& \{\textbackslash{}mathop\{∑
\}\}\_\{k\textless{}j\}\{ \textbackslash{}mathop\{∑
\}\}\_\{1≤\{i\}\_\{1\}\textless{}\{i\}\_\{2\}\textless{}\textbackslash{}mathop\{\ldots{}\}\textless{}\{i\}\_\{p\}≤n\}\textbackslash{}left
(\{
\{∂\}\^{}\{2\}\{a\}\_\{\{i\}\_\{1\},\textbackslash{}mathop\{\ldots{}\},\{i\}\_\{p\}\}
\textbackslash{}over ∂\{x\}\_\{k\}∂\{x\}\_\{j\}\} −\{
\{∂\}\^{}\{2\}\{a\}\_\{\{i\}\_\{1\},\textbackslash{}mathop\{\ldots{}\},\{i\}\_\{p\}\}
\textbackslash{}over ∂\{x\}\_\{j\}∂\{x\}\_\{k\}\} \textbackslash{}right
)d\{x\}\_\{k\} ∧ d\{x\}\_\{j\} ∧ d\{x\}\_\{\{i\}\_\{1\}\}
∧\textbackslash{}mathop\{\ldots{}\} ∧ d\{x\}\_\{\{i\}\_\{p\}\}\%\&
\textbackslash{}\textbackslash{} \textbackslash{}end\{eqnarray*\}

en tenant compte de d\{x\}\_\{j\} ∧ d\{x\}\_\{k\} = 0 si j = k et
d\{x\}\_\{j\} ∧ d\{x\}\_\{k\} = −d\{x\}\_\{k\} ∧ d\{x\}\_\{j\} si
j\textbackslash{}mathrel\{≠\}k. Mais le théorème de Schwarz montre que
\{ \{∂\}\^{}\{2\}\{a\}\_\{\{
i\}\_\{1\},\textbackslash{}mathop\{\textbackslash{}mathop\{\ldots{}\}\},\{i\}\_\{p\}\}
\textbackslash{}over ∂\{x\}\_\{k\}∂\{x\}\_\{j\}\} =\{
\{∂\}\^{}\{2\}\{a\}\_\{\{
i\}\_\{1\},\textbackslash{}mathop\{\textbackslash{}mathop\{\ldots{}\}\},\{i\}\_\{p\}\}
\textbackslash{}over ∂\{x\}\_\{j\}∂\{x\}\_\{k\}\} et donc d(dω) = 0.

En tenant compte des expressions trouvées pour dω dans le cas n = 3, on
obtient donc le corollaire suivant

Corollaire~15.3.4 (i) Soit f une fonction de classe \{C\}\^{}\{2\} sur
un ouvert U de \{ℝ\}\^{}\{3\}. Alors
\textbackslash{}mathop\{rot\}\textbackslash{}mathop\{grad\}f = 0 (ii)
Soit V un champ de vecteurs de classe \{C\}\^{}\{2\} sur un ouvert U de
\{ℝ\}\^{}\{3\}. Alors
\textbackslash{}mathop\{div\}\textbackslash{}mathop\{rot\}V = 0

Nous allons maintenant nous intéresser à la réciproque du théorème
précédent

Théorème~15.3.5 (Poincaré). Soit U ⊂ \{ℝ\}\^{}\{n\} un ouvert étoilé en
a ∈ U (c'est-à-dire que \textbackslash{}mathop\{∀\}x ∈ U, {[}a,x{]} ⊂
U). Soit ω une forme différentielle de degré p ≥ 1 de classe
\{C\}\^{}\{1\} sur U. Alors les conditions suivantes sont équivalentes
(i) dω = 0 (ii) ω est exacte~: il existe une forme différentielle α de
degré p − 1 de classe \{C\}\^{}\{2\} sur U telle que ω = dα.

Démonstration Le théorème précédent implique clairement que (ii) ⇒(i).
Nous nous contenterons de démontrer que (i) ⇒(ii) lorsque p = 1, en
admettant le cas général. Par une translation, sans nuire à la
généralité, on peut supposer que a = 0. Soit U ⊂ \{ℝ\}\^{}\{n\} un
ouvert étoilé en 0 ∈ U et soit ω =\{\textbackslash{}mathop\{
\textbackslash{}mathop\{∑ \}\}
\}\_\{i=1\}\^{}\{n\}\{c\}\_\{i\}(x)d\{x\}\_\{i\}. On a par un calcul
facile

dω =\{ \textbackslash{}mathop\{∑
\}\}\_\{i\textless{}j\}\textbackslash{}left (\{ ∂\{c\}\_\{j\}
\textbackslash{}over ∂\{x\}\_\{i\}\} −\{ ∂\{c\}\_\{i\}
\textbackslash{}over ∂\{x\}\_\{j\}\} \textbackslash{}right
)d\{x\}\_\{i\} ∧ d\{x\}\_\{j\}

Donc dω = 0 \textbackslash{}mathrel\{⇔\}
\textbackslash{}mathop\{∀\}i,j,\{ ∂\{c\}\_\{j\} \textbackslash{}over
∂\{x\}\_\{i\}\} =\{ ∂\{c\}\_\{i\} \textbackslash{}over ∂\{x\}\_\{j\}\} .

Définissons f : U → ℝ par f(x) =\{\textbackslash{}mathop\{
\textbackslash{}mathop\{∑ \}\}
\}\_\{i=1\}\^{}\{n\}\{x\}\_\{i\}\{\textbackslash{}mathop\{∫ \}
\}\_\{0\}\^{}\{1\}\{c\}\_\{i\}(tx) dt. Comme
(t,\{x\}\_\{j\})\textbackslash{}mathrel\{↦\}\{c\}\_\{i\}(tx) =
\{c\}\_\{i\}(t\{x\}\_\{1\},\textbackslash{}mathop\{\textbackslash{}mathop\{\ldots{}\}\},t\{x\}\_\{n\})
admet une dérivée partielle par rapport à \{x\}\_\{j\}, \{ ∂
\textbackslash{}over ∂\{x\}\_\{j\}\}
(\{c\}\_\{i\}(t\{x\}\_\{1\},\textbackslash{}mathop\{\textbackslash{}mathop\{\ldots{}\}\},t\{x\}\_\{n\}))
= t\{ ∂\{c\}\_\{i\} \textbackslash{}over ∂\{x\}\_\{j\}\}
(t\{x\}\_\{1\},\textbackslash{}mathop\{\textbackslash{}mathop\{\ldots{}\}\},t\{x\}\_\{n\})
qui est une fonction continue du couple (t,\{x\}\_\{j\}), l'application
\{x\}\_\{j\}\textbackslash{}mathrel\{↦\}\{\textbackslash{}mathop\{∫ \}
\}\_\{0\}\^{}\{1\}\{c\}\_\{i\}(tx) dt est dérivable et \{ ∂
\textbackslash{}over ∂\{x\}\_\{j\}\} \{\textbackslash{}mathop\{∫ \}
\}\_\{0\}\^{}\{1\}\{c\}\_\{i\}(tx) dt =\{\textbackslash{}mathop\{∫ \}
\}\_\{0\}\^{}\{1\}t\{ ∂\{c\}\_\{i\} \textbackslash{}over ∂\{x\}\_\{j\}\}
(tx) dt. On en déduit que

\textbackslash{}begin\{eqnarray*\}\{ ∂f \textbackslash{}over
∂\{x\}\_\{j\}\} (x)\& =\& \{\textbackslash{}mathop\{∑
\}\}\_\{i=1\}\^{}\{n\}\{ ∂\{x\}\_\{i\} \textbackslash{}over
∂\{x\}\_\{j\}\} \{ \textbackslash{}mathop\{\textbackslash{}mathop\{∫ \}
\} \}\_\{0\}\^{}\{1\}\{c\}\_\{ i\}(tx) dt +\{ \textbackslash{}mathop\{∑
\}\}\_\{i=1\}\^{}\{n\}\{x\}\_\{ i\}\{ ∂ \textbackslash{}over
∂\{x\}\_\{j\}\} \{ \textbackslash{}mathop\{\textbackslash{}mathop\{∫ \}
\} \}\_\{0\}\^{}\{1\}\{c\}\_\{ i\}(tx) dt\%\&
\textbackslash{}\textbackslash{} \& =\& \{\textbackslash{}mathop\{∫ \}
\}\_\{0\}\^{}\{1\}\{c\}\_\{ j\}(tx) dt +\{ \textbackslash{}mathop\{∑
\}\}\_\{i=1\}\^{}\{n\}\{x\}\_\{ i\}\{
\textbackslash{}mathop\{\textbackslash{}mathop\{∫ \} \}
\}\_\{0\}\^{}\{1\}t\{ ∂\{c\}\_\{i\} \textbackslash{}over ∂\{x\}\_\{j\}\}
(tx) dt \%\& \textbackslash{}\textbackslash{} \& =\&
\{\textbackslash{}mathop\{∫ \} \}\_\{0\}\^{}\{1\}\textbackslash{}left
(\{c\}\_\{ j\}(tx) +\{ \textbackslash{}mathop\{∑
\}\}\_\{i=1\}\^{}\{n\}t\{x\}\_\{ i\}\{ ∂\{c\}\_\{i\}
\textbackslash{}over ∂\{x\}\_\{j\}\} (tx)\textbackslash{}right ) dt \%\&
\textbackslash{}\textbackslash{} \textbackslash{}end\{eqnarray*\}

Utilisons alors \{ ∂\{c\}\_\{j\} \textbackslash{}over ∂\{x\}\_\{i\}\}
=\{ ∂\{c\}\_\{i\} \textbackslash{}over ∂\{x\}\_\{j\}\} . On obtient

\textbackslash{}begin\{eqnarray*\}\{ ∂f \textbackslash{}over
∂\{x\}\_\{j\}\} (x)\& =\& \{\textbackslash{}mathop\{∫ \}
\}\_\{0\}\^{}\{1\}\textbackslash{}left (\{c\}\_\{ j\}(tx) +\{
\textbackslash{}mathop\{∑ \}\}\_\{i=1\}\^{}\{n\}t\{x\}\_\{ i\}\{
∂\{c\}\_\{j\} \textbackslash{}over ∂\{x\}\_\{i\}\}
(tx)\textbackslash{}right ) dt\%\& \textbackslash{}\textbackslash{} \&
=\& \{\textbackslash{}mathop\{∫ \} \}\_\{0\}\^{}\{1\}\{ d
\textbackslash{}over dt\} \textbackslash{}left
(t\{c\}\_\{j\}(t\{x\}\_\{1\},\textbackslash{}mathop\{\textbackslash{}mathop\{\ldots{}\}\},t\{x\}\_\{n\})\textbackslash{}right
) dt \%\& \textbackslash{}\textbackslash{} \& =\& \textbackslash{}big
\{{[}t\{c\}\_\{j\}(tx)\textbackslash{}big {]}\}\_\{0\}\^{}\{1\} =
\{c\}\_\{ j\}(x) \%\& \textbackslash{}\textbackslash{}
\textbackslash{}end\{eqnarray*\}

Ceci montre à la fois que f est de classe \{C\}\^{}\{2\} et que ω = df.

En réutilisant les calculs faits dans \{ℝ\}\^{}\{3\}, nous pouvons
traduire ce résultat sous la forme

Corollaire~15.3.6 Soit U ⊂ \{ℝ\}\^{}\{n\} un ouvert étoilé en a ∈ U
(c'est-à-dire que \textbackslash{}mathop\{∀\}x ∈ U, {[}a,x{]} ⊂ U), soit
V un champ de vecteurs de classe \{C\}\^{}\{1\} sur U. Alors les
conditions suivantes sont équivalentes (i) il existe une fonction f de
classe \{C\}\^{}\{2\} telle que V =\textbackslash{}mathop\{ grad\}f
(resp. il existe un champ de vecteurs W de classe \{C\}\^{}\{2\} tel que
V =\textbackslash{}mathop\{ rot\}W) (ii) \textbackslash{}mathop\{rot\}V
= 0 (resp. \textbackslash{}mathop\{div\} V = 0)

Remarque~15.3.6 Dans le premier cas, on dit que V dérive du potentiel
scalaire f, dans le deuxième cas qu'il dérive du potentiel vecteur W.

{[}\href{coursse85.html}{next}{]} {[}\href{coursse83.html}{prev}{]}
{[}\href{coursse83.html\#tailcoursse83.html}{prev-tail}{]}
{[}\href{coursse84.html}{front}{]}
{[}\href{coursch16.html\#coursse84.html}{up}{]}

\end{document}
