\documentclass[]{article}
\usepackage[T1]{fontenc}
\usepackage{lmodern}
\usepackage{amssymb,amsmath}
\usepackage{ifxetex,ifluatex}
\usepackage{fixltx2e} % provides \textsubscript
% use upquote if available, for straight quotes in verbatim environments
\IfFileExists{upquote.sty}{\usepackage{upquote}}{}
\ifnum 0\ifxetex 1\fi\ifluatex 1\fi=0 % if pdftex
  \usepackage[utf8]{inputenc}
\else % if luatex or xelatex
  \ifxetex
    \usepackage{mathspec}
    \usepackage{xltxtra,xunicode}
  \else
    \usepackage{fontspec}
  \fi
  \defaultfontfeatures{Mapping=tex-text,Scale=MatchLowercase}
  \newcommand{\euro}{€}
\fi
% use microtype if available
\IfFileExists{microtype.sty}{\usepackage{microtype}}{}
\ifxetex
  \usepackage[setpagesize=false, % page size defined by xetex
              unicode=false, % unicode breaks when used with xetex
              xetex]{hyperref}
\else
  \usepackage[unicode=true]{hyperref}
\fi
\hypersetup{breaklinks=true,
            bookmarks=true,
            pdfauthor={},
            pdftitle={Formes differentielles},
            colorlinks=true,
            citecolor=blue,
            urlcolor=blue,
            linkcolor=magenta,
            pdfborder={0 0 0}}
\urlstyle{same}  % don't use monospace font for urls
\setlength{\parindent}{0pt}
\setlength{\parskip}{6pt plus 2pt minus 1pt}
\setlength{\emergencystretch}{3em}  % prevent overfull lines
\setcounter{secnumdepth}{0}
 
/* start css.sty */
.cmr-5{font-size:50%;}
.cmr-7{font-size:70%;}
.cmmi-5{font-size:50%;font-style: italic;}
.cmmi-7{font-size:70%;font-style: italic;}
.cmmi-10{font-style: italic;}
.cmsy-5{font-size:50%;}
.cmsy-7{font-size:70%;}
.cmex-7{font-size:70%;}
.cmex-7x-x-71{font-size:49%;}
.msbm-7{font-size:70%;}
.cmtt-10{font-family: monospace;}
.cmti-10{ font-style: italic;}
.cmbx-10{ font-weight: bold;}
.cmr-17x-x-120{font-size:204%;}
.cmsl-10{font-style: oblique;}
.cmti-7x-x-71{font-size:49%; font-style: italic;}
.cmbxti-10{ font-weight: bold; font-style: italic;}
p.noindent { text-indent: 0em }
td p.noindent { text-indent: 0em; margin-top:0em; }
p.nopar { text-indent: 0em; }
p.indent{ text-indent: 1.5em }
@media print {div.crosslinks {visibility:hidden;}}
a img { border-top: 0; border-left: 0; border-right: 0; }
center { margin-top:1em; margin-bottom:1em; }
td center { margin-top:0em; margin-bottom:0em; }
.Canvas { position:relative; }
li p.indent { text-indent: 0em }
.enumerate1 {list-style-type:decimal;}
.enumerate2 {list-style-type:lower-alpha;}
.enumerate3 {list-style-type:lower-roman;}
.enumerate4 {list-style-type:upper-alpha;}
div.newtheorem { margin-bottom: 2em; margin-top: 2em;}
.obeylines-h,.obeylines-v {white-space: nowrap; }
div.obeylines-v p { margin-top:0; margin-bottom:0; }
.overline{ text-decoration:overline; }
.overline img{ border-top: 1px solid black; }
td.displaylines {text-align:center; white-space:nowrap;}
.centerline {text-align:center;}
.rightline {text-align:right;}
div.verbatim {font-family: monospace; white-space: nowrap; text-align:left; clear:both; }
.fbox {padding-left:3.0pt; padding-right:3.0pt; text-indent:0pt; border:solid black 0.4pt; }
div.fbox {display:table}
div.center div.fbox {text-align:center; clear:both; padding-left:3.0pt; padding-right:3.0pt; text-indent:0pt; border:solid black 0.4pt; }
div.minipage{width:100%;}
div.center, div.center div.center {text-align: center; margin-left:1em; margin-right:1em;}
div.center div {text-align: left;}
div.flushright, div.flushright div.flushright {text-align: right;}
div.flushright div {text-align: left;}
div.flushleft {text-align: left;}
.underline{ text-decoration:underline; }
.underline img{ border-bottom: 1px solid black; margin-bottom:1pt; }
.framebox-c, .framebox-l, .framebox-r { padding-left:3.0pt; padding-right:3.0pt; text-indent:0pt; border:solid black 0.4pt; }
.framebox-c {text-align:center;}
.framebox-l {text-align:left;}
.framebox-r {text-align:right;}
span.thank-mark{ vertical-align: super }
span.footnote-mark sup.textsuperscript, span.footnote-mark a sup.textsuperscript{ font-size:80%; }
div.tabular, div.center div.tabular {text-align: center; margin-top:0.5em; margin-bottom:0.5em; }
table.tabular td p{margin-top:0em;}
table.tabular {margin-left: auto; margin-right: auto;}
div.td00{ margin-left:0pt; margin-right:0pt; }
div.td01{ margin-left:0pt; margin-right:5pt; }
div.td10{ margin-left:5pt; margin-right:0pt; }
div.td11{ margin-left:5pt; margin-right:5pt; }
table[rules] {border-left:solid black 0.4pt; border-right:solid black 0.4pt; }
td.td00{ padding-left:0pt; padding-right:0pt; }
td.td01{ padding-left:0pt; padding-right:5pt; }
td.td10{ padding-left:5pt; padding-right:0pt; }
td.td11{ padding-left:5pt; padding-right:5pt; }
table[rules] {border-left:solid black 0.4pt; border-right:solid black 0.4pt; }
.hline hr, .cline hr{ height : 1px; margin:0px; }
.tabbing-right {text-align:right;}
span.TEX {letter-spacing: -0.125em; }
span.TEX span.E{ position:relative;top:0.5ex;left:-0.0417em;}
a span.TEX span.E {text-decoration: none; }
span.LATEX span.A{ position:relative; top:-0.5ex; left:-0.4em; font-size:85%;}
span.LATEX span.TEX{ position:relative; left: -0.4em; }
div.float img, div.float .caption {text-align:center;}
div.figure img, div.figure .caption {text-align:center;}
.marginpar {width:20%; float:right; text-align:left; margin-left:auto; margin-top:0.5em; font-size:85%; text-decoration:underline;}
.marginpar p{margin-top:0.4em; margin-bottom:0.4em;}
.equation td{text-align:center; vertical-align:middle; }
td.eq-no{ width:5%; }
table.equation { width:100%; } 
div.math-display, div.par-math-display{text-align:center;}
math .texttt { font-family: monospace; }
math .textit { font-style: italic; }
math .textsl { font-style: oblique; }
math .textsf { font-family: sans-serif; }
math .textbf { font-weight: bold; }
.partToc a, .partToc, .likepartToc a, .likepartToc {line-height: 200%; font-weight:bold; font-size:110%;}
.chapterToc a, .chapterToc, .likechapterToc a, .likechapterToc, .appendixToc a, .appendixToc {line-height: 200%; font-weight:bold;}
.index-item, .index-subitem, .index-subsubitem {display:block}
.caption td.id{font-weight: bold; white-space: nowrap; }
table.caption {text-align:center;}
h1.partHead{text-align: center}
p.bibitem { text-indent: -2em; margin-left: 2em; margin-top:0.6em; margin-bottom:0.6em; }
p.bibitem-p { text-indent: 0em; margin-left: 2em; margin-top:0.6em; margin-bottom:0.6em; }
.paragraphHead, .likeparagraphHead { margin-top:2em; font-weight: bold;}
.subparagraphHead, .likesubparagraphHead { font-weight: bold;}
.quote {margin-bottom:0.25em; margin-top:0.25em; margin-left:1em; margin-right:1em; text-align:\jmathustify;}
.verse{white-space:nowrap; margin-left:2em}
div.maketitle {text-align:center;}
h2.titleHead{text-align:center;}
div.maketitle{ margin-bottom: 2em; }
div.author, div.date {text-align:center;}
div.thanks{text-align:left; margin-left:10%; font-size:85%; font-style:italic; }
div.author{white-space: nowrap;}
.quotation {margin-bottom:0.25em; margin-top:0.25em; margin-left:1em; }
h1.partHead{text-align: center}
.sectionToc, .likesectionToc {margin-left:2em;}
.subsectionToc, .likesubsectionToc {margin-left:4em;}
.subsubsectionToc, .likesubsubsectionToc {margin-left:6em;}
.frenchb-nbsp{font-size:75%;}
.frenchb-thinspace{font-size:75%;}
.figure img.graphics {margin-left:10%;}
/* end css.sty */

\title{Formes differentielles}
\author{}
\date{}

\begin{document}
\maketitle

\textbf{Warning: 
requires JavaScript to process the mathematics on this page.\\ If your
browser supports JavaScript, be sure it is enabled.}

\begin{center}\rule{3in}{0.4pt}\end{center}

{[}
{[}
{[}{]}
{[}

\subsubsection{15.3 Formes différentielles}

Remarque~15.3.1 En dehors de la notion de gradient, cette section ne
fait pas partie du programme des classes préparatoires. Cependant, les
formes différentielles de degré 1 sont un outil particulièrement commode
même à ce niveau.

\paragraph{15.3.1 Rappels sur les formes linéaires alternées}

Proposition~15.3.1 Soit E un \mathbb{R}~-espace vectoriel,
f\_1,\\ldots,f\_p~
\in E^∗. Alors f\_1
∧\\ldots~ ∧
f\_p : E^p \rightarrow~ K définie par
(x\_1,\\ldots,x\_p)\mapsto~\\mathrm{det}~
(f\_i(x\_\jmath))\_1\leqi\leqp,1\leq\jmath\leqp est une forme p-
linéaire alternée sur E. L'application (E^∗)^p \rightarrow~
A\_p(E),
(f\_1,\\ldots,f\_p)\mapsto~f\_1~
∧\\ldots~ ∧
f\_p est elle même p-linéaire et alternée.

Ceci permet d'exhiber une base de l'espace A\_p(E) des formes
p-linéaires alternées sur E. Pour cela soit E un K-espace vectoriel de
dimension n et
(e\_1,\\ldots,e\_n~)
une base de E.

Théorème~15.3.2 La famille des
(e\_i\_1^∗∧\\ldots~
∧
e\_i\_p^∗)\_1\leqi\_1\textless{}i\_2\textless{}\\ldots\textless{}i\_p\leqn~
est une base de A\_p(E) (qui est donc de dimension
C\_n^p).

\paragraph{15.3.2 Notion de forme différentielle}

Définition~15.3.1 Soit U un ouvert de \mathbb{R}~^n. On appelle forme
différentielle de degré p sur U toute application de U dans
A\_p(\mathbb{R}~^n) (en posant par convention
A\_0(\mathbb{R}~^n) = \mathbb{R}~).

Remarque~15.3.2 Soit \omega : U \rightarrow~ A\_p(\mathbb{R}~^n) une forme
différentielle de degré p. Soit
(e\_1,\\ldots,e\_n~)
la base canonique de \mathbb{R}~^n et
(e\_i\_1^∗∧\\ldots~
∧
e\_i\_p^∗)\_1\leqi\_1\textless{}i\_2\textless{}\\ldots\textless{}i\_p\leqn~
la base correspondante de A\_p(\mathbb{R}~^n). On a alors, pour
x \in U, \omega(x) = \\sum ~
\_1\leqi\_1\textless{}i\_2\textless{}\\ldots\textless{}i\_p\leqna\_i\_1,\\\ldots,i\_p(x)e\_i\_1^∗∧\\\ldots~
∧ e\_i\_p^∗. On dit que \omega est de classe
C^k si toutes les applications
a\_i\_1,\\ldots,i\_p~
: U \rightarrow~ \mathbb{R}~ sont de classe C^k.

Remarque~15.3.3 Soit f : U \rightarrow~ \mathbb{R}~ de classe \mathcal{C}^1. Alors pour tout
x \in U, df(x) est une application linéaire de \mathbb{R}~^n dans \mathbb{R}~ donc
une forme linéaire sur \mathbb{R}~^n, donc un élément de
(\mathbb{R}~^n)^∗ = A\_1(E). On en déduit que df :
x\mapsto~df(x) est une forme différentielle de degré
1 sur U. On sait que

df(x).h = \sum \_i=1^n~ \partial~f
\over \partial~x\_i (x)h\_i =
\sum \_i=1^n~ \partial~f
\over \partial~x\_i (x)e\_i^∗(h)

On en déduit que df(x) =\
\sum  \_i=1^n~ \partial~f
\over \partial~x\_i (x)e\_i^∗. Prenons
par exemple f = e\_i^∗. On a \forall~~x
\in U, df(x) = e\_i^∗. Si on note x\_i,
l'application i-ième coordonnée (c'est-à-dire encore
e\_i^∗), on a donc dx\_i = e\_i^∗
si bien que l'on peut noter df(x) =\
\sum  \_i=1^n~ \partial~f
\over \partial~x\_i (x)dx\_i. Plus
généralement, une forme différentielle de degré p sur U sera de la forme

\omega(x) = \\sum
\_1\leqi\_1\textless{}i\_2\textless{}\ldots\textless{}i\_p\leqna\_i\_1,\\ldots,i\_p(x)dx\_i\_1~
∧\ldots ∧ dx\_i\_p~

C'est cette dernière forme que nous utiliserons par la suite, avec comme
seule propriété à connaître le fait que ∧ est multilinéaire et alternée.

Exemple~15.3.1 Dans le cas de la dimension 3 et de p = 2, on préfère
utiliser une base invariante par permutation circulaire, à savoir
dx\_2 ∧ dx\_3,dx\_3 ∧
dx\_1,dx\_1 ∧ dx\_2. On aura ainsi les
expressions générales de formes différentielles de degré p sur un ouvert
de \mathbb{R}~^n.

p = 0~: dans tous les cas, une forme différentielle de degré 0 est
simplement une fonction à valeurs réelles et une forme différentielle de
degré 1 s'écrit

\omega(x\_1,\\ldots,x\_n~))
=
a\_1(x\_1,\\ldots,x\_n)dx\_1~
+ \\ldots~ +
a\_n(x\_1,\\ldots,x\_n)dx\_n~

\begin{align*} n = 2,p = 2& :&
\omega(x\_1,x\_2) = a(x\_1,x\_2)dx\_1
∧ dx\_2 \%& \\ n = 3,p = 2& :&
\omega(x\_1,x\_2,x\_3) =
a\_1(x\_1,x\_2,x\_3)dx\_2 ∧
dx\_3 \%& \\ & &
+a\_2(x\_1,x\_2,x\_3)dx\_3 ∧
dx\_1 +
a\_3(x\_1,x\_2,x\_3)dx\_1 ∧
dx\_2\%& \\ n = 3,p = 3& :&
\omega(x\_1,x\_2,x\_3) =
a(x\_1,x\_2,x\_3)dx\_1 ∧ dx\_2 ∧
dx\_3 \%& \\
\end{align*}

\paragraph{15.3.3 Notion de gradient d'une fonction}

Soit E un espace euclidien, U un ouvert de E et f : U \rightarrow~ \mathbb{R}~ de classe
\mathcal{C}^1. Alors, pour x \in E, df(x) est une forme linéaire sur E~;
on sait qu'il existe un unique vecteur noté
grad~f(x) dans E tel que
\forall~~h \in E, df(x).h =
(gradf(x)\mathrel∣~h).

Définition~15.3.2 Le vecteur grad~f(x) défini
par \forall~~h \in E, df(x).h =
(gradf(x)\mathrel∣~h), est
appelé gradient de f au point x.

Remarque~15.3.4 Supposons que E = \mathbb{R}~^n muni de sa structure
euclidienne naturelle (celle qui rend la base canonique orthonormée).
Alors

df(x).h = \sum \_i=1^nh~\_
i \partial~f \over \partial~x\_i (x) =
(\sum \_i=1^n~ \partial~f
\over \partial~x\_i
(x)e\_i∣h)

si bien que l'on retrouve l'expression classique du gradient de f

grad~f(x) = \\sum
\_i=1^n \partial~f \over \partial~x\_i
(x)e\_i = ( \partial~f \over \partial~x\_1
(x),\ldots~, \partial~f \over
\partial~x\_n (x))

\paragraph{15.3.4 Invariance de la différentielle}

Soit U un ouvert de \mathbb{R}~^p et f : U \rightarrow~ \mathbb{R}~ de classe
\mathcal{C}^1. Soit V un ouvert de \mathbb{R}~^n et soit \phi =
(\phi\_1,\\ldots,\phi\_p~)
: V \rightarrow~ U. Posons y\_1 =
\phi\_1(x\_1,\\ldots,x\_n),\\\ldots,y\_p~
=
\phi\_p(x\_1,\\ldots,x\_n~).
On a donc dy\_\jmath =\
\sum  \_i=1^n \partial~\phi\_\jmath~
\over \partial~x\_i dx\_i. De plus,
f(y\_1,\\ldots,y\_p~)
=
f(\phi\_1(x\_1,\\ldots,x\_n),\\\ldots,\phi\_p(x\_1,\\\ldots,x\_n~))
si bien que

\begin{align*}
d(f(y\_1,\\ldots,y\_p~))&&
\%& \\ & =& \\sum
\_i=1^n \partial~ \over \partial~x\_i
\left
(f(\phi\_1(x\_1,\ldots,x\_n),\\ldots,\phi\_p(x\_1,\\ldots,x\_n~))\right
)dx\_i \%& \\ & =&
\sum \_i=1^n~\left
(\sum \_\jmath=1^p~ \partial~f
\over \partial~y\_\jmath
(\phi\_1(x\_1,\ldots,x\_n),\\ldots,\phi\_p(x\_1,\\ldots,x\_n~))
\partial~\phi\_\jmath \over \partial~x\_i \right
)dx\_i\%& \\ & =&
\sum \_\jmath=1^p~ \partial~f
\over \partial~y\_\jmath
(y\_1,\ldots,y\_p~)\left
(\sum \_i=1^n \partial~\phi\_\jmath~
\over \partial~x\_i dx\_i\right
) \%& \\ & =&
\sum \_\jmath=1^p~ \partial~f
\over \partial~y\_\jmath
(y\_1,\ldots,y\_p)dy\_\jmath~
\%& \\ \end{align*}

en utilisant la règle de dérivation partielle des fonctions composées et
en intervertissant les deux sommations.

On voit donc que la formule
d(f(y\_1,\\ldots,y\_p~))
= \\sum ~
\_\jmath=1^p \partial~f \over \partial~y\_\jmath
(y\_1,\\ldots,y\_p)dy\_\jmath~
est valable aussi bien quand
y\_1,\\ldots,y\_p~
désignent des variables libres (c'est-à-dire qui varient dans un ouvert
de \mathbb{R}~^p) que lorsque
y\_1,\\ldots,y\_p~
désignent des fonctions d'autres variables (ici
x\_1,\\ldots,x\_n~).
C'est une propriété essentielle de la différentielle qui fait tout
l'intérêt des formes différentielles (en particulier de degré 1)~: on
peut différentier une expression sans savoir quelles sont les variables
et quelles sont les fonctions.

On prendra simplement garde au fait suivant~: lorsque
y\_1,\\ldots,y\_p~
désignent des variables libres, qui varient dans des ouverts de
\mathbb{R}~^p, on a dy\_\jmath = e\_\jmath^∗, et donc les
formes différentielles
dy\_1,\\ldots,dy\_p~
forment une famille libre (ce qui permet en particulier des
identifications)~; il n'en est évidemment plus de même lorsque
y\_1,\\ldots,y\_p~
sont elles mêmes des fonctions d'autres variables
x\_1,\\ldots,x\_n~.

\paragraph{15.3.5 Différentielle extérieure}

Définition~15.3.3 Soit \omega(x) =\
\sum ~
\_1\leqi\_1\textless{}i\_2\textless{}\\ldots\textless{}i\_p\leqna\_i\_1,\\\ldots,i\_p(x)dx\_i\_1~
∧\\ldots~ ∧
dx\_i\_p une forme différentielle de degré p, de classe
\mathcal{C}^1 sur l'ouvert U de \mathbb{R}~^n. On appelle
différentielle extérieure de \omega, la forme différentielle de degré p + 1
définie par

d\omega(x) = \\sum
\_1\leqi\_1\textless{}i\_2\textless{}\ldots\textless{}i\_p\leqnda\_i\_1,\\ldots,i\_p~(x)
∧ dx\_i\_1 ∧\ldots~ ∧
dx\_i\_p

Remarque~15.3.5 Le calcul effectif se fait en utilisant la définition de
da\_i\_1,\\ldots,i\_p~(x)
et les propriétés de l'opérateur ∧~: linéaire par rapport à chaque
terme, alterné, antisymétrique. On a donc

\begin{align*}
da\_i\_1,\\ldots,i\_p~(x)
∧ dx\_i\_1
∧\\ldots~ ∧
dx\_i\_p&& \%& \\ &
=& \left (\\sum
\_\jmath=1^n
\partial~a\_i\_1,\ldots,i\_p~
\over \partial~x\_\jmath
\,dx\_\jmath\right ) ∧
dx\_i\_1
∧\\ldots~ ∧
dx\_i\_p\%& \\ & =&
\sum \_\jmath=1^n~
\partial~a\_i\_1,\ldots,i\_p~
\over \partial~x\_\jmath \,dx\_\jmath ∧
dx\_i\_1 ∧\ldots~ ∧
dx\_i\_p \%& \\
\end{align*}

avec dx\_\jmath ∧ dx\_i\_1
∧\\ldots~ ∧
dx\_i\_p = 0 si \jmath
\in\i\_1,\\ldots,i\_p\~
et dx\_\jmath ∧ dx\_i\_1
∧\\ldots~ ∧
dx\_i\_p = ±dx\_i\_1
∧\\ldots~ ∧
dx\_\jmath
∧\\ldots~ ∧
dx\_i\_p où l'on met de signe + ou le signe - suivant la
parité du nombre de transpositions nécessaires pour intercaler \jmath à la
bonne place dans la suite
\i\_1,\\ldots,i\_p\~.
Dans le cas d'une forme différentielle de degré 0 (une fonction), on
trouve bien entendu tout simplement la différentielle de la fonction.

Exemple~15.3.2 Calcul dans le cas n = 3. Si p = 0, on a \omega = f et d\omega =
\partial~f \over \partial~x\_1 dx\_1 + \partial~f
\over \partial~x\_2 dx\_2 + \partial~f
\over \partial~x\_3 dx\_3 et on retrouve
l'expression du gradient de la fonction f.

Si p = 1, on a \omega(x) = a\_1(x)dx\_1 +
a\_2(x)dx\_2 + a\_3(x)dx\_3, et donc

\begin{align*} d\omega(x)& =& da\_1(x) ∧
dx\_1 + da\_2(x) ∧ dx\_2 + da\_3(x) ∧
dx\_3 \%& \\ & =& (
\partial~a\_1 \over \partial~x\_1 (x)dx\_1 +
\partial~a\_1 \over \partial~x\_2 (x)dx\_2 +
\partial~a\_1 \over \partial~x\_3 (x)dx\_3) ∧
dx\_1 \%& \\ & & +(
\partial~a\_2 \over \partial~x\_1 (x)dx\_1 +
\partial~a\_2 \over \partial~x\_2 (x)dx\_2 +
\partial~a\_2 \over \partial~x\_3 (x)dx\_3) ∧
dx\_2\%& \\ & & +(
\partial~a\_3 \over \partial~x\_1 (x)dx\_1 +
\partial~a\_3 \over \partial~x\_2 (x)dx\_2 +
\partial~a\_3 \over \partial~x\_3 (x)dx\_3) ∧
dx\_3\%& \\ & =& (
\partial~a\_3 \over \partial~x\_2 (x) - \partial~a\_2
\over \partial~x\_3 (x))dx\_2 ∧ dx\_3
\%& \\ & & +( \partial~a\_1
\over \partial~x\_3 (x) - \partial~a\_3
\over \partial~x\_1 (x))dx\_3 ∧ dx\_1
\%& \\ & & +( \partial~a\_2
\over \partial~x\_1 (x) - \partial~a\_1
\over \partial~x\_2 (x))dx\_1 ∧ dx\_2
\%& \\ \end{align*}

en tenant compte de dx\_i ∧ dx\_i = 0 et de
dx\_i ∧ dx\_\jmath = -dx\_\jmath ∧ dx\_i. On
reconnaît là l'expression classique du rotationnel du champ de vecteurs
de composantes (a\_1(x),a\_2(x),a\_3(x)).

Si p = 2, on a \omega(x) = a\_1(x)dx\_2 ∧ dx\_3 +
a\_2(x)dx\_3 ∧ dx\_1 +
a\_3(x)dx\_1 ∧ dx\_2 et donc

\begin{align*} d\omega(x)& =& da\_1(x) ∧
dx\_2 ∧ dx\_3 + da\_2(x) ∧ dx\_3 ∧
dx\_1 \%& \\ & &
+da\_3(x) ∧ dx\_1 ∧ dx\_2 \%&
\\ & =& ( \partial~a\_1
\over \partial~x\_1 (x)dx\_1 + \partial~a\_1
\over \partial~x\_2 (x)dx\_2 + \partial~a\_1
\over \partial~x\_3 (x)dx\_3) ∧ dx\_2 ∧
dx\_3 \%& \\ & & +(
\partial~a\_2 \over \partial~x\_1 (x)dx\_1 +
\partial~a\_2 \over \partial~x\_2 (x)dx\_2 +
\partial~a\_2 \over \partial~x\_3 (x)dx\_3) ∧
dx\_3 ∧ dx\_1\%& \\ & &
+( \partial~a\_3 \over \partial~x\_1 (x)dx\_1
+ \partial~a\_3 \over \partial~x\_2 (x)dx\_2
+ \partial~a\_3 \over \partial~x\_3 (x)dx\_3)
∧ dx\_1 ∧ dx\_2\%& \\ &
=& \left ( \partial~a\_1 \over
\partial~x\_1 (x) + \partial~a\_2 \over
\partial~x\_2 (x) + \partial~a\_3 \over
\partial~x\_3 (x)\right )dx\_1 ∧ dx\_2
∧ dx\_3 \%& \\
\end{align*}

en tenant compte de dx\_i ∧ dx\_\jmath ∧ dx\_k = 0 si
i,\jmath et k ne sont pas distincts et de dx\_\jmath ∧ dx\_k ∧
dx\_i = dx\_i ∧ dx\_\jmath ∧ dx\_k si i,\jmath,k
sont distincts (les permutations circulaires de trois éléments sont de
signature + 1). On reconnaît là l'expression classique de la divergence
du champ de vecteurs de composantes
(a\_1(x),a\_2(x),a\_3(x)).

La différentielle extérieure des formes différentielles est donc une
généralisation (et une unification) des notions classiques de gradient
d'une fonction et de rotationnel ou divergence d'un champ de vecteurs.

\paragraph{15.3.6 Théorème de Poincaré}

Théorème~15.3.3 Soit U un ouvert de \mathbb{R}~^n et \omega une forme
différentielle de degré p de classe C^2 sur U. Alors d(d\omega) =
0.

Démonstration On a

\begin{align*} d\omega& =& \\sum
\_1\leqi\_1\textless{}i\_2\textless{}\ldots\textless{}i\_p\leqnda\_i\_1,\\ldots,i\_p~
∧ dx\_i\_1 ∧\ldots~ ∧
dx\_i\_p \%& \\ & =&
\\sum
\_1\leqi\_1\textless{}i\_2\textless{}\ldots\textless{}i\_p\leqn~\left
(\sum \_\jmath=1^n~
\partial~a\_i\_1,\ldots,i\_p~
\over \partial~x\_\jmath
\,dx\_\jmath\right ) ∧
dx\_i\_1 ∧\ldots~ ∧
dx\_i\_p\%& \\ & =&
\sum \_\jmath=1^n~
\\sum
\_1\leqi\_1\textless{}i\_2\textless{}\ldots\textless{}i\_p\leqn~
\partial~a\_i\_1,\ldots,i\_p~
\over \partial~x\_\jmath dx\_\jmath ∧
dx\_i\_1 ∧\ldots~ ∧
dx\_i\_p \%& \\
\end{align*}

d'où

\begin{align*} d(d\omega)&& \%&
\\ & =& \\sum
\_\jmath=1^n \\sum
\_1\leqi\_1\textless{}i\_2\textless{}\ldots\textless{}i\_p\leqn~d\left
(
\partial~a\_i\_1,\ldots,i\_p~
\over \partial~x\_\jmath \right ) ∧
dx\_\jmath ∧ dx\_i\_1
∧\ldots ∧ dx\_i\_p~ \%&
\\ & =& \\sum
\_\jmath=1^n \\sum
\_1\leqi\_1\textless{}i\_2\textless{}\ldots\textless{}i\_p\leqn~\left
(\sum \_k=1^n~
\partial~^2a\_
i\_1,\ldots,i\_p~
\over \partial~x\_k\partial~x\_\jmath
dx\_k\right ) ∧ dx\_\jmath ∧
dx\_i\_1 ∧\ldots~ ∧
dx\_i\_p \%& \\ & =&
\sum \_k=1^n~
\sum \_\jmath=1^n~
\\sum
\_1\leqi\_1\textless{}i\_2\textless{}\ldots\textless{}i\_p\leqn~
\partial~^2a\_i\_1,\ldots,i\_p~
\over \partial~x\_k\partial~x\_\jmath dx\_k ∧
dx\_\jmath ∧ dx\_i\_1
∧\ldots ∧ dx\_i\_p~ \%&
\\ & =& \\sum
\_k\textless{}\jmath \\sum
\_1\leqi\_1\textless{}i\_2\textless{}\ldots\textless{}i\_p\leqn~\left
(
\partial~^2a\_i\_1,\ldots,i\_p~
\over \partial~x\_k\partial~x\_\jmath -
\partial~^2a\_i\_1,\ldots,i\_p~
\over \partial~x\_\jmath\partial~x\_k \right
)dx\_k ∧ dx\_\jmath ∧ dx\_i\_1
∧\ldots ∧ dx\_i\_p~\%&
\\ \end{align*}

en tenant compte de dx\_\jmath ∧ dx\_k = 0 si \jmath = k et
dx\_\jmath ∧ dx\_k = -dx\_k ∧ dx\_\jmath si
\jmath\neq~k. Mais le théorème de Schwarz montre que
 \partial~^2a\_
i\_1,\\ldots,i\_p~
\over \partial~x\_k\partial~x\_\jmath =
\partial~^2a\_
i\_1,\\ldots,i\_p~
\over \partial~x\_\jmath\partial~x\_k et donc d(d\omega) = 0.

En tenant compte des expressions trouvées pour d\omega dans le cas n = 3, on
obtient donc le corollaire suivant

Corollaire~15.3.4 (i) Soit f une fonction de classe C^2 sur
un ouvert U de \mathbb{R}~^3. Alors
rot\grad~f = 0 (ii)
Soit V un champ de vecteurs de classe C^2 sur un ouvert U de
\mathbb{R}~^3. Alors
div\rot~V = 0

Nous allons maintenant nous intéresser à la réciproque du théorème
précédent

Théorème~15.3.5 (Poincaré). Soit U \subset~ \mathbb{R}~^n un ouvert étoilé en
a \in U (c'est-à-dire que \forall~~x \in U, {[}a,x{]} \subset~
U). Soit \omega une forme différentielle de degré p ≥ 1 de classe
\mathcal{C}^1 sur U. Alors les conditions suivantes sont équivalentes
(i) d\omega = 0 (ii) \omega est exacte~: il existe une forme différentielle \alpha~ de
degré p - 1 de classe C^2 sur U telle que \omega = d\alpha~.

Démonstration Le théorème précédent implique clairement que (ii) \rigtharrow~(i).
Nous nous contenterons de démontrer que (i) \rigtharrow~(ii) lorsque p = 1, en
admettant le cas général. Par une translation, sans nuire à la
généralité, on peut supposer que a = 0. Soit U \subset~ \mathbb{R}~^n un
ouvert étoilé en 0 \in U et soit \omega =\
\sum ~
\_i=1^nc\_i(x)dx\_i. On a par un calcul
facile

d\omega = \\sum
\_i\textless{}\jmath\left ( \partial~c\_\jmath
\over \partial~x\_i - \partial~c\_i
\over \partial~x\_\jmath \right
)dx\_i ∧ dx\_\jmath

Donc d\omega = 0 \Leftrightarrow
\forall~i,\jmath, \partial~c\_\jmath~ \over
\partial~x\_i = \partial~c\_i \over \partial~x\_\jmath .

Définissons f : U \rightarrow~ \mathbb{R}~ par f(x) =\
\sum ~
\_i=1^nx\_i\int ~
\_0^1c\_i(tx) dt. Comme
(t,x\_\jmath)\mapsto~c\_i(tx) =
c\_i(tx\_1,\\ldots,tx\_n~)
admet une dérivée partielle par rapport à x\_\jmath,  \partial~
\over \partial~x\_\jmath
(c\_i(tx\_1,\\ldots,tx\_n~))
= t \partial~c\_i \over \partial~x\_\jmath
(tx\_1,\\ldots,tx\_n~)
qui est une fonction continue du couple (t,x\_\jmath), l'application
x\_\jmath\mapsto~\int ~
\_0^1c\_i(tx) dt est dérivable et  \partial~
\over \partial~x\_\jmath \int ~
\_0^1c\_i(tx) dt =\int ~
\_0^1t \partial~c\_i \over \partial~x\_\jmath
(tx) dt. On en déduit que

\begin{align*} \partial~f \over
\partial~x\_\jmath (x)& =& \\sum
\_i=1^n \partial~x\_i \over
\partial~x\_\jmath  \\int ~
 \_0^1c\_ i(tx) dt + \\sum
\_i=1^nx\_ i \partial~ \over
\partial~x\_\jmath  \\int ~
 \_0^1c\_ i(tx) dt\%&
\\ & =& \int ~
\_0^1c\_ \jmath(tx) dt + \\sum
\_i=1^nx\_ i
\\int  ~
\_0^1t \partial~c\_i \over \partial~x\_\jmath
(tx) dt \%& \\ & =&
\int  \_0^1~\left
(c\_ \jmath(tx) + \\sum
\_i=1^ntx\_ i \partial~c\_i
\over \partial~x\_\jmath (tx)\right ) dt \%&
\\ \end{align*}

Utilisons alors  \partial~c\_\jmath \over \partial~x\_i
= \partial~c\_i \over \partial~x\_\jmath . On obtient

\begin{align*} \partial~f \over
\partial~x\_\jmath (x)& =& \int ~
\_0^1\left (c\_ \jmath(tx) +
\sum \_i=1^ntx\_ i~
\partial~c\_\jmath \over \partial~x\_i
(tx)\right ) dt\%& \\ &
=& \int  \_0^1~ d
\over dt \left
(tc\_\jmath(tx\_1,\\ldots,tx\_n~)\right
) dt \%& \\ & =& \big
{[}tc\_\jmath(tx)\big {]}\_0^1 =
c\_ \jmath(x) \%& \\
\end{align*}

Ceci montre à la fois que f est de classe C^2 et que \omega = df.

En réutilisant les calculs faits dans \mathbb{R}~^3, nous pouvons
traduire ce résultat sous la forme

Corollaire~15.3.6 Soit U \subset~ \mathbb{R}~^n un ouvert étoilé en a \in U
(c'est-à-dire que \forall~~x \in U, {[}a,x{]} \subset~ U), soit
V un champ de vecteurs de classe \mathcal{C}^1 sur U. Alors les
conditions suivantes sont équivalentes (i) il existe une fonction f de
classe C^2 telle que V = grad~f
(resp. il existe un champ de vecteurs W de classe C^2 tel que
V = rotW) (ii) \rot~V
= 0 (resp. div~ V = 0)

Remarque~15.3.6 Dans le premier cas, on dit que V dérive du potentiel
scalaire f, dans le deuxième cas qu'il dérive du potentiel vecteur W.

{[}
{[}
{[}
{[}

\end{document}
