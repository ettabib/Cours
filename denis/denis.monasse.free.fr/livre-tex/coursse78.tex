

\subsubsection{Séries trigonométriques}

\paragraph{Rappels d'intégration}

\begin{lem}
Soit $f : \mathbb{R} \rightarrow \mathbb{C}$ périodique de période $T$, continue par
morceaux. Alors, pour tout $a \in \mathbb{R}$, $\int_a^{a+T} f(t) dt = \int_0^T f(t) dt$.
\end{lem}

\begin{proof}
On écrit 
\begin{align*}
\int_a^{a+T} f(t) dt &= \int_a^0 f(t) dt + \int_0^T f(t) dt + \int_T^{a+T} f(t) dt \\
&= \int_a^0 f(t) dt + \int_0^T f(t) dt + \int_0^a f(u+T) du \\
&= \int_a^0 f(t) dt + \int_0^T f(t) dt + \int_0^a f(u) du \\
&= \int_0^T f(t) dt
\end{align*}
en faisant le changement de variable $u = t - T$.
\end{proof}

\begin{lem}
Pour tout $n \in \mathbb{Z}$, $\int_0^{2\pi} e^{int} dt = 2\pi \delta_n^0$.
\end{lem}

\paragraph{Généralités}

\begin{de}[forme réelle]
Soit $(a_n)_{n \geq 0}$ et
$(b_n)_{n \geq 1}$ deux suites de nombres complexes. On appelle
série trigonométrique associée la série de fonctions de $\mathbb{R}$ dans $\mathbb{C}$,

$a_0 + \sum_{n \geq 1} (a_n \cos nx + b_n \sin nx)$
\end{de}

\begin{rem}
Soit $n \in \mathbb{N}^*$ et $a_n$ et $b_n$
deux nombres complexes. On a alors 
$a_n \cos nx + b_n \sin nx = c_n e^{inx} + c_{-n} e^{-inx}$ 
avec
$c_n = \frac{a_n - ib_n}{2}$ et
$c_{-n} = \frac{a_n + ib_n}{2}$.
Inversement, si on se donne deux nombres complexes $c_n$ et
$c_{-n}$, on a $c_n e^{inx} + c_{-n} e^{-inx} = a_n \cos nx + b_n \sin nx$ avec
$a_n = c_n + c_{-n}$ et $b_n = i(c_n - c_{-n})$. Ceci amène également à poser
\end{rem}

\begin{de}[forme complexe]
Soit $(c_n)_{n \in \mathbb{Z}}$ une
suite de nombres complexes. On appelle série trigonométrique associée la
série de fonctions de $\mathbb{R}$ dans $\mathbb{C}$,

$c_0 + \sum_{n \geq 1} (c_n e^{inx} + c_{-n} e^{-inx})$
\end{de}

On passe donc de la forme réelle à la forme complexe ou vice versa par
les formules

\begin{align*} 
a_0 &= c_0 \\
\forall n \geq 1, \quad c_n &= \frac{a_n - ib_n}{2}, \quad c_{-n} = \frac{a_n + ib_n}{2} \\
\forall n \geq 1, \quad a_n &= c_n + c_{-n}, \quad b_n = i(c_n - c_{-n})
\end{align*}

\paragraph{Un cas de convergence normale}

\begin{thm}
On considère une série trigonométrique vérifiant les
conditions équivalentes

\begin{itemize}
\itemsep1pt\parskip0pt\parsep0pt
\item
  (i) les deux séries $\sum a_n$ et
  $\sum b_n$ sont convergentes.
\item
  (ii) les deux séries
  $\sum_{n \geq 0} c_n$ et
  $\sum_{n \geq 0} c_{-n}$ sont convergentes.
\end{itemize}

Alors la série trigonométrique converge normalement sur $\mathbb{R}$, sa somme $f$
est une fonction continue périodique de période $2\pi$ et on a

\begin{align*} 
\forall n \in \mathbb{Z}, \quad c_n &= \frac{1}{2\pi} \int_0^{2\pi} f(t) e^{-int} dt \\
\forall n \geq 1, \quad a_n &= \frac{1}{\pi} \int_0^{2\pi} f(t) \cos nt dt, \quad b_n = \frac{1}{\pi} \int_0^{2\pi} f(t) \sin nt dt
\end{align*}
\end{thm}

\begin{proof}
Les relations
$|a_n| \leq |c_n| + |c_{-n}|$, 
$|b_n| \leq |c_n| + |c_{-n}|$,
$|c_n| \leq \frac{1}{2} (|a_n| + |b_n|)$ et
$|c_{-n}| \leq \frac{1}{2} (|a_n| + |b_n|)$
(que l'on déduit facilement des relations du paragraphe précédent)
montrent clairement l'équivalence. Alors on a

$\forall x \in \mathbb{R}, |c_n e^{inx} + c_{-n} e^{-inx}| \leq |c_n| + |c_{-n}|$

qui est une série convergente indépendante de $x$. On a donc la
convergence normale de la série et en particulier la continuité de sa
somme. Cette somme est évidemment périodique de période $2\pi$ puisque
toutes les applications
$x \mapsto c_n e^{inx} + c_{-n} e^{-inx}$ le sont. Soit $p \in \mathbb{Z}$. On a aussi
$\forall x \in \mathbb{R},
|(c_n e^{inx} + c_{-n} e^{-inx}) e^{-ipx}| \leq |c_n| + |c_{-n}|$ ce qui montre que la série
$c_0 e^{-ipx} + \sum_{n \geq 1} (c_n e^{inx} + c_{-n} e^{-inx}) e^{-ipx}$ converge normalement
sur $\mathbb{R}$, donc sur $[0,2\pi]$. Ceci justifie donc dans le calcul suivant
l'interversion du signe d'intégrale et du signe somme

\begin{align*} 
\int_0^{2\pi} f(t) e^{-ipt} dt 
&= \int_0^{2\pi} \left(c_0 e^{-ipt} + \sum_{n \geq 1} (c_n e^{int} + c_{-n} e^{-int}) e^{-ipt} \right) dt \\
&= c_0 \int_0^{2\pi} e^{-ipt} dt \\
&\quad + \sum_{n=1}^{+\infty} \left(c_n \int_0^{2\pi} e^{i(n-p)t} dt + c_{-n} \int_0^{2\pi} e^{-i(n+p)t} dt \right) \\
&= 2\pi \left(c_0 \delta_p^0 + \sum_{n=1}^{+\infty} (c_n \delta_p^n + c_{-n} \delta_p^{-n}) \right) = 2\pi c_p
\end{align*}

en distinguant les différents cas possibles $p = 0$, $p \geq 1$ ou $p \leq -1$. Les
relations sur les $a_n$ et $b_n$ s'en déduisent facilement
par les formules du premier paragraphe.  
\end{proof}

\begin{rem}
La même technique permet d'aboutir aux mêmes formules
dès que la série trigonométrique converge uniformément sur un segment de
longueur $2\pi$.
\end{rem}