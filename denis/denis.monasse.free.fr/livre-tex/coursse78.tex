\documentclass[]{article}
\usepackage[T1]{fontenc}
\usepackage{lmodern}
\usepackage{amssymb,amsmath}
\usepackage{ifxetex,ifluatex}
\usepackage{fixltx2e} % provides \textsubscript
% use upquote if available, for straight quotes in verbatim environments
\IfFileExists{upquote.sty}{\usepackage{upquote}}{}
\ifnum 0\ifxetex 1\fi\ifluatex 1\fi=0 % if pdftex
  \usepackage[utf8]{inputenc}
\else % if luatex or xelatex
  \ifxetex
    \usepackage{mathspec}
    \usepackage{xltxtra,xunicode}
  \else
    \usepackage{fontspec}
  \fi
  \defaultfontfeatures{Mapping=tex-text,Scale=MatchLowercase}
  \newcommand{\euro}{€}
\fi
% use microtype if available
\IfFileExists{microtype.sty}{\usepackage{microtype}}{}
\ifxetex
  \usepackage[setpagesize=false, % page size defined by xetex
              unicode=false, % unicode breaks when used with xetex
              xetex]{hyperref}
\else
  \usepackage[unicode=true]{hyperref}
\fi
\hypersetup{breaklinks=true,
            bookmarks=true,
            pdfauthor={},
            pdftitle={Series trigonometriques},
            colorlinks=true,
            citecolor=blue,
            urlcolor=blue,
            linkcolor=magenta,
            pdfborder={0 0 0}}
\urlstyle{same}  % don't use monospace font for urls
\setlength{\parindent}{0pt}
\setlength{\parskip}{6pt plus 2pt minus 1pt}
\setlength{\emergencystretch}{3em}  % prevent overfull lines
\setcounter{secnumdepth}{0}
 
/* start css.sty */
.cmr-5{font-size:50%;}
.cmr-7{font-size:70%;}
.cmmi-5{font-size:50%;font-style: italic;}
.cmmi-7{font-size:70%;font-style: italic;}
.cmmi-10{font-style: italic;}
.cmsy-5{font-size:50%;}
.cmsy-7{font-size:70%;}
.cmex-7{font-size:70%;}
.cmex-7x-x-71{font-size:49%;}
.msbm-7{font-size:70%;}
.cmtt-10{font-family: monospace;}
.cmti-10{ font-style: italic;}
.cmbx-10{ font-weight: bold;}
.cmr-17x-x-120{font-size:204%;}
.cmsl-10{font-style: oblique;}
.cmti-7x-x-71{font-size:49%; font-style: italic;}
.cmbxti-10{ font-weight: bold; font-style: italic;}
p.noindent { text-indent: 0em }
td p.noindent { text-indent: 0em; margin-top:0em; }
p.nopar { text-indent: 0em; }
p.indent{ text-indent: 1.5em }
@media print {div.crosslinks {visibility:hidden;}}
a img { border-top: 0; border-left: 0; border-right: 0; }
center { margin-top:1em; margin-bottom:1em; }
td center { margin-top:0em; margin-bottom:0em; }
.Canvas { position:relative; }
li p.indent { text-indent: 0em }
.enumerate1 {list-style-type:decimal;}
.enumerate2 {list-style-type:lower-alpha;}
.enumerate3 {list-style-type:lower-roman;}
.enumerate4 {list-style-type:upper-alpha;}
div.newtheorem { margin-bottom: 2em; margin-top: 2em;}
.obeylines-h,.obeylines-v {white-space: nowrap; }
div.obeylines-v p { margin-top:0; margin-bottom:0; }
.overline{ text-decoration:overline; }
.overline img{ border-top: 1px solid black; }
td.displaylines {text-align:center; white-space:nowrap;}
.centerline {text-align:center;}
.rightline {text-align:right;}
div.verbatim {font-family: monospace; white-space: nowrap; text-align:left; clear:both; }
.fbox {padding-left:3.0pt; padding-right:3.0pt; text-indent:0pt; border:solid black 0.4pt; }
div.fbox {display:table}
div.center div.fbox {text-align:center; clear:both; padding-left:3.0pt; padding-right:3.0pt; text-indent:0pt; border:solid black 0.4pt; }
div.minipage{width:100%;}
div.center, div.center div.center {text-align: center; margin-left:1em; margin-right:1em;}
div.center div {text-align: left;}
div.flushright, div.flushright div.flushright {text-align: right;}
div.flushright div {text-align: left;}
div.flushleft {text-align: left;}
.underline{ text-decoration:underline; }
.underline img{ border-bottom: 1px solid black; margin-bottom:1pt; }
.framebox-c, .framebox-l, .framebox-r { padding-left:3.0pt; padding-right:3.0pt; text-indent:0pt; border:solid black 0.4pt; }
.framebox-c {text-align:center;}
.framebox-l {text-align:left;}
.framebox-r {text-align:right;}
span.thank-mark{ vertical-align: super }
span.footnote-mark sup.textsuperscript, span.footnote-mark a sup.textsuperscript{ font-size:80%; }
div.tabular, div.center div.tabular {text-align: center; margin-top:0.5em; margin-bottom:0.5em; }
table.tabular td p{margin-top:0em;}
table.tabular {margin-left: auto; margin-right: auto;}
div.td00{ margin-left:0pt; margin-right:0pt; }
div.td01{ margin-left:0pt; margin-right:5pt; }
div.td10{ margin-left:5pt; margin-right:0pt; }
div.td11{ margin-left:5pt; margin-right:5pt; }
table[rules] {border-left:solid black 0.4pt; border-right:solid black 0.4pt; }
td.td00{ padding-left:0pt; padding-right:0pt; }
td.td01{ padding-left:0pt; padding-right:5pt; }
td.td10{ padding-left:5pt; padding-right:0pt; }
td.td11{ padding-left:5pt; padding-right:5pt; }
table[rules] {border-left:solid black 0.4pt; border-right:solid black 0.4pt; }
.hline hr, .cline hr{ height : 1px; margin:0px; }
.tabbing-right {text-align:right;}
span.TEX {letter-spacing: -0.125em; }
span.TEX span.E{ position:relative;top:0.5ex;left:-0.0417em;}
a span.TEX span.E {text-decoration: none; }
span.LATEX span.A{ position:relative; top:-0.5ex; left:-0.4em; font-size:85%;}
span.LATEX span.TEX{ position:relative; left: -0.4em; }
div.float img, div.float .caption {text-align:center;}
div.figure img, div.figure .caption {text-align:center;}
.marginpar {width:20%; float:right; text-align:left; margin-left:auto; margin-top:0.5em; font-size:85%; text-decoration:underline;}
.marginpar p{margin-top:0.4em; margin-bottom:0.4em;}
.equation td{text-align:center; vertical-align:middle; }
td.eq-no{ width:5%; }
table.equation { width:100%; } 
div.math-display, div.par-math-display{text-align:center;}
math .texttt { font-family: monospace; }
math .textit { font-style: italic; }
math .textsl { font-style: oblique; }
math .textsf { font-family: sans-serif; }
math .textbf { font-weight: bold; }
.partToc a, .partToc, .likepartToc a, .likepartToc {line-height: 200%; font-weight:bold; font-size:110%;}
.chapterToc a, .chapterToc, .likechapterToc a, .likechapterToc, .appendixToc a, .appendixToc {line-height: 200%; font-weight:bold;}
.index-item, .index-subitem, .index-subsubitem {display:block}
.caption td.id{font-weight: bold; white-space: nowrap; }
table.caption {text-align:center;}
h1.partHead{text-align: center}
p.bibitem { text-indent: -2em; margin-left: 2em; margin-top:0.6em; margin-bottom:0.6em; }
p.bibitem-p { text-indent: 0em; margin-left: 2em; margin-top:0.6em; margin-bottom:0.6em; }
.paragraphHead, .likeparagraphHead { margin-top:2em; font-weight: bold;}
.subparagraphHead, .likesubparagraphHead { font-weight: bold;}
.quote {margin-bottom:0.25em; margin-top:0.25em; margin-left:1em; margin-right:1em; text-align:justify;}
.verse{white-space:nowrap; margin-left:2em}
div.maketitle {text-align:center;}
h2.titleHead{text-align:center;}
div.maketitle{ margin-bottom: 2em; }
div.author, div.date {text-align:center;}
div.thanks{text-align:left; margin-left:10%; font-size:85%; font-style:italic; }
div.author{white-space: nowrap;}
.quotation {margin-bottom:0.25em; margin-top:0.25em; margin-left:1em; }
h1.partHead{text-align: center}
.sectionToc, .likesectionToc {margin-left:2em;}
.subsectionToc, .likesubsectionToc {margin-left:4em;}
.subsubsectionToc, .likesubsubsectionToc {margin-left:6em;}
.frenchb-nbsp{font-size:75%;}
.frenchb-thinspace{font-size:75%;}
.figure img.graphics {margin-left:10%;}
/* end css.sty */

\title{Series trigonometriques}
\author{}
\date{}

\begin{document}
\maketitle

\textbf{Warning: 
requires JavaScript to process the mathematics on this page.\\ If your
browser supports JavaScript, be sure it is enabled.}

\begin{center}\rule{3in}{0.4pt}\end{center}

[
[
[]
[

\subsubsection{14.2 Séries trigonométriques}

\paragraph{14.2.1 Rappels d'intégration}

Lemme~14.2.1 Soit f : \mathbb{R}~ \rightarrow~ \mathbb{C} périodique de période T, continue par
morceaux. Alors, pour tout a \in \mathbb{R}~, \int ~
_a^a+Tf(t) dt =\int ~
_0^Tf(t) dt

Démonstration On écrit \int ~
_a^a+Tf(t) dt =\int ~
_a^0f(t) dt+\int ~
_0^Tf(t) dt+\int ~
_T^a+Tf(t) dt =\int ~
_a^0f(t) dt+\int ~
_0^Tf(t) dt+\int ~
_0^af(u+T) du = \int ~
_a^0f(t) dt +\int ~
_0^Tf(t) dt +\int ~
_0^af(u) du =\int ~
_0^Tf(t) dt en faisant le changement de variable u = t -
T.

Lemme~14.2.2 Pour tout n \in \mathbb{Z}, \int ~
_0^2\pi~e^int dt = 2\pi~\delta_n^0.

\paragraph{14.2.2 Généralités}

Définition~14.2.1 (forme réelle). Soit (a_n)_n≥0 et
(b_n)_n≥1 deux suites de nombres complexes. On appelle
série trigonométrique associée la série de fonctions de \mathbb{R}~ dans \mathbb{C},

a_0 + \sum _n≥1(a_n~
\cos nx + b_n \sin nx)

Remarque~14.2.1 Soit n \in \mathbb{N}~^∗ et a_n et b_n
deux nombres complexes. On a alors a_n\
cos nx + b_n sin~ nx =
c_ne^inx + c_-ne^-inx avec
c_n = a_n-ib_n \over 2 et
c_-n = a_n+ib_n \over 2 .
Inversement, si on se donne deux nombres complexes c_n et
c_-n, on a c_ne^inx +
c_-ne^-inx = a_n\
sin nx + b_n cos~ nx avec
a_n = c_n + c_-n et b_n =
i(c_n - c_-n). Ceci amène également à poser

Définition~14.2.2 (forme complexe). Soit (c_n)_n\in\mathbb{Z} une
suite de nombres complexes. On appelle série trigonométrique associée la
série de fonctions de \mathbb{R}~ dans \mathbb{C},

c_0 + \\sum
_n≥1(c_ne^inx + c_
-ne^-inx)

On passe donc de la forme réelle à la forme complexe ou vice versa par
les formules

\begin{align*} a_0& =& c_0 \%&
\\ \forall~~n ≥
1,\quad c_n& =& a_n - ib_n
\over 2 ,\quad c_-n =
a_n + ib_n \over 2 \%&
\\ \forall~~n ≥
1,\quad a_n& =& c_n +
c_-n,\quad b_n = i(c_n -
c_-n)\%& \\
\end{align*}

\paragraph{14.2.3 Un cas de convergence normale}

Théorème~14.2.3 On considère une série trigonométrique vérifiant les
conditions équivalentes

\begin{itemize}
\itemsep1pt\parskip0pt\parsep0pt
\item
  (i) les deux séries \\\sum
   a_n et
  \\sum ~
  b_n sont convergentes.
\item
  (ii) les deux séries
  \\sum ~
  _n≥0c_n et
  \\sum ~
  _n≥0c_-n sont convergentes.
\end{itemize}

Alors la série trigonométrique converge normalement sur \mathbb{R}~, sa somme f
est une fonction continue périodique de période 2\pi~ et on a

\begin{align*} \forall~~n \in
\mathbb{Z},\quad c_n& =& 1 \over 2\pi~
\int  _0^2\pi~f(t)e^-int~
dt \%& \\ \forall~~n ≥
1,\quad a_n& =& 1 \over \pi~
\int ~
_0^2\pi~f(t)cos~ nt
dt,\quad b_ n = 1 \over \pi~
\int ~
_0^2\pi~f(t)sin~ nt dt\%&
\\ \end{align*}

Démonstration Les relations
a_n\leqc_n +
c_-n,
b_n\leqc_n +
c_-n, c_n\leq 1
\over 2 (a_n +
b_n) et
c_-n\leq 1 \over 2
(a_n + b_n)
(que l'on déduit facilement des relations du paragraphe précédent)
montrent clairement l'équivalence. Alors on a

\forall~x \in \mathbb{R}~, c_ne^inx~
+ c_ -ne^-inx\leqc_
n + c_-n

qui est une série convergente indépendante de x. On a donc la
convergence normale de la série et en particulier la continuité de sa
somme. Cette somme est évidemment périodique de période 2\pi~ puisque
toutes les applications
x\mapsto~c_ne^inx +
c_-ne^-inx le sont. Soit p \in \mathbb{Z}. On a aussi
\forall~~x \in \mathbb{R}~,
(c_ne^inx +
c_-ne^-inx)e^-ipx\leqc_n
+ c_-n ce qui montre que la série
c_0e^-ipx +\
\sum  _n≥1(c_ne^inx~
+ c_-ne^-inx)e^-ipx converge normalement
sur \mathbb{R}~, donc sur [0,2\pi~]. Ceci justifie donc dans le calcul suivant
l'interversion du signe d'intégrale et du signe somme

\begin{align*} \int ~
_0^2\pi~f(t)e^-ipt dt&& \%&
\\ & =& \int ~
_0^2\pi~\left (c_ 0e^-ipt
+ \sum _n≥1(c_ne^int~
+ c_ -ne^-int)e^-ipt\right
) dt \%& \\ & =&
c_0\int ~
_0^2\pi~e^-ipt dt \%&
\\ & \text &
+\sum _n=1^+\infty~~\left
(c_ n \\int  ~
_0^2\pi~e^i(n-p)t dt + c_ -n
\\int  ~
_0^2\pi~e^-i(n+p)t dt\right )\%&
\\ & =& 2\pi~\left
(c_0\delta_p^0 + \\sum
_n=1^+\infty~(c_ n\delta_p^n + c_
-n\delta_p^-n)\right ) = 2\pi~c_ p
\%& \\ \end{align*}

en distinguant les différents cas possibles p = 0, p ≥ 1 ou p \leq-1. Les
relations sur les a_n et b_n s'en déduisent facilement
par les formules du premier paragraphe.

Remarque~14.2.2 La même technique permet d'aboutir aux mêmes formules
dès que la série trigonométrique converge uniformément sur un segment de
longueur 2\pi~.

[
[
[
[

\end{document}
