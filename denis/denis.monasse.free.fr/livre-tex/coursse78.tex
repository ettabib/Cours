
\subsubsection{14.2 Séries trigonométriques}

\paragraph{14.2.1 Rappels d'intégration}

Lemme~14.2.1 Soit f : \mathbb{R}~ \rightarrow~ \mathbb{C} périodique de période T, continue par
morceaux. Alors, pour tout a \in \mathbb{R}~, \int ~
_a^a+Tf(t) dt =\int ~
_0^Tf(t) dt

Démonstration On écrit \int ~
_a^a+Tf(t) dt =\int ~
_a^0f(t) dt+\int ~
_0^Tf(t) dt+\int ~
_T^a+Tf(t) dt =\int ~
_a^0f(t) dt+\int ~
_0^Tf(t) dt+\int ~
_0^af(u+T) du = \int ~
_a^0f(t) dt +\int ~
_0^Tf(t) dt +\int ~
_0^af(u) du =\int ~
_0^Tf(t) dt en faisant le changement de variable u = t -
T.

Lemme~14.2.2 Pour tout n \in \mathbb{Z}, \int ~
_0^2\pi~e^int dt = 2\pi~\delta_n^0.

\paragraph{14.2.2 Généralités}

Définition~14.2.1 (forme réelle). Soit (a_n)_n≥0 et
(b_n)_n≥1 deux suites de nombres complexes. On appelle
série trigonométrique associée la série de fonctions de \mathbb{R}~ dans \mathbb{C},

a_0 + \sum _n≥1(a_n~
\cos nx + b_n \sin nx)

Remarque~14.2.1 Soit n \in \mathbb{N}~^∗ et a_n et b_n
deux nombres complexes. On a alors a_n\
cos nx + b_n sin~ nx =
c_ne^inx + c_-ne^-inx avec
c_n = a_n-ib_n \over 2 et
c_-n = a_n+ib_n \over 2 .
Inversement, si on se donne deux nombres complexes c_n et
c_-n, on a c_ne^inx +
c_-ne^-inx = a_n\
sin nx + b_n cos~ nx avec
a_n = c_n + c_-n et b_n =
i(c_n - c_-n). Ceci amène également à poser

Définition~14.2.2 (forme complexe). Soit (c_n)_n\in\mathbb{Z} une
suite de nombres complexes. On appelle série trigonométrique associée la
série de fonctions de \mathbb{R}~ dans \mathbb{C},

c_0 + \\sum
_n≥1(c_ne^inx + c_
-ne^-inx)

On passe donc de la forme réelle à la forme complexe ou vice versa par
les formules

\begin{align*} a_0& =& c_0 \%&
\\ \forall~~n ≥
1,\quad c_n& =& a_n - ib_n
\over 2 ,\quad c_-n =
a_n + ib_n \over 2 \%&
\\ \forall~~n ≥
1,\quad a_n& =& c_n +
c_-n,\quad b_n = i(c_n -
c_-n)\%& \\
\end{align*}

\paragraph{14.2.3 Un cas de convergence normale}

Théorème~14.2.3 On considère une série trigonométrique vérifiant les
conditions équivalentes

\begin{itemize}
\itemsep1pt\parskip0pt\parsep0pt
\item
  (i) les deux séries \\\sum
   a_n et
  \\sum ~
  b_n sont convergentes.
\item
  (ii) les deux séries
  \\sum ~
  _n≥0c_n et
  \\sum ~
  _n≥0c_-n sont convergentes.
\end{itemize}

Alors la série trigonométrique converge normalement sur \mathbb{R}~, sa somme f
est une fonction continue périodique de période 2\pi~ et on a

\begin{align*} \forall~~n \in
\mathbb{Z},\quad c_n& =& 1 \over 2\pi~
\int  _0^2\pi~f(t)e^-int~
dt \%& \\ \forall~~n ≥
1,\quad a_n& =& 1 \over \pi~
\int ~
_0^2\pi~f(t)cos~ nt
dt,\quad b_ n = 1 \over \pi~
\int ~
_0^2\pi~f(t)sin~ nt dt\%&
\\ \end{align*}

Démonstration Les relations
a_n\leqc_n +
c_-n,
b_n\leqc_n +
c_-n, c_n\leq 1
\over 2 (a_n +
b_n) et
c_-n\leq 1 \over 2
(a_n + b_n)
(que l'on déduit facilement des relations du paragraphe précédent)
montrent clairement l'équivalence. Alors on a

\forall~x \in \mathbb{R}~, c_ne^inx~
+ c_ -ne^-inx\leqc_
n + c_-n

qui est une série convergente indépendante de x. On a donc la
convergence normale de la série et en particulier la continuité de sa
somme. Cette somme est évidemment périodique de période 2\pi~ puisque
toutes les applications
x\mapsto~c_ne^inx +
c_-ne^-inx le sont. Soit p \in \mathbb{Z}. On a aussi
\forall~~x \in \mathbb{R}~,
(c_ne^inx +
c_-ne^-inx)e^-ipx\leqc_n
+ c_-n ce qui montre que la série
c_0e^-ipx +\
\sum  _n≥1(c_ne^inx~
+ c_-ne^-inx)e^-ipx converge normalement
sur \mathbb{R}~, donc sur [0,2\pi~]. Ceci justifie donc dans le calcul suivant
l'interversion du signe d'intégrale et du signe somme

\begin{align*} \int ~
_0^2\pi~f(t)e^-ipt dt&& \%&
\\ & =& \int ~
_0^2\pi~\left (c_ 0e^-ipt
+ \sum _n≥1(c_ne^int~
+ c_ -ne^-int)e^-ipt\right
) dt \%& \\ & =&
c_0\int ~
_0^2\pi~e^-ipt dt \%&
\\ & \text &
+\sum _n=1^+\infty~~\left
(c_ n \\int  ~
_0^2\pi~e^i(n-p)t dt + c_ -n
\\int  ~
_0^2\pi~e^-i(n+p)t dt\right )\%&
\\ & =& 2\pi~\left
(c_0\delta_p^0 + \\sum
_n=1^+\infty~(c_ n\delta_p^n + c_
-n\delta_p^-n)\right ) = 2\pi~c_ p
\%& \\ \end{align*}

en distinguant les différents cas possibles p = 0, p ≥ 1 ou p \leq-1. Les
relations sur les a_n et b_n s'en déduisent facilement
par les formules du premier paragraphe.

Remarque~14.2.2 La même technique permet d'aboutir aux mêmes formules
dès que la série trigonométrique converge uniformément sur un segment de
longueur 2\pi~.

