\documentclass[]{article}
\usepackage[T1]{fontenc}
\usepackage{lmodern}
\usepackage{amssymb,amsmath}
\usepackage{ifxetex,ifluatex}
\usepackage{fixltx2e} % provides \textsubscript
% use upquote if available, for straight quotes in verbatim environments
\IfFileExists{upquote.sty}{\usepackage{upquote}}{}
\ifnum 0\ifxetex 1\fi\ifluatex 1\fi=0 % if pdftex
  \usepackage[utf8]{inputenc}
\else % if luatex or xelatex
  \ifxetex
    \usepackage{mathspec}
    \usepackage{xltxtra,xunicode}
  \else
    \usepackage{fontspec}
  \fi
  \defaultfontfeatures{Mapping=tex-text,Scale=MatchLowercase}
  \newcommand{\euro}{€}
\fi
% use microtype if available
\IfFileExists{microtype.sty}{\usepackage{microtype}}{}
\ifxetex
  \usepackage[setpagesize=false, % page size defined by xetex
              unicode=false, % unicode breaks when used with xetex
              xetex]{hyperref}
\else
  \usepackage[unicode=true]{hyperref}
\fi
\hypersetup{breaklinks=true,
            bookmarks=true,
            pdfauthor={},
            pdftitle={Series trigonometriques},
            colorlinks=true,
            citecolor=blue,
            urlcolor=blue,
            linkcolor=magenta,
            pdfborder={0 0 0}}
\urlstyle{same}  % don't use monospace font for urls
\setlength{\parindent}{0pt}
\setlength{\parskip}{6pt plus 2pt minus 1pt}
\setlength{\emergencystretch}{3em}  % prevent overfull lines
\setcounter{secnumdepth}{0}
 
/* start css.sty */
.cmr-5{font-size:50%;}
.cmr-7{font-size:70%;}
.cmmi-5{font-size:50%;font-style: italic;}
.cmmi-7{font-size:70%;font-style: italic;}
.cmmi-10{font-style: italic;}
.cmsy-5{font-size:50%;}
.cmsy-7{font-size:70%;}
.cmex-7{font-size:70%;}
.cmex-7x-x-71{font-size:49%;}
.msbm-7{font-size:70%;}
.cmtt-10{font-family: monospace;}
.cmti-10{ font-style: italic;}
.cmbx-10{ font-weight: bold;}
.cmr-17x-x-120{font-size:204%;}
.cmsl-10{font-style: oblique;}
.cmti-7x-x-71{font-size:49%; font-style: italic;}
.cmbxti-10{ font-weight: bold; font-style: italic;}
p.noindent { text-indent: 0em }
td p.noindent { text-indent: 0em; margin-top:0em; }
p.nopar { text-indent: 0em; }
p.indent{ text-indent: 1.5em }
@media print {div.crosslinks {visibility:hidden;}}
a img { border-top: 0; border-left: 0; border-right: 0; }
center { margin-top:1em; margin-bottom:1em; }
td center { margin-top:0em; margin-bottom:0em; }
.Canvas { position:relative; }
li p.indent { text-indent: 0em }
.enumerate1 {list-style-type:decimal;}
.enumerate2 {list-style-type:lower-alpha;}
.enumerate3 {list-style-type:lower-roman;}
.enumerate4 {list-style-type:upper-alpha;}
div.newtheorem { margin-bottom: 2em; margin-top: 2em;}
.obeylines-h,.obeylines-v {white-space: nowrap; }
div.obeylines-v p { margin-top:0; margin-bottom:0; }
.overline{ text-decoration:overline; }
.overline img{ border-top: 1px solid black; }
td.displaylines {text-align:center; white-space:nowrap;}
.centerline {text-align:center;}
.rightline {text-align:right;}
div.verbatim {font-family: monospace; white-space: nowrap; text-align:left; clear:both; }
.fbox {padding-left:3.0pt; padding-right:3.0pt; text-indent:0pt; border:solid black 0.4pt; }
div.fbox {display:table}
div.center div.fbox {text-align:center; clear:both; padding-left:3.0pt; padding-right:3.0pt; text-indent:0pt; border:solid black 0.4pt; }
div.minipage{width:100%;}
div.center, div.center div.center {text-align: center; margin-left:1em; margin-right:1em;}
div.center div {text-align: left;}
div.flushright, div.flushright div.flushright {text-align: right;}
div.flushright div {text-align: left;}
div.flushleft {text-align: left;}
.underline{ text-decoration:underline; }
.underline img{ border-bottom: 1px solid black; margin-bottom:1pt; }
.framebox-c, .framebox-l, .framebox-r { padding-left:3.0pt; padding-right:3.0pt; text-indent:0pt; border:solid black 0.4pt; }
.framebox-c {text-align:center;}
.framebox-l {text-align:left;}
.framebox-r {text-align:right;}
span.thank-mark{ vertical-align: super }
span.footnote-mark sup.textsuperscript, span.footnote-mark a sup.textsuperscript{ font-size:80%; }
div.tabular, div.center div.tabular {text-align: center; margin-top:0.5em; margin-bottom:0.5em; }
table.tabular td p{margin-top:0em;}
table.tabular {margin-left: auto; margin-right: auto;}
div.td00{ margin-left:0pt; margin-right:0pt; }
div.td01{ margin-left:0pt; margin-right:5pt; }
div.td10{ margin-left:5pt; margin-right:0pt; }
div.td11{ margin-left:5pt; margin-right:5pt; }
table[rules] {border-left:solid black 0.4pt; border-right:solid black 0.4pt; }
td.td00{ padding-left:0pt; padding-right:0pt; }
td.td01{ padding-left:0pt; padding-right:5pt; }
td.td10{ padding-left:5pt; padding-right:0pt; }
td.td11{ padding-left:5pt; padding-right:5pt; }
table[rules] {border-left:solid black 0.4pt; border-right:solid black 0.4pt; }
.hline hr, .cline hr{ height : 1px; margin:0px; }
.tabbing-right {text-align:right;}
span.TEX {letter-spacing: -0.125em; }
span.TEX span.E{ position:relative;top:0.5ex;left:-0.0417em;}
a span.TEX span.E {text-decoration: none; }
span.LATEX span.A{ position:relative; top:-0.5ex; left:-0.4em; font-size:85%;}
span.LATEX span.TEX{ position:relative; left: -0.4em; }
div.float img, div.float .caption {text-align:center;}
div.figure img, div.figure .caption {text-align:center;}
.marginpar {width:20%; float:right; text-align:left; margin-left:auto; margin-top:0.5em; font-size:85%; text-decoration:underline;}
.marginpar p{margin-top:0.4em; margin-bottom:0.4em;}
.equation td{text-align:center; vertical-align:middle; }
td.eq-no{ width:5%; }
table.equation { width:100%; } 
div.math-display, div.par-math-display{text-align:center;}
math .texttt { font-family: monospace; }
math .textit { font-style: italic; }
math .textsl { font-style: oblique; }
math .textsf { font-family: sans-serif; }
math .textbf { font-weight: bold; }
.partToc a, .partToc, .likepartToc a, .likepartToc {line-height: 200%; font-weight:bold; font-size:110%;}
.chapterToc a, .chapterToc, .likechapterToc a, .likechapterToc, .appendixToc a, .appendixToc {line-height: 200%; font-weight:bold;}
.index-item, .index-subitem, .index-subsubitem {display:block}
.caption td.id{font-weight: bold; white-space: nowrap; }
table.caption {text-align:center;}
h1.partHead{text-align: center}
p.bibitem { text-indent: -2em; margin-left: 2em; margin-top:0.6em; margin-bottom:0.6em; }
p.bibitem-p { text-indent: 0em; margin-left: 2em; margin-top:0.6em; margin-bottom:0.6em; }
.paragraphHead, .likeparagraphHead { margin-top:2em; font-weight: bold;}
.subparagraphHead, .likesubparagraphHead { font-weight: bold;}
.quote {margin-bottom:0.25em; margin-top:0.25em; margin-left:1em; margin-right:1em; text-align:justify;}
.verse{white-space:nowrap; margin-left:2em}
div.maketitle {text-align:center;}
h2.titleHead{text-align:center;}
div.maketitle{ margin-bottom: 2em; }
div.author, div.date {text-align:center;}
div.thanks{text-align:left; margin-left:10%; font-size:85%; font-style:italic; }
div.author{white-space: nowrap;}
.quotation {margin-bottom:0.25em; margin-top:0.25em; margin-left:1em; }
h1.partHead{text-align: center}
.sectionToc, .likesectionToc {margin-left:2em;}
.subsectionToc, .likesubsectionToc {margin-left:4em;}
.subsubsectionToc, .likesubsubsectionToc {margin-left:6em;}
.frenchb-nbsp{font-size:75%;}
.frenchb-thinspace{font-size:75%;}
.figure img.graphics {margin-left:10%;}
/* end css.sty */

\title{Series trigonometriques}
\author{}
\date{}

\begin{document}
\maketitle

\textbf{Warning: \href{http://www.math.union.edu/locate/jsMath}{jsMath}
requires JavaScript to process the mathematics on this page.\\ If your
browser supports JavaScript, be sure it is enabled.}

\begin{center}\rule{3in}{0.4pt}\end{center}

{[}\href{coursse79.html}{next}{]} {[}\href{coursse77.html}{prev}{]}
{[}\href{coursse77.html\#tailcoursse77.html}{prev-tail}{]}
{[}\hyperref[tailcoursse78.html]{tail}{]}
{[}\href{coursch15.html\#coursse78.html}{up}{]}

\subsubsection{14.2 Séries trigonométriques}

\paragraph{14.2.1 Rappels d'intégration}

Lemme~14.2.1 Soit f : ℝ → ℂ périodique de période T, continue par
morceaux. Alors, pour tout a ∈ ℝ, \{\textbackslash{}mathop\{∫ \}
\}\_\{a\}\^{}\{a+T\}f(t) dt =\{\textbackslash{}mathop\{∫ \}
\}\_\{0\}\^{}\{T\}f(t) dt

Démonstration On écrit \{\textbackslash{}mathop\{∫ \}
\}\_\{a\}\^{}\{a+T\}f(t) dt =\{\textbackslash{}mathop\{∫ \}
\}\_\{a\}\^{}\{0\}f(t) dt+\{\textbackslash{}mathop\{∫ \}
\}\_\{0\}\^{}\{T\}f(t) dt+\{\textbackslash{}mathop\{∫ \}
\}\_\{T\}\^{}\{a+T\}f(t) dt =\{\textbackslash{}mathop\{∫ \}
\}\_\{a\}\^{}\{0\}f(t) dt+\{\textbackslash{}mathop\{∫ \}
\}\_\{0\}\^{}\{T\}f(t) dt+\{\textbackslash{}mathop\{∫ \}
\}\_\{0\}\^{}\{a\}f(u+T) du = \{\textbackslash{}mathop\{∫ \}
\}\_\{a\}\^{}\{0\}f(t) dt +\{\textbackslash{}mathop\{∫ \}
\}\_\{0\}\^{}\{T\}f(t) dt +\{\textbackslash{}mathop\{∫ \}
\}\_\{0\}\^{}\{a\}f(u) du =\{\textbackslash{}mathop\{∫ \}
\}\_\{0\}\^{}\{T\}f(t) dt en faisant le changement de variable u = t −
T.

Lemme~14.2.2 Pour tout n ∈ ℤ, \{\textbackslash{}mathop\{∫ \}
\}\_\{0\}\^{}\{2π\}\{e\}\^{}\{int\} dt = 2π\{δ\}\_\{n\}\^{}\{0\}.

\paragraph{14.2.2 Généralités}

Définition~14.2.1 (forme réelle). Soit \{(\{a\}\_\{n\})\}\_\{n≥0\} et
\{(\{b\}\_\{n\})\}\_\{n≥1\} deux suites de nombres complexes. On appelle
série trigonométrique associée la série de fonctions de ℝ dans ℂ,

\{a\}\_\{0\} +\{ \textbackslash{}mathop\{∑ \}\}\_\{n≥1\}(\{a\}\_\{n\}
\textbackslash{}cos nx + \{b\}\_\{n\} \textbackslash{}sin nx)

Remarque~14.2.1 Soit n ∈ \{ℕ\}\^{}\{∗\} et \{a\}\_\{n\} et \{b\}\_\{n\}
deux nombres complexes. On a alors \{a\}\_\{n\}\textbackslash{}mathop\{
cos\} nx + \{b\}\_\{n\}\textbackslash{}mathop\{ sin\} nx =
\{c\}\_\{n\}\{e\}\^{}\{inx\} + \{c\}\_\{−n\}\{e\}\^{}\{−inx\} avec
\{c\}\_\{n\} =\{ \{a\}\_\{n\}−i\{b\}\_\{n\} \textbackslash{}over 2\} et
\{c\}\_\{−n\} =\{ \{a\}\_\{n\}+i\{b\}\_\{n\} \textbackslash{}over 2\} .
Inversement, si on se donne deux nombres complexes \{c\}\_\{n\} et
\{c\}\_\{−n\}, on a \{c\}\_\{n\}\{e\}\^{}\{inx\} +
\{c\}\_\{−n\}\{e\}\^{}\{−inx\} = \{a\}\_\{n\}\textbackslash{}mathop\{
sin\} nx + \{b\}\_\{n\}\textbackslash{}mathop\{ cos\} nx avec
\{a\}\_\{n\} = \{c\}\_\{n\} + \{c\}\_\{−n\} et \{b\}\_\{n\} =
i(\{c\}\_\{n\} − \{c\}\_\{−n\}). Ceci amène également à poser

Définition~14.2.2 (forme complexe). Soit \{(\{c\}\_\{n\})\}\_\{n∈ℤ\} une
suite de nombres complexes. On appelle série trigonométrique associée la
série de fonctions de ℝ dans ℂ,

\{c\}\_\{0\} +\{ \textbackslash{}mathop\{∑
\}\}\_\{n≥1\}(\{c\}\_\{n\}\{e\}\^{}\{inx\} + \{c\}\_\{
−n\}\{e\}\^{}\{−inx\})

On passe donc de la forme réelle à la forme complexe ou vice versa par
les formules

\textbackslash{}begin\{eqnarray*\} \{a\}\_\{0\}\& =\& \{c\}\_\{0\} \%\&
\textbackslash{}\textbackslash{} \textbackslash{}mathop\{∀\}n ≥
1,\textbackslash{}quad \{c\}\_\{n\}\& =\&\{ \{a\}\_\{n\} − i\{b\}\_\{n\}
\textbackslash{}over 2\} ,\textbackslash{}quad \{c\}\_\{−n\} =\{
\{a\}\_\{n\} + i\{b\}\_\{n\} \textbackslash{}over 2\} \%\&
\textbackslash{}\textbackslash{} \textbackslash{}mathop\{∀\}n ≥
1,\textbackslash{}quad \{a\}\_\{n\}\& =\& \{c\}\_\{n\} +
\{c\}\_\{−n\},\textbackslash{}quad \{b\}\_\{n\} = i(\{c\}\_\{n\} −
\{c\}\_\{−n\})\%\& \textbackslash{}\textbackslash{}
\textbackslash{}end\{eqnarray*\}

\paragraph{14.2.3 Un cas de convergence normale}

Théorème~14.2.3 On considère une série trigonométrique vérifiant les
conditions équivalentes

\begin{itemize}
\itemsep1pt\parskip0pt\parsep0pt
\item
  (i) les deux séries \textbackslash{}mathop\{\textbackslash{}mathop\{∑
  \}\} \textbar{}\{a\}\_\{n\}\textbar{} et
  \textbackslash{}mathop\{\textbackslash{}mathop\{∑ \}\}
  \textbar{}\{b\}\_\{n\}\textbar{} sont convergentes.
\item
  (ii) les deux séries
  \{\textbackslash{}mathop\{\textbackslash{}mathop\{∑ \}\}
  \}\_\{n≥0\}\textbar{}\{c\}\_\{n\}\textbar{} et
  \{\textbackslash{}mathop\{\textbackslash{}mathop\{∑ \}\}
  \}\_\{n≥0\}\textbar{}\{c\}\_\{−n\}\textbar{} sont convergentes.
\end{itemize}

Alors la série trigonométrique converge normalement sur ℝ, sa somme f
est une fonction continue périodique de période 2π et on a

\textbackslash{}begin\{eqnarray*\} \textbackslash{}mathop\{∀\}n ∈
ℤ,\textbackslash{}quad \{c\}\_\{n\}\& =\&\{ 1 \textbackslash{}over 2π\}
\{\textbackslash{}mathop\{∫ \} \}\_\{0\}\^{}\{2π\}f(t)\{e\}\^{}\{−int\}
dt \%\& \textbackslash{}\textbackslash{} \textbackslash{}mathop\{∀\}n ≥
1,\textbackslash{}quad \{a\}\_\{n\}\& =\&\{ 1 \textbackslash{}over π\}
\{\textbackslash{}mathop\{∫ \}
\}\_\{0\}\^{}\{2π\}f(t)\textbackslash{}mathop\{cos\} nt
dt,\textbackslash{}quad \{b\}\_\{ n\} =\{ 1 \textbackslash{}over π\}
\{\textbackslash{}mathop\{∫ \}
\}\_\{0\}\^{}\{2π\}f(t)\textbackslash{}mathop\{sin\} nt dt\%\&
\textbackslash{}\textbackslash{} \textbackslash{}end\{eqnarray*\}

Démonstration Les relations
\textbar{}\{a\}\_\{n\}\textbar{}≤\textbar{}\{c\}\_\{n\}\textbar{} +
\textbar{}\{c\}\_\{−n\}\textbar{},
\textbar{}\{b\}\_\{n\}\textbar{}≤\textbar{}\{c\}\_\{n\}\textbar{} +
\textbar{}\{c\}\_\{−n\}\textbar{}, \textbar{}\{c\}\_\{n\}\textbar{}≤\{ 1
\textbackslash{}over 2\} (\textbar{}\{a\}\_\{n\}\textbar{} +
\textbar{}\{b\}\_\{n\}\textbar{}) et
\textbar{}\{c\}\_\{−n\}\textbar{}≤\{ 1 \textbackslash{}over 2\}
(\textbar{}\{a\}\_\{n\}\textbar{} + \textbar{}\{b\}\_\{n\}\textbar{})
(que l'on déduit facilement des relations du paragraphe précédent)
montrent clairement l'équivalence. Alors on a

\textbackslash{}mathop\{∀\}x ∈ ℝ, \textbar{}\{c\}\_\{n\}\{e\}\^{}\{inx\}
+ \{c\}\_\{ −n\}\{e\}\^{}\{−inx\}\textbar{}≤\textbar{}\{c\}\_\{
n\}\textbar{} + \textbar{}\{c\}\_\{−n\}\textbar{}

qui est une série convergente indépendante de x. On a donc la
convergence normale de la série et en particulier la continuité de sa
somme. Cette somme est évidemment périodique de période 2π puisque
toutes les applications
x\textbackslash{}mathrel\{↦\}\{c\}\_\{n\}\{e\}\^{}\{inx\} +
\{c\}\_\{−n\}\{e\}\^{}\{−inx\} le sont. Soit p ∈ ℤ. On a aussi
\textbackslash{}mathop\{∀\}x ∈ ℝ,
\textbar{}(\{c\}\_\{n\}\{e\}\^{}\{inx\} +
\{c\}\_\{−n\}\{e\}\^{}\{−inx\})\{e\}\^{}\{−ipx\}\textbar{}≤\textbar{}\{c\}\_\{n\}\textbar{}
+ \textbar{}\{c\}\_\{−n\}\textbar{} ce qui montre que la série
\{c\}\_\{0\}\{e\}\^{}\{−ipx\} +\{\textbackslash{}mathop\{
\textbackslash{}mathop\{∑ \}\} \}\_\{n≥1\}(\{c\}\_\{n\}\{e\}\^{}\{inx\}
+ \{c\}\_\{−n\}\{e\}\^{}\{−inx\})\{e\}\^{}\{−ipx\} converge normalement
sur ℝ, donc sur {[}0,2π{]}. Ceci justifie donc dans le calcul suivant
l'interversion du signe d'intégrale et du signe somme

\textbackslash{}begin\{eqnarray*\} \{\textbackslash{}mathop\{∫ \}
\}\_\{0\}\^{}\{2π\}f(t)\{e\}\^{}\{−ipt\} dt\&\& \%\&
\textbackslash{}\textbackslash{} \& =\& \{\textbackslash{}mathop\{∫ \}
\}\_\{0\}\^{}\{2π\}\textbackslash{}left (\{c\}\_\{ 0\}\{e\}\^{}\{−ipt\}
+\{ \textbackslash{}mathop\{∑ \}\}\_\{n≥1\}(\{c\}\_\{n\}\{e\}\^{}\{int\}
+ \{c\}\_\{ −n\}\{e\}\^{}\{−int\})\{e\}\^{}\{−ipt\}\textbackslash{}right
) dt \%\& \textbackslash{}\textbackslash{} \& =\&
\{c\}\_\{0\}\{\textbackslash{}mathop\{∫ \}
\}\_\{0\}\^{}\{2π\}\{e\}\^{}\{−ipt\} dt \%\&
\textbackslash{}\textbackslash{} \& \textbackslash{}text\{\} \&
+\{\textbackslash{}mathop\{∑ \}\}\_\{n=1\}\^{}\{+∞\}\textbackslash{}left
(\{c\}\_\{ n\}\{ \textbackslash{}mathop\{\textbackslash{}mathop\{∫ \} \}
\}\_\{0\}\^{}\{2π\}\{e\}\^{}\{i(n−p)t\} dt + \{c\}\_\{ −n\}\{
\textbackslash{}mathop\{\textbackslash{}mathop\{∫ \} \}
\}\_\{0\}\^{}\{2π\}\{e\}\^{}\{−i(n+p)t\} dt\textbackslash{}right )\%\&
\textbackslash{}\textbackslash{} \& =\& 2π\textbackslash{}left
(\{c\}\_\{0\}\{δ\}\_\{p\}\^{}\{0\} +\{ \textbackslash{}mathop\{∑
\}\}\_\{n=1\}\^{}\{+∞\}(\{c\}\_\{ n\}\{δ\}\_\{p\}\^{}\{n\} + \{c\}\_\{
−n\}\{δ\}\_\{p\}\^{}\{−n\})\textbackslash{}right ) = 2π\{c\}\_\{ p\}
\%\& \textbackslash{}\textbackslash{} \textbackslash{}end\{eqnarray*\}

en distinguant les différents cas possibles p = 0, p ≥ 1 ou p ≤−1. Les
relations sur les \{a\}\_\{n\} et \{b\}\_\{n\} s'en déduisent facilement
par les formules du premier paragraphe.

Remarque~14.2.2 La même technique permet d'aboutir aux mêmes formules
dès que la série trigonométrique converge uniformément sur un segment de
longueur 2π.

{[}\href{coursse79.html}{next}{]} {[}\href{coursse77.html}{prev}{]}
{[}\href{coursse77.html\#tailcoursse77.html}{prev-tail}{]}
{[}\href{coursse78.html}{front}{]}
{[}\href{coursch15.html\#coursse78.html}{up}{]}

\end{document}
