\documentclass[]{article}
\usepackage[T1]{fontenc}
\usepackage{lmodern}
\usepackage{amssymb,amsmath}
\usepackage{ifxetex,ifluatex}
\usepackage{fixltx2e} % provides \textsubscript
% use upquote if available, for straight quotes in verbatim environments
\IfFileExists{upquote.sty}{\usepackage{upquote}}{}
\ifnum 0\ifxetex 1\fi\ifluatex 1\fi=0 % if pdftex
  \usepackage[utf8]{inputenc}
\else % if luatex or xelatex
  \ifxetex
    \usepackage{mathspec}
    \usepackage{xltxtra,xunicode}
  \else
    \usepackage{fontspec}
  \fi
  \defaultfontfeatures{Mapping=tex-text,Scale=MatchLowercase}
  \newcommand{\euro}{€}
\fi
% use microtype if available
\IfFileExists{microtype.sty}{\usepackage{microtype}}{}
\ifxetex
  \usepackage[setpagesize=false, % page size defined by xetex
              unicode=false, % unicode breaks when used with xetex
              xetex]{hyperref}
\else
  \usepackage[unicode=true]{hyperref}
\fi
\hypersetup{breaklinks=true,
            bookmarks=true,
            pdfauthor={},
            pdftitle={Etude metrique des arcs},
            colorlinks=true,
            citecolor=blue,
            urlcolor=blue,
            linkcolor=magenta,
            pdfborder={0 0 0}}
\urlstyle{same}  % don't use monospace font for urls
\setlength{\parindent}{0pt}
\setlength{\parskip}{6pt plus 2pt minus 1pt}
\setlength{\emergencystretch}{3em}  % prevent overfull lines
\setcounter{secnumdepth}{0}
 
/* start css.sty */
.cmr-5{font-size:50%;}
.cmr-7{font-size:70%;}
.cmmi-5{font-size:50%;font-style: italic;}
.cmmi-7{font-size:70%;font-style: italic;}
.cmmi-10{font-style: italic;}
.cmsy-5{font-size:50%;}
.cmsy-7{font-size:70%;}
.cmex-7{font-size:70%;}
.cmex-7x-x-71{font-size:49%;}
.msbm-7{font-size:70%;}
.cmtt-10{font-family: monospace;}
.cmti-10{ font-style: italic;}
.cmbx-10{ font-weight: bold;}
.cmr-17x-x-120{font-size:204%;}
.cmsl-10{font-style: oblique;}
.cmti-7x-x-71{font-size:49%; font-style: italic;}
.cmbxti-10{ font-weight: bold; font-style: italic;}
p.noindent { text-indent: 0em }
td p.noindent { text-indent: 0em; margin-top:0em; }
p.nopar { text-indent: 0em; }
p.indent{ text-indent: 1.5em }
@media print {div.crosslinks {visibility:hidden;}}
a img { border-top: 0; border-left: 0; border-right: 0; }
center { margin-top:1em; margin-bottom:1em; }
td center { margin-top:0em; margin-bottom:0em; }
.Canvas { position:relative; }
li p.indent { text-indent: 0em }
.enumerate1 {list-style-type:decimal;}
.enumerate2 {list-style-type:lower-alpha;}
.enumerate3 {list-style-type:lower-roman;}
.enumerate4 {list-style-type:upper-alpha;}
div.newtheorem { margin-bottom: 2em; margin-top: 2em;}
.obeylines-h,.obeylines-v {white-space: nowrap; }
div.obeylines-v p { margin-top:0; margin-bottom:0; }
.overline{ text-decoration:overline; }
.overline img{ border-top: 1px solid black; }
td.displaylines {text-align:center; white-space:nowrap;}
.centerline {text-align:center;}
.rightline {text-align:right;}
div.verbatim {font-family: monospace; white-space: nowrap; text-align:left; clear:both; }
.fbox {padding-left:3.0pt; padding-right:3.0pt; text-indent:0pt; border:solid black 0.4pt; }
div.fbox {display:table}
div.center div.fbox {text-align:center; clear:both; padding-left:3.0pt; padding-right:3.0pt; text-indent:0pt; border:solid black 0.4pt; }
div.minipage{width:100%;}
div.center, div.center div.center {text-align: center; margin-left:1em; margin-right:1em;}
div.center div {text-align: left;}
div.flushright, div.flushright div.flushright {text-align: right;}
div.flushright div {text-align: left;}
div.flushleft {text-align: left;}
.underline{ text-decoration:underline; }
.underline img{ border-bottom: 1px solid black; margin-bottom:1pt; }
.framebox-c, .framebox-l, .framebox-r { padding-left:3.0pt; padding-right:3.0pt; text-indent:0pt; border:solid black 0.4pt; }
.framebox-c {text-align:center;}
.framebox-l {text-align:left;}
.framebox-r {text-align:right;}
span.thank-mark{ vertical-align: super }
span.footnote-mark sup.textsuperscript, span.footnote-mark a sup.textsuperscript{ font-size:80%; }
div.tabular, div.center div.tabular {text-align: center; margin-top:0.5em; margin-bottom:0.5em; }
table.tabular td p{margin-top:0em;}
table.tabular {margin-left: auto; margin-right: auto;}
div.td00{ margin-left:0pt; margin-right:0pt; }
div.td01{ margin-left:0pt; margin-right:5pt; }
div.td10{ margin-left:5pt; margin-right:0pt; }
div.td11{ margin-left:5pt; margin-right:5pt; }
table[rules] {border-left:solid black 0.4pt; border-right:solid black 0.4pt; }
td.td00{ padding-left:0pt; padding-right:0pt; }
td.td01{ padding-left:0pt; padding-right:5pt; }
td.td10{ padding-left:5pt; padding-right:0pt; }
td.td11{ padding-left:5pt; padding-right:5pt; }
table[rules] {border-left:solid black 0.4pt; border-right:solid black 0.4pt; }
.hline hr, .cline hr{ height : 1px; margin:0px; }
.tabbing-right {text-align:right;}
span.TEX {letter-spacing: -0.125em; }
span.TEX span.E{ position:relative;top:0.5ex;left:-0.0417em;}
a span.TEX span.E {text-decoration: none; }
span.LATEX span.A{ position:relative; top:-0.5ex; left:-0.4em; font-size:85%;}
span.LATEX span.TEX{ position:relative; left: -0.4em; }
div.float img, div.float .caption {text-align:center;}
div.figure img, div.figure .caption {text-align:center;}
.marginpar {width:20%; float:right; text-align:left; margin-left:auto; margin-top:0.5em; font-size:85%; text-decoration:underline;}
.marginpar p{margin-top:0.4em; margin-bottom:0.4em;}
.equation td{text-align:center; vertical-align:middle; }
td.eq-no{ width:5%; }
table.equation { width:100%; } 
div.math-display, div.par-math-display{text-align:center;}
math .texttt { font-family: monospace; }
math .textit { font-style: italic; }
math .textsl { font-style: oblique; }
math .textsf { font-family: sans-serif; }
math .textbf { font-weight: bold; }
.partToc a, .partToc, .likepartToc a, .likepartToc {line-height: 200%; font-weight:bold; font-size:110%;}
.chapterToc a, .chapterToc, .likechapterToc a, .likechapterToc, .appendixToc a, .appendixToc {line-height: 200%; font-weight:bold;}
.index-item, .index-subitem, .index-subsubitem {display:block}
.caption td.id{font-weight: bold; white-space: nowrap; }
table.caption {text-align:center;}
h1.partHead{text-align: center}
p.bibitem { text-indent: -2em; margin-left: 2em; margin-top:0.6em; margin-bottom:0.6em; }
p.bibitem-p { text-indent: 0em; margin-left: 2em; margin-top:0.6em; margin-bottom:0.6em; }
.paragraphHead, .likeparagraphHead { margin-top:2em; font-weight: bold;}
.subparagraphHead, .likesubparagraphHead { font-weight: bold;}
.quote {margin-bottom:0.25em; margin-top:0.25em; margin-left:1em; margin-right:1em; text-align:\jmathustify;}
.verse{white-space:nowrap; margin-left:2em}
div.maketitle {text-align:center;}
h2.titleHead{text-align:center;}
div.maketitle{ margin-bottom: 2em; }
div.author, div.date {text-align:center;}
div.thanks{text-align:left; margin-left:10%; font-size:85%; font-style:italic; }
div.author{white-space: nowrap;}
.quotation {margin-bottom:0.25em; margin-top:0.25em; margin-left:1em; }
h1.partHead{text-align: center}
.sectionToc, .likesectionToc {margin-left:2em;}
.subsectionToc, .likesubsectionToc {margin-left:4em;}
.subsubsectionToc, .likesubsubsectionToc {margin-left:6em;}
.frenchb-nbsp{font-size:75%;}
.frenchb-thinspace{font-size:75%;}
.figure img.graphics {margin-left:10%;}
/* end css.sty */

\title{Etude metrique des arcs}
\author{}
\date{}

\begin{document}
\maketitle

\textbf{Warning: 
requires JavaScript to process the mathematics on this page.\\ If your
browser supports JavaScript, be sure it is enabled.}

\begin{center}\rule{3in}{0.4pt}\end{center}

{[}
{[}
{[}{]}
{[}

\subsubsection{18.4 Etude métrique des arcs}

\paragraph{18.4.1 Arcs rectifiables}

Soit E un espace vectoriel normé de dimension finie et \Gamma = ({[}a,b{]},F)
un arc paramétré dont l'intervalle de définition est un segment de \mathbb{R}~. A
toute subdivision \sigma = (a\_i)\_0\leqi\leqn de {[}a,b{]} on
associe la longueur

L(\Gamma,\sigma)

de la ligne polygonale inscrite dans \Gamma dont les sommets sont les
F(a\_i), c'est-à-dire

L(\Gamma,\sigma) = \sum \_i=1^nd(F(a~\_
i-1,F(a\_i)) = \\sum
\_i=1^n\\textbar{}F(a\_ i) -
F(a\_i-1)\\textbar{}

Lemme~18.4.1 Soit \sigma et \sigma' deux subdivisions de {[}a,b{]}. Si \sigma' est plus
fine que \sigma, on a L(\Gamma,\sigma) \leq L(\Gamma,\sigma').

Démonstration Par une récurrence évidente sur le nombre de points
a\jmathoutés à \sigma pour obtenir \sigma', il suffit de montrer le résultat lorsque \sigma
est composée de a\_0 = a \textless{} a\_1 \textless{}
\\ldots~ \textless{}
a\_n = b et \sigma' est composée de a\_0 = a \textless{}
\\ldots~ \textless{}
a\_k-1 \textless{} c \textless{} a\_k \textless{}
\\ldots~ \textless{}
a\_n = b. Dans ce cas on a

\begin{align*} L(\Gamma,\sigma)& =&
\\sum
\_i=1^n\\textbar{}F(a\_ i) -
F(a\_i-1)\\textbar{} =
\\sum
\_i=1^k-1\\textbar{}F(a\_ i) -
F(a\_i-1)\\textbar{}\%&
\\ & &
+\\textbar{}F(a\_k) -
F(a\_k-1)\\textbar{} +
\\sum
\_i=k+1^n\\textbar{}F(a\_ i) -
F(a\_i-1)\\textbar{}\ \%&
\\ \end{align*}

alors que

\begin{align*} L(\Gamma,\sigma')& =&
\\sum
\_i=1^k-1\\textbar{}F(a\_ i) -
F(a\_i-1)\\textbar{} +\\textbar{}
F(c) - F(a\_k-1)\\textbar{}\%&
\\ & &
+\\textbar{}F(a\_k) -
F(c)\\textbar{} + \\sum
\_i=k+1^n\\textbar{}F(a\_ i) -
F(a\_i-1)\\textbar{}\%&
\\ \end{align*}

et le résultat découle immédiatement de l'inégalité triangulaire.

Intuitivement, si l'on peut donner un sens à la longueur d'un arc
paramétré, suivant le principe la ligne droite est le plus court chemin
d'un point à un autre, la longueur de cet arc doit être plus grande que
la longueur de toute ligne polygonale inscrite dans l'arc. Ceci \jmathustifie
l'introduction de la définition suivante.

Définition~18.4.1 Soit E un espace vectoriel normé de dimension finie et
\Gamma = ({[}a,b{]},F) un arc paramétré dont l'intervalle de définition est
un segment de \mathbb{R}~. On dit que \Gamma est rectifiable si l'ensemble des L(\Gamma,\sigma)
est ma\jmathoré, \sigma décrivant l'ensemble des subdivisions de {[}a,b{]}. On
appelle alors longueur de l'arc \Gamma le nombre l(\Gamma)
=\
sup\L(\Gamma,\sigma)∣\sigma\text
subdivision de {[}a,b{]}\.

Remarque~18.4.1 En général les arcs continus ne sont pas rectifiables~;
les fractales donnent de bons exemples d'arcs paramétrés dont tout sous
arc est de longueur infini. Sans aller \jmathusque là, nous pouvons
construire facilement un graphe de fonction continue qui n'est pas
rectifiable. Prenons une fonction continue sur {[}0,1{]}, de classe
\mathcal{C}^1 sur {]}0,1{]} mais telle que la fonction
t\mapsto~\sqrt1 +
f'(t)^2 ne soit pas intégrable sur {]}0,1{]} (par exemple
f(t) = \sqrttsin~
1\over  t^2 si
t\neq~0 et f(0) = 0). Comme nous le verrons par
la suite, la longueur du graphe de x à 1 est égale à
\int  \_x^1~\sqrt1
+ f'(t)^2 dt et elle tend vers + \infty~ quand x tend vers 0,
bien que f soit continue.

Proposition~18.4.2 Soit \Gamma\_1 et \Gamma\_2 deux arcs définis
sur des segments et C^k-équivalents. Alors \Gamma\_1 est
rectifiable si et seulement si~\Gamma\_2 est rectifiable et dans ce
cas ils ont même longueur.

Démonstration Soit \Gamma\_1 = ({[}a,b{]},F) et \Gamma\_2 =
({[}c,d{]},G). Soit \theta un difféomorphisme de {[}a,b{]} sur {[}c,d{]} tel
que F = G \cdot \theta~; \theta est donc strictement monotone. A toute subdivision \sigma
de {[}a,b{]}, on peut associer une subdivision \theta^∗(\sigma) de
{[}c,d{]} de la manière suivante~: si \sigma est donnée par a\_0 = a
\textless{} a\_1 \textless{}
\\ldots~ \textless{}
a\_n = b, \theta^∗(\sigma) est la subdivision \theta(a\_0) =
c \textless{} \theta(a\_1) \textless{}
\\ldots~ \textless{}
\theta(a\_n) = d si \theta est croissant et la subdivision \theta(a\_n)
= c \textless{} \theta(a\_n-1) \textless{}
\\ldots~ \textless{}
\theta(a\_1) \textless{} \theta(a\_0) = d si \theta est décroissant~;
on obtient ainsi une bi\jmathection de l'ensemble des subdivisions de
{[}a,b{]} sur l'ensemble des subdivisions de {[}c,d{]}. La ligne
polygonale \jmathoignant les points F(a\_i) est encore la ligne
polygonale \jmathoignant les points G(\theta(a\_i)), et donc
L(\Gamma\_1,\sigma) = L(\Gamma\_2,\theta^∗(\sigma)). Comme
\theta^∗ est bi\jmathective, \theta^∗(\sigma) parcourt toutes les
subdivisions de {[}c,d{]}, et donc (en prenant des bornes supérieures
dans \overline\mathbb{R}~)

\begin{align*}
sup\L(\Gamma\_1,\sigma)\mathrel∣~\sigma\text
subdivision de {[}a,b{]}\&& \%&
\\ & =&
sup\L(\Gamma\_2,\theta^∗(\sigma))\mathrel∣~\sigma\text
subdivision de {[}a,b{]}\\%&
\\ & =&
sup\L(\Gamma\_2,\sigma')\mathrel∣~\sigma'\text
subdivision de {[}c,d{]}\ \%&
\\ \end{align*}

ce qui montre la proposition.

Proposition~18.4.3 Soit E un espace vectoriel normé de dimension finie
et \Gamma = ({[}a,b{]},F) un arc paramétré dont l'intervalle de définition
est un segment de \mathbb{R}~. Soit c \in{]}a,b{[}. Alors \Gamma est rectifiable si et
seulement si~les deux sous arcs \Gamma\_1 =
({[}a,c{]},F\_\textbar{}\_{[}a,c{]}) et \Gamma\_2 =
({[}c,b{]},F\_\textbar{}\_{[}c,b{]}) sont rectifiables.
Dans ce cas on a l(\Gamma) = l(\Gamma\_1) + l(\Gamma\_2).

Démonstration Supposons tout d'abord que \Gamma est rectifiable et soit
\sigma\_1 une subdivision de {[}a,c{]}, a\_0 = a \textless{}
a\_1 \textless{}
\\ldots~ \textless{}
a\_p = c. En a\jmathoutant le point b, on obtient une subdivision \sigma
de {[}a,b{]} et on a L(\Gamma,\sigma) = L(\Gamma\_1,\sigma\_1)
+\\textbar{} f(b) - f(c)\\textbar{}. On en
déduit que L(\Gamma\_1,\sigma\_1) \leq l(\Gamma)
-\\textbar{} f(b) - f(c)\\textbar{} ce qui
montre que \Gamma\_1 est rectifiable. On montre de même que
\Gamma\_2 est rectifiable.

Inversement, supposons \Gamma\_1 et \Gamma\_2 rectifiables et soit
\sigma une subdivision de {[}a,b{]}. En a\jmathoutant éventuellement à \sigma le point
c on obtient une subdivision \sigma' de {[}a,b{]}, plus fine que \sigma et qui est
la \jmathuxtaposition d'une subdivision \sigma\_1 de {[}a,c{]} et d'une
subdivision \sigma\_2 de {[}c,b{]}. La longueur de la ligne
polygonale correspondant à \sigma' est donc la somme des longueurs des lignes
polygonales correspondant à \sigma\_1 et \sigma\_2. On a donc

L(\Gamma,\sigma) \leq L(\Gamma,\sigma') = L(\Gamma\_1,\sigma\_1) +
L(\Gamma\_2,\sigma\_2) \leq l(\Gamma\_1) + l(\Gamma\_2)

Ceci montre que \Gamma est rectifiable et que l(\Gamma) \leq l(\Gamma\_1) +
l(\Gamma\_2).

Inversement, soit \epsilon \textgreater{} 0. Par définition de la borne
supérieure, il existe \sigma\_1 subdivision de {[}a,c{]} telle que
L(\Gamma\_1,\sigma\_1) ≥ l(\Gamma\_1) - \epsilon
\over 2 . De même, il existe \sigma\_2 subdivision
de {[}c,b{]} telle que L(\Gamma\_2,\sigma\_2) ≥ l(\Gamma\_2)
- \epsilon \over 2 . La \jmathuxtaposition \sigma de \sigma\_1 et
\sigma\_2 est une subdivision de {[}a,b{]} et on a

l(\Gamma) ≥ L(\Gamma,\sigma) = L(\Gamma\_1,\sigma\_1) +
L(\Gamma\_2,\sigma\_2) ≥ l(\Gamma\_1) + l(\Gamma\_2) - \epsilon

Donc \forall~~\epsilon \textgreater{} 0, l(\Gamma) ≥
l(\Gamma\_1) + l(\Gamma\_2) - \epsilon et donc l(\Gamma) ≥ l(\Gamma\_1) +
l(\Gamma\_2). Comme l'inégalité en sens inverse était dé\jmathà connue, on
a l'égalité.

On déduit immédiatement de ce résultat que si ({[}a,b{]},F) est
rectifiable et si {[}c,d{]} est un segment contenu dans {[}a,b{]}, alors
({[}c,d{]},F\_\textbar{}\_{[}c,d{]}) est encore
rectifiable, autrement dit que tout sous arc d'un arc rectifiable est
rectifiable. Ceci \jmathustifie la définition suivante

Définition~18.4.2 Soit \Gamma = (I,F) un arc paramétré. On dit que \Gamma est
rectifiable si pour tout segment {[}a,b{]} \subset~ I, le sous arc
({[}a,b{]},F\_\textbar{}\_{[}a,b{]}) est rectifiable.

Dans ce cas, pour tout couple a,b de I tel que a \textless{} b, on peut
définir \ell\_\Gamma(a,b) =
l({[}a,b{]},F\_\textbar{}\_{[}a,b{]}). La proposition
précédente montre clairement que si a \textless{} b \textless{} c, on a
\ell\_\Gamma(a,c) = \ell\_\Gamma(a,b) + \ell\_\Gamma(b,c). On prolonge
la définition de \ell\_\Gamma en posant \ell\_\Gamma(a,b) = 0 si a = b
et \ell\_\Gamma(a,b) = -\ell\_\Gamma(b,a) si b \textless{} a (convention
de Chasles). On obtient alors facilement

Proposition~18.4.4 (relation de Chasles). Soit E un espace vectoriel
normé, \Gamma = (I,F) un arc paramétré rectifiable de E. Alors

\forall~a,b,c \in I, \ell\_\Gamma~(a,c) =
\ell\_\Gamma(a,b) + \ell\_\Gamma(b,c)

Remarque~18.4.2 Il découle immédiatement des résultats précédents que si
deux arcs sont équivalents, ils sont simultanément rectifiables ou non
rectifiables~; de plus si \Gamma\_1 est équivalent à \Gamma\_2 et
de même sens (de manière à conserver la convention de Chasles), et si \theta
est le changement de paramétrage croissant qui permet de passer de l'un
à l'autre, on a

\ell\_\Gamma\_1(a,b) = \ell\_\Gamma\_2(\theta(a),\theta(b))

Par contre si \theta était décroissant, on aurait a \textless{} b \rigtharrow~ \theta(a)
\textgreater{} \theta(b) et la convention de Chasles donnerait
\ell\_\Gamma\_1(a,b) = -\ell\_\Gamma\_2(\theta(a),\theta(b))

\paragraph{18.4.2 Arcs de classe \mathcal{C}^1}

Théorème~18.4.5 Tout arc de classe \mathcal{C}^1 est rectifiable. Plus
précisément, si \Gamma = (I,F) est un arc de classe \mathcal{C}^1 de E,
alors \Gamma est rectifiable et

\forall~a,b \in I, \ell\_\Gamma~(a,b)
=\int ~
\_a^b\\textbar{}F'(t)\\textbar{}
dt

Démonstration Soit {[}a,b{]} un segment inclus dans I, \Gamma\_0 le
sous arc correspondant et \sigma = (a\_i)\_0\leqi\leqn une
subdivision de {[}a,b{]}. Comme F est de classe \mathcal{C}^1, on a

\begin{align*} L(\Gamma\_0,\sigma)& =&
\\sum
\_i=1^n\\textbar{}F(a\_ i) -
F(a\_i-1)\\textbar{} =
\\sum
\_i=1^n\\textbar{}
\\int  ~
\_a\_i-1^a\_i F'(t)
dt\\textbar{}\%& \\ & \leq&
\sum \_i=1^n~
\\int  ~
\_a\_i-1^a\_i
\\textbar{}F'(t)\\textbar{} dt =
\\int  ~
\_a^b\\textbar{}F'(t)\\textbar{}
dt \%& \\
\end{align*}

Ceci montre que \Gamma\_0 est rectifiable et que \ell\_\Gamma(a,b) =
l(\Gamma\_0) \leq\int ~
\_a^b\\textbar{}F'(t)\\textbar{}
dt.

Fixons maintenant a \in I~; nous allons montrer que
\forall~t \in I, \ell\_\Gamma~(a,t)
=\int ~
\_a^t\\textbar{}F'(u)\\textbar{}
du. Comme \ell\_\Gamma(a,a) = 0, il suffit de montrer que
t\mapsto~\ell\_\Gamma(a,t) est dérivable et que sa
dérivée est \\textbar{}F'(t)\\textbar{}~;
on en déduira que t\mapsto~\ell\_\Gamma(a,t) est de
classe \mathcal{C}^1 et que donc \ell\_\Gamma(a,t) = \ell\_\Gamma(a,a)
+\int  \_a^t~ d
\over du \ell\_\Gamma(a,u) du
=\int ~
\_a^t\\textbar{}F'(u)\\textbar{}
du. Montrons tout d'abord la dérivabilité à droite. Soit h
\textgreater{} 0. On a alors \ell\_\Gamma(a,t + h) - \ell\_\Gamma(a,t) =
\ell\_\Gamma(t,t + h). Mais en utilisant d'une part la ligne polygonale
triviale qui \jmathoint par un seul segment les points F(t) et F(t + h), et
d'autre part la ma\jmathoration de la longueur par l'intégrale de
\\textbar{}F'\\textbar{} dé\jmathà démontrée,
on a

\\textbar{}F(t + h) - F(t)\\textbar{} \leq
\ell\_\Gamma(t,t + h) \leq\int ~
\_t^t+h\\textbar{}F'(u)\\textbar{}
du

soit encore, en divisant par h

\begin{align*} \\textbar{} F(t + h)
- F(t) \over h \\textbar{}& \leq&
\ell\_\Gamma(a,t + h) - \ell\_\Gamma(a,t) \over h \%&
\\ & \leq& 1 \over h
\int ~
\_t^t+h\\textbar{}F'(u)\\textbar{}du
\%& \\ & =& 1 \over
h \left (\int ~
\_a^t+h\\textbar{}F'(u)\\textbar{}
du -\int ~
\_a^t\\textbar{}F'(u)\\textbar{}
du\right )\%& \\
\end{align*}

Quand h tend vers 0 le terme de gauche (par définition de la dérivée) et
le terme de droite (dérivée d'une intégrale par rapport à sa borne
supérieure) tendent tous les deux vers
\\textbar{}F'(t)\\textbar{}. On en déduit
que lim\_h\rightarrow~0^+~
\ell\_\Gamma(a,t+h)-\ell\_\Gamma(a,t) \over h
=\\textbar{} F'(t)\\textbar{}. Donc
t\mapsto~\ell\_\Gamma(a,t) est dérivable à droite,
de dérivée \\textbar{}F'\\textbar{}. Le
raisonnement est similaire à gauche.

Si maintenant b et c sont dans I, on a \ell\_\Gamma(b,c) =
\ell\_\Gamma(a,c) - \ell\_\Gamma(a,b) =\int ~
\_a^c\\textbar{}F'(t)\\textbar{}
dt -\int ~
\_a^b\\textbar{}F'(t)\\textbar{}
dt =\int ~
\_b^c\\textbar{}F'(t)\\textbar{}
dt ce qui achève la démonstration.

Exemple~18.4.1 (i) Pour un arc paramétré plan
t\mapsto~x(t)\vec\imath +
y(t)\vecȷ dans un repère orthonormé, on a
\\textbar{}F'(t)\\textbar{} =
\sqrtx'(t)^2  + y'(t)^2 et donc
\ell\_\Gamma(a,b) =\int ~
\_a^b\sqrtx'(t)^2  +
y'(t)^2 dt. En particulier le graphe d'une fonction f de
classe \mathcal{C}^1 sur {[}a,b{]} est rectifiable et sa longueur est
\int  \_a^b~\sqrt1
+ f'(t)^2 dt.

(ii) Pour un arc plan donné par une équation polaire \rho = f(\theta), on a
F'(\theta) = f'(\theta)\vecu(\theta) +
f(\theta)\vecu'(\theta), soit
\\textbar{}F'(\theta)\\textbar{} =
\sqrtf(\theta)^2  + f'(\theta)^2 et donc
\ell\_\Gamma(a,b) =\int ~
\_a^b\sqrtf(\theta)^2  +
f'(\theta)^2 d\theta

(iii) Pour un arc paramétré de l'espace \mathbb{R}~^3,
t\mapsto~x(t)\vec\imath +
y(t)\vecȷ + z(t)\veck dans un
repère orthonormé, on a
\\textbar{}F'(t)\\textbar{} =
\sqrtx'(t)^2  + y'(t)^2  +
z'(t)^2 et donc \ell\_\Gamma(a,b)
=\int ~
\_a^b\sqrtx'(t)^2  +
y'(t)^2  + z'(t)^2 dt

Le lecteur adaptera ces formules pour un arc donné en coordonnées
cylindriques ou sphériques.

\paragraph{18.4.3 Abscisses curvilignes}

Définition~18.4.3 Soit \Gamma = (I,F) un arc rectifiable. On appelle abscisse
curviligne sur \Gamma toute application s : I \rightarrow~ \mathbb{R}~ telle que
\forall~a,b \in I, \ell\_\Gamma~(a,b) = s(b) - s(a). On
dit que \Gamma est paramétré par abscisse curviligne si s(t) = t est une
abscisse curviligne, autrement dit si \forall~~a,b \in
I, \ell\_\Gamma(a,b) = b - a.

Remarque~18.4.3 Choisissons sur \Gamma une origine a\_0. La relation
de Chasles montre de manière évidente que s(t) =
\ell\_\Gamma(a\_0,t) est une abscisse curviligne sur \Gamma.

Proposition~18.4.6 Sur un arc rectifiable, deux abscisses curvilignes
diffèrent d'une constante.

Démonstration Soit a\_0 \in I. On a \ell\_\Gamma(a\_0,t) =
s\_1(t) - s\_1(a\_0) = s\_2(t) -
s\_2(a\_0), si bien que s\_2(t) =
s\_1(t) + K avec K = s\_2(a\_0) -
s\_1(a\_0).

Théorème~18.4.7 Un arc \Gamma = (I,F) de classe \mathcal{C}^1 est paramétré
par abscisse curviligne si et seulement si~\forall~~t
\in I, \\textbar{}F'(t)\\textbar{} = 1.

Démonstration Supposons tout d'abord que \forall~~t \in
I, \\textbar{}F'(t)\\textbar{} = 1. Alors
on a \forall~a,b \in \Gamma, \ell\_\Gamma~(a,b)
=\int ~
\_a^b\\textbar{}F'(t)\\textbar{}
dt =\int  \_a^b~ dt = b - a.
Inversement, supposons l'arc paramétré par abscisse curviligne, alors si
a \in I, on a \forall~t \in I, t - a = \ell\_\Gamma~(a,t)
=\int ~
\_a^t\\textbar{}F'(u)\\textbar{}
du et en dérivant par rapport à t, on obtient 1
=\\textbar{} F'(t)\\textbar{}.

Théorème~18.4.8 Un arc de classe C^k (k ≥ 1) est
C^k-équivalent à un arc paramétré par abscisse curviligne si
et seulement si~il est régulier.

Démonstration Le théorème précédent implique que tout arc paramétré par
abscisse curviligne est régulier et comme tout arc équivalent à un arc
régulier est lui même régulier, la condition est évidemment nécessaire.
Inversement, soit \Gamma = (I,F) un arc paramétré régulier. Soit s(t) une
abscisse curviligne sur \Gamma. On sait que, si a \in I, s(t) = s(a)
+\int ~
\_a^t\\textbar{}F'(u)\\textbar{}
du et donc s est de classe C^k (car
u\mapsto~\\textbar{}F'(u)\\textbar{}
est de classe C^k-1 si F' ne s'annule pas) et s'(t)
=\\textbar{} F'(t)\\textbar{}
\textgreater{} 0. On en déduit que s est un C^k
difféomorphisme de I sur J = s(I). Soit G = F \cdot s^-1 : J \rightarrow~ E.
L'arc (J,G) est C^k équivalent à (I,F) et on a (puisque
s^-1 est un difféomorphisme croissant)

\begin{align*}
\ell\_J,G(u\_1,u\_2)& =&
\ell\_(I,F)(s^-1(u\_
1),s^-1(u\_ 2)) \%&
\\ & =& s(s^-1(u\_
1)) - s(s^-1(u\_ 2)) = u\_1 -
u\_2\%& \\
\end{align*}

donc (J,G) est paramétré par abscisse curviligne.

Remarque~18.4.4 Soit (I,F) et (J,G) deux arcs paramétrés équivalents et
de même sens et \theta un difféomorphisme croissant tel que F = G \cdot \theta.
Supposons que (J,G) est paramétré par abscisse curviligne. On a alors
F'(t) = \theta'(t)G'(\theta(t)) et comme G'(\theta(t)) est de norme 1 et \theta'(t)
\textgreater{} 0, on a \theta'(t) =\\textbar{}
F'(t)\\textbar{}. On voit que \theta est déterminé à une
constante près. Si l'on pose s = \theta(t), on aura donc  ds
\over dt =\\textbar{}
F'(t)\\textbar{} ce qui peut encore s'écrire de la
manière suivante

\begin{itemize}
\itemsep1pt\parskip0pt\parsep0pt
\item
  (i) Pour un arc paramétré plan
  t\mapsto~x(t)\vec\imath +
  y(t)\vecȷ dans un repère orthonormé, on a
  ds^2 = dx^2 + dy^2 (c'est-à-dire que
  \left ( ds \over dt
  \right )^2 = \left ( dx
  \over dt \right )^2 +
  \left ( dy \over dt
  \right )^2)
\item
  (ii) Pour un arc plan donné par une équation polaire \rho = f(\theta), on a
  ds^2 = d\rho^2 + \rho^2 d\theta^2
  (c'est-à-dire que \left ( ds \over
  d\theta \right )^2 = \left (
  d\rho \over d\theta \right )^2 +
  \rho^2)
\item
  (iii) Pour un arc paramétré de l'espace \mathbb{R}~^3,
  t\mapsto~x(t)\vec\imath +
  y(t)\vecȷ + z(t)\veck dans un
  repère orthonormé, on a ds^2 = dx^2 +
  dy^2 + dz^2 (c'est-à-dire que
  \left ( ds \over dt
  \right )^2 = \left ( dx
  \over dt \right )^2 +
  \left ( dy \over dt
  \right )^2 + \left ( dz
  \over dt \right )^2)
\end{itemize}

\paragraph{18.4.4 Introduction à la méthode du repère mobile}

Soit E un espace affine euclidien de direction \vecE.
On désigne par \mathcal{R} l'ensemble des repères orthonormés de E.

Définition~18.4.4 On appelle repère mobile tout couple (I,R) d'un
intervalle I de \mathbb{R}~ et d'une application t\mapsto~R(t)
de I dans \mathcal{R}, de classe \mathcal{C}^1.

On notera a(t) \in E l'origine du repère R(t) et \mathcal{E}(t) =
(\overrightarrowe\_1(t),\\ldots,\overrightarrowe\_n~(t))
la base orthonormée définissant R(t). Le résultat essentiel est le
suivant

Théorème~18.4.9 Soit t\mapsto~R(t) =
(a(t),\overrightarrowe\_1(t),\\ldots,\overrightarrowe\_n~(t))
un repère mobile. Alors, pour tout t \in I, la matrice des coordonnées des
vecteurs  d\overrightarrowe\_1
\over dt
(t),\\ldots~,
d\overrightarrowe\_n \over
dt (t) dans la base
(\overrightarrowe\_1(t),\\ldots,\overrightarrowe\_n~(t))
est antisymétrique.

Démonstration Posons  d\overrightarrowe\_\jmath
\over dt (t) =\
\sum ~
\_i=1^na\_i,\jmath(t)\overrightarrowe\_i(t).
Par dérivation, la relation \forall~~t \in I,
(\overrightarrowe\_i(t)∣\overrightarrowe\_i(t))
= 1 donne \forall~~t \in I, 2(
d\overrightarrowe\_i \over
dt
(t)∣\overrightarrowe\_i(t))
= 0 soit encore a\_i,i(t) = 0. De même la relation
\forall~~t \in I,
(\overrightarrowe\_i(t)∣\overrightarrowe\_\jmath(t))
= 0 donne par dérivation \forall~~t \in I, (
d\overrightarrowe\_i \over
dt
(t)∣\overrightarrowe\_\jmath(t))
+ ( d\overrightarrowe\_\jmath
\over dt
(t)∣\overrightarrowe\_i(t))
= 0 soit encore a\_i,\jmath(t) + a\_\jmath,i(t) = 0. Ceci montre
bien que la matrice est antisymétrique.

\paragraph{18.4.5 Repère de Frénet et courbure des arcs d'un plan
euclidien orienté}

On désigne par E un plan euclidien orienté.

Définition~18.4.5 Soit \Gamma = (I,F) un arc paramétré par abscisse
curviligne de classe \mathcal{C}^1. On appelle repère de Frénet au
point s \in I le repère orthonormé direct
(F(s),\vect(s),\vecn(s)) dont
l'origine est le point F(s) et tel que \vect(s) =
F'(s).

Justification~: on sait que
\\textbar{}F'(s)\\textbar{} = 1. De plus
la connaissance de \vect(s) détermine parfaitement
\vecn(s) qui doit être l'image de
\vect(s) par la rotation d'angle + \pi~
\over 2 .

Supposons que (I,F) est un arc paramétré par abscisse curviligne de
classe C^2. Alors l'application
s\mapsto~\vect(s) = F'(s) est de
classe \mathcal{C}^1 et il en est de même de
s\mapsto~\vecn(s) =
r\_\pi~\diagup2(\vect(s)). Le théorème sur le repère
mobile nous dit que la matrice des coordonnées des vecteurs 
d\vect \over ds (s),
d\vecn \over ds (s) dans la base
(\vect(s),\vecn(s)) est
antisymétrique, donc de la forme \left
(\matrix\,0 &-c(s) \cr
c(s)&0 \right ).

Définition~18.4.6 Soit \Gamma = (I,F) un arc paramétré par abscisse
curviligne de classe C^2. On appelle courbure de \Gamma au point s
de I le réel c(s) défini par la relation F''(s) =
d\vect \over ds (s) =
c(s)\vecn(s).

Remarque~18.4.5 On a donc les formules (dites formules de Frénet)

 d\vect \over ds (s) =
c(s)\vecn(s), d\vecn
\over ds (s) = -c(s)\vect(s)

Définition~18.4.7 Soit \Gamma = (I,F) un arc paramétré régulier de classe
C^2. Soit (J,G) un arc paramétré par abscisse curviligne
équivalent et de même sens, \theta : I \rightarrow~ J un difféomorphisme croissant tel
que F = G \cdot \theta. On appelle repère de Frénet (resp. courbure) à \Gamma au point
t \in I le repère de Frénet (resp. la courbure) au point \theta(t) à l'arc
(J,G).

Remarque~18.4.6 Il faut voir bien entendu que cette définition ne dépend
pas du choix de (J,G). Cela découle de la quasi-unicité de \theta que nous
avons vue précédemment, ou bien plus simplement du théorème suivant

Théorème~18.4.10 Soit \Gamma = (I,F) un arc paramétré régulier de classe
C^2. Alors le repère de Frénet et la courbure au point t à \Gamma
sont donnés respectivement par

\vect\_\Gamma(t) = F'(t) \over
\\textbar{}F'(t)\\textbar{}
,\quad \vecn\_\Gamma(t) =
r\_\pi~\diagup2(\vect\_\Gamma(t)),\quad
c\_\Gamma(t) = {[}F'(t),F'`(t){]} \over
\\textbar{}F'(t)\\textbar{}^3

où {[}F'(t),F''(t){]} désigne le produit mixte des vecteurs F'(t) et
F''(t).

Démonstration Par définition, si (J,G) désigne un arc paramétré par
abscisse curviligne équivalent et de même sens, \theta : I \rightarrow~ J un
difféomorphisme tel que F = G \cdot \theta, on a
\vect\_\Gamma(t) = G'(\theta(t)) et on a G''(\theta(t)) =
c\_\Gamma(t)\vecn\_\Gamma(t). On a alors

F'(t) = (G \cdot \theta)'(t) = \theta'(t)G'(\theta(t)) =
\theta'(t)\vect\_\Gamma(t)

Comme \theta'(t) \textgreater{} 0, on a \theta'(t) =\\textbar{}
F'(t)\\textbar{} et donc
\vect\_\Gamma(t) = F'(t) \over
\\textbar{}F'(t)\\textbar{} . De plus on
a

\begin{align*} F'`(t)& =& \theta'`(t)G'(\theta(t)) +
\theta'(t)^2G'`(\theta(t)) \%& \\ & =&
\theta'`(t)\vect\_\Gamma(t) +\\textbar{}
F'(t)\\textbar{}^2c\_
\Gamma(t)\vecn\_\Gamma(t)\%&
\\ \end{align*}

d'où l'on déduit que

\begin{align*} {[}F'(t),F'`(t){]}& =&
{[}\\textbar{}F'(t)\\textbar{}\vect\_\Gamma(t),\theta'`(t)\vect\_\Gamma(t)
+\\textbar{}
F'(t)\\textbar{}^2c\_
\Gamma(t)\vecn\_\Gamma(t){]}\%&
\\ & =&
c\_\Gamma(t)\\textbar{}F'(t)\\textbar{}^3{[}\vect\_
\Gamma(t),\vecn\_\Gamma(t){]} =
c\_\Gamma(t)\\textbar{}F'(t)\\textbar{}^3
\%& \\ \end{align*}

ce qui démontre la dernière formule.

Corollaire~18.4.11 Soit \Gamma = (I,F) un arc paramétré régulier de classe
C^2 et t \in I. On a équivalence de (i) t est un point
birégulier de \Gamma (ii) c\_\Gamma(t)\neq~0

Démonstration En effet t est birégulier si et seulement si~la famille
(F'(t),F''(t)) est libre c'est-à-dire si et seulement si~le produit
mixte {[}F'(t),F''(t){]} est non nul.

Corollaire~18.4.12 Soit \Gamma l'arc défini en polaire par \rho = f(\theta). Alors la
courbure au point \theta est donnée par

c = \rho^2 + 2\rho'^2 - \rho\rho'`\over
(\rho^2 + \rho'^2)^3\diagup2

(avec \rho' = f'(\theta) et \rho'`= f''(\theta)).

Démonstration On a F'(\theta) = f'(\theta)\vecu(\theta) +
f(\theta)\vecu'(\theta) et F'`(\theta) = (f'`(\theta) -
f(\theta))\vecu(\theta) + 2f'(\theta)\vecu'(\theta).
On en déduit que {[}F'(\theta),F'`(\theta){]} = \rho^2 + 2\rho'^2
- \rho\rho'' et que \\textbar{}F'(\theta)\\textbar{}
= \sqrt\rho^2  + \rho'^2, d'où la
formule.

Nous allons maintenant donner une autre interprétation de la courbure à
l'aide d'une détermination de l'angle \phi du vecteur tangent unitaire
\vect avec un vecteur fixe \vec\imath.
Pour pouvoir dériver une telle détermination, nous utiliserons le lemme
suivant

Lemme~18.4.13 (lemme de relèvement \mathcal{C}^1). Soit I un intervalle
de \mathbb{R}~ et f : I \rightarrow~ U = \z \in
\mathbb{C}∣\textbar{}z\textbar{} = 1\
une application de classe \mathcal{C}^1. Alors il existe g : I \rightarrow~ \mathbb{R}~ de
classe \mathcal{C}^1 telle que \forall~~t \in I, f(t) =
e^ig(t). Si g\_1 et g\_2 sont deux telles
applications, il existe k \in ℤ tel que \forall~~t \in I,
g\_2(t) = g\_1(t) + 2k\pi~.

Démonstration Traitons tout d'abord la question de l'unicité. Soit
g\_1 et g\_2 qui conviennent. Alors
\forall~~t \in I,
e^i(g\_2(t)-g\_1(t)) = 1 soit
\forall~t \in I, g\_2(t) - g\_1~(t) \in
2\pi~ℤ. Or, d'après le théorème des valeurs intermédiaires, (g\_2 -
g\_1)(I) est un intervalle~; cet intervalle devant être contenu
dans 2\pi~ℤ, c'est forcément un singleton et donc il existe k \in ℤ tel que
\forall~t \in I, g\_2(t) = g\_1~(t) +
2k\pi~. En ce qui concerne l'existence, remarquons que si g existe, on doit
avoir f'(t) = ig'(t)e^ig(t) = ig'(t)f(t) et donc g'(t) =
f'(t) \over if(t) .

Puisque \forall~~t \in
I,f(t)\overlinef(t) = 1, par dérivation on obtient
\forall~t \in I,f(t)\overlinef'(t)~ +
f'(t)\overlinef(t) = 0, donc  1
\over i f'(t)\overlinef(t) =
f'(t)\over if(t) \in \mathbb{R}~. Soit donc a \in I~; choisissons \alpha~
\in \mathbb{R}~ tel que e^i\alpha~ = f(a) et soit g(t) = \alpha~
+\int  \_a^t~ f'(u)
\over if(u) du = \alpha~ - i\int ~
\_a^tf'(u)\overlinef(u) du. La fonction
g est de classe \mathcal{C}^1 de I dans \mathbb{R}~ et g'(t) = f'(t)
\over if(t) ~; posons h(t) = f(t)e^-ig(t)~;
h est de classe \mathcal{C}^1 et on a h'(t) = f'(t)e^-ig(t)
- ig'(t)f(t)e^-ig(t) = \left (f'(t) -
ig'(t)f(t)\right )e^-ig(t) = 0. Donc h est
constante. Comme h(a) = f(a)e^-ig(a) = f(a)e^-i\alpha~ =
1, h est la constante 1 et donc \forall~~t \in I, f(t) =
e^ig(t).

Lemme~18.4.14 Soit E un plan vectoriel euclidien,
(\vec\imath,\vecȷ) une base orthonormée
de E, S = \x \in
E∣\\textbar{}x\\textbar{}
= 1\. Soit I un intervalle de \mathbb{R}~ et f : I \rightarrow~ S de classe
\mathcal{C}^1. Alors il existe g : I \rightarrow~ \mathbb{R}~ de classe \mathcal{C}^1 telle
que \forall~~t \in I, f(t) =\
cos g(t)\,\vec\imath
+ sin~
g(t)\,\vecȷ. Si g\_1 et
g\_2 sont deux telles applications, il existe k \in ℤ tel que
\forall~t \in I, g\_2(t) = g\_1~(t) +
2k\pi~.

Démonstration Il suffit d'utiliser l'isométrie L de E dans \mathbb{C} (considérés
tous deux comme espaces euclidiens) définie par
L(a\vec\imath + b\vecȷ) = a + ib et
d'appliquer le lemme précédent à L \cdot f. On a alors L \cdot f(t) =
e^ig(t), soit f(t) = L^-1(e^ig(t))
= cos~
g(t)\,\vec\imath +\
sin g(t)\,\vecȷ.

Théorème~18.4.15 Soit \Gamma = (I,F) un arc paramétré régulier de classe
C^2 du plan euclidien E. Soit
(\vec\imath,\vecȷ) une base orthonormée
de E. Pour chaque t \in I soit \vect\_\Gamma(t) le
vecteur tangent unitaire au point t, soit \phi : I \rightarrow~ \mathbb{R}~ une application de
classe \mathcal{C}^1 telle que \forall~~t \in I,
\vect\_\Gamma(t) = cos~
\phi(t)\,\vec\imath +\
sin \phi(t)\,\vecȷ et
t\mapsto~s(t) une abscisse curviligne sur \Gamma. Alors

\forall~t \in I, \phi'(t) = c\_\Gamma~(t)s'(t)

Démonstration Si (J,G) désigne un arc paramétré par abscisse curviligne
équivalent et de même sens, \theta : I \rightarrow~ J un difféomorphisme tel que F = G \cdot
\theta, on a \vect\_\Gamma(t) = G'(\theta(t))~; on sait
aussi que \theta'(t) =\\textbar{}
F'(t)\\textbar{} = s'(t). On a donc F'(t) =
\theta'(t)\vect\_\Gamma(t) =
s'(t)(cos~
\phi(t)\,\vec\imath +\
sin \phi(t)\,\vecȷ) d'où F'`(t) =
s'`(t)(cos~
\phi(t)\,\vec\imath +\
sin \phi(t)\,\vecȷ) +
s'(t)\phi'(t)(-sin~
\phi(t)\,\vec\imath +\
cos \phi(t)\,\vecȷ) ce qui nous donne
{[}F'(t),F'`(t){]} = s'(t)^2\phi'(t) alors que
\\textbar{}F'(t)\\textbar{} = s'(t). La
formule s'en déduit immédiatement.

Remarque~18.4.7 Sur un arc paramétré par abscisse curviligne
s\mapsto~F(s), on peut prendre le paramètre comme
abscisse curviligne et on a donc \forall~~s \in I,
c\_\Gamma(s) = d\phi \over ds (s). La courbure mesure
donc la vitesse avec laquelle tourne le vecteur tangent en fonction de
l'abscisse curviligne.

Formulaire

Pour un arc paramétré
t\mapsto~x(t)\vec\imath +
y(t)\vecȷ, on a vu que \left ( ds
\over dt \right )^2 =
\left ( dx \over dt
\right )^2 + \left ( dy
\over dt \right )^2~; de plus
F'(t) =\\textbar{}
F'(t)\\textbar{}(cos~
\phi(t)\,\vec\imath +\
sin \phi(t)\,\vecȷ) = ds
\over dt (cos~
\phi(t)\,\vec\imath +\
sin \phi(t)\,\vecȷ) ce qui nous donne
 dx \over dt = ds \over dt
 cos \phi et  dy \over dt~ =
ds \over dt  sin~ \phi, avec
enfin  d\phi \over dt = c\_\Gamma\,
ds \over dt . On retiendra ces formules en termes de
formes différentielles sous la forme

dx = ds\,cos~
\phi,\quad dy =
ds\,sin~ \phi,\quad
d\phi = c\,ds

d'où l'on retrouve que ds^2 = dx^2 +
dy^2.

Pour un arc en polaires donné par \rho = f(\theta), on a F'(\theta) =
f'(\theta)\vecu(\theta) + f(\theta)\vecu'(\theta). Par
le théorème de relèvement, il existe une application
\theta\mapsto~\alpha~(\theta) de classe C^1 telle que
\vect(\theta) = cos~
\alpha~(\theta)\,\vecu(\theta)
+ sin~
\alpha~(\theta)\,\vecu'(\theta) et on a alors

\begin{align*} F'(\theta)& =&
f'(\theta)\vecu(\theta) + f(\theta)\vecu'(\theta) \%&
\\ & =&
\\textbar{}F'(\theta)\\textbar{}(cos~
\alpha~(\theta)\,\vecu(\theta)
+ sin~
\alpha~(\theta)\,\vecu'(\theta))\%&
\\ & =& ds \over d\theta
(cos~
\alpha~(\theta)\,\vecu(\theta)
+ sin~
\alpha~(\theta)\,\vecu'(\theta)) \%&
\\ \end{align*}

On en déduit que \rho = f(\theta) = ds \over d\theta
 sin~ \alpha~(\theta) et que  d\rho \over
d\theta = ds \over d\theta  sin~
\alpha~(\theta). D'autre part on peut évidemment prendre \phi(\theta) = \theta + \alpha~(\theta) ce qui
nous donne les formules suivantes en termes de formes différentielles

d\rho = ds\,cos~
\alpha~,\quad \rho\,d\theta =
ds\,sin~ \alpha~, \phi = \theta +
\alpha~,\quad d\phi = c\,ds

Exemple~18.4.2 Considérons la cardioïde d'équation \rho = a(1
+ cos~ \theta), \theta \in{]} - \pi~,\pi~{[}. On a alors
ds\,cos~ \alpha~ = d\rho =
-asin~ \theta\,d\theta =
-2acos  \theta \over 2~
 sin  \theta \over 2~
\,d\theta et ds\,sin~
\alpha~ = \rho\,d\theta = a(1 + cos~
\theta)\,d\theta = 2acos ^2~
\theta \over 2 \,d\theta. Comme  ds
\over d\theta \textgreater{} 0, ceci nécessite ds =
2acos  \theta \over 2~ d\theta. On a
alors cos \alpha~ = -\sin~
 \theta \over 2 , sin~ \alpha~
= cos  \theta \over 2~ , ce qui
détermine \alpha~ à une constante près par \alpha~ = \theta \over 2
+ \pi~ \over 2 . On en déduit que \phi = \alpha~ + \theta = 3\theta
\over 2 + \pi~ \over 2 et donc d\phi = 3
\over 2 \,d\theta = c\,ds =
2accos  \theta \over 2~ d\theta. Par
conséquent c = 3 \over 4a cos~
 \theta \over 2  .

\paragraph{18.4.6 Centre de courbure, cercle osculateur}

Définition~18.4.8 Soit \Gamma = (I,F) un arc paramétré régulier de classe
C^2 et t \in I un point birégulier de \Gamma. On appelle rayon de
courbure au point t le nombre réel R\_\Gamma(t) = 1
\over c\_\Gamma(t) .

Remarque~18.4.8 Dans un arc paramétré par abscisse curviligne
s\mapsto~F(s), les formules de Frénet peuvent alors
s'écrire

 d\vect \over ds (s) = 1
\over R(s) \vecn(s),
d\vecn \over ds (s) = - 1
\over R(s) \vect(s)

Définition~18.4.9 Soit \Gamma = (I,F) un arc paramétré régulier de classe
C^2 et t \in I un point birégulier de \Gamma. Soit
(F(t),\vect\_\Gamma(t),\vecn\_\Gamma(t))
le repère de Frénet au point t et R\_\Gamma(t) le rayon de courbure
au point t. On appelle centre de courbure au point t le point
C\_\Gamma(t) = F(t) +
R\_\Gamma(t)\vecn\_\Gamma(t), on appelle cercle
de courbure au point t le cercle ayant pour centre le centre de courbure
et pour rayon le rayon de courbure (et qui passe donc par le point
F(t)).

Remarque~18.4.9 Par définition même, il s'agit de notions invariantes
par changement de paramétrage admissible~: si \Gamma = (I,F) et \Gamma' = (J,G)
sont deux arcs équivalents et de même sens et si \theta est un
difféomorphisme croissant tel que F = G \cdot \theta, alors le rayon de courbure
(resp. le centre de courbure, resp. le cercle de courbure) au point \theta(t)
à \Gamma' est le rayon de courbure (resp. le centre de courbure, resp. le
cercle de courbure) au point t à \Gamma.

Théorème~18.4.16 Soit \Gamma = (I,F) un arc paramétré régulier de classe
C^2 et t\_0 \in I un point birégulier de \Gamma. Alors le
cercle de courbure au point t\_0 est l'unique cercle osculateur
à \Gamma au point t\_0.

Démonstration La notion de contact étant invariante par changement de
paramétrage admissible, on peut supposer que l'arc est paramétré par
abscisse curviligne, s\mapsto~F(s) et nous noterons
s\_0 à la place de t\_0. L'étude du contact pouvant se
faire dans n'importe quel repère, nous pouvons à cet effet utiliser le
repère de Frénet au point s\_0. Si on note alors
\overrightarrowF(s\_0)F(s) =
x(s)\vect(s\_0) +
y(s)\vecn(s\_0), les formules
F'(s\_0) =\vec t(s\_0) et
F''(s\_0) =
c(s\_0)\vecn(s\_0) se traduisent par
x'(s\_0) = 1,y'(s\_0) = 0,x'`(s\_0) =
0,y''(s\_0) = c(s\_0) avec bien entendu x(s\_0)
= y(s\_0) = 0 puisque F(s\_0) est l'origine du repère.
Considérons un cercle passant par F(s\_0)~; dans ce repère de
Frénet il admet une équation du type X^2 + Y ^2 -
2\alpha~X - 2\beta~Y = 0, son centre ayant pour coordonnées (\alpha~,\beta~). Si on introduit
l'équation aux points d'intersection du cercle et de \Gamma, on pose \phi(s) =
x(s)^2 + y(s)^2 - 2\alpha~x(s) - 2\beta~y(s) et la condition
d'osculation se traduit par \phi(s\_0) = \phi'(s\_0) =
\phi''(s\_0) = 0. Mais compte tenu des formules x(s\_0) =
y(s\_0) = 0,x'(s\_0) = 1,y'(s\_0) =
0,x'`(s\_0) = 0,y''(s\_0) = c(s\_0), on a
\phi(s\_0) = 0, \phi'(s\_0) = 2x(s\_0)x'(s\_0)
+ 2y(s\_0)y'(s\_0) - 2\alpha~x'(s\_0) -
2\beta~y'(s\_0) = -2\alpha~ et \phi'`(s\_0) =
2x'(s\_0)^2 + 2x(s\_0)x'`(s\_0) +
2y'(s\_0)^2 + 2y(s\_0)y'`(s\_0) -
2\alpha~x'`(s\_0) - 2\beta~y''(s\_0) = 2 - 2\beta~c(s\_0). On
voit donc que le cercle est tangent si et seulement si~\alpha~ = 0 (autrement
dit le cercle est centré sur la normale, c'était prévisible) et qu'il
est osculateur si et seulement si~\alpha~ = 0,\beta~ = 1 \over
c(s\_0) = R(s\_0), c'est-à-dire si et seulement si~son
centre est le point F(s\_0) +
R(s\_0)\vecn(s\_0), soit le centre de
courbure.

\paragraph{18.4.7 Développée, développantes}

Ces notions ne sont pas au programme des classes préparatoires.

Définition~18.4.10 Soit \Gamma = (I,F) un arc paramétré birégulier de classe
C^3. On appelle développée de \Gamma l'arc (I,C) tel que pour tout
t \in I, C(t) soit le centre de courbure au point t à \Gamma.

Théorème~18.4.17 Soit \Gamma = (I,F) un arc paramétré birégulier de classe
C^k (k ≥ 3) de développée (I,C) et soit t\_0 \in I un
point non totalement singulier de (I,C). Alors la tangente au point
t\_0 à la développée est la normale au point t\_0 à \Gamma,
c'est-à-dire la droite F(t\_0) +
\mathbb{R}~\vecn\_\Gamma(t\_0)~: la développée de
l'arc \Gamma est l'enveloppe des normales à \Gamma.

Démonstration Par définition même cette normale passe par
C(t\_0). Il nous suffit donc de montrer qu'elle est parallèle à
un vecteur tangent à (I,C). Toutes les notions étant invariantes par
changement de paramétrage admissible, on peut supposer que l'arc est
paramétré par abscisse curviligne, s\mapsto~F(s) et
nous noterons s\_0 à la place de t\_0. On a alors C(s) =
F(s) + R(s)\vecn(s). Les formules de calcul de la
courbure (et donc du rayon de courbure) montrent que
s\mapsto~R(s) est de classe C^k-2~;
d'autre part s\mapsto~\vecn(s)
est (comme s\mapsto~\vect(s) dont
elle se déduit par rotation) de classe C^k-1. On en déduit
que s\mapsto~C(s) est de classe C^k-2. On
a alors

\begin{align*} C'(s)& =& F'(s) +
R'(s)\vecn(s) + R(s) d\vecn
\over dt (s) \%& \\ &
=& \vect(s) + R'(s)\vecn(s) -
R(s) \vect(s) \over R(s) =
R'(s)\vecn(s)\%& \\
\end{align*}

On en déduit que s\_0 est un point régulier de la développée si
et seulement si~R'(s\_0)\neq~0 et que
dans ce cas le vecteur \vecn(s\_0) est un
vecteur tangent à la développée, ce qui démontre dans ce cas le
résultat. La formule de Leibnitz appliquée à C'(s) =
R'(s)\vecn(s) nous montre clairement que s\_0
est un point non totalement singulier de la développée si et seulement
si~il existe n tel que
R^(n)(s\_0)\neq~0. Si l'on
suppose alors que R'(s\_0) =
\\ldots~ =
R^(p-1)(s\_0) = 0 et
R^(p)(s\_0)\neq~0, cette même
formule de Leibnitz montre que C'(s\_0) =
\\ldots~ =
C^(p-1)(s\_0) = 0 et C^(p)(s\_0) =
R^(p)(s\_0)\vecn(s\_0)\neq~0
(tous les autres termes faisant intervenir des dérivées d'ordre
inférieur de R qui sont nulles au point s\_0). Le vecteur
\vecn(s\_0) est donc un vecteur tangent à la
développée, ce qui démontre dans le résultat.

Remarque~18.4.10 Il est parfois beaucoup plus rapide de rechercher la
développée comme enveloppe des normales que de calculer repère de Frénet
et rayon de courbure. On pourra également, lorsque l'arc est paramétré
par t\mapsto~x(t)\vec\imath +
y(t)\vecȷ paramétrer la développée par
t\mapsto~\xi(t)\vec\imath +
\eta(t)\vecȷ avec

\begin{align*} \xi(t)\vec\imath +
\eta(t)\vecȷ& =& F(t) +
R(t)\vecn(t) \%& \\ &
=& x(t)\vec\imath + y(t)\vecȷ +
s'(t) \over \phi'(t) (-sin~
\phi(t)\vec\imath + cos~
\phi(t)\vecȷ)\%& \\
\end{align*}

et en tenant compte de x'(t) = s'(t)cos~ \phi(t)
et de y'(t) = s'(t)sin~ \phi(t) (qui traduit
simplement F'(t) =\\textbar{}
F'(t)\\textbar{}\vect(t), on obtient
\xi(t) = x(t) - y'(t) \over \phi'(t) , \eta(t) = y(t) +
x'(t) \over \phi'(t) qu'on retiendra en terme de formes
différentielles

\xi = x - dy \over d\phi ,\quad \eta = y +
dx \over d\phi

Nous avons vu également que s\_0 est un point régulier de la
développée si et seulement
si~R'(s\_0)\neq~0. On en déduit que les
points qui annulent R', et en particulier les points qui réalisent des
extremums locaux de R (qui sont appelés les sommets de \Gamma), donnent des
points singuliers de la développée (le plus souvent des points de
rebroussement de première espèce).

Considérons maintenant le problème inverse~: étant donné un arc \Gamma =
(I,F) (que nous supposerons régulier), peut-on trouver un arc (I,G) dont
\Gamma soit la développée. On peut, sans nuire à la généralité, supposer que
(I,F) est paramétré par abscisse curviligne. La tangente F(s) +
\mathbb{R}~\vect(s) à l'arc \Gamma doit être la normale à l'arc
(I,G) et en particulier le point G(s) doit appartenir à cette tangente.
On doit donc avoir G(s) = F(s) + \lambda~(s)\vect(s) avec
\lambda~(s) =
(\overrightarrowG(s)F(s)∣\vect(s))
qui doit donc être de classe \mathcal{C}^1. On en déduit que

\begin{align*} G'(s)& =& F'(s) +
\lambda~'(s)\vect(s) +
\lambda~(s)c(s)\vecn(s)\%& \\
& =& (1 + \lambda~'(s))\vect(s) +
\lambda~(s)c(s)\vecn(s) \%&
\\ \end{align*}

Comme la tangente à l'arc \Gamma doit être la normale à l'arc (I,G), il est
nécessaire que G'(s) soit orthogonal à \vect(s),
c'est-à-dire que 1 + \lambda~'(s) = 0. Ceci exige que \lambda~(s) = s\_0 - s.
Inversement, si cette condition est vérifiée, on a G(s) = F(s) +
(s\_0 - s)\vect(s) et G'(s) = (s\_0 -
s)c(s)\vecn(s). Si l'arc \Gamma est birégulier et si
s\_0∉I, alors l'arc (I,G) est
régulier et \Gamma est l'enveloppe des normales à (I,G), donc la développée
de (I,G). Pour un arc quelconque, ceci amène à introduire la définition
suivante

Définition~18.4.11 Soit \Gamma = (I,F) un arc paramétré régulier de classe
\mathcal{C}^1, t\mapsto~s(t) une abscisse
curviligne sur \Gamma et \vect(t) le vecteur tangent
unitaire au point t. Les arcs paramétrés
t\mapsto~F(t) + (s\_0 -
s(t))\vect(t) où s\_0 \in \mathbb{R}~, sont appelés les
développantes de l'arc \Gamma.

Remarque~18.4.11 L'arc n'est la développée d'une de ses développantes
t\mapsto~F(t) + (s\_0 -
s(t))\vect(t) que s'il est suffisamment dérivable,
birégulier et si s\_0 - s(t) ne s'annule pas sur I.

\paragraph{18.4.8 Equations intrinsèques}

Cette notion n'est pas au programme des classes préparatoires.

Une question supplémentaire se pose~: étant donnée une fonction continue
\gamma : I \rightarrow~ \mathbb{R}~, peut-on trouver un arc paramétré par abscisse curviligne \Gamma
tel que la courbure au point s \in I soit égale à \gamma(s)~? Le théorème
suivant répond à cette question

Théorème~18.4.18 Soit \gamma : I \rightarrow~ \mathbb{R}~ une fonction continue, E un plan
euclidien orienté, a \in E et \vecu
\in\overrightarrow E tel que
\\textbar{}u\\textbar{} = 1. Soit
s\_0 \in I. Alors il existe un unique arc paramétré par abscisse
curviligne (I,F) de classe C^2 tel que F(s\_0) = a et
F'(s\_0) =\vec u.

Démonstration Soit (\vec\imath,\vecȷ)
une base orthonormée de \overrightarrowE et
s\mapsto~\phi(s) une détermination de classe
\mathcal{C}^1 de l'angle de F'(s) =\vec t(s) avec
\vec\imath. Alors nécessairement F'(s)
= cos \phi(s)\vec\imath~
+ sin \phi(s)\vecȷ~. Mais on
sait de plus que \phi'(s) = c\_\Gamma(s) = \gamma(s). Alors nécessairement
\phi(s) = \alpha~ +\int ~
\_s\_0^s\gamma(u) du (où \alpha~ désigne une mesure de l'angle
de \vec\imath avec \vecu)~; ceci
détermine \phi(s) à un élément de 2\pi~ℤ près et donc détermine parfaitement
\vect(s). Mais alors nécessairement

\begin{align*} F(s)& =& F(s\_0)
+\int  \_s\_0^s~F'(u) du =
a +\int ~
\_s\_0^s\vect(u) du\%&
\\ & =& a +\\int
 \_s\_0^s(cos~
\phi(u)\vec\imath + sin~
\phi(u)\vecȷ) du \%& \\
\end{align*}

ce qui montre l'unicité de F. Mais en faisant les calculs en sens
inverse, on constate que si on définit F(s) par cette formule, l'arc
(I,F) convient~: il est paramétré par abscisse curviligne car
\\textbar{}F'(s)\\textbar{}
=\\textbar{} cos~
\phi(s)\vec\imath + sin~
\phi(s)\vecȷ\\textbar{} = 1, \phi(s) est une
détermination de classe \mathcal{C}^1 de l'angle de F'(s)
=\vec t(s) avec \vec\imath et comme
\phi'(s) = \gamma(s), la courbure au point s est bien \gamma(s).

Corollaire~18.4.19 Soit \Gamma\_1 = (I,F\_1) et \Gamma\_2
= (I,F\_2) deux arcs paramétrés par abscisse curviligne tels que
\forall~s \in I, c\_\Gamma\_1~(s) =
c\_\Gamma\_2(s). Alors les deux arcs sont isométriques.

Démonstration Soit s\_0 \in I et soit D l'unique déplacement de E
qui vérifie D(F\_1(s\_0)) = F\_2(s\_0)
et \overrightarrowD(F\_1'(s\_0)) =
F\_2'(s\_0). Les deux arcs \Gamma\_2 et
D(\Gamma\_1) ont la même fonction de courbure en fonction de
l'abscisse curviligne (car la courbure, notion métrique, est bien
entendu invariante par déplacement), coïncident en s\_0 ainsi
que leurs vecteurs tangents. D'après l'unicité du théorème précédent,
ces deux arcs sont égaux.

Remarque~18.4.12 Autrement dit, la fonction
s\mapsto~c\_\Gamma(s) définit l'arc \Gamma à un
déplacement près du plan affine euclidien. On dira encore que l'équation
c\_\Gamma(s) = \gamma(s) est une équation intrinsèque de l'arc \Gamma.

\paragraph{18.4.9 Courbure des arcs gauches}

Nous supposerons maintenant que E est un espace euclidien de dimension
3, orienté. Soit \Gamma = (I,F) un arc paramétré par abscisse curviligne de
classe C^2. On a donc \forall~~s \in I,
\\textbar{}F'(s)\\textbar{} = 1 et donc
\forall~~s \in I,
(F'(s)∣F'(s)) = 1. En dérivant, on obtient
\forall~~s \in I,(F'(s),F''(s)) = 0 ce qui montre que le
vecteur F''(s) est orthogonal au vecteur F'(s). Si le point s est
birégulier, on a  F'`(s) \over
\\textbar{}F''(s)\\textbar{} qui est un
vecteur unitaire orthogonal au vecteur unitaire F'(s). Ceci définit de
manière unique une base orthonormée directe dont les deux premiers
vecteurs sont F'(s) et  F'`(s) \over
\\textbar{}F''(s)\\textbar{} .

Définition~18.4.12 Soit E un espace euclidien de dimension 3 orienté.
Soit \Gamma = (I,F) un arc paramétré par abscisse curviligne de classe
C^2 et s un point birégulier de \Gamma. On appelle repère de
Frénet au point s le repère orthonormé direct
(F(s),\vect(s),\vecn(s),\vecb(s))
dont l'origine est le point image F(s) et tel que
\vect(s) = F'(s), \vecn(s) =
F'`(s) \over
\\textbar{}F''(s)\\textbar{}\\textbar{}
, \vecb(s) =\vec t(s)
∧\vec n(s).

Si l'arc est birégulier de classe C^3, on dispose ainsi d'un
repère mobile
s\mapsto~(F(s),\vect(s),\vecn(s),\vecb(s)).
La théorie du repère mobile assure que la matrice des coordonnées des
dérivées des vecteurs
\vect(s),\vecn(s),\vecb(s)
dans la base
(\vect(s),\vecn(s),\vecb(s))
est antisymétrique. En tenant compte de ce que 
d\vect \over ds (s) = F''(s) est
colinéaire à \vecn, cette matrice est nécessairement
de la forme

\left (\matrix\,0
&-c(s)&0 \cr c(s)&0 &-\tau(s) \cr 0
&\tau(s) &0 \right )

Définition~18.4.13 Soit E un espace euclidien de dimension 3 orienté.
Soit \Gamma = (I,F) un arc paramétré par abscisse curviligne birégulier de
classe C^3. Pour s \in I, on appelle courbure et torsion au
point s les nombres réels définis par les formules

 d\vect \over ds (s) =
c(s)\vecn(s),\quad 
d\vecb \over ds (s) =
-\tau(s)\vecn(s)

On a également  d\vecn \over ds
(s) = -c(s)\vect(s) +
\tau(s)\vecb(s).

Remarque~18.4.13 On a  d\vect \over
ds (s) = F''(s) = c(s)\vecn(s), d'où nécessairement
c(s) =\\textbar{} F''(s)\\textbar{}
\textgreater{} 0. Contrairement aux arcs plans, la courbure d'un arc
gauche est nécessairement strictement positive.

Comme dans le cas d'un arc plan, on étend ces définitions aux arcs
biréguliers par équivalence d'arcs.

Définition~18.4.14 Soit \Gamma = (I,F) un arc paramétré birégulier de classe
C^3 de E espace euclidien de dimension 3 orienté. Soit (J,G)
un arc paramétré par abscisse curviligne équivalent et de même sens, \theta :
I \rightarrow~ J un difféomorphisme croissant tel que F = G \cdot \theta. On appelle repère
de Frénet (resp. courbure, resp. torsion) à \Gamma au point t \in I le repère
de Frénet (resp. la courbure, resp. la torsion) au point \theta(t) à l'arc
(J,G).

Méthodes de calcul

Par définition, si s\_0 = \theta(t\_0), on a
\vect(t\_0) = G'(s\_0) et
\vecn(t\_0) = G'`(s\_0)
\over
\\textbar{}G''(s\_0)\\textbar{}
. Mais on a G = F \cdot \theta ce qui nous donne F'(t\_0) =
\theta'(t\_0)G'(\theta(t\_0)) =
\theta'(t\_0)\vect(t\_0) et
F'`(t\_0) = \theta'`(t\_0)G'(\theta(t\_0)) +
\theta'(t\_0)^2G''(\theta(t\_0)). On en déduit que
\theta'(t\_0) =\\textbar{}
F'(t\_0)\\textbar{} (comme d'habitude) et que

\begin{align*} F'(t\_0) ∧
F'`(t\_0)& =& \theta'(t\_0)^3G'(s\_ 0) ∧
G'`(s\_0) \%& \\ & =&
\theta'(t\_0)^3c(t\_
0)\vect(t\_0) ∧\vec
n(t\_0) = \theta'(t\_0)^3c(t\_
0)\vecb(t\_0)\%&
\\ \end{align*}

Comme \theta'(t\_0)^3c(t\_0) \textgreater{} 0,
c'est que \vecb(t\_0) =
F'(t\_0)∧F'`(t\_0) \over
\\textbar{}F'(t\_0)∧F''(t\_0)\\textbar{}
. On a ensuite \vecn(t\_0)
=\vec b(t\_0) ∧\vec
t(b\_0).

Reprenons alors le calcul ci dessus. On a vu que F'(t\_0) ∧
F'`(t\_0) =
\theta'(t\_0)^3c(t\_0)\vecb(t\_0).
On en déduit, puisque la courbure est positive, que c(t\_0) =
\\textbar{}F'(t\_0)∧F'`(t\_0)\\textbar{}
\over \theta'(t\_0)^3 =
\\textbar{}F'(t\_0)∧F'`(t\_0)\\textbar{}
\over
\\textbar{}F'(t\_0)\\textbar{}^3
. On en déduit la proposition

Proposition~18.4.20 Soit \Gamma = (I,F) un arc paramétré birégulier de classe
C^3 de E espace euclidien de dimension 3 orienté et
t\_0 \in I. Alors le repère de Frénet au point t\_0 est
donné par les formules

\vect(t\_0) = F'(t\_0)
\over
\\textbar{}F'(t\_0)\\textbar{}
,\quad \vecb(t\_0) =
F'(t\_0) ∧ F'`(t\_0) \over
\\textbar{}F'(t\_0) ∧
F''(t\_0)\\textbar{}

\vecn(t\_0) =\vec
b(t\_0) ∧\vec t(b\_0)

La courbure au point t\_0 est donnée par

c(t\_0) = \\textbar{}F'(t\_0) ∧
F'`(t\_0)\\textbar{} \over
\\textbar{}F'(t\_0)\\textbar{}^3

Remarque~18.4.14 Avec la même technique, le lecteur montrera sans
difficulté que la torsion est donnée par \tau(t\_0) =
{[}F'(t\_0),F'`(t\_0),F'`'(t\_0){]}
\over
\\textbar{}F'(t\_0)∧F''(t\_0)\\textbar{}^2
.

{[}
{[}
{[}
{[}

\end{document}
