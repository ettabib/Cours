\section{Bases et dimension}

\subsection{Existence de bases}

\begin{lem}[fondamental]
Soit $E$ un $K$-espace vectoriel et $\mathcal{E}$ une famille de $E$. On a équivalence de 
\begin{enumerate}
\item $\mathcal{E}$ est une \index{base} base de $E$ 
\item $\mathcal{E}$ est une \index{famille libre!maximale} famille libre maximale (toute surfamille stricte est liée) 
\item $\mathcal{E}$ est une \index{famille génératrice!minimale} famille génératrice minimale (toute sous-famille stricte est non génératrice).
\end{enumerate}
\end{lem}

\begin{proof}
Si $\mathcal{E}$ est une base, on ne peut lui adjoindre aucun vecteur $x$ sans obtenir une famille liée puisque $x$ est combinaison linéaire de $\mathcal{E}$; on ne peut lui retirer aucun vecteur $e_{i_0}$ sans obtenir une famille non génératrice puisque $e_{i_0}$ n'est pas combinaison linéaire des autres vecteurs. Donc une base est une famille libre maximale et une famille génératrice minimale.

Inversement, soit $\mathcal{E}$ une famille libre maximale; si on lui adjoint un vecteur $x$, la famille devient liée, c'est donc que $x$ est combinaison linéaire de $\mathcal{E}$ et donc $\mathcal{E}$ est également génératrice, donc c'est une base.

De même soit $\mathcal{E}$ une famille génératrice minimale; si elle était liée, l'un des vecteurs serait combinaison linéaire des autres vecteurs et la famille obtenue en retirant ce vecteur serait encore génératrice, ce qui n'est pas; la famille est donc libre, donc c'est une base.
\end{proof}

\begin{thm}[de la base incomplète]
\index{théorème!base incomplète}
Soit $E$ un $K$-espace vectoriel, $(e_i)_{i\in I}$ une famille génératrice de $E$. Soit $L$ une partie de $I$ telle que la famille $(e_i)_{i\in L}$ soit libre. Alors il existe une partie $J$ avec $L \subset J \subset I$ telle que $(e_i)_{i\in J}$ est une base de $E$.
\end{thm}

\begin{proof}
On considère $X = \{J | L \subset J \subset I \text{ et } (e_i)_{i\in J} \text{ libre}\}$ ordonné par l'inclusion. Si $(J_s)_{s\in S}$ est une famille totalement ordonnée de $X$, alors $J = \cup_{s\in S} J_s$ est encore dans $X$ (facile car si on a une sous-famille finie de la famille $(e_j)_{j\in J}$, ils sont tous dans un même $J_s$, donc forment une famille libre) et c'est un majorant de la famille. Le théorème de Zorn garantit que l'ensemble $X$ admet un élément maximal. Soit $J$ un tel élément maximal. La famille $(e_i)_{i\in J}$ est libre. D'autre part, pour tout $j \in I \setminus J$, la famille $(e_i)_{i\in I \cup \{j\}}$ est liée, donc $e_j$ est combinaison linéaire de la famille $(e_i)_{i\in J}$. On en déduit que la famille est aussi génératrice.
\end{proof}

\begin{thm}
Tout $K$-espace vectoriel admet des bases.
\end{thm}

\begin{proof}
Prendre la famille $(x)_{x\in E}$ comme famille génératrice et la famille vide comme sous-famille libre.
\end{proof}

\begin{thm}
Soit $E$ un $K$-espace vectoriel et $F$ un \index{sous-espace vectoriel} sous-espace vectoriel de $E$. Alors $F$ admet des \index{supplémentaire} supplémentaires dans $E$.
\end{thm}

\begin{proof}
Il suffit de prendre une base de $F$, de la compléter en une base de $E$ et de prendre pour supplémentaire le sous-espace vectoriel engendré par les vecteurs de la base introduits en supplément.
\end{proof}

\subsection{Espaces vectoriels de dimension finie. Dimension}

\begin{de}
\index{espace vectoriel!dimension finie}
On dit qu'un espace vectoriel $E$ est de dimension finie, s'il admet une famille génératrice finie.
\end{de}

\begin{lem}
Soit $E$ un espace vectoriel de dimension finie, $(x_1,\ldots,x_n)$ une famille génératrice finie. Alors toute famille libre a un cardinal inférieur ou égal à $n$.
\end{lem}

\begin{proof}
Il suffit de démontrer par récurrence sur $n$ que si $n+1$ vecteurs $y_1,\ldots,y_{n+1}$ sont combinaisons linéaires de $n$ vecteurs $x_1,\ldots,x_n$, alors la famille $(y_1,\ldots,y_{n+1})$ est liée (c'est évident pour $n=1$). Pour cela on écrit $y_j = \sum_{i=1}^n a_{i,j} x_j$. Si, pour tout $i$, $a_{i,n} = 0$ alors $y_1,\ldots,y_n$ sont combinaisons linéaires de $x_1,\ldots,x_{n-1}$, donc forment une famille liée; il en est de même a fortiori de la famille complète. Sinon par exemple $a_{n+1,n} \neq 0$. On pose alors $\forall i \in [1,n], z_i = y_i - \frac{a_{i,n}}{a_{n+1,n}} y_{n+1} = \sum_{j=1}^{n-1} b_{i,j} x_j$. Par récurrence, la famille $z_1,\ldots,z_n$ est liée (combinaisons linéaires de $x_1,\ldots,x_{n-1}$), donc il existe $\alpha_1,\ldots,\alpha_n$ non tous nuls tels que $\alpha_1 z_1 + \ldots + \alpha_n z_n = 0$ ce qui donne après remplacement $\alpha_1 y_1 + \ldots + \alpha_n y_n + \beta y_{n+1} = 0$ et montre que la famille est liée. Ceci achève la démonstration.
\end{proof}

\begin{rem}
Ceci permet de redémontrer dans ce cas le théorème de la base incomplète sans faire appel au théorème de Zorn, l'existence d'un sous-ensemble $J$ de $X$ maximal étant garanti par la limitation sur son cardinal induite par le lemme précédent. Le lemme précédent montre aussi que le cardinal d'une famille libre est nécessairement fini et que le cardinal d'une famille libre est inférieur au cardinal d'une famille génératrice. Comme les bases sont à la fois libres et génératrices, on a le théorème suivant :
\end{rem}

\begin{thm}
\index{dimension!espace vectoriel}
Soit $E$ un $K$-espace vectoriel de dimension finie. Alors toutes les bases de $E$ ont un même cardinal fini, appelé la dimension de $E$. Toute famille libre est finie et a un cardinal inférieur ou égal à $\dim E$ avec égalité si et seulement si c'est une base de $E$. Toute famille génératrice a un cardinal supérieur ou égal à $\dim E$ avec égalité si et seulement si c'est une base de $E$.
\end{thm}

\subsection{Résultats sur la dimension}

\begin{thm}
\index{isomorphisme!espaces vectoriels}
Deux espaces sont isomorphes si et seulement si ils ont la même dimension.
\end{thm}

\begin{proof}
Deux espaces de même dimension sont isomorphes : prendre deux bases et envoyer l'une sur l'autre. Inversement, l'image d'une base par un isomorphisme étant une base, deux espaces isomorphes ont même dimension.
\end{proof}

\begin{thm}
\begin{enumerate}
\item \index{dimension!produit} $\dim(E \times F) = \dim E + \dim F$
\item \index{dimension!somme directe} $\dim(F_1 \oplus \cdots \oplus F_k) = \sum \dim F_i$
\item \index{dimension!quotient} $\dim(E/F) = \dim E - \dim F$
\item \index{théorème!du rang} $f : E \rightarrow F$ linéaire, $\dim E = \dim \Ker f + \dim \Im f$ (théorème du rang)
\item \index{dimension!somme} $\dim(F + G) = \dim F + \dim G - \dim(F \cap G)$
\item $F \subset E, \quad \dim F \leq \dim E$ avec égalité si et seulement si $F = E$
\item \index{dimension!applications linéaires} $\dim L(E,F) = \dim E \cdot \dim F$
\end{enumerate}
\end{thm}

\begin{proof}
\begin{enumerate}
\item Si $(e_i)_{i\in I}$ et $(f_j)_{j\in J}$ sont des bases respectives de $E$ et $F$, on vérifie immédiatement que la famille $\left((e_i,0)\right)_{i\in I} \cup \left((0,f_j)\right)_{j\in J})$ est une base de $E \times F$ en écrivant $(x,y) = (x,0) + (0,y)$.
\item $F_1 \oplus \cdots \oplus F_k$ est isomorphe à $F_1 \times \cdots \times F_k$ (prendre $(x_1,\ldots,x_k) \mapsto x_1 + \ldots + x_k$) qui est de dimension $\sum \dim F_i$ par récurrence à partir du résultat précédent.
\item L'espace $E/F$ est isomorphe à tout supplémentaire de $F$ dans $E$ qui est de dimension $\dim E - \dim F$ d'après le résultat précédent.
\item Il existe un isomorphisme $\overline{f}$ de $E/\Ker f$ sur $\Im f$, d'où $\dim \Im f = \dim E/\Ker f = \dim E - \dim \Ker f$.
\item L'application linéaire $f : F \times G \rightarrow E, (x,y) \mapsto x + y$ a pour image $F + G$ et pour noyau $\{(x,-x) | x \in F \cap G\}$ qui est naturellement isomorphe à $F \cap G$. Il suffit donc d'appliquer le résultat précédent à $f$.
\item D'après le théorème de la base incomplète, toute base de $F$ est une famille libre de $E$, donc peut être complétée en une base de $E$, d'où $\dim F \leq \dim E$ avec égalité si et seulement si $F = E$
\item On peut soit utiliser le résultat analogue sur les matrices que l'on verra plus loin, soit montrer que si $(e_i)_{i\in I}$ et $(f_j)_{j\in J}$ sont des bases respectives de $E$ et $F$, alors les applications linéaires $\phi_{j,i} \in L(E,F)$ définies par

$\phi_{j,i}(e_k) = \begin{cases} f_j & \text{si } k = i \\ 0 & \text{si } k \neq i \end{cases}$

forment une base de $L(E,F)$, ce qui est laissé en exercice.
\end{enumerate}
\end{proof}

\begin{thm}
\index{supplémentaires!caractérisation}
Soit $F$ et $G$ deux sous-espaces vectoriels de $E$ de dimension finie. On a équivalence de 
\begin{enumerate}
\item $F$ et $G$ sont supplémentaires dans $E$ 
\item $E = F + G$ et $\dim E = \dim F + \dim G$
\item $F \cap G = \{0\}$ et $\dim E = \dim F + \dim G$
\end{enumerate}
\end{thm}

\begin{proof}
Élémentaire.
\end{proof}

\begin{rem}
On sait que si $F$ est un sous-espace vectoriel de $E$, la restriction à tout supplémentaire $G$ de $F$ dans $E$ de la projection canonique de $E$ sur $E/F$ est un isomorphisme d'espace vectoriel. Ceci justifie la définition suivante :
\end{rem}

\begin{de}
\index{codimension}
Soit $E$ un $K$ espace vectoriel et $F$ un sous-espace vectoriel de $E$. On appelle codimension de $F$ la dimension (éventuellement infinie) de l'espace vectoriel quotient $E/F$, qui est encore la dimension de tout supplémentaire de $F$ dans $E$.
\end{de}