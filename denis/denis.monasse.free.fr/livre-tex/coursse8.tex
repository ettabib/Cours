\documentclass[]{article}
\usepackage[T1]{fontenc}
\usepackage{lmodern}
\usepackage{amssymb,amsmath}
\usepackage{ifxetex,ifluatex}
\usepackage{fixltx2e} % provides \textsubscript
% use upquote if available, for straight quotes in verbatim environments
\IfFileExists{upquote.sty}{\usepackage{upquote}}{}
\ifnum 0\ifxetex 1\fi\ifluatex 1\fi=0 % if pdftex
  \usepackage[utf8]{inputenc}
\else % if luatex or xelatex
  \ifxetex
    \usepackage{mathspec}
    \usepackage{xltxtra,xunicode}
  \else
    \usepackage{fontspec}
  \fi
  \defaultfontfeatures{Mapping=tex-text,Scale=MatchLowercase}
  \newcommand{\euro}{€}
\fi
% use microtype if available
\IfFileExists{microtype.sty}{\usepackage{microtype}}{}
\ifxetex
  \usepackage[setpagesize=false, % page size defined by xetex
              unicode=false, % unicode breaks when used with xetex
              xetex]{hyperref}
\else
  \usepackage[unicode=true]{hyperref}
\fi
\hypersetup{breaklinks=true,
            bookmarks=true,
            pdfauthor={},
            pdftitle={Bases et dimension},
            colorlinks=true,
            citecolor=blue,
            urlcolor=blue,
            linkcolor=magenta,
            pdfborder={0 0 0}}
\urlstyle{same}  % don't use monospace font for urls
\setlength{\parindent}{0pt}
\setlength{\parskip}{6pt plus 2pt minus 1pt}
\setlength{\emergencystretch}{3em}  % prevent overfull lines
\setcounter{secnumdepth}{0}
 
/* start css.sty */
.cmr-5{font-size:50%;}
.cmr-7{font-size:70%;}
.cmmi-5{font-size:50%;font-style: italic;}
.cmmi-7{font-size:70%;font-style: italic;}
.cmmi-10{font-style: italic;}
.cmsy-5{font-size:50%;}
.cmsy-7{font-size:70%;}
.cmex-7{font-size:70%;}
.cmex-7x-x-71{font-size:49%;}
.msbm-7{font-size:70%;}
.cmtt-10{font-family: monospace;}
.cmti-10{ font-style: italic;}
.cmbx-10{ font-weight: bold;}
.cmr-17x-x-120{font-size:204%;}
.cmsl-10{font-style: oblique;}
.cmti-7x-x-71{font-size:49%; font-style: italic;}
.cmbxti-10{ font-weight: bold; font-style: italic;}
p.noindent { text-indent: 0em }
td p.noindent { text-indent: 0em; margin-top:0em; }
p.nopar { text-indent: 0em; }
p.indent{ text-indent: 1.5em }
@media print {div.crosslinks {visibility:hidden;}}
a img { border-top: 0; border-left: 0; border-right: 0; }
center { margin-top:1em; margin-bottom:1em; }
td center { margin-top:0em; margin-bottom:0em; }
.Canvas { position:relative; }
li p.indent { text-indent: 0em }
.enumerate1 {list-style-type:decimal;}
.enumerate2 {list-style-type:lower-alpha;}
.enumerate3 {list-style-type:lower-roman;}
.enumerate4 {list-style-type:upper-alpha;}
div.newtheorem { margin-bottom: 2em; margin-top: 2em;}
.obeylines-h,.obeylines-v {white-space: nowrap; }
div.obeylines-v p { margin-top:0; margin-bottom:0; }
.overline{ text-decoration:overline; }
.overline img{ border-top: 1px solid black; }
td.displaylines {text-align:center; white-space:nowrap;}
.centerline {text-align:center;}
.rightline {text-align:right;}
div.verbatim {font-family: monospace; white-space: nowrap; text-align:left; clear:both; }
.fbox {padding-left:3.0pt; padding-right:3.0pt; text-indent:0pt; border:solid black 0.4pt; }
div.fbox {display:table}
div.center div.fbox {text-align:center; clear:both; padding-left:3.0pt; padding-right:3.0pt; text-indent:0pt; border:solid black 0.4pt; }
div.minipage{width:100%;}
div.center, div.center div.center {text-align: center; margin-left:1em; margin-right:1em;}
div.center div {text-align: left;}
div.flushright, div.flushright div.flushright {text-align: right;}
div.flushright div {text-align: left;}
div.flushleft {text-align: left;}
.underline{ text-decoration:underline; }
.underline img{ border-bottom: 1px solid black; margin-bottom:1pt; }
.framebox-c, .framebox-l, .framebox-r { padding-left:3.0pt; padding-right:3.0pt; text-indent:0pt; border:solid black 0.4pt; }
.framebox-c {text-align:center;}
.framebox-l {text-align:left;}
.framebox-r {text-align:right;}
span.thank-mark{ vertical-align: super }
span.footnote-mark sup.textsuperscript, span.footnote-mark a sup.textsuperscript{ font-size:80%; }
div.tabular, div.center div.tabular {text-align: center; margin-top:0.5em; margin-bottom:0.5em; }
table.tabular td p{margin-top:0em;}
table.tabular {margin-left: auto; margin-right: auto;}
div.td00{ margin-left:0pt; margin-right:0pt; }
div.td01{ margin-left:0pt; margin-right:5pt; }
div.td10{ margin-left:5pt; margin-right:0pt; }
div.td11{ margin-left:5pt; margin-right:5pt; }
table[rules] {border-left:solid black 0.4pt; border-right:solid black 0.4pt; }
td.td00{ padding-left:0pt; padding-right:0pt; }
td.td01{ padding-left:0pt; padding-right:5pt; }
td.td10{ padding-left:5pt; padding-right:0pt; }
td.td11{ padding-left:5pt; padding-right:5pt; }
table[rules] {border-left:solid black 0.4pt; border-right:solid black 0.4pt; }
.hline hr, .cline hr{ height : 1px; margin:0px; }
.tabbing-right {text-align:right;}
span.TEX {letter-spacing: -0.125em; }
span.TEX span.E{ position:relative;top:0.5ex;left:-0.0417em;}
a span.TEX span.E {text-decoration: none; }
span.LATEX span.A{ position:relative; top:-0.5ex; left:-0.4em; font-size:85%;}
span.LATEX span.TEX{ position:relative; left: -0.4em; }
div.float img, div.float .caption {text-align:center;}
div.figure img, div.figure .caption {text-align:center;}
.marginpar {width:20%; float:right; text-align:left; margin-left:auto; margin-top:0.5em; font-size:85%; text-decoration:underline;}
.marginpar p{margin-top:0.4em; margin-bottom:0.4em;}
.equation td{text-align:center; vertical-align:middle; }
td.eq-no{ width:5%; }
table.equation { width:100%; } 
div.math-display, div.par-math-display{text-align:center;}
math .texttt { font-family: monospace; }
math .textit { font-style: italic; }
math .textsl { font-style: oblique; }
math .textsf { font-family: sans-serif; }
math .textbf { font-weight: bold; }
.partToc a, .partToc, .likepartToc a, .likepartToc {line-height: 200%; font-weight:bold; font-size:110%;}
.chapterToc a, .chapterToc, .likechapterToc a, .likechapterToc, .appendixToc a, .appendixToc {line-height: 200%; font-weight:bold;}
.index-item, .index-subitem, .index-subsubitem {display:block}
.caption td.id{font-weight: bold; white-space: nowrap; }
table.caption {text-align:center;}
h1.partHead{text-align: center}
p.bibitem { text-indent: -2em; margin-left: 2em; margin-top:0.6em; margin-bottom:0.6em; }
p.bibitem-p { text-indent: 0em; margin-left: 2em; margin-top:0.6em; margin-bottom:0.6em; }
.paragraphHead, .likeparagraphHead { margin-top:2em; font-weight: bold;}
.subparagraphHead, .likesubparagraphHead { font-weight: bold;}
.quote {margin-bottom:0.25em; margin-top:0.25em; margin-left:1em; margin-right:1em; text-align:justify;}
.verse{white-space:nowrap; margin-left:2em}
div.maketitle {text-align:center;}
h2.titleHead{text-align:center;}
div.maketitle{ margin-bottom: 2em; }
div.author, div.date {text-align:center;}
div.thanks{text-align:left; margin-left:10%; font-size:85%; font-style:italic; }
div.author{white-space: nowrap;}
.quotation {margin-bottom:0.25em; margin-top:0.25em; margin-left:1em; }
h1.partHead{text-align: center}
.sectionToc, .likesectionToc {margin-left:2em;}
.subsectionToc, .likesubsectionToc {margin-left:4em;}
.subsubsectionToc, .likesubsubsectionToc {margin-left:6em;}
.frenchb-nbsp{font-size:75%;}
.frenchb-thinspace{font-size:75%;}
.figure img.graphics {margin-left:10%;}
/* end css.sty */

\title{Bases et dimension}
\author{}
\date{}

\begin{document}
\maketitle

\textbf{Warning: \href{http://www.math.union.edu/locate/jsMath}{jsMath}
requires JavaScript to process the mathematics on this page.\\ If your
browser supports JavaScript, be sure it is enabled.}

\begin{center}\rule{3in}{0.4pt}\end{center}

{[}\href{coursse9.html}{next}{]} {[}\href{coursse7.html}{prev}{]}
{[}\href{coursse7.html\#tailcoursse7.html}{prev-tail}{]}
{[}\hyperref[tailcoursse8.html]{tail}{]}
{[}\href{coursch3.html\#coursse8.html}{up}{]}

\subsubsection{2.2 Bases et dimension}

\paragraph{2.2.1 Existence de bases}

Lemme~2.2.1 fondamental. Soit E un K-espace vectoriel et ℰ une famille
de E. On a équivalence de (i) ℰ est une base de E (ii) ℰ est une famille
libre maximale (toute surfamille stricte est liée) (iii) ℰ est une
famille génératrice minimale (toute sous-famille stricte est non
génératrice).

Démonstration Si ℰ est une base, on ne peut lui adjoindre aucun vecteur
x sans obtenir une famille liée puisque x est combinaison linéaire de
ℰ~; on ne peut lui retirer aucun vecteur \{e\}\_\{\{i\}\_\{0\}\} sans
obtenir une famille non génératrice puisque \{e\}\_\{\{i\}\_\{0\}\}
n'est pas combinaison linéaire des autres vecteurs. Donc une base est
une famille libre maximale et une famille génératrice minimale.

Inversement, soit ℰ une famille libre maximale~; si on lui adjoint un
vecteur x, la famille devient liée, c'est donc que x est combinaison
linéaire de ℰ et donc ℰ est également génératrice, donc c'est une base.

De même soit ℰ une famille génératrice minimale~; si elle était liée,
l'un des vecteurs serait combinaison linéaire des autres vecteurs et la
famille obtenue en retirant ce vecteur serait encore génératrice, ce qui
n'est pas~; la famille est donc libre, donc c'est une base.

Théorème~2.2.2 (de la base incomplète). Soit E un K-espace vectoriel ,
\{(\{e\}\_\{i\})\}\_\{i∈I\} une famille génératrice de E. Soit L une
partie de I telle que la famille \{(\{e\}\_\{i\})\}\_\{i∈L\} soit libre.
Alors il existe une partie J avec L ⊂ J ⊂ I telle que
\{(\{e\}\_\{i\})\}\_\{i∈J\} est une base de E.

Démonstration On considère X =
\textbackslash{}\{J\textbackslash{}mathrel\{∣\}L ⊂ J ⊂
I\textbackslash{}text\{ et
\}\{(\{e\}\_\{i\})\}\_\{i∈J\}\textbackslash{}text\{
libre\}\textbackslash{}\} ordonné par l'inclusion. Si
\{(\{J\}\_\{s\})\}\_\{s∈S\} est une famille totalement ordonnée de X,
alors J = \{∪\}\_\{s∈S\}\{J\}\_\{s\} est encore dans X (facile car si on
a une sous-famille finie de la famille \{(\{e\}\_\{j\})\}\_\{j∈J\}, ils
sont tous dans un même \{J\}\_\{s\}, donc forment une famille libre) et
c'est un majorant de la famille. Le théorème de Zorn garantit que
l'ensemble X admet un élément maximal. Soit J un tel élément maximal. La
famille \{(\{e\}\_\{i\})\}\_\{i∈J\} est libre. D'autre part, pour tout j
∈ I ∖ J, la famille
\{(\{e\}\_\{i\})\}\_\{i∈I∪\textbackslash{}\{j\textbackslash{}\}\} est
liée, donc \{e\}\_\{j\} est combinaison linéaire de la famille
\{(\{e\}\_\{i\})\}\_\{i∈J\}. On en déduit que la famille est aussi
génératrice.

Corollaire~2.2.3 Tout K-espace vectoriel admet des bases.

Démonstration Prendre la famille \{(x)\}\_\{x∈E\} comme famille
génératrice et la famille vide comme sous-famille libre.

Corollaire~2.2.4 Soit E un K-espace vectoriel et F un sous-espace
vectoriel de E. Alors F admet des supplémentaires dans E.

Démonstration Il suffit de prendre une base de F, de la compléter en une
base de E et de prendre pour supplémentaire le sous-espace vectoriel
engendré par les vecteurs de la base introduits en supplément.

\paragraph{2.2.2 Espaces vectoriels de dimension finie. Dimension}

Définition~2.2.1 On dit qu'un espace vectoriel E est de dimension finie,
s'il admet une famille génératrice finie.

Lemme~2.2.5 Soit E un espace vectoriel de dimension finie,
(\{x\}\_\{1\},\textbackslash{}mathop\{\textbackslash{}mathop\{\ldots{}\}\},\{x\}\_\{n\})
une famille génératrice finie. Alors toute famille libre a un cardinal
inférieur ou égal à n.

Démonstration Il suffit de démontrer par récurrence sur n que si n + 1
vecteurs
\{y\}\_\{1\},\textbackslash{}mathop\{\textbackslash{}mathop\{\ldots{}\}\},\{y\}\_\{n+1\}
sont combinaisons linéaires de n vecteurs
\{x\}\_\{1\},\textbackslash{}mathop\{\textbackslash{}mathop\{\ldots{}\}\},\{x\}\_\{n\},
alors la famille
(\{y\}\_\{1\},\textbackslash{}mathop\{\textbackslash{}mathop\{\ldots{}\}\},\{y\}\_\{n+1\})
est liée (c'est évident pour n = 1). Pour cela on écrit \{y\}\_\{j\}
=\{\textbackslash{}mathop\{ \textbackslash{}mathop\{∑ \}\}
\}\_\{i=1\}\^{}\{n\}\{a\}\_\{i,j\}\{x\}\_\{j\}. Si, pour tout i,
\{a\}\_\{i,n\} = 0 alors
\{y\}\_\{1\},\textbackslash{}mathop\{\textbackslash{}mathop\{\ldots{}\}\},\{y\}\_\{n\}
sont combinaisons linéaires de
\{x\}\_\{1\},\textbackslash{}mathop\{\textbackslash{}mathop\{\ldots{}\}\},\{x\}\_\{n−1\},
donc forment une famille liée~; il en est de même a fortiori de la
famille complète. Sinon par exemple
\{a\}\_\{n+1,n\}\textbackslash{}mathrel\{≠\}0. On pose alors
\textbackslash{}mathop\{∀\}i ∈ {[}1,n{]}, \{z\}\_\{i\} = \{y\}\_\{i\}
−\{ \{a\}\_\{i,n\} \textbackslash{}over \{a\}\_\{n+1,n\}\}
\{y\}\_\{n+1\} =\{\textbackslash{}mathop\{ \textbackslash{}mathop\{∑
\}\} \}\_\{j=1\}\^{}\{n−1\}\{b\}\_\{i,j\}\{x\}\_\{j\}. Par récurrence,
la famille
\{z\}\_\{1\},\textbackslash{}mathop\{\textbackslash{}mathop\{\ldots{}\}\},\{z\}\_\{n\}
est liée (combinaisons linéaires de
\{x\}\_\{1\},\textbackslash{}mathop\{\textbackslash{}mathop\{\ldots{}\}\},\{x\}\_\{n−1\}),
donc il existe
\{α\}\_\{1\},\textbackslash{}mathop\{\textbackslash{}mathop\{\ldots{}\}\},\{α\}\_\{n\}
non tous nuls tels que \{α\}\_\{1\}\{z\}\_\{1\} +
\textbackslash{}mathop\{\textbackslash{}mathop\{\ldots{}\}\} +
\{α\}\_\{n\}\{z\}\_\{n\} = 0 ce qui donne après remplacement
\{α\}\_\{1\}\{y\}\_\{1\} +
\textbackslash{}mathop\{\textbackslash{}mathop\{\ldots{}\}\} +
\{α\}\_\{n\}\{y\}\_\{n\} + β\{y\}\_\{n+1\} = 0 et montre que la famille
est liée. Ceci achève la démonstration.

Remarque~2.2.1 Ceci permet de redémontrer dans ce cas le théorème de la
base incomplète sans faire appel au théorème de Zorn, l'existence d'un
sous-ensemble J de X maximal étant garanti par la limitation sur son
cardinal induite par le lemme précédent. Le lemme précédent montre aussi
que le cardinal d'une famille libre est nécessairement fini et que le
cardinal d'une famille libre est inférieur au cardinal d'une famille
génératrice. Comme les bases sont à la fois libres et génératrices, on a
le théorème suivant~:

Théorème~2.2.6 Soit E un K-espace vectoriel de dimension finie. Alors
toutes les bases de E ont un même cardinal fini, appelé la dimension de
E. Toute famille libre est finie et a un cardinal inférieur ou égal à
\textbackslash{}mathop\{dim\} E avec égalité si et seulement si c'est
une base de E. Toute famille génératrice a un cardinal supérieur ou égal
à \textbackslash{}mathop\{dim\} E avec égalité si et seulement si c'est
une base de E.

\paragraph{2.2.3 Résultats sur la dimension}

Théorème~2.2.7 Deux espaces sont isomorphes si et seulement si ils ont
la même dimension.

Démonstration Deux espaces de même dimension sont isomorphes~: prendre
deux bases et envoyer l'une sur l'autre. Inversement, l'image d'une base
par un isomorphisme étant une base, deux espaces isomorphes ont même
dimension.

Théorème~2.2.8

\begin{itemize}
\itemsep1pt\parskip0pt\parsep0pt
\item
  \textbackslash{}mathop\{dim\} (E × F) =\textbackslash{}mathop\{ dim\}
  E +\textbackslash{}mathop\{ dim\} F
\item
  \textbackslash{}mathop\{dim\} (\{F\}\_\{1\}
  ⊕\textbackslash{}mathrel\{⋯\} ⊕ \{F\}\_\{k\})
  =\textbackslash{}mathop\{ \textbackslash{}mathop\{∑ \}\}
  \textbackslash{}mathop\{dim\} \{F\}\_\{i\}
\item
  \textbackslash{}mathop\{ dim\} (E∕F) =\textbackslash{}mathop\{ dim\} E
  −\textbackslash{}mathop\{ dim\} F
\item
  f : E → F linéaire, \textbackslash{}mathop\{dim\} E
  =\textbackslash{}mathop\{ dim\}
  \textbackslash{}mathop\{\textbackslash{}mathrm\{Ker\}\}f
  +\textbackslash{}mathop\{ dim\}
  \textbackslash{}mathop\{\textbackslash{}mathrm\{Im\}\}f (théorème du
  rang)
\item
  \textbackslash{}mathop\{dim\} (F + G) =\textbackslash{}mathop\{ dim\}
  F +\textbackslash{}mathop\{ dim\} G −\textbackslash{}mathop\{ dim\} F
  ∩ G
\item
  F ⊂ E,\textbackslash{}quad \textbackslash{}mathop\{dim\} F
  ≤\textbackslash{}mathop\{ dim\} E avec égalité si et seulement si F =
  E
\item
  \textbackslash{}mathop\{dim\} L(E,F) =\textbackslash{}mathop\{ dim\}
  E.\textbackslash{}mathop\{dim\} F
\end{itemize}

Démonstration

\begin{itemize}
\item
  Si \{(\{e\}\_\{i\})\}\_\{i∈I\} et \{(\{f\}\_\{j\})\}\_\{j∈J\} sont des
  bases respectives de E et F, on vérifie immédiatement que la famille
  \{\textbackslash{}left ((\{e\}\_\{i\},0)\textbackslash{}right
  )\}\_\{i∈I\} ∪\{\textbackslash{}left
  ((0,\{f\}\_\{j\})\textbackslash{}right )\}\_\{j∈J\}) est une base de E
  × F en écrivant (x,y) = (x,0) + (0,y).
\item
  \{F\}\_\{1\} ⊕\textbackslash{}mathrel\{⋯\} ⊕ \{F\}\_\{k\} est
  isomorphe à \{F\}\_\{1\} ×\textbackslash{}mathrel\{⋯\} × \{F\}\_\{k\}
  (prendre
  (\{x\}\_\{1\},\textbackslash{}mathop\{\textbackslash{}mathop\{\ldots{}\}\},\{x\}\_\{k\})\textbackslash{}mathrel\{↦\}\{x\}\_\{1\}
  + \textbackslash{}mathop\{\textbackslash{}mathop\{\ldots{}\}\} +
  \{x\}\_\{k\}) qui est de dimension
  \textbackslash{}mathop\{\textbackslash{}mathop\{∑ \}\}
  \textbackslash{}mathop\{dim\} \{F\}\_\{i\} par récurrence à partir du
  résultat précédent.
\item
  L'espace E∕F est isomorphe à tout supplémentaire de F dans E qui est
  de dimension \textbackslash{}mathop\{dim\} E −\textbackslash{}mathop\{
  dim\} F d'après le résultat précédent.
\item
  Il existe un isomorphisme \textbackslash{}overline\{f\} de
  E∕\textbackslash{}mathop\{\textbackslash{}mathrm\{Ker\}\}f sur
  \textbackslash{}mathop\{\textbackslash{}mathrm\{Im\}\}f, d'où
  \textbackslash{}mathop\{dim\}
  \textbackslash{}mathop\{\textbackslash{}mathrm\{Im\}\}f
  =\textbackslash{}mathop\{ dim\}
  E∕\textbackslash{}mathop\{\textbackslash{}mathrm\{Ker\}\}f
  =\textbackslash{}mathop\{ dim\} E −\textbackslash{}mathop\{ dim\}
  \textbackslash{}mathop\{\textbackslash{}mathrm\{Ker\}\}f.
\item
  L'application linéaire f : F × G → E,
  (x,y)\textbackslash{}mathrel\{↦\}x + y a pour image F + G et pour
  noyau \textbackslash{}\{(x,−x)\textbackslash{}mathrel\{∣\}x ∈ F ∩
  G\textbackslash{}\} qui est naturellement isomorphe à F ∩ G. Il suffit
  donc d'appliquer le résultat précédent à f.
\item
  D'après le théorème de la base incomplète, toute base de F est une
  famille libre de E, donc peut être complétée en une base de E, d'où
  \textbackslash{}mathop\{dim\} F ≤\textbackslash{}mathop\{ dim\} E avec
  égalité si et seulement si F = E
\item
  On peut soit utiliser le résultat analogue sur les matrices que l'on
  verra plus loin, soit montrer que si \{(\{e\}\_\{i\})\}\_\{i∈I\} et
  \{(\{f\}\_\{j\})\}\_\{j∈J\} sont des bases respectives de E et F,
  alors les applications linéaires \{φ\}\_\{j,i\} ∈ L(E,F) définies par

  \{ φ\}\_\{j,i\}(\{e\}\_\{k\}) = \textbackslash{}left
  \textbackslash{}\{\textbackslash{}cases\{ \{f\}\_\{j\}\&si k = i
  \textbackslash{}cr 0 \&si k\textbackslash{}mathrel\{≠\}i \}
  \textbackslash{}right .

  forment une base de L(E,F), ce qui est laissé en exercice.
\end{itemize}

Théorème~2.2.9 Soit F et G deux sous-espaces vectoriels de E de
dimension finie. On a équivalence de (i) F et G sont supplémentaires
dans E (ii) E = F + G et \textbackslash{}mathop\{dim\} E
=\textbackslash{}mathop\{ dim\} F +\textbackslash{}mathop\{ dim\} G
(iii) F ∩ G = \textbackslash{}\{0\textbackslash{}\} et
\textbackslash{}mathop\{dim\} E =\textbackslash{}mathop\{ dim\} F
+\textbackslash{}mathop\{ dim\} G

Démonstration Elémentaire.

Remarque~2.2.2 On sait que si F est un sous-espace vectoriel de E, la
restriction à tout supplémentaire G de F dans E de la projection
canonique de E sur E∕F est un isomorphisme d'espace vectoriel. Ceci
justifie la définition suivante~:

Définition~2.2.2 Soit E un K espace vectoriel et F un sous-espace
vectoriel de E. On appelle codimension de F la dimension (éventuellement
infinie) de l'espace vectoriel quotient E∕F, qui est encore la dimension
de tout supplémentaire de F dans E.

{[}\href{coursse9.html}{next}{]} {[}\href{coursse7.html}{prev}{]}
{[}\href{coursse7.html\#tailcoursse7.html}{prev-tail}{]}
{[}\href{coursse8.html}{front}{]}
{[}\href{coursch3.html\#coursse8.html}{up}{]}

\end{document}
