\section{Dualité : approche restreinte}

\subsection{Formes linéaires, dual, formes coordonnées}

\begin{de}
\index{forme linéaire}
\index{dual!espace vectoriel}
Soit $E$ un $K$-espace vectoriel. On appelle forme linéaire sur $E$ toute application linéaire de $E$ dans $K$. On appelle dual de $E$ le $K$-espace vectoriel $E^*=L(E,K)$.
\end{de}

\begin{rem}
\index{forme linéaire!coordonnée}
Soit $(e_i)_{i\in I}$ une base de $E$ et $i_0 \in I$. Tout vecteur $x$ de $E$ s'écrit de manière unique sous la forme $x = \sum_{i\in I} x_i e_i$. L'application $\phi_{i_0} : x \mapsto x_{i_0}$ est clairement une forme linéaire sur $E$, appelée forme linéaire coordonnée d'indice $i_0$ dans la base $(e_i)_{i\in I}$. Elle est définie par $\phi_{i_0}(e_{i_0}) = 1$ et $\phi_{i_0}(e_i) = 0$ si $i\neq i_0$, soit encore par $\phi_{i_0}(e_i) = \delta_{i_0}^i$.
\end{rem}

\begin{prop}
\index{forme linéaire!existence}
Soit $E$ un $K$-espace vectoriel et $x \in E$, $x\neq 0$. Alors il existe une forme linéaire $\phi$ sur $E$ telle que $\phi(x) = 1$.
\end{prop}

\begin{proof}
Le vecteur $x$ forme à lui tout seul une famille libre que l'on peut compléter en une base $(e_i)_{i\in I}$ de $E$ avec $x = e_{i_0}$. Soit $\phi$ la forme linéaire qui associe à tout vecteur de $E$ sa $i_0$-ième coordonnée dans cette base. On a bien entendu $\phi(x) = 1$.
\end{proof}

\begin{rem}
Le résultat précédent peut encore s'interpréter sous la forme : si $x \in E$,
\[ \forall \phi \in E^*, \phi(x) = 0 \quad \Leftrightarrow x = 0 \]
\end{rem}

\subsection{Base duale d'un espace vectoriel de dimension finie}

\begin{de}
\index{base!duale}
Soit $E$ un espace vectoriel de dimension finie, $\mathcal{E} = (e_1,\ldots,e_n)$ une base de $E$. Pour $i \in [1,n]$, soit $\phi_i$ la forme linéaire coordonnée d'indice $i$ dans la base $\mathcal{E}$. Alors $\mathcal{E}^* = (\phi_1,\ldots,\phi_n)$ est une base du dual $E^*$, appelée la base duale de la base $\mathcal{E}$. Elle est caractérisée par les relations $\forall i,j \in [1,n], \phi_i(e_j) = \delta_i^j$.
\end{de}

\begin{proof}
Tout d'abord, montrons que $\mathcal{E}^*$ est une famille libre en utilisant les relations de définition des formes coordonnées $\forall i,j \in [1,n], \phi_i(e_j) = \delta_i^j$

Soit $\lambda_1,\ldots,\lambda_n \in K$ tels que $\lambda_1\phi_1 + \ldots + \lambda_n\phi_n = 0$; on a alors, pour tout $i \in [1,n]$
\[ 0 = 0(e_i) = \lambda_1\phi_1(e_i) + \ldots + \lambda_n\phi_n(e_i) = \lambda_i \]
ce qui montre bien que la famille est libre. Pour montrer qu'elle est génératrice, soit $\phi \in E$ et considérons $\psi = \sum_{i=1}^n \phi(e_i)\phi_i$; on a alors pour tout $j \in [1,n]$,
\[ \psi(e_j) = \sum_{i=1}^n \phi(e_i)\phi_i(e_j) = \sum_{i=1}^n \phi(e_i)\delta_i^j = \phi(e_j) \]

Les deux applications linéaires $\phi$ et $\psi$ coïncidant sur une base, sont égales, ce qui montre que la famille est génératrice.
\end{proof}

\begin{rem}
\index{dimension!infinie}
Attention : la dimension finie est essentielle; elle garantit qu'il n'y a qu'un nombre fini de $\phi(e_i)$ non nuls et permet de considérer la somme $\sum_{i\in I} \phi(e_i)\phi_i$; en dimension infinie, $\mathcal{E}^*$ n'est pas une base de $E^*$, car elle n'est pas génératrice (considérer la forme linéaire $\phi$ qui à tout $e_i$ associe $1$).
\end{rem}

\begin{cor}
\index{dual!dimension}
La dimension de l'espace dual d'un espace vectoriel de dimension finie est égale à la dimension de l'espace.
\end{cor}

\subsection{Orthogonalité 1}

\begin{prop}
\index{orthogonalité!famille libre}
Soit $E$ un $K$-espace vectoriel de dimension finie, $(e_1,\ldots,e_p)$ une famille d'éléments de $E$. Nous pouvons associer à cette famille l'application $u : E^* \to K^p$, définie par $u(\phi) = (\phi(e_1),\ldots,\phi(e_p))$. Si $(e_1,\ldots,e_p)$ est une base de $E$, alors $u$ est un isomorphisme d'espace vectoriel de $E^*$ sur $K^p$.
\end{prop}

\begin{prop}
\index{orthogonalité!noyau}
La famille $(e_1,\ldots,e_p)$ est libre si et seulement si $u$ est surjective. Sous ces conditions, $\operatorname{Ker}(u)$ est de codimension $p$ et
\[ \forall x \in E, \quad (x \in \operatorname{Vect}(e_1,\ldots,e_p) \Leftrightarrow \forall \phi \in \operatorname{Ker}(u), \phi(x) = 0) \]
\end{prop}

\begin{rem}
Application : Soit $F$ un sous-espace vectoriel de $E$, $(e_1,\ldots,e_p)$ une base de $F$, $u : E^* \to K^p$ l'application linéaire associée, $(\phi_1,\ldots,\phi_{n-p})$ une base de $\operatorname{Ker}(u)$; alors $x \in F \Leftrightarrow \forall i \in [1,n-p], \phi_i(x) = 0$. On dit encore que $F$ est défini par le système d'équations linéaires $\phi_1(x) = 0,\ldots,\phi_{n-p}(x) = 0$.
\end{rem}

\subsection{Hyperplans}

\begin{de}
\index{hyperplan}
On appelle hyperplan de $E$ tout sous-espace vectoriel $H$ de $E$ vérifiant les conditions équivalentes :
\begin{enumerate}
\item $\dim E/H = 1$
\item $\exists f \in E^* \setminus \{0\}, H = \operatorname{Ker}(f)$
\item $H$ admet une droite comme supplémentaire
\end{enumerate}
\end{de}

\begin{proof}
\begin{enumerate}
\item (i) $\Rightarrow$ (ii) : prendre $\overline{e}$ une base de $E/H$ et écrire $\pi(x) = f(x)\overline{e}$.
\item (ii) $\Rightarrow$ (iii) : on prend $a \in E$ tel que $f(a)\neq 0$. Tout élément $x$ s'écrit de manière unique sous la forme $x = (x - \frac{f(x)}{f(a)}a) + \frac{f(x)}{f(a)}a$ avec $x - \frac{f(x)}{f(a)}a \in \operatorname{Ker}(f)$, soit $E = \operatorname{Ker}(f) \oplus Ka$.
\item (iii) $\Rightarrow$ (i) : tout supplémentaire de $H$ est isomorphe à $E/H$.
\end{enumerate}
\end{proof}

\begin{thm}
\index{hyperplan!formes linéaires}
Soit $H$ un hyperplan de $E$. Alors deux formes linéaires nulles sur $H$ sont proportionnelles.
\end{thm}

\begin{proof}
Si $E = H \oplus Ka$ et $H = \operatorname{Ker}(f)$, soit $g \in E^*$ nulle sur $H$. Alors $g$ et $\frac{g(a)}{f(a)}f$ coïncident sur $H$ et sur $Ka$, donc sont égales.
\end{proof}

\subsection{Orthogonalité 2}

\begin{rem}
Soit $E$ un $K$-espace vectoriel de dimension finie, $(\phi_1,\ldots,\phi_p)$ une famille d'éléments de $E^*$. Nous pouvons associer à cette famille l'application $v : E \to K^p$, définie par $v(x) = (\phi_1(x),\ldots,\phi_p(x))$. Son noyau est constitué de l'intersection des $\operatorname{Ker}(\phi_i)$ (en général des hyperplans, sauf si la forme linéaire est nulle).
\end{rem}

\begin{prop}
\index{base!duale!caractérisation}
Si $(\phi_1,\ldots,\phi_p)$ est une base de $E^*$, alors $v$ est un isomorphisme d'espace vectoriel de $E$ sur $K^p$.
\end{prop}

\begin{thm}
\index{base!duale!unicité}
Soit $(\phi_1,\ldots,\phi_p)$ une base de $E^*$; alors il existe une unique base $(e_1,\ldots,e_p)$ de $E$ dont $(\phi_1,\ldots,\phi_p)$ soit la base duale.
\end{thm}

\begin{prop}
\index{orthogonalité!formes linéaires}
La famille $(\phi_1,\ldots,\phi_p)$ est libre si et seulement si $v$ est surjective. Sous ces conditions, $\operatorname{Ker}(v)$ est de codimension $p$ et $\forall \phi \in E^*$,
\[ (\phi \in \operatorname{Vect}(\phi_1,\ldots,\phi_p) \Leftrightarrow \forall x \in \operatorname{Ker}(v), \phi(x) = 0) \]
\end{prop}

\subsection{Application : polynômes d'interpolation de Lagrange}

\begin{thm}
\index{polynômes!interpolation de Lagrange}
Soit $K$ un corps commutatif, $x_1,\ldots,x_n \in K$ distincts. Soit $a_1,\ldots,a_n \in K$. Alors il existe un unique polynôme $P \in K[X]$ tel que $\deg P \leq n-1$ et $\forall i \in [1,n], P(x_i) = a_i$.
\end{thm}

\begin{rem}
\index{polynômes!Lagrange}
Les polynômes de Lagrange sont donnés par :
\[ P_i(X) = \prod_{j\neq i}\frac{X - x_j}{x_i - x_j} \]
et le polynôme d'interpolation s'écrit :
\[ P = \sum_{i=1}^n P(x_i)P_i = \sum_{i=1}^n a_i\prod_{j\neq i}\frac{X - x_j}{x_i - x_j} \]
\end{rem}