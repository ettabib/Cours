\documentclass[]{article}
\usepackage[T1]{fontenc}
\usepackage{lmodern}
\usepackage{amssymb,amsmath}
\usepackage{ifxetex,ifluatex}
\usepackage{fixltx2e} % provides \textsubscript
% use upquote if available, for straight quotes in verbatim environments
\IfFileExists{upquote.sty}{\usepackage{upquote}}{}
\ifnum 0\ifxetex 1\fi\ifluatex 1\fi=0 % if pdftex
  \usepackage[utf8]{inputenc}
\else % if luatex or xelatex
  \ifxetex
    \usepackage{mathspec}
    \usepackage{xltxtra,xunicode}
  \else
    \usepackage{fontspec}
  \fi
  \defaultfontfeatures{Mapping=tex-text,Scale=MatchLowercase}
  \newcommand{\euro}{€}
\fi
% use microtype if available
\IfFileExists{microtype.sty}{\usepackage{microtype}}{}
\ifxetex
  \usepackage[setpagesize=false, % page size defined by xetex
              unicode=false, % unicode breaks when used with xetex
              xetex]{hyperref}
\else
  \usepackage[unicode=true]{hyperref}
\fi
\hypersetup{breaklinks=true,
            bookmarks=true,
            pdfauthor={},
            pdftitle={Dualite : approche restreinte},
            colorlinks=true,
            citecolor=blue,
            urlcolor=blue,
            linkcolor=magenta,
            pdfborder={0 0 0}}
\urlstyle{same}  % don't use monospace font for urls
\setlength{\parindent}{0pt}
\setlength{\parskip}{6pt plus 2pt minus 1pt}
\setlength{\emergencystretch}{3em}  % prevent overfull lines
\setcounter{secnumdepth}{0}
 
/* start css.sty */
.cmr-5{font-size:50%;}
.cmr-7{font-size:70%;}
.cmmi-5{font-size:50%;font-style: italic;}
.cmmi-7{font-size:70%;font-style: italic;}
.cmmi-10{font-style: italic;}
.cmsy-5{font-size:50%;}
.cmsy-7{font-size:70%;}
.cmex-7{font-size:70%;}
.cmex-7x-x-71{font-size:49%;}
.msbm-7{font-size:70%;}
.cmtt-10{font-family: monospace;}
.cmti-10{ font-style: italic;}
.cmbx-10{ font-weight: bold;}
.cmr-17x-x-120{font-size:204%;}
.cmsl-10{font-style: oblique;}
.cmti-7x-x-71{font-size:49%; font-style: italic;}
.cmbxti-10{ font-weight: bold; font-style: italic;}
p.noindent { text-indent: 0em }
td p.noindent { text-indent: 0em; margin-top:0em; }
p.nopar { text-indent: 0em; }
p.indent{ text-indent: 1.5em }
@media print {div.crosslinks {visibility:hidden;}}
a img { border-top: 0; border-left: 0; border-right: 0; }
center { margin-top:1em; margin-bottom:1em; }
td center { margin-top:0em; margin-bottom:0em; }
.Canvas { position:relative; }
li p.indent { text-indent: 0em }
.enumerate1 {list-style-type:decimal;}
.enumerate2 {list-style-type:lower-alpha;}
.enumerate3 {list-style-type:lower-roman;}
.enumerate4 {list-style-type:upper-alpha;}
div.newtheorem { margin-bottom: 2em; margin-top: 2em;}
.obeylines-h,.obeylines-v {white-space: nowrap; }
div.obeylines-v p { margin-top:0; margin-bottom:0; }
.overline{ text-decoration:overline; }
.overline img{ border-top: 1px solid black; }
td.displaylines {text-align:center; white-space:nowrap;}
.centerline {text-align:center;}
.rightline {text-align:right;}
div.verbatim {font-family: monospace; white-space: nowrap; text-align:left; clear:both; }
.fbox {padding-left:3.0pt; padding-right:3.0pt; text-indent:0pt; border:solid black 0.4pt; }
div.fbox {display:table}
div.center div.fbox {text-align:center; clear:both; padding-left:3.0pt; padding-right:3.0pt; text-indent:0pt; border:solid black 0.4pt; }
div.minipage{width:100%;}
div.center, div.center div.center {text-align: center; margin-left:1em; margin-right:1em;}
div.center div {text-align: left;}
div.flushright, div.flushright div.flushright {text-align: right;}
div.flushright div {text-align: left;}
div.flushleft {text-align: left;}
.underline{ text-decoration:underline; }
.underline img{ border-bottom: 1px solid black; margin-bottom:1pt; }
.framebox-c, .framebox-l, .framebox-r { padding-left:3.0pt; padding-right:3.0pt; text-indent:0pt; border:solid black 0.4pt; }
.framebox-c {text-align:center;}
.framebox-l {text-align:left;}
.framebox-r {text-align:right;}
span.thank-mark{ vertical-align: super }
span.footnote-mark sup.textsuperscript, span.footnote-mark a sup.textsuperscript{ font-size:80%; }
div.tabular, div.center div.tabular {text-align: center; margin-top:0.5em; margin-bottom:0.5em; }
table.tabular td p{margin-top:0em;}
table.tabular {margin-left: auto; margin-right: auto;}
div.td00{ margin-left:0pt; margin-right:0pt; }
div.td01{ margin-left:0pt; margin-right:5pt; }
div.td10{ margin-left:5pt; margin-right:0pt; }
div.td11{ margin-left:5pt; margin-right:5pt; }
table[rules] {border-left:solid black 0.4pt; border-right:solid black 0.4pt; }
td.td00{ padding-left:0pt; padding-right:0pt; }
td.td01{ padding-left:0pt; padding-right:5pt; }
td.td10{ padding-left:5pt; padding-right:0pt; }
td.td11{ padding-left:5pt; padding-right:5pt; }
table[rules] {border-left:solid black 0.4pt; border-right:solid black 0.4pt; }
.hline hr, .cline hr{ height : 1px; margin:0px; }
.tabbing-right {text-align:right;}
span.TEX {letter-spacing: -0.125em; }
span.TEX span.E{ position:relative;top:0.5ex;left:-0.0417em;}
a span.TEX span.E {text-decoration: none; }
span.LATEX span.A{ position:relative; top:-0.5ex; left:-0.4em; font-size:85%;}
span.LATEX span.TEX{ position:relative; left: -0.4em; }
div.float img, div.float .caption {text-align:center;}
div.figure img, div.figure .caption {text-align:center;}
.marginpar {width:20%; float:right; text-align:left; margin-left:auto; margin-top:0.5em; font-size:85%; text-decoration:underline;}
.marginpar p{margin-top:0.4em; margin-bottom:0.4em;}
.equation td{text-align:center; vertical-align:middle; }
td.eq-no{ width:5%; }
table.equation { width:100%; } 
div.math-display, div.par-math-display{text-align:center;}
math .texttt { font-family: monospace; }
math .textit { font-style: italic; }
math .textsl { font-style: oblique; }
math .textsf { font-family: sans-serif; }
math .textbf { font-weight: bold; }
.partToc a, .partToc, .likepartToc a, .likepartToc {line-height: 200%; font-weight:bold; font-size:110%;}
.chapterToc a, .chapterToc, .likechapterToc a, .likechapterToc, .appendixToc a, .appendixToc {line-height: 200%; font-weight:bold;}
.index-item, .index-subitem, .index-subsubitem {display:block}
.caption td.id{font-weight: bold; white-space: nowrap; }
table.caption {text-align:center;}
h1.partHead{text-align: center}
p.bibitem { text-indent: -2em; margin-left: 2em; margin-top:0.6em; margin-bottom:0.6em; }
p.bibitem-p { text-indent: 0em; margin-left: 2em; margin-top:0.6em; margin-bottom:0.6em; }
.paragraphHead, .likeparagraphHead { margin-top:2em; font-weight: bold;}
.subparagraphHead, .likesubparagraphHead { font-weight: bold;}
.quote {margin-bottom:0.25em; margin-top:0.25em; margin-left:1em; margin-right:1em; text-align:justify;}
.verse{white-space:nowrap; margin-left:2em}
div.maketitle {text-align:center;}
h2.titleHead{text-align:center;}
div.maketitle{ margin-bottom: 2em; }
div.author, div.date {text-align:center;}
div.thanks{text-align:left; margin-left:10%; font-size:85%; font-style:italic; }
div.author{white-space: nowrap;}
.quotation {margin-bottom:0.25em; margin-top:0.25em; margin-left:1em; }
h1.partHead{text-align: center}
.sectionToc, .likesectionToc {margin-left:2em;}
.subsectionToc, .likesubsectionToc {margin-left:4em;}
.subsubsectionToc, .likesubsubsectionToc {margin-left:6em;}
.frenchb-nbsp{font-size:75%;}
.frenchb-thinspace{font-size:75%;}
.figure img.graphics {margin-left:10%;}
/* end css.sty */

\title{Dualite : approche restreinte}
\author{}
\date{}

\begin{document}
\maketitle

\textbf{Warning: \href{http://www.math.union.edu/locate/jsMath}{jsMath}
requires JavaScript to process the mathematics on this page.\\ If your
browser supports JavaScript, be sure it is enabled.}

\begin{center}\rule{3in}{0.4pt}\end{center}

{[}\href{coursse11.html}{next}{]} {[}\href{coursse9.html}{prev}{]}
{[}\href{coursse9.html\#tailcoursse9.html}{prev-tail}{]}
{[}\hyperref[tailcoursse10.html]{tail}{]}
{[}\href{coursch3.html\#coursse10.html}{up}{]}

\subsubsection{2.4 Dualité~: approche restreinte}

\paragraph{2.4.1 Formes linéaires, dual, formes coordonnées}

Définition~2.4.1 Soit E un K-espace vectoriel . On appelle forme
linéaire sur E toute application linéaire de E dans K. On appelle dual
de E le K-espace vectoriel \{E\}\^{}\{∗\} = L(E,K).

Remarque~2.4.1 Soit \{(\{e\}\_\{i\})\}\_\{i∈I\} une base de E et
\{i\}\_\{0\} ∈ I. Tout vecteur x de E s'écrit de manière unique sous la
forme x =\{\textbackslash{}mathop\{ \textbackslash{}mathop\{∑ \}\}
\}\_\{i∈I\}\{x\}\_\{i\}\{e\}\_\{i\}. L'application
\{φ\}\_\{\{i\}\_\{0\}\} :
x\textbackslash{}mathrel\{↦\}\{x\}\_\{\{i\}\_\{0\}\} est clairement une
forme linéaire sur E, appelée forme linéaire coordonnée d'indice
\{i\}\_\{0\} dans la base \{(\{e\}\_\{i\})\}\_\{i∈I\}. Elle est définie
par \{φ\}\_\{\{i\}\_\{0\}\}(\{e\}\_\{\{i\}\_\{0\}\}) = 1 et
\{φ\}\_\{\{i\}\_\{0\}\}(\{e\}\_\{i\}) = 0 si
i\textbackslash{}mathrel\{≠\}\{i\}\_\{0\}, soit encore par
\{φ\}\_\{\{i\}\_\{0\}\}(\{e\}\_\{i\}) =
\{δ\}\_\{\{i\}\_\{0\}\}\^{}\{i\}.

Proposition~2.4.1 Soit E un K-espace vectoriel et x ∈ E,
x\textbackslash{}mathrel\{≠\}0. Alors il existe une forme linéaire φ sur
E telle que φ(x) = 1.

Démonstration Le vecteur x forme à lui tout seul une famille libre que
l'on peut compléter en une base \{(\{e\}\_\{i\})\}\_\{i∈I\} de E avec x
= \{e\}\_\{\{i\}\_\{0\}\}. Soit φ la forme linéaire qui associe à tout
vecteur de E sa \{i\}\_\{0\}-ième coordonnée dans cette base. On a bien
entendu φ(x) = 1.

Remarque~2.4.2 Le résultat précédent peut encore s'interpréter sous la
forme~: si x ∈ E,

\textbackslash{}mathop\{∀\}φ ∈ \{E\}\^{}\{∗\}, φ(x) =
0\textbackslash{}quad \textbackslash{}mathrel\{⇔\} x = 0

\paragraph{2.4.2 Base duale d'un espace vectoriel de dimension finie}

Définition~2.4.2 Soit E un espace vectoriel de dimension finie, ℰ =
(\{e\}\_\{1\},\textbackslash{}mathop\{\textbackslash{}mathop\{\ldots{}\}\},\{e\}\_\{n\})
une base de E. Pour i ∈ {[}1,n{]}, soit \{φ\}\_\{i\} la forme linéaire
coordonnée d'indice i dans la base ℰ. Alors \{ℰ\}\^{}\{∗\} =
(\{φ\}\_\{1\},\textbackslash{}mathop\{\textbackslash{}mathop\{\ldots{}\}\},\{φ\}\_\{n\})
est une base du dual \{E\}\^{}\{∗\}, appelée la base duale de la base ℰ.
Elle est caractérisée par les relations \textbackslash{}mathop\{∀\}i,j ∈
{[}1,n{]}, \{φ\}\_\{i\}(\{e\}\_\{j\}) = \{δ\}\_\{i\}\^{}\{j\}.

Démonstration Tout d'abord, montrons que \{ℰ\}\^{}\{∗\} est une famille
libre en utilisant les relations de définition des formes coordonnées

\textbackslash{}mathop\{∀\}i,j ∈ {[}1,n{]}, \{φ\}\_\{i\}(\{e\}\_\{j\}) =
\{δ\}\_\{i\}\^{}\{j\}

Soit
\{λ\}\_\{1\},\textbackslash{}mathop\{\textbackslash{}mathop\{\ldots{}\}\},\{λ\}\_\{n\}
∈ K tels que \{λ\}\_\{1\}\{φ\}\_\{1\} +
\textbackslash{}mathop\{\textbackslash{}mathop\{\ldots{}\}\} +
\{λ\}\_\{n\}\{φ\}\_\{n\} = 0~; on a alors, pout tout i ∈ {[}1,n{]}

0 = 0(\{e\}\_\{i\}) = \{λ\}\_\{1\}\{φ\}\_\{1\}(\{e\}\_\{i\}) +
\textbackslash{}mathop\{\textbackslash{}mathop\{\ldots{}\}\} +
\{λ\}\_\{n\}\{φ\}\_\{n\}(\{e\}\_\{i\}) = \{λ\}\_\{i\}

ce qui montre bien que la famille est libre. Pour montrer qu'elle est
génératrice, soit φ ∈ E et considérons ψ =\{\textbackslash{}mathop\{
\textbackslash{}mathop\{∑ \}\}
\}\_\{i=1\}\^{}\{n\}φ(\{e\}\_\{i\})\{φ\}\_\{i\}~; on a alors pour tout j
∈ {[}1,n{]},

ψ(\{e\}\_\{j\}) =\{ \textbackslash{}mathop\{∑
\}\}\_\{i=1\}\^{}\{n\}φ(\{e\}\_\{ i\})\{φ\}\_\{i\}(\{e\}\_\{j\}) =\{
\textbackslash{}mathop\{∑ \}\}\_\{i=1\}\^{}\{n\}φ(\{e\}\_\{
i\})\{δ\}\_\{i\}\^{}\{j\} = φ(\{e\}\_\{ j\})

Les deux applications linéaires φ et ψ coïncidant sur une base, sont
égales, ce qui montre que la famille est génératrice.

Remarque~2.4.3 Attention~: la dimension finie est essentielle~; elle
garantit qu'il n'y a qu'un nombre fini de φ(\{e\}\_\{i\}) non nuls et
permet de considérer la somme
\{\textbackslash{}mathop\{\textbackslash{}mathop\{∑ \}\}
\}\_\{i∈I\}φ(\{e\}\_\{i\})\{φ\}\_\{i\}~; en dimension infinie,
\{ℰ\}\^{}\{∗\} n'est pas une base de \{E\}\^{}\{∗\}, car elle n'est pas
génératrice (considérer la forme linéaire φ qui à tout \{e\}\_\{i\}
associe 1).

Corollaire~2.4.2 La dimension de l'espace dual d'un espace vectoriel de
dimension finie est égale à la dimension de l'espace.

\paragraph{2.4.3 Orthogonalité 1}

Soit E un K-espace vectoriel de dimension finie,
(\{e\}\_\{1\},\textbackslash{}mathop\{\textbackslash{}mathop\{\ldots{}\}\},\{e\}\_\{p\})
une famille d'éléments de E. Nous pouvons associer à cette famille
l'application u : \{E\}\^{}\{∗\}→ \{K\}\^{}\{p\}, définie par u(φ) =
(φ(\{e\}\_\{1\}),\textbackslash{}mathop\{\textbackslash{}mathop\{\ldots{}\}\},φ(\{e\}\_\{p\})).
On vérifie immédiatement que u est linéaire. Son noyau est constitué des
φ ∈ \{E\}\^{}\{∗\} vérifiant \textbackslash{}mathop\{∀\}i ∈ {[}1,p{]},
φ(\{e\}\_\{i\}) = 0.

Proposition~2.4.3 Si
(\{e\}\_\{1\},\textbackslash{}mathop\{\textbackslash{}mathop\{\ldots{}\}\},\{e\}\_\{p\})
est une base de E, alors u est un isomorphisme d'espace vectoriel de
\{E\}\^{}\{∗\} sur \{K\}\^{}\{p\}.

Démonstration En effet dans ce cas, u envoie la base duale
\{ℰ\}\^{}\{∗\} sur la base canonique de \{K\}\^{}\{p\}~; c'est donc un
isomorphisme.

Proposition~2.4.4 La famille
(\{e\}\_\{1\},\textbackslash{}mathop\{\textbackslash{}mathop\{\ldots{}\}\},\{e\}\_\{p\})
est libre si et seulement si u est surjective. Sous ces conditions,
\textbackslash{}mathop\{\textbackslash{}mathrm\{Ker\}\}u est de
codimension p et

\textbackslash{}mathop\{∀\}x ∈ E,\textbackslash{}quad (x
∈\textbackslash{}mathop\{\textbackslash{}mathrm\{Vect\}\}(\{e\}\_\{1\},\textbackslash{}mathop\{\textbackslash{}mathop\{\ldots{}\}\},\{e\}\_\{p\})
\textbackslash{}mathrel\{⇔\} \textbackslash{}mathop\{∀\}φ
∈\textbackslash{}mathop\{\textbackslash{}mathrm\{Ker\}\}u, φ(x) = 0)

Démonstration Si u est surjective, notons
(\{ε\}\_\{1\},\textbackslash{}mathop\{\textbackslash{}mathop\{\ldots{}\}\},\{ε\}\_\{p\})
la base canonique de \{K\}\^{}\{p\} et soit \{φ\}\_\{i\} ∈
\{E\}\^{}\{∗\} tel que u(\{φ\}\_\{i\}) = \{ε\}\_\{i\}~; on a donc
\textbackslash{}mathop\{∀\}i,j ∈ {[}1,p{]}, \{φ\}\_\{i\}(\{e\}\_\{j\}) =
\{δ\}\_\{i\}\^{}\{j\} ce qui implique évidemment que la famille est
libre~: si \{\textbackslash{}mathop\{\textbackslash{}mathop\{∑ \}\}
\}\_\{j=1\}\^{}\{p\}\{λ\}\_\{j\}\{e\}\_\{j\} = 0, on a pour tout i ∈
{[}1,p{]}

0 = \{φ\}\_\{i\}(0) = \{φ\}\_\{i\}(\{\textbackslash{}mathop\{∑
\}\}\_\{j=1\}\^{}\{p\}\{λ\}\_\{ j\}\{e\}\_\{j\}) =\{
\textbackslash{}mathop\{∑ \}\}\_\{j=1\}\^{}\{p\}\{λ\}\_\{
j\}\{φ\}\_\{i\}(\{e\}\_\{j\}) = \{λ\}\_\{i\}

Inversement, si la famille est libre, on peut compléter la famille
(\{e\}\_\{1\},\textbackslash{}mathop\{\textbackslash{}mathop\{\ldots{}\}\},\{e\}\_\{p\})
en une base
(\{e\}\_\{1\},\textbackslash{}mathop\{\textbackslash{}mathop\{\ldots{}\}\},\{e\}\_\{n\})
de E et soit
(\{φ\}\_\{1\},\textbackslash{}mathop\{\textbackslash{}mathop\{\ldots{}\}\},\{φ\}\_\{n\})
la base duale. On a alors \textbackslash{}mathop\{∀\}i,j ∈ {[}1,p{]},
\{φ\}\_\{i\}(\{e\}\_\{j\}) = \{δ\}\_\{i\}\^{}\{j\}, soit
\textbackslash{}mathop\{∀\}i ∈ {[}1,p{]}, u(\{φ\}\_\{i\}) =
\{ε\}\_\{i\}. L'image de u contient une base de \{K\}\^{}\{p\}, c'est
donc \{K\}\^{}\{p\} et u est surjective. Dans ces conditions, on peut
appliquer le théorème du rang, et donc \textbackslash{}mathop\{dim\}
\textbackslash{}mathop\{\textbackslash{}mathrm\{Ker\}\}u
=\textbackslash{}mathop\{ dim\} \{E\}\^{}\{∗\}− p
=\textbackslash{}mathop\{ dim\} E − p.

Soit φ ∈\textbackslash{}mathop\{\textbackslash{}mathrm\{Ker\}\}u~; alors
\textbackslash{}mathop\{∀\}i ∈ {[}1,p{]}, φ(\{e\}\_\{i\}) = 0 et donc
\textbackslash{}mathop\{∀\}x
∈\textbackslash{}mathop\{\textbackslash{}mathrm\{Vect\}\}(\{e\}\_\{1\},\textbackslash{}mathop\{\textbackslash{}mathop\{\ldots{}\}\},\{e\}\_\{p\}),
φ(x) = 0. Inversement, supposons que
x\textbackslash{}mathrel\{∉\}\textbackslash{}mathop\{\textbackslash{}mathrm\{Vect\}\}(\{e\}\_\{1\},\textbackslash{}mathop\{\textbackslash{}mathop\{\ldots{}\}\},\{e\}\_\{p\})~;
alors la famille
(\{e\}\_\{1\},\textbackslash{}mathop\{\textbackslash{}mathop\{\ldots{}\}\},\{e\}\_\{p\},x)
est libre, on peut la compléter en une base de E et la forme coordonnée
suivant x dans cette base, soit φ, appartient à
\textbackslash{}mathop\{\textbackslash{}mathrm\{Ker\}\}u alors que φ(x)
= 1. On a donc bien l'équivalence

\textbackslash{}mathop\{∀\}x ∈ E,\textbackslash{}quad (x
∈\textbackslash{}mathop\{\textbackslash{}mathrm\{Vect\}\}(\{e\}\_\{1\},\textbackslash{}mathop\{\textbackslash{}mathop\{\ldots{}\}\},\{e\}\_\{p\})
\textbackslash{}mathrel\{⇔\} \textbackslash{}mathop\{∀\}φ
∈\textbackslash{}mathop\{\textbackslash{}mathrm\{Ker\}\}u, φ(x) = 0)

Remarque~2.4.4 Application~: Soit F un sous-espace vectoriel de E,
(\{e\}\_\{1\},\textbackslash{}mathop\{\textbackslash{}mathop\{\ldots{}\}\},\{e\}\_\{p\})
une base de F, u : \{E\}\^{}\{∗\}→ \{K\}\^{}\{p\} l'application linéaire
associée,
(\{φ\}\_\{1\},\textbackslash{}mathop\{\textbackslash{}mathop\{\ldots{}\}\},\{φ\}\_\{n−p\})
une base de \textbackslash{}mathop\{\textbackslash{}mathrm\{Ker\}\}u~;
alors x ∈ F \textbackslash{}mathrel\{⇔\} \textbackslash{}mathop\{∀\}i ∈
{[}1,n − p{]}, \{φ\}\_\{i\}(x) = 0. On dit encore que F est défini par
le système d'équations linéaires \{φ\}\_\{1\}(x) =
0,\textbackslash{}mathop\{\textbackslash{}mathop\{\ldots{}\}\},\{φ\}\_\{n−p\}(x)
= 0.

\paragraph{2.4.4 Hyperplans}

Définition~2.4.3 On appelle hyperplan de E tout sous-espace vectoriel H
de E vérifiant les conditions équivalentes

\begin{itemize}
\itemsep1pt\parskip0pt\parsep0pt
\item
  (i) \textbackslash{}mathop\{dim\} E∕H = 1
\item
  (ii) \textbackslash{}mathop\{∃\}f ∈ E
  ∖\textbackslash{}\{0\textbackslash{}\}, H =\textbackslash{}mathop\{
  \textbackslash{}mathrm\{Ker\}\}f
\item
  (iii) H admet une droite comme supplémentaire.
\end{itemize}

Démonstration

\begin{itemize}
\itemsep1pt\parskip0pt\parsep0pt
\item
  (i) ⇒(ii)~: prendre \textbackslash{}overline\{e\} une base de E∕H et
  écrire π(x) = f(x)\textbackslash{}overline\{e\}.
\item
  (ii) ⇒ (iii)~: on prend a ∈ E tel que
  f(a)\textbackslash{}mathrel\{≠\}0. Tout élément x s'écrit de manière
  unique sous la forme x = (x −\{ f(x) \textbackslash{}over f(a)\} a)
  +\{ f(x) \textbackslash{}over f(a)\} a avec x −\{ f(x)
  \textbackslash{}over f(a)\} a
  ∈\textbackslash{}mathop\{\textbackslash{}mathrm\{Ker\}\}f, soit E
  =\textbackslash{}mathop\{ \textbackslash{}mathrm\{Ker\}\}f ⊕ Ka.
\item
  (iii) ⇒(i)~: tout supplémentaire de H est isomorphe à E∕H.
\end{itemize}

Théorème~2.4.5 Soit H un hyperplan de E. Alors deux formes linéaires
nulles sur H sont proportionnelles.

Démonstration Si E = H ⊕ Ka et H =\textbackslash{}mathop\{
\textbackslash{}mathrm\{Ker\}\}f, soit g ∈ \{E\}\^{}\{∗\} nulle sur H.
Alors g et \{ g(a) \textbackslash{}over f(a)\} f coïncident sur H et sur
Ka, donc sont égales.

\paragraph{2.4.5 Orthogonalité 2}

Remarque~2.4.5 Soit E un K-espace vectoriel de dimension finie,
(\{φ\}\_\{1\},\textbackslash{}mathop\{\textbackslash{}mathop\{\ldots{}\}\},\{φ\}\_\{p\})
une famille d'éléments de \{E\}\^{}\{∗\}. Nous pouvons associer à cette
famille l'application v : E → \{K\}\^{}\{p\}, définie par v(x) =
(\{φ\}\_\{1\}(x),\textbackslash{}mathop\{\textbackslash{}mathop\{\ldots{}\}\},\{φ\}\_\{p\}(x)).
On vérifie immédiatement que v est linéaire. Son noyau est constitué de
l'intersection des
\textbackslash{}mathop\{\textbackslash{}mathrm\{Ker\}\}\{φ\}\_\{i\} (en
général des hyperplans, sauf si la forme linéaire est nulle).

Proposition~2.4.6 Si
(\{φ\}\_\{1\},\textbackslash{}mathop\{\textbackslash{}mathop\{\ldots{}\}\},\{φ\}\_\{p\})
est une base de \{E\}\^{}\{∗\}, alors v est un isomorphisme d'espace
vectoriel de \{E\}\^{}\{∗\} sur \{K\}\^{}\{p\}.

Démonstration En effet dans ce cas, v est injective car

\textbackslash{}begin\{eqnarray*\} v(x) = 0\&
\textbackslash{}mathrel\{⇔\} \& \textbackslash{}mathop\{∀\}i ∈
{[}1,p{]}, \{φ\}\_\{i\}(x) = 0\%\& \textbackslash{}\textbackslash{} \&
\textbackslash{}mathrel\{⇔\} \& \textbackslash{}mathop\{∀\}φ ∈
\{E\}\^{}\{∗\}, φ(x) = 0 \%\& \textbackslash{}\textbackslash{} \&
\textbackslash{}mathrel\{⇔\} \& x = 0 \%\&
\textbackslash{}\textbackslash{} \textbackslash{}end\{eqnarray*\}

Comme \textbackslash{}mathop\{dim\} E =\textbackslash{}mathop\{ dim\}
\{E\}\^{}\{∗\} = p =\textbackslash{}mathop\{ dim\} \{K\}\^{}\{p\}, il
s'agit d'un isomorphisme.

Théorème~2.4.7 Soit
(\{φ\}\_\{1\},\textbackslash{}mathop\{\textbackslash{}mathop\{\ldots{}\}\},\{φ\}\_\{p\})
une base de \{E\}\^{}\{∗\}~; alors il existe une unique base
(\{e\}\_\{1\},\textbackslash{}mathop\{\textbackslash{}mathop\{\ldots{}\}\},\{e\}\_\{p\})
de E dont
(\{φ\}\_\{1\},\textbackslash{}mathop\{\textbackslash{}mathop\{\ldots{}\}\},\{φ\}\_\{p\})
soit la base duale.

Démonstration On a en effet \textbackslash{}mathop\{∀\}i ∈
{[}1,p{]},\{φ\}\_\{i\}(\{e\}\_\{j\}) = \{δ\}\_\{i\}\^{}\{j\}
\textbackslash{}mathrel\{⇔\} v(\{e\}\_\{j\}) = \{ε\}\_\{j\} (j-ième
vecteur de la base canonique). La famille
(\{e\}\_\{1\},\textbackslash{}mathop\{\textbackslash{}mathop\{\ldots{}\}\},\{e\}\_\{p\})
est donc l'image de la base canonique de \{K\}\^{}\{p\} par
l'isomorphisme \{v\}\^{}\{−1\}.

Proposition~2.4.8 La famille
(\{φ\}\_\{1\},\textbackslash{}mathop\{\textbackslash{}mathop\{\ldots{}\}\},\{φ\}\_\{p\})
est libre si et seulement si v est surjective. Sous ces conditions,
\textbackslash{}mathop\{\textbackslash{}mathrm\{Ker\}\}v est de
codimension p et \textbackslash{}mathop\{∀\}φ ∈ \{E\}\^{}\{∗\},

(φ
∈\textbackslash{}mathop\{\textbackslash{}mathrm\{Vect\}\}(\{φ\}\_\{1\},\textbackslash{}mathop\{\textbackslash{}mathop\{\ldots{}\}\},\{φ\}\_\{p\})
\textbackslash{}mathrel\{⇔\} \textbackslash{}mathop\{∀\}x
∈\textbackslash{}mathop\{\textbackslash{}mathrm\{Ker\}\}v, φ(x) = 0)

Démonstration Si v est surjective, notons
(\{ε\}\_\{1\},\textbackslash{}mathop\{\textbackslash{}mathop\{\ldots{}\}\},\{ε\}\_\{p\})
la base canonique de \{K\}\^{}\{p\} et soit \{e\}\_\{i\} ∈ E tel que
v(\{e\}\_\{i\}) = \{ε\}\_\{i\}~; on a donc
\textbackslash{}mathop\{∀\}i,j ∈ {[}1,p{]}, \{φ\}\_\{i\}(\{e\}\_\{j\}) =
\{δ\}\_\{i\}\^{}\{j\} ce qui implique évidemment que la famille est
libre~: si \{\textbackslash{}mathop\{\textbackslash{}mathop\{∑ \}\}
\}\_\{j=1\}\^{}\{p\}\{λ\}\_\{j\}\{φ\}\_\{j\} = 0, on a pour tout i ∈
{[}1,p{]}

0 = 0(\{e\}\_\{i\}) =\{ \textbackslash{}mathop\{∑
\}\}\_\{j=1\}\^{}\{p\}\{λ\}\_\{ j\}\{φ\}\_\{j\}(\{e\}\_\{i\}) =\{
\textbackslash{}mathop\{∑ \}\}\_\{j=1\}\^{}\{p\}\{λ\}\_\{
j\}\{φ\}\_\{j\}(\{e\}\_\{i\}) = \{λ\}\_\{i\}

Inversement, si la famille est libre, on peut compléter la famille
(\{φ\}\_\{1\},\textbackslash{}mathop\{\textbackslash{}mathop\{\ldots{}\}\},\{φ\}\_\{p\})
en une base
(\{φ\}\_\{1\},\textbackslash{}mathop\{\textbackslash{}mathop\{\ldots{}\}\},\{φ\}\_\{n\})
de \{E\}\^{}\{∗\} qui est la base duale de la base
(\{e\}\_\{1\},\textbackslash{}mathop\{\textbackslash{}mathop\{\ldots{}\}\},\{e\}\_\{n\})
de E. On a alors \textbackslash{}mathop\{∀\}i,j ∈ {[}1,p{]},
\{φ\}\_\{i\}(\{e\}\_\{j\}) = \{δ\}\_\{i\}\^{}\{j\}, soit
\textbackslash{}mathop\{∀\}i ∈ {[}1,p{]}, v(\{e\}\_\{i\}) =
\{ε\}\_\{i\}. L'image de v contient une base de \{K\}\^{}\{p\}, c'est
donc \{K\}\^{}\{p\} et v est surjective. Dans ces conditions, on peut
appliquer le théorème du rang, et donc \textbackslash{}mathop\{dim\}
\textbackslash{}mathop\{\textbackslash{}mathrm\{Ker\}\}v
=\textbackslash{}mathop\{ dim\} E − p.

Soit x ∈\textbackslash{}mathop\{\textbackslash{}mathrm\{Ker\}\}v~; alors
\textbackslash{}mathop\{∀\}i ∈ {[}1,p{]}, \{φ\}\_\{i\}(x) = 0 et donc
\textbackslash{}mathop\{∀\}φ
∈\textbackslash{}mathop\{\textbackslash{}mathrm\{Vect\}\}(\{φ\}\_\{1\},\textbackslash{}mathop\{\textbackslash{}mathop\{\ldots{}\}\},\{φ\}\_\{p\}),
φ(x) = 0. Inversement, supposons que
φ\textbackslash{}mathrel\{∉\}\textbackslash{}mathop\{\textbackslash{}mathrm\{Vect\}\}(\{φ\}\_\{1\},\textbackslash{}mathop\{\textbackslash{}mathop\{\ldots{}\}\},\{φ\}\_\{p\})~;
alors la famille
(\{φ\}\_\{1\},\textbackslash{}mathop\{\textbackslash{}mathop\{\ldots{}\}\},\{φ\}\_\{p\},φ)
est libre, on peut la compléter en une base de \{E\}\^{}\{∗\} qui est la
base duale d'une base
(\{e\}\_\{1\},\textbackslash{}mathop\{\textbackslash{}mathop\{\ldots{}\}\},\{e\}\_\{n\})
de E~; on a alors \{e\}\_\{p+1\}
∈\textbackslash{}mathop\{\textbackslash{}mathrm\{Ker\}\}v et
φ(\{e\}\_\{p+1\}) = 1. On a donc bien l'équivalence

\textbackslash{}begin\{eqnarray*\} \textbackslash{}mathop\{∀\}φ ∈
\{E\}\^{}\{∗\},\textbackslash{}quad (φ
∈\textbackslash{}mathop\{\textbackslash{}mathrm\{Vect\}\}(\{φ\}\_\{
1\},\textbackslash{}mathop\{\textbackslash{}mathop\{\ldots{}\}\},\{φ\}\_\{p\})\&\&\%\&
\textbackslash{}\textbackslash{} \& \textbackslash{}mathrel\{⇔\} \&
\textbackslash{}mathop\{∀\}x
∈\textbackslash{}mathop\{\textbackslash{}mathrm\{Ker\}\}v, φ(x) = 0)\%\&
\textbackslash{}\textbackslash{} \textbackslash{}end\{eqnarray*\}

Remarque~2.4.6 Application~: soit
\{H\}\_\{1\},\textbackslash{}mathop\{\textbackslash{}mathop\{\ldots{}\}\},\{H\}\_\{p\}
des hyperplans de E d'équations respectives \{φ\}\_\{1\}(x) =
0,\textbackslash{}mathop\{\textbackslash{}mathop\{\ldots{}\}\},\{φ\}\_\{p\}(x)
= 0~; soit r =\textbackslash{}mathop\{
\textbackslash{}mathrm\{rg\}\}(\{φ\}\_\{1\},\textbackslash{}mathop\{\textbackslash{}mathop\{\ldots{}\}\},\{φ\}\_\{p\}).
Quitte à renuméroter les \{H\}\_\{i\}, on peut supposer que
(\{φ\}\_\{1\},\textbackslash{}mathop\{\textbackslash{}mathop\{\ldots{}\}\},\{φ\}\_\{r\})
est une base de
\textbackslash{}mathop\{\textbackslash{}mathrm\{Vect\}\}(\{φ\}\_\{1\},\textbackslash{}mathop\{\textbackslash{}mathop\{\ldots{}\}\},\{φ\}\_\{p\}).
On a alors, si v : E → \{K\}\^{}\{r\} est l'application linéaire
associée à cette famille,

\textbackslash{}begin\{eqnarray*\} x ∈\{\textbackslash{}mathop\{⋂
\}\}\_\{i=1\}\^{}\{p\}\{H\}\_\{ i\}\& \textbackslash{}mathrel\{⇔\} \&
\textbackslash{}mathop\{∀\}i ∈ {[}1,p{]}, \{φ\}\_\{i\}(x) = 0\%\&
\textbackslash{}\textbackslash{} \& \textbackslash{}mathrel\{⇔\} \&
\textbackslash{}mathop\{∀\}i ∈ {[}1,r{]}, \{φ\}\_\{i\}(x) = 0\%\&
\textbackslash{}\textbackslash{} \& \textbackslash{}mathrel\{⇔\} \& x
∈\textbackslash{}mathop\{\textbackslash{}mathrm\{Ker\}\}v \%\&
\textbackslash{}\textbackslash{} \textbackslash{}end\{eqnarray*\}

ce qui montre que \{\textbackslash{}mathop\{\textbackslash{}mathop\{⋂
\}\} \}\_\{i=1\}\^{}\{p\}\{H\}\_\{i\} est un sous-espace vectoriel de
dimension \textbackslash{}mathop\{dim\} E − r. Soit alors H un hyperplan
de E d'équation φ(x) = 0. On a alors

\textbackslash{}begin\{eqnarray*\} \{\textbackslash{}mathop\{⋂
\}\}\_\{i=1\}\^{}\{p\}\{H\}\_\{ i\} ⊂ H\& \textbackslash{}mathrel\{⇔\}
\& \textbackslash{}mathop\{\textbackslash{}mathrm\{Ker\}\}v
⊂\textbackslash{}mathop\{\textbackslash{}mathrm\{Ker\}\}φ \%\&
\textbackslash{}\textbackslash{} \& \textbackslash{}mathrel\{⇔\} \& φ
∈\textbackslash{}mathop\{\textbackslash{}mathrm\{Vect\}\}(\{φ\}\_\{1\},\textbackslash{}mathop\{\textbackslash{}mathop\{\ldots{}\}\},\{φ\}\_\{r\})
=\textbackslash{}mathop\{
\textbackslash{}mathrm\{Vect\}\}(\{φ\}\_\{1\},\textbackslash{}mathop\{\textbackslash{}mathop\{\ldots{}\}\},\{φ\}\_\{p\})\%\&
\textbackslash{}\textbackslash{} \textbackslash{}end\{eqnarray*\}

\paragraph{2.4.6 Application~: polynômes d'interpolation de Lagrange}

Théorème~2.4.9 Soit K un corps commutatif,
\{x\}\_\{1\},\textbackslash{}mathop\{\textbackslash{}mathop\{\ldots{}\}\},\{x\}\_\{n\}
∈ K distincts. Soit
\{a\}\_\{1\},\textbackslash{}mathop\{\textbackslash{}mathop\{\ldots{}\}\},\{a\}\_\{n\}
∈ K. Alors il existe un unique polynôme P ∈ K{[}X{]} tel que
\textbackslash{}mathop\{deg\} P ≤ n − 1 et \textbackslash{}mathop\{∀\}i
∈ {[}1,n{]}, P(\{x\}\_\{i\}) = \{a\}\_\{i\}.

Démonstration Soit \{φ\}\_\{i\} : \{K\}\_\{n−1\}{[}X{]} → K,
P\textbackslash{}mathrel\{↦\}P(\{x\}\_\{i\}) (où \{K\}\_\{n−1\}{[}X{]} =
\textbackslash{}\{P ∈
K{[}X{]}\textbackslash{}mathrel\{∣\}\textbackslash{}mathop\{deg\} P ≤ n
− 1\textbackslash{}\}). Les \{φ\}\_\{i\} sont des formes linéaires sur
l'espace vectoriel \{K\}\_\{n−1\}{[}X{]} de dimension n~; soit v :
\{K\}\_\{n−1\}{[}X{]} → \{K\}\^{}\{n\},
P\textbackslash{}mathrel\{↦\}(\{φ\}\_\{1\}(P),\textbackslash{}mathop\{\textbackslash{}mathop\{\ldots{}\}\},\{φ\}\_\{n\}(P))
=
(P(\{x\}\_\{1\}),\textbackslash{}mathop\{\textbackslash{}mathop\{\ldots{}\}\},P(\{x\}\_\{n\})).
Alors v est une application linéaire injective car

\textbackslash{}begin\{eqnarray*\} v(P) = 0\&
\textbackslash{}mathrel\{⇔\} \& \textbackslash{}mathop\{∀\}i ∈
{[}1,n{]}, P(\{x\}\_\{i\}) = 0 \%\& \textbackslash{}\textbackslash{} \&
\textbackslash{}mathrel\{⇔\} \& \{\textbackslash{}mathop\{∏
\}\}\_\{i=1\}\^{}\{n\}(X − \{x\}\_\{
i\})\textbackslash{}mathrel\{∣\}P(X) \textbackslash{}mathrel\{⇔\} P =
0\%\& \textbackslash{}\textbackslash{} \textbackslash{}end\{eqnarray*\}

pour des raisons de degré évidentes. On en déduit que v est un
isomorphisme d'espaces vectoriels, ce qui démontre le résultat.

Remarque~2.4.7 Comme v est surjective, la famille
(\{φ\}\_\{1\},\textbackslash{}mathop\{\textbackslash{}mathop\{\ldots{}\}\},\{φ\}\_\{n\})
est libre~; comme son cardinal est n, c'est une base du dual
\{K\}\_\{n−1\}\{{[}X{]}\}\^{}\{∗\}. Cherchons la base dont c'est la
duale, c'est-à-dire des polynômes \{P\}\_\{i\} vérifiant
\{P\}\_\{i\}(\{x\}\_\{j\}) = \{δ\}\_\{i\}\^{}\{j\}~; un tel polynôme
doit être divisible par
\{\textbackslash{}mathop\{\textbackslash{}mathop\{∏ \}\}
\}\_\{j\textbackslash{}mathrel\{≠\}i\}(X − \{x\}\_\{j\}). Pour des
raisons de degrés, il doit lui être proportionnel et le fait que
\{P\}\_\{i\}(\{x\}\_\{i\}) = 1 nécessite

\{P\}\_\{i\}(X) =\{ \{\textbackslash{}mathop\{∏
\}\}\_\{j\textbackslash{}mathrel\{≠\}i\}(X − \{x\}\_\{j\})
\textbackslash{}over \{\textbackslash{}mathop\{∏
\}\}\_\{j\textbackslash{}mathrel\{≠\}i\}(\{x\}\_\{i\} − \{x\}\_\{j\})\}

On a alors

\textbackslash{}mathop\{∀\}P ∈ \{K\}\_\{n−1\}{[}X{]}, P =\{
\textbackslash{}mathop\{∑ \}\}\_\{i=1\}\^{}\{n\}\{φ\}\_\{
i\}(P)\{P\}\_\{i\} =\{ \textbackslash{}mathop\{∑
\}\}\_\{i=1\}\^{}\{n\}P(\{x\}\_\{ i\})\{ \{\textbackslash{}mathop\{∏
\}\}\_\{j\textbackslash{}mathrel\{≠\}i\}(X − \{x\}\_\{j\})
\textbackslash{}over \{\textbackslash{}mathop\{∏
\}\}\_\{j\textbackslash{}mathrel\{≠\}i\}(\{x\}\_\{i\} − \{x\}\_\{j\})\}

{[}\href{coursse11.html}{next}{]} {[}\href{coursse9.html}{prev}{]}
{[}\href{coursse9.html\#tailcoursse9.html}{prev-tail}{]}
{[}\href{coursse10.html}{front}{]}
{[}\href{coursch3.html\#coursse10.html}{up}{]}

\end{document}
