\documentclass[]{article}
\usepackage[T1]{fontenc}
\usepackage{lmodern}
\usepackage{amssymb,amsmath}
\usepackage{ifxetex,ifluatex}
\usepackage{fixltx2e} % provides \textsubscript
% use upquote if available, for straight quotes in verbatim environments
\IfFileExists{upquote.sty}{\usepackage{upquote}}{}
\ifnum 0\ifxetex 1\fi\ifluatex 1\fi=0 % if pdftex
  \usepackage[utf8]{inputenc}
\else % if luatex or xelatex
  \ifxetex
    \usepackage{mathspec}
    \usepackage{xltxtra,xunicode}
  \else
    \usepackage{fontspec}
  \fi
  \defaultfontfeatures{Mapping=tex-text,Scale=MatchLowercase}
  \newcommand{\euro}{€}
\fi
% use microtype if available
\IfFileExists{microtype.sty}{\usepackage{microtype}}{}
\ifxetex
  \usepackage[setpagesize=false, % page size defined by xetex
              unicode=false, % unicode breaks when used with xetex
              xetex]{hyperref}
\else
  \usepackage[unicode=true]{hyperref}
\fi
\hypersetup{breaklinks=true,
            bookmarks=true,
            pdfauthor={},
            pdftitle={Dualite : approche restreinte},
            colorlinks=true,
            citecolor=blue,
            urlcolor=blue,
            linkcolor=magenta,
            pdfborder={0 0 0}}
\urlstyle{same}  % don't use monospace font for urls
\setlength{\parindent}{0pt}
\setlength{\parskip}{6pt plus 2pt minus 1pt}
\setlength{\emergencystretch}{3em}  % prevent overfull lines
\setcounter{secnumdepth}{0}
 
/* start css.sty */
.cmr-5{font-size:50%;}
.cmr-7{font-size:70%;}
.cmmi-5{font-size:50%;font-style: italic;}
.cmmi-7{font-size:70%;font-style: italic;}
.cmmi-10{font-style: italic;}
.cmsy-5{font-size:50%;}
.cmsy-7{font-size:70%;}
.cmex-7{font-size:70%;}
.cmex-7x-x-71{font-size:49%;}
.msbm-7{font-size:70%;}
.cmtt-10{font-family: monospace;}
.cmti-10{ font-style: italic;}
.cmbx-10{ font-weight: bold;}
.cmr-17x-x-120{font-size:204%;}
.cmsl-10{font-style: oblique;}
.cmti-7x-x-71{font-size:49%; font-style: italic;}
.cmbxti-10{ font-weight: bold; font-style: italic;}
p.noindent { text-indent: 0em }
td p.noindent { text-indent: 0em; margin-top:0em; }
p.nopar { text-indent: 0em; }
p.indent{ text-indent: 1.5em }
@media print {div.crosslinks {visibility:hidden;}}
a img { border-top: 0; border-left: 0; border-right: 0; }
center { margin-top:1em; margin-bottom:1em; }
td center { margin-top:0em; margin-bottom:0em; }
.Canvas { position:relative; }
li p.indent { text-indent: 0em }
.enumerate1 {list-style-type:decimal;}
.enumerate2 {list-style-type:lower-alpha;}
.enumerate3 {list-style-type:lower-roman;}
.enumerate4 {list-style-type:upper-alpha;}
div.newtheorem { margin-bottom: 2em; margin-top: 2em;}
.obeylines-h,.obeylines-v {white-space: nowrap; }
div.obeylines-v p { margin-top:0; margin-bottom:0; }
.overline{ text-decoration:overline; }
.overline img{ border-top: 1px solid black; }
td.displaylines {text-align:center; white-space:nowrap;}
.centerline {text-align:center;}
.rightline {text-align:right;}
div.verbatim {font-family: monospace; white-space: nowrap; text-align:left; clear:both; }
.fbox {padding-left:3.0pt; padding-right:3.0pt; text-indent:0pt; border:solid black 0.4pt; }
div.fbox {display:table}
div.center div.fbox {text-align:center; clear:both; padding-left:3.0pt; padding-right:3.0pt; text-indent:0pt; border:solid black 0.4pt; }
div.minipage{width:100%;}
div.center, div.center div.center {text-align: center; margin-left:1em; margin-right:1em;}
div.center div {text-align: left;}
div.flushright, div.flushright div.flushright {text-align: right;}
div.flushright div {text-align: left;}
div.flushleft {text-align: left;}
.underline{ text-decoration:underline; }
.underline img{ border-bottom: 1px solid black; margin-bottom:1pt; }
.framebox-c, .framebox-l, .framebox-r { padding-left:3.0pt; padding-right:3.0pt; text-indent:0pt; border:solid black 0.4pt; }
.framebox-c {text-align:center;}
.framebox-l {text-align:left;}
.framebox-r {text-align:right;}
span.thank-mark{ vertical-align: super }
span.footnote-mark sup.textsuperscript, span.footnote-mark a sup.textsuperscript{ font-size:80%; }
div.tabular, div.center div.tabular {text-align: center; margin-top:0.5em; margin-bottom:0.5em; }
table.tabular td p{margin-top:0em;}
table.tabular {margin-left: auto; margin-right: auto;}
div.td00{ margin-left:0pt; margin-right:0pt; }
div.td01{ margin-left:0pt; margin-right:5pt; }
div.td10{ margin-left:5pt; margin-right:0pt; }
div.td11{ margin-left:5pt; margin-right:5pt; }
table[rules] {border-left:solid black 0.4pt; border-right:solid black 0.4pt; }
td.td00{ padding-left:0pt; padding-right:0pt; }
td.td01{ padding-left:0pt; padding-right:5pt; }
td.td10{ padding-left:5pt; padding-right:0pt; }
td.td11{ padding-left:5pt; padding-right:5pt; }
table[rules] {border-left:solid black 0.4pt; border-right:solid black 0.4pt; }
.hline hr, .cline hr{ height : 1px; margin:0px; }
.tabbing-right {text-align:right;}
span.TEX {letter-spacing: -0.125em; }
span.TEX span.E{ position:relative;top:0.5ex;left:-0.0417em;}
a span.TEX span.E {text-decoration: none; }
span.LATEX span.A{ position:relative; top:-0.5ex; left:-0.4em; font-size:85%;}
span.LATEX span.TEX{ position:relative; left: -0.4em; }
div.float img, div.float .caption {text-align:center;}
div.figure img, div.figure .caption {text-align:center;}
.marginpar {width:20%; float:right; text-align:left; margin-left:auto; margin-top:0.5em; font-size:85%; text-decoration:underline;}
.marginpar p{margin-top:0.4em; margin-bottom:0.4em;}
.equation td{text-align:center; vertical-align:middle; }
td.eq-no{ width:5%; }
table.equation { width:100%; } 
div.math-display, div.par-math-display{text-align:center;}
math .texttt { font-family: monospace; }
math .textit { font-style: italic; }
math .textsl { font-style: oblique; }
math .textsf { font-family: sans-serif; }
math .textbf { font-weight: bold; }
.partToc a, .partToc, .likepartToc a, .likepartToc {line-height: 200%; font-weight:bold; font-size:110%;}
.chapterToc a, .chapterToc, .likechapterToc a, .likechapterToc, .appendixToc a, .appendixToc {line-height: 200%; font-weight:bold;}
.index-item, .index-subitem, .index-subsubitem {display:block}
.caption td.id{font-weight: bold; white-space: nowrap; }
table.caption {text-align:center;}
h1.partHead{text-align: center}
p.bibitem { text-indent: -2em; margin-left: 2em; margin-top:0.6em; margin-bottom:0.6em; }
p.bibitem-p { text-indent: 0em; margin-left: 2em; margin-top:0.6em; margin-bottom:0.6em; }
.paragraphHead, .likeparagraphHead { margin-top:2em; font-weight: bold;}
.subparagraphHead, .likesubparagraphHead { font-weight: bold;}
.quote {margin-bottom:0.25em; margin-top:0.25em; margin-left:1em; margin-right:1em; text-align:\jmathustify;}
.verse{white-space:nowrap; margin-left:2em}
div.maketitle {text-align:center;}
h2.titleHead{text-align:center;}
div.maketitle{ margin-bottom: 2em; }
div.author, div.date {text-align:center;}
div.thanks{text-align:left; margin-left:10%; font-size:85%; font-style:italic; }
div.author{white-space: nowrap;}
.quotation {margin-bottom:0.25em; margin-top:0.25em; margin-left:1em; }
h1.partHead{text-align: center}
.sectionToc, .likesectionToc {margin-left:2em;}
.subsectionToc, .likesubsectionToc {margin-left:4em;}
.subsubsectionToc, .likesubsubsectionToc {margin-left:6em;}
.frenchb-nbsp{font-size:75%;}
.frenchb-thinspace{font-size:75%;}
.figure img.graphics {margin-left:10%;}
/* end css.sty */

\title{Dualite : approche restreinte}
\author{}
\date{}

\begin{document}
\maketitle

\textbf{Warning: 
requires JavaScript to process the mathematics on this page.\\ If your
browser supports JavaScript, be sure it is enabled.}

\begin{center}\rule{3in}{0.4pt}\end{center}

{[}
{[}
{[}{]}
{[}

\subsubsection{2.4 Dualité~: approche restreinte}

\paragraph{2.4.1 Formes linéaires, dual, formes coordonnées}

Définition~2.4.1 Soit E un K-espace vectoriel . On appelle forme
linéaire sur E toute application linéaire de E dans K. On appelle dual
de E le K-espace vectoriel E^∗ = L(E,K).

Remarque~2.4.1 Soit (e\_i)\_i\inI une base de E et
i\_0 \in I. Tout vecteur x de E s'écrit de manière unique sous la
forme x = \\sum ~
\_i\inIx\_ie\_i. L'application
\phi\_i\_0 :
x\mapsto~x\_i\_0 est clairement une
forme linéaire sur E, appelée forme linéaire coordonnée d'indice
i\_0 dans la base (e\_i)\_i\inI. Elle est définie
par \phi\_i\_0(e\_i\_0) = 1 et
\phi\_i\_0(e\_i) = 0 si
i\neq~i\_0, soit encore par
\phi\_i\_0(e\_i) =
\delta\_i\_0^i.

Proposition~2.4.1 Soit E un K-espace vectoriel et x \in E,
x\neq~0. Alors il existe une forme linéaire \phi sur
E telle que \phi(x) = 1.

Démonstration Le vecteur x forme à lui tout seul une famille libre que
l'on peut compléter en une base (e\_i)\_i\inI de E avec x
= e\_i\_0. Soit \phi la forme linéaire qui associe à tout
vecteur de E sa i\_0-ième coordonnée dans cette base. On a bien
entendu \phi(x) = 1.

Remarque~2.4.2 Le résultat précédent peut encore s'interpréter sous la
forme~: si x \in E,

\forall~\phi \in E^∗~, \phi(x) =
0\quad \Leftrightarrow x = 0

\paragraph{2.4.2 Base duale d'un espace vectoriel de dimension finie}

Définition~2.4.2 Soit E un espace vectoriel de dimension finie, \mathcal{E} =
(e\_1,\\ldots,e\_n~)
une base de E. Pour i \in {[}1,n{]}, soit \phi\_i la forme linéaire
coordonnée d'indice i dans la base \mathcal{E}. Alors \mathcal{E}^∗ =
(\phi\_1,\\ldots,\phi\_n~)
est une base du dual E^∗, appelée la base duale de la base \mathcal{E}.
Elle est caractérisée par les relations \forall~~i,\jmath \in
{[}1,n{]}, \phi\_i(e\_\jmath) = \delta\_i^\jmath.

Démonstration Tout d'abord, montrons que \mathcal{E}^∗ est une famille
libre en utilisant les relations de définition des formes coordonnées

\forall~i,\jmath \in {[}1,n{]}, \phi\_i(e\_\jmath~) =
\delta\_i^\jmath

Soit
\lambda~\_1,\\ldots,\lambda~\_n~
\in K tels que \lambda~\_1\phi\_1 +
\\ldots~ +
\lambda~\_n\phi\_n = 0~; on a alors, pout tout i \in {[}1,n{]}

0 = 0(e\_i) = \lambda~\_1\phi\_1(e\_i) +
\\ldots~ +
\lambda~\_n\phi\_n(e\_i) = \lambda~\_i

ce qui montre bien que la famille est libre. Pour montrer qu'elle est
génératrice, soit \phi \in E et considérons \psi =\
\sum ~
\_i=1^n\phi(e\_i)\phi\_i~; on a alors pour tout \jmath
\in {[}1,n{]},

\psi(e\_\jmath) = \\sum
\_i=1^n\phi(e\_ i)\phi\_i(e\_\jmath) =
\sum \_i=1^n\phi(e~\_
i)\delta\_i^\jmath = \phi(e\_ \jmath)

Les deux applications linéaires \phi et \psi coïncidant sur une base, sont
égales, ce qui montre que la famille est génératrice.

Remarque~2.4.3 Attention~: la dimension finie est essentielle~; elle
garantit qu'il n'y a qu'un nombre fini de \phi(e\_i) non nuls et
permet de considérer la somme
\\sum ~
\_i\inI\phi(e\_i)\phi\_i~; en dimension infinie,
\mathcal{E}^∗ n'est pas une base de E^∗, car elle n'est pas
génératrice (considérer la forme linéaire \phi qui à tout e\_i
associe 1).

Corollaire~2.4.2 La dimension de l'espace dual d'un espace vectoriel de
dimension finie est égale à la dimension de l'espace.

\paragraph{2.4.3 Orthogonalité 1}

Soit E un K-espace vectoriel de dimension finie,
(e\_1,\\ldots,e\_p~)
une famille d'éléments de E. Nous pouvons associer à cette famille
l'application u : E^∗\rightarrow~ K^p, définie par u(\phi) =
(\phi(e\_1),\\ldots,\phi(e\_p~)).
On vérifie immédiatement que u est linéaire. Son noyau est constitué des
\phi \in E^∗ vérifiant \forall~~i \in {[}1,p{]},
\phi(e\_i) = 0.

Proposition~2.4.3 Si
(e\_1,\\ldots,e\_p~)
est une base de E, alors u est un isomorphisme d'espace vectoriel de
E^∗ sur K^p.

Démonstration En effet dans ce cas, u envoie la base duale
\mathcal{E}^∗ sur la base canonique de K^p~; c'est donc un
isomorphisme.

Proposition~2.4.4 La famille
(e\_1,\\ldots,e\_p~)
est libre si et seulement si u est sur\jmathective. Sous ces conditions,
\mathrmKer~u est de
codimension p et

\forall~~x \in E,\quad (x
\in\mathrmVect(e\_1,\\\ldots,e\_p~)
\Leftrightarrow \forall~~\phi
\in\mathrmKer~u, \phi(x) = 0)

Démonstration Si u est sur\jmathective, notons
(\epsilon\_1,\\ldots,\epsilon\_p~)
la base canonique de K^p et soit \phi\_i \in
E^∗ tel que u(\phi\_i) = \epsilon\_i~; on a donc
\forall~i,\jmath \in {[}1,p{]}, \phi\_i(e\_\jmath~) =
\delta\_i^\jmath ce qui implique évidemment que la famille est
libre~: si \\sum ~
\_\jmath=1^p\lambda~\_\jmathe\_\jmath = 0, on a pour tout i \in
{[}1,p{]}

0 = \phi\_i(0) = \phi\_i(\\sum
\_\jmath=1^p\lambda~\_ \jmathe\_\jmath) =
\sum \_\jmath=1^p\lambda~~\_
\jmath\phi\_i(e\_\jmath) = \lambda~\_i

Inversement, si la famille est libre, on peut compléter la famille
(e\_1,\\ldots,e\_p~)
en une base
(e\_1,\\ldots,e\_n~)
de E et soit
(\phi\_1,\\ldots,\phi\_n~)
la base duale. On a alors \forall~~i,\jmath \in {[}1,p{]},
\phi\_i(e\_\jmath) = \delta\_i^\jmath, soit
\forall~i \in {[}1,p{]}, u(\phi\_i~) =
\epsilon\_i. L'image de u contient une base de K^p, c'est
donc K^p et u est sur\jmathective. Dans ces conditions, on peut
appliquer le théorème du rang, et donc dim~
\mathrmKer~u
= dim E^∗~- p
= dim~ E - p.

Soit \phi \in\mathrmKer~u~; alors
\forall~i \in {[}1,p{]}, \phi(e\_i~) = 0 et donc
\forall~~x
\in\mathrmVect(e\_1,\\\ldots,e\_p~),
\phi(x) = 0. Inversement, supposons que
x∉\mathrmVect(e\_1,\\\ldots,e\_p~)~;
alors la famille
(e\_1,\\ldots,e\_p~,x)
est libre, on peut la compléter en une base de E et la forme coordonnée
suivant x dans cette base, soit \phi, appartient à
\mathrmKer~u alors que \phi(x)
= 1. On a donc bien l'équivalence

\forall~~x \in E,\quad (x
\in\mathrmVect(e\_1,\\\ldots,e\_p~)
\Leftrightarrow \forall~~\phi
\in\mathrmKer~u, \phi(x) = 0)

Remarque~2.4.4 Application~: Soit F un sous-espace vectoriel de E,
(e\_1,\\ldots,e\_p~)
une base de F, u : E^∗\rightarrow~ K^p l'application linéaire
associée,
(\phi\_1,\\ldots,\phi\_n-p~)
une base de \mathrmKer~u~;
alors x \in F \Leftrightarrow \forall~~i \in
{[}1,n - p{]}, \phi\_i(x) = 0. On dit encore que F est défini par
le système d'équations linéaires \phi\_1(x) =
0,\\ldots,\phi\_n-p~(x)
= 0.

\paragraph{2.4.4 Hyperplans}

Définition~2.4.3 On appelle hyperplan de E tout sous-espace vectoriel H
de E vérifiant les conditions équivalentes

\begin{itemize}
\itemsep1pt\parskip0pt\parsep0pt
\item
  (i) dim~ E\diagupH = 1
\item
  (ii) \exists~f \in E
  \diagdown\0\, H =\
  \mathrmKerf
\item
  (iii) H admet une droite comme supplémentaire.
\end{itemize}

Démonstration

\begin{itemize}
\itemsep1pt\parskip0pt\parsep0pt
\item
  (i) \rigtharrow~(ii)~: prendre \overlinee une base de E\diagupH et
  écrire \pi~(x) = f(x)\overlinee.
\item
  (ii) \rigtharrow~ (iii)~: on prend a \in E tel que
  f(a)\neq~0. Tout élément x s'écrit de manière
  unique sous la forme x = (x - f(x) \over f(a) a)
  + f(x) \over f(a) a avec x - f(x)
  \over f(a) a
  \in\mathrmKer~f, soit E
  = \mathrmKer~f \oplus~ Ka.
\item
  (iii) \rigtharrow~(i)~: tout supplémentaire de H est isomorphe à E\diagupH.
\end{itemize}

Théorème~2.4.5 Soit H un hyperplan de E. Alors deux formes linéaires
nulles sur H sont proportionnelles.

Démonstration Si E = H \oplus~ Ka et H =\
\mathrmKerf, soit g \in E^∗ nulle sur H.
Alors g et  g(a) \over f(a) f coïncident sur H et sur
Ka, donc sont égales.

\paragraph{2.4.5 Orthogonalité 2}

Remarque~2.4.5 Soit E un K-espace vectoriel de dimension finie,
(\phi\_1,\\ldots,\phi\_p~)
une famille d'éléments de E^∗. Nous pouvons associer à cette
famille l'application v : E \rightarrow~ K^p, définie par v(x) =
(\phi\_1(x),\\ldots,\phi\_p~(x)).
On vérifie immédiatement que v est linéaire. Son noyau est constitué de
l'intersection des
\mathrmKer\phi\_i~ (en
général des hyperplans, sauf si la forme linéaire est nulle).

Proposition~2.4.6 Si
(\phi\_1,\\ldots,\phi\_p~)
est une base de E^∗, alors v est un isomorphisme d'espace
vectoriel de E^∗ sur K^p.

Démonstration En effet dans ce cas, v est in\jmathective car

\begin{align*} v(x) = 0&
\Leftrightarrow & \forall~~i \in
{[}1,p{]}, \phi\_i(x) = 0\%& \\ &
\Leftrightarrow & \forall~~\phi \in
E^∗, \phi(x) = 0 \%& \\ &
\Leftrightarrow & x = 0 \%&
\\ \end{align*}

Comme dim E =\ dim~
E^∗ = p = dim K^p~, il
s'agit d'un isomorphisme.

Théorème~2.4.7 Soit
(\phi\_1,\\ldots,\phi\_p~)
une base de E^∗~; alors il existe une unique base
(e\_1,\\ldots,e\_p~)
de E dont
(\phi\_1,\\ldots,\phi\_p~)
soit la base duale.

Démonstration On a en effet \forall~~i \in
{[}1,p{]},\phi\_i(e\_\jmath) = \delta\_i^\jmath
\Leftrightarrow v(e\_\jmath) = \epsilon\_\jmath (\jmath-ième
vecteur de la base canonique). La famille
(e\_1,\\ldots,e\_p~)
est donc l'image de la base canonique de K^p par
l'isomorphisme v^-1.

Proposition~2.4.8 La famille
(\phi\_1,\\ldots,\phi\_p~)
est libre si et seulement si v est sur\jmathective. Sous ces conditions,
\mathrmKer~v est de
codimension p et \forall~\phi \in E^∗~,

(\phi
\in\mathrmVect(\phi\_1,\\\ldots,\phi\_p~)
\Leftrightarrow \forall~~x
\in\mathrmKer~v, \phi(x) = 0)

Démonstration Si v est sur\jmathective, notons
(\epsilon\_1,\\ldots,\epsilon\_p~)
la base canonique de K^p et soit e\_i \in E tel que
v(e\_i) = \epsilon\_i~; on a donc
\forall~i,\jmath \in {[}1,p{]}, \phi\_i(e\_\jmath~) =
\delta\_i^\jmath ce qui implique évidemment que la famille est
libre~: si \\sum ~
\_\jmath=1^p\lambda~\_\jmath\phi\_\jmath = 0, on a pour tout i \in
{[}1,p{]}

0 = 0(e\_i) = \\sum
\_\jmath=1^p\lambda~\_ \jmath\phi\_\jmath(e\_i) =
\sum \_\jmath=1^p\lambda~~\_
\jmath\phi\_\jmath(e\_i) = \lambda~\_i

Inversement, si la famille est libre, on peut compléter la famille
(\phi\_1,\\ldots,\phi\_p~)
en une base
(\phi\_1,\\ldots,\phi\_n~)
de E^∗ qui est la base duale de la base
(e\_1,\\ldots,e\_n~)
de E. On a alors \forall~~i,\jmath \in {[}1,p{]},
\phi\_i(e\_\jmath) = \delta\_i^\jmath, soit
\forall~i \in {[}1,p{]}, v(e\_i~) =
\epsilon\_i. L'image de v contient une base de K^p, c'est
donc K^p et v est sur\jmathective. Dans ces conditions, on peut
appliquer le théorème du rang, et donc dim~
\mathrmKer~v
= dim~ E - p.

Soit x \in\mathrmKer~v~; alors
\forall~i \in {[}1,p{]}, \phi\_i~(x) = 0 et donc
\forall~~\phi
\in\mathrmVect(\phi\_1,\\\ldots,\phi\_p~),
\phi(x) = 0. Inversement, supposons que
\phi∉\mathrmVect(\phi\_1,\\\ldots,\phi\_p~)~;
alors la famille
(\phi\_1,\\ldots,\phi\_p~,\phi)
est libre, on peut la compléter en une base de E^∗ qui est la
base duale d'une base
(e\_1,\\ldots,e\_n~)
de E~; on a alors e\_p+1
\in\mathrmKer~v et
\phi(e\_p+1) = 1. On a donc bien l'équivalence

\begin{align*} \forall~~\phi \in
E^∗,\quad (\phi
\in\mathrmVect(\phi~\_
1,\\ldots,\phi\_p~)&&\%&
\\ & \Leftrightarrow &
\forall~~x
\in\mathrmKer~v, \phi(x) = 0)\%&
\\ \end{align*}

Remarque~2.4.6 Application~: soit
H\_1,\\ldots,H\_p~
des hyperplans de E d'équations respectives \phi\_1(x) =
0,\\ldots,\phi\_p~(x)
= 0~; soit r =\
\mathrmrg(\phi\_1,\\ldots,\phi\_p~).
Quitte à renuméroter les H\_i, on peut supposer que
(\phi\_1,\\ldots,\phi\_r~)
est une base de
\mathrmVect(\phi\_1,\\\ldots,\phi\_p~).
On a alors, si v : E \rightarrow~ K^r est l'application linéaire
associée à cette famille,

\begin{align*} x \in\⋂
\_i=1^pH\_ i& \Leftrightarrow &
\forall~i \in {[}1,p{]}, \phi\_i~(x) = 0\%&
\\ & \Leftrightarrow &
\forall~i \in {[}1,r{]}, \phi\_i~(x) = 0\%&
\\ & \Leftrightarrow & x
\in\mathrmKer~v \%&
\\ \end{align*}

ce qui montre que \\⋂
 \_i=1^pH\_i est un sous-espace vectoriel de
dimension dim~ E - r. Soit alors H un hyperplan
de E d'équation \phi(x) = 0. On a alors

\begin{align*} \⋂
\_i=1^pH\_ i \subset~ H& \Leftrightarrow
& \mathrmKer~v
\subset~\mathrmKer~\phi \%&
\\ & \Leftrightarrow & \phi
\in\mathrmVect(\phi\_1,\\\ldots,\phi\_r~)
=\
\mathrmVect(\phi\_1,\\ldots,\phi\_p~)\%&
\\ \end{align*}

\paragraph{2.4.6 Application~: polynômes d'interpolation de Lagrange}

Théorème~2.4.9 Soit K un corps commutatif,
x\_1,\\ldots,x\_n~
\in K distincts. Soit
a\_1,\\ldots,a\_n~
\in K. Alors il existe un unique polynôme P \in K{[}X{]} tel que
deg P \leq n - 1 et \\forall~~i
\in {[}1,n{]}, P(x\_i) = a\_i.

Démonstration Soit \phi\_i : K\_n-1{[}X{]} \rightarrow~ K,
P\mapsto~P(x\_i) (où K\_n-1{[}X{]} =
\P \in
K{[}X{]}∣deg~ P \leq n
- 1\). Les \phi\_i sont des formes linéaires sur
l'espace vectoriel K\_n-1{[}X{]} de dimension n~; soit v :
K\_n-1{[}X{]} \rightarrow~ K^n,
P\mapsto~(\phi\_1(P),\\ldots,\phi\_n~(P))
=
(P(x\_1),\\ldots,P(x\_n~)).
Alors v est une application linéaire in\jmathective car

\begin{align*} v(P) = 0&
\Leftrightarrow & \forall~~i \in
{[}1,n{]}, P(x\_i) = 0 \%& \\ &
\Leftrightarrow & \∏
\_i=1^n(X - x\_
i)∣P(X) \mathrel\Leftrightarrow P =
0\%& \\ \end{align*}

pour des raisons de degré évidentes. On en déduit que v est un
isomorphisme d'espaces vectoriels, ce qui démontre le résultat.

Remarque~2.4.7 Comme v est sur\jmathective, la famille
(\phi\_1,\\ldots,\phi\_n~)
est libre~; comme son cardinal est n, c'est une base du dual
K\_n-1{[}X{]}^∗. Cherchons la base dont c'est la
duale, c'est-à-dire des polynômes P\_i vérifiant
P\_i(x\_\jmath) = \delta\_i^\jmath~; un tel polynôme
doit être divisible par
\∏ ~
\_\jmath\neq~i(X - x\_\jmath). Pour des
raisons de degrés, il doit lui être proportionnel et le fait que
P\_i(x\_i) = 1 nécessite

P\_i(X) = \∏
\_\jmath\neq~i(X - x\_\jmath)
\over \∏
\_\jmath\neq~i(x\_i - x\_\jmath)

On a alors

\forall~P \in K\_n-1~{[}X{]}, P =
\sum \_i=1^n\phi~\_
i(P)P\_i = \\sum
\_i=1^nP(x\_ i) \∏
\_\jmath\neq~i(X - x\_\jmath)
\over \∏
\_\jmath\neq~i(x\_i - x\_\jmath)

{[}
{[}
{[}
{[}

\end{document}
