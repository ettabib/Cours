\documentclass[]{article}
\usepackage[T1]{fontenc}
\usepackage{lmodern}
\usepackage{amssymb,amsmath}
\usepackage{ifxetex,ifluatex}
\usepackage{fixltx2e} % provides \textsubscript
% use upquote if available, for straight quotes in verbatim environments
\IfFileExists{upquote.sty}{\usepackage{upquote}}{}
\ifnum 0\ifxetex 1\fi\ifluatex 1\fi=0 % if pdftex
  \usepackage[utf8]{inputenc}
\else % if luatex or xelatex
  \ifxetex
    \usepackage{mathspec}
    \usepackage{xltxtra,xunicode}
  \else
    \usepackage{fontspec}
  \fi
  \defaultfontfeatures{Mapping=tex-text,Scale=MatchLowercase}
  \newcommand{\euro}{€}
\fi
% use microtype if available
\IfFileExists{microtype.sty}{\usepackage{microtype}}{}
\ifxetex
  \usepackage[setpagesize=false, % page size defined by xetex
              unicode=false, % unicode breaks when used with xetex
              xetex]{hyperref}
\else
  \usepackage[unicode=true]{hyperref}
\fi
\hypersetup{breaklinks=true,
            bookmarks=true,
            pdfauthor={},
            pdftitle={Developpements en series enti`eres},
            colorlinks=true,
            citecolor=blue,
            urlcolor=blue,
            linkcolor=magenta,
            pdfborder={0 0 0}}
\urlstyle{same}  % don't use monospace font for urls
\setlength{\parindent}{0pt}
\setlength{\parskip}{6pt plus 2pt minus 1pt}
\setlength{\emergencystretch}{3em}  % prevent overfull lines
\setcounter{secnumdepth}{0}
 
/* start css.sty */
.cmr-5{font-size:50%;}
.cmr-7{font-size:70%;}
.cmmi-5{font-size:50%;font-style: italic;}
.cmmi-7{font-size:70%;font-style: italic;}
.cmmi-10{font-style: italic;}
.cmsy-5{font-size:50%;}
.cmsy-7{font-size:70%;}
.cmex-7{font-size:70%;}
.cmex-7x-x-71{font-size:49%;}
.msbm-7{font-size:70%;}
.cmtt-10{font-family: monospace;}
.cmti-10{ font-style: italic;}
.cmbx-10{ font-weight: bold;}
.cmr-17x-x-120{font-size:204%;}
.cmsl-10{font-style: oblique;}
.cmti-7x-x-71{font-size:49%; font-style: italic;}
.cmbxti-10{ font-weight: bold; font-style: italic;}
p.noindent { text-indent: 0em }
td p.noindent { text-indent: 0em; margin-top:0em; }
p.nopar { text-indent: 0em; }
p.indent{ text-indent: 1.5em }
@media print {div.crosslinks {visibility:hidden;}}
a img { border-top: 0; border-left: 0; border-right: 0; }
center { margin-top:1em; margin-bottom:1em; }
td center { margin-top:0em; margin-bottom:0em; }
.Canvas { position:relative; }
li p.indent { text-indent: 0em }
.enumerate1 {list-style-type:decimal;}
.enumerate2 {list-style-type:lower-alpha;}
.enumerate3 {list-style-type:lower-roman;}
.enumerate4 {list-style-type:upper-alpha;}
div.newtheorem { margin-bottom: 2em; margin-top: 2em;}
.obeylines-h,.obeylines-v {white-space: nowrap; }
div.obeylines-v p { margin-top:0; margin-bottom:0; }
.overline{ text-decoration:overline; }
.overline img{ border-top: 1px solid black; }
td.displaylines {text-align:center; white-space:nowrap;}
.centerline {text-align:center;}
.rightline {text-align:right;}
div.verbatim {font-family: monospace; white-space: nowrap; text-align:left; clear:both; }
.fbox {padding-left:3.0pt; padding-right:3.0pt; text-indent:0pt; border:solid black 0.4pt; }
div.fbox {display:table}
div.center div.fbox {text-align:center; clear:both; padding-left:3.0pt; padding-right:3.0pt; text-indent:0pt; border:solid black 0.4pt; }
div.minipage{width:100%;}
div.center, div.center div.center {text-align: center; margin-left:1em; margin-right:1em;}
div.center div {text-align: left;}
div.flushright, div.flushright div.flushright {text-align: right;}
div.flushright div {text-align: left;}
div.flushleft {text-align: left;}
.underline{ text-decoration:underline; }
.underline img{ border-bottom: 1px solid black; margin-bottom:1pt; }
.framebox-c, .framebox-l, .framebox-r { padding-left:3.0pt; padding-right:3.0pt; text-indent:0pt; border:solid black 0.4pt; }
.framebox-c {text-align:center;}
.framebox-l {text-align:left;}
.framebox-r {text-align:right;}
span.thank-mark{ vertical-align: super }
span.footnote-mark sup.textsuperscript, span.footnote-mark a sup.textsuperscript{ font-size:80%; }
div.tabular, div.center div.tabular {text-align: center; margin-top:0.5em; margin-bottom:0.5em; }
table.tabular td p{margin-top:0em;}
table.tabular {margin-left: auto; margin-right: auto;}
div.td00{ margin-left:0pt; margin-right:0pt; }
div.td01{ margin-left:0pt; margin-right:5pt; }
div.td10{ margin-left:5pt; margin-right:0pt; }
div.td11{ margin-left:5pt; margin-right:5pt; }
table[rules] {border-left:solid black 0.4pt; border-right:solid black 0.4pt; }
td.td00{ padding-left:0pt; padding-right:0pt; }
td.td01{ padding-left:0pt; padding-right:5pt; }
td.td10{ padding-left:5pt; padding-right:0pt; }
td.td11{ padding-left:5pt; padding-right:5pt; }
table[rules] {border-left:solid black 0.4pt; border-right:solid black 0.4pt; }
.hline hr, .cline hr{ height : 1px; margin:0px; }
.tabbing-right {text-align:right;}
span.TEX {letter-spacing: -0.125em; }
span.TEX span.E{ position:relative;top:0.5ex;left:-0.0417em;}
a span.TEX span.E {text-decoration: none; }
span.LATEX span.A{ position:relative; top:-0.5ex; left:-0.4em; font-size:85%;}
span.LATEX span.TEX{ position:relative; left: -0.4em; }
div.float img, div.float .caption {text-align:center;}
div.figure img, div.figure .caption {text-align:center;}
.marginpar {width:20%; float:right; text-align:left; margin-left:auto; margin-top:0.5em; font-size:85%; text-decoration:underline;}
.marginpar p{margin-top:0.4em; margin-bottom:0.4em;}
.equation td{text-align:center; vertical-align:middle; }
td.eq-no{ width:5%; }
table.equation { width:100%; } 
div.math-display, div.par-math-display{text-align:center;}
math .texttt { font-family: monospace; }
math .textit { font-style: italic; }
math .textsl { font-style: oblique; }
math .textsf { font-family: sans-serif; }
math .textbf { font-weight: bold; }
.partToc a, .partToc, .likepartToc a, .likepartToc {line-height: 200%; font-weight:bold; font-size:110%;}
.chapterToc a, .chapterToc, .likechapterToc a, .likechapterToc, .appendixToc a, .appendixToc {line-height: 200%; font-weight:bold;}
.index-item, .index-subitem, .index-subsubitem {display:block}
.caption td.id{font-weight: bold; white-space: nowrap; }
table.caption {text-align:center;}
h1.partHead{text-align: center}
p.bibitem { text-indent: -2em; margin-left: 2em; margin-top:0.6em; margin-bottom:0.6em; }
p.bibitem-p { text-indent: 0em; margin-left: 2em; margin-top:0.6em; margin-bottom:0.6em; }
.paragraphHead, .likeparagraphHead { margin-top:2em; font-weight: bold;}
.subparagraphHead, .likesubparagraphHead { font-weight: bold;}
.quote {margin-bottom:0.25em; margin-top:0.25em; margin-left:1em; margin-right:1em; text-align:justify;}
.verse{white-space:nowrap; margin-left:2em}
div.maketitle {text-align:center;}
h2.titleHead{text-align:center;}
div.maketitle{ margin-bottom: 2em; }
div.author, div.date {text-align:center;}
div.thanks{text-align:left; margin-left:10%; font-size:85%; font-style:italic; }
div.author{white-space: nowrap;}
.quotation {margin-bottom:0.25em; margin-top:0.25em; margin-left:1em; }
h1.partHead{text-align: center}
.sectionToc, .likesectionToc {margin-left:2em;}
.subsectionToc, .likesubsectionToc {margin-left:4em;}
.subsubsectionToc, .likesubsubsectionToc {margin-left:6em;}
.frenchb-nbsp{font-size:75%;}
.frenchb-thinspace{font-size:75%;}
.figure img.graphics {margin-left:10%;}
/* end css.sty */

\title{Developpements en series enti`eres}
\author{}
\date{}

\begin{document}
\maketitle

\textbf{Warning: \href{http://www.math.union.edu/locate/jsMath}{jsMath}
requires JavaScript to process the mathematics on this page.\\ If your
browser supports JavaScript, be sure it is enabled.}

\begin{center}\rule{3in}{0.4pt}\end{center}

{[}\href{coursse66.html}{next}{]} {[}\href{coursse64.html}{prev}{]}
{[}\href{coursse64.html\#tailcoursse64.html}{prev-tail}{]}
{[}\hyperref[tailcoursse65.html]{tail}{]}
{[}\href{coursch12.html\#coursse65.html}{up}{]}

\subsubsection{11.3 Développements en séries entières}

\paragraph{11.3.1 Problème local, problème global}

Définition~11.3.1 (globale). Soit U un voisinage de 0 dans K et f : U →
E (espace vectoriel normé complet). Soit r \textgreater{} 0 tel que
D(0,r) = \textbackslash{}\{z ∈
K\textbackslash{}mathrel\{∣\}\textbar{}z\textbar{} \textless{}
r\textbackslash{}\} ⊂ U. On dit que f est développable en série entière
sur le disque D(0,r) s'il existe une série entière
\textbackslash{}mathop\{\textbackslash{}mathop\{∑ \}\}
\{a\}\_\{n\}\{z\}\^{}\{n\} de rayon de convergence R ≥ r tel que
\textbackslash{}mathop\{∀\}z ∈ D(0,r), f(z) =\textbackslash{}mathop\{
\textbackslash{}mathop\{∑ \}\} \{a\}\_\{n\}\{z\}\^{}\{n\}.

Définition~11.3.2 (locale). Soit U un voisinage de 0 dans K et f : U → E
(espace vectoriel normé complet). On dit que f est développable en série
entière au voisinage de 0 s'il existe r \textgreater{} 0 tel que D(0,r)
= \textbackslash{}\{z ∈
K\textbackslash{}mathrel\{∣\}\textbar{}z\textbar{} \textless{}
r\textbackslash{}\} ⊂ U et f est développable en série entière dans
D(0,r).

Remarque~11.3.1 Dans le premier cas, on impose donc le disque dans
lequel f doit être développable en série entière, alors que dans le
second cas on laisse tout latitude à r de prendre des valeurs aussi
petites que nécessaires.

Remarque~11.3.2 Il est facile de construire des fonctions non
développables en séries entières au voisinage de 0. Par exemple toute
fonction admettant 0 comme zéro non isolé ne peut être développable en
série entière au voisinage de 0. Par exemple f(x) =
\{e\}\^{}\{−1∕\{x\}\^{}\{2\} \}\textbackslash{}mathop\{ sin\} \{ 1
\textbackslash{}over x\} si x\textbackslash{}mathrel\{≠\}0, f(0) = 0~;
cette fonction, bien que \{C\}\^{}\{∞\} sur ℝ, ne peut être développable
en série entière au voisinage de 0.

\paragraph{11.3.2 Méthodes de développement}

On peut utiliser tout d'abord les résultats sur les opérations
algébriques ou analytiques concernant les séries entières. On obtiendra
de manière évidente les résultats suivants

Théorème~11.3.1 (i) si f et g : U → E sont développables en séries
entières~sur D(0,r) ⊂ U (resp. au voisinage de 0), alors αf + βg est
développable en série entière~sur D(0,r) (resp. au voisinage de 0). (ii)
si f et g : U → K sont développables en séries entières~sur D(0,r) ⊂ U
(resp. au voisinage de 0), alors fg est développable en série
entière~sur D(0,r) (resp. au voisinage de 0). (iii) si f : U → K est
développable en série entière~au voisinage de 0 et si
f(0)\textbackslash{}mathrel\{≠\}0, alors \{ 1 \textbackslash{}over f\}
est développable en série entière~au voisinage de 0. (iv) si U ⊂ ℝ, f :
U → E est dérivable et si f' est développable en série entière~sur
D(0,r) ⊂ U (resp. au voisinage de 0), alors f est développable en série
entière~sur D(0,r) (resp. au voisinage de 0). (v) si U ⊂ ℝ, f : U → E
est développable en série entière~sur D(0,r) ⊂ U (resp. au voisinage de
0), alors f est \{C\}\^{}\{∞\} sur {]} − r,r{[} (resp. au voisinage de
0)et \{f\}\^{}\{(p)\} est développable en série entière~sur D(0,r)
(resp. au voisinage de 0).

Dans un cadre général, on pourra utiliser le résultat suivant

Théorème~11.3.2 (Mac Laurin). Soit U un voisinage de 0, f : U → E de
classe \{C\}\^{}\{∞\} et r \textgreater{} 0 tel que {]} − r,r{[}⊂ U.
Alors f est développable en série entière dans {]} − r,r{[} si et
seulement si

\textbackslash{}mathop\{∀\}t ∈{]} − r,r{[},
\{\textbackslash{}mathop\{lim\}\}\_\{n→+∞\}\textbackslash{}left (f(t)
−\{\textbackslash{}mathop\{∑ \}\}\_\{k=0\}\^{}\{n\}\{
\{f\}\^{}\{(k)\}(0) \textbackslash{}over k!\}
\{t\}\^{}\{k\}\textbackslash{}right ) = 0

Démonstration On sait en effet que f est développable en série entière
dans {]} − r,r{[} si et seulement si

\textbackslash{}mathop\{∀\}t ∈{]} − r,r{[}, f(t) =\{
\textbackslash{}mathop\{∑ \}\}\_\{k=0\}\^{}\{+∞\}\{ \{f\}\^{}\{(k)\}(0)
\textbackslash{}over k!\} \{t\}\^{}\{k\}

ce qui est la même chose que l'assertion ci dessus.

Méthode~: pour démontrer que f est développable en série entière dans
{]} − r,r{[}, on peut donc chercher à estimer (par exemple par
utilisation de la formule de Taylor-Lagrange ou de la formule de Taylor
avec reste intégral) l'expression \{R\}\_\{n\}(t) = f(t)
−\{\textbackslash{}mathop\{\textbackslash{}mathop\{∑ \}\}
\}\_\{k=0\}\^{}\{n\}\{ \{f\}\^{}\{(k)\}(0) \textbackslash{}over k!\}
\{t\}\^{}\{k\} et à montrer qu'elle admet la limite 0 quand n tend vers
+ ∞.

Exemple~11.3.1 Soit f :{]} − r,r{[}→ E une fonction de classe
\{C\}\^{}\{∞\} telle que \textbackslash{}mathop\{∀\}n ∈ ℕ,
\textbackslash{}mathop\{∀\}t ∈{]} − r,r{[},
\textbackslash{}\textbar{}\{f\}\^{}\{(n)\}(t)\textbackslash{}\textbar{}
≤ M. On peut alors écrire la formule de Taylor avec reste intégral, et
on obtient, pour t ∈{]} − r,r{[}, \{R\}\_\{n\}(t)
=\{\textbackslash{}mathop\{∫ \} \}\_\{0\}\^{}\{t\}\{ \{(t−u)\}\^{}\{n\}
\textbackslash{}over n!\} \{f\}\^{}\{(n+1)\}(u) du soit encore
\textbackslash{}\textbar{}\{R\}\_\{n\}(t)\textbackslash{}\textbar{} ≤
M\{\textbackslash{}mathop\{∫ \} \}\_\{0\}\^{}\{t\}\{ \{(t−u)\}\^{}\{n\}
\textbackslash{}over n!\} du = M\{ \{t\}\^{}\{n+1\} \textbackslash{}over
(n+1)!\} qui tend vers 0 quand n tend vers + ∞ (la série converge
d'après la règle de d'Alembert). On en déduit que la fonction f est
développable en série entière dans {]} − r,r{[}. C'est ainsi que quelle
que soit la méthode utilisée pour définir une fonction exponentielle,
qui vérifiera (\textbackslash{}mathop\{exp\} )'
=\textbackslash{}mathop\{ exp\} , on aura (pour un r quelconque)
\textbackslash{}mathop\{∀\}n ∈ ℕ, \textbackslash{}mathop\{∀\}t ∈{]} −
r,r{[}, \textbar{}\{f\}\^{}\{(n)\}(t)\textbar{}≤ \{e\}\^{}\{r\}, donc
\textbackslash{}mathop\{∀\}t ∈{]} − r,r{[},
\textbackslash{}mathop\{exp\} (t) =\{\textbackslash{}mathop\{
\textbackslash{}mathop\{∑ \}\} \}\_\{n=0\}\^{}\{+∞\}\{ \{t\}\^{}\{n\}
\textbackslash{}over n!\} et donc \textbackslash{}mathop\{∀\}t ∈ ℝ,
\textbackslash{}mathop\{exp\} (t) =\{\textbackslash{}mathop\{
\textbackslash{}mathop\{∑ \}\} \}\_\{n=0\}\^{}\{+∞\}\{ \{t\}\^{}\{n\}
\textbackslash{}over n!\} . Une étude similaire pourrait être faite pour
les fonctions trigonométriques (cosinus et sinus). En fait, dans le
paragraphe suivant, nous allons définir directement ces fonctions comme
sommes de séries entières.

Enfin, une dernière technique utilise le théorème de Cauchy-Lipschitz
sur l'unicité des solutions d'équations différentielles à conditions
initiales données. Soit f : U → K une fonction de classe \{C\}\^{}\{∞\}
vérifiant une équation différentielle \{y\}\^{}\{(n)\} =
G(t,y,y',\textbackslash{}mathop\{\textbackslash{}mathop\{\ldots{}\}\},\{y\}\^{}\{(n−1)\}).
Supposons également que nous connaissions une série entière
\textbackslash{}mathop\{\textbackslash{}mathop\{∑ \}\}
\{a\}\_\{n\}\{t\}\^{}\{n\} de rayon de convergence R \textgreater{} 0
telle que sa somme S vérifie la même équation différentielle avec de
plus S(0) =
f(0),\textbackslash{}mathop\{\textbackslash{}mathop\{\ldots{}\}\},\{S\}\^{}\{(n−1)\}(0)
= \{f\}\^{}\{(n−1)\}(0). Si l'équation différentielle relève de l'un des
deux théorèmes de Cauchy Lipschitz, et si en particulier elle est
linéaire à coefficients continus, on aura \textbackslash{}mathop\{∀\}t
∈{]} − R,R{[}∩U, f(t) = S(t) =\{\textbackslash{}mathop\{
\textbackslash{}mathop\{∑ \}\}
\}\_\{k=0\}\^{}\{+∞\}\{a\}\_\{k\}\{t\}\^{}\{k\}. La démarche est alors
la suivante

\begin{itemize}
\item
  (i) trouver une équation différentielle vérifiée par f
\item
  (ii) trouver une série entière
  \textbackslash{}mathop\{\textbackslash{}mathop\{∑ \}\}
  \{a\}\_\{n\}\{t\}\^{}\{n\} vérifiant formellement l'équation
  différentielle en question avec

  \{a\}\_\{0\} =
  f(0),\textbackslash{}mathop\{\textbackslash{}mathop\{\ldots{}\}\},\{a\}\_\{n−1\}
  = (n − 1)!\{f\}\^{}\{(n−1)\}(0)
\item
  (iii) montrer que cette série entière a un rayon de convergence R non
  nul
\item
  (iv) utiliser un théorème de Cauchy Lipschitz pour garantir que
  \textbackslash{}mathop\{∀\}t ∈{]} − R,R{[}∩U, f(t) = S(t)
  =\{\textbackslash{}mathop\{ \textbackslash{}mathop\{∑ \}\}
  \}\_\{k=0\}\^{}\{+∞\}\{a\}\_\{k\}\{t\}\^{}\{k\}
\end{itemize}

Exemple~11.3.2 Soit m ∈ ℝ et f : ℝ → ℝ définie par f(x) = \{(x +
\textbackslash{}sqrt\{1 + \{x\}\^{}\{2\}\})\}\^{}\{m\}. On a

\textbackslash{}begin\{eqnarray*\} f'(x)\& =\& m\textbackslash{}left (1
+ \{x\textbackslash{}over \textbackslash{}sqrt\{1 +
\{x\}\^{}\{2\}\}\}\textbackslash{}right )\{(x + \textbackslash{}sqrt\{1
+ \{x\}\^{}\{2\}\})\}\^{}\{m−1\}\%\& \textbackslash{}\textbackslash{} \&
=\& m \{f(x)\textbackslash{}over \textbackslash{}sqrt\{1 +
\{x\}\^{}\{2\}\}\} \%\& \textbackslash{}\textbackslash{}
\textbackslash{}end\{eqnarray*\}

d'où

\textbackslash{}begin\{eqnarray*\} f'`(x)\& =\& −
\{2mx\textbackslash{}over \{ (1 + \{x\}\^{}\{2\})\}\^{}\{3∕2\}\}f(x) +
\{m\textbackslash{}over \textbackslash{}sqrt\{1 +
\{x\}\^{}\{2\}\}\}f'(x)\%\& \textbackslash{}\textbackslash{} \& =\& −
\{x\textbackslash{}over 1 + \{x\}\^{}\{2\}\}f'(x) +
\{\{m\}\^{}\{2\}\textbackslash{}over 1 + \{x\}\^{}\{2\}\}f(x) \%\&
\textbackslash{}\textbackslash{} \textbackslash{}end\{eqnarray*\}

en rempla\textbackslash{}c\{c\}ant simultanément m
\{f(x)\textbackslash{}over \textbackslash{}sqrt\{1+\{x\}\^{}\{2\}\}\}
par f'(x) et f'(x) par m \{f(x)\textbackslash{}over
\textbackslash{}sqrt\{1+\{x\}\^{}\{2\}\}\} . Donc f vérifie l'équation
différentielle

(1 + \{x\}\^{}\{2\})y'`+ xy' − \{m\}\^{}\{2\}y = 0

avec les conditions initiales f(0) = 1, f'(0) = m.

Cherchons inversement une fonction S développable en série entière sur
{]} − R,R{[} vérifiant cette équation différentielle. On a alors S(t)
=\{\textbackslash{}mathop\{ \textbackslash{}mathop\{∑ \}\}
\}\_\{n=0\}\^{}\{+∞\}\{a\}\_\{n\}\{t\}\^{}\{n\}, tS'(t)
=\{\textbackslash{}mathop\{ \textbackslash{}mathop\{∑ \}\}
\}\_\{n=0\}\^{}\{+∞\}n\{a\}\_\{n\}\{t\}\^{}\{n\}, \{t\}\^{}\{2\}S''(t)
=\{\textbackslash{}mathop\{ \textbackslash{}mathop\{∑ \}\}
\}\_\{n=0\}\^{}\{+∞\}n(n − 1)\{a\}\_\{n\}\{t\}\^{}\{n\} et S''(t)
=\{\textbackslash{}mathop\{ \textbackslash{}mathop\{∑ \}\}
\}\_\{n=0\}\^{}\{+∞\}(n + 1)(n + 2)\{a\}\_\{n+2\}\{t\}\^{}\{n\}
(vérifications faciles laissées au lecteur, en remarquant que
n\{a\}\_\{n\} = 0 pour n = 0 et n(n − 1)\{a\}\_\{n\} = 0 pour n = 0 et n
= 1). On a donc

\textbackslash{}begin\{eqnarray*\} (1 + \{t\}\^{}\{2\})S'`(t) + tS'(t) −
\{m\}\^{}\{2\}S(t)\&\& \%\& \textbackslash{}\textbackslash{} \& =\&
\{\textbackslash{}mathop\{∑ \}\}\_\{n=0\}\^{}\{+∞\}\textbackslash{}left
((n + 1)(n + 2)\{a\}\_\{ n+2\} + n(n − 1)\{a\}\_\{n\} + n\{a\}\_\{n\} −
\{m\}\^{}\{2\}\{a\}\_\{ n\}\textbackslash{}right )\{t\}\^{}\{n\}\%\&
\textbackslash{}\textbackslash{} \& =\& \{\textbackslash{}mathop\{∑
\}\}\_\{n=0\}\^{}\{+∞\}\textbackslash{}left ((n + 1)(n + 2)\{a\}\_\{
n+2\} + (\{n\}\^{}\{2\} − \{m\}\^{}\{2\})\{a\}\_\{
n\}\textbackslash{}right )\{t\}\^{}\{n\} \%\&
\textbackslash{}\textbackslash{} \textbackslash{}end\{eqnarray*\}

L'unicité du développement en série entière de la fonction nulle montre
que S est solution de l'équation différentielle si et seulement si~

\textbackslash{}mathop\{∀\}n ∈ ℕ, (n + 1)(n + 2)\{a\}\_\{n+2\} +
(\{n\}\^{}\{2\} − \{m\}\^{}\{2\})\{a\}\_\{ n\}

On veut de plus que \{a\}\_\{0\} = 1 et \{a\}\_\{1\} = m. Ces trois
relations définissent parfaitement les deux suites
\{(\{a\}\_\{2n\})\}\_\{n∈ℕ\} et \{(\{a\}\_\{2n+1\})\}\_\{n∈ℕ\} (l'une
des deux, celle correspondant à la parité de m, étant nulle à partir du
rang \textbar{}m\textbar{} + 2 si m ∈ ℤ). Des deux séries entières
\textbackslash{}mathop\{\textbackslash{}mathop\{∑ \}\}
\{a\}\_\{2n\}\{t\}\^{}\{2n\} et
\textbackslash{}mathop\{\textbackslash{}mathop\{∑ \}\}
\{a\}\_\{2n+1\}\{t\}\^{}\{2n+1\}, l'une des deux est de rayon de
convergence 1, l'autre de rayon de convergence soit 1 soit + ∞. Donc la
série entière \textbackslash{}mathop\{\textbackslash{}mathop\{∑ \}\}
\{a\}\_\{n\}\{t\}\^{}\{n\} est de rayon de convergence 1
=\textbackslash{}mathop\{ inf\}(1,+∞).

Sur {]} − 1,1{[}, on peut donc définir S(t) =\{\textbackslash{}mathop\{
\textbackslash{}mathop\{∑ \}\}
\}\_\{n=0\}\^{}\{+∞\}\{a\}\_\{n\}\{t\}\^{}\{n\}. Cette fonction vérifie
la même équation différentielle que f avec les mêmes conditions
initiales. Le théorème de Cauchy-Lipschitz pour les équations
différentielles linéaire à coefficients continus garantit que f et S
coïncident sur l'intersection de leurs intervalles de définition,
c'est-à-dire sur {]} − 1,1{[}. Donc f est développable en série entière
sur {]} − 1,1{[}.

\paragraph{11.3.3 Fonction exponentielle. Fonctions trigonométriques}

La règle de d'Alembert montre que la série entière
\{\textbackslash{}mathop\{\textbackslash{}mathop\{∑ \}\} \}\_\{n≥0\}\{
\{z\}\^{}\{n\} \textbackslash{}over n!\} a un rayon de convergence
infini. Ceci justifie l'introduction de la définition suivante

Définition~11.3.3 Pour z ∈ ℂ, on pose \textbackslash{}mathop\{exp\} (z)
=\{\textbackslash{}mathop\{ \textbackslash{}mathop\{∑ \}\}
\}\_\{n=0\}\^{}\{+∞\}\{ \{z\}\^{}\{n\} \textbackslash{}over n!\} .

Proposition~11.3.3 (i) z\textbackslash{}mathrel\{↦\}exp(z) est un
morphisme de groupes de (ℂ,+) dans (\{ℂ\}\^{}\{∗\},.), autrement dit
\textbackslash{}mathop\{exp\} (0) = 1, \textbackslash{}mathop\{exp\}
(\{z\}\_\{1\} + \{z\}\_\{2\}) =\textbackslash{}mathop\{ exp\}
\{z\}\_\{1\}\textbackslash{}mathop\{ exp\} \{z\}\_\{2\},
\textbackslash{}mathop\{exp\} (z)\textbackslash{}mathrel\{≠\}0 et
\{(\textbackslash{}mathop\{exp\} (z))\}\^{}\{−1\}
=\textbackslash{}mathop\{ exp\} (−z). (ii) \textbackslash{}mathop\{∀\}z
∈ ℂ, \textbackslash{}mathop\{exp\} (\textbackslash{}overline\{z\}) =
\textbackslash{}overline\{\textbackslash{}mathop\{exp\} z\} (iii) pour
tout z ∈ ℂ, l'application
t\textbackslash{}mathrel\{↦\}\textbackslash{}mathop\{exp\} (tz) est
\{C\}\^{}\{∞\} de ℝ dans ℂ et \{ \{d\}\^{}\{n\} \textbackslash{}over
d\{t\}\^{}\{n\}\} (\textbackslash{}mathop\{exp\} (tz)) =
\{z\}\^{}\{n\}\textbackslash{}mathop\{ exp\} (tz).

Démonstration (i) Soit \{z\}\_\{1\},\{z\}\_\{2\} ∈ ℂ. On pose
\{a\}\_\{n\} =\{ \{z\}\_\{1\}\^{}\{n\} \textbackslash{}over n!\} et
\{b\}\_\{n\} =\{ \{z\}\_\{2\}\^{}\{n\} \textbackslash{}over n!\} . Ces
séries sont absolument convergentes. On peut donc faire le produit de
Cauchy de ces deux séries et on a alors \{c\}\_\{n\}
=\{\textbackslash{}mathop\{ \textbackslash{}mathop\{∑ \}\}
\}\_\{k=0\}\^{}\{n\}\{ 1 \textbackslash{}over k!(n−k)!\}
\{z\}\_\{1\}\^{}\{k\}\{z\}\_\{ 2\}\^{}\{n−k\} =\{ 1 \textbackslash{}over
n!\} \{(\{z\}\_\{1\} + \{z\}\_\{2\})\}\^{}\{n\} d'après la formule du
binôme. On a donc

\{\textbackslash{}mathop\{∑ \}\}\_\{n=0\}\^{}\{+∞\}\{ \{(\{z\}\_\{1\} +
\{z\}\_\{2\})\}\^{}\{n\} \textbackslash{}over n!\} =
\textbackslash{}left (\{\textbackslash{}mathop\{∑
\}\}\_\{n=0\}\^{}\{+∞\}\{ \{z\}\_\{1\}\^{}\{n\} \textbackslash{}over
n!\} \textbackslash{}right )\textbackslash{}left
(\{\textbackslash{}mathop\{∑ \}\}\_\{n=0\}\^{}\{+∞\}\{
\{z\}\_\{2\}\^{}\{n\} \textbackslash{}over n!\} \textbackslash{}right )

(ii) Il suffit de faire tendre N vers + ∞ dans la formule évidente
\{\textbackslash{}mathop\{\textbackslash{}mathop\{∑ \}\}
\}\_\{n=0\}\^{}\{N\}\{ \{\textbackslash{}overline\{z\}\}\^{}\{n\}
\textbackslash{}over n!\} =
\textbackslash{}overline\{\{\textbackslash{}mathop\{\textbackslash{}mathop\{∑
\}\} \}\_\{n=0\}\^{}\{N\}\{ \{z\}\^{}\{n\} \textbackslash{}over n!\} \}

(iii) On a \textbackslash{}mathop\{exp\} (tz)
=\{\textbackslash{}mathop\{ \textbackslash{}mathop\{∑ \}\}
\}\_\{n=0\}\^{}\{+∞\}\{ \{z\}\^{}\{n\} \textbackslash{}over n!\}
\{t\}\^{}\{n\} qui est une série entière en t de rayon de convergence
infini. On en déduit que sa somme est de classe \{C\}\^{}\{∞\} sur ℝ et
que

\textbackslash{}begin\{eqnarray*\}\{ \{d\}\^{}\{n\} \textbackslash{}over
d\{t\}\^{}\{n\}\} (\textbackslash{}mathop\{exp\} (tz))\& =\&
\{\textbackslash{}mathop\{∑ \}\}\_\{k=n\}\^{}\{+∞\}\{ \{z\}\^{}\{k\}
\textbackslash{}over k!\} k(k − 1)\textbackslash{}mathop\{\ldots{}\}(k −
n + 1)\{t\}\^{}\{k−n\}\%\& \textbackslash{}\textbackslash{} \& =\&
\{\textbackslash{}mathop\{∑ \}\}\_\{k=n\}\^{}\{+∞\}\{ \{z\}\^{}\{k\}
\textbackslash{}over (k − n)!\} \{t\}\^{}\{k−n\} = \{z\}\^{}\{n\} exp
(tz) \%\& \textbackslash{}\textbackslash{}
\textbackslash{}end\{eqnarray*\}

après le changement de k − n en k.

Définition~11.3.4 On pose e =\textbackslash{}mathop\{ exp\} 1~; on a
bien entendu, \textbackslash{}mathop\{∀\}n ∈ ℤ,
\textbackslash{}mathop\{exp\} (n) = \{e\}\^{}\{n\}. On en déduit
facilement que \textbackslash{}mathop\{∀\}r ∈ ℚ,
\textbackslash{}mathop\{exp\} (r) = \{e\}\^{}\{r\}, ce qui justifie
ensuite la notation \textbackslash{}mathop\{exp\} (z) = \{e\}\^{}\{z\}
pour z ∈ ℂ.

Définition~11.3.5 Pour z ∈ ℂ, on pose \textbackslash{}mathop\{cos\} z
=\{ \{e\}\^{}\{iz\}+\{e\}\^{}\{−iz\} \textbackslash{}over 2\} ,
\textbackslash{}mathop\{sin\} z =\{ \{e\}\^{}\{iz\}−\{e\}\^{}\{−iz\}
\textbackslash{}over 2i\} ,
\textbackslash{}mathop\{\textbackslash{}mathrm\{ch\}\} z =\{
\{e\}\^{}\{z\}+\{e\}\^{}\{−z\} \textbackslash{}over 2\} ,
\textbackslash{}mathop\{\textbackslash{}mathrm\{sh\}\} z =\{
\{e\}\^{}\{z\}−\{e\}\^{}\{−z\} \textbackslash{}over 2\} .

Remarque~11.3.3 Il est clair que les fonctions
\textbackslash{}mathop\{cos\} et
\textbackslash{}mathop\{\textbackslash{}mathrm\{ch\}\} sont paires et
que les fonctions \textbackslash{}mathop\{sin\} et
\textbackslash{}mathop\{\textbackslash{}mathrm\{sh\}\} sont impaires.

Proposition~11.3.4 (i) \textbackslash{}mathop\{∀\}z ∈ ℂ,
\textbackslash{}mathop\{\textbackslash{}mathrm\{ch\}\} iz
=\textbackslash{}mathop\{ cos\} z,
\textbackslash{}mathop\{\textbackslash{}mathrm\{sh\}\} iz =
i\textbackslash{}mathop\{sin\} z, \{\textbackslash{}mathop\{cos\}
\}\^{}\{2\}z +\{\textbackslash{}mathop\{ sin\} \}\^{}\{2\}z = 1,
\{\textbackslash{}mathop\{\textbackslash{}mathrm\{ch\}\} \}\^{}\{2\}z
−\{\textbackslash{}mathop\{\textbackslash{}mathrm\{sh\}\} \}\^{}\{2\}z =
1 (ii) \textbackslash{}mathop\{∀\}a,b ∈ ℂ

\textbackslash{}begin\{eqnarray*\} \textbackslash{}mathop\{cos\} (a +
b)\& =\& \textbackslash{}mathop\{cos\} a\textbackslash{}mathop\{cos\} b
−\textbackslash{}mathop\{ sin\} a\textbackslash{}mathop\{sin\} b\%\&
\textbackslash{}\textbackslash{} \textbackslash{}mathop\{sin\} (a + b)\&
=\& \textbackslash{}mathop\{sin\} a\textbackslash{}mathop\{cos\} b
+\textbackslash{}mathop\{ cos\} a\textbackslash{}mathop\{sin\} b\%\&
\textbackslash{}\textbackslash{}
\textbackslash{}mathop\{\textbackslash{}mathrm\{ch\}\} (a + b)\& =\&
\textbackslash{}mathop\{\textbackslash{}mathrm\{ch\}\}
a\textbackslash{}mathop\{\textbackslash{}mathrm\{ch\}\} b
+\textbackslash{}mathop\{ \textbackslash{}mathrm\{sh\}\}
a\textbackslash{}mathop\{\textbackslash{}mathrm\{sh\}\} b \%\&
\textbackslash{}\textbackslash{}
\textbackslash{}mathop\{\textbackslash{}mathrm\{sh\}\} (a + b)\& =\&
\textbackslash{}mathop\{\textbackslash{}mathrm\{sh\}\}
a\textbackslash{}mathop\{\textbackslash{}mathrm\{ch\}\} b
+\textbackslash{}mathop\{ \textbackslash{}mathrm\{ch\}\}
a\textbackslash{}mathop\{\textbackslash{}mathrm\{sh\}\} b \%\&
\textbackslash{}\textbackslash{} \textbackslash{}end\{eqnarray*\}

Démonstration Par le calcul à partir de la définition.

Remarque~11.3.4 On déduit de ces formules de manière évidente toutes les
formules usuelles de la trigonométrie circulaire ou hyperbolique dont
nous ne citerons que celles qu'il est absolument indispensable de
connaître par coeur~:

\textbackslash{}mathop\{∀\}a,b ∈ ℂ

\textbackslash{}begin\{eqnarray*\} \textbackslash{}mathop\{cos\} (a +
b)\& =\& \textbackslash{}mathop\{cos\} a\textbackslash{}mathop\{cos\} b
−\textbackslash{}mathop\{ sin\} a\textbackslash{}mathop\{sin\} b \%\&
\textbackslash{}\textbackslash{} \textbackslash{}mathop\{cos\} (a − b)\&
=\& \textbackslash{}mathop\{cos\} a\textbackslash{}mathop\{cos\} b
+\textbackslash{}mathop\{ sin\} a\textbackslash{}mathop\{sin\} b \%\&
\textbackslash{}\textbackslash{} \textbackslash{}mathop\{sin\} (a + b)\&
=\& \textbackslash{}mathop\{sin\} a\textbackslash{}mathop\{cos\} b
+\textbackslash{}mathop\{ cos\} a\textbackslash{}mathop\{sin\} b \%\&
\textbackslash{}\textbackslash{} \textbackslash{}mathop\{sin\} (a − b)\&
=\& \textbackslash{}mathop\{sin\} a\textbackslash{}mathop\{cos\} b
−\textbackslash{}mathop\{ cos\} a\textbackslash{}mathop\{sin\} b \%\&
\textbackslash{}\textbackslash{} \textbackslash{}mathop\{cos\} (2a)\&
=\& \{\textbackslash{}mathop\{cos\} \}\^{}\{2\}a
−\{\textbackslash{}mathop\{ sin\} \}\^{}\{2\}a =
2\{\textbackslash{}mathop\{cos\} \}\^{}\{2\}a − 1 = 1 −
2\{\textbackslash{}mathop\{sin\} \}\^{}\{2\}a\%\&
\textbackslash{}\textbackslash{} \textbackslash{}mathop\{sin\} (2a)\&
=\& 2\textbackslash{}mathop\{sin\} a\textbackslash{}mathop\{cos\} a \%\&
\textbackslash{}\textbackslash{} \{\textbackslash{}mathop\{cos\}
\}\^{}\{2\}a\& =\&\{ 1 +\textbackslash{}mathop\{ cos\} 2a
\textbackslash{}over 2\} ,\textbackslash{}quad
\{\textbackslash{}mathop\{sin\} \}\^{}\{2\}a =\{ 1
−\textbackslash{}mathop\{ cos\} 2a \textbackslash{}over 2\} \%\&
\textbackslash{}\textbackslash{} \textbackslash{}mathop\{cos\} p
+\textbackslash{}mathop\{ cos\} q\& =\& 2\textbackslash{}mathop\{cos\}
\{ p − q \textbackslash{}over 2\} \textbackslash{}mathop\{cos\} \{ p + q
\textbackslash{}over 2\} \%\& \textbackslash{}\textbackslash{}
\textbackslash{}mathop\{cos\} p −\textbackslash{}mathop\{ cos\} q\& =\&
−2\textbackslash{}mathop\{sin\} \{ p − q \textbackslash{}over 2\}
\textbackslash{}mathop\{sin\} \{ p + q \textbackslash{}over 2\} \%\&
\textbackslash{}\textbackslash{} \textbackslash{}mathop\{sin\} p
+\textbackslash{}mathop\{ sin\} q\& =\& 2\textbackslash{}mathop\{cos\}
\{ p − q \textbackslash{}over 2\} \textbackslash{}mathop\{sin\} \{ p + q
\textbackslash{}over 2\} \%\& \textbackslash{}\textbackslash{}
\textbackslash{}mathop\{sin\} p −\textbackslash{}mathop\{ sin\} q\& =\&
2\textbackslash{}mathop\{sin\} \{ p − q \textbackslash{}over 2\}
\textbackslash{}mathop\{cos\} \{ p + q \textbackslash{}over 2\} \%\&
\textbackslash{}\textbackslash{} \textbackslash{}end\{eqnarray*\}

\textbackslash{}mathop\{∀\}a,b ∈ ℂ

\textbackslash{}begin\{eqnarray*\}
\textbackslash{}mathop\{\textbackslash{}mathrm\{ch\}\} (a + b)\& =\&
\textbackslash{}mathop\{\textbackslash{}mathrm\{ch\}\}
a\textbackslash{}mathop\{\textbackslash{}mathrm\{ch\}\} b
+\textbackslash{}mathop\{ \textbackslash{}mathrm\{sh\}\}
a\textbackslash{}mathop\{\textbackslash{}mathrm\{sh\}\} b \%\&
\textbackslash{}\textbackslash{}
\textbackslash{}mathop\{\textbackslash{}mathrm\{ch\}\} (a − b)\& =\&
\textbackslash{}mathop\{\textbackslash{}mathrm\{ch\}\}
a\textbackslash{}mathop\{\textbackslash{}mathrm\{ch\}\} b
−\textbackslash{}mathop\{\textbackslash{}mathrm\{sh\}\}
a\textbackslash{}mathop\{\textbackslash{}mathrm\{sh\}\} b \%\&
\textbackslash{}\textbackslash{}
\textbackslash{}mathop\{\textbackslash{}mathrm\{sh\}\} (a + b)\& =\&
\textbackslash{}mathop\{\textbackslash{}mathrm\{sh\}\}
a\textbackslash{}mathop\{\textbackslash{}mathrm\{ch\}\} b
+\textbackslash{}mathop\{ \textbackslash{}mathrm\{ch\}\}
a\textbackslash{}mathop\{\textbackslash{}mathrm\{sh\}\} b \%\&
\textbackslash{}\textbackslash{}
\textbackslash{}mathop\{\textbackslash{}mathrm\{sh\}\} (a − b)\& =\&
\textbackslash{}mathop\{\textbackslash{}mathrm\{sh\}\}
a\textbackslash{}mathop\{\textbackslash{}mathrm\{ch\}\} b
−\textbackslash{}mathop\{\textbackslash{}mathrm\{ch\}\}
a\textbackslash{}mathop\{\textbackslash{}mathrm\{sh\}\} b \%\&
\textbackslash{}\textbackslash{}
\textbackslash{}mathop\{\textbackslash{}mathrm\{ch\}\} (2a)\& =\&
\{\textbackslash{}mathop\{\textbackslash{}mathrm\{ch\}\} \}\^{}\{2\}a
+\{\textbackslash{}mathop\{ \textbackslash{}mathrm\{sh\}\} \}\^{}\{2\}a
= 2\{\textbackslash{}mathop\{\textbackslash{}mathrm\{ch\}\} \}\^{}\{2\}a
− 1 = 1 + 2\{\textbackslash{}mathop\{\textbackslash{}mathrm\{sh\}\}
\}\^{}\{2\}a\%\& \textbackslash{}\textbackslash{}
\textbackslash{}mathop\{\textbackslash{}mathrm\{sh\}\} (2a)\& =\&
2\textbackslash{}mathop\{\textbackslash{}mathrm\{sh\}\}
a\textbackslash{}mathop\{\textbackslash{}mathrm\{ch\}\} a \%\&
\textbackslash{}\textbackslash{} \textbackslash{}end\{eqnarray*\}

Etude sur ℝ des fonctions exponentielle et logarithme

Théorème~11.3.5 L'application
t\textbackslash{}mathrel\{↦\}\{e\}\^{}\{t\} est un \{C\}\^{}\{∞\}
difféomorphisme de ℝ sur {]}0,+∞{[}. Le difféomorphisme réciproque est
appelé le logarithme (naturel ou népérien). Pour t \textgreater{} 0 et α
∈ ℝ, on pose \{t\}\^{}\{α\} = \{e\}\^{}\{α\textbackslash{}mathop\{ log\}
t\} (la notation est cohérente pour α ∈ ℚ).On a alors (entre autres
propriétés)

\begin{itemize}
\item
  (i) \textbackslash{}mathop\{∀\}\{t\}\_\{1\},\{t\}\_\{2\} ∈{]}0,+∞{[},
  \textbackslash{}mathop\{log\} (\{t\}\_\{1\}\{t\}\_\{2\})
  =\textbackslash{}mathop\{ log\} \{t\}\_\{1\} +\textbackslash{}mathop\{
  log\} \{t\}\_\{2\}
\item
  (ii) \{ d \textbackslash{}over dt\} (\textbackslash{}mathop\{log\} t)
  =\{ 1 \textbackslash{}over t\}
\item
  (iii) On a également

  \textbackslash{}begin\{eqnarray*\} \textbackslash{}mathop\{∀\}α
  \textgreater{} 0, \&
  \{\textbackslash{}mathop\{lim\}\}\_\{t→+∞\}\{t\}\^{}\{−α\}\{e\}\^{}\{t\}
  = +∞,\textbackslash{}quad
  \{\textbackslash{}mathop\{lim\}\}\_\{t→+∞\}\{t\}\^{}\{−α\}\textbackslash{}mathop\{
  log\} t = 0\& \%\& \textbackslash{}\textbackslash{} \&
  \{\textbackslash{}mathop\{lim\}\}\_\{t→+∞\}\{t\}\^{}\{α\}\{e\}\^{}\{−t\}
  = 0,\textbackslash{}quad
  \{\textbackslash{}mathop\{lim\}\}\_\{t→0\}\{t\}\^{}\{α\}\textbackslash{}mathop\{
  log\} t = 0 \& \%\& \textbackslash{}\textbackslash{}
  \textbackslash{}end\{eqnarray*\}
\end{itemize}

Démonstration On a clairement \textbackslash{}mathop\{exp\} t
\textgreater{} 0 pour t ≥ 0~; pour t ≤ 0, on a
\textbackslash{}mathop\{exp\} t = \{(\textbackslash{}mathop\{exp\}
\textbar{}t\textbar{})\}\^{}\{−1\} \textgreater{} 0~; on en déduit que
\{ d \textbackslash{}over dt\} (\textbackslash{}mathop\{exp\} t)
=\textbackslash{}mathop\{ exp\} t \textgreater{} 0 pour tout t ∈ ℝ donc
t\textbackslash{}mathrel\{↦\}\textbackslash{}mathop\{exp\} t est un
\{C\}\^{}\{∞\} difféomorphisme croissant de ℝ sur son image. Mais, pour
t ≥ 0, on a \textbackslash{}mathop\{exp\} t =\{\textbackslash{}mathop\{
\textbackslash{}mathop\{∑ \}\} \}\_\{n=0\}\^{}\{+∞\}\{ \{t\}\^{}\{n\}
\textbackslash{}over n!\} ≥ 1 + t donc
\{\textbackslash{}mathop\{lim\}\}\_\{t→+∞\}\textbackslash{}mathop\{exp\}
t = +∞. Comme pour t \textless{} 0, \textbackslash{}mathop\{exp\} t =
\{(\textbackslash{}mathop\{exp\} \textbar{}t\textbar{})\}\^{}\{−1\}
\textgreater{} 0, on a
\{\textbackslash{}mathop\{lim\}\}\_\{t→−∞\}\textbackslash{}mathop\{exp\}
t = 0. Donc \textbackslash{}mathop\{exp\} (ℝ) ={]}0,+∞{[}. La cohérence
de la définition de \{t\}\^{}\{α\} =
\{e\}\^{}\{α\textbackslash{}mathop\{ log\} t\} est laissée aux soins du
lecteur ainsi que les propriétés évidentes de l'application
t\textbackslash{}mathrel\{↦\}\{t\}\^{}\{α\}.

(i) Comme \textbackslash{}mathop\{exp\} est un isomorphisme de (ℝ,+) sur
(\{ℝ\}\^{}\{∗⋅\},.), \textbackslash{}mathop\{log\} est également un
isomorphisme de groupes de (\{ℝ\}\^{}\{∗⋅\},.) sur (ℝ,+)

(ii) Le théorème sur la dérivation des fonctions réciproques montre que
\{ d \textbackslash{}over dt\} (\textbackslash{}mathop\{log\} t) =\{ 1
\textbackslash{}over \textbackslash{}mathop\{exp\}
'(\textbackslash{}mathop\{log\} t)\} =\{ 1 \textbackslash{}over
\textbackslash{}mathop\{exp\} \textbackslash{}mathop\{ log\} t\} =\{ 1
\textbackslash{}over t\} .

(iii) pour t \textgreater{} 0, on a \textbackslash{}mathop\{exp\} t
=\{\textbackslash{}mathop\{ \textbackslash{}mathop\{∑ \}\}
\}\_\{n=0\}\^{}\{+∞\}\{ \{t\}\^{}\{n\} \textbackslash{}over n!\} ≥\{
\{t\}\^{}\{N\} \textbackslash{}over N!\} . Soit α \textgreater{} 0, on a
alors pour N \textgreater{} α, \{t\}\^{}\{−α\}\textbackslash{}mathop\{
exp\} t \textgreater{}\{ \{t\}\^{}\{N−α\} \textbackslash{}over N!\} , ce
qui montre que
\{\textbackslash{}mathop\{lim\}\}\_\{t→+∞\}\{t\}\^{}\{−α\}\{e\}\^{}\{t\}
= +∞. En passant à l'inverse, on trouve
\{\textbackslash{}mathop\{lim\}\}\_\{t→∞\}\{t\}\^{}\{α\}\{e\}\^{}\{−t\}
= 0. Comme
\{\textbackslash{}mathop\{lim\}\}\_\{x→+∞\}\textbackslash{}mathop\{log\}
x = +∞, le théorème de composition des limites donne
\{\textbackslash{}mathop\{lim\}\}\_\{x→+∞\}\{(\textbackslash{}mathop\{log\}
x)\}\^{}\{−1∕α\}x = 0, soit encore
\{\textbackslash{}mathop\{lim\}\}\_\{x→+∞\}\{x\}\^{}\{−α\}\textbackslash{}mathop\{
log\} x = 0~; en changeant x en 1∕x, on obtient alors
\{\textbackslash{}mathop\{lim\}\}\_\{x→0\}\{x\}\^{}\{α\}\textbackslash{}mathop\{
log\} x = 0.

Etude sur ℝ des fonctions cosinus et sinus

On a pour t ∈ ℝ, \textbackslash{}overline\{\{e\}\^{}\{it\}\}
=\textbackslash{}mathop\{ exp\} (\textbackslash{}overline\{it\}) =
\{e\}\^{}\{−it\}. On a donc \textbackslash{}mathop\{cos\} t
=\textbackslash{}mathop\{
\textbackslash{}mathrm\{Re\}\}(\{e\}\^{}\{it\}) et
\textbackslash{}mathop\{sin\} t =\textbackslash{}mathop\{
\textbackslash{}mathrm\{Im\}\}(\{e\}\^{}\{it\}). On en déduit que
\textbackslash{}mathop\{cos\} t =\{\textbackslash{}mathop\{
\textbackslash{}mathop\{∑ \}\} \}\_\{n=0\}\^{}\{+∞\}\{(−1)\}\^{}\{n\}\{
\{t\}\^{}\{2n\} \textbackslash{}over (2n)!\} et
\textbackslash{}mathop\{sin\} t =\{\textbackslash{}mathop\{
\textbackslash{}mathop\{∑ \}\} \}\_\{n=0\}\^{}\{+∞\}\{(−1)\}\^{}\{n\}\{
\{t\}\^{}\{2n+1\} \textbackslash{}over (2n+1)!\} . Ces formules (ou
celles de définition) montrent immédiatement que
\textbackslash{}mathop\{cos\} ' = −\textbackslash{}mathop\{sin\} et
\textbackslash{}mathop\{sin\} ' =\textbackslash{}mathop\{ cos\} . On a,
pour t ∈{]}0,2{]}, \{ \textbackslash{}mathop\{sin\} t
\textbackslash{}over t\} =\{\textbackslash{}mathop\{
\textbackslash{}mathop\{∑ \}\} \}\_\{n=0\}\^{}\{+∞\}\{(−1)\}\^{}\{n\}\{
\{t\}\^{}\{2n\} \textbackslash{}over (2n+1)!\} qui est la somme d'une
série alternée à partir de n = 1 car

\{ \{ \{t\}\^{}\{2n+2\} \textbackslash{}over (2n+3)!\}
\textbackslash{}over \{ \{t\}\^{}\{2n\} \textbackslash{}over (2n+1)!\}
\} =\{ \{t\}\^{}\{2\} \textbackslash{}over (2n + 1)(2n + 3)\}
\textless{} 1

pour t ∈{]}0,2{]} et n ≥ 1. On en déduit (théorème sur l'encadrement des
sommes d'une série alternée) que \textbackslash{}mathop\{∀\}t
∈{]}0,2{]},\{ \textbackslash{}mathop\{sin\} t \textbackslash{}over t\} ≥
1 −\{ \{t\}\^{}\{2\} \textbackslash{}over 6\} \textgreater{} 0, donc
\textbackslash{}mathop\{sin\} t \textgreater{} 0. Comme
\textbackslash{}mathop\{cos\} ' = −\textbackslash{}mathop\{sin\} , la
fonction cosinus est strictement décroissante sur {[}0,2{]}. On a
\textbackslash{}mathop\{cos\} 0 = 1 \textgreater{} 0 et
\textbackslash{}mathop\{cos\} 2 =\{\textbackslash{}mathop\{
\textbackslash{}mathop\{∑ \}\} \}\_\{n=0\}\^{}\{+∞\}\{(−1)\}\^{}\{n\}\{
\{2\}\^{}\{2n\} \textbackslash{}over (2n)!\} qui est la somme d'une
série alternée à partir de n = 1 car

\{ \{ \{2\}\^{}\{2n+2\} \textbackslash{}over (2n+2)!\}
\textbackslash{}over \{ \{2\}\^{}\{2n\} \textbackslash{}over (2n)!\} \}
=\{ 4 \textbackslash{}over (2n + 1)(2n + 2)\} \textless{} 1

pour n ≥ 1. On en déduit (théorème sur l'encadrement des sommes d'une
série alternée) que

\textbackslash{}mathop\{cos\} 2 ≤ 1 −\{ 4 \textbackslash{}over 2!\} +\{
16 \textbackslash{}over 4!\} \textless{} 0

Le théorème des valeurs intermédiaires assure alors que la fonction
cosinus s'annule une et une seule fois sur {]}0,2{[} en un point α. On
posera π = 2α. On a donc \textbackslash{}mathop\{cos\} \{ π
\textbackslash{}over 2\} = 0 et \textbackslash{}mathop\{cos\} t
\textgreater{} 0 pour t ∈ {[}0,\{ π \textbackslash{}over 2\} {[}. Comme
\textbackslash{}mathop\{sin\} ' =\textbackslash{}mathop\{ cos\} ,
\textbackslash{}mathop\{sin\} est strictement croissante sur {[}0,\{ π
\textbackslash{}over 2\} {]}, comme \textbackslash{}mathop\{sin\} 0 = 0
et \{\textbackslash{}mathop\{cos\} \}\^{}\{2\}\{ π \textbackslash{}over
2\} +\{\textbackslash{}mathop\{ sin\} \}\^{}\{2\}\{ π
\textbackslash{}over 2\} = 1, on a donc \textbackslash{}mathop\{sin\} \{
π \textbackslash{}over 2\} = 1. Les formules d'addition donnent alors
\textbackslash{}mathop\{cos\} (x +\{ π \textbackslash{}over 2\} ) =
−\textbackslash{}mathop\{sin\} x, \textbackslash{}mathop\{sin\} (x +\{ π
\textbackslash{}over 2\} ) =\textbackslash{}mathop\{ cos\} x et on a
donc les variations suivantes

\begin{center}\rule{3in}{0.4pt}\end{center}

\begin{center}\rule{3in}{0.4pt}\end{center}

\begin{center}\rule{3in}{0.4pt}\end{center}

\begin{center}\rule{3in}{0.4pt}\end{center}

\begin{center}\rule{3in}{0.4pt}\end{center}

\begin{center}\rule{3in}{0.4pt}\end{center}

\begin{center}\rule{3in}{0.4pt}\end{center}

\begin{center}\rule{3in}{0.4pt}\end{center}

\begin{center}\rule{3in}{0.4pt}\end{center}

\begin{center}\rule{3in}{0.4pt}\end{center}

t

0

\{ π \textbackslash{}over 2\}

π

\{ 3π \textbackslash{}over 2\}

2π

\begin{center}\rule{3in}{0.4pt}\end{center}

\begin{center}\rule{3in}{0.4pt}\end{center}

\begin{center}\rule{3in}{0.4pt}\end{center}

\begin{center}\rule{3in}{0.4pt}\end{center}

\begin{center}\rule{3in}{0.4pt}\end{center}

\begin{center}\rule{3in}{0.4pt}\end{center}

\begin{center}\rule{3in}{0.4pt}\end{center}

\begin{center}\rule{3in}{0.4pt}\end{center}

\begin{center}\rule{3in}{0.4pt}\end{center}

\begin{center}\rule{3in}{0.4pt}\end{center}

\textbackslash{}mathop\{ cos\} t

1

↘

0

↘

− 1

↗

0

↗

1

\begin{center}\rule{3in}{0.4pt}\end{center}

\begin{center}\rule{3in}{0.4pt}\end{center}

\begin{center}\rule{3in}{0.4pt}\end{center}

\begin{center}\rule{3in}{0.4pt}\end{center}

\begin{center}\rule{3in}{0.4pt}\end{center}

\begin{center}\rule{3in}{0.4pt}\end{center}

\begin{center}\rule{3in}{0.4pt}\end{center}

\begin{center}\rule{3in}{0.4pt}\end{center}

\begin{center}\rule{3in}{0.4pt}\end{center}

\begin{center}\rule{3in}{0.4pt}\end{center}

\textbackslash{}mathop\{sin\} t

0

↗

1

↘

0

↘

− 1

↗

0

\begin{center}\rule{3in}{0.4pt}\end{center}

\begin{center}\rule{3in}{0.4pt}\end{center}

\begin{center}\rule{3in}{0.4pt}\end{center}

\begin{center}\rule{3in}{0.4pt}\end{center}

\begin{center}\rule{3in}{0.4pt}\end{center}

\begin{center}\rule{3in}{0.4pt}\end{center}

\begin{center}\rule{3in}{0.4pt}\end{center}

\begin{center}\rule{3in}{0.4pt}\end{center}

\begin{center}\rule{3in}{0.4pt}\end{center}

\begin{center}\rule{3in}{0.4pt}\end{center}

On a \textbackslash{}mathop\{cos\} (x + 2π) =\textbackslash{}mathop\{
cos\} x et \textbackslash{}mathop\{sin\} (x + 2π)
=\textbackslash{}mathop\{ sin\} x (de nouveau par les formules
d'addition). L'ensemble des périodes des fonctions
\textbackslash{}mathop\{cos\} et \textbackslash{}mathop\{sin\} (qui est
évidemment le même puisque chacune est au signe près la dérivée de
l'autre) est un sous-groupe fermé de ℝ donc de la forme aℤ avec donc a
=\{ 2π \textbackslash{}over k\} , où k ∈ ℤ~; le tableau de variation
montre clairement que k ne peut pas être strictement supérieur à 1~;
donc les deux fonctions sont périodiques de plus petite période 2π.

\paragraph{11.3.4 Nombres complexes de module 1}

Avec l'aimable autorisation de Hervé Pépin.

Théorème~11.3.6 L'application φ :
t\textbackslash{}mathrel\{↦\}\{e\}\^{}\{it\} est un morphisme surjectif
de groupes de (ℝ,+) sur (U,.) (où U désigne l'ensemble des nombres
complexes de module 1) dont le noyau est 2πℤ. Par factorisation
canonique, il définit un isomorphisme \textbackslash{}overline\{φ\} du
groupe quotient (ℝ∕2πℤ) sur le groupe (U,.).

Démonstration Il est clair que φ(\{t\}\_\{1\} + \{t\}\_\{2\}) =
φ(\{t\}\_\{1\})φ(\{t\}\_\{2\}), donc φ est un morphisme de groupes. Soit
z = a + ib un nombre complexe de module 1. On a a ∈ {[}−1,1{]}, donc il
existe t ∈ ℝ tel que \textbackslash{}mathop\{cos\} t = a. Mais alors
\textbackslash{}mathop\{sin\} t = 1 − \{a\}\^{}\{2\} = \{b\}\^{}\{2\} et
donc b vaut soit \textbackslash{}mathop\{sin\} t soit
\textbackslash{}mathop\{sin\} (−t). Dans les deux cas on a trouvé un x ∈
ℝ tel que z = \{e\}\^{}\{ix\} ce qui montre que φ est surjectif. On a
alors x ∈\textbackslash{}mathop\{\textbackslash{}mathrm\{Ker\}\}φ
\textbackslash{}mathrel\{⇔\} φ(x) = 1 \textbackslash{}mathrel\{⇔\}
\textbackslash{}mathop\{cos\} x = 1 \textbackslash{}mathrel\{⇔\} x ∈ 2πℤ
ce qui assure que
\textbackslash{}mathop\{\textbackslash{}mathrm\{Ker\}\}φ = 2πℤ.

Définition~11.3.6 Soit ζ ∈ ℂ, z\textbackslash{}mathrel\{≠\}0. On pose
alors \textbackslash{}mathop\{Arg\} z =\{
\textbackslash{}overline\{φ\}\}\^{}\{−1\}(\{ z \textbackslash{}over
\textbar{}z\textbar{}\} ) ∈ ℝ∕2πℤ (argument du nombre complexe z).

Remarque~11.3.5 Si \textbackslash{}mathop\{Arg\} z = θ + 2πℤ, on a par
définition z = \textbar{}z\textbar{}\{e\}\^{}\{iθ\}. On a bien entendu,
\textbackslash{}mathop\{Arg\}(\{z\}\_\{1\}\{z\}\_\{2\})
=\textbackslash{}mathop\{ Arg\} \{z\}\_\{1\} +\textbackslash{}mathop\{
Arg\} \{z\}\_\{2\}. On peut prendre un représentant
\textbackslash{}mathop\{arg\} z dans ℝ de \textbackslash{}mathop\{Arg\}
z (par exemple classiquement dans {]} − π,π{]}), mais alors en général
\textbackslash{}mathop\{arg\}(\{z\}\_\{1\}\{z\}\_\{2\})\textbackslash{}mathrel\{≠\}\textbackslash{}mathop\{arg\}
\{z\}\_\{1\} +\textbackslash{}mathop\{ arg\} \{z\}\_\{2\}.

Définition~11.3.7 Soit X un espace métrique et f : X → U une application
continue. On dit que f est relevable s'il existe φ : X → ℝ continue
telle que f = \{e\}\^{}\{iφ\}.

Proposition~11.3.7 Soit X un espace métrique et f : X → U une
application continue non surjective. Alors f est relevable.

Démonstration Soit α ∈ ℝ tel que
\{e\}\^{}\{iα\}\textbackslash{}mathrel\{∉\}f(X). L'application ω :{]}α,α
+ 2π{[}→ U ∖\textbackslash{}\{\{e\}\^{}\{iα\}\textbackslash{}\},
t\textbackslash{}mathrel\{↦\}\{e\}\^{}\{it\} est un homéomorphisme. Donc
φ = \{ω\}\^{}\{−1\} ∘ f convient.

Corollaire~11.3.8 Soit X un espace métrique et f,g : X → U telles que
\textbackslash{}mathop\{∀\}x ∈ X, \textbar{}f(x) − g(x)\textbar{}
\textless{} 2. Alors \{ f \textbackslash{}over g\} est relevable.

Démonstration On a \textbackslash{}mathop\{∀\}x ∈ X,\{ f(x)
\textbackslash{}over g(x)\} \textbackslash{}mathrel\{≠\} − 1 et donc \{
f \textbackslash{}over g\} n'est pas surjective.

Nous allons maintenant introduire un concept important en topologie,
l'homotopie.

Définition~11.3.8 Soit X et E deux espaces métriques, f,g : X → E. On
dit que f et g sont homotopes s'il existe F : X × {[}0,1{]} → E continue
telle que \textbackslash{}mathop\{∀\}x ∈ X, f(x) = F(x,0) et g(x) =
F(x,1).

Théorème~11.3.9 Soit X un espace métrique compact et f,g : X → U
homotopes. Alors \{ f \textbackslash{}over g\} est relevable et donc (f
relevable) \textbackslash{}mathrel\{⇔\} (g relevable).

Démonstration Soit F comme ci dessus. Alors F est continue sur X ×
{[}0,1{]} compact, donc uniformément continue. Soit donc η
\textgreater{} 0 tel que \textbar{}t − t'\textbar{} \textless{} η
⇒\textbackslash{}mathop\{∀\}x ∈ X,\textbar{}F(x,t) − F(x,t')\textbar{}
\textless{} 2 et soit \{t\}\_\{0\} =
1,\textbackslash{}mathop\{\textbackslash{}mathop\{\ldots{}\}\},\{t\}\_\{n\}
= 1 une subdivision de {[}0,1{]} de pas plus petit que η. Alors, si
\{f\}\_\{i\}(x) = F(x,\{t\}\_\{i\}), on a \textbackslash{}mathop\{∀\}x ∈
X, \textbar{}\{f\}\_\{i+1\}(x) − \{f\}\_\{i\}(x)\textbar{} \textless{} 2
donc \{ \{f\}\_\{i\} \textbackslash{}over \{f\}\_\{i+1\}\} est
relevable. Mais alors \{ f \textbackslash{}over g\} =\{ \{f\}\_\{0\}
\textbackslash{}over \{f\}\_\{n\}\} =\{\textbackslash{}mathop\{
\textbackslash{}mathop\{∏ \}\} \}\_\{i=0\}\^{}\{n−1\}\{ \{f\}\_\{i\}
\textbackslash{}over \{f\}\_\{i+1\}\} est relevable.

Corollaire~11.3.10 Soit f : {[}a,b{]} → U une application continue.
Alors f est relevable.

Démonstration f est évidemment homotope à l'application constante
x\textbackslash{}mathrel\{↦\}f(a) qui est relevable (car non surjective)
par F(x,t) = f(a + t(x − a)).

Corollaire~11.3.11 Soit I un intervalle de ℝ et f : I → U continue.
Alors f est relevable.

\paragraph{11.3.5 Fonctions classiques}

Fonctions d'une variable complexe

Lemme~11.3.12 La fonction z\textbackslash{}mathrel\{↦\}\{ 1
\textbackslash{}over 1−z\} est développable en série entière dans le
disque D(0,1) (R = 1).

Démonstration On a \{ 1 \textbackslash{}over 1−z\} = 1 + z +
\textbackslash{}mathrel\{⋯\} + \{z\}\^{}\{n\} +\{ \{z\}\^{}\{n+1\}
\textbackslash{}over 1−z\} ce dernier terme tendant vers 0 quand n tend
vers + ∞ si \textbar{}z\textbar{} \textless{} 1. On en déduit que pour
\textbar{}z\textbar{} \textless{} 1, \{ 1 \textbackslash{}over 1−z\}
=\{\textbackslash{}mathop\{ \textbackslash{}mathop\{∑ \}\}
\}\_\{n=0\}\^{}\{+∞\}\{z\}\^{}\{n\} avec un rayon de convergence
évidemment égal à 1 par la règle de d'Alembert.

Proposition~11.3.13 Pour \textbar{}z\textbar{} \textless{} 1, on a \{ 1
\textbackslash{}over \{(1−z)\}\^{}\{k+1\}\} =\{\textbackslash{}mathop\{
\textbackslash{}mathop\{∑ \}\}
\}\_\{n=0\}\^{}\{+∞\}\{C\}\_\{n+k\}\^{}\{k\}\{z\}\^{}\{n\} avec un rayon
de convergence égal à 1.

Démonstration Soit z\textbackslash{}mathrel\{≠\}0~; le lemme précédent
montre que pour t ∈{]} −\{ 1 \textbackslash{}over
\textbar{}z\textbar{}\} ,\{ 1 \textbackslash{}over
\textbar{}z\textbar{}\} {[}, on a \{ 1 \textbackslash{}over 1−tz\}
=\{\textbackslash{}mathop\{ \textbackslash{}mathop\{∑ \}\}
\}\_\{n=0\}\^{}\{+∞\}\{z\}\^{}\{n\}\{t\}\^{}\{n\}. Dérivons k fois par
rapport à t cette série entière en t. On obtient donc

\{ \{z\}\^{}\{k\}k! \textbackslash{}over \{(1 − tz)\}\^{}\{k+1\}\} =\{
\textbackslash{}mathop\{∑ \}\}\_\{n=k\}\^{}\{+∞\}\{z\}\^{}\{n\}\{ n!
\textbackslash{}over (n − k)!\} \{t\}\^{}\{n−k\} =\{
\textbackslash{}mathop\{∑ \}\}\_\{n=0\}\^{}\{+∞\}\{z\}\^{}\{n+k\}\{ (n +
k)! \textbackslash{}over n!\} \{t\}\^{}\{n\}

après le changement de n en n + k. Si \textbar{}z\textbar{} \textless{}
1, en divisant les deux membres par \{z\}\^{}\{k\}k! et en faisant t = 1
∈{]} −\{ 1 \textbackslash{}over \textbar{}z\textbar{}\} ,\{ 1
\textbackslash{}over \textbar{}z\textbar{}\} {[}, on obtient la formule
désirée. Il est clair que le rayon de convergence de la série obtenue
est 1 (règle de d'Alembert).

Corollaire~11.3.14 Soit R(z) =\{ P(z) \textbackslash{}over Q(z)\} une
fraction rationnelle à coefficients complexes n'admettant pas 0 pour
pôle et soit ρ =\textbackslash{}mathop\{
min\}\textbackslash{}\{\textbar{}α\textbar{}\textbackslash{}mathrel\{∣\}α\textbackslash{}text\{
pôle de \}R\textbackslash{}\}. Alors R est développable en série entière
dans le disque D(0,ρ).

Démonstration Par décomposition en éléments simples sur le corps des
nombres complexes, il suffit de montrer que si α est un pôle de R(z), un
élément simple de la forme \{ 1 \textbackslash{}over
\{(z−α)\}\^{}\{k\}\} est développable en série entière dans le disque
D(0,ρ). Mais

\{ 1 \textbackslash{}over \{(z − α)\}\^{}\{k\}\} =\{ \{(−1)\}\^{}\{k\}
\textbackslash{}over \{α\}\^{}\{k\}\} \{ 1 \textbackslash{}over \{(1 −\{
z \textbackslash{}over α\} )\}\^{}\{k\}\} =\{ \{(−1)\}\^{}\{k\}
\textbackslash{}over \{α\}\^{}\{k\}\} \{ \textbackslash{}mathop\{∑
\}\}\_\{n=0\}\^{}\{+∞\}\{C\}\_\{ n+k−1\}\^{}\{k−1\}\{ \{z\}\^{}\{n\}
\textbackslash{}over \{α\}\^{}\{n\}\}

avec un rayon de convergence égal à \textbar{}α\textbar{}≥ ρ.

Remarque~11.3.6 On montre facilement que le rayon de convergence est
exactement égal à ρ, puisque la somme de la série entière ne peut
admettre un prolongement continu en un pôle de la fraction rationnelle.

Fonctions d'une variable réelle

Lemme~11.3.15 Soit α ∈ ℝ. La fonction t\textbackslash{}mathrel\{↦\}\{(1
+ t)\}\^{}\{α\} est développable en série entière dans {]} − 1,1{[} si
α\textbackslash{}mathrel\{∉\}ℕ, dans ℝ si α ∈ ℕ.

Démonstration Le résultat est évident si α ∈ ℕ puisqu'alors la fonction
est polynomiale. Nous supposerons donc α\textbackslash{}mathrel\{∉\}ℕ.
Considérons la série entière 1 +\{\textbackslash{}mathop\{
\textbackslash{}mathop\{∑ \}\} \}\_\{n≥1\}\{
α(α−1)\textbackslash{}mathop\{\textbackslash{}mathop\{\ldots{}\}\}(α−n+1)
\textbackslash{}over n!\} \{t\}\^{}\{n\}. On a \{
α(α−1)\textbackslash{}mathop\{\textbackslash{}mathop\{\ldots{}\}\}(α−n+1)
\textbackslash{}over n!\} \textbackslash{}mathrel\{≠\}0 et

\textbackslash{}mathop\{lim\}\{ \{
α(α−1)\textbackslash{}mathop\{\textbackslash{}mathop\{\ldots{}\}\}(α−n)
\textbackslash{}over (n+1)!\} \textbackslash{}over \{
α(α−1)\textbackslash{}mathop\{\textbackslash{}mathop\{\ldots{}\}\}(α−n+1)
\textbackslash{}over n!\} \} =\textbackslash{}mathop\{ lim\}\{ α − n
\textbackslash{}over n + 1\} = −1

Donc son rayon de convergence est 1. Posons donc, pour t ∈{]} − 1,1{[},
S(t) = 1 +\{\textbackslash{}mathop\{ \textbackslash{}mathop\{∑ \}\}
\}\_\{n=1\}\^{}\{+∞\}\{
α(α−1)\textbackslash{}mathop\{\textbackslash{}mathop\{\ldots{}\}\}(α−n+1)
\textbackslash{}over n!\} \{t\}\^{}\{n\}. On a

\textbackslash{}begin\{eqnarray*\} S'(t)\& =\&
\{\textbackslash{}mathop\{∑ \}\}\_\{n=1\}\^{}\{+∞\}\{ α(α −
1)\textbackslash{}mathop\{\ldots{}\}(α − n + 1) \textbackslash{}over (n
− 1)!\} \{t\}\^{}\{n−1\}\%\& \textbackslash{}\textbackslash{} \& =\&
\{\textbackslash{}mathop\{∑ \}\}\_\{n=0\}\^{}\{+∞\}\{ α(α −
1)\textbackslash{}mathop\{\ldots{}\}(α − n) \textbackslash{}over n!\}
\{t\}\^{}\{n\} \%\& \textbackslash{}\textbackslash{}
\textbackslash{}end\{eqnarray*\}

après un changement d'indice. On en déduit que

\textbackslash{}begin\{eqnarray*\} (1 + t)S'(t) = S'(t) + tS'(t)\&\&
\%\& \textbackslash{}\textbackslash{} \& =\& α +\{
\textbackslash{}mathop\{∑ \}\}\_\{n=1\}\^{}\{+∞\}\textbackslash{}left
(\{ α(α − 1)\textbackslash{}mathop\{\ldots{}\}(α − n)
\textbackslash{}over n!\} +\{ α(α −
1)\textbackslash{}mathop\{\ldots{}\}(α − n + 1) \textbackslash{}over (n
− 1)!\} \textbackslash{}right )\{t\}\^{}\{n\}\%\&
\textbackslash{}\textbackslash{} \textbackslash{}end\{eqnarray*\}

en utilisant d'abord la seconde expression de S'(t) puis la première.
Mais

\textbackslash{}begin\{eqnarray*\}\{ α(α −
1)\textbackslash{}mathop\{\textbackslash{}mathop\{\ldots{}\}\}(α − n)
\textbackslash{}over n!\} +\{ α(α −
1)\textbackslash{}mathop\{\textbackslash{}mathop\{\ldots{}\}\}(α − n +
1) \textbackslash{}over (n − 1)!\} \&\& \%\&
\textbackslash{}\textbackslash{} \& =\&\{ α(α −
1)\textbackslash{}mathop\{\textbackslash{}mathop\{\ldots{}\}\}(α − n +
1) \textbackslash{}over (n − 1)!\} \textbackslash{}left (\{ α − n
\textbackslash{}over n\} + 1\textbackslash{}right )\%\&
\textbackslash{}\textbackslash{} \& =\&\{ α(α −
1)\textbackslash{}mathop\{\textbackslash{}mathop\{\ldots{}\}\}(α − n +
1) \textbackslash{}over (n − 1)!\} \{ α \textbackslash{}over n\} \%\&
\textbackslash{}\textbackslash{} \textbackslash{}end\{eqnarray*\}

On en déduit que

(1 + t)S'(t) = α + α\{\textbackslash{}mathop\{∑
\}\}\_\{n=1\}\^{}\{+∞\}\{ α(α − 1)\textbackslash{}mathop\{\ldots{}\}(α −
n + 1) \textbackslash{}over n!\} \{t\}\^{}\{n\} = αS(t)

Alors

\{ d \textbackslash{}over dt\} \textbackslash{}left (\{ S(t)
\textbackslash{}over \{(1 + t)\}\^{}\{α\}\} \textbackslash{}right ) =\{
(1 + t)S'(t) − αS(t) \textbackslash{}over \{(1 + t)\}\^{}\{α+1\}\} = 0

donc la fonction est constante sur {]} − 1,1{[}. Comme elle vaut 1 au
point 0, elle est constamment égale à 1 et donc
\textbackslash{}mathop\{∀\}t ∈{]} − 1,1{[}, \{(1 + t)\}\^{}\{α\} = S(t)
= 1 +\{\textbackslash{}mathop\{ \textbackslash{}mathop\{∑ \}\}
\}\_\{n=1\}\^{}\{+∞\}\{
α(α−1)\textbackslash{}mathop\{\textbackslash{}mathop\{\ldots{}\}\}(α−n+1)
\textbackslash{}over n!\} \{t\}\^{}\{n\}.

On obtient ainsi facilement des développements en série entière de \{(1
+ t)\}\^{}\{−1\}, \{(1 + \{t\}\^{}\{2\})\}\^{}\{−1\}, \{(1 −
\{t\}\^{}\{2\})\}\^{}\{−1\}, \{(1 − \{t\}\^{}\{2\})\}\^{}\{−1∕2\}, \{(1
+ \{t\}\^{}\{2\})\}\^{}\{−1∕2\}, puis par intégration des développements
de \textbackslash{}mathop\{log\} (1 + t),
\textbackslash{}mathop\{\textbackslash{}mathrm\{arctg\}\} t,
\textbackslash{}mathop\{arg\}
\textbackslash{}mathop\{\textbackslash{}mathrm\{th\}\} t,
\textbackslash{}mathop\{arcsin\} t et \textbackslash{}mathop\{arg\}
\textbackslash{}mathop\{\textbackslash{}mathrm\{sh\}\} t. En
récapitulant, on obtient la table suivante de développements en série
entière

\textbackslash{}begin\{eqnarray*\} \{e\}\^{}\{t\}\& =\&
\{\textbackslash{}mathop\{∑ \}\}\_\{n=0\}\^{}\{+∞\}\{ \{t\}\^{}\{n\}
\textbackslash{}over n!\} ,\textbackslash{}quad R = +∞ \%\&
\textbackslash{}\textbackslash{} \textbackslash{}mathop\{cos\} t\& =\&
\{\textbackslash{}mathop\{∑ \}\}\_\{n=0\}\^{}\{+∞\}\{(−1)\}\^{}\{n\}\{
\{t\}\^{}\{2n\} \textbackslash{}over (2n)!\} ,\textbackslash{}quad R =
+∞ \%\& \textbackslash{}\textbackslash{} \textbackslash{}mathop\{sin\}
t\& =\& \{\textbackslash{}mathop\{∑
\}\}\_\{n=0\}\^{}\{+∞\}\{(−1)\}\^{}\{n\}\{ \{t\}\^{}\{2n+1\}
\textbackslash{}over (2n + 1)!\} ,\textbackslash{}quad R = +∞ \%\&
\textbackslash{}\textbackslash{}
\textbackslash{}mathop\{\textbackslash{}mathrm\{ch\}\} t\& =\&
\{\textbackslash{}mathop\{∑ \}\}\_\{n=0\}\^{}\{+∞\}\{ \{t\}\^{}\{2n\}
\textbackslash{}over (2n)!\} ,\textbackslash{}quad R = +∞ \%\&
\textbackslash{}\textbackslash{}
\textbackslash{}mathop\{\textbackslash{}mathrm\{sh\}\} t\& =\&
\{\textbackslash{}mathop\{∑ \}\}\_\{n=0\}\^{}\{+∞\}\{ \{t\}\^{}\{2n+1\}
\textbackslash{}over (2n + 1)!\} ,\textbackslash{}quad R = +∞ \%\&
\textbackslash{}\textbackslash{} \{(1 + t)\}\^{}\{α\}\& =\& 1 +\{
\textbackslash{}mathop\{∑ \}\}\_\{n=1\}\^{}\{+∞\}\{ α(α −
1)\textbackslash{}mathop\{\ldots{}\}(α − n + 1) \textbackslash{}over
n!\} \{t\}\^{}\{n\},\textbackslash{}quad R = 1\textbackslash{}text\{ ou
\} + ∞\%\& \textbackslash{}\textbackslash{} \{ 1 \textbackslash{}over 1
+ t\} \& =\& \{\textbackslash{}mathop\{∑
\}\}\_\{n=0\}\^{}\{+∞\}\{(−1)\}\^{}\{n\}\{t\}\^{}\{n\},\textbackslash{}quad
R = 1 \%\& \textbackslash{}\textbackslash{} \{ 1 \textbackslash{}over 1
− t\} \& =\& \{\textbackslash{}mathop\{∑
\}\}\_\{n=0\}\^{}\{+∞\}\{t\}\^{}\{n\},\textbackslash{}quad R = 1 \%\&
\textbackslash{}\textbackslash{} \textbackslash{}mathop\{log\} (1 + t)\&
=\& \{\textbackslash{}mathop\{∑
\}\}\_\{n=1\}\^{}\{+∞\}\{(−1)\}\^{}\{n−1\}\{ \{t\}\^{}\{n\}
\textbackslash{}over n\} ,\textbackslash{}quad R = 1 \%\&
\textbackslash{}\textbackslash{} \textbackslash{}mathop\{log\} (1 − t)\&
=\& −\{\textbackslash{}mathop\{∑ \}\}\_\{n=1\}\^{}\{+∞\}\{
\{t\}\^{}\{n\} \textbackslash{}over n\} ,\textbackslash{}quad R = 1 \%\&
\textbackslash{}\textbackslash{}
\textbackslash{}mathop\{\textbackslash{}mathrm\{arctg\}\} t\& =\&
\{\textbackslash{}mathop\{∑ \}\}\_\{n=0\}\^{}\{+∞\}\{(−1)\}\^{}\{n\}\{
\{t\}\^{}\{2n+1\} \textbackslash{}over 2n + 1\} ,\textbackslash{}quad R
= 1 \%\& \textbackslash{}\textbackslash{} \textbackslash{}mathop\{arg\}
\textbackslash{}mathop\{\textbackslash{}mathrm\{th\}\} t\& =\&
\{\textbackslash{}mathop\{∑ \}\}\_\{n=0\}\^{}\{+∞\}\{ \{t\}\^{}\{2n+1\}
\textbackslash{}over 2n + 1\} ,\textbackslash{}quad R = 1 \%\&
\textbackslash{}\textbackslash{} \textbackslash{}mathop\{arcsin\} t\&
=\& t +\{ \textbackslash{}mathop\{∑ \}\}\_\{n=1\}\^{}\{+∞\}\{
1.3\textbackslash{}mathop\{\ldots{}\}(2n − 1) \textbackslash{}over
2.4\textbackslash{}mathop\{\ldots{}\}(2n)\} \{ \{t\}\^{}\{2n+1\}
\textbackslash{}over 2n + 1\} ,\textbackslash{}quad R = 1 \%\&
\textbackslash{}\textbackslash{} \textbackslash{}mathop\{arg\}
\textbackslash{}mathop\{\textbackslash{}mathrm\{sh\}\} t\& =\& t +\{
\textbackslash{}mathop\{∑ \}\}\_\{n=1\}\^{}\{+∞\}\{(−1)\}\^{}\{n\}\{
1.3\textbackslash{}mathop\{\ldots{}\}(2n − 1) \textbackslash{}over
2.4\textbackslash{}mathop\{\ldots{}\}(2n)\} \{ \{t\}\^{}\{2n+1\}
\textbackslash{}over 2n + 1\} ,\textbackslash{}quad R = 1 \%\&
\textbackslash{}\textbackslash{} \textbackslash{}end\{eqnarray*\}

\paragraph{11.3.6 Méthodes de sommation}

Le problème est ici, étant donné une série entière
\textbackslash{}mathop\{\textbackslash{}mathop\{∑ \}\}
\{a\}\_\{n\}\{t\}\^{}\{n\}, de reconnaître une fonction exprimable avec
des fonctions classiques dont c'est le développement en série entière.
Bien entendu, on utilise toutes les méthodes inverses des méthodes ci
dessus (combinaisons linéaires, produits, changement de variable,
dérivation, intégration, équations différentielles). Nous décrirons ici
seulement deux cas classiques qui se traitent avec des méthodes
spécifiques.

Séries entières \textbackslash{}mathop\{\textbackslash{}mathop\{∑ \}\}
P(n)\{t\}\^{}\{n\}, où P est un polynôme Le rayon de convergence est
évidemment égal à 1. On a vu que

\{ 1 \textbackslash{}over \{(1 − t)\}\^{}\{k+1\}\} =\{
\textbackslash{}mathop\{∑ \}\}\_\{n=0\}\^{}\{+∞\}\{C\}\_\{
n+k\}\^{}\{k\}\{t\}\^{}\{n\} =\{ \textbackslash{}mathop\{∑
\}\}\_\{n=0\}\^{}\{+∞\}\{ (n + k)(n + k −
1)\textbackslash{}mathop\{\ldots{}\}(n + 1) \textbackslash{}over k!\}
\{t\}\^{}\{n\}

Posons donc \{P\}\_\{0\}(X) = 1 et

\{P\}\_\{k\}(X) =\{ (X + k)(X + k −
1)\textbackslash{}mathop\{\textbackslash{}mathop\{\ldots{}\}\}(X + 1)
\textbackslash{}over k!\}

pour k ≥ 1 et soit d =\textbackslash{}mathop\{ deg\} P. La famille
(\{P\}\_\{0\},\{P\}\_\{1\},\textbackslash{}mathop\{\textbackslash{}mathop\{\ldots{}\}\},\{P\}\_\{d\})
est une base de \{ℝ\}\_\{d\}{[}X{]} (espace vectoriel des polynômes de
degré inférieur ou égal à d) car elle est échelonnée en degré. On peut
donc écrire P(X) = \{λ\}\_\{0\}\{P\}\_\{0\}(X) +
\{λ\}\_\{1\}\{P\}\_\{1\}(X) +
\textbackslash{}mathop\{\textbackslash{}mathop\{\ldots{}\}\} +
\{λ\}\_\{d\}\{P\}\_\{d\}(X) et alors, pour t ∈{]} − 1,1{[},

\{\textbackslash{}mathop\{∑ \}\}\_\{n=0\}\^{}\{+∞\}P(n)\{t\}\^{}\{n\}
=\{ \{λ\}\_\{0\} \textbackslash{}over 1 − t\} +\{ \{λ\}\_\{1\}
\textbackslash{}over \{(1 − t)\}\^{}\{2\}\} +
\textbackslash{}mathop\{\ldots{}\} +\{ \{λ\}\_\{d\} \textbackslash{}over
\{(1 − t)\}\^{}\{d+1\}\}

Remarque~11.3.7 Cette méthode s'étend à des séries entières du type
\textbackslash{}mathop\{\textbackslash{}mathop\{∑ \}\} \{ P(n)
\textbackslash{}over n+k\} \{t\}\^{}\{n\} où k est un entier. Il suffit
en effet de multiplier par \{t\}\^{}\{k\} et de dériver pour tomber sur
une série du type précédent. On peut ensuite, par décomposition en
éléments simples sommer les séries du type
\textbackslash{}mathop\{\textbackslash{}mathop\{∑ \}\} \{ P(n)
\textbackslash{}over
(n+\{k\}\_\{1\})\textbackslash{}mathop\{\textbackslash{}mathop\{\ldots{}\}\}(n+\{k\}\_\{m\})\}
\{t\}\^{}\{n\} où
\{k\}\_\{1\},\textbackslash{}mathop\{\textbackslash{}mathop\{\ldots{}\}\},\{k\}\_\{m\}
sont des entiers distincts.

Séries entières \textbackslash{}mathop\{\textbackslash{}mathop\{∑ \}\}
P(n)\{ \{t\}\^{}\{n\} \textbackslash{}over n!\} , où P est un polynôme
Le rayon de convergence est évidemment égal à + ∞. On remarque ici que
\{\textbackslash{}mathop\{\textbackslash{}mathop\{∑ \}\}
\}\_\{n=0\}\^{}\{+∞\}n(n −
1)\textbackslash{}mathop\{\textbackslash{}mathop\{\ldots{}\}\}(n − k +
1)\{ \{t\}\^{}\{n\} \textbackslash{}over n!\}
=\{\textbackslash{}mathop\{ \textbackslash{}mathop\{∑ \}\}
\}\_\{n=k\}\^{}\{+∞\}\{ \{t\}\^{}\{n\} \textbackslash{}over (n−k)!\} =
\{t\}\^{}\{k\}\{e\}\^{}\{t\}. Posons donc \{P\}\_\{0\}(X) = 1 et
\{P\}\_\{k\}(X) = X(X −
1)\textbackslash{}mathop\{\textbackslash{}mathop\{\ldots{}\}\}(X − k +
1) pour k ≥ 1 et soit d =\textbackslash{}mathop\{ deg\} P. La famille
(\{P\}\_\{0\},\{P\}\_\{1\},\textbackslash{}mathop\{\textbackslash{}mathop\{\ldots{}\}\},\{P\}\_\{d\})
est une base de \{ℝ\}\_\{d\}{[}X{]} (espace vectoriel des polynômes de
degré inférieur ou égal à d) car elle est échelonnée en degré. On peut
donc écrire P(X) = \{λ\}\_\{0\}\{P\}\_\{0\}(X) +
\{λ\}\_\{1\}\{P\}\_\{1\}(X) +
\textbackslash{}mathop\{\textbackslash{}mathop\{\ldots{}\}\} +
\{λ\}\_\{d\}\{P\}\_\{d\}(X) et alors, pour t ∈ ℝ, on a

\{\textbackslash{}mathop\{∑ \}\}\_\{n=0\}\^{}\{+∞\}P(n)\{ \{t\}\^{}\{n\}
\textbackslash{}over n!\} = (\{λ\}\_\{0\} + \{λ\}\_\{1\}t +
\textbackslash{}mathop\{\ldots{}\} +
\{λ\}\_\{d\}\{t\}\^{}\{d\})\{e\}\^{}\{t\}

{[}\href{coursse66.html}{next}{]} {[}\href{coursse64.html}{prev}{]}
{[}\href{coursse64.html\#tailcoursse64.html}{prev-tail}{]}
{[}\href{coursse65.html}{front}{]}
{[}\href{coursch12.html\#coursse65.html}{up}{]}

\end{document}
