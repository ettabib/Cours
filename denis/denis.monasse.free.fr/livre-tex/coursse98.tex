\documentclass[]{article}
\usepackage[T1]{fontenc}
\usepackage{lmodern}
\usepackage{amssymb,amsmath}
\usepackage{ifxetex,ifluatex}
\usepackage{fixltx2e} % provides \textsubscript
% use upquote if available, for straight quotes in verbatim environments
\IfFileExists{upquote.sty}{\usepackage{upquote}}{}
\ifnum 0\ifxetex 1\fi\ifluatex 1\fi=0 % if pdftex
  \usepackage[utf8]{inputenc}
\else % if luatex or xelatex
  \ifxetex
    \usepackage{mathspec}
    \usepackage{xltxtra,xunicode}
  \else
    \usepackage{fontspec}
  \fi
  \defaultfontfeatures{Mapping=tex-text,Scale=MatchLowercase}
  \newcommand{\euro}{€}
\fi
% use microtype if available
\IfFileExists{microtype.sty}{\usepackage{microtype}}{}
\ifxetex
  \usepackage[setpagesize=false, % page size defined by xetex
              unicode=false, % unicode breaks when used with xetex
              xetex]{hyperref}
\else
  \usepackage[unicode=true]{hyperref}
\fi
\hypersetup{breaklinks=true,
            bookmarks=true,
            pdfauthor={},
            pdftitle={Probl`emes classiques sur les courbes},
            colorlinks=true,
            citecolor=blue,
            urlcolor=blue,
            linkcolor=magenta,
            pdfborder={0 0 0}}
\urlstyle{same}  % don't use monospace font for urls
\setlength{\parindent}{0pt}
\setlength{\parskip}{6pt plus 2pt minus 1pt}
\setlength{\emergencystretch}{3em}  % prevent overfull lines
\setcounter{secnumdepth}{0}
 
/* start css.sty */
.cmr-5{font-size:50%;}
.cmr-7{font-size:70%;}
.cmmi-5{font-size:50%;font-style: italic;}
.cmmi-7{font-size:70%;font-style: italic;}
.cmmi-10{font-style: italic;}
.cmsy-5{font-size:50%;}
.cmsy-7{font-size:70%;}
.cmex-7{font-size:70%;}
.cmex-7x-x-71{font-size:49%;}
.msbm-7{font-size:70%;}
.cmtt-10{font-family: monospace;}
.cmti-10{ font-style: italic;}
.cmbx-10{ font-weight: bold;}
.cmr-17x-x-120{font-size:204%;}
.cmsl-10{font-style: oblique;}
.cmti-7x-x-71{font-size:49%; font-style: italic;}
.cmbxti-10{ font-weight: bold; font-style: italic;}
p.noindent { text-indent: 0em }
td p.noindent { text-indent: 0em; margin-top:0em; }
p.nopar { text-indent: 0em; }
p.indent{ text-indent: 1.5em }
@media print {div.crosslinks {visibility:hidden;}}
a img { border-top: 0; border-left: 0; border-right: 0; }
center { margin-top:1em; margin-bottom:1em; }
td center { margin-top:0em; margin-bottom:0em; }
.Canvas { position:relative; }
li p.indent { text-indent: 0em }
.enumerate1 {list-style-type:decimal;}
.enumerate2 {list-style-type:lower-alpha;}
.enumerate3 {list-style-type:lower-roman;}
.enumerate4 {list-style-type:upper-alpha;}
div.newtheorem { margin-bottom: 2em; margin-top: 2em;}
.obeylines-h,.obeylines-v {white-space: nowrap; }
div.obeylines-v p { margin-top:0; margin-bottom:0; }
.overline{ text-decoration:overline; }
.overline img{ border-top: 1px solid black; }
td.displaylines {text-align:center; white-space:nowrap;}
.centerline {text-align:center;}
.rightline {text-align:right;}
div.verbatim {font-family: monospace; white-space: nowrap; text-align:left; clear:both; }
.fbox {padding-left:3.0pt; padding-right:3.0pt; text-indent:0pt; border:solid black 0.4pt; }
div.fbox {display:table}
div.center div.fbox {text-align:center; clear:both; padding-left:3.0pt; padding-right:3.0pt; text-indent:0pt; border:solid black 0.4pt; }
div.minipage{width:100%;}
div.center, div.center div.center {text-align: center; margin-left:1em; margin-right:1em;}
div.center div {text-align: left;}
div.flushright, div.flushright div.flushright {text-align: right;}
div.flushright div {text-align: left;}
div.flushleft {text-align: left;}
.underline{ text-decoration:underline; }
.underline img{ border-bottom: 1px solid black; margin-bottom:1pt; }
.framebox-c, .framebox-l, .framebox-r { padding-left:3.0pt; padding-right:3.0pt; text-indent:0pt; border:solid black 0.4pt; }
.framebox-c {text-align:center;}
.framebox-l {text-align:left;}
.framebox-r {text-align:right;}
span.thank-mark{ vertical-align: super }
span.footnote-mark sup.textsuperscript, span.footnote-mark a sup.textsuperscript{ font-size:80%; }
div.tabular, div.center div.tabular {text-align: center; margin-top:0.5em; margin-bottom:0.5em; }
table.tabular td p{margin-top:0em;}
table.tabular {margin-left: auto; margin-right: auto;}
div.td00{ margin-left:0pt; margin-right:0pt; }
div.td01{ margin-left:0pt; margin-right:5pt; }
div.td10{ margin-left:5pt; margin-right:0pt; }
div.td11{ margin-left:5pt; margin-right:5pt; }
table[rules] {border-left:solid black 0.4pt; border-right:solid black 0.4pt; }
td.td00{ padding-left:0pt; padding-right:0pt; }
td.td01{ padding-left:0pt; padding-right:5pt; }
td.td10{ padding-left:5pt; padding-right:0pt; }
td.td11{ padding-left:5pt; padding-right:5pt; }
table[rules] {border-left:solid black 0.4pt; border-right:solid black 0.4pt; }
.hline hr, .cline hr{ height : 1px; margin:0px; }
.tabbing-right {text-align:right;}
span.TEX {letter-spacing: -0.125em; }
span.TEX span.E{ position:relative;top:0.5ex;left:-0.0417em;}
a span.TEX span.E {text-decoration: none; }
span.LATEX span.A{ position:relative; top:-0.5ex; left:-0.4em; font-size:85%;}
span.LATEX span.TEX{ position:relative; left: -0.4em; }
div.float img, div.float .caption {text-align:center;}
div.figure img, div.figure .caption {text-align:center;}
.marginpar {width:20%; float:right; text-align:left; margin-left:auto; margin-top:0.5em; font-size:85%; text-decoration:underline;}
.marginpar p{margin-top:0.4em; margin-bottom:0.4em;}
.equation td{text-align:center; vertical-align:middle; }
td.eq-no{ width:5%; }
table.equation { width:100%; } 
div.math-display, div.par-math-display{text-align:center;}
math .texttt { font-family: monospace; }
math .textit { font-style: italic; }
math .textsl { font-style: oblique; }
math .textsf { font-family: sans-serif; }
math .textbf { font-weight: bold; }
.partToc a, .partToc, .likepartToc a, .likepartToc {line-height: 200%; font-weight:bold; font-size:110%;}
.chapterToc a, .chapterToc, .likechapterToc a, .likechapterToc, .appendixToc a, .appendixToc {line-height: 200%; font-weight:bold;}
.index-item, .index-subitem, .index-subsubitem {display:block}
.caption td.id{font-weight: bold; white-space: nowrap; }
table.caption {text-align:center;}
h1.partHead{text-align: center}
p.bibitem { text-indent: -2em; margin-left: 2em; margin-top:0.6em; margin-bottom:0.6em; }
p.bibitem-p { text-indent: 0em; margin-left: 2em; margin-top:0.6em; margin-bottom:0.6em; }
.paragraphHead, .likeparagraphHead { margin-top:2em; font-weight: bold;}
.subparagraphHead, .likesubparagraphHead { font-weight: bold;}
.quote {margin-bottom:0.25em; margin-top:0.25em; margin-left:1em; margin-right:1em; text-align:\\jmathmathustify;}
.verse{white-space:nowrap; margin-left:2em}
div.maketitle {text-align:center;}
h2.titleHead{text-align:center;}
div.maketitle{ margin-bottom: 2em; }
div.author, div.date {text-align:center;}
div.thanks{text-align:left; margin-left:10%; font-size:85%; font-style:italic; }
div.author{white-space: nowrap;}
.quotation {margin-bottom:0.25em; margin-top:0.25em; margin-left:1em; }
h1.partHead{text-align: center}
.sectionToc, .likesectionToc {margin-left:2em;}
.subsectionToc, .likesubsectionToc {margin-left:4em;}
.subsubsectionToc, .likesubsubsectionToc {margin-left:6em;}
.frenchb-nbsp{font-size:75%;}
.frenchb-thinspace{font-size:75%;}
.figure img.graphics {margin-left:10%;}
/* end css.sty */

\title{Probl`emes classiques sur les courbes}
\author{}
\date{}

\begin{document}
\maketitle

\textbf{Warning: 
requires JavaScript to process the mathematics on this page.\\ If your
browser supports JavaScript, be sure it is enabled.}

\begin{center}\rule{3in}{0.4pt}\end{center}

{[}
{[}
{[}{]}
{[}

\subsubsection{18.3 Problèmes classiques sur les courbes}

\paragraph{18.3.1 Tra\\jmathmathectoires orthogonales}

Ce paragraphe ne fait pas partie du programme des classes préparatoires.

Soit E un espace euclidien.

Soit (\Gamma_\lambda~)_\lambda~\inJ une famille d'arcs paramétrés indexée
par un intervalle J de \mathbb{R}~ où \Gamma_\lambda~ = (I,F_\lambda~). On posera
encore F_\lambda~(t) = F(t,\lambda~) et on supposera que F : I \times J \rightarrow~ E est de
classe \mathcal{C}^1.

Donnons nous un arc paramétré (K,G) qui rencontre tous les \Gamma_\lambda~.
Le point u \in K de G est donc le point t(u) de l'arc \Gamma_\lambda~(u) si
bien que G(u) = F_\lambda~(u)(t(u)) = F(t(u),\lambda~(u)). On obtient ainsi
un arc paramétré u\mapsto~F(t(u),\lambda~(u)). On dira que
c'est une tra\\jmathmathectoire orthogonale de la famille (\Gamma_\lambda~) si cet
arc est orthogonal à l'arc \Gamma_\lambda~(u) au point t(u). On obtient
donc la définition suivante~:

Définition~18.3.1 On appelle tra\\jmathmathectoire orthogonale des arcs
\Gamma_\lambda~ tout arc paramétré (K,G) de la forme G(u) = F(t(u),\lambda~(u)),
où u\mapsto~t(u) est une application de classe
\mathcal{C}^1 de K dans I et u\mapsto~\lambda~(u) une
application de classe \mathcal{C}^1 de K dans J telles que
\forall~u \in K, F_\lambda~(u)~'(t(u)) \bot G'(u)

La recherche s'effectue en remarquant que F_\lambda~'(t) = \partial~F
\over \partial~t (t,\lambda~), soit F_\lambda~(u)'(t(u)) = \partial~F
\over \partial~t (t(u),\lambda~(u)) et que G'(u) = dt
\over du (u) \partial~F \over \partial~t (t(u),\lambda~(u))
+ d\lambda~ \over du (u) \partial~F \over \partial~\lambda~
(t(u),\lambda~(u)). La condition F_\lambda~(u)'(t(u),\lambda~(u)) \bot G'(u) s'écrit
donc

\left ( \partial~F \over \partial~t
(t(u),\lambda~(u))∣ dt \over du
(u) \partial~F \over \partial~t (t(u),\lambda~(u)) + d\lambda~
\over du (u) \partial~F \over \partial~\lambda~
(t(u),\lambda~(u))\right ) = 0

soit encore, en termes de formes différentielles

\left ( \partial~F \over \partial~t
(t,\lambda~)∣ \partial~F \over \partial~t (t,\lambda~)
dt + \partial~F \over \partial~\lambda~ (t,\lambda~) d\lambda~\right ) = 0

qui conduit à une équation différentielle reliant t et \lambda~.

Exemples~: prenons la famille de paraboles y = x^2 + \lambda~, \lambda~ \in
\mathbb{R}~. On peut les paramétrer par F(t,\lambda~) = (t,t^2 + \lambda~). On a
alors  \partial~F \over \partial~t (t,\lambda~) = (1,2t) et  \partial~F
\over \partial~\lambda~ (t,\lambda~) = (0,1). L'équation ci dessus s'écrit
encore \ \partial~F \over \partial~t
(t,\lambda~)\^2 dt + \left
( \partial~F \over \partial~t (t,\lambda~)∣ \partial~F
\over \partial~\lambda~ (t,\lambda~)\right ) d\lambda~ = 0, soit ici
(1 + 4t^2) dt + 2t d\lambda~ = 0. Il s'agit d'une équation à
variable séparable. Elle s'écrit encore (2t + 1 \over
2t )dt = -d\lambda~. On trouve donc \lambda~ = -t^2 - 1
\over 2  log~
t + \lambda_0, soit (t,t^2 + \lambda~) =
(t,- 1 \over 2  log~
t + \lambda_0) et donc les tra\\jmathmathectoires
orthogonales sont équivalentes aux arcs
t\mapsto~(t,- 1 \over 2
 log t + \lambda_0~). En fait la division
par t nous a fait perdre une solution évidente t = 0 correspondant à
l'axe Oy.

\paragraph{18.3.2 Inverse d'une courbe}

Ce paragraphe ne fait pas partie du programme des classes préparatoires.

Rappelons que si E est un espace affine euclidien, on appelle inversion
de pôle O \in E l'application de E \diagdown\O\
dans lui même qui à M \in E \diagdown\O\ associe
l'unique point M' défini par

\begin{itemize}
\itemsep1pt\parskip0pt\parsep0pt
\item
  (i) O,M et M' sont alignés
\item
  (ii) \overlineOM.\overlineOM' =
  1
\end{itemize}

On vérifie immédiatement que M' = O +
\overrightarrowOM \over
\\overrightarrowOM\^2
.

Etant donné un arc (I,F) de E dont l'image est contenue dans E
\diagdown\O\, on peut alors définir son
inverse de pôle O~; c'est l'arc (I,G) tel que, pour tout t \in I, G(t)
soit l'inverse de F(t).

Supposons que E soit un plan euclidien rapporté à un repère orthonormé
(O,\vec\imath,\vecȷ) et que
\overrightarrowOF(t) = x(t)\vec\imath +
y(t)\vecȷ. On a alors
\overrightarrowOG(t) = X(t)\vec\imath +
Y (t)\vecȷ avec

X(t) = x(t) \over x(t)^2 +
y(t)^2 ,\quad Y (t) = y(t)
\over x(t)^2 + y(t)^2

Si \Gamma = (I,F) est donné en polaires par l'équation \rho = f(\theta), on a
\overrightarrowOF(\theta) =
f(\theta)\vecu(\theta) et alors
\overrightarrowOG(\theta) = 1 \over
f(\theta) \vecu(\theta), si bien que l'inverse est donnée par
l'équation polaire \rho = 1 \over f(\theta) .

Exemple~18.3.1 Inverse des coniques de foyer O. On a vu qu'une telle
conique admettait une équation polaire \rho = p \over
1+e cos (\theta-\theta_0)~ . L'inverse d'une
telle conique est donc une courbe d'équation polaire \rho =
1+e cos (\theta-\theta_0~) \over
p , soit encore (à une rotation près autour de l'origine) \rho = a(1 +
ecos~ \theta). On obtient la famille des
lima\ccons de Pascal (avec le cas particulier de la
cardioïde, pour e = 1, qui est l'inverse d'une parabole par rapport à
son foyer).

\paragraph{18.3.3 Podaire d'une courbe}

Ce paragraphe ne fait pas partie du programme des classes préparatoires.

Soit E un espace affine euclidien, (I,F) un arc paramétré régulier de E
et A un point de E.

Définition~18.3.2 On appelle podaire de l'arc (I,F) par rapport au point
A l'arc (I,G) où pour chaque t \in I, G(t) est la pro\\jmathmathection orthogonale
de A sur la tangente au point t à l'arc (I,F).

Comme cette tangente est définie comme F(t) + \mathbb{R}~F'(t), il suffit donc
d'exprimer que la famille
(F'(t),\overrightarrowF(t)G(t)) est liée et que
\overrightarrowAG(t) \bot F'(t).

Supposons que E soit un plan euclidien rapporté à un repère orthonormé
(O,\vec\imath,\vecȷ) et que
\overrightarrowOF(t) = x(t)\vec\imath +
y(t)\vecȷ. Posons
\overrightarrowOA = a\vec\imath +
b\vecȷ et \overrightarrowOG(t) =
X(t)\vec\imath + Y (t)\vecȷ. On doit
donc écrire

\left
\matrix\,X(t) - x(t)&x'(t)
\cr Y (t) - y(t)&y'(t)\right 
= 0\text et (X(t) - a)x'(t) + (Y (t) - b)y'(t) = 0

ce qui conduit à un système de Cramer aux inconnues X(t) et Y (t)

\left
\\matrix\,y'(t)X(t) -
x'(t)Y (t) = y'(t)x(t) - x'(t)y(t) \cr X(t)x'(t) + Y
(t)y'(t) = ax'(t) + by'(t)\right .

Exemple~18.3.2 Recherchons la podaire d'un cercle par rapport à un point
du plan. On choisissant convenablement le repère, on peut supposer que
le cercle est paramétré par t\mapsto~(a +
Rcos t,R\sin~ t) avec
a ≥ 0 et R \textgreater{} 0 et que le point A a pour coordonnées (0,0).
Le système ci dessus devient alors (après simplification par R)

\left
\\matrix\,cos~
tX(t) + sin~ tY (t) =
acos~ t + R \cr
-sin tX(t) +\ cos~ tY
(t) = 0\right .

d'où l'on déduit X(t) = (acos~ t +
R)cos~ t et Y (t) =
(acos t + R)\sin~ t.
On en déduit que la podaire est la courbe d'équation polaire \rho =
acos~ \theta + R. Il s'agit d'un
lima\ccon de Pascal (évidemment dégénéré en un cercle
si a = 0 c'est-à-dire si le cercle de départ est centré en A). On trouve
une cardioïde lorsque a = R, c'est-à-dire lorsque le cercle passe par A.

\paragraph{18.3.4 Conchoïdes d'une courbe}

Ce paragraphe ne fait pas partie du programme des classes préparatoires.

Soit \Gamma = (I,F) un arc paramétré d'un plan euclidien E, O un point de E
n'appartenant pas à l'image de \Gamma et a \textgreater{} 0. On associe à \Gamma
les arcs \Gamma_1 = (I,F_1) et \Gamma_2 =
(I,F_2), appelés conchoïdes de centre O pour la longueur a, où
F_i(t) est défini pour i
\in\1,2\ par

\begin{itemize}
\itemsep1pt\parskip0pt\parsep0pt
\item
  (i) O,F(t) et F_i(t) sont alignés
\item
  (ii) la distance de F(t) à F_i(t) est égale à a.
\end{itemize}

Supposons que E soit un plan euclidien rapporté à un repère orthonormé
(O,\vec\imath,\vecȷ) et que
\overrightarrowOF(t) = x(t)\vec\imath +
y(t)\vecȷ. On a alors
\overrightarrowOF_i(t) =
X_i(t)\vec\imath + Y
_i(t)\vecȷ avec

X(t) = x(t) ± a x(t) \over
\sqrtx(t)^2  + y(t)^2
,\quad Y (t) = y(t) ± a y(t) \over
\sqrtx(t)^2  + y(t)^2

Si \Gamma = (I,F) est donné en polaires par l'équation \rho = f(\theta), on a
\overrightarrowOF(\theta) =
f(\theta)\vecu(\theta) et alors
\overrightarrowOF_i(\theta) = (f(\theta) ±
a)\vecu(\theta), si bien que les conchoïdes sont donnés
par l'équation polaire \rho = f(\theta) ± a.

Exemple~18.3.3 Conchoïdes d'un cercle par rapport à l'un de ses points.
En choisissant convenablement le repère d'origine 0, le cercle a pour
équation polaire \rho = 2Rcos~ \theta si bien que les
deux conchoïdes ont pour équation polaire \rho =
2Rcos~ \theta ± a. Remarquons que les deux courbes
d'équations polaires \rho = 2Rcos~ \theta + a et \rho =
2Rcos~ \theta - a se déduisent l'une de l'autre par
le changement de (\rho,\theta) en (-\rho,\theta + \pi~) ce qui redonne le même point
géométrique. Elles ont donc la même image. Ce sont des
lima\ccons de Pascal, le cas de la cardioïde
correspondant à a = 2R (la longueur a est égale au diamètre du cercle).

{[}
{[}
{[}
{[}

\end{document}
