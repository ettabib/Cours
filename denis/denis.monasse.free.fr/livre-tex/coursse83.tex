\documentclass[]{article}
\usepackage[T1]{fontenc}
\usepackage{lmodern}
\usepackage{amssymb,amsmath}
\usepackage{ifxetex,ifluatex}
\usepackage{fixltx2e} % provides \textsubscript
% use upquote if available, for straight quotes in verbatim environments
\IfFileExists{upquote.sty}{\usepackage{upquote}}{}
\ifnum 0\ifxetex 1\fi\ifluatex 1\fi=0 % if pdftex
  \usepackage[utf8]{inputenc}
\else % if luatex or xelatex
  \ifxetex
    \usepackage{mathspec}
    \usepackage{xltxtra,xunicode}
  \else
    \usepackage{fontspec}
  \fi
  \defaultfontfeatures{Mapping=tex-text,Scale=MatchLowercase}
  \newcommand{\euro}{€}
\fi
% use microtype if available
\IfFileExists{microtype.sty}{\usepackage{microtype}}{}
\ifxetex
  \usepackage[setpagesize=false, % page size defined by xetex
              unicode=false, % unicode breaks when used with xetex
              xetex]{hyperref}
\else
  \usepackage[unicode=true]{hyperref}
\fi
\hypersetup{breaklinks=true,
            bookmarks=true,
            pdfauthor={},
            pdftitle={Differentielle},
            colorlinks=true,
            citecolor=blue,
            urlcolor=blue,
            linkcolor=magenta,
            pdfborder={0 0 0}}
\urlstyle{same}  % don't use monospace font for urls
\setlength{\parindent}{0pt}
\setlength{\parskip}{6pt plus 2pt minus 1pt}
\setlength{\emergencystretch}{3em}  % prevent overfull lines
\setcounter{secnumdepth}{0}
 
/* start css.sty */
.cmr-5{font-size:50%;}
.cmr-7{font-size:70%;}
.cmmi-5{font-size:50%;font-style: italic;}
.cmmi-7{font-size:70%;font-style: italic;}
.cmmi-10{font-style: italic;}
.cmsy-5{font-size:50%;}
.cmsy-7{font-size:70%;}
.cmex-7{font-size:70%;}
.cmex-7x-x-71{font-size:49%;}
.msbm-7{font-size:70%;}
.cmtt-10{font-family: monospace;}
.cmti-10{ font-style: italic;}
.cmbx-10{ font-weight: bold;}
.cmr-17x-x-120{font-size:204%;}
.cmsl-10{font-style: oblique;}
.cmti-7x-x-71{font-size:49%; font-style: italic;}
.cmbxti-10{ font-weight: bold; font-style: italic;}
p.noindent { text-indent: 0em }
td p.noindent { text-indent: 0em; margin-top:0em; }
p.nopar { text-indent: 0em; }
p.indent{ text-indent: 1.5em }
@media print {div.crosslinks {visibility:hidden;}}
a img { border-top: 0; border-left: 0; border-right: 0; }
center { margin-top:1em; margin-bottom:1em; }
td center { margin-top:0em; margin-bottom:0em; }
.Canvas { position:relative; }
li p.indent { text-indent: 0em }
.enumerate1 {list-style-type:decimal;}
.enumerate2 {list-style-type:lower-alpha;}
.enumerate3 {list-style-type:lower-roman;}
.enumerate4 {list-style-type:upper-alpha;}
div.newtheorem { margin-bottom: 2em; margin-top: 2em;}
.obeylines-h,.obeylines-v {white-space: nowrap; }
div.obeylines-v p { margin-top:0; margin-bottom:0; }
.overline{ text-decoration:overline; }
.overline img{ border-top: 1px solid black; }
td.displaylines {text-align:center; white-space:nowrap;}
.centerline {text-align:center;}
.rightline {text-align:right;}
div.verbatim {font-family: monospace; white-space: nowrap; text-align:left; clear:both; }
.fbox {padding-left:3.0pt; padding-right:3.0pt; text-indent:0pt; border:solid black 0.4pt; }
div.fbox {display:table}
div.center div.fbox {text-align:center; clear:both; padding-left:3.0pt; padding-right:3.0pt; text-indent:0pt; border:solid black 0.4pt; }
div.minipage{width:100%;}
div.center, div.center div.center {text-align: center; margin-left:1em; margin-right:1em;}
div.center div {text-align: left;}
div.flushright, div.flushright div.flushright {text-align: right;}
div.flushright div {text-align: left;}
div.flushleft {text-align: left;}
.underline{ text-decoration:underline; }
.underline img{ border-bottom: 1px solid black; margin-bottom:1pt; }
.framebox-c, .framebox-l, .framebox-r { padding-left:3.0pt; padding-right:3.0pt; text-indent:0pt; border:solid black 0.4pt; }
.framebox-c {text-align:center;}
.framebox-l {text-align:left;}
.framebox-r {text-align:right;}
span.thank-mark{ vertical-align: super }
span.footnote-mark sup.textsuperscript, span.footnote-mark a sup.textsuperscript{ font-size:80%; }
div.tabular, div.center div.tabular {text-align: center; margin-top:0.5em; margin-bottom:0.5em; }
table.tabular td p{margin-top:0em;}
table.tabular {margin-left: auto; margin-right: auto;}
div.td00{ margin-left:0pt; margin-right:0pt; }
div.td01{ margin-left:0pt; margin-right:5pt; }
div.td10{ margin-left:5pt; margin-right:0pt; }
div.td11{ margin-left:5pt; margin-right:5pt; }
table[rules] {border-left:solid black 0.4pt; border-right:solid black 0.4pt; }
td.td00{ padding-left:0pt; padding-right:0pt; }
td.td01{ padding-left:0pt; padding-right:5pt; }
td.td10{ padding-left:5pt; padding-right:0pt; }
td.td11{ padding-left:5pt; padding-right:5pt; }
table[rules] {border-left:solid black 0.4pt; border-right:solid black 0.4pt; }
.hline hr, .cline hr{ height : 1px; margin:0px; }
.tabbing-right {text-align:right;}
span.TEX {letter-spacing: -0.125em; }
span.TEX span.E{ position:relative;top:0.5ex;left:-0.0417em;}
a span.TEX span.E {text-decoration: none; }
span.LATEX span.A{ position:relative; top:-0.5ex; left:-0.4em; font-size:85%;}
span.LATEX span.TEX{ position:relative; left: -0.4em; }
div.float img, div.float .caption {text-align:center;}
div.figure img, div.figure .caption {text-align:center;}
.marginpar {width:20%; float:right; text-align:left; margin-left:auto; margin-top:0.5em; font-size:85%; text-decoration:underline;}
.marginpar p{margin-top:0.4em; margin-bottom:0.4em;}
.equation td{text-align:center; vertical-align:middle; }
td.eq-no{ width:5%; }
table.equation { width:100%; } 
div.math-display, div.par-math-display{text-align:center;}
math .texttt { font-family: monospace; }
math .textit { font-style: italic; }
math .textsl { font-style: oblique; }
math .textsf { font-family: sans-serif; }
math .textbf { font-weight: bold; }
.partToc a, .partToc, .likepartToc a, .likepartToc {line-height: 200%; font-weight:bold; font-size:110%;}
.chapterToc a, .chapterToc, .likechapterToc a, .likechapterToc, .appendixToc a, .appendixToc {line-height: 200%; font-weight:bold;}
.index-item, .index-subitem, .index-subsubitem {display:block}
.caption td.id{font-weight: bold; white-space: nowrap; }
table.caption {text-align:center;}
h1.partHead{text-align: center}
p.bibitem { text-indent: -2em; margin-left: 2em; margin-top:0.6em; margin-bottom:0.6em; }
p.bibitem-p { text-indent: 0em; margin-left: 2em; margin-top:0.6em; margin-bottom:0.6em; }
.paragraphHead, .likeparagraphHead { margin-top:2em; font-weight: bold;}
.subparagraphHead, .likesubparagraphHead { font-weight: bold;}
.quote {margin-bottom:0.25em; margin-top:0.25em; margin-left:1em; margin-right:1em; text-align:\jmathustify;}
.verse{white-space:nowrap; margin-left:2em}
div.maketitle {text-align:center;}
h2.titleHead{text-align:center;}
div.maketitle{ margin-bottom: 2em; }
div.author, div.date {text-align:center;}
div.thanks{text-align:left; margin-left:10%; font-size:85%; font-style:italic; }
div.author{white-space: nowrap;}
.quotation {margin-bottom:0.25em; margin-top:0.25em; margin-left:1em; }
h1.partHead{text-align: center}
.sectionToc, .likesectionToc {margin-left:2em;}
.subsectionToc, .likesubsectionToc {margin-left:4em;}
.subsubsectionToc, .likesubsubsectionToc {margin-left:6em;}
.frenchb-nbsp{font-size:75%;}
.frenchb-thinspace{font-size:75%;}
.figure img.graphics {margin-left:10%;}
/* end css.sty */

\title{Differentielle}
\author{}
\date{}

\begin{document}
\maketitle

\textbf{Warning: 
requires JavaScript to process the mathematics on this page.\\ If your
browser supports JavaScript, be sure it is enabled.}

\begin{center}\rule{3in}{0.4pt}\end{center}

{[}
{[}
{[}{]}
{[}

\subsubsection{15.2 Différentielle}

\paragraph{15.2.1 Applications différentiables}

Définition~15.2.1 Soit E et F deux espaces vectoriels normés, U un
ouvert de E, a \in U et f : U \rightarrow~ F. On dit que f est différentiable au
point a s'il existe une application linéaire continue L : E \rightarrow~ F telle
que, pour h voisin de 0,

f(a + h) = f(a) + L(h) +
o(\\textbar{}h\\textbar{})

Dans ce cas, l'application L est unique et est appelée la différentielle
de f au point a, notée df(a) ou encore d\_af.

Démonstration Supposons que L\_1 et L\_2 conviennent.
Par différence, on a L\_1(h) - L\_2(h) =
o(\\textbar{}h\\textbar{}). On a donc,
pour x \in E \diagdown\0\

lim\_t\rightarrow~0,t\textgreater{}0L\_1~(tx)
- L\_2(tx)\over
\\textbar{}tx\\textbar{} = 0

Mais pour t \textgreater{} 0, on a
L\_1(tx)-L\_2(tx)\over
\\textbar{}tx\\textbar{} =
L\_1(x)-L\_2(x)\over
\\textbar{}x\\textbar{} ~; ceci montre
que L\_1(x) = L\_2(x) et donc L\_1 =
L\_2.

Remarque~15.2.1 Pour alléger les notations, on écrira df(a).h à la place
de \big {[}df(a)\big {]}(h). On a donc par
définition f(a + h) = f(a) + df(a).h +
o(\\textbar{}h\\textbar{}) ou encore f(a +
h) = f(a) + df(a).h +\\textbar{}
h\\textbar{}\epsilon(h) avec
lim\_h\rightarrow~0~\epsilon(h) = 0.

Remarque~15.2.2 Si E est de dimension finie, une application linéaire de
E dans F est automatiquement continue. Il est clair d'autre part que la
différentiabilité est une notion locale et que le changement des normes
sur E et F en normes équivalentes ne change ni la différentiabilité, ni
la différentielle.

Proposition~15.2.1 Si f est différentiable au point a, elle est continue
au point a.

Démonstration On a f(a + h) = f(a) + df(a).h +\\textbar{}
h\\textbar{}\epsilon(h) avec
lim\_h\rightarrow~0~\epsilon(h) = 0. Comme df(a) est une
application linéaire continue, on a
lim\_h\rightarrow~0~df(a).h = df(a).0 = 0 et donc
lim\_h\rightarrow~0~f(a + h) = f(a).

\paragraph{15.2.2 Exemples d'applications différentiables}

Proposition~15.2.2 Soit E et F deux espaces vectoriels normés, u une
application linéaire continue de E dans F. Alors u est différentiable en
tout point a de E et du(a) = u.

Démonstration On a en effet u(a + h) = u(a) + u(h) + 0.

Proposition~15.2.3 Soit E, F et G trois espaces vectoriels normés, u : E
\times F \rightarrow~ G une application bilinéaire continue. Alors f est différentiable
en tout point (a,b) de E \times F et du(a,b).(h,k) = u(a,k) + u(h,b).

Démonstration On a u((a,b) + (h,k)) = u(a + h,b + k) = u(a,b) +
\left (u(a,k) + u(h,b)\right ) + u(h,k).
Mais comme u est bilinéaire continue, il existe une constante A telle
que \\textbar{}u(h,k)\\textbar{} \leq
A\\textbar{}h\\textbar{}\\textbar{}k\\textbar{}
soit encore \\textbar{}u(h,k)\\textbar{} \leq
Amax(\\textbar{}h\\textbar{},\\textbar{}k\\textbar{})^2~
=
A\\textbar{}(h,k)\\textbar{}^2.
On a donc u((a,b) + (h,k)) = u(a,b) + \left (u(a,k) +
u(h,b)\right ) +
O(\\textbar{}(h,k)\\textbar{}^2).
Comme (h,k)\mapsto~u(a,k) + u(h,b) est clairement
linéaire et continue, c'est la différentielle de u au point (a,b).

Exemple~15.2.1 Tous les produits usuels (produit dans K, produits
scalaires, produits vectoriels, produits matriciels) sont donc
différentiables en tout point.

\paragraph{15.2.3 Opérations sur les différentielles}

Proposition~15.2.4 Soit E et F deux espaces vectoriels normés, U un
ouvert de E, a \in U et f,g : U \rightarrow~ F. Si f et g sont différentiables en a,
il en est de même pour \alpha~f + \beta~g et d(\alpha~f + \beta~g)(a) = \alpha~df(a) + \beta~dg(a).

Démonstration On a f(a + h) = f(a) + df(a).h +
o(\\textbar{}h\\textbar{}) et g(a + h) =
g(a) + dg(a).h +
o(\\textbar{}h\\textbar{}), d'où (\alpha~f +
\beta~g)(a + h) = (\alpha~f + \beta~g)(a) + \alpha~df(a).h + \beta~dg(a).h +
o(\\textbar{}h\\textbar{}) avec \alpha~df(a) +
\beta~dg(a) application linéaire continue.

Théorème~15.2.5 Soit E, F et G trois espaces vectoriels normés, U un
ouvert de E, V un ouvert de F, f : U \rightarrow~ F tel que f(U) \subset~ V et g : V \rightarrow~ G.
Soit a \in U. Si f est différentiable au point a et g différentiable au
point f(a), alors g \cdot f est différentiable au point a et d(g \cdot f)(a) =
dg\left (f(a)\right ) \cdot df(a).

Démonstration On a, en posant b = f(a), f(a + h) = f(a) + df(a).h
+\\textbar{} h\\textbar{}\epsilon(h) avec
lim\_h\rightarrow~0~\epsilon(h) = 0 et g(b + k) = g(b) +
dg(b).k +\\textbar{} k\\textbar{}\eta(k) avec
lim\_k\rightarrow~0~\eta(k) = 0. Prenons en
particulier

k = \phi(h) = f(a + h) - f(a) = df(a).h +\\textbar{}
h\\textbar{}\epsilon(h)

On a b + k = f(a + h), et donc

g(f(a + h)) = g(f(a)) + dg(b).\phi(h) +\\textbar{}
\phi(h)\\textbar{}\eta(\phi(h))

Mais on a

\begin{align*} dg(b).\phi(h)& =&
dg(b).\left (df(a).h +\\textbar{}
h\\textbar{}\epsilon(h)\right ) \%&
\\ & =& dg(b) \cdot df(a).h
+\\textbar{} h\\textbar{}dg(b).\epsilon(h)\%&
\\ & =& dg(b) \cdot df(a).h +
o(\\textbar{}h\\textbar{}) \%&
\\ \end{align*}

puisque lim\_h\rightarrow~0~dg(b).\epsilon(h) = dg(b).0 =
0. D'autre part,

\begin{align*}
\\textbar{}\phi(h)\\textbar{}& \leq&
\\textbar{}df(a).h +\\textbar{}
h\\textbar{}\epsilon(h)\\textbar{} \%&
\\ & \leq&
\\textbar{}df(a)\\textbar{}\,\\textbar{}h\\textbar{}
+\\textbar{}
\epsilon(h)\\textbar{}\,\\textbar{}h\\textbar{}
= O(\\textbar{}h\\textbar{})\%&
\\ \end{align*}

donc \\textbar{}\phi(h)\\textbar{}\eta(\phi(h)) =
o(\\textbar{}h\\textbar{}) puisque
lim\_h\rightarrow~0~\phi(h) = 0 (continuité de f au
point a) et donc lim\_h\rightarrow~0~\eta(\phi(h)) = 0.
On a donc en définitive, g(f(a + h)) = g(f(a)) + dg(b) \cdot df(a).h +
o(\\textbar{}h\\textbar{}) ce qui termine
la démonstration.

Remarque~15.2.3 En particulier, si u est une application linéaire
continue et si f est différentiable, u \cdot f est différentiable et d(u \cdot
f)(a) = u \cdot df(a).

\paragraph{15.2.4 Différentielle et dérivées partielles}

Regardons tout d'abord le cas des fonctions d'une variable. On a un
résultat très simple qui montre que la notion de différentiabilité est
une généralisation de la notion de dérivabilité.

Théorème~15.2.6 Soit U un ouvert de \mathbb{R}~, a \in U et F un K-espace vectoriel
normé. Soit f : U \rightarrow~ F. Alors f est différentiable au point a si et
seulement si~elle est dérivable au point a et on a f'(a) = df(a).1 et
df(a).h = hf'(a).

Démonstration Si f est dérivable au point a, on a f(a + h) = f(a) +
hf'(a) + o(h), ce qui montre que f est différentiable en a et que
df(a).h = hf'(a). Inversement si f est différentiable au point a, on a
f(a + h) = f(a) + df(a).h + o(h) = f(a) + hdf(a).1 + o(h) (car h est un
réel), soit encore lim\_h\rightarrow~0~
f(a+h)-f(a) \over h = df(a).1, ce qui montre que f est
dérivable au point 1 et que f'(a) = df(a).1.

Exemple~15.2.2 Soit U un ouvert de \mathbb{R}~, V un ouvert de E, soit \phi : U \rightarrow~ E
telle que \phi(U) \subset~ V et f : V \rightarrow~ F. Soit a \in U. Supposons que \phi est
dérivable (donc différentiable) au point a et que f est différentiable
au point \phi(a). Alors f \cdot \phi est différentiable (donc dérivable) au point
a et (f \cdot \phi)'(a) = d(f \cdot \phi)(a).1 = df(\phi(a)) \cdot d\phi(a).1 = df(\phi(a)).\phi'(a).
On retiendra donc la formule importante (f \cdot \phi)'(a) = df(\phi(a)).\phi'(a).

En ce qui concerne les fonctions de plusieurs variables, le lien entre
différentiabilité et dérivée partielle est plus complexe puisque l'on a
vu que l'existence de dérivées partielles n'impliquait même pas la
continuité, et donc certainement pas la différentiabilité. On a le
résultat suivant

Théorème~15.2.7 Soit E et F deux espaces vectoriels normés de dimensions
finies, U un ouvert de E et f : U \rightarrow~ F. (i) si f est différentiable au
point a \in U, alors f admet en a une dérivée partielle suivant tout
vecteur v et \partial~\_vf(a) = df(a).v (ii) inversement, si E =
\mathbb{R}~^n et si f est de classe \mathcal{C}^1 sur U, alors f est
différentiable en tout point a de U et df(a).h
= \\sum ~
\_i=1^nh\_i \partial~f \over
\partial~x\_i (a).

Démonstration (i) On a f(a + h) = f(a) + df(a).h
+\\textbar{} h\\textbar{}\epsilon(h), soit encore
f(a + tv) = f(a) + tdf(a).v +
\textbar{}t\textbar{}\,\\textbar{}v\\textbar{}\epsilon(tv),
c'est-à-dire lim\_t\rightarrow~0~ f(a+tv)-f(a)
\over t = df(a).v, donc f admet en a une dérivée
partielle suivant v et \partial~\_vf(a) = df(a).v.

(ii) La formule de Taylor Young à l'ordre 1 montre en effet que f(a + h)
= f(a) + \\sum ~
\_i=1^nh\_i \partial~f \over
\partial~x\_i (a) +
o(\\textbar{}h\\textbar{}) ce qui montre
que f est différentiable en a et que df(a).h =\
\sum  \_i=1^nh\_i~ \partial~f
\over \partial~x\_i (a).

Remarque~15.2.4 Si E = \mathbb{R}~^n, et si f est différentiable au
point a, on a nécessairement

\begin{align*} df(a).h& =&
df(a).(\sum \_i=1^nh~\_
ie\_i) = \\sum
\_i=1^nh\_ idf(a).e\_i\%&
\\ & =& \\sum
\_i=1^nh\_ i\partial~\_e\_if(a) =
\sum \_i=1^nh\_ i~ \partial~f
\over \partial~x\_i (a) \%&
\\ \end{align*}

Donc en fait montrer la différentiabilité de f en a, c'est montrer que
les  \partial~f \over \partial~x\_i (a) existent et que f(a +
h) - f(a) -\\sum ~
\_i=1^nh\_i \partial~f \over
\partial~x\_i (a) =
o(\\textbar{}h\\textbar{}).

\paragraph{15.2.5 Matrices \jmathacobiennes, \jmathacobiens}

Définition~15.2.2 Soit U un ouvert de \mathbb{R}~^n et f : U \rightarrow~
\mathbb{R}~^p. Soit a \in U tel que f soit différentiable au point a. On
appelle matrice \jmathacobienne de f au point a la matrice J\_f(a) de
l'application linéaire df(a) dans les bases canoniques de \mathbb{R}~^n
et \mathbb{R}~^p. Si
f(x\_1,\\ldots,x\_n~)
=
(f\_1(x\_1,\\ldots,x\_n),\\\ldots,f\_p(x\_1,\\\ldots,x\_n~)),
c'est la matrice

\begin{align*} J\_f(a) =
\left ( \partial~f\_i \over
\partial~x\_\jmath (a)\right )\_ 1\leqi\leqp
\atop 1\leq\jmath\leqn  = \left
(\matrix\, \partial~f\_1
\over \partial~x\_1
(a)&\\ldots~&
\partial~f\_1 \over \partial~x\_n (a)
\cr
\\ldots~
&\\ldots&\\\ldots~
\cr  \partial~f\_p \over \partial~x\_1
(a)&\\ldots~&
\partial~f\_p \over \partial~x\_n
(a)\right ) \in M\_\mathbb{R}~(p,n)& & \%&
\\ \end{align*}

Démonstration Il faut en effet mettre dans la \jmath-ième colonne de
J\_f(a) les coordonnées du vecteur df(a).e\_\jmath =
\partial~\_e\_\jmathf(a) = \partial~f \over \partial~x\_\jmath
(a) = ( \partial~f\_1 \over \partial~x\_\jmath
(a),\\ldots~,
\partial~f\_p \over \partial~x\_\jmath (a)).

Le théorème de composition des applications différentiables va ainsi se
traduire de la manière suivante sur les matrices \jmathacobiennes

Définition~15.2.3 Soit U un ouvert de \mathbb{R}~^n, V un ouvert de
\mathbb{R}~^p, f : U \rightarrow~ \mathbb{R}~^p telle que f(U) \subset~ V et g : V \rightarrow~
\mathbb{R}~^q. Soit a \in U tel que f soit différentiable au point a et g
différentiable au point f(a). Alors on a J\_g\cdotf(a) =
J\_g(f(a))J\_f(a).

Démonstration La matrice de la composée de deux applications linéaires
est le produit des matrices de ces applications linéaires dans des bases
adéquates (ici les bases canoniques).

Remarque~15.2.5 Utilisons alors les formules donnant le produit de deux
matrices. On va ainsi obtenir

 \left ( \partial~g \cdot f \over \partial~x\_\jmath
(a)\right )\_i = \partial~g\_i \cdot f
\over \partial~x\_\jmath (a) = \\sum
\_k=1^p \partial~g\_i \over
\partial~y\_k (f(a)) \partial~f\_k \over
\partial~x\_\jmath (a)

ce qui n'est autre que la formule trouvée dans la première section pour
les dérivées partielles d'une fonction composée, mais avec des
hypothèses plus faibles (la condition g est \mathcal{C}^1 a été
remplacée par g est différentiable au point a).

Définition~15.2.4 Soit U un ouvert de \mathbb{R}~^n et f : U \rightarrow~
\mathbb{R}~^n. Soit a \in U tel que f soit différentiable au point a. On
appelle \jmathacobien de f au point a le nombre réel

\jmath\_f(a) =\
\mathrm{det} J\_f(a) = \left
\textbar{}\matrix\, \partial~f\_1
\over \partial~x\_1
(a)&\\ldots~&
\partial~f\_1 \over \partial~x\_n (a)
\cr \⋮~
&\\ldots&\\⋮~
\cr  \partial~f\_n \over \partial~x\_1
(a)&\\ldots~&
\partial~f\_n \over \partial~x\_n
(a)\right \textbar{}

Remarque~15.2.6 Soit U un ouvert de \mathbb{R}~^n, V un ouvert de
\mathbb{R}~^n, f : U \rightarrow~ \mathbb{R}~^n telle que f(U) \subset~ V et g : V \rightarrow~
\mathbb{R}~^n. Soit a \in U tel que f soit différentiable au point a et g
différentiable au point f(a). Alors on a \jmath\_g\cdotf(a) =
\jmath\_g(f(a))\jmath\_f(a).

\paragraph{15.2.6 Inégalité des accroissements finis}

Théorème~15.2.8 Soit E et F deux espaces vectoriels normés, U un ouvert
de E et f : U \rightarrow~ F. Soit a,b \in U tels que {[}a,b{]} \subset~ U. On suppose que f
est différentiable en tout point x de {[}a,b{]} et que, pour tout x \in
{[}a,b{]}, la norme de l'application linéaire df(x) est ma\jmathorée par M ≥
0. Alors

\\textbar{}f(b) - f(a)\\textbar{} \leq
M\\textbar{}b - a\\textbar{}

Démonstration Considérons \phi : {[}0,1{]} \rightarrow~ F définie par \phi(t) = f((1 -
t)a + tb). On a \phi'(t) = df((1 - t)a + tb). d \over dt
((1 - t)a + tb) = df((1 - t)a + tb).(b - a). On en déduit que
\forall~~t \in {[}a,b{]},
\\textbar{}\phi'(t)\\textbar{}
\leq\\textbar{} df((1 - t)a +
tb)\\textbar{}.\\textbar{}b -
a\\textbar{} \leq M\\textbar{}b -
a\\textbar{}. L'inégalité des accroissements finis pour
les fonctions d'une variable donne alors \\textbar{}\phi(1)
- \phi(0)\\textbar{} \leq M\\textbar{}b -
a\\textbar{}(1 - 0) = M\\textbar{}b -
a\\textbar{}, ce qu'il fallait démontrer.

Remarque~15.2.7 Cette inégalité des accroissements finis a des
conséquences similaires à celles de l'inégalité des accroissements finis
pour les fonctions d'une variable (en prenant soin de respecter la
condition restrictive~: {[}a,b{]} \subset~ U)~; parmi les plus importantes
citons celle là

Corollaire~15.2.9 Soit E et F deux espaces vectoriels normés, U un
ouvert convexe de E et f : U \rightarrow~ F une application différentiable telle
que \forall~~x \in U,
\\textbar{}df(x)\\textbar{} \leq M. Alors f
est M-lipschitzienne.

{[}
{[}
{[}
{[}

\end{document}
