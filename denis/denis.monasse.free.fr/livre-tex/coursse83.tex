\documentclass[]{article}
\usepackage[T1]{fontenc}
\usepackage{lmodern}
\usepackage{amssymb,amsmath}
\usepackage{ifxetex,ifluatex}
\usepackage{fixltx2e} % provides \textsubscript
% use upquote if available, for straight quotes in verbatim environments
\IfFileExists{upquote.sty}{\usepackage{upquote}}{}
\ifnum 0\ifxetex 1\fi\ifluatex 1\fi=0 % if pdftex
  \usepackage[utf8]{inputenc}
\else % if luatex or xelatex
  \ifxetex
    \usepackage{mathspec}
    \usepackage{xltxtra,xunicode}
  \else
    \usepackage{fontspec}
  \fi
  \defaultfontfeatures{Mapping=tex-text,Scale=MatchLowercase}
  \newcommand{\euro}{€}
\fi
% use microtype if available
\IfFileExists{microtype.sty}{\usepackage{microtype}}{}
\ifxetex
  \usepackage[setpagesize=false, % page size defined by xetex
              unicode=false, % unicode breaks when used with xetex
              xetex]{hyperref}
\else
  \usepackage[unicode=true]{hyperref}
\fi
\hypersetup{breaklinks=true,
            bookmarks=true,
            pdfauthor={},
            pdftitle={Differentielle},
            colorlinks=true,
            citecolor=blue,
            urlcolor=blue,
            linkcolor=magenta,
            pdfborder={0 0 0}}
\urlstyle{same}  % don't use monospace font for urls
\setlength{\parindent}{0pt}
\setlength{\parskip}{6pt plus 2pt minus 1pt}
\setlength{\emergencystretch}{3em}  % prevent overfull lines
\setcounter{secnumdepth}{0}
 
/* start css.sty */
.cmr-5{font-size:50%;}
.cmr-7{font-size:70%;}
.cmmi-5{font-size:50%;font-style: italic;}
.cmmi-7{font-size:70%;font-style: italic;}
.cmmi-10{font-style: italic;}
.cmsy-5{font-size:50%;}
.cmsy-7{font-size:70%;}
.cmex-7{font-size:70%;}
.cmex-7x-x-71{font-size:49%;}
.msbm-7{font-size:70%;}
.cmtt-10{font-family: monospace;}
.cmti-10{ font-style: italic;}
.cmbx-10{ font-weight: bold;}
.cmr-17x-x-120{font-size:204%;}
.cmsl-10{font-style: oblique;}
.cmti-7x-x-71{font-size:49%; font-style: italic;}
.cmbxti-10{ font-weight: bold; font-style: italic;}
p.noindent { text-indent: 0em }
td p.noindent { text-indent: 0em; margin-top:0em; }
p.nopar { text-indent: 0em; }
p.indent{ text-indent: 1.5em }
@media print {div.crosslinks {visibility:hidden;}}
a img { border-top: 0; border-left: 0; border-right: 0; }
center { margin-top:1em; margin-bottom:1em; }
td center { margin-top:0em; margin-bottom:0em; }
.Canvas { position:relative; }
li p.indent { text-indent: 0em }
.enumerate1 {list-style-type:decimal;}
.enumerate2 {list-style-type:lower-alpha;}
.enumerate3 {list-style-type:lower-roman;}
.enumerate4 {list-style-type:upper-alpha;}
div.newtheorem { margin-bottom: 2em; margin-top: 2em;}
.obeylines-h,.obeylines-v {white-space: nowrap; }
div.obeylines-v p { margin-top:0; margin-bottom:0; }
.overline{ text-decoration:overline; }
.overline img{ border-top: 1px solid black; }
td.displaylines {text-align:center; white-space:nowrap;}
.centerline {text-align:center;}
.rightline {text-align:right;}
div.verbatim {font-family: monospace; white-space: nowrap; text-align:left; clear:both; }
.fbox {padding-left:3.0pt; padding-right:3.0pt; text-indent:0pt; border:solid black 0.4pt; }
div.fbox {display:table}
div.center div.fbox {text-align:center; clear:both; padding-left:3.0pt; padding-right:3.0pt; text-indent:0pt; border:solid black 0.4pt; }
div.minipage{width:100%;}
div.center, div.center div.center {text-align: center; margin-left:1em; margin-right:1em;}
div.center div {text-align: left;}
div.flushright, div.flushright div.flushright {text-align: right;}
div.flushright div {text-align: left;}
div.flushleft {text-align: left;}
.underline{ text-decoration:underline; }
.underline img{ border-bottom: 1px solid black; margin-bottom:1pt; }
.framebox-c, .framebox-l, .framebox-r { padding-left:3.0pt; padding-right:3.0pt; text-indent:0pt; border:solid black 0.4pt; }
.framebox-c {text-align:center;}
.framebox-l {text-align:left;}
.framebox-r {text-align:right;}
span.thank-mark{ vertical-align: super }
span.footnote-mark sup.textsuperscript, span.footnote-mark a sup.textsuperscript{ font-size:80%; }
div.tabular, div.center div.tabular {text-align: center; margin-top:0.5em; margin-bottom:0.5em; }
table.tabular td p{margin-top:0em;}
table.tabular {margin-left: auto; margin-right: auto;}
div.td00{ margin-left:0pt; margin-right:0pt; }
div.td01{ margin-left:0pt; margin-right:5pt; }
div.td10{ margin-left:5pt; margin-right:0pt; }
div.td11{ margin-left:5pt; margin-right:5pt; }
table[rules] {border-left:solid black 0.4pt; border-right:solid black 0.4pt; }
td.td00{ padding-left:0pt; padding-right:0pt; }
td.td01{ padding-left:0pt; padding-right:5pt; }
td.td10{ padding-left:5pt; padding-right:0pt; }
td.td11{ padding-left:5pt; padding-right:5pt; }
table[rules] {border-left:solid black 0.4pt; border-right:solid black 0.4pt; }
.hline hr, .cline hr{ height : 1px; margin:0px; }
.tabbing-right {text-align:right;}
span.TEX {letter-spacing: -0.125em; }
span.TEX span.E{ position:relative;top:0.5ex;left:-0.0417em;}
a span.TEX span.E {text-decoration: none; }
span.LATEX span.A{ position:relative; top:-0.5ex; left:-0.4em; font-size:85%;}
span.LATEX span.TEX{ position:relative; left: -0.4em; }
div.float img, div.float .caption {text-align:center;}
div.figure img, div.figure .caption {text-align:center;}
.marginpar {width:20%; float:right; text-align:left; margin-left:auto; margin-top:0.5em; font-size:85%; text-decoration:underline;}
.marginpar p{margin-top:0.4em; margin-bottom:0.4em;}
.equation td{text-align:center; vertical-align:middle; }
td.eq-no{ width:5%; }
table.equation { width:100%; } 
div.math-display, div.par-math-display{text-align:center;}
math .texttt { font-family: monospace; }
math .textit { font-style: italic; }
math .textsl { font-style: oblique; }
math .textsf { font-family: sans-serif; }
math .textbf { font-weight: bold; }
.partToc a, .partToc, .likepartToc a, .likepartToc {line-height: 200%; font-weight:bold; font-size:110%;}
.chapterToc a, .chapterToc, .likechapterToc a, .likechapterToc, .appendixToc a, .appendixToc {line-height: 200%; font-weight:bold;}
.index-item, .index-subitem, .index-subsubitem {display:block}
.caption td.id{font-weight: bold; white-space: nowrap; }
table.caption {text-align:center;}
h1.partHead{text-align: center}
p.bibitem { text-indent: -2em; margin-left: 2em; margin-top:0.6em; margin-bottom:0.6em; }
p.bibitem-p { text-indent: 0em; margin-left: 2em; margin-top:0.6em; margin-bottom:0.6em; }
.paragraphHead, .likeparagraphHead { margin-top:2em; font-weight: bold;}
.subparagraphHead, .likesubparagraphHead { font-weight: bold;}
.quote {margin-bottom:0.25em; margin-top:0.25em; margin-left:1em; margin-right:1em; text-align:justify;}
.verse{white-space:nowrap; margin-left:2em}
div.maketitle {text-align:center;}
h2.titleHead{text-align:center;}
div.maketitle{ margin-bottom: 2em; }
div.author, div.date {text-align:center;}
div.thanks{text-align:left; margin-left:10%; font-size:85%; font-style:italic; }
div.author{white-space: nowrap;}
.quotation {margin-bottom:0.25em; margin-top:0.25em; margin-left:1em; }
h1.partHead{text-align: center}
.sectionToc, .likesectionToc {margin-left:2em;}
.subsectionToc, .likesubsectionToc {margin-left:4em;}
.subsubsectionToc, .likesubsubsectionToc {margin-left:6em;}
.frenchb-nbsp{font-size:75%;}
.frenchb-thinspace{font-size:75%;}
.figure img.graphics {margin-left:10%;}
/* end css.sty */

\title{Differentielle}
\author{}
\date{}

\begin{document}
\maketitle

\textbf{Warning: \href{http://www.math.union.edu/locate/jsMath}{jsMath}
requires JavaScript to process the mathematics on this page.\\ If your
browser supports JavaScript, be sure it is enabled.}

\begin{center}\rule{3in}{0.4pt}\end{center}

{[}\href{coursse84.html}{next}{]} {[}\href{coursse82.html}{prev}{]}
{[}\href{coursse82.html\#tailcoursse82.html}{prev-tail}{]}
{[}\hyperref[tailcoursse83.html]{tail}{]}
{[}\href{coursch16.html\#coursse83.html}{up}{]}

\subsubsection{15.2 Différentielle}

\paragraph{15.2.1 Applications différentiables}

Définition~15.2.1 Soit E et F deux espaces vectoriels normés, U un
ouvert de E, a ∈ U et f : U → F. On dit que f est différentiable au
point a s'il existe une application linéaire continue L : E → F telle
que, pour h voisin de 0,

f(a + h) = f(a) + L(h) +
o(\textbackslash{}\textbar{}h\textbackslash{}\textbar{})

Dans ce cas, l'application L est unique et est appelée la différentielle
de f au point a, notée df(a) ou encore \{d\}\_\{a\}f.

Démonstration Supposons que \{L\}\_\{1\} et \{L\}\_\{2\} conviennent.
Par différence, on a \{L\}\_\{1\}(h) − \{L\}\_\{2\}(h) =
o(\textbackslash{}\textbar{}h\textbackslash{}\textbar{}). On a donc,
pour x ∈ E ∖\textbackslash{}\{0\textbackslash{}\}

\{\textbackslash{}mathop\{lim\}\}\_\{t→0,t\textgreater{}0\}\{\{L\}\_\{1\}(tx)
− \{L\}\_\{2\}(tx)\textbackslash{}over
\textbackslash{}\textbar{}tx\textbackslash{}\textbar{}\} = 0

Mais pour t \textgreater{} 0, on a
\{\{L\}\_\{1\}(tx)−\{L\}\_\{2\}(tx)\textbackslash{}over
\textbackslash{}\textbar{}tx\textbackslash{}\textbar{}\} =
\{\{L\}\_\{1\}(x)−\{L\}\_\{2\}(x)\textbackslash{}over
\textbackslash{}\textbar{}x\textbackslash{}\textbar{}\} ~; ceci montre
que \{L\}\_\{1\}(x) = \{L\}\_\{2\}(x) et donc \{L\}\_\{1\} =
\{L\}\_\{2\}.

Remarque~15.2.1 Pour alléger les notations, on écrira df(a).h à la place
de \textbackslash{}big {[}df(a)\textbackslash{}big {]}(h). On a donc par
définition f(a + h) = f(a) + df(a).h +
o(\textbackslash{}\textbar{}h\textbackslash{}\textbar{}) ou encore f(a +
h) = f(a) + df(a).h +\textbackslash{}\textbar{}
h\textbackslash{}\textbar{}ε(h) avec
\{\textbackslash{}mathop\{lim\}\}\_\{h→0\}ε(h) = 0.

Remarque~15.2.2 Si E est de dimension finie, une application linéaire de
E dans F est automatiquement continue. Il est clair d'autre part que la
différentiabilité est une notion locale et que le changement des normes
sur E et F en normes équivalentes ne change ni la différentiabilité, ni
la différentielle.

Proposition~15.2.1 Si f est différentiable au point a, elle est continue
au point a.

Démonstration On a f(a + h) = f(a) + df(a).h +\textbackslash{}\textbar{}
h\textbackslash{}\textbar{}ε(h) avec
\{\textbackslash{}mathop\{lim\}\}\_\{h→0\}ε(h) = 0. Comme df(a) est une
application linéaire continue, on a
\{\textbackslash{}mathop\{lim\}\}\_\{h→0\}df(a).h = df(a).0 = 0 et donc
\{\textbackslash{}mathop\{lim\}\}\_\{h→0\}f(a + h) = f(a).

\paragraph{15.2.2 Exemples d'applications différentiables}

Proposition~15.2.2 Soit E et F deux espaces vectoriels normés, u une
application linéaire continue de E dans F. Alors u est différentiable en
tout point a de E et du(a) = u.

Démonstration On a en effet u(a + h) = u(a) + u(h) + 0.

Proposition~15.2.3 Soit E, F et G trois espaces vectoriels normés, u : E
× F → G une application bilinéaire continue. Alors f est différentiable
en tout point (a,b) de E × F et du(a,b).(h,k) = u(a,k) + u(h,b).

Démonstration On a u((a,b) + (h,k)) = u(a + h,b + k) = u(a,b) +
\textbackslash{}left (u(a,k) + u(h,b)\textbackslash{}right ) + u(h,k).
Mais comme u est bilinéaire continue, il existe une constante A telle
que \textbackslash{}\textbar{}u(h,k)\textbackslash{}\textbar{} ≤
A\textbackslash{}\textbar{}h\textbackslash{}\textbar{}\textbackslash{}\textbar{}k\textbackslash{}\textbar{}
soit encore \textbackslash{}\textbar{}u(h,k)\textbackslash{}\textbar{} ≤
A\textbackslash{}mathop\{max\}\{(\textbackslash{}\textbar{}h\textbackslash{}\textbar{},\textbackslash{}\textbar{}k\textbackslash{}\textbar{})\}\^{}\{2\}
=
A\textbackslash{}\textbar{}\{(h,k)\textbackslash{}\textbar{}\}\^{}\{2\}.
On a donc u((a,b) + (h,k)) = u(a,b) + \textbackslash{}left (u(a,k) +
u(h,b)\textbackslash{}right ) +
O(\textbackslash{}\textbar{}\{(h,k)\textbackslash{}\textbar{}\}\^{}\{2\}).
Comme (h,k)\textbackslash{}mathrel\{↦\}u(a,k) + u(h,b) est clairement
linéaire et continue, c'est la différentielle de u au point (a,b).

Exemple~15.2.1 Tous les produits usuels (produit dans K, produits
scalaires, produits vectoriels, produits matriciels) sont donc
différentiables en tout point.

\paragraph{15.2.3 Opérations sur les différentielles}

Proposition~15.2.4 Soit E et F deux espaces vectoriels normés, U un
ouvert de E, a ∈ U et f,g : U → F. Si f et g sont différentiables en a,
il en est de même pour αf + βg et d(αf + βg)(a) = αdf(a) + βdg(a).

Démonstration On a f(a + h) = f(a) + df(a).h +
o(\textbackslash{}\textbar{}h\textbackslash{}\textbar{}) et g(a + h) =
g(a) + dg(a).h +
o(\textbackslash{}\textbar{}h\textbackslash{}\textbar{}), d'où (αf +
βg)(a + h) = (αf + βg)(a) + αdf(a).h + βdg(a).h +
o(\textbackslash{}\textbar{}h\textbackslash{}\textbar{}) avec αdf(a) +
βdg(a) application linéaire continue.

Théorème~15.2.5 Soit E, F et G trois espaces vectoriels normés, U un
ouvert de E, V un ouvert de F, f : U → F tel que f(U) ⊂ V et g : V → G.
Soit a ∈ U. Si f est différentiable au point a et g différentiable au
point f(a), alors g ∘ f est différentiable au point a et d(g ∘ f)(a) =
dg\textbackslash{}left (f(a)\textbackslash{}right ) ∘ df(a).

Démonstration On a, en posant b = f(a), f(a + h) = f(a) + df(a).h
+\textbackslash{}\textbar{} h\textbackslash{}\textbar{}ε(h) avec
\{\textbackslash{}mathop\{lim\}\}\_\{h→0\}ε(h) = 0 et g(b + k) = g(b) +
dg(b).k +\textbackslash{}\textbar{} k\textbackslash{}\textbar{}η(k) avec
\{\textbackslash{}mathop\{lim\}\}\_\{k→0\}η(k) = 0. Prenons en
particulier

k = φ(h) = f(a + h) − f(a) = df(a).h +\textbackslash{}\textbar{}
h\textbackslash{}\textbar{}ε(h)

On a b + k = f(a + h), et donc

g(f(a + h)) = g(f(a)) + dg(b).φ(h) +\textbackslash{}\textbar{}
φ(h)\textbackslash{}\textbar{}η(φ(h))

Mais on a

\textbackslash{}begin\{eqnarray*\} dg(b).φ(h)\& =\&
dg(b).\textbackslash{}left (df(a).h +\textbackslash{}\textbar{}
h\textbackslash{}\textbar{}ε(h)\textbackslash{}right ) \%\&
\textbackslash{}\textbackslash{} \& =\& dg(b) ∘ df(a).h
+\textbackslash{}\textbar{} h\textbackslash{}\textbar{}dg(b).ε(h)\%\&
\textbackslash{}\textbackslash{} \& =\& dg(b) ∘ df(a).h +
o(\textbackslash{}\textbar{}h\textbackslash{}\textbar{}) \%\&
\textbackslash{}\textbackslash{} \textbackslash{}end\{eqnarray*\}

puisque \{\textbackslash{}mathop\{lim\}\}\_\{h→0\}dg(b).ε(h) = dg(b).0 =
0. D'autre part,

\textbackslash{}begin\{eqnarray*\}
\textbackslash{}\textbar{}φ(h)\textbackslash{}\textbar{}\& ≤\&
\textbackslash{}\textbar{}df(a).h +\textbackslash{}\textbar{}
h\textbackslash{}\textbar{}ε(h)\textbackslash{}\textbar{} \%\&
\textbackslash{}\textbackslash{} \& ≤\&
\textbackslash{}\textbar{}df(a)\textbackslash{}\textbar{}\textbackslash{},\textbackslash{}\textbar{}h\textbackslash{}\textbar{}
+\textbackslash{}\textbar{}
ε(h)\textbackslash{}\textbar{}\textbackslash{},\textbackslash{}\textbar{}h\textbackslash{}\textbar{}
= O(\textbackslash{}\textbar{}h\textbackslash{}\textbar{})\%\&
\textbackslash{}\textbackslash{} \textbackslash{}end\{eqnarray*\}

donc \textbackslash{}\textbar{}φ(h)\textbackslash{}\textbar{}η(φ(h)) =
o(\textbackslash{}\textbar{}h\textbackslash{}\textbar{}) puisque
\{\textbackslash{}mathop\{lim\}\}\_\{h→0\}φ(h) = 0 (continuité de f au
point a) et donc \{\textbackslash{}mathop\{lim\}\}\_\{h→0\}η(φ(h)) = 0.
On a donc en définitive, g(f(a + h)) = g(f(a)) + dg(b) ∘ df(a).h +
o(\textbackslash{}\textbar{}h\textbackslash{}\textbar{}) ce qui termine
la démonstration.

Remarque~15.2.3 En particulier, si u est une application linéaire
continue et si f est différentiable, u ∘ f est différentiable et d(u ∘
f)(a) = u ∘ df(a).

\paragraph{15.2.4 Différentielle et dérivées partielles}

Regardons tout d'abord le cas des fonctions d'une variable. On a un
résultat très simple qui montre que la notion de différentiabilité est
une généralisation de la notion de dérivabilité.

Théorème~15.2.6 Soit U un ouvert de ℝ, a ∈ U et F un K-espace vectoriel
normé. Soit f : U → F. Alors f est différentiable au point a si et
seulement si~elle est dérivable au point a et on a f'(a) = df(a).1 et
df(a).h = hf'(a).

Démonstration Si f est dérivable au point a, on a f(a + h) = f(a) +
hf'(a) + o(h), ce qui montre que f est différentiable en a et que
df(a).h = hf'(a). Inversement si f est différentiable au point a, on a
f(a + h) = f(a) + df(a).h + o(h) = f(a) + hdf(a).1 + o(h) (car h est un
réel), soit encore \{\textbackslash{}mathop\{lim\}\}\_\{h→0\}\{
f(a+h)−f(a) \textbackslash{}over h\} = df(a).1, ce qui montre que f est
dérivable au point 1 et que f'(a) = df(a).1.

Exemple~15.2.2 Soit U un ouvert de ℝ, V un ouvert de E, soit φ : U → E
telle que φ(U) ⊂ V et f : V → F. Soit a ∈ U. Supposons que φ est
dérivable (donc différentiable) au point a et que f est différentiable
au point φ(a). Alors f ∘ φ est différentiable (donc dérivable) au point
a et (f ∘ φ)'(a) = d(f ∘ φ)(a).1 = df(φ(a)) ∘ dφ(a).1 = df(φ(a)).φ'(a).
On retiendra donc la formule importante (f ∘ φ)'(a) = df(φ(a)).φ'(a).

En ce qui concerne les fonctions de plusieurs variables, le lien entre
différentiabilité et dérivée partielle est plus complexe puisque l'on a
vu que l'existence de dérivées partielles n'impliquait même pas la
continuité, et donc certainement pas la différentiabilité. On a le
résultat suivant

Théorème~15.2.7 Soit E et F deux espaces vectoriels normés de dimensions
finies, U un ouvert de E et f : U → F. (i) si f est différentiable au
point a ∈ U, alors f admet en a une dérivée partielle suivant tout
vecteur v et \{∂\}\_\{v\}f(a) = df(a).v (ii) inversement, si E =
\{ℝ\}\^{}\{n\} et si f est de classe \{C\}\^{}\{1\} sur U, alors f est
différentiable en tout point a de U et df(a).h
=\{\textbackslash{}mathop\{ \textbackslash{}mathop\{∑ \}\}
\}\_\{i=1\}\^{}\{n\}\{h\}\_\{i\}\{ ∂f \textbackslash{}over
∂\{x\}\_\{i\}\} (a).

Démonstration (i) On a f(a + h) = f(a) + df(a).h
+\textbackslash{}\textbar{} h\textbackslash{}\textbar{}ε(h), soit encore
f(a + tv) = f(a) + tdf(a).v +
\textbar{}t\textbar{}\textbackslash{},\textbackslash{}\textbar{}v\textbackslash{}\textbar{}ε(tv),
c'est-à-dire \{\textbackslash{}mathop\{lim\}\}\_\{t→0\}\{ f(a+tv)−f(a)
\textbackslash{}over t\} = df(a).v, donc f admet en a une dérivée
partielle suivant v et \{∂\}\_\{v\}f(a) = df(a).v.

(ii) La formule de Taylor Young à l'ordre 1 montre en effet que f(a + h)
= f(a) +\{\textbackslash{}mathop\{ \textbackslash{}mathop\{∑ \}\}
\}\_\{i=1\}\^{}\{n\}\{h\}\_\{i\}\{ ∂f \textbackslash{}over
∂\{x\}\_\{i\}\} (a) +
o(\textbackslash{}\textbar{}h\textbackslash{}\textbar{}) ce qui montre
que f est différentiable en a et que df(a).h =\{\textbackslash{}mathop\{
\textbackslash{}mathop\{∑ \}\} \}\_\{i=1\}\^{}\{n\}\{h\}\_\{i\}\{ ∂f
\textbackslash{}over ∂\{x\}\_\{i\}\} (a).

Remarque~15.2.4 Si E = \{ℝ\}\^{}\{n\}, et si f est différentiable au
point a, on a nécessairement

\textbackslash{}begin\{eqnarray*\} df(a).h\& =\&
df(a).(\{\textbackslash{}mathop\{∑ \}\}\_\{i=1\}\^{}\{n\}\{h\}\_\{
i\}\{e\}\_\{i\}) =\{ \textbackslash{}mathop\{∑
\}\}\_\{i=1\}\^{}\{n\}\{h\}\_\{ i\}df(a).\{e\}\_\{i\}\%\&
\textbackslash{}\textbackslash{} \& =\& \{\textbackslash{}mathop\{∑
\}\}\_\{i=1\}\^{}\{n\}\{h\}\_\{ i\}\{∂\}\_\{\{e\}\_\{i\}\}f(a) =\{
\textbackslash{}mathop\{∑ \}\}\_\{i=1\}\^{}\{n\}\{h\}\_\{ i\}\{ ∂f
\textbackslash{}over ∂\{x\}\_\{i\}\} (a) \%\&
\textbackslash{}\textbackslash{} \textbackslash{}end\{eqnarray*\}

Donc en fait montrer la différentiabilité de f en a, c'est montrer que
les \{ ∂f \textbackslash{}over ∂\{x\}\_\{i\}\} (a) existent et que f(a +
h) − f(a) −\{\textbackslash{}mathop\{\textbackslash{}mathop\{∑ \}\}
\}\_\{i=1\}\^{}\{n\}\{h\}\_\{i\}\{ ∂f \textbackslash{}over
∂\{x\}\_\{i\}\} (a) =
o(\textbackslash{}\textbar{}h\textbackslash{}\textbar{}).

\paragraph{15.2.5 Matrices jacobiennes, jacobiens}

Définition~15.2.2 Soit U un ouvert de \{ℝ\}\^{}\{n\} et f : U →
\{ℝ\}\^{}\{p\}. Soit a ∈ U tel que f soit différentiable au point a. On
appelle matrice jacobienne de f au point a la matrice \{J\}\_\{f\}(a) de
l'application linéaire df(a) dans les bases canoniques de \{ℝ\}\^{}\{n\}
et \{ℝ\}\^{}\{p\}. Si
f(\{x\}\_\{1\},\textbackslash{}mathop\{\textbackslash{}mathop\{\ldots{}\}\},\{x\}\_\{n\})
=
(\{f\}\_\{1\}(\{x\}\_\{1\},\textbackslash{}mathop\{\textbackslash{}mathop\{\ldots{}\}\},\{x\}\_\{n\}),\textbackslash{}mathop\{\textbackslash{}mathop\{\ldots{}\}\},\{f\}\_\{p\}(\{x\}\_\{1\},\textbackslash{}mathop\{\textbackslash{}mathop\{\ldots{}\}\},\{x\}\_\{n\})),
c'est la matrice

\textbackslash{}begin\{eqnarray*\}\{ J\}\_\{f\}(a) =\{
\textbackslash{}left (\{ ∂\{f\}\_\{i\} \textbackslash{}over
∂\{x\}\_\{j\}\} (a)\textbackslash{}right )\}\_\{\{ 1≤i≤p
\textbackslash{}atop 1≤j≤n\} \} = \textbackslash{}left
(\textbackslash{}matrix\{\textbackslash{},\{ ∂\{f\}\_\{1\}
\textbackslash{}over ∂\{x\}\_\{1\}\}
(a)\&\textbackslash{}mathop\{\textbackslash{}mathop\{\ldots{}\}\}\&\{
∂\{f\}\_\{1\} \textbackslash{}over ∂\{x\}\_\{n\}\} (a)
\textbackslash{}cr
\textbackslash{}mathop\{\textbackslash{}mathop\{\ldots{}\}\}
\&\textbackslash{}mathop\{\textbackslash{}mathop\{\ldots{}\}\}\&\textbackslash{}mathop\{\textbackslash{}mathop\{\ldots{}\}\}
\textbackslash{}cr \{ ∂\{f\}\_\{p\} \textbackslash{}over ∂\{x\}\_\{1\}\}
(a)\&\textbackslash{}mathop\{\textbackslash{}mathop\{\ldots{}\}\}\&\{
∂\{f\}\_\{p\} \textbackslash{}over ∂\{x\}\_\{n\}\}
(a)\}\textbackslash{}right ) ∈ \{M\}\_\{ℝ\}(p,n)\& \& \%\&
\textbackslash{}\textbackslash{} \textbackslash{}end\{eqnarray*\}

Démonstration Il faut en effet mettre dans la j-ième colonne de
\{J\}\_\{f\}(a) les coordonnées du vecteur df(a).\{e\}\_\{j\} =
\{∂\}\_\{\{e\}\_\{j\}\}f(a) =\{ ∂f \textbackslash{}over ∂\{x\}\_\{j\}\}
(a) = (\{ ∂\{f\}\_\{1\} \textbackslash{}over ∂\{x\}\_\{j\}\}
(a),\textbackslash{}mathop\{\textbackslash{}mathop\{\ldots{}\}\},\{
∂\{f\}\_\{p\} \textbackslash{}over ∂\{x\}\_\{j\}\} (a)).

Le théorème de composition des applications différentiables va ainsi se
traduire de la manière suivante sur les matrices jacobiennes

Définition~15.2.3 Soit U un ouvert de \{ℝ\}\^{}\{n\}, V un ouvert de
\{ℝ\}\^{}\{p\}, f : U → \{ℝ\}\^{}\{p\} telle que f(U) ⊂ V et g : V →
\{ℝ\}\^{}\{q\}. Soit a ∈ U tel que f soit différentiable au point a et g
différentiable au point f(a). Alors on a \{J\}\_\{g∘f\}(a) =
\{J\}\_\{g\}(f(a))\{J\}\_\{f\}(a).

Démonstration La matrice de la composée de deux applications linéaires
est le produit des matrices de ces applications linéaires dans des bases
adéquates (ici les bases canoniques).

Remarque~15.2.5 Utilisons alors les formules donnant le produit de deux
matrices. On va ainsi obtenir

\{ \textbackslash{}left (\{ ∂g ∘ f \textbackslash{}over ∂\{x\}\_\{j\}\}
(a)\textbackslash{}right )\}\_\{i\} =\{ ∂\{g\}\_\{i\} ∘ f
\textbackslash{}over ∂\{x\}\_\{j\}\} (a) =\{ \textbackslash{}mathop\{∑
\}\}\_\{k=1\}\^{}\{p\}\{ ∂\{g\}\_\{i\} \textbackslash{}over
∂\{y\}\_\{k\}\} (f(a))\{ ∂\{f\}\_\{k\} \textbackslash{}over
∂\{x\}\_\{j\}\} (a)

ce qui n'est autre que la formule trouvée dans la première section pour
les dérivées partielles d'une fonction composée, mais avec des
hypothèses plus faibles (la condition g est \{C\}\^{}\{1\} a été
remplacée par g est différentiable au point a).

Définition~15.2.4 Soit U un ouvert de \{ℝ\}\^{}\{n\} et f : U →
\{ℝ\}\^{}\{n\}. Soit a ∈ U tel que f soit différentiable au point a. On
appelle jacobien de f au point a le nombre réel

\{j\}\_\{f\}(a) =\textbackslash{}mathop\{
\textbackslash{}mathrm\{det\}\} \{J\}\_\{f\}(a) = \textbackslash{}left
\textbar{}\textbackslash{}matrix\{\textbackslash{},\{ ∂\{f\}\_\{1\}
\textbackslash{}over ∂\{x\}\_\{1\}\}
(a)\&\textbackslash{}mathop\{\textbackslash{}mathop\{\ldots{}\}\}\&\{
∂\{f\}\_\{1\} \textbackslash{}over ∂\{x\}\_\{n\}\} (a)
\textbackslash{}cr \textbackslash{}mathop\{\textbackslash{}mathop\{⋮\}\}
\&\textbackslash{}mathop\{\textbackslash{}mathop\{\ldots{}\}\}\&\textbackslash{}mathop\{\textbackslash{}mathop\{⋮\}\}
\textbackslash{}cr \{ ∂\{f\}\_\{n\} \textbackslash{}over ∂\{x\}\_\{1\}\}
(a)\&\textbackslash{}mathop\{\textbackslash{}mathop\{\ldots{}\}\}\&\{
∂\{f\}\_\{n\} \textbackslash{}over ∂\{x\}\_\{n\}\}
(a)\}\textbackslash{}right \textbar{}

Remarque~15.2.6 Soit U un ouvert de \{ℝ\}\^{}\{n\}, V un ouvert de
\{ℝ\}\^{}\{n\}, f : U → \{ℝ\}\^{}\{n\} telle que f(U) ⊂ V et g : V →
\{ℝ\}\^{}\{n\}. Soit a ∈ U tel que f soit différentiable au point a et g
différentiable au point f(a). Alors on a \{j\}\_\{g∘f\}(a) =
\{j\}\_\{g\}(f(a))\{j\}\_\{f\}(a).

\paragraph{15.2.6 Inégalité des accroissements finis}

Théorème~15.2.8 Soit E et F deux espaces vectoriels normés, U un ouvert
de E et f : U → F. Soit a,b ∈ U tels que {[}a,b{]} ⊂ U. On suppose que f
est différentiable en tout point x de {[}a,b{]} et que, pour tout x ∈
{[}a,b{]}, la norme de l'application linéaire df(x) est majorée par M ≥
0. Alors

\textbackslash{}\textbar{}f(b) − f(a)\textbackslash{}\textbar{} ≤
M\textbackslash{}\textbar{}b − a\textbackslash{}\textbar{}

Démonstration Considérons φ : {[}0,1{]} → F définie par φ(t) = f((1 −
t)a + tb). On a φ'(t) = df((1 − t)a + tb).\{ d \textbackslash{}over dt\}
((1 − t)a + tb) = df((1 − t)a + tb).(b − a). On en déduit que
\textbackslash{}mathop\{∀\}t ∈ {[}a,b{]},
\textbackslash{}\textbar{}φ'(t)\textbackslash{}\textbar{}
≤\textbackslash{}\textbar{} df((1 − t)a +
tb)\textbackslash{}\textbar{}.\textbackslash{}\textbar{}b −
a\textbackslash{}\textbar{} ≤ M\textbackslash{}\textbar{}b −
a\textbackslash{}\textbar{}. L'inégalité des accroissements finis pour
les fonctions d'une variable donne alors \textbackslash{}\textbar{}φ(1)
− φ(0)\textbackslash{}\textbar{} ≤ M\textbackslash{}\textbar{}b −
a\textbackslash{}\textbar{}(1 − 0) = M\textbackslash{}\textbar{}b −
a\textbackslash{}\textbar{}, ce qu'il fallait démontrer.

Remarque~15.2.7 Cette inégalité des accroissements finis a des
conséquences similaires à celles de l'inégalité des accroissements finis
pour les fonctions d'une variable (en prenant soin de respecter la
condition restrictive~: {[}a,b{]} ⊂ U)~; parmi les plus importantes
citons celle là

Corollaire~15.2.9 Soit E et F deux espaces vectoriels normés, U un
ouvert convexe de E et f : U → F une application différentiable telle
que \textbackslash{}mathop\{∀\}x ∈ U,
\textbackslash{}\textbar{}df(x)\textbackslash{}\textbar{} ≤ M. Alors f
est M-lipschitzienne.

{[}\href{coursse84.html}{next}{]} {[}\href{coursse82.html}{prev}{]}
{[}\href{coursse82.html\#tailcoursse82.html}{prev-tail}{]}
{[}\href{coursse83.html}{front}{]}
{[}\href{coursch16.html\#coursse83.html}{up}{]}

\end{document}
