\documentclass[]{article}
\usepackage[T1]{fontenc}
\usepackage{lmodern}
\usepackage{amssymb,amsmath}
\usepackage{ifxetex,ifluatex}
\usepackage{fixltx2e} % provides \textsubscript
% use upquote if available, for straight quotes in verbatim environments
\IfFileExists{upquote.sty}{\usepackage{upquote}}{}
\ifnum 0\ifxetex 1\fi\ifluatex 1\fi=0 % if pdftex
  \usepackage[utf8]{inputenc}
\else % if luatex or xelatex
  \ifxetex
    \usepackage{mathspec}
    \usepackage{xltxtra,xunicode}
  \else
    \usepackage{fontspec}
  \fi
  \defaultfontfeatures{Mapping=tex-text,Scale=MatchLowercase}
  \newcommand{\euro}{€}
\fi
% use microtype if available
\IfFileExists{microtype.sty}{\usepackage{microtype}}{}
\ifxetex
  \usepackage[setpagesize=false, % page size defined by xetex
              unicode=false, % unicode breaks when used with xetex
              xetex]{hyperref}
\else
  \usepackage[unicode=true]{hyperref}
\fi
\hypersetup{breaklinks=true,
            bookmarks=true,
            pdfauthor={},
            pdftitle={Polynomes d'endomorphismes},
            colorlinks=true,
            citecolor=blue,
            urlcolor=blue,
            linkcolor=magenta,
            pdfborder={0 0 0}}
\urlstyle{same}  % don't use monospace font for urls
\setlength{\parindent}{0pt}
\setlength{\parskip}{6pt plus 2pt minus 1pt}
\setlength{\emergencystretch}{3em}  % prevent overfull lines
\setcounter{secnumdepth}{0}
 
/* start css.sty */
.cmr-5{font-size:50%;}
.cmr-7{font-size:70%;}
.cmmi-5{font-size:50%;font-style: italic;}
.cmmi-7{font-size:70%;font-style: italic;}
.cmmi-10{font-style: italic;}
.cmsy-5{font-size:50%;}
.cmsy-7{font-size:70%;}
.cmex-7{font-size:70%;}
.cmex-7x-x-71{font-size:49%;}
.msbm-7{font-size:70%;}
.cmtt-10{font-family: monospace;}
.cmti-10{ font-style: italic;}
.cmbx-10{ font-weight: bold;}
.cmr-17x-x-120{font-size:204%;}
.cmsl-10{font-style: oblique;}
.cmti-7x-x-71{font-size:49%; font-style: italic;}
.cmbxti-10{ font-weight: bold; font-style: italic;}
p.noindent { text-indent: 0em }
td p.noindent { text-indent: 0em; margin-top:0em; }
p.nopar { text-indent: 0em; }
p.indent{ text-indent: 1.5em }
@media print {div.crosslinks {visibility:hidden;}}
a img { border-top: 0; border-left: 0; border-right: 0; }
center { margin-top:1em; margin-bottom:1em; }
td center { margin-top:0em; margin-bottom:0em; }
.Canvas { position:relative; }
li p.indent { text-indent: 0em }
.enumerate1 {list-style-type:decimal;}
.enumerate2 {list-style-type:lower-alpha;}
.enumerate3 {list-style-type:lower-roman;}
.enumerate4 {list-style-type:upper-alpha;}
div.newtheorem { margin-bottom: 2em; margin-top: 2em;}
.obeylines-h,.obeylines-v {white-space: nowrap; }
div.obeylines-v p { margin-top:0; margin-bottom:0; }
.overline{ text-decoration:overline; }
.overline img{ border-top: 1px solid black; }
td.displaylines {text-align:center; white-space:nowrap;}
.centerline {text-align:center;}
.rightline {text-align:right;}
div.verbatim {font-family: monospace; white-space: nowrap; text-align:left; clear:both; }
.fbox {padding-left:3.0pt; padding-right:3.0pt; text-indent:0pt; border:solid black 0.4pt; }
div.fbox {display:table}
div.center div.fbox {text-align:center; clear:both; padding-left:3.0pt; padding-right:3.0pt; text-indent:0pt; border:solid black 0.4pt; }
div.minipage{width:100%;}
div.center, div.center div.center {text-align: center; margin-left:1em; margin-right:1em;}
div.center div {text-align: left;}
div.flushright, div.flushright div.flushright {text-align: right;}
div.flushright div {text-align: left;}
div.flushleft {text-align: left;}
.underline{ text-decoration:underline; }
.underline img{ border-bottom: 1px solid black; margin-bottom:1pt; }
.framebox-c, .framebox-l, .framebox-r { padding-left:3.0pt; padding-right:3.0pt; text-indent:0pt; border:solid black 0.4pt; }
.framebox-c {text-align:center;}
.framebox-l {text-align:left;}
.framebox-r {text-align:right;}
span.thank-mark{ vertical-align: super }
span.footnote-mark sup.textsuperscript, span.footnote-mark a sup.textsuperscript{ font-size:80%; }
div.tabular, div.center div.tabular {text-align: center; margin-top:0.5em; margin-bottom:0.5em; }
table.tabular td p{margin-top:0em;}
table.tabular {margin-left: auto; margin-right: auto;}
div.td00{ margin-left:0pt; margin-right:0pt; }
div.td01{ margin-left:0pt; margin-right:5pt; }
div.td10{ margin-left:5pt; margin-right:0pt; }
div.td11{ margin-left:5pt; margin-right:5pt; }
table[rules] {border-left:solid black 0.4pt; border-right:solid black 0.4pt; }
td.td00{ padding-left:0pt; padding-right:0pt; }
td.td01{ padding-left:0pt; padding-right:5pt; }
td.td10{ padding-left:5pt; padding-right:0pt; }
td.td11{ padding-left:5pt; padding-right:5pt; }
table[rules] {border-left:solid black 0.4pt; border-right:solid black 0.4pt; }
.hline hr, .cline hr{ height : 1px; margin:0px; }
.tabbing-right {text-align:right;}
span.TEX {letter-spacing: -0.125em; }
span.TEX span.E{ position:relative;top:0.5ex;left:-0.0417em;}
a span.TEX span.E {text-decoration: none; }
span.LATEX span.A{ position:relative; top:-0.5ex; left:-0.4em; font-size:85%;}
span.LATEX span.TEX{ position:relative; left: -0.4em; }
div.float img, div.float .caption {text-align:center;}
div.figure img, div.figure .caption {text-align:center;}
.marginpar {width:20%; float:right; text-align:left; margin-left:auto; margin-top:0.5em; font-size:85%; text-decoration:underline;}
.marginpar p{margin-top:0.4em; margin-bottom:0.4em;}
.equation td{text-align:center; vertical-align:middle; }
td.eq-no{ width:5%; }
table.equation { width:100%; } 
div.math-display, div.par-math-display{text-align:center;}
math .texttt { font-family: monospace; }
math .textit { font-style: italic; }
math .textsl { font-style: oblique; }
math .textsf { font-family: sans-serif; }
math .textbf { font-weight: bold; }
.partToc a, .partToc, .likepartToc a, .likepartToc {line-height: 200%; font-weight:bold; font-size:110%;}
.chapterToc a, .chapterToc, .likechapterToc a, .likechapterToc, .appendixToc a, .appendixToc {line-height: 200%; font-weight:bold;}
.index-item, .index-subitem, .index-subsubitem {display:block}
.caption td.id{font-weight: bold; white-space: nowrap; }
table.caption {text-align:center;}
h1.partHead{text-align: center}
p.bibitem { text-indent: -2em; margin-left: 2em; margin-top:0.6em; margin-bottom:0.6em; }
p.bibitem-p { text-indent: 0em; margin-left: 2em; margin-top:0.6em; margin-bottom:0.6em; }
.paragraphHead, .likeparagraphHead { margin-top:2em; font-weight: bold;}
.subparagraphHead, .likesubparagraphHead { font-weight: bold;}
.quote {margin-bottom:0.25em; margin-top:0.25em; margin-left:1em; margin-right:1em; text-align:justify;}
.verse{white-space:nowrap; margin-left:2em}
div.maketitle {text-align:center;}
h2.titleHead{text-align:center;}
div.maketitle{ margin-bottom: 2em; }
div.author, div.date {text-align:center;}
div.thanks{text-align:left; margin-left:10%; font-size:85%; font-style:italic; }
div.author{white-space: nowrap;}
.quotation {margin-bottom:0.25em; margin-top:0.25em; margin-left:1em; }
h1.partHead{text-align: center}
.sectionToc, .likesectionToc {margin-left:2em;}
.subsectionToc, .likesubsectionToc {margin-left:4em;}
.subsubsectionToc, .likesubsubsectionToc {margin-left:6em;}
.frenchb-nbsp{font-size:75%;}
.frenchb-thinspace{font-size:75%;}
.figure img.graphics {margin-left:10%;}
/* end css.sty */

\title{Polynomes d'endomorphismes}
\author{}
\date{}

\begin{document}
\maketitle

\textbf{Warning: \href{http://www.math.union.edu/locate/jsMath}{jsMath}
requires JavaScript to process the mathematics on this page.\\ If your
browser supports JavaScript, be sure it is enabled.}

\begin{center}\rule{3in}{0.4pt}\end{center}

{[}\href{coursse17.html}{next}{]} {[}\href{coursse15.html}{prev}{]}
{[}\href{coursse15.html\#tailcoursse15.html}{prev-tail}{]}
{[}\hyperref[tailcoursse16.html]{tail}{]}
{[}\href{coursch4.html\#coursse16.html}{up}{]}

\subsubsection{3.2 Polynômes d'endomorphismes}

\paragraph{3.2.1 Généralités}

Soit E un K-espace vectoriel et u ∈ L(E). On pose \{u\}\^{}\{0\} =\{
\textbackslash{}mathrm\{Id\}\}\_\{E\} et pour k ≥ 1, \{u\}\^{}\{k\} = u
∘ \{u\}\^{}\{k−1\}. Si P ∈ K{[}X{]}, P =\{\textbackslash{}mathop\{
\textbackslash{}mathop\{∑ \}\}
\}\_\{k=0\}\^{}\{d\}\{a\}\_\{k\}\{X\}\^{}\{k\}, on pose

P(u) =\{ \textbackslash{}mathop\{∑ \}\}\_\{k=0\}\^{}\{d\}\{a\}\_\{
k\}\{u\}\^{}\{k\}

Proposition~3.2.1 L'application P\textbackslash{}mathrel\{↦\}P(u) est un
morphisme de K-algèbres de K{[}X{]} dans L(E). Son image K{[}u{]} est la
plus petite sous-algèbre de L(E) contenant u~; elle est commutative.

Démonstration Vérifications immédiates.

\paragraph{3.2.2 Idéal annulateur. Polynôme minimal}

Définition~3.2.1 On appelle idéal annulateur de u ∈ L(E) l'idéal
\{I\}\_\{u\} = \textbackslash{}\{P ∈
K{[}X{]}\textbackslash{}mathrel\{∣\}P(u) = 0\textbackslash{}\}. On dit
que u admet un polynôme minimal si
\{I\}\_\{u\}\textbackslash{}mathrel\{≠\}\textbackslash{}\{0\textbackslash{}\}.
Dans ce cas \{I\}\_\{u\} est engendré par un unique polynôme normalisé
\{μ\}\_\{u\}(X) appelé le polynôme minimal de u.

Exemple~3.2.1 Si E = K{[}X{]} et u :
P(X)\textbackslash{}mathrel\{↦\}XP(X), on a Q(u) :
P(X)\textbackslash{}mathrel\{↦\}Q(X)P(X) soit
Q(u)\textbackslash{}mathrel\{≠\}0 et donc u n'admet pas de polynôme
minimal. Ce phénomène ne peut pas se produire en dimension finie

Théorème~3.2.2 Soit E un K-espace vectoriel de dimension finie et u ∈
L(E). Alors u admet un polynôme minimal.

Démonstration Premier argument~: comme \textbackslash{}mathop\{dim\}
K{[}X{]} = +∞ et \textbackslash{}mathop\{dim\} L(E) \textless{} +∞,
l'application P(X)\textbackslash{}mathrel\{↦\}P(u) ne peut être
injective. Deuxième argument~: comme \textbackslash{}mathop\{dim\} L(E)
= \{n\}\^{}\{2\} (où n =\textbackslash{}mathop\{ dim\} E), la famille de
cardinal \{n\}\^{}\{2\} + 1,
\{(\{u\}\^{}\{k\})\}\_\{0≤k≤\{n\}\^{}\{2\}\} doit être liée~; il existe
donc
\{λ\}\_\{0\},\textbackslash{}mathop\{\textbackslash{}mathop\{\ldots{}\}\},\{λ\}\_\{\{n\}\^{}\{2\}\}
non tous nuls tels que \{λ\}\_\{0\}\{u\}\^{}\{0\} +
\textbackslash{}mathop\{\textbackslash{}mathop\{\ldots{}\}\} +
\{λ\}\_\{\{n\}\^{}\{2\}\}\{u\}\^{}\{\{n\}\^{}\{2\} \} = 0~; le polynôme
\{λ\}\_\{0\}1 +
\textbackslash{}mathop\{\textbackslash{}mathop\{\ldots{}\}\} +
\{λ\}\_\{\{n\}\^{}\{2\}\}\{X\}\^{}\{\{n\}\^{}\{2\} \} n'est pas nul et
il est dans \{I\}\_\{u\}.

On suppose désormais \textbackslash{}mathop\{dim\} E \textless{} +∞

Proposition~3.2.3 Soit F un sous-espace de E stable par u et v
l'endomorphisme de F induit par u. Alors \{μ\}\_\{v\} divise
\{μ\}\_\{u\}.

Démonstration On vérifie facilement que pour tout k dans ℕ,
\{v\}\^{}\{k\} est l'endomorphisme de F induit par \{u\}\^{}\{k\}, donc
si P(X) ∈ K{[}X{]}, P(v) l'endomorphisme de F induit par P(u). En
particulier \{μ\}\_\{u\}(v) est l'endomorphisme de F induit par
\{μ\}\_\{u\}(u) = 0 donc \{μ\}\_\{u\}(v) = 0 et \{μ\}\_\{v\} divise
\{μ\}\_\{u\}.

Théorème~3.2.4 Les racines de \{μ\}\_\{u\} sont exactement les valeurs
propres de u.

Démonstration Soit λ une valeur propre de u et x un vecteur propre
associé. On a \textbackslash{}mathop\{∀\}k ∈ ℕ, \{u\}\^{}\{k\}(x) =
\{λ\}\^{}\{k\}x et donc \textbackslash{}mathop\{∀\}P ∈ K{[}X{]}, P(u)(x)
= P(λ)x. En particulier 0 = \{μ\}\_\{u\}(u)(x) = \{μ\}\_\{u\}(λ)x et
comme x\textbackslash{}mathrel\{≠\}0, on a \{μ\}\_\{u\}(λ) = 0.
Inversement soit λ une racine de \{μ\}\_\{u\} et écrivons
\{μ\}\_\{u\}(X) = (X − λ)Q(X). Supposons que λ n'est pas valeur propre
de u. On a 0 = \{μ\}\_\{u\}(u) = (u −
λ\{\textbackslash{}mathrm\{Id\}\}\_\{E\}) ∘ Q(u). Mais comme λ n'est pas
valeur propre de u, u − λ\{\textbackslash{}mathrm\{Id\}\}\_\{E\} est
injectif et donc Q(u) = 0. Ceci impose que \{μ\}\_\{u\} divise Q ce qui
est impossible pour une question de degrés.

\paragraph{3.2.3 Théorème de Cayley-Hamilton}

Théorème~3.2.5 Soit E un K-espace vectoriel de dimension finie et u ∈
L(E). Alors \{χ\}\_\{u\}(u) = 0.

Remarque~3.2.1 Une version équivalente est~: soit M ∈ \{M\}\_\{K\}(n).
Alors \{χ\}\_\{M\}(M) = 0.

Démonstration Démonstration 1. Soit x ∈ E,
x\textbackslash{}mathrel\{≠\}0 et soit d =\textbackslash{}mathop\{
max\}\textbackslash{}\{k\textbackslash{}mathrel\{∣\}(x,u(x),\textbackslash{}mathop\{\textbackslash{}mathop\{\ldots{}\}\},\{u\}\^{}\{k−1\}(x))\textbackslash{}text\{
libre \}\textbackslash{}\}. On a alors \{u\}\^{}\{d\}(x) = \{λ\}\_\{0\}x
+ \textbackslash{}mathop\{\textbackslash{}mathop\{\ldots{}\}\} +
\{λ\}\_\{d−1\}\{u\}\^{}\{d−1\}(x). Soit \{E\}\_\{x\}
=\textbackslash{}mathop\{
\textbackslash{}mathrm\{Vect\}\}(x,u(x),\textbackslash{}mathop\{\textbackslash{}mathop\{\ldots{}\}\},\{u\}\^{}\{d−1\}(x)).
Alors \{E\}\_\{x\} est stable par u. Soit v la restriction de u à
\{E\}\_\{x\}. \{ℰ\}\_\{x\} =
(x,u(x),\textbackslash{}mathop\{\textbackslash{}mathop\{\ldots{}\}\},\{u\}\^{}\{d−1\}(x))
est une base de \{E\}\_\{x\} et

\textbackslash{}mathop\{\textbackslash{}mathrm\{Mat\}\} (v,\{ℰ\}\_\{x\})
= \textbackslash{}left
(\textbackslash{}matrix\{\textbackslash{},0\&0\&\textbackslash{}mathop\{\textbackslash{}mathop\{\ldots{}\}\}\&\textbackslash{}mathop\{\textbackslash{}mathop\{\ldots{}\}\}\&\{λ\}\_\{0\}
\textbackslash{}cr
1\&0\&\textbackslash{}mathop\{\textbackslash{}mathop\{\ldots{}\}\}\&\textbackslash{}mathop\{\textbackslash{}mathop\{\ldots{}\}\}\&\{λ\}\_\{1\}
\textbackslash{}cr
\textbackslash{}mathop\{\textbackslash{}mathop\{\ldots{}\}\}\&\textbackslash{}mathrel\{⋱\}\&\textbackslash{}mathrel\{⋱\}\&\textbackslash{}mathop\{\textbackslash{}mathop\{\ldots{}\}\}\&\textbackslash{}mathop\{\textbackslash{}mathop\{\ldots{}\}\}
\textbackslash{}cr
\textbackslash{}mathop\{\textbackslash{}mathop\{\ldots{}\}\}\&\textbackslash{}mathop\{\textbackslash{}mathop\{\ldots{}\}\}\&\textbackslash{}mathrel\{⋱\}\&\textbackslash{}mathrel\{⋱\}\&\textbackslash{}mathop\{\textbackslash{}mathop\{\ldots{}\}\}
\textbackslash{}cr
0\&\textbackslash{}mathop\{\textbackslash{}mathop\{\ldots{}\}\}\&\textbackslash{}mathop\{\textbackslash{}mathop\{\ldots{}\}\}\&1\&\{λ\}\_\{d−1\}\}\textbackslash{}right
)

Un calcul simple montre que \{χ\}\_\{v\}(X) = \{X\}\^{}\{d\} −
\{λ\}\_\{d−1\}\{X\}\^{}\{d−1\}
−\textbackslash{}mathop\{\textbackslash{}mathop\{\ldots{}\}\} −
\{λ\}\_\{0\}. On a donc \{χ\}\_\{v\}(v)(x) = 0 et donc
\{χ\}\_\{v\}(u)(x) = 0. Mais \{χ\}\_\{v\} divise \{χ\}\_\{u\} et donc on
a aussi \{χ\}\_\{u\}(u)(x) = 0. Comme x est quelconque, la relation
\{χ\}\_\{u\}(u)(x) = 0 étant évidente si x = 0, on a \{χ\}\_\{u\}(u) =
0.

Démonstration 2. Montrons que si K est un sous-corps de ℂ et M ∈
\{M\}\_\{K\}(n), alors \{χ\}\_\{M\}(M) = 0. Il suffit bien entendu de le
montrer lorsque M ∈ \{M\}\_\{ℂ\}(n). Si M est diagonalisable, on a M =
\{P\}\^{}\{−1\}DP avec D =\textbackslash{}mathop\{
\textbackslash{}mathrm\{diag\}\}(\{λ\}\_\{1\},\textbackslash{}mathop\{\textbackslash{}mathop\{\ldots{}\}\},\{λ\}\_\{n\}).
On a alors \{χ\}\_\{M\}(M) = \{χ\}\_\{D\}(M) =
\{P\}\^{}\{−1\}\{χ\}\_\{D\}(D)P =
\{P\}\^{}\{−1\}\textbackslash{}mathop\{
\textbackslash{}mathrm\{diag\}\}(\{χ\}\_\{D\}(\{λ\}\_\{1\}),\textbackslash{}mathop\{\textbackslash{}mathop\{\ldots{}\}\},\{χ\}\_\{D\}(\{λ\}\_\{n\}))P
= \{P\}\^{}\{−1\}0P = 0. Si M n'est pas diagonalisable, soit
\{M\}\_\{n\} une suite de matrices diagonalisables qui converge vers M.
Comme l'application A\textbackslash{}mathrel\{↦\}\{χ\}\_\{A\}(A) est
polynomiale en les coefficients de A, elle est continue et on a

\{χ\}\_\{M\}(M) =\textbackslash{}mathop\{
lim\}\{χ\}\_\{\{M\}\_\{n\}\}(\{M\}\_\{n\}) =\textbackslash{}mathop\{
lim\}0 = 0

Corollaire~3.2.6 Soit E un K-espace vectoriel de dimension finie et u ∈
L(E). Alors \{μ\}\_\{u\} divise \{χ\}\_\{u\}.

\paragraph{3.2.4 Polynôme annulateur et trigonalisation}

Théorème~3.2.7 Soit E un K-espace vectoriel de dimension finie et u ∈
L(E). Alors u est trigonalisable si et seulement si il existe un
polynôme P ∈ K{[}X{]} ∖ 0, scindé sur K tel que P(u) = 0.

Démonstration Si u est trigonalisable, le polynôme caractéristique
\{χ\}\_\{u\} de u est scindé et vérifie \{χ\}\_\{u\}(u) = 0.

Inversement, supposons qu'il existe un polynôme non nul, scindé P tel
que P(u) = 0. On peut supposer que P est normalisé si bien que l'on peut
écrire P =\{\textbackslash{}mathop\{ \textbackslash{}mathop\{∏ \}\}
\}\_\{i=1\}\^{}\{k\}(X − \{λ\}\_\{i\}). On a 0
=\{\textbackslash{}mathop\{ \textbackslash{}mathop\{∏ \}\}
\}\_\{i=1\}\^{}\{k\}(u − \{λ\}\_\{i\}\textbackslash{}mathrm\{Id\}), si
bien que l'un au moins des u − \{λ\}\_\{i\}\textbackslash{}mathrm\{Id\}
est non injectif. Donc u possède au moins une valeur propre.

Montrons donc le résultat par récurrence sur n =\textbackslash{}mathop\{
dim\} E~; il n'y a rien à démontrer si n = 1. Soit λ une valeur propre
de u. Soit \{e\}\_\{1\} un vecteur propre associé à λ, que l'on complète
en
(\{e\}\_\{1\},\textbackslash{}mathop\{\textbackslash{}mathop\{\ldots{}\}\},\{e\}\_\{n\})
base de E. Soit F =\textbackslash{}mathop\{
\textbackslash{}mathrm\{Vect\}\}(\{e\}\_\{2\},\textbackslash{}mathop\{\textbackslash{}mathop\{\ldots{}\}\},\{e\}\_\{n\}),
p la projection sur F parallèlement à K\{e\}\_\{1\} et v : F → F défini
par v(x) = p(u(x)) si x ∈ F. Alors M =\textbackslash{}mathop\{
\textbackslash{}mathrm\{Mat\}\} (u,ℰ) = \textbackslash{}left
(\textbackslash{}matrix\{\textbackslash{},λ\&∗∗∗ \textbackslash{}cr
\textbackslash{}matrix\{\textbackslash{},0 \textbackslash{}cr
\textbackslash{}mathop\{\textbackslash{}mathop\{⋮\}\} \textbackslash{}cr
0\}\&A \}\textbackslash{}right ) avec A =\textbackslash{}mathop\{
\textbackslash{}mathrm\{Mat\}\}
(v,(\{e\}\_\{2\},\textbackslash{}mathop\{\textbackslash{}mathop\{\ldots{}\}\},\{e\}\_\{n\})).
Le produit de matrices par blocs et une récurrence élémentaire montrent
que \textbackslash{}mathop\{∀\}p ∈ ℕ,

\{M\}\^{}\{p\} = \textbackslash{}left
(\textbackslash{}matrix\{\textbackslash{},λ\&∗∗∗ \textbackslash{}cr
\textbackslash{}matrix\{\textbackslash{},0 \textbackslash{}cr
\textbackslash{}mathop\{\textbackslash{}mathop\{⋮\}\} \textbackslash{}cr
0\}\&\{A\}\^{}\{p\} \}\textbackslash{}right )

et par combinaisons linéaires que

P(M) = \textbackslash{}left
(\textbackslash{}matrix\{\textbackslash{},λ\&∗ ∗ ∗ \textbackslash{}cr
\textbackslash{}matrix\{\textbackslash{},0 \textbackslash{}cr
\textbackslash{}mathop\{\textbackslash{}mathop\{⋮\}\} \textbackslash{}cr
0\}\&P(A)\}\textbackslash{}right )

On en déduit que P(A) = 0 et comme P(A) =\textbackslash{}mathop\{
\textbackslash{}mathrm\{Mat\}\}
(P(v),(\{e\}\_\{2\},\textbackslash{}mathop\{\textbackslash{}mathop\{\ldots{}\}\},\{e\}\_\{n\})),
on a aussi P(v) = 0. Par hypothèse de récurrence, il existe une base
(\{ε\}\_\{2\},\textbackslash{}mathop\{\textbackslash{}mathop\{\ldots{}\}\},\{ε\}\_\{n\})
de F telle que \textbackslash{}mathop\{\textbackslash{}mathrm\{Mat\}\}
(v,(\{ε\}\_\{2\},\textbackslash{}mathop\{\textbackslash{}mathop\{\ldots{}\}\},\{ε\}\_\{n\}))
soit triangulaire supérieure et alors
\textbackslash{}mathop\{\textbackslash{}mathrm\{Mat\}\}
(u,(\{e\}\_\{1\},\{ε\}\_\{2\},\textbackslash{}mathop\{\textbackslash{}mathop\{\ldots{}\}\},\{ε\}\_\{n\}))
= \textbackslash{}left (\textbackslash{}matrix\{\textbackslash{},λ\&∗
\textbackslash{}cr \textbackslash{}matrix\{\textbackslash{},0
\textbackslash{}cr \textbackslash{}mathop\{\textbackslash{}mathop\{⋮\}\}
\textbackslash{}cr
0\}\&\textbackslash{}mathop\{\textbackslash{}mathrm\{Mat\}\}
(v,(\{ε\}\_\{2\},\textbackslash{}mathop\{\textbackslash{}mathop\{\ldots{}\}\},\{ε\}\_\{n\}))\}\textbackslash{}right
) est triangulaire supérieure.

\paragraph{3.2.5 Décomposition des noyaux}

Théorème~3.2.8 Soit E un K-espace vectoriel et u ∈ L(E). Soit P ∈
K{[}X{]} et P =
\{P\}\_\{1\}\textbackslash{}mathop\{\textbackslash{}mathop\{\ldots{}\}\}\{P\}\_\{k\}
une décomposition de P en produit de polynômes deux à deux premiers
entre eux. Alors
\textbackslash{}mathop\{\textbackslash{}mathrm\{Ker\}\}P(u)
=\textbackslash{}mathop\{ \textbackslash{}mathrm\{Ker\}\}\{P\}\_\{1\}(u)
⊕\textbackslash{}mathrel\{⋯\}
⊕\textbackslash{}mathop\{\textbackslash{}mathrm\{Ker\}\}\{P\}\_\{k\}(u).

Démonstration Par récurrence sur k ≥ 2. Pour k = 2, écrivons P =
\{P\}\_\{1\}\{P\}\_\{2\} avec \{P\}\_\{1\} et \{P\}\_\{2\} premiers
entre eux. Soit U et V tels que U\{P\}\_\{1\} + V \{P\}\_\{2\} = 1
(Bezout). On a déjà
\textbackslash{}mathop\{\textbackslash{}mathrm\{Ker\}\}\{P\}\_\{i\}(u)
⊂\textbackslash{}mathop\{\textbackslash{}mathrm\{Ker\}\}P(u). De plus on
a

U(u)\{P\}\_\{1\}(u) + V (u)\{P\}\_\{2\}(u) =\{
\textbackslash{}mathrm\{Id\}\}\_\{E\}

Soit x
∈\textbackslash{}mathop\{\textbackslash{}mathrm\{Ker\}\}\{P\}\_\{1\}(u)
∩\textbackslash{}mathop\{\textbackslash{}mathrm\{Ker\}\}\{P\}\_\{2\}(u).
On a x = U(u)\{P\}\_\{1\}(u)(x) + V (u)\{P\}\_\{2\}(u)(x) = 0 + 0 = 0,
donc
\textbackslash{}mathop\{\textbackslash{}mathrm\{Ker\}\}\{P\}\_\{1\}(u)
∩\textbackslash{}mathop\{\textbackslash{}mathrm\{Ker\}\}\{P\}\_\{2\}(u)
= \textbackslash{}\{0\textbackslash{}\}. Soit maintenant x
∈\textbackslash{}mathop\{\textbackslash{}mathrm\{Ker\}\}P(u). On a
encore x = \{x\}\_\{1\} + \{x\}\_\{2\} avec \{x\}\_\{1\} = V
(u)\{P\}\_\{2\}(u)(x) et \{x\}\_\{2\} = U(u)\{P\}\_\{1\}(u)(x). Mais
alors

\textbackslash{}begin\{eqnarray*\}\{ P\}\_\{1\}(u)(\{x\}\_\{1\})\& =\&
\{P\}\_\{1\}(u)V (u)\{P\}\_\{2\}(u)(x) = V
(u)(\{P\}\_\{1\}\{P\}\_\{2\})(u)(x)\%\& \textbackslash{}\textbackslash{}
\& =\& V (u)P(u)(x) = 0 \%\& \textbackslash{}\textbackslash{}
\textbackslash{}end\{eqnarray*\}

donc \{x\}\_\{1\}
∈\textbackslash{}mathop\{\textbackslash{}mathrm\{Ker\}\}\{P\}\_\{1\}(u).
De même \{x\}\_\{2\}
∈\textbackslash{}mathop\{\textbackslash{}mathrm\{Ker\}\}\{P\}\_\{2\}(u)
et donc \textbackslash{}mathop\{\textbackslash{}mathrm\{Ker\}\}P(u)
=\textbackslash{}mathop\{ \textbackslash{}mathrm\{Ker\}\}\{P\}\_\{1\}(u)
⊕\textbackslash{}mathop\{\textbackslash{}mathrm\{Ker\}\}\{P\}\_\{2\}(u).
Supposons donc le résultat vrai pour k − 1. Comme \{P\}\_\{k\} et
\{P\}\_\{1\}\textbackslash{}mathop\{\textbackslash{}mathop\{\ldots{}\}\}\{P\}\_\{k−1\}
sont premiers entre eux, le cas k = 2 nous donne
\textbackslash{}mathop\{\textbackslash{}mathrm\{Ker\}\}P(u)
=\textbackslash{}mathop\{
\textbackslash{}mathrm\{Ker\}\}\{P\}\_\{1\}\textbackslash{}mathop\{\textbackslash{}mathop\{\ldots{}\}\}\{P\}\_\{k−1\}(u)
⊕\textbackslash{}mathop\{\textbackslash{}mathrm\{Ker\}\}\{P\}\_\{k\}(u)
puis grâce à l'hypothèse de récurrence
\textbackslash{}mathop\{\textbackslash{}mathrm\{Ker\}\}P(u)
=\textbackslash{}mathop\{ \textbackslash{}mathrm\{Ker\}\}\{P\}\_\{1\}(u)
⊕\textbackslash{}mathrel\{⋯\}
⊕\textbackslash{}mathop\{\textbackslash{}mathrm\{Ker\}\}\{P\}\_\{k\}(u).

Corollaire~3.2.9 Soit E un K-espace vectoriel et u ∈ L(E). Soit P ∈
K{[}X{]} et P =
\{P\}\_\{1\}\textbackslash{}mathop\{\textbackslash{}mathop\{\ldots{}\}\}\{P\}\_\{k\}
une décomposition de P en produit de polynômes deux à deux premiers
entre eux. On suppose que P(u) = 0. Alors E =\textbackslash{}mathop\{
\textbackslash{}mathrm\{Ker\}\}\{P\}\_\{1\}(u)
⊕\textbackslash{}mathrel\{⋯\}
⊕\textbackslash{}mathop\{\textbackslash{}mathrm\{Ker\}\}\{P\}\_\{k\}(u).

Proposition~3.2.10

\begin{itemize}
\itemsep1pt\parskip0pt\parsep0pt
\item
  (i) Soit E un K-espace vectoriel et u ∈ L(E) tel que \{u\}\^{}\{2\} =
  u. Alors u est un projecteur
\item
  (ii) Soit E un K-espace vectoriel et u ∈ L(E) tel que \{u\}\^{}\{2\} =
  \textbackslash{}mathrm\{Id\}. Si
  \textbackslash{}mathop\{car\}K\textbackslash{}mathrel\{≠\}2, alors u
  est la symétrie par rapport à un sous-espace V parallèlement à un
  supplémentaire.
\end{itemize}

Démonstration

\begin{itemize}
\itemsep1pt\parskip0pt\parsep0pt
\item
  (i) On écrit \{X\}\^{}\{2\} − X = X(X − 1)~: les polynômes X et X − 1
  sont premiers entre eux d'où E =\textbackslash{}mathop\{
  \textbackslash{}mathrm\{Ker\}\}(u −\textbackslash{}mathrm\{Id\})
  ⊕\textbackslash{}mathop\{\textbackslash{}mathrm\{Ker\}\}u. Alors u est
  la projection sur
  \textbackslash{}mathop\{\textbackslash{}mathrm\{Ker\}\}(u
  −\textbackslash{}mathrm\{Id\}) parallèlement à
  \textbackslash{}mathop\{\textbackslash{}mathrm\{Ker\}\}u.
\item
  (ii) On écrit \{X\}\^{}\{2\} − 1 = (X − 1)(X + 1)~: les polynômes X +
  1 et X − 1 sont premiers entre eux (si
  \textbackslash{}mathop\{car\}K\textbackslash{}mathrel\{≠\}2) d'où E
  =\textbackslash{}mathop\{ \textbackslash{}mathrm\{Ker\}\}(u
  −\textbackslash{}mathrm\{Id\})
  ⊕\textbackslash{}mathop\{\textbackslash{}mathrm\{Ker\}\}(u +
  \textbackslash{}mathrm\{Id\}). Alors u est la symétrie par rapport à
  \textbackslash{}mathop\{\textbackslash{}mathrm\{Ker\}\}(u
  −\textbackslash{}mathrm\{Id\}) parallèlement à
  \textbackslash{}mathop\{\textbackslash{}mathrm\{Ker\}\}(u +
  \textbackslash{}mathrm\{Id\}).
\end{itemize}

Théorème~3.2.11 Soit E un K-espace vectoriel de dimension finie et u ∈
L(E). Alors u est diagonalisable si et seulement si il existe P ∈
K{[}X{]} scindé à racines simples tel que P(u) = 0.

Démonstration

\begin{itemize}
\itemsep1pt\parskip0pt\parsep0pt
\item
  ( ⇒). Soit ℰ une base de E telle que D =\textbackslash{}mathop\{
  \textbackslash{}mathrm\{Mat\}\} (u,ℰ) soit diagonale, D
  =\textbackslash{}mathop\{
  \textbackslash{}mathrm\{diag\}\}(\{λ\}\_\{1\},\textbackslash{}mathop\{\textbackslash{}mathop\{\ldots{}\}\},\{λ\}\_\{n\}).
  Supposons que
  \{λ\}\_\{1\},\textbackslash{}mathop\{\textbackslash{}mathop\{\ldots{}\}\},\{λ\}\_\{k\}
  sont distinctes et que pour i ≥ k + 1, \{λ\}\_\{i\}
  ∈\textbackslash{}\{\{λ\}\_\{1\},\textbackslash{}mathop\{\textbackslash{}mathop\{\ldots{}\}\}\{λ\}\_\{k\}\textbackslash{}\}.
  Soit P =\{\textbackslash{}mathop\{ \textbackslash{}mathop\{∏ \}\}
  \}\_\{i=1\}\^{}\{k\}(X − \{λ\}\_\{i\}). On a P(D)
  =\textbackslash{}mathop\{
  \textbackslash{}mathrm\{diag\}\}(P(\{λ\}\_\{1\}),\textbackslash{}mathop\{\textbackslash{}mathop\{\ldots{}\}\},P(\{λ\}\_\{n\}))
  = 0 et donc P(u) = 0 avec P scindé à racines simples.
\item
  ( ⇐) Soit P(X) =\{\textbackslash{}mathop\{ \textbackslash{}mathop\{∏
  \}\} \}\_\{i=1\}\^{}\{k\}(X − \{λ\}\_\{i\}). On a donc E
  =\textbackslash{}mathop\{ \textbackslash{}mathrm\{Ker\}\}(u −
  \{λ\}\_\{1\}\textbackslash{}mathrm\{Id\})
  ⊕\textbackslash{}mathrel\{⋯\}
  ⊕\textbackslash{}mathop\{\textbackslash{}mathrm\{Ker\}\}(u −
  \{λ\}\_\{k\}\textbackslash{}mathrm\{Id\}). En réunissant des bases de
  tous ces sous-espaces, on obtient une base de E formée de vecteurs
  propres de u. Donc u est diagonalisable.
\end{itemize}

Remarque~3.2.2 Ce théorème peut encore s'exprimer sous la forme~: u est
diagonalisable si et seulement si \{μ\}\_\{u\} est scindé à racines
simples.

\paragraph{3.2.6 Sous-espaces caractéristiques}

Remarque~3.2.3 Soit E un K-espace vectoriel de dimension finie, u ∈ L(E)
et P tel que P(u) = 0. Soit P =
\{P\}\_\{1\}\textbackslash{}mathop\{\textbackslash{}mathop\{\ldots{}\}\}\{P\}\_\{k\}
une décomposition de \{χ\}\_\{u\} en produit de polynômes deux à deux
premiers entre eux. Soit \{E\}\_\{i\} =\textbackslash{}mathop\{
\textbackslash{}mathrm\{Ker\}\}\{P\}\_\{i\}(u). On a E = \{E\}\_\{1\}
⊕\textbackslash{}mathrel\{⋯\} ⊕ \{E\}\_\{k\} et chacun des sous-espaces
\{E\}\_\{i\} est stable par u. Soit \{u\}\_\{i\} la restriction de u à
\{E\}\_\{i\}. Dans une base ℰ = \{ℰ\}\_\{1\}
∪\textbackslash{}mathop\{\textbackslash{}mathop\{\ldots{}\}\}
∪\{ℰ\}\_\{k\} adaptée à la décomposition en somme directe, on a M
=\textbackslash{}mathop\{ \textbackslash{}mathrm\{Mat\}\} (u,ℰ)
=\textbackslash{}mathop\{
\textbackslash{}mathrm\{diag\}\}(\{M\}\_\{1\},\textbackslash{}mathop\{\textbackslash{}mathop\{\ldots{}\}\},\{M\}\_\{k\})
avec \{M\}\_\{i\} =\textbackslash{}mathop\{
\textbackslash{}mathrm\{Mat\}\} (\{u\}\_\{i\},\{ℰ\}\_\{i\}). On en
déduit (calcul par blocs d'un déterminant) que \{χ\}\_\{u\}
=\textbackslash{}mathop\{ \textbackslash{}mathop\{∏ \}\}
\{χ\}\_\{\{u\}\_\{i\}\}. De même, on a, si Q ∈ K{[}X{]}, Q(M)
=\textbackslash{}mathop\{
\textbackslash{}mathrm\{diag\}\}(Q(\{M\}\_\{1\}),\textbackslash{}mathop\{\textbackslash{}mathop\{\ldots{}\}\},Q(\{M\}\_\{k\}))
et donc Q(u) = 0 \textbackslash{}mathrel\{⇔\}
\textbackslash{}mathop\{∀\}i, Q(\{u\}\_\{i\}) = 0. On en déduit que
\{μ\}\_\{u\} =\textbackslash{}mathop\{ ppcm\}\{μ\}\_\{\{u\}\_\{i\}\}.
Mais \{P\}\_\{i\}(\{u\}\_\{i\}) = 0 et donc \{μ\}\_\{\{u\}\_\{i\}\}
divise \{P\}\_\{i\}. On en déduit que les \{μ\}\_\{\{u\}\_\{i\}\} sont
deux à deux premiers entre eux et donc \{μ\}\_\{u\}
=\textbackslash{}mathop\{ \textbackslash{}mathop\{∏ \}\}
\{μ\}\_\{\{u\}\_\{i\}\}.

Supposons maintenant que le polynôme caractéristique de u est scindé,
\{χ\}\_\{u\}(X) = \{\textbackslash{}mathop\{\textbackslash{}mathop\{∏
\}\} \}\_\{i=1\}\^{}\{k\}\{(X − \{λ\}\_\{i\})\}\^{}\{\{m\}\_\{i\}\} avec
\{λ\}\_\{1\},\textbackslash{}mathop\{\textbackslash{}mathop\{\ldots{}\}\},\{λ\}\_\{k\}
distincts. Appliquons les résultats précédents avec P = \{χ\}\_\{u\} et
\{P\}\_\{i\} = \{(X − \{λ\}\_\{i\})\}\^{}\{\{m\}\_\{i\}\}. Ceci nous
conduit à la définition~:

Définition~3.2.2 Soit u ∈ L(E) et λ une valeur propre de u de
multiplicité m. Le sous-espace
\textbackslash{}mathop\{\textbackslash{}mathrm\{Ker\}\}\{(u −
λ\textbackslash{}mathrm\{Id\})\}\^{}\{m\} est appelé sous-espace
caractéristique de u associé à λ, il est stable par u.

Remarque~3.2.4 Appelons donc \{E\}\_\{i\} le sous-espace caractéristique
de u associé à \{λ\}\_\{i\}, et \{u\}\_\{i\} la restriction de u à
\{E\}\_\{i\}. On sait que \{μ\}\_\{\{u\}\_\{i\}\} divise \{(X −
\{λ\}\_\{i\})\}\^{}\{\{m\}\_\{i\}\}, donc \{μ\}\_\{\{u\}\_\{i\}\} = \{(X
− \{λ\}\_\{i\})\}\^{}\{\{r\}\_\{i\}\}. Dans ce cas, \{μ\}\_\{u\}(X)
=\{\textbackslash{}mathop\{ \textbackslash{}mathop\{∏ \}\}
\}\_\{i=1\}\^{}\{k\}\{(X − \{λ\}\_\{i\})\}\^{}\{\{r\}\_\{i\}\}. De plus,
\{χ\}\_\{u\}(X) =\textbackslash{}mathop\{ \textbackslash{}mathop\{∏ \}\}
\{χ\}\_\{\{u\}\_\{i\}\}, chacun des \{χ\}\_\{\{u\}\_\{i\}\} est scindé
et a les mêmes racines que \{μ\}\_\{\{u\}\_\{i\}\}. On en déduit que
\{χ\}\_\{\{u\}\_\{i\}\}(X) = \{(X −
\{λ\}\_\{i\})\}\^{}\{\textbackslash{}mathop\{dim\} \{E\}\_\{i\}\}. Mais
alors \{χ\}\_\{u\}(X) =\{\textbackslash{}mathop\{
\textbackslash{}mathop\{∏ \}\} \}\_\{i=1\}\^{}\{k\}\{(X −
\{λ\}\_\{i\})\}\^{}\{\{m\}\_\{i\}\} =\{\textbackslash{}mathop\{
\textbackslash{}mathop\{∏\}\} \}\_\{i=1\}\^{}\{k\}\{(X −
\{λ\}\_\{i\})\}\^{}\{\textbackslash{}mathop\{dim\} \{E\}\_\{i\}\}. Ceci
démontre que \textbackslash{}mathop\{dim\} \{E\}\_\{i\} = \{m\}\_\{i\}.
D'où le théorème

Théorème~3.2.12 Soit E un K-espace vectoriel de dimension finie et u ∈
L(E) dont le polynôme caractéristique est scindé sur K. Alors E est
somme directe des sous-espaces caractéristiques de u~; chaque
sous-espace caractéristique a pour dimension la multiplicité de la
valeur propre correspondante.

\paragraph{3.2.7 Application~: récurrences linéaires d'ordre 2}

Théorème~3.2.13 Soit E un K-espace vectoriel de dimension 2 et u ∈ L(E)
dont le polynôme caractéristique est scindé. Alors, il existe une base ℰ
de E telle que la matrice de u dans cette base soit de l'une des deux
formes suivantes~: \textbackslash{}left
(\textbackslash{}matrix\{\textbackslash{},\{λ\}\_\{1\}\&0
\textbackslash{}cr 0 \&\{λ\}\_\{2\}\}\textbackslash{}right ) ou
\textbackslash{}left
(\textbackslash{}matrix\{\textbackslash{},λ\&1\textbackslash{}cr 0
\&λ\}\textbackslash{}right ).

Démonstration Si u est diagonalisable, il existe une base ℰ de E telle
que la matrice de u dans cette base soit \textbackslash{}left
(\textbackslash{}matrix\{\textbackslash{},\{λ\}\_\{1\}\&0
\textbackslash{}cr 0 \&\{λ\}\_\{2\}\}\textbackslash{}right ). Supposons
donc u non diagonalisable. Le polynôme caractéristique de u a
nécessairement une racine double λ (sinon u serait diagonalisable) et le
sous-espace propre associé \{E\}\_\{u\}(λ) est nécessairement de
dimension 1 (pour la même raison). Soit donc \{e\}\_\{2\} ∈ E ∖
\{E\}\_\{u\}(λ) et posons \{e\}\_\{1\} = u(\{e\}\_\{2\}) − λ\{e\}\_\{2\}
= (u − λ\textbackslash{}mathrm\{Id\})(\{e\}\_\{2\})~; le vecteur
\{e\}\_\{1\} est non nul car \{e\}\_\{2\} n'est pas vecteur propre de u.
Le théorème de Cayley Hamilton garantit que \{(u −
λ\textbackslash{}mathrm\{Id\})\}\^{}\{2\} = 0 et donc (u −
λ\textbackslash{}mathrm\{Id\})(\{e\}\_\{1\}) = 0. Donc \{e\}\_\{1\} est
vecteur propre de u. Ceci garantit que (\{e\}\_\{1\},\{e\}\_\{2\}) est
libre (puisque \{e\}\_\{2\} n'est pas vecteur propre de u), et donc est
une base de E. On a u(\{e\}\_\{1\}) = λ\{e\}\_\{1\} et u(\{e\}\_\{2\}) =
\{e\}\_\{1\} + λ\{e\}\_\{2\} et donc la matrice de u dans cette base est
\textbackslash{}left
(\textbackslash{}matrix\{\textbackslash{},λ\&1\textbackslash{}cr 0
\&λ\}\textbackslash{}right ).

Proposition~3.2.14 Soit n ∈ ℕ, alors

\{ \textbackslash{}left
(\textbackslash{}matrix\{\textbackslash{},\{λ\}\_\{1\}\&0
\textbackslash{}cr 0 \&\{λ\}\_\{2\}\}\textbackslash{}right )\}\^{}\{n\}
= \textbackslash{}left
(\textbackslash{}matrix\{\textbackslash{},\{λ\}\_\{1\}\^{}\{n\}\&0
\textbackslash{}cr 0 \&\{λ\}\_\{2\}\^{}\{n\}\}\textbackslash{}right )

et

\{ \textbackslash{}left
(\textbackslash{}matrix\{\textbackslash{},λ\&1\textbackslash{}cr 0
\&λ\}\textbackslash{}right )\}\^{}\{n\} = \textbackslash{}left
(\textbackslash{}matrix\{\textbackslash{},\{λ\}\^{}\{n\}\&n\{λ\}\^{}\{n−1\}
\textbackslash{}cr 0 \&\{λ\}\^{}\{n\} \}\textbackslash{}right )

Démonstration La première formule est évidente~; la deuxième peut se
montrer soit par récurrence, soit en appliquant la formule du binôme~;
on écrit que

\textbackslash{}left
(\textbackslash{}matrix\{\textbackslash{},λ\&1\textbackslash{}cr 0
\&λ\}\textbackslash{}right ) = λ\{I\}\_\{2\} + \textbackslash{}left
(\textbackslash{}matrix\{\textbackslash{},0\&1 \textbackslash{}cr
0\&0\}\textbackslash{}right )

et on remarque que \{\textbackslash{}left
(\textbackslash{}matrix\{\textbackslash{},0\&1 \textbackslash{}cr
0\&0\}\textbackslash{}right )\}\^{}\{2\} = 0.

Considérons a,b ∈ ℂ, b\textbackslash{}mathrel\{≠\}0 et soit E l'ensemble
des suites \{(\{u\}\_\{n\})\}\_\{n∈ℕ\} de nombres complexes vérifiant

\textbackslash{}mathop\{∀\}n ∈ ℕ, \{u\}\_\{n+2\} = a\{u\}\_\{n+1\} +
b\{u\}\_\{n\}

Proposition~3.2.15 E est un ℂ-espace vectoriel et l'application E →
\{ℂ\}\^{}\{2\},
\{(\{u\}\_\{n\})\}\_\{n∈ℕ\}\textbackslash{}mathrel\{↦\}(\{u\}\_\{0\},\{u\}\_\{1\})
est un isomorphisme d'espaces vectoriels~; en particulier,
\textbackslash{}mathop\{dim\} E = 2.

Démonstration La vérification du premier point est élémentaire.
L'application
\{(\{u\}\_\{n\})\}\_\{n∈ℕ\}\textbackslash{}mathrel\{↦\}(\{u\}\_\{0\},\{u\}\_\{1\})
est visiblement linéaire et elle est bijective car une suite de E est
entièrement déterminée par la donnée de ses deux premiers éléments.

Soit \{(\{u\}\_\{n\})\}\_\{n\}∈ ℕ ∈ E et posons \{U\}\_\{n\} =
\textbackslash{}left
(\textbackslash{}matrix\{\textbackslash{},\{u\}\_\{n\}
\textbackslash{}cr \{u\}\_\{n+1\}\}\textbackslash{}right ). On a alors

\textbackslash{}begin\{eqnarray*\}\{ U\}\_\{n+1\}\& =\&
\textbackslash{}left
(\textbackslash{}matrix\{\textbackslash{},\{u\}\_\{n+1\}
\textbackslash{}cr \{u\}\_\{n+2\}\}\textbackslash{}right ) =
\textbackslash{}left
(\textbackslash{}matrix\{\textbackslash{},\{u\}\_\{n+1\}
\textbackslash{}cr a\{u\}\_\{n+1\} +
b\{u\}\_\{n\}\}\textbackslash{}right )\%\&
\textbackslash{}\textbackslash{} \& =\& \textbackslash{}left
(\textbackslash{}matrix\{\textbackslash{},0\&1 \textbackslash{}cr
b\&a\}\textbackslash{}right )\textbackslash{}left
(\textbackslash{}matrix\{\textbackslash{},\{u\}\_\{n\}
\textbackslash{}cr \{u\}\_\{n+1\}\}\textbackslash{}right ) =
A\{U\}\_\{n\} \%\& \textbackslash{}\textbackslash{}
\textbackslash{}end\{eqnarray*\}

avec A = \textbackslash{}left
(\textbackslash{}matrix\{\textbackslash{},0\&1 \textbackslash{}cr
b\&a\}\textbackslash{}right ). On en déduit que \{U\}\_\{n\} =
\{A\}\^{}\{n\}\{U\}\_\{0\}. Comme ℂ est algébriquement clos,
\{χ\}\_\{A\} est scindé. Si A est diagonalisable, il existe P inversible
telle que A = \{P\}\^{}\{−1\}\textbackslash{}left
(\textbackslash{}matrix\{\textbackslash{},\{λ\}\_\{1\}\&0
\textbackslash{}cr 0 \&\{λ\}\_\{2\}\}\textbackslash{}right )P, d'où

\{U\}\_\{n\} = \{P\}\^{}\{−1\}\textbackslash{}left
(\textbackslash{}matrix\{\textbackslash{},\{λ\}\_\{1\}\^{}\{n\}\&0
\textbackslash{}cr 0 \&\{λ\}\_\{2\}\^{}\{n\}\}\textbackslash{}right
)P\{U\}\_\{0\}

et en prenant la première coordonnée, \{u\}\_\{n\} =
α\{λ\}\_\{1\}\^{}\{n\} + β\{λ\}\_\{2\}\^{}\{n\}. Si par contre, A n'est
pas diagonalisable, il existe P inversible telle que A =
\{P\}\^{}\{−1\}\textbackslash{}left
(\textbackslash{}matrix\{\textbackslash{},λ\&1\textbackslash{}cr 0
\&λ\}\textbackslash{}right )P, d'où

\{U\}\_\{n\} = \{P\}\^{}\{−1\}\textbackslash{}left
(\textbackslash{}matrix\{\textbackslash{},\{λ\}\^{}\{n\}\&n\{λ\}\^{}\{n−1\}
\textbackslash{}cr 0 \&\{λ\}\^{}\{n\} \}\textbackslash{}right
)P\{U\}\_\{0\}

et en prenant la première coordonnée, \{u\}\_\{n\} = α\{λ\}\^{}\{n\} +
βn\{λ\}\^{}\{n\}.

On a donc E ⊂ F avec F =\textbackslash{}mathop\{
\textbackslash{}mathrm\{Vect\}\}(\{(\{λ\}\_\{1\}\^{}\{n\})\}\_\{n∈ℕ\},\{(\{λ\}\_\{2\}\^{}\{n\})\}\_\{n∈ℕ\})
ou F =\textbackslash{}mathop\{
\textbackslash{}mathrm\{Vect\}\}(\{(\{λ\}\^{}\{n\})\}\_\{n∈ℕ\},\{(n\{λ\}\^{}\{n\})\}\_\{n∈ℕ\})
suivant le cas. Comme \textbackslash{}mathop\{dim\} E = 2 et
\textbackslash{}mathop\{dim\} F ≤ 2, on a nécessairement égalité.
Remarquons alors que, pour λ\textbackslash{}mathrel\{≠\}0,

\{(\{λ\}\^{}\{n\})\}\_\{ n∈ℕ\} ∈ E\textbackslash{}quad
\textbackslash{}mathrel\{⇔\} \{λ\}\^{}\{2\} = aλ + b
\textbackslash{}mathrel\{⇔\} \{χ\}\_\{ A\}(λ) = 0

Ceci nous conduit à la méthode suivante de résolution de la récurrence
linéaire \textbackslash{}mathop\{∀\}n ∈ ℕ, \{u\}\_\{n+2\} =
a\{u\}\_\{n+1\} + b\{u\}\_\{n\}

\begin{itemize}
\itemsep1pt\parskip0pt\parsep0pt
\item
  rechercher les solutions particulières de la forme \{u\}\_\{n\} =
  \{λ\}\^{}\{n\} ceci conduit à une équation du second degré en λ, P(λ)
  = 0
\item
  si cette équation a deux racines simples \{λ\}\_\{1\} et \{λ\}\_\{2\},
  les solutions sont les suites de la forme \{u\}\_\{n\} =
  α\{λ\}\_\{1\}\^{}\{n\} + β\{λ\}\_\{2\}\^{}\{n\}
\item
  si cette équation a une racine double λ, les solutions sont les suites
  de la forme \{u\}\_\{n\} = α\{λ\}\^{}\{n\} + βn\{λ\}\^{}\{n\}.
\end{itemize}

{[}\href{coursse17.html}{next}{]} {[}\href{coursse15.html}{prev}{]}
{[}\href{coursse15.html\#tailcoursse15.html}{prev-tail}{]}
{[}\href{coursse16.html}{front}{]}
{[}\href{coursch4.html\#coursse16.html}{up}{]}

\end{document}
