\documentclass[]{article}
\usepackage[T1]{fontenc}
\usepackage{lmodern}
\usepackage{amssymb,amsmath}
\usepackage{ifxetex,ifluatex}
\usepackage{fixltx2e} % provides \textsubscript
% use upquote if available, for straight quotes in verbatim environments
\IfFileExists{upquote.sty}{\usepackage{upquote}}{}
\ifnum 0\ifxetex 1\fi\ifluatex 1\fi=0 % if pdftex
  \usepackage[utf8]{inputenc}
\else % if luatex or xelatex
  \ifxetex
    \usepackage{mathspec}
    \usepackage{xltxtra,xunicode}
  \else
    \usepackage{fontspec}
  \fi
  \defaultfontfeatures{Mapping=tex-text,Scale=MatchLowercase}
  \newcommand{\euro}{€}
\fi
% use microtype if available
\IfFileExists{microtype.sty}{\usepackage{microtype}}{}
\ifxetex
  \usepackage[setpagesize=false, % page size defined by xetex
              unicode=false, % unicode breaks when used with xetex
              xetex]{hyperref}
\else
  \usepackage[unicode=true]{hyperref}
\fi
\hypersetup{breaklinks=true,
            bookmarks=true,
            pdfauthor={},
            pdftitle={},
            colorlinks=true,
            citecolor=blue,
            urlcolor=blue,
            linkcolor=magenta,
            pdfborder={0 0 0}}
\urlstyle{same}  % don't use monospace font for urls
\setlength{\parindent}{0pt}
\setlength{\parskip}{6pt plus 2pt minus 1pt}
\setlength{\emergencystretch}{3em}  % prevent overfull lines
\setcounter{secnumdepth}{0}
 
/* start css.sty */
.cmr-5{font-size:50%;}
.cmr-7{font-size:70%;}
.cmmi-5{font-size:50%;font-style: italic;}
.cmmi-7{font-size:70%;font-style: italic;}
.cmmi-10{font-style: italic;}
.cmsy-5{font-size:50%;}
.cmsy-7{font-size:70%;}
.cmex-7{font-size:70%;}
.cmex-7x-x-71{font-size:49%;}
.msbm-7{font-size:70%;}
.cmtt-10{font-family: monospace;}
.cmti-10{ font-style: italic;}
.cmbx-10{ font-weight: bold;}
.cmr-17x-x-120{font-size:204%;}
.cmsl-10{font-style: oblique;}
.cmti-7x-x-71{font-size:49%; font-style: italic;}
.cmbxti-10{ font-weight: bold; font-style: italic;}
p.noindent { text-indent: 0em }
td p.noindent { text-indent: 0em; margin-top:0em; }
p.nopar { text-indent: 0em; }
p.indent{ text-indent: 1.5em }
@media print {div.crosslinks {visibility:hidden;}}
a img { border-top: 0; border-left: 0; border-right: 0; }
center { margin-top:1em; margin-bottom:1em; }
td center { margin-top:0em; margin-bottom:0em; }
.Canvas { position:relative; }
li p.indent { text-indent: 0em }
.enumerate1 {list-style-type:decimal;}
.enumerate2 {list-style-type:lower-alpha;}
.enumerate3 {list-style-type:lower-roman;}
.enumerate4 {list-style-type:upper-alpha;}
div.newtheorem { margin-bottom: 2em; margin-top: 2em;}
.obeylines-h,.obeylines-v {white-space: nowrap; }
div.obeylines-v p { margin-top:0; margin-bottom:0; }
.overline{ text-decoration:overline; }
.overline img{ border-top: 1px solid black; }
td.displaylines {text-align:center; white-space:nowrap;}
.centerline {text-align:center;}
.rightline {text-align:right;}
div.verbatim {font-family: monospace; white-space: nowrap; text-align:left; clear:both; }
.fbox {padding-left:3.0pt; padding-right:3.0pt; text-indent:0pt; border:solid black 0.4pt; }
div.fbox {display:table}
div.center div.fbox {text-align:center; clear:both; padding-left:3.0pt; padding-right:3.0pt; text-indent:0pt; border:solid black 0.4pt; }
div.minipage{width:100%;}
div.center, div.center div.center {text-align: center; margin-left:1em; margin-right:1em;}
div.center div {text-align: left;}
div.flushright, div.flushright div.flushright {text-align: right;}
div.flushright div {text-align: left;}
div.flushleft {text-align: left;}
.underline{ text-decoration:underline; }
.underline img{ border-bottom: 1px solid black; margin-bottom:1pt; }
.framebox-c, .framebox-l, .framebox-r { padding-left:3.0pt; padding-right:3.0pt; text-indent:0pt; border:solid black 0.4pt; }
.framebox-c {text-align:center;}
.framebox-l {text-align:left;}
.framebox-r {text-align:right;}
span.thank-mark{ vertical-align: super }
span.footnote-mark sup.textsuperscript, span.footnote-mark a sup.textsuperscript{ font-size:80%; }
div.tabular, div.center div.tabular {text-align: center; margin-top:0.5em; margin-bottom:0.5em; }
table.tabular td p{margin-top:0em;}
table.tabular {margin-left: auto; margin-right: auto;}
div.td00{ margin-left:0pt; margin-right:0pt; }
div.td01{ margin-left:0pt; margin-right:5pt; }
div.td10{ margin-left:5pt; margin-right:0pt; }
div.td11{ margin-left:5pt; margin-right:5pt; }
table[rules] {border-left:solid black 0.4pt; border-right:solid black 0.4pt; }
td.td00{ padding-left:0pt; padding-right:0pt; }
td.td01{ padding-left:0pt; padding-right:5pt; }
td.td10{ padding-left:5pt; padding-right:0pt; }
td.td11{ padding-left:5pt; padding-right:5pt; }
table[rules] {border-left:solid black 0.4pt; border-right:solid black 0.4pt; }
.hline hr, .cline hr{ height : 1px; margin:0px; }
.tabbing-right {text-align:right;}
span.TEX {letter-spacing: -0.125em; }
span.TEX span.E{ position:relative;top:0.5ex;left:-0.0417em;}
a span.TEX span.E {text-decoration: none; }
span.LATEX span.A{ position:relative; top:-0.5ex; left:-0.4em; font-size:85%;}
span.LATEX span.TEX{ position:relative; left: -0.4em; }
div.float img, div.float .caption {text-align:center;}
div.figure img, div.figure .caption {text-align:center;}
.marginpar {width:20%; float:right; text-align:left; margin-left:auto; margin-top:0.5em; font-size:85%; text-decoration:underline;}
.marginpar p{margin-top:0.4em; margin-bottom:0.4em;}
.equation td{text-align:center; vertical-align:middle; }
td.eq-no{ width:5%; }
table.equation { width:100%; } 
div.math-display, div.par-math-display{text-align:center;}
math .texttt { font-family: monospace; }
math .textit { font-style: italic; }
math .textsl { font-style: oblique; }
math .textsf { font-family: sans-serif; }
math .textbf { font-weight: bold; }
.partToc a, .partToc, .likepartToc a, .likepartToc {line-height: 200%; font-weight:bold; font-size:110%;}
.chapterToc a, .chapterToc, .likechapterToc a, .likechapterToc, .appendixToc a, .appendixToc {line-height: 200%; font-weight:bold;}
.index-item, .index-subitem, .index-subsubitem {display:block}
.caption td.id{font-weight: bold; white-space: nowrap; }
table.caption {text-align:center;}
h1.partHead{text-align: center}
p.bibitem { text-indent: -2em; margin-left: 2em; margin-top:0.6em; margin-bottom:0.6em; }
p.bibitem-p { text-indent: 0em; margin-left: 2em; margin-top:0.6em; margin-bottom:0.6em; }
.paragraphHead, .likeparagraphHead { margin-top:2em; font-weight: bold;}
.subparagraphHead, .likesubparagraphHead { font-weight: bold;}
.quote {margin-bottom:0.25em; margin-top:0.25em; margin-left:1em; margin-right:1em; text-align:justify;}
.verse{white-space:nowrap; margin-left:2em}
div.maketitle {text-align:center;}
h2.titleHead{text-align:center;}
div.maketitle{ margin-bottom: 2em; }
div.author, div.date {text-align:center;}
div.thanks{text-align:left; margin-left:10%; font-size:85%; font-style:italic; }
div.author{white-space: nowrap;}
.quotation {margin-bottom:0.25em; margin-top:0.25em; margin-left:1em; }
h1.partHead{text-align: center}
.sectionToc, .likesectionToc {margin-left:2em;}
.subsectionToc, .likesubsectionToc {margin-left:4em;}
.subsubsectionToc, .likesubsubsectionToc {margin-left:6em;}
.frenchb-nbsp{font-size:75%;}
.frenchb-thinspace{font-size:75%;}
.figure img.graphics {margin-left:10%;}
/* end css.sty */

\author{}
\date{}

\begin{document}

\textbf{Warning: \href{http://www.math.union.edu/locate/jsMath}{jsMath}
requires JavaScript to process the mathematics on this page.\\ If your
browser supports JavaScript, be sure it is enabled.}

\begin{center}\rule{3in}{0.4pt}\end{center}

\href{coursch1.html\#x2-1000}{Avant-propos} \\
\href{coursli1.html\#x3-5000}{Table des matières} \\ 1
\href{coursch2.html\#x4-60001}{Ensembles et structures} \\ ~1.1
\href{coursse1.html\#x5-70001.1}{Ensembles et relations} \\ ~1.2
\href{coursse2.html\#x6-120001.2}{Cardinaux et entiers naturels} \\ ~1.3
\href{coursse3.html\#x7-150001.3}{Groupes} \\ ~1.4
\href{coursse4.html\#x8-260001.4}{Anneaux et corps} \\ ~1.5
\href{coursse5.html\#x9-360001.5}{Polynômes à une variable} \\ ~1.6
\href{coursse6.html\#x10-460001.6}{Polynômes à plusieurs variables} \\ 2
\href{coursch3.html\#x11-510002}{Algèbre linéaire élémentaire} \\ ~2.1
\href{coursse7.html\#x12-520002.1}{Généralités sur les espaces
vectoriels} \\ ~2.2 \href{coursse8.html\#x13-610002.2}{Bases et
dimension} \\ ~2.3 \href{coursse9.html\#x14-650002.3}{Rang} \\ ~2.4
\href{coursse10.html\#x15-680002.4}{Dualité: approche restreinte} \\
~2.5 \href{coursse11.html\#x16-750002.5}{Dualité: approche générale} \\
~2.6 \href{coursse12.html\#x17-810002.6}{Matrices} \\ ~2.7
\href{coursse13.html\#x18-890002.7}{Déterminants} \\ ~2.8
\href{coursse14.html\#x19-960002.8}{Systèmes linéaires} \\ 3
\href{coursch4.html\#x20-1010003}{Réduction des endomorphismes} \\ ~3.1
\href{coursse15.html\#x21-1020003.1}{Valeurs propres. Vecteurs propres}
\\ ~3.2 \href{coursse16.html\#x22-1090003.2}{Polynômes d'endomorphismes}
\\ ~3.3 \href{coursse17.html\#x23-1170003.3}{A propos de Jordan} \\ 4
\href{coursch5.html\#x24-1240004}{Topologie des espaces métriques} \\
~4.1 \href{coursse18.html\#x25-1250004.1}{Eléments de topologie
générale} \\ ~4.2 \href{coursse19.html\#x26-1310004.2}{Espaces
métriques} \\ ~4.3 \href{coursse20.html\#x27-1380004.3}{Suites} \\ ~4.4
\href{coursse21.html\#x28-1420004.4}{Limites de fonctions} \\ ~4.5
\href{coursse22.html\#x29-1470004.5}{Continuité} \\ ~4.6
\href{coursse23.html\#x30-1510004.6}{Continuité uniforme} \\ ~4.7
\href{coursse24.html\#x31-1540004.7}{Espaces complets} \\ ~4.8
\href{coursse25.html\#x32-1580004.8}{Espaces et parties compactes} \\
~4.9 \href{coursse26.html\#x33-1620004.9}{Espaces et parties connexes}
\\ 5 \href{coursch6.html\#x34-1670005}{Espaces vectoriels normés} \\
~5.1 \href{coursse27.html\#x35-1680005.1}{Notion d'espace vectoriel
normé} \\ ~5.2 \href{coursse28.html\#x36-1720005.2}{Applications
linéaires continues} \\ ~5.3
\href{coursse29.html\#x37-1770005.3}{Espaces vectoriels normés de
dimensions finies} \\ ~5.4
\href{coursse30.html\#x38-1810005.4}{Compléments: le théorème de Baire
et ses conséquences} \\ ~5.5
\href{coursse31.html\#x39-1840005.5}{Compléments: convexité dans les
espaces vectoriels normés} \\ 6
\href{coursch7.html\#x40-1890006}{Comparaison des fonctions} \\ ~6.1
\href{coursse32.html\#x41-1900006.1}{Relations de comparaison} \\ ~6.2
\href{coursse33.html\#x42-1950006.2}{Développements limités} \\ ~6.3
\href{coursse34.html\#x43-1990006.3}{Développements asymptotiques} \\ 7
\href{coursch8.html\#x44-2030007}{Suites et séries} \\ ~7.1
\href{coursse35.html\#x45-2040007.1}{Convergence des suites} \\ ~7.2
\href{coursse36.html\#x46-2090007.2}{Généralités sur les séries} \\ ~7.3
\href{coursse37.html\#x47-2120007.3}{Séries à termes réels positifs} \\
~7.4 \href{coursse38.html\#x48-2170007.4}{Séries absolument
convergentes} \\ ~7.5 \href{coursse39.html\#x49-2230007.5}{Séries
semi-convergentes} \\ ~7.6
\href{coursse40.html\#x50-2260007.6}{Opérations sur les séries} \\ ~7.7
\href{coursse41.html\#x51-2310007.7}{Séries doubles} \\ ~7.8
\href{coursse42.html\#x52-2320007.8}{Espaces de suites} \\ ~7.9
\href{coursse43.html\#x53-2330007.9}{Compléments: développements
asymptotiques, analyse numérique} \\ 8
\href{coursch9.html\#x54-2360008}{Fonctions d'une variable réelle} \\
~8.1 \href{coursse44.html\#x55-2370008.1}{Monotonie, continuité} \\ ~8.2
\href{coursse45.html\#x56-2400008.2}{Dérivée} \\ ~8.3
\href{coursse46.html\#x57-2440008.3}{Fonctions réelles d'une variable
réelle} \\ ~8.4 \href{coursse47.html\#x58-2510008.4}{Fonctions
vectorielles d'une variable réelle} \\ ~8.5
\href{coursse48.html\#x59-2550008.5}{Fonctions classiques} \\ ~8.6
\href{coursse49.html\#x60-2590008.6}{Analyse numérique des fonctions
d'une variable} \\ 9 \href{coursch10.html\#x61-2630009}{Intégration} \\
~9.1 \href{coursse50.html\#x62-2640009.1}{Subdivisions, approximation
des fonctions} \\ ~9.2 \href{coursse51.html\#x63-2680009.2}{Intégrale
des fonctions réglées sur un segment} \\ ~9.3
\href{coursse52.html\#x64-2740009.3}{Primitives et intégrales} \\ ~9.4
\href{coursse53.html\#x65-2790009.4}{Recherches de primitives} \\ ~9.5
\href{coursse54.html\#x66-2870009.5}{Intégration sur un intervalle
quelconque: fonctions à valeurs réelles positives} \\ ~9.6
\href{coursse55.html\#x67-2910009.6}{Intégration sur un intervalle
quelconque: fonctions à valeurs complexes} \\ ~9.7
\href{coursse56.html\#x68-2980009.7}{Développements asymptotiques et
analyse numérique} \\ ~9.8
\href{coursse57.html\#x69-3020009.8}{Généralités sur les intégrales
impropres} \\ ~9.9 \href{coursse58.html\#x70-3070009.9}{Intégrale des
fonctions réelles positives} \\ ~9.10
\href{coursse59.html\#x71-3110009.10}{Convergence absolue,
semi-convergence} \\ 10 \href{coursch11.html\#x72-31600010}{Suites et
séries de fonctions} \\ ~10.1
\href{coursse60.html\#x73-31700010.1}{Suites de fonctions} \\ ~10.2
\href{coursse61.html\#x74-32500010.2}{Séries de fonctions} \\ ~10.3
\href{coursse62.html\#x75-33000010.3}{Intégrales dépendant d'un
paramètre} \\ 11 \href{coursch12.html\#x76-33900011}{Séries entières} \\
~11.1 \href{coursse63.html\#x77-34000011.1}{Convergence des séries
entières} \\ ~11.2 \href{coursse64.html\#x78-34500011.2}{Somme d'une
série entière} \\ ~11.3
\href{coursse65.html\#x79-34900011.3}{Développements en séries entières}
\\ ~11.4 \href{coursse66.html\#x80-35800011.4}{Application aux
endomorphismes continus et aux matrices} \\ 12
\href{coursch13.html\#x81-36200012}{Formes quadratiques} \\ ~12.1
\href{coursse67.html\#x82-36300012.1}{Formes bilinéaires} \\ ~12.2
\href{coursse68.html\#x83-37100012.2}{Formes quadratiques} \\ ~12.3
\href{coursse69.html\#x84-37500012.3}{Réduction des formes quadratiques
en dimension finie} \\ ~12.4
\href{coursse70.html\#x85-37800012.4}{Formes quadratiques réelles} \\
~12.5 \href{coursse71.html\#x86-38600012.5}{Endomorphismes et formes
quadratiques} \\ ~12.6
\href{coursse72.html\#x87-39200012.6}{Endomorphismes d'un espace
euclidien} \\ 13 \href{coursch14.html\#x88-40000013}{Formes
hermitiennes} \\ ~13.1 \href{coursse73.html\#x89-40100013.1}{Compléments
sur la conjugaison} \\ ~13.2
\href{coursse74.html\#x90-40500013.2}{Formes sesquilinéaires} \\ ~13.3
\href{coursse75.html\#x91-41200013.3}{Formes quadratiques hermitiennes}
\\ ~13.4 \href{coursse76.html\#x92-41700013.4}{Endomorphismes d'un
espace hermitien} \\ 14 \href{coursch15.html\#x93-42400014}{Séries de
Fourier} \\ ~14.1 \href{coursse77.html\#x94-42500014.1}{Introduction:
transformée de Fourier sur les groupes abéliens finis} \\ ~14.2
\href{coursse78.html\#x95-42800014.2}{Séries trigonométriques} \\ ~14.3
\href{coursse79.html\#x96-43200014.3}{Série de Fourier d'une fonction}
\\ ~14.4 \href{coursse80.html\#x97-43900014.4}{Fonctions périodiques de
période T} \\ ~14.5 \href{coursse81.html\#x98-44000014.5}{Produit de
convolution} \\ 15 \href{coursch16.html\#x99-44300015}{Calcul
différentiel} \\ ~15.1 \href{coursse82.html\#x100-44400015.1}{Dérivées
partielles} \\ ~15.2
\href{coursse83.html\#x101-45100015.2}{Différentielle} \\ ~15.3
\href{coursse84.html\#x102-45800015.3}{Formes différentielles} \\ ~15.4
\href{coursse85.html\#x103-46500015.4}{Fonctions implicites et inversion
locale} \\ 16 \href{coursch17.html\#x104-47000016}{Equations
différentielles} \\ ~16.1 \href{coursse86.html\#x105-47100016.1}{Notions
générales} \\ ~16.2 \href{coursse87.html\#x106-47700016.2}{Théorie de
Cauchy-Lipschitz} \\ ~16.3
\href{coursse88.html\#x107-47900016.3}{Equations différentielles
linéaires d'ordre 1} \\ ~16.4
\href{coursse89.html\#x108-48600016.4}{Equation différentielle linéaire
d'ordre n} \\ ~16.5 \href{coursse90.html\#x109-49500016.5}{Equations
différentielles non linéaires} \\ ~16.6
\href{coursse91.html\#x110-50400016.6}{Analyse numérique des équations
différentielles} \\ 17 \href{coursch18.html\#x111-50800017}{Espaces
affines} \\ ~17.1 \href{coursse92.html\#x112-50900017.1}{Généralités sur
les espaces affines} \\ ~17.2
\href{coursse93.html\#x113-51700017.2}{Barycentres} \\ ~17.3
\href{coursse94.html\#x114-52200017.3}{Espaces affines euclidiens} \\
~17.4 \href{coursse95.html\#x115-52900017.4}{Cercles, sphères, triangle}
\\ 18 \href{coursch19.html\#x116-53300018}{Courbes} \\ ~18.1
\href{coursse96.html\#x117-53400018.1}{Arcs paramétrés} \\ ~18.2
\href{coursse97.html\#x118-54500018.2}{Arcs en polaires} \\ ~18.3
\href{coursse98.html\#x119-55100018.3}{Problèmes classiques sur les
courbes} \\ ~18.4 \href{coursse99.html\#x120-55600018.4}{Etude métrique
des arcs} \\ 19 \href{coursch20.html\#x121-56600019}{Surfaces} \\ ~19.1
\href{coursse100.html\#x122-56700019.1}{Nappes paramétrées} \\ ~19.2
\href{coursse101.html\#x123-57400019.2}{Nappes réglées} \\ ~19.3
\href{coursse102.html\#x124-57800019.3}{Equations de surfaces} \\ ~19.4
\href{coursse103.html\#x125-58300019.4}{Quadriques} \\ 20
\href{coursch21.html\#x126-58800020}{Intégrales curvilignes, intégrales
multiples} \\ ~20.1 \href{coursse104.html\#x127-58900020.1}{Intégrales
curvilignes} \\ ~20.2 \href{coursse105.html\#x128-59300020.2}{Intégrales
multiples} \\ ~20.3 \href{coursse106.html\#x129-59800020.3}{Calcul des
intégrales doubles et triples} \\ ~20.4
\href{coursse107.html\#x130-60300020.4}{Introduction aux intégrales de
surface}

\end{document}
