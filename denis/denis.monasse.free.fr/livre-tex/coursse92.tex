\documentclass[]{article}
\usepackage[T1]{fontenc}
\usepackage{lmodern}
\usepackage{amssymb,amsmath}
\usepackage{ifxetex,ifluatex}
\usepackage{fixltx2e} % provides \textsubscript
% use upquote if available, for straight quotes in verbatim environments
\IfFileExists{upquote.sty}{\usepackage{upquote}}{}
\ifnum 0\ifxetex 1\fi\ifluatex 1\fi=0 % if pdftex
  \usepackage[utf8]{inputenc}
\else % if luatex or xelatex
  \ifxetex
    \usepackage{mathspec}
    \usepackage{xltxtra,xunicode}
  \else
    \usepackage{fontspec}
  \fi
  \defaultfontfeatures{Mapping=tex-text,Scale=MatchLowercase}
  \newcommand{\euro}{€}
\fi
% use microtype if available
\IfFileExists{microtype.sty}{\usepackage{microtype}}{}
\ifxetex
  \usepackage[setpagesize=false, % page size defined by xetex
              unicode=false, % unicode breaks when used with xetex
              xetex]{hyperref}
\else
  \usepackage[unicode=true]{hyperref}
\fi
\hypersetup{breaklinks=true,
            bookmarks=true,
            pdfauthor={},
            pdftitle={Generalites sur les espaces affines},
            colorlinks=true,
            citecolor=blue,
            urlcolor=blue,
            linkcolor=magenta,
            pdfborder={0 0 0}}
\urlstyle{same}  % don't use monospace font for urls
\setlength{\parindent}{0pt}
\setlength{\parskip}{6pt plus 2pt minus 1pt}
\setlength{\emergencystretch}{3em}  % prevent overfull lines
\setcounter{secnumdepth}{0}
 
/* start css.sty */
.cmr-5{font-size:50%;}
.cmr-7{font-size:70%;}
.cmmi-5{font-size:50%;font-style: italic;}
.cmmi-7{font-size:70%;font-style: italic;}
.cmmi-10{font-style: italic;}
.cmsy-5{font-size:50%;}
.cmsy-7{font-size:70%;}
.cmex-7{font-size:70%;}
.cmex-7x-x-71{font-size:49%;}
.msbm-7{font-size:70%;}
.cmtt-10{font-family: monospace;}
.cmti-10{ font-style: italic;}
.cmbx-10{ font-weight: bold;}
.cmr-17x-x-120{font-size:204%;}
.cmsl-10{font-style: oblique;}
.cmti-7x-x-71{font-size:49%; font-style: italic;}
.cmbxti-10{ font-weight: bold; font-style: italic;}
p.noindent { text-indent: 0em }
td p.noindent { text-indent: 0em; margin-top:0em; }
p.nopar { text-indent: 0em; }
p.indent{ text-indent: 1.5em }
@media print {div.crosslinks {visibility:hidden;}}
a img { border-top: 0; border-left: 0; border-right: 0; }
center { margin-top:1em; margin-bottom:1em; }
td center { margin-top:0em; margin-bottom:0em; }
.Canvas { position:relative; }
li p.indent { text-indent: 0em }
.enumerate1 {list-style-type:decimal;}
.enumerate2 {list-style-type:lower-alpha;}
.enumerate3 {list-style-type:lower-roman;}
.enumerate4 {list-style-type:upper-alpha;}
div.newtheorem { margin-bottom: 2em; margin-top: 2em;}
.obeylines-h,.obeylines-v {white-space: nowrap; }
div.obeylines-v p { margin-top:0; margin-bottom:0; }
.overline{ text-decoration:overline; }
.overline img{ border-top: 1px solid black; }
td.displaylines {text-align:center; white-space:nowrap;}
.centerline {text-align:center;}
.rightline {text-align:right;}
div.verbatim {font-family: monospace; white-space: nowrap; text-align:left; clear:both; }
.fbox {padding-left:3.0pt; padding-right:3.0pt; text-indent:0pt; border:solid black 0.4pt; }
div.fbox {display:table}
div.center div.fbox {text-align:center; clear:both; padding-left:3.0pt; padding-right:3.0pt; text-indent:0pt; border:solid black 0.4pt; }
div.minipage{width:100%;}
div.center, div.center div.center {text-align: center; margin-left:1em; margin-right:1em;}
div.center div {text-align: left;}
div.flushright, div.flushright div.flushright {text-align: right;}
div.flushright div {text-align: left;}
div.flushleft {text-align: left;}
.underline{ text-decoration:underline; }
.underline img{ border-bottom: 1px solid black; margin-bottom:1pt; }
.framebox-c, .framebox-l, .framebox-r { padding-left:3.0pt; padding-right:3.0pt; text-indent:0pt; border:solid black 0.4pt; }
.framebox-c {text-align:center;}
.framebox-l {text-align:left;}
.framebox-r {text-align:right;}
span.thank-mark{ vertical-align: super }
span.footnote-mark sup.textsuperscript, span.footnote-mark a sup.textsuperscript{ font-size:80%; }
div.tabular, div.center div.tabular {text-align: center; margin-top:0.5em; margin-bottom:0.5em; }
table.tabular td p{margin-top:0em;}
table.tabular {margin-left: auto; margin-right: auto;}
div.td00{ margin-left:0pt; margin-right:0pt; }
div.td01{ margin-left:0pt; margin-right:5pt; }
div.td10{ margin-left:5pt; margin-right:0pt; }
div.td11{ margin-left:5pt; margin-right:5pt; }
table[rules] {border-left:solid black 0.4pt; border-right:solid black 0.4pt; }
td.td00{ padding-left:0pt; padding-right:0pt; }
td.td01{ padding-left:0pt; padding-right:5pt; }
td.td10{ padding-left:5pt; padding-right:0pt; }
td.td11{ padding-left:5pt; padding-right:5pt; }
table[rules] {border-left:solid black 0.4pt; border-right:solid black 0.4pt; }
.hline hr, .cline hr{ height : 1px; margin:0px; }
.tabbing-right {text-align:right;}
span.TEX {letter-spacing: -0.125em; }
span.TEX span.E{ position:relative;top:0.5ex;left:-0.0417em;}
a span.TEX span.E {text-decoration: none; }
span.LATEX span.A{ position:relative; top:-0.5ex; left:-0.4em; font-size:85%;}
span.LATEX span.TEX{ position:relative; left: -0.4em; }
div.float img, div.float .caption {text-align:center;}
div.figure img, div.figure .caption {text-align:center;}
.marginpar {width:20%; float:right; text-align:left; margin-left:auto; margin-top:0.5em; font-size:85%; text-decoration:underline;}
.marginpar p{margin-top:0.4em; margin-bottom:0.4em;}
.equation td{text-align:center; vertical-align:middle; }
td.eq-no{ width:5%; }
table.equation { width:100%; } 
div.math-display, div.par-math-display{text-align:center;}
math .texttt { font-family: monospace; }
math .textit { font-style: italic; }
math .textsl { font-style: oblique; }
math .textsf { font-family: sans-serif; }
math .textbf { font-weight: bold; }
.partToc a, .partToc, .likepartToc a, .likepartToc {line-height: 200%; font-weight:bold; font-size:110%;}
.chapterToc a, .chapterToc, .likechapterToc a, .likechapterToc, .appendixToc a, .appendixToc {line-height: 200%; font-weight:bold;}
.index-item, .index-subitem, .index-subsubitem {display:block}
.caption td.id{font-weight: bold; white-space: nowrap; }
table.caption {text-align:center;}
h1.partHead{text-align: center}
p.bibitem { text-indent: -2em; margin-left: 2em; margin-top:0.6em; margin-bottom:0.6em; }
p.bibitem-p { text-indent: 0em; margin-left: 2em; margin-top:0.6em; margin-bottom:0.6em; }
.paragraphHead, .likeparagraphHead { margin-top:2em; font-weight: bold;}
.subparagraphHead, .likesubparagraphHead { font-weight: bold;}
.quote {margin-bottom:0.25em; margin-top:0.25em; margin-left:1em; margin-right:1em; text-align:justify;}
.verse{white-space:nowrap; margin-left:2em}
div.maketitle {text-align:center;}
h2.titleHead{text-align:center;}
div.maketitle{ margin-bottom: 2em; }
div.author, div.date {text-align:center;}
div.thanks{text-align:left; margin-left:10%; font-size:85%; font-style:italic; }
div.author{white-space: nowrap;}
.quotation {margin-bottom:0.25em; margin-top:0.25em; margin-left:1em; }
h1.partHead{text-align: center}
.sectionToc, .likesectionToc {margin-left:2em;}
.subsectionToc, .likesubsectionToc {margin-left:4em;}
.subsubsectionToc, .likesubsubsectionToc {margin-left:6em;}
.frenchb-nbsp{font-size:75%;}
.frenchb-thinspace{font-size:75%;}
.figure img.graphics {margin-left:10%;}
/* end css.sty */

\title{Generalites sur les espaces affines}
\author{}
\date{}

\begin{document}
\maketitle

\textbf{Warning: \href{http://www.math.union.edu/locate/jsMath}{jsMath}
requires JavaScript to process the mathematics on this page.\\ If your
browser supports JavaScript, be sure it is enabled.}

\begin{center}\rule{3in}{0.4pt}\end{center}

{[}\href{coursse93.html}{next}{]}
{[}\hyperref[tailcoursse92.html]{tail}{]}
{[}\href{coursch18.html\#coursse92.html}{up}{]}

\subsubsection{17.1 Généralités sur les espaces affines}

\paragraph{17.1.1 Notion d'espace affine}

Définition~17.1.1 On appelle espace affine un triplet
(E,\textbackslash{}overrightarrow\{E\},+) d'un ensemble E (l'ensemble
des points), d'un espace vectoriel \textbackslash{}overrightarrow\{E\}
(l'espace des vecteurs) et d'une application + : E
×\textbackslash{}overrightarrow\{ E\} → E,
(x,\textbackslash{}overrightarrow\{ξ\})\textbackslash{}mathrel\{↦\}x
+\textbackslash{}overrightarrow\{ ξ\} vérifiant les propriétés

\begin{itemize}
\itemsep1pt\parskip0pt\parsep0pt
\item
  (i)
  \textbackslash{}mathop\{∀\}(x,\textbackslash{}overrightarrow\{ξ\},\textbackslash{}overrightarrow\{η\})
  ∈ E ×\textbackslash{}overrightarrow\{ E\}
  ×\textbackslash{}overrightarrow\{ E\}, (x
  +\textbackslash{}overrightarrow\{ ξ\})
  +\textbackslash{}overrightarrow\{ η\} = x +
  (\textbackslash{}overrightarrow\{ξ\} +\textbackslash{}overrightarrow\{
  η\})
\item
  (ii)
  \textbackslash{}mathop\{∀\}(x,\textbackslash{}overrightarrow\{ξ\}) ∈ E
  ×\textbackslash{}overrightarrow\{ E\}, \textbackslash{}left (x
  +\textbackslash{}overrightarrow\{ ξ\} = x \textbackslash{}mathrel\{⇔\}
  \textbackslash{}overrightarrow\{ξ\} =\textbackslash{}overrightarrow\{
  0\}\textbackslash{}right )
\item
  (iii) \textbackslash{}mathop\{∀\}x,y ∈ E,
  \textbackslash{}mathop\{∃\}\textbackslash{}overrightarrow\{ξ\}
  ∈\textbackslash{}overrightarrow\{ E\}, x
  +\textbackslash{}overrightarrow\{ ξ\} = y
\end{itemize}

Remarque~17.1.1 L'assertion (i) et la moitié '' ⇐'' de l'assertion (iii)
traduisent que l'application
(x,\textbackslash{}overrightarrow\{ξ\})\textbackslash{}mathrel\{↦\}x
+\textbackslash{}overrightarrow\{ ξ\} est une loi de groupe opérant sur
un ensemble du groupe additif de l'espace vectoriel
\textbackslash{}overrightarrow\{E\} sur l'ensemble E. La propriété (iii)
traduit le fait que cette opération est transitive, c'est-à-dire qu'il y
a une seule orbite pour cette opération. La moitié '' ⇒'' de la
propriété (ii) est appelée la fidélité de l'opération. Un espace affine
est donc une opération transitive et fidèle du groupe additif d'un
espace vectoriel sur un ensemble. L'espace vectoriel
\textbackslash{}overrightarrow\{E\} est appelé la direction de l'espace
affine. Par la suite on confondra abusivement l'espace affine
(E,\textbackslash{}overrightarrow\{E\},+) avec l'ensemble E de ses
points.

Proposition~17.1.1 Etant donné x,y ∈ E, il existe un unique vecteur noté
\textbackslash{}overrightarrow\{xy\} ∈\textbackslash{}overrightarrow\{
E\} vérifiant x +\textbackslash{}overrightarrow\{ xy\} = y. On a
\textbackslash{}overrightarrow\{xy\} = 0 \textbackslash{}mathrel\{⇔\} x
= y et on a la relation de Chasles

\textbackslash{}mathop\{∀\}x,y,z ∈ E,
\textbackslash{}overrightarrow\{xz\} =\textbackslash{}overrightarrow\{
xy\} +\textbackslash{}overrightarrow\{ yz\}

Démonstration L'existence est garantie par la propriété (iii). Pour
l'unicité, si on a à la fois y = x +\textbackslash{}overrightarrow\{ ξ\}
= x +\textbackslash{}overrightarrow\{ η\}, on a x = x
+\textbackslash{}overrightarrow\{ 0\} = (x
+\textbackslash{}overrightarrow\{ ξ\}) +
(−\textbackslash{}overrightarrow\{ξ\}) = (x
+\textbackslash{}overrightarrow\{ η\}) +
(−\textbackslash{}overrightarrow\{ξ\}) = x +
(\textbackslash{}overrightarrow\{η\} −\textbackslash{}overrightarrow\{
ξ\}) d'où \textbackslash{}overrightarrow\{ξ\}
−\textbackslash{}overrightarrow\{ η\} =\textbackslash{}overrightarrow\{
0\} et donc \textbackslash{}overrightarrow\{ξ\}
=\textbackslash{}overrightarrow\{ η\}. La propriété
\textbackslash{}overrightarrow\{xy\} = 0 \textbackslash{}mathrel\{⇔\} x
= y est une conséquence évidente de (ii). On a z = y
+\textbackslash{}overrightarrow\{ yz\} = (x
+\textbackslash{}overrightarrow\{ xy\})
+\textbackslash{}overrightarrow\{ yz\} = x +
(\textbackslash{}overrightarrow\{xy\} +\textbackslash{}overrightarrow\{
yz\}) d'où \textbackslash{}overrightarrow\{xz\}
=\textbackslash{}overrightarrow\{ xy\} +\textbackslash{}overrightarrow\{
yz\}.

Proposition~17.1.2 Soit a ∈ E. L'application \{φ\}\_\{a\} :
x\textbackslash{}mathrel\{↦\}\textbackslash{}overrightarrow\{ax\} est
une bijection de l'espace affine E sur l'espace vectoriel
\textbackslash{}overrightarrow\{E\}.

Démonstration L'application réciproque est bien entendu l'application
\{ψ\}\_\{a\} :\textbackslash{}overrightarrow\{
ξ\}\textbackslash{}mathrel\{↦\}a +\textbackslash{}overrightarrow\{ ξ\}.
On vérifie immédiatement que \{ψ\}\_\{a\} ∘ \{φ\}\_\{a\} =\{
\textbackslash{}mathrm\{Id\}\}\_\{E\} et que \{φ\}\_\{a\} ∘ \{ψ\}\_\{a\}
=\{
\textbackslash{}mathrm\{Id\}\}\_\{\textbackslash{}overrightarrow\{E\}\}.

Remarque~17.1.2 Par transport des opérations algébriques de
\textbackslash{}overrightarrow\{E\} sur E, on munit ainsi E d'une
structure d'espace vectoriel isomorphe à
\textbackslash{}overrightarrow\{E\}. On dira que l'espace vectoriel
ainsi obtenu est le vectorialisé de E en l'origine a et on le notera par
la suite \{E\}\_\{a\}. On retiendra donc que le choix d'une origine dans
l'espace affine transforme cet espace affine en un espace vectoriel.

Définition~17.1.2 On appelle dimension de l'espace affine E la dimension
de sa direction \textbackslash{}overrightarrow\{E\} comme espace
vectoriel.

\paragraph{17.1.2 Repères affines, bases affines}

Définition~17.1.3 On appelle repère affine de E tout couple (a,ℰ) d'un
point a de E (l'origine du repère) et d'une base ℰ de
\textbackslash{}overrightarrow\{E\}. Si ℰ =
\{(\textbackslash{}vec\{\{e\}\}\_\{i\})\}\_\{i∈I\} tout point x ∈ E
s'écrit de manière unique sous la forme x = a
+\{\textbackslash{}mathop\{ \textbackslash{}mathop\{∑ \}\}
\}\_\{i∈I\}\{x\}\_\{i\}\textbackslash{}vec\{\{e\}\}\_\{i\}~; on dit que
les \{x\}\_\{i\} sont les coordonnées du point x dans le repère affine~;
ce sont également les coordonnées du vecteur
\textbackslash{}overrightarrow\{ax\} dans la base ℰ.

Soit (a,ℰ) et (b,ℱ) deux repères affines de E. Ecrivons alors (avec des
notations évidentes) b = a +\{\textbackslash{}mathop\{
\textbackslash{}mathop\{∑ \}\}
\}\_\{i∈I\}\{β\}\_\{i\}\textbackslash{}vec\{\{e\}\}\_\{i\} et
\textbackslash{}vec\{\{f\}\}\_\{j\} =\{\textbackslash{}mathop\{
\textbackslash{}mathop\{∑ \}\}
\}\_\{i∈I\}\{u\}\_\{i,j\}\textbackslash{}vec\{\{e\}\}\_\{i\}. On a
alors, si les \{x\}\_\{i\} désignent les coordonnées de x dans le repère
(a,ℰ) et les \{y\}\_\{j\} celles de x dans le repère (b,ℱ)

\textbackslash{}begin\{eqnarray*\} x\& =\& b +\{
\textbackslash{}mathop\{∑
\}\}\_\{j∈J\}\{y\}\_\{j\}\textbackslash{}vec\{\{f\}\}\_\{j\} = a +\{
\textbackslash{}mathop\{∑
\}\}\_\{i∈I\}\{β\}\_\{i\}\textbackslash{}vec\{\{e\}\}\_\{i\} +\{
\textbackslash{}mathop\{∑ \}\}\_\{j∈J\}\{y\}\_\{j\}\{
\textbackslash{}mathop\{∑
\}\}\_\{i∈I\}\{u\}\_\{i,j\}\textbackslash{}vec\{\{e\}\}\_\{i\}\%\&
\textbackslash{}\textbackslash{} \& =\& a +\{ \textbackslash{}mathop\{∑
\}\}\_\{i∈I\}\{β\}\_\{i\}\textbackslash{}vec\{\{e\}\}\_\{i\} +\{
\textbackslash{}mathop\{∑ \}\}\_\{i∈I\}\textbackslash{}left
(\{\textbackslash{}mathop\{∑
\}\}\_\{j∈J\}\{u\}\_\{i,j\}\{y\}\_\{j\}\textbackslash{}right
)\{e\}\_\{i\} \%\& \textbackslash{}\textbackslash{} \& =\& a +\{
\textbackslash{}mathop\{∑ \}\}\_\{i∈I\}\textbackslash{}left
(\{β\}\_\{i\} +\{ \textbackslash{}mathop\{∑
\}\}\_\{j∈J\}\{u\}\_\{i,j\}\{y\}\_\{j\}\textbackslash{}right
)\{e\}\_\{i\} \%\& \textbackslash{}\textbackslash{}
\textbackslash{}end\{eqnarray*\}

si bien que \textbackslash{}mathop\{∀\}i ∈ I,\{x\}\_\{i\} = \{β\}\_\{i\}
+\{\textbackslash{}mathop\{ \textbackslash{}mathop\{∑ \}\}
\}\_\{j∈J\}\{u\}\_\{i,j\}\{y\}\_\{j\} ce qui fournit les formules de
changement de repère.

Proposition~17.1.3 Soit \{(\{a\}\_\{i\})\}\_\{i∈I\} une famille non vide
de points de E. Alors les propriétés suivantes sont équivalentes

\begin{itemize}
\itemsep1pt\parskip0pt\parsep0pt
\item
  (i) il existe \{i\}\_\{0\} ∈ I tel que la famille
  \{(\textbackslash{}overrightarrow\{\{a\}\_\{\{i\}\_\{0\}\}\{a\}\_\{i\}\})\}\_\{i∈I∖\textbackslash{}\{\{i\}\_\{0\}\textbackslash{}\}\}
  soit libre (resp. génératrice, resp. une base) dans
  \textbackslash{}overrightarrow\{E\}
\item
  (ii) pour tout \{i\}\_\{0\} ∈ I, la famille
  \{(\textbackslash{}overrightarrow\{\{a\}\_\{\{i\}\_\{0\}\}\{a\}\_\{i\}\})\}\_\{i∈I∖\textbackslash{}\{\{i\}\_\{0\}\textbackslash{}\}\}
  est libre (resp. génératrice, resp. une base) dans
  \textbackslash{}overrightarrow\{E\}
\end{itemize}

On dit dans ce cas que la famille \{(\{a\}\_\{i\})\}\_\{i∈I\} est
affinement libre (resp. affinement génératrice, resp. une base affine)
de E.

Démonstration Libre~: Il est clair que (ii) ⇒(i). Supposons (i) vérifiée
et soit \{i\}\_\{1\} ∈ I. Soit
\{(\{λ\}\_\{i\})\}\_\{i∈I∖\textbackslash{}\{\{i\}\_\{1\}\textbackslash{}\}\}
une famille de scalaires (avec seulement un nombre fini de scalaires non
nuls) tels que \{\textbackslash{}mathop\{\textbackslash{}mathop\{∑ \}\}
\}\_\{i∈I∖\textbackslash{}\{\{i\}\_\{1\}\textbackslash{}\}\}\{λ\}\_\{i\}\textbackslash{}overrightarrow\{\{a\}\_\{\{i\}\_\{1\}\}\{a\}\_\{i\}\}
=\textbackslash{}overrightarrow\{ 0\}. On a alors en notant que
\textbackslash{}overrightarrow\{\{a\}\_\{\{i\}\_\{0\}\}\{a\}\_\{\{i\}\_\{0\}\}\}
=\textbackslash{}overrightarrow\{ 0\})

\textbackslash{}begin\{eqnarray*\} \textbackslash{}overrightarrow\{0\}\&
=\& \{\textbackslash{}mathop\{∑
\}\}\_\{i∈I∖\textbackslash{}\{\{i\}\_\{1\}\textbackslash{}\}\}\{λ\}\_\{i\}(\textbackslash{}overrightarrow\{\{a\}\_\{\{i\}\_\{1\}\}\{a\}\_\{\{i\}\_\{0\}\}\}
+\textbackslash{}overrightarrow\{ \{a\}\_\{\{i\}\_\{0\}\}\{a\}\_\{i\}\})
\%\& \textbackslash{}\textbackslash{} \& =\& \{\textbackslash{}mathop\{∑
\}\}\_\{i∈I∖\textbackslash{}\{\{i\}\_\{0\},\{i\}\_\{1\}\textbackslash{}\}\}\{λ\}\_\{i\}\textbackslash{}overrightarrow\{\{a\}\_\{\{i\}\_\{0\}\}\{a\}\_\{i\}\}
−\textbackslash{}left (\{\textbackslash{}mathop\{∑
\}\}\_\{i∈I∖\textbackslash{}\{\{i\}\_\{1\}\textbackslash{}\}\}\{λ\}\_\{i\}\textbackslash{}right
)\textbackslash{}overrightarrow\{\{a\}\_\{\{i\}\_\{0\}\}\{a\}\_\{\{i\}\_\{1\}\}\}\%\&
\textbackslash{}\textbackslash{} \textbackslash{}end\{eqnarray*\}

Comme la famille
\{(\textbackslash{}overrightarrow\{\{a\}\_\{\{i\}\_\{0\}\}\{a\}\_\{i\}\})\}\_\{i∈I∖\textbackslash{}\{\{i\}\_\{0\}\textbackslash{}\}\}
est libre, on doit donc avoir \textbackslash{}mathop\{∀\}i ∈ I
∖\textbackslash{}\{\{i\}\_\{0\},\{i\}\_\{1\}\textbackslash{}\},
\{λ\}\_\{i\} = 0 et également
\{\textbackslash{}mathop\{\textbackslash{}mathop\{∑ \}\}
\}\_\{i∈I∖\textbackslash{}\{\{i\}\_\{1\}\textbackslash{}\}\}\{λ\}\_\{i\}
= 0, ce qui nous donne évidemment que \{λ\}\_\{\{i\}\_\{0\}\} est
également nul. D'où \textbackslash{}mathop\{∀\}i ∈ I
∖\textbackslash{}\{\{i\}\_\{1\}\textbackslash{}\}, \{λ\}\_\{i\} = 0.

Génératrice Il est clair que (ii) ⇒(i). Supposons (i) vérifiée et soit
\{i\}\_\{1\} ∈ I. On peut alors écrire, si
\textbackslash{}overrightarrow\{ξ\} ∈\textbackslash{}overrightarrow\{
E\},

\textbackslash{}begin\{eqnarray*\} \textbackslash{}overrightarrow\{ξ\}\&
=\& \{\textbackslash{}mathop\{∑
\}\}\_\{i∈I∖\textbackslash{}\{\{i\}\_\{0\}\textbackslash{}\}\}\{λ\}\_\{i\}\textbackslash{}overrightarrow\{\{a\}\_\{\{i\}\_\{0\}\}\{a\}\_\{i\}\}
=\{ \textbackslash{}mathop\{∑
\}\}\_\{i∈I∖\textbackslash{}\{\{i\}\_\{0\}\textbackslash{}\}\}\{λ\}\_\{i\}(\textbackslash{}overrightarrow\{\{a\}\_\{\{i\}\_\{0\}\}\{a\}\_\{\{i\}\_\{1\}\}\}
+\textbackslash{}overrightarrow\{
\{a\}\_\{\{i\}\_\{1\}\}\{a\}\_\{i\}\})\%\&
\textbackslash{}\textbackslash{} \& =\& \{\textbackslash{}mathop\{∑
\}\}\_\{i∈I∖\textbackslash{}\{\{i\}\_\{0\},\{i\}\_\{1\}\textbackslash{}\}\}\{λ\}\_\{i\}\textbackslash{}overrightarrow\{\{a\}\_\{\{i\}\_\{1\}\}\{a\}\_\{i\}\}
−\textbackslash{}left (\{\textbackslash{}mathop\{∑
\}\}\_\{i∈I∖\textbackslash{}\{\{i\}\_\{0\}\textbackslash{}\}\}\{λ\}\_\{i\}\textbackslash{}right
)\textbackslash{}overrightarrow\{\{a\}\_\{\{i\}\_\{1\}\}\{a\}\_\{\{i\}\_\{0\}\}\}
\%\& \textbackslash{}\textbackslash{} \& =\& \{\textbackslash{}mathop\{∑
\}\}\_\{i∈I∖\textbackslash{}\{\{i\}\_\{1\}\textbackslash{}\}\}\{μ\}\_\{i\}\textbackslash{}overrightarrow\{\{a\}\_\{\{i\}\_\{1\}\}\{a\}\_\{i\}\}
\%\& \textbackslash{}\textbackslash{} \textbackslash{}end\{eqnarray*\}

avec \{μ\}\_\{i\} = \{λ\}\_\{i\} pour i ∈ I
∖\textbackslash{}\{\{i\}\_\{0\},\{i\}\_\{1\}\textbackslash{}\} et
\{μ\}\_\{\{i\}\_\{0\}\} =\{\textbackslash{}mathop\{
\textbackslash{}mathop\{∑ \}\}
\}\_\{i∈I∖\textbackslash{}\{\{i\}\_\{0\}\textbackslash{}\}\}\{λ\}\_\{i\}
ce qui montre que la famille
\{(\textbackslash{}overrightarrow\{\{a\}\_\{\{i\}\_\{1\}\}\{a\}\_\{i\}\})\}\_\{i∈I∖\textbackslash{}\{\{i\}\_\{1\}\textbackslash{}\}\}
est encore génératrice.

Le résultat pour les bases se déduit immédiatement de la combinaison des
deux résultats précédents.

\paragraph{17.1.3 Sous-espace affine}

Définition~17.1.4 Soit E un espace affine et F une partie de E. On dit
que F est un sous espace affine de E si on a les propriétés équivalentes

\begin{itemize}
\itemsep1pt\parskip0pt\parsep0pt
\item
  (i) il existe a ∈ F tel que
  \textbackslash{}\{\textbackslash{}overrightarrow\{ax\}\textbackslash{}mathrel\{∣\}x
  ∈ F\textbackslash{}\} soit un sous-espace vectoriel de
  \textbackslash{}overrightarrow\{E\}
\item
  (ii) F\textbackslash{}mathrel\{≠\}∅ et pour tout a ∈ F,
  \textbackslash{}\{\textbackslash{}overrightarrow\{ax\}\textbackslash{}mathrel\{∣\}x
  ∈ F\textbackslash{}\} est un sous espace vectoriel de
  \textbackslash{}overrightarrow\{E\}
\item
  (iii) il existe a ∈ F et un sous-espace vectoriel
  \textbackslash{}overrightarrow\{\{F\}\_\{a\}\} de
  \textbackslash{}overrightarrow\{E\} tel que F = a
  +\textbackslash{}overrightarrow\{ \{F\}\_\{a\}\}
\item
  (iv) F\textbackslash{}mathrel\{≠\}∅ et pour tout a ∈ F, il existe un
  sous-espace vectoriel \textbackslash{}overrightarrow\{\{F\}\_\{a\}\}
  de \textbackslash{}overrightarrow\{E\} tel que F = a
  +\textbackslash{}overrightarrow\{ \{F\}\_\{a\}\}
\end{itemize}

Le sous-espace vectoriel \textbackslash{}overrightarrow\{\{F\}\_\{a\}\}
est alors indépendant du choix de a ∈ F~; on l'appelle la direction de F
et on le note \textbackslash{}overrightarrow\{F\}

Démonstration Il est clair que (ii) ⇒(i) et que (iv) ⇒(iii). D'autre
part

\textbackslash{}begin\{eqnarray*\} F = a
+\textbackslash{}overrightarrow\{ \{F\}\_\{a\}\}\&
\textbackslash{}mathrel\{⇔\} \& F = \textbackslash{}\{a
+\textbackslash{}overrightarrow\{
ξ\}\textbackslash{}mathrel\{∣\}\textbackslash{}overrightarrow\{ξ\}
∈\textbackslash{}overrightarrow\{ \{F\}\_\{a\}\}\textbackslash{}\}\%\&
\textbackslash{}\textbackslash{} \& \textbackslash{}mathrel\{⇔\} \& F =
a +
\textbackslash{}\{\textbackslash{}overrightarrow\{ξ\}\textbackslash{}mathrel\{∣\}\textbackslash{}overrightarrow\{ξ\}
∈\textbackslash{}overrightarrow\{ \{F\}\_\{a\}\}\textbackslash{}\}\%\&
\textbackslash{}\textbackslash{} \& \textbackslash{}mathrel\{⇔\} \&
\textbackslash{}overrightarrow\{\{F\}\_\{a\}\} =
\textbackslash{}\{\textbackslash{}overrightarrow\{ax\}\textbackslash{}mathrel\{∣\}x
∈ F\textbackslash{}\} \%\& \textbackslash{}\textbackslash{}
\textbackslash{}end\{eqnarray*\}

ce qui assure que (i) ⇔(iii) et que (ii) ⇔(iv). Il ne nous reste plus à
montrer que (i) ⇒(iii) pour avoir les équivalences. Mais si b ∈ E, on a

\textbackslash{}\{\textbackslash{}overrightarrow\{bx\}\textbackslash{}mathrel\{∣\}x
∈ F\textbackslash{}\} =
\textbackslash{}\{\textbackslash{}overrightarrow\{ba\}
+\textbackslash{}overrightarrow\{ ax\}\textbackslash{}mathrel\{∣\}x ∈
F\textbackslash{}\} = −\textbackslash{}overrightarrow\{ab\} +
\textbackslash{}\{\textbackslash{}overrightarrow\{ax\}\textbackslash{}mathrel\{∣\}x
∈ F\textbackslash{}\}

avec \textbackslash{}overrightarrow\{ab\}
∈\textbackslash{}\{\textbackslash{}overrightarrow\{ax\}\textbackslash{}mathrel\{∣\}x
∈ F\textbackslash{}\}~; or un sous-espace vectoriel est stable par ses
translations et donc −\textbackslash{}overrightarrow\{ ab\} +
\textbackslash{}\{\textbackslash{}overrightarrow\{ax\}\textbackslash{}mathrel\{∣\}x
∈ F\textbackslash{}\} =
\textbackslash{}\{\textbackslash{}overrightarrow\{ax\}\textbackslash{}mathrel\{∣\}x
∈ F\textbackslash{}\} ce qui montre à la fois que
\textbackslash{}\{\textbackslash{}overrightarrow\{bx\}\textbackslash{}mathrel\{∣\}x
∈ F\textbackslash{}\} =
\textbackslash{}\{\textbackslash{}overrightarrow\{ax\}\textbackslash{}mathrel\{∣\}x
∈ F\textbackslash{}\} et que
\textbackslash{}\{\textbackslash{}overrightarrow\{bx\}\textbackslash{}mathrel\{∣\}x
∈ F\textbackslash{}\} est un sous-espace vectoriel.

Comme on l'a vu, on a alors

\textbackslash{}overrightarrow\{\{F\}\_\{b\}\} =
\textbackslash{}\{\textbackslash{}overrightarrow\{bx\}\textbackslash{}mathrel\{∣\}x
∈ F\textbackslash{}\} =
\textbackslash{}\{\textbackslash{}overrightarrow\{ax\}\textbackslash{}mathrel\{∣\}x
∈ F\textbackslash{}\} =\textbackslash{}overrightarrow\{ \{F\}\_\{a\}\}

ce qui montre que \textbackslash{}overrightarrow\{\{F\}\_\{a\}\} est
indépendant du choix de a ∈ F.

Remarque~17.1.3 On peut également traduire les propriétés (i) ou (ii)
sous la forme~: F est un sous-espace vectoriel de \{E\}\_\{a\}
(vectorialisé de E en l'origine a). Remarquons qu'un sous-espace affine
est nécessairement non vide.

Définition~17.1.5 On appelle dimension du sous-espace affine F de E la
dimension de \textbackslash{}overrightarrow\{F\}.

Proposition~17.1.4 Soit \{(\{F\}\_\{i\})\}\_\{i∈I\} une famille de
sous-espaces affines de E. Alors
\{\textbackslash{}mathop\{\textbackslash{}mathop\{⋂ \}\}
\}\_\{i∈I\}\{F\}\_\{i\} est soit l'ensemble vide, soit un sous-espace
affine de direction \{\textbackslash{}mathop\{\textbackslash{}mathop\{⋂
\}\} \}\_\{i∈I\}\textbackslash{}overrightarrow\{\{F\}\_\{i\}\}.

Démonstration Supposons l'intersection non vide et soit a
∈\{\textbackslash{}mathop\{\textbackslash{}mathop\{⋂ \}\}
\}\_\{i∈I\}\{F\}\_\{i\}. Alors

x ∈\{\textbackslash{}mathop\{⋂ \}\}\_\{i∈I\}\{F\}\_\{i\}
\textbackslash{}mathrel\{⇔\} \textbackslash{}mathop\{∀\}i ∈ I,
\textbackslash{}overrightarrow\{ax\} ∈\textbackslash{}overrightarrow\{
\{F\}\_\{i\}\} \textbackslash{}mathrel\{⇔\}
\textbackslash{}overrightarrow\{ax\} ∈\{\textbackslash{}mathop\{⋂
\}\}\_\{i∈I\}\textbackslash{}overrightarrow\{\{F\}\_\{i\}\}

ce dernier étant un sous-espace vectoriel de
\textbackslash{}overrightarrow\{E\}. Ceci montre que
\{\textbackslash{}mathop\{\textbackslash{}mathop\{⋂ \}\}
\}\_\{i∈I\}\{F\}\_\{i\} est un sous-espace affine et que sa direction
est \{\textbackslash{}mathop\{\textbackslash{}mathop\{⋂ \}\}
\}\_\{i∈I\}\textbackslash{}overrightarrow\{\{F\}\_\{i\}\}.

Remarque~17.1.4 Ceci permet ensuite de parler de sous-espace affine
engendré par une partie non vide de E~: c'est l'intersection de tous les
sous-espaces affines contenant A. Cette intersection est un sous-espace
affine contenant A et il est contenu dans tout espace affine contenant
A. Comme dans les espaces vectoriels, ceci permet de définir le
sous-espace affine \textbackslash{}mathop\{Aff\}(\{a\}\_\{i\},i ∈ I)
engendrée par une famille \{(\{a\}\_\{i\})\}\_\{i∈I\} que l'on doit ici
supposer non vide. Si \{i\}\_\{0\} ∈ I, on vérifie immédiatement que

\textbackslash{}begin\{eqnarray*\}
\textbackslash{}mathop\{Aff\}(\{a\}\_\{i\},i ∈ I)\& =\&
\{a\}\_\{\{i\}\_\{0\}\} +\textbackslash{}mathop\{
\textbackslash{}mathrm\{Vect\}\}(\textbackslash{}overrightarrow\{\{a\}\_\{\{i\}\_\{0\}\}\{a\}\_\{i\}\},
i ∈ I ∖\textbackslash{}\{\{i\}\_\{0\}\textbackslash{}\})\%\&
\textbackslash{}\textbackslash{} \& =\& \{a\}\_\{\{i\}\_\{0\}\} +
\textbackslash{}\{\{\textbackslash{}mathop\{∑
\}\}\_\{i∈I∖\textbackslash{}\{\{i\}\_\{0\}\textbackslash{}\}\}\{λ\}\_\{i\}\textbackslash{}overrightarrow\{\{a\}\_\{\{i\}\_\{0\}\}\{a\}\_\{i\}\}\textbackslash{}\}
\%\& \textbackslash{}\textbackslash{} \textbackslash{}end\{eqnarray*\}

On en déduit que \textbackslash{}mathop\{dim\}
\textbackslash{}mathop\{Aff\}(\{a\}\_\{i\},i ∈ I) est inférieur ou égal
au cardinal de I moins 1. Ceci permet également de parler de rang d'une
famille de points.

\paragraph{17.1.4 Parallélisme, intersection}

Définition~17.1.6 Soit deux sous-espaces affines F et G. On dit que

\begin{itemize}
\itemsep1pt\parskip0pt\parsep0pt
\item
  (i) F et G sont parallèles si \textbackslash{}overrightarrow\{F\}
  =\textbackslash{}overrightarrow\{ G\}
\item
  (ii) F est faiblement parallèle à G si
  \textbackslash{}overrightarrow\{F\} ⊂\textbackslash{}overrightarrow\{
  G\}
\end{itemize}

Théorème~17.1.5

\begin{itemize}
\itemsep1pt\parskip0pt\parsep0pt
\item
  (i) Si a ∈ E, il existe un unique sous-espace affine passant par a et
  parallèle à un sous-espace affine donné F
\item
  (ii) Si F et G sont parallèles, on a soit F = G, soit F ∩ G = ∅
\item
  (iii) Si F est faiblement parallèle à G , on a soit F ⊂ G, soit F ∩ G
  = ∅
\end{itemize}

Démonstration (i) G = a +\textbackslash{}overrightarrow\{ F\} est bien
entendu le seul qui convient. (ii) et (iii) sont évidents.

Le théorème suivant va jouer un rôle important pour garantir que deux
sous-espaces affines ont une intersection non vide

Théorème~17.1.6 Soit F et G deux sous-espaces affines de E, a ∈ F et b ∈
G. On a équivalence de

\begin{itemize}
\itemsep1pt\parskip0pt\parsep0pt
\item
  (i) F ∩ G\textbackslash{}mathrel\{≠\}∅
\item
  (ii) \textbackslash{}overrightarrow\{ab\}
  ∈\textbackslash{}overrightarrow\{ F\}
  +\textbackslash{}overrightarrow\{ G\}
\end{itemize}

Démonstration (i) ⇒(ii) Soit x ∈ F ∩ G. On a donc
\textbackslash{}overrightarrow\{ax\} ∈\textbackslash{}overrightarrow\{
F\} et \textbackslash{}overrightarrow\{xb\}
∈\textbackslash{}overrightarrow\{ G\}, d'où
\textbackslash{}overrightarrow\{ab\} =\textbackslash{}overrightarrow\{
ax\} +\textbackslash{}overrightarrow\{ bx\}
∈\textbackslash{}overrightarrow\{ F\} +\textbackslash{}overrightarrow\{
G\}.

(ii) ⇒(i) Ecrivons \textbackslash{}overrightarrow\{ab\}
=\textbackslash{}overrightarrow\{ ξ\} +\textbackslash{}overrightarrow\{
η\} avec \textbackslash{}overrightarrow\{ξ\}
∈\textbackslash{}overrightarrow\{ F\} et
\textbackslash{}overrightarrow\{η\} ∈\textbackslash{}overrightarrow\{
G\}. On a alors b −\textbackslash{}overrightarrow\{ η\} = a
+\textbackslash{}overrightarrow\{ ab\} −\textbackslash{}overrightarrow\{
η\} = a +\textbackslash{}overrightarrow\{ ξ\}. Alors le point x = b
−\textbackslash{}overrightarrow\{ η\} = a
+\textbackslash{}overrightarrow\{ ξ\} appartient à la fois à F et à G.

Remarque~17.1.5 En dimension 3 et pour deux droites non parallèles
\{D\}\_\{1\} = a + K\textbackslash{}vec\{u\} et \{D\}\_\{2\} = b +
K\textbackslash{}vec\{v\}, on en déduit que

\{D\}\_\{1\} ∩
\{D\}\_\{2\}\textbackslash{}mathrel\{≠\}∅\textbackslash{}mathrel\{⇔\}
\textbackslash{}overrightarrow\{ab\} ∈ K\textbackslash{}vec\{u\} +
K\textbackslash{}vec\{v\} \textbackslash{}mathrel\{⇔\}
(\textbackslash{}overrightarrow\{ab\},\textbackslash{}vec\{u\},\textbackslash{}vec\{v\})\textbackslash{}text\{
est liée \}

Corollaire~17.1.7 Soit F et G deux sous-espaces affines de E tels que
\textbackslash{}overrightarrow\{F\} et
\textbackslash{}overrightarrow\{G\} soient deux sous-espaces
supplémentaires de \textbackslash{}overrightarrow\{E\}. Alors F ∩ G est
un point.

Démonstration Soit a ∈ F et b ∈ G~; on a
\textbackslash{}overrightarrow\{ab\} ∈\textbackslash{}overrightarrow\{
E\} =\textbackslash{}overrightarrow\{ F\}
+\textbackslash{}overrightarrow\{ G\} donc F ∩ G n'est pas vide~; mais
alors \textbackslash{}overrightarrow\{F ∩ G\}
=\textbackslash{}overrightarrow\{ F\} ∩\textbackslash{}overrightarrow\{
G\} =
\textbackslash{}\{\textbackslash{}overrightarrow\{0\}\textbackslash{}\}
et donc l'intersection est un point.

\paragraph{17.1.5 Applications affines}

Définition~17.1.7 Soit E et F deux espaces affines et f : E → F. On dit
que f est une application affine si elle vérifie

\begin{itemize}
\itemsep1pt\parskip0pt\parsep0pt
\item
  (i) il existe a ∈ E et une application linéaire
  \textbackslash{}vec\{f\} de \textbackslash{}overrightarrow\{E\} dans
  \textbackslash{}overrightarrow\{F\} tels que
  \textbackslash{}mathop\{∀\}x ∈ E, f(x) = f(a) +\textbackslash{}vec\{
  f\}(\textbackslash{}overrightarrow\{ax\})
\item
  (ii) il existe une application linéaire \textbackslash{}vec\{f\} de
  \textbackslash{}overrightarrow\{E\} dans
  \textbackslash{}overrightarrow\{F\} telle que
  \textbackslash{}mathop\{∀\}x ∈ E,
  \textbackslash{}mathop\{∀\}\textbackslash{}overrightarrow\{ξ\}
  ∈\textbackslash{}overrightarrow\{ E\}, f(x
  +\textbackslash{}overrightarrow\{ ξ\}) = f(x) +\textbackslash{}vec\{
  f\}(\textbackslash{}overrightarrow\{ξ\})
\end{itemize}

L'application linéaire \textbackslash{}vec\{f\} est unique, on l'appelle
l'application linéaire tangente à l'application affine f.

Démonstration Il est clair que (ii) ⇒(i). Montrons donc que (i) ⇒(ii).
On a en effet

\textbackslash{}begin\{eqnarray*\} f(x +\textbackslash{}overrightarrow\{
ξ\})\& =\& f(a +\textbackslash{}overrightarrow\{ ax\}
+\textbackslash{}overrightarrow\{ ξ\}) = f(a) +\textbackslash{}vec\{
f\}(\textbackslash{}overrightarrow\{ax\}
+\textbackslash{}overrightarrow\{ ξ\}) \%\&
\textbackslash{}\textbackslash{} \& =\& f(a) +\textbackslash{}vec\{
f\}(\textbackslash{}overrightarrow\{ax\}) +\textbackslash{}vec\{
f\}(\textbackslash{}overrightarrow\{ξ\}) = f(x) +\textbackslash{}vec\{
f\}(\textbackslash{}overrightarrow\{ξ\})\%\&
\textbackslash{}\textbackslash{} \textbackslash{}end\{eqnarray*\}

La propriété (ii) montre que
\textbackslash{}vec\{f\}(\textbackslash{}overrightarrow\{ξ\})
=\textbackslash{}overrightarrow\{ f(x)f(x
+\textbackslash{}overrightarrow\{ ξ\})\} ce qui assure l'unicité de
\textbackslash{}vec\{f\}.

Proposition~17.1.8

\begin{itemize}
\itemsep1pt\parskip0pt\parsep0pt
\item
  (i) Si f : E → F et g : F → G sont deux applications affines, alors g
  ∘ f est affine et on a \textbackslash{}overrightarrow\{g ∘ f\}
  =\textbackslash{}vec\{ g\} ∘\textbackslash{}vec\{ f\}
\item
  (ii) f est injective (resp. surjective, resp. bijective) si et
  seulement si~\textbackslash{}vec\{f\} est injective (resp. surjective,
  resp. bijective)
\end{itemize}

Démonstration Elémentaire.

Remarque~17.1.6 Soit f une application affine de E dans E~; supposons
que f a un point fixe a. On a alors \textbackslash{}mathop\{∀\}x ∈ E,
\textbackslash{}overrightarrow\{af(x)\}
=\textbackslash{}overrightarrow\{ f(a)f(x)\} =\textbackslash{}vec\{
f\}(\textbackslash{}overrightarrow\{ax\}) si bien qu'en identifiant E et
\textbackslash{}overrightarrow\{E\} par le choix de l'origine a
(c'est-à-dire en identifiant x et \textbackslash{}overrightarrow\{ax\}),
f s'identifie à \textbackslash{}vec\{f\}. Une application affine ayant
un point fixe s'identifie, par le choix d'un tel point fixe comme
origine, à son application linéaire tangente. Il est donc
particulièrement important de savoir qu'une application affine a un
point fixe.

Théorème~17.1.9 Soit E un espace affine de dimension finie, f : E → E
une application affine. Si 1 n'est pas valeur propre de
\textbackslash{}vec\{f\}, alors f a un unique point fixe.

Démonstration Soit a ∈ E. On a

\textbackslash{}begin\{eqnarray*\}
\textbackslash{}overrightarrow\{xf(x)\}\& =\&
\textbackslash{}overrightarrow\{xa\} +\textbackslash{}overrightarrow\{
af(a)\} +\textbackslash{}overrightarrow\{ f(a)f(x)\} =
−\textbackslash{}overrightarrow\{ax\} +\textbackslash{}overrightarrow\{
af(a)\} +\textbackslash{}vec\{
f\}(\textbackslash{}overrightarrow\{ax\})\%\&
\textbackslash{}\textbackslash{} \& =\& (\textbackslash{}vec\{f\}
−\textbackslash{}mathrm\{Id\})(\textbackslash{}overrightarrow\{ax\})
+\textbackslash{}overrightarrow\{ af(a)\} \%\&
\textbackslash{}\textbackslash{} \textbackslash{}end\{eqnarray*\}

On en déduit que

\textbackslash{}begin\{eqnarray*\} x = f(x)\&
\textbackslash{}mathrel\{⇔\} \& (\textbackslash{}vec\{f\}
−\textbackslash{}mathrm\{Id\})(\textbackslash{}overrightarrow\{ax\})
+\textbackslash{}overrightarrow\{ af(a)\}
=\textbackslash{}overrightarrow\{ 0\}\%\&
\textbackslash{}\textbackslash{} \& \textbackslash{}mathrel\{⇔\} \&
(\textbackslash{}vec\{f\}
−\textbackslash{}mathrm\{Id\})(\textbackslash{}overrightarrow\{ax\})
=\textbackslash{}overrightarrow\{ f(a)a\} \%\&
\textbackslash{}\textbackslash{} \textbackslash{}end\{eqnarray*\}

Mais comme 1 n'est pas valeur propre de \textbackslash{}vec\{f\},
\textbackslash{}vec\{f\} −\textbackslash{}mathrm\{Id\} est bijective et
donc l'équation précédente définit un unique vecteur
\textbackslash{}overrightarrow\{ax\} et donc un unique point x.

Ceci nous amène à étudier certaines applications affines particulières.

Définition~17.1.8 Soit \textbackslash{}overrightarrow\{ξ\}
∈\textbackslash{}overrightarrow\{ E\}. On appelle translation de vecteur
\textbackslash{}overrightarrow\{ξ\} l'application
\{t\}\_\{\textbackslash{}overrightarrow\{ξ\}\} :
x\textbackslash{}mathrel\{↦\}x +\textbackslash{}overrightarrow\{ ξ\}.

Définition~17.1.9 Soit a ∈ E et λ ∈ \{K\}\^{}\{∗\}. On appelle
homothétie de centre a de rapport λ l'application \{h\}\_\{a,λ\} :
x\textbackslash{}mathrel\{↦\}a + λ \textbackslash{}overrightarrow\{ax\}.

Proposition~17.1.10 Soit f une application affine de E dans E. Alors

\begin{itemize}
\itemsep1pt\parskip0pt\parsep0pt
\item
  (i) f est une translation si et seulement si~\textbackslash{}vec\{f\}
  = \textbackslash{}mathrm\{Id\}
\item
  (ii) f est une homothétie de rapport λ\textbackslash{}mathrel\{≠\}1 si
  et seulement si~\textbackslash{}vec\{f\} =
  λ\textbackslash{}mathrm\{Id\}.
\end{itemize}

Démonstration Si f est une translation de vecteur
\textbackslash{}overrightarrow\{ξ\}, on a f(x
+\textbackslash{}overrightarrow\{ η\}) = x
+\textbackslash{}overrightarrow\{ η\} +\textbackslash{}overrightarrow\{
ξ\} = f(x) +\textbackslash{}overrightarrow\{ η\} ce qui montre que
\textbackslash{}vec\{f\} = \textbackslash{}mathrm\{Id\}. Inversement, si
\textbackslash{}vec\{f\} = \textbackslash{}mathrm\{Id\}, soit a ∈ E et
\textbackslash{}overrightarrow\{ξ\} =\textbackslash{}overrightarrow\{
af(a)\}~; alors f(x) = f(a) +\textbackslash{}vec\{
f\}(\textbackslash{}overrightarrow\{ax\}) = f(a)
+\textbackslash{}overrightarrow\{ ax\} = a
+\textbackslash{}overrightarrow\{ ξ\} +\textbackslash{}overrightarrow\{
ax\} = x +\textbackslash{}overrightarrow\{ ξ\} ce qui montre que f est
la translation \{t\}\_\{\textbackslash{}overrightarrow\{ξ\}\}.

Si f = \{h\}\_\{a,λ\}, la définition même montre que
\textbackslash{}vec\{f\} = λ\textbackslash{}mathrm\{Id\}. Inversement,
si \textbackslash{}vec\{f\} = λ\textbackslash{}mathrm\{Id\} avec
λ\textbackslash{}mathrel\{≠\}1, \textbackslash{}vec\{f\}
−\textbackslash{}mathrm\{Id\} est bijective et le même raisonnement que
ci dessus montre que f a un unique point fixe a. Mais alors f(x) = f(a)
+\textbackslash{}vec\{ f\}(\textbackslash{}overrightarrow\{ax\}) = a + λ
\textbackslash{}overrightarrow\{ax\}, donc f = \{h\}\_\{a,λ\}.

Remarque~17.1.7 Considérons le groupe GA(E) des applications affines
bijectives de E dans E~; on dispose de l'application
f\textbackslash{}mathrel\{↦\}\textbackslash{}vec\{f\} de GA(E) dans
GL(E) qui est visiblement un morphisme de groupes. L'ensemble constitué
des homothéties et des translations est l'image réciproque par cette
application du sous-groupe \{K\}\^{}\{∗\}\textbackslash{}mathrm\{Id\} de
GL(E). Il s'agit donc d'un sous-groupe de GA(E). On peut préciser ce
résultat en utilisant \textbackslash{}overrightarrow\{g ∘ f\}
=\textbackslash{}vec\{ g\} ∘\textbackslash{}vec\{ f\}~:la composée de
deux translations est une translation, la composée d'une homothétie et
d'une translation (dans n'importe quel ordre) est une homothétie, la
composée de deux homothéties est en général une homothétie, à moins que
le produit des deux rapports soit égal à 1 auquel cas la composée est
une translation.

\paragraph{17.1.6 Utilisation de repères affines}

Théorème~17.1.11 Soit
(a,\{(\textbackslash{}vec\{\{e\}\}\_\{i\})\}\_\{i∈I\}) un repère affine
de l'espace affine E. Soit b un point de l'espace affine F et
\{(\textbackslash{}vec\{\{u\}\}\_\{i\})\}\_\{i∈I\} une famille de
vecteurs de \textbackslash{}overrightarrow\{F\}. Alors il existe une
unique application affine f : E → F vérifiant f(a) = b et
\textbackslash{}mathop\{∀\}i ∈ I,
\textbackslash{}vec\{f\}(\textbackslash{}vec\{\{e\}\}\_\{i\})
=\textbackslash{}vec\{ \{u\}\}\_\{i\}.

Démonstration On a en effet nécessairement f(x) = f(a
+\textbackslash{}mathop\{ \textbackslash{}mathop\{∑ \}\}
\{x\}\_\{i\}\textbackslash{}vec\{\{e\}\}\_\{i\}) = f(a)
+\textbackslash{}mathop\{ \textbackslash{}mathop\{∑ \}\}
\{x\}\_\{i\}\textbackslash{}vec\{f\}(\textbackslash{}vec\{\{e\}\}\_\{i\})
= b +\{\textbackslash{}mathop\{ \textbackslash{}mathop\{∑ \}\}
\}\_\{i∈I\}\{x\}\_\{i\}\textbackslash{}vec\{\{u\}\}\_\{i\}. Inversement,
il est clair que f ainsi définie convient.

Corollaire~17.1.12 Soit \{(\{a\}\_\{i\})\}\_\{i∈I\} une base affine de E
et \{(\{b\}\_\{i\})\}\_\{i∈I\} une famille de points de F. Il existe une
unique application affine f : E → F vérifiant
\textbackslash{}mathop\{∀\}i ∈ I,f(\{a\}\_\{i\}) = \{b\}\_\{i\}.

Démonstration En effet, si \{i\}\_\{0\} ∈ I,
(\{a\}\_\{\{i\}\_\{0\}\},\{(\textbackslash{}overrightarrow\{\{a\}\_\{\{i\}\_\{0\}\}\{a\}\_\{i\}\})\}\_\{i∈I∖\textbackslash{}\{\{i\}\_\{0\}\textbackslash{}\}\})
est un repère de E et

\textbackslash{}begin\{eqnarray*\} \textbackslash{}mathop\{∀\}i ∈
I,f(\{a\}\_\{i\}) = \{b\}\_\{i\}\&\& \%\&
\textbackslash{}\textbackslash{} \& \textbackslash{}mathrel\{⇔\} \&
f(\{a\}\_\{\{i\}\_\{0\}\}) =
\{b\}\_\{\{i\}\_\{0\}\}\textbackslash{}text\{ et
\}\textbackslash{}mathop\{∀\}i ∈ I
∖\textbackslash{}\{\{i\}\_\{0\}\textbackslash{}\},
\textbackslash{}overrightarrow\{f(\{a\}\_\{\{i\}\_\{0\}\})f(\{a\}\_\{i\})\}
=\textbackslash{}overrightarrow\{
\{b\}\_\{\{i\}\_\{0\}\}\{b\}\_\{i\}\}\%\&
\textbackslash{}\textbackslash{} \& \textbackslash{}mathrel\{⇔\} \&
f(\{a\}\_\{\{i\}\_\{0\}\}) =
\{b\}\_\{\{i\}\_\{0\}\}\textbackslash{}text\{ et
\}\textbackslash{}mathop\{∀\}i ∈ I
∖\textbackslash{}\{\{i\}\_\{0\}\textbackslash{}\},
\textbackslash{}vec\{f\}(\textbackslash{}overrightarrow\{\{a\}\_\{\{i\}\_\{0\}\}\{a\}\_\{i\}\})
=\textbackslash{}overrightarrow\{ \{b\}\_\{\{i\}\_\{0\}\}\{b\}\_\{i\}\}
\%\& \textbackslash{}\textbackslash{} \textbackslash{}end\{eqnarray*\}

ce qui ramène au problème précédent.

Supposons E et F de dimensions finies. Soit
(a,\{(\textbackslash{}vec\{\{e\}\}\_\{j\})\}\_\{1≤i≤n\}) un repère
affine de l'espace affine E et
(b,\{(\textbackslash{}vec\{\{f\}\}\_\{i\})\}\_\{1≤i≤p\}) un repère
affine de F, soit u : E → F une application affine. Considérons A =
\{(\{a\}\_\{i,j\})\}\_\{1≤i≤p,1≤j≤n\} =\textbackslash{}mathop\{
\textbackslash{}mathrm\{Mat\}\}
(\textbackslash{}vec\{u\},\{(\textbackslash{}vec\{\{e\}\}\_\{j\})\}\_\{1≤j≤n\},\{(\textbackslash{}vec\{\{f\}\}\_\{i\})\}\_\{1≤i≤p\})
la matrice de l' application linéaire \textbackslash{}vec\{u\} dans les
bases respectives de \textbackslash{}overrightarrow\{E\} et
\textbackslash{}overrightarrow\{F\}. On a donc
\textbackslash{}vec\{u\}(\textbackslash{}vec\{\{e\}\}\_\{j\})
=\{\textbackslash{}mathop\{ \textbackslash{}mathop\{∑ \}\}
\}\_\{i=1\}\^{}\{p\}\{a\}\_\{i,j\}\textbackslash{}vec\{\{f\}\}\_\{i\}.
Posons u(a) = b +\{\textbackslash{}mathop\{ \textbackslash{}mathop\{∑
\}\}
\}\_\{i=1\}\^{}\{p\}\{α\}\_\{i\}\textbackslash{}vec\{\{f\}\}\_\{i\}.
Soit x = a +\{\textbackslash{}mathop\{ \textbackslash{}mathop\{∑ \}\}
\}\_\{j=1\}\^{}\{n\}\{x\}\_\{j\}\textbackslash{}vec\{\{e\}\}\_\{j\} ∈ E.
On a alors

\textbackslash{}begin\{eqnarray*\} u(x)\& =\& u(a) +\{
\textbackslash{}mathop\{∑ \}\}\_\{j=1\}\^{}\{n\}\{x\}\_\{
j\}\textbackslash{}vec\{u\}(\textbackslash{}vec\{\{e\}\}\_\{j\}) = b +\{
\textbackslash{}mathop\{∑ \}\}\_\{i=1\}\^{}\{p\}\{α\}\_\{
i\}\textbackslash{}vec\{\{f\}\}\_\{i\} +\{ \textbackslash{}mathop\{∑
\}\}\_\{j=1\}\^{}\{n\}\{x\}\_\{ j\}\{ \textbackslash{}mathop\{∑
\}\}\_\{i=1\}\^{}\{p\}\{a\}\_\{
i,j\}\textbackslash{}vec\{\{f\}\}\_\{i\}\%\&
\textbackslash{}\textbackslash{} \& =\& b +\{ \textbackslash{}mathop\{∑
\}\}\_\{i=1\}\^{}\{p\}\textbackslash{}left (\{α\}\_\{ i\} +\{
\textbackslash{}mathop\{∑ \}\}\_\{j=1\}\^{}\{n\}\{a\}\_\{
i,j\}\{x\}\_\{j\}\textbackslash{}right
)\textbackslash{}vec\{\{f\}\}\_\{i\} \%\&
\textbackslash{}\textbackslash{} \textbackslash{}end\{eqnarray*\}

Donc les coordonnées
\{y\}\_\{1\},\textbackslash{}mathop\{\textbackslash{}mathop\{\ldots{}\}\},\{y\}\_\{p\}
de u(x) dans le repère
(b,\{(\textbackslash{}vec\{\{f\}\}\_\{i\})\}\_\{1≤i≤p\}) de F sont
données par

\textbackslash{}mathop\{∀\}i ∈ {[}1,p{]}, \{y\}\_\{i\} =\{
\textbackslash{}mathop\{∑ \}\}\_\{j=1\}\^{}\{n\}\{a\}\_\{
i,j\}\{x\}\_\{j\} + \{α\}\_\{i\}

Introduisons les vecteurs colonnes X = \textbackslash{}left
(\textbackslash{}matrix\{\textbackslash{},\{x\}\_\{1\}
\textbackslash{}cr \textbackslash{}mathop\{\textbackslash{}mathop\{⋮\}\}
\textbackslash{}cr \{x\}\_\{n\} \textbackslash{}cr 1
\}\textbackslash{}right ), Y = \textbackslash{}left
(\textbackslash{}matrix\{\textbackslash{},\{y\}\_\{1\}
\textbackslash{}cr \textbackslash{}mathop\{\textbackslash{}mathop\{⋮\}\}
\textbackslash{}cr \{y\}\_\{p\} \textbackslash{}cr 1
\}\textbackslash{}right ) et A' = \textbackslash{}left
(\textbackslash{}matrix\{\textbackslash{},A
\&\textbackslash{}matrix\{\textbackslash{},\{α\}\_\{1\}
\textbackslash{}cr \textbackslash{}mathop\{\textbackslash{}mathop\{⋮\}\}
\textbackslash{}cr \{α\}\_\{p\}\} \textbackslash{}cr
\textbackslash{}matrix\{\textbackslash{},0\&\textbackslash{}mathop\{\textbackslash{}mathop\{\ldots{}\}\}\&0\}\&1
\}\textbackslash{}right ). On a alors

y = u(x) \textbackslash{}mathrel\{⇔\} Y = A'X

Définition~17.1.10 On dira que X = \textbackslash{}left
(\textbackslash{}matrix\{\textbackslash{},\{x\}\_\{1\}
\textbackslash{}cr \textbackslash{}mathop\{\textbackslash{}mathop\{⋮\}\}
\textbackslash{}cr \{x\}\_\{n\} \textbackslash{}cr 1
\}\textbackslash{}right ) est le vecteur colonne des coordonnées du
point x = a +\{\textbackslash{}mathop\{ \textbackslash{}mathop\{∑ \}\}
\}\_\{j=1\}\^{}\{n\}\{x\}\_\{j\}\textbackslash{}vec\{\{e\}\}\_\{j\} dans
le repère affine
(a,\{(\textbackslash{}vec\{\{e\}\}\_\{j\})\}\_\{1≤j≤n\}) de E. La
matrice A' = \textbackslash{}left
(\textbackslash{}matrix\{\textbackslash{},A
\&\textbackslash{}matrix\{\textbackslash{},\{α\}\_\{1\}
\textbackslash{}cr \textbackslash{}mathop\{\textbackslash{}mathop\{⋮\}\}
\textbackslash{}cr \{α\}\_\{p\}\} \textbackslash{}cr
\textbackslash{}matrix\{\textbackslash{},0\&\textbackslash{}mathop\{\textbackslash{}mathop\{\ldots{}\}\}\&0\}\&1
\}\textbackslash{}right ) (où A = \{(\{a\}\_\{i,j\})\}\_\{1≤i≤p,1≤j≤n\}
=\textbackslash{}mathop\{ \textbackslash{}mathrm\{Mat\}\}
(\textbackslash{}vec\{u\},\{(\textbackslash{}vec\{\{e\}\}\_\{j\})\}\_\{1≤j≤n\},\{(\textbackslash{}vec\{\{f\}\}\_\{i\})\}\_\{1≤i≤p\})
est la matrice de l' application linéaire \textbackslash{}vec\{u\} dans
les bases respectives de \textbackslash{}overrightarrow\{E\} et
\textbackslash{}overrightarrow\{F\} et u(a) = b
+\{\textbackslash{}mathop\{ \textbackslash{}mathop\{∑ \}\}
\}\_\{i=1\}\^{}\{p\}\{α\}\_\{i\}\textbackslash{}vec\{\{f\}\}\_\{i\})
sera appelée la matrice de l'application affine u dans les repères
(a,\{(\textbackslash{}vec\{\{e\}\}\_\{j\})\}\_\{1≤j≤n\}) et
(b,\{(\textbackslash{}vec\{\{f\}\}\_\{i\})\}\_\{1≤i≤p\}).

\paragraph{17.1.7 Formes affines et sous-espaces affines}

Définition~17.1.11 On appelle forme affine sur E toute application
affine f de E dans K.

Remarque~17.1.8 Si E est de dimension finie n, soit
(a,\{(\textbackslash{}vec\{\{e\}\}\_\{j\})\}\_\{1≤j≤n\}) un repère
affine de E. L'étude précédente sur la matrice d'une application affine
montre que f : E → K est une forme affine si et seulement si~elle est de
la forme x = a +\{\textbackslash{}mathop\{ \textbackslash{}mathop\{∑
\}\}
\}\_\{j=1\}\^{}\{n\}\{x\}\_\{j\}\textbackslash{}vec\{\{e\}\}\_\{j\}\textbackslash{}mathrel\{↦\}\{\textbackslash{}mathop\{\textbackslash{}mathop\{∑
\}\} \}\_\{j=1\}\^{}\{n\}\{a\}\_\{j\}\{x\}\_\{j\} + α.

Théorème~17.1.13 Soit E un espace affine de dimension n et F une partie
de E. Alors on a équivalence de

\begin{itemize}
\itemsep1pt\parskip0pt\parsep0pt
\item
  (i) F est un sous-espace affine de E de dimension p
\item
  (ii) il existe des formes affines
  \{f\}\_\{1\},\textbackslash{}mathop\{\textbackslash{}mathop\{\ldots{}\}\},\{f\}\_\{n−p\}
  telles que la famille
  (\textbackslash{}vec\{\{f\}\}\_\{1\},\textbackslash{}mathop\{\textbackslash{}mathop\{\ldots{}\}\},\textbackslash{}vec\{\{f\}\}\_\{n−p\})
  soit libre et F = \textbackslash{}\{x ∈
  E\textbackslash{}mathrel\{∣\}\{f\}\_\{1\}(x) =
  \textbackslash{}mathop\{\textbackslash{}mathop\{\ldots{}\}\} =
  \{f\}\_\{n−p\}(x) = 0\textbackslash{}\}.
\end{itemize}

Démonstration Soit tout d'abord F un sous-espace affine de dimension p
de direction \textbackslash{}overrightarrow\{F\}. Soit
(\textbackslash{}vec\{\{f\}\}\_\{1\},\textbackslash{}mathop\{\textbackslash{}mathop\{\ldots{}\}\},\textbackslash{}vec\{\{f\}\}\_\{n−p\})
une base de l'orthogonal de \textbackslash{}overrightarrow\{F\} dans le
dual \textbackslash{}overrightarrow\{\{E\}\}\^{}\{∗\} de l'espace
vectoriel \textbackslash{}overrightarrow\{E\}. On a alors

\textbackslash{}overrightarrow\{ξ\} ∈\textbackslash{}overrightarrow\{
F\} \textbackslash{}mathrel\{⇔\} \textbackslash{}mathop\{∀\}i ∈ {[}1,n −
p{]},
\textbackslash{}vec\{\{f\}\}\_\{i\}(\textbackslash{}overrightarrow\{ξ\})
= 0

Soit a ∈ F et soit \{f\}\_\{i\} la forme affine qui vaut 0 au point a et
d'application linéaire tangente \textbackslash{}vec\{\{f\}\}\_\{i\},
c'est-à-dire définie par \{f\}\_\{i\}(x) =\textbackslash{}vec\{
\{f\}\}\_\{i\}(\textbackslash{}overrightarrow\{ax\}). Alors

\textbackslash{}begin\{eqnarray*\} x ∈ F\& \textbackslash{}mathrel\{⇔\}
\& \textbackslash{}overrightarrow\{ax\}
∈\textbackslash{}overrightarrow\{ F\} \textbackslash{}mathrel\{⇔\}
\textbackslash{}mathop\{∀\}i ∈ {[}1,n − p{]},
\textbackslash{}vec\{\{f\}\}\_\{i\}(\textbackslash{}overrightarrow\{ax\})
=\textbackslash{}overrightarrow\{ 0\}\%\&
\textbackslash{}\textbackslash{} \& \textbackslash{}mathrel\{⇔\} \&
\textbackslash{}mathop\{∀\}i ∈ {[}1,n − p{]}, \{f\}\_\{i\}(x) = 0 \%\&
\textbackslash{}\textbackslash{} \textbackslash{}end\{eqnarray*\}

ce qui montre que (i) ⇒(ii).

Inversement, soit a ∈ E. Nous allons tout d'abord démontrer que si (ii)
est vérifiée alors F n'est pas vide. Pour cela, complétons
(\textbackslash{}vec\{\{f\}\}\_\{1\},\textbackslash{}mathop\{\textbackslash{}mathop\{\ldots{}\}\},\textbackslash{}vec\{\{f\}\}\_\{n−p\})
en une base
(\textbackslash{}vec\{\{f\}\}\_\{1\},\textbackslash{}mathop\{\textbackslash{}mathop\{\ldots{}\}\},\textbackslash{}vec\{\{f\}\}\_\{n\})
de \textbackslash{}overrightarrow\{\{E\}\}\^{}\{∗\}. Soit
(\textbackslash{}vec\{\{e\}\}\_\{1\},\textbackslash{}mathop\{\textbackslash{}mathop\{\ldots{}\}\},\textbackslash{}vec\{\{e\}\}\_\{n\})
une base dont c'est la duale, si bien que
\textbackslash{}vec\{\{f\}\}\_\{i\}(\textbackslash{}vec\{\{e\}\}\_\{j\})
= \{δ\}\_\{i\}\^{}\{j\}. Posons b = a
−\{\textbackslash{}mathop\{\textbackslash{}mathop\{∑ \}\}
\}\_\{i=1\}\^{}\{n−p\}\{f\}\_\{i\}(a)\textbackslash{}vec\{\{e\}\}\_\{i\}.
On a alors \{f\}\_\{j\}(b) = \{f\}\_\{j\}(a)
−\{\textbackslash{}mathop\{\textbackslash{}mathop\{∑ \}\}
\}\_\{i=1\}\^{}\{n−p\}\{f\}\_\{i\}(a)\textbackslash{}vec\{\{f\}\}\_\{j\}(\textbackslash{}vec\{\{e\}\}\_\{i\})
= \{f\}\_\{j\}(a) −\{\textbackslash{}mathop\{\textbackslash{}mathop\{∑
\}\} \}\_\{i=1\}\^{}\{p\}\{f\}\_\{i\}(a)\{δ\}\_\{i\}\^{}\{j\} =
\{f\}\_\{j\}(a) − \{f\}\_\{j\}(a) = 0 si bien que b appartient à F. On a
alors

\textbackslash{}begin\{eqnarray*\} x ∈ F\& \textbackslash{}mathrel\{⇔\}
\& \textbackslash{}mathop\{∀\}i ∈ {[}1,n − p{]}, \{f\}\_\{i\}(x) = 0
\%\& \textbackslash{}\textbackslash{} \& \textbackslash{}mathrel\{⇔\} \&
\textbackslash{}mathop\{∀\}i ∈ {[}1,n − p{]}, \{f\}\_\{i\}(b)
+\textbackslash{}vec\{
\{f\}\}\_\{i\}(\textbackslash{}overrightarrow\{bx\}) = 0 \%\&
\textbackslash{}\textbackslash{} \& \textbackslash{}mathrel\{⇔\} \&
\textbackslash{}mathop\{∀\}i ∈ {[}1,n − p{]},
\textbackslash{}vec\{\{f\}\}\_\{i\}(\textbackslash{}overrightarrow\{bx\})
= 0 \%\& \textbackslash{}\textbackslash{} \&
\textbackslash{}mathrel\{⇔\} \& \textbackslash{}mathop\{∀\}i ∈ {[}1,n −
p{]}, \textbackslash{}overrightarrow\{bx\} ∈
\{(\textbackslash{}vec\{\{f\}\}\_\{1\},\textbackslash{}mathop\{\textbackslash{}mathop\{\ldots{}\}\},\textbackslash{}vec\{\{f\}\}\_\{n−p\})\}\^{}\{⊥\}\%\&
\textbackslash{}\textbackslash{} \textbackslash{}end\{eqnarray*\}

On a donc F = b +
\{(\textbackslash{}vec\{\{f\}\}\_\{1\},\textbackslash{}mathop\{\textbackslash{}mathop\{\ldots{}\}\},\textbackslash{}vec\{\{f\}\}\_\{n−p\})\}\^{}\{⊥\}
qui est un sous-espace affine de dimension p.

De fa\textbackslash{}c\{c\}on plus pratique, dans un repère de E, on en
déduit que F est un sous-espace affine de dimension p si et seulement
si~F est l'ensemble des solutions d'un système de n − p équations
linéaires de rang n − p

x ∈ F\textbackslash{}quad \textbackslash{}mathrel\{⇔\}
\textbackslash{}quad \textbackslash{}left
\textbackslash{}\{\textbackslash{}array\{ \{α\}\_\{1,1\}\{x\}\_\{1\} +
\textbackslash{}mathop\{\textbackslash{}mathop\{\ldots{}\}\} +
\{α\}\_\{1,n\}\{x\}\_\{n\} \& = \{β\}\_\{1\} \textbackslash{}cr
\&\textbackslash{}mathop\{\textbackslash{}mathop\{\ldots{}\}\}\textbackslash{}cr
\{α\}\_\{ n−p,1\}\{x\}\_\{1\} +
\textbackslash{}mathop\{\textbackslash{}mathop\{\ldots{}\}\} +
\{α\}\_\{n−p,n\}\{x\}\_\{n\}\& = \{β\}\_\{n−p\} \} \textbackslash{}right
.

en posant \{f\}\_\{i\}(x) = \{α\}\_\{i,1\}\{x\}\_\{1\} +
\textbackslash{}mathop\{\textbackslash{}mathop\{\ldots{}\}\} +
\{α\}\_\{i,n\}\{x\}\_\{n\} − \{β\}\_\{i\} (et donc
\textbackslash{}vec\{\{f\}\}\_\{i\}(\textbackslash{}overrightarrow\{ξ\})
= \{α\}\_\{i,1\}\{x\}\_\{1\} +
\textbackslash{}mathop\{\textbackslash{}mathop\{\ldots{}\}\} +
\{α\}\_\{i,n\}\{x\}\_\{n\}).

Exemple~17.1.1 Un hyperplan affine (sous-espace affine de dimension n −
1) est toujours défini par F = \textbackslash{}\{x ∈
E\textbackslash{}mathrel\{∣\}f(x) = 0\textbackslash{}\} où f est une
forme affine telle que
\textbackslash{}vec\{f\}\textbackslash{}mathrel\{≠\}0 (soit f non
constante). Autrement dit, avec des coordonnées dans un repère, un
hyperplan est défini par une équation \{u\}\_\{1\}\{x\}\_\{1\} +
\textbackslash{}mathop\{\textbackslash{}mathop\{\ldots{}\}\} +
\{u\}\_\{n\}\{x\}\_\{n\} + h = 0 avec
(\{u\}\_\{1\},\textbackslash{}mathop\{\textbackslash{}mathop\{\ldots{}\}\},\{u\}\_\{n\})\textbackslash{}mathrel\{≠\}(0,\textbackslash{}mathop\{\textbackslash{}mathop\{\ldots{}\}\},0).

De la même fa\textbackslash{}c\{c\}on, dans un espace de dimension 3,
une droite est définie par deux équations

\textbackslash{}left \textbackslash{}\{\textbackslash{}array\{ ax + by +
cz + h\& = 0\textbackslash{}cr a'x + b'y + c'z + h' \& = 0 \}
\textbackslash{}right .

où la matrice \textbackslash{}left
(\textbackslash{}matrix\{\textbackslash{},a\&b\&c\textbackslash{}cr a'
\&b' \&c'\}\textbackslash{}right ) est de rang 2.

Théorème~17.1.14 (faisceaux d'hyperplans). Soit E un espace affine de
dimension finie, \{H\}\_\{1\} et \{H\}\_\{2\} deux hyperplans non
parallèles d'équations \{f\}\_\{1\}(x) = 0 et \{f\}\_\{2\}(x) = 0. Alors
les hyperplans contenant \{H\}\_\{1\} ∩ \{H\}\_\{2\} sont exactement les
hyperplans d'équations \{λ\}\_\{1\}\{f\}\_\{1\}(x) +
\{λ\}\_\{2\}\{f\}\_\{2\}(x) = 0 avec
(\{λ\}\_\{1\},\{λ\}\_\{2\})\textbackslash{}mathrel\{≠\}(0,0).

Démonstration Les hyperplans vectoriels
\textbackslash{}overrightarrow\{\{H\}\_\{1\}\} et
\textbackslash{}overrightarrow\{\{H\}\_\{2\}\} sont définis par les
équations
\textbackslash{}vec\{\{f\}\}\_\{1\}(\textbackslash{}overrightarrow\{ξ\})
= 0 et
\textbackslash{}vec\{\{f\}\}\_\{2\}(\textbackslash{}overrightarrow\{ξ\})
= 0. Comme ils sont distincts, ces formes linéaires
\textbackslash{}vec\{\{f\}\}\_\{1\} et
\textbackslash{}vec\{\{f\}\}\_\{2\} ne sont pas proportionnelles, et
donc
(\textbackslash{}vec\{\{f\}\}\_\{1\},\textbackslash{}vec\{\{f\}\}\_\{2\})
est libre. On en déduit que F = \{H\}\_\{1\} ∩ \{H\}\_\{2\} =
\textbackslash{}\{x ∈ E\textbackslash{}mathrel\{∣\}\{f\}\_\{1\}(x) =
\{f\}\_\{2\}(x) = 0\textbackslash{}\} est un sous-espace affine de
dimension n − 2 de E, en particulier F\textbackslash{}mathrel\{≠\}∅.
Soit a ∈ F. Alors

x ∈ F \textbackslash{}mathrel\{⇔\}
\textbackslash{}vec\{\{f\}\}\_\{1\}(\textbackslash{}overrightarrow\{ax\})
=\textbackslash{}vec\{
\{f\}\}\_\{2\}(\textbackslash{}overrightarrow\{ax\}) = 0
\textbackslash{}mathrel\{⇔\} \textbackslash{}overrightarrow\{ax\}
∈\textbackslash{}mathop\{\textbackslash{}mathrm\{Vect\}\}\{(\textbackslash{}vec\{\{f\}\}\_\{1\},\textbackslash{}vec\{\{f\}\}\_\{2\})\}\^{}\{⊥\}

si bien que \textbackslash{}overrightarrow\{F\}
=\textbackslash{}mathop\{
\textbackslash{}mathrm\{Vect\}\}\{(\textbackslash{}vec\{\{f\}\}\_\{1\},\textbackslash{}vec\{\{f\}\}\_\{2\})\}\^{}\{⊥\}
Soit H un hyperplan et f une équation de H. Alors

\textbackslash{}begin\{eqnarray*\} F ⊂ H\& \textbackslash{}mathrel\{⇔\}
\& f(a) = 0\textbackslash{}text\{ et
\}\textbackslash{}mathop\{∀\}\textbackslash{}overrightarrow\{ξ\} ∈ F,
\textbackslash{}vec\{f\}(\textbackslash{}overrightarrow\{ξ\}) = 0 \%\&
\textbackslash{}\textbackslash{} \& \textbackslash{}mathrel\{⇔\} \& f(a)
= 0\textbackslash{}text\{ et \}\textbackslash{}vec\{f\}
∈\{\textbackslash{}left
(\textbackslash{}mathop\{\textbackslash{}mathrm\{Vect\}\}\{(\textbackslash{}vec\{\{f\}\}\_\{1\},\textbackslash{}vec\{\{f\}\}\_\{2\})\}\^{}\{⊥\}\textbackslash{}right
)\}\^{}\{⊥\}\%\& \textbackslash{}\textbackslash{} \&
\textbackslash{}mathrel\{⇔\} \& f(a) = 0\textbackslash{}text\{ et
\}\textbackslash{}vec\{f\}
∈\textbackslash{}mathop\{\textbackslash{}mathrm\{Vect\}\}(\textbackslash{}vec\{\{f\}\}\_\{1\},\textbackslash{}vec\{\{f\}\}\_\{2\})
\%\& \textbackslash{}\textbackslash{} \textbackslash{}end\{eqnarray*\}

Mais comme \{f\}\_\{1\}(a) = \{f\}\_\{2\}(a) = 0, la condition f =
\{λ\}\_\{1\}\{f\}\_\{1\} + \{λ\}\_\{2\}\{f\}\_\{2\} est manifestement
équivalente à f(a) = 0 et \textbackslash{}vec\{f\} =
\{λ\}\_\{1\}\textbackslash{}vec\{\{f\}\}\_\{1\} +
\{λ\}\_\{2\}\textbackslash{}vec\{\{f\}\}\_\{2\} ce qui nous donne le
résultat souhaité.

Exemple~17.1.2 Dans un espace de dimension 3, soit D la droite
d'équations

\textbackslash{}left \textbackslash{}\{\textbackslash{}array\{ ax + by +
cz + h\& = 0\textbackslash{}cr a'x + b'y + c'z + h' \& = 0 \}
\textbackslash{}right .

où la matrice \textbackslash{}left
(\textbackslash{}matrix\{\textbackslash{},a\&b\&c\textbackslash{}cr a'
\&b' \&c'\}\textbackslash{}right ) est de rang 2. Alors les plans
contenant D sont exactement les plans d'équations \{λ\}\_\{1\}(ax + by +
cz + h) + \{λ\}\_\{2\}(a'x + b'y + c'z + h') = 0 avec
(\{λ\}\_\{1\},\{λ\}\_\{2\})\textbackslash{}mathrel\{≠\}(0,0).

De même en dimension 2, si un point A est défini comme intersection de
deux droites non parallèles d'équations ax + by + c = 0 et a'x + b'y +
c' = 0, les droites contenant A sont exactement les droites admettant
comme équations \{λ\}\_\{1\}(ax + by + c) + \{λ\}\_\{2\}(a'x + b'y + c')
= 0 avec (\{λ\}\_\{1\},\{λ\}\_\{2\})\textbackslash{}mathrel\{≠\}(0,0).

{[}\href{coursse93.html}{next}{]} {[}\href{coursse92.html}{front}{]}
{[}\href{coursch18.html\#coursse92.html}{up}{]}

\end{document}
