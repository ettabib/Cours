\documentclass[]{article}
\usepackage[T1]{fontenc}
\usepackage{lmodern}
\usepackage{amssymb,amsmath}
\usepackage{ifxetex,ifluatex}
\usepackage{fixltx2e} % provides \textsubscript
% use upquote if available, for straight quotes in verbatim environments
\IfFileExists{upquote.sty}{\usepackage{upquote}}{}
\ifnum 0\ifxetex 1\fi\ifluatex 1\fi=0 % if pdftex
  \usepackage[utf8]{inputenc}
\else % if luatex or xelatex
  \ifxetex
    \usepackage{mathspec}
    \usepackage{xltxtra,xunicode}
  \else
    \usepackage{fontspec}
  \fi
  \defaultfontfeatures{Mapping=tex-text,Scale=MatchLowercase}
  \newcommand{\euro}{€}
\fi
% use microtype if available
\IfFileExists{microtype.sty}{\usepackage{microtype}}{}
\ifxetex
  \usepackage[setpagesize=false, % page size defined by xetex
              unicode=false, % unicode breaks when used with xetex
              xetex]{hyperref}
\else
  \usepackage[unicode=true]{hyperref}
\fi
\hypersetup{breaklinks=true,
            bookmarks=true,
            pdfauthor={},
            pdftitle={Elements de topologie generale},
            colorlinks=true,
            citecolor=blue,
            urlcolor=blue,
            linkcolor=magenta,
            pdfborder={0 0 0}}
\urlstyle{same}  % don't use monospace font for urls
\setlength{\parindent}{0pt}
\setlength{\parskip}{6pt plus 2pt minus 1pt}
\setlength{\emergencystretch}{3em}  % prevent overfull lines
\setcounter{secnumdepth}{0}
 
/* start css.sty */
.cmr-5{font-size:50%;}
.cmr-7{font-size:70%;}
.cmmi-5{font-size:50%;font-style: italic;}
.cmmi-7{font-size:70%;font-style: italic;}
.cmmi-10{font-style: italic;}
.cmsy-5{font-size:50%;}
.cmsy-7{font-size:70%;}
.cmex-7{font-size:70%;}
.cmex-7x-x-71{font-size:49%;}
.msbm-7{font-size:70%;}
.cmtt-10{font-family: monospace;}
.cmti-10{ font-style: italic;}
.cmbx-10{ font-weight: bold;}
.cmr-17x-x-120{font-size:204%;}
.cmsl-10{font-style: oblique;}
.cmti-7x-x-71{font-size:49%; font-style: italic;}
.cmbxti-10{ font-weight: bold; font-style: italic;}
p.noindent { text-indent: 0em }
td p.noindent { text-indent: 0em; margin-top:0em; }
p.nopar { text-indent: 0em; }
p.indent{ text-indent: 1.5em }
@media print {div.crosslinks {visibility:hidden;}}
a img { border-top: 0; border-left: 0; border-right: 0; }
center { margin-top:1em; margin-bottom:1em; }
td center { margin-top:0em; margin-bottom:0em; }
.Canvas { position:relative; }
li p.indent { text-indent: 0em }
.enumerate1 {list-style-type:decimal;}
.enumerate2 {list-style-type:lower-alpha;}
.enumerate3 {list-style-type:lower-roman;}
.enumerate4 {list-style-type:upper-alpha;}
div.newtheorem { margin-bottom: 2em; margin-top: 2em;}
.obeylines-h,.obeylines-v {white-space: nowrap; }
div.obeylines-v p { margin-top:0; margin-bottom:0; }
.overline{ text-decoration:overline; }
.overline img{ border-top: 1px solid black; }
td.displaylines {text-align:center; white-space:nowrap;}
.centerline {text-align:center;}
.rightline {text-align:right;}
div.verbatim {font-family: monospace; white-space: nowrap; text-align:left; clear:both; }
.fbox {padding-left:3.0pt; padding-right:3.0pt; text-indent:0pt; border:solid black 0.4pt; }
div.fbox {display:table}
div.center div.fbox {text-align:center; clear:both; padding-left:3.0pt; padding-right:3.0pt; text-indent:0pt; border:solid black 0.4pt; }
div.minipage{width:100%;}
div.center, div.center div.center {text-align: center; margin-left:1em; margin-right:1em;}
div.center div {text-align: left;}
div.flushright, div.flushright div.flushright {text-align: right;}
div.flushright div {text-align: left;}
div.flushleft {text-align: left;}
.underline{ text-decoration:underline; }
.underline img{ border-bottom: 1px solid black; margin-bottom:1pt; }
.framebox-c, .framebox-l, .framebox-r { padding-left:3.0pt; padding-right:3.0pt; text-indent:0pt; border:solid black 0.4pt; }
.framebox-c {text-align:center;}
.framebox-l {text-align:left;}
.framebox-r {text-align:right;}
span.thank-mark{ vertical-align: super }
span.footnote-mark sup.textsuperscript, span.footnote-mark a sup.textsuperscript{ font-size:80%; }
div.tabular, div.center div.tabular {text-align: center; margin-top:0.5em; margin-bottom:0.5em; }
table.tabular td p{margin-top:0em;}
table.tabular {margin-left: auto; margin-right: auto;}
div.td00{ margin-left:0pt; margin-right:0pt; }
div.td01{ margin-left:0pt; margin-right:5pt; }
div.td10{ margin-left:5pt; margin-right:0pt; }
div.td11{ margin-left:5pt; margin-right:5pt; }
table[rules] {border-left:solid black 0.4pt; border-right:solid black 0.4pt; }
td.td00{ padding-left:0pt; padding-right:0pt; }
td.td01{ padding-left:0pt; padding-right:5pt; }
td.td10{ padding-left:5pt; padding-right:0pt; }
td.td11{ padding-left:5pt; padding-right:5pt; }
table[rules] {border-left:solid black 0.4pt; border-right:solid black 0.4pt; }
.hline hr, .cline hr{ height : 1px; margin:0px; }
.tabbing-right {text-align:right;}
span.TEX {letter-spacing: -0.125em; }
span.TEX span.E{ position:relative;top:0.5ex;left:-0.0417em;}
a span.TEX span.E {text-decoration: none; }
span.LATEX span.A{ position:relative; top:-0.5ex; left:-0.4em; font-size:85%;}
span.LATEX span.TEX{ position:relative; left: -0.4em; }
div.float img, div.float .caption {text-align:center;}
div.figure img, div.figure .caption {text-align:center;}
.marginpar {width:20%; float:right; text-align:left; margin-left:auto; margin-top:0.5em; font-size:85%; text-decoration:underline;}
.marginpar p{margin-top:0.4em; margin-bottom:0.4em;}
.equation td{text-align:center; vertical-align:middle; }
td.eq-no{ width:5%; }
table.equation { width:100%; } 
div.math-display, div.par-math-display{text-align:center;}
math .texttt { font-family: monospace; }
math .textit { font-style: italic; }
math .textsl { font-style: oblique; }
math .textsf { font-family: sans-serif; }
math .textbf { font-weight: bold; }
.partToc a, .partToc, .likepartToc a, .likepartToc {line-height: 200%; font-weight:bold; font-size:110%;}
.chapterToc a, .chapterToc, .likechapterToc a, .likechapterToc, .appendixToc a, .appendixToc {line-height: 200%; font-weight:bold;}
.index-item, .index-subitem, .index-subsubitem {display:block}
.caption td.id{font-weight: bold; white-space: nowrap; }
table.caption {text-align:center;}
h1.partHead{text-align: center}
p.bibitem { text-indent: -2em; margin-left: 2em; margin-top:0.6em; margin-bottom:0.6em; }
p.bibitem-p { text-indent: 0em; margin-left: 2em; margin-top:0.6em; margin-bottom:0.6em; }
.paragraphHead, .likeparagraphHead { margin-top:2em; font-weight: bold;}
.subparagraphHead, .likesubparagraphHead { font-weight: bold;}
.quote {margin-bottom:0.25em; margin-top:0.25em; margin-left:1em; margin-right:1em; text-align:\\jmathmathustify;}
.verse{white-space:nowrap; margin-left:2em}
div.maketitle {text-align:center;}
h2.titleHead{text-align:center;}
div.maketitle{ margin-bottom: 2em; }
div.author, div.date {text-align:center;}
div.thanks{text-align:left; margin-left:10%; font-size:85%; font-style:italic; }
div.author{white-space: nowrap;}
.quotation {margin-bottom:0.25em; margin-top:0.25em; margin-left:1em; }
h1.partHead{text-align: center}
.sectionToc, .likesectionToc {margin-left:2em;}
.subsectionToc, .likesubsectionToc {margin-left:4em;}
.subsubsectionToc, .likesubsubsectionToc {margin-left:6em;}
.frenchb-nbsp{font-size:75%;}
.frenchb-thinspace{font-size:75%;}
.figure img.graphics {margin-left:10%;}
/* end css.sty */

\title{Elements de topologie generale}
\author{}
\date{}

\begin{document}
\maketitle

\textbf{Warning: 
requires JavaScript to process the mathematics on this page.\\ If your
browser supports JavaScript, be sure it is enabled.}

\begin{center}\rule{3in}{0.4pt}\end{center}

{[}
{[}{]}
{[}

\subsubsection{4.1 Eléments de topologie générale}

\paragraph{4.1.1 Espaces topologiques}

Définition~4.1.1 Soit E un ensemble. On appelle topologie sur E toute
partie T de P(E) vérifiant les propriétés

\begin{itemize}
\itemsep1pt\parskip0pt\parsep0pt
\item
  (i) \varnothing~\inT et E \inT
\item
  (ii) A,B \inT \rigtharrow~ A \bigcap B \inT
\item
  (iii) Pour toute famille (U_i)_i\inI d'éléments de T,
  \⋃ ~
  _i\inIU_i appartient à T.
\end{itemize}

Les éléments de T s'appellent les ouverts de la topologie. On appelle
espace topologique un couple (E,T ) d'un ensemble E et d'une topologie T
sur E.

Remarque~4.1.1 On déduit de la propriété (ii) que toute intersection
finie d'ouverts est encore un ouvert.

Exemple~4.1.1 \\varnothing~,E\ est une topologie
sur E appelée la topologie grossière~; de même P(E) est une topologie
sur E appelée la topologie discrète.

\paragraph{4.1.2 La topologie de \mathbb{R}~}

Définition~4.1.2 On dit qu'une partie I de \mathbb{R}~ est un intervalle ouvert si
elle est de l'une des formes suivantes

\begin{itemize}
\itemsep1pt\parskip0pt\parsep0pt
\item
  (i) I ={]}a,b{[}= \x \in \mathbb{R}~∣a
  \textless{} x \textless{} b\
\item
  (ii) I ={]}a,+\infty~{[}= \x \in
  \mathbb{R}~∣a \textless{} x\ ou I
  ={]} -\infty~,a{[}= \x \in \mathbb{R}~∣x
  \textless{} a\
\item
  (iii) I ={]} -\infty~,+\infty~{[}= \mathbb{R}~
\end{itemize}

On vérifie facilement que cet ensemble noté ℐ est stable par
intersection finie (car on a un ordre total). Soit alors T l'ensemble
des réunions de familles d'intervalles ouverts. On vérifie facilement la
proposition suivante

Proposition~4.1.1 T est une topologie sur \mathbb{R}~ appelée la topologie
usuelle.

Théorème~4.1.2 Soit U une partie de \mathbb{R}~. On a équivalence de

\begin{itemize}
\itemsep1pt\parskip0pt\parsep0pt
\item
  (i) U est un ouvert pour la topologie usuelle
\item
  (ii) \forall~~x \in U,
  \existsI_x~ \inℐ,\quad x \in
  I_x \subset~ U
\end{itemize}

Démonstration ((i) \rigtharrow~ (ii)) Si U =\
⋃  _k\inKI_k~ et x \in U, alors
\existsk \in K, x \in I_k et I_x~ =
I_k convient.

((ii) \rigtharrow~ (i)) Soit V =\
⋃  _x\inUI_x~. V est une
réunion de parties de U donc V \subset~ U~; mais \forall~~x \in
U,x \in I_x \subset~ V , donc U \subset~ V . On a donc U = V \inT.

Corollaire~4.1.3 Dans \mathbb{R}~, une partie U est ouverte si et seulement si
elle vérifie

\forall~x \in U, \\exists~\epsilon
\textgreater{} 0,\quad {]}x - \epsilon,x + \epsilon{[}\subset~ U

Démonstration En effet {]}x - \epsilon,x + \epsilon{[} est un intervalle ouvert
contenant x, et inversement tout intervalle ouvert contenant x contient
un {]}x - \epsilon,x + \epsilon{[}, pour un \epsilon \textgreater{} 0 assez petit.

\paragraph{4.1.3 Fermés et voisinages}

Définition~4.1.3 Soit (E,T ) un espace topologique. On dit qu'une partie
A de E est fermée si son complémentaire est ouvert.

Proposition~4.1.4 (i) \varnothing~ et E sont fermés

\begin{itemize}
\itemsep1pt\parskip0pt\parsep0pt
\item
  (ii) si A,B sont des fermés, A \cup B est fermé (par récurrence, la
  réunion d'un nombre fini de fermés est fermée).
\item
  (iii) Pour toute famille (F_i)_i\inI de fermés,
  \⋂ ~
  _i\inIF_i est fermée.
\end{itemize}

Démonstration Par passage au complémentaire à partir des trois
propriétés des ouverts.

Remarque~4.1.2 Les parties \varnothing~ et E sont à la fois ouvertes et fermées~;
dans \mathbb{R}~ muni de la topologie usuelle, la partie {]}0,1{]} n'est ni
ouverte, ni fermée. Fermé n'est en aucun cas le contraire d'ouvert.

Définition~4.1.4 Soit (E,T ) un espace topologique, x \in E et V une
partie contenant x. On dit que V est un voisinage de x si il existe un
ouvert U tel que x \in U \subset~ V .

Proposition~4.1.5 Toute intersection finie de voisinages de x est un
voisinage de x~; toute partie contenant un voisinage de x est un
voisinage de x.

Démonstration Elémentaire.

Exemple~4.1.2 Dans \mathbb{R}~, V est un voisinage de x si et seulement si,
\exists~\epsilon \textgreater{} 0, {]}x - \epsilon,x + \epsilon{[}\subset~ V .

Théorème~4.1.6 Soit (E,T ) un espace topologique. Une partie U de E est
ouverte si et seulement si U est voisinage de tous ses points.

Démonstration Si U est ouverte, on a \forall~~x \in U, x
\in U \subset~ U, donc U est un voisinage de tous ses points. Inversement, si U
est voisinage de tous ses points, pour chaque x \in U, il existe
U_x ouvert tel que x \in U_x \subset~ U. Soit V
= \⋃ ~
_x\inUU_x. U est une réunion d'ouverts, donc un ouvert. On
a V \subset~ U comme réunion de parties de U et U \subset~ V car
\forall~x \in U,x \in U_x~ \subset~ V . Donc U = V et U
est ouvert.

Définition~4.1.5 On notera V (a) l'ensemble des voisinages de a.

\paragraph{4.1.4 Intérieur, adhérence, frontière}

Proposition~4.1.7 Soit A une partie de E (espace topologique).

\begin{itemize}
\itemsep1pt\parskip0pt\parsep0pt
\item
  (i) L'ensemble des ouverts contenus dans A a un plus grand élément
  appelé l'intérieur de A et noté A^o.
\item
  (ii) L'ensemble des fermés contenant A a un plus petit élément appelé
  l'adhérence de A et noté \overlineA.
\end{itemize}

Démonstration \⋃ ~
_ U\textouvert \atop U\subset~A U
est un ouvert contenu dans A et c'est bien entendu le plus grand. De
même \⋂  _
F\textfermé \atop A\subset~F F est un
fermé contenant A et c'est évidemment le plus petit.

Proposition~4.1.8 c(\overlineA) = (cA)^o
et c(A^o) = \overlinecA

Démonstration Il suffit de remarquer que les ouverts sont les
complémentaires des fermés et que U \subset~ A \Leftrightarrow cA
\subset~cU et que A \subset~ F \Leftrightarrow cF \subset~cA.

Théorème~4.1.9

\begin{itemize}
\itemsep1pt\parskip0pt\parsep0pt
\item
  (i) A^o = \x \in
  A∣A \in V (x)\
\item
  (ii) \overlineA = \x \in
  E∣\forall~~V \in V (x), V \bigcap
  A\neq~\varnothing~\
\end{itemize}

Démonstration (i) Soit U = \x \in
A∣A \in V (x)\. On a U \subset~ A. On
remarque que U est ouvert~; en effet si x \in U, on a A \in V (x), donc il
existe U_o ouvert tel que x \in U_o \subset~ A~; mais alors
\forall~y \in U_o, y \in U_o~ \subset~ A, donc
A \in V (y) soit x \in U_o \subset~ U et U \in V (x). Donc U est voisinage
de tous ses points, il est donc ouvert. On a donc U \subset~ A^o.
Mais inversement, si x \in A^o, comme A^o est
ouvert, on a x \in A^o \subset~ A, donc A \in V (x). On a donc
A^o \subset~ U soit A^o = U.

(ii) On a donc c(\overlineA) = (cA)^o =
\x∣cA \in V
(x)\ =
\x∣\exists~V
\in V (x), V \subset~cA\. On en déduit que
\overlineA = \x \in
E∣\forall~~V \in V (x), V \bigcap
A\neq~\varnothing~\.

Remarque~4.1.3 Pour la topologie naturelle de \mathbb{R}~, on a
(\overlineℚ)^o = \mathbb{R}~ et
\overlineℚ^o = \varnothing~~; ces applications ne
sont donc en rien réciproques.

Proposition~4.1.10 A est ouvert si et seulement si~A^o = A. A
est fermé si et seulement si~\overlineA = A.

Démonstration Evident.

Définition~4.1.6 Une partie A de E est dite dense dans E si elle vérifie
les conditions équivalentes

\begin{itemize}
\itemsep1pt\parskip0pt\parsep0pt
\item
  (i) \overlineA = E
\item
  (ii) \forall~x \in E, \\forall~~V \in
  V (x),\quad V \bigcap A\neq~\varnothing~.
\item
  (iii) Tout ouvert non vide de E contient un point de A
\end{itemize}

Démonstration L'équivalence de (i) et (ii) est claire d'après le
théorème précédent~; l'équivalence entre (ii) et (iii) est tout à fait
élémentaire~: tout ouvert est un voisinage, tout voisinage contient un
ouvert.

Définition~4.1.7 La frontière d'une partie A est
\mathrmFr~(A) =
\overlineA \diagdown A^o =
\overlineA \bigcap\overlinecA. C'est un
fermé de E.

Démonstration Elle est fermée comme intersection de deux fermés.

\paragraph{4.1.5 Topologie induite}

Définition~4.1.8 Soit (E,T ) un espace topologique, F une partie de E et
soit T_F = \U \bigcap F∣U
\inT\. Alors T_F est une topologie sur F appelée
la topologie induite par celle de E

Proposition~4.1.11 Soit A \subset~ F.

\begin{itemize}
\itemsep1pt\parskip0pt\parsep0pt
\item
  (i) A est fermée dans F si et seulement si~il existe B fermé de E tel
  que A = B \bigcap F.
\item
  (ii) Soit a \in A~; A est un voisinage de a dans F si et seulement si~il
  existe B voisinage de a dans E tel que A = B \bigcap F.
\end{itemize}

Démonstration ((i) \rigtharrow~) Supposons que A est fermé dans F. Alors F \diagdown A est
ouvert dans F et donc il existe U ouvert de E tel que F \diagdown A = U \bigcap F.
Mais on a alors A = (E \diagdown U) \bigcap F et donc A est l'intersection avec F d'un
fermé de E.

((i) ⇐) Si A = B \bigcap F, on a F \diagdown A = (E \diagdown B) \bigcap F donc F \diagdown A est ouvert
dans F, donc A est fermé dans F.

((ii) \rigtharrow~) Si A est un voisinage de a dans F, il existe U ouvert de F tel
que a \in U \subset~ A. Mais on a U = V \bigcap F, où V est un ouvert de E. Alors V \cup A
est un voisinage de a dans E tel que (V \cup A) \bigcap F = U \cup A = A.

((ii) ⇐) Si A = B \bigcap F où B est un voisinage de a dans E, il existe V
ouvert de E tel que a \in V \subset~ B. On a alors a \in V \bigcap F \subset~ A, ce qui montre
que A est un voisinage de a dans F.

Remarque~4.1.4 On prendra soin de ne pas confondre ouvert dans F et
ouvert dans E, fermé dans F et fermé dans E, etc.

{[}
{[}

\end{document}
