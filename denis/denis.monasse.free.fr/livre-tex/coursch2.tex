\documentclass[]{article}
\usepackage[T1]{fontenc}
\usepackage{lmodern}
\usepackage{amssymb,amsmath}
\usepackage{ifxetex,ifluatex}
\usepackage{fixltx2e} % provides \textsubscript
% use upquote if available, for straight quotes in verbatim environments
\IfFileExists{upquote.sty}{\usepackage{upquote}}{}
\ifnum 0\ifxetex 1\fi\ifluatex 1\fi=0 % if pdftex
  \usepackage[utf8]{inputenc}
\else % if luatex or xelatex
  \ifxetex
    \usepackage{mathspec}
    \usepackage{xltxtra,xunicode}
  \else
    \usepackage{fontspec}
  \fi
  \defaultfontfeatures{Mapping=tex-text,Scale=MatchLowercase}
  \newcommand{\euro}{€}
\fi
% use microtype if available
\IfFileExists{microtype.sty}{\usepackage{microtype}}{}
\ifxetex
  \usepackage[setpagesize=false, % page size defined by xetex
              unicode=false, % unicode breaks when used with xetex
              xetex]{hyperref}
\else
  \usepackage[unicode=true]{hyperref}
\fi
\hypersetup{breaklinks=true,
            bookmarks=true,
            pdfauthor={},
            pdftitle={1 Ensembles et structures},
            colorlinks=true,
            citecolor=blue,
            urlcolor=blue,
            linkcolor=magenta,
            pdfborder={0 0 0}}
\urlstyle{same}  % don't use monospace font for urls
\setlength{\parindent}{0pt}
\setlength{\parskip}{6pt plus 2pt minus 1pt}
\setlength{\emergencystretch}{3em}  % prevent overfull lines
\setcounter{secnumdepth}{0}
 
/* start css.sty */
.cmr-5{font-size:50%;}
.cmr-7{font-size:70%;}
.cmmi-5{font-size:50%;font-style: italic;}
.cmmi-7{font-size:70%;font-style: italic;}
.cmmi-10{font-style: italic;}
.cmsy-5{font-size:50%;}
.cmsy-7{font-size:70%;}
.cmex-7{font-size:70%;}
.cmex-7x-x-71{font-size:49%;}
.msbm-7{font-size:70%;}
.cmtt-10{font-family: monospace;}
.cmti-10{ font-style: italic;}
.cmbx-10{ font-weight: bold;}
.cmr-17x-x-120{font-size:204%;}
.cmsl-10{font-style: oblique;}
.cmti-7x-x-71{font-size:49%; font-style: italic;}
.cmbxti-10{ font-weight: bold; font-style: italic;}
p.noindent { text-indent: 0em }
td p.noindent { text-indent: 0em; margin-top:0em; }
p.nopar { text-indent: 0em; }
p.indent{ text-indent: 1.5em }
@media print {div.crosslinks {visibility:hidden;}}
a img { border-top: 0; border-left: 0; border-right: 0; }
center { margin-top:1em; margin-bottom:1em; }
td center { margin-top:0em; margin-bottom:0em; }
.Canvas { position:relative; }
li p.indent { text-indent: 0em }
.enumerate1 {list-style-type:decimal;}
.enumerate2 {list-style-type:lower-alpha;}
.enumerate3 {list-style-type:lower-roman;}
.enumerate4 {list-style-type:upper-alpha;}
div.newtheorem { margin-bottom: 2em; margin-top: 2em;}
.obeylines-h,.obeylines-v {white-space: nowrap; }
div.obeylines-v p { margin-top:0; margin-bottom:0; }
.overline{ text-decoration:overline; }
.overline img{ border-top: 1px solid black; }
td.displaylines {text-align:center; white-space:nowrap;}
.centerline {text-align:center;}
.rightline {text-align:right;}
div.verbatim {font-family: monospace; white-space: nowrap; text-align:left; clear:both; }
.fbox {padding-left:3.0pt; padding-right:3.0pt; text-indent:0pt; border:solid black 0.4pt; }
div.fbox {display:table}
div.center div.fbox {text-align:center; clear:both; padding-left:3.0pt; padding-right:3.0pt; text-indent:0pt; border:solid black 0.4pt; }
div.minipage{width:100%;}
div.center, div.center div.center {text-align: center; margin-left:1em; margin-right:1em;}
div.center div {text-align: left;}
div.flushright, div.flushright div.flushright {text-align: right;}
div.flushright div {text-align: left;}
div.flushleft {text-align: left;}
.underline{ text-decoration:underline; }
.underline img{ border-bottom: 1px solid black; margin-bottom:1pt; }
.framebox-c, .framebox-l, .framebox-r { padding-left:3.0pt; padding-right:3.0pt; text-indent:0pt; border:solid black 0.4pt; }
.framebox-c {text-align:center;}
.framebox-l {text-align:left;}
.framebox-r {text-align:right;}
span.thank-mark{ vertical-align: super }
span.footnote-mark sup.textsuperscript, span.footnote-mark a sup.textsuperscript{ font-size:80%; }
div.tabular, div.center div.tabular {text-align: center; margin-top:0.5em; margin-bottom:0.5em; }
table.tabular td p{margin-top:0em;}
table.tabular {margin-left: auto; margin-right: auto;}
div.td00{ margin-left:0pt; margin-right:0pt; }
div.td01{ margin-left:0pt; margin-right:5pt; }
div.td10{ margin-left:5pt; margin-right:0pt; }
div.td11{ margin-left:5pt; margin-right:5pt; }
table[rules] {border-left:solid black 0.4pt; border-right:solid black 0.4pt; }
td.td00{ padding-left:0pt; padding-right:0pt; }
td.td01{ padding-left:0pt; padding-right:5pt; }
td.td10{ padding-left:5pt; padding-right:0pt; }
td.td11{ padding-left:5pt; padding-right:5pt; }
table[rules] {border-left:solid black 0.4pt; border-right:solid black 0.4pt; }
.hline hr, .cline hr{ height : 1px; margin:0px; }
.tabbing-right {text-align:right;}
span.TEX {letter-spacing: -0.125em; }
span.TEX span.E{ position:relative;top:0.5ex;left:-0.0417em;}
a span.TEX span.E {text-decoration: none; }
span.LATEX span.A{ position:relative; top:-0.5ex; left:-0.4em; font-size:85%;}
span.LATEX span.TEX{ position:relative; left: -0.4em; }
div.float img, div.float .caption {text-align:center;}
div.figure img, div.figure .caption {text-align:center;}
.marginpar {width:20%; float:right; text-align:left; margin-left:auto; margin-top:0.5em; font-size:85%; text-decoration:underline;}
.marginpar p{margin-top:0.4em; margin-bottom:0.4em;}
.equation td{text-align:center; vertical-align:middle; }
td.eq-no{ width:5%; }
table.equation { width:100%; } 
div.math-display, div.par-math-display{text-align:center;}
math .texttt { font-family: monospace; }
math .textit { font-style: italic; }
math .textsl { font-style: oblique; }
math .textsf { font-family: sans-serif; }
math .textbf { font-weight: bold; }
.partToc a, .partToc, .likepartToc a, .likepartToc {line-height: 200%; font-weight:bold; font-size:110%;}
.chapterToc a, .chapterToc, .likechapterToc a, .likechapterToc, .appendixToc a, .appendixToc {line-height: 200%; font-weight:bold;}
.index-item, .index-subitem, .index-subsubitem {display:block}
.caption td.id{font-weight: bold; white-space: nowrap; }
table.caption {text-align:center;}
h1.partHead{text-align: center}
p.bibitem { text-indent: -2em; margin-left: 2em; margin-top:0.6em; margin-bottom:0.6em; }
p.bibitem-p { text-indent: 0em; margin-left: 2em; margin-top:0.6em; margin-bottom:0.6em; }
.paragraphHead, .likeparagraphHead { margin-top:2em; font-weight: bold;}
.subparagraphHead, .likesubparagraphHead { font-weight: bold;}
.quote {margin-bottom:0.25em; margin-top:0.25em; margin-left:1em; margin-right:1em; text-align:justify;}
.verse{white-space:nowrap; margin-left:2em}
div.maketitle {text-align:center;}
h2.titleHead{text-align:center;}
div.maketitle{ margin-bottom: 2em; }
div.author, div.date {text-align:center;}
div.thanks{text-align:left; margin-left:10%; font-size:85%; font-style:italic; }
div.author{white-space: nowrap;}
.quotation {margin-bottom:0.25em; margin-top:0.25em; margin-left:1em; }
h1.partHead{text-align: center}
.sectionToc, .likesectionToc {margin-left:2em;}
.subsectionToc, .likesubsectionToc {margin-left:4em;}
.subsubsectionToc, .likesubsubsectionToc {margin-left:6em;}
.frenchb-nbsp{font-size:75%;}
.frenchb-thinspace{font-size:75%;}
.figure img.graphics {margin-left:10%;}
/* end css.sty */

\title{1 Ensembles et structures}
\author{}
\date{}

\begin{document}
\maketitle

\textbf{Warning: \href{http://www.math.union.edu/locate/jsMath}{jsMath}
requires JavaScript to process the mathematics on this page.\\ If your
browser supports JavaScript, be sure it is enabled.}

\begin{center}\rule{3in}{0.4pt}\end{center}

{[}\href{coursch3.html}{next}{]} {[}\href{coursli1.html}{prev}{]}
{[}\href{coursli1.html\#tailcoursli1.html}{prev-tail}{]}
{[}\hyperref[tailcoursch2.html]{tail}{]}
{[}\href{cours.html\#coursch2.html}{up}{]}

\subsection{Chapitre~1\\Ensembles et structures}

~1.1 \href{coursse1.html\#x5-70001.1}{Ensembles et relations} \\ ~~1.1.1
\href{coursse1.html\#x5-80001.1.1}{Relations d'équivalences} \\ ~~1.1.2
\href{coursse1.html\#x5-90001.1.2}{Relations d'ordre} \\ ~~1.1.3
\href{coursse1.html\#x5-100001.1.3}{Eléments extrémaux} \\ ~~1.1.4
\href{coursse1.html\#x5-110001.1.4}{L'axiome de Zorn} \\ ~1.2
\href{coursse2.html\#x6-120001.2}{Cardinaux et entiers naturels} \\
~~1.2.1 \href{coursse2.html\#x6-130001.2.1}{Notion de cardinal} \\
~~1.2.2 \href{coursse2.html\#x6-140001.2.2}{Les entiers naturels} \\
~1.3 \href{coursse3.html\#x7-150001.3}{Groupes} \\ ~~1.3.1
\href{coursse3.html\#x7-160001.3.1}{Définitions et première propriété}
\\ ~~1.3.2 \href{coursse3.html\#x7-170001.3.2}{Sous-groupes} \\ ~~1.3.3
\href{coursse3.html\#x7-180001.3.3}{Quotient par un sous-groupe} \\
~~1.3.4 \href{coursse3.html\#x7-190001.3.4}{Morphisme de groupes} \\
~~1.3.5 \href{coursse3.html\#x7-200001.3.5}{Le groupe ℤ} \\ ~~1.3.6
\href{coursse3.html\#x7-210001.3.6}{Ordre d'un élément} \\ ~~1.3.7
\href{coursse3.html\#x7-220001.3.7}{Groupes finis} \\ ~~1.3.8
\href{coursse3.html\#x7-230001.3.8}{Groupes cycliques} \\ ~~1.3.9
\href{coursse3.html\#x7-240001.3.9}{Groupe opérant sur un ensemble} \\
~~1.3.10 \href{coursse3.html\#x7-250001.3.10}{Groupe des permutations
d'un ensemble fini} \\ ~1.4 \href{coursse4.html\#x8-260001.4}{Anneaux et
corps} \\ ~~1.4.1 \href{coursse4.html\#x8-270001.4.1}{Généralités sur
les anneaux} \\ ~~1.4.2 \href{coursse4.html\#x8-280001.4.2}{Idéaux et
quotients} \\ ~~1.4.3 \href{coursse4.html\#x8-290001.4.3}{Morphisme
d'anneaux} \\ ~~1.4.4 \href{coursse4.html\#x8-300001.4.4}{Corps} \\
~~1.4.5 \href{coursse4.html\#x8-310001.4.5}{Idéaux maximaux} \\ ~~1.4.6
\href{coursse4.html\#x8-320001.4.6}{Idéaux et anneaux principaux} \\
~~1.4.7 \href{coursse4.html\#x8-330001.4.7}{Anneaux euclidiens} \\
~~1.4.8 \href{coursse4.html\#x8-340001.4.8}{L'anneau ℤ. Caractéristique
d'un anneau} \\ ~~1.4.9 \href{coursse4.html\#x8-350001.4.9}{Théorème
chinois, indicateur d'Euler} \\ ~1.5
\href{coursse5.html\#x9-360001.5}{Polynômes à une variable} \\ ~~1.5.1
\href{coursse5.html\#x9-370001.5.1}{L'anneau des séries formelles à
coefficients dans A} \\ ~~1.5.2
\href{coursse5.html\#x9-380001.5.2}{L'anneau des polynômes à
coefficients dans A} \\ ~~1.5.3
\href{coursse5.html\#x9-390001.5.3}{Division euclidienne et racines} \\
~~1.5.4 \href{coursse5.html\#x9-400001.5.4}{Dérivation} \\ ~~1.5.5
\href{coursse5.html\#x9-410001.5.5}{L'anneau principal K{[}X{]}} \\
~~1.5.6 \href{coursse5.html\#x9-420001.5.6}{Formule de Taylor.
Multiplicité d'une racine} \\ ~~1.5.7
\href{coursse5.html\#x9-430001.5.7}{Racines et extensions de corps} \\
~~1.5.8 \href{coursse5.html\#x9-440001.5.8}{Polynômes sur ℂ et ℝ} \\
~~1.5.9 \href{coursse5.html\#x9-450001.5.9}{Division suivant les
puissances croissantes} \\ ~1.6
\href{coursse6.html\#x10-460001.6}{Polynômes à plusieurs variables} \\
~~1.6.1 \href{coursse6.html\#x10-470001.6.1}{Généralités} \\ ~~1.6.2
\href{coursse6.html\#x10-480001.6.2}{Dérivées partielles, formule de
Taylor} \\ ~~1.6.3 \href{coursse6.html\#x10-490001.6.3}{Degré total,
polynômes homogènes} \\ ~~1.6.4
\href{coursse6.html\#x10-500001.6.4}{Polynômes symétriques}

{[}\href{coursch3.html}{next}{]} {[}\href{coursli1.html}{prev}{]}
{[}\href{coursli1.html\#tailcoursli1.html}{prev-tail}{]}
{[}\href{coursch2.html}{front}{]}
{[}\href{cours.html\#coursch2.html}{up}{]}

\end{document}
