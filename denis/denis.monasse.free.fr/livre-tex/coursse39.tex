\documentclass[]{article}
\usepackage[T1]{fontenc}
\usepackage{lmodern}
\usepackage{amssymb,amsmath}
\usepackage{ifxetex,ifluatex}
\usepackage{fixltx2e} % provides \textsubscript
% use upquote if available, for straight quotes in verbatim environments
\IfFileExists{upquote.sty}{\usepackage{upquote}}{}
\ifnum 0\ifxetex 1\fi\ifluatex 1\fi=0 % if pdftex
  \usepackage[utf8]{inputenc}
\else % if luatex or xelatex
  \ifxetex
    \usepackage{mathspec}
    \usepackage{xltxtra,xunicode}
  \else
    \usepackage{fontspec}
  \fi
  \defaultfontfeatures{Mapping=tex-text,Scale=MatchLowercase}
  \newcommand{\euro}{€}
\fi
% use microtype if available
\IfFileExists{microtype.sty}{\usepackage{microtype}}{}
\ifxetex
  \usepackage[setpagesize=false, % page size defined by xetex
              unicode=false, % unicode breaks when used with xetex
              xetex]{hyperref}
\else
  \usepackage[unicode=true]{hyperref}
\fi
\hypersetup{breaklinks=true,
            bookmarks=true,
            pdfauthor={},
            pdftitle={Series semi-convergentes},
            colorlinks=true,
            citecolor=blue,
            urlcolor=blue,
            linkcolor=magenta,
            pdfborder={0 0 0}}
\urlstyle{same}  % don't use monospace font for urls
\setlength{\parindent}{0pt}
\setlength{\parskip}{6pt plus 2pt minus 1pt}
\setlength{\emergencystretch}{3em}  % prevent overfull lines
\setcounter{secnumdepth}{0}
 
/* start css.sty */
.cmr-5{font-size:50%;}
.cmr-7{font-size:70%;}
.cmmi-5{font-size:50%;font-style: italic;}
.cmmi-7{font-size:70%;font-style: italic;}
.cmmi-10{font-style: italic;}
.cmsy-5{font-size:50%;}
.cmsy-7{font-size:70%;}
.cmex-7{font-size:70%;}
.cmex-7x-x-71{font-size:49%;}
.msbm-7{font-size:70%;}
.cmtt-10{font-family: monospace;}
.cmti-10{ font-style: italic;}
.cmbx-10{ font-weight: bold;}
.cmr-17x-x-120{font-size:204%;}
.cmsl-10{font-style: oblique;}
.cmti-7x-x-71{font-size:49%; font-style: italic;}
.cmbxti-10{ font-weight: bold; font-style: italic;}
p.noindent { text-indent: 0em }
td p.noindent { text-indent: 0em; margin-top:0em; }
p.nopar { text-indent: 0em; }
p.indent{ text-indent: 1.5em }
@media print {div.crosslinks {visibility:hidden;}}
a img { border-top: 0; border-left: 0; border-right: 0; }
center { margin-top:1em; margin-bottom:1em; }
td center { margin-top:0em; margin-bottom:0em; }
.Canvas { position:relative; }
li p.indent { text-indent: 0em }
.enumerate1 {list-style-type:decimal;}
.enumerate2 {list-style-type:lower-alpha;}
.enumerate3 {list-style-type:lower-roman;}
.enumerate4 {list-style-type:upper-alpha;}
div.newtheorem { margin-bottom: 2em; margin-top: 2em;}
.obeylines-h,.obeylines-v {white-space: nowrap; }
div.obeylines-v p { margin-top:0; margin-bottom:0; }
.overline{ text-decoration:overline; }
.overline img{ border-top: 1px solid black; }
td.displaylines {text-align:center; white-space:nowrap;}
.centerline {text-align:center;}
.rightline {text-align:right;}
div.verbatim {font-family: monospace; white-space: nowrap; text-align:left; clear:both; }
.fbox {padding-left:3.0pt; padding-right:3.0pt; text-indent:0pt; border:solid black 0.4pt; }
div.fbox {display:table}
div.center div.fbox {text-align:center; clear:both; padding-left:3.0pt; padding-right:3.0pt; text-indent:0pt; border:solid black 0.4pt; }
div.minipage{width:100%;}
div.center, div.center div.center {text-align: center; margin-left:1em; margin-right:1em;}
div.center div {text-align: left;}
div.flushright, div.flushright div.flushright {text-align: right;}
div.flushright div {text-align: left;}
div.flushleft {text-align: left;}
.underline{ text-decoration:underline; }
.underline img{ border-bottom: 1px solid black; margin-bottom:1pt; }
.framebox-c, .framebox-l, .framebox-r { padding-left:3.0pt; padding-right:3.0pt; text-indent:0pt; border:solid black 0.4pt; }
.framebox-c {text-align:center;}
.framebox-l {text-align:left;}
.framebox-r {text-align:right;}
span.thank-mark{ vertical-align: super }
span.footnote-mark sup.textsuperscript, span.footnote-mark a sup.textsuperscript{ font-size:80%; }
div.tabular, div.center div.tabular {text-align: center; margin-top:0.5em; margin-bottom:0.5em; }
table.tabular td p{margin-top:0em;}
table.tabular {margin-left: auto; margin-right: auto;}
div.td00{ margin-left:0pt; margin-right:0pt; }
div.td01{ margin-left:0pt; margin-right:5pt; }
div.td10{ margin-left:5pt; margin-right:0pt; }
div.td11{ margin-left:5pt; margin-right:5pt; }
table[rules] {border-left:solid black 0.4pt; border-right:solid black 0.4pt; }
td.td00{ padding-left:0pt; padding-right:0pt; }
td.td01{ padding-left:0pt; padding-right:5pt; }
td.td10{ padding-left:5pt; padding-right:0pt; }
td.td11{ padding-left:5pt; padding-right:5pt; }
table[rules] {border-left:solid black 0.4pt; border-right:solid black 0.4pt; }
.hline hr, .cline hr{ height : 1px; margin:0px; }
.tabbing-right {text-align:right;}
span.TEX {letter-spacing: -0.125em; }
span.TEX span.E{ position:relative;top:0.5ex;left:-0.0417em;}
a span.TEX span.E {text-decoration: none; }
span.LATEX span.A{ position:relative; top:-0.5ex; left:-0.4em; font-size:85%;}
span.LATEX span.TEX{ position:relative; left: -0.4em; }
div.float img, div.float .caption {text-align:center;}
div.figure img, div.figure .caption {text-align:center;}
.marginpar {width:20%; float:right; text-align:left; margin-left:auto; margin-top:0.5em; font-size:85%; text-decoration:underline;}
.marginpar p{margin-top:0.4em; margin-bottom:0.4em;}
.equation td{text-align:center; vertical-align:middle; }
td.eq-no{ width:5%; }
table.equation { width:100%; } 
div.math-display, div.par-math-display{text-align:center;}
math .texttt { font-family: monospace; }
math .textit { font-style: italic; }
math .textsl { font-style: oblique; }
math .textsf { font-family: sans-serif; }
math .textbf { font-weight: bold; }
.partToc a, .partToc, .likepartToc a, .likepartToc {line-height: 200%; font-weight:bold; font-size:110%;}
.chapterToc a, .chapterToc, .likechapterToc a, .likechapterToc, .appendixToc a, .appendixToc {line-height: 200%; font-weight:bold;}
.index-item, .index-subitem, .index-subsubitem {display:block}
.caption td.id{font-weight: bold; white-space: nowrap; }
table.caption {text-align:center;}
h1.partHead{text-align: center}
p.bibitem { text-indent: -2em; margin-left: 2em; margin-top:0.6em; margin-bottom:0.6em; }
p.bibitem-p { text-indent: 0em; margin-left: 2em; margin-top:0.6em; margin-bottom:0.6em; }
.paragraphHead, .likeparagraphHead { margin-top:2em; font-weight: bold;}
.subparagraphHead, .likesubparagraphHead { font-weight: bold;}
.quote {margin-bottom:0.25em; margin-top:0.25em; margin-left:1em; margin-right:1em; text-align:\\jmathmathustify;}
.verse{white-space:nowrap; margin-left:2em}
div.maketitle {text-align:center;}
h2.titleHead{text-align:center;}
div.maketitle{ margin-bottom: 2em; }
div.author, div.date {text-align:center;}
div.thanks{text-align:left; margin-left:10%; font-size:85%; font-style:italic; }
div.author{white-space: nowrap;}
.quotation {margin-bottom:0.25em; margin-top:0.25em; margin-left:1em; }
h1.partHead{text-align: center}
.sectionToc, .likesectionToc {margin-left:2em;}
.subsectionToc, .likesubsectionToc {margin-left:4em;}
.subsubsectionToc, .likesubsubsectionToc {margin-left:6em;}
.frenchb-nbsp{font-size:75%;}
.frenchb-thinspace{font-size:75%;}
.figure img.graphics {margin-left:10%;}
/* end css.sty */

\title{Series semi-convergentes}
\author{}
\date{}

\begin{document}
\maketitle

\textbf{Warning: 
requires JavaScript to process the mathematics on this page.\\ If your
browser supports JavaScript, be sure it is enabled.}

\begin{center}\rule{3in}{0.4pt}\end{center}

{[}
{[}
{[}{]}
{[}

\subsubsection{7.5 Séries semi-convergentes}

\paragraph{7.5.1 Séries alternées}

Théorème~7.5.1 (convergence des séries alternées). Soit (a_n)
une suite de nombres réels, décroissante, de limite 0. Alors la série
\\sum ~
(-1)^na_n converge~; le reste d'ordre n est du signe
de son premier terme (c'est-à-dire (-1)^n+1) et sa valeur
absolue est ma\\jmathmathorée par la valeur absolue de ce premier terme
(c'est-à-dire a_n+1).

Démonstration On a S_2n+2 - S_2n = a_2n+2 -
a_2n+1 \leq 0 et S_2n+3 - S_2n+1 =
a_2n+2 - a_2n+3 ≥ 0. La suite (S_2n) est donc
décroissante, la suite S_2n+1 est croissante~; comme
S_2n - S_2n+1 = a_2n+1 est positif et tend
vers 0, ces deux suites forment un couple de suites ad\\jmathmathacentes~; elles
admettent donc une limite commune S qui est limite de la suite
S_n. On a pour tout n, S_2n-1 \leq S_2n+1 \leq S \leq
S_2n. Ceci nous montre que 0 \leq-R_2n = S_2n -
S \leq S_2n - S_2n+1 = a_2n+1 et que 0 \leq
R_2n-1 = S - S_2n-1 \leq S_2n - S_2n-1
= a_2n d'où les assertions sur le reste.

\paragraph{7.5.2 Etude de séries semi-convergentes}

Les théorèmes de comparaison ne s'appliquent pas aux séries quelconques.
Ainsi on a  (-1)^n \over
\sqrtn ∼ (-1)^n \over
\sqrtn + 1 \over n alors que la
première est convergente et la deuxième divergente (somme d'une série
convergente et d'une série divergente). Pour une série à termes réels,
on peut envisager le plan suivant

(i) regarder si le critère de convergence des séries alternées
s'applique (a_n =
(-1)^na_n avec
a_n décroissant de limite 0).

(ii) si a_n = (-1)^na_n
mais si on ne peut pas appliquer le critère de convergence des séries
alternées, on peut essayer de trouver une série alternée
\\sum  b_n~ qui
relève de ce critère telle que a_n ∼ b_n~; alors,
comme la série \\sum ~
b_n converge, la nature de la série
\\sum  a_n~ sera
la même que celle de la série
\\sum  (a_n~ -
b_n), avec a_n - b_n = o(a_n)~; on
essayera de poursuivre le processus \\jmathmathusqu'à tomber soit sur une série
divergente, soit sur une série absolument convergente

(iii) si a_n n'est pas alterné en signes, on peut utiliser une
sommation par paquets (cf plus loin)~: en regroupant les termes
consécutifs de même signe, on aboutira à une série alternée en signe à
laquelle on pourra appliquer l'une des méthodes précédentes

Enfin, pour une série à termes non réels ou qui ne relève pas d'une des
méthodes précédentes, on pourra utiliser un théorème d'Abel comme le
suivant

Théorème~7.5.2 Soit (a_n) une suite de nombres réels et
(x_n) une suite de l'espace vectoriel normé~complet E telles
que

\begin{itemize}
\itemsep1pt\parskip0pt\parsep0pt
\item
  (i) \existsM ≥ 0, \\forall~~n \in
  \mathbb{N}~,
  \\\\sum
   _p=0^nx_p\ \leq M
\item
  (ii) la suite (a_n) tend vers 0 en décroissant.
\end{itemize}

Alors la série \\sum ~
a_nx_n converge.

Démonstration On a

\begin{align*} \\sum
_n=p^qa_ nx_n& =&
\sum _n=p^qa_
n(S_n(x) - S_n-1(x)) \%&
\\ & =& \\sum
_n=p^qa_ nS_n(x)
-\sum _n=p^qa_
nS_n-1(x) \%& \\ & =&
\sum _n=p^qa_
nS_n(x) -\\sum
_n=p-1^q-1a_ n+1S_n(x) \%&
\\ \text (changement
d'indices \$n - 1\mapsto~n\$)&& \%&
\\ & & \%&
\\ & =& a_qS_q(x) -
a_pS_p-1(x) + \\sum
_n=p^q-1(a_ n -
a_n+1)S_n(x)\%& \\
\end{align*}

On a effectué ici une transformation dite transformation d'Abel.

Comme \forall~~n,
\S_n(x)\ \leq M
on a

\\\sum
_n=p^qa_
nx_n\ \leq
M(a_q + a_p
+ \\sum
_n=p^q-1a_ n -
a_n+1) = 2Ma_p

en tenant compte de a_n ≥ 0 et a_n - a_n+1 ≥
0. Comme lima_p~ = 0, la série
\\sum ~
a_nx_n vérifie le critère de Cauchy, donc elle
converge.

{[}
{[}
{[}
{[}

\end{document}
