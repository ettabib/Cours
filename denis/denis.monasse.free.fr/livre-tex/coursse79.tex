\documentclass[]{article}
\usepackage[T1]{fontenc}
\usepackage{lmodern}
\usepackage{amssymb,amsmath}
\usepackage{ifxetex,ifluatex}
\usepackage{fixltx2e} % provides \textsubscript
% use upquote if available, for straight quotes in verbatim environments
\IfFileExists{upquote.sty}{\usepackage{upquote}}{}
\ifnum 0\ifxetex 1\fi\ifluatex 1\fi=0 % if pdftex
  \usepackage[utf8]{inputenc}
\else % if luatex or xelatex
  \ifxetex
    \usepackage{mathspec}
    \usepackage{xltxtra,xunicode}
  \else
    \usepackage{fontspec}
  \fi
  \defaultfontfeatures{Mapping=tex-text,Scale=MatchLowercase}
  \newcommand{\euro}{€}
\fi
% use microtype if available
\IfFileExists{microtype.sty}{\usepackage{microtype}}{}
\ifxetex
  \usepackage[setpagesize=false, % page size defined by xetex
              unicode=false, % unicode breaks when used with xetex
              xetex]{hyperref}
\else
  \usepackage[unicode=true]{hyperref}
\fi
\hypersetup{breaklinks=true,
            bookmarks=true,
            pdfauthor={},
            pdftitle={Serie de Fourier d'une fonction},
            colorlinks=true,
            citecolor=blue,
            urlcolor=blue,
            linkcolor=magenta,
            pdfborder={0 0 0}}
\urlstyle{same}  % don't use monospace font for urls
\setlength{\parindent}{0pt}
\setlength{\parskip}{6pt plus 2pt minus 1pt}
\setlength{\emergencystretch}{3em}  % prevent overfull lines
\setcounter{secnumdepth}{0}
 
/* start css.sty */
.cmr-5{font-size:50%;}
.cmr-7{font-size:70%;}
.cmmi-5{font-size:50%;font-style: italic;}
.cmmi-7{font-size:70%;font-style: italic;}
.cmmi-10{font-style: italic;}
.cmsy-5{font-size:50%;}
.cmsy-7{font-size:70%;}
.cmex-7{font-size:70%;}
.cmex-7x-x-71{font-size:49%;}
.msbm-7{font-size:70%;}
.cmtt-10{font-family: monospace;}
.cmti-10{ font-style: italic;}
.cmbx-10{ font-weight: bold;}
.cmr-17x-x-120{font-size:204%;}
.cmsl-10{font-style: oblique;}
.cmti-7x-x-71{font-size:49%; font-style: italic;}
.cmbxti-10{ font-weight: bold; font-style: italic;}
p.noindent { text-indent: 0em }
td p.noindent { text-indent: 0em; margin-top:0em; }
p.nopar { text-indent: 0em; }
p.indent{ text-indent: 1.5em }
@media print {div.crosslinks {visibility:hidden;}}
a img { border-top: 0; border-left: 0; border-right: 0; }
center { margin-top:1em; margin-bottom:1em; }
td center { margin-top:0em; margin-bottom:0em; }
.Canvas { position:relative; }
li p.indent { text-indent: 0em }
.enumerate1 {list-style-type:decimal;}
.enumerate2 {list-style-type:lower-alpha;}
.enumerate3 {list-style-type:lower-roman;}
.enumerate4 {list-style-type:upper-alpha;}
div.newtheorem { margin-bottom: 2em; margin-top: 2em;}
.obeylines-h,.obeylines-v {white-space: nowrap; }
div.obeylines-v p { margin-top:0; margin-bottom:0; }
.overline{ text-decoration:overline; }
.overline img{ border-top: 1px solid black; }
td.displaylines {text-align:center; white-space:nowrap;}
.centerline {text-align:center;}
.rightline {text-align:right;}
div.verbatim {font-family: monospace; white-space: nowrap; text-align:left; clear:both; }
.fbox {padding-left:3.0pt; padding-right:3.0pt; text-indent:0pt; border:solid black 0.4pt; }
div.fbox {display:table}
div.center div.fbox {text-align:center; clear:both; padding-left:3.0pt; padding-right:3.0pt; text-indent:0pt; border:solid black 0.4pt; }
div.minipage{width:100%;}
div.center, div.center div.center {text-align: center; margin-left:1em; margin-right:1em;}
div.center div {text-align: left;}
div.flushright, div.flushright div.flushright {text-align: right;}
div.flushright div {text-align: left;}
div.flushleft {text-align: left;}
.underline{ text-decoration:underline; }
.underline img{ border-bottom: 1px solid black; margin-bottom:1pt; }
.framebox-c, .framebox-l, .framebox-r { padding-left:3.0pt; padding-right:3.0pt; text-indent:0pt; border:solid black 0.4pt; }
.framebox-c {text-align:center;}
.framebox-l {text-align:left;}
.framebox-r {text-align:right;}
span.thank-mark{ vertical-align: super }
span.footnote-mark sup.textsuperscript, span.footnote-mark a sup.textsuperscript{ font-size:80%; }
div.tabular, div.center div.tabular {text-align: center; margin-top:0.5em; margin-bottom:0.5em; }
table.tabular td p{margin-top:0em;}
table.tabular {margin-left: auto; margin-right: auto;}
div.td00{ margin-left:0pt; margin-right:0pt; }
div.td01{ margin-left:0pt; margin-right:5pt; }
div.td10{ margin-left:5pt; margin-right:0pt; }
div.td11{ margin-left:5pt; margin-right:5pt; }
table[rules] {border-left:solid black 0.4pt; border-right:solid black 0.4pt; }
td.td00{ padding-left:0pt; padding-right:0pt; }
td.td01{ padding-left:0pt; padding-right:5pt; }
td.td10{ padding-left:5pt; padding-right:0pt; }
td.td11{ padding-left:5pt; padding-right:5pt; }
table[rules] {border-left:solid black 0.4pt; border-right:solid black 0.4pt; }
.hline hr, .cline hr{ height : 1px; margin:0px; }
.tabbing-right {text-align:right;}
span.TEX {letter-spacing: -0.125em; }
span.TEX span.E{ position:relative;top:0.5ex;left:-0.0417em;}
a span.TEX span.E {text-decoration: none; }
span.LATEX span.A{ position:relative; top:-0.5ex; left:-0.4em; font-size:85%;}
span.LATEX span.TEX{ position:relative; left: -0.4em; }
div.float img, div.float .caption {text-align:center;}
div.figure img, div.figure .caption {text-align:center;}
.marginpar {width:20%; float:right; text-align:left; margin-left:auto; margin-top:0.5em; font-size:85%; text-decoration:underline;}
.marginpar p{margin-top:0.4em; margin-bottom:0.4em;}
.equation td{text-align:center; vertical-align:middle; }
td.eq-no{ width:5%; }
table.equation { width:100%; } 
div.math-display, div.par-math-display{text-align:center;}
math .texttt { font-family: monospace; }
math .textit { font-style: italic; }
math .textsl { font-style: oblique; }
math .textsf { font-family: sans-serif; }
math .textbf { font-weight: bold; }
.partToc a, .partToc, .likepartToc a, .likepartToc {line-height: 200%; font-weight:bold; font-size:110%;}
.chapterToc a, .chapterToc, .likechapterToc a, .likechapterToc, .appendixToc a, .appendixToc {line-height: 200%; font-weight:bold;}
.index-item, .index-subitem, .index-subsubitem {display:block}
.caption td.id{font-weight: bold; white-space: nowrap; }
table.caption {text-align:center;}
h1.partHead{text-align: center}
p.bibitem { text-indent: -2em; margin-left: 2em; margin-top:0.6em; margin-bottom:0.6em; }
p.bibitem-p { text-indent: 0em; margin-left: 2em; margin-top:0.6em; margin-bottom:0.6em; }
.paragraphHead, .likeparagraphHead { margin-top:2em; font-weight: bold;}
.subparagraphHead, .likesubparagraphHead { font-weight: bold;}
.quote {margin-bottom:0.25em; margin-top:0.25em; margin-left:1em; margin-right:1em; text-align:justify;}
.verse{white-space:nowrap; margin-left:2em}
div.maketitle {text-align:center;}
h2.titleHead{text-align:center;}
div.maketitle{ margin-bottom: 2em; }
div.author, div.date {text-align:center;}
div.thanks{text-align:left; margin-left:10%; font-size:85%; font-style:italic; }
div.author{white-space: nowrap;}
.quotation {margin-bottom:0.25em; margin-top:0.25em; margin-left:1em; }
h1.partHead{text-align: center}
.sectionToc, .likesectionToc {margin-left:2em;}
.subsectionToc, .likesubsectionToc {margin-left:4em;}
.subsubsectionToc, .likesubsubsectionToc {margin-left:6em;}
.frenchb-nbsp{font-size:75%;}
.frenchb-thinspace{font-size:75%;}
.figure img.graphics {margin-left:10%;}
/* end css.sty */

\title{Serie de Fourier d'une fonction}
\author{}
\date{}

\begin{document}
\maketitle

\textbf{Warning: \href{http://www.math.union.edu/locate/jsMath}{jsMath}
requires JavaScript to process the mathematics on this page.\\ If your
browser supports JavaScript, be sure it is enabled.}

\begin{center}\rule{3in}{0.4pt}\end{center}

{[}\href{coursse80.html}{next}{]} {[}\href{coursse78.html}{prev}{]}
{[}\href{coursse78.html\#tailcoursse78.html}{prev-tail}{]}
{[}\hyperref[tailcoursse79.html]{tail}{]}
{[}\href{coursch15.html\#coursse79.html}{up}{]}

\subsubsection{14.3 Série de Fourier d'une fonction}

\paragraph{14.3.1 Les espaces C et D}

Définition~14.3.1 On considère l'espace vectoriel C des fonctions de ℝ
dans ℂ, continues par morceaux et périodiques de période 2π. On
désignera par D le sous-espace vectoriel des applications f : ℝ → ℂ,
continues par morceaux, périodiques de période 2π et vérifiant
\textbackslash{}mathop\{∀\}x ∈ ℝ, f(x) =\{
f(\{x\}\^{}\{+\})+f(\{x\}\^{}\{−\}) \textbackslash{}over 2\} (où
f(\{x\}\^{}\{+\}) et f(\{x\}\^{}\{−\}) désignent respectivement les
limites à gauche et à droite de f au point x). Pour f,g ∈C, on posera
(f\textbackslash{}mathrel\{∣\}g) =\{ 1 \textbackslash{}over 2π\}
\{\textbackslash{}mathop\{∫ \}
\}\_\{0\}\^{}\{2π\}\textbackslash{}overline\{f(t)\}g(t) dt,
\textbackslash{}\textbar{}\{f\textbackslash{}\textbar{}\}\_\{2\} =
\textbackslash{}sqrt\{(f\textbackslash{}mathrel\{∣\} f)\} et
\{e\}\_\{n\} : t\textbackslash{}mathrel\{↦\}\{e\}\^{}\{int\}.

Théorème~14.3.1 L'application
(f,g)\textbackslash{}mathrel\{↦\}(f\textbackslash{}mathrel\{∣\}g) est
une forme hermitienne positive sur C dont la restriction à D est définie
positive. La famille \{(\{e\}\_\{n\})\}\_\{n∈ℤ\} est une famille
orthonormée de C. Pour toute f ∈C, on a
\textbackslash{}\textbar{}\{f\textbackslash{}\textbar{}\}\_\{2\}
≤\textbackslash{}\textbar{} \{f\textbackslash{}\textbar{}\}\_\{∞\}
(norme de la convergence uniforme).

Démonstration Le caractère sesquilinéaire et la symétrie hermitienne
sont évidents. Si f ∈C, on a (f\textbackslash{}mathrel\{∣\}f) =\{ 1
\textbackslash{}over 2π\} \{\textbackslash{}mathop\{∫ \}
\}\_\{0\}\^{}\{2π\}\textbar{}f(t)\{\textbar{}\}\^{}\{2\} dt ≥ 0. La
nullité de (f\textbackslash{}mathrel\{∣\}f) nécessite que f soit nulle
en tout point de {[}0,2π{]} où elle est continue, soit sur {[}0,2π{]}
privé d'un nombre fini de points. Si f est dans D, alors en chacun de
ces points on a f(\{x\}\^{}\{+\}) = f(\{x\}\^{}\{−\}) = 0 (car il existe
tout un intervalle ouvert à gauche de x sur lequel f est nul, et de même
à droite) et donc f(x) = 0, par conséquent f est la fonction nulle sur
{[}0,2π{]}, donc sur ℝ.

Remarque~14.3.1 On prendra garde que si f est seulement continue par
morceaux,
\textbackslash{}\textbar{}\{f\textbackslash{}\textbar{}\}\_\{2\} = 0
n'implique pas f = 0.

\paragraph{14.3.2 Coefficients de Fourier d'une fonction continue par
morceaux}

Définition~14.3.2 Soit f : ℝ → ℂ continue par morceaux et périodique de
période 2π. On définit les coefficients de Fourier de la fonction f par

\textbackslash{}begin\{eqnarray*\} \textbackslash{}mathop\{∀\}n ∈
ℤ,\textbackslash{}quad \{c\}\_\{n\}(f)\& =\&
(\{e\}\_\{n\}\textbackslash{}mathrel\{∣\}f) =\{ 1 \textbackslash{}over
2π\} \{\textbackslash{}mathop\{∫ \}
\}\_\{0\}\^{}\{2π\}f(t)\{e\}\^{}\{−int\} dt\%\&
\textbackslash{}\textbackslash{} \textbackslash{}mathop\{∀\}n ≥
0,\textbackslash{}quad \{a\}\_\{n\}(f)\& =\&\{ 1 \textbackslash{}over
π\} \{\textbackslash{}mathop\{∫ \}
\}\_\{0\}\^{}\{2π\}f(t)\textbackslash{}mathop\{cos\} nt dt \%\&
\textbackslash{}\textbackslash{} \textbackslash{}mathop\{∀\}n ≥
1,\textbackslash{}quad \{b\}\_\{n\}(f)\& =\&\{ 1 \textbackslash{}over
π\} \{\textbackslash{}mathop\{∫ \}
\}\_\{0\}\^{}\{2π\}f(t)\textbackslash{}mathop\{sin\} nt dt \%\&
\textbackslash{}\textbackslash{} \textbackslash{}end\{eqnarray*\}

Remarque~14.3.2 Les fonctions intégrées étant périodiques de période 2π,
on a aussi pour tout a ∈ ℝ, \{c\}\_\{n\}(f) =\{ 1 \textbackslash{}over
2π\} \{\textbackslash{}mathop\{∫ \}
\}\_\{a\}\^{}\{a+2π\}f(t)\{e\}\^{}\{−int\} dt, \{a\}\_\{n\}(f) =\{ 1
\textbackslash{}over π\} \{\textbackslash{}mathop\{∫ \}
\}\_\{a\}\^{}\{a+2π\}f(t)\textbackslash{}mathop\{cos\} nt dt,
\{b\}\_\{n\}(f) =\{ 1 \textbackslash{}over π\}
\{\textbackslash{}mathop\{∫ \}
\}\_\{a\}\^{}\{a+2π\}f(t)\textbackslash{}mathop\{sin\} nt dt et en
particulier \{c\}\_\{n\}(f) =\{ 1 \textbackslash{}over 2π\}
\{\textbackslash{}mathop\{∫ \} \}\_\{−π\}\^{}\{π\}f(t)\{e\}\^{}\{−int\}
dt, \{a\}\_\{n\}(f) =\{ 1 \textbackslash{}over π\}
\{\textbackslash{}mathop\{∫ \}
\}\_\{−π\}\^{}\{π\}f(t)\textbackslash{}mathop\{cos\} nt dt,
\{b\}\_\{n\}(f) =\{ 1 \textbackslash{}over π\}
\{\textbackslash{}mathop\{∫ \}
\}\_\{−π\}\^{}\{π\}f(t)\textbackslash{}mathop\{sin\} nt dt

Proposition~14.3.2 On a les relations suivantes

\textbackslash{}begin\{eqnarray*\} \{c\}\_\{0\}(f)\& =\&\{
\{a\}\_\{0\}(f) \textbackslash{}over 2\} \%\&
\textbackslash{}\textbackslash{} \textbackslash{}mathop\{∀\}n ≥
1,\textbackslash{}quad \{c\}\_\{n\}(f)\& =\&\{ \{a\}\_\{n\}(f) −
i\{b\}\_\{n\}(f) \textbackslash{}over 2\} ,\textbackslash{}quad
\{c\}\_\{−n\}(f) =\{ \{a\}\_\{n\}(f) + i\{b\}\_\{n\}(f)
\textbackslash{}over 2\} \%\& \textbackslash{}\textbackslash{}
\textbackslash{}mathop\{∀\}n ≥ 1,\textbackslash{}quad \{a\}\_\{n\}(f)\&
=\& \{c\}\_\{n\}(f) + \{c\}\_\{−n\}(f),\textbackslash{}quad \{b\}\_\{n\}
= i(\{c\}\_\{n\}(f) − \{c\}\_\{−n\}(f)) \%\&
\textbackslash{}\textbackslash{} \textbackslash{}end\{eqnarray*\}

Démonstration Elémentaire

Proposition~14.3.3 Soit f : ℝ → ℂ continue par morceaux et périodique de
période 2π. Si f est à valeurs réelles, on a \{a\}\_\{n\}(f) ∈ ℝ,
\{b\}\_\{n\}(f) ∈ ℝ et \{c\}\_\{−n\}(f) =
\textbackslash{}overline\{\{c\}\_\{n\}(f)\}. Si f est paire (resp.
impaire) on a \{b\}\_\{n\}(f) = 0 (resp. \{a\}\_\{n\}(f) = 0).

Démonstration Si f est à valeurs réelles, il en est de même de
x\textbackslash{}mathrel\{↦\}f(x)\textbackslash{}mathop\{cos\} nx et de
x\textbackslash{}mathrel\{↦\}f(x)\textbackslash{}mathop\{sin\} nx ce qui
montre que \{a\}\_\{n\}(f) et \{b\}\_\{n\}(f) sont réels~; de plus
f(x)\{e\}\^{}\{inx\} = \textbackslash{}overline\{f(x)\{e\}\^{}\{−inx\}\}
ce qui montre que \{c\}\_\{−n\}(f) =
\textbackslash{}overline\{\{c\}\_\{n\}(f)\}. Si f est paire, on a
\{b\}\_\{n\}(f) =\{ 1 \textbackslash{}over 2π\}
\{\textbackslash{}mathop\{∫ \}
\}\_\{−π\}\^{}\{π\}f(x)\textbackslash{}mathop\{sin\} nx dx = 0 puisque
la fonction f(x)\textbackslash{}mathop\{sin\} nx est impaire. Le
raisonnement est similaire si f est impaire avec les \{a\}\_\{n\}(f).

Définition~14.3.3 Soit f : ℝ → ℂ continue par morceaux et périodique de
période 2π. On appelle série de Fourier de la fonction f la série
trigonométrique

\textbackslash{}begin\{eqnarray*\}\{ c\}\_\{0\}(f) +\{
\textbackslash{}mathop\{∑ \}\}\_\{n≥1\}(\{c\}\_\{n\}(f)\{e\}\^{}\{inx\}
+ \{c\}\_\{ −n\}(f)\{e\}\^{}\{−inx\})\&\& \%\&
\textbackslash{}\textbackslash{} \& \& =\{ \{a\}\_\{0\}(f)
\textbackslash{}over 2\} +\{ \textbackslash{}mathop\{∑
\}\}\_\{n≥1\}(\{a\}\_\{n\}(f)\textbackslash{}cos nx +
\{b\}\_\{n\}(f)\textbackslash{}sin nx)\%\&
\textbackslash{}\textbackslash{} \textbackslash{}end\{eqnarray*\}

Définition~14.3.4 Pour n ≥ 1, on posera (sommes partielles de la série
de Fourier)

\textbackslash{}begin\{eqnarray*\}\{ S\}\_\{n\}(f)(x)\& =\&
\{c\}\_\{0\}(f) +\{ \textbackslash{}mathop\{∑
\}\}\_\{p=1\}\^{}\{n\}(\{c\}\_\{ p\}(f)\{e\}\^{}\{ipx\} + \{c\}\_\{
−p\}(f)\{e\}\^{}\{−ipx\}) \%\& \textbackslash{}\textbackslash{} \& =\&\{
\{a\}\_\{0\}(f) \textbackslash{}over 2\} +\{ \textbackslash{}mathop\{∑
\}\}\_\{p=1\}\^{}\{n\}(\{a\}\_\{ p\}(f)\textbackslash{}cos px +
\{b\}\_\{p\}(f)\textbackslash{}sin px)\%\&
\textbackslash{}\textbackslash{} \textbackslash{}end\{eqnarray*\}

\paragraph{14.3.3 Inégalité de Bessel et théorème de Riemann-Lebesgue}

Définition~14.3.5 Pour N ≥ 1, on posera \{T\}\_\{N\}
=\textbackslash{}mathop\{
\textbackslash{}mathrm\{Vect\}\}(\{e\}\_\{−N\},\{e\}\_\{−N+1\},\textbackslash{}mathop\{\textbackslash{}mathop\{\ldots{}\}\},\{e\}\_\{−1\},\{e\}\_\{0\},\{e\}\_\{1\},\textbackslash{}mathop\{\textbackslash{}mathop\{\ldots{}\}\},\{e\}\_\{N−1\},\{e\}\_\{N\})
(espace vectoriel des polynômes trigonométriques de degré inférieur ou
égal à N.

Remarque~14.3.3 On a également

\{T\}\_\{N\} = \textbackslash{}\{x\textbackslash{}mathrel\{↦\}\{
\{a\}\_\{0\} \textbackslash{}over 2\} +\{ \textbackslash{}mathop\{∑
\}\}\_\{p=1\}\^{}\{N\}(\{a\}\_\{ p\} \textbackslash{}cos px +
\{b\}\_\{p\} \textbackslash{}sin px)\textbackslash{}\}

Par définition même
(\{e\}\_\{−N\},\{e\}\_\{−N+1\},\textbackslash{}mathop\{\textbackslash{}mathop\{\ldots{}\}\},\{e\}\_\{−1\},\{e\}\_\{0\},\{e\}\_\{1\},\textbackslash{}mathop\{\textbackslash{}mathop\{\ldots{}\}\},\{e\}\_\{N−1\},\{e\}\_\{N\})
est une base orthonormée de \{T\}\_\{N\}.

Lemme~14.3.4 Soit f : ℝ → ℂ continue par morceaux et périodique de
période 2π. Alors
\textbackslash{}\textbar{}\{S\}\_\{N\}\{(f)\textbackslash{}\textbar{}\}\_\{2\}\^{}\{2\}
=\{\textbackslash{}mathop\{ \textbackslash{}mathop\{∑ \}\}
\}\_\{k=−N\}\^{}\{N\}\textbar{}\{c\}\_\{k\}(f)\{\textbar{}\}\^{}\{2\}.

Démonstration
\{c\}\_\{−N\}(f),\textbackslash{}mathop\{\textbackslash{}mathop\{\ldots{}\}\},\{c\}\_\{0\}(f),\textbackslash{}mathop\{\textbackslash{}mathop\{\ldots{}\}\},\{c\}\_\{N\}(f)
sont les coordonnées de \{S\}\_\{N\}(f) dans la base orthonormée
(\{e\}\_\{−N\},\{e\}\_\{−N+1\},\textbackslash{}mathop\{\textbackslash{}mathop\{\ldots{}\}\},\{e\}\_\{−1\},\{e\}\_\{0\},\{e\}\_\{1\},\textbackslash{}mathop\{\textbackslash{}mathop\{\ldots{}\}\},\{e\}\_\{N−1\},\{e\}\_\{N\})~;
la norme au carré de \{S\}\_\{N\}(f) est donc la somme des carrés des
modules de ces coordonnées~; d'où le résultat.

Lemme~14.3.5 Soit f : ℝ → ℂ continue par morceaux et périodique de
période 2π. Alors \{S\}\_\{N\}(f) est la projection orthogonale de f sur
le sous-espace vectoriel \{T\}\_\{N\}.

Démonstration Puisque \{S\}\_\{N\}(f) appartient à \{T\}\_\{N\}, il
suffit de montrer que f − \{S\}\_\{N\}(f) ⊥ \{T\}\_\{N\} ou encore que
\textbackslash{}mathop\{∀\}n ∈ {[}−N,N{]},
(\{e\}\_\{n\}\textbackslash{}mathrel\{∣\}f − \{S\}\_\{N\}(f)) = 0, ou
encore que \textbackslash{}mathop\{∀\}n ∈ {[}−N,N{]},
(\{e\}\_\{n\}\textbackslash{}mathrel\{∣\}f) =
(\{e\}\_\{n\}\textbackslash{}mathrel\{∣\}\{S\}\_\{N\}(f)). Mais
(\{e\}\_\{n\}\textbackslash{}mathrel\{∣\}\{S\}\_\{N\}(f)) est la
coordonnée suivant \{e\}\_\{n\} de \{S\}\_\{N\}(f) (puisque la base est
orthonormée), c'est donc \{c\}\_\{n\}(f) =
(\{e\}\_\{n\}\textbackslash{}mathrel\{∣\}f) par définition, ce qui
montre le résultat.

Théorème~14.3.6 (Bessel). Soit f : ℝ → ℂ continue par morceaux et
périodique de période 2π. Alors la série
\textbar{}\{c\}\_\{0\}(f)\{\textbar{}\}\^{}\{2\}
+\{\textbackslash{}mathop\{ \textbackslash{}mathop\{∑ \}\}
\}\_\{n≥1\}(\textbar{}\{c\}\_\{n\}(f)\{\textbar{}\}\^{}\{2\} +
\textbar{}\{c\}\_\{−n\}(f)\{\textbar{}\}\^{}\{2\}) est convergente et on
a

\textbar{}\{c\}\_\{0\}(f)\{\textbar{}\}\^{}\{2\} +\{
\textbackslash{}mathop\{∑ \}\}\_\{n=1\}\^{}\{+∞\}(\textbar{}\{c\}\_\{
n\}(f)\{\textbar{}\}\^{}\{2\} + \textbar{}\{c\}\_\{
−n\}(f)\{\textbar{}\}\^{}\{2\}) ≤\textbackslash{}\textbar{}
\{f\textbackslash{}\textbar{}\}\_\{ 2\}\^{}\{2\}

Démonstration Puisque \{S\}\_\{N\}(f) est la projection orthogonale de f
sur \{T\}\_\{N\}, on a f = \{S\}\_\{N\}(f) + (f − \{S\}\_\{N\}(f)) avec
\{S\}\_\{N\}(f) ⊥ f − \{S\}\_\{N\}(f). Le théorème de Pythagore assure
que
\textbackslash{}\textbar{}\{f\textbackslash{}\textbar{}\}\_\{2\}\^{}\{2\}
=\textbackslash{}\textbar{}
\{S\}\_\{N\}\{(f)\textbackslash{}\textbar{}\}\_\{2\}\^{}\{2\}
+\textbackslash{}\textbar{} f −
\{S\}\_\{N\}\{(f)\textbackslash{}\textbar{}\}\_\{2\}\^{}\{2\}, d'où
encore d'après le lemme 1

\textbar{}\{c\}\_\{0\}(f)\{\textbar{}\}\^{}\{2\} +\{
\textbackslash{}mathop\{∑ \}\}\_\{n=1\}\^{}\{N\}(\textbar{}\{c\}\_\{
n\}(f)\{\textbar{}\}\^{}\{2\} + \textbar{}\{c\}\_\{
−n\}(f)\{\textbar{}\}\^{}\{2\}) =\textbackslash{}\textbar{} \{S\}\_\{
N\}\{(f)\textbackslash{}\textbar{}\}\_\{2\}\^{}\{2\}
≤\textbackslash{}\textbar{} \{f\textbackslash{}\textbar{}\}\_\{
2\}\^{}\{2\}

La série à termes positifs
\textbar{}\{c\}\_\{0\}(f)\{\textbar{}\}\^{}\{2\}
+\{\textbackslash{}mathop\{ \textbackslash{}mathop\{∑ \}\}
\}\_\{n≥1\}(\textbar{}\{c\}\_\{n\}(f)\{\textbar{}\}\^{}\{2\} +
\textbar{}\{c\}\_\{−n\}(f)\{\textbar{}\}\^{}\{2\}) a ses sommes
partielles majorées par
\textbackslash{}\textbar{}\{f\textbackslash{}\textbar{}\}\_\{2\}\^{}\{2\},
donc elle converge et sa somme est majorée par
\textbackslash{}\textbar{}\{f\textbackslash{}\textbar{}\}\_\{2\}\^{}\{2\},
ce qui achève la démonstration.

Remarque~14.3.4 Un calcul élémentaire montre que pour n ≥ 1,

\textbar{}\{c\}\_\{n\}(f)\{\textbar{}\}\^{}\{2\} + \textbar{}\{c\}\_\{
−n\}(f)\{\textbar{}\}\^{}\{2\} =\{ 1 \textbackslash{}over 2\}
(\textbar{}\{a\}\_\{n\}(f)\{\textbar{}\}\^{}\{2\} + \textbar{}\{b\}\_\{
n\}(f)\{\textbar{}\}\^{}\{2\})

ce qui montre que les séries
\textbackslash{}mathop\{\textbackslash{}mathop\{∑ \}\}
\textbar{}\{a\}\_\{n\}(f)\{\textbar{}\}\^{}\{2\} et
\textbackslash{}mathop\{\textbackslash{}mathop\{∑ \}\}
\textbar{}\{b\}\_\{n\}(f)\{\textbar{}\}\^{}\{2\} convergent et que (en
tenant compte de \{a\}\_\{0\}(f) =\{ \{c\}\_\{0\}(f)
\textbackslash{}over 2\} )

\{ \textbar{}\{a\}\_\{0\}(f)\{\textbar{}\}\^{}\{2\} \textbackslash{}over
4\} +\{ 1 \textbackslash{}over 2\} \{\textbackslash{}mathop\{∑
\}\}\_\{n=1\}\^{}\{+∞\}(\textbar{}\{a\}\_\{
n\}(f)\{\textbar{}\}\^{}\{2\} + \textbar{}\{b\}\_\{
n\}(f)\{\textbar{}\}\^{}\{2\}) ≤\textbackslash{}\textbar{}
\{f\textbackslash{}\textbar{}\}\^{}\{2\}

Théorème~14.3.7 (Riemann-Lebesgue). Soit f : ℝ → ℂ continue par morceaux
et périodique de période 2π. Alors

\{\textbackslash{}mathop\{lim\}\}\_\{n→±∞\}\{c\}\_\{n\}(f)
=\{\textbackslash{}mathop\{ lim\}\}\_\{n→+∞\}\{a\}\_\{n\}(f)
=\{\textbackslash{}mathop\{ lim\}\}\_\{n→+∞\}\{b\}\_\{n\}(f) = 0

Démonstration Puisque les séries
\{\textbackslash{}mathop\{\textbackslash{}mathop\{∑ \}\}
\}\_\{n≥1\}(\textbar{}\{c\}\_\{n\}(f)\{\textbar{}\}\^{}\{2\} +
\textbar{}\{c\}\_\{−n\}(f)\{\textbar{}\}\^{}\{2\}),
\textbackslash{}mathop\{\textbackslash{}mathop\{∑ \}\}
\textbar{}\{a\}\_\{n\}(f)\{\textbar{}\}\^{}\{2\} et
\textbackslash{}mathop\{\textbackslash{}mathop\{∑ \}\}
\textbar{}\{b\}\_\{n\}(f)\{\textbar{}\}\^{}\{2\} sont convergentes,
leurs termes généraux admettent la limite 0, ce qui montre le résultat.

\paragraph{14.3.4 Les théorèmes de Dirichlet}

Nous aurons besoin par la suite du lemme suivant

Lemme~14.3.8 Pour tout entier n ≥ 1 et pour
t\textbackslash{}mathrel\{∉\}2πℤ,
\{\textbackslash{}mathop\{\textbackslash{}mathop\{∑ \}\}
\}\_\{k=−n\}\^{}\{n\}\{e\}\^{}\{ikt\} =\{ \textbackslash{}mathop\{sin\}
(2n+1)\{ t \textbackslash{}over 2\} \textbackslash{}over
\textbackslash{}mathop\{sin\} \{ t \textbackslash{}over 2\} \} .

Démonstration On a en effet

\textbackslash{}begin\{eqnarray*\} \{\textbackslash{}mathop\{∑
\}\}\_\{k=−n\}\^{}\{n\}\{e\}\^{}\{ikt\}\& =\& \{e\}\^{}\{−int\}\{
\textbackslash{}mathop\{∑ \}\}\_\{k=0\}\^{}\{2n\}\{e\}\^{}\{ikt\} =
\{e\}\^{}\{−int\}\{ \{e\}\^{}\{(2n+1)it\} − 1 \textbackslash{}over
\{e\}\^{}\{it\} − 1\} \%\& \textbackslash{}\textbackslash{} \& =\&\{
\{e\}\^{}\{(n+1)it\} − \{e\}\^{}\{−int\} \textbackslash{}over
\{e\}\^{}\{it\} − 1\} =\{ \{e\}\^{}\{(n+\{ 1 \textbackslash{}over 2\}
)it\} − \{e\}\^{}\{−(n+\{ 1 \textbackslash{}over 2\} )it\}
\textbackslash{}over \{e\}\^{}\{i\{ t \textbackslash{}over 2\} \} −
\{e\}\^{}\{−i\{ t \textbackslash{}over 2\} \}\} \%\&
\textbackslash{}\textbackslash{} \textbackslash{}end\{eqnarray*\}

en multipliant numérateur et dénominateur par \{e\}\^{}\{−it∕2\}. On en
déduit immédiatement la formule souhaitée.

Théorème~14.3.9 (Dirichlet). Soit f : ℝ → ℂ de classe \{C\}\^{}\{1\} par
morceaux et périodique de période 2π. Alors la série de Fourier de f
converge sur ℝ et

\textbackslash{}mathop\{∀\}x ∈ ℝ,\{ f(\{x\}\^{}\{+\}) +
f(\{x\}\^{}\{−\}) \textbackslash{}over 2\} = \{c\}\_\{0\}(f) +\{
\textbackslash{}mathop\{∑ \}\}\_\{n=1\}\^{}\{+∞\}(\{c\}\_\{
n\}(f)\{e\}\^{}\{inx\} + \{c\}\_\{ −n\}(f)\{e\}\^{}\{−inx\})

Démonstration On a

\textbackslash{}begin\{eqnarray*\}\{ S\}\_\{n\}(f)(x)\& =\&\{ 1
\textbackslash{}over 2π\} \{\textbackslash{}mathop\{∑
\}\}\_\{k=−n\}\^{}\{n\}\{e\}\^{}\{inx\}\{
\textbackslash{}mathop\{\textbackslash{}mathop\{∫ \} \}
\}\_\{0\}\^{}\{2π\}f(t)\{e\}\^{}\{−int\} dt\%\&
\textbackslash{}\textbackslash{} \& =\&\{ 1 \textbackslash{}over 2π\}
\{\textbackslash{}mathop\{∫ \}
\}\_\{0\}\^{}\{2π\}f(t)\textbackslash{}left (\{\textbackslash{}mathop\{∑
\}\}\_\{k=−n\}\^{}\{n\}\{e\}\^{}\{in(x−t)\}\textbackslash{}right )
dt\%\& \textbackslash{}\textbackslash{} \& =\&\{ 1 \textbackslash{}over
2π\} \{\textbackslash{}mathop\{∫ \} \}\_\{0\}\^{}\{2π\}f(t)\{
\textbackslash{}mathop\{sin\} (2n + 1)\{ x−t \textbackslash{}over 2\}
\textbackslash{}over \textbackslash{}mathop\{sin\} \{ x−t
\textbackslash{}over 2\} \} dt \%\& \textbackslash{}\textbackslash{}
\textbackslash{}end\{eqnarray*\}

Faisons le changement de variable t = x + u, on obtient

\textbackslash{}begin\{eqnarray*\}\{ S\}\_\{n\}(f)(x)\& =\&\{ 1
\textbackslash{}over 2π\} \{\textbackslash{}mathop\{∫ \}
\}\_\{−x\}\^{}\{2π−x\}f(x + u)\{ \textbackslash{}mathop\{sin\} (2n +
1)\{ u \textbackslash{}over 2\} \textbackslash{}over
\textbackslash{}mathop\{sin\} \{ u \textbackslash{}over 2\} \} du\%\&
\textbackslash{}\textbackslash{} \& =\&\{ 1 \textbackslash{}over 2π\}
\{\textbackslash{}mathop\{∫ \} \}\_\{−π\}\^{}\{π\}f(x + u)\{
\textbackslash{}mathop\{sin\} (2n + 1)\{ u \textbackslash{}over 2\}
\textbackslash{}over \textbackslash{}mathop\{sin\} \{ u
\textbackslash{}over 2\} \} du \%\& \textbackslash{}\textbackslash{}
\textbackslash{}end\{eqnarray*\}

puisque la fonction intégrée est périodique de période 2π et que donc
son intégrale sur tout intervalle de longueur 2π est la même. Coupons
l'intégrale en deux, l'une de − π à 0, l'autre de 0 à π. Dans la
première faisons le changement de variable u = −2v et dans la seconde le
changement de variable u = 2v. On obtient

\textbackslash{}begin\{eqnarray*\}\{ S\}\_\{n\}(f)(x)\& =\&\{ 1
\textbackslash{}over π\} \{\textbackslash{}mathop\{∫ \}
\}\_\{0\}\^{}\{π∕2\}f(x − 2v)\{ \textbackslash{}mathop\{sin\} (2n + 1)v
\textbackslash{}over \textbackslash{}mathop\{sin\} v\} dv \%\&
\textbackslash{}\textbackslash{} \& \textbackslash{}text\{\} \& +\{ 1
\textbackslash{}over π\} \{\textbackslash{}mathop\{∫ \}
\}\_\{0\}\^{}\{π∕2\}f(x + 2v)\{ \textbackslash{}mathop\{sin\} (2n + 1)v
\textbackslash{}over \textbackslash{}mathop\{sin\} v\} dv \%\&
\textbackslash{}\textbackslash{} \& =\&\{ 1 \textbackslash{}over π\}
\{\textbackslash{}mathop\{∫ \} \}\_\{0\}\^{}\{π∕2\}(f(x + 2v) + f(x −
2v))\{ \textbackslash{}mathop\{sin\} (2n + 1)v \textbackslash{}over
\textbackslash{}mathop\{sin\} v\} dv\%\&
\textbackslash{}\textbackslash{} \textbackslash{}end\{eqnarray*\}

Appliquons le résultat précédent à la fonction constante \{f\}\_\{0\} :
x\textbackslash{}mathrel\{↦\}1. On a bien entendu
\{S\}\_\{n\}(\{f\}\_\{0\})(x) = 1 puisque \{c\}\_\{0\}(\{f\}\_\{0\}) = 1
et \{c\}\_\{n\}(\{f\}\_\{0\}) = 0 pour n\textbackslash{}mathrel\{≠\}0~;
on obtient

1 =\{ 2 \textbackslash{}over π\} \{\textbackslash{}mathop\{∫ \}
\}\_\{0\}\^{}\{π∕2\}\{ \textbackslash{}mathop\{sin\} (2n + 1)v
\textbackslash{}over \textbackslash{}mathop\{sin\} v\} dv

On en déduit que

\textbackslash{}begin\{eqnarray*\}\{ S\}\_\{n\}(f)(x) −\{
f(\{x\}\^{}\{+\}) + f(\{x\}\^{}\{−\}) \textbackslash{}over 2\} =\&\&
\%\& \textbackslash{}\textbackslash{} \& \&\{ 1 \textbackslash{}over π\}
\{\textbackslash{}mathop\{∫ \} \}\_\{0\}\^{}\{π∕2\}\{ f(x + 2v) −
f(\{x\}\^{}\{+\}) + f(x − 2v) − f(\{x\}\_\{ −\}) \textbackslash{}over
\textbackslash{}mathop\{sin\} v\} \textbackslash{}mathop\{sin\} (2n +
1)v dv\%\& \textbackslash{}\textbackslash{}
\textbackslash{}end\{eqnarray*\}

Considérons la fonction g périodique de période 2π définie par

\textbackslash{}begin\{eqnarray*\} g(v)\& =\&\{ f(x + 2v) −
f(\{x\}\^{}\{+\}) + f(x − 2v) − f(\{x\}\_\{−\}) \textbackslash{}over
\textbackslash{}mathop\{sin\} v\} \textbackslash{}text\{ pour \}v
∈{]}0,\{ π \textbackslash{}over 2\} {]}\%\&
\textbackslash{}\textbackslash{} g(0)\& =\& 2(f'(\{x\}\^{}\{+\}) −
f'(\{x\}\^{}\{−\})) \%\& \textbackslash{}\textbackslash{} g(v)\& =\&
0\textbackslash{}text\{ pour \}v ∈{]}\{ π \textbackslash{}over 2\}
,2π{[} \%\& \textbackslash{}\textbackslash{}
\textbackslash{}end\{eqnarray*\}

Comme la fonction \textbackslash{}tilde\{f\} définie par
\textbackslash{}tilde\{f\}(x) = f(\{x\}\^{}\{+\}) et
\textbackslash{}tilde\{f\}(t) = f(t) pour t \textgreater{} x est
dérivable à droite au point x (puisque f est de classe \{C\}\^{}\{1\}
par morceaux), on a, quand v tend vers 0 par valeurs supérieures,

\textbackslash{}begin\{eqnarray*\}\{ f(x + 2v) − f(\{x\}\^{}\{+\}))
\textbackslash{}over \textbackslash{}mathop\{sin\} v\} \&
\{∼\}\_\{v→0,v\textgreater{}0\}\&\{ f(x + 2v) − f(\{x\}\^{}\{+\})
\textbackslash{}over v\} \%\& \textbackslash{}\textbackslash{} \& = \&
2\{ \textbackslash{}tilde\{f\}(x + 2v) −\textbackslash{}tilde\{ f\}(x)
\textbackslash{}over 2v\} \%\& \textbackslash{}\textbackslash{}
\textbackslash{}end\{eqnarray*\}

de limite 2f'(\{x\}\^{}\{+\}). De même on a

\{\textbackslash{}mathop\{lim\}\}\_\{v→0,v\textgreater{}0\}\{ f(x − 2v)
− f(\{x\}\_\{−\}) \textbackslash{}over \textbackslash{}mathop\{sin\} v\}
= −2f'(\{x\}\^{}\{−\})

ce qui montre que g est continue à droite au point 0. On en déduit
immédiatement que g est continue par morceaux. Mais alors

\textbackslash{}begin\{eqnarray*\}\{ S\}\_\{n\}(f)(x) −\{
f(\{x\}\^{}\{+\}) + f(\{x\}\^{}\{−\}) \textbackslash{}over 2\} \&\& \%\&
\textbackslash{}\textbackslash{} \& =\&\{ 1 \textbackslash{}over π\}
\{\textbackslash{}mathop\{∫ \}
\}\_\{0\}\^{}\{2π\}g(v)\textbackslash{}mathop\{sin\} (2n + 1)v dv =
\{b\}\_\{ 2n+1\}(g)\%\& \textbackslash{}\textbackslash{}
\textbackslash{}end\{eqnarray*\}

D'après le théorème de Riemann-Lebesgue, cette expression tend vers 0
quand n tend vers + ∞, ce qui montre à la fois la convergence de la
série et donne la valeur de sa somme.

Lemme~14.3.10 Soit f : ℝ → ℂ périodique de période 2πde classe
\{C\}\^{}\{1\} par morceaux et continue. Alors
\textbackslash{}mathop\{∀\}n ∈ ℤ, \{c\}\_\{n\}(f') = in\{c\}\_\{n\}(f)
(où f' désigne la fonction de D égale à la dérivée de f sauf en un
nombre fini de points modulo 2π).

Démonstration Soit σ = \{(\{a\}\_\{i\})\}\_\{0≤i≤p\} une subdivision de
{[}0,2π{]} adaptée à f. En tout point de {[}0,2π{]}
∖\textbackslash{}\{\{a\}\_\{0\},\textbackslash{}mathop\{\textbackslash{}mathop\{\ldots{}\}\},\{a\}\_\{p\}\textbackslash{}\},
f'(t) est la dérivée de f et on pose f'(\{a\}\_\{i\}) =\{ 1
\textbackslash{}over 2\} (f'(\{a\}\_\{i\}\^{}\{+\}) + f'(\{a\}\_\{
i\}\^{}\{−\})), si bien que f' ∈D. Une intégration par parties donne, si
{[}a,b{]} ⊂{]}\{a\}\_\{i−1\},\{a\}\_\{i\}{[},

\textbackslash{}begin\{eqnarray*\} \{\textbackslash{}mathop\{∫ \}
\}\_\{a\}\^{}\{b\}f'(t)\{e\}\^{}\{−int\} dt\& =\&\{ \textbackslash{}left
{[}f(t)\{e\}\^{}\{−int\}\textbackslash{}right {]}\}\_\{ a\}\^{}\{b\} +
in\{\textbackslash{}mathop\{∫ \} \}\_\{a\}\^{}\{b\}f(t)\{e\}\^{}\{−int\}
dt \%\& \textbackslash{}\textbackslash{} \& =\& f(b)\{e\}\^{}\{−inb\} −
f(a)\{e\}\^{}\{−ina\} + in\{\textbackslash{}mathop\{∫ \}
\}\_\{a\}\^{}\{b\}f(t)\{e\}\^{}\{−int\} dt\%\&
\textbackslash{}\textbackslash{} \textbackslash{}end\{eqnarray*\}

En faisant tendre a vers \{a\}\_\{i−1\} et b vers \{a\}\_\{i\}, en
tenant compte de la continuité de f aux points \{a\}\_\{i−1\} et
\{a\}\_\{i\} on obtient

\textbackslash{}begin\{eqnarray*\} \{\textbackslash{}mathop\{∫ \}
\}\_\{\{a\}\_\{i−1\}\}\^{}\{\{a\}\_\{i\} \}f'(t)\{e\}\^{}\{−int\} dt\&
=\& f(\{a\}\_\{ i\})\{e\}\^{}\{−in\{a\}\_\{i\} \} −
f(\{a\}\_\{i−1\})\{e\}\^{}\{−in\{a\}\_\{i−1\} \}\%\&
\textbackslash{}\textbackslash{} \& \textbackslash{}text\{\} \&
+in\{\textbackslash{}mathop\{∫ \}
\}\_\{\{a\}\_\{i−1\}\}\^{}\{\{a\}\_\{i\} \}f(t)\{e\}\^{}\{−int\} dt \%\&
\textbackslash{}\textbackslash{} \textbackslash{}end\{eqnarray*\}

et en sommant

\textbackslash{}begin\{eqnarray*\} \{\textbackslash{}mathop\{∫ \}
\}\_\{0\}\^{}\{2π\}f'(t)\{e\}\^{}\{−int\} dt\&\& \%\&
\textbackslash{}\textbackslash{} \& =\& \{\textbackslash{}mathop\{∑
\}\}\_\{i=1\}\^{}\{p\}\{
\textbackslash{}mathop\{\textbackslash{}mathop\{∫ \} \}
\}\_\{\{a\}\_\{i−1\}\}\^{}\{\{a\}\_\{i\} \}f'(t)\{e\}\^{}\{−int\} dt
\%\& \textbackslash{}\textbackslash{} \& =\& \{\textbackslash{}mathop\{∑
\}\}\_\{i=1\}\^{}\{p\}\textbackslash{}left (f(\{a\}\_\{
i\})\{e\}\^{}\{−in\{a\}\_\{i\} \} −
f(\{a\}\_\{i−1\})\{e\}\^{}\{−in\{a\}\_\{i−1\} \}\textbackslash{}right )
+ in\{\textbackslash{}mathop\{\textbackslash{}mathop\{∫ \} \}
\}\_\{\{a\}\_\{i−1\}\}\^{}\{\{a\}\_\{i\} \}f(t)\{e\}\^{}\{−int\} dt\%\&
\textbackslash{}\textbackslash{} \& =\&
f(\{a\}\_\{p\})\{e\}\^{}\{−in\{a\}\_\{p\} \} −
f(\{a\}\_\{0\})\{e\}\^{}\{−in\{a\}\_\{0\} \} +
in\{\textbackslash{}mathop\{∫ \} \}\_\{\{a\}\_\{0\}\}\^{}\{\{a\}\_\{p\}
\}f(t)\{e\}\^{}\{−int\} dt \%\& \textbackslash{}\textbackslash{} \& =\&
in\{\textbackslash{}mathop\{∫ \}
\}\_\{0\}\^{}\{2π\}f(t)\{e\}\^{}\{−int\} dt \%\&
\textbackslash{}\textbackslash{} \textbackslash{}end\{eqnarray*\}

puisque \{a\}\_\{0\} = 0, \{a\}\_\{p\} = 2π,
f(\{a\}\_\{p\})\{e\}\^{}\{−in\{a\}\_\{p\}\} = f(2π)\{e\}\^{}\{−in2π\} =
f(2π) = f(0) = f(\{a\}\_\{0\})\{e\}\^{}\{−in\{a\}\_\{0\}\}. En divisant
par 2π, on obtient \{c\}\_\{n\}(f') = in\{c\}\_\{n\}(f).

Théorème~14.3.11 (Dirichlet). Soit f : ℝ → ℂ périodique de période 2π de
classe \{C\}\^{}\{1\} par morceaux et continue. Alors la série
\textbar{}\{c\}\_\{0\}(f)\textbar{} +\{\textbackslash{}mathop\{
\textbackslash{}mathop\{∑ \}\}
\}\_\{n≥1\}(\textbar{}\{c\}\_\{n\}(f)\textbar{} +
\textbar{}\{c\}\_\{−n\}(f)\textbar{}) converge, la série de Fourier de f
converge normalement sur ℝ et on a

\textbackslash{}mathop\{∀\}x ∈ ℝ, f(x) = \{c\}\_\{0\}(f) +\{
\textbackslash{}mathop\{∑ \}\}\_\{n=1\}\^{}\{+∞\}(\{c\}\_\{
n\}(f)\{e\}\^{}\{inx\} + \{c\}\_\{ −n\}(f)\{e\}\^{}\{−inx\})

(autrement dit f est somme de sa série de Fourier).

Démonstration Pour a et b réels on a ab ≤\{ 1 \textbackslash{}over 2\}
(\{a\}\^{}\{2\} + \{b\}\^{}\{2\})~; on en déduit que si
n\textbackslash{}mathrel\{≠\}0, on a 0
≤\textbar{}\{c\}\_\{n\}(f)\textbar{} = \textbackslash{}left \textbar{}\{
\{c\}\_\{n\}(f') \textbackslash{}over in\} \textbackslash{}right
\textbar{}≤\{ 1 \textbackslash{}over 2\}
(\textbar{}\{c\}\_\{n\}(f)\{\textbar{}\}\^{}\{2\} +\{ 1
\textbackslash{}over \{n\}\^{}\{2\}\} ). D'après le théorème de Bessel,
la série \{\textbackslash{}mathop\{\textbackslash{}mathop\{∑ \}\}
\}\_\{n≥0\}\textbar{}\{c\}\_\{n\}(f')\{\textbar{}\}\^{}\{2\} converge et
d'après la théorie des séries de Riemann la série
\textbackslash{}mathop\{\textbackslash{}mathop\{∑ \}\} \{ 1
\textbackslash{}over \{n\}\^{}\{2\}\} converge. On en déduit que la
série \{\textbackslash{}mathop\{\textbackslash{}mathop\{∑ \}\}
\}\_\{n≥1\}\textbar{}\{c\}\_\{n\}(f)\textbar{} converge. On montre de la
même fa\textbackslash{}c\{c\}on que la série
\{\textbackslash{}mathop\{\textbackslash{}mathop\{∑ \}\}
\}\_\{n≥1\}\textbar{}\{c\}\_\{−n\}(f)\textbar{} converge, d'où la
convergence de la série
\{\textbackslash{}mathop\{\textbackslash{}mathop\{∑ \}\}
\}\_\{n≥1\}(\textbar{}\{c\}\_\{n\}(f)\textbar{} +
\textbar{}\{c\}\_\{−n\}(f)\textbar{}). La convergence normale de la
série de Fourier en résulte immédiatement puisque

\textbackslash{}mathop\{∀\}x ∈ ℝ,
\textbar{}\{c\}\_\{n\}(f)\{e\}\^{}\{inx\} + \{c\}\_\{
−n\}(f)\{e\}\^{}\{−inx\}\textbar{}≤\textbar{}\{c\}\_\{ n\}(f)\textbar{}
+ \textbar{}\{c\}\_\{−n\}(f)\textbar{}

qui est une série convergente indépendante de x. La formule résulte du
premier théorème de Dirichlet en remarquant que si f est continue, f(x)
=\{ f(\{x\}\^{}\{+\})+f(\{x\}\^{}\{−\}) \textbackslash{}over 2\} .

\paragraph{14.3.5 Coefficients de Fourier des fonctions de classe
\{C\}\^{}\{k\}}

Théorème~14.3.12 Soit f : ℝ → ℂ périodique de période 2π de classe
\{C\}\^{}\{k\}. Alors

\textbackslash{}mathop\{∀\}n ∈ ℤ, \{c\}\_\{n\}(f) =
\{(in)\}\^{}\{k\}\{c\}\_\{ n\}(\{f\}\^{}\{(k)\})

et, quand \textbar{}n\textbar{} tend vers + ∞, \{c\}\_\{n\}(f) = o(\{ 1
\textbackslash{}over \{n\}\^{}\{k\}\} ).

Démonstration On a vu que \{c\}\_\{n\}(f') = in\{c\}\_\{n\}(f) et il
suffit de faire une récurrence évidente sur k pour obtenir
\{c\}\_\{n\}(f) = \{(in)\}\^{}\{k\}\{c\}\_\{n\}(\{f\}\^{}\{(k)\}). Comme
le théorème de Riemann-Lebesgue assure que
\{\textbackslash{}mathop\{lim\}\}\_\{\textbar{}n\textbar{}→+∞\}\{c\}\_\{n\}(\{f\}\^{}\{(k)\})
= 0, on a \{c\}\_\{n\}(f) = o(\{ 1 \textbackslash{}over \{n\}\^{}\{k\}\}
).

Remarque~14.3.5 Autrement dit, plus la fonction est régulière, plus vite
les coefficients de Fourier tendent vers 0 à l'infini. Si f est de
classe \{C\}\^{}\{∞\}, on a pour tout k ∈ ℕ,
\{\textbackslash{}mathop\{lim\}\}\_\{\textbar{}n\textbar{}→+∞\}\{n\}\^{}\{k\}\{c\}\_\{n\}(f)
= 0 (typiquement les coefficients de Fourier seront à décroissance
exponentielle).

\paragraph{14.3.6 Le théorème de Parseval}

Lemme~14.3.13 Soit f : ℝ → ℂ périodique de période 2π et continue par
morceaux. Alors, pour tout ε \textgreater{} 0, il existe g : ℝ → ℂ
périodique de période 2π, de classe \{C\}\^{}\{1\} par morceaux et
continue telle que \textbackslash{}\textbar{}f −
\{g\textbackslash{}\textbar{}\}\_\{2\} \textless{} ε.

Démonstration Supposons tout d'abord que f est en escalier et soit 0 =
\{a\}\_\{0\} \textless{} \{a\}\_\{1\} \textless{}
\textbackslash{}mathop\{\textbackslash{}mathop\{\ldots{}\}\} \textless{}
\{a\}\_\{p\} = 2π une subdivision de {[}0,2π{]} adaptée à f avec f(t) =
\{λ\}\_\{i\} pour t ∈{]}\{a\}\_\{i−1\},\{a\}\_\{i\}{[}. Soit δ le pas de
la subdivision. Pour \{ 2 \textbackslash{}over n\} \textless{} η
définissons une fonction \{g\}\_\{n\} par

\begin{itemize}
\itemsep1pt\parskip0pt\parsep0pt
\item
  (i) \textbackslash{}mathop\{∀\}i ∈ {[}0,p{]},
  \{g\}\_\{n\}(\{a\}\_\{i\}) = 0
\item
  (ii) \textbackslash{}mathop\{∀\}i ∈ {[}1,p{]},
  \textbackslash{}mathop\{∀\}t ∈ {[}\{a\}\_\{i−1\} +\{ 1
  \textbackslash{}over n\} ,\{a\}\_\{i\} −\{ 1 \textbackslash{}over n\}
  {]}, \{g\}\_\{n\}(t) = \{λ\}\_\{i\}
\item
  (iii) \{g\}\_\{n\} est affine sur chacun des intervalles
  {[}\{a\}\_\{i−1\},\{a\}\_\{i−1\} +\{ 1 \textbackslash{}over n\} {]} et
  {[}\{a\}\_\{i\} −\{ 1 \textbackslash{}over n\} ,\{a\}\_\{i\}{]}.
\end{itemize}

Il est clair que \{g\}\_\{n\} est continue, affine par morceaux. Comme
de plus \{g\}\_\{n\}(0) = \{g\}\_\{n\}(2π) = 0 elle se prolonge en une
application continue et périodique de période 2π sur ℝ. Puisque
\{g\}\_\{n\} est affine par morceaux, elle est a fortiori de classe
\{C\}\^{}\{1\} par morceaux. On a

\textbackslash{}begin\{eqnarray*\} \{\textbackslash{}mathop\{∫ \}
\}\_\{\{a\}\_\{i−1\}\}\^{}\{\{a\}\_\{i\} \}\textbar{}f(t) −
\{g\}\_\{n\}(t)\{\textbar{}\}\^{}\{2\} dt\& =\&
\{\textbackslash{}mathop\{∫ \}
\}\_\{\{a\}\_\{i−1\}\}\^{}\{\{a\}\_\{i−1\}+\{ 1 \textbackslash{}over n\}
\}\textbar{}f(t) − \{g\}\_\{n\}(t)\{\textbar{}\}\^{}\{2\} dt + \%\&
\textbackslash{}\textbackslash{} \& \textbackslash{}text\{\} \&
\{\textbackslash{}mathop\{∫ \} \}\_\{\{a\}\_\{i\}−\{ 1
\textbackslash{}over n\} \}\^{}\{\{a\}\_\{i\} \}\textbar{}f(t) −
\{g\}\_\{n\}(t)\{\textbar{}\}\^{}\{2\} dt \%\&
\textbackslash{}\textbackslash{} \textbackslash{}end\{eqnarray*\}

Mais on a g(t) = n\{λ\}\_\{i\}(t − \{a\}\_\{i\}) pour t ∈
{[}\{a\}\_\{i−1\},\{a\}\_\{i−1\} +\{ 1 \textbackslash{}over n\} {]} et
g(t) = −n\{λ\}\_\{i\}(t − \{a\}\_\{i\}) pour t ∈ {[}\{a\}\_\{i\} −\{ 1
\textbackslash{}over n\} ,\{a\}\_\{i\}{]}. On a donc

\textbackslash{}begin\{eqnarray*\} \{\textbackslash{}mathop\{∫ \}
\}\_\{\{a\}\_\{i−1\}\}\^{}\{\{a\}\_\{i\} \}\textbar{}f(t) −
\{g\}\_\{n\}(t)\{\textbar{}\}\^{}\{2\} dt\&\& \%\&
\textbackslash{}\textbackslash{} \& =\&
\textbar{}\{λ\}\_\{i\}\{\textbar{}\}\^{}\{2\}\textbackslash{}left
(\{\textbackslash{}mathop\{∫ \}
\}\_\{\{a\}\_\{i−1\}\}\^{}\{\{a\}\_\{i−1\}+\{ 1 \textbackslash{}over n\}
\}\{(1 − n(t − \{a\}\_\{i−1\}))\}\^{}\{2\} dt\textbackslash{}right .
\%\& \textbackslash{}\textbackslash{} \& \textbackslash{}text\{\} \&
\textbackslash{}quad \textbackslash{}quad + \textbackslash{}left
.\{\textbackslash{}mathop\{∫ \} \}\_\{\{a\}\_\{i\}−\{ 1
\textbackslash{}over n\} \}\^{}\{\{a\}\_\{i\} \}\{(1 + n(t −
\{a\}\_\{i\}))\}\^{}\{2\} dt\textbackslash{}right ) \%\&
\textbackslash{}\textbackslash{} \& =\&
\textbar{}\{λ\}\_\{i\}\{\textbar{}\}\^{}\{2\}\textbackslash{}left
(\{\textbackslash{}mathop\{∫ \} \}\_\{0\}\^{}\{\{ 1 \textbackslash{}over
n\} \}\{(1 − nu)\}\^{}\{2\} dt +\{\textbackslash{}mathop\{∫ \} \}\_\{−\{
1 \textbackslash{}over n\} \}\^{}\{0\}\{(1 + nu)\}\^{}\{2\}
dt\textbackslash{}right )\%\& \textbackslash{}\textbackslash{} \& =\&\{
\textbar{}\{λ\}\_\{i\}\{\textbar{}\}\^{}\{2\} \textbackslash{}over 3n\}
\textbackslash{}left (\{\textbackslash{}left {[}−\{(1 −
nu)\}\^{}\{3\}\textbackslash{}right {]}\}\_\{ 0\}\^{}\{\{ 1
\textbackslash{}over n\} \} +\{ \textbackslash{}left {[}\{(1 +
nu)\}\^{}\{3\}\textbackslash{}right {]}\}\_\{−\{ 1 \textbackslash{}over
n\} \}\^{}\{0\}\textbackslash{}right ) \%\&
\textbackslash{}\textbackslash{} \& =\&\{
2\textbar{}\{λ\}\_\{i\}\{\textbar{}\}\^{}\{2\} \textbackslash{}over 3n\}
\%\& \textbackslash{}\textbackslash{} \textbackslash{}end\{eqnarray*\}

soit encore

2π\textbackslash{}\textbar{}f −
\{g\{\}\_\{n\}\textbackslash{}\textbar{}\}\_\{2\}\^{}\{2\}
=\{\textbackslash{}mathop\{∫ \} \}\_\{0\}\^{}\{2π\}\textbar{}f(t) −
\{g\}\_\{ n\}(t)\{\textbar{}\}\^{}\{2\} dt =\{ 2 \textbackslash{}over
3n\} \{\textbackslash{}mathop\{∑
\}\}\_\{i=1\}\^{}\{p\}\textbar{}\{λ\}\_\{ i\}\{\textbar{}\}\^{}\{2\}

On en déduit que
\{\textbackslash{}mathop\{lim\}\}\_\{n→+∞\}\textbackslash{}\textbar{}f −
\{g\{\}\_\{n\}\textbackslash{}\textbar{}\}\_\{2\} = 0 et que donc on
peut trouver un n tel que \{ 2 \textbackslash{}over n\} \textless{} η
avec \textbackslash{}\textbar{}f −
\{g\{\}\_\{n\}\textbackslash{}\textbar{}\}\_\{2\} \textless{} ε.

Supposons maintenant que f est continue par morceaux. Sa restriction à
{[}0,2π{]} est réglée et donc on peut trouver φ en escalier sur
{[}0,2π{[} (et que l'on prolonge par périodicité) telle que
\textbackslash{}\textbar{}f − \{φ\textbackslash{}\textbar{}\}\_\{∞\}
\textless{}\{ ε \textbackslash{}over 2\} . On a alors

\textbackslash{}begin\{eqnarray*\} \textbackslash{}\textbar{}f −
\{φ\textbackslash{}\textbar{}\}\_\{2\}\^{}\{2\}\& =\&\{ 1
\textbackslash{}over 2π\} \{\textbackslash{}mathop\{∫ \}
\}\_\{0\}\^{}\{2π\}\textbar{}f(t) − φ(t)\{\textbar{}\}\^{}\{2\} dt ≤\{ 1
\textbackslash{}over 2π\} \{\textbackslash{}mathop\{∫ \}
\}\_\{0\}\^{}\{2π\}\textbackslash{}\textbar{}f −
\{φ\textbackslash{}\textbar{}\}\_\{ ∞\}\^{}\{2\} dt\%\&
\textbackslash{}\textbackslash{} \& =\& \textbackslash{}\textbar{}f −
\{φ\textbackslash{}\textbar{}\}\_\{∞\}\^{}\{2\} \%\&
\textbackslash{}\textbackslash{} \textbackslash{}end\{eqnarray*\}

soit encore \textbackslash{}\textbar{}f −
\{φ\textbackslash{}\textbar{}\}\_\{2\} ≤\textbackslash{}\textbar{} f −
\{φ\textbackslash{}\textbar{}\}\_\{∞\} \textless{}\{ ε
\textbackslash{}over 2\} . Mais d'autre part, comme φ est en escalier,
on sait qu'on peut trouver g continue et affine par morceaux telle que
\textbackslash{}\textbar{}φ − \{g\textbackslash{}\textbar{}\}\_\{2\}
\textless{}\{ ε \textbackslash{}over 2\} . On a alors
\textbackslash{}\textbar{}f − \{g\textbackslash{}\textbar{}\}\_\{2\}
≤\textbackslash{}\textbar{} f − \{φ\textbackslash{}\textbar{}\}\_\{2\}
+\textbackslash{}\textbar{} φ − \{g\textbackslash{}\textbar{}\}\_\{2\}
\textless{} ε ce qui démontre le lemme.

Théorème~14.3.14 (Parseval-Plancherel). Soit f : ℝ → ℂ périodique de
période 2π et continue par morceaux. Alors

\textbackslash{}begin\{eqnarray*\}
\textbackslash{}\textbar{}\{f\textbackslash{}\textbar{}\}\_\{2\}\^{}\{2\}\&
=\&\{ 1 \textbackslash{}over 2π\} \{\textbackslash{}mathop\{∫ \}
\}\_\{0\}\^{}\{2π\}\textbar{}f(t)\{\textbar{}\}\^{}\{2\} dt \%\&
\textbackslash{}\textbackslash{} \& =\&
\textbar{}\{c\}\_\{0\}(f)\{\textbar{}\}\^{}\{2\} +\{
\textbackslash{}mathop\{∑ \}\}\_\{n=1\}\^{}\{+∞\}(\textbar{}\{c\}\_\{
n\}(f)\{\textbar{}\}\^{}\{2\} + \textbar{}\{c\}\_\{
−n\}(f)\{\textbar{}\}\^{}\{2\}) \%\& \textbackslash{}\textbackslash{} \&
=\&\{ \textbar{}\{a\}\_\{0\}(f)\{\textbar{}\}\^{}\{2\}
\textbackslash{}over 4\} +\{ 1 \textbackslash{}over 2\}
\{\textbackslash{}mathop\{∑ \}\}\_\{n=1\}\^{}\{+∞\}(\textbar{}\{a\}\_\{
n\}(f)\{\textbar{}\}\^{}\{2\} + \textbar{}\{b\}\_\{
n\}(f)\{\textbar{}\}\^{}\{2\})\%\& \textbackslash{}\textbackslash{}
\textbackslash{}end\{eqnarray*\}

Démonstration On sait que
\textbar{}\{c\}\_\{0\}(f)\{\textbar{}\}\^{}\{2\}
+\{\textbackslash{}mathop\{ \textbackslash{}mathop\{∑ \}\}
\}\_\{n=1\}\^{}\{N\}(\textbar{}\{c\}\_\{n\}(f)\{\textbar{}\}\^{}\{2\} +
\textbar{}\{c\}\_\{−n\}(f)\{\textbar{}\}\^{}\{2\})
=\textbackslash{}\textbar{}
\{S\}\_\{N\}\{(f)\textbackslash{}\textbar{}\}\_\{2\}\^{}\{2\}. D'autre
part, à l'aide du théorème de Pythagore et puisque \{S\}\_\{N\}(f) est
la projection orthogonale de f sur le sous-espace \{T\}\_\{N\} des
polynômes trigonométriques de degré au plus N, on a
\textbackslash{}\textbar{}\{f\textbackslash{}\textbar{}\}\_\{2\}\^{}\{2\}
=\textbackslash{}\textbar{}
\{S\}\_\{N\}\{(f)\textbackslash{}\textbar{}\}\_\{2\}\^{}\{2\}
+\textbackslash{}\textbar{} f −
\{S\}\_\{N\}\{(f)\textbackslash{}\textbar{}\}\_\{2\}\^{}\{2\}. Le
résultat à démontrer est donc équivalent à
\{\textbackslash{}mathop\{lim\}\}\_\{N→+∞\}\textbackslash{}\textbar{}\{S\}\_\{N\}\{(f)\textbackslash{}\textbar{}\}\_\{2\}\^{}\{2\}
=\textbackslash{}\textbar{}
\{f\textbackslash{}\textbar{}\}\_\{2\}\^{}\{2\}, soit encore à
\{\textbackslash{}mathop\{lim\}\}\_\{N→+∞\}\textbackslash{}\textbar{}f −
\{S\}\_\{N\}\{(f)\textbackslash{}\textbar{}\}\_\{2\} = 0.

Supposons tout d'abord que f est \{C\}\^{}\{1\} par morceaux et
continue. On sait que la série de Fourier de f converge normalement,
donc uniformément vers f. On a donc
\{\textbackslash{}mathop\{lim\}\}\_\{N→+∞\}\textbackslash{}\textbar{}f −
\{S\}\_\{N\}\{(f)\textbackslash{}\textbar{}\}\_\{∞\} = 0, mais comme ci
dessus, on a \textbackslash{}\textbar{}f −
\{S\}\_\{N\}\{(f)\textbackslash{}\textbar{}\}\_\{2\}
≤\textbackslash{}\textbar{} f −
\{S\}\_\{N\}\{(f)\textbackslash{}\textbar{}\}\_\{∞\} ce qui montre que
\{\textbackslash{}mathop\{lim\}\}\_\{N→+∞\}\textbackslash{}\textbar{}f −
\{S\}\_\{N\}\{(f)\textbackslash{}\textbar{}\}\_\{2\} = 0.

Si maintenant f est seulement continue par morceaux, soit ε
\textgreater{} 0 et g : ℝ → ℂ périodique de période 2π, de classe
\{C\}\^{}\{1\} par morceaux et continue telle que
\textbackslash{}\textbar{}f − \{g\textbackslash{}\textbar{}\}\_\{2\}
\textless{}\{ ε \textbackslash{}over 2\} . D'après le premier cas, on a
\{\textbackslash{}mathop\{lim\}\}\_\{N→+∞\}\textbackslash{}\textbar{}g −
\{S\}\_\{N\}\{(g)\textbackslash{}\textbar{}\}\_\{2\} = 0 et donc il
existe \{N\}\_\{0\} ∈ ℕ tel que N ≥ \{N\}\_\{0\}
⇒\textbackslash{}\textbar{} g −
\{S\}\_\{N\}\{(g)\textbackslash{}\textbar{}\}\_\{2\} \textless{}\{ ε
\textbackslash{}over 2\} . Mais comme \{S\}\_\{N\}(g) ∈ \{T\}\_\{N\} et
que \{S\}\_\{N\}(f) est la projection orthogonale de f sur \{T\}\_\{N\},
on a \textbackslash{}\textbar{}f −
\{S\}\_\{N\}\{(f)\textbackslash{}\textbar{}\}\_\{2\}
≤\textbackslash{}\textbar{} f −
\{S\}\_\{N\}\{(g)\textbackslash{}\textbar{}\}\_\{2\} soit encore, pour N
≥ \{N\}\_\{0\},

\textbackslash{}begin\{eqnarray*\} \textbackslash{}\textbar{}f −
\{S\}\_\{N\}\{(f)\textbackslash{}\textbar{}\}\_\{2\}\& ≤\&
\textbackslash{}\textbar{}f −
\{S\}\_\{N\}\{(g)\textbackslash{}\textbar{}\}\_\{2\}
≤\textbackslash{}\textbar{} f − \{g\textbackslash{}\textbar{}\}\_\{2\}
+\textbackslash{}\textbar{} g −
\{S\}\_\{N\}\{(g)\textbackslash{}\textbar{}\}\_\{2\}\%\&
\textbackslash{}\textbackslash{} \& \textless{}\&\{ ε
\textbackslash{}over 2\} +\{ ε \textbackslash{}over 2\} = ε \%\&
\textbackslash{}\textbackslash{} \textbackslash{}end\{eqnarray*\}

ce qui démontre le résultat. La deuxième formule résulte d'un calcul
précédent qui montre que

\textbackslash{}begin\{eqnarray*\}
\textbar{}\{c\}\_\{0\}(f)\{\textbar{}\}\^{}\{2\}\& +\&
\{\textbackslash{}mathop\{∑ \}\}\_\{n=1\}\^{}\{+∞\}(\textbar{}\{c\}\_\{
n\}(f)\{\textbar{}\}\^{}\{2\} + \textbar{}\{c\}\_\{
−n\}(f)\{\textbar{}\}\^{}\{2\}) \%\& \textbackslash{}\textbackslash{} \&
=\&\{ \textbar{}\{a\}\_\{0\}(f)\{\textbar{}\}\^{}\{2\}
\textbackslash{}over 4\} +\{ 1 \textbackslash{}over 2\}
\{\textbackslash{}mathop\{∑ \}\}\_\{n=1\}\^{}\{+∞\}(\textbar{}\{a\}\_\{
n\}(f)\{\textbar{}\}\^{}\{2\} + \textbar{}\{b\}\_\{
n\}(f)\{\textbar{}\}\^{}\{2\})\%\& \textbackslash{}\textbackslash{}
\textbackslash{}end\{eqnarray*\}

Corollaire~14.3.15 (injectivité de la transformation de Fourier). Soit f
et g deux fonctions de D telles que \textbackslash{}mathop\{∀\}n ∈ ℕ,
\{c\}\_\{n\}(f) = \{c\}\_\{n\}(g). Alors f = g.

Démonstration On a \textbackslash{}mathop\{∀\}n ∈ ℕ, \{c\}\_\{n\}(f − g)
= 0, soit encore d'après le théorème de Parseval,
\textbackslash{}\textbar{}f − \{g\textbackslash{}\textbar{}\}\_\{2\} =
0. Comme f − g appartient à D sur laquelle le produit scalaire est
défini positif, on a f − g = 0.

Remarque~14.3.6 Si on suppose seulement que f et g sont continues par
morceaux, on obtient seulement que f et g coïncident sauf en un nombre
fini de points (sur un intervalle de longueur 2π).

{[}\href{coursse80.html}{next}{]} {[}\href{coursse78.html}{prev}{]}
{[}\href{coursse78.html\#tailcoursse78.html}{prev-tail}{]}
{[}\href{coursse79.html}{front}{]}
{[}\href{coursch15.html\#coursse79.html}{up}{]}

\end{document}
