\documentclass[]{article}
\usepackage[T1]{fontenc}
\usepackage{lmodern}
\usepackage{amssymb,amsmath}
\usepackage{ifxetex,ifluatex}
\usepackage{fixltx2e} % provides \textsubscript
% use upquote if available, for straight quotes in verbatim environments
\IfFileExists{upquote.sty}{\usepackage{upquote}}{}
\ifnum 0\ifxetex 1\fi\ifluatex 1\fi=0 % if pdftex
  \usepackage[utf8]{inputenc}
\else % if luatex or xelatex
  \ifxetex
    \usepackage{mathspec}
    \usepackage{xltxtra,xunicode}
  \else
    \usepackage{fontspec}
  \fi
  \defaultfontfeatures{Mapping=tex-text,Scale=MatchLowercase}
  \newcommand{\euro}{€}
\fi
% use microtype if available
\IfFileExists{microtype.sty}{\usepackage{microtype}}{}
\ifxetex
  \usepackage[setpagesize=false, % page size defined by xetex
              unicode=false, % unicode breaks when used with xetex
              xetex]{hyperref}
\else
  \usepackage[unicode=true]{hyperref}
\fi
\hypersetup{breaklinks=true,
            bookmarks=true,
            pdfauthor={},
            pdftitle={Polynomes `a une variable},
            colorlinks=true,
            citecolor=blue,
            urlcolor=blue,
            linkcolor=magenta,
            pdfborder={0 0 0}}
\urlstyle{same}  % don't use monospace font for urls
\setlength{\parindent}{0pt}
\setlength{\parskip}{6pt plus 2pt minus 1pt}
\setlength{\emergencystretch}{3em}  % prevent overfull lines
\setcounter{secnumdepth}{0}
 
/* start css.sty */
.cmr-5{font-size:50%;}
.cmr-7{font-size:70%;}
.cmmi-5{font-size:50%;font-style: italic;}
.cmmi-7{font-size:70%;font-style: italic;}
.cmmi-10{font-style: italic;}
.cmsy-5{font-size:50%;}
.cmsy-7{font-size:70%;}
.cmex-7{font-size:70%;}
.cmex-7x-x-71{font-size:49%;}
.msbm-7{font-size:70%;}
.cmtt-10{font-family: monospace;}
.cmti-10{ font-style: italic;}
.cmbx-10{ font-weight: bold;}
.cmr-17x-x-120{font-size:204%;}
.cmsl-10{font-style: oblique;}
.cmti-7x-x-71{font-size:49%; font-style: italic;}
.cmbxti-10{ font-weight: bold; font-style: italic;}
p.noindent { text-indent: 0em }
td p.noindent { text-indent: 0em; margin-top:0em; }
p.nopar { text-indent: 0em; }
p.indent{ text-indent: 1.5em }
@media print {div.crosslinks {visibility:hidden;}}
a img { border-top: 0; border-left: 0; border-right: 0; }
center { margin-top:1em; margin-bottom:1em; }
td center { margin-top:0em; margin-bottom:0em; }
.Canvas { position:relative; }
li p.indent { text-indent: 0em }
.enumerate1 {list-style-type:decimal;}
.enumerate2 {list-style-type:lower-alpha;}
.enumerate3 {list-style-type:lower-roman;}
.enumerate4 {list-style-type:upper-alpha;}
div.newtheorem { margin-bottom: 2em; margin-top: 2em;}
.obeylines-h,.obeylines-v {white-space: nowrap; }
div.obeylines-v p { margin-top:0; margin-bottom:0; }
.overline{ text-decoration:overline; }
.overline img{ border-top: 1px solid black; }
td.displaylines {text-align:center; white-space:nowrap;}
.centerline {text-align:center;}
.rightline {text-align:right;}
div.verbatim {font-family: monospace; white-space: nowrap; text-align:left; clear:both; }
.fbox {padding-left:3.0pt; padding-right:3.0pt; text-indent:0pt; border:solid black 0.4pt; }
div.fbox {display:table}
div.center div.fbox {text-align:center; clear:both; padding-left:3.0pt; padding-right:3.0pt; text-indent:0pt; border:solid black 0.4pt; }
div.minipage{width:100%;}
div.center, div.center div.center {text-align: center; margin-left:1em; margin-right:1em;}
div.center div {text-align: left;}
div.flushright, div.flushright div.flushright {text-align: right;}
div.flushright div {text-align: left;}
div.flushleft {text-align: left;}
.underline{ text-decoration:underline; }
.underline img{ border-bottom: 1px solid black; margin-bottom:1pt; }
.framebox-c, .framebox-l, .framebox-r { padding-left:3.0pt; padding-right:3.0pt; text-indent:0pt; border:solid black 0.4pt; }
.framebox-c {text-align:center;}
.framebox-l {text-align:left;}
.framebox-r {text-align:right;}
span.thank-mark{ vertical-align: super }
span.footnote-mark sup.textsuperscript, span.footnote-mark a sup.textsuperscript{ font-size:80%; }
div.tabular, div.center div.tabular {text-align: center; margin-top:0.5em; margin-bottom:0.5em; }
table.tabular td p{margin-top:0em;}
table.tabular {margin-left: auto; margin-right: auto;}
div.td00{ margin-left:0pt; margin-right:0pt; }
div.td01{ margin-left:0pt; margin-right:5pt; }
div.td10{ margin-left:5pt; margin-right:0pt; }
div.td11{ margin-left:5pt; margin-right:5pt; }
table[rules] {border-left:solid black 0.4pt; border-right:solid black 0.4pt; }
td.td00{ padding-left:0pt; padding-right:0pt; }
td.td01{ padding-left:0pt; padding-right:5pt; }
td.td10{ padding-left:5pt; padding-right:0pt; }
td.td11{ padding-left:5pt; padding-right:5pt; }
table[rules] {border-left:solid black 0.4pt; border-right:solid black 0.4pt; }
.hline hr, .cline hr{ height : 1px; margin:0px; }
.tabbing-right {text-align:right;}
span.TEX {letter-spacing: -0.125em; }
span.TEX span.E{ position:relative;top:0.5ex;left:-0.0417em;}
a span.TEX span.E {text-decoration: none; }
span.LATEX span.A{ position:relative; top:-0.5ex; left:-0.4em; font-size:85%;}
span.LATEX span.TEX{ position:relative; left: -0.4em; }
div.float img, div.float .caption {text-align:center;}
div.figure img, div.figure .caption {text-align:center;}
.marginpar {width:20%; float:right; text-align:left; margin-left:auto; margin-top:0.5em; font-size:85%; text-decoration:underline;}
.marginpar p{margin-top:0.4em; margin-bottom:0.4em;}
.equation td{text-align:center; vertical-align:middle; }
td.eq-no{ width:5%; }
table.equation { width:100%; } 
div.math-display, div.par-math-display{text-align:center;}
math .texttt { font-family: monospace; }
math .textit { font-style: italic; }
math .textsl { font-style: oblique; }
math .textsf { font-family: sans-serif; }
math .textbf { font-weight: bold; }
.partToc a, .partToc, .likepartToc a, .likepartToc {line-height: 200%; font-weight:bold; font-size:110%;}
.chapterToc a, .chapterToc, .likechapterToc a, .likechapterToc, .appendixToc a, .appendixToc {line-height: 200%; font-weight:bold;}
.index-item, .index-subitem, .index-subsubitem {display:block}
.caption td.id{font-weight: bold; white-space: nowrap; }
table.caption {text-align:center;}
h1.partHead{text-align: center}
p.bibitem { text-indent: -2em; margin-left: 2em; margin-top:0.6em; margin-bottom:0.6em; }
p.bibitem-p { text-indent: 0em; margin-left: 2em; margin-top:0.6em; margin-bottom:0.6em; }
.paragraphHead, .likeparagraphHead { margin-top:2em; font-weight: bold;}
.subparagraphHead, .likesubparagraphHead { font-weight: bold;}
.quote {margin-bottom:0.25em; margin-top:0.25em; margin-left:1em; margin-right:1em; text-align:justify;}
.verse{white-space:nowrap; margin-left:2em}
div.maketitle {text-align:center;}
h2.titleHead{text-align:center;}
div.maketitle{ margin-bottom: 2em; }
div.author, div.date {text-align:center;}
div.thanks{text-align:left; margin-left:10%; font-size:85%; font-style:italic; }
div.author{white-space: nowrap;}
.quotation {margin-bottom:0.25em; margin-top:0.25em; margin-left:1em; }
h1.partHead{text-align: center}
.sectionToc, .likesectionToc {margin-left:2em;}
.subsectionToc, .likesubsectionToc {margin-left:4em;}
.subsubsectionToc, .likesubsubsectionToc {margin-left:6em;}
.frenchb-nbsp{font-size:75%;}
.frenchb-thinspace{font-size:75%;}
.figure img.graphics {margin-left:10%;}
/* end css.sty */

\title{Polynomes `a une variable}
\author{}
\date{}

\begin{document}
\maketitle

\textbf{Warning: \href{http://www.math.union.edu/locate/jsMath}{jsMath}
requires JavaScript to process the mathematics on this page.\\ If your
browser supports JavaScript, be sure it is enabled.}

\begin{center}\rule{3in}{0.4pt}\end{center}

{[}\href{coursse6.html}{next}{]} {[}\href{coursse4.html}{prev}{]}
{[}\href{coursse4.html\#tailcoursse4.html}{prev-tail}{]}
{[}\hyperref[tailcoursse5.html]{tail}{]}
{[}\href{coursch2.html\#coursse5.html}{up}{]}

\subsubsection{1.5 Polynômes à une variable}

On désignera par A un anneau commutatif.

\paragraph{1.5.1 L'anneau des séries formelles à coefficients dans A}

Définition~1.5.1 On appelle anneau des séries formelles à coefficients
dans A, l'anneau ainsi défini~: son ensemble de base est l'ensemble
\{A\}\^{}\{ℕ\} des suites d'éléments de A muni des lois suivantes

(\{a\}\_\{n\}) + (\{b\}\_\{n\}) = (\{a\}\_\{n\} +
\{b\}\_\{n\})\textbackslash{}quad (\{a\}\_\{n\})(\{b\}\_\{n\}) =
(\{c\}\_\{n\})

avec

\{c\}\_\{n\} =\{ \textbackslash{}mathop\{∑
\}\}\_\{p+q=n\}\{a\}\_\{p\}\{b\}\_\{q\} =\{ \textbackslash{}mathop\{∑
\}\}\_\{p=0\}\^{}\{n\}\{a\}\_\{ p\}\{b\}\_\{n−p\}

L'élément unité est la suite
(1,0,\textbackslash{}mathop\{\textbackslash{}mathop\{\ldots{}\}\},0,\textbackslash{}mathop\{\textbackslash{}mathop\{\ldots{}\}\})
et on note X =
(0,1,0,\textbackslash{}mathop\{\textbackslash{}mathop\{\ldots{}\}\},0,\textbackslash{}mathop\{\textbackslash{}mathop\{\ldots{}\}\}).

Démonstration Facile, sauf peut-être en ce qui concerne l'associativité
de la multiplication. Mais on a~:
((\{a\}\_\{n\})(\{b\}\_\{n\}))(\{c\}\_\{n\}) = (\{d\}\_\{n\}) avec
\{d\}\_\{n\} =\{\textbackslash{}mathop\{ \textbackslash{}mathop\{∑ \}\}
\}\_\{p+q+r=n\}\{a\}\_\{p\}\{b\}\_\{q\}\{c\}\_\{r\}, ce qui conduit à
une vérification facile de cette associativité.

Remarque~1.5.1 On vérifie immédiatement que \{X\}\^{}\{n\} est la suite
qui a un 1 à la (n + 1)-ième place et des 0 partout ailleurs si bien que
(\{a\}\_\{n\}) =\{\textbackslash{}mathop\{ \textbackslash{}mathop\{∑
\}\} \}\_\{n≥0\}\{a\}\_\{n\}\{X\}\^{}\{n\}. Pour cette raison, cet
anneau est noté A{[}{[}X{]}{]}. On utilisera systématiquement la
notation \{\textbackslash{}mathop\{\textbackslash{}mathop\{∑ \}\}
\}\_\{n≥0\}\{a\}\_\{n\}\{X\}\^{}\{n\} pour désigner un de ses éléments.

Exercice Montrer que les éléments inversibles de A{[}{[}X{]}{]} sont les
séries formelles \{\textbackslash{}mathop\{\textbackslash{}mathop\{∑
\}\} \}\_\{n≥0\}\{a\}\_\{n\}\{X\}\^{}\{n\} telles que \{a\}\_\{0\} soit
un élément inversible de A. Montrer que si A est un corps, l'anneau
A{[}{[}X{]}{]} est un anneau principal n'ayant qu'un seul élément
irréductible (à multiplication près par un élément inversible).

\paragraph{1.5.2 L'anneau des polynômes à coefficients dans A}

Définition~1.5.2 On appelle polynôme à coefficients dans A une série
formelle à support fini, c'est-à-dire
\{\textbackslash{}mathop\{\textbackslash{}mathop\{∑ \}\}
\}\_\{n≥0\}\{a\}\_\{n\}\{X\}\^{}\{n\} telle que
\textbackslash{}\{n\textbackslash{}mathrel\{∣\}\{a\}\_\{n\}\textbackslash{}mathrel\{≠\}0\textbackslash{}\}
est fini.

Proposition~1.5.1 L'ensemble A{[}X{]} des polynômes à coefficients dans
A est un sous anneau de A{[}{[}X{]}{]}.

Définition~1.5.3 Soit P ∈ A{[}X{]}. On appelle degré et valuation de P~:

\textbackslash{}mathop\{deg\} P = \textbackslash{}left
\textbackslash{}\{ \textbackslash{}cases\{ −∞ \&si P = 0
\textbackslash{}cr
\textbackslash{}mathop\{max\}\textbackslash{}\{k\textbackslash{}mathrel\{∣\}\{a\}\_\{k\}\textbackslash{}mathrel\{≠\}0\textbackslash{}\}\&si
P = \textbackslash{}mathop\{∑
\}\{a\}\_\{k\}\{X\}\^{}\{k\}\textbackslash{}mathrel\{≠\}0 \}
\textbackslash{}right .

v(P) = \textbackslash{}left \textbackslash{}\{ \textbackslash{}cases\{
+∞ \&si P = 0 \textbackslash{}cr
\textbackslash{}mathop\{min\}\textbackslash{}\{k\textbackslash{}mathrel\{∣\}\{a\}\_\{k\}\textbackslash{}mathrel\{≠\}0\textbackslash{}\}\&si
P = \textbackslash{}mathop\{∑
\}\{a\}\_\{k\}\{X\}\^{}\{k\}\textbackslash{}mathrel\{≠\}0 \}
\textbackslash{}right .

Proposition~1.5.2 On a (i) \textbackslash{}mathop\{deg\} (P + Q)
≤\textbackslash{}mathop\{ max\}(\textbackslash{}mathop\{deg\}
P,\textbackslash{}mathop\{deg\} Q) avec égalité si
\textbackslash{}mathop\{deg\}
P\textbackslash{}mathrel\{≠\}\textbackslash{}mathop\{deg\} Q (ii)
\textbackslash{}mathop\{deg\} (PQ) ≤\textbackslash{}mathop\{ deg\} P
+\textbackslash{}mathop\{ deg\} Q avec égalité sauf si le produit des
termes de plus haut degré de P et Q est nul. En particulier on a égalité
si A est intègre.

Remarque~1.5.2 On a des résultats similaires pour la valuation.

Proposition~1.5.3 (règle de substitution). Soit A et B deux anneaux
commutatifs et φ:A → B un morphisme d'anneaux. Soit β ∈ B. Alors
l'application \{T\}\_\{φ,β\}:A{[}X{]} → B,
\{\textbackslash{}mathop\{\textbackslash{}mathop\{∑ \}\}
\}\_\{k=0\}\^{}\{n\}\{a\}\_\{k\}\{X\}\^{}\{k\}\textbackslash{}mathrel\{↦\}\{\textbackslash{}mathop\{\textbackslash{}mathop\{∑
\}\} \}\_\{k=0\}\^{}\{n\}φ(\{a\}\_\{k\})\{β\}\^{}\{k\} est un morphisme
d'anneaux.

Démonstration Faire le calcul.

Remarque~1.5.3 Lorsque A ⊂ B et φ(a) = a, on note P(β) =
\{T\}\_\{φ,β\}(P).

\paragraph{1.5.3 Division euclidienne et racines}

Théorème~1.5.4 (division euclidienne). Soit \{P\}\_\{1\},\{P\}\_\{2\} ∈
A{[}X{]} tels que \{P\}\_\{2\}\textbackslash{}mathrel\{≠\}0 et le terme
de plus haut degré de \{P\}\_\{2\} est inversible dans A. Alors il
existe un unique couple (Q,R) ∈ A\{{[}X{]}\}\^{}\{2\} tel que
\{P\}\_\{1\} = \{P\}\_\{2\}Q + R avec \textbackslash{}mathop\{deg\} R
\textless{}\textbackslash{}mathop\{ deg\} \{P\}\_\{2\}.

Démonstration On démontre l'existence par récurrence sur n
=\textbackslash{}mathop\{ deg\} \{P\}\_\{1\}. Si n
\textless{}\textbackslash{}mathop\{ deg\} \{P\}\_\{2\}, alors Q = 0 et R
= \{P\}\_\{1\} conviennent. Supposons le résultat montré pour tout
polynôme de degré strictement inférieur à n ≥\textbackslash{}mathop\{
deg\} \{P\}\_\{2\} et soit \textbackslash{}mathop\{deg\} \{P\}\_\{1\} =
n. On écrit \{P\}\_\{1\}(X) = \{a\}\_\{n\}\{X\}\^{}\{n\} +
\textbackslash{}mathop\{\textbackslash{}mathop\{\ldots{}\}\} et
\{P\}\_\{2\}(X) = \{b\}\_\{m\}\{X\}\^{}\{m\} +
\textbackslash{}mathop\{\textbackslash{}mathop\{\ldots{}\}\} avec
\{b\}\_\{m\} inversible. Posons \{S\}\_\{1\}(X) = \{P\}\_\{1\}(X) −
\{a\}\_\{n\}\{b\}\_\{m\}\^{}\{−1\}\{X\}\^{}\{n−m\}\{P\}\_\{2\}(X)~;
alors \textbackslash{}mathop\{deg\} \{S\}\_\{1\}
≤\textbackslash{}mathop\{ max\}(\textbackslash{}mathop\{deg\}
P,\textbackslash{}mathop\{deg\} (\{X\}\^{}\{n−m\}\{P\}\_\{2\})) = n et
\{S\}\_\{1\} n'a plus de terme de degré n. D'après l'hypothèse de
récurrence, on peut donc écrire \{S\}\_\{1\}(X) =
\{Q\}\_\{1\}(X)\{P\}\_\{2\}(X) + R(X) avec \textbackslash{}mathop\{deg\}
R \textless{}\textbackslash{}mathop\{ deg\} \{P\}\_\{2\}. Mais alors
P(X) = \{S\}\_\{1\}(X) +
\{a\}\_\{n\}\{b\}\_\{m\}\^{}\{−1\}\{X\}\^{}\{n−m\}\{P\}\_\{2\}(X) =
(\{Q\}\_\{1\}(X) +
\{a\}\_\{n\}\{b\}\_\{m\}\^{}\{−1\}\{X\}\^{}\{n−m\})\{P\}\_\{2\}(X) +
R(X) et donc Q(X) = \{Q\}\_\{1\}(X) +
\{a\}\_\{n\}\{b\}\_\{m\}\^{}\{−1\}\{X\}\^{}\{n−m\} et R(X) répondent aux
exigences voulues.

Pour l'unicité, on remarque que P = \{P\}\_\{2\}Q + R = \{P\}\_\{2\}Q' +
R' exige \{P\}\_\{2\}(Q − Q') = R' − R. Or \textbackslash{}mathop\{deg\}
(R' − R) \textless{}\textbackslash{}mathop\{ deg\} \{P\}\_\{2\}, et, si
Q − Q'\textbackslash{}mathrel\{≠\}0, \textbackslash{}mathop\{deg\}
(\{P\}\_\{2\}(Q − Q')) =\textbackslash{}mathop\{ deg\} \{P\}\_\{2\}
+\textbackslash{}mathop\{ deg\} (Q − Q') ≥\textbackslash{}mathop\{ deg\}
\{P\}\_\{2\} (car le coefficient de plus haut degré de \{Q\}\_\{2\} est
inversible, et donc le produit des coefficients de plus haut degré de
\{P\}\_\{2\} et Q − Q' ne peut pas être nul). C'est absurde. Donc Q =
Q', ce qui implique également R = R', et montre l'unicité.

Corollaire~1.5.5 Soit P ∈ A{[}X{]} et a ∈ A. Alors X −
a\textbackslash{}mathrel\{∣\}P \textbackslash{}mathrel\{⇔\} P(a) = 0.

Démonstration Si X − a divise P, on a P(X) = (X − a)Q(X) et donc P(a) =
0. Inversement, supposons que P(a) = 0 et effectuons la division
euclidienne de P(X) par X − a (dont le coefficient de plus haut degré
est 1, donc inversible)~; on peut écrire P(X) = (X − a)Q(X) + R(X) avec
\textbackslash{}mathop\{deg\} R \textless{} 1. Donc R est une constante
b~; mais alors 0 = P(a) = (a − a)Q(a) + b = b et donc P(X) = (X − a)Q(X)
et X − a divise P.

Remarque~1.5.4 On dit que a est racine de P si P(a) = 0.

Corollaire~1.5.6 Supposons A intègre. Soit P ∈ A{[}X{]} et
\{a\}\_\{1\},\textbackslash{}mathop\{\textbackslash{}mathop\{\ldots{}\}\},\{a\}\_\{k\}
des racines de P. Alors (X −
\{a\}\_\{1\})\textbackslash{}mathop\{\textbackslash{}mathop\{\ldots{}\}\}(X
− \{a\}\_\{k\})\textbackslash{}mathrel\{∣\}P. Si
P\textbackslash{}mathrel\{≠\}0, on a k ≤\textbackslash{}mathop\{ deg\}
P.

Démonstration Par récurrence sur k. On a déjà vu le résultat pour k = 1.
Supposons le vérifié pour k − 1~; on a donc déjà le fait que (X −
\{a\}\_\{1\})\textbackslash{}mathop\{\textbackslash{}mathop\{\ldots{}\}\}(X
− \{a\}\_\{k−1\}) divise P~; donc P(X) = (X −
\{a\}\_\{1\})\textbackslash{}mathop\{\textbackslash{}mathop\{\ldots{}\}\}(X
− \{a\}\_\{k−1\})Q(X). Mais alors 0 = P(\{a\}\_\{k\}) = (\{a\}\_\{k\} −
\{a\}\_\{1\})\textbackslash{}mathop\{\textbackslash{}mathop\{\ldots{}\}\}(\{a\}\_\{k\}
− \{a\}\_\{k−1\})Q(\{a\}\_\{k\}), et donc Q(\{a\}\_\{k\}) = 0 (les
autres termes sont non nuls et A est intègre)~; en particulier X −
\{a\}\_\{k\} divise Q et donc (X −
\{a\}\_\{1\})\textbackslash{}mathop\{\textbackslash{}mathop\{\ldots{}\}\}(X
− \{a\}\_\{k\})\textbackslash{}mathrel\{∣\}P.

Remarque~1.5.5 Sur un anneau intègre un polynôme non nul n'a donc qu'un
nombre fini de racines. En particulier on en déduit

Corollaire~1.5.7 Soit A un anneau intègre infini. Alors l'application
A{[}X{]} → \{A\}\^{}\{A\},
P\textbackslash{}mathrel\{↦\}\textbackslash{}tilde\{P\} avec
\textbackslash{}tilde\{P\}(x) = P(x), qui à un polynôme associe sa
fonction polynomiale, est un morphisme d'anneaux injectif.

\paragraph{1.5.4 Dérivation}

Définition~1.5.4 Soit P =\{\textbackslash{}mathop\{
\textbackslash{}mathop\{∑ \}\} \}\_\{k≥0\}\{a\}\_\{k\}\{X\}\^{}\{k\} ∈
A{[}X{]}. On appelle dérivée de P le polynôme

P' =\{ \textbackslash{}mathop\{∑
\}\}\_\{k≥1\}k\{a\}\_\{k\}\{X\}\^{}\{k−1\}

Proposition~1.5.8 On a les formules (αP + βQ)' = αP' + βQ', (PQ)' = P'Q
+ PQ', (\{P\}\^{}\{n\})' = nP'\{P\}\^{}\{n−1\}.

Démonstration On montre les résultats sur les monômes et on les étend
aux polynômes par linéarité.

\paragraph{1.5.5 L'anneau principal K{[}X{]}}

Soit K un corps commutatif. Le coefficient de plus haut degré d'un
polynôme non nul de K{[}X{]} étant par essence même différent de 0 donc
inversible, la division euclidienne est toujours possible. L'anneau
K{[}X{]} est donc un anneau euclidien, et par conséquent un anneau
principal. Tous les résultats sur les anneaux principaux s'appliquent
donc à K{[}X{]}~: existence du PGCD et du PPCM, identité de Bézout,
théorème de Gauss, existence et unicité de la décomposition en polynômes
irréductibles normalisés (ceux-ci étant des représentants naturels des
classes d'éléments irréductibles). Il en est de même des résultats sur
les anneaux euclidiens, et en particulier de l'algorithme de calcul du
PGCD.

Dans toute la suite du chapitre, les corps seront toujours supposés
commutatifs.

\paragraph{1.5.6 Formule de Taylor. Multiplicité d'une racine}

Définition~1.5.5 Soit K un corps, P ∈ K{[}X{]} et a ∈ K. On dit que a
est racine de multiplicité k de P si \{(X −
a)\}\^{}\{k\}\textbackslash{}mathrel\{∣\}P et \{(X − a)\}\^{}\{k+1\} ne
divise pas P.

Proposition~1.5.9 Soit K un corps, P ∈ K{[}X{]},
P\textbackslash{}mathrel\{≠\}0. Soit
\{a\}\_\{1\},\textbackslash{}mathop\{\textbackslash{}mathop\{\ldots{}\}\},\{a\}\_\{k\}
les racines de P de multiplicités respectives
\{m\}\_\{1\},\textbackslash{}mathop\{\textbackslash{}mathop\{\ldots{}\}\},\{m\}\_\{k\}.
Alors \{(X −
\{a\}\_\{1\})\}\^{}\{\{m\}\_\{1\}\}\textbackslash{}mathop\{\textbackslash{}mathop\{\ldots{}\}\}\{(X
− \{a\}\_\{k\})\}\^{}\{\{m\}\_\{k\}\}\textbackslash{}mathrel\{∣\}P et
donc \{m\}\_\{1\} +
\textbackslash{}mathop\{\textbackslash{}mathop\{\ldots{}\}\} +
\{m\}\_\{k\} ≤\textbackslash{}mathop\{ deg\} P. Si on a égalité, on a P
= λ\{(X −
\{a\}\_\{1\})\}\^{}\{\{m\}\_\{1\}\}\textbackslash{}mathop\{\textbackslash{}mathop\{\ldots{}\}\}\{(X
− \{a\}\_\{k\})\}\^{}\{\{m\}\_\{k\}\}, avec λ ∈ K. On dit alors que P
est scindé sur A.

Démonstration En effet les polynômes \{(X −
\{a\}\_\{i\})\}\^{}\{\{m\}\_\{i\}\} sont deux à deux premiers entre eux.

Théorème~1.5.10 (formule de Taylor pour les polynômes). Soit K un corps
de caractéristique 0, P ∈ K{[}X{]} et a ∈ K. Alors P(X + a)
=\{\textbackslash{}mathop\{ \textbackslash{}mathop\{∑ \}\}
\}\_\{k=0\}\^{}\{\textbackslash{}mathop\{deg\} P\}\{ \{P\}\^{}\{(k)\}(a)
\textbackslash{}over k!\} \{X\}\^{}\{k\}. Si A est un sous-anneau de K
qui contient à la fois a et les coefficients de P, alors
\textbackslash{}mathop\{∀\}k,\{ \{P\}\^{}\{(k)\}(a) \textbackslash{}over
k!\} ∈ A.

Démonstration Le polynôme P(X + a) s'écrit sous la forme Q(X)
=\{\textbackslash{}mathop\{ \textbackslash{}mathop\{∑ \}\}
\}\_\{k=0\}\^{}\{\textbackslash{}mathop\{deg\}
P\}\{b\}\_\{k\}\{X\}\^{}\{k\} (en développant chacun des \{(X +
a)\}\^{}\{k\} par la formule du binôme) et un calcul trivial montre que
\{P\}\^{}\{(k)\}(a) = \{Q\}\^{}\{(k)\}(0) = k!\{b\}\_\{k\}, soit
\{b\}\_\{k\} =\{ \{P\}\^{}\{(k)\}(a) \textbackslash{}over k!\} ~; de
plus, si les \{a\}\_\{k\} et a sont dans A, il en est de même des
\{b\}\_\{k\} (toujours par la formule du binôme).

Corollaire~1.5.11 Soit K un corps de caractéristique 0, P ∈ K{[}X{]} et
a ∈ K. Alors a est racine de multiplicité m de P si et seulement si P(a)
= P'(a) = \textbackslash{}mathop\{\textbackslash{}mathop\{\ldots{}\}\} =
\{P\}\^{}\{(m−1)\}(a) = 0 et
\{P\}\^{}\{(m)\}(a)\textbackslash{}mathrel\{≠\}0.

Démonstration Si P(X) = \{(X − a)\}\^{}\{m\}Q(X), une récurrence facile
montre que \{P\}\^{}\{(k)\}(X) = m(m −
1)\textbackslash{}mathop\{\textbackslash{}mathop\{\ldots{}\}\}(m − k +
1)\{(X − a)\}\^{}\{m−k\}Q(X) + \{(X − a)\}\^{}\{m−k+1\}\{R\}\_\{k\}(X),
pour k ≤ m~; on en déduit que \{P\}\^{}\{(k)\}(a) = 0 pour k ≤ m − 1 et
que \{P\}\^{}\{(m)\}(a) = m!Q(a)\textbackslash{}mathrel\{≠\}0 si X − a
ne divise pas Q, soit \{(X − a)\}\^{}\{m+1\} ne divise pas P.

Inversement, si P(a) = P'(a) =
\textbackslash{}mathop\{\textbackslash{}mathop\{\ldots{}\}\} =
\{P\}\^{}\{(m−1)\}(a) = 0, la formule de Taylor (dont les m premiers
termes sont nuls) montre que \{(X − a)\}\^{}\{m\} divise P. Mais, le
fait que \{P\}\^{}\{(m)\}(a)\textbackslash{}mathrel\{≠\}0, montre
d'après le sens direct, que \{(X − a)\}\^{}\{m+1\} ne divise pas P.

Remarque~1.5.6 En particulier P a une racine multiple si et seulement si
P et P' ont une racine commune.

\paragraph{1.5.7 Racines et extensions de corps}

Théorème~1.5.12 Soit K un corps et P ∈ K{[}X{]}. Alors il existe un
corps L contenant K dans lequel P a une racine.

Démonstration Il suffit bien entendu de démontrer ce résultat lorsque P
est un polynôme irréductible. On prend alors L = K{[}X{]}∕PK{[}X{]} (qui
contient K{[}X{]} en identifiant a ∈ K à π(a) ∈ L) et alors, si on pose
x = π(X) on a, puisque π est un morphisme d'anneaux, P(x) = P(π(X)) =
π(P(X)) = 0. Donc P a bien une racine dans L.

On montre alors par récurrence sur \textbackslash{}mathop\{deg\} P le
corollaire suivant

Corollaire~1.5.13 Soit K un corps et P ∈ K{[}X{]}. Alors il existe un
corps L contenant K sur lequel P est scindé.

Définition~1.5.6 On dit qu'un corps K est algébriquement clos s'il
vérifie les conditions équivalentes suivantes (i) Tout polynôme de
K{[}X{]} non constant a une racine dans K (ii) Tout polynôme de K{[}X{]}
est scindé sur K (iii) Les seuls polynômes irréductibles de K{[}X{]}
sont les polynômes de degré 1

Théorème~1.5.14 On montre, et on admettra que le corps des nombres
complexes est algébriquement clos (théorème de d'Alembert-Gauss)

\paragraph{1.5.8 Polynômes sur ℂ et ℝ}

On a vu que ℂ est algébriquement clos. On en déduit le résultat suivant

Théorème~1.5.15 Tout polynôme non constant de ℂ{[}X{]} a une racine dans
ℂ. Les seuls polynômes irréductibles de ℂ{[}X{]} sont les polynômes de
degré 1. Si
\{α\}\_\{1\},\textbackslash{}mathop\{\textbackslash{}mathop\{\ldots{}\}\},\{α\}\_\{k\}
sont les racines dans ℂ du polynôme P ∈ ℂ{[}X{]}, de multiplicités
respectives
\{m\}\_\{1\},\textbackslash{}mathop\{\textbackslash{}mathop\{\ldots{}\}\},\{m\}\_\{k\},
on a \{m\}\_\{1\} +
\textbackslash{}mathop\{\textbackslash{}mathop\{\ldots{}\}\} +
\{m\}\_\{k\} =\textbackslash{}mathop\{ deg\} P et P(X) =
\{a\}\_\{n\}\{\textbackslash{}mathop\{ \textbackslash{}mathop\{∏ \}\}
\}\_\{i=1\}\^{}\{k\}\{(X − \{α\}\_\{i\})\}\^{}\{\{m\}\_\{i\}\}.

Théorème~1.5.16 Soit P ∈ ℝ{[}X{]}.

\begin{itemize}
\item
  Si α ∈ ℂ est racine de P de multiplicité m, il en est de même de
  \textbackslash{}overline\{α\}.
\item
  Le nombre de racines non réelles de P est pair.
\item
  Si P est de degré impair, il a au moins une racine réelle.
\item
  Les polynômes irréductibles sur ℝ sont d'une part les polynômes de
  degré 1, d'autre part les polynômes de degré 2 sans racine réelle (Δ
  \textless{} 0).
\item
  Soit P ∈ ℝ{[}X{]}, soit
  \{α\}\_\{1\},\textbackslash{}mathop\{\textbackslash{}mathop\{\ldots{}\}\},\{α\}\_\{k\}
  ses racines réelles de multiplicités
  \{m\}\_\{1\},\textbackslash{}mathop\{\textbackslash{}mathop\{\ldots{}\}\},\{m\}\_\{k\},
  \{β\}\_\{1\},\textbackslash{}mathop\{\textbackslash{}mathop\{\ldots{}\}\},\{β\}\_\{l\}
  ses racines complexes de parties imaginaires strictement positives, de
  multiplicités
  \{n\}\_\{1\},\textbackslash{}mathop\{\textbackslash{}mathop\{\ldots{}\}\},\{n\}\_\{l\}.
  Alors la décomposition de P en produit de polynômes irréductibles dans
  ℝ{[}X{]} est

  P(X) = \{a\}\_\{n\}\{ \textbackslash{}mathop\{∏
  \}\}\_\{i=1\}\^{}\{k\}\{(X − \{α\}\_\{ i\})\}\^{}\{\{m\}\_\{i\} \}\{
  \textbackslash{}mathop\{∏ \}\}\_\{j=1\}\^{}\{l\}\{(\{X\}\^{}\{2\} −
  2\textbackslash{}mathrm\{Re\}(\{β\}\_\{ j\})X +
  \textbar{}\{β\}\_\{j\}\{\textbar{}\}\^{}\{2\})\}\^{}\{\{n\}\_\{j\} \}
\end{itemize}

Démonstration Pour les racines, on remarque que
\{P\}\^{}\{(k)\}(\textbackslash{}overline\{α\}) =
\textbackslash{}overline\{\{P\}\^{}\{(k)\}(α)\} si P est à coefficients
réels. Il suffit pour obtenir la décomposition de regrouper les racines
non réelles deux à deux conjuguées en remarquant que (X − β)(X
−\textbackslash{}overline\{β\}) = \{X\}\^{}\{2\} −
2\textbackslash{}mathop\{\textbackslash{}mathrm\{Re\}\}(β)X +
\textbar{}β\{\textbar{}\}\^{}\{2\} La caractérisation des polynômes
irréductibles en découle immédiatement.

\paragraph{1.5.9 Division suivant les puissances croissantes}

Théorème~1.5.17 Soit A ∈ K{[}X{]} et P ∈ K{[}X{]} tel que
P(0)\textbackslash{}mathrel\{≠\}0. Pour tout n ∈ ℕ, il existe un unique
couple (Q,R) de polynômes vérifiant A(X) = P(X)Q(X) +
\{X\}\^{}\{n+1\}R(X) avec \textbackslash{}mathop\{deg\} Q ≤ n.

Démonstration Similaire à celle de la division euclidienne, la valuation
prenant la place du degré.

{[}\href{coursse6.html}{next}{]} {[}\href{coursse4.html}{prev}{]}
{[}\href{coursse4.html\#tailcoursse4.html}{prev-tail}{]}
{[}\href{coursse5.html}{front}{]}
{[}\href{coursch2.html\#coursse5.html}{up}{]}

\end{document}
