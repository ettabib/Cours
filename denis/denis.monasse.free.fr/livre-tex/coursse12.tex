\section{Matrices}

\subsection{Généralités}

\begin{de}
\index{matrice}
$M_K(m,n) = \{(a_{i,j})_{1\leq i\leq m,1\leq j\leq n}\}$ est un K-espace vectoriel de dimension $mn$. Il admet pour base la famille $(E_{k,l})_{1\leq k\leq m,1\leq l\leq n}$ avec 
\[ E_{k,l} = (\delta_i^k\delta_j^l)_{1\leq i\leq m,1\leq j\leq n} \]
\end{de}

\begin{de}
\index{matrice!application linéaire}
Soit $E$ et $F$ deux espaces vectoriels de dimensions finies $n$ et $m$ respectivement et $u \in L(E,F)$. Soit $\mathcal{E} = (e_1,\ldots,e_n)$ une base de $E$ et $\mathcal{F} = (f_1,\ldots,f_m)$ une base de $F$ de base duale $\mathcal{F}^* = (f_1^*,\ldots,f_m^*)$. On définit la matrice de $u$ dans les bases $\mathcal{E}$ et $\mathcal{F}$ comme étant la matrice $\operatorname{Mat}(u,\mathcal{E},\mathcal{F}) = (a_{i,j})_{1\leq i\leq m,1\leq j\leq n}$ construite de façon équivalente par :
\begin{enumerate}
\item $\forall j \in [1,n], u(e_j) = \sum_{i=1}^m a_{i,j}f_i$ 
\item $\forall i \in [1,m], \forall j \in [1,n], a_{i,j} = f_i^*(u(e_j)) = \langle f_i^*|u(e_j)\rangle$
\end{enumerate}
\end{de}

\begin{prop}
L'application $L(E,F) \to M_K(m,n), u \mapsto \operatorname{Mat}(u,\mathcal{E},\mathcal{F})$ est un isomorphisme de K-espaces vectoriels.
\end{prop}

\subsection{Produit matriciel}
\index{matrice!produit}

Soit $A = (a_{i,j}) \in M_K(m,n)$ et $B = (b_{i,j}) \in M_K(n,p)$. On définit $AB = (c_{i,j}) \in M_K(m,p)$ par :
\[ \forall (i,j) \in [1,m] \times [1,p], c_{i,j} = \sum_{k=1}^n a_{i,k}b_{k,j} \]

\begin{thm}
\index{matrice!composition}
Soit $u \in L(E,F)$, $v \in L(F,G)$ où $E$, $F$ et $G$ sont trois espaces vectoriels de dimensions finies admettant des bases $\mathcal{E}$, $\mathcal{F}$ et $\mathcal{G}$. Alors on a :
\[ \operatorname{Mat}(v \circ u,\mathcal{E},\mathcal{G}) = \operatorname{Mat}(v,\mathcal{F},\mathcal{G})\operatorname{Mat}(u,\mathcal{E},\mathcal{F}) \]
\end{thm}

\subsection{Matrices carrées}

\begin{lem}
\index{matrice!base canonique}
Soit $(E_{i,j})$ la base canonique de $M_K(n)$. On a $E_{i,j}E_{k,l} = \delta_j^k E_{i,l}$.
\end{lem}

\begin{prop}
\index{matrice!carrée}
\index{groupe linéaire}
$M_K(n)$ (= $M_K(n,n)$) est une K-algèbre de dimension $n^2$ dont le centre est constitué des matrices scalaires. Le groupe de ses éléments inversibles est noté $GL_K(n)$ (groupe linéaire d'indice $n$).
\end{prop}

\begin{de}
\index{matrice!trace}
Si $A = (a_{i,j}) \in M_K(n)$, on définit la trace de $A$ comme :
\[ \mathrm{tr}(A) = \sum_{i=1}^n a_{i,i} \]
\end{de}

\begin{thm}
\index{trace!propriétés}
Soit $A \in M_K(m,n)$ et $B \in M_K(n,m)$. Alors $\mathrm{tr}(AB) = \mathrm{tr}(BA)$. En particulier, si $A \in M_K(n)$ et $P \in GL_K(n)$, alors $\mathrm{tr}(P^{-1}AP) = \mathrm{tr}(A)$.
\end{thm}

\subsection{Transposée}

\begin{de}
\index{matrice!transposée}
Si $A = (a_{i,j}) \in M_K(m,n)$ on pose ${}^tA = (b_{i,j}) \in M_K(n,m)$ définie par :
\[ \forall (i,j) \in [1,n] \times [1,m], b_{i,j} = a_{j,i} \]
\end{de}

\begin{prop}
\index{transposée!propriétés}
L'application $M \mapsto {}^tM$ est linéaire bijective de $M_K(m,n)$ sur $M_K(n,m)$ et on a ${}^t({}^tM) = M$. Si $M \in M_K(m,n), N \in M_K(n,p)$, alors ${}^t(MN) = {}^tN{}^tM$. Si $M \in M_K(n)$ est inversible, ${}^tM$ aussi et $({}^tM)^{-1} = {}^t(M^{-1})$.
\end{prop}

\begin{thm}
\index{transposée!application linéaire}
Soit $E$ et $F$ deux K-espaces vectoriels de dimensions finies admettant des bases $\mathcal{E}$ et $\mathcal{F}$. Alors :
\[ \operatorname{Mat}({}^tu,\mathcal{F}^*,\mathcal{E}^*) = {}^t\operatorname{Mat}(u,\mathcal{E},\mathcal{F}) \]
\end{thm}

\begin{de}
\index{matrice!symétrique}
\index{matrice!antisymétrique}
Soit $M \in M_K(n)$. On dit que $M$ est symétrique si ${}^tM = M$ et antisymétrique si ${}^tM = -M$.
\end{de}

\subsection{Rang d'une matrice}

\begin{de}
\index{matrice!rang}
Soit $A \in M_K(m,n)$. On appelle rang de $A$ le rang dans $K^m$ de la famille $(c_1,\ldots,c_n)$ de ses vecteurs colonnes.
\end{de}

\begin{thm}
\index{rang!matrice}
Si $u \in L(E,F)$, $\mathcal{E}$ une base de $E$, $\mathcal{F}$ une base de $F$, $A = \operatorname{Mat}(u,\mathcal{E},\mathcal{F})$. Alors $\operatorname{rg}(A) = \operatorname{rg}(u)$.
\end{thm}

\begin{cor}
\index{rang!produit}
$\operatorname{rg}(AB) \leq \min(\operatorname{rg}(A),\operatorname{rg}(B))$
\end{cor}

\begin{thm}
\index{matrice!inversible}
Soit $A \in M_K(n)$. Les propriétés suivantes sont équivalentes :
\begin{enumerate}
\item $A$ est inversible
\item $\operatorname{rg}(A) = n$
\item $\exists B \in M_K(n), AB = I_n$
\item $\exists B \in M_K(n), BA = I_n$
\end{enumerate}
\end{thm}

\subsection{Méthode du pivot}

\begin{de}
\index{pivot!opérations élémentaires}
On définit les opérations élémentaires sur les vecteurs colonnes $(c_1,\ldots,c_n)$ d'une matrice $A \in M_K(m,n)$ :
\begin{enumerate}
\item Ajouter à une colonne $c_j$ une combinaison linéaire des autres vecteurs colonnes :
\[ c_i \leftarrow c_i + \sum_{j\neq i} \lambda_j c_j \]
\item Multiplier la colonne $c_i$ par un scalaire non nul :
\[ c_i \leftarrow \lambda c_i \]
\item Effectuer une permutation $\sigma$ sur les vecteurs colonnes :
\[ (c_1,\ldots,c_n) \leftarrow (c_{\sigma(1)},\ldots,c_{\sigma(n)}) \]
\end{enumerate}
\end{de}

\begin{thm}
\index{pivot!forme échelonnée}
Soit $M \in M_K(n,p)$. Il existe une matrice $M'$ qui se déduit de $M$ par une suite d'opérations élémentaires, de la forme échelonnée réduite. Dans toute telle écriture, $\operatorname{rg}(M)$ est égal au nombre de pivots.
\end{thm}

\subsection{Changement de bases}

\begin{de}
\index{matrice!changement de base}
Soit $\mathcal{E}$ et $\mathcal{E}'$ deux bases de $E$. On définit $P_\mathcal{E}^{\mathcal{E}'}$ comme étant la matrice inversible $(a_{i,j}) \in M_K(n)$ définie par :
\[ e'_j = \sum_{i=1}^n a_{i,j}e_i \]
\end{de}

\begin{thm}
\index{changement de base!propriétés}
\begin{enumerate}
\item $P_{\mathcal{E}'}^\mathcal{E} = (P_\mathcal{E}^{\mathcal{E}'})^{-1}$ et $P_\mathcal{E}^{\mathcal{E}''} = P_\mathcal{E}^{\mathcal{E}'}P_{\mathcal{E}'}^{\mathcal{E}''}$
\item Si $X$ et $X'$ sont les vecteurs colonnes des coordonnées de $x \in E$ dans les bases $\mathcal{E}$ et $\mathcal{E}'$ et $P = P_\mathcal{E}^{\mathcal{E}'}$, on a $X = PX'$
\end{enumerate}
\end{thm}

\subsection{Produit des matrices par blocs}

\begin{thm}
\index{matrice!produit par blocs}
Soit $A$, $B$, $C$, $D$ des matrices de tailles compatibles. Alors :
\[ \begin{pmatrix} A & B \\ C & D \end{pmatrix} \begin{pmatrix} A' & B' \\ C' & D' \end{pmatrix} = \begin{pmatrix} AA' + BC' & AB' + BD' \\ CA' + DC' & CB' + DD' \end{pmatrix} \]
\end{thm}

\begin{thm}
\index{matrice!similitude}
Deux matrices de $M_K(m,n)$ sont équivalentes si et seulement si elles ont même rang. Pour les matrices carrées, la relation de similitude (existence de $P$ inversible telle que $M' = P^{-1}MP$) est plus fine que l'équivalence.
\end{thm}