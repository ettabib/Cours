\documentclass[]{article}
\usepackage[T1]{fontenc}
\usepackage{lmodern}
\usepackage{amssymb,amsmath}
\usepackage{ifxetex,ifluatex}
\usepackage{fixltx2e} % provides \textsubscript
% use upquote if available, for straight quotes in verbatim environments
\IfFileExists{upquote.sty}{\usepackage{upquote}}{}
\ifnum 0\ifxetex 1\fi\ifluatex 1\fi=0 % if pdftex
  \usepackage[utf8]{inputenc}
\else % if luatex or xelatex
  \ifxetex
    \usepackage{mathspec}
    \usepackage{xltxtra,xunicode}
  \else
    \usepackage{fontspec}
  \fi
  \defaultfontfeatures{Mapping=tex-text,Scale=MatchLowercase}
  \newcommand{\euro}{€}
\fi
% use microtype if available
\IfFileExists{microtype.sty}{\usepackage{microtype}}{}
\ifxetex
  \usepackage[setpagesize=false, % page size defined by xetex
              unicode=false, % unicode breaks when used with xetex
              xetex]{hyperref}
\else
  \usepackage[unicode=true]{hyperref}
\fi
\hypersetup{breaklinks=true,
            bookmarks=true,
            pdfauthor={},
            pdftitle={Limites de fonctions},
            colorlinks=true,
            citecolor=blue,
            urlcolor=blue,
            linkcolor=magenta,
            pdfborder={0 0 0}}
\urlstyle{same}  % don't use monospace font for urls
\setlength{\parindent}{0pt}
\setlength{\parskip}{6pt plus 2pt minus 1pt}
\setlength{\emergencystretch}{3em}  % prevent overfull lines
\setcounter{secnumdepth}{0}
 
/* start css.sty */
.cmr-5{font-size:50%;}
.cmr-7{font-size:70%;}
.cmmi-5{font-size:50%;font-style: italic;}
.cmmi-7{font-size:70%;font-style: italic;}
.cmmi-10{font-style: italic;}
.cmsy-5{font-size:50%;}
.cmsy-7{font-size:70%;}
.cmex-7{font-size:70%;}
.cmex-7x-x-71{font-size:49%;}
.msbm-7{font-size:70%;}
.cmtt-10{font-family: monospace;}
.cmti-10{ font-style: italic;}
.cmbx-10{ font-weight: bold;}
.cmr-17x-x-120{font-size:204%;}
.cmsl-10{font-style: oblique;}
.cmti-7x-x-71{font-size:49%; font-style: italic;}
.cmbxti-10{ font-weight: bold; font-style: italic;}
p.noindent { text-indent: 0em }
td p.noindent { text-indent: 0em; margin-top:0em; }
p.nopar { text-indent: 0em; }
p.indent{ text-indent: 1.5em }
@media print {div.crosslinks {visibility:hidden;}}
a img { border-top: 0; border-left: 0; border-right: 0; }
center { margin-top:1em; margin-bottom:1em; }
td center { margin-top:0em; margin-bottom:0em; }
.Canvas { position:relative; }
li p.indent { text-indent: 0em }
.enumerate1 {list-style-type:decimal;}
.enumerate2 {list-style-type:lower-alpha;}
.enumerate3 {list-style-type:lower-roman;}
.enumerate4 {list-style-type:upper-alpha;}
div.newtheorem { margin-bottom: 2em; margin-top: 2em;}
.obeylines-h,.obeylines-v {white-space: nowrap; }
div.obeylines-v p { margin-top:0; margin-bottom:0; }
.overline{ text-decoration:overline; }
.overline img{ border-top: 1px solid black; }
td.displaylines {text-align:center; white-space:nowrap;}
.centerline {text-align:center;}
.rightline {text-align:right;}
div.verbatim {font-family: monospace; white-space: nowrap; text-align:left; clear:both; }
.fbox {padding-left:3.0pt; padding-right:3.0pt; text-indent:0pt; border:solid black 0.4pt; }
div.fbox {display:table}
div.center div.fbox {text-align:center; clear:both; padding-left:3.0pt; padding-right:3.0pt; text-indent:0pt; border:solid black 0.4pt; }
div.minipage{width:100%;}
div.center, div.center div.center {text-align: center; margin-left:1em; margin-right:1em;}
div.center div {text-align: left;}
div.flushright, div.flushright div.flushright {text-align: right;}
div.flushright div {text-align: left;}
div.flushleft {text-align: left;}
.underline{ text-decoration:underline; }
.underline img{ border-bottom: 1px solid black; margin-bottom:1pt; }
.framebox-c, .framebox-l, .framebox-r { padding-left:3.0pt; padding-right:3.0pt; text-indent:0pt; border:solid black 0.4pt; }
.framebox-c {text-align:center;}
.framebox-l {text-align:left;}
.framebox-r {text-align:right;}
span.thank-mark{ vertical-align: super }
span.footnote-mark sup.textsuperscript, span.footnote-mark a sup.textsuperscript{ font-size:80%; }
div.tabular, div.center div.tabular {text-align: center; margin-top:0.5em; margin-bottom:0.5em; }
table.tabular td p{margin-top:0em;}
table.tabular {margin-left: auto; margin-right: auto;}
div.td00{ margin-left:0pt; margin-right:0pt; }
div.td01{ margin-left:0pt; margin-right:5pt; }
div.td10{ margin-left:5pt; margin-right:0pt; }
div.td11{ margin-left:5pt; margin-right:5pt; }
table[rules] {border-left:solid black 0.4pt; border-right:solid black 0.4pt; }
td.td00{ padding-left:0pt; padding-right:0pt; }
td.td01{ padding-left:0pt; padding-right:5pt; }
td.td10{ padding-left:5pt; padding-right:0pt; }
td.td11{ padding-left:5pt; padding-right:5pt; }
table[rules] {border-left:solid black 0.4pt; border-right:solid black 0.4pt; }
.hline hr, .cline hr{ height : 1px; margin:0px; }
.tabbing-right {text-align:right;}
span.TEX {letter-spacing: -0.125em; }
span.TEX span.E{ position:relative;top:0.5ex;left:-0.0417em;}
a span.TEX span.E {text-decoration: none; }
span.LATEX span.A{ position:relative; top:-0.5ex; left:-0.4em; font-size:85%;}
span.LATEX span.TEX{ position:relative; left: -0.4em; }
div.float img, div.float .caption {text-align:center;}
div.figure img, div.figure .caption {text-align:center;}
.marginpar {width:20%; float:right; text-align:left; margin-left:auto; margin-top:0.5em; font-size:85%; text-decoration:underline;}
.marginpar p{margin-top:0.4em; margin-bottom:0.4em;}
.equation td{text-align:center; vertical-align:middle; }
td.eq-no{ width:5%; }
table.equation { width:100%; } 
div.math-display, div.par-math-display{text-align:center;}
math .texttt { font-family: monospace; }
math .textit { font-style: italic; }
math .textsl { font-style: oblique; }
math .textsf { font-family: sans-serif; }
math .textbf { font-weight: bold; }
.partToc a, .partToc, .likepartToc a, .likepartToc {line-height: 200%; font-weight:bold; font-size:110%;}
.chapterToc a, .chapterToc, .likechapterToc a, .likechapterToc, .appendixToc a, .appendixToc {line-height: 200%; font-weight:bold;}
.index-item, .index-subitem, .index-subsubitem {display:block}
.caption td.id{font-weight: bold; white-space: nowrap; }
table.caption {text-align:center;}
h1.partHead{text-align: center}
p.bibitem { text-indent: -2em; margin-left: 2em; margin-top:0.6em; margin-bottom:0.6em; }
p.bibitem-p { text-indent: 0em; margin-left: 2em; margin-top:0.6em; margin-bottom:0.6em; }
.paragraphHead, .likeparagraphHead { margin-top:2em; font-weight: bold;}
.subparagraphHead, .likesubparagraphHead { font-weight: bold;}
.quote {margin-bottom:0.25em; margin-top:0.25em; margin-left:1em; margin-right:1em; text-align:\jmathustify;}
.verse{white-space:nowrap; margin-left:2em}
div.maketitle {text-align:center;}
h2.titleHead{text-align:center;}
div.maketitle{ margin-bottom: 2em; }
div.author, div.date {text-align:center;}
div.thanks{text-align:left; margin-left:10%; font-size:85%; font-style:italic; }
div.author{white-space: nowrap;}
.quotation {margin-bottom:0.25em; margin-top:0.25em; margin-left:1em; }
h1.partHead{text-align: center}
.sectionToc, .likesectionToc {margin-left:2em;}
.subsectionToc, .likesubsectionToc {margin-left:4em;}
.subsubsectionToc, .likesubsubsectionToc {margin-left:6em;}
.frenchb-nbsp{font-size:75%;}
.frenchb-thinspace{font-size:75%;}
.figure img.graphics {margin-left:10%;}
/* end css.sty */

\title{Limites de fonctions}
\author{}
\date{}

\begin{document}
\maketitle

\textbf{Warning: 
requires JavaScript to process the mathematics on this page.\\ If your
browser supports JavaScript, be sure it is enabled.}

\begin{center}\rule{3in}{0.4pt}\end{center}

{[}
{[}
{[}{]}
{[}

\subsubsection{4.4 Limites de fonctions}

Définition~4.4.1 Si E et F sont deux ensembles, on appellera fonction de
E vers F toute application d'une partie X de E (le domaine de définition
de la fonction) dans F. On notera Def~ (f) le
domaine de définition de la fonction f.

\paragraph{4.4.1 Notion de limite suivant une partie}

Définition~4.4.2 Soit E et F deux espaces métriques, A \subset~ F , a
\in\overlineA. Soit f une fonction de E vers F telle
que A \subset~ Def~ (f). On dit que f admet une limite
en a suivant A s'il existe \ell \in F vérifiant les conditions équivalentes

\begin{itemize}
\itemsep1pt\parskip0pt\parsep0pt
\item
  (i) \forall~V \in V (\ell), \\exists~U
  \in V (a),\quad f(U \bigcap A) \subset~ V
\item
  (ii) \forall~~\epsilon \textgreater{} 0,
  \exists~\eta \textgreater{} 0,\quad (x
  \in A\text et d(x,a) \textless{} \eta) \rigtharrow~ d(f(x),\ell)
  \textless{} \epsilon.
\end{itemize}

Démonstration De nouveau, (ii) n'est qu'une reformulation de (i) en
termes de boules~: toute boule est un voisinage, tout voisinage contient
une boule.

Remarque~4.4.1 Sans la condition a \in\overlineA, on
pourrait avoir U \bigcap A = \varnothing~ et la notion deviendrait triviale, tout élément
\ell vérifiant la condition.

Proposition~4.4.1 Si la fonction f admet une limite en a suivant A,
l'élément \ell de E est unique~; on l'appelle la limite de la fonction en a
suivant A. On pose \ell =\
lim\_x\rightarrow~a,x\inAf(x).

Démonstration Si \ell et \ell' conviennent avec \ell\neq~\ell', il existe d'après la
propriété de séparation V ouvert contenant \ell et V ' ouvert contenant \ell'
tels que V \bigcap V ' = \varnothing~. Mais \exists~U \in V (a), f(U \bigcap
A) \subset~ V et \exists~U' \in V (a), f(U' \bigcap A) \subset~ V '. On a
alors f(U \bigcap U' \bigcap A) \subset~ V \bigcap V ' = \varnothing~ alors que U \bigcap U' \bigcap
A\neq~\varnothing~ puisque U \bigcap U' est un voisinage de a et
que a \in\overlineA. C'est absurde.

Remarque~4.4.2 On prendra garde à ne pas introduire le symbole
lim\_x\rightarrow~a,x\inA~f(x) de manière opératoire
avant d'avoir démontré son existence. On remarquera d'autre part que les
notions de convergence et de limites sont purement topologiques
puisqu'on peut les exprimer en terme de voisinages~; elles sont donc
inchangées par changement de distance topologiquement équivalente.

Théorème~4.4.2 Soit U\_0 un ouvert contenant a. Alors f admet
une limite en a suivant A si et seulement si il admet une limite suivant
U\_0 \bigcap A et dans ce cas la limite est la même (on dit que la
notion de limite est une notion locale).

Démonstration Si f(U \bigcap A) \subset~ V , on a à fortiori f(U \bigcap U\_0 \bigcap A)
\subset~ V . Inversement, si f(U \bigcap U\_0 \bigcap A) \subset~ V , U' = U \bigcap
U\_0 est un voisinage de a tel que f(U' \bigcap A) \subset~ V .

Exemple~4.4.1

\begin{enumerate}
\itemsep1pt\parskip0pt\parsep0pt
\item
  Pour E = \overline\mathbb{R}~, A = \mathbb{N}~, a = +\infty~ et f(n) =
  x\_n on retrouve le cas particulier des suites.
\item
  Pour A ={]}a,+\infty~{[}\bigcapDef~ (f) on trouve le cas
  particulier d'une limite à droite (si cela a un sens, c'est à dire si
  a est dans l'adhérence de cet ensemble)~; de même pour les limites à
  gauche.
\item
  Pour A = Def~ (f)
  \diagdown\a\ on trouve le cas important de
  limite quand x tend vers a en étant distinct de a.
\item
  Pour a \in Def~ (f) et A
  = Def~ (f), la seule limite possible est f(a)
  (facile).
\end{enumerate}

\paragraph{4.4.2 Propriétés élémentaires}

Proposition~4.4.3 Si f admet \ell pour limite en a suivant A, alors \ell
\in\overlinef(A).

Démonstration Si V \in V (\ell), il existe U \in V (a) tel que f(U \bigcap A) \subset~ V
(\bigcapf(A)) et comme U \bigcap A\neq~\varnothing~, on a V \bigcap
f(A)\neq~\varnothing~. Donc \ell
\in\overlinef(A).

Remarque~4.4.3 Si f admet \ell pour limite en a suivant A et si A' \subset~ A est
tel que a \in\overlineA', il est clair que f(U \bigcap A') \subset~
f(U \bigcap A) et donc f admet encore \ell comme limite en a suivant A'. La
réciproque est évidemment fausse mais on a

Théorème~4.4.4 Soit A et A' deux parties de E telles que A \cup A'
\subset~ Def (f) et a \in\overlineA~
\bigcap\overlineA'. Alors on a équivalence de

\begin{itemize}
\itemsep1pt\parskip0pt\parsep0pt
\item
  (i) f admet une limite en a suivant A \cup A'
\item
  (ii) f admet une limite suivant A, une limite suivant A' et ces
  limites sont égales.
\end{itemize}

Démonstration D'après la remarque précédente, on a (i) \rigtharrow~(ii).
Inversement supposons que \ell =\
lim\_Af(x) = lim\_A'~f(x). Soit
V un voisinage de \ell. Soit U \in V (a) tel que f(U \bigcap A) \subset~ V et U' \in V (a)
tel que f(U' \bigcap A') \subset~ V . Alors U \bigcap U' est un voisinage de a et f((U \bigcap
U') \bigcap (A \cup A')) \subset~ V (facile). Donc \ell est limite suivant A \cup A'.

Exemple~4.4.2 Une suite (x\_n) converge si et seulement si~les
deux sous suites (x\_2n) et (x\_2n+1) convergent et ont
la même limite. De même, une fonction admet une limite en a si et
seulement si~elle a une limite à gauche et une limite à droite et ces
limites sont égales (à condition que tout cela ait un sens).

\paragraph{4.4.3 Composition des limites}

Théorème~4.4.5 Soit E,F et G trois espaces métriques, f fonction de E
vers F, g une fonction de F vers G. Soit A une partie de E et B une
partie de F. On suppose que A \subset~ Def~ (f), B
\subset~ Def~ (g) et f(A) \subset~ B (si bien que g \cdot f est
définie sur A). Si f admet une limite b en a suivant A et si g admet une
limite \ell en b suivant B, alors g \cdot f admet \ell pour limite en a suivant A.

Démonstration Remarquons que b \in\overlinef(A)
\subset~\overlineB. Soit alors W \in V (\ell). Il existe V \in V
(b) tel que g(V \bigcap B) \subset~ W. Pour ce voisinage V , il existe U \in V (a) tel
que f(U \bigcap A) \subset~ V . Mais on a f(U \bigcap A) \subset~ f(A) \subset~ B et donc f(U \bigcap A) \subset~ V \bigcap
B soit g \cdot f(U \bigcap A) \subset~ W, ce qui achève la démonstration.

Proposition~4.4.6 Soit
E,F\_1,\\ldots,F\_p~
des espaces métriques, F = F\_1 \times⋯ \times
F\_p, p\_i la pro\jmathection de F sur F\_i définie
par
p\_i(y\_1,\\ldots,y\_p~)
= y\_i. Soit f une fonction de E vers F, f\_i =
p\_i \cdot f si bien que f(x) =
(f\_1(x),\\ldots,f\_p~(x)).
Alors f admet une limite \ell en a suivant A si et seulement si~chacune des
f\_i admet une limite \ell\_i en a suivant A. dans ce cas \ell
=
(\ell\_1,\\ldots,\ell\_p~).

Lemme~4.4.7 Pour tout b \in F on a
lim\_y\rightarrow~bp\_i~(y) =
p\_i(b).

Démonstration Soit V \_i un voisinage ouvert de b\_i =
p\_i(b). Alors U = F\_1 \times⋯ \times
V \_i \times⋯ \times F\_p est un ouvert
contenant b tel que p\_i(U) \subset~ V .

Démonstration La condition est nécessaire d'après le théorème de
composition des limites~: si f admet \ell pour limite, alors p\_i \cdot
f admet pour limite p\_i(\ell) que l'on nomme \ell\_i. On a
alors bien entendu, \ell =
(\ell\_1,\\ldots,\ell\_p~).
Réciproquement, supposons que chacune des f\_i admet
\ell\_i pour limite en a suivant A et soit \ell =
(\ell\_1,\\ldots,\ell\_p~).
Soit V un voisinage de \ell. Il existe alors un ouvert élémentaire V
\_1 \times⋯ \times V \_p tel que
(\ell\_1,\\ldots,\ell\_p~)
\subset~ V \_1 \times⋯ \times V \_p \subset~ V . Pour
chaque i, il existe U\_i voisinage de a tel que
f\_i(U\_i \bigcap A) \subset~ V \_i. Soit U = U\_1
\bigcap\\ldots~ \bigcap
U\_p. On a alors f(U \bigcap A) \subset~ V \_1
\times⋯ \times V \_p \subset~ V , ce qui montre que f
a \ell pour limite en a suivant A.

\paragraph{4.4.4 Limites et suites}

Théorème~4.4.8 Soit E et F deux espaces métriques. Alors f admet \ell pour
limite en a suivant A si et seulement si~pour toute suite (a\_n)
d'éléments de A de limite a, la suite (f(a\_n))\_n\in\mathbb{N}~
admet \ell pour limite.

Démonstration Le fait que la condition soit nécessaire résulte du
théorème de composition des limites~: soit V voisinage de \ell~; il existe
U \in V (a) tel que f(U \bigcap A) \subset~ V ~; il existe N \in \mathbb{N}~ tel que n ≥ N \rigtharrow~
a\_n \in U(\bigcapA)~; alors, pour n ≥ N, on a f(a\_n) \in V ,
donc \ell est limite de la suite (f(a\_n)). Supposons maintenant
que f n'admet pas \ell pour limite en a suivant A~; ceci signifie que

\exists~\epsilon \textgreater{} 0,
\forall~~\eta \textgreater{}
0\exists~a' \in A\text tel que
d(a,a') \textless{} \eta\text et d(f(x),\ell) ≥ \epsilon

Pour \eta = 1 \over n+1 , on a donc a\_n \in A tel
que d(a,a\_n) \textless{} 1 \over n+1 avec
d(\ell,f(a\_n)) ≥ \epsilon, d'où une suite d'éléments de A de limite a
telle que la suite (f(a\_n)) n'admet pas \ell pour limite. Ceci
démontre la réciproque par contraposition.

Corollaire~4.4.9 Avec les mêmes notations, f admet une limite en a
suivant A si et seulement si~pour toute suite (a\_n) d'éléments
de A de limite a, la suite (f(a\_n)) converge.

Démonstration La condition est évidemment nécessaire. Pour la
réciproque, il suffit de montrer que la limite de la suite
(f(a\_n)) ne dépend pas de la suite (a\_n)~; or si
(a\_n) et (b\_n) sont deux telles suites, on définit
(c\_n) par c\_2n = a\_n et c\_2n+1 =
b\_n~; cette suite converge vers a, donc la suite
(f(c\_n)) converge et donc ses deux sous suites
(f(a\_n)) et (f(b\_n)) ont la même limite.

Remarque~4.4.4 Ce corollaire permet d'assurer l'existence d'une limite
sans expliciter celle-ci à condition d'avoir un critère de convergence
des suites (par exemple le critère de Cauchy).

{[}
{[}
{[}
{[}

\end{document}
