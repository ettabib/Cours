\documentclass[]{article}
\usepackage[T1]{fontenc}
\usepackage{lmodern}
\usepackage{amssymb,amsmath}
\usepackage{ifxetex,ifluatex}
\usepackage{fixltx2e} % provides \textsubscript
% use upquote if available, for straight quotes in verbatim environments
\IfFileExists{upquote.sty}{\usepackage{upquote}}{}
\ifnum 0\ifxetex 1\fi\ifluatex 1\fi=0 % if pdftex
  \usepackage[utf8]{inputenc}
\else % if luatex or xelatex
  \ifxetex
    \usepackage{mathspec}
    \usepackage{xltxtra,xunicode}
  \else
    \usepackage{fontspec}
  \fi
  \defaultfontfeatures{Mapping=tex-text,Scale=MatchLowercase}
  \newcommand{\euro}{€}
\fi
% use microtype if available
\IfFileExists{microtype.sty}{\usepackage{microtype}}{}
\ifxetex
  \usepackage[setpagesize=false, % page size defined by xetex
              unicode=false, % unicode breaks when used with xetex
              xetex]{hyperref}
\else
  \usepackage[unicode=true]{hyperref}
\fi
\hypersetup{breaklinks=true,
            bookmarks=true,
            pdfauthor={},
            pdftitle={Espaces metriques},
            colorlinks=true,
            citecolor=blue,
            urlcolor=blue,
            linkcolor=magenta,
            pdfborder={0 0 0}}
\urlstyle{same}  % don't use monospace font for urls
\setlength{\parindent}{0pt}
\setlength{\parskip}{6pt plus 2pt minus 1pt}
\setlength{\emergencystretch}{3em}  % prevent overfull lines
\setcounter{secnumdepth}{0}
 
/* start css.sty */
.cmr-5{font-size:50%;}
.cmr-7{font-size:70%;}
.cmmi-5{font-size:50%;font-style: italic;}
.cmmi-7{font-size:70%;font-style: italic;}
.cmmi-10{font-style: italic;}
.cmsy-5{font-size:50%;}
.cmsy-7{font-size:70%;}
.cmex-7{font-size:70%;}
.cmex-7x-x-71{font-size:49%;}
.msbm-7{font-size:70%;}
.cmtt-10{font-family: monospace;}
.cmti-10{ font-style: italic;}
.cmbx-10{ font-weight: bold;}
.cmr-17x-x-120{font-size:204%;}
.cmsl-10{font-style: oblique;}
.cmti-7x-x-71{font-size:49%; font-style: italic;}
.cmbxti-10{ font-weight: bold; font-style: italic;}
p.noindent { text-indent: 0em }
td p.noindent { text-indent: 0em; margin-top:0em; }
p.nopar { text-indent: 0em; }
p.indent{ text-indent: 1.5em }
@media print {div.crosslinks {visibility:hidden;}}
a img { border-top: 0; border-left: 0; border-right: 0; }
center { margin-top:1em; margin-bottom:1em; }
td center { margin-top:0em; margin-bottom:0em; }
.Canvas { position:relative; }
li p.indent { text-indent: 0em }
.enumerate1 {list-style-type:decimal;}
.enumerate2 {list-style-type:lower-alpha;}
.enumerate3 {list-style-type:lower-roman;}
.enumerate4 {list-style-type:upper-alpha;}
div.newtheorem { margin-bottom: 2em; margin-top: 2em;}
.obeylines-h,.obeylines-v {white-space: nowrap; }
div.obeylines-v p { margin-top:0; margin-bottom:0; }
.overline{ text-decoration:overline; }
.overline img{ border-top: 1px solid black; }
td.displaylines {text-align:center; white-space:nowrap;}
.centerline {text-align:center;}
.rightline {text-align:right;}
div.verbatim {font-family: monospace; white-space: nowrap; text-align:left; clear:both; }
.fbox {padding-left:3.0pt; padding-right:3.0pt; text-indent:0pt; border:solid black 0.4pt; }
div.fbox {display:table}
div.center div.fbox {text-align:center; clear:both; padding-left:3.0pt; padding-right:3.0pt; text-indent:0pt; border:solid black 0.4pt; }
div.minipage{width:100%;}
div.center, div.center div.center {text-align: center; margin-left:1em; margin-right:1em;}
div.center div {text-align: left;}
div.flushright, div.flushright div.flushright {text-align: right;}
div.flushright div {text-align: left;}
div.flushleft {text-align: left;}
.underline{ text-decoration:underline; }
.underline img{ border-bottom: 1px solid black; margin-bottom:1pt; }
.framebox-c, .framebox-l, .framebox-r { padding-left:3.0pt; padding-right:3.0pt; text-indent:0pt; border:solid black 0.4pt; }
.framebox-c {text-align:center;}
.framebox-l {text-align:left;}
.framebox-r {text-align:right;}
span.thank-mark{ vertical-align: super }
span.footnote-mark sup.textsuperscript, span.footnote-mark a sup.textsuperscript{ font-size:80%; }
div.tabular, div.center div.tabular {text-align: center; margin-top:0.5em; margin-bottom:0.5em; }
table.tabular td p{margin-top:0em;}
table.tabular {margin-left: auto; margin-right: auto;}
div.td00{ margin-left:0pt; margin-right:0pt; }
div.td01{ margin-left:0pt; margin-right:5pt; }
div.td10{ margin-left:5pt; margin-right:0pt; }
div.td11{ margin-left:5pt; margin-right:5pt; }
table[rules] {border-left:solid black 0.4pt; border-right:solid black 0.4pt; }
td.td00{ padding-left:0pt; padding-right:0pt; }
td.td01{ padding-left:0pt; padding-right:5pt; }
td.td10{ padding-left:5pt; padding-right:0pt; }
td.td11{ padding-left:5pt; padding-right:5pt; }
table[rules] {border-left:solid black 0.4pt; border-right:solid black 0.4pt; }
.hline hr, .cline hr{ height : 1px; margin:0px; }
.tabbing-right {text-align:right;}
span.TEX {letter-spacing: -0.125em; }
span.TEX span.E{ position:relative;top:0.5ex;left:-0.0417em;}
a span.TEX span.E {text-decoration: none; }
span.LATEX span.A{ position:relative; top:-0.5ex; left:-0.4em; font-size:85%;}
span.LATEX span.TEX{ position:relative; left: -0.4em; }
div.float img, div.float .caption {text-align:center;}
div.figure img, div.figure .caption {text-align:center;}
.marginpar {width:20%; float:right; text-align:left; margin-left:auto; margin-top:0.5em; font-size:85%; text-decoration:underline;}
.marginpar p{margin-top:0.4em; margin-bottom:0.4em;}
.equation td{text-align:center; vertical-align:middle; }
td.eq-no{ width:5%; }
table.equation { width:100%; } 
div.math-display, div.par-math-display{text-align:center;}
math .texttt { font-family: monospace; }
math .textit { font-style: italic; }
math .textsl { font-style: oblique; }
math .textsf { font-family: sans-serif; }
math .textbf { font-weight: bold; }
.partToc a, .partToc, .likepartToc a, .likepartToc {line-height: 200%; font-weight:bold; font-size:110%;}
.chapterToc a, .chapterToc, .likechapterToc a, .likechapterToc, .appendixToc a, .appendixToc {line-height: 200%; font-weight:bold;}
.index-item, .index-subitem, .index-subsubitem {display:block}
.caption td.id{font-weight: bold; white-space: nowrap; }
table.caption {text-align:center;}
h1.partHead{text-align: center}
p.bibitem { text-indent: -2em; margin-left: 2em; margin-top:0.6em; margin-bottom:0.6em; }
p.bibitem-p { text-indent: 0em; margin-left: 2em; margin-top:0.6em; margin-bottom:0.6em; }
.paragraphHead, .likeparagraphHead { margin-top:2em; font-weight: bold;}
.subparagraphHead, .likesubparagraphHead { font-weight: bold;}
.quote {margin-bottom:0.25em; margin-top:0.25em; margin-left:1em; margin-right:1em; text-align:justify;}
.verse{white-space:nowrap; margin-left:2em}
div.maketitle {text-align:center;}
h2.titleHead{text-align:center;}
div.maketitle{ margin-bottom: 2em; }
div.author, div.date {text-align:center;}
div.thanks{text-align:left; margin-left:10%; font-size:85%; font-style:italic; }
div.author{white-space: nowrap;}
.quotation {margin-bottom:0.25em; margin-top:0.25em; margin-left:1em; }
h1.partHead{text-align: center}
.sectionToc, .likesectionToc {margin-left:2em;}
.subsectionToc, .likesubsectionToc {margin-left:4em;}
.subsubsectionToc, .likesubsubsectionToc {margin-left:6em;}
.frenchb-nbsp{font-size:75%;}
.frenchb-thinspace{font-size:75%;}
.figure img.graphics {margin-left:10%;}
/* end css.sty */

\title{Espaces metriques}
\author{}
\date{}

\begin{document}
\maketitle

\textbf{Warning: \href{http://www.math.union.edu/locate/jsMath}{jsMath}
requires JavaScript to process the mathematics on this page.\\ If your
browser supports JavaScript, be sure it is enabled.}

\begin{center}\rule{3in}{0.4pt}\end{center}

{[}\href{coursse20.html}{next}{]} {[}\href{coursse18.html}{prev}{]}
{[}\href{coursse18.html\#tailcoursse18.html}{prev-tail}{]}
{[}\hyperref[tailcoursse19.html]{tail}{]}
{[}\href{coursch5.html\#coursse19.html}{up}{]}

\subsubsection{4.2 Espaces métriques}

\paragraph{4.2.1 Distances}

Définition~4.2.1 Soit E un ensemble. On appelle distance sur E toute
application d : E × E → \{ℝ\}\^{}\{+\} vérifiant pour tout x,y,z ∈ E

\begin{itemize}
\itemsep1pt\parskip0pt\parsep0pt
\item
  (i) d(x,y) = 0 \textbackslash{}mathrel\{⇔\} x = y (propriété de
  séparation)
\item
  (ii) d(x,y) = d(y,x) (propriété de symétrie)
\item
  (iii) d(x,z) ≤ d(x,y) + d(y,z) (inégalité triangulaire)
\end{itemize}

On appelle espace métrique un couple (E,d) d'un ensemble E et d'une
distance d sur E.

Proposition~4.2.1 Soit d une distance sur E Alors

\textbackslash{}mathop\{∀\}x,y,z ∈ E, \textbar{}d(x,z) −
d(y,z)\textbar{}≤ d(x,y)

Démonstration On a d(x,z) − d(y,z) ≤ d(x,y) d'après l'inégalité
triangulaire. En échangeant x et y, on a aussi d(y,z) − d(x,z) ≤ d(x,y),
d'où le résultat.

Exemple~4.2.1 Sur tout ensemble, d(x,y) = \textbackslash{}left
\textbackslash{}\{ \textbackslash{}cases\{ 1\&si
x\textbackslash{}mathrel\{≠\}y \textbackslash{}cr 0\&si x = y
\textbackslash{}cr \} \textbackslash{}right . est une distance sur E
appelée la distance discrète. Sur K = ℝ ou K = ℂ, d(x,y) = \textbar{}x −
y\textbar{} est une distance appelée la distance usuelle. Sur
\{K\}\^{}\{n\} on trouve classiquement trois distances utiles
\{d\}\_\{1\}(x,y) =\{\textbackslash{}mathop\{ \textbackslash{}mathop\{∑
\}\} \}\_\{i\}\textbar{}\{x\}\_\{i\} − \{y\}\_\{i\}\textbar{},
\{d\}\_\{2\}(x,y) =
\textbackslash{}sqrt\{\{\textbackslash{}mathop\{\textbackslash{}mathop\{∑
\}\} \}\_\{i\}\textbar{}\{x\}\_\{i\} −
\{y\}\_\{i\}\{\textbar{}\}\^{}\{2\}\} et \{d\}\_\{∞\}(x,y)
=\{\textbackslash{}mathop\{ max\}\}\_\{i\}\textbar{}\{x\}\_\{i\} −
\{y\}\_\{i\}\textbar{} si x =
(\{x\}\_\{1\},\textbackslash{}mathop\{\textbackslash{}mathop\{\ldots{}\}\},\{x\}\_\{n\})
et y =
(\{y\}\_\{1\},\textbackslash{}mathop\{\textbackslash{}mathop\{\ldots{}\}\},\{y\}\_\{n\}).

Définition~4.2.2 On appelle boule ouverte de centre a de rayon r
\textgreater{} 0~: B(a,r) = \textbackslash{}\{x ∈
E\textbackslash{}mathrel\{∣\}d(a,x) \textless{} r\textbackslash{}\}.

On appelle boule fermée de centre a de rayon r \textgreater{} 0~:
B'(a,r) = \textbackslash{}\{x ∈ E\textbackslash{}mathrel\{∣\}d(a,x) ≤
r\textbackslash{}\}.

On appelle sphère de centre a de rayon r \textgreater{} 0~: S(a,r) =
\textbackslash{}\{x ∈ E\textbackslash{}mathrel\{∣\}d(a,x) =
r\textbackslash{}\}

Définition~4.2.3 Soit (E,d) un espace métrique et \{d\}\_\{F\} la
restriction de d à F × F. Alors \{d\}\_\{F\} est encore une distance sur
F appelée la distance induite par d.

Remarque~4.2.1 On a clairement \{B\}\_\{\{d\}\_\{F\}\}(a,r) =
\{B\}\_\{d\}(a,r) ∩ F et le résultat similaire pour les boules fermées,
si a ∈ F.

Définition~4.2.4 Soit
(\{E\}\_\{1\},\{d\}\_\{1\}),\textbackslash{}mathop\{\textbackslash{}mathop\{\ldots{}\}\},(\{E\}\_\{k\},\{d\}\_\{k\})
des espaces métriques. Soit E = \{E\}\_\{1\}
×\textbackslash{}mathrel\{⋯\} × \{E\}\_\{k\}. On définit alors sur E une
distance produit par d(x,y) =\{\textbackslash{}mathop\{
max\}\}\_\{i\}\{d\}\_\{i\}(\{x\}\_\{i\},\{y\}\_\{i\}) si x =
(\{x\}\_\{1\},\textbackslash{}mathop\{\textbackslash{}mathop\{\ldots{}\}\},\{x\}\_\{k\})
et y =
(\{y\}\_\{1\},\textbackslash{}mathop\{\textbackslash{}mathop\{\ldots{}\}\},\{y\}\_\{k\}).

Définition~4.2.5

\begin{itemize}
\itemsep1pt\parskip0pt\parsep0pt
\item
  (i) Soit x ∈ E et A ⊂ E, A\textbackslash{}mathrel\{≠\}∅. On appelle
  distance de x à A le réel d(x,A) =\textbackslash{}mathop\{ inf\}
  \textbackslash{}\{d(x,a)\textbackslash{}mathrel\{∣\}a ∈
  A\textbackslash{}\}
\item
  (ii) A,B ⊂ E non vides. On appelle distance de A et B le réel d(A,B)
  =\textbackslash{}mathop\{ inf\}
  \textbackslash{}\{d(a,b)\textbackslash{}mathrel\{∣\}a ∈ A,b ∈
  B\textbackslash{}\}
\item
  (iii) On appelle diamètre de A ⊂ E, A\textbackslash{}mathrel\{≠\}∅, le
  nombre δ(A) =\textbackslash{}mathop\{
  sup\}\textbackslash{}\{d(a,a')\textbackslash{}mathrel\{∣\}a,a' ∈
  A\textbackslash{}\} ∈ ℝ ∪\textbackslash{}\{ + ∞\textbackslash{}\}~; on
  dit que A est bornée si δ(A) \textless{} +∞.
\end{itemize}

Définition~4.2.6 Soit (E,d) et (F,δ) deux espaces métriques. On appelle
isométrie de E sur F toute application f : E → F bijective qui conserve
la distance~:

\textbackslash{}mathop\{∀\}x,y ∈ E, δ(f(x),f(y)) = d(x,y)

\paragraph{4.2.2 Topologie définie par une distance}

Définition~4.2.7 Soit (E,d) un espace métrique. On appelle topologie
définie sur E par la distance d l'ensemble des parties U de E (les
ouverts de la topologie) vérifiant

\textbackslash{}mathop\{∀\}x ∈ U, \textbackslash{}mathop\{∃\}r
\textgreater{} 0,\textbackslash{}quad B(x,r) ⊂ U

Démonstration C'est bien une topologie~: clairement E et ∅ sont des
ouverts~; si U et U' sont des ouverts et x ∈ U ∩ U', il existe r
\textgreater{} 0 et r' \textgreater{} 0 tels que B(x,r) ⊂ U et B(x,r') ⊂
U' et alors \{r\}\_\{0\} =\textbackslash{}mathop\{ min\}(r,r')
\textgreater{} 0 est tel que B(x,\{r\}\_\{0\}) ⊂ U ∩ U'. Si les
\{U\}\_\{i\}, i ∈ I sont des ouverts, soit x
∈\{\textbackslash{}mathop\{\textbackslash{}mathop\{⋃ \}\}
\}\_\{i∈I\}\{U\}\_\{i\}. Il existe \{i\}\_\{0\} tel que x ∈
\{U\}\_\{\{i\}\_\{0\}\} puis r \textgreater{} 0 tel que B(x,r) ⊂
\{U\}\_\{\{i\}\_\{0\}\}. On a alors B(x,r)
⊂\{\textbackslash{}mathop\{\textbackslash{}mathop\{⋃ \}\}
\}\_\{i∈I\}\{U\}\_\{i\}.

Proposition~4.2.2 Dans un espace métrique, toute boule ouverte est un
ouvert, toute boule fermée est un fermé.

Démonstration Soit x ∈ B(a,r) et ρ = r − d(a,x) \textgreater{} 0. Si y ∈
B(x,ρ), on a d(a,y) ≤ d(a,x) + d(x,y) \textless{} d(a,x) + ρ = r soit
B(x,ρ) ⊂ B(a,r). De même on montre que si
x\textbackslash{}mathrel\{∉\}B'(a,r) et si ρ = d(a,x) − r \textgreater{}
0, alors B(x,ρ) ⊂ E ∖ B'(a,r). Donc E ∖ B'(a,r) est ouvert et B'(a,r)
est fermé.

Remarque~4.2.2

\begin{itemize}
\itemsep1pt\parskip0pt\parsep0pt
\item
  (i) V ∈ V (a) \textbackslash{}mathrel\{⇔\}
  \textbackslash{}mathop\{∃\}r \textgreater{} 0, B(a,r) ⊂ V
\item
  (ii) a ∈ \{A\}\^{}\{o\} \textbackslash{}mathrel\{⇔\}
  \textbackslash{}mathop\{∃\}r \textgreater{} 0, B(a,r) ⊂ A
\item
  (iii) \textbackslash{}overline\{A\} = \textbackslash{}\{x ∈
  E\textbackslash{}mathrel\{∣\}\textbackslash{}mathop\{∀\}r
  \textgreater{} 0, B(x,r) ∩
  A\textbackslash{}mathrel\{≠\}∅\textbackslash{}\}
\item
  (iv) \textbackslash{}mathop\{\textbackslash{}mathrm\{Fr\}\}(A) =
  \textbackslash{}\{x ∈
  E\textbackslash{}mathrel\{∣\}\textbackslash{}mathop\{∀\}r
  \textgreater{} 0, B(x,r) ∩
  A\textbackslash{}mathrel\{≠\}∅\textbackslash{}text\{ et \}B(x,r)
  ∩cA\textbackslash{}mathrel\{≠\}∅\textbackslash{}\}
\end{itemize}

Proposition~4.2.3 Soit (E,d) un espace métrique et F ⊂ E. Alors la
topologie induite sur F est la topologie définie par la distance
\{d\}\_\{F\}.

Démonstration On remarque que si a ∈ F, \{B\}\_\{\{d\}\_\{F\}\}(a,r) =
\{B\}\_\{d\}(a,r) ∩ F. Soit V un ouvert pour la topologie induite, soit
U ouvert de E tel que V = U ∩ F. On a a ∈ U, donc il existe r
\textgreater{} 0 tel que \{B\}\_\{d\}(a,r) ⊂ U. On a alors
\{B\}\_\{\{d\}\_\{F\}\}(a,r) = \{B\}\_\{d\}(a,r) ∩ F ⊂ U ∩ F ⊂ U.
Inversement, soit V un ouvert pour la distance \{d\}\_\{F\}. Pour tout x
∈ V , il existe \{r\}\_\{x\} \textgreater{} 0 tel que
\{B\}\_\{\{d\}\_\{F\}\}(x,\{r\}\_\{x\}) ⊂ V . On a alors V
=\{\textbackslash{}mathop\{ \textbackslash{}mathop\{⋃ \}\} \}\_\{x∈V
\}\{B\}\_\{\{d\}\_\{F\}\}(x,\{r\}\_\{x\}) (cette réunion contient V de
manière évidente et est contenue dans V car réunion de parties de V ).
On pose alors U =\{\textbackslash{}mathop\{ \textbackslash{}mathop\{⋃
\}\} \}\_\{x∈V \}\{B\}\_\{d\}(x,\{r\}\_\{x\}). C'est un ouvert de E et
on a V = U ∩ F.

Remarque~4.2.3 Ceci montre que la topologie définie par la distance
\{d\}\_\{F\} ne dépend que de la topologie sur E et pas vraiment de la
distance d. Montrons de même que la topologie définie par la distance
produit ne dépend que des topologies sur les espaces et pas des
distances elles-mêmes

Proposition~4.2.4 Soit (\{E\}\_\{1\},\{d\}\_\{1\}) et
(\{E\}\_\{2\},\{d\}\_\{2\}) deux espaces métriques et (\{E\}\_\{1\} ×
\{E\}\_\{2\},δ) l'espace métrique produit. Soit U ⊂ \{E\}\_\{1\} ×
\{E\}\_\{2\}. Alors U est ouvert si et seulement si~

\textbackslash{}mathop\{∀\}(\{a\}\_\{1\},\{a\}\_\{2\}) ∈ U,
\textbackslash{}mathop\{∃\}\{V \}\_\{1\} ∈ V (\{a\}\_\{1\}),
\textbackslash{}mathop\{∃\}\{V \}\_\{2\} ∈ V
(\{a\}\_\{2\}),\textbackslash{}quad \{V \}\_\{1\} × \{V \}\_\{2\} ⊂ U

Démonstration Supposons que U est ouvert pour la distance produit. Si
(\{a\}\_\{1\},\{a\}\_\{2\}) ∈ U, il existe r \textgreater{} 0 tel que
\{B\}\_\{δ\}((\{a\}\_\{1\},\{a\}\_\{2\}),r) ⊂ U. Mais on a

\textbackslash{}begin\{eqnarray*\}\{
B\}\_\{δ\}((\{a\}\_\{1\},\{a\}\_\{2\}),r)\& =\&
\textbackslash{}\{(\{x\}\_\{1\},\{x\}\_\{2\})\textbackslash{}mathrel\{∣\}\textbackslash{}mathop\{max\}(\{d\}\_\{1\}(\{x\}\_\{1\},\{a\}\_\{1\}),\{d\}\_\{2\}(\{a\}\_\{2\},\{r\}\_\{2\}))
\textless{} r\textbackslash{}\}\%\& \textbackslash{}\textbackslash{} \&
=\& \{B\}\_\{\{d\}\_\{1\}\}(\{a\}\_\{1\},r) ×
\{B\}\_\{\{d\}\_\{2\}\}(\{a\}\_\{2\},r) \%\&
\textbackslash{}\textbackslash{} \textbackslash{}end\{eqnarray*\}

et donc \{V \}\_\{1\} = \{B\}\_\{\{d\}\_\{1\}\}(\{a\}\_\{1\},r) et \{V
\}\_\{2\} = \{B\}\_\{\{d\}\_\{2\}\}(\{a\}\_\{2\},r) sont des voisinages
de \{a\}\_\{1\} et \{a\}\_\{2\} tels que \{V \}\_\{1\} × \{V \}\_\{2\} ⊂
U. Inversement, si U vérifie cette propriété, soit
(\{a\}\_\{1\},\{a\}\_\{2\}) ∈ U et soit \{V \}\_\{1\} ∈ V
(\{a\}\_\{1\}), \{V \}\_\{2\} ∈ V (\{a\}\_\{2\}) tels que \{V \}\_\{1\}
× \{V \}\_\{2\} ⊂ U. Il existe \{r\}\_\{1\} \textgreater{} 0 et
\{r\}\_\{2\} \textgreater{} 0 tels que
\{B\}\_\{\{d\}\_\{i\}\}(\{a\}\_\{i\},\{r\}\_\{i\}) ⊂ \{V \}\_\{i\}. Soit
r =\textbackslash{}mathop\{ min\}(\{r\}\_\{1\},\{r\}\_\{2\})
\textgreater{} 0. On a

\textbackslash{}begin\{eqnarray*\}\{
B\}\_\{δ\}((\{a\}\_\{1\},\{a\}\_\{2\}),r)\& =\&
\{B\}\_\{\{d\}\_\{1\}\}(\{a\}\_\{1\},r) ×
\{B\}\_\{\{d\}\_\{2\}\}(\{a\}\_\{2\},r) \%\&
\textbackslash{}\textbackslash{} \& ⊂\&
\{B\}\_\{\{d\}\_\{1\}\}(\{a\}\_\{1\},\{r\}\_\{1\}) ×
\{B\}\_\{\{d\}\_\{2\}\}(\{a\}\_\{2\},\{r\}\_\{2\}) ⊂ \{V \}\_\{1\} × \{V
\}\_\{2\} ⊂ U\%\& \textbackslash{}\textbackslash{}
\textbackslash{}end\{eqnarray*\}

donc U est un ouvert pour δ, ce qui achève la démonstration.

Remarque~4.2.4 En particulier, si \{U\}\_\{1\} et \{U\}\_\{2\} sont des
ouverts de \{E\}\_\{1\} et \{E\}\_\{2\}, alors \{U\}\_\{1\} ×
\{U\}\_\{2\} est un ouvert de \{E\}\_\{1\} × \{E\}\_\{2\}~; un tel
ouvert sera dit ouvert élémentaire.

\paragraph{4.2.3 Points isolés, points d'accumulation}

Soit toujours F une partie de E et x ∈\textbackslash{}overline\{F\}. On
sait que \textbackslash{}mathop\{∀\}V ∈ V (x) V ∩
F\textbackslash{}mathrel\{≠\}∅. On a alors deux possibilités suivant que
V ∩ F peut être réduit à \textbackslash{}\{x\textbackslash{}\} ou non.

Définition~4.2.8

\begin{itemize}
\itemsep1pt\parskip0pt\parsep0pt
\item
  (i) On dit que x ∈ F est point isolé de F, s'il existe V voisinage de
  x dans E tel que V ∩ F = \textbackslash{}\{x\textbackslash{}\}
\item
  (ii) On dit que x ∈ E est point d'accumulation de F si pour tout
  voisinage V de x dans E, V ∩ F
  ∖\textbackslash{}\{x\textbackslash{}\}\textbackslash{}mathrel\{≠\}∅.
\end{itemize}

Théorème~4.2.5 Soit E un espace métrique.

\begin{itemize}
\itemsep1pt\parskip0pt\parsep0pt
\item
  (i) x ∈ F est point isolé de F si et seulement
  si~\textbackslash{}\{x\textbackslash{}\} est ouvert dans F
\item
  (ii) x ∈ E est point d'accumulation de F si et seulement si~pour tout
  voisinage V de x dans E, V ∩ F est infini.
\end{itemize}

Démonstration (i) est tout à fait élémentaire et résulte de la
définition de la topologie induite. En ce qui concerne (ii), la partie (
⇐) est évidente. Montrons donc la partie ( ⇒). Soit x un point
d'accumulation de F, V un voisinage de x et r \textgreater{} 0 tel que
B(x,r) ⊂ V . Alors (B(x,r) ∖\textbackslash{}\{x\textbackslash{}\}) ∩
F\textbackslash{}mathrel\{≠\}∅. Soit \{x\}\_\{1\} ∈ (B(x,r)
∖\textbackslash{}\{x\textbackslash{}\}) ∩ F. Si \{x\}\_\{n\} est supposé
construit, on pose \{r\}\_\{n\} = d(x,\{x\}\_\{n\}) \textgreater{} 0 et
on choisit \{x\}\_\{n+1\} ∈ (B(x,\{r\}\_\{n\})
∖\textbackslash{}\{x\textbackslash{}\}) ∩
F\textbackslash{}mathrel\{≠\}∅. Alors la suite (d(x,\{x\}\_\{n\})) est
strictement décroissante, ce qui montre que les \{x\}\_\{n\} sont deux à
deux distincts. Ils sont tous dans F et dans B(x,r) donc dans V .

\paragraph{4.2.4 Propriété de séparation}

Théorème~4.2.6 Soit E un espace métrique, a et b deux points distincts
de E. Alors il existe U ouvert contenant a et V ouvert contenant b tels
que U ∩ V = ∅.

Démonstration Soit r =\{ 1 \textbackslash{}over 3\} d(a,b), U = B(a,r)
et V = B(b,r) conviennent.

Corollaire~4.2.7 Soit E un espace métrique et Δ =
\textbackslash{}\{(x,x) ∈ E × E\textbackslash{}\}. Alors Δ est fermée
dans E × E.

Démonstration Soit (a,b) ∈ E × E ∖ Δ. On a donc
a\textbackslash{}mathrel\{≠\}b. Il existe U ouvert contenant a et V
ouvert contenant b tels que U ∩ V = ∅. Alors U × V est un ouvert de E ×
E (élémentaire) et (U × V ) ∩ Δ = ∅, soit U × V ⊂ E × E ∖ Δ. Donc E × E
∖ Δ est voisinage de tous ses points et il est ouvert. Donc Δ est
fermée.

\paragraph{4.2.5 Changement de distances}

Définition~4.2.9 Soit E un ensemble. On dit que deux distances
\{d\}\_\{1\} et \{d\}\_\{2\} sur E sont topologiquement équivalentes si
elles définissent la même topologie (il s'agit clairement d'une relation
d'équivalence).

Théorème~4.2.8 Soit E un ensemble, d et d' deux distances sur E. Ces
distances sont topologiquement équivalentes si et seulement si~elles
vérifient

\begin{itemize}
\itemsep1pt\parskip0pt\parsep0pt
\item
  (i) \textbackslash{}mathop\{∀\}a ∈ E, \textbackslash{}mathop\{∀\}r
  \textgreater{} 0, \textbackslash{}mathop\{∃\}r' \textgreater{}
  0,\textbackslash{}quad \{B\}\_\{d'\}(a,r') ⊂ \{B\}\_\{d\}(a,r)
\item
  (ii) \textbackslash{}mathop\{∀\}a ∈ E, \textbackslash{}mathop\{∀\}r'
  \textgreater{} 0, \textbackslash{}mathop\{∃\}r \textgreater{}
  0,\textbackslash{}quad \{B\}\_\{d\}(a,r) ⊂ \{B\}\_\{d'\}(a,r')
\end{itemize}

Démonstration Ces conditions sont évidemment nécessaires puisque les
boules ouvertes pour d doivent être des ouverts pour d' et
réciproquement. Supposons maintenant que (i) est vérifiée et soit U un
ouvert pour d. Soit a ∈ U. Il existe r \textgreater{} 0 tel que
\{B\}\_\{d\}(a,r) ⊂ U. Alors \textbackslash{}mathop\{∃\}r'
\textgreater{} 0,\textbackslash{}quad \{B\}\_\{d'\}(a,r') ⊂
\{B\}\_\{d\}(a,r) ⊂ U. On en déduit que U est ouvert pour d', donc
\{T\}\_\{d\} ⊂\{T\}\_\{d'\}. De même (ii) traduit l'inclusion
\{T\}\_\{d'\} ⊂\{T\}\_\{d\}.

Définition~4.2.10 Soit E un ensemble. On dit que deux distances
\{d\}\_\{1\} et \{d\}\_\{2\} sur E sont équivalentes s'il existe α et β
strictement positifs tels que

\textbackslash{}mathop\{∀\}x,y ∈ E,\textbackslash{}quad
α\{d\}\_\{1\}(x,y) ≤ \{d\}\_\{2\}(x,y) ≤ β\{d\}\_\{1\}(x,y)

Proposition~4.2.9 Deux distances équivalentes sont topologiquement
équivalentes.

Démonstration On a \{d\}\_\{1\}(a,x) \textless{}\{ r
\textbackslash{}over β\} ⇒ \{d\}\_\{2\}(x,y) \textless{} r soit
\{B\}\_\{\{d\}\_\{1\}\}(a,\{ r \textbackslash{}over β\} ) ⊂
\{B\}\_\{\{d\}\_\{2\}\}(a,r). De même \{B\}\_\{\{d\}\_\{2\}\}(a,αr) ⊂
\{B\}\_\{\{d\}\_\{1\}\}(a,r).

Remarque~4.2.5 Soit d une distance sur E et posons d'(x,y)
=\textbackslash{}mathop\{ min\}(1,d(x,y)). On vérifie facilement que d
est une distance sur E, que d et d' sont topologiquement équivalentes
(\{B\}\_\{d\}(a,r) ⊂ \{B\}\_\{d'\}(a,r) et
\{B\}\_\{d'\}(a,\textbackslash{}mathop\{min\}(\{ 1 \textbackslash{}over
2\} ,r)) ⊂ \{B\}\_\{d\}(a,r)). Mais en général, d et d' ne sont pas
équivalentes (d' est toujours bornée alors que d ne l'est pas en
général).

\paragraph{4.2.6 La droite numérique achevée}

On pose \textbackslash{}overline\{ℝ\} = ℝ
∪\textbackslash{}\{−∞,+∞\textbackslash{}\} muni de la relation d'ordre
évidente. Les intervalles ouverts sont donc les intervalles de la forme

\begin{itemize}
\itemsep1pt\parskip0pt\parsep0pt
\item
  (i) I ={]}a,b{[}= \textbackslash{}\{x ∈ ℝ\textbackslash{}mathrel\{∣\}a
  \textless{} x \textless{} b\textbackslash{}\} pour a,b
  ∈\textbackslash{}overline\{ℝ\}
\item
  (ii) I ={]}a,+∞{]} = \textbackslash{}\{x
  ∈\textbackslash{}overline\{ℝ\}\textbackslash{}mathrel\{∣\}a
  \textless{} x\textbackslash{}\} ou I = {[}−∞,a{[}= \textbackslash{}\{x
  ∈\textbackslash{}overline\{ℝ\}\textbackslash{}mathrel\{∣\}x
  \textless{} a\textbackslash{}\}
\item
  (iii) I = {[}−∞,+∞{]} = \textbackslash{}overline\{ℝ\}
\end{itemize}

Comme sur ℝ, ces intervalles ouverts engendrent une topologie appelée la
topologie usuelle de \textbackslash{}overline\{ℝ\}. On a alors

\begin{itemize}
\item
  (i) si a ∈ ℝ, V ∈ V (a) \textbackslash{}mathrel\{⇔\}
  \textbackslash{}mathop\{∃\}ε \textgreater{} 0,\textbackslash{}quad
  {]}a − ε,a + ε{[}⊂ V
\item
  (ii) si a = +∞,

  V ∈ V (+∞) \textbackslash{}mathrel\{⇔\} \textbackslash{}mathop\{∃\}A
  \textgreater{} 0,\textbackslash{}quad {]}A,+∞{]} ⊂ V
\item
  (iii) si a = −∞,

  V ∈ V (−∞) \textbackslash{}mathrel\{⇔\} \textbackslash{}mathop\{∃\}A
  \textless{} 0,\textbackslash{}quad {[}−∞,A{[}⊂ V
\end{itemize}

Théorème~4.2.10 La topologie usuelle sur \textbackslash{}overline\{ℝ\}
est définie par une distance.

Démonstration Soit φ : \textbackslash{}overline\{ℝ\} → {[}−1,1{]}
définie par φ(x) = \textbackslash{}left \textbackslash{}\{
\textbackslash{}cases\{ \{ x \textbackslash{}over
1+\textbar{}x\textbar{}\} \&si x ∈ ℝ \textbackslash{}cr 1 \&si x = +∞
\textbackslash{}cr −1 \&si x = −∞ \textbackslash{}cr \}
\textbackslash{}right .. L'application φ est une bijection strictement
croissante donc respecte les intervalles ouverts, donc les topologies
usuelles~: si U ⊂\textbackslash{}overline\{ℝ\}, U est ouvert dans
\textbackslash{}overline\{ℝ\} si et seulement si~φ(U) est ouvert dans
{[}−1,1{]}~; comme la topologie sur {[}−1,1{]} est définie par la
distance \textbar{}x − y\textbar{}, la topologie sur
\textbackslash{}overline\{ℝ\} est définie par la distance d(x,y) =
\textbar{}φ(x) − φ(y)\textbar{} (pour cette distance, φ devient une
isométrie).

Remarque~4.2.6 On vérifie immédiatement que la topologie usuelle de
\textbackslash{}overline\{ℝ\} induit sur ℝ la topologie usuelle de ℝ.

{[}\href{coursse20.html}{next}{]} {[}\href{coursse18.html}{prev}{]}
{[}\href{coursse18.html\#tailcoursse18.html}{prev-tail}{]}
{[}\href{coursse19.html}{front}{]}
{[}\href{coursch5.html\#coursse19.html}{up}{]}

\end{document}
