\documentclass[]{article}
\usepackage[T1]{fontenc}
\usepackage{lmodern}
\usepackage{amssymb,amsmath}
\usepackage{ifxetex,ifluatex}
\usepackage{fixltx2e} % provides \textsubscript
% use upquote if available, for straight quotes in verbatim environments
\IfFileExists{upquote.sty}{\usepackage{upquote}}{}
\ifnum 0\ifxetex 1\fi\ifluatex 1\fi=0 % if pdftex
  \usepackage[utf8]{inputenc}
\else % if luatex or xelatex
  \ifxetex
    \usepackage{mathspec}
    \usepackage{xltxtra,xunicode}
  \else
    \usepackage{fontspec}
  \fi
  \defaultfontfeatures{Mapping=tex-text,Scale=MatchLowercase}
  \newcommand{\euro}{€}
\fi
% use microtype if available
\IfFileExists{microtype.sty}{\usepackage{microtype}}{}
\usepackage{graphicx}
% Redefine \includegraphics so that, unless explicit options are
% given, the image width will not exceed the width of the page.
% Images get their normal width if they fit onto the page, but
% are scaled down if they would overflow the margins.
\makeatletter
\def\ScaleIfNeeded{%
  \ifdim\Gin@nat@width>\linewidth
    \linewidth
  \else
    \Gin@nat@width
  \fi
}
\makeatother
\let\Oldincludegraphics\includegraphics
{%
 \catcode`\@=11\relax%
 \gdef\includegraphics{\@ifnextchar[{\Oldincludegraphics}{\Oldincludegraphics[width=\ScaleIfNeeded]}}%
}%
\ifxetex
  \usepackage[setpagesize=false, % page size defined by xetex
              unicode=false, % unicode breaks when used with xetex
              xetex]{hyperref}
\else
  \usepackage[unicode=true]{hyperref}
\fi
\hypersetup{breaklinks=true,
            bookmarks=true,
            pdfauthor={},
            pdftitle={Developpements asymptotiques et analyse numerique},
            colorlinks=true,
            citecolor=blue,
            urlcolor=blue,
            linkcolor=magenta,
            pdfborder={0 0 0}}
\urlstyle{same}  % don't use monospace font for urls
\setlength{\parindent}{0pt}
\setlength{\parskip}{6pt plus 2pt minus 1pt}
\setlength{\emergencystretch}{3em}  % prevent overfull lines
\setcounter{secnumdepth}{0}
 
/* start css.sty */
.cmr-5{font-size:50%;}
.cmr-7{font-size:70%;}
.cmmi-5{font-size:50%;font-style: italic;}
.cmmi-7{font-size:70%;font-style: italic;}
.cmmi-10{font-style: italic;}
.cmsy-5{font-size:50%;}
.cmsy-7{font-size:70%;}
.cmex-7{font-size:70%;}
.cmex-7x-x-71{font-size:49%;}
.msbm-7{font-size:70%;}
.cmtt-10{font-family: monospace;}
.cmti-10{ font-style: italic;}
.cmbx-10{ font-weight: bold;}
.cmr-17x-x-120{font-size:204%;}
.cmsl-10{font-style: oblique;}
.cmti-7x-x-71{font-size:49%; font-style: italic;}
.cmbxti-10{ font-weight: bold; font-style: italic;}
p.noindent { text-indent: 0em }
td p.noindent { text-indent: 0em; margin-top:0em; }
p.nopar { text-indent: 0em; }
p.indent{ text-indent: 1.5em }
@media print {div.crosslinks {visibility:hidden;}}
a img { border-top: 0; border-left: 0; border-right: 0; }
center { margin-top:1em; margin-bottom:1em; }
td center { margin-top:0em; margin-bottom:0em; }
.Canvas { position:relative; }
li p.indent { text-indent: 0em }
.enumerate1 {list-style-type:decimal;}
.enumerate2 {list-style-type:lower-alpha;}
.enumerate3 {list-style-type:lower-roman;}
.enumerate4 {list-style-type:upper-alpha;}
div.newtheorem { margin-bottom: 2em; margin-top: 2em;}
.obeylines-h,.obeylines-v {white-space: nowrap; }
div.obeylines-v p { margin-top:0; margin-bottom:0; }
.overline{ text-decoration:overline; }
.overline img{ border-top: 1px solid black; }
td.displaylines {text-align:center; white-space:nowrap;}
.centerline {text-align:center;}
.rightline {text-align:right;}
div.verbatim {font-family: monospace; white-space: nowrap; text-align:left; clear:both; }
.fbox {padding-left:3.0pt; padding-right:3.0pt; text-indent:0pt; border:solid black 0.4pt; }
div.fbox {display:table}
div.center div.fbox {text-align:center; clear:both; padding-left:3.0pt; padding-right:3.0pt; text-indent:0pt; border:solid black 0.4pt; }
div.minipage{width:100%;}
div.center, div.center div.center {text-align: center; margin-left:1em; margin-right:1em;}
div.center div {text-align: left;}
div.flushright, div.flushright div.flushright {text-align: right;}
div.flushright div {text-align: left;}
div.flushleft {text-align: left;}
.underline{ text-decoration:underline; }
.underline img{ border-bottom: 1px solid black; margin-bottom:1pt; }
.framebox-c, .framebox-l, .framebox-r { padding-left:3.0pt; padding-right:3.0pt; text-indent:0pt; border:solid black 0.4pt; }
.framebox-c {text-align:center;}
.framebox-l {text-align:left;}
.framebox-r {text-align:right;}
span.thank-mark{ vertical-align: super }
span.footnote-mark sup.textsuperscript, span.footnote-mark a sup.textsuperscript{ font-size:80%; }
div.tabular, div.center div.tabular {text-align: center; margin-top:0.5em; margin-bottom:0.5em; }
table.tabular td p{margin-top:0em;}
table.tabular {margin-left: auto; margin-right: auto;}
div.td00{ margin-left:0pt; margin-right:0pt; }
div.td01{ margin-left:0pt; margin-right:5pt; }
div.td10{ margin-left:5pt; margin-right:0pt; }
div.td11{ margin-left:5pt; margin-right:5pt; }
table[rules] {border-left:solid black 0.4pt; border-right:solid black 0.4pt; }
td.td00{ padding-left:0pt; padding-right:0pt; }
td.td01{ padding-left:0pt; padding-right:5pt; }
td.td10{ padding-left:5pt; padding-right:0pt; }
td.td11{ padding-left:5pt; padding-right:5pt; }
table[rules] {border-left:solid black 0.4pt; border-right:solid black 0.4pt; }
.hline hr, .cline hr{ height : 1px; margin:0px; }
.tabbing-right {text-align:right;}
span.TEX {letter-spacing: -0.125em; }
span.TEX span.E{ position:relative;top:0.5ex;left:-0.0417em;}
a span.TEX span.E {text-decoration: none; }
span.LATEX span.A{ position:relative; top:-0.5ex; left:-0.4em; font-size:85%;}
span.LATEX span.TEX{ position:relative; left: -0.4em; }
div.float img, div.float .caption {text-align:center;}
div.figure img, div.figure .caption {text-align:center;}
.marginpar {width:20%; float:right; text-align:left; margin-left:auto; margin-top:0.5em; font-size:85%; text-decoration:underline;}
.marginpar p{margin-top:0.4em; margin-bottom:0.4em;}
.equation td{text-align:center; vertical-align:middle; }
td.eq-no{ width:5%; }
table.equation { width:100%; } 
div.math-display, div.par-math-display{text-align:center;}
math .texttt { font-family: monospace; }
math .textit { font-style: italic; }
math .textsl { font-style: oblique; }
math .textsf { font-family: sans-serif; }
math .textbf { font-weight: bold; }
.partToc a, .partToc, .likepartToc a, .likepartToc {line-height: 200%; font-weight:bold; font-size:110%;}
.chapterToc a, .chapterToc, .likechapterToc a, .likechapterToc, .appendixToc a, .appendixToc {line-height: 200%; font-weight:bold;}
.index-item, .index-subitem, .index-subsubitem {display:block}
.caption td.id{font-weight: bold; white-space: nowrap; }
table.caption {text-align:center;}
h1.partHead{text-align: center}
p.bibitem { text-indent: -2em; margin-left: 2em; margin-top:0.6em; margin-bottom:0.6em; }
p.bibitem-p { text-indent: 0em; margin-left: 2em; margin-top:0.6em; margin-bottom:0.6em; }
.paragraphHead, .likeparagraphHead { margin-top:2em; font-weight: bold;}
.subparagraphHead, .likesubparagraphHead { font-weight: bold;}
.quote {margin-bottom:0.25em; margin-top:0.25em; margin-left:1em; margin-right:1em; text-align:justify;}
.verse{white-space:nowrap; margin-left:2em}
div.maketitle {text-align:center;}
h2.titleHead{text-align:center;}
div.maketitle{ margin-bottom: 2em; }
div.author, div.date {text-align:center;}
div.thanks{text-align:left; margin-left:10%; font-size:85%; font-style:italic; }
div.author{white-space: nowrap;}
.quotation {margin-bottom:0.25em; margin-top:0.25em; margin-left:1em; }
h1.partHead{text-align: center}
.sectionToc, .likesectionToc {margin-left:2em;}
.subsectionToc, .likesubsectionToc {margin-left:4em;}
.subsubsectionToc, .likesubsubsectionToc {margin-left:6em;}
.frenchb-nbsp{font-size:75%;}
.frenchb-thinspace{font-size:75%;}
.figure img.graphics {margin-left:10%;}
/* end css.sty */

\title{Developpements asymptotiques et analyse numerique}
\author{}
\date{}

\begin{document}
\maketitle

\textbf{Warning: 
requires JavaScript to process the mathematics on this page.\\ If your
browser supports JavaScript, be sure it is enabled.}

\begin{center}\rule{3in}{0.4pt}\end{center}

[
[
[]
[

\subsubsection{9.7 Développements asymptotiques et analyse numérique}

\paragraph{9.7.1 La formule d'Euler-Mac Laurin}

Proposition~9.7.1 Il existe une unique famille de polynômes
B_n(X) (polynômes de Bernoulli) dans \mathbb{R}~[X] vérifiant les
relations

\begin{itemize}
\itemsep1pt\parskip0pt\parsep0pt
\item
  (i) B_0(X) = 1, B_1(X) = X - 1
  \over 2 ,
\item
  (ii) B_n'(X) = nB_n-1(X) pour n ≥ 1
\item
  (iii) B_2n+1(0) = B_2n+1(1) = 0 pour n ≥ 1
\end{itemize}

Démonstration La relation (ii) définit B_n à une constante près
et la relation (iii) fixe les deux constantes d'intégration qui se sont
introduites pour le passage de B_2n-1 à B_2n+1.

Théorème~9.7.2

\begin{itemize}
\itemsep1pt\parskip0pt\parsep0pt
\item
  (i) B_n est un polynôme normalisé de degré n.
\item
  (ii) On a B_n(1 - X) = (-1)^nB_n(X) et en
  particulier B_2n+1( 1 \over 2 ) = 0,
  B_2n(1) = B_2n(0) (noté b_2n)
\item
  (iii) B_n(X + 1) - B_n(X) = nX^n-1
\end{itemize}

Démonstration (i) est évident par récurrence à partir de
B_n'(X) = nB_n-1(X). Pour démontrer (ii), il suffit de
démontrer que, si l'on pose C_n(X) =
(-1)^nB_n(1 - X), la suite (C_n) vérifie
les mêmes relations que la suite (B_n(X)), ce qui est immédiat.
On montre (iii) par récurrence sur n. La relation est vérifiée pour n =
1 et si elle est vérifiée pour n - 1, soit P(X) = B_n(X + 1) -
B_n(X) - nX^n-1. On a P'(X) = n(B_n-1(X +
1) - B_n-1(X) - (n - 1)X^n-2) = 0 par l'hypothèse de
récurrence. Mais d'autre part P(0) = B_n(1) - B_n(0) =
0 (par définition si n est impair, d'après l'assertion précédente si n
est pair), donc P est le polynôme nul.

Théorème~9.7.3 (formule d'Euler-Mac Laurin). Soit f : [0,1] \rightarrow~ E de
classe C^2n+1. Alors

\begin{align*} \int ~
_0^1f(t) dt& =& 1 \over 2 (f(1) +
f(0)) \%& \\ & \text
& -\\sum
_k=1^n(f^(2k-1)(1) -
f^(2k-1)(0)) b_2k \over (2k)!
\%& \\ & \text & -
1 \over (2n + 1)! \int ~
_0^1f^(2n+1)(t)B_ 2n+1(t) dt \%&
\\ \end{align*}

Démonstration Par récurrence sur n. Pour n = 0, on écrit

\begin{align*} \int ~
_0^1f(t) dt& =& \int ~
_0^1f(t)B_ 0(t) dt = \left
[f(t)B_1(t)\right ]_0^1
-\int  _0^1f'(t)B_ 1~(t)
dt\%& \\ & =& 1 \over
2 (f(1) + f(0)) -\int ~
_0^1f'(t)B_ 1(t) dt \%&
\\ \end{align*}

Si la formule est vérifiée pour n, deux intégrations par parties donnent

\begin{align*} 1 \over (2n + 1)!
\int ~
_0^1f^(2n+1)(t)B_ 2n+1(t)
dt\quad && \%& \\ &
=& 1 \over (2n + 1)! \left
[f^(2n+1)(t) B_2n+2(t) \over 2n +
2 \right ]_0^1 \%&
\\ & \text & - 1
\over (2n + 2)! \int ~
_0^1f^(2n+2)(t)B_ 2n+2(t) dt\%&
\\ & =& b_2n+2
\over (2n + 2)! (f^(2n+1)(1) -
f^(2n+1)(0)) \%& \\ &
\text & -\Biggl ( 1
\over (2n + 2)! \left
[f^(2n+2)(t) B_2n+3(t) \over 2n +
3 \right ]_0^1 \%&
\\ & \text & - 1
\over (2n + 3)! \int ~
_0^1f^(2n+3)(t)B_ 2n+3(t)
dt\Biggr )\%& \\ &
=& b_2n+2 \over (2n + 2)!
(f^(2n+1)(1) - f^(2n+1)(0)) \%&
\\ & \text & + 1
\over (2n + 3)! \int ~
_0^1f^(2n+3)(t)B_ 2n+3(t) dt\%&
\\ \end{align*}

en tenant compte de B_2n+2(0) = B_2n+2(1) =
b_2n+2 et de B_2n+3(0) = B_2n+3(1) = 0.

Remarque~9.7.1 On peut montrer que \forall~~x \in
[0,1], B_n(x)\leq 4e^2\pi~ n!
\over (2\pi~)^n ce qui permet d'avoir une
estimation du reste. Si f est à valeurs réelles, on peut obtenir une
autre estimation du reste en montrant par récurrence que les polynômes
B_n ont les variations suivantes

̲ ̲ ̲ ̲ ̲ ̲ ̲ ̲ ̲ ̲

x

0

1\diagup2

1 ̲ ̲ ̲ ̲ ̲ ̲ ̲ ̲ ̲ ̲

B_4n(x)

b_4n < 0

\nearrow

0

\nearrow

> 0

\searrow

0

\searrow

b_4n < 0 ̲ ̲ ̲ ̲ ̲ ̲ ̲ ̲ ̲ ̲

B_4n+1(x)

0

\searrow

\nearrow

0

\nearrow

\searrow

0 ̲ ̲ ̲ ̲ ̲ ̲ ̲ ̲ ̲ ̲

B_4n+2(x)

b_4n+2 > 0

\searrow

0

\searrow

< 0

\nearrow

0

\nearrow

b_4n+2 > 0 ̲ ̲ ̲ ̲ ̲ ̲ ̲ ̲ ̲ ̲

B_4n+3(x)

0

\nearrow

\searrow

0

\searrow

\nearrow

0 ̲ ̲ ̲ ̲ ̲ ̲ ̲ ̲ ̲ ̲

\includegraphics{cours7x.png}

Ceci montre que les polynômes B_2p - b_2p sont de
signe constant sur [0,1]. On peut donc utiliser la première formule
de la moyenne, ce qui nous donne

\begin{align*} \int ~
_0^1f^(2n+2)(t)B_ 2n+2(t)
dt\quad && \%& \\ & =&
b_2n+2\int ~
_0^1f^(2n+2)(t) dt +\\int
 _0^1f^(2n+2)(t)(B_ 2n+2(t) -
b_2n+2) dt\%& \\ & =&
b_2n+2(f^(2n+1)(1) - f^(2n+1)(0)) \%&
\\ & \text &
+f^(2n+2)(\xi)\int ~
_0^1(B_ 2n+2(t) - b_2n+2) dt \%&
\\ & =&
b_2n+2(f^(2n+1)(1) - f^(2n+1)(0)) -
b_ 2n+2f^(2n+2)(\xi) \%&
\\ \end{align*}

car \int  _0^1B_2n+2~(t)
dt = \left [ B_2n+3(t)
\over 2n+3 \right ]_0^1
= 0. On obtient, en reprenant la démonstration du lemme, la formule sous
la forme

\begin{align*} \int ~
_0^1f(t) dt& =& 1 \over 2 (f(1) +
f(0)) \%& \\ & \text
& -\\sum
_k=1^n+1(f^(2k-1)(1) -
f^(2k-1)(0)) b_2k \over (2k)!
\%& \\ & \text & +
b_2n+2 \over (2n + 1)! f^(2n+2)(\xi)
\%& \\ \end{align*}

Exemple~9.7.1 Appliquons cette formule à f(t) = 1 \over
t+p . On va obtenir

\begin{align*} log~ (p + 1)
- log (p)& =& 1 \over 2~
( 1 \over p + 1 + 1 \over p ) \%&
\\ & \text &
+\sum _k=1^n+1~( 1
\over (p + 1)^2k - 1 \over
p^2k ) b_2k \over 2k \%&
\\ & \text & + (2n
+ 2)b_2n+2 \over \xi_p^2n+3 \%&
\\ \end{align*}

avec \xi_p \in [p,p + 1] et donc \xi_p ∼ p. En sommant
de p = 1 jusque N - 1, on obtient

\begin{align*} log~ N& =&
\sum _p=1^N~ 1
\over p - 1 \over 2 - 1
\over 2N + \\sum
_k=1^n+1( 1 \over N^2k -
1) b_2k \over 2k \%&
\\ & \text & +(2n +
2)b_2n+2 \sum _p=1^N-1~
1 \over \xi_p^2n+3 \%&
\\ \end{align*}

et en utilisant \\sum ~
_p=N^+\infty~ 1 \over
\xi_p^2n+3 = O( 1 \over
N^2n+2 ) on obtient, après amalgame de tous les termes ne
dépendant pas de N en une constante \gamma,

\begin{align*} \\sum
_p=1^N 1 \over p & =&
log N + \gamma + 1 \over 2N~
-\sum _k=1^n b_2k~
\over 2kN^2k + O( 1 \over
N^2n+2 )\%& \\
\end{align*}

\paragraph{9.7.2 Calcul approché d'intégrales}

Méthode des trapèzes

Soit f : [a,b] \rightarrow~ \mathbb{R}~ de classe C^2 et p \in \mathbb{N}~^∗.
Pour k \in [0,p] posons a_k = a + k b-a
\over p . On approche la fonction f par la fonction \phi :
[a,b] \rightarrow~ E qui vérifie \forall~~k \in [0,p],
\phi(a_k) = f(a_k) et qui est linéaire sur chaque
intervalle [a_k-1,a_k]. On a immédiatement
\int ~
_a_k-1^a_k\phi = (a_ k -
a_k-1) f(a_k)+f(a_k-1) \over
2 (aire d'un trapèze). D'où, \int ~
_a^b\phi = b-a \over n
\left ( f(a) \over 2
+ \\sum ~
_k=1^p-1f(a_k) + f(b) \over 2
\right ) = T_p avec les notations du paragraphe
précédent. On prendra donc comme valeur approchée de I
=\int  _a^b~f,
\overlineI =\int ~
_a^b\phi = T_p.

Majoration de l'erreur~: on cherche à majorer I
-\overlineI =
\int  _a^b~(f -
\phi). Posons g = f - \phi et calculons à l'aide d'une intégration
par parties l'intégrale suivante (en remarquant que la restriction de g
à [a_k-1,a_k] est de classe C^2 avec
g'`= f'')

\begin{align*} \int ~
_a_k-1^a_k f'`(t)(t -
a_k-1)(a_k - t) dt&& \%&
\\ & =& \int ~
_a_k-1^a_k g'`(t)(t -
a_k-1)(a_k - t) dt \%&
\\ & =& \left
[g'(t)(t - a_k-1)(a_k - t)\right
]_a_k-1^a_k 
+\int  _a_k-1^a_k~
g'(t)(2t - a_k-1 - a_k) dt\%&
\\ & =& \int ~
_a_k-1^a_k g'(t)(2t - a_k-1 -
a_k) dt \%& \\ & =&
\left [g(t)(2t - a_k-1 -
a_k)\right
]_a_k-1^a_k  -
2\int  _a_k-1^a_k~
g(t) dt \%& \\ & =&
-2\int ~
_a_k-1^a_k g(t) dt \%&
\\ \end{align*}

puisque g(a_k-1) = g(a_k) = 0. On a donc

\begin{align*} \left
\int ~
_a_k-1^a_k g(t) dt\right
& =& 1 \over 2 \left
\int ~
_a_k-1^a_k f''(t)(t -
a_k-1)(a_k - t) dt\right \%&
\\ & \leq& M_2
\over 2 \int ~
_a_k-1^a_k (t -
a_k-1)(a_k - t) dt \%&
\\ & =& M_2
\over 12 (a_k - a_k-1)^3
= M_2(b - a)^3 \over
12p^3 \%& \\
\end{align*}

En sommant de k = 1 à p, on obtient

I -\overlineI\leq M_2(b
- a)^3 \over 12p^2

Application de la formule d'Euler-Mac Laurin

Soit alors f : [a,b] \rightarrow~ E de classe C^2n+1 et p \in
\mathbb{N}~^∗. Pour k \in [0,p] posons a_k = a + k b-a
\over p ~; appliquons la formule précédente à
t\mapsto~f(a_k-1 + t b-a
\over p ). On obtient alors (avec le changement de
variable x = a_k-1 + t b-a \over p )

\begin{align*} \int ~
_a_k-1^a_k f(x) dx& =& b - a
\over n \int ~
_0^1f(a_ k-1 + t b - a \over p
) dt\%& \\ & =& b - a
\over 2p (f(a_k-1) + f(a_k)) \%&
\\ & -\\sum
_k=1^n (b - a)^2k \over
p^2k (f^(2k-1)(a_ k) -
f^(2k-1)(a_ k-1)) b_2k
\over (2k)! &\%& \\ &
\text & + (b - a)^2n+2
\over p^2n+2(2n + 1)! \rho_n,k \%&
\\ \end{align*}

avec \rho_n,k =\int ~
_0^1f^(2n+1)(a_k-1 + t b-a
\over p )B_2n+1(t) dt. Posons M_2n+1
=\
sup_t\in[a,b]\f^(2n+1)(t)\.
On a alors
\\rho_n,k\ \leq
M_2n+1\int ~
_0^1B_2n+1(t) dt. Sommons
alors les égalités ci dessus, en posant

T_p = b - a \over n \left (
f(a) \over 2 + \\sum
_k=1^p-1f(a_ k) + f(b) \over
2 \right )

on obtient,

\begin{align*} \int ~
_a^bf(x) dx& =& T_ p
-\sum _k=1^n~ (b -
a)^2k \over p^2k
(f^(2k-1)(b) - f^(2k-1)(a)) b_2k
\over (2k)! \%& \\ &
\text & + (b - a)^2n+2
\over p^2n+2(2n + 1)! S_n,p \%&
\\ \end{align*}

avec S_n,p =\
\sum  _k=1^p\rho_n,k~ et
donc \S_n,p\
\leq\\sum ~
_k=1^p\\rho_n,k\
\leq pM_2n+1\int ~
_0^1B_2n+1(t) dt. On obtient
donc

Théorème~9.7.4 Soit f : [a,b] \rightarrow~ E de classe C^2n+1 et p \in
\mathbb{N}~^∗. Pour k \in [0,p] posons a_k = a + k b-a
\over p . Soit T_p = b-a
\over n \left ( f(a)
\over 2 +\
\sum  _k=1^p-1f(a_k~) +
f(b) \over 2 \right ) et M_2n+1
=\
sup_t\in[a,b]\f^(2n+1)(t)\.
Alors

\begin{align*} \int ~
_a^bf(x) dx& =& T_ p
-\sum _k=1^n b_2k~(b
- a)^2k \over p^2k(2k)!
(f^(2k-1)(b) - f^(2k-1)(a))\%&
\\ & \text & + (b
- a)^2n+2 \over p^2n+1(2n + 1)!
R_n,p \%& \\
\end{align*}

avec \R_n,p\
\leq M_2n+1\int ~
_0^1B_2n+1(t) dt.

Remarque~9.7.2 Ce théorème nous donne un développement à un ordre
arbitraire de la différence entre l'intégrale et sa valeur approchée par
la méthode des trapèzes

Méthode de Simpson

La formule d'Euler-Mac Laurin, nous montre que si f est de classe
C^4, on a

I - T_p = \lambda~ \over p^2 + O( 1
\over p^4 )

On a donc également I - T_2p = \lambda~ \over
4p^2 + O( 1 \over p^4 ) puis
4(I - T_2p) - (I - T_p) = O( 1 \over
p^4 ) ou encore

I - 1 \over 3 (4T_2p - T_p) = O(
1 \over p^4 )

Posons donc a_k = a + k b-a \over 2p , on a

\begin{align*} S_p& =& (b - a)
\over 6p (T_p + 4T_2p) \%&
\\ & =& (b - a) \over
6p (f(a) + 4f(a_1) + 2f(a_2) + 4f(a_3) +
\\ldots~\%&
\\ & \text &
+2f(a_2p-2) + 4f(a_2p-1) + f(b)) \%&
\\ \end{align*}

On sait donc que l'on a une majoration du type

I - S_p\leq M \over
p^4

Remarque~9.7.3 On peut montrer qu'en fait I -
S_p\leq M_4(b-a)^5
\over 2880p^4 avec M_4
=\
sup_t\in[a,b]\f^(4)(t)\,
majoration de peu d'intérêt dans la pratique vu la difficulté qu'il y a
habituellement à trouver un majorant de M_4.

Méthode de Romberg

Elle consiste à généraliser la méthode qui nous a fait passer de la
méthode des trapèzes à la méthode de Simpson en utilisant le calcul de
T_p,T_2p,T_4p,\\ldots,T_2^kp~
pour éliminer successivement les termes en  1 \over
p^2 , 1 \over p^4
,\\ldots~, 1
\over p^2k du développement asymptotique
donné par la formule d'Euler-Mac Laurin.

\paragraph{9.7.3 La méthode de Laplace}

C'est une méthode classique de recherche d'équivalents d'intégrales
dépendant d'un paramètre (ici n) consistant à remarquer qu'un intégrande
du type f(t)e^ng(t) va privilégier, pour n grand, les valeurs
de t pour lesquelles la fonction g atteint son maximum (car si x
< y, e^nx = o(e^ny)).

Proposition~9.7.5 Soit f :]a,b[\rightarrow~ \mathbb{R}~ continue intégrable sur
]a,b[. Soit g :]a,b[\rightarrow~ \mathbb{R}~ de classe C^2. On suppose que
g atteint son maximum en un point c \in]a,b[ avec g''(c) <
0, f(c)\neq~0 et que \forall~~\eta
> 0,
sup_t-c≥\eta~g(t)
< g(c). Alors, quand n tend vers + \infty~

\int  _a^bf(t)e^ng(t)~
dt ∼ f(c)e^ng(c)\sqrt 2\pi~
\over ng''(c) 

Démonstration Quitte à changer f en - f, on peut supposer f(c)
> 0. Soit \alpha~ tel que 0 < \alpha~
<\
min(g''(c),f(c)) et soit \eta > 0 tel
que t - c\leq \eta \rigtharrow~f(t) - f(c)\leq \alpha~ et
g(t) - g(c) - (t-c)^2 \over 2
g''(c) < \alpha~ (t-c)^2 \over
2 (puisque g'(c) = 0). Sur [c - \eta,c + \eta], on a f(c) - \alpha~
< f(t) < f(c) + \alpha~ et

g(c) + (t - c)^2 \over 2 (g'`(c) - \alpha~) \leq
g(t) \leq g(c) + (t - c)^2 \over 2 (g''(c) +
\alpha~)

On obtient donc

\begin{align*} (f(c) -
\alpha~)e^ng(c)\int ~
_c-\eta^c+\eta exp~
\left (n (t - c)^2 \over 2
(g'`(c) - \alpha~)\right ) dt&& \%&
\\ & \leq& \int ~
_c-\eta^c+\etaf(t)e^ng(t) dt \%&
\\ & \leq& (f(c) +
\alpha~)e^ng(c)\int ~
_c-\eta^c+\eta exp~
\left (n (t - c)^2 \over 2
(g''(c) + \alpha~)\right ) dt\%&
\\ \end{align*}

Mais, si \lambda~ < 0, le changement de variable u =
\sqrt n\lambda~ \over 2
 (t - c) donne

\begin{align*} \int ~
_c-\eta^c+\eta exp~
\left (n\lambda~ (t - c)^2 \over 2
\right ) dt&& \%& \\ &
=& \sqrt 2 \over
n\lambda~ \int  _
-\sqrt n\lambda~ \over
2  \eta^\sqrt n\lambda~
\over 2  \etae^-u^2  du \%&
\\ & ∼& \sqrt 2
\over n\lambda~
\int ~
_-\infty~^+\infty~e^-u^2  du =
\sqrt 2\pi~ \over
n\lambda~ \%& \\
\end{align*}

Posons I_n =
\sqrtne^-ng(c)\\int
 _c-\eta^c+\etaf(t)e^ng(t) dt. On a donc
u_n \leq I_n \leq v_n avec

\begin{align*} u_n& =& (f(c) -
\alpha~)\sqrtn\int ~
_c-\eta^c+\eta exp~
\left (n (t - c)^2 \over 2
(g'`(c) - \alpha~)\right ) dt\%&
\\ & ∼& (f(c) -
\alpha~)\sqrt 2\pi~ \over g''(c) -
\alpha~  \%& \\
\end{align*}

et de même v_n ∼ (f(c) + \alpha~)\sqrt 2\pi~
\over g''(c)+\alpha~ . Donnons nous \epsilon
> 0 et soit \alpha~ tel que

\begin{itemize}
\itemsep1pt\parskip0pt\parsep0pt
\item
  (i) 0 < \alpha~ <\
  min(g''(c),f(c))
\item
  (ii) (f(c) - \alpha~)\sqrt 2\pi~ \over
  g'`(c)-\alpha~  >
  f(c)\sqrt 2\pi~ \over
  g''(c)  - \epsilon \over 2
\item
  (iii) (f(c) + \alpha~)\sqrt 2\pi~ \over
  g'`(c)+\alpha~  <
  f(c)\sqrt 2\pi~ \over
  g''(c)  + \epsilon \over 2
\end{itemize}

On prend le \eta correspondant comme ci-dessus. Alors comme
limu_n~ = (f(c) -
\alpha~)\sqrt 2\pi~ \over
g''(c)-\alpha~  et
limv_n~ = (f(c) +
\alpha~)\sqrt 2\pi~ \over
g''(c)+\alpha~ , il existe N \in \mathbb{N}~ tel que

\begin{align*} n ≥ N \rigtharrow~ f(c)\sqrt
2\pi~ \over g'`(c) - \epsilon
\over 2 < u_n \leq v_n
< f(c)\sqrt 2\pi~ \over
g''(c)  + \epsilon \over 2 & &
\%& \\ \end{align*}

Pour n ≥ N on a donc f(c)\sqrt 2\pi~
\over g'`(c)  - \epsilon
\over 2 < I_n <
f(c)\sqrt 2\pi~ \over
g''(c)  + \epsilon \over 2 .

Soit M =\
sup_t-c≥\etag(t) < g(c). On a alors

\left
\sqrtne^-ng(c)\\int
 _t-c≥\etaf(t)e^ng(t)
dt\right
\leq\sqrtne^-n(g(c)-M)\\int
 _a^bf(t) dt

qui tend vers 0 quand n tend vers + \infty~. Soit donc N' \in \mathbb{N}~ tel que n \leq N'
\rigtharrow~\sqrtne^-n(g(c)-M)\\int
 _a^bf(t) dt < \epsilon
\over 2 . Alors, pour n ≥\
max(N,N'), on a

\begin{align*} f(c)\sqrt 2\pi~
\over g'`(c) - \epsilon& <&
\sqrtne^-ng(c)\\int
 _a^bf(t)e^ng(t) dt\%&
\\ & <&
f(c)\sqrt 2\pi~ \over
g''(c)  + \epsilon \%&
\\ \end{align*}

ce qui montre le résultat.

Remarque~9.7.4 Pour appliquer la méthode précédente, il suffit en fait
qu'il existe un n_0 \in \mathbb{N}~ tel que \int ~
_a^bf(t)e^n_0g(t)
dt converge~: on peut alors écrire, en posant f_1(t) =
f(t)e^n_0g(t)

\begin{align*} \int ~
_a^bf(t)e^ng(t) dt& =&
\int  _a^bf_
1(t)e^(n-n_0)g(t) dt \%&
\\ & ∼&
f_1(c)e^(n-n_0)g(c)\sqrt
2\pi~ \over (n - n_0)g'`(c)
\%& \\ & ∼&
f(c)e^ng(c)\sqrt 2\pi~ \over
ng''(c)  \%& \\
\end{align*}

Exemple~9.7.2 Ecrivons

n! = \Gamma(n + 1) =\int ~
_0^+\infty~t^ne^-t dt =
n^n+1\int ~
_0^+\infty~(ue^-u)^n du

avec le changement de variable t = nu. Pour trouver un équivalent de
\int ~
_0^+\infty~(ue^-u)^n du, on peut appliquer
la méthode de Laplace, en tenant compte de la remarque ci-dessus avec
n_0 = 1. On prend donc f(u) = 1, g(u) =\
log (ue^-u) = -u + log~ u qui
atteint son maximum au point 1 avec g''(1) = -1, g(1) = -1. D'où

\int ~
_0^+\infty~(ue^-u)^n du ∼
e^-n\sqrt 2\pi~ \over n 

et donc n! ∼\sqrt2\pi~n n^n
\over e^n .

[
[
[
[

\end{document}
