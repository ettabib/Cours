\documentclass[]{article}
\usepackage[T1]{fontenc}
\usepackage{lmodern}
\usepackage{amssymb,amsmath}
\usepackage{ifxetex,ifluatex}
\usepackage{fixltx2e} % provides \textsubscript
% use upquote if available, for straight quotes in verbatim environments
\IfFileExists{upquote.sty}{\usepackage{upquote}}{}
\ifnum 0\ifxetex 1\fi\ifluatex 1\fi=0 % if pdftex
  \usepackage[utf8]{inputenc}
\else % if luatex or xelatex
  \ifxetex
    \usepackage{mathspec}
    \usepackage{xltxtra,xunicode}
  \else
    \usepackage{fontspec}
  \fi
  \defaultfontfeatures{Mapping=tex-text,Scale=MatchLowercase}
  \newcommand{\euro}{€}
\fi
% use microtype if available
\IfFileExists{microtype.sty}{\usepackage{microtype}}{}
\usepackage{graphicx}
% Redefine \includegraphics so that, unless explicit options are
% given, the image width will not exceed the width of the page.
% Images get their normal width if they fit onto the page, but
% are scaled down if they would overflow the margins.
\makeatletter
\def\ScaleIfNeeded{%
  \ifdim\Gin@nat@width>\linewidth
    \linewidth
  \else
    \Gin@nat@width
  \fi
}
\makeatother
\let\Oldincludegraphics\includegraphics
{%
 \catcode`\@=11\relax%
 \gdef\includegraphics{\@ifnextchar[{\Oldincludegraphics}{\Oldincludegraphics[width=\ScaleIfNeeded]}}%
}%
\ifxetex
  \usepackage[setpagesize=false, % page size defined by xetex
              unicode=false, % unicode breaks when used with xetex
              xetex]{hyperref}
\else
  \usepackage[unicode=true]{hyperref}
\fi
\hypersetup{breaklinks=true,
            bookmarks=true,
            pdfauthor={},
            pdftitle={Developpements asymptotiques et analyse numerique},
            colorlinks=true,
            citecolor=blue,
            urlcolor=blue,
            linkcolor=magenta,
            pdfborder={0 0 0}}
\urlstyle{same}  % don't use monospace font for urls
\setlength{\parindent}{0pt}
\setlength{\parskip}{6pt plus 2pt minus 1pt}
\setlength{\emergencystretch}{3em}  % prevent overfull lines
\setcounter{secnumdepth}{0}
 
/* start css.sty */
.cmr-5{font-size:50%;}
.cmr-7{font-size:70%;}
.cmmi-5{font-size:50%;font-style: italic;}
.cmmi-7{font-size:70%;font-style: italic;}
.cmmi-10{font-style: italic;}
.cmsy-5{font-size:50%;}
.cmsy-7{font-size:70%;}
.cmex-7{font-size:70%;}
.cmex-7x-x-71{font-size:49%;}
.msbm-7{font-size:70%;}
.cmtt-10{font-family: monospace;}
.cmti-10{ font-style: italic;}
.cmbx-10{ font-weight: bold;}
.cmr-17x-x-120{font-size:204%;}
.cmsl-10{font-style: oblique;}
.cmti-7x-x-71{font-size:49%; font-style: italic;}
.cmbxti-10{ font-weight: bold; font-style: italic;}
p.noindent { text-indent: 0em }
td p.noindent { text-indent: 0em; margin-top:0em; }
p.nopar { text-indent: 0em; }
p.indent{ text-indent: 1.5em }
@media print {div.crosslinks {visibility:hidden;}}
a img { border-top: 0; border-left: 0; border-right: 0; }
center { margin-top:1em; margin-bottom:1em; }
td center { margin-top:0em; margin-bottom:0em; }
.Canvas { position:relative; }
li p.indent { text-indent: 0em }
.enumerate1 {list-style-type:decimal;}
.enumerate2 {list-style-type:lower-alpha;}
.enumerate3 {list-style-type:lower-roman;}
.enumerate4 {list-style-type:upper-alpha;}
div.newtheorem { margin-bottom: 2em; margin-top: 2em;}
.obeylines-h,.obeylines-v {white-space: nowrap; }
div.obeylines-v p { margin-top:0; margin-bottom:0; }
.overline{ text-decoration:overline; }
.overline img{ border-top: 1px solid black; }
td.displaylines {text-align:center; white-space:nowrap;}
.centerline {text-align:center;}
.rightline {text-align:right;}
div.verbatim {font-family: monospace; white-space: nowrap; text-align:left; clear:both; }
.fbox {padding-left:3.0pt; padding-right:3.0pt; text-indent:0pt; border:solid black 0.4pt; }
div.fbox {display:table}
div.center div.fbox {text-align:center; clear:both; padding-left:3.0pt; padding-right:3.0pt; text-indent:0pt; border:solid black 0.4pt; }
div.minipage{width:100%;}
div.center, div.center div.center {text-align: center; margin-left:1em; margin-right:1em;}
div.center div {text-align: left;}
div.flushright, div.flushright div.flushright {text-align: right;}
div.flushright div {text-align: left;}
div.flushleft {text-align: left;}
.underline{ text-decoration:underline; }
.underline img{ border-bottom: 1px solid black; margin-bottom:1pt; }
.framebox-c, .framebox-l, .framebox-r { padding-left:3.0pt; padding-right:3.0pt; text-indent:0pt; border:solid black 0.4pt; }
.framebox-c {text-align:center;}
.framebox-l {text-align:left;}
.framebox-r {text-align:right;}
span.thank-mark{ vertical-align: super }
span.footnote-mark sup.textsuperscript, span.footnote-mark a sup.textsuperscript{ font-size:80%; }
div.tabular, div.center div.tabular {text-align: center; margin-top:0.5em; margin-bottom:0.5em; }
table.tabular td p{margin-top:0em;}
table.tabular {margin-left: auto; margin-right: auto;}
div.td00{ margin-left:0pt; margin-right:0pt; }
div.td01{ margin-left:0pt; margin-right:5pt; }
div.td10{ margin-left:5pt; margin-right:0pt; }
div.td11{ margin-left:5pt; margin-right:5pt; }
table[rules] {border-left:solid black 0.4pt; border-right:solid black 0.4pt; }
td.td00{ padding-left:0pt; padding-right:0pt; }
td.td01{ padding-left:0pt; padding-right:5pt; }
td.td10{ padding-left:5pt; padding-right:0pt; }
td.td11{ padding-left:5pt; padding-right:5pt; }
table[rules] {border-left:solid black 0.4pt; border-right:solid black 0.4pt; }
.hline hr, .cline hr{ height : 1px; margin:0px; }
.tabbing-right {text-align:right;}
span.TEX {letter-spacing: -0.125em; }
span.TEX span.E{ position:relative;top:0.5ex;left:-0.0417em;}
a span.TEX span.E {text-decoration: none; }
span.LATEX span.A{ position:relative; top:-0.5ex; left:-0.4em; font-size:85%;}
span.LATEX span.TEX{ position:relative; left: -0.4em; }
div.float img, div.float .caption {text-align:center;}
div.figure img, div.figure .caption {text-align:center;}
.marginpar {width:20%; float:right; text-align:left; margin-left:auto; margin-top:0.5em; font-size:85%; text-decoration:underline;}
.marginpar p{margin-top:0.4em; margin-bottom:0.4em;}
.equation td{text-align:center; vertical-align:middle; }
td.eq-no{ width:5%; }
table.equation { width:100%; } 
div.math-display, div.par-math-display{text-align:center;}
math .texttt { font-family: monospace; }
math .textit { font-style: italic; }
math .textsl { font-style: oblique; }
math .textsf { font-family: sans-serif; }
math .textbf { font-weight: bold; }
.partToc a, .partToc, .likepartToc a, .likepartToc {line-height: 200%; font-weight:bold; font-size:110%;}
.chapterToc a, .chapterToc, .likechapterToc a, .likechapterToc, .appendixToc a, .appendixToc {line-height: 200%; font-weight:bold;}
.index-item, .index-subitem, .index-subsubitem {display:block}
.caption td.id{font-weight: bold; white-space: nowrap; }
table.caption {text-align:center;}
h1.partHead{text-align: center}
p.bibitem { text-indent: -2em; margin-left: 2em; margin-top:0.6em; margin-bottom:0.6em; }
p.bibitem-p { text-indent: 0em; margin-left: 2em; margin-top:0.6em; margin-bottom:0.6em; }
.paragraphHead, .likeparagraphHead { margin-top:2em; font-weight: bold;}
.subparagraphHead, .likesubparagraphHead { font-weight: bold;}
.quote {margin-bottom:0.25em; margin-top:0.25em; margin-left:1em; margin-right:1em; text-align:justify;}
.verse{white-space:nowrap; margin-left:2em}
div.maketitle {text-align:center;}
h2.titleHead{text-align:center;}
div.maketitle{ margin-bottom: 2em; }
div.author, div.date {text-align:center;}
div.thanks{text-align:left; margin-left:10%; font-size:85%; font-style:italic; }
div.author{white-space: nowrap;}
.quotation {margin-bottom:0.25em; margin-top:0.25em; margin-left:1em; }
h1.partHead{text-align: center}
.sectionToc, .likesectionToc {margin-left:2em;}
.subsectionToc, .likesubsectionToc {margin-left:4em;}
.subsubsectionToc, .likesubsubsectionToc {margin-left:6em;}
.frenchb-nbsp{font-size:75%;}
.frenchb-thinspace{font-size:75%;}
.figure img.graphics {margin-left:10%;}
/* end css.sty */

\title{Developpements asymptotiques et analyse numerique}
\author{}
\date{}

\begin{document}
\maketitle

\textbf{Warning: \href{http://www.math.union.edu/locate/jsMath}{jsMath}
requires JavaScript to process the mathematics on this page.\\ If your
browser supports JavaScript, be sure it is enabled.}

\begin{center}\rule{3in}{0.4pt}\end{center}

{[}\href{coursse57.html}{next}{]} {[}\href{coursse55.html}{prev}{]}
{[}\href{coursse55.html\#tailcoursse55.html}{prev-tail}{]}
{[}\hyperref[tailcoursse56.html]{tail}{]}
{[}\href{coursch10.html\#coursse56.html}{up}{]}

\subsubsection{9.7 Développements asymptotiques et analyse numérique}

\paragraph{9.7.1 La formule d'Euler-Mac Laurin}

Proposition~9.7.1 Il existe une unique famille de polynômes
\{B\}\_\{n\}(X) (polynômes de Bernoulli) dans ℝ{[}X{]} vérifiant les
relations

\begin{itemize}
\itemsep1pt\parskip0pt\parsep0pt
\item
  (i) \{B\}\_\{0\}(X) = 1, \{B\}\_\{1\}(X) = X −\{ 1
  \textbackslash{}over 2\} ,
\item
  (ii) \{B\}\_\{n\}'(X) = n\{B\}\_\{n−1\}(X) pour n ≥ 1
\item
  (iii) \{B\}\_\{2n+1\}(0) = \{B\}\_\{2n+1\}(1) = 0 pour n ≥ 1
\end{itemize}

Démonstration La relation (ii) définit \{B\}\_\{n\} à une constante près
et la relation (iii) fixe les deux constantes d'intégration qui se sont
introduites pour le passage de \{B\}\_\{2n−1\} à \{B\}\_\{2n+1\}.

Théorème~9.7.2

\begin{itemize}
\itemsep1pt\parskip0pt\parsep0pt
\item
  (i) \{B\}\_\{n\} est un polynôme normalisé de degré n.
\item
  (ii) On a \{B\}\_\{n\}(1 − X) = \{(−1)\}\^{}\{n\}\{B\}\_\{n\}(X) et en
  particulier \{B\}\_\{2n+1\}(\{ 1 \textbackslash{}over 2\} ) = 0,
  \{B\}\_\{2n\}(1) = \{B\}\_\{2n\}(0) (noté \{b\}\_\{2n\})
\item
  (iii) \{B\}\_\{n\}(X + 1) − \{B\}\_\{n\}(X) = n\{X\}\^{}\{n−1\}
\end{itemize}

Démonstration (i) est évident par récurrence à partir de
\{B\}\_\{n\}'(X) = n\{B\}\_\{n−1\}(X). Pour démontrer (ii), il suffit de
démontrer que, si l'on pose \{C\}\_\{n\}(X) =
\{(−1)\}\^{}\{n\}\{B\}\_\{n\}(1 − X), la suite (\{C\}\_\{n\}) vérifie
les mêmes relations que la suite (\{B\}\_\{n\}(X)), ce qui est immédiat.
On montre (iii) par récurrence sur n. La relation est vérifiée pour n =
1 et si elle est vérifiée pour n − 1, soit P(X) = \{B\}\_\{n\}(X + 1) −
\{B\}\_\{n\}(X) − n\{X\}\^{}\{n−1\}. On a P'(X) = n(\{B\}\_\{n−1\}(X +
1) − \{B\}\_\{n−1\}(X) − (n − 1)\{X\}\^{}\{n−2\}) = 0 par l'hypothèse de
récurrence. Mais d'autre part P(0) = \{B\}\_\{n\}(1) − \{B\}\_\{n\}(0) =
0 (par définition si n est impair, d'après l'assertion précédente si n
est pair), donc P est le polynôme nul.

Théorème~9.7.3 (formule d'Euler-Mac Laurin). Soit f : {[}0,1{]} → E de
classe \{C\}\^{}\{2n+1\}. Alors

\textbackslash{}begin\{eqnarray*\} \{\textbackslash{}mathop\{∫ \}
\}\_\{0\}\^{}\{1\}f(t) dt\& =\&\{ 1 \textbackslash{}over 2\} (f(1) +
f(0)) \%\& \textbackslash{}\textbackslash{} \& \textbackslash{}text\{\}
\& −\{\textbackslash{}mathop\{∑
\}\}\_\{k=1\}\^{}\{n\}(\{f\}\^{}\{(2k−1)\}(1) −
\{f\}\^{}\{(2k−1)\}(0))\{ \{b\}\_\{2k\} \textbackslash{}over (2k)!\}
\%\& \textbackslash{}\textbackslash{} \& \textbackslash{}text\{\} \& −\{
1 \textbackslash{}over (2n + 1)!\} \{\textbackslash{}mathop\{∫ \}
\}\_\{0\}\^{}\{1\}\{f\}\^{}\{(2n+1)\}(t)\{B\}\_\{ 2n+1\}(t) dt \%\&
\textbackslash{}\textbackslash{} \textbackslash{}end\{eqnarray*\}

Démonstration Par récurrence sur n. Pour n = 0, on écrit

\textbackslash{}begin\{eqnarray*\} \{\textbackslash{}mathop\{∫ \}
\}\_\{0\}\^{}\{1\}f(t) dt\& =\& \{\textbackslash{}mathop\{∫ \}
\}\_\{0\}\^{}\{1\}f(t)\{B\}\_\{ 0\}(t) dt =\{ \textbackslash{}left
{[}f(t)\{B\}\_\{1\}(t)\textbackslash{}right {]}\}\_\{0\}\^{}\{1\}
−\{\textbackslash{}mathop\{∫ \} \}\_\{0\}\^{}\{1\}f'(t)\{B\}\_\{ 1\}(t)
dt\%\& \textbackslash{}\textbackslash{} \& =\&\{ 1 \textbackslash{}over
2\} (f(1) + f(0)) −\{\textbackslash{}mathop\{∫ \}
\}\_\{0\}\^{}\{1\}f'(t)\{B\}\_\{ 1\}(t) dt \%\&
\textbackslash{}\textbackslash{} \textbackslash{}end\{eqnarray*\}

Si la formule est vérifiée pour n, deux intégrations par parties donnent

\textbackslash{}begin\{eqnarray*\}\{ 1 \textbackslash{}over (2n + 1)!\}
\{\textbackslash{}mathop\{∫ \}
\}\_\{0\}\^{}\{1\}\{f\}\^{}\{(2n+1)\}(t)\{B\}\_\{ 2n+1\}(t)
dt\textbackslash{}quad \&\& \%\& \textbackslash{}\textbackslash{} \&
=\&\{ 1 \textbackslash{}over (2n + 1)!\} \{\textbackslash{}left
{[}\{f\}\^{}\{(2n+1)\}(t)\{ \{B\}\_\{2n+2\}(t) \textbackslash{}over 2n +
2\} \textbackslash{}right {]}\}\_\{0\}\^{}\{1\} \%\&
\textbackslash{}\textbackslash{} \& \textbackslash{}text\{\} \& −\{ 1
\textbackslash{}over (2n + 2)!\} \{\textbackslash{}mathop\{∫ \}
\}\_\{0\}\^{}\{1\}\{f\}\^{}\{(2n+2)\}(t)\{B\}\_\{ 2n+2\}(t) dt\%\&
\textbackslash{}\textbackslash{} \& =\&\{ \{b\}\_\{2n+2\}
\textbackslash{}over (2n + 2)!\} (\{f\}\^{}\{(2n+1)\}(1) −
\{f\}\^{}\{(2n+1)\}(0)) \%\& \textbackslash{}\textbackslash{} \&
\textbackslash{}text\{\} \& −\{\textbackslash{}Biggl (\{ 1
\textbackslash{}over (2n + 2)!\} \{\textbackslash{}left
{[}\{f\}\^{}\{(2n+2)\}(t)\{ \{B\}\_\{2n+3\}(t) \textbackslash{}over 2n +
3\} \textbackslash{}right {]}\}\_\{0\}\^{}\{1\} \%\&
\textbackslash{}\textbackslash{} \& \textbackslash{}text\{\} \& −\{ 1
\textbackslash{}over (2n + 3)!\} \{\textbackslash{}mathop\{∫ \}
\}\_\{0\}\^{}\{1\}\{f\}\^{}\{(2n+3)\}(t)\{B\}\_\{ 2n+3\}(t)
dt\textbackslash{}Biggr )\}\%\& \textbackslash{}\textbackslash{} \&
=\&\{ \{b\}\_\{2n+2\} \textbackslash{}over (2n + 2)!\}
(\{f\}\^{}\{(2n+1)\}(1) − \{f\}\^{}\{(2n+1)\}(0)) \%\&
\textbackslash{}\textbackslash{} \& \textbackslash{}text\{\} \& +\{ 1
\textbackslash{}over (2n + 3)!\} \{\textbackslash{}mathop\{∫ \}
\}\_\{0\}\^{}\{1\}\{f\}\^{}\{(2n+3)\}(t)\{B\}\_\{ 2n+3\}(t) dt\%\&
\textbackslash{}\textbackslash{} \textbackslash{}end\{eqnarray*\}

en tenant compte de \{B\}\_\{2n+2\}(0) = \{B\}\_\{2n+2\}(1) =
\{b\}\_\{2n+2\} et de \{B\}\_\{2n+3\}(0) = \{B\}\_\{2n+3\}(1) = 0.

Remarque~9.7.1 On peut montrer que \textbackslash{}mathop\{∀\}x ∈
{[}0,1{]}, \textbar{}\{B\}\_\{n\}(x)\textbar{}≤ 4\{e\}\^{}\{2π\}\{ n!
\textbackslash{}over \{(2π)\}\^{}\{n\}\} ce qui permet d'avoir une
estimation du reste. Si f est à valeurs réelles, on peut obtenir une
autre estimation du reste en montrant par récurrence que les polynômes
\{B\}\_\{n\} ont les variations suivantes

̲ ̲ ̲ ̲ ̲ ̲ ̲ ̲ ̲ ̲

x

0

1∕2

1 ̲ ̲ ̲ ̲ ̲ ̲ ̲ ̲ ̲ ̲

\{B\}\_\{4n\}(x)

\{b\}\_\{4n\} \textless{} 0

↗

0

↗

\textgreater{} 0

↘

0

↘

\{b\}\_\{4n\} \textless{} 0 ̲ ̲ ̲ ̲ ̲ ̲ ̲ ̲ ̲ ̲

\{B\}\_\{4n+1\}(x)

0

↘

↗

0

↗

↘

0 ̲ ̲ ̲ ̲ ̲ ̲ ̲ ̲ ̲ ̲

\{B\}\_\{4n+2\}(x)

\{b\}\_\{4n+2\} \textgreater{} 0

↘

0

↘

\textless{} 0

↗

0

↗

\{b\}\_\{4n+2\} \textgreater{} 0 ̲ ̲ ̲ ̲ ̲ ̲ ̲ ̲ ̲ ̲

\{B\}\_\{4n+3\}(x)

0

↗

↘

0

↘

↗

0 ̲ ̲ ̲ ̲ ̲ ̲ ̲ ̲ ̲ ̲

\includegraphics{cours7x.png}

Ceci montre que les polynômes \{B\}\_\{2p\} − \{b\}\_\{2p\} sont de
signe constant sur {[}0,1{]}. On peut donc utiliser la première formule
de la moyenne, ce qui nous donne

\textbackslash{}begin\{eqnarray*\} \{\textbackslash{}mathop\{∫ \}
\}\_\{0\}\^{}\{1\}\{f\}\^{}\{(2n+2)\}(t)\{B\}\_\{ 2n+2\}(t)
dt\textbackslash{}quad \&\& \%\& \textbackslash{}\textbackslash{} \& =\&
\{b\}\_\{2n+2\}\{\textbackslash{}mathop\{∫ \}
\}\_\{0\}\^{}\{1\}\{f\}\^{}\{(2n+2)\}(t) dt +\{\textbackslash{}mathop\{∫
\} \}\_\{0\}\^{}\{1\}\{f\}\^{}\{(2n+2)\}(t)(\{B\}\_\{ 2n+2\}(t) −
\{b\}\_\{2n+2\}) dt\%\& \textbackslash{}\textbackslash{} \& =\&
\{b\}\_\{2n+2\}(\{f\}\^{}\{(2n+1)\}(1) − \{f\}\^{}\{(2n+1)\}(0)) \%\&
\textbackslash{}\textbackslash{} \& \textbackslash{}text\{\} \&
+\{f\}\^{}\{(2n+2)\}(ξ)\{\textbackslash{}mathop\{∫ \}
\}\_\{0\}\^{}\{1\}(\{B\}\_\{ 2n+2\}(t) − \{b\}\_\{2n+2\}) dt \%\&
\textbackslash{}\textbackslash{} \& =\&
\{b\}\_\{2n+2\}(\{f\}\^{}\{(2n+1)\}(1) − \{f\}\^{}\{(2n+1)\}(0)) −
\{b\}\_\{ 2n+2\}\{f\}\^{}\{(2n+2)\}(ξ) \%\&
\textbackslash{}\textbackslash{} \textbackslash{}end\{eqnarray*\}

car \{\textbackslash{}mathop\{∫ \} \}\_\{0\}\^{}\{1\}\{B\}\_\{2n+2\}(t)
dt =\{ \textbackslash{}left {[}\{ \{B\}\_\{2n+3\}(t)
\textbackslash{}over 2n+3\} \textbackslash{}right {]}\}\_\{0\}\^{}\{1\}
= 0. On obtient, en reprenant la démonstration du lemme, la formule sous
la forme

\textbackslash{}begin\{eqnarray*\} \{\textbackslash{}mathop\{∫ \}
\}\_\{0\}\^{}\{1\}f(t) dt\& =\&\{ 1 \textbackslash{}over 2\} (f(1) +
f(0)) \%\& \textbackslash{}\textbackslash{} \& \textbackslash{}text\{\}
\& −\{\textbackslash{}mathop\{∑
\}\}\_\{k=1\}\^{}\{n+1\}(\{f\}\^{}\{(2k−1)\}(1) −
\{f\}\^{}\{(2k−1)\}(0))\{ \{b\}\_\{2k\} \textbackslash{}over (2k)!\}
\%\& \textbackslash{}\textbackslash{} \& \textbackslash{}text\{\} \& +\{
\{b\}\_\{2n+2\} \textbackslash{}over (2n + 1)!\} \{f\}\^{}\{(2n+2)\}(ξ)
\%\& \textbackslash{}\textbackslash{} \textbackslash{}end\{eqnarray*\}

Exemple~9.7.1 Appliquons cette formule à f(t) =\{ 1 \textbackslash{}over
t+p\} . On va obtenir

\textbackslash{}begin\{eqnarray*\} \textbackslash{}mathop\{log\} (p + 1)
−\textbackslash{}mathop\{ log\} (p)\& =\&\{ 1 \textbackslash{}over 2\}
(\{ 1 \textbackslash{}over p + 1\} +\{ 1 \textbackslash{}over p\} ) \%\&
\textbackslash{}\textbackslash{} \& \textbackslash{}text\{\} \&
+\{\textbackslash{}mathop\{∑ \}\}\_\{k=1\}\^{}\{n+1\}(\{ 1
\textbackslash{}over \{(p + 1)\}\^{}\{2k\}\} −\{ 1 \textbackslash{}over
\{p\}\^{}\{2k\}\} )\{ \{b\}\_\{2k\} \textbackslash{}over 2k\} \%\&
\textbackslash{}\textbackslash{} \& \textbackslash{}text\{\} \& +\{ (2n
+ 2)\{b\}\_\{2n+2\} \textbackslash{}over \{ξ\}\_\{p\}\^{}\{2n+3\}\} \%\&
\textbackslash{}\textbackslash{} \textbackslash{}end\{eqnarray*\}

avec \{ξ\}\_\{p\} ∈ {[}p,p + 1{]} et donc \{ξ\}\_\{p\} ∼ p. En sommant
de p = 1 jusque N − 1, on obtient

\textbackslash{}begin\{eqnarray*\} \textbackslash{}mathop\{log\} N\& =\&
\{\textbackslash{}mathop\{∑ \}\}\_\{p=1\}\^{}\{N\}\{ 1
\textbackslash{}over p\} −\{ 1 \textbackslash{}over 2\} −\{ 1
\textbackslash{}over 2N\} +\{ \textbackslash{}mathop\{∑
\}\}\_\{k=1\}\^{}\{n+1\}(\{ 1 \textbackslash{}over \{N\}\^{}\{2k\}\} −
1)\{ \{b\}\_\{2k\} \textbackslash{}over 2k\} \%\&
\textbackslash{}\textbackslash{} \& \textbackslash{}text\{\} \& +(2n +
2)\{b\}\_\{2n+2\}\{ \textbackslash{}mathop\{∑ \}\}\_\{p=1\}\^{}\{N−1\}\{
1 \textbackslash{}over \{ξ\}\_\{p\}\^{}\{2n+3\}\} \%\&
\textbackslash{}\textbackslash{} \textbackslash{}end\{eqnarray*\}

et en utilisant \{\textbackslash{}mathop\{\textbackslash{}mathop\{∑ \}\}
\}\_\{p=N\}\^{}\{+∞\}\{ 1 \textbackslash{}over
\{ξ\}\_\{p\}\^{}\{2n+3\}\} = O(\{ 1 \textbackslash{}over
\{N\}\^{}\{2n+2\}\} ) on obtient, après amalgame de tous les termes ne
dépendant pas de N en une constante γ,

\textbackslash{}begin\{eqnarray*\} \{\textbackslash{}mathop\{∑
\}\}\_\{p=1\}\^{}\{N\}\{ 1 \textbackslash{}over p\} \& =\&
\textbackslash{}mathop\{log\} N + γ +\{ 1 \textbackslash{}over 2N\}
−\{\textbackslash{}mathop\{∑ \}\}\_\{k=1\}\^{}\{n\}\{ \{b\}\_\{2k\}
\textbackslash{}over 2k\{N\}\^{}\{2k\}\} + O(\{ 1 \textbackslash{}over
\{N\}\^{}\{2n+2\}\} )\%\& \textbackslash{}\textbackslash{}
\textbackslash{}end\{eqnarray*\}

\paragraph{9.7.2 Calcul approché d'intégrales}

Méthode des trapèzes

Soit f : {[}a,b{]} → ℝ de classe \{C\}\^{}\{2\} et p ∈ \{ℕ\}\^{}\{∗\}.
Pour k ∈ {[}0,p{]} posons \{a\}\_\{k\} = a + k\{ b−a
\textbackslash{}over p\} . On approche la fonction f par la fonction φ :
{[}a,b{]} → E qui vérifie \textbackslash{}mathop\{∀\}k ∈ {[}0,p{]},
φ(\{a\}\_\{k\}) = f(\{a\}\_\{k\}) et qui est linéaire sur chaque
intervalle {[}\{a\}\_\{k−1\},\{a\}\_\{k\}{]}. On a immédiatement
\{\textbackslash{}mathop\{∫ \}
\}\_\{\{a\}\_\{k−1\}\}\^{}\{\{a\}\_\{k\}\}φ = (\{a\}\_\{ k\} −
\{a\}\_\{k−1\})\{ f(\{a\}\_\{k\})+f(\{a\}\_\{k−1\}) \textbackslash{}over
2\} (aire d'un trapèze). D'où, \{\textbackslash{}mathop\{∫ \}
\}\_\{a\}\^{}\{b\}φ =\{ b−a \textbackslash{}over n\}
\textbackslash{}left (\{ f(a) \textbackslash{}over 2\}
+\{\textbackslash{}mathop\{ \textbackslash{}mathop\{∑ \}\}
\}\_\{k=1\}\^{}\{p−1\}f(\{a\}\_\{k\}) +\{ f(b) \textbackslash{}over 2\}
\textbackslash{}right ) = \{T\}\_\{p\} avec les notations du paragraphe
précédent. On prendra donc comme valeur approchée de I
=\{\textbackslash{}mathop\{∫ \} \}\_\{a\}\^{}\{b\}f,
\textbackslash{}overline\{I\} =\{\textbackslash{}mathop\{∫ \}
\}\_\{a\}\^{}\{b\}φ = \{T\}\_\{p\}.

Majoration de l'erreur~: on cherche à majorer \textbar{}I
−\textbackslash{}overline\{I\}\textbar{} =
\textbar{}\{\textbackslash{}mathop\{∫ \} \}\_\{a\}\^{}\{b\}(f −
φ)\textbar{}. Posons g = f − φ et calculons à l'aide d'une intégration
par parties l'intégrale suivante (en remarquant que la restriction de g
à {[}\{a\}\_\{k−1\},\{a\}\_\{k\}{]} est de classe \{C\}\^{}\{2\} avec
g'`= f'')

\textbackslash{}begin\{eqnarray*\} \{\textbackslash{}mathop\{∫ \}
\}\_\{\{a\}\_\{k−1\}\}\^{}\{\{a\}\_\{k\} \}f'`(t)(t −
\{a\}\_\{k−1\})(\{a\}\_\{k\} − t) dt\&\& \%\&
\textbackslash{}\textbackslash{} \& =\& \{\textbackslash{}mathop\{∫ \}
\}\_\{\{a\}\_\{k−1\}\}\^{}\{\{a\}\_\{k\} \}g'`(t)(t −
\{a\}\_\{k−1\})(\{a\}\_\{k\} − t) dt \%\&
\textbackslash{}\textbackslash{} \& =\&\{ \textbackslash{}left
{[}g'(t)(t − \{a\}\_\{k−1\})(\{a\}\_\{k\} − t)\textbackslash{}right
{]}\}\_\{\{a\}\_\{k−1\}\}\^{}\{\{a\}\_\{k\} \}
+\{\textbackslash{}mathop\{∫ \} \}\_\{\{a\}\_\{k−1\}\}\^{}\{\{a\}\_\{k\}
\}g'(t)(2t − \{a\}\_\{k−1\} − \{a\}\_\{k\}) dt\%\&
\textbackslash{}\textbackslash{} \& =\& \{\textbackslash{}mathop\{∫ \}
\}\_\{\{a\}\_\{k−1\}\}\^{}\{\{a\}\_\{k\} \}g'(t)(2t − \{a\}\_\{k−1\} −
\{a\}\_\{k\}) dt \%\& \textbackslash{}\textbackslash{} \& =\&\{
\textbackslash{}left {[}g(t)(2t − \{a\}\_\{k−1\} −
\{a\}\_\{k\})\textbackslash{}right
{]}\}\_\{\{a\}\_\{k−1\}\}\^{}\{\{a\}\_\{k\} \} −
2\{\textbackslash{}mathop\{∫ \} \}\_\{\{a\}\_\{k−1\}\}\^{}\{\{a\}\_\{k\}
\}g(t) dt \%\& \textbackslash{}\textbackslash{} \& =\&
−2\{\textbackslash{}mathop\{∫ \}
\}\_\{\{a\}\_\{k−1\}\}\^{}\{\{a\}\_\{k\} \}g(t) dt \%\&
\textbackslash{}\textbackslash{} \textbackslash{}end\{eqnarray*\}

puisque g(\{a\}\_\{k−1\}) = g(\{a\}\_\{k\}) = 0. On a donc

\textbackslash{}begin\{eqnarray*\} \textbackslash{}left
\textbar{}\{\textbackslash{}mathop\{∫ \}
\}\_\{\{a\}\_\{k−1\}\}\^{}\{\{a\}\_\{k\} \}g(t) dt\textbackslash{}right
\textbar{}\& =\&\{ 1 \textbackslash{}over 2\} \textbackslash{}left
\textbar{}\{\textbackslash{}mathop\{∫ \}
\}\_\{\{a\}\_\{k−1\}\}\^{}\{\{a\}\_\{k\} \}f''(t)(t −
\{a\}\_\{k−1\})(\{a\}\_\{k\} − t) dt\textbackslash{}right \textbar{}\%\&
\textbackslash{}\textbackslash{} \& ≤\&\{ \{M\}\_\{2\}
\textbackslash{}over 2\} \{\textbackslash{}mathop\{∫ \}
\}\_\{\{a\}\_\{k−1\}\}\^{}\{\{a\}\_\{k\} \}(t −
\{a\}\_\{k−1\})(\{a\}\_\{k\} − t) dt \%\&
\textbackslash{}\textbackslash{} \& =\&\{ \{M\}\_\{2\}
\textbackslash{}over 12\} \{(\{a\}\_\{k\} − \{a\}\_\{k−1\})\}\^{}\{3\}
=\{ \{M\}\_\{2\}\{(b − a)\}\^{}\{3\} \textbackslash{}over
12\{p\}\^{}\{3\}\} \%\& \textbackslash{}\textbackslash{}
\textbackslash{}end\{eqnarray*\}

En sommant de k = 1 à p, on obtient

\textbar{}I −\textbackslash{}overline\{I\}\textbar{}≤\{ \{M\}\_\{2\}\{(b
− a)\}\^{}\{3\} \textbackslash{}over 12\{p\}\^{}\{2\}\}

Application de la formule d'Euler-Mac Laurin

Soit alors f : {[}a,b{]} → E de classe \{C\}\^{}\{2n+1\} et p ∈
\{ℕ\}\^{}\{∗\}. Pour k ∈ {[}0,p{]} posons \{a\}\_\{k\} = a + k\{ b−a
\textbackslash{}over p\} ~; appliquons la formule précédente à
t\textbackslash{}mathrel\{↦\}f(\{a\}\_\{k−1\} + t\{ b−a
\textbackslash{}over p\} ). On obtient alors (avec le changement de
variable x = \{a\}\_\{k−1\} + t\{ b−a \textbackslash{}over p\} )

\textbackslash{}begin\{eqnarray*\} \{\textbackslash{}mathop\{∫ \}
\}\_\{\{a\}\_\{k−1\}\}\^{}\{\{a\}\_\{k\} \}f(x) dx\& =\&\{ b − a
\textbackslash{}over n\} \{\textbackslash{}mathop\{∫ \}
\}\_\{0\}\^{}\{1\}f(\{a\}\_\{ k−1\} + t\{ b − a \textbackslash{}over p\}
) dt\%\& \textbackslash{}\textbackslash{} \& =\&\{ b − a
\textbackslash{}over 2p\} (f(\{a\}\_\{k−1\}) + f(\{a\}\_\{k\})) \%\&
\textbackslash{}\textbackslash{} \& −\{\textbackslash{}mathop\{∑
\}\}\_\{k=1\}\^{}\{n\}\{ \{(b − a)\}\^{}\{2k\} \textbackslash{}over
\{p\}\^{}\{2k\}\} (\{f\}\^{}\{(2k−1)\}(\{a\}\_\{ k\}) −
\{f\}\^{}\{(2k−1)\}(\{a\}\_\{ k−1\}))\{ \{b\}\_\{2k\}
\textbackslash{}over (2k)!\} \&\%\& \textbackslash{}\textbackslash{} \&
\textbackslash{}text\{\} \& +\{ \{(b − a)\}\^{}\{2n+2\}
\textbackslash{}over \{p\}\^{}\{2n+2\}(2n + 1)!\} \{ρ\}\_\{n,k\} \%\&
\textbackslash{}\textbackslash{} \textbackslash{}end\{eqnarray*\}

avec \{ρ\}\_\{n,k\} =\{\textbackslash{}mathop\{∫ \}
\}\_\{0\}\^{}\{1\}\{f\}\^{}\{(2n+1)\}(\{a\}\_\{k−1\} + t\{ b−a
\textbackslash{}over p\} )\{B\}\_\{2n+1\}(t) dt. Posons \{M\}\_\{2n+1\}
=\{\textbackslash{}mathop\{
sup\}\}\_\{t∈{[}a,b{]}\}\textbackslash{}\textbar{}\{f\}\^{}\{(2n+1)\}(t)\textbackslash{}\textbar{}.
On a alors
\textbackslash{}\textbar{}\{ρ\}\_\{n,k\}\textbackslash{}\textbar{} ≤
\{M\}\_\{2n+1\}\{\textbackslash{}mathop\{∫ \}
\}\_\{0\}\^{}\{1\}\textbar{}\{B\}\_\{2n+1\}(t)\textbar{} dt. Sommons
alors les égalités ci dessus, en posant

\{T\}\_\{p\} =\{ b − a \textbackslash{}over n\} \textbackslash{}left (\{
f(a) \textbackslash{}over 2\} +\{ \textbackslash{}mathop\{∑
\}\}\_\{k=1\}\^{}\{p−1\}f(\{a\}\_\{ k\}) +\{ f(b) \textbackslash{}over
2\} \textbackslash{}right )

on obtient,

\textbackslash{}begin\{eqnarray*\} \{\textbackslash{}mathop\{∫ \}
\}\_\{a\}\^{}\{b\}f(x) dx\& =\& \{T\}\_\{ p\}
−\{\textbackslash{}mathop\{∑ \}\}\_\{k=1\}\^{}\{n\}\{ \{(b −
a)\}\^{}\{2k\} \textbackslash{}over \{p\}\^{}\{2k\}\}
(\{f\}\^{}\{(2k−1)\}(b) − \{f\}\^{}\{(2k−1)\}(a))\{ \{b\}\_\{2k\}
\textbackslash{}over (2k)!\} \%\& \textbackslash{}\textbackslash{} \&
\textbackslash{}text\{\} \& +\{ \{(b − a)\}\^{}\{2n+2\}
\textbackslash{}over \{p\}\^{}\{2n+2\}(2n + 1)!\} \{S\}\_\{n,p\} \%\&
\textbackslash{}\textbackslash{} \textbackslash{}end\{eqnarray*\}

avec \{S\}\_\{n,p\} =\{\textbackslash{}mathop\{
\textbackslash{}mathop\{∑ \}\} \}\_\{k=1\}\^{}\{p\}\{ρ\}\_\{n,k\} et
donc \textbackslash{}\textbar{}\{S\}\_\{n,p\}\textbackslash{}\textbar{}
≤\{\textbackslash{}mathop\{\textbackslash{}mathop\{∑ \}\}
\}\_\{k=1\}\^{}\{p\}\textbackslash{}\textbar{}\{ρ\}\_\{n,k\}\textbackslash{}\textbar{}
≤ p\{M\}\_\{2n+1\}\{\textbackslash{}mathop\{∫ \}
\}\_\{0\}\^{}\{1\}\textbar{}\{B\}\_\{2n+1\}(t)\textbar{} dt. On obtient
donc

Théorème~9.7.4 Soit f : {[}a,b{]} → E de classe \{C\}\^{}\{2n+1\} et p ∈
\{ℕ\}\^{}\{∗\}. Pour k ∈ {[}0,p{]} posons \{a\}\_\{k\} = a + k\{ b−a
\textbackslash{}over p\} . Soit \{T\}\_\{p\} =\{ b−a
\textbackslash{}over n\} \textbackslash{}left (\{ f(a)
\textbackslash{}over 2\} +\{\textbackslash{}mathop\{
\textbackslash{}mathop\{∑ \}\} \}\_\{k=1\}\^{}\{p−1\}f(\{a\}\_\{k\}) +\{
f(b) \textbackslash{}over 2\} \textbackslash{}right ) et \{M\}\_\{2n+1\}
=\{\textbackslash{}mathop\{
sup\}\}\_\{t∈{[}a,b{]}\}\textbackslash{}\textbar{}\{f\}\^{}\{(2n+1)\}(t)\textbackslash{}\textbar{}.
Alors

\textbackslash{}begin\{eqnarray*\} \{\textbackslash{}mathop\{∫ \}
\}\_\{a\}\^{}\{b\}f(x) dx\& =\& \{T\}\_\{ p\}
−\{\textbackslash{}mathop\{∑ \}\}\_\{k=1\}\^{}\{n\}\{ \{b\}\_\{2k\}\{(b
− a)\}\^{}\{2k\} \textbackslash{}over \{p\}\^{}\{2k\}(2k)!\}
(\{f\}\^{}\{(2k−1)\}(b) − \{f\}\^{}\{(2k−1)\}(a))\%\&
\textbackslash{}\textbackslash{} \& \textbackslash{}text\{\} \& +\{ \{(b
− a)\}\^{}\{2n+2\} \textbackslash{}over \{p\}\^{}\{2n+1\}(2n + 1)!\}
\{R\}\_\{n,p\} \%\& \textbackslash{}\textbackslash{}
\textbackslash{}end\{eqnarray*\}

avec \textbackslash{}\textbar{}\{R\}\_\{n,p\}\textbackslash{}\textbar{}
≤ \{M\}\_\{2n+1\}\{\textbackslash{}mathop\{∫ \}
\}\_\{0\}\^{}\{1\}\textbar{}\{B\}\_\{2n+1\}(t)\textbar{} dt.

Remarque~9.7.2 Ce théorème nous donne un développement à un ordre
arbitraire de la différence entre l'intégrale et sa valeur approchée par
la méthode des trapèzes

Méthode de Simpson

La formule d'Euler-Mac Laurin, nous montre que si f est de classe
\{C\}\^{}\{4\}, on a

I − \{T\}\_\{p\} =\{ λ \textbackslash{}over \{p\}\^{}\{2\}\} + O(\{ 1
\textbackslash{}over \{p\}\^{}\{4\}\} )

On a donc également I − \{T\}\_\{2p\} =\{ λ \textbackslash{}over
4\{p\}\^{}\{2\}\} + O(\{ 1 \textbackslash{}over \{p\}\^{}\{4\}\} ) puis
4(I − \{T\}\_\{2p\}) − (I − \{T\}\_\{p\}) = O(\{ 1 \textbackslash{}over
\{p\}\^{}\{4\}\} ) ou encore

I −\{ 1 \textbackslash{}over 3\} (4\{T\}\_\{2p\} − \{T\}\_\{p\}) = O(\{
1 \textbackslash{}over \{p\}\^{}\{4\}\} )

Posons donc \{a\}\_\{k\} = a + k\{ b−a \textbackslash{}over 2p\} , on a

\textbackslash{}begin\{eqnarray*\}\{ S\}\_\{p\}\& =\&\{ (b − a)
\textbackslash{}over 6p\} (\{T\}\_\{p\} + 4\{T\}\_\{2p\}) \%\&
\textbackslash{}\textbackslash{} \& =\&\{ (b − a) \textbackslash{}over
6p\} (f(a) + 4f(\{a\}\_\{1\}) + 2f(\{a\}\_\{2\}) + 4f(\{a\}\_\{3\}) +
\textbackslash{}mathop\{\textbackslash{}mathop\{\ldots{}\}\}\%\&
\textbackslash{}\textbackslash{} \& \textbackslash{}text\{\} \&
+2f(\{a\}\_\{2p−2\}) + 4f(\{a\}\_\{2p−1\}) + f(b)) \%\&
\textbackslash{}\textbackslash{} \textbackslash{}end\{eqnarray*\}

On sait donc que l'on a une majoration du type

\textbar{}I − \{S\}\_\{p\}\textbar{}≤\{ M \textbackslash{}over
\{p\}\^{}\{4\}\}

Remarque~9.7.3 On peut montrer qu'en fait \textbar{}I −
\{S\}\_\{p\}\textbar{}≤\{ \{M\}\_\{4\}\{(b−a)\}\^{}\{5\}
\textbackslash{}over 2880\{p\}\^{}\{4\}\} avec \{M\}\_\{4\}
=\{\textbackslash{}mathop\{
sup\}\}\_\{t∈{[}a,b{]}\}\textbackslash{}\textbar{}\{f\}\^{}\{(4)\}(t)\textbackslash{}\textbar{},
majoration de peu d'intérêt dans la pratique vu la difficulté qu'il y a
habituellement à trouver un majorant de \{M\}\_\{4\}.

Méthode de Romberg

Elle consiste à généraliser la méthode qui nous a fait passer de la
méthode des trapèzes à la méthode de Simpson en utilisant le calcul de
\{T\}\_\{p\},\{T\}\_\{2p\},\{T\}\_\{4p\},\textbackslash{}mathop\{\textbackslash{}mathop\{\ldots{}\}\},\{T\}\_\{\{2\}\^{}\{k\}p\}
pour éliminer successivement les termes en \{ 1 \textbackslash{}over
\{p\}\^{}\{2\}\} ,\{ 1 \textbackslash{}over \{p\}\^{}\{4\}\}
,\textbackslash{}mathop\{\textbackslash{}mathop\{\ldots{}\}\},\{ 1
\textbackslash{}over \{p\}\^{}\{2k\}\} du développement asymptotique
donné par la formule d'Euler-Mac Laurin.

\paragraph{9.7.3 La méthode de Laplace}

C'est une méthode classique de recherche d'équivalents d'intégrales
dépendant d'un paramètre (ici n) consistant à remarquer qu'un intégrande
du type f(t)\{e\}\^{}\{ng(t)\} va privilégier, pour n grand, les valeurs
de t pour lesquelles la fonction g atteint son maximum (car si x
\textless{} y, \{e\}\^{}\{nx\} = o(\{e\}\^{}\{ny\})).

Proposition~9.7.5 Soit f :{]}a,b{[}→ ℝ continue intégrable sur
{]}a,b{[}. Soit g :{]}a,b{[}→ ℝ de classe \{C\}\^{}\{2\}. On suppose que
g atteint son maximum en un point c ∈{]}a,b{[} avec g''(c) \textless{}
0, f(c)\textbackslash{}mathrel\{≠\}0 et que \textbackslash{}mathop\{∀\}η
\textgreater{} 0,
\{\textbackslash{}mathop\{sup\}\}\_\{\textbar{}t−c\textbar{}≥η\}g(t)
\textless{} g(c). Alors, quand n tend vers + ∞

\{\textbackslash{}mathop\{∫ \} \}\_\{a\}\^{}\{b\}f(t)\{e\}\^{}\{ng(t)\}
dt ∼ f(c)\{e\}\^{}\{ng(c)\}\textbackslash{}sqrt\{\{ 2π
\textbackslash{}over n\textbar{}g''(c)\textbar{}\} \}

Démonstration Quitte à changer f en − f, on peut supposer f(c)
\textgreater{} 0. Soit α tel que 0 \textless{} α
\textless{}\textbackslash{}mathop\{
min\}(\textbar{}g''(c)\textbar{},f(c)) et soit η \textgreater{} 0 tel
que \textbar{}t − c\textbar{}≤ η ⇒\textbar{}f(t) − f(c)\textbar{}≤ α et
\textbar{}g(t) − g(c) −\{ \{(t−c)\}\^{}\{2\} \textbackslash{}over 2\}
g''(c)\textbar{} \textless{} α\{ \{(t−c)\}\^{}\{2\} \textbackslash{}over
2\} (puisque g'(c) = 0). Sur {[}c − η,c + η{]}, on a f(c) − α
\textless{} f(t) \textless{} f(c) + α et

g(c) +\{ \{(t − c)\}\^{}\{2\} \textbackslash{}over 2\} (g'`(c) − α) ≤
g(t) ≤ g(c) +\{ \{(t − c)\}\^{}\{2\} \textbackslash{}over 2\} (g''(c) +
α)

On obtient donc

\textbackslash{}begin\{eqnarray*\} (f(c) −
α)\{e\}\^{}\{ng(c)\}\{\textbackslash{}mathop\{∫ \}
\}\_\{c−η\}\^{}\{c+η\}\textbackslash{}mathop\{ exp\}
\textbackslash{}left (n\{ \{(t − c)\}\^{}\{2\} \textbackslash{}over 2\}
(g'`(c) − α)\textbackslash{}right ) dt\&\& \%\&
\textbackslash{}\textbackslash{} \& ≤\& \{\textbackslash{}mathop\{∫ \}
\}\_\{c−η\}\^{}\{c+η\}f(t)\{e\}\^{}\{ng(t)\} dt \%\&
\textbackslash{}\textbackslash{} \& ≤\& (f(c) +
α)\{e\}\^{}\{ng(c)\}\{\textbackslash{}mathop\{∫ \}
\}\_\{c−η\}\^{}\{c+η\}\textbackslash{}mathop\{ exp\}
\textbackslash{}left (n\{ \{(t − c)\}\^{}\{2\} \textbackslash{}over 2\}
(g''(c) + α)\textbackslash{}right ) dt\%\&
\textbackslash{}\textbackslash{} \textbackslash{}end\{eqnarray*\}

Mais, si λ \textless{} 0, le changement de variable u =
\textbackslash{}sqrt\{\{ n\textbar{}λ\textbar{} \textbackslash{}over 2\}
\} (t − c) donne

\textbackslash{}begin\{eqnarray*\} \{\textbackslash{}mathop\{∫ \}
\}\_\{c−η\}\^{}\{c+η\}\textbackslash{}mathop\{ exp\}
\textbackslash{}left (nλ\{ \{(t − c)\}\^{}\{2\} \textbackslash{}over 2\}
\textbackslash{}right ) dt\&\& \%\& \textbackslash{}\textbackslash{} \&
=\& \textbackslash{}sqrt\{\{ 2 \textbackslash{}over
n\textbar{}λ\textbar{}\} \}\{\textbackslash{}mathop\{∫ \} \}\_\{
−\textbackslash{}sqrt\{\{ n\textbar{}λ\textbar{} \textbackslash{}over
2\} \} η\}\^{}\{\textbackslash{}sqrt\{\{ n\textbar{}λ\textbar{}
\textbackslash{}over 2\} \} η\}\{e\}\^{}\{−\{u\}\^{}\{2\} \} du \%\&
\textbackslash{}\textbackslash{} \& ∼\& \textbackslash{}sqrt\{\{ 2
\textbackslash{}over n\textbar{}λ\textbar{}\}
\}\{\textbackslash{}mathop\{∫ \}
\}\_\{−∞\}\^{}\{+∞\}\{e\}\^{}\{−\{u\}\^{}\{2\} \} du =
\textbackslash{}sqrt\{\{ 2π \textbackslash{}over
n\textbar{}λ\textbar{}\} \}\%\& \textbackslash{}\textbackslash{}
\textbackslash{}end\{eqnarray*\}

Posons \{I\}\_\{n\} =
\textbackslash{}sqrt\{n\}\{e\}\^{}\{−ng(c)\}\{\textbackslash{}mathop\{∫
\} \}\_\{c−η\}\^{}\{c+η\}f(t)\{e\}\^{}\{ng(t)\} dt. On a donc
\{u\}\_\{n\} ≤ \{I\}\_\{n\} ≤ \{v\}\_\{n\} avec

\textbackslash{}begin\{eqnarray*\}\{ u\}\_\{n\}\& =\& (f(c) −
α)\textbackslash{}sqrt\{n\}\{\textbackslash{}mathop\{∫ \}
\}\_\{c−η\}\^{}\{c+η\}\textbackslash{}mathop\{ exp\}
\textbackslash{}left (n\{ \{(t − c)\}\^{}\{2\} \textbackslash{}over 2\}
(g'`(c) − α)\textbackslash{}right ) dt\%\&
\textbackslash{}\textbackslash{} \& ∼\& (f(c) −
α)\textbackslash{}sqrt\{\{ 2π \textbackslash{}over \textbar{}g''(c) −
α\textbar{}\} \} \%\& \textbackslash{}\textbackslash{}
\textbackslash{}end\{eqnarray*\}

et de même \{v\}\_\{n\} ∼ (f(c) + α)\textbackslash{}sqrt\{\{ 2π
\textbackslash{}over \textbar{}g''(c)+α\textbar{}\} \}. Donnons nous ε
\textgreater{} 0 et soit α tel que

\begin{itemize}
\itemsep1pt\parskip0pt\parsep0pt
\item
  (i) 0 \textless{} α \textless{}\textbackslash{}mathop\{
  min\}(\textbar{}g''(c)\textbar{},f(c))
\item
  (ii) (f(c) − α)\textbackslash{}sqrt\{\{ 2π \textbackslash{}over
  \textbar{}g'`(c)−α\textbar{}\} \} \textgreater{}
  f(c)\textbackslash{}sqrt\{\{ 2π \textbackslash{}over
  \textbar{}g''(c)\textbar{}\} \} −\{ ε \textbackslash{}over 2\}
\item
  (iii) (f(c) + α)\textbackslash{}sqrt\{\{ 2π \textbackslash{}over
  \textbar{}g'`(c)+α\textbar{}\} \} \textless{}
  f(c)\textbackslash{}sqrt\{\{ 2π \textbackslash{}over
  \textbar{}g''(c)\textbar{}\} \} +\{ ε \textbackslash{}over 2\}
\end{itemize}

On prend le η correspondant comme ci-dessus. Alors comme
\textbackslash{}mathop\{lim\}\{u\}\_\{n\} = (f(c) −
α)\textbackslash{}sqrt\{\{ 2π \textbackslash{}over
\textbar{}g''(c)−α\textbar{}\} \} et
\textbackslash{}mathop\{lim\}\{v\}\_\{n\} = (f(c) +
α)\textbackslash{}sqrt\{\{ 2π \textbackslash{}over
\textbar{}g''(c)+α\textbar{}\} \}, il existe N ∈ ℕ tel que

\textbackslash{}begin\{eqnarray*\} n ≥ N ⇒ f(c)\textbackslash{}sqrt\{\{
2π \textbackslash{}over \textbar{}g'`(c)\textbar{}\} \}−\{ ε
\textbackslash{}over 2\} \textless{} \{u\}\_\{n\} ≤ \{v\}\_\{n\}
\textless{} f(c)\textbackslash{}sqrt\{\{ 2π \textbackslash{}over
\textbar{}g''(c)\textbar{}\} \} +\{ ε \textbackslash{}over 2\} \& \&
\%\& \textbackslash{}\textbackslash{} \textbackslash{}end\{eqnarray*\}

Pour n ≥ N on a donc f(c)\textbackslash{}sqrt\{\{ 2π
\textbackslash{}over \textbar{}g'`(c)\textbar{}\} \} −\{ ε
\textbackslash{}over 2\} \textless{} \{I\}\_\{n\} \textless{}
f(c)\textbackslash{}sqrt\{\{ 2π \textbackslash{}over
\textbar{}g''(c)\textbar{}\} \} +\{ ε \textbackslash{}over 2\} .

Soit M =\{\textbackslash{}mathop\{
sup\}\}\_\{\textbar{}t−c\textbar{}≥η\}g(t) \textless{} g(c). On a alors

\textbackslash{}left
\textbar{}\textbackslash{}sqrt\{n\}\{e\}\^{}\{−ng(c)\}\{\textbackslash{}mathop\{∫
\} \}\_\{\textbar{}t−c\textbar{}≥η\}f(t)\{e\}\^{}\{ng(t)\}
dt\textbackslash{}right
\textbar{}≤\textbackslash{}sqrt\{n\}\{e\}\^{}\{−n(g(c)−M)\}\{\textbackslash{}mathop\{∫
\} \}\_\{a\}\^{}\{b\}\textbar{}f(t)\textbar{} dt

qui tend vers 0 quand n tend vers + ∞. Soit donc N' ∈ ℕ tel que n ≤ N'
⇒\textbackslash{}sqrt\{n\}\{e\}\^{}\{−n(g(c)−M)\}\{\textbackslash{}mathop\{∫
\} \}\_\{a\}\^{}\{b\}\textbar{}f(t)\textbar{} dt \textless{}\{ ε
\textbackslash{}over 2\} . Alors, pour n ≥\textbackslash{}mathop\{
max\}(N,N'), on a

\textbackslash{}begin\{eqnarray*\} f(c)\textbackslash{}sqrt\{\{ 2π
\textbackslash{}over \textbar{}g'`(c)\textbar{}\} \}− ε\& \textless{}\&
\textbackslash{}sqrt\{n\}\{e\}\^{}\{−ng(c)\}\{\textbackslash{}mathop\{∫
\} \}\_\{a\}\^{}\{b\}f(t)\{e\}\^{}\{ng(t)\} dt\%\&
\textbackslash{}\textbackslash{} \& \textless{}\&
f(c)\textbackslash{}sqrt\{\{ 2π \textbackslash{}over
\textbar{}g''(c)\textbar{}\} \} + ε \%\&
\textbackslash{}\textbackslash{} \textbackslash{}end\{eqnarray*\}

ce qui montre le résultat.

Remarque~9.7.4 Pour appliquer la méthode précédente, il suffit en fait
qu'il existe un \{n\}\_\{0\} ∈ ℕ tel que \{\textbackslash{}mathop\{∫ \}
\}\_\{a\}\^{}\{b\}\textbar{}f(t)\textbar{}\{e\}\^{}\{\{n\}\_\{0\}g(t)\}
dt converge~: on peut alors écrire, en posant \{f\}\_\{1\}(t) =
f(t)\{e\}\^{}\{\{n\}\_\{0\}g(t)\}

\textbackslash{}begin\{eqnarray*\} \{\textbackslash{}mathop\{∫ \}
\}\_\{a\}\^{}\{b\}f(t)\{e\}\^{}\{ng(t)\} dt\& =\&
\{\textbackslash{}mathop\{∫ \} \}\_\{a\}\^{}\{b\}\{f\}\_\{
1\}(t)\{e\}\^{}\{(n−\{n\}\_\{0\})g(t)\} dt \%\&
\textbackslash{}\textbackslash{} \& ∼\&
\{f\}\_\{1\}(c)\{e\}\^{}\{(n−\{n\}\_\{0\})g(c)\}\textbackslash{}sqrt\{\{
2π \textbackslash{}over (n − \{n\}\_\{0\})\textbar{}g'`(c)\textbar{}\}
\}\%\& \textbackslash{}\textbackslash{} \& ∼\&
f(c)\{e\}\^{}\{ng(c)\}\textbackslash{}sqrt\{\{ 2π \textbackslash{}over
n\textbar{}g''(c)\textbar{}\} \} \%\& \textbackslash{}\textbackslash{}
\textbackslash{}end\{eqnarray*\}

Exemple~9.7.2 Ecrivons

n! = Γ(n + 1) =\{\textbackslash{}mathop\{∫ \}
\}\_\{0\}\^{}\{+∞\}\{t\}\^{}\{n\}\{e\}\^{}\{−t\} dt =
\{n\}\^{}\{n+1\}\{\textbackslash{}mathop\{∫ \}
\}\_\{0\}\^{}\{+∞\}\{(u\{e\}\^{}\{−u\})\}\^{}\{n\} du

avec le changement de variable t = nu. Pour trouver un équivalent de
\{\textbackslash{}mathop\{∫ \}
\}\_\{0\}\^{}\{+∞\}\{(u\{e\}\^{}\{−u\})\}\^{}\{n\} du, on peut appliquer
la méthode de Laplace, en tenant compte de la remarque ci-dessus avec
\{n\}\_\{0\} = 1. On prend donc f(u) = 1, g(u) =\textbackslash{}mathop\{
log\} (u\{e\}\^{}\{−u\}) = −u +\textbackslash{}mathop\{ log\} u qui
atteint son maximum au point 1 avec g''(1) = −1, g(1) = −1. D'où

\{\textbackslash{}mathop\{∫ \}
\}\_\{0\}\^{}\{+∞\}\{(u\{e\}\^{}\{−u\})\}\^{}\{n\} du ∼
\{e\}\^{}\{−n\}\textbackslash{}sqrt\{\{ 2π \textbackslash{}over n\} \}

et donc n! ∼\textbackslash{}sqrt\{2πn\}\{ \{n\}\^{}\{n\}
\textbackslash{}over \{e\}\^{}\{n\}\} .

{[}\href{coursse57.html}{next}{]} {[}\href{coursse55.html}{prev}{]}
{[}\href{coursse55.html\#tailcoursse55.html}{prev-tail}{]}
{[}\href{coursse56.html}{front}{]}
{[}\href{coursch10.html\#coursse56.html}{up}{]}

\end{document}
