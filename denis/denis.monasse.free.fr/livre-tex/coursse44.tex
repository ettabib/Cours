\documentclass[]{article}
\usepackage[T1]{fontenc}
\usepackage{lmodern}
\usepackage{amssymb,amsmath}
\usepackage{ifxetex,ifluatex}
\usepackage{fixltx2e} % provides \textsubscript
% use upquote if available, for straight quotes in verbatim environments
\IfFileExists{upquote.sty}{\usepackage{upquote}}{}
\ifnum 0\ifxetex 1\fi\ifluatex 1\fi=0 % if pdftex
  \usepackage[utf8]{inputenc}
\else % if luatex or xelatex
  \ifxetex
    \usepackage{mathspec}
    \usepackage{xltxtra,xunicode}
  \else
    \usepackage{fontspec}
  \fi
  \defaultfontfeatures{Mapping=tex-text,Scale=MatchLowercase}
  \newcommand{\euro}{€}
\fi
% use microtype if available
\IfFileExists{microtype.sty}{\usepackage{microtype}}{}
\ifxetex
  \usepackage[setpagesize=false, % page size defined by xetex
              unicode=false, % unicode breaks when used with xetex
              xetex]{hyperref}
\else
  \usepackage[unicode=true]{hyperref}
\fi
\hypersetup{breaklinks=true,
            bookmarks=true,
            pdfauthor={},
            pdftitle={Monotonie, continuite},
            colorlinks=true,
            citecolor=blue,
            urlcolor=blue,
            linkcolor=magenta,
            pdfborder={0 0 0}}
\urlstyle{same}  % don't use monospace font for urls
\setlength{\parindent}{0pt}
\setlength{\parskip}{6pt plus 2pt minus 1pt}
\setlength{\emergencystretch}{3em}  % prevent overfull lines
\setcounter{secnumdepth}{0}
 
/* start css.sty */
.cmr-5{font-size:50%;}
.cmr-7{font-size:70%;}
.cmmi-5{font-size:50%;font-style: italic;}
.cmmi-7{font-size:70%;font-style: italic;}
.cmmi-10{font-style: italic;}
.cmsy-5{font-size:50%;}
.cmsy-7{font-size:70%;}
.cmex-7{font-size:70%;}
.cmex-7x-x-71{font-size:49%;}
.msbm-7{font-size:70%;}
.cmtt-10{font-family: monospace;}
.cmti-10{ font-style: italic;}
.cmbx-10{ font-weight: bold;}
.cmr-17x-x-120{font-size:204%;}
.cmsl-10{font-style: oblique;}
.cmti-7x-x-71{font-size:49%; font-style: italic;}
.cmbxti-10{ font-weight: bold; font-style: italic;}
p.noindent { text-indent: 0em }
td p.noindent { text-indent: 0em; margin-top:0em; }
p.nopar { text-indent: 0em; }
p.indent{ text-indent: 1.5em }
@media print {div.crosslinks {visibility:hidden;}}
a img { border-top: 0; border-left: 0; border-right: 0; }
center { margin-top:1em; margin-bottom:1em; }
td center { margin-top:0em; margin-bottom:0em; }
.Canvas { position:relative; }
li p.indent { text-indent: 0em }
.enumerate1 {list-style-type:decimal;}
.enumerate2 {list-style-type:lower-alpha;}
.enumerate3 {list-style-type:lower-roman;}
.enumerate4 {list-style-type:upper-alpha;}
div.newtheorem { margin-bottom: 2em; margin-top: 2em;}
.obeylines-h,.obeylines-v {white-space: nowrap; }
div.obeylines-v p { margin-top:0; margin-bottom:0; }
.overline{ text-decoration:overline; }
.overline img{ border-top: 1px solid black; }
td.displaylines {text-align:center; white-space:nowrap;}
.centerline {text-align:center;}
.rightline {text-align:right;}
div.verbatim {font-family: monospace; white-space: nowrap; text-align:left; clear:both; }
.fbox {padding-left:3.0pt; padding-right:3.0pt; text-indent:0pt; border:solid black 0.4pt; }
div.fbox {display:table}
div.center div.fbox {text-align:center; clear:both; padding-left:3.0pt; padding-right:3.0pt; text-indent:0pt; border:solid black 0.4pt; }
div.minipage{width:100%;}
div.center, div.center div.center {text-align: center; margin-left:1em; margin-right:1em;}
div.center div {text-align: left;}
div.flushright, div.flushright div.flushright {text-align: right;}
div.flushright div {text-align: left;}
div.flushleft {text-align: left;}
.underline{ text-decoration:underline; }
.underline img{ border-bottom: 1px solid black; margin-bottom:1pt; }
.framebox-c, .framebox-l, .framebox-r { padding-left:3.0pt; padding-right:3.0pt; text-indent:0pt; border:solid black 0.4pt; }
.framebox-c {text-align:center;}
.framebox-l {text-align:left;}
.framebox-r {text-align:right;}
span.thank-mark{ vertical-align: super }
span.footnote-mark sup.textsuperscript, span.footnote-mark a sup.textsuperscript{ font-size:80%; }
div.tabular, div.center div.tabular {text-align: center; margin-top:0.5em; margin-bottom:0.5em; }
table.tabular td p{margin-top:0em;}
table.tabular {margin-left: auto; margin-right: auto;}
div.td00{ margin-left:0pt; margin-right:0pt; }
div.td01{ margin-left:0pt; margin-right:5pt; }
div.td10{ margin-left:5pt; margin-right:0pt; }
div.td11{ margin-left:5pt; margin-right:5pt; }
table[rules] {border-left:solid black 0.4pt; border-right:solid black 0.4pt; }
td.td00{ padding-left:0pt; padding-right:0pt; }
td.td01{ padding-left:0pt; padding-right:5pt; }
td.td10{ padding-left:5pt; padding-right:0pt; }
td.td11{ padding-left:5pt; padding-right:5pt; }
table[rules] {border-left:solid black 0.4pt; border-right:solid black 0.4pt; }
.hline hr, .cline hr{ height : 1px; margin:0px; }
.tabbing-right {text-align:right;}
span.TEX {letter-spacing: -0.125em; }
span.TEX span.E{ position:relative;top:0.5ex;left:-0.0417em;}
a span.TEX span.E {text-decoration: none; }
span.LATEX span.A{ position:relative; top:-0.5ex; left:-0.4em; font-size:85%;}
span.LATEX span.TEX{ position:relative; left: -0.4em; }
div.float img, div.float .caption {text-align:center;}
div.figure img, div.figure .caption {text-align:center;}
.marginpar {width:20%; float:right; text-align:left; margin-left:auto; margin-top:0.5em; font-size:85%; text-decoration:underline;}
.marginpar p{margin-top:0.4em; margin-bottom:0.4em;}
.equation td{text-align:center; vertical-align:middle; }
td.eq-no{ width:5%; }
table.equation { width:100%; } 
div.math-display, div.par-math-display{text-align:center;}
math .texttt { font-family: monospace; }
math .textit { font-style: italic; }
math .textsl { font-style: oblique; }
math .textsf { font-family: sans-serif; }
math .textbf { font-weight: bold; }
.partToc a, .partToc, .likepartToc a, .likepartToc {line-height: 200%; font-weight:bold; font-size:110%;}
.chapterToc a, .chapterToc, .likechapterToc a, .likechapterToc, .appendixToc a, .appendixToc {line-height: 200%; font-weight:bold;}
.index-item, .index-subitem, .index-subsubitem {display:block}
.caption td.id{font-weight: bold; white-space: nowrap; }
table.caption {text-align:center;}
h1.partHead{text-align: center}
p.bibitem { text-indent: -2em; margin-left: 2em; margin-top:0.6em; margin-bottom:0.6em; }
p.bibitem-p { text-indent: 0em; margin-left: 2em; margin-top:0.6em; margin-bottom:0.6em; }
.paragraphHead, .likeparagraphHead { margin-top:2em; font-weight: bold;}
.subparagraphHead, .likesubparagraphHead { font-weight: bold;}
.quote {margin-bottom:0.25em; margin-top:0.25em; margin-left:1em; margin-right:1em; text-align:\\jmathmathustify;}
.verse{white-space:nowrap; margin-left:2em}
div.maketitle {text-align:center;}
h2.titleHead{text-align:center;}
div.maketitle{ margin-bottom: 2em; }
div.author, div.date {text-align:center;}
div.thanks{text-align:left; margin-left:10%; font-size:85%; font-style:italic; }
div.author{white-space: nowrap;}
.quotation {margin-bottom:0.25em; margin-top:0.25em; margin-left:1em; }
h1.partHead{text-align: center}
.sectionToc, .likesectionToc {margin-left:2em;}
.subsectionToc, .likesubsectionToc {margin-left:4em;}
.subsubsectionToc, .likesubsubsectionToc {margin-left:6em;}
.frenchb-nbsp{font-size:75%;}
.frenchb-thinspace{font-size:75%;}
.figure img.graphics {margin-left:10%;}
/* end css.sty */

\title{Monotonie, continuite}
\author{}
\date{}

\begin{document}
\maketitle

\textbf{Warning: 
requires JavaScript to process the mathematics on this page.\\ If your
browser supports JavaScript, be sure it is enabled.}

\begin{center}\rule{3in}{0.4pt}\end{center}

{[}
{[}{]}
{[}

\subsubsection{8.1 Monotonie, continuité}

\paragraph{8.1.1 Limites et monotonie}

Proposition~8.1.1 Soit I un intervalle de \mathbb{R}~ et f : I \rightarrow~ \mathbb{R}~ croissante.
Alors

\begin{itemize}
\itemsep1pt\parskip0pt\parsep0pt
\item
  (i) f admet en tout point a de I (dans la mesure où cela a un sens)
  une limite à gauche f(a-) et une limite à droite f(a+) dans \mathbb{R}~ avec
  f(a-) \leq f(a) \leq f(a+)
\item
  (ii) f admet en l'extrémité droite b de I une limite si et seulement
  si~elle est ma\\jmathmathorée sur I~; dans le cas contraire
  lim_x\rightarrow~b,x\textless{}b~f(x) = +\infty~
\item
  (iii) f admet en l'extrémité gauche a de I une limite si et seulement
  si~elle est minorée sur I~; dans le cas contraire
  lim_x\rightarrow~a,x\textgreater{}a~f(x) = -\infty~
\end{itemize}

Démonstration (i) Supposons que a n'est pas l'extrémité gauche de I.
Pour x \textless{} a on a f(x) \leq f(a). Soit m
= sup_x\textless{}a~f(x) \leq f(a). Soit
\epsilon \textgreater{} 0. Il existe x_0 \textless{} a tel que m - \epsilon
\textless{} f(x_0) \leq m. Alors x \in{]}x_0,a{[}\rigtharrow~ m - \epsilon
\textless{} f(x_0) \leq f(x) \leq m et donc m
= lim_x\rightarrow~a,x\textless{}a~f(x). De même,
si a n'est pas l'extrémité gauche de I et si M
= inf _x\textgreater{}a~f(x) ≥ f(a),
on a M =\
lim_x\rightarrow~a,x\textgreater{}af(x).

(ii) f admet de toute fa\ccon dans
\overline\mathbb{R}~ la limite
sup_x\inI~f(x) (comme ci dessus)~; cette
limite est dans \mathbb{R}~ si et seulement si~f est ma\\jmathmathorée sur I~; similaire
pour (iii).

Remarque~8.1.1 On a un résultat similaire pour les applications
décroissantes~:

Proposition~8.1.2 Soit I un intervalle de \mathbb{R}~ et f : I \rightarrow~ \mathbb{R}~ décroissante.
Alors (i) f admet en tout point a de I (dans la mesure où cela a un
sens) une limite à gauche f(a-) et une limite à droite f(a+) dans \mathbb{R}~ avec
f(a-) ≥ f(a) ≥ f(a+) (ii) f admet en l'extrémité droite b de I une
limite si et seulement si~elle est minorée sur I~; dans le cas contraire
lim_x\rightarrow~b,x\textless{}b~f(x) = -\infty~ (iii)
f admet en l'extrémité gauche a de I une limite si et seulement si~elle
est ma\\jmathmathorée sur I~; dans le cas contraire
lim_x\rightarrow~a,x\textgreater{}a~f(x) = +\infty~

\paragraph{8.1.2 Continuité et monotonie}

Lemme~8.1.3 Soit I un intervalle de \mathbb{R}~ et f : I \rightarrow~ \mathbb{R}~ monotone. Alors f est
continue si et seulement si~f(I) est un intervalle.

Démonstration La condition est évidemment nécessaire d'après le théorème
des valeurs intermédiaires. Inversement supposons que f(I) est un
intervalle et a \in I. On peut par exemple supposer que f est croissante.
Supposons que f(a) \textless{} f(a+) (ce qui sous entend que a n'est pas
l'extrémité droite de I). Soit x \textgreater{} a. On a alors f(a)
\textless{} f(a+) \leq f(x) (puisque f(a+) =\
inf _t\textgreater{}af(t)). En particulier {]}f(a),f(a+){[}\subset~
{[}f(a),f(x){]} \subset~ f(I) (convexité des intervalles). Soit alors y
\in{]}f(a),f(a+){[}~; on peut poser y = f(t) pour t \in I. Mais si t \leq a, on
a y = f(t) \leq f(a) et si t \textgreater{} a on a y = f(t) ≥ f(a+). C'est
absurde. Donc f(a) = f(a+). On montre de même que si a n'est pas
l'extrémité gauche de I, f(a) = f(a-). Donc f est continue sur I.

Théorème~8.1.4 Soit I un intervalle de \mathbb{R}~ et f : I \rightarrow~ \mathbb{R}~ continue
strictement monotone. Alors J = f(I) est un intervalle de \mathbb{R}~ et f induit
un homéomorphisme de I sur J.

Démonstration On sait dé\\jmathmathà que J est un intervalle~; alors
f^-1 : J \rightarrow~ I est encore strictement monotone et
f^-1(J) = I est un intervalle, donc f^-1 est
continue. Donc f induit un homéomorphisme de I sur J.

Le théorème suivant montre que réciproquement, la condition de stricte
monotonie est une condition nécessaire pour un homéomorphisme d'un
intervalle sur un autre.

Théorème~8.1.5 Soit I un intervalle de \mathbb{R}~ et f : I \rightarrow~ \mathbb{R}~ continue. Alors f
est in\\jmathmathective si et seulement si~elle est strictement monotone.

Démonstration La condition est évidemment suffisante. Inversement,
supposons f continue et in\\jmathmathective. Soit X = \(x,y) \in I
\times I∣x \textless{} y\ et g :
X \rightarrow~ \mathbb{R}~ définie par g(x,y) = f(y) - f(x). Alors X est connexe (car
convexe) et g est continue. L'ensemble g(X) est donc un intervalle de \mathbb{R}~
et cet intervalle ne contient pas 0 car g est in\\jmathmathective. Donc soit g(X)
\subset~{]}0,+\infty~{[} (auquel cas f est strictement croissante), soit g(X) \subset~{]}
-\infty~,0{[} (auquel cas f est strictement décroissante).

{[}
{[}

\end{document}
