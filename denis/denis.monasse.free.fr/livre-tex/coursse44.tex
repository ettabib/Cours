\subsection{8.1 Monotonie, continuité}

\subsubsection{Limites et monotonie}
\label{sec:limites-et-monotonie}



\begin{prop}
Soit I un intervalle de $\mathbb{R}$~ et$ f : I \rightarrow~ \mathbb{R}~ croissante$.
Alors

\begin{itemize}

\item
f admet en tout point a de I (dans la mesure où cela a un sens)
  une limite à gauche $f(a-)$ et une limite à droite $ f(a+)$ dans $\mathbb{R}$ avec
  $f(a-) \leq f(a) \leq f(a+)$
\item
f admet en l'extrémité droite b de I une limite si et seulement
  si~elle est majorée sur I~; dans le cas contraire
 $ lim_x\rightarrow~b,x<b~f(x) = +\infty~$
\item
f admet en l'extrémité gauche a de I une limite si et seulement
  si~elle est minorée sur I~; dans le cas contraire
$lim_x \rightarrow ~a, x >a~f(x) = -\infty~$
\end{itemize}
  
\end{prop}

Démonstration (i) Supposons que a n'est pas l'extrémité gauche de I.
Pour x < a on a f(x) \leq f(a). Soit m
= sup_x<a~f(x) \leq f(a). Soit
\epsilon > 0. Il existe x_0 < a tel que m - \epsilon
< f(x_0) \leq m. Alors x \in]x_0,a[\rigtharrow~ m - \epsilon
< f(x_0) \leq f(x) \leq m et donc m
= lim_x\rightarrow~a,x<a~f(x). De même,
si a n'est pas l'extrémité gauche de I et si M
= inf _x>a~f(x) ≥ f(a),
on a M =\
lim_x\rightarrow~a,x>af(x).

(ii) f admet de toute fa\ccon dans
\overline\mathbb{R}~ la limite
sup_x\inI~f(x) (comme ci dessus)~; cette
limite est dans \mathbb{R}~ si et seulement si~f est majorée sur I~; similaire
pour (iii).

Remarque~8.1.1 On a un résultat similaire pour les applications
décroissantes~:

Proposition~8.1.2 Soit I un intervalle de \mathbb{R}~ et f : I \rightarrow~ \mathbb{R}~ décroissante.
Alors (i) f admet en tout point a de I (dans la mesure où cela a un
sens) une limite à gauche f(a-) et une limite à droite f(a+) dans \mathbb{R}~ avec
f(a-) ≥ f(a) ≥ f(a+) (ii) f admet en l'extrémité droite b de I une
limite si et seulement si~elle est minorée sur I~; dans le cas contraire
lim_x\rightarrow~b,x<b~f(x) = -\infty~ (iii)
f admet en l'extrémité gauche a de I une limite si et seulement si~elle
est majorée sur I~; dans le cas contraire
lim_x\rightarrow~a,x>a~f(x) = +\infty~

\subsubsection{Continuité et monotonie}
\label{sec:cont-et-monot}



\begin{lem}
  Soit I un intervalle de $\mathbb{R}$~ et $f : I \rightarrow~ \mathbb{R}$~ monotone. Alors f est
continue si et seulement si~f(I) est un intervalle.

\end{lem}
Démonstration La condition est évidemment nécessaire d'après le théorème
des valeurs intermédiaires. Inversement supposons que f(I) est un
intervalle et a \in I. On peut par exemple supposer que f est croissante.
Supposons que f(a) < f(a+) (ce qui sous entend que a n'est pas
l'extrémité droite de I). Soit x > a. On a alors f(a)
< f(a+) \leq f(x) (puisque f(a+) =\
inf _t>af(t)). En particulier ]f(a),f(a+)[\subset~
[f(a),f(x)] \subset~ f(I) (convexité des intervalles). Soit alors y
\in]f(a),f(a+)[~; on peut poser y = f(t) pour t \in I. Mais si t \leq a, on
a y = f(t) \leq f(a) et si t > a on a y = f(t) ≥ f(a+). C'est
absurde. Donc f(a) = f(a+). On montre de même que si a n'est pas
l'extrémité gauche de I, f(a) = f(a-). Donc f est continue sur I.

Théorème~8.1.4 Soit I un intervalle de \mathbb{R}~ et f : I \rightarrow~ \mathbb{R}~ continue
strictement monotone. Alors J = f(I) est un intervalle de \mathbb{R}~ et f induit
un homéomorphisme de I sur J.

Démonstration On sait déjà que J est un intervalle~; alors
f^-1 : J \rightarrow~ I est encore strictement monotone et
f^-1(J) = I est un intervalle, donc f^-1 est
continue. Donc f induit un homéomorphisme de I sur J.

Le théorème suivant montre que réciproquement, la condition de stricte
monotonie est une condition nécessaire pour un homéomorphisme d'un
intervalle sur un autre.

Théorème~8.1.5 Soit I un intervalle de \mathbb{R}~ et f : I \rightarrow~ \mathbb{R}~ continue. Alors f
est injective si et seulement si~elle est strictement monotone.

Démonstration La condition est évidemment suffisante. Inversement,
supposons f continue et injective. Soit X = \(x,y) \in I
\times I∣x < y\ et g :
X \rightarrow~ \mathbb{R}~ définie par g(x,y) = f(y) - f(x). Alors X est connexe (car
convexe) et g est continue. L'ensemble g(X) est donc un intervalle de \mathbb{R}~
et cet intervalle ne contient pas 0 car g est injective. Donc soit g(X)
\subset~]0,+\infty~[ (auquel cas f est strictement croissante), soit g(X) \subset~]
-\infty~,0[ (auquel cas f est strictement décroissante).
