Voici le texte LaTeX corrigé avec les environnements demandés et l'indexation des mots-clés :

\section{Intégrales dépendant d'un paramètre}

\subsection{Position du problème}

Soit $E$ un espace métrique, $a,b \in \mathbb{R}$, $E'$ un espace vectoriel normé complet et $f : E \times [a,b] \to E'$, $(x,t) \mapsto f(x,t)$. On suppose que $\forall x \in E$, l'application $t \mapsto f(x,t)$ est réglée de $[a,b]$ dans $E'$. On peut donc définir une application $F : E \to E'$ par $F(x) = \int_a^b f(x,t) dt$. Nous allons nous intéresser ici aux propriétés de la fonction $F$ (continuité, dérivabilité, intégration) en fonction de celles de $f$.

\subsection{Continuité}

\begin{thm}[Continuité par convergence dominée]
\index{continuité!par convergence dominée}
Soit $E$ un espace métrique, $I$ un intervalle de $\mathbb{R}$, $f : E \times I \to \mathbb{C}$, $(x,t) \mapsto f(x,t)$. On suppose
\begin{enumerate}
  \item pour chaque $x \in E$, l'application $t \mapsto f(x,t)$ est continue par morceaux sur $I$
  \item pour chaque $t \in I$, l'application $x \mapsto f(x,t)$ est continue sur $E$
  \item il existe une fonction $\phi : I \to \mathbb{R}^+$, intégrable, telle que $\forall (x,t) \in E \times I, |f(x,t)| \leq \phi(t)$ (hypothèse de domination).
\end{enumerate}
Alors, pour tout $x \in E$, la fonction $t \mapsto f(x,t)$ est intégrable sur $I$ et l'application $F : E \to \mathbb{C}$, $x \mapsto \int_I f(x,t) dt$ est continue sur $E$.
\end{thm}

\begin{proof}
L'intégrabilité de $t \mapsto f(x,t)$ est claire avec la majoration $|f(x,t)| \leq \phi(t)$. Soit alors $x \in E$ et $(x_n)$ une suite de $E$ de limite $x$. Posons $g_n(t) = f(x_n,t)$ et $g(t) = f(x,t)$. La suite $(g_n)$ est une suite de fonctions continues par morceaux sur $I$ qui converge vers $g$ continue par morceaux et on a $|g_n| \leq \phi$ avec $\phi$ intégrable ; le théorème de convergence dominée assure que $\lim \int_I g_n = \int_I g$ soit encore $\lim_{n \to +\infty} F(x_n) = F(x)$. Donc $F$ est bien continue.
\end{proof}

\begin{rem}
Pour montrer que $F$ est continue, il suffit de montrer que sa restriction à tout compact $K$ contenu dans $E$ est continue, donc qu'à tout compact $K$ contenu dans $E$, on peut associer une fonction $\phi_K$ intégrable telle que $\forall (x,t) \in K \times I, |f(x,t)| \leq \phi_K(t)$.
\end{rem}

\begin{cor}
\index{intégrale!continuité}
Soit $U$ un ouvert de $\mathbb{R}^n$, $a,b \in \mathbb{R}$ et $f : U \times [a,b] \to \mathbb{C}$ continue, $(x,t) \mapsto f(x,t)$. Alors l'application $F : U \to \mathbb{C}$ définie par $F(x) = \int_a^b f(x,t) dt$ est continue.
\end{cor}

\begin{proof}
Soit $x_0 \in U$ et $r > 0$ tel que la boule fermée $B'(x_0,r)$ soit contenue dans $U$. La fonction $f$ est continue sur le compact $B'(x_0,r) \times [a,b]$, donc elle y est bornée : soit $M \geq 0$ tel que $\forall (x,t) \in B'(x_0,r) \times [a,b], |f(x,t)| \leq M$. Comme la fonction constante $t \mapsto M$ est intégrable sur $[a,b]$ et bien entendue indépendante de $x$, le théorème de continuité par convergence dominée montre que $F$ est continue sur $B'(x_0,r)$ et en particulier qu'elle est continue au point $x_0$.
\end{proof}

\begin{example}
\index{fonction Gamma}
Soit $0 < a < b < +\infty$ et soit $\Gamma_{a,b}(x) = \int_a^b t^{x-1} e^{-t} dt$. On a ici, $f(x,t) = t^{x-1} e^{-t} = \exp(-t + (x-1)\log t)$ qui est une fonction continue de $\mathbb{R} \times [a,b]$ dans $\mathbb{R}$ (composée de fonctions continues). On en déduit que $\Gamma_{a,b}$ est continue sur $\mathbb{R}$.
\end{example}

\subsection{Dérivabilité}

Nous supposerons ici que $E = I$ intervalle de $\mathbb{R}$. Soit $t_0 \in [a,b]$ ; lorsque l'application $x \mapsto f(x,t_0)$ est dérivable en un point $x_0 \in I$, sa dérivée au point $x_0$ sera notée $\frac{\partial f}{\partial x}(x_0,t_0)$.

\begin{thm}[Dérivabilité par convergence dominée]
\index{dérivabilité!par convergence dominée}
Soit $J$ un intervalle de $\mathbb{R}$, $I$ un intervalle de $\mathbb{R}$, $f : J \times I \to \mathbb{C}$, $(x,t) \mapsto f(x,t)$ continue, admettant une dérivée partielle par rapport à $x$, $(x,t) \mapsto \frac{\partial f}{\partial x}(x,t)$, continue sur $J \times I$. On suppose que pour tout $x \in E$, la fonction $t \mapsto f(x,t)$ est intégrable sur $I$ et qu'il existe une fonction $\phi : I \to \mathbb{R}^+$, intégrable, telle que $\forall (x,t) \in J \times I, |\frac{\partial f}{\partial x}(x,t)| \leq \phi(t)$ (hypothèse de domination). Alors, l'application $F : J \to \mathbb{C}$, $x \mapsto \int_I f(x,t) dt$ est de classe $\mathcal{C}^1$ sur $J$ et

$\forall x \in J, F'(x) = \int_I \frac{\partial f}{\partial x}(x,t) dt$
\end{thm}

\begin{proof}
L'intégrabilité de $t \mapsto \frac{\partial f}{\partial x}(x,t)$ est claire avec la majoration $|\frac{\partial f}{\partial x}(x,t)| \leq \phi(t)$. Soit alors $x \in J$ et $x_n$ une suite de $J \setminus \{x\}$ de limite $x$. Posons $g_n(t) = \frac{f(x,t)-f(x_n,t)}{x-x_n}$ et $g(t) = \frac{\partial f}{\partial x}(x,t)$. La suite $(g_n)$ est une suite de fonctions continues sur $I$ qui converge vers $g$ continue. L'inégalité des accroissements finis assure que $|f(x,t) - f(x_n,t)| \leq |x-x_n| \sup_{y \in ]x,x_n[} |\frac{\partial f}{\partial x}(y,t)| \leq |x-x_n| \phi(t)$, d'où

$|g_n(t)| = |\frac{f(x,t) - f(x_n,t)}{x-x_n}| \leq \phi(t)$

avec $\phi$ intégrable. Le théorème de convergence dominée assure alors que $\lim \int_I g_n = \int_I g$, soit encore que $\lim_{n \to +\infty} \frac{F(x_n)-F(x)}{x_n-x} = \int_I \frac{\partial f}{\partial x}(x,t) dt$. Comme la suite $(x_n)$ est quelconque, on a

$\lim_{t \to x} \frac{F(t) - F(x)}{t-x} = \int_I \frac{\partial f}{\partial x}(x,t) dt$

donc $F$ est dérivable et $F'(x) = \int_I \frac{\partial f}{\partial x}(x,t) dt$. La continuité de $F'$ relève du théorème précédent relatif à la continuité d'une intégrale dépendant d'un paramètre, la fonction étant dominée indépendamment du paramètre.
\end{proof}

\begin{cor}
\index{intégrale!dérivabilité}
Soit $J$ un intervalle de $\mathbb{R}$, $I$ un intervalle de $\mathbb{R}$, $f : J \times I \to \mathbb{C}$, $(x,t) \mapsto f(x,t)$ continue, admettant des dérivées partielles par rapport à $x$, $(x,t) \mapsto \frac{\partial^i f}{\partial x^i}(x,t)$, continues sur $J \times I$, $i = 1,\ldots,k$. On suppose que pour tout $x \in E$, la fonction $t \mapsto f(x,t)$ est intégrable sur $I$ et qu'il existe des fonctions $\phi_1,\ldots,\phi_k : I \to \mathbb{R}^+$, intégrables, telles que $\forall (x,t) \in J \times I, |\frac{\partial^i f}{\partial x^i}(x,t)| \leq \phi_i(t)$ (hypothèses de domination). Alors, l'application $F : J \to \mathbb{C}$, $x \mapsto \int_I f(x,t) dt$ est de classe $C^k$ sur $J$ et

$\forall i \in [1,k], \forall x \in J, F^{(i)}(x) = \int_I \frac{\partial^i f}{\partial x^i}(x,t) dt$
\end{cor}

\begin{proof}
Récurrence évidente à partir du théorème précédent.
\end{proof}

\begin{thm}
\index{intégrale!dérivabilité}
Soit $I$ un intervalle de $\mathbb{R}$, $a,b \in \mathbb{R}$ et $f : I \times [a,b] \to \mathbb{C}$, $(x,t) \mapsto f(x,t)$. On suppose que
\begin{enumerate}
  \item Pour chaque $x \in I$, l'application $t \mapsto f(x,t)$ est continue par morceaux sur $[a,b]$
  \item Pour chaque $(x,t) \in I \times [a,b]$, $f$ admet une dérivée partielle par rapport à $x$, $\frac{\partial f}{\partial x}(x,t)$ et que l'application $(x,t) \mapsto \frac{\partial f}{\partial x}(x,t)$ est continue.
\end{enumerate}
Alors $F : I \to \mathbb{C}$, $x \mapsto \int_a^b f(x,t) dt$ est de classe $\mathcal{C}^1$ et $\forall x_0 \in I, F'(x_0) = \int_a^b \frac{\partial f}{\partial x}(x_0,t) dt$.
\end{thm}

\begin{proof}
Il suffit, comme dans le théorème correspondant de continuité pour une intégrale dépendant d'un paramètre sur un segment, de prendre $x_0 \in I$, un segment $K$, voisinage de $x_0$ dans $I$, et d'utiliser le fait que la fonction $(x,t) \mapsto \frac{\partial f}{\partial x}(x,t)$ est continue, donc bornée par un certain $M$ sur le compact $K \times [a,b]$. La fonction constante $M$ ainsi introduite est intégrable sur le segment $[a,b]$ et fournit ainsi une fonction dominante de la fonction $(x,t) \mapsto \frac{\partial f}{\partial x}(x,t)$ sur $K \times [a,b]$. Par le théorème de dérivation par convergence dominée, $F$ est dérivable sur $K$ et en particulier elle est dérivable au point $x_0$ avec $F'(x_0) = \int_a^b \frac{\partial f}{\partial x}(x_0,t) dt$.
\end{proof}

\begin{example}
\index{fonction Gamma}
Soit $0 < a < b < +\infty$ et soit $\Gamma_{a,b}(x) = \int_a^b t^{x-1} e^{-t} dt$. On a ici, $f(x,t) = t^{x-1} e^{-t} = \exp(-t + (x-1)\log t)$ qui admet une dérivée par rapport à $x$, $\frac{\partial f}{\partial x}(x,t) = \log t \cdot t^{x-1} e^{-t}$ qui est une fonction continue de $\mathbb{R} \times [a,b]$ dans $\mathbb{R}$ (composée de fonctions continues). On en déduit que $\Gamma_{a,b}$ est de classe $\mathcal{C}^1$ sur $\mathbb{R}$ et que $\Gamma_{a,b}'(x) = \int_a^b t^{x-1} \log t \cdot e^{-t} dt$. Une récurrence évidente montrera alors que $\Gamma_{a,b}$ est de classe $C^\infty$ et que $\Gamma_{a,b}^{(n)}(x) = \int_a^b t^{x-1} (\log t)^n e^{-t} dt$.
\end{example}

\subsection{Théorème de Fubini sur un produit de segments}

\begin{thm}[Fubini sur un produit de segments]
  \index{théorème de Fubini}
  Soit $f : [a,b] \times [c,d] \to \mathbb{C}$ continue. Alors
  
  $\int_a^b \left(\int_c^d f(x,y) dy\right) dx = \int_c^d \left(\int_a^b f(x,y) dx\right) dy$
  \end{thm}
  
  \begin{proof}
  On va démontrer que $\forall t \in [a,b], F(t) = G(t)$ où $F(t) = \int_a^t \left(\int_c^d f(x,y) dy\right) dx$ et $G(t) = \int_c^d \left(\int_a^t f(x,y) dx\right) dy$. Pour $t = b$, on aura alors le résultat voulu. Comme $F(a) = G(a) = 0$, il suffit de démontrer que $F$ et $G$ sont dérivables et que $F' = G'$. Mais on a $F(t) = \int_a^t \phi(x) dx$ avec $\phi(x) = \int_c^d f(x,y) dy$. Le théorème de continuité des intégrales dépendant d'un paramètre sur un segment (conséquence du théorème de continuité par convergence dominée) assure que $\phi$ est continue, donc que $F$ est dérivable et que $F'(t) = \phi(t) = \int_c^d f(t,y) dy$.
  
  On a aussi $G(t) = \int_c^d \psi(t,y) dy$ avec $\psi(t,y) = \int_a^t f(x,y) dx$. Comme $x \mapsto f(x,y)$ est continue, $t \mapsto \psi(t,y)$ est de classe $C^1$ et $\frac{\partial \psi}{\partial t}(t,y) = f(t,y)$. Mais $f$ étant continue sur le compact $[a,b] \times [c,d]$, elle y est bornée et on a donc
  
  $\forall (t,y) \in [a,b] \times [c,d], |\frac{\partial \psi}{\partial t}(t,y)| \leq \|f\|_\infty$
  
  qui est une fonction (constante) de la variable $y$, intégrable sur $[c,d]$ et indépendante de $t$. Le théorème de dérivation par convergence dominée assure que l'application $G : t \mapsto \int_c^d \psi(t,y) dy$ est dérivable et que $G'(t) = \int_c^d \frac{\partial \psi}{\partial t}(t,y) dy$, soit encore $G'(t) = \int_c^d f(t,y) dy = \phi(t) = F'(t)$, ce qui achève la démonstration.
  \end{proof}
  
  \subsection{Intégrales sur un pavé ou un rectangle}
  
  \begin{de}
  On appelle pavé de $\mathbb{R}^2$ (resp. rectangle de $\mathbb{R}^2$) toute partie de $\mathbb{R}^2$ de la forme $[a,b] \times [c,d]$ (resp $I \times I'$ où $I$ et $I'$ sont des intervalles de $\mathbb{R}$).
  \index{pavé}
  \index{rectangle}
  \end{de}
  
  En appliquant le théorème de Fubini pour une fonction continue sur un produit de segments, on est amené à donner la définition suivante :
  
  \begin{de}
  Soit $P = [a,b] \times [c,d]$ un pavé de $\mathbb{R}^2$ et $f : P \to \mathbb{C}$ une fonction continue. On appelle intégrale de la fonction $f$ sur le pavé $P$ le nombre complexe noté indifféremment $\iint_P f$ ou $\iint_P f(x,y) dx dy$ défini par
  
  $\iint_P f = \iint_P f(x,y) dx dy = \int_a^b \left(\int_c^d f(x,y) dy\right) dx = \int_c^d \left(\int_a^b f(x,y) dx\right) dy$
  \index{intégrale!sur un pavé}
  \end{de}
  
  On peut alors répéter les définitions et résultats qui nous ont permis de définir les fonctions intégrables sur un intervalle à partir de l'intégrale sur un segment, pour définir des fonctions intégrables sur un rectangle à partir de la notion de intégrale sur un pavé. On donnera donc les définitions et propriétés suivantes sans commentaire ou démonstration.
  
  \begin{itemize}
  \item Soit $R$ un rectangle et $f : R \to \mathbb{R}$ continue positive. On dit que $f$ est intégrable sur $R$ s'il existe $M \geq 0$ tel que pour tout pavé $P \subset R$ on ait $\iint_P f \leq M$. On pose alors $\iint_R f = \sup_{P \subset R} f$ ; on montre que si $(P_n)_{n \in \mathbb{N}}$ est une suite croissante de pavés contenus dans $R$ dont la réunion est $R$, alors $\iint_R f = \lim_{n \to +\infty} \iint_{P_n} f$.
  \index{fonction intégrable}
  
  \item Soit $R$ un rectangle et $f : R \to \mathbb{C}$ continue. On dit que $f$ est intégrable sur $R$ si la fonction continue positive $|f|$ est intégrable sur $R$ ; on montre alors que si $(P_n)_{n \in \mathbb{N}}$ est une suite croissante de pavés contenus dans $R$ dont la réunion est $R$, la suite $\left(\iint_{P_n} f\right)_{n \in \mathbb{N}}$ converge et que sa limite est indépendante du choix de la suite $(P_n)_{n \in \mathbb{N}}$ ; on pose donc $\iint_R f = \lim_{n \to +\infty} \iint_{P_n} f$.
  
  \item L'ensemble des fonctions continues de $R$ dans $\mathbb{C}$ intégrables sur $R$ est un sous-espace vectoriel de l'espace des fonctions continues et l'application $f \mapsto \iint_R f$ est linéaire.
  
  \item Si un rectangle $R$ est la réunion de deux rectangles $R_1$ et $R_2$ ne se rencontrant que suivant un de leurs côtés, alors $f$ est intégrable sur $R$ si et seulement si elle est intégrable sur $R_1$ et $R_2$ et dans ce cas, $\iint_R f = \iint_{R_1} f + \iint_{R_2} f$.
  \end{itemize}
  
  On utilisera plusieurs fois le lemme suivant
  
  \begin{lem}
  Soit $R$ et $R'$ deux rectangles tels que $R \subset R'$ et soit $f : R' \to \mathbb{C}$ continue et intégrable sur $R'$. Alors $f$ est intégrable sur $R$ et
  
  $|\iint_{R'} f - \iint_R f| \leq \iint_{R'} |f| - \iint_R |f|$
  \end{lem}
  
  \begin{proof}
  Tout pavé $P$ inclus dans $R$ est inclus dans $R'$ et donc $\sup_{P \subset R} \iint_P |f| \leq \sup_{P \subset R'} \iint_P |f| = \iint_{R'} |f| < +\infty$ ce qui garantit l'intégrabilité de $f$ sur $R$. De plus (avec quelques conventions d'écritures évidentes pour l'intégrale sur la différence de deux rectangles)
  
  $|\iint_{R'} f - \iint_R f| = |\iint_{R' \setminus R} f| \leq \iint_{R' \setminus R} |f| \leq \iint_{R'} |f| - \iint_R |f|$
  \end{proof}
  
  \begin{lem}
  Soit $R = I \times I'$ un rectangle, $f : R \to \mathbb{C}$ une fonction continue, intégrable sur $R$. Soit $(J_n)_{n \in \mathbb{N}}$ et $(K_n)_{n \in \mathbb{N}}$ des suites croissantes de segments dont les réunions sont respectivement $I$ et $I'$. Alors
  
  $\iint_R f = \lim_{n \to +\infty} \iint_{I \times K_n} f = \lim_{n \to +\infty} \iint_{J_n \times I'} f$
  \end{lem}
  
  \begin{proof}
  Premier cas : $f$ est à valeurs réelles positives. On a alors
  
  $\iint_{J_n \times K_n} f \leq \iint_{I \times K_n} f \leq \iint_{I \times I'} f$
  
  et comme $\iint_{I \times I'} f = \lim_{n \to +\infty} \iint_{J_n \times K_n} f$ (les $J_n \times K_n$ forment une suite croissante de pavés dont la réunion est $I \times I'$), on a $\iint_{I \times I'} f = \lim_{n \to +\infty} \iint_{I \times K_n} f$. On démontre l'autre formule de manière similaire.
  
  Deuxième cas : $f$ est à valeurs complexes. On remarque que, d'après le lemme précédent
  
  $|\iint_{I \times I'} f - \iint_{I \times K_n} f| \leq \iint_{I \times I'} |f| - \iint_{I \times K_n} |f|$
  
  qui tend vers $0$ d'après le premier cas. Donc $\iint_{I \times I'} f = \lim_{n \to +\infty} \iint_{I \times K_n} f$. On démontre l'autre formule de manière similaire.
  \end{proof}
  
  \subsection{Théorème de Fubini sur un produit d'intervalles}
  
  \begin{lem}
  Soit $I'$ un intervalle de $\mathbb{R}$, $f : [a,b] \times I' \to \mathbb{C}$ continue. On fait les hypothèses suivantes :
  \begin{itemize}
  \item pour tout $x \in [a,b]$, $y \mapsto f(x,y)$ est intégrable sur $I'$
  \item l'application $g : x \mapsto \int_{I'} f(x,y) dy$ est continue par morceaux sur $[a,b]$
  \item $f$ est intégrable sur le rectangle $[a,b] \times I'$
  \end{itemize}
  Alors $\iint_{[a,b] \times I'} f = \int_a^b g = \int_a^b \left(\int_{I'} f(x,y) dy\right) dx$.
  \end{lem}
  
  \begin{proof}
  Soit $K_n$ une suite croissante de segments dont la réunion est $I'$. On sait que
  
  $\iint_{[a,b] \times I'} f = \lim_{n \to +\infty} \iint_{[a,b] \times K_n} f = \lim_{n \to +\infty} \int_a^b \left(\int_{K_n} f(x,y) dy\right) dx = \lim_{n \to +\infty} \int_a^b g_n(x) dx$
  
  en posant $g_n(x) = \int_{K_n} f(x,y) dy$ ; le théorème de continuité des intégrales dépendant d'un paramètre sur un segment nous garantit que $g_n$ est continue ; de plus, comme la fonction $y \mapsto f(x,y)$ est intégrable sur $I'$, la suite $(g_n)_{n \in \mathbb{N}}$ converge simplement vers $g$ et on peut écrire
  
  \begin{align*}
  g_{n+1}(x) - g_n(x) &= \int_{K_{n+1}} f(x,y) dy - \int_{K_n} f(x,y) dy \\
  &= \int_{K_{n+1} \setminus K_n} f(x,y) dy \leq \int_{K_{n+1} \setminus K_n} |f(x,y)| dy \\
  &= \int_{K_{n+1}} |f(x,y)| dy - \int_{K_n} |f(x,y)| dy
  \end{align*}
  
  Posons $u_n = \int_a^b \left(\int_{K_n} f(x,y) dy\right) dx$. En intégrant l'inégalité ci-dessus de $a$ à $b$, on obtient
  
  $\int_a^b g_{n+1}(x) - g_n(x) dx \leq \int_a^b \left(\int_{K_{n+1}} |f(x,y)| dy\right) - \int_a^b \left(\int_{K_n} |f(x,y)| dy\right) = u_{n+1} - u_n$
  
  Mais $\sum_{n=0}^N (u_{n+1} - u_n) = u_{N+1} - u_0$ qui admet la limite $\iint_{[a,b] \times I'} f$. Donc la série $\sum (u_{n+1} - u_n)$ converge, et par conséquent, il en est de même de la série $\sum \int_a^b g_{n+1}(x) - g_n(x) dx$. Le théorème d'intégration termes à termes pour les séries de fonctions assure que
  
  $\sum_{n=0}^{+\infty} \int_a^b (g_{n+1} - g_n) = \int_a^b \sum_{n=0}^{+\infty} (g_{n+1} - g_n) = \int_a^b (g - g_0)$
  
  Mais
  
  $\sum_{n=0}^{N-1} \int_a^b (g_{n+1} - g_n) = \int_a^b g_N - \int_a^b g_0$
  
  Autrement dit on a $\lim_{N \to +\infty} \left(\int_a^b g_N - \int_a^b g_0\right) = \int_a^b (g - g_0)$, soit encore $\lim_{N \to +\infty} \int_a^b g_N = \int_a^b g$, c'est à dire $\iint_{[a,b] \times I'} f = \int_a^b g$, ce que nous cherchions à démontrer.
  \end{proof}
  
  Nous sommes maintenant en mesure de démontrer un premier théorème reliant l'intégrale sur un rectangle et les intégrales partielles sur les intervalles projections de ce rectangle sur les deux axes.
  
  \\begin{thm}
    \index{théorème de Fubini!sur un rectangle}
    Soit $R = I \times I'$ un rectangle, $f : R \to \mathbb{C}$ continue. On suppose que
    \begin{itemize}
    \item $f$ est intégrable sur $R$
    \item pour tout $x \in I$, $y \mapsto f(x,y)$ est intégrable sur $I'$
    \item les applications $x \mapsto \int_{I'} f(x,y) dy$ et $g : x \mapsto \int_{I'} f(x,y) dy$ sont continues par morceaux sur $I$
    \end{itemize}
    Alors $g$ est intégrable sur $I$ et $\iint_{I \times I'} f = \int_I g = \int_I \left(\int_{I'} f(x,y) dy\right) dx$.
    \end{thm}
    
    \begin{proof}
    Soit $J$ un segment inclus dans $I$. D'après le lemme précédent dont les hypothèses sont évidemment vérifiées,
    
    $\int_J g = \int_J \left(\int_{I'} f(x,y) dy\right) dx = \iint_{J \times I'} f \leq \iint_{I \times I'} f$
    
    ce qui garantit que $g$ est intégrable sur $I$.
    
    Soit maintenant $(J_n)_{n \in \mathbb{N}}$ une suite croissante de segments dont la réunion est $I$. Alors, en combinant les deux lemmes précédents, on a
    
    $\iint_{I \times I'} f = \lim_{n \to +\infty} \iint_{J_n \times I'} f = \lim_{n \to +\infty} \int_{J_n} g = \int_I g$
    
    ce que nous voulions démontrer.
    \end{proof}
    
    Nous allons en déduire un théorème nous permettant d'intervertir les signes d'intégration sur des intervalles.
    
    \begin{thm}[Fubini]
    \index{théorème de Fubini!sur des intervalles}
    Soit $I$ et $I'$ deux intervalles de $\mathbb{R}$, $f : I \times I' \to \mathbb{C}$ continue. On suppose que
    \begin{itemize}
    \item pour tout $x \in I$, $y \mapsto f(x,y)$ est intégrable sur $I'$
    \item pour tout $y \in I'$, $x \mapsto f(x,y)$ est intégrable sur $I$ et l'application $y \mapsto \int_I f(x,y) dx$ est continue par morceaux
    \item l'application $x \mapsto \int_{I'} f(x,y) dy$ est continue par morceaux sur $I$ et $x \mapsto \int_{I'} f(x,y) dy$ est continue par morceaux et intégrable sur $I$
    \end{itemize}
    Alors l'application $y \mapsto \int_I f(x,y) dx$ est intégrable sur $I'$ et on a
    
    $\int_I \left(\int_{I'} f(x,y) dy\right) dx = \int_{I'} \left(\int_I f(x,y) dx\right) dy$
    \end{thm}
    
    \begin{proof}
    Soit $P = J \times K$ un pavé contenu dans $I \times I'$. On a alors
    
    $\iint_{J \times K} f = \int_J \left(\int_K f(x,y) dy\right) dx \leq \int_J \left(\int_{I'} f(x,y) dy\right) dx \leq \int_I \left(\int_{I'} f(x,y) dy\right) dx$
    
    ce qui montre que $f$ est intégrable sur le rectangle $I \times I'$. Il en est donc de même pour $|f|$.
    
    Le théorème précédent assure que l'application $x \mapsto \int_{I'} f(x,y) dy$ est intégrable sur $I$ et que
    
    $\iint_{I \times I'} f = \int_I \left(\int_{I'} f(x,y) dy\right) dx$
    
    On applique à nouveau le théorème précédent en intervertissant le rôle de $I$ et $I'$, donc des variables $x$ et $y$. On obtient que l'application $y \mapsto \int_I f(x,y) dx$ est intégrable sur $I$ et que $\int_{I'} \left(\int_I f(x,y) dx\right) dy = \iint_{I \times I'} f$ ce qui nous donne l'égalité recherchée.
    \end{proof}
    
    \begin{rem}
    Moyennant la vérification que toutes les fonctions intégrées sont continues par morceaux, on pourra retenir ce théorème sous la forme
    
    $\left\{\begin{array}{l}
    \int_I \left(\int_{I'} f(x,y) dy\right) dx < +\infty \\
    \forall y \in E, \int_I f(x,y) dx < +\infty 
    \end{array}\right. \Rightarrow \int_I \left(\int_{I'} f(x,y) dy\right) dx = \int_{I'} \left(\int_I f(x,y) dx\right) dy$
    \end{rem}
    
    \subsection{La fonction $\Gamma$}
    
    \begin{de}
    \index{fonction Gamma}
    Pour $x \in ]0,+\infty[$, on pose $\Gamma(x) = \int_0^{+\infty} t^{x-1} e^{-t} dt$.
    \end{de}
    
    \begin{proof}
    En $+\infty$, $t^{x-1} e^{-t} = o(\frac{1}{t^2})$ donc la fonction est intégrable sur $[1,+\infty[$. En $0$, on a $t^{x-1} e^{-t} \sim t^{x-1} > 0$ donc la fonction est intégrable sur $]0,1]$ si et seulement si $x > 0$. La fonction $\Gamma$ est donc définie pour $x > 0$.
    \end{proof}
    
    \begin{prop}
    La fonction $\Gamma$ est de classe $C^\infty$ sur $]0,+\infty[$ et
    
    $\forall k \in \mathbb{N}, \forall x \in ]0,+\infty[, \Gamma^{(k)}(x) = \int_0^{+\infty} (\log t)^k e^{-t} t^{x-1} dt$
    \end{prop}
    
    \begin{proof}
    Soit $0 < a < 1 < b < +\infty$ et posons $f(x,t) = t^{x-1} e^{-t}$ pour $(x,t) \in [a,b] \times ]0,+\infty[$. Soit $J = [a,b]$ et $I = ]0,+\infty[$ ; la fonction $f : J \times I \to \mathbb{C}$, $(x,t) \mapsto f(x,t)$ est continue et admet des dérivées partielles par rapport à $x$, $(x,t) \mapsto \frac{\partial^i f}{\partial x^i}(x,t) = (\log t)^i e^{-t} t^{x-1}$, continues sur $J \times I$, $i = 1,\ldots,k$. Soit $\phi_i : ]0,+\infty[ \to \mathbb{R}^+$ définie par
    
    $\phi_i(t) = \begin{cases}
    (\log t)^i e^{-t} t^{a-1} &\text{si } t \in ]0,1] \\
    (\log t)^i e^{-t} t^{b-1} &\text{si } t \geq 1
    \end{cases}$
    
    Alors $\phi_i$ est continue par morceaux, intégrable sur $]0,+\infty[$ et on a
    
    $\forall (x,t) \in [a,b] \times ]0,+\infty[, |\frac{\partial^i f}{\partial x^i}(x,t)| \leq \phi_i(t)$
    
    D'après le théorème de dérivation des intégrales dépendant d'un paramètre, $\Gamma(x) = \int_0^{+\infty} f(x,t) dt$ est de classe $C^k$ sur $[a,b]$ et $\Gamma^{(i)}(x) = \int_{]0,+\infty[} \frac{\partial^i f}{\partial x^i}(x,t) dt$. Comme $a$ et $b$ sont quelconques, le résultat reste valide sur la réunion des intervalles $[a,b]$, donc $\Gamma$ est de classe $C^k$ sur $]0,+\infty[$ et
    
    $\forall k \in \mathbb{N}, \forall x \in ]0,+\infty[, \Gamma^{(k)}(x) = \int_0^{+\infty} (\log t)^k e^{-t} t^{x-1} dt$
    \end{proof}
    
    \begin{prop}
    Pour tout $x \in ]0,+\infty[$, $\Gamma(x + 1) = x \Gamma(x)$. En particulier $\forall n \in \mathbb{N}, \Gamma(n + 1) = n!$.
    \end{prop}
    
    \begin{proof}
    Soit $0 < a < b < +\infty$. On a par une intégration par parties
    
    $\int_a^b e^{-t} t^x dt = \left[-e^{-t} t^x\right]_a^b + x \int_a^b e^{-t} t^{x-1} dt$
    
    Il suffit alors de faire tendre $a$ vers $0$ et $b$ vers $+\infty$ ; le crochet admet la limite $0$ et on obtient $\Gamma(x + 1) = x \Gamma(x)$. Comme $\Gamma(1) = 1$, une récurrence immédiate donne $\Gamma(n + 1) = n!$.
    \end{proof}
    
    \begin{prop}
    $\Gamma(\frac{1}{2}) = 2 \int_0^{+\infty} e^{-t^2} dt = \sqrt{\pi}$.
    \end{prop}
    
    \begin{proof}
    Soit $0 < a < b < +\infty$. Le changement de variable $t = u^2$ donne
    
    $\int_a^b e^{-t} t^{-1/2} dt = 2 \int_{\sqrt{a}}^{\sqrt{b}} e^{-u^2} du$
    
    Il suffit alors de faire tendre $a$ vers $0$ et $b$ vers $+\infty$, pour avoir $\Gamma(\frac{1}{2}) = 2 \int_0^{+\infty} e^{-t^2} dt$. La valeur $\frac{\sqrt{\pi}}{2}$ de cette dernière intégrale sera admise (démontrée en exercice).
    \end{proof}
    
    \subsection{Méthodes directes}
    
    En ce qui concerne la continuité et la dérivabilité de $F$, on peut aussi tenter de majorer directement les expressions $F(x) - F(x_0) = \int_a^b (f(x,t) - f(x_0,t)) dt$ et $F(x) - F(x_0) - (x - x_0) \int_a^b \frac{\partial f}{\partial x}(x_0,t) dt = \int_a^b \left(f(x,t) - f(x_0,t) - (x - x_0) \frac{\partial f}{\partial x}(x_0,t)\right) dt$ en utilisant en particulier l'inégalité des accroissements finis et l'inégalité de Taylor-Lagrange à l'ordre 2 qui nous donneront, moyennant des hypothèses raisonnables,
    
    $|f(x,t) - f(x_0,t)| \leq |x - x_0| \sup_{y \in [x_0,x]} |\frac{\partial f}{\partial x}(y,t)|$
    
    et
    
    $|f(x,t) - f(x_0,t) - (x - x_0) \frac{\partial f}{\partial x}(x_0,t)| \leq \frac{|x - x_0|^2}{2} \sup_{y \in [x_0,x]} |\frac{\partial^2 f}{\partial x^2}(y,t)|$
    
    Des méthodes directes similaires peuvent être utilisées pour l'intégration.
    
    \begin{example}
    Soit $F(x) = \int_0^{+\infty} \frac{\sin t}{t} e^{-tx} dt$. Il est clair que la fonction est intégrable pour $x > 0$. Montrons que $F$ est dérivable sur $]0,+\infty[$. On a, pour $x_0 > 0$ et $x > \frac{x_0}{2}$
    
    $e^{-tx} - e^{-tx_0} + (x - x_0) t e^{-tx_0} = \frac{(x - x_0)^2}{2} t^2 e^{-t\xi}, \xi \in [x_0,x]$
    
    en appliquant la formule de Taylor-Lagrange à l'application $y \mapsto e^{-ty}$ sur $[x_0,x]$ (ou $[x,x_0]$), soit encore $|e^{-tx} - e^{-tx_0} + (x - x_0) t e^{-tx_0}| \leq \frac{(x-x_0)^2}{2} t^2 e^{-\frac{tx_0}{2}}$. On en déduit que $|F(x) - F(x_0) + (x - x_0) \int_0^{+\infty} \sin t e^{-tx_0} dt| \leq \frac{|x-x_0|^2}{2} \int_0^{+\infty} |\sin t| t e^{-\frac{tx_0}{2}} dt$, toutes les intégrales ayant manifestement un sens. En divisant par $|x - x_0|$, on en déduit que $F$ est dérivable en $x_0$ et que $F'(x_0) = - \int_0^{+\infty} \sin t e^{-tx_0} dt = - \frac{1}{1+x_0^2}$ (facile). On obtient donc $F(x) = K - \arctan x$, pour $x > 0$. Nous laissons en exercice au lecteur le soin de montrer que $\lim_{x \to \infty} F(x) = 0$, ce qui montrera que $K = \frac{\pi}{2}$. Bien entendu, on aurait pu aussi utiliser le théorème de convergence dominée pour démontrer la dérivabilité de $F$ sur $]0,+\infty[$ (ou plutôt sur $[a,+\infty[$ pour tout $a > 0$).
  \end{example}
  
  Ceci conclut la section sur les intégrales dépendant d'un paramètre. Les principaux concepts et théorèmes ont été présentés, avec leurs démonstrations et quelques exemples d'application. Les mots-clés importants ont été indexés pour faciliter la référence. Le texte est structuré en sections et sous-sections cohérentes, utilisant les environnements LaTeX demandés pour les théorèmes, définitions, exemples, etc.