\documentclass[]{article}
\usepackage[T1]{fontenc}
\usepackage{lmodern}
\usepackage{amssymb,amsmath}
\usepackage{ifxetex,ifluatex}
\usepackage{fixltx2e} % provides \textsubscript
% use upquote if available, for straight quotes in verbatim environments
\IfFileExists{upquote.sty}{\usepackage{upquote}}{}
\ifnum 0\ifxetex 1\fi\ifluatex 1\fi=0 % if pdftex
  \usepackage[utf8]{inputenc}
\else % if luatex or xelatex
  \ifxetex
    \usepackage{mathspec}
    \usepackage{xltxtra,xunicode}
  \else
    \usepackage{fontspec}
  \fi
  \defaultfontfeatures{Mapping=tex-text,Scale=MatchLowercase}
  \newcommand{\euro}{€}
\fi
% use microtype if available
\IfFileExists{microtype.sty}{\usepackage{microtype}}{}
\ifxetex
  \usepackage[setpagesize=false, % page size defined by xetex
              unicode=false, % unicode breaks when used with xetex
              xetex]{hyperref}
\else
  \usepackage[unicode=true]{hyperref}
\fi
\hypersetup{breaklinks=true,
            bookmarks=true,
            pdfauthor={},
            pdftitle={Fonctions implicites et inversion locale},
            colorlinks=true,
            citecolor=blue,
            urlcolor=blue,
            linkcolor=magenta,
            pdfborder={0 0 0}}
\urlstyle{same}  % don't use monospace font for urls
\setlength{\parindent}{0pt}
\setlength{\parskip}{6pt plus 2pt minus 1pt}
\setlength{\emergencystretch}{3em}  % prevent overfull lines
\setcounter{secnumdepth}{0}
 
/* start css.sty */
.cmr-5{font-size:50%;}
.cmr-7{font-size:70%;}
.cmmi-5{font-size:50%;font-style: italic;}
.cmmi-7{font-size:70%;font-style: italic;}
.cmmi-10{font-style: italic;}
.cmsy-5{font-size:50%;}
.cmsy-7{font-size:70%;}
.cmex-7{font-size:70%;}
.cmex-7x-x-71{font-size:49%;}
.msbm-7{font-size:70%;}
.cmtt-10{font-family: monospace;}
.cmti-10{ font-style: italic;}
.cmbx-10{ font-weight: bold;}
.cmr-17x-x-120{font-size:204%;}
.cmsl-10{font-style: oblique;}
.cmti-7x-x-71{font-size:49%; font-style: italic;}
.cmbxti-10{ font-weight: bold; font-style: italic;}
p.noindent { text-indent: 0em }
td p.noindent { text-indent: 0em; margin-top:0em; }
p.nopar { text-indent: 0em; }
p.indent{ text-indent: 1.5em }
@media print {div.crosslinks {visibility:hidden;}}
a img { border-top: 0; border-left: 0; border-right: 0; }
center { margin-top:1em; margin-bottom:1em; }
td center { margin-top:0em; margin-bottom:0em; }
.Canvas { position:relative; }
li p.indent { text-indent: 0em }
.enumerate1 {list-style-type:decimal;}
.enumerate2 {list-style-type:lower-alpha;}
.enumerate3 {list-style-type:lower-roman;}
.enumerate4 {list-style-type:upper-alpha;}
div.newtheorem { margin-bottom: 2em; margin-top: 2em;}
.obeylines-h,.obeylines-v {white-space: nowrap; }
div.obeylines-v p { margin-top:0; margin-bottom:0; }
.overline{ text-decoration:overline; }
.overline img{ border-top: 1px solid black; }
td.displaylines {text-align:center; white-space:nowrap;}
.centerline {text-align:center;}
.rightline {text-align:right;}
div.verbatim {font-family: monospace; white-space: nowrap; text-align:left; clear:both; }
.fbox {padding-left:3.0pt; padding-right:3.0pt; text-indent:0pt; border:solid black 0.4pt; }
div.fbox {display:table}
div.center div.fbox {text-align:center; clear:both; padding-left:3.0pt; padding-right:3.0pt; text-indent:0pt; border:solid black 0.4pt; }
div.minipage{width:100%;}
div.center, div.center div.center {text-align: center; margin-left:1em; margin-right:1em;}
div.center div {text-align: left;}
div.flushright, div.flushright div.flushright {text-align: right;}
div.flushright div {text-align: left;}
div.flushleft {text-align: left;}
.underline{ text-decoration:underline; }
.underline img{ border-bottom: 1px solid black; margin-bottom:1pt; }
.framebox-c, .framebox-l, .framebox-r { padding-left:3.0pt; padding-right:3.0pt; text-indent:0pt; border:solid black 0.4pt; }
.framebox-c {text-align:center;}
.framebox-l {text-align:left;}
.framebox-r {text-align:right;}
span.thank-mark{ vertical-align: super }
span.footnote-mark sup.textsuperscript, span.footnote-mark a sup.textsuperscript{ font-size:80%; }
div.tabular, div.center div.tabular {text-align: center; margin-top:0.5em; margin-bottom:0.5em; }
table.tabular td p{margin-top:0em;}
table.tabular {margin-left: auto; margin-right: auto;}
div.td00{ margin-left:0pt; margin-right:0pt; }
div.td01{ margin-left:0pt; margin-right:5pt; }
div.td10{ margin-left:5pt; margin-right:0pt; }
div.td11{ margin-left:5pt; margin-right:5pt; }
table[rules] {border-left:solid black 0.4pt; border-right:solid black 0.4pt; }
td.td00{ padding-left:0pt; padding-right:0pt; }
td.td01{ padding-left:0pt; padding-right:5pt; }
td.td10{ padding-left:5pt; padding-right:0pt; }
td.td11{ padding-left:5pt; padding-right:5pt; }
table[rules] {border-left:solid black 0.4pt; border-right:solid black 0.4pt; }
.hline hr, .cline hr{ height : 1px; margin:0px; }
.tabbing-right {text-align:right;}
span.TEX {letter-spacing: -0.125em; }
span.TEX span.E{ position:relative;top:0.5ex;left:-0.0417em;}
a span.TEX span.E {text-decoration: none; }
span.LATEX span.A{ position:relative; top:-0.5ex; left:-0.4em; font-size:85%;}
span.LATEX span.TEX{ position:relative; left: -0.4em; }
div.float img, div.float .caption {text-align:center;}
div.figure img, div.figure .caption {text-align:center;}
.marginpar {width:20%; float:right; text-align:left; margin-left:auto; margin-top:0.5em; font-size:85%; text-decoration:underline;}
.marginpar p{margin-top:0.4em; margin-bottom:0.4em;}
.equation td{text-align:center; vertical-align:middle; }
td.eq-no{ width:5%; }
table.equation { width:100%; } 
div.math-display, div.par-math-display{text-align:center;}
math .texttt { font-family: monospace; }
math .textit { font-style: italic; }
math .textsl { font-style: oblique; }
math .textsf { font-family: sans-serif; }
math .textbf { font-weight: bold; }
.partToc a, .partToc, .likepartToc a, .likepartToc {line-height: 200%; font-weight:bold; font-size:110%;}
.chapterToc a, .chapterToc, .likechapterToc a, .likechapterToc, .appendixToc a, .appendixToc {line-height: 200%; font-weight:bold;}
.index-item, .index-subitem, .index-subsubitem {display:block}
.caption td.id{font-weight: bold; white-space: nowrap; }
table.caption {text-align:center;}
h1.partHead{text-align: center}
p.bibitem { text-indent: -2em; margin-left: 2em; margin-top:0.6em; margin-bottom:0.6em; }
p.bibitem-p { text-indent: 0em; margin-left: 2em; margin-top:0.6em; margin-bottom:0.6em; }
.paragraphHead, .likeparagraphHead { margin-top:2em; font-weight: bold;}
.subparagraphHead, .likesubparagraphHead { font-weight: bold;}
.quote {margin-bottom:0.25em; margin-top:0.25em; margin-left:1em; margin-right:1em; text-align:\\jmathmathustify;}
.verse{white-space:nowrap; margin-left:2em}
div.maketitle {text-align:center;}
h2.titleHead{text-align:center;}
div.maketitle{ margin-bottom: 2em; }
div.author, div.date {text-align:center;}
div.thanks{text-align:left; margin-left:10%; font-size:85%; font-style:italic; }
div.author{white-space: nowrap;}
.quotation {margin-bottom:0.25em; margin-top:0.25em; margin-left:1em; }
h1.partHead{text-align: center}
.sectionToc, .likesectionToc {margin-left:2em;}
.subsectionToc, .likesubsectionToc {margin-left:4em;}
.subsubsectionToc, .likesubsubsectionToc {margin-left:6em;}
.frenchb-nbsp{font-size:75%;}
.frenchb-thinspace{font-size:75%;}
.figure img.graphics {margin-left:10%;}
/* end css.sty */

\title{Fonctions implicites et inversion locale}
\author{}
\date{}

\begin{document}
\maketitle

\textbf{Warning: 
requires JavaScript to process the mathematics on this page.\\ If your
browser supports JavaScript, be sure it is enabled.}

\begin{center}\rule{3in}{0.4pt}\end{center}

{[}
{[}
{[}{]}
{[}

\subsubsection{15.4 Fonctions implicites et inversion locale}

\paragraph{15.4.1 Position du problème des fonctions implicites}

Soit E,F et G trois espaces vectoriels normés, W un ouvert de E \times F, f :
W \rightarrow~ G. On considère la courbe \Gamma = \(x,y) \in
W∣f(x,y) = 0\. On se pose la
question de savoir si \Gamma est le graphe d'une fonction \phi d'un ouvert U de
E dans F, autrement dit si f(x,y) = 0 \Leftrightarrow y =
\phi(x).

Cette question globale n'admet pas vraiment de réponse satisfaisante et
nous allons la transformer en une question locale. Soit (a,b) \in \Gamma. On se
pose la question de savoir si \Gamma, au voisinage de (a,b), est le graphe
d'une fonction \phi d'un ouvert U de E dans F, autrement dit si il existe U
ouvert contenant a et V ouvert contenant b tels que, pour (a,b) \in U \times V
, f(x,y) = 0 \Leftrightarrow y = \phi(x). Cela revient à
demander que \Gamma \bigcap (U \times V ) soit un graphe, autrement dit que

\forall~x \in U, \\exists~!y \in V,
f(x,y) = 0

Nous cherchons en plus des propriétés de la fonction \phi (lorsqu'elle
existe) à partir de propriétés de la fonction f.

Supposons que E = \mathbb{R}~^n, F = \mathbb{R}~^p et G =
\mathbb{R}~^q. On a f(x,y) =
(f_1(x,y),\\ldots,f_q~(x,y))
si bien que

f(x,y) = 0 \Leftrightarrow \left
\\matrix\,f_1(x_1,\\ldots,x_n,y_1,\\\ldots,y_p~)
= 0 \cr \cr
f_q(x_1,\\ldots,x_n,y_1,\\\ldots,y_p~)
= 0\right .

Pour
(x_1,\\ldots,x_n~)
fixé dans U \inV(a), ce système doit déterminer un unique
(y_1,\\ldots,y_p~)
dans V \inV(b). Ceci semble nécessiter qu'il y ait autant d'équations que
d'inconnues, c'est-à-dire que p = q.

Même, dans ce cas, l'exemple n = p = q = 1 et f(x,y) = x^2 +
y^2 - 1 montre que la réponse est positive en (a,b) \in
\mathbb{R}~^2 si b\neq~0, mais qu'elle est
négative aux points (1,0) et (-1,0) de \Gamma, points où l'on a  \partial~f
\over \partial~y (a,b) = 0.

Le théorème des fonctions implicites va nous donner une condition
suffisante pour que la réponse au problème local soit positive.

\paragraph{15.4.2 Théorème des fonctions implicites}

Théorème~15.4.1 Soit W un ouvert de \mathbb{R}~^n \times \mathbb{R}~^p et f
: W \rightarrow~ \mathbb{R}~^p de classe \mathcal{C}^1. On pose x =
(x_1,\\ldots,x_n~),
y =
(y_1,\\ldots,y_p~)
et f =
(f_1,\\ldots,f_p~).
Soit (a,b) \in W tel que f(a,b) = 0 et Q = \left (
\partial~f_i \over \partial~y_\\jmathmath
(a,b)\right )_1\leqi,\\jmathmath\leqp est inversible. Alors, il
existe U \inV(a) et V \inV(b) (ouverts) tels que

\forall~x \in U \\exists~!y \in V
\text tel que f(x,y) = 0.

Si l'on pose y = \phi(x), \phi est continue sur U et de classe \mathcal{C}^1
sur un voisinage U_0 de a.

Démonstration Soit \psi : W \rightarrow~ \mathbb{R}~^p,
(x,y)\mapsto~\psi(x,y) = \psi_x(y) = y -
Q^-1(f(x,y)). On a de manière évidente

f(x,y) = 0 \Leftrightarrow \psi_x(y) = y.

On va essayer d'appliquer le théorème du point fixe à l'équation
\psi_x(y) = y. Notons Q(x,y) = \left (
\partial~f_i \over \partial~y_\\jmathmath
(x,y)\right )_1\leqi,\\jmathmath\leqp, de sorte que Q = Q(a,b).
Puisque Q^-1 est une application linéaire, elle est sa propre
différentielle en tout point et la matrice de d\psi_x(y) est donc
la matrice

J_\psi_x(y) = I_p - Q^-1 \cdotQ(x,y)

Donc d\psi_a(b) = 0 et l'application
(x,y)\mapsto~d\psi_x(y) est continue. On en
déduit qu'il existe r \textgreater{} 0 tel que

\x - a\ \leq
r\text et \y -
b\ \leq r \rigtharrow~\
d\psi_x(y)\ \leq 1 \over
2 .

Soit x \in B'(a,r),y,y' \in B'(b,r). On a alors

\\psi_x(y) -
\psi_x(y')\ \leq\ y
-
y'\sup_z\in{[}y,y'{]}\d\psi_x~(z)\
\leq 1 \over 2 \y -
y'\

d'après l'inégalité des accroissements finis. Puisque \psi est continue en
(a,b), il existe U_1 voisinage ouvert de a inclus dans B'(a,r)
tel que x \in U_1 \rigtharrow~\ \psi(x,b) -
\psi(a,b)\ \leq r \over 2 , soit
encore, puisque \psi(a,b) = b, x \in U_1 \rigtharrow~\
\psi(x,b) - b\ \leq r \over 2 .
Pour x \in U_1 et y \in B'(b,r), on a donc

\begin{align*}
\\psi_x(y) -
b& \leq&
\\psi_x(y) -
\psi_x(b)\ +\
\psi(x,b) - \psi(a,b)\\%&
\\ & \leq& 1 \over 2
\y - b\ + r
\over 2 \leq r. \%& \\
\end{align*}

Donc, si x \in U_1, \psi_x est une application de B'(b,r)
dans B'(b,r) qui est  1 \over 2
-\textcontractante~; mais B'(b,r) est un espace
métrique complet (fermé dans un complet). Donc pour x \in U_1, il
existe un unique y \in V _1 = B'(b,r) tel que \psi_x(y) =
y, c'est-à-dire f(x,y) = 0.

Appelons \phi(x) cet unique y, on définit ainsi \phi : U_1 \rightarrow~ V
_1 telle que \psi_x(\phi(x)) = \phi(x). Montrons que \phi est
continue. Soit x et x_0 dans U_1. On a

\begin{align*} \\phi(x) -
\phi(x_0)\ =\
\psi_x(\phi(x)) -
\psi_x_0(\phi(x_0))&&
\%& \\ & & \%&
\\ & \leq&
\\psi_x(\phi(x)) -
\psi_x(\phi(x_0))\
+\ \psi(x,\phi(x_0)) -
\psi(x_0,\phi(x_0))\\%&
\\ & \leq& 1 \over 2
\\phi(x) -
\phi(x_0)\ +\
\psi(x,\phi(x_0)) -
\psi(x_0,\phi(x_0)\ \%&
\\ \end{align*}

soit encore

\\phi(x) -
\phi(x_0)\ \leq
2\\psi(x,\phi(x_0)) -
\psi(x_0,\phi(x_0)\.

Comme x\mapsto~\psi(x,\phi(x_0)) est continue en
x_0, il en est de même de x\mapsto~\phi(x).

Soit alors V = B(b,r) et U = U_1 \bigcap \phi^-1(V ). V est
ouvert, et il en est de même de U comme intersection de l'ouvert
U_1 et de l'image réciproque de l'ouvert V par l'application
continue \phi. Pour x \in U, il existe un unique y \in B'(b,r) tel que f(x,y) =
0, avec y = \phi(x). Mais comme x \in \phi^-1(V ), on a en fait y \in V
et en définitive

\forall~x \in U \\exists~!y \in V
\text tel que f(x,y) = 0.

Soit P = \left ( \partial~f_i \over
\partial~x_\\jmathmath (a,b)\right )_1\leqi\leqp,1\leq\\jmathmath\leqn et soit
h \in \mathbb{R}~^n et k \in \mathbb{R}~^p. Les formules

\begin{align*} f(a + h,b + k) =
\sum _i=1^n~ \partial~f
\over \partial~x_i (a,b)h_i +
\sum _i=1^p~ \partial~f
\over \partial~y_i (a,b)k_i +
o(\(h,k)\)& & \%&
\\ \end{align*}

se traduisent par

f(a + h,b + k) = Ph + Qk +
o(\h\
+\ k\).

Prenons k = \theta(h) = \phi(a + h) - \phi(a) = \phi(a + h) - b. On a alors

\begin{align*} 0& =& f(a + h,\phi(a + h)) = f(a + h,b
+ \theta(h))\%& \\ & =& Ph + Q\theta(h) +
(\h\
+\ \theta(h)\)\epsilon(h) \%&
\\ \end{align*}

soit encore

\theta(h) = -Q^-1Ph +
(\h\
+\ \theta(h)\)\eta(h)

avec \eta(h) = -Q^-1(\epsilon(h)). Comme on a
lim_h\rightarrow~0~\eta(h) = 0, soit \rho
\textgreater{} 0 tel que h \textless{} \rho \rigtharrow~\eta(h)
\textless{} 1 \over 2 . Alors pour h \textless{} \rho,
on a

\\theta(h)\
\leq\
Q^-1P\\h\
+ 1 \over 2
(\h\
+\ \theta(h)\),

soit encore

\\theta(h)\ \leq
(2\Q^-1P\ +
1)\h\.

On a donc \\theta(h)\ =
O(\h\), soit encore
(\h\
+\ \theta(h)\)\eta(h) =
o(\h\), ce qui montre
que

\phi(a + h) - \phi(a) = \theta(h) = -Q^-1Ph +
o(\h\).

Donc \phi est différentiable au point a et sa différentielle est -
Q^-1P. On montre de même que \phi est différentiable en tout
point x assez voisin de a pour que la matrice Q(x,\phi(x)) reste inversible
et que l'on a encore d\phi(x) = -Q(x,\phi(x))^-1P(x,\phi(x)), ce qui
montre que \phi est de classe \mathcal{C}^1 sur un tel voisinage.

\paragraph{15.4.3 Applications du théorème des fonctions implicites}

Nous nous intéresserons tout particulièrement au cas p = 1~; dans ce cas
Q = \left (\matrix\, \partial~f
\over \partial~y (a,b)\right ) et la matrice
est inversible si et seulement si  \partial~f \over \partial~y
(a,b)\neq~0. On obtient donc la formulation
suivante

Théorème~15.4.2 Soit W un ouvert de \mathbb{R}~^n \times \mathbb{R}~ et f : W \rightarrow~ \mathbb{R}~ de
classe \mathcal{C}^1,
(x_1,\\ldots,x_n,y)\mapsto~f(x_1,\\\ldots,x_n~,y).
Soit
(a_1,\\ldots,a_n~,b)
\in W tel que
f(a_1,\\ldots,a_n~,b)
= 0 et  \partial~f \over \partial~y
(a_1,\\ldots,a_n,b)\mathrel\neq~~0.
Alors, il existe U
\inV(a_1,\\ldots,a_n~)
et V \inV(b) (ouverts) tels que

\forall~(x_1,\\\ldots,x_n~)
\in U, \exists~!y \in V \text tel que
f(x_1,\\ldots,x_n~,y)
= 0.

Si l'on pose y =
\phi(x_1,\\ldots,x_n~),
\phi est continue sur U et de classe \mathcal{C}^1 sur un voisinage
U_0 de a.

Remarque~15.4.1 Le calcul des dérivées partielles de \phi se fait très
facilement en utilisant les formes différentielles. Les variables
x_1,\\ldots,x_n~
et y étant liées par la relation
f(x_1,\\ldots,x_n~,y)
= 0, on obtient par différentiation

\sum _i=1^n~ \partial~f
\over \partial~x_i
(x_1,\ldots,x_n,y)dx_i~
+ \partial~f \over \partial~y
(x_1,\ldots,x_n~,y)dy = 0

soit encore

dy = -\sum _i=1^n~  \partial~f
\over \partial~x_i
(x_1,\ldots,x_n~,y)
\over  \partial~f \over \partial~y
(x_1,\ldots,x_n,y)~
dx_i

On en déduit que

 \partial~\phi \over \partial~x_i
(x_1,\\ldots,x_n~)
= \partial~y \over \partial~x_i
(x_1,\\ldots,x_n~)
= -  \partial~f \over \partial~x_i
(x_1,\\ldots,x_n~,y)
\over  \partial~f \over \partial~y
(x_1,\\ldots,x_n,y)~

si y =
\phi(x_1,\\ldots,x_n~).

Théorème~15.4.3 Soit W un ouvert de \mathbb{R}~^2 et f : W \rightarrow~ \mathbb{R}~ de
classe \mathcal{C}^1. Soit \Gamma = \(x,y) \in
W∣f(x,y) = 0\. On suppose
que \forall~~(a,b) \in \Gamma, \left ( \partial~f
\over \partial~x (a,b), \partial~f \over \partial~y
(a,b)\right )\neq~(0,0). Alors, au
voisinage de chacun de ses points, \Gamma est soit le graphe d'une
application de classe \mathcal{C}^1 x\mapsto~y =
\phi(x), soit le graphe d'une application de classe \mathcal{C}^1
y\mapsto~x = \psi(y). La tangente à ce graphe au point
(a,b) est la droite d'équation

(x - a) \partial~f \over \partial~x (a,b) + (y - b) \partial~f
\over \partial~y (a,b) = 0

Démonstration Si par exemple  \partial~f \over \partial~y
(a,b)\neq~0, le théorème précédent s'applique et
permet de conclure, qu'au voisinage de (a,b), \Gamma est le graphe d'une
application de classe \mathcal{C}^1, x\mapsto~y =
\phi(x). La tangente à ce graphe est la droite d'équation y - b = \phi'(a)(x -
a) avec \phi'(a) = -  \partial~f \over \partial~x (a,b)
\over  \partial~f \over \partial~y (a,b) ce qui
donne l'équation ci dessus. Si  \partial~f \over \partial~x
(a,b)\neq~0, il suffit d'échanger les rôles \\jmathmathoués
par x et y.

Nous avons un théorème similaire pour les surfaces de \mathbb{R}~^3

Théorème~15.4.4 Soit W un ouvert de \mathbb{R}~^3 et f : W \rightarrow~ \mathbb{R}~ de
classe \mathcal{C}^1. Soit \Sigma = \(x,y,z) \in
W∣f(x,y,z) = 0\. On suppose
que \forall~~(a,b,c) \in \Gamma, \left ( \partial~f
\over \partial~x (a,b,c), \partial~f \over \partial~y
(a,b,c), \partial~f \over \partial~z (a,b,c)\right
)\neq~(0,0,0). Alors, au voisinage de chacun de
ses points, \Sigma est soit le graphe d'une application de classe
\mathcal{C}^1, (x,y)\mapsto~z = \phi(x,y), soit le
graphe d'une application de classe \mathcal{C}^1,
(y,z)\mapsto~x = \psi(y,z), soit le graphe d'une
application de classe \mathcal{C}^1, (x,z)\mapsto~y
= \psi(x,z). Le plan tangent à ce graphe au point (a,b,c) est le plan
d'équation

(x - a) \partial~f \over \partial~x (a,b,c) + (y - b) \partial~f
\over \partial~y (a,b,c) + (z - c) \partial~f \over
\partial~z (a,b,c) = 0

Démonstration La même que précédemment sauf pour ce qui concerne
l'équation du plan tangent. Supposons que localement \Sigma est le graphe
d'une application de classe \mathcal{C}^1
(x,y)\mapsto~z = \phi(x,y). La surface est paramétrée
par (x,y)\mapsto~(x,y,\phi(x,y)) et les deux vecteurs
tangents dérivés partiels sont  \partial~ \over \partial~x
(x,y,\phi(x,y)) = (1,0, \partial~\phi \over \partial~x (x,y)) et  \partial~
\over \partial~y (x,y,\phi(x,y)) = (0,1, \partial~\phi \over
\partial~x (x,y)). Le plan tangent est le plan parallèle à ces deux vecteurs
(pour (x,y) = (a,b)) et contenant le point (a,b,c), c'est-à-dire le plan
d'équation

\left
\matrix\,x - a&1 &0
\cr y - b&0 &1 \cr z - c& \partial~\phi
\over \partial~x (a,b)& \partial~\phi \over \partial~y
(a,b)\right  = 0

Mais on a

\begin{align*} \partial~\phi \over \partial~x (a,b)
= -  \partial~f \over \partial~x (a,b,c) \over  \partial~f
\over \partial~z (a,b,c) \text et  \partial~\phi
\over \partial~y (a,b) = -  \partial~f \over \partial~y
(a,b,c) \over  \partial~f \over \partial~z (a,b,c)
& & \%& \\
\end{align*}

Il suffit de reporter et de développer le déterminant suivant la
première colonne pour obtenir l'équation du plan tangent sous la forme
souhaitée.

\paragraph{15.4.4 Difféomorphismes et inversion locale}

Définition~15.4.1 Soit E et F deux K-espaces vectoriels normés, U un
ouvert de E et V un ouvert de F. On dit que f : U \rightarrow~ V est un
difféomorphisme de classe \mathcal{C}^1 si (i) f est bi\\jmathmathective de U sur
V (ii) f et f^-1 sont de classe \mathcal{C}^1.

Remarque~15.4.2 Comme pour les fonctions d'une variable, le fait que f
soit bi\\jmathmathective et de classe \mathcal{C}^1 n'implique évidemment pas que
f^-1 soit de classe \mathcal{C}^1.

Théorème~15.4.5 Soit E et F deux K-espaces vectoriels normés, U un
ouvert de E, V un ouvert de F, f : U \rightarrow~ V un difféomorphisme de classe
\mathcal{C}^1. Alors, pour tout x \in U, df(x) est un isomorphisme
d'espace vectoriel de E sur F et on a

\forall~y \in V, d(f^-1~)(y) =
\left (df(f^-1(y))\right
)^-1

Démonstration Soit g = f^-1 : V \rightarrow~ U. On a g \cdot f =
\mathrmId_U. Comme f est différentiable en x
et g en f(x), on a \mathrmId_E =
d(\mathrmId_U)(x) = d(g \cdot f)(x) = dg(f(x)) \cdot
df(x). On montre de la même fa\ccon que
\mathrmId_F =
d(\mathrmId_V )(f(x)) = d(f \cdot g)(f(x)) =
df(x) \cdot dg(f(x)). On en déduit que df(x) est un isomorphisme de E sur F
d'isomorphisme réciproque dg(f(x)).

Remarque~15.4.3 On vérifie facilement à partir de la formule ci dessus
que si f est à la fois de classe C^k et un \mathcal{C}^1
difféomorphisme, alors f^-1 est aussi de classe
C^k. On dit alors que f est un
C^k-difféomorphisme.

Remarque~15.4.4 Si E et F sont de dimensions finies, l'existence d'un
\mathcal{C}^1 difféomorphisme d'un ouvert de E sur un ouvert de F
nécessite que E et F aient même dimension~; il ne peut y avoir de
difféomorphisme d'un ouvert de \mathbb{R}~^n sur un ouvert de
\mathbb{R}~^p pour n\neq~p. Si f : U \rightarrow~ V est un
difféomorphisme d'un ouvert U de \mathbb{R}~^n sur un ouvert V de
\mathbb{R}~^n, on peut calculer les dérivées partielles de
f^-1 à l'aide des matrices \\jmathmathacobiennes grâce à la formule
J_f^-1(y) = \left
(J_f(f^-1(y))\right )^-1.

Nous allons maintenant nous intéresser à une réciproque partielle (en
fait locale) du théorème précédent.

Théorème~15.4.6 (inversion locale). Soit U un ouvert de \mathbb{R}~^n
et f : U \rightarrow~ \mathbb{R}~^n une application de classe \mathcal{C}^1. Soit
a \in U tel que df(a) soit un isomorphisme d'espace vectoriel de
\mathbb{R}~^n sur \mathbb{R}~^n (autrement dit la matrice
J_f(a) est inversible). Alors il existe un ouvert U_0
contenant a et un ouvert V _0 contenant f(a) tel que f induise
un difféomorphisme de classe \mathcal{C}^1 de U_0 sur V
_0.

Démonstration Considérons l'application g : \mathbb{R}~^n \times U \rightarrow~
\mathbb{R}~^n, (y,x)\mapsto~f(x) - y. La matrice
\left ( \partial~g_i \over
\partial~x_\\jmathmath (f(a),a)\right )_1\leqi,\\jmathmath\leqn n'est
autre que la matrice J_f(a) qui est inversible. On peut donc
appliquer le théorème des fonctions implicites. On en déduit qu'il
existe V _1 ouvert contenant f(a) et un ouvert U_1
contenant a tel que \forall~y \in V _1~,
\exists!x \in U_1~, g(y,x) = 0, autrement dit

\forall~y \in V _1~,
\exists!x \in U_1~, f(x) = y

Si on pose x = g(y), on sait que quitte à restreindre V _1, on
peut supposer que g est de classe \mathcal{C}^1. Par définition même,
on a f(g(y)) = y pour y \in V _1. Par contre, il n'est pas vrai
en général que, pour x \in U_1, on ait g(f(x)) = x car il n'y a
pas de raison que f(x) appartienne à V _1. Mais comme f est
continue et V _1 ouvert, f^-1(V _1) est un
ouvert contenant a et donc il en est de même de U_1 \bigcap
f^-1(V _1) = U_0. Pour x \in U_0, on
a x \in U_1 et f(x) \in V _1. Comme g(f(x)) est l'unique
x' dans U_1 tel que f(x') = f(x) et comme x convient bien
évidemment, on a g(f(x)) = x pour x \in U_0. Comme on a aussi
f(g(y)) = y pour y \in V _1, f est bi\\jmathmathective de U_0 sur
V _1, et son inverse est g qui est encore de classe
\mathcal{C}^1.

Remarque~15.4.5 Le théorème précédent est uniquement local. A partir de
la dimension 2, il n'existe pas de moyen local simple qui permette de
garantir l'in\\jmathmathectivité globale de f, comme le montre l'exemple de f
:{]}0,+\infty~{[}\times\mathbb{R}~ \rightarrow~ \mathbb{R}~^2
\diagdown\(0,0)\,
(\rho,\theta)\mapsto~(\rhocos~
\theta,\rhosin~ \theta). La matrice \\jmathmathacobienne est
J_f(\rho,\theta) = \left
(\matrix\,cos~
\theta&-\rhosin~ \theta \cr
sin \theta&\rho\cos~ \theta
\right ) qui est inversible (de déterminant
\rho\neq~0)~; l'application f est localement
in\\jmathmathective (et même localement un difféomorphisme), mais elle ne l'est
pas globalement puisque f(\rho,\theta + 2\pi~) = f(\rho,\theta).

Corollaire~15.4.7 Soit U un ouvert de \mathbb{R}~^n et f : U \rightarrow~
\mathbb{R}~^n une application de classe \mathcal{C}^1. On suppose que
(i) f est in\\jmathmathective (ii) \forall~~x \in U,
J_f(x) est une matrice inversible. Alors f(U) = V est un ouvert
de \mathbb{R}~^n et f est un \mathcal{C}^1 difféomorphisme de U sur V
.

Démonstration Soit g = f^-1 : V \rightarrow~ U. Soit y \in V , y = f(x).
Comme J_f(x) est une matrice inversible, le théorème
d'inversion locale assure l'existence d'un ouvert U_0 contenant
x et d'un ouvert V _0 contenant y = f(x) tel que f induise un
\mathcal{C}^1 difféomorphisme de U_0 sur V _0. Mais
alors V _0 = f(U_0) \subset~ V . Ceci nous garantit que V est
un voisinage de y. Donc V est un voisinage de chacun de ses points, et
donc il est ouvert. Mais d'autre part, le difféomorphisme réciproque de
f_∣_U_ 0 :
U_0 \rightarrow~ V _0 ne peut être que
g_∣_V_ 0. On en
déduit que g est de classe \mathcal{C}^1 au point y. Donc g est de
classe \mathcal{C}^1 sur V et f est un \mathcal{C}^1 difféomorphisme
de U sur V .

Remarque~15.4.6 Un difféomorphisme f d'un ouvert U de \mathbb{R}~^n sur
un ouvert V de \mathbb{R}~^n,
(\alpha_1,\\ldots,\alpha_n)\mapsto~(x_1,\\\ldots,x_n~)
=
(f_1(\alpha_1,\\ldots,\alpha_n),\\\ldots,f_n(\alpha_1,\\\ldots,\alpha_n~))
est souvent appelé un système de coordonnées curvilignes sur V . Un tel
système permet de repérer un point
(x_1,\\ldots,x_n~)
de V par ses coordonnées curvilignes
(\alpha_1,\\ldots\alpha_n~).
Les coordonnées polaires, cylindriques ou sphériques sont typiques de
coordonnées curvilignes locales.

{[}
{[}
{[}
{[}

\end{document}
