\documentclass[]{article}
\usepackage[T1]{fontenc}
\usepackage{lmodern}
\usepackage{amssymb,amsmath}
\usepackage{ifxetex,ifluatex}
\usepackage{fixltx2e} % provides \textsubscript
% use upquote if available, for straight quotes in verbatim environments
\IfFileExists{upquote.sty}{\usepackage{upquote}}{}
\ifnum 0\ifxetex 1\fi\ifluatex 1\fi=0 % if pdftex
  \usepackage[utf8]{inputenc}
\else % if luatex or xelatex
  \ifxetex
    \usepackage{mathspec}
    \usepackage{xltxtra,xunicode}
  \else
    \usepackage{fontspec}
  \fi
  \defaultfontfeatures{Mapping=tex-text,Scale=MatchLowercase}
  \newcommand{\euro}{€}
\fi
% use microtype if available
\IfFileExists{microtype.sty}{\usepackage{microtype}}{}
\ifxetex
  \usepackage[setpagesize=false, % page size defined by xetex
              unicode=false, % unicode breaks when used with xetex
              xetex]{hyperref}
\else
  \usepackage[unicode=true]{hyperref}
\fi
\hypersetup{breaklinks=true,
            bookmarks=true,
            pdfauthor={},
            pdftitle={Endomorphismes d'un espace hermitien},
            colorlinks=true,
            citecolor=blue,
            urlcolor=blue,
            linkcolor=magenta,
            pdfborder={0 0 0}}
\urlstyle{same}  % don't use monospace font for urls
\setlength{\parindent}{0pt}
\setlength{\parskip}{6pt plus 2pt minus 1pt}
\setlength{\emergencystretch}{3em}  % prevent overfull lines
\setcounter{secnumdepth}{0}
 
/* start css.sty */
.cmr-5{font-size:50%;}
.cmr-7{font-size:70%;}
.cmmi-5{font-size:50%;font-style: italic;}
.cmmi-7{font-size:70%;font-style: italic;}
.cmmi-10{font-style: italic;}
.cmsy-5{font-size:50%;}
.cmsy-7{font-size:70%;}
.cmex-7{font-size:70%;}
.cmex-7x-x-71{font-size:49%;}
.msbm-7{font-size:70%;}
.cmtt-10{font-family: monospace;}
.cmti-10{ font-style: italic;}
.cmbx-10{ font-weight: bold;}
.cmr-17x-x-120{font-size:204%;}
.cmsl-10{font-style: oblique;}
.cmti-7x-x-71{font-size:49%; font-style: italic;}
.cmbxti-10{ font-weight: bold; font-style: italic;}
p.noindent { text-indent: 0em }
td p.noindent { text-indent: 0em; margin-top:0em; }
p.nopar { text-indent: 0em; }
p.indent{ text-indent: 1.5em }
@media print {div.crosslinks {visibility:hidden;}}
a img { border-top: 0; border-left: 0; border-right: 0; }
center { margin-top:1em; margin-bottom:1em; }
td center { margin-top:0em; margin-bottom:0em; }
.Canvas { position:relative; }
li p.indent { text-indent: 0em }
.enumerate1 {list-style-type:decimal;}
.enumerate2 {list-style-type:lower-alpha;}
.enumerate3 {list-style-type:lower-roman;}
.enumerate4 {list-style-type:upper-alpha;}
div.newtheorem { margin-bottom: 2em; margin-top: 2em;}
.obeylines-h,.obeylines-v {white-space: nowrap; }
div.obeylines-v p { margin-top:0; margin-bottom:0; }
.overline{ text-decoration:overline; }
.overline img{ border-top: 1px solid black; }
td.displaylines {text-align:center; white-space:nowrap;}
.centerline {text-align:center;}
.rightline {text-align:right;}
div.verbatim {font-family: monospace; white-space: nowrap; text-align:left; clear:both; }
.fbox {padding-left:3.0pt; padding-right:3.0pt; text-indent:0pt; border:solid black 0.4pt; }
div.fbox {display:table}
div.center div.fbox {text-align:center; clear:both; padding-left:3.0pt; padding-right:3.0pt; text-indent:0pt; border:solid black 0.4pt; }
div.minipage{width:100%;}
div.center, div.center div.center {text-align: center; margin-left:1em; margin-right:1em;}
div.center div {text-align: left;}
div.flushright, div.flushright div.flushright {text-align: right;}
div.flushright div {text-align: left;}
div.flushleft {text-align: left;}
.underline{ text-decoration:underline; }
.underline img{ border-bottom: 1px solid black; margin-bottom:1pt; }
.framebox-c, .framebox-l, .framebox-r { padding-left:3.0pt; padding-right:3.0pt; text-indent:0pt; border:solid black 0.4pt; }
.framebox-c {text-align:center;}
.framebox-l {text-align:left;}
.framebox-r {text-align:right;}
span.thank-mark{ vertical-align: super }
span.footnote-mark sup.textsuperscript, span.footnote-mark a sup.textsuperscript{ font-size:80%; }
div.tabular, div.center div.tabular {text-align: center; margin-top:0.5em; margin-bottom:0.5em; }
table.tabular td p{margin-top:0em;}
table.tabular {margin-left: auto; margin-right: auto;}
div.td00{ margin-left:0pt; margin-right:0pt; }
div.td01{ margin-left:0pt; margin-right:5pt; }
div.td10{ margin-left:5pt; margin-right:0pt; }
div.td11{ margin-left:5pt; margin-right:5pt; }
table[rules] {border-left:solid black 0.4pt; border-right:solid black 0.4pt; }
td.td00{ padding-left:0pt; padding-right:0pt; }
td.td01{ padding-left:0pt; padding-right:5pt; }
td.td10{ padding-left:5pt; padding-right:0pt; }
td.td11{ padding-left:5pt; padding-right:5pt; }
table[rules] {border-left:solid black 0.4pt; border-right:solid black 0.4pt; }
.hline hr, .cline hr{ height : 1px; margin:0px; }
.tabbing-right {text-align:right;}
span.TEX {letter-spacing: -0.125em; }
span.TEX span.E{ position:relative;top:0.5ex;left:-0.0417em;}
a span.TEX span.E {text-decoration: none; }
span.LATEX span.A{ position:relative; top:-0.5ex; left:-0.4em; font-size:85%;}
span.LATEX span.TEX{ position:relative; left: -0.4em; }
div.float img, div.float .caption {text-align:center;}
div.figure img, div.figure .caption {text-align:center;}
.marginpar {width:20%; float:right; text-align:left; margin-left:auto; margin-top:0.5em; font-size:85%; text-decoration:underline;}
.marginpar p{margin-top:0.4em; margin-bottom:0.4em;}
.equation td{text-align:center; vertical-align:middle; }
td.eq-no{ width:5%; }
table.equation { width:100%; } 
div.math-display, div.par-math-display{text-align:center;}
math .texttt { font-family: monospace; }
math .textit { font-style: italic; }
math .textsl { font-style: oblique; }
math .textsf { font-family: sans-serif; }
math .textbf { font-weight: bold; }
.partToc a, .partToc, .likepartToc a, .likepartToc {line-height: 200%; font-weight:bold; font-size:110%;}
.chapterToc a, .chapterToc, .likechapterToc a, .likechapterToc, .appendixToc a, .appendixToc {line-height: 200%; font-weight:bold;}
.index-item, .index-subitem, .index-subsubitem {display:block}
.caption td.id{font-weight: bold; white-space: nowrap; }
table.caption {text-align:center;}
h1.partHead{text-align: center}
p.bibitem { text-indent: -2em; margin-left: 2em; margin-top:0.6em; margin-bottom:0.6em; }
p.bibitem-p { text-indent: 0em; margin-left: 2em; margin-top:0.6em; margin-bottom:0.6em; }
.paragraphHead, .likeparagraphHead { margin-top:2em; font-weight: bold;}
.subparagraphHead, .likesubparagraphHead { font-weight: bold;}
.quote {margin-bottom:0.25em; margin-top:0.25em; margin-left:1em; margin-right:1em; text-align:justify;}
.verse{white-space:nowrap; margin-left:2em}
div.maketitle {text-align:center;}
h2.titleHead{text-align:center;}
div.maketitle{ margin-bottom: 2em; }
div.author, div.date {text-align:center;}
div.thanks{text-align:left; margin-left:10%; font-size:85%; font-style:italic; }
div.author{white-space: nowrap;}
.quotation {margin-bottom:0.25em; margin-top:0.25em; margin-left:1em; }
h1.partHead{text-align: center}
.sectionToc, .likesectionToc {margin-left:2em;}
.subsectionToc, .likesubsectionToc {margin-left:4em;}
.subsubsectionToc, .likesubsubsectionToc {margin-left:6em;}
.frenchb-nbsp{font-size:75%;}
.frenchb-thinspace{font-size:75%;}
.figure img.graphics {margin-left:10%;}
/* end css.sty */

\title{Endomorphismes d'un espace hermitien}
\author{}
\date{}

\begin{document}
\maketitle

\textbf{Warning: \href{http://www.math.union.edu/locate/jsMath}{jsMath}
requires JavaScript to process the mathematics on this page.\\ If your
browser supports JavaScript, be sure it is enabled.}

\begin{center}\rule{3in}{0.4pt}\end{center}

{[}\href{coursse75.html}{prev}{]}
{[}\href{coursse75.html\#tailcoursse75.html}{prev-tail}{]}
{[}\hyperref[tailcoursse76.html]{tail}{]}
{[}\href{coursch14.html\#coursse76.html}{up}{]}

\subsubsection{13.4 Endomorphismes d'un espace hermitien}

\paragraph{13.4.1 Notion d'adjoint}

Soit E un espace préhilbertien complexe

Définition~13.4.1 Soit E un espace préhilbertien complexe. Soit u,v ∈
L(E). On dit que u et v sont des endomorphismes adjoints si

\textbackslash{}mathop\{∀\}x,y ∈ E, (u(x)\textbackslash{}mathrel\{∣\}y)
= (x\textbackslash{}mathrel\{∣\}v(y))

Remarque~13.4.1 La symétrie hermitienne du produit scalaire montre
clairement que u et v jouent des rôles symétriques, donc que u est
adjoint de v si et seulement si~v est adjoint de u.

Théorème~13.4.1 Soit E un espace hermitien. Tout endomorphisme de E
admet un unique adjoint \{u\}\^{}\{∗\}. Si u ∈ L(E), ℰ une base de E, Ω
=\textbackslash{}mathop\{ \textbackslash{}mathrm\{Mat\}\} (φ,ℰ) et A
=\textbackslash{}mathop\{ \textbackslash{}mathrm\{Mat\}\} (u,ℰ), alors

\textbackslash{}mathop\{\textbackslash{}mathrm\{Mat\}\}
(\{u\}\^{}\{∗\},ℰ) = \{Ω\}\^{}\{−1\}\{A\}\^{}\{∗\}Ω

Démonstration Soit ℰ une base de E et Ω =\textbackslash{}mathop\{
\textbackslash{}mathrm\{Mat\}\} (φ,ℰ). Comme φ est non dégénérée, la
matrice Ω est inversible. Soit u,v ∈ L(E), A =\textbackslash{}mathop\{
\textbackslash{}mathrm\{Mat\}\} (u,ℰ) et B =\textbackslash{}mathop\{
\textbackslash{}mathrm\{Mat\}\} (v,ℰ). Si x,y ∈ E, on a
(u(x)\textbackslash{}mathrel\{∣\}y) = \{(AX)\}\^{}\{∗\}ΩY =
\{X\}\^{}\{∗\}\{A\}\^{}\{∗\}ΩY et (x\textbackslash{}mathrel\{∣\}v(y)) =
\{X\}\^{}\{∗\}ΩBY . L'unicité de la matrice de la forme sesquilinéaire
(x,y)\textbackslash{}mathrel\{↦\}(u(x)\textbackslash{}mathrel\{∣\}y)
montre que

\textbackslash{}begin\{eqnarray*\} \textbackslash{}mathop\{∀\}x,y ∈ E,
(u(x)\textbackslash{}mathrel\{∣\}y) =
(x\textbackslash{}mathrel\{∣\}v(y))\&\& \%\&
\textbackslash{}\textbackslash{} \& \textbackslash{}mathrel\{⇔\} \&
\{A\}\^{}\{∗\}Ω = ΩB \textbackslash{}mathrel\{⇔\} B =
\{Ω\}\^{}\{−1\}\{A\}\^{}\{∗\}Ω\%\& \textbackslash{}\textbackslash{}
\textbackslash{}end\{eqnarray*\}

ce qui montre à la fois l'existence et l'unicité de l'adjoint et la
formule voulue.

Proposition~13.4.2 Soit E un espace hermitien. L'application
u\textbackslash{}mathrel\{↦\}\{u\}\^{}\{∗\} est un endomorphisme
semi-linéaire involutif de L(E). Si u,v ∈ L(E), alors u ∘ v aussi et
\{(u ∘ v)\}\^{}\{∗\} = \{v\}\^{}\{∗\}∘ \{u\}\^{}\{∗\}. Si u ∈ L(E) est
inversible, alors \{u\}\^{}\{∗\} est inversible et
\{(\{u\}\^{}\{−1\})\}\^{}\{∗\} = \{(\{u\}\^{}\{∗\})\}\^{}\{−1\}.

Démonstration On a déjà vu que la relation u et v sont adjoints était
symétrique, donc si u ∈ L(E), \{u\}\^{}\{∗\} aussi et \{u\}\^{}\{∗∗\} =
u. Si u,v ∈ L(E), α,β ∈ ℂ, on a

\textbackslash{}begin\{eqnarray*\} ((αu +
βv)(x)\textbackslash{}mathrel\{∣\}y)\& =\& (αu(x) +
βv(x)\textbackslash{}mathrel\{∣\}y) \%\&
\textbackslash{}\textbackslash{} \& =\&
\textbackslash{}overline\{α\}(u(x)\textbackslash{}mathrel\{∣\}y) +
\textbackslash{}overline\{β\}(v(x)\textbackslash{}mathrel\{∣\}y) \%\&
\textbackslash{}\textbackslash{} \& =\&
\textbackslash{}overline\{α\}(x\textbackslash{}mathrel\{∣\}\{u\}\^{}\{∗\}(y))
+
\textbackslash{}overline\{β\}(x\textbackslash{}mathrel\{∣\}\{v\}\^{}\{∗\}(y))\%\&
\textbackslash{}\textbackslash{} \& =\&
(x\textbackslash{}mathrel\{∣\}(\textbackslash{}overline\{α\}\{u\}\^{}\{∗\}
+ \textbackslash{}overline\{β\}\{v\}\^{}\{∗\})(y)) \%\&
\textbackslash{}\textbackslash{} \textbackslash{}end\{eqnarray*\}

ce qui montre que \{(αu + βv)\}\^{}\{∗\} =
\textbackslash{}overline\{α\}\{u\}\^{}\{∗\} +
\textbackslash{}overline\{β\}\{v\}\^{}\{∗\} et donc la semilinéarité de
u\textbackslash{}mathrel\{↦\}\{u\}\^{}\{∗\}. Si u,v ∈ L(E), on a

(u ∘ v(x)\textbackslash{}mathrel\{∣\}y) =
(v(x)\textbackslash{}mathrel\{∣\}\{u\}\^{}\{∗\}(y)) =
(x\textbackslash{}mathrel\{∣\}\{v\}\^{}\{∗\}∘ \{u\}\^{}\{∗\}(y))

ce qui montre que u ∘ v admet \{v\}\^{}\{∗\}∘ \{u\}\^{}\{∗\} comme
adjoint.

Si u est inversible, on a \{u\}\^{}\{−1\} ∘ u =\{
\textbackslash{}mathrm\{Id\}\}\_\{E\} d'où \{(\{u\}\^{}\{−1\} ∘
u)\}\^{}\{∗\} =\{ \textbackslash{}mathrm\{Id\}\}\_\{E\}\^{}\{∗\}, soit
\{u\}\^{}\{∗\}∘ \{(\{u\}\^{}\{−1\})\}\^{}\{∗\} =\{
\textbackslash{}mathrm\{Id\}\}\_\{E\}. De même u ∘ \{u\}\^{}\{−1\} =\{
\textbackslash{}mathrm\{Id\}\}\_\{E\} donne par passage à l'adjoint
\{(\{u\}\^{}\{−1\})\}\^{}\{∗\}∘ \{u\}\^{}\{∗\} =\{
\textbackslash{}mathrm\{Id\}\}\_\{E\}. Ceci montre que \{u\}\^{}\{∗\}
est inversible et que \{(\{u\}\^{}\{−1\})\}\^{}\{∗\} =
\{(\{u\}\^{}\{∗\})\}\^{}\{−1\}

Proposition~13.4.3 Soit E un espace hermitien, u ∈ L(E). Alors

\begin{itemize}
\itemsep1pt\parskip0pt\parsep0pt
\item
  (i) \textbackslash{}mathop\{\textbackslash{}mathrm\{det\}\}
  \{u\}\^{}\{∗\} =
  \textbackslash{}overline\{\textbackslash{}mathop\{\textbackslash{}mathrm\{det\}\}
  u\},
  \textbackslash{}mathop\{\textbackslash{}mathrm\{tr\}\}\{u\}\^{}\{∗\} =
  \textbackslash{}overline\{\textbackslash{}mathop\{\textbackslash{}mathrm\{tr\}\}u\},
  \{χ\}\_\{\{u\}\^{}\{∗\}\} = \textbackslash{}overline\{\{χ\}\_\{u\}\}
\item
  (ii)
  \textbackslash{}mathop\{\textbackslash{}mathrm\{Ker\}\}\{u\}\^{}\{∗\}
  =
  \{(\textbackslash{}mathop\{\textbackslash{}mathrm\{Im\}\}u)\}\^{}\{⊥\},
  \textbackslash{}mathop\{\textbackslash{}mathrm\{Im\}\}\{u\}\^{}\{∗\} =
  \{(\textbackslash{}mathop\{\textbackslash{}mathrm\{Ker\}\}u)\}\^{}\{⊥\}
\item
  (iii)
  \textbackslash{}mathop\{\textbackslash{}mathrm\{Ker\}\}\{u\}\^{}\{∗\}u
  =\textbackslash{}mathop\{ \textbackslash{}mathrm\{Ker\}\}u et
  \textbackslash{}mathop\{\textbackslash{}mathrm\{Im\}\}\{u\}\^{}\{∗\}u
  =\textbackslash{}mathop\{ \textbackslash{}mathrm\{Im\}\}\{u\}\^{}\{∗\}
\end{itemize}

Démonstration (i) Soit ℰ une base de E, Ω =\textbackslash{}mathop\{
\textbackslash{}mathrm\{Mat\}\} (φ,ℰ) et A =\textbackslash{}mathop\{
\textbackslash{}mathrm\{Mat\}\} (u,ℰ), alors
\textbackslash{}mathop\{\textbackslash{}mathrm\{Mat\}\}
(\{u\}\^{}\{∗\},ℰ) = \{Ω\}\^{}\{−1\}\{A\}\^{}\{∗\}Ω. On a donc
\textbackslash{}mathop\{\textbackslash{}mathrm\{det\}\} \{u\}\^{}\{∗\}
=\textbackslash{}mathop\{ \textbackslash{}mathrm\{det\}\}
\{Ω\}\^{}\{−1\}\{A\}\^{}\{∗\}Ω =\textbackslash{}mathop\{
\textbackslash{}mathrm\{det\}\} \{A\}\^{}\{∗\} =
\textbackslash{}overline\{\textbackslash{}mathop\{\textbackslash{}mathrm\{det\}\}
A\} =
\textbackslash{}overline\{\textbackslash{}mathop\{\textbackslash{}mathrm\{det\}\}
u\}. La démonstration est la même pour la trace et pour le polynôme
caractéristique.

(ii) On a

\textbackslash{}begin\{eqnarray*\} x
∈\textbackslash{}mathop\{\textbackslash{}mathrm\{Ker\}\}\{u\}\^{}\{∗\}\&
\textbackslash{}mathrel\{⇔\} \& \{u\}\^{}\{∗\}(x) = 0
\textbackslash{}mathrel\{⇔\} \textbackslash{}mathop\{∀\}y ∈ E,
(\{u\}\^{}\{∗\}(x)\textbackslash{}mathrel\{∣\}y) = 0 \%\&
\textbackslash{}\textbackslash{} \& \textbackslash{}mathrel\{⇔\} \&
\textbackslash{}mathop\{∀\}y ∈ E, (x\textbackslash{}mathrel\{∣\}u(y)) =
0 \textbackslash{}mathrel\{⇔\} x ∈
\{(\textbackslash{}mathop\{\textbackslash{}mathrm\{Im\}\}u)\}\^{}\{⊥\}\%\&
\textbackslash{}\textbackslash{} \textbackslash{}end\{eqnarray*\}

En appliquant ce résultat à \{u\}\^{}\{∗\} on obtient,
\textbackslash{}mathop\{\textbackslash{}mathrm\{Ker\}\}u =
\{(\textbackslash{}mathop\{\textbackslash{}mathrm\{Im\}\}\{u\}\^{}\{∗\})\}\^{}\{⊥\}
et en prenant l'orthogonal,
\textbackslash{}mathop\{\textbackslash{}mathrm\{Im\}\}\{u\}\^{}\{∗\} =
\{(\textbackslash{}mathop\{\textbackslash{}mathrm\{Ker\}\}u)\}\^{}\{⊥\}

(iii) On a visiblement u(x) = 0 ⇒ \{u\}\^{}\{∗\}u(x) = 0, donc
\textbackslash{}mathop\{\textbackslash{}mathrm\{Ker\}\}u
⊂\textbackslash{}mathop\{\textbackslash{}mathrm\{Ker\}\}\{u\}\^{}\{∗\}u~;
mais d'autre part, si x
∈\textbackslash{}mathop\{\textbackslash{}mathrm\{Ker\}\}\{u\}\^{}\{∗\}u,
on a

\textbackslash{}\textbar{}u\{(x)\textbackslash{}\textbar{}\}\^{}\{2\} =
(u(x)\textbackslash{}mathrel\{∣\}u(x)) =
(\{u\}\^{}\{∗\}u(x)\textbackslash{}mathrel\{∣\}x) =
(0\textbackslash{}mathrel\{∣\}x) = 0

et donc u(x) = 0, soit
\textbackslash{}mathop\{\textbackslash{}mathrm\{Ker\}\}\{u\}\^{}\{∗\}u
⊂\textbackslash{}mathop\{\textbackslash{}mathrm\{Ker\}\}u et l'égalité.
On en déduit alors que

\textbackslash{}mathop\{\textbackslash{}mathrm\{Im\}\}\{u\}\^{}\{∗\}u =
\{(\textbackslash{}mathop\{\textbackslash{}mathrm\{Ker\}\}\{(\{u\}\^{}\{∗\}u)\}\^{}\{∗\})\}\^{}\{⊥\}
=
\{(\textbackslash{}mathop\{\textbackslash{}mathrm\{Ker\}\}\{u\}\^{}\{∗\}u)\}\^{}\{⊥\}
=
\{(\textbackslash{}mathop\{\textbackslash{}mathrm\{Ker\}\}u)\}\^{}\{⊥\}
=\textbackslash{}mathop\{ \textbackslash{}mathrm\{Im\}\}\{u\}\^{}\{∗\}

Une des propriétés essentielles de l'adjoint que nous utiliserons de
fa\textbackslash{}c\{c\}on systématique pour la réduction des
endomorphismes est la suivante

Théorème~13.4.4 Soit u ∈ L(E). Soit F un sous-espace de E stable par u~;
alors \{F\}\^{}\{⊥\} est stable par \{u\}\^{}\{∗\}.

Démonstration Soit x ∈ \{F\}\^{}\{⊥\}. Si y ∈ F, on a
φ(\{u\}\^{}\{∗\}(x),y) = φ(x,u(y)) = 0 puisque u(y) ∈ F et x ∈
\{F\}\^{}\{⊥\}. Donc \{u\}\^{}\{∗\}(x) ∈ \{F\}\^{}\{⊥\} et
\{F\}\^{}\{⊥\} est stable par \{u\}\^{}\{∗\}.

\paragraph{13.4.2 Endomorphismes hermitiens}

Définition~13.4.2 Soit E un espace hermitien, u ∈ L(E). On dit que u est
hermitien (ou autoadjoint) s'il vérifie les conditions équivalentes

\begin{itemize}
\itemsep1pt\parskip0pt\parsep0pt
\item
  (i) \{u\}\^{}\{∗\} = u
\item
  (ii) \textbackslash{}mathop\{∀\}x,y ∈ E,
  (u(x)\textbackslash{}mathrel\{∣\}y) =
  (x\textbackslash{}mathrel\{∣\}u(y))
\end{itemize}

Remarque~13.4.2 Si la base ℰ est orthonormée, alors
\textbackslash{}mathop\{\textbackslash{}mathrm\{Mat\}\} ((
\textbackslash{}mathrel\{∣\} ),ℰ) = \{I\}\_\{n\} et
\textbackslash{}mathop\{\textbackslash{}mathrm\{Mat\}\}
(\{u\}\^{}\{∗\},ℰ) =\textbackslash{}mathop\{
\textbackslash{}mathrm\{Mat\}\} \{(u,ℰ)\}\^{}\{∗\}~; en particulier

Théorème~13.4.5 Soit ℰ une base orthonormée de E~; alors u est hermitien
si et seulement
si~\textbackslash{}mathop\{\textbackslash{}mathrm\{Mat\}\} (u,ℰ) est une
matrice hermitienne.

Proposition~13.4.6 L'ensemble H(E) des endomorphismes hermitiens est un
ℝ-sous-espace vectoriel de L(E) (mais pas un ℂ sous-espace vectoriel).
On a L(E) = H(E) ⊕ iH(E)

Démonstration L'endomorphisme de \{L\}\^{}\{∗\}(E),
u\textbackslash{}mathrel\{↦\}\{u\}\^{}\{∗\} étant ℝ linéaire et
involutif, l'espace L(E) est somme directe du sous-espace propre associé
à la valeur propre 1 (les endomorphismes hermitiens) et du sous-espace
propre associé à la valeur propre -1 (les endomorphismes antihermitiens,
qui ne sont autre que les endomorphismes hermitiens multipliés par i).

\paragraph{13.4.3 Groupe unitaire}

Soit E un espace hermitien

Définition~13.4.3 On dit que u ∈ L(E) est un endomorphisme unitaire si
on a les propriétés équivalentes

\begin{itemize}
\itemsep1pt\parskip0pt\parsep0pt
\item
  (i) \textbackslash{}mathop\{∀\}x ∈ E,
  \textbackslash{}\textbar{}u(x)\textbackslash{}\textbar{}
  =\textbackslash{}\textbar{} x\textbackslash{}\textbar{}
\item
  (ii) \textbackslash{}mathop\{∀\}x,y ∈ E,
  (u(x)\textbackslash{}mathrel\{∣\}u(y)) =
  (x\textbackslash{}mathrel\{∣\}y)
\item
  (iii) u est inversible et \{u\}\^{}\{−1\} = \{u\}\^{}\{∗\}
\item
  (iv) u ∘ \{u\}\^{}\{∗\} =\{ \textbackslash{}mathrm\{Id\}\}\_\{E\}
\item
  (v) \{u\}\^{}\{∗\}∘ u =\{ \textbackslash{}mathrm\{Id\}\}\_\{E\}
\end{itemize}

Démonstration (ii) ⇒(i) est évident (faire y = x). (i) ⇒(ii) provient de
l'identité de polarisation et de la linéarité de u. Pour un
endomorphisme d'un espace vectoriel de dimension finie, on sait que
l'inversibilité est équivalente à l'inversibilité à gauche ou à droite.
On a donc (iii) \textbackslash{}mathrel\{⇔\} (iv)
\textbackslash{}mathrel\{⇔\} (v). Supposons (ii) vérifié. Alors φ(x,y) =
φ(u(x),u(y)) = φ(x,\{u\}\^{}\{∗\}∘ u(y)), ce qui montre (puisque φ est
non dégénérée) que \{u\}\^{}\{∗\}∘ u =\{
\textbackslash{}mathrm\{Id\}\}\_\{E\}~; donc (ii) ⇒(v). De même (v)
⇒(ii) puisque φ(u(x),u(y)) = φ(x,\{u\}\^{}\{∗\}∘ u(y)).

Théorème~13.4.7 L'ensemble U(E) des endomorphismes unitaires de E est un
sous-groupe de (GL(E),∘). Pour tout endomorphisme unitaire u de E, on a
\textbar{}\textbackslash{}mathop\{\textbackslash{}mathrm\{det\}\}
u\textbar{} = 1. L'ensemble SU(E) des endomorphismes unitaires de
déterminant 1 est un sous-groupe distingué de U(E).

Démonstration On a clairement \{\textbackslash{}mathrm\{Id\}\}\_\{E\} ∈
U(E). La définition (i) montre évidemment que si u et v sont unitaires,
il en est de même de u ∘ v. De plus, soit u ∈ U(E)~; on a
\textbackslash{}\textbar{}\{u\}\^{}\{−1\}(x)\textbackslash{}\textbar{}
=\textbackslash{}\textbar{}
u(\{u\}\^{}\{−1\}(x))\textbackslash{}\textbar{}
=\textbackslash{}\textbar{} x\textbackslash{}\textbar{} ce qui montre
que \{u\}\^{}\{−1\} ∈ U(E). Donc U(E) est un sous-groupe de (GL(E),∘).
On a alors 1 =\textbackslash{}mathop\{ \textbackslash{}mathrm\{det\}\}
\{\textbackslash{}mathrm\{Id\}\}\_\{E\} =\textbackslash{}mathop\{
\textbackslash{}mathrm\{det\}\} (\{u\}\^{}\{∗\}∘ u)
=\textbackslash{}mathop\{ \textbackslash{}mathrm\{det\}\}
\{u\}\^{}\{∗\}\textbackslash{}mathop\{\textbackslash{}mathrm\{det\}\} u
= \textbar{}\textbackslash{}mathop\{\textbackslash{}mathrm\{det\}\}
u\{\textbar{}\}\^{}\{2\}, soit
\textbar{}\textbackslash{}mathop\{\textbackslash{}mathrm\{det\}\}
u\textbar{} = 1. L'application de U(E) dans le groupe multiplicatif des
nombres complexes de module 1,
u\textbackslash{}mathrel\{↦\}\textbackslash{}mathop\{\textbackslash{}mathrm\{det\}\}
u est un morphisme de groupes~; son noyau SU(E) est donc un sous groupe
distingué.

Théorème~13.4.8 Soit u ∈ L(E).

\begin{itemize}
\itemsep1pt\parskip0pt\parsep0pt
\item
  (i) Si u est unitaire, il envoie toute base orthonormée sur une base
  orthonormée.
\item
  (ii) Inversement, s'il existe une base orthonormée ℰ de E telle que
  u(ℰ) soit encore orthonormée, alors u est un endomorphisme unitaire.
\end{itemize}

Démonstration (i) On a
(u(\{e\}\_\{i\})\textbackslash{}mathrel\{∣\}u(\{e\}\_\{j\})) =
(\{e\}\_\{i\}\textbackslash{}mathrel\{∣\}\{e\}\_\{j\}) =
\{δ\}\_\{i\}\^{}\{j\}.

(ii) Soit x =\textbackslash{}mathop\{ \textbackslash{}mathop\{∑ \}\}
\{x\}\_\{i\}\{e\}\_\{i\} ∈ E. On a
\textbackslash{}\textbar{}\{x\textbackslash{}\textbar{}\}\^{}\{2\}
=\textbackslash{}mathop\{ \textbackslash{}mathop\{∑ \}\}
\textbar{}\{x\}\_\{i\}\{\textbar{}\}\^{}\{2\}. Mais on a aussi u(x)
=\textbackslash{}mathop\{ \textbackslash{}mathop\{∑ \}\}
\{x\}\_\{i\}u(\{e\}\_\{i\}) et comme u(ℰ) est orthonormée,
\textbackslash{}\textbar{}u\{(x)\textbackslash{}\textbar{}\}\^{}\{2\}
=\textbackslash{}mathop\{ \textbackslash{}mathop\{∑ \}\}
\textbar{}\{x\}\_\{i\}\{\textbar{}\}\^{}\{2\}~; on a donc
\textbackslash{}mathop\{∀\}x ∈ E,
\textbackslash{}\textbar{}u(x)\textbackslash{}\textbar{}
=\textbackslash{}\textbar{} x\textbackslash{}\textbar{}.

Théorème~13.4.9 Soit u un endomorphisme unitaire et F un sous-espace de
E stable par u. Alors \{F\}\^{}\{⊥\} est stable par u.

Démonstration On a u(F) ⊂ F et comme u est inversible, on a
\textbackslash{}mathop\{dim\} u(F) =\textbackslash{}mathop\{ dim\} F. On
a donc u(F) = F. Soit donc x ∈ \{F\}\^{}\{⊥\} et y ∈ F~; il existe z ∈ F
tel que u(z) = y, d'où (u(x)\textbackslash{}mathrel\{∣\}y) =
(u(x)\textbackslash{}mathrel\{∣\}u(z)) =
(x\textbackslash{}mathrel\{∣\}z) = 0, et donc u(x) ∈ \{F\}\^{}\{⊥\}.

\paragraph{13.4.4 Matrices unitaires}

Proposition~13.4.10 Soit E un espace hermitien. Soit u ∈ L(E), ℰ une
base de E, Ω =\textbackslash{}mathop\{ \textbackslash{}mathrm\{Mat\}\}
(( \textbackslash{}mathrel\{∣\} ),ℰ) et A =\textbackslash{}mathop\{
\textbackslash{}mathrm\{Mat\}\} (u,ℰ). Alors u est un endomorphisme
unitaire si et seulement si~\{A\}\^{}\{∗\}ΩA = Ω.

Démonstration On a φ(u(x),u(y)) = \{(AX)\}\^{}\{∗\}Ω(AY ) =
\{X\}\^{}\{∗\}\{A\}\^{}\{∗\}ΩAY . L'unicité de la matrice d'une forme
bilinéaire montre que

\textbackslash{}mathop\{∀\}x,y ∈ E,
(u(x)\textbackslash{}mathrel\{∣\}u(y)) =
(x\textbackslash{}mathrel\{∣\}y) \textbackslash{}mathrel\{⇔\}
\{A\}\^{}\{∗\}ΩA = Ω

En particulier, si ℰ est une base orthonormée de E, u est un
endomorphisme unitaire si et seulement si~\{A\}\^{}\{∗\}A =
\{I\}\_\{n\}. Ceci conduit à la définition suivante

Définition~13.4.4 Soit A ∈ \{M\}\_\{ℂ\}(n). On dit que A est une matrice
unitaire si elle vérifie les conditions équivalentes

\begin{itemize}
\itemsep1pt\parskip0pt\parsep0pt
\item
  (i) A est inversible et \{A\}\^{}\{−1\} = \{A\}\^{}\{∗\}
\item
  (ii) \{A\}\^{}\{∗\}A = \{I\}\_\{n\}
\item
  (iii) A\{A\}\^{}\{∗\} = \{I\}\_\{n\}
\end{itemize}

Théorème~13.4.11 L'ensemble U(n) des matrices carrées unitaires d'ordre
n est un sous-groupe de (G\{L\}\_\{ℂ\}(n),.). Pour toute matrice
unitaire A, on a
\textbar{}\textbackslash{}mathop\{\textbackslash{}mathrm\{det\}\}
A\textbar{} = 1. L'ensemble SU(n) des matrices unitaires de déterminant
1 est un sous-groupe distingué de U(n) .

Démonstration On a clairement \{I\}\_\{n\} ∈ U(n). La définition (i)
montre évidemment que si A et B sont unitaires, il en est de même de AB.
De plus, soit A ∈ U(n)~; on a
\{A\}\^{}\{−1\}\{(\{A\}\^{}\{−1\})\}\^{}\{∗\} =
\{A\}\^{}\{−1\}\{(\{A\}\^{}\{∗\})\}\^{}\{∗\} = \{A\}\^{}\{−1\}A =
\{I\}\_\{n\} ce qui montre que \{A\}\^{}\{−1\} ∈ U(n). Donc U(n) est un
sous-groupe de (G\{L\}\_\{ℂ\}(n),.). On a alors 1
=\textbackslash{}mathop\{ \textbackslash{}mathrm\{det\}\} \{I\}\_\{n\}
=\textbackslash{}mathop\{ \textbackslash{}mathrm\{det\}\}
(\{A\}\^{}\{∗\}A) =
\textbar{}\textbackslash{}mathop\{\textbackslash{}mathrm\{det\}\}
A\{\textbar{}\}\^{}\{2\}, soit
\textbar{}\textbackslash{}mathop\{\textbackslash{}mathrm\{det\}\}
A\textbar{} = 1. L'application de U(n) dans le groupe multiplicatif des
nombres complexes de module 1,
A\textbackslash{}mathrel\{↦\}\textbackslash{}mathop\{\textbackslash{}mathrm\{det\}\}
A est un morphisme de groupes multiplicatifs~; son noyau SU(n) est donc
un sous-groupe distingué.

Dans ce paragraphe, on munira \{ℂ\}\^{}\{n\} de la forme sesquilinéaire
hermitienne naturelle (qui rend la base canonique orthonormée),
c'est-à-dire que l'on posera (x\textbackslash{}mathrel\{∣\}y)
=\{\textbackslash{}mathop\{ \textbackslash{}mathop\{∑ \}\}
\}\_\{i=1\}\^{}\{n\}\textbackslash{}overline\{\{x\}\_\{i\}\}\{y\}\_\{i\}

Théorème~13.4.12 Une matrice A ∈ \{M\}\_\{ℂ\}(n) est unitaire si et
seulement si~ses vecteurs colonnes (resp. lignes) forment une base
orthonormée de \{ℂ\}\^{}\{n\}.

Démonstration Soit
(\{c\}\_\{1\},\textbackslash{}mathop\{\textbackslash{}mathop\{\ldots{}\}\},\{c\}\_\{n\})
les vecteurs colonnes de A,
(\{l\}\_\{1\},\textbackslash{}mathop\{\textbackslash{}mathop\{\ldots{}\}\},\{l\}\_\{n\})
ses vecteurs lignes. On a

\textbackslash{}begin\{eqnarray*\} A ∈ U(n)\&
\textbackslash{}mathrel\{⇔\} \& \{A\}\^{}\{∗\}A = \{I\}\_\{ n\}
\textbackslash{}mathrel\{⇔\} \textbackslash{}mathop\{∀\}i,j,
\{(\{A\}\^{}\{∗\}A)\}\_\{ i,j\} = \{δ\}\_\{i\}\^{}\{j\} \%\&
\textbackslash{}\textbackslash{} \& \textbackslash{}mathrel\{⇔\} \&
\textbackslash{}mathop\{∀\}i,j, \{\textbackslash{}mathop\{∑
\}\}\_\{k=1\}\^{}\{n\}\textbackslash{}overline\{\{a\}\_\{
k,i\}\}\{a\}\_\{k,j\} = \{δ\}\_\{i\}\^{}\{j\}
\textbackslash{}mathrel\{⇔\} \textbackslash{}mathop\{∀\}i,j, (\{c\}\_\{
i\}\textbackslash{}mathrel\{∣\}\{c\}\_\{j\}) = \{δ\}\_\{i\}\^{}\{j\}\%\&
\textbackslash{}\textbackslash{} \textbackslash{}end\{eqnarray*\}

De la même fa\textbackslash{}c\{c\}on, en traduisant la relation
A\{A\}\^{}\{∗\} = \{I\}\_\{n\}, on obtiendrait
(\{l\}\_\{i\}\textbackslash{}mathrel\{∣\}\{l\}\_\{j\}) =
\{δ\}\_\{i\}\^{}\{j\}.

Théorème~13.4.13 Soit E un espace hermitien. Soit ℰ une base orthonormée
de E, ℰ' une base de E. Alors on a équivalence de

\begin{itemize}
\itemsep1pt\parskip0pt\parsep0pt
\item
  (i) ℰ' est orthonormée
\item
  (ii) la matrice \{P\}\_\{ℰ\}\^{}\{ℰ'\} de passage de la base ℰ à la
  base ℰ' est unitaire.
\end{itemize}

Démonstration On sait que \{P\}\_\{ℰ\}\^{}\{ℰ'\}
=\textbackslash{}mathop\{ \textbackslash{}mathrm\{Mat\}\} (u,ℰ) où u est
l'endomorphisme de E défini par \textbackslash{}mathop\{∀\}i,
u(\{e\}\_\{i\}) = \{e\}\_\{i\}'. Or d'après les résultats du paragraphe
précédent, u est un endomorphisme unitaire si et seulement si~ℰ' est
orthonormée~; mais d'autre part, comme ℰ est orthonormée, u est unitaire
si et seulement
si~\textbackslash{}mathop\{\textbackslash{}mathrm\{Mat\}\} (u,ℰ) est une
matrice unitaire, d'où l'équivalence entre (i) et (ii).

\paragraph{13.4.5 Réduction des endomorphismes normaux}

Définition~13.4.5 Soit E un espace hermitien et u ∈ L(E). On dit que u
est un endomorphisme normal si

\{u\}\^{}\{∗\}u = u\{u\}\^{}\{∗\}

Lemme~13.4.14 Soit u un endomorphisme normal. Alors
\textbackslash{}mathop\{\textbackslash{}mathrm\{Ker\}\}\{u\}\^{}\{∗\}
=\textbackslash{}mathop\{ \textbackslash{}mathrm\{Ker\}\}u.

Démonstration On a

\textbackslash{}begin\{eqnarray*\} x
∈\textbackslash{}mathop\{\textbackslash{}mathrm\{Ker\}\}\{u\}\^{}\{∗\}\&
\textbackslash{}mathrel\{⇔\} \&
(\{u\}\^{}\{∗\}(x)\textbackslash{}mathrel\{∣\}\{u\}\^{}\{∗\}(x)) = 0
\textbackslash{}mathrel\{⇔\}
(u\{u\}\^{}\{∗\}(x)\textbackslash{}mathrel\{∣\}x) = 0\%\&
\textbackslash{}\textbackslash{} \& \textbackslash{}mathrel\{⇔\} \&
(\{u\}\^{}\{∗\}u(x)\textbackslash{}mathrel\{∣\}x) = 0
\textbackslash{}mathrel\{⇔\} (u(x)\textbackslash{}mathrel\{∣\}u(x)) = 0
\%\& \textbackslash{}\textbackslash{} \& \textbackslash{}mathrel\{⇔\} \&
x ∈\textbackslash{}mathop\{\textbackslash{}mathrm\{Ker\}\}u \%\&
\textbackslash{}\textbackslash{} \textbackslash{}end\{eqnarray*\}

Lemme~13.4.15 2. Soit u un endomorphisme normal. Alors, pour tout λ ∈ ℂ,
\textbackslash{}mathop\{\textbackslash{}mathrm\{Ker\}\}(\{u\}\^{}\{∗\}−\textbackslash{}overline\{λ\}\{\textbackslash{}mathrm\{Id\}\}\_\{E\})
=\textbackslash{}mathop\{ \textbackslash{}mathrm\{Ker\}\}(u −
λ\{\textbackslash{}mathrm\{Id\}\}\_\{E\}).

Démonstration Il suffit de remarquer que u −
λ\textbackslash{}mathrm\{Id\} est encore normal (élémentaire) et de lui
appliquer le lemme précédent en remarquant que
\{u\}\^{}\{∗\}−\textbackslash{}overline\{λ\}\{\textbackslash{}mathrm\{Id\}\}\_\{E\}
= \{(u − λ\{\textbackslash{}mathrm\{Id\}\}\_\{E\})\}\^{}\{∗\}

Théorème~13.4.16 Soit u un endomorphisme d'un espace hermitien. On a
équivalence de

\begin{itemize}
\itemsep1pt\parskip0pt\parsep0pt
\item
  (i) u est normal
\item
  (ii) u est diagonalisable dans une base orthonormée.
\end{itemize}

Démonstration (ii) ⇒(i) Soit ℰ une base orthonormée de diagonalisation
de u. Alors \textbackslash{}mathop\{\textbackslash{}mathrm\{Mat\}\}
(u,ℰ) =\textbackslash{}mathop\{
diag\}(\{λ\}\_\{1\},\textbackslash{}mathop\{\textbackslash{}mathop\{\ldots{}\}\},\{λ\}\_\{n\}).
Comme ℰ est orthonormée, on a
\textbackslash{}mathop\{\textbackslash{}mathrm\{Mat\}\}
(\{u\}\^{}\{∗\},ℰ) =\textbackslash{}mathop\{
\textbackslash{}mathrm\{Mat\}\} \{(u,ℰ)\}\^{}\{∗\}
=\textbackslash{}mathop\{
diag\}(\textbackslash{}overline\{\{λ\}\_\{1\}\},\textbackslash{}mathop\{\textbackslash{}mathop\{\ldots{}\}\},\textbackslash{}overline\{\{λ\}\_\{n\}\}).
Les deux matrices diagonales commutant, on a u\{u\}\^{}\{∗\} =
\{u\}\^{}\{∗\}u, donc u est normal.

(i) ⇒(ii) Montrons le résultat par récurrence sur
\textbackslash{}mathop\{dim\} E, le résultat étant évident si
\textbackslash{}mathop\{dim\} E = 1. Supposons que u est normal. Comme ℂ
est algébriquement clos, u admet une valeur propre λ. Comme
\textbackslash{}mathop\{\textbackslash{}mathrm\{Ker\}\}(\{u\}\^{}\{∗\}−\textbackslash{}overline\{λ\}\{\textbackslash{}mathrm\{Id\}\}\_\{E\})
=\textbackslash{}mathop\{ \textbackslash{}mathrm\{Ker\}\}(u −
λ\{\textbackslash{}mathrm\{Id\}\}\_\{E\}), \{E\}\_\{u\}(λ)
=\textbackslash{}mathop\{ \textbackslash{}mathrm\{Ker\}\}(u −
λ\{\textbackslash{}mathrm\{Id\}\}\_\{E\}) est stable par \{u\}\^{}\{∗\}
et donc \{E\}\_\{u\}\{(λ)\}\^{}\{⊥\} est stable par \{u\}\^{}\{∗∗\} = u.
Mais comme \{E\}\_\{u\}(λ) est stable par u, le sous-espace
\{E\}\_\{u\}\{(λ)\}\^{}\{⊥\} est stable par \{u\}\^{}\{∗\}. Soit v =
\{u\}\_\{\{\textbar{}\}\_\{\{ E\}\_\{u\}\{(λ)\}\^{}\{⊥\}\}\}. La
relation (v(x)\textbackslash{}mathrel\{∣\}y) =
(u(x)\textbackslash{}mathrel\{∣\}y) =
(x\textbackslash{}mathrel\{∣\}\{u\}\^{}\{∗\}(y)) pour x,y ∈
\{E\}\_\{u\}\{(λ)\}\^{}\{⊥\} montre que \{v\}\^{}\{∗\} =
\{u\}\_\{\{\textbar{}\}\_\{\{ E\}\_\{u\}\{(λ)\}\^{}\{⊥\}\}\}\^{}\{∗\},
donc \{v\}\^{}\{∗\}v = v\{v\}\^{}\{∗\} et donc v est un endomorphisme
normal de \{E\}\_\{u\}\{(λ)\}\^{}\{⊥\}. Par hypothèse de récurrence, il
existe une base orthonormée de \{E\}\_\{u\}\{(λ)\}\^{}\{⊥\} formée de
vecteurs propres de v donc de u. Comme E = \{E\}\_\{u\}(λ) ⊥ ⊕
\{E\}\_\{u\}\{(λ)\}\^{}\{⊥\}, si on réunit cette base avec une base
orthonormée de \{E\}\_\{u\}(λ), on obtient une base orthonormée de E
formée évidemment de vecteurs propres de u, ce qui achève la
démonstration.

Remarque~13.4.3 Soit ℰ une telle base. Alors
\textbackslash{}mathop\{\textbackslash{}mathrm\{Mat\}\} (u,ℰ)
=\textbackslash{}mathop\{
diag\}(\{λ\}\_\{1\},\textbackslash{}mathop\{\textbackslash{}mathop\{\ldots{}\}\},\{λ\}\_\{n\}).
L'endomorphisme u est hermitien si et seulement si~sa matrice dans la
base orthonormée ℰ est hermitienne, c'est-à-dire si et seulement
si~\textbackslash{}mathop\{∀\}i, \{λ\}\_\{i\} ∈ ℝ~; de même u est
unitaire si et seulement si~sa matrice dans la base orthonormée ℰ est
unitaire, c'est-à-dire si et seulement si~\textbackslash{}mathop\{∀\}i,
\textbar{}\{λ\}\_\{i\}\textbar{} = 1. Comme il est clair que tout
endomorphisme hermitien ou unitaire est normal on obtient les deux
corollaires

Corollaire~13.4.17 Soit u un endomorphisme d'un espace hermitien. On a
équivalence de

\begin{itemize}
\itemsep1pt\parskip0pt\parsep0pt
\item
  (i) u est hermitien
\item
  (ii) u est diagonalisable dans une base orthonormée et
  \textbackslash{}mathop\{\textbackslash{}mathrm\{Sp\}\}(u) ⊂ ℝ
\end{itemize}

Corollaire~13.4.18 Soit u un endomorphisme d'un espace hermitien. On a
équivalence de

\begin{itemize}
\itemsep1pt\parskip0pt\parsep0pt
\item
  (i) u est unitaire
\item
  (ii) u est diagonalisable dans une base orthonormée et
  \textbackslash{}mathop\{\textbackslash{}mathrm\{Sp\}\}(u) ⊂ U
  (ensemble des nombres complexes de module 1)
\end{itemize}

\paragraph{13.4.6 Réduction des matrices normales}

En traduisant le paragraphe précédent en terme de matrices (en utilisant
le produit hermitien canonique sur \{ℂ\}\^{}\{2\} défini par
(x\textbackslash{}mathrel\{∣\}y) =\{\textbackslash{}mathop\{
\textbackslash{}mathop\{∑ \}\}
\}\_\{i\}\textbackslash{}overline\{\{x\}\_\{i\}\}\{y\}\_\{i\}) on
obtient la définition et les résultats suivants.

Définition~13.4.6 Soit A ∈ \{M\}\_\{ℂ\}(n). On dit que A est une matrice
normale si

\{A\}\^{}\{∗\}A = A\{A\}\^{}\{∗\}

Théorème~13.4.19 Soit A ∈ \{M\}\_\{ℂ\}(n). On a équivalence de

\begin{itemize}
\itemsep1pt\parskip0pt\parsep0pt
\item
  (i) A est normal
\item
  (ii) Il existe P unitaire telle que \{P\}\^{}\{−1\}AP =
  \{P\}\^{}\{∗\}AP soit diagonale.
\end{itemize}

Corollaire~13.4.20 Soit A ∈ \{M\}\_\{ℂ\}(n). On a équivalence de

\begin{itemize}
\itemsep1pt\parskip0pt\parsep0pt
\item
  (i) A est hermitienne
\item
  (ii) Il existe P unitaire telle que \{P\}\^{}\{−1\}AP =
  \{P\}\^{}\{∗\}AP soit diagonale réelle
\end{itemize}

Corollaire~13.4.21 Soit A ∈ \{M\}\_\{ℂ\}(n). On a équivalence de

\begin{itemize}
\itemsep1pt\parskip0pt\parsep0pt
\item
  (i) A est unitaire
\item
  (ii) Il existe P unitaire telle que \{P\}\^{}\{−1\}AP =
  \{P\}\^{}\{∗\}AP soit diagonale à éléments diagonaux dans U (ensemble
  des nombres complexes de module 1)
\end{itemize}

{[}\href{coursse75.html}{prev}{]}
{[}\href{coursse75.html\#tailcoursse75.html}{prev-tail}{]}
{[}\href{coursse76.html}{front}{]}
{[}\href{coursch14.html\#coursse76.html}{up}{]}

\end{document}
