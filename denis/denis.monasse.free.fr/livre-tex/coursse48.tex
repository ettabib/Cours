\documentclass[]{article}
\usepackage[T1]{fontenc}
\usepackage{lmodern}
\usepackage{amssymb,amsmath}
\usepackage{ifxetex,ifluatex}
\usepackage{fixltx2e} % provides \textsubscript
% use upquote if available, for straight quotes in verbatim environments
\IfFileExists{upquote.sty}{\usepackage{upquote}}{}
\ifnum 0\ifxetex 1\fi\ifluatex 1\fi=0 % if pdftex
  \usepackage[utf8]{inputenc}
\else % if luatex or xelatex
  \ifxetex
    \usepackage{mathspec}
    \usepackage{xltxtra,xunicode}
  \else
    \usepackage{fontspec}
  \fi
  \defaultfontfeatures{Mapping=tex-text,Scale=MatchLowercase}
  \newcommand{\euro}{€}
\fi
% use microtype if available
\IfFileExists{microtype.sty}{\usepackage{microtype}}{}
\ifxetex
  \usepackage[setpagesize=false, % page size defined by xetex
              unicode=false, % unicode breaks when used with xetex
              xetex]{hyperref}
\else
  \usepackage[unicode=true]{hyperref}
\fi
\hypersetup{breaklinks=true,
            bookmarks=true,
            pdfauthor={},
            pdftitle={Fonctions classiques},
            colorlinks=true,
            citecolor=blue,
            urlcolor=blue,
            linkcolor=magenta,
            pdfborder={0 0 0}}
\urlstyle{same}  % don't use monospace font for urls
\setlength{\parindent}{0pt}
\setlength{\parskip}{6pt plus 2pt minus 1pt}
\setlength{\emergencystretch}{3em}  % prevent overfull lines
\setcounter{secnumdepth}{0}
 
/* start css.sty */
.cmr-5{font-size:50%;}
.cmr-7{font-size:70%;}
.cmmi-5{font-size:50%;font-style: italic;}
.cmmi-7{font-size:70%;font-style: italic;}
.cmmi-10{font-style: italic;}
.cmsy-5{font-size:50%;}
.cmsy-7{font-size:70%;}
.cmex-7{font-size:70%;}
.cmex-7x-x-71{font-size:49%;}
.msbm-7{font-size:70%;}
.cmtt-10{font-family: monospace;}
.cmti-10{ font-style: italic;}
.cmbx-10{ font-weight: bold;}
.cmr-17x-x-120{font-size:204%;}
.cmsl-10{font-style: oblique;}
.cmti-7x-x-71{font-size:49%; font-style: italic;}
.cmbxti-10{ font-weight: bold; font-style: italic;}
p.noindent { text-indent: 0em }
td p.noindent { text-indent: 0em; margin-top:0em; }
p.nopar { text-indent: 0em; }
p.indent{ text-indent: 1.5em }
@media print {div.crosslinks {visibility:hidden;}}
a img { border-top: 0; border-left: 0; border-right: 0; }
center { margin-top:1em; margin-bottom:1em; }
td center { margin-top:0em; margin-bottom:0em; }
.Canvas { position:relative; }
li p.indent { text-indent: 0em }
.enumerate1 {list-style-type:decimal;}
.enumerate2 {list-style-type:lower-alpha;}
.enumerate3 {list-style-type:lower-roman;}
.enumerate4 {list-style-type:upper-alpha;}
div.newtheorem { margin-bottom: 2em; margin-top: 2em;}
.obeylines-h,.obeylines-v {white-space: nowrap; }
div.obeylines-v p { margin-top:0; margin-bottom:0; }
.overline{ text-decoration:overline; }
.overline img{ border-top: 1px solid black; }
td.displaylines {text-align:center; white-space:nowrap;}
.centerline {text-align:center;}
.rightline {text-align:right;}
div.verbatim {font-family: monospace; white-space: nowrap; text-align:left; clear:both; }
.fbox {padding-left:3.0pt; padding-right:3.0pt; text-indent:0pt; border:solid black 0.4pt; }
div.fbox {display:table}
div.center div.fbox {text-align:center; clear:both; padding-left:3.0pt; padding-right:3.0pt; text-indent:0pt; border:solid black 0.4pt; }
div.minipage{width:100%;}
div.center, div.center div.center {text-align: center; margin-left:1em; margin-right:1em;}
div.center div {text-align: left;}
div.flushright, div.flushright div.flushright {text-align: right;}
div.flushright div {text-align: left;}
div.flushleft {text-align: left;}
.underline{ text-decoration:underline; }
.underline img{ border-bottom: 1px solid black; margin-bottom:1pt; }
.framebox-c, .framebox-l, .framebox-r { padding-left:3.0pt; padding-right:3.0pt; text-indent:0pt; border:solid black 0.4pt; }
.framebox-c {text-align:center;}
.framebox-l {text-align:left;}
.framebox-r {text-align:right;}
span.thank-mark{ vertical-align: super }
span.footnote-mark sup.textsuperscript, span.footnote-mark a sup.textsuperscript{ font-size:80%; }
div.tabular, div.center div.tabular {text-align: center; margin-top:0.5em; margin-bottom:0.5em; }
table.tabular td p{margin-top:0em;}
table.tabular {margin-left: auto; margin-right: auto;}
div.td00{ margin-left:0pt; margin-right:0pt; }
div.td01{ margin-left:0pt; margin-right:5pt; }
div.td10{ margin-left:5pt; margin-right:0pt; }
div.td11{ margin-left:5pt; margin-right:5pt; }
table[rules] {border-left:solid black 0.4pt; border-right:solid black 0.4pt; }
td.td00{ padding-left:0pt; padding-right:0pt; }
td.td01{ padding-left:0pt; padding-right:5pt; }
td.td10{ padding-left:5pt; padding-right:0pt; }
td.td11{ padding-left:5pt; padding-right:5pt; }
table[rules] {border-left:solid black 0.4pt; border-right:solid black 0.4pt; }
.hline hr, .cline hr{ height : 1px; margin:0px; }
.tabbing-right {text-align:right;}
span.TEX {letter-spacing: -0.125em; }
span.TEX span.E{ position:relative;top:0.5ex;left:-0.0417em;}
a span.TEX span.E {text-decoration: none; }
span.LATEX span.A{ position:relative; top:-0.5ex; left:-0.4em; font-size:85%;}
span.LATEX span.TEX{ position:relative; left: -0.4em; }
div.float img, div.float .caption {text-align:center;}
div.figure img, div.figure .caption {text-align:center;}
.marginpar {width:20%; float:right; text-align:left; margin-left:auto; margin-top:0.5em; font-size:85%; text-decoration:underline;}
.marginpar p{margin-top:0.4em; margin-bottom:0.4em;}
.equation td{text-align:center; vertical-align:middle; }
td.eq-no{ width:5%; }
table.equation { width:100%; } 
div.math-display, div.par-math-display{text-align:center;}
math .texttt { font-family: monospace; }
math .textit { font-style: italic; }
math .textsl { font-style: oblique; }
math .textsf { font-family: sans-serif; }
math .textbf { font-weight: bold; }
.partToc a, .partToc, .likepartToc a, .likepartToc {line-height: 200%; font-weight:bold; font-size:110%;}
.chapterToc a, .chapterToc, .likechapterToc a, .likechapterToc, .appendixToc a, .appendixToc {line-height: 200%; font-weight:bold;}
.index-item, .index-subitem, .index-subsubitem {display:block}
.caption td.id{font-weight: bold; white-space: nowrap; }
table.caption {text-align:center;}
h1.partHead{text-align: center}
p.bibitem { text-indent: -2em; margin-left: 2em; margin-top:0.6em; margin-bottom:0.6em; }
p.bibitem-p { text-indent: 0em; margin-left: 2em; margin-top:0.6em; margin-bottom:0.6em; }
.paragraphHead, .likeparagraphHead { margin-top:2em; font-weight: bold;}
.subparagraphHead, .likesubparagraphHead { font-weight: bold;}
.quote {margin-bottom:0.25em; margin-top:0.25em; margin-left:1em; margin-right:1em; text-align:\jmathustify;}
.verse{white-space:nowrap; margin-left:2em}
div.maketitle {text-align:center;}
h2.titleHead{text-align:center;}
div.maketitle{ margin-bottom: 2em; }
div.author, div.date {text-align:center;}
div.thanks{text-align:left; margin-left:10%; font-size:85%; font-style:italic; }
div.author{white-space: nowrap;}
.quotation {margin-bottom:0.25em; margin-top:0.25em; margin-left:1em; }
h1.partHead{text-align: center}
.sectionToc, .likesectionToc {margin-left:2em;}
.subsectionToc, .likesubsectionToc {margin-left:4em;}
.subsubsectionToc, .likesubsubsectionToc {margin-left:6em;}
.frenchb-nbsp{font-size:75%;}
.frenchb-thinspace{font-size:75%;}
.figure img.graphics {margin-left:10%;}
/* end css.sty */

\title{Fonctions classiques}
\author{}
\date{}

\begin{document}
\maketitle

\textbf{Warning: 
requires JavaScript to process the mathematics on this page.\\ If your
browser supports JavaScript, be sure it is enabled.}

\begin{center}\rule{3in}{0.4pt}\end{center}

{[}
{[}
{[}{]}
{[}

\subsubsection{8.5 Fonctions classiques}

\paragraph{8.5.1 Fonctions circulaires réciproques}

Le lecteur démontrera sans difficulté les résultats suivants qui
découlent immédiatement des caractérisations des homéomorphismes et des
difféomorphismes d'un intervalle sur un autre intervalle de \mathbb{R}~.

Théorème~8.5.1 (i)
x\mapsto~cos~ x est un
homéomorphisme décroissant de {[}0,\pi~{]} sur {[}-1,1{]}~;
l'homéomorphisme réciproque est noté arccos~ :
{[}-1,1{]} \rightarrow~ {[}0,\pi~{]}~; arccos~ est
C^\infty~ sur {]} - 1,1{[} et arccos~ '(x)
= - 1 \over \sqrt1-x^2
. (ii) x\mapsto~sin~ x est
un homéomorphisme croissant de {[}-\pi~\diagup2,\pi~\diagup2{]} sur {[}-1,1{]}~;
l'homéomorphisme réciproque est noté arcsin~ :
{[}-1,1{]} \rightarrow~ {[}-\pi~\diagup2,\pi~\diagup2{]}~; arcsin~ est
C^\infty~ sur {]} - 1,1{[} et arcsin~ '(x)
= 1 \over \sqrt1-x^2 .
(iii) x\mapsto~tan~ x est
un C^\infty~ difféomorphisme croissant de {]} - \pi~\diagup2,\pi~\diagup2{[} sur {]}
-\infty~,+\infty~{[}~; le difféomorphisme réciproque est noté
arctan~ :{]} -\infty~,+\infty~{[}\rightarrow~{]} - \pi~\diagup2,\pi~\diagup2{[} et
arctan~ '(x) = 1 \over
1+x^2 .

Remarque~8.5.1 Pour t \in {[}-1,1{]},

x = arccos t \mathrel\Leftrightarrow~ t
= cos x\text et ~x \in
{[}0,\pi~{]}

x = arcsin t \mathrel\Leftrightarrow~ t
= sin x\text et ~x \in
{[}-\pi~\diagup2,\pi~\diagup2{]}

Pour t \in \mathbb{R}~,

x = arctan t \mathrel\Leftrightarrow~ t
= tan x\text et ~x \in{]} -
\pi~\diagup2,\pi~\diagup2{[}

cos (\arccos~ t) = t,
sin (\arccos~ t) =
\sqrt1 - t^2,
tan (\arccos~ t) =
\\ldots~

cos (\arcsin~ t) =
\sqrt1 - t^2,
sin (\arcsin~ t) = t,
tan (\arcsin~ t) =
\\ldots~

cos (\arctan~ t) = 1
\over \sqrt1 + t^2 ,
sin (\arctan~ t) =
\\ldots~,
tan (\arctan~ t) = t

\paragraph{8.5.2 Fonctions hyperboliques directes}

\mathrmch~ x = 1
\over 2 (e^x + e^-x),
\mathrmsh~ x = 1
\over 2 (e^x - e^-x),
\mathrmth~ x =
\mathrmsh~ x
\over
\mathrmch x~ =
e^2x-1 \over e^2x+1

 \mathrmch~
`= \mathrmsh~ ,
\mathrmsh~'
= \mathrmch~ ,
\mathrmth~ ' = 1
-\mathrmth ^2~

 \mathrmch~ x
+ \mathrmsh~ x =
e^x, \mathrmch~ x
-\mathrmsh~ x =
e^-x

 \mathrmch ^2~x
-\mathrmsh ^2~x =
1

\mathrmch~ (a + b)
= \mathrmch~
a\mathrmch~ b
+ \mathrmsh~
a\mathrmsh~ b

\mathrmch~ (a - b)
= \mathrmch~
a\mathrmch~ b
-\mathrmsh~
a\mathrmsh~ b

\mathrmsh~ (a + b)
= \mathrmsh~
a\mathrmch~ b
+ \mathrmch~
a\mathrmsh~ b

\mathrmsh~ (a - b)
= \mathrmsh~
a\mathrmch~ b
-\mathrmch~
a\mathrmsh~ b

\mathrmch~ 2a =
2\mathrmch ^2~a -
1 = 1 + 2\mathrmsh~
^2a = \mathrmch~
^2a + \mathrmsh~
^2a

\mathrmsh~ 2a =
2\mathrmsh~
a\mathrmch~ a,
\mathrmth~ 2a =
2 \mathrmth~ a
\over
1+\mathrmth~
^2a

Si t = \mathrmth~ ( x
\over 2 ),

\mathrmch~ x = 1 +
t^2 \over 1 - t^2 ,
\mathrmsh~ x = 2t
\over 1 - t^2 ,
\mathrmth~ x = 2t
\over 1 + t^2

\paragraph{8.5.3 Fonctions hyperboliques réciproques}

Théorème~8.5.2 (i)
x\mapsto~\mathrmch~
x est un homéomorphisme croissant de {[}0,+\infty~{[} sur {[}1,+\infty~{[}~;
l'homéomorphisme réciproque est noté arg~
\mathrmch~ : {[}1,+\infty~{[}\rightarrow~
{[}0,+\infty~{[}~; arg~
\mathrmch~ est
C^\infty~ sur {]}1,+\infty~{[} et arg~
\mathrmch~ '(x) = 1
\over \sqrtx^2  -1 . (ii)
x\mapsto~\mathrmsh~
x est un C^\infty~ difféomorphisme croissant de {]} -\infty~,+\infty~{[} sur
{]} -\infty~,+\infty~{[}~; le difféomorphisme réciproque est noté
arg~
\mathrmsh~ :{]} -\infty~,+\infty~{[}\rightarrow~{]}
-\infty~,+\infty~{[} et on a arg~
\mathrmsh~ '(x) = 1
\over \sqrt1+x^2 . (iii)
x\mapsto~\mathrmth~
x est un C^\infty~ difféomorphisme croissant de {]} -\infty~,+\infty~{[} sur
{]} - 1,+1{[}~; le difféomorphisme réciproque est noté
arg~
\mathrmth~ :{]} - 1,1{[}\rightarrow~{]}
-\infty~,+\infty~{[} et on a arg~
\mathrmth~ '(x) = 1
\over 1-x^2 .

Pour t ≥ 1,\quad x = arg~
\mathrmch~ t
\Leftrightarrow t =\
\mathrmch x\text et x ≥ 0.

Pour t \in \mathbb{R}~,\quad x = arg~
\mathrmsh~ t
\Leftrightarrow t =\
\mathrmsh x.

Pour t \in{]} - 1,1{[}, \quad x =\
arg \mathrmth~ t
\Leftrightarrow t =\
\mathrmth x.

arg~
\mathrmch~ t
= log (t + \sqrtt~^2
 - 1),\quad arg~
\mathrmsh~ t
= log (t + \sqrtt~^2
 + 1),\quad arg~
\mathrmth~ t = 1
\over 2  log~ ( 1+t
\over 1-t )

{[}
{[}
{[}
{[}

\end{document}
