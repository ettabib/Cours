\documentclass[]{article}
\usepackage[T1]{fontenc}
\usepackage{lmodern}
\usepackage{amssymb,amsmath}
\usepackage{ifxetex,ifluatex}
\usepackage{fixltx2e} % provides \textsubscript
% use upquote if available, for straight quotes in verbatim environments
\IfFileExists{upquote.sty}{\usepackage{upquote}}{}
\ifnum 0\ifxetex 1\fi\ifluatex 1\fi=0 % if pdftex
  \usepackage[utf8]{inputenc}
\else % if luatex or xelatex
  \ifxetex
    \usepackage{mathspec}
    \usepackage{xltxtra,xunicode}
  \else
    \usepackage{fontspec}
  \fi
  \defaultfontfeatures{Mapping=tex-text,Scale=MatchLowercase}
  \newcommand{\euro}{€}
\fi
% use microtype if available
\IfFileExists{microtype.sty}{\usepackage{microtype}}{}
\ifxetex
  \usepackage[setpagesize=false, % page size defined by xetex
              unicode=false, % unicode breaks when used with xetex
              xetex]{hyperref}
\else
  \usepackage[unicode=true]{hyperref}
\fi
\hypersetup{breaklinks=true,
            bookmarks=true,
            pdfauthor={},
            pdftitle={7 Suites et series},
            colorlinks=true,
            citecolor=blue,
            urlcolor=blue,
            linkcolor=magenta,
            pdfborder={0 0 0}}
\urlstyle{same}  % don't use monospace font for urls
\setlength{\parindent}{0pt}
\setlength{\parskip}{6pt plus 2pt minus 1pt}
\setlength{\emergencystretch}{3em}  % prevent overfull lines
\setcounter{secnumdepth}{0}
 
/* start css.sty */
.cmr-5{font-size:50%;}
.cmr-7{font-size:70%;}
.cmmi-5{font-size:50%;font-style: italic;}
.cmmi-7{font-size:70%;font-style: italic;}
.cmmi-10{font-style: italic;}
.cmsy-5{font-size:50%;}
.cmsy-7{font-size:70%;}
.cmex-7{font-size:70%;}
.cmex-7x-x-71{font-size:49%;}
.msbm-7{font-size:70%;}
.cmtt-10{font-family: monospace;}
.cmti-10{ font-style: italic;}
.cmbx-10{ font-weight: bold;}
.cmr-17x-x-120{font-size:204%;}
.cmsl-10{font-style: oblique;}
.cmti-7x-x-71{font-size:49%; font-style: italic;}
.cmbxti-10{ font-weight: bold; font-style: italic;}
p.noindent { text-indent: 0em }
td p.noindent { text-indent: 0em; margin-top:0em; }
p.nopar { text-indent: 0em; }
p.indent{ text-indent: 1.5em }
@media print {div.crosslinks {visibility:hidden;}}
a img { border-top: 0; border-left: 0; border-right: 0; }
center { margin-top:1em; margin-bottom:1em; }
td center { margin-top:0em; margin-bottom:0em; }
.Canvas { position:relative; }
li p.indent { text-indent: 0em }
.enumerate1 {list-style-type:decimal;}
.enumerate2 {list-style-type:lower-alpha;}
.enumerate3 {list-style-type:lower-roman;}
.enumerate4 {list-style-type:upper-alpha;}
div.newtheorem { margin-bottom: 2em; margin-top: 2em;}
.obeylines-h,.obeylines-v {white-space: nowrap; }
div.obeylines-v p { margin-top:0; margin-bottom:0; }
.overline{ text-decoration:overline; }
.overline img{ border-top: 1px solid black; }
td.displaylines {text-align:center; white-space:nowrap;}
.centerline {text-align:center;}
.rightline {text-align:right;}
div.verbatim {font-family: monospace; white-space: nowrap; text-align:left; clear:both; }
.fbox {padding-left:3.0pt; padding-right:3.0pt; text-indent:0pt; border:solid black 0.4pt; }
div.fbox {display:table}
div.center div.fbox {text-align:center; clear:both; padding-left:3.0pt; padding-right:3.0pt; text-indent:0pt; border:solid black 0.4pt; }
div.minipage{width:100%;}
div.center, div.center div.center {text-align: center; margin-left:1em; margin-right:1em;}
div.center div {text-align: left;}
div.flushright, div.flushright div.flushright {text-align: right;}
div.flushright div {text-align: left;}
div.flushleft {text-align: left;}
.underline{ text-decoration:underline; }
.underline img{ border-bottom: 1px solid black; margin-bottom:1pt; }
.framebox-c, .framebox-l, .framebox-r { padding-left:3.0pt; padding-right:3.0pt; text-indent:0pt; border:solid black 0.4pt; }
.framebox-c {text-align:center;}
.framebox-l {text-align:left;}
.framebox-r {text-align:right;}
span.thank-mark{ vertical-align: super }
span.footnote-mark sup.textsuperscript, span.footnote-mark a sup.textsuperscript{ font-size:80%; }
div.tabular, div.center div.tabular {text-align: center; margin-top:0.5em; margin-bottom:0.5em; }
table.tabular td p{margin-top:0em;}
table.tabular {margin-left: auto; margin-right: auto;}
div.td00{ margin-left:0pt; margin-right:0pt; }
div.td01{ margin-left:0pt; margin-right:5pt; }
div.td10{ margin-left:5pt; margin-right:0pt; }
div.td11{ margin-left:5pt; margin-right:5pt; }
table[rules] {border-left:solid black 0.4pt; border-right:solid black 0.4pt; }
td.td00{ padding-left:0pt; padding-right:0pt; }
td.td01{ padding-left:0pt; padding-right:5pt; }
td.td10{ padding-left:5pt; padding-right:0pt; }
td.td11{ padding-left:5pt; padding-right:5pt; }
table[rules] {border-left:solid black 0.4pt; border-right:solid black 0.4pt; }
.hline hr, .cline hr{ height : 1px; margin:0px; }
.tabbing-right {text-align:right;}
span.TEX {letter-spacing: -0.125em; }
span.TEX span.E{ position:relative;top:0.5ex;left:-0.0417em;}
a span.TEX span.E {text-decoration: none; }
span.LATEX span.A{ position:relative; top:-0.5ex; left:-0.4em; font-size:85%;}
span.LATEX span.TEX{ position:relative; left: -0.4em; }
div.float img, div.float .caption {text-align:center;}
div.figure img, div.figure .caption {text-align:center;}
.marginpar {width:20%; float:right; text-align:left; margin-left:auto; margin-top:0.5em; font-size:85%; text-decoration:underline;}
.marginpar p{margin-top:0.4em; margin-bottom:0.4em;}
.equation td{text-align:center; vertical-align:middle; }
td.eq-no{ width:5%; }
table.equation { width:100%; } 
div.math-display, div.par-math-display{text-align:center;}
math .texttt { font-family: monospace; }
math .textit { font-style: italic; }
math .textsl { font-style: oblique; }
math .textsf { font-family: sans-serif; }
math .textbf { font-weight: bold; }
.partToc a, .partToc, .likepartToc a, .likepartToc {line-height: 200%; font-weight:bold; font-size:110%;}
.chapterToc a, .chapterToc, .likechapterToc a, .likechapterToc, .appendixToc a, .appendixToc {line-height: 200%; font-weight:bold;}
.index-item, .index-subitem, .index-subsubitem {display:block}
.caption td.id{font-weight: bold; white-space: nowrap; }
table.caption {text-align:center;}
h1.partHead{text-align: center}
p.bibitem { text-indent: -2em; margin-left: 2em; margin-top:0.6em; margin-bottom:0.6em; }
p.bibitem-p { text-indent: 0em; margin-left: 2em; margin-top:0.6em; margin-bottom:0.6em; }
.paragraphHead, .likeparagraphHead { margin-top:2em; font-weight: bold;}
.subparagraphHead, .likesubparagraphHead { font-weight: bold;}
.quote {margin-bottom:0.25em; margin-top:0.25em; margin-left:1em; margin-right:1em; text-align:justify;}
.verse{white-space:nowrap; margin-left:2em}
div.maketitle {text-align:center;}
h2.titleHead{text-align:center;}
div.maketitle{ margin-bottom: 2em; }
div.author, div.date {text-align:center;}
div.thanks{text-align:left; margin-left:10%; font-size:85%; font-style:italic; }
div.author{white-space: nowrap;}
.quotation {margin-bottom:0.25em; margin-top:0.25em; margin-left:1em; }
h1.partHead{text-align: center}
.sectionToc, .likesectionToc {margin-left:2em;}
.subsectionToc, .likesubsectionToc {margin-left:4em;}
.subsubsectionToc, .likesubsubsectionToc {margin-left:6em;}
.frenchb-nbsp{font-size:75%;}
.frenchb-thinspace{font-size:75%;}
.figure img.graphics {margin-left:10%;}
/* end css.sty */

\title{7 Suites et series}
\author{}
\date{}

\begin{document}
\maketitle

\textbf{Warning: \href{http://www.math.union.edu/locate/jsMath}{jsMath}
requires JavaScript to process the mathematics on this page.\\ If your
browser supports JavaScript, be sure it is enabled.}

\begin{center}\rule{3in}{0.4pt}\end{center}

{[}\href{coursch9.html}{next}{]} {[}\href{coursch7.html}{prev}{]}
{[}\href{coursch7.html\#tailcoursch7.html}{prev-tail}{]}
{[}\hyperref[tailcoursch8.html]{tail}{]}
{[}\href{cours.html\#coursch8.html}{up}{]}

\subsection{Chapitre~7\\Suites et séries}

~7.1 \href{coursse35.html\#x45-2040007.1}{Convergence des suites} \\
~~7.1.1 \href{coursse35.html\#x45-2050007.1.1}{Monotonie (suites à
termes réels)} \\ ~~7.1.2 \href{coursse35.html\#x45-2060007.1.2}{Critère
de Cauchy} \\ ~~7.1.3 \href{coursse35.html\#x45-2070007.1.3}{Valeurs
d'adhérences, limites inférieures et supérieures} \\ ~~7.1.4
\href{coursse35.html\#x45-2080007.1.4}{Récurrences d'ordre 1} \\ ~7.2
\href{coursse36.html\#x46-2090007.2}{Généralités sur les séries} \\
~~7.2.1 \href{coursse36.html\#x46-2100007.2.1}{Notion de série} \\
~~7.2.2 \href{coursse36.html\#x46-2110007.2.2}{Terme général, critère de
Cauchy} \\ ~7.3 \href{coursse37.html\#x47-2120007.3}{Séries à termes
réels positifs} \\ ~~7.3.1
\href{coursse37.html\#x47-2130007.3.1}{Convergence des séries à termes
réels positifs} \\ ~~7.3.2
\href{coursse37.html\#x47-2140007.3.2}{Comparaison des séries à termes
réels positifs} \\ ~~7.3.3 \href{coursse37.html\#x47-2150007.3.3}{Séries
de Riemann et de Bertrand} \\ ~~7.3.4
\href{coursse37.html\#x47-2160007.3.4}{Comparaison à des intégrales} \\
~7.4 \href{coursse38.html\#x48-2170007.4}{Séries absolument
convergentes} \\ ~~7.4.1 \href{coursse38.html\#x48-2180007.4.1}{Notion
de convergence absolue} \\ ~~7.4.2
\href{coursse38.html\#x48-2190007.4.2}{Critères de convergence absolue}
\\ ~~7.4.3 \href{coursse38.html\#x48-2200007.4.3}{Règles classiques} \\
~~7.4.4 \href{coursse38.html\#x48-2210007.4.4}{Règles complémentaires}
\\ ~~7.4.5 \href{coursse38.html\#x48-2220007.4.5}{Comparaison à une
intégrale} \\ ~7.5 \href{coursse39.html\#x49-2230007.5}{Séries
semi-convergentes} \\ ~~7.5.1
\href{coursse39.html\#x49-2240007.5.1}{Séries alternées} \\ ~~7.5.2
\href{coursse39.html\#x49-2250007.5.2}{Etude de séries
semi-convergentes} \\ ~7.6
\href{coursse40.html\#x50-2260007.6}{Opérations sur les séries} \\
~~7.6.1 \href{coursse40.html\#x50-2270007.6.1}{Combinaisons linéaires}
\\ ~~7.6.2 \href{coursse40.html\#x50-2280007.6.2}{Sommation par paquets}
\\ ~~7.6.3 \href{coursse40.html\#x50-2290007.6.3}{Modification de
l'ordre des termes} \\ ~~7.6.4
\href{coursse40.html\#x50-2300007.6.4}{Produit de Cauchy} \\ ~7.7
\href{coursse41.html\#x51-2310007.7}{Séries doubles} \\ ~7.8
\href{coursse42.html\#x52-2320007.8}{Espaces de suites} \\ ~7.9
\href{coursse43.html\#x53-2330007.9}{Compléments: développements
asymptotiques, analyse numérique} \\ ~~7.9.1
\href{coursse43.html\#x53-2340007.9.1}{Calcul approché de la somme d'une
série} \\ ~~7.9.2 \href{coursse43.html\#x53-2350007.9.2}{Accélération de
la convergence}

{[}\href{coursch9.html}{next}{]} {[}\href{coursch7.html}{prev}{]}
{[}\href{coursch7.html\#tailcoursch7.html}{prev-tail}{]}
{[}\href{coursch8.html}{front}{]}
{[}\href{cours.html\#coursch8.html}{up}{]}

\end{document}
