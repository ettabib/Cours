\section{Dualité : approche générale}

\begin{rem}
Cette section ne figure pas au programme des classes préparatoires. Elle reprend les définitions et les résultats de la section précédente en les généralisant.
\end{rem}

\subsection{Notion de dual. Orthogonalité}

\begin{de}
\index{dualité!forme linéaire}
\index{dual}
Soit $E$ un $K$-espace vectoriel. On appelle forme linéaire sur $E$ toute application linéaire de $E$ dans $K$. On appelle dual de $E$ le $K$-espace vectoriel $E^* = L(E,K)$.
\end{de}

\begin{rem}
\index{forme bilinéaire!canonique}
\index{orthogonalité}
On dispose d'une application bilinéaire de $E^*\times E$ dans $K$ donnée par $\langle f|x\rangle = f(x)$ appelée la forme bilinéaire canonique. À cette forme bilinéaire est associée une notion d'orthogonalité. On notera donc :
\begin{enumerate}
\item si $A \subset E$, $A^\perp = \{f \in E^* | \forall x \in A, f(x) = 0\}$
\item si $B \subset E^*$, $B^o = \{x \in E | \forall f \in B, f(x) = 0\}$
\end{enumerate}
\end{rem}

\begin{prop}
\index{orthogonalité!propriétés}
Les notations $A,A_1,A_2$ désignant des parties de $E$ et $B,B_1,B_2$ désignant des parties de $E^*$, on a :
\begin{enumerate}
\item $A^\perp$ et $B^o$ sont des sous-espaces vectoriels de $E^*$ et $E$; $A^\perp = \operatorname{Vect}(A)^\perp$ et $B^o = \operatorname{Vect}(B)^o$
\item $A_1 \subset A_2 \Rightarrow A_1^\perp \supset A_2^\perp$ et $B_1 \subset B_2 \Rightarrow B_1^o \supset B_2^o$
\item $A \subset (A^\perp)^o$ et $B \subset (B^o)^\perp$
\item Soit $A$ un sous-espace vectoriel de $E$, alors $A^\perp = \{0\} \Leftrightarrow A = E$ et $A^\perp = E^* \Leftrightarrow A = \{0\}$
\item Soit $B$ un sous-espace vectoriel de $E^*$, alors $B^o = E \Leftrightarrow B = \{0\}$
\end{enumerate}
\end{prop}

\subsection{Hyperplans}

\begin{de}
\index{hyperplan!définition}
On appelle hyperplan de $E$ tout sous-espace vectoriel $H$ de $E$ vérifiant les conditions équivalentes :
\begin{enumerate}
\item $\dim E/H = 1$
\item $\exists f \in E^*\setminus\{0\}, H = \operatorname{Ker}(f)$
\item $H$ admet une droite comme supplémentaire
\end{enumerate}
\end{de}

\begin{thm}
\index{hyperplan!orthogonal}
Soit $H$ un hyperplan de $E$. Alors $H^\perp$ est de dimension 1 (droite vectorielle) : deux formes linéaires nulles sur $H$ sont proportionnelles.
\end{thm}

\subsection{Bidual}

\begin{de}
\index{bidual}
On désigne par $E^{**}$ le dual de $E^*$.
\end{de}

\begin{rem}
Si $E$ est de dimension finie, $E^*$ aussi et $\dim E^* = \dim E$. On en déduit que $E^{**}$ est aussi de dimension finie encore égale à $\dim E$.
\end{rem}

\begin{thm}
\index{bidual!isomorphisme canonique}
L'application $u : E \to E^{**}, x \mapsto u_x$ définie par $u_x(f) = f(x)$ est une application linéaire injective. Si $E$ est un espace vectoriel de dimension finie, c'est un isomorphisme d'espaces vectoriels.
\end{thm}

\subsection{Transposée}

\begin{de}
\index{transposée!application linéaire}
Soit $u \in L(E,F)$. On note ${}^tu : F^* \to E^*$ définie par ${}^tu(g) = g \circ u$ (c'est une application linéaire).
\end{de}

\begin{rem}
Cela revient à poser, pour $x \in E$ et $g \in F^*$, $\langle {}^tu(g)|x\rangle_E = \langle g|u(x)\rangle_F$.
\end{rem}

\begin{thm}
\index{transposée!propriétés}
On a les propriétés suivantes :
\begin{enumerate}
\item $u \mapsto {}^tu$ est linéaire de $L(E,F)$ dans $L(F^*,E^*)$
\item $u \in L(E,F), v \in L(F,G)$; alors ${}^t(v \circ u) = {}^tu \circ {}^tv$
\item Si $u$ est bijective, ${}^tu$ aussi et $({}^tu)^{-1} = {}^t(u^{-1})$
\item $\operatorname{Ker}({}^tu) = (\operatorname{Im}(u))^\perp$
\item $\operatorname{Im}({}^tu) = (\operatorname{Ker}(u))^\perp$
\end{enumerate}
\end{thm}

\subsection{Dualité en dimension finie}

\begin{prop}
\index{base!duale}
Soit $E$ un espace vectoriel de dimension finie, $\mathcal{E} = (e_1,\ldots,e_n)$ une base de $E$. La famille $\mathcal{E}' = (e_1^*,\ldots,e_n^*)$ de $E^*$ définie par $e_i^*(e_j) = \delta_i^j$ est une base de $E^*$ appelée la base duale de la base $\mathcal{E}$.
\end{prop}

\begin{thm}
\index{base!duale!bijection}
Soit $E$ un espace vectoriel de dimension finie. L'application $\mathcal{E} \mapsto \mathcal{E}'$ est une bijection de l'ensemble des bases de $E$ sur l'ensemble des bases de $E^*$.
\end{thm}

\begin{cor}
\index{dualité!dimension finie}
Soit $E$ un espace vectoriel de dimension finie.
\begin{enumerate}
\item Soit $A$ un sous-espace vectoriel de $E$. On a :
\[ \dim E = \dim A + \dim A^\perp \text{ et } (A^\perp)^o = A \]
\item Soit $B$ un sous-espace vectoriel de $E^*$. On a :
\[ \dim E = \dim B + \dim B^o \text{ et } (B^o)^\perp = B \]
\end{enumerate}
\end{cor}

\begin{cor}
\index{dualité!système d'équations}
Soit $E$ un espace vectoriel de dimension finie, $f_1,\ldots,f_k \in E^*$, $V = \{x \in E | f_1(x) = \cdots = f_k(x) = 0\}$. Alors
\[ \dim V = \dim E - \operatorname{rg}(f_1,\ldots,f_k) \]
\end{cor}

\begin{thm}
\index{transposée!rang}
Soit $E$ et $F$ deux espaces vectoriels de dimensions finies, $u \in L(E,F)$. Alors 
\[ \operatorname{rg}(u) = \operatorname{rg}({}^tu) \]
\end{thm}