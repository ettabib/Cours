\documentclass[]{article}
\usepackage[T1]{fontenc}
\usepackage{lmodern}
\usepackage{amssymb,amsmath}
\usepackage{ifxetex,ifluatex}
\usepackage{fixltx2e} % provides \textsubscript
% use upquote if available, for straight quotes in verbatim environments
\IfFileExists{upquote.sty}{\usepackage{upquote}}{}
\ifnum 0\ifxetex 1\fi\ifluatex 1\fi=0 % if pdftex
  \usepackage[utf8]{inputenc}
\else % if luatex or xelatex
  \ifxetex
    \usepackage{mathspec}
    \usepackage{xltxtra,xunicode}
  \else
    \usepackage{fontspec}
  \fi
  \defaultfontfeatures{Mapping=tex-text,Scale=MatchLowercase}
  \newcommand{\euro}{€}
\fi
% use microtype if available
\IfFileExists{microtype.sty}{\usepackage{microtype}}{}
\ifxetex
  \usepackage[setpagesize=false, % page size defined by xetex
              unicode=false, % unicode breaks when used with xetex
              xetex]{hyperref}
\else
  \usepackage[unicode=true]{hyperref}
\fi
\hypersetup{breaklinks=true,
            bookmarks=true,
            pdfauthor={},
            pdftitle={Dualite : approche generale},
            colorlinks=true,
            citecolor=blue,
            urlcolor=blue,
            linkcolor=magenta,
            pdfborder={0 0 0}}
\urlstyle{same}  % don't use monospace font for urls
\setlength{\parindent}{0pt}
\setlength{\parskip}{6pt plus 2pt minus 1pt}
\setlength{\emergencystretch}{3em}  % prevent overfull lines
\setcounter{secnumdepth}{0}
 
/* start css.sty */
.cmr-5{font-size:50%;}
.cmr-7{font-size:70%;}
.cmmi-5{font-size:50%;font-style: italic;}
.cmmi-7{font-size:70%;font-style: italic;}
.cmmi-10{font-style: italic;}
.cmsy-5{font-size:50%;}
.cmsy-7{font-size:70%;}
.cmex-7{font-size:70%;}
.cmex-7x-x-71{font-size:49%;}
.msbm-7{font-size:70%;}
.cmtt-10{font-family: monospace;}
.cmti-10{ font-style: italic;}
.cmbx-10{ font-weight: bold;}
.cmr-17x-x-120{font-size:204%;}
.cmsl-10{font-style: oblique;}
.cmti-7x-x-71{font-size:49%; font-style: italic;}
.cmbxti-10{ font-weight: bold; font-style: italic;}
p.noindent { text-indent: 0em }
td p.noindent { text-indent: 0em; margin-top:0em; }
p.nopar { text-indent: 0em; }
p.indent{ text-indent: 1.5em }
@media print {div.crosslinks {visibility:hidden;}}
a img { border-top: 0; border-left: 0; border-right: 0; }
center { margin-top:1em; margin-bottom:1em; }
td center { margin-top:0em; margin-bottom:0em; }
.Canvas { position:relative; }
li p.indent { text-indent: 0em }
.enumerate1 {list-style-type:decimal;}
.enumerate2 {list-style-type:lower-alpha;}
.enumerate3 {list-style-type:lower-roman;}
.enumerate4 {list-style-type:upper-alpha;}
div.newtheorem { margin-bottom: 2em; margin-top: 2em;}
.obeylines-h,.obeylines-v {white-space: nowrap; }
div.obeylines-v p { margin-top:0; margin-bottom:0; }
.overline{ text-decoration:overline; }
.overline img{ border-top: 1px solid black; }
td.displaylines {text-align:center; white-space:nowrap;}
.centerline {text-align:center;}
.rightline {text-align:right;}
div.verbatim {font-family: monospace; white-space: nowrap; text-align:left; clear:both; }
.fbox {padding-left:3.0pt; padding-right:3.0pt; text-indent:0pt; border:solid black 0.4pt; }
div.fbox {display:table}
div.center div.fbox {text-align:center; clear:both; padding-left:3.0pt; padding-right:3.0pt; text-indent:0pt; border:solid black 0.4pt; }
div.minipage{width:100%;}
div.center, div.center div.center {text-align: center; margin-left:1em; margin-right:1em;}
div.center div {text-align: left;}
div.flushright, div.flushright div.flushright {text-align: right;}
div.flushright div {text-align: left;}
div.flushleft {text-align: left;}
.underline{ text-decoration:underline; }
.underline img{ border-bottom: 1px solid black; margin-bottom:1pt; }
.framebox-c, .framebox-l, .framebox-r { padding-left:3.0pt; padding-right:3.0pt; text-indent:0pt; border:solid black 0.4pt; }
.framebox-c {text-align:center;}
.framebox-l {text-align:left;}
.framebox-r {text-align:right;}
span.thank-mark{ vertical-align: super }
span.footnote-mark sup.textsuperscript, span.footnote-mark a sup.textsuperscript{ font-size:80%; }
div.tabular, div.center div.tabular {text-align: center; margin-top:0.5em; margin-bottom:0.5em; }
table.tabular td p{margin-top:0em;}
table.tabular {margin-left: auto; margin-right: auto;}
div.td00{ margin-left:0pt; margin-right:0pt; }
div.td01{ margin-left:0pt; margin-right:5pt; }
div.td10{ margin-left:5pt; margin-right:0pt; }
div.td11{ margin-left:5pt; margin-right:5pt; }
table[rules] {border-left:solid black 0.4pt; border-right:solid black 0.4pt; }
td.td00{ padding-left:0pt; padding-right:0pt; }
td.td01{ padding-left:0pt; padding-right:5pt; }
td.td10{ padding-left:5pt; padding-right:0pt; }
td.td11{ padding-left:5pt; padding-right:5pt; }
table[rules] {border-left:solid black 0.4pt; border-right:solid black 0.4pt; }
.hline hr, .cline hr{ height : 1px; margin:0px; }
.tabbing-right {text-align:right;}
span.TEX {letter-spacing: -0.125em; }
span.TEX span.E{ position:relative;top:0.5ex;left:-0.0417em;}
a span.TEX span.E {text-decoration: none; }
span.LATEX span.A{ position:relative; top:-0.5ex; left:-0.4em; font-size:85%;}
span.LATEX span.TEX{ position:relative; left: -0.4em; }
div.float img, div.float .caption {text-align:center;}
div.figure img, div.figure .caption {text-align:center;}
.marginpar {width:20%; float:right; text-align:left; margin-left:auto; margin-top:0.5em; font-size:85%; text-decoration:underline;}
.marginpar p{margin-top:0.4em; margin-bottom:0.4em;}
.equation td{text-align:center; vertical-align:middle; }
td.eq-no{ width:5%; }
table.equation { width:100%; } 
div.math-display, div.par-math-display{text-align:center;}
math .texttt { font-family: monospace; }
math .textit { font-style: italic; }
math .textsl { font-style: oblique; }
math .textsf { font-family: sans-serif; }
math .textbf { font-weight: bold; }
.partToc a, .partToc, .likepartToc a, .likepartToc {line-height: 200%; font-weight:bold; font-size:110%;}
.chapterToc a, .chapterToc, .likechapterToc a, .likechapterToc, .appendixToc a, .appendixToc {line-height: 200%; font-weight:bold;}
.index-item, .index-subitem, .index-subsubitem {display:block}
.caption td.id{font-weight: bold; white-space: nowrap; }
table.caption {text-align:center;}
h1.partHead{text-align: center}
p.bibitem { text-indent: -2em; margin-left: 2em; margin-top:0.6em; margin-bottom:0.6em; }
p.bibitem-p { text-indent: 0em; margin-left: 2em; margin-top:0.6em; margin-bottom:0.6em; }
.paragraphHead, .likeparagraphHead { margin-top:2em; font-weight: bold;}
.subparagraphHead, .likesubparagraphHead { font-weight: bold;}
.quote {margin-bottom:0.25em; margin-top:0.25em; margin-left:1em; margin-right:1em; text-align:justify;}
.verse{white-space:nowrap; margin-left:2em}
div.maketitle {text-align:center;}
h2.titleHead{text-align:center;}
div.maketitle{ margin-bottom: 2em; }
div.author, div.date {text-align:center;}
div.thanks{text-align:left; margin-left:10%; font-size:85%; font-style:italic; }
div.author{white-space: nowrap;}
.quotation {margin-bottom:0.25em; margin-top:0.25em; margin-left:1em; }
h1.partHead{text-align: center}
.sectionToc, .likesectionToc {margin-left:2em;}
.subsectionToc, .likesubsectionToc {margin-left:4em;}
.subsubsectionToc, .likesubsubsectionToc {margin-left:6em;}
.frenchb-nbsp{font-size:75%;}
.frenchb-thinspace{font-size:75%;}
.figure img.graphics {margin-left:10%;}
/* end css.sty */

\title{Dualite : approche generale}
\author{}
\date{}

\begin{document}
\maketitle

\textbf{Warning: \href{http://www.math.union.edu/locate/jsMath}{jsMath}
requires JavaScript to process the mathematics on this page.\\ If your
browser supports JavaScript, be sure it is enabled.}

\begin{center}\rule{3in}{0.4pt}\end{center}

{[}\href{coursse12.html}{next}{]} {[}\href{coursse10.html}{prev}{]}
{[}\href{coursse10.html\#tailcoursse10.html}{prev-tail}{]}
{[}\hyperref[tailcoursse11.html]{tail}{]}
{[}\href{coursch3.html\#coursse11.html}{up}{]}

\subsubsection{2.5 Dualité~: approche générale}

Cette section ne figure pas au programme des classes préparatoires. Elle
reprend les définitions et les résultats de la section précédente en les
généralisant.

\paragraph{2.5.1 Notion de dual. Orthogonalité}

Définition~2.5.1 Soit E un K-espace vectoriel . On appelle forme
linéaire sur E toute application linéaire de E dans K. On appelle dual
de E le K-espace vectoriel \{E\}\^{}\{∗\} = L(E,K).

Remarque~2.5.1 On dispose d'une application bilinéaire de
\{E\}\^{}\{∗\}× E dans K donnée par \textbackslash{}langle
f\textbackslash{}mathrel\{∣\}x\textbackslash{}rangle = f(x) appelée la
forme bilinéaire canonique. A cette forme bilinéaire est associée une
notion d'orthogonalité. On notera donc

\begin{itemize}
\itemsep1pt\parskip0pt\parsep0pt
\item
  (i) si A ⊂ E, \{A\}\^{}\{⊥\} = \textbackslash{}\{f ∈
  \{E\}\^{}\{∗\}\textbackslash{}mathrel\{∣\}\textbackslash{}mathop\{∀\}x
  ∈ A, f(x) = 0\textbackslash{}\}
\item
  (ii) si B ⊂ \{E\}\^{}\{∗\}, \{B\}\^{}\{o\} = \textbackslash{}\{x ∈
  E\textbackslash{}mathrel\{∣\}\textbackslash{}mathop\{∀\}f ∈ B, f(x) =
  0\textbackslash{}\}
\end{itemize}

Proposition~2.5.1 Les notations A,\{A\}\_\{1\},\{A\}\_\{2\} désignant
des parties de E et B,\{B\}\_\{1\},\{B\}\_\{2\} désignant des parties de
\{E\}\^{}\{∗\}, on a

\begin{itemize}
\itemsep1pt\parskip0pt\parsep0pt
\item
  (i) \{A\}\^{}\{⊥\} et \{B\}\^{}\{o\} sont des sous-espaces vectoriels
  de \{E\}\^{}\{∗\} et E~; \{A\}\^{}\{⊥\} =\textbackslash{}mathop\{
  \textbackslash{}mathrm\{Vect\}\}\{(A)\}\^{}\{⊥\} et \{B\}\^{}\{o\}
  =\textbackslash{}mathop\{
  \textbackslash{}mathrm\{Vect\}\}\{(B)\}\^{}\{o\}
\item
  (ii) \{A\}\_\{1\} ⊂ \{A\}\_\{2\} ⇒ \{A\}\_\{1\}\^{}\{⊥\}⊃
  \{A\}\_\{2\}\^{}\{⊥\} et \{B\}\_\{1\} ⊂ \{B\}\_\{2\} ⇒
  \{B\}\_\{1\}\^{}\{o\} ⊃ \{B\}\_\{2\}\^{}\{o\}
\item
  (iii) A ⊂ \{(\{A\}\^{}\{⊥\})\}\^{}\{o\} et B ⊂
  \{(\{B\}\^{}\{o\})\}\^{}\{⊥\}
\item
  (iv) Soit A un sous-espace vectoriel de E, alors \{A\}\^{}\{⊥\} =
  \textbackslash{}\{0\textbackslash{}\} \textbackslash{}mathrel\{⇔\} A =
  E et \{A\}\^{}\{⊥\} = \{E\}\^{}\{∗\}\textbackslash{}mathrel\{⇔\} A =
  \textbackslash{}\{0\textbackslash{}\}.
\item
  (v) Soit B un sous-espace vectoriel de \{E\}\^{}\{∗\}, alors
  \{B\}\^{}\{o\} = E \textbackslash{}mathrel\{⇔\} B =
  \textbackslash{}\{0\textbackslash{}\}.
\end{itemize}

Démonstration Les propriétés (i),(ii) et (iii) sont évidentes ainsi que
les parties '' ⇐'' de (iv) et (v).

Montrons donc que A\textbackslash{}mathrel\{≠\}E ⇒
\{A\}\^{}\{⊥\}\textbackslash{}mathrel\{≠\}\textbackslash{}\{0\textbackslash{}\}.
Soit \{(\{e\}\_\{i\})\}\_\{i∈I\} une base de A que l'on complète en
\{(\{e\}\_\{i\})\}\_\{i∈J\} base de E. Soit \{i\}\_\{0\} ∈ J ∖ I et f
l'application qui à x associe sa \{i\}\_\{0\}-ième coordonnée dans la
base. On a f\textbackslash{}mathrel\{≠\}0 et f ∈ \{A\}\^{}\{⊥\}.

Montrons maintenant que
A\textbackslash{}mathrel\{≠\}\textbackslash{}\{0\textbackslash{}\} ⇒
\{A\}\^{}\{⊥\}\textbackslash{}mathrel\{≠\}\{E\}\^{}\{∗\}. Pour cela soit
x ∈ A ∖\textbackslash{}\{0\textbackslash{}\}. On complète x en une base
\{(\{e\}\_\{i\})\}\_\{i∈I\} de E avec x = \{e\}\_\{\{i\}\_\{0\}\}. Soit
f l'application qui à x associe sa \{i\}\_\{0\}-ième coordonnée dans la
base. On a f(x)\textbackslash{}mathrel\{≠\}0, donc
f\textbackslash{}mathrel\{∉\}\{A\}\^{}\{⊥\}.

Montrons maintenant que
B\textbackslash{}mathrel\{≠\}\textbackslash{}\{0\textbackslash{}\} ⇒
\{B\}\^{}\{o\}\textbackslash{}mathrel\{≠\}E. Soit f ∈ B
∖\textbackslash{}\{0\textbackslash{}\}. On a
f\textbackslash{}mathrel\{≠\}0, donc \textbackslash{}mathop\{∃\}x ∈ E,
f(x)\textbackslash{}mathrel\{≠\}0. Dans ce cas
x\textbackslash{}mathrel\{∉\}\{B\}\^{}\{o\}, ce qui achève la
démonstration.

On prendra garde qu'on peut avoir \{B\}\^{}\{o\} =
\textbackslash{}\{0\textbackslash{}\} avec
B\textbackslash{}mathrel\{≠\}\{E\}\^{}\{∗\} (prendre par exemple E =
ℝ{[}X{]} et B =\textbackslash{}mathop\{
\textbackslash{}mathrm\{Vect\}\}(\{ε\}\_\{x\},x ∈ ℤ) où \{ε\}\_\{x\}(P)
= P(x)~; on a \{B\}\^{}\{o\} = \textbackslash{}\{0\textbackslash{}\}
alors que \{ε\}\_\{1∕2\}\textbackslash{}mathrel\{∉\}B).

\paragraph{2.5.2 Hyperplans}

Définition~2.5.2 On appelle hyperplan de E tout sous-espace vectoriel H
de E vérifiant les conditions équivalentes

\begin{itemize}
\itemsep1pt\parskip0pt\parsep0pt
\item
  (i) \textbackslash{}mathop\{dim\} E∕H = 1
\item
  (ii) \textbackslash{}mathop\{∃\}f ∈
  \{E\}\^{}\{∗\}∖\textbackslash{}\{0\textbackslash{}\}, H
  =\textbackslash{}mathop\{ \textbackslash{}mathrm\{Ker\}\}f
\item
  (iii) H admet une droite comme supplémentaire.
\end{itemize}

Démonstration (i) ⇒(ii)~: prendre \textbackslash{}overline\{e\} une base
de E∕H et écrire π(x) = f(x)\textbackslash{}overline\{e\}.

(ii) ⇒ (iii)~: on prend a ∈ E tel que f(a)\textbackslash{}mathrel\{≠\}0.
Tout élément x s'écrit de manière unique sous la forme x = (x −\{ f(x)
\textbackslash{}over f(a)\} a) +\{ f(x) \textbackslash{}over f(a)\} a
avec x −\{ f(x) \textbackslash{}over f(a)\} a
∈\textbackslash{}mathop\{\textbackslash{}mathrm\{Ker\}\}f, soit E
=\textbackslash{}mathop\{ \textbackslash{}mathrm\{Ker\}\}f ⊕ Ka.

(iii) ⇒(i)~: tout supplémentaire de H est isomorphe à E∕H.

Théorème~2.5.2 Soit H un hyperplan de E. Alors \{H\}\^{}\{⊥\} est de
dimension 1 (droite vectorielle)~: deux formes linéaires nulles sur H
sont proportionnelles.

Démonstration Si E = H ⊕ Ka et H =\textbackslash{}mathop\{
\textbackslash{}mathrm\{Ker\}\}f, soit g ∈ \{H\}\^{}\{⊥\}. Alors g et \{
g(a) \textbackslash{}over f(a)\} f coïncident sur H et sur Ka, donc sont
égales.

\paragraph{2.5.3 Bidual}

Définition~2.5.3 On désigne par \{E\}\^{}\{∗∗\} le dual de
\{E\}\^{}\{∗\}.

Remarque~2.5.2 Si E est de dimension finie, \{E\}\^{}\{∗\} aussi et
\textbackslash{}mathop\{dim\} \{E\}\^{}\{∗\} =\textbackslash{}mathop\{
dim\} E. On en déduit que \{E\}\^{}\{∗∗\} est aussi de dimension finie
encore égale à \textbackslash{}mathop\{dim\} E.

Théorème~2.5.3 L'application u : E → \{E\}\^{}\{∗∗\},
x\textbackslash{}mathrel\{↦\}\{u\}\_\{x\} définie par \{u\}\_\{x\}(f) =
f(x) est une application linéaire injective. Si E est un espace
vectoriel de dimension finie, c'est un isomorphisme d'espaces
vectoriels.

Démonstration En effet, cette application est visiblement linéaire et si
x ∈\textbackslash{}mathop\{\textbackslash{}mathrm\{Ker\}\}u, on a

\textbackslash{}mathop\{∀\}f ∈ \{E\}\^{}\{∗\}, f(x) = \{u\}\_\{ x\}(f) =
0(f) = 0

et donc x ∈ \{(\{E\}\^{}\{∗\})\}\^{}\{o\} =
\textbackslash{}\{0\textbackslash{}\}~; elle est donc injective. Si E
est un espace vectoriel de dimension finie, on a une application
linéaire injective entre deux espaces de même dimension finie, elle est
donc bijective.

\paragraph{2.5.4 Transposée}

Définition~2.5.4 Soit u ∈ L(E,F). On note \{\}\^{}\{t\}u :
\{F\}\^{}\{∗\}→ \{E\}\^{}\{∗\} définie par \{\}\^{}\{t\}u(g) = g ∘ u
(c'est une application linéaire).

Remarque~2.5.3 Cela revient à poser, pour x ∈ E et g ∈ \{F\}\^{}\{∗\},
\{\textbackslash{}langle
\}\^{}\{t\}u(g)\textbackslash{}mathrel\{∣\}\{x\textbackslash{}rangle
\}\_\{E\} =\textbackslash{}langle
g\textbackslash{}mathrel\{∣\}u\{(x)\textbackslash{}rangle \}\_\{F\}.

Théorème~2.5.4 On a les propriétés suivantes

\begin{itemize}
\itemsep1pt\parskip0pt\parsep0pt
\item
  (i) u\{\textbackslash{}mathrel\{↦\}\}\^{}\{t\}u est linéaire de L(E,F)
  dans L(\{F\}\^{}\{∗\},\{E\}\^{}\{∗\}).
\item
  (ii) u ∈ L(E,F),v ∈ L(F,G)~; alors \{\}\^{}\{t\}(v ∘ u) \{=
  \}\^{}\{t\}u \{∘\}\^{}\{t\}v
\item
  (iii) Si u est bijective, \{\}\^{}\{t\}u aussi et
  \{\{(\}\^{}\{t\}u)\}\^{}\{−1\} \{= \}\^{}\{t\}(\{u\}\^{}\{−1\})
\item
  (iv)
  \{\textbackslash{}mathop\{\textbackslash{}mathrm\{Ker\}\}\}\^{}\{t\}u
  =
  \{(\textbackslash{}mathop\{\textbackslash{}mathrm\{Im\}\}u)\}\^{}\{⊥\}
\item
  (v)
  \{\textbackslash{}mathop\{\textbackslash{}mathrm\{Im\}\}\}\^{}\{t\}u =
  \{(\textbackslash{}mathop\{\textbackslash{}mathrm\{Ker\}\}u)\}\^{}\{⊥\}
\end{itemize}

Démonstration (i) et (ii) sont très faciles à partir de la définition.
(iii) découle immédiatement de (ii) en écrivant que v ∘ u =\{
\textbackslash{}mathrm\{Id\}\}\_\{E\} et u ∘ v =\{
\textbackslash{}mathrm\{Id\}\}\_\{F\}.

Pour (iv), on a g
∈\{\textbackslash{}mathop\{\textbackslash{}mathrm\{Ker\}\}\}\^{}\{t\}u
\textbackslash{}mathrel\{⇔\} \textbackslash{}mathop\{∀\}x ∈ E\{,
\}\^{}\{t\}u(g)(x) = 0 \textbackslash{}mathrel\{⇔\}
\textbackslash{}mathop\{∀\}x ∈ E, g(u(x)) = 0
\textbackslash{}mathrel\{⇔\} g ∈
\{(\textbackslash{}mathop\{\textbackslash{}mathrm\{Im\}\}u)\}\^{}\{⊥\}.

Pour (v), on remarque d'abord que f
∈\{\textbackslash{}mathop\{\textbackslash{}mathrm\{Im\}\}\}\^{}\{t\}u
⇒\textbackslash{}mathop\{∃\}g, f = g ∘ u ⇒\textbackslash{}mathop\{∀\}x
∈\textbackslash{}mathop\{\textbackslash{}mathrm\{Ker\}\}u, f(x) = 0 ⇒ f
∈
\{(\textbackslash{}mathop\{\textbackslash{}mathrm\{Ker\}\}u)\}\^{}\{⊥\},
soit
\{\textbackslash{}mathop\{\textbackslash{}mathrm\{Im\}\}\}\^{}\{t\}u ⊂
\{(\textbackslash{}mathop\{\textbackslash{}mathrm\{Ker\}\}u)\}\^{}\{⊥\}.
Inversement, soit f ∈
\{(\textbackslash{}mathop\{\textbackslash{}mathrm\{Ker\}\}u)\}\^{}\{⊥\}.
On définit \{g\}\_\{1\} forme linéaire sur
\textbackslash{}mathop\{\textbackslash{}mathrm\{Im\}\}u par
\{g\}\_\{1\}(y) = f(x) si y = u(x)~; on vérifie en effet que f(x) est
indépendant du choix de x tel que y = u(x) car f est nulle sur
\textbackslash{}mathop\{\textbackslash{}mathrm\{Ker\}\}u. Soit alors V
un supplémentaire de
\textbackslash{}mathop\{\textbackslash{}mathrm\{Im\}\}u dans F. On
définit g : F → K par g(\{y\}\_\{1\} + \{y\}\_\{2\}) =
\{g\}\_\{1\}(\{y\}\_\{1\}) si \{y\}\_\{1\}
∈\textbackslash{}mathop\{\textbackslash{}mathrm\{Im\}\}u et \{y\}\_\{2\}
∈ V . On a bien f = g ∘ u \{= \}\^{}\{t\}u(g). Donc
\{(\textbackslash{}mathop\{\textbackslash{}mathrm\{Ker\}\}u)\}\^{}\{⊥\}⊂\{\textbackslash{}mathop\{\textbackslash{}mathrm\{Im\}\}\}\^{}\{t\}u
et donc l'égalité.

\paragraph{2.5.5 Dualité en dimension finie}

Proposition~2.5.5 Soit E un espace vectoriel de dimension finie, ℰ =
(\{e\}\_\{1\},\textbackslash{}mathop\{\textbackslash{}mathop\{\ldots{}\}\},\{e\}\_\{n\})
une base de E. La famille ℰ' =
(\{e\}\_\{1\}\^{}\{∗\},\textbackslash{}mathop\{\textbackslash{}mathop\{\ldots{}\}\},\{e\}\_\{n\}\^{}\{∗\})
de \{E\}\^{}\{∗\} définie par \{e\}\_\{i\}\^{}\{∗\}(\{e\}\_\{j\}) =
\{δ\}\_\{i\}\^{}\{j\} est une base de \{E\}\^{}\{∗\} appelée la base
duale de la base ℰ

Démonstration On vérifie en effet immédiatement qu'elle est libre et
elle a le bon cardinal.

Théorème~2.5.6 Soit E un espace vectoriel de dimension finie.
L'application ℰ→ℰ' est une bijection de l'ensemble des bases de E sur
l'ensemble des bases de \{E\}\^{}\{∗\}.

Démonstration Injectivité~: si
((\{e\}\_\{1\},\textbackslash{}mathop\{\textbackslash{}mathop\{\ldots{}\}\},\{e\}\_\{n\})
et
(\{e\}\_\{1\}',\textbackslash{}mathop\{\textbackslash{}mathop\{\ldots{}\}\},\{e\}\_\{n\}')
sont deux bases qui ont même base duale, on a pour toute f ∈
\{E\}\^{}\{∗\}, f(\{e\}\_\{i\}) = f(\{e\}\_\{i\}') et donc \{e\}\_\{i\}
= \{e\}\_\{i\}'. Surjectivité~: soit ℱ =
(\{f\}\_\{1\},\textbackslash{}mathop\{\textbackslash{}mathop\{\ldots{}\}\},\{f\}\_\{n\})
une base de \{E\}\^{}\{∗\} et soit ℱ sa base duale (dans
\{E\}\^{}\{∗∗\}). Soit u l'isomorphisme de E sur \{E\}\^{}\{∗∗\}, et ℰ =
\{u\}\^{}\{−1\}(ℱ'), base de E. On a alors \{f\}\_\{i\}(\{e\}\_\{j\}) =
\{u\}\_\{\{e\}\_\{j\}\}(\{f\}\_\{i\}) =
\{f\}\_\{j\}\^{}\{∗\}(\{f\}\_\{i\}) = \{δ\}\_\{i\}\^{}\{j\}, donc ℱ est
la base duale de la base ℰ.

Corollaire~2.5.7 Soit E un espace vectoriel de dimension finie.

\begin{itemize}
\item
  (i) Soit A un sous-espace vectoriel de E. On a

  \textbackslash{}mathop\{dim\} E =\textbackslash{}mathop\{ dim\} A
  +\textbackslash{}mathop\{ dim\} \{A\}\^{}\{⊥\}\textbackslash{}text\{
  et \}\{(\{A\}\^{}\{⊥\})\}\^{}\{o\} = A
\item
  (ii) Soit B un sous-espace vectoriel de \{E\}\^{}\{∗\}. On a

  \textbackslash{}mathop\{dim\} E =\textbackslash{}mathop\{ dim\} B
  +\textbackslash{}mathop\{ dim\} \{B\}\^{}\{o\}\textbackslash{}text\{
  et \}\{(\{B\}\^{}\{o\})\}\^{}\{⊥\} = B
\end{itemize}

Démonstration Soit
(\{e\}\_\{1\},\textbackslash{}mathop\{\textbackslash{}mathop\{\ldots{}\}\},\{e\}\_\{p\})
une base de A que l'on complète en
(\{e\}\_\{1\},\textbackslash{}mathop\{\textbackslash{}mathop\{\ldots{}\}\},\{e\}\_\{n\})
base de E. On vérifie immédiatement que \{A\}\^{}\{⊥\}
=\textbackslash{}mathop\{
\textbackslash{}mathrm\{Vect\}\}(\{e\}\_\{p+1\}\^{}\{∗\},\textbackslash{}mathop\{\textbackslash{}mathop\{\ldots{}\}\},\{e\}\_\{n\}\^{}\{∗\})
d'où le résultat sur la dimension. On montre de même le résultat sur la
dimension de \{B\}\^{}\{o\}. Les égalités découlent alors des inclusions
et du fait que les espaces ont même dimension.

Corollaire~2.5.8 Soit E un espace vectoriel de dimension finie,
\{f\}\_\{1\},\textbackslash{}mathop\{\textbackslash{}mathop\{\ldots{}\}\},\{f\}\_\{k\}
∈ \{E\}\^{}\{∗\}, V = \textbackslash{}\{x ∈
E\textbackslash{}mathrel\{∣\}\{f\}\_\{1\}(x) =
\textbackslash{}mathop\{\textbackslash{}mathop\{\ldots{}\}\} =
\{f\}\_\{k\}(x) = 0\textbackslash{}\}. Alors
\textbackslash{}mathop\{dim\} V =\textbackslash{}mathop\{ dim\} E
−\textbackslash{}mathop\{\textbackslash{}mathrm\{rg\}\}(\{f\}\_\{1\},\textbackslash{}mathop\{\textbackslash{}mathop\{\ldots{}\}\},\{f\}\_\{k\}).

Théorème~2.5.9 Soit E et F deux espaces vectoriels de dimensions finies,
u ∈ L(E,F). Alors
\textbackslash{}mathop\{\textbackslash{}mathrm\{rg\}\}u
=\{\textbackslash{}mathop\{ \textbackslash{}mathrm\{rg\}\}\}\^{}\{t\}u.

Démonstration
\{\textbackslash{}mathop\{\textbackslash{}mathrm\{rg\}\}\}\^{}\{t\}u
=\textbackslash{}mathop\{ dim\}
\{\textbackslash{}mathop\{\textbackslash{}mathrm\{Im\}\}\}\^{}\{t\}u
=\textbackslash{}mathop\{ dim\}
\{(\textbackslash{}mathop\{\textbackslash{}mathrm\{Ker\}\}u)\}\^{}\{⊥\}
=\textbackslash{}mathop\{ dim\} E −\textbackslash{}mathop\{ dim\}
\textbackslash{}mathop\{\textbackslash{}mathrm\{Ker\}\}u
=\textbackslash{}mathop\{ \textbackslash{}mathrm\{rg\}\}u.

{[}\href{coursse12.html}{next}{]} {[}\href{coursse10.html}{prev}{]}
{[}\href{coursse10.html\#tailcoursse10.html}{prev-tail}{]}
{[}\href{coursse11.html}{front}{]}
{[}\href{coursch3.html\#coursse11.html}{up}{]}

\end{document}
