\documentclass[]{article}
\usepackage[T1]{fontenc}
\usepackage{lmodern}
\usepackage{amssymb,amsmath}
\usepackage{ifxetex,ifluatex}
\usepackage{fixltx2e} % provides \textsubscript
% use upquote if available, for straight quotes in verbatim environments
\IfFileExists{upquote.sty}{\usepackage{upquote}}{}
\ifnum 0\ifxetex 1\fi\ifluatex 1\fi=0 % if pdftex
  \usepackage[utf8]{inputenc}
\else % if luatex or xelatex
  \ifxetex
    \usepackage{mathspec}
    \usepackage{xltxtra,xunicode}
  \else
    \usepackage{fontspec}
  \fi
  \defaultfontfeatures{Mapping=tex-text,Scale=MatchLowercase}
  \newcommand{\euro}{€}
\fi
% use microtype if available
\IfFileExists{microtype.sty}{\usepackage{microtype}}{}
\ifxetex
  \usepackage[setpagesize=false, % page size defined by xetex
              unicode=false, % unicode breaks when used with xetex
              xetex]{hyperref}
\else
  \usepackage[unicode=true]{hyperref}
\fi
\hypersetup{breaklinks=true,
            bookmarks=true,
            pdfauthor={},
            pdftitle={Dualite : approche generale},
            colorlinks=true,
            citecolor=blue,
            urlcolor=blue,
            linkcolor=magenta,
            pdfborder={0 0 0}}
\urlstyle{same}  % don't use monospace font for urls
\setlength{\parindent}{0pt}
\setlength{\parskip}{6pt plus 2pt minus 1pt}
\setlength{\emergencystretch}{3em}  % prevent overfull lines
\setcounter{secnumdepth}{0}
 
/* start css.sty */
.cmr-5{font-size:50%;}
.cmr-7{font-size:70%;}
.cmmi-5{font-size:50%;font-style: italic;}
.cmmi-7{font-size:70%;font-style: italic;}
.cmmi-10{font-style: italic;}
.cmsy-5{font-size:50%;}
.cmsy-7{font-size:70%;}
.cmex-7{font-size:70%;}
.cmex-7x-x-71{font-size:49%;}
.msbm-7{font-size:70%;}
.cmtt-10{font-family: monospace;}
.cmti-10{ font-style: italic;}
.cmbx-10{ font-weight: bold;}
.cmr-17x-x-120{font-size:204%;}
.cmsl-10{font-style: oblique;}
.cmti-7x-x-71{font-size:49%; font-style: italic;}
.cmbxti-10{ font-weight: bold; font-style: italic;}
p.noindent { text-indent: 0em }
td p.noindent { text-indent: 0em; margin-top:0em; }
p.nopar { text-indent: 0em; }
p.indent{ text-indent: 1.5em }
@media print {div.crosslinks {visibility:hidden;}}
a img { border-top: 0; border-left: 0; border-right: 0; }
center { margin-top:1em; margin-bottom:1em; }
td center { margin-top:0em; margin-bottom:0em; }
.Canvas { position:relative; }
li p.indent { text-indent: 0em }
.enumerate1 {list-style-type:decimal;}
.enumerate2 {list-style-type:lower-alpha;}
.enumerate3 {list-style-type:lower-roman;}
.enumerate4 {list-style-type:upper-alpha;}
div.newtheorem { margin-bottom: 2em; margin-top: 2em;}
.obeylines-h,.obeylines-v {white-space: nowrap; }
div.obeylines-v p { margin-top:0; margin-bottom:0; }
.overline{ text-decoration:overline; }
.overline img{ border-top: 1px solid black; }
td.displaylines {text-align:center; white-space:nowrap;}
.centerline {text-align:center;}
.rightline {text-align:right;}
div.verbatim {font-family: monospace; white-space: nowrap; text-align:left; clear:both; }
.fbox {padding-left:3.0pt; padding-right:3.0pt; text-indent:0pt; border:solid black 0.4pt; }
div.fbox {display:table}
div.center div.fbox {text-align:center; clear:both; padding-left:3.0pt; padding-right:3.0pt; text-indent:0pt; border:solid black 0.4pt; }
div.minipage{width:100%;}
div.center, div.center div.center {text-align: center; margin-left:1em; margin-right:1em;}
div.center div {text-align: left;}
div.flushright, div.flushright div.flushright {text-align: right;}
div.flushright div {text-align: left;}
div.flushleft {text-align: left;}
.underline{ text-decoration:underline; }
.underline img{ border-bottom: 1px solid black; margin-bottom:1pt; }
.framebox-c, .framebox-l, .framebox-r { padding-left:3.0pt; padding-right:3.0pt; text-indent:0pt; border:solid black 0.4pt; }
.framebox-c {text-align:center;}
.framebox-l {text-align:left;}
.framebox-r {text-align:right;}
span.thank-mark{ vertical-align: super }
span.footnote-mark sup.textsuperscript, span.footnote-mark a sup.textsuperscript{ font-size:80%; }
div.tabular, div.center div.tabular {text-align: center; margin-top:0.5em; margin-bottom:0.5em; }
table.tabular td p{margin-top:0em;}
table.tabular {margin-left: auto; margin-right: auto;}
div.td00{ margin-left:0pt; margin-right:0pt; }
div.td01{ margin-left:0pt; margin-right:5pt; }
div.td10{ margin-left:5pt; margin-right:0pt; }
div.td11{ margin-left:5pt; margin-right:5pt; }
table[rules] {border-left:solid black 0.4pt; border-right:solid black 0.4pt; }
td.td00{ padding-left:0pt; padding-right:0pt; }
td.td01{ padding-left:0pt; padding-right:5pt; }
td.td10{ padding-left:5pt; padding-right:0pt; }
td.td11{ padding-left:5pt; padding-right:5pt; }
table[rules] {border-left:solid black 0.4pt; border-right:solid black 0.4pt; }
.hline hr, .cline hr{ height : 1px; margin:0px; }
.tabbing-right {text-align:right;}
span.TEX {letter-spacing: -0.125em; }
span.TEX span.E{ position:relative;top:0.5ex;left:-0.0417em;}
a span.TEX span.E {text-decoration: none; }
span.LATEX span.A{ position:relative; top:-0.5ex; left:-0.4em; font-size:85%;}
span.LATEX span.TEX{ position:relative; left: -0.4em; }
div.float img, div.float .caption {text-align:center;}
div.figure img, div.figure .caption {text-align:center;}
.marginpar {width:20%; float:right; text-align:left; margin-left:auto; margin-top:0.5em; font-size:85%; text-decoration:underline;}
.marginpar p{margin-top:0.4em; margin-bottom:0.4em;}
.equation td{text-align:center; vertical-align:middle; }
td.eq-no{ width:5%; }
table.equation { width:100%; } 
div.math-display, div.par-math-display{text-align:center;}
math .texttt { font-family: monospace; }
math .textit { font-style: italic; }
math .textsl { font-style: oblique; }
math .textsf { font-family: sans-serif; }
math .textbf { font-weight: bold; }
.partToc a, .partToc, .likepartToc a, .likepartToc {line-height: 200%; font-weight:bold; font-size:110%;}
.chapterToc a, .chapterToc, .likechapterToc a, .likechapterToc, .appendixToc a, .appendixToc {line-height: 200%; font-weight:bold;}
.index-item, .index-subitem, .index-subsubitem {display:block}
.caption td.id{font-weight: bold; white-space: nowrap; }
table.caption {text-align:center;}
h1.partHead{text-align: center}
p.bibitem { text-indent: -2em; margin-left: 2em; margin-top:0.6em; margin-bottom:0.6em; }
p.bibitem-p { text-indent: 0em; margin-left: 2em; margin-top:0.6em; margin-bottom:0.6em; }
.paragraphHead, .likeparagraphHead { margin-top:2em; font-weight: bold;}
.subparagraphHead, .likesubparagraphHead { font-weight: bold;}
.quote {margin-bottom:0.25em; margin-top:0.25em; margin-left:1em; margin-right:1em; text-align:\jmathustify;}
.verse{white-space:nowrap; margin-left:2em}
div.maketitle {text-align:center;}
h2.titleHead{text-align:center;}
div.maketitle{ margin-bottom: 2em; }
div.author, div.date {text-align:center;}
div.thanks{text-align:left; margin-left:10%; font-size:85%; font-style:italic; }
div.author{white-space: nowrap;}
.quotation {margin-bottom:0.25em; margin-top:0.25em; margin-left:1em; }
h1.partHead{text-align: center}
.sectionToc, .likesectionToc {margin-left:2em;}
.subsectionToc, .likesubsectionToc {margin-left:4em;}
.subsubsectionToc, .likesubsubsectionToc {margin-left:6em;}
.frenchb-nbsp{font-size:75%;}
.frenchb-thinspace{font-size:75%;}
.figure img.graphics {margin-left:10%;}
/* end css.sty */

\title{Dualite : approche generale}
\author{}
\date{}

\begin{document}
\maketitle

\textbf{Warning: 
requires JavaScript to process the mathematics on this page.\\ If your
browser supports JavaScript, be sure it is enabled.}

\begin{center}\rule{3in}{0.4pt}\end{center}

{[}
{[}
{[}{]}
{[}

\subsubsection{2.5 Dualité~: approche générale}

Cette section ne figure pas au programme des classes préparatoires. Elle
reprend les définitions et les résultats de la section précédente en les
généralisant.

\paragraph{2.5.1 Notion de dual. Orthogonalité}

Définition~2.5.1 Soit E un K-espace vectoriel . On appelle forme
linéaire sur E toute application linéaire de E dans K. On appelle dual
de E le K-espace vectoriel E^∗ = L(E,K).

Remarque~2.5.1 On dispose d'une application bilinéaire de
E^∗\times E dans K donnée par \langle
f∣x\rangle = f(x) appelée la
forme bilinéaire canonique. A cette forme bilinéaire est associée une
notion d'orthogonalité. On notera donc

\begin{itemize}
\itemsep1pt\parskip0pt\parsep0pt
\item
  (i) si A \subset~ E, A^\bot = \f \in
  E^∗∣\forall~~x
  \in A, f(x) = 0\
\item
  (ii) si B \subset~ E^∗, B^o = \x \in
  E∣\forall~~f \in B, f(x) =
  0\
\end{itemize}

Proposition~2.5.1 Les notations A,A\_1,A\_2 désignant
des parties de E et B,B\_1,B\_2 désignant des parties de
E^∗, on a

\begin{itemize}
\itemsep1pt\parskip0pt\parsep0pt
\item
  (i) A^\bot et B^o sont des sous-espaces vectoriels
  de E^∗ et E~; A^\bot =\
  \mathrmVect(A)^\bot et B^o
  =\
  \mathrmVect(B)^o
\item
  (ii) A\_1 \subset~ A\_2 \rigtharrow~ A\_1^\bot⊃
  A\_2^\bot et B\_1 \subset~ B\_2 \rigtharrow~
  B\_1^o ⊃ B\_2^o
\item
  (iii) A \subset~ (A^\bot)^o et B \subset~
  (B^o)^\bot
\item
  (iv) Soit A un sous-espace vectoriel de E, alors A^\bot =
  \0\ \Leftrightarrow A =
  E et A^\bot = E^∗\Leftrightarrow A =
  \0\.
\item
  (v) Soit B un sous-espace vectoriel de E^∗, alors
  B^o = E \Leftrightarrow B =
  \0\.
\end{itemize}

Démonstration Les propriétés (i),(ii) et (iii) sont évidentes ainsi que
les parties '' ⇐'' de (iv) et (v).

Montrons donc que A\neq~E \rigtharrow~
A^\bot\neq~\0\.
Soit (e\_i)\_i\inI une base de A que l'on complète en
(e\_i)\_i\inJ base de E. Soit i\_0 \in J \diagdown I et f
l'application qui à x associe sa i\_0-ième coordonnée dans la
base. On a f\neq~0 et f \in A^\bot.

Montrons maintenant que
A\neq~\0\ \rigtharrow~
A^\bot\neq~E^∗. Pour cela soit
x \in A \diagdown\0\. On complète x en une base
(e\_i)\_i\inI de E avec x = e\_i\_0. Soit
f l'application qui à x associe sa i\_0-ième coordonnée dans la
base. On a f(x)\neq~0, donc
f∉A^\bot.

Montrons maintenant que
B\neq~\0\ \rigtharrow~
B^o\neq~E. Soit f \in B
\diagdown\0\. On a
f\neq~0, donc \exists~x \in E,
f(x)\neq~0. Dans ce cas
x∉B^o, ce qui achève la
démonstration.

On prendra garde qu'on peut avoir B^o =
\0\ avec
B\neq~E^∗ (prendre par exemple E =
\mathbb{R}~{[}X{]} et B =\
\mathrmVect(\epsilon\_x,x \in ℤ) où \epsilon\_x(P)
= P(x)~; on a B^o = \0\
alors que \epsilon\_1\diagup2∉B).

\paragraph{2.5.2 Hyperplans}

Définition~2.5.2 On appelle hyperplan de E tout sous-espace vectoriel H
de E vérifiant les conditions équivalentes

\begin{itemize}
\itemsep1pt\parskip0pt\parsep0pt
\item
  (i) dim~ E\diagupH = 1
\item
  (ii) \exists~f \in
  E^∗\diagdown\0\, H
  = \mathrmKer~f
\item
  (iii) H admet une droite comme supplémentaire.
\end{itemize}

Démonstration (i) \rigtharrow~(ii)~: prendre \overlinee une base
de E\diagupH et écrire \pi~(x) = f(x)\overlinee.

(ii) \rigtharrow~ (iii)~: on prend a \in E tel que f(a)\neq~0.
Tout élément x s'écrit de manière unique sous la forme x = (x - f(x)
\over f(a) a) + f(x) \over f(a) a
avec x - f(x) \over f(a) a
\in\mathrmKer~f, soit E
= \mathrmKer~f \oplus~ Ka.

(iii) \rigtharrow~(i)~: tout supplémentaire de H est isomorphe à E\diagupH.

Théorème~2.5.2 Soit H un hyperplan de E. Alors H^\bot est de
dimension 1 (droite vectorielle)~: deux formes linéaires nulles sur H
sont proportionnelles.

Démonstration Si E = H \oplus~ Ka et H =\
\mathrmKerf, soit g \in H^\bot. Alors g et 
g(a) \over f(a) f coïncident sur H et sur Ka, donc sont
égales.

\paragraph{2.5.3 Bidual}

Définition~2.5.3 On désigne par E^∗∗ le dual de
E^∗.

Remarque~2.5.2 Si E est de dimension finie, E^∗ aussi et
dim E^∗~ =\
dim E. On en déduit que E^∗∗ est aussi de dimension finie
encore égale à dim~ E.

Théorème~2.5.3 L'application u : E \rightarrow~ E^∗∗,
x\mapsto~u\_x définie par u\_x(f) =
f(x) est une application linéaire in\jmathective. Si E est un espace
vectoriel de dimension finie, c'est un isomorphisme d'espaces
vectoriels.

Démonstration En effet, cette application est visiblement linéaire et si
x \in\mathrmKer~u, on a

\forall~f \in E^∗, f(x) = u\_ x~(f) =
0(f) = 0

et donc x \in (E^∗)^o =
\0\~; elle est donc in\jmathective. Si E
est un espace vectoriel de dimension finie, on a une application
linéaire in\jmathective entre deux espaces de même dimension finie, elle est
donc bi\jmathective.

\paragraph{2.5.4 Transposée}

Définition~2.5.4 Soit u \in L(E,F). On note ^tu :
F^∗\rightarrow~ E^∗ définie par ^tu(g) = g \cdot u
(c'est une application linéaire).

Remarque~2.5.3 Cela revient à poser, pour x \in E et g \in F^∗,
\langle
^tu(g)∣x\rangle
\_E =\langle
g∣u(x)\rangle \_F.

Théorème~2.5.4 On a les propriétés suivantes

\begin{itemize}
\itemsep1pt\parskip0pt\parsep0pt
\item
  (i) u\mapsto~^tu est linéaire de L(E,F)
  dans L(F^∗,E^∗).
\item
  (ii) u \in L(E,F),v \in L(F,G)~; alors ^t(v \cdot u) =
  ^tu \cdot^tv
\item
  (iii) Si u est bi\jmathective, ^tu aussi et
  (^tu)^-1 = ^t(u^-1)
\item
  (iv)
  \mathrmKer^t~u
  =
  (\mathrmImu)^\bot~
\item
  (v)
  \mathrmIm^t~u =
  (\mathrmKeru)^\bot~
\end{itemize}

Démonstration (i) et (ii) sont très faciles à partir de la définition.
(iii) découle immédiatement de (ii) en écrivant que v \cdot u =
\mathrmId\_E et u \cdot v =
\mathrmId\_F.

Pour (iv), on a g
\in\mathrmKer^t~u
\Leftrightarrow \forall~~x \in E,
^tu(g)(x) = 0 \Leftrightarrow
\forall~~x \in E, g(u(x)) = 0
\Leftrightarrow g \in
(\mathrmImu)^\bot~.

Pour (v), on remarque d'abord que f
\in\mathrmIm^t~u
\rigtharrow~\existsg, f = g \cdot u \rigtharrow~\\forall~~x
\in\mathrmKer~u, f(x) = 0 \rigtharrow~ f
\in
(\mathrmKeru)^\bot~,
soit
\mathrmIm^t~u \subset~
(\mathrmKeru)^\bot~.
Inversement, soit f \in
(\mathrmKeru)^\bot~.
On définit g\_1 forme linéaire sur
\mathrmIm~u par
g\_1(y) = f(x) si y = u(x)~; on vérifie en effet que f(x) est
indépendant du choix de x tel que y = u(x) car f est nulle sur
\mathrmKer~u. Soit alors V
un supplémentaire de
\mathrmIm~u dans F. On
définit g : F \rightarrow~ K par g(y\_1 + y\_2) =
g\_1(y\_1) si y\_1
\in\mathrmImu et y\_2~
\in V . On a bien f = g \cdot u = ^tu(g). Donc
(\mathrmKeru)^\bot\subset~\\mathrmIm^t~u
et donc l'égalité.

\paragraph{2.5.5 Dualité en dimension finie}

Proposition~2.5.5 Soit E un espace vectoriel de dimension finie, \mathcal{E} =
(e\_1,\\ldots,e\_n~)
une base de E. La famille \mathcal{E}' =
(e\_1^∗,\\ldots,e\_n^∗~)
de E^∗ définie par e\_i^∗(e\_\jmath) =
\delta\_i^\jmath est une base de E^∗ appelée la base
duale de la base \mathcal{E}

Démonstration On vérifie en effet immédiatement qu'elle est libre et
elle a le bon cardinal.

Théorème~2.5.6 Soit E un espace vectoriel de dimension finie.
L'application \mathcal{E}\rightarrow~\mathcal{E}' est une bi\jmathection de l'ensemble des bases de E sur
l'ensemble des bases de E^∗.

Démonstration In\jmathectivité~: si
((e\_1,\\ldots,e\_n~)
et
(e\_1',\\ldots,e\_n~')
sont deux bases qui ont même base duale, on a pour toute f \in
E^∗, f(e\_i) = f(e\_i') et donc e\_i
= e\_i'. Sur\jmathectivité~: soit ℱ =
(f\_1,\\ldots,f\_n~)
une base de E^∗ et soit ℱ sa base duale (dans
E^∗∗). Soit u l'isomorphisme de E sur E^∗∗, et \mathcal{E} =
u^-1(ℱ'), base de E. On a alors f\_i(e\_\jmath) =
u\_e\_\jmath(f\_i) =
f\_\jmath^∗(f\_i) = \delta\_i^\jmath, donc ℱ est
la base duale de la base \mathcal{E}.

Corollaire~2.5.7 Soit E un espace vectoriel de dimension finie.

\begin{itemize}
\item
  (i) Soit A un sous-espace vectoriel de E. On a

  dim E =\ dim~ A
  + dim A^\bot~\text
  et (A^\bot)^o = A
\item
  (ii) Soit B un sous-espace vectoriel de E^∗. On a

  dim E =\ dim~ B
  + dim B^o~\text
  et (B^o)^\bot = B
\end{itemize}

Démonstration Soit
(e\_1,\\ldots,e\_p~)
une base de A que l'on complète en
(e\_1,\\ldots,e\_n~)
base de E. On vérifie immédiatement que A^\bot
=\
\mathrmVect(e\_p+1^∗,\\ldots,e\_n^∗~)
d'où le résultat sur la dimension. On montre de même le résultat sur la
dimension de B^o. Les égalités découlent alors des inclusions
et du fait que les espaces ont même dimension.

Corollaire~2.5.8 Soit E un espace vectoriel de dimension finie,
f\_1,\\ldots,f\_k~
\in E^∗, V = \x \in
E∣f\_1(x) =
\\ldots~ =
f\_k(x) = 0\. Alors
dim V =\ dim~ E
-\mathrmrg(f\_1,\\\ldots,f\_k~).

Théorème~2.5.9 Soit E et F deux espaces vectoriels de dimensions finies,
u \in L(E,F). Alors
\mathrmrg~u
= \mathrmrg^t~u.

Démonstration
\mathrmrg^t~u
= dim~
\mathrmIm^t~u
= dim~
(\mathrmKeru)^\bot~
= dim E -\ dim~
\mathrmKer~u
= \mathrmrg~u.

{[}
{[}
{[}
{[}

\end{document}
