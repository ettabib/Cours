\documentclass[]{article}
\usepackage[T1]{fontenc}
\usepackage{lmodern}
\usepackage{amssymb,amsmath}
\usepackage{ifxetex,ifluatex}
\usepackage{fixltx2e} % provides \textsubscript
% use upquote if available, for straight quotes in verbatim environments
\IfFileExists{upquote.sty}{\usepackage{upquote}}{}
\ifnum 0\ifxetex 1\fi\ifluatex 1\fi=0 % if pdftex
  \usepackage[utf8]{inputenc}
\else % if luatex or xelatex
  \ifxetex
    \usepackage{mathspec}
    \usepackage{xltxtra,xunicode}
  \else
    \usepackage{fontspec}
  \fi
  \defaultfontfeatures{Mapping=tex-text,Scale=MatchLowercase}
  \newcommand{\euro}{€}
\fi
% use microtype if available
\IfFileExists{microtype.sty}{\usepackage{microtype}}{}
\ifxetex
  \usepackage[setpagesize=false, % page size defined by xetex
              unicode=false, % unicode breaks when used with xetex
              xetex]{hyperref}
\else
  \usepackage[unicode=true]{hyperref}
\fi
\hypersetup{breaklinks=true,
            bookmarks=true,
            pdfauthor={},
            pdftitle={Endomorphismes et formes quadratiques},
            colorlinks=true,
            citecolor=blue,
            urlcolor=blue,
            linkcolor=magenta,
            pdfborder={0 0 0}}
\urlstyle{same}  % don't use monospace font for urls
\setlength{\parindent}{0pt}
\setlength{\parskip}{6pt plus 2pt minus 1pt}
\setlength{\emergencystretch}{3em}  % prevent overfull lines
\setcounter{secnumdepth}{0}
 
/* start css.sty */
.cmr-5{font-size:50%;}
.cmr-7{font-size:70%;}
.cmmi-5{font-size:50%;font-style: italic;}
.cmmi-7{font-size:70%;font-style: italic;}
.cmmi-10{font-style: italic;}
.cmsy-5{font-size:50%;}
.cmsy-7{font-size:70%;}
.cmex-7{font-size:70%;}
.cmex-7x-x-71{font-size:49%;}
.msbm-7{font-size:70%;}
.cmtt-10{font-family: monospace;}
.cmti-10{ font-style: italic;}
.cmbx-10{ font-weight: bold;}
.cmr-17x-x-120{font-size:204%;}
.cmsl-10{font-style: oblique;}
.cmti-7x-x-71{font-size:49%; font-style: italic;}
.cmbxti-10{ font-weight: bold; font-style: italic;}
p.noindent { text-indent: 0em }
td p.noindent { text-indent: 0em; margin-top:0em; }
p.nopar { text-indent: 0em; }
p.indent{ text-indent: 1.5em }
@media print {div.crosslinks {visibility:hidden;}}
a img { border-top: 0; border-left: 0; border-right: 0; }
center { margin-top:1em; margin-bottom:1em; }
td center { margin-top:0em; margin-bottom:0em; }
.Canvas { position:relative; }
li p.indent { text-indent: 0em }
.enumerate1 {list-style-type:decimal;}
.enumerate2 {list-style-type:lower-alpha;}
.enumerate3 {list-style-type:lower-roman;}
.enumerate4 {list-style-type:upper-alpha;}
div.newtheorem { margin-bottom: 2em; margin-top: 2em;}
.obeylines-h,.obeylines-v {white-space: nowrap; }
div.obeylines-v p { margin-top:0; margin-bottom:0; }
.overline{ text-decoration:overline; }
.overline img{ border-top: 1px solid black; }
td.displaylines {text-align:center; white-space:nowrap;}
.centerline {text-align:center;}
.rightline {text-align:right;}
div.verbatim {font-family: monospace; white-space: nowrap; text-align:left; clear:both; }
.fbox {padding-left:3.0pt; padding-right:3.0pt; text-indent:0pt; border:solid black 0.4pt; }
div.fbox {display:table}
div.center div.fbox {text-align:center; clear:both; padding-left:3.0pt; padding-right:3.0pt; text-indent:0pt; border:solid black 0.4pt; }
div.minipage{width:100%;}
div.center, div.center div.center {text-align: center; margin-left:1em; margin-right:1em;}
div.center div {text-align: left;}
div.flushright, div.flushright div.flushright {text-align: right;}
div.flushright div {text-align: left;}
div.flushleft {text-align: left;}
.underline{ text-decoration:underline; }
.underline img{ border-bottom: 1px solid black; margin-bottom:1pt; }
.framebox-c, .framebox-l, .framebox-r { padding-left:3.0pt; padding-right:3.0pt; text-indent:0pt; border:solid black 0.4pt; }
.framebox-c {text-align:center;}
.framebox-l {text-align:left;}
.framebox-r {text-align:right;}
span.thank-mark{ vertical-align: super }
span.footnote-mark sup.textsuperscript, span.footnote-mark a sup.textsuperscript{ font-size:80%; }
div.tabular, div.center div.tabular {text-align: center; margin-top:0.5em; margin-bottom:0.5em; }
table.tabular td p{margin-top:0em;}
table.tabular {margin-left: auto; margin-right: auto;}
div.td00{ margin-left:0pt; margin-right:0pt; }
div.td01{ margin-left:0pt; margin-right:5pt; }
div.td10{ margin-left:5pt; margin-right:0pt; }
div.td11{ margin-left:5pt; margin-right:5pt; }
table[rules] {border-left:solid black 0.4pt; border-right:solid black 0.4pt; }
td.td00{ padding-left:0pt; padding-right:0pt; }
td.td01{ padding-left:0pt; padding-right:5pt; }
td.td10{ padding-left:5pt; padding-right:0pt; }
td.td11{ padding-left:5pt; padding-right:5pt; }
table[rules] {border-left:solid black 0.4pt; border-right:solid black 0.4pt; }
.hline hr, .cline hr{ height : 1px; margin:0px; }
.tabbing-right {text-align:right;}
span.TEX {letter-spacing: -0.125em; }
span.TEX span.E{ position:relative;top:0.5ex;left:-0.0417em;}
a span.TEX span.E {text-decoration: none; }
span.LATEX span.A{ position:relative; top:-0.5ex; left:-0.4em; font-size:85%;}
span.LATEX span.TEX{ position:relative; left: -0.4em; }
div.float img, div.float .caption {text-align:center;}
div.figure img, div.figure .caption {text-align:center;}
.marginpar {width:20%; float:right; text-align:left; margin-left:auto; margin-top:0.5em; font-size:85%; text-decoration:underline;}
.marginpar p{margin-top:0.4em; margin-bottom:0.4em;}
.equation td{text-align:center; vertical-align:middle; }
td.eq-no{ width:5%; }
table.equation { width:100%; } 
div.math-display, div.par-math-display{text-align:center;}
math .texttt { font-family: monospace; }
math .textit { font-style: italic; }
math .textsl { font-style: oblique; }
math .textsf { font-family: sans-serif; }
math .textbf { font-weight: bold; }
.partToc a, .partToc, .likepartToc a, .likepartToc {line-height: 200%; font-weight:bold; font-size:110%;}
.chapterToc a, .chapterToc, .likechapterToc a, .likechapterToc, .appendixToc a, .appendixToc {line-height: 200%; font-weight:bold;}
.index-item, .index-subitem, .index-subsubitem {display:block}
.caption td.id{font-weight: bold; white-space: nowrap; }
table.caption {text-align:center;}
h1.partHead{text-align: center}
p.bibitem { text-indent: -2em; margin-left: 2em; margin-top:0.6em; margin-bottom:0.6em; }
p.bibitem-p { text-indent: 0em; margin-left: 2em; margin-top:0.6em; margin-bottom:0.6em; }
.paragraphHead, .likeparagraphHead { margin-top:2em; font-weight: bold;}
.subparagraphHead, .likesubparagraphHead { font-weight: bold;}
.quote {margin-bottom:0.25em; margin-top:0.25em; margin-left:1em; margin-right:1em; text-align:justify;}
.verse{white-space:nowrap; margin-left:2em}
div.maketitle {text-align:center;}
h2.titleHead{text-align:center;}
div.maketitle{ margin-bottom: 2em; }
div.author, div.date {text-align:center;}
div.thanks{text-align:left; margin-left:10%; font-size:85%; font-style:italic; }
div.author{white-space: nowrap;}
.quotation {margin-bottom:0.25em; margin-top:0.25em; margin-left:1em; }
h1.partHead{text-align: center}
.sectionToc, .likesectionToc {margin-left:2em;}
.subsectionToc, .likesubsectionToc {margin-left:4em;}
.subsubsectionToc, .likesubsubsectionToc {margin-left:6em;}
.frenchb-nbsp{font-size:75%;}
.frenchb-thinspace{font-size:75%;}
.figure img.graphics {margin-left:10%;}
/* end css.sty */

\title{Endomorphismes et formes quadratiques}
\author{}
\date{}

\begin{document}
\maketitle

\textbf{Warning: \href{http://www.math.union.edu/locate/jsMath}{jsMath}
requires JavaScript to process the mathematics on this page.\\ If your
browser supports JavaScript, be sure it is enabled.}

\begin{center}\rule{3in}{0.4pt}\end{center}

{[}\href{coursse72.html}{next}{]} {[}\href{coursse70.html}{prev}{]}
{[}\href{coursse70.html\#tailcoursse70.html}{prev-tail}{]}
{[}\hyperref[tailcoursse71.html]{tail}{]}
{[}\href{coursch13.html\#coursse71.html}{up}{]}

\subsubsection{12.5 Endomorphismes et formes quadratiques}

\paragraph{12.5.1 Notion d'adjoint}

Soit E un K-espace vectoriel , Φ une forme quadratique non dégénérée sur
E de forme polaire φ.

Définition~12.5.1 Soit u,v ∈ L(E). On dit que u et v sont des
endomorphismes adjoints si

\textbackslash{}mathop\{∀\}x,y ∈ E, φ(u(x),y) = φ(x,v(y))

Remarque~12.5.1 La symétrie de φ montre clairement que u et v jouent des
rôles symétriques, donc que u est adjoint de v si et seulement si~v est
adjoint de u.

Proposition~12.5.1 Si u ∈ L(E) admet un adjoint, celui-ci est unique.

Démonstration Si \{v\}\_\{1\} et \{v\}\_\{2\} sont adjoints de u, on a
\textbackslash{}mathop\{∀\}x,y ∈ E, φ(u(x),y) = φ(x,\{v\}\_\{1\}(y)) =
φ(x,\{v\}\_\{2\}(y)). On a donc \textbackslash{}mathop\{∀\}x,y ∈ E,
φ(\{v\}\_\{1\}(y) − \{v\}\_\{2\}(y),x) = 0, donc pour y ∈ E,
\{v\}\_\{1\}(y) − \{v\}\_\{2\}(y)
∈\textbackslash{}mathop\{\textbackslash{}mathrm\{Ker\}\}φ =
\textbackslash{}\{0\textbackslash{}\} et donc \{v\}\_\{1\} =
\{v\}\_\{2\}.

Définition~12.5.2 Lorsque u ∈ L(E) admet un adjoint, nous le noterons
\{u\}\^{}\{∗\} et nous noterons \{L\}\^{}\{∗\}(E) l'ensemble des
endomorphismes de E qui admettent un adjoint. Il est clair que
\{\textbackslash{}mathrm\{Id\}\}\_\{E\} appartient à \{L\}\^{}\{∗\}(E)
et que \{\textbackslash{}mathrm\{Id\}\}\^{}\{∗\} =
\textbackslash{}mathrm\{Id\}.

Proposition~12.5.2 \{L\}\^{}\{∗\}(E) est un sous-espace vectoriel de
L(E). L'application u\textbackslash{}mathrel\{↦\}\{u\}\^{}\{∗\} est un
endomorphisme involutif de \{L\}\^{}\{∗\}(E). Si u,v ∈
\{L\}\^{}\{∗\}(E), alors u ∘ v aussi et \{(u ∘ v)\}\^{}\{∗\} =
\{v\}\^{}\{∗\}∘ \{u\}\^{}\{∗\}.

Démonstration On a déjà vu que la relation u et v sont adjoints était
symétrique, donc si u ∈ \{L\}\^{}\{∗\}(E), \{u\}\^{}\{∗\} aussi et
\{u\}\^{}\{∗∗\} = u. Si u,v ∈ \{L\}\^{}\{∗\}(E), α,β ∈ K, on a

\textbackslash{}begin\{eqnarray*\} φ((αu + βv)(x),y)\& =\& φ(αu(x) +
βv(x),y) \%\& \textbackslash{}\textbackslash{} \& =\& αφ(u(x),y) +
βφ(v(x),y) \%\& \textbackslash{}\textbackslash{} \& =\&
αφ(x,\{u\}\^{}\{∗\}(y)) + βφ(x,\{v\}\^{}\{∗\}(y))\%\&
\textbackslash{}\textbackslash{} \& =\& φ(x,(α\{u\}\^{}\{∗\} +
β\{v\}\^{}\{∗\})(y)) \%\& \textbackslash{}\textbackslash{}
\textbackslash{}end\{eqnarray*\}

ce qui montre que αu + βv ∈ \{L\}\^{}\{∗\}(E) et que \{(αu +
βv)\}\^{}\{∗\} = α\{u\}\^{}\{∗\} + β\{v\}\^{}\{∗\}~; donc
\{L\}\^{}\{∗\}(E) est un sous-espace vectoriel de L(E) et
u\textbackslash{}mathrel\{↦\}\{u\}\^{}\{∗\} est linéaire. Si u,v ∈
\{L\}\^{}\{∗\}(E), on a

φ(u ∘ v(x),y) = φ(v(x),\{u\}\^{}\{∗\}(y)) = φ(x,\{v\}\^{}\{∗\}∘
\{u\}\^{}\{∗\}(y))

ce qui montre que u ∘ v admet \{v\}\^{}\{∗\}∘ \{u\}\^{}\{∗\} comme
adjoint.

Une des propriétés essentielles de l'adjoint que nous utiliserons de
fa\textbackslash{}c\{c\}on assez systématique pour la réduction des
endomorphismes est la suivante

Théorème~12.5.3 Soit u ∈ \{L\}\^{}\{∗\}(E). Soit F un sous-espace de E
stable par u~; alors \{F\}\^{}\{⊥\} est stable par \{u\}\^{}\{∗\}.

Démonstration Soit x ∈ \{F\}\^{}\{⊥\}. Si y ∈ F, on a
φ(\{u\}\^{}\{∗\}(x),y) = φ(x,u(y)) = 0 puisque u(y) ∈ F et x ∈
\{F\}\^{}\{⊥\}. Donc \{u\}\^{}\{∗\}(x) ∈ \{F\}\^{}\{⊥\} et
\{F\}\^{}\{⊥\} est stable par \{u\}\^{}\{∗\}.

Définition~12.5.3 On dit que u ∈ L(E) est symétrique ou autoadjoint si
\{u\}\^{}\{∗\} = u. On dit que u est antisymétrique si \{u\}\^{}\{∗\} =
−u.

Proposition~12.5.4 L'espace \{L\}\^{}\{∗\}(E) est somme directe du
sous-espace des endomorphismes symétriques et du sous-espace des
endomorphismes antisymétriques.

Démonstration L'endomorphisme de \{L\}\^{}\{∗\}(E),
u\textbackslash{}mathrel\{↦\}\{u\}\^{}\{∗\} étant involutif, l'espace
\{L\}\^{}\{∗\}(E) est somme directe du sous-espace propre associé à la
valeur propre 1 (les endomorphismes symétriques) et du sous-espace
propre associé à la valeur propre -1 (les endomorphismes
antisymétriques).

\paragraph{12.5.2 Adjoint en dimension finie}

Soit E un K-espace vectoriel ~de dimension finie, Φ une forme
quadratique non dégénérée sur E de forme polaire φ.

Théorème~12.5.5 Tout endomorphisme de E admet un unique adjoint
\{u\}\^{}\{∗\} (autrement dit \{L\}\^{}\{∗\}(E) = L(E)). Si u ∈ L(E), ℰ
une base de E, Ω =\textbackslash{}mathop\{
\textbackslash{}mathrm\{Mat\}\} (φ,ℰ) et A =\textbackslash{}mathop\{
\textbackslash{}mathrm\{Mat\}\} (u,ℰ), alors

\textbackslash{}mathop\{\textbackslash{}mathrm\{Mat\}\}
(\{u\}\^{}\{∗\},ℰ) = \{Ω\{\}\^{}\{−1\}\}\^{}\{t\}AΩ

Démonstration Soit ℰ une base de E et Ω =\textbackslash{}mathop\{
\textbackslash{}mathrm\{Mat\}\} (φ,ℰ). Comme φ est non dégénérée, la
matrice Ω est inversible. Soit u,v ∈ L(E), A =\textbackslash{}mathop\{
\textbackslash{}mathrm\{Mat\}\} (u,ℰ) et B =\textbackslash{}mathop\{
\textbackslash{}mathrm\{Mat\}\} (v,ℰ). Si x,y ∈ E, on a φ(u(x),y) \{=
\}\^{}\{t\}(AX)ΩY \{= \}\^{}\{t\}\{X\}\^{}\{t\}AΩY et φ(x,v(y)) \{=
\}\^{}\{t\}XΩBY . L'unicité de la matrice de la forme bilinéaire
(x,y)\textbackslash{}mathrel\{↦\}φ(u(x),y) montre que

\textbackslash{}begin\{eqnarray*\} \textbackslash{}mathop\{∀\}x,y ∈ E,
φ(u(x),y) = φ(x,v(y))\{\& \textbackslash{}mathrel\{⇔\} \& \}\^{}\{t\}AΩ
= ΩB \%\& \textbackslash{}\textbackslash{} \&
\textbackslash{}mathrel\{⇔\} \& B = \{Ω\{\}\^{}\{−1\}\}\^{}\{t\}AΩ\%\&
\textbackslash{}\textbackslash{} \textbackslash{}end\{eqnarray*\}

ce qui montre à la fois l'existence (et l'unicité) de l'adjoint et la
formule voulue.

Remarque~12.5.2 Si la base ℰ est orthonormée, alors Ω = \{I\}\_\{n\} et
\textbackslash{}mathop\{\textbackslash{}mathrm\{Mat\}\}
(\{u\}\^{}\{∗\},ℰ) \{= \}\^{}\{t\}\textbackslash{}mathop\{
\textbackslash{}mathrm\{Mat\}\} (u,ℰ)~; en particulier

Corollaire~12.5.6 Soit ℰ une base orthonormée de E~; alors u est
symétrique (resp. antisymétrique) si et seulement
si~\textbackslash{}mathop\{\textbackslash{}mathrm\{Mat\}\} (u,ℰ) est une
matrice symétrique (resp. antisymétrique).

Corollaire~12.5.7 Si u ∈ L(E) est inversible, alors \{u\}\^{}\{∗\} est
inversible et \{(\{u\}\^{}\{−1\})\}\^{}\{∗\} =
\{(\{u\}\^{}\{∗\})\}\^{}\{−1\}.

Démonstration On a \{u\}\^{}\{−1\} ∘ u =\{
\textbackslash{}mathrm\{Id\}\}\_\{E\} d'où \{(\{u\}\^{}\{−1\} ∘
u)\}\^{}\{∗\} =\{ \textbackslash{}mathrm\{Id\}\}\_\{E\}\^{}\{∗\}, soit
\{u\}\^{}\{∗\}∘ \{(\{u\}\^{}\{−1\})\}\^{}\{∗\} =\{
\textbackslash{}mathrm\{Id\}\}\_\{E\}. De même u ∘ \{u\}\^{}\{−1\} =\{
\textbackslash{}mathrm\{Id\}\}\_\{E\} donne par passage à l'adjoint
\{(\{u\}\^{}\{−1\})\}\^{}\{∗\}∘ \{u\}\^{}\{∗\} =\{
\textbackslash{}mathrm\{Id\}\}\_\{E\}. Ceci montre que \{u\}\^{}\{∗\}
est inversible et que \{(\{u\}\^{}\{−1\})\}\^{}\{∗\} =
\{(\{u\}\^{}\{∗\})\}\^{}\{−1\}

Corollaire~12.5.8
\textbackslash{}mathop\{\textbackslash{}mathrm\{det\}\} \{u\}\^{}\{∗\}
=\textbackslash{}mathop\{ \textbackslash{}mathrm\{det\}\} u,
\textbackslash{}mathop\{\textbackslash{}mathrm\{tr\}\}\{u\}\^{}\{∗\}
=\textbackslash{}mathop\{ \textbackslash{}mathrm\{tr\}\}u,
\{χ\}\_\{\{u\}\^{}\{∗\}\} = \{χ\}\_\{u\}.

Démonstration Soit ℰ une base de E, Ω =\textbackslash{}mathop\{
\textbackslash{}mathrm\{Mat\}\} (φ,ℰ) et A =\textbackslash{}mathop\{
\textbackslash{}mathrm\{Mat\}\} (u,ℰ), alors
\textbackslash{}mathop\{\textbackslash{}mathrm\{Mat\}\}
(\{u\}\^{}\{∗\},ℰ) = \{Ω\{\}\^{}\{−1\}\}\^{}\{t\}AΩ. On a donc
\textbackslash{}mathop\{\textbackslash{}mathrm\{det\}\} \{u\}\^{}\{∗\}
=\textbackslash{}mathop\{ \textbackslash{}mathrm\{det\}\}
\{Ω\{\}\^{}\{−1\}\}\^{}\{t\}AΩ =\{\textbackslash{}mathop\{
\textbackslash{}mathrm\{det\}\} \}\^{}\{t\}A =\textbackslash{}mathop\{
\textbackslash{}mathrm\{det\}\} A =\textbackslash{}mathop\{
\textbackslash{}mathrm\{det\}\} u. La démonstration est la même pour la
trace et pour le polynôme caractéristique.

Proposition~12.5.9 Soit E un espace euclidien, u ∈ L(E). Alors

\begin{itemize}
\itemsep1pt\parskip0pt\parsep0pt
\item
  (i)
  \textbackslash{}mathop\{\textbackslash{}mathrm\{Ker\}\}\{u\}\^{}\{∗\}
  =
  \{(\textbackslash{}mathop\{\textbackslash{}mathrm\{Im\}\}u)\}\^{}\{⊥\},
  \textbackslash{}mathop\{\textbackslash{}mathrm\{Im\}\}\{u\}\^{}\{∗\} =
  \{(\textbackslash{}mathop\{\textbackslash{}mathrm\{Ker\}\}u)\}\^{}\{⊥\}
\item
  (ii)
  \textbackslash{}mathop\{\textbackslash{}mathrm\{Ker\}\}\{u\}\^{}\{∗\}u
  =\textbackslash{}mathop\{ \textbackslash{}mathrm\{Ker\}\}u et
  \textbackslash{}mathop\{\textbackslash{}mathrm\{Im\}\}\{u\}\^{}\{∗\}u
  =\textbackslash{}mathop\{ \textbackslash{}mathrm\{Im\}\}\{u\}\^{}\{∗\}
\end{itemize}

Démonstration (ii) On a

\textbackslash{}begin\{eqnarray*\} x
∈\textbackslash{}mathop\{\textbackslash{}mathrm\{Ker\}\}\{u\}\^{}\{∗\}\&
\textbackslash{}mathrel\{⇔\} \& \{u\}\^{}\{∗\}(x) = 0
\textbackslash{}mathrel\{⇔\} \textbackslash{}mathop\{∀\}y ∈ E,
(\{u\}\^{}\{∗\}(x)\textbackslash{}mathrel\{∣\}y) = 0 \%\&
\textbackslash{}\textbackslash{} \& \textbackslash{}mathrel\{⇔\} \&
\textbackslash{}mathop\{∀\}y ∈ E, (x\textbackslash{}mathrel\{∣\}u(y)) =
0 \textbackslash{}mathrel\{⇔\} x ∈
\{(\textbackslash{}mathop\{\textbackslash{}mathrm\{Im\}\}u)\}\^{}\{⊥\}\%\&
\textbackslash{}\textbackslash{} \textbackslash{}end\{eqnarray*\}

En appliquant ce résultat à \{u\}\^{}\{∗\} on obtient,
\textbackslash{}mathop\{\textbackslash{}mathrm\{Ker\}\}u =
\{(\textbackslash{}mathop\{\textbackslash{}mathrm\{Im\}\}\{u\}\^{}\{∗\})\}\^{}\{⊥\}
et en prenant l'orthogonal,
\textbackslash{}mathop\{\textbackslash{}mathrm\{Im\}\}\{u\}\^{}\{∗\} =
\{(\textbackslash{}mathop\{\textbackslash{}mathrm\{Ker\}\}u)\}\^{}\{⊥\}

(ii) On a visiblement u(x) = 0 ⇒ \{u\}\^{}\{∗\}u(x) = 0, donc
\textbackslash{}mathop\{\textbackslash{}mathrm\{Ker\}\}u
⊂\textbackslash{}mathop\{\textbackslash{}mathrm\{Ker\}\}\{u\}\^{}\{∗\}u~;
mais d'autre part, si x
∈\textbackslash{}mathop\{\textbackslash{}mathrm\{Ker\}\}\{u\}\^{}\{∗\}u,
on a

\textbackslash{}\textbar{}u\{(x)\textbackslash{}\textbar{}\}\^{}\{2\} =
(u(x)\textbackslash{}mathrel\{∣\}u(x)) =
(\{u\}\^{}\{∗\}u(x)\textbackslash{}mathrel\{∣\}x) =
(0\textbackslash{}mathrel\{∣\}x) = 0

et donc u(x) = 0, soit
\textbackslash{}mathop\{\textbackslash{}mathrm\{Ker\}\}\{u\}\^{}\{∗\}u
⊂\textbackslash{}mathop\{\textbackslash{}mathrm\{Ker\}\}u et l'égalité.
On en déduit alors que

\textbackslash{}mathop\{\textbackslash{}mathrm\{Im\}\}\{u\}\^{}\{∗\}u =
\{(\textbackslash{}mathop\{\textbackslash{}mathrm\{Ker\}\}\{(\{u\}\^{}\{∗\}u)\}\^{}\{∗\})\}\^{}\{⊥\}
=
\{(\textbackslash{}mathop\{\textbackslash{}mathrm\{Ker\}\}\{u\}\^{}\{∗\}u)\}\^{}\{⊥\}
=
\{(\textbackslash{}mathop\{\textbackslash{}mathrm\{Ker\}\}u)\}\^{}\{⊥\}
=\textbackslash{}mathop\{ \textbackslash{}mathrm\{Im\}\}\{u\}\^{}\{∗\}

\paragraph{12.5.3 Endomorphismes symétriques et formes quadratiques}

Soit E un K-espace vectoriel ~de dimension finie, Φ une forme
quadratique non dégénérée sur E de forme polaire φ. A tout endomorphisme
u de E, on peut associer la forme bilinéaire \{ψ\}\_\{u\} :
(x,y)\textbackslash{}mathrel\{↦\}φ(x,u(y)). Il est clair que u est
symétrique (resp. antisymétrique) si et seulement si~\{ψ\}\_\{u\} est
une forme bilinéaire symétrique (resp. antisymétrique).

Théorème~12.5.10 L'application u\textbackslash{}mathrel\{↦\}\{ψ\}\_\{u\}
est un isomorphisme d'espaces vectoriels de L(E) sur l'espace
\{L\}\_\{2\}(E) des formes bilinéaires sur E.

Démonstration Soit ℰ une base de E, Ω =\textbackslash{}mathop\{
\textbackslash{}mathrm\{Mat\}\} (φ,ℰ) et A =\textbackslash{}mathop\{
\textbackslash{}mathrm\{Mat\}\} (u,ℰ). Alors \{ψ\}\_\{u\}(x,y) =
φ(x,u(y)) \{= \}\^{}\{t\}XΩAY et donc
\textbackslash{}mathop\{\textbackslash{}mathrm\{Mat\}\} (\{ψ\}\_\{u\},ℰ)
= ΩA. Comme l'application A\textbackslash{}mathrel\{↦\}ΩA est un
isomorphisme, il en est de même de
u\textbackslash{}mathrel\{↦\}\{ψ\}\_\{u\}.

Remarque~12.5.3 Supposons que la base ℰ est orthonormée, si bien que Ω =
\{I\}\_\{n\}. Alors
\textbackslash{}mathop\{\textbackslash{}mathrm\{Mat\}\} (\{ψ\}\_\{u\},ℰ)
=\textbackslash{}mathop\{ \textbackslash{}mathrm\{Mat\}\} (u,ℰ). Une
matrice carrée est à la fois la matrice d'un endomorphisme u de E et
d'une forme bilinéaire \{ψ\}\_\{u\} sur E. Mais le lecteur prendra garde
au fait que les formules de changement de bases ne sont évidemment pas
les mêmes pour l'endomorphisme
(A\textbackslash{}mathrel\{↦\}\{P\}\^{}\{−1\}AP) et pour la forme
bilinéaire (A\{\textbackslash{}mathrel\{↦\}\}\^{}\{t\}PAP).

\paragraph{12.5.4 Groupe orthogonal}

Soit E un K-espace vectoriel ~de dimension finie, Φ une forme
quadratique non dégénérée sur E de forme polaire φ.

Définition~12.5.4 On dit que u ∈ L(E) est un endomorphisme orthogonal si
on a les propriétés équivalentes

\begin{itemize}
\itemsep1pt\parskip0pt\parsep0pt
\item
  (i) \textbackslash{}mathop\{∀\}x ∈ E, Φ(u(x)) = Φ(x)
\item
  (ii) \textbackslash{}mathop\{∀\}x,y ∈ E, φ(u(x),u(y)) = φ(x,y)
\item
  (iii) u est inversible et \{u\}\^{}\{−1\} = \{u\}\^{}\{∗\}
\item
  (iv) u ∘ \{u\}\^{}\{∗\} =\{ \textbackslash{}mathrm\{Id\}\}\_\{E\}
\item
  (v) \{u\}\^{}\{∗\}∘ u =\{ \textbackslash{}mathrm\{Id\}\}\_\{E\}
\end{itemize}

Démonstration (ii) ⇒(i) est évident (faire y = x). (i) ⇒(ii) provient de
l'identité de polarisation pour Φ et de la linéarité de u

\textbackslash{}begin\{eqnarray*\} φ(u(x),u(y))\& =\&\{ 1
\textbackslash{}over 2\} (Φ(u(x) + u(y)) − Φ(u(x)) − Φ(u(y)))\%\&
\textbackslash{}\textbackslash{} \& =\&\{ 1 \textbackslash{}over 2\}
(Φ(u(x + y)) − Φ(u(x)) − Φ(u(y))) \%\& \textbackslash{}\textbackslash{}
\& =\&\{ 1 \textbackslash{}over 2\} (Φ(x + y) − Φ(x) − Φ(y)) = φ(x,y)
\%\& \textbackslash{}\textbackslash{} \textbackslash{}end\{eqnarray*\}

Pour un endomorphisme d'un espace vectoriel de dimension finie, on sait
que l'inversibilité est équivalente à l'inversibilité à gauche ou à
droite. On a donc (iii) \textbackslash{}mathrel\{⇔\} (iv)
\textbackslash{}mathrel\{⇔\} (v). Supposons (ii) vérifié. Alors φ(x,y) =
φ(u(x),u(y)) = φ(x,\{u\}\^{}\{∗\}∘ u(y)), ce qui montre (puisque φ est
non dégénérée) que \{u\}\^{}\{∗\}∘ u =\{
\textbackslash{}mathrm\{Id\}\}\_\{E\}~; donc (ii) ⇒(v). De même (v)
⇒(ii) puisque φ(u(x),u(y)) = φ(x,\{u\}\^{}\{∗\}∘ u(y)).

Remarque~12.5.4 La définition peut s'étendre au cas de la dimension
infinie, à condition d'imposer a priori que u est inversible.

Théorème~12.5.11 L'ensemble \{O\}\_\{Φ\}(E) des endomorphismes
orthogonaux de E est un sous groupe de (GL(E),∘). Pour tout
endomorphisme orthogonal u de E, on a
\textbackslash{}mathop\{\textbackslash{}mathrm\{det\}\} u = ±1.
L'ensemble S\{O\}\_\{Φ\}(E) des endomorphismes orthogonaux de
déterminant 1 est un sous-groupe distingué de \{O\}\_\{Φ\}(E) dont les
éléments sont appelés des rotations.

Démonstration On a clairement \{\textbackslash{}mathrm\{Id\}\}\_\{E\} ∈
\{O\}\_\{Φ\}(E). La définition (i) montre évidemment que si u et v sont
orthogonaux, il en est de même de u ∘ v. De plus, soit u ∈
\{O\}\_\{Φ\}(E)~; on a Φ(\{u\}\^{}\{−1\}(x)) = Φ(u(\{u\}\^{}\{−1\}(x)))
= Φ(x) ce qui montre que \{u\}\^{}\{−1\} ∈ \{O\}\_\{Φ\}(E). Donc
\{O\}\_\{Φ\}(E) est un sous-groupe de (GL(E),∘). On a alors 1
=\textbackslash{}mathop\{ \textbackslash{}mathrm\{det\}\}
\{\textbackslash{}mathrm\{Id\}\}\_\{E\} =\textbackslash{}mathop\{
\textbackslash{}mathrm\{det\}\} (\{u\}\^{}\{∗\}∘ u)
=\textbackslash{}mathop\{ \textbackslash{}mathrm\{det\}\}
\{u\}\^{}\{∗\}\textbackslash{}mathop\{\textbackslash{}mathrm\{det\}\} u
= \{(\textbackslash{}mathop\{\textbackslash{}mathrm\{det\}\}
u)\}\^{}\{2\}, soit
\textbackslash{}mathop\{\textbackslash{}mathrm\{det\}\} u = ±1.
L'application \{O\}\_\{Φ\}(E) →\textbackslash{}\{−
1,1\textbackslash{}\},
u\textbackslash{}mathrel\{↦\}\textbackslash{}mathop\{\textbackslash{}mathrm\{det\}\}
u est un morphisme de groupes multiplicatifs~; son noyau
S\{O\}\_\{Φ\}(E) est donc un sous-groupe distingué.

Théorème~12.5.12 On suppose qu'il existe dans E des bases orthonormées.
Soit u ∈ L(E).

\begin{itemize}
\itemsep1pt\parskip0pt\parsep0pt
\item
  (i) Si u est orthogonal, il envoie toute base orthonormée sur une base
  orthonormée.
\item
  (ii) Inversement, s'il existe une base orthonormée ℰ de E telle que
  u(ℰ) soit encore orthonormée, alors u est un endomorphisme orthogonal.
\end{itemize}

Démonstration (i) On a φ(u(\{e\}\_\{i\}),u(\{e\}\_\{j\})) =
φ(\{e\}\_\{i\},\{e\}\_\{j\}) = \{δ\}\_\{i\}\^{}\{j\}.

(ii) Soit x =\textbackslash{}mathop\{ \textbackslash{}mathop\{∑ \}\}
\{x\}\_\{i\}\{e\}\_\{i\} ∈ E. On a Φ(x) =\textbackslash{}mathop\{
\textbackslash{}mathop\{∑ \}\} \{x\}\_\{i\}\^{}\{2\}. Mais on a aussi
u(x) =\textbackslash{}mathop\{ \textbackslash{}mathop\{∑ \}\}
\{x\}\_\{i\}u(\{e\}\_\{i\}) et comme u(ℰ) est orthonormée, Φ(u(x))
=\textbackslash{}mathop\{ \textbackslash{}mathop\{∑ \}\}
\{x\}\_\{i\}\^{}\{2\}~; on a donc \textbackslash{}mathop\{∀\}x ∈ E,
Φ(u(x)) = Φ(x).

Théorème~12.5.13 Soit u un endomorphisme orthogonal et F un sous-espace
de E stable par u. Alors \{F\}\^{}\{⊥\} est stable par u.

Démonstration On a u(F) ⊂ F et comme u est inversible, on a
\textbackslash{}mathop\{dim\} u(F) =\textbackslash{}mathop\{ dim\} F. On
a donc u(F) = F. Soit donc x ∈ \{F\}\^{}\{⊥\} et y ∈ F~; il existe z ∈ F
tel que u(z) = y, d'où φ(u(x),y) = φ(u(x),u(z)) = φ(x,z) = 0, et donc
u(x) ∈ \{F\}\^{}\{⊥\}.

\paragraph{12.5.5 Matrices orthogonales}

Proposition~12.5.14 Soit E un K-espace vectoriel ~de dimension finie, Φ
une forme quadratique non dégénérée sur E de forme polaire φ. Soit u ∈
L(E), ℰ une base de E, Ω =\textbackslash{}mathop\{
\textbackslash{}mathrm\{Mat\}\} (φ,ℰ) et A =\textbackslash{}mathop\{
\textbackslash{}mathrm\{Mat\}\} (u,ℰ). Alors u est un endomorphisme
orthogonal si et seulement si~\{\}\^{}\{t\}AΩA = Ω.

Démonstration On a φ(u(x),u(y)) \{= \}\^{}\{t\}(AX)Ω(AY ) \{=
\}\^{}\{t\}\{X\}\^{}\{t\}AΩAY . L'unicité de la matrice d'une forme
bilinéaire montre que

\textbackslash{}mathop\{∀\}x,y ∈ E, φ(u(x),u(y)) = φ(x,y)\{
\textbackslash{}mathrel\{⇔\} \}\^{}\{t\}AΩA = Ω

En particulier, si ℰ est une base orthonormée de E, u est un
endomorphisme orthogonal si et seulement si~\{\}\^{}\{t\}AA =
\{I\}\_\{n\}. Ceci conduit à la définition suivante

Définition~12.5.5 Soit A ∈ \{M\}\_\{K\}(n)~; On dit que A est une
matrice orthogonale si elle vérifie les conditions équivalentes

\begin{itemize}
\itemsep1pt\parskip0pt\parsep0pt
\item
  (i) A est inversible et \{A\}\^{}\{−1\} \{= \}\^{}\{t\}A
\item
  (ii) \{\}\^{}\{t\}AA = \{I\}\_\{n\}
\item
  (iii) \{A\}\^{}\{t\}A = \{I\}\_\{n\}
\end{itemize}

Théorème~12.5.15 L'ensemble \{O\}\_\{K\}(n) des matrices carrées
orthogonales d'ordre n est un sous groupe de (G\{L\}\_\{K\}(n),.). Pour
toute matrice orthogonale A, on a
\textbackslash{}mathop\{\textbackslash{}mathrm\{det\}\} A = ±1.
L'ensemble S\{O\}\_\{K\}(n) des matrices orthogonales de déterminant 1
est un sous-groupe distingué de \{O\}\_\{K\}(n) dont les éléments sont
appelés des matrices de rotations.

Démonstration On a clairement \{I\}\_\{n\} ∈ \{O\}\_\{K\}(n). La
définition (i) montre évidemment que si A et B sont orthogonales, il en
est de même de AB. De plus, soit A ∈ \{O\}\_\{K\}(n)~; on a
\{A\{\}\^{}\{−1\}\}\^{}\{t\}(\{A\}\^{}\{−1\}) =
\{A\{\}\^{}\{−1\}\}\^{}\{t\}\{(\}\^{}\{t\}A) = \{A\}\^{}\{−1\}A =
\{I\}\_\{n\} ce qui montre que \{A\}\^{}\{−1\} ∈ \{O\}\_\{K\}(n). Donc
\{O\}\_\{K\}(n) est un sous-groupe de (G\{L\}\_\{K\}(n),.). On a alors 1
=\textbackslash{}mathop\{ \textbackslash{}mathrm\{det\}\} \{I\}\_\{n\}
=\textbackslash{}mathop\{ \textbackslash{}mathrm\{det\}\}
\{(\}\^{}\{t\}AA) =
\{(\textbackslash{}mathop\{\textbackslash{}mathrm\{det\}\}
A)\}\^{}\{2\}, soit
\textbackslash{}mathop\{\textbackslash{}mathrm\{det\}\} A = ±1.
L'application \{O\}\_\{K\}(n) →\textbackslash{}\{−
1,1\textbackslash{}\},
A\textbackslash{}mathrel\{↦\}\textbackslash{}mathop\{\textbackslash{}mathrm\{det\}\}
A est un morphisme de groupes multiplicatifs~; son noyau
S\{O\}\_\{K\}(n) est donc un sous-groupe distingué.

Dans ce paragraphe, on munira \{K\}\^{}\{n\} de la forme bilinéaire
symétrique naturelle (qui rend la base canonique orthonormée),
c'est-à-dire que l'on posera (x\textbackslash{}mathrel\{∣\}y)
=\{\textbackslash{}mathop\{ \textbackslash{}mathop\{∑ \}\}
\}\_\{i=1\}\^{}\{n\}\{x\}\_\{i\}\{y\}\_\{i\}

Théorème~12.5.16 Une matrice A ∈ \{M\}\_\{K\}(n) est orthogonale si et
seulement si~ses vecteurs colonnes (resp. lignes) forment une base
orthonormée de \{K\}\^{}\{n\}.

Démonstration Soit
(\{c\}\_\{1\},\textbackslash{}mathop\{\textbackslash{}mathop\{\ldots{}\}\},\{c\}\_\{n\})
les vecteurs colonnes de A,
(\{l\}\_\{1\},\textbackslash{}mathop\{\textbackslash{}mathop\{\ldots{}\}\},\{l\}\_\{n\})
ses vecteurs lignes. On a

\textbackslash{}begin\{eqnarray*\} A ∈ \{O\}\_\{K\}(n)\{\&
\textbackslash{}mathrel\{⇔\} \& \}\^{}\{t\}AA = \{I\}\_\{ n\}
\textbackslash{}mathrel\{⇔\} \textbackslash{}mathop\{∀\}i,j,
\{\{(\}\^{}\{t\}AA)\}\_\{ i,j\} = \{δ\}\_\{i\}\^{}\{j\}\%\&
\textbackslash{}\textbackslash{} \& \textbackslash{}mathrel\{⇔\} \&
\textbackslash{}mathop\{∀\}i,j, \{\textbackslash{}mathop\{∑
\}\}\_\{k=1\}\^{}\{n\}\{a\}\_\{ k,i\}\{a\}\_\{k,j\} =
\{δ\}\_\{i\}\^{}\{j\} \%\& \textbackslash{}\textbackslash{} \&
\textbackslash{}mathrel\{⇔\} \& \textbackslash{}mathop\{∀\}i,j,
(\{c\}\_\{i\}\textbackslash{}mathrel\{∣\}\{c\}\_\{j\}) =
\{δ\}\_\{i\}\^{}\{j\} \%\& \textbackslash{}\textbackslash{}
\textbackslash{}end\{eqnarray*\}

De la même fa\textbackslash{}c\{c\}on, en traduisant la relation
\{A\}\^{}\{t\}A = \{I\}\_\{n\}, on obtiendrait
(\{l\}\_\{i\}\textbackslash{}mathrel\{∣\}\{l\}\_\{j\}) =
\{δ\}\_\{i\}\^{}\{j\}.

Théorème~12.5.17 Soit E un K-espace vectoriel ~de dimension finie, Φ une
forme quadratique non dégénérée sur E de forme polaire φ. Soit ℰ une
base orthonormée de E, ℰ' une base de E. Alors on a équivalence de

\begin{itemize}
\itemsep1pt\parskip0pt\parsep0pt
\item
  (i) ℰ' est orthonormée
\item
  (ii) la matrice \{P\}\_\{ℰ\}\^{}\{ℰ'\} de passage de la base ℰ à la
  base ℰ' est orthogonale.
\end{itemize}

Démonstration On sait que \{P\}\_\{ℰ\}\^{}\{ℰ'\}
=\textbackslash{}mathop\{ \textbackslash{}mathrm\{Mat\}\} (u,ℰ) où u est
l'endomorphisme de E défini par \textbackslash{}mathop\{∀\}i,
u(\{e\}\_\{i\}) = \{e\}\_\{i\}'. Or d'après les résultats du paragraphe
précédent, u est un endomorphisme orthogonal si et seulement si~ℰ' est
orthonormée~; mais d'autre part, comme ℰ est orthonormée, u est
orthogonal si et seulement
si~\textbackslash{}mathop\{\textbackslash{}mathrm\{Mat\}\} (u,ℰ) est une
matrice orthogonale, d'où l'équivalence entre (i) et (ii).

{[}\href{coursse72.html}{next}{]} {[}\href{coursse70.html}{prev}{]}
{[}\href{coursse70.html\#tailcoursse70.html}{prev-tail}{]}
{[}\href{coursse71.html}{front}{]}
{[}\href{coursch13.html\#coursse71.html}{up}{]}

\end{document}
