\documentclass[]{article}
\usepackage[T1]{fontenc}
\usepackage{lmodern}
\usepackage{amssymb,amsmath}
\usepackage{ifxetex,ifluatex}
\usepackage{fixltx2e} % provides \textsubscript
% use upquote if available, for straight quotes in verbatim environments
\IfFileExists{upquote.sty}{\usepackage{upquote}}{}
\ifnum 0\ifxetex 1\fi\ifluatex 1\fi=0 % if pdftex
  \usepackage[utf8]{inputenc}
\else % if luatex or xelatex
  \ifxetex
    \usepackage{mathspec}
    \usepackage{xltxtra,xunicode}
  \else
    \usepackage{fontspec}
  \fi
  \defaultfontfeatures{Mapping=tex-text,Scale=MatchLowercase}
  \newcommand{\euro}{€}
\fi
% use microtype if available
\IfFileExists{microtype.sty}{\usepackage{microtype}}{}
\ifxetex
  \usepackage[setpagesize=false, % page size defined by xetex
              unicode=false, % unicode breaks when used with xetex
              xetex]{hyperref}
\else
  \usepackage[unicode=true]{hyperref}
\fi
\hypersetup{breaklinks=true,
            bookmarks=true,
            pdfauthor={},
            pdftitle={Espaces et parties connexes},
            colorlinks=true,
            citecolor=blue,
            urlcolor=blue,
            linkcolor=magenta,
            pdfborder={0 0 0}}
\urlstyle{same}  % don't use monospace font for urls
\setlength{\parindent}{0pt}
\setlength{\parskip}{6pt plus 2pt minus 1pt}
\setlength{\emergencystretch}{3em}  % prevent overfull lines
\setcounter{secnumdepth}{0}
 
/* start css.sty */
.cmr-5{font-size:50%;}
.cmr-7{font-size:70%;}
.cmmi-5{font-size:50%;font-style: italic;}
.cmmi-7{font-size:70%;font-style: italic;}
.cmmi-10{font-style: italic;}
.cmsy-5{font-size:50%;}
.cmsy-7{font-size:70%;}
.cmex-7{font-size:70%;}
.cmex-7x-x-71{font-size:49%;}
.msbm-7{font-size:70%;}
.cmtt-10{font-family: monospace;}
.cmti-10{ font-style: italic;}
.cmbx-10{ font-weight: bold;}
.cmr-17x-x-120{font-size:204%;}
.cmsl-10{font-style: oblique;}
.cmti-7x-x-71{font-size:49%; font-style: italic;}
.cmbxti-10{ font-weight: bold; font-style: italic;}
p.noindent { text-indent: 0em }
td p.noindent { text-indent: 0em; margin-top:0em; }
p.nopar { text-indent: 0em; }
p.indent{ text-indent: 1.5em }
@media print {div.crosslinks {visibility:hidden;}}
a img { border-top: 0; border-left: 0; border-right: 0; }
center { margin-top:1em; margin-bottom:1em; }
td center { margin-top:0em; margin-bottom:0em; }
.Canvas { position:relative; }
li p.indent { text-indent: 0em }
.enumerate1 {list-style-type:decimal;}
.enumerate2 {list-style-type:lower-alpha;}
.enumerate3 {list-style-type:lower-roman;}
.enumerate4 {list-style-type:upper-alpha;}
div.newtheorem { margin-bottom: 2em; margin-top: 2em;}
.obeylines-h,.obeylines-v {white-space: nowrap; }
div.obeylines-v p { margin-top:0; margin-bottom:0; }
.overline{ text-decoration:overline; }
.overline img{ border-top: 1px solid black; }
td.displaylines {text-align:center; white-space:nowrap;}
.centerline {text-align:center;}
.rightline {text-align:right;}
div.verbatim {font-family: monospace; white-space: nowrap; text-align:left; clear:both; }
.fbox {padding-left:3.0pt; padding-right:3.0pt; text-indent:0pt; border:solid black 0.4pt; }
div.fbox {display:table}
div.center div.fbox {text-align:center; clear:both; padding-left:3.0pt; padding-right:3.0pt; text-indent:0pt; border:solid black 0.4pt; }
div.minipage{width:100%;}
div.center, div.center div.center {text-align: center; margin-left:1em; margin-right:1em;}
div.center div {text-align: left;}
div.flushright, div.flushright div.flushright {text-align: right;}
div.flushright div {text-align: left;}
div.flushleft {text-align: left;}
.underline{ text-decoration:underline; }
.underline img{ border-bottom: 1px solid black; margin-bottom:1pt; }
.framebox-c, .framebox-l, .framebox-r { padding-left:3.0pt; padding-right:3.0pt; text-indent:0pt; border:solid black 0.4pt; }
.framebox-c {text-align:center;}
.framebox-l {text-align:left;}
.framebox-r {text-align:right;}
span.thank-mark{ vertical-align: super }
span.footnote-mark sup.textsuperscript, span.footnote-mark a sup.textsuperscript{ font-size:80%; }
div.tabular, div.center div.tabular {text-align: center; margin-top:0.5em; margin-bottom:0.5em; }
table.tabular td p{margin-top:0em;}
table.tabular {margin-left: auto; margin-right: auto;}
div.td00{ margin-left:0pt; margin-right:0pt; }
div.td01{ margin-left:0pt; margin-right:5pt; }
div.td10{ margin-left:5pt; margin-right:0pt; }
div.td11{ margin-left:5pt; margin-right:5pt; }
table[rules] {border-left:solid black 0.4pt; border-right:solid black 0.4pt; }
td.td00{ padding-left:0pt; padding-right:0pt; }
td.td01{ padding-left:0pt; padding-right:5pt; }
td.td10{ padding-left:5pt; padding-right:0pt; }
td.td11{ padding-left:5pt; padding-right:5pt; }
table[rules] {border-left:solid black 0.4pt; border-right:solid black 0.4pt; }
.hline hr, .cline hr{ height : 1px; margin:0px; }
.tabbing-right {text-align:right;}
span.TEX {letter-spacing: -0.125em; }
span.TEX span.E{ position:relative;top:0.5ex;left:-0.0417em;}
a span.TEX span.E {text-decoration: none; }
span.LATEX span.A{ position:relative; top:-0.5ex; left:-0.4em; font-size:85%;}
span.LATEX span.TEX{ position:relative; left: -0.4em; }
div.float img, div.float .caption {text-align:center;}
div.figure img, div.figure .caption {text-align:center;}
.marginpar {width:20%; float:right; text-align:left; margin-left:auto; margin-top:0.5em; font-size:85%; text-decoration:underline;}
.marginpar p{margin-top:0.4em; margin-bottom:0.4em;}
.equation td{text-align:center; vertical-align:middle; }
td.eq-no{ width:5%; }
table.equation { width:100%; } 
div.math-display, div.par-math-display{text-align:center;}
math .texttt { font-family: monospace; }
math .textit { font-style: italic; }
math .textsl { font-style: oblique; }
math .textsf { font-family: sans-serif; }
math .textbf { font-weight: bold; }
.partToc a, .partToc, .likepartToc a, .likepartToc {line-height: 200%; font-weight:bold; font-size:110%;}
.chapterToc a, .chapterToc, .likechapterToc a, .likechapterToc, .appendixToc a, .appendixToc {line-height: 200%; font-weight:bold;}
.index-item, .index-subitem, .index-subsubitem {display:block}
.caption td.id{font-weight: bold; white-space: nowrap; }
table.caption {text-align:center;}
h1.partHead{text-align: center}
p.bibitem { text-indent: -2em; margin-left: 2em; margin-top:0.6em; margin-bottom:0.6em; }
p.bibitem-p { text-indent: 0em; margin-left: 2em; margin-top:0.6em; margin-bottom:0.6em; }
.paragraphHead, .likeparagraphHead { margin-top:2em; font-weight: bold;}
.subparagraphHead, .likesubparagraphHead { font-weight: bold;}
.quote {margin-bottom:0.25em; margin-top:0.25em; margin-left:1em; margin-right:1em; text-align:justify;}
.verse{white-space:nowrap; margin-left:2em}
div.maketitle {text-align:center;}
h2.titleHead{text-align:center;}
div.maketitle{ margin-bottom: 2em; }
div.author, div.date {text-align:center;}
div.thanks{text-align:left; margin-left:10%; font-size:85%; font-style:italic; }
div.author{white-space: nowrap;}
.quotation {margin-bottom:0.25em; margin-top:0.25em; margin-left:1em; }
h1.partHead{text-align: center}
.sectionToc, .likesectionToc {margin-left:2em;}
.subsectionToc, .likesubsectionToc {margin-left:4em;}
.subsubsectionToc, .likesubsubsectionToc {margin-left:6em;}
.frenchb-nbsp{font-size:75%;}
.frenchb-thinspace{font-size:75%;}
.figure img.graphics {margin-left:10%;}
/* end css.sty */

\title{Espaces et parties connexes}
\author{}
\date{}

\begin{document}
\maketitle

\textbf{Warning: \href{http://www.math.union.edu/locate/jsMath}{jsMath}
requires JavaScript to process the mathematics on this page.\\ If your
browser supports JavaScript, be sure it is enabled.}

\begin{center}\rule{3in}{0.4pt}\end{center}

{[}\href{coursse25.html}{prev}{]}
{[}\href{coursse25.html\#tailcoursse25.html}{prev-tail}{]}
{[}\hyperref[tailcoursse26.html]{tail}{]}
{[}\href{coursch5.html\#coursse26.html}{up}{]}

\subsubsection{4.9 Espaces et parties connexes}

\paragraph{4.9.1 Notion de connexe}

Définition~4.9.1 Soit E un espace topologique. On dit que E n'est pas
connexe s'il vérifie les conditions équivalentes

\begin{itemize}
\itemsep1pt\parskip0pt\parsep0pt
\item
  (i) E est réunion de deux ouverts non vides disjoints
\item
  (ii) E est réunion de deux fermés non vides disjoints
\item
  (iii) il existe une partie de E distincte de ∅ et de E qui est à la
  fois ouverte et fermée dans E.
\end{itemize}

Démonstration (i) ⇒(ii). Si E = \{U\}\_\{1\} ∪ \{U\}\_\{2\} avec
\{U\}\_\{1\} et \{U\}\_\{2\} ouverts non vides disjoints, \{U\}\_\{1\}
et \{U\}\_\{2\} sont aussi fermés puisque \{U\}\_\{1\} = c\{U\}\_\{2\}
et \{U\}\_\{2\} = c\{U\}\_\{1\}, d'où la propriété (ii). On montre de
même que (ii) ⇒(i).

(i) ⇒(iii). Si E = \{U\}\_\{1\} ∪ \{U\}\_\{2\} avec \{U\}\_\{1\} et
\{U\}\_\{2\} ouverts non vides disjoints, alors \{U\}\_\{1\} est ouvert,
fermé (car \{U\}\_\{1\} = c\{U\}\_\{2\}), distinct de ∅ et de E.

(iii) ⇒(i). Si A est à la fois ouverte et fermée, distincte de ∅ et de
E, on écrit E = A ∪cA, avec cA ouvert (complémentaire d'un fermé), les
deux étant non vides et disjoints.

Définition~4.9.2 Soit E un espace topologique et F une partie de E. On
dit que F est connexe si F muni de la topologie induite est connexe.

\paragraph{4.9.2 Propriétés des connexes}

Théorème~4.9.1 Soit E,F deux espaces topologiques, f : E → F continue.
Si E est connexe, alors f(E) est une partie connexe de F.

Démonstration En effet si f(E) est réunion de deux ouverts non vides
disjoints \{U\}\_\{1\} et \{U\}\_\{2\} de f(E), alors E est réunion des
deux ouverts \{f\}\^{}\{−1\}(\{U\}\_\{1\}) et
\{f\}\^{}\{−1\}(\{U\}\_\{2\}) qui sont encore disjoints et non vides

Corollaire~4.9.2 Soit E,F deux espaces topologiques, f : E → F continue
et A une partie connexe de E. Alors f(A) est une partie connexe de F.

Démonstration En effet la restriction de f à A est encore continue, et
on peut lui appliquer le théorème précédent.

Proposition~4.9.3 Soit A une partie connexe de E. Alors toute partie B
telle que A ⊂ B ⊂\textbackslash{}overline\{A\} est connexe.

Démonstration En effet si B est réunion de deux ouverts de B non vides
disjoints \{U\}\_\{1\} et \{U\}\_\{2\}, alors A est réunion des deux
ouverts de A disjoints~: \{U\}\_\{1\} ∩ A et \{U\}\_\{2\} ∩ A~; or ces
deux ouverts sont non vides car, A étant dense dans B, tout ouvert non
vide de B rencontre A. C'est absurde. Donc \textbackslash{}overline\{A\}
est connexe.

Proposition~4.9.4 Soit \{(\{A\}\_\{i\})\}\_\{i∈I\} une famille de
parties connexes de E telle que
\{\textbackslash{}mathop\{\textbackslash{}mathop\{⋂ \}\}
\}\_\{i∈I\}\{A\}\_\{i\}\textbackslash{}mathrel\{≠\}∅. Alors
\{\textbackslash{}mathop\{\textbackslash{}mathop\{⋃ \}\}
\}\_\{i∈I\}\{A\}\_\{i\} est connexe.

Démonstration Soit a
∈\{\textbackslash{}mathop\{\textbackslash{}mathop\{⋂ \}\}
\}\_\{i∈I\}\{A\}\_\{i\}. Si A =\{\textbackslash{}mathop\{
\textbackslash{}mathop\{⋃ \}\} \}\_\{i∈I\}\{A\}\_\{i\} = \{U\}\_\{1\} ∪
\{U\}\_\{2\} avec \{U\}\_\{1\} et \{U\}\_\{2\} ouverts disjoints de A,
alors chacun des \{A\}\_\{i\} doit être contenu soit dans \{U\}\_\{1\},
soit dans \{U\}\_\{2\}, sinon \{A\}\_\{i\} serait réunion des deux
ouverts non vides disjoints \{A\}\_\{i\} ∩ \{U\}\_\{1\} et \{A\}\_\{i\}
∩ \{U\}\_\{2\}. Comme \{A\}\_\{i\} contient a, il est forcément contenu
dans celui des deux ouverts \{U\}\_\{1\} et \{U\}\_\{2\} qui contient a.
Mais alors A lui-même est contenu dans cet ouvert, et donc l'autre est
vide.

Corollaire~4.9.5 Soit E un espace topologique et a un point de E. Alors
l'ensemble des connexes contenant a a un plus grand élément appelé la
composante connexe de a dans E~; deux composantes connexes sont soit
disjointes soit confondues~; toute composante connexe est fermée.

Démonstration La composante connexe de a est bien entendu
\{\textbackslash{}mathop\{\textbackslash{}mathop\{⋃ \}\} \}\_\{a∈A,
A\textbackslash{}text\{connexe\}\}A~; c'est un connexe d'après la
proposition précédente et c'est bien entendu le plus grand~; si A est la
composante connexe de a, B celle de b et si A ∩
B\textbackslash{}mathrel\{≠\}∅, alors A ∪ B est connexe et donc A ∪ B ⊂
A, soit B ⊂ A et de même A ⊂ B soit A = B. D'autre part,
\textbackslash{}overline\{A\} est encore connexe contenant a, donc
\textbackslash{}overline\{A\} ⊂ A et donc A est fermé.

Proposition~4.9.6 Soit
\{E\}\_\{1\},\textbackslash{}mathop\{\textbackslash{}mathop\{\ldots{}\}\},\{E\}\_\{k\}
des espaces connexes. Alors \{E\}\_\{1\} ×\textbackslash{}mathrel\{⋯\} ×
\{E\}\_\{k\} est connexe.

Démonstration Il suffit évidemment de montrer le résultat pour k = 2.
Soit \{E\}\_\{1\} × \{E\}\_\{2\} = \{U\}\_\{1\} ∪ \{U\}\_\{2\} avec
\{U\}\_\{1\} et \{U\}\_\{2\} ouverts disjoints. Remarquons que si a ∈
\{E\}\_\{1\}, l'application y\textbackslash{}mathrel\{↦\}(a,y) est un
homéomorphisme de \{E\}\_\{2\} sur \textbackslash{}\{a\textbackslash{}\}
× \{E\}\_\{2\} qui est donc aussi connexe. Donc V doit être contenu soit
dans \{U\}\_\{1\} soit dans \{U\}\_\{2\} (sinon il serait réunion des
deux ouverts non vides disjoints (\textbackslash{}\{a\textbackslash{}\}
× \{E\}\_\{2\}) ∩ \{U\}\_\{1\} et (\textbackslash{}\{a\textbackslash{}\}
× \{E\}\_\{2\}) ∩ \{U\}\_\{2\}). Soit b ∈ \{E\}\_\{2\}~; pour la même
raison, on a par exemple \{E\}\_\{1\}
×\textbackslash{}\{b\textbackslash{}\} ⊂ \{U\}\_\{1\}. Alors, pour tout
a ∈ \{E\}\_\{1\}, comme (a,b) ∈\textbackslash{}\{a\textbackslash{}\} ×
\{E\}\_\{2\} et \{E\}\_\{1\} ×\textbackslash{}\{b\textbackslash{}\} ⊂
\{U\}\_\{1\}, on a nécessairement \textbackslash{}\{a\textbackslash{}\}
× \{E\}\_\{2\} ⊂ \{U\}\_\{1\} et donc \{E\}\_\{1\} × \{E\}\_\{2\} ⊂
\{U\}\_\{1\}, soit encore \{U\}\_\{2\} = ∅.

\paragraph{4.9.3 Connexes de ℝ}

Définition~4.9.3 Une partie A de ℝ est dite convexe si
\textbackslash{}mathop\{∀\}a,b ∈ ℝ, {[}a,b{]} ⊂ ℝ.

Proposition~4.9.7 Les parties convexes de ℝ sont les intervalles.

Démonstration Il est clair que tout intervalle est convexe. L'ensemble
vide est bien entendu un intervalle. Soit donc A une partie convexe non
vide, m =\textbackslash{}mathop\{ inf\} A ∈ ℝ
∪\textbackslash{}\{−∞\textbackslash{}\} et M =\textbackslash{}mathop\{
sup\}A ∈ ℝ ∪\textbackslash{}\{ + ∞\textbackslash{}\}. On a A ⊂
{[}m,M{]}. Pour montrer que A est un intervalle, il suffit de montrer
que {]}m,M{[}⊂ A. Or, soit x ∈{]}m,M{[}. Il existe a ∈ A tel que m ≤ a
\textless{} x (propriété caractéristique de la borne inférieure) et de
même, il existe b ∈ A tel que x \textless{} b ≤ M. On a donc x
∈{]}a,b{[}⊂ A~; d'où l'inclusion et le résultat.

Théorème~4.9.8 Les parties connexes de ℝ sont les intervalles.

Démonstration Soit A une partie connexe~; si A n'était pas convexe, il
existerait a,b ∈ A tel que {]}a,b{[}⊄A (car a,b sont dans A)~; soit x
∈{]}a,b{[} tel que x\textbackslash{}mathrel\{∉\}A. On a alors A = (A∩{]}
−∞,x{[}) ∪ (A∩{]}x,+∞{[}), réunion de deux ouverts de A non vides et
disjoints~; c'est absurde. Donc A est convexe et donc un intervalle.

Inversement, soit I un intervalle~; alors il existe J intervalle ouvert
tel que J ⊂ I ⊂\textbackslash{}overline\{J\} donc il suffit de montrer
que les intervalles ouverts sont connexes.

Soit I ={]}a,b{[} un intervalle ouvert de ℝ et A une partie ouverte et
fermée, non vide et distincte de I. Soit x ∈ I ∖ A. Alors A =
(A∩{]}a,x{[}) ∪ (A∩{]}x,b{[})~; au moins une des deux parties est non
vide, par exemple B = A∩{]}x,b{[}= A ∩ {[}x,b{[}. Cette partie est à la
fois ouverte et fermée dans I (intersection de deux ouverts de I et
aussi de deux fermés de I). Soit m =\textbackslash{}mathop\{ inf\} B ≥
x. On a m ∈ I et comme B est fermé dans I, on a m ∈ B. Mais alors
\textbackslash{}mathop\{∃\}ε \textgreater{} 0, {]}m − ε,m + ε{[}⊂ B, ce
qui contredit la définition de la borne inférieure. C'est absurde, donc
I est connexe.

Corollaire~4.9.9 (théorème des valeurs intermédiaires). Soit E un espace
topologique connexe et f : E → ℝ continue. Alors
\textbackslash{}mathop\{\textbackslash{}mathrm\{Im\}\}f est un
intervalle de ℝ.

Démonstration f(E) est connexe, donc un intervalle.

\paragraph{4.9.4 Connexité par arcs}

Définition~4.9.4 Soit E un espace topologique, a,b ∈ E. On appelle
chemin d'origine a et d'extrémité b dans E toute application continue γ
: {[}0,1{]} → E telle que γ(0) = a et γ(1) = b.

Proposition~4.9.10 Soit E un espace topologique. La relation ''il existe
un chemin d'origine a et d'extrémité b'' est une relation d'équivalence
sur E.

Démonstration Cette relation est clairement réflexive (prendre γ
constant) et symétrique (prendre \{γ\}\_\{1\}(t) = γ(1 − t)). Pour la
transitivité, soit \{γ\}\_\{1\} une chemin de a à b et \{γ\}\_\{2\} un
chemin de b à c. On définit γ : {[}0,1{]} → E par γ(t) =
\textbackslash{}left \textbackslash{}\{ \textbackslash{}cases\{
\{γ\}\_\{1\}(2t) \&si t ∈ {[}0,1∕2{]} \textbackslash{}cr \{γ\}\_\{2\}(2t
− 1)\&si t ∈ {[}1∕2,1{]} \textbackslash{}cr \} \textbackslash{}right ..
Alors γ est un chemin de a à c.

Définition~4.9.5 On dit que E est connexe par arcs si, pour tout couple
(a,b) ∈ \{E\}\^{}\{2\} il existe un chemin de a à b dans E.

Proposition~4.9.11 Tout espace topologique connexe par arcs est connexe.

Démonstration Soit a ∈ E et pour x ∈ E soit \{γ\}\_\{x\} un chemin
d'origine a et d'extrémité x. Les \{γ\}\_\{x\}({[}0,1{]}) sont des
images de connexes par une application continue, ils sont donc connexes.
Leur intersection contient a, et donc leur réunion est connexe. Mais on
a évidemment E =\{\textbackslash{}mathop\{ \textbackslash{}mathop\{⋃
\}\} \}\_\{x∈E\}\{γ\}\_\{x\}({[}0,1{]}) (une réunion de parties de E est
contenue dans E et de plus tout élément x de E appartient à
\{γ\}\_\{x\}({[}0,1{]})). Donc E est connexe.

{[}\href{coursse25.html}{prev}{]}
{[}\href{coursse25.html\#tailcoursse25.html}{prev-tail}{]}
{[}\href{coursse26.html}{front}{]}
{[}\href{coursch5.html\#coursse26.html}{up}{]}

\end{document}
