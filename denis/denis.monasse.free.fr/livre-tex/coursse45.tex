\documentclass[]{article}
\usepackage[T1]{fontenc}
\usepackage{lmodern}
\usepackage{amssymb,amsmath}
\usepackage{ifxetex,ifluatex}
\usepackage{fixltx2e} % provides \textsubscript
% use upquote if available, for straight quotes in verbatim environments
\IfFileExists{upquote.sty}{\usepackage{upquote}}{}
\ifnum 0\ifxetex 1\fi\ifluatex 1\fi=0 % if pdftex
  \usepackage[utf8]{inputenc}
\else % if luatex or xelatex
  \ifxetex
    \usepackage{mathspec}
    \usepackage{xltxtra,xunicode}
  \else
    \usepackage{fontspec}
  \fi
  \defaultfontfeatures{Mapping=tex-text,Scale=MatchLowercase}
  \newcommand{\euro}{€}
\fi
% use microtype if available
\IfFileExists{microtype.sty}{\usepackage{microtype}}{}
\ifxetex
  \usepackage[setpagesize=false, % page size defined by xetex
              unicode=false, % unicode breaks when used with xetex
              xetex]{hyperref}
\else
  \usepackage[unicode=true]{hyperref}
\fi
\hypersetup{breaklinks=true,
            bookmarks=true,
            pdfauthor={},
            pdftitle={Derivee},
            colorlinks=true,
            citecolor=blue,
            urlcolor=blue,
            linkcolor=magenta,
            pdfborder={0 0 0}}
\urlstyle{same}  % don't use monospace font for urls
\setlength{\parindent}{0pt}
\setlength{\parskip}{6pt plus 2pt minus 1pt}
\setlength{\emergencystretch}{3em}  % prevent overfull lines
\setcounter{secnumdepth}{0}
 
/* start css.sty */
.cmr-5{font-size:50%;}
.cmr-7{font-size:70%;}
.cmmi-5{font-size:50%;font-style: italic;}
.cmmi-7{font-size:70%;font-style: italic;}
.cmmi-10{font-style: italic;}
.cmsy-5{font-size:50%;}
.cmsy-7{font-size:70%;}
.cmex-7{font-size:70%;}
.cmex-7x-x-71{font-size:49%;}
.msbm-7{font-size:70%;}
.cmtt-10{font-family: monospace;}
.cmti-10{ font-style: italic;}
.cmbx-10{ font-weight: bold;}
.cmr-17x-x-120{font-size:204%;}
.cmsl-10{font-style: oblique;}
.cmti-7x-x-71{font-size:49%; font-style: italic;}
.cmbxti-10{ font-weight: bold; font-style: italic;}
p.noindent { text-indent: 0em }
td p.noindent { text-indent: 0em; margin-top:0em; }
p.nopar { text-indent: 0em; }
p.indent{ text-indent: 1.5em }
@media print {div.crosslinks {visibility:hidden;}}
a img { border-top: 0; border-left: 0; border-right: 0; }
center { margin-top:1em; margin-bottom:1em; }
td center { margin-top:0em; margin-bottom:0em; }
.Canvas { position:relative; }
li p.indent { text-indent: 0em }
.enumerate1 {list-style-type:decimal;}
.enumerate2 {list-style-type:lower-alpha;}
.enumerate3 {list-style-type:lower-roman;}
.enumerate4 {list-style-type:upper-alpha;}
div.newtheorem { margin-bottom: 2em; margin-top: 2em;}
.obeylines-h,.obeylines-v {white-space: nowrap; }
div.obeylines-v p { margin-top:0; margin-bottom:0; }
.overline{ text-decoration:overline; }
.overline img{ border-top: 1px solid black; }
td.displaylines {text-align:center; white-space:nowrap;}
.centerline {text-align:center;}
.rightline {text-align:right;}
div.verbatim {font-family: monospace; white-space: nowrap; text-align:left; clear:both; }
.fbox {padding-left:3.0pt; padding-right:3.0pt; text-indent:0pt; border:solid black 0.4pt; }
div.fbox {display:table}
div.center div.fbox {text-align:center; clear:both; padding-left:3.0pt; padding-right:3.0pt; text-indent:0pt; border:solid black 0.4pt; }
div.minipage{width:100%;}
div.center, div.center div.center {text-align: center; margin-left:1em; margin-right:1em;}
div.center div {text-align: left;}
div.flushright, div.flushright div.flushright {text-align: right;}
div.flushright div {text-align: left;}
div.flushleft {text-align: left;}
.underline{ text-decoration:underline; }
.underline img{ border-bottom: 1px solid black; margin-bottom:1pt; }
.framebox-c, .framebox-l, .framebox-r { padding-left:3.0pt; padding-right:3.0pt; text-indent:0pt; border:solid black 0.4pt; }
.framebox-c {text-align:center;}
.framebox-l {text-align:left;}
.framebox-r {text-align:right;}
span.thank-mark{ vertical-align: super }
span.footnote-mark sup.textsuperscript, span.footnote-mark a sup.textsuperscript{ font-size:80%; }
div.tabular, div.center div.tabular {text-align: center; margin-top:0.5em; margin-bottom:0.5em; }
table.tabular td p{margin-top:0em;}
table.tabular {margin-left: auto; margin-right: auto;}
div.td00{ margin-left:0pt; margin-right:0pt; }
div.td01{ margin-left:0pt; margin-right:5pt; }
div.td10{ margin-left:5pt; margin-right:0pt; }
div.td11{ margin-left:5pt; margin-right:5pt; }
table[rules] {border-left:solid black 0.4pt; border-right:solid black 0.4pt; }
td.td00{ padding-left:0pt; padding-right:0pt; }
td.td01{ padding-left:0pt; padding-right:5pt; }
td.td10{ padding-left:5pt; padding-right:0pt; }
td.td11{ padding-left:5pt; padding-right:5pt; }
table[rules] {border-left:solid black 0.4pt; border-right:solid black 0.4pt; }
.hline hr, .cline hr{ height : 1px; margin:0px; }
.tabbing-right {text-align:right;}
span.TEX {letter-spacing: -0.125em; }
span.TEX span.E{ position:relative;top:0.5ex;left:-0.0417em;}
a span.TEX span.E {text-decoration: none; }
span.LATEX span.A{ position:relative; top:-0.5ex; left:-0.4em; font-size:85%;}
span.LATEX span.TEX{ position:relative; left: -0.4em; }
div.float img, div.float .caption {text-align:center;}
div.figure img, div.figure .caption {text-align:center;}
.marginpar {width:20%; float:right; text-align:left; margin-left:auto; margin-top:0.5em; font-size:85%; text-decoration:underline;}
.marginpar p{margin-top:0.4em; margin-bottom:0.4em;}
.equation td{text-align:center; vertical-align:middle; }
td.eq-no{ width:5%; }
table.equation { width:100%; } 
div.math-display, div.par-math-display{text-align:center;}
math .texttt { font-family: monospace; }
math .textit { font-style: italic; }
math .textsl { font-style: oblique; }
math .textsf { font-family: sans-serif; }
math .textbf { font-weight: bold; }
.partToc a, .partToc, .likepartToc a, .likepartToc {line-height: 200%; font-weight:bold; font-size:110%;}
.chapterToc a, .chapterToc, .likechapterToc a, .likechapterToc, .appendixToc a, .appendixToc {line-height: 200%; font-weight:bold;}
.index-item, .index-subitem, .index-subsubitem {display:block}
.caption td.id{font-weight: bold; white-space: nowrap; }
table.caption {text-align:center;}
h1.partHead{text-align: center}
p.bibitem { text-indent: -2em; margin-left: 2em; margin-top:0.6em; margin-bottom:0.6em; }
p.bibitem-p { text-indent: 0em; margin-left: 2em; margin-top:0.6em; margin-bottom:0.6em; }
.paragraphHead, .likeparagraphHead { margin-top:2em; font-weight: bold;}
.subparagraphHead, .likesubparagraphHead { font-weight: bold;}
.quote {margin-bottom:0.25em; margin-top:0.25em; margin-left:1em; margin-right:1em; text-align:\\jmathmathustify;}
.verse{white-space:nowrap; margin-left:2em}
div.maketitle {text-align:center;}
h2.titleHead{text-align:center;}
div.maketitle{ margin-bottom: 2em; }
div.author, div.date {text-align:center;}
div.thanks{text-align:left; margin-left:10%; font-size:85%; font-style:italic; }
div.author{white-space: nowrap;}
.quotation {margin-bottom:0.25em; margin-top:0.25em; margin-left:1em; }
h1.partHead{text-align: center}
.sectionToc, .likesectionToc {margin-left:2em;}
.subsectionToc, .likesubsectionToc {margin-left:4em;}
.subsubsectionToc, .likesubsubsectionToc {margin-left:6em;}
.frenchb-nbsp{font-size:75%;}
.frenchb-thinspace{font-size:75%;}
.figure img.graphics {margin-left:10%;}
/* end css.sty */

\title{Derivee}
\author{}
\date{}

\begin{document}
\maketitle

\textbf{Warning: 
requires JavaScript to process the mathematics on this page.\\ If your
browser supports JavaScript, be sure it is enabled.}

\begin{center}\rule{3in}{0.4pt}\end{center}

{[}
{[}
{[}{]}
{[}

\subsubsection{8.2 Dérivée}

\paragraph{8.2.1 Notion de dérivée}

Définition~8.2.1 Soit I un intervalle de \mathbb{R}~, E un espace vectoriel
normé~et f : I \rightarrow~ E. On dit que f est dérivable en a \in I si existe
lim_x\rightarrow~a,x\neq~a~
f(x)-f(a) \over x-a ~; dans ce cas cette limite est
appelée la dérivée de f au point a et notée f'(a).

Remarque~8.2.1 Comme toute notion de limite, il s'agit d'une notion
locale~: f : I \rightarrow~ E est dérivable au point a si et seulement si~sa
restriction à {]}a - \eta,a + \eta{[}\bigcapI est dérivable au point a.

Proposition~8.2.1 Si f est dérivable au point a \in I elle est continue au
point a.

Démonstration On écrit pour x\neq~a,  f(x)-f(a)
\over x-a = f'(a) + \epsilon(x - a) avec
lim_h\rightarrow~0~\epsilon(h) = 0~; on a donc f(x) =
f(a) + (x - a)f'(a) + (x - a)\epsilon(x - a) ce qui montre que
lim_x\rightarrow~a,x\neq~a~f(x)
= f(a)~; donc f est continue au point a.

Définition~8.2.2 Soit I un intervalle de \mathbb{R}~, E un espace vectoriel
normé~et f : I \rightarrow~ E. On dit que f est dérivable si elle est dérivable en
tout point de I~; l'application f' : a\mapsto~f'(a)
est appelée la dérivée de f.

Remarque~8.2.2 On a donc~: f dérivable \rigtharrow~ f continue.

\paragraph{8.2.2 Opérations sur les dérivées}

Théorème~8.2.2 Soit I un intervalle de \mathbb{R}~, E un espace vectoriel normé~et
f et g des applications de I dans E dérivables au point a~; si \alpha~ et \beta~
sont des scalaires, \alpha~f + \beta~g est dérivable au point a et (\alpha~f + \beta~g)'(a) =
\alpha~f'(a) + \beta~g'(a).

Démonstration Il suffit de remarquer que  (\alpha~f+\beta~g)(x)-(\alpha~f+\beta~g)(a)
\over x-a = \alpha~ f(x)-f(a) \over x-a +
\beta~ g(x)-g(a) \over x-a et d'appliquer les théorèmes
sur les limites.

Théorème~8.2.3 Soit I un intervalle de \mathbb{R}~, E,F,G trois espaces vectoriels
normés, f : I \rightarrow~ E, g : I \rightarrow~ F~; soit u : E \times F \rightarrow~ G une application
bilinéaire continue et h : I \rightarrow~ G,
t\mapsto~u(f(t),g(t)). Si f et g sont dérivables au
point a, alors h est dérivable au point a et h'(a) = u(f'(a),g(a)) +
u(f(a),g'(a)).

Démonstration On vérifie immédiatement que  h(x)-h(a)
\over x-a = u( f(x)-f(a) \over x-a
,g(x)) + u(f(a), g(x)-g(a) \over x-a ). Or
lim_x\rightarrow~a,x\neq~a~
f(x)-f(a) \over x-a = f'(a),
lim_x\rightarrow~a,x\neq~a~g(x)
= g(a) et
lim_x\rightarrow~a,x\neq~a~
g(x)-g(a) \over x-a = g'(a). Comme u est continue, on a
lim_x\rightarrow~a,x\neq~a~
h(x)-h(a) \over x-a = u(f'(a),g(a)) + u(f(a),g'(a)).

Remarque~8.2.3 Ce théorème s'étend immédiatement au cas d'une
application p-linéaire continue u : E_1
\times⋯ \times E_p dans G. Dans ce cas on a

h'(a) = \sum _i=1^pu(f_
1(a),\ldots,f_i-1(a),f_i'(a),f_i+1(a),\\ldots,f_p~(a))

En particulier, dans le cas d'un déterminant on retiendra

\begin{align*}
{[}\mathrm{det}~
_\mathcal{E}(f_1,\\ldots,f_n~){]}'(a)&&
\%& \\ & =& \\sum
_i=1^n \mathrm{det} _
\mathcal{E}(f_1(a),\ldots,f_i-1(a),f_i'(a),f_i+1(a),\\ldots,f_n~(a))\%&
\\ \end{align*}

Théorème~8.2.4 Soit \phi : I \rightarrow~ \mathbb{R}~ et f : J \rightarrow~ E avec \phi(I) \subset~ J~; soit a \in I.
Si \phi est dérivable au point a et si f est dérivable au point f(a), alors
f \cdot \phi est dérivable au point a et (f \cdot \phi)'(a) = \phi'(a)f'(\phi(a)).

Démonstration En effet la dérivabilité de f au point \phi(a) peut se
traduire par

f(x) - f(\phi(a)) = (x - \phi(a))f'(\phi(a)) + (x - \phi(a))\epsilon(x)

avec lim_x\rightarrow~\phi(a)~\epsilon(x) = 0. On a donc

f(\phi(t)) - f(\phi(a)) = (\phi(t) - \phi(a))f'(\phi(a)) + (\phi(t) - \phi(a))\epsilon(\phi(t))

avec lim_t\rightarrow~a~\epsilon(\phi(t)) = 0 puisque \phi est
continue au point a.

De même la dérivabilité de \phi au point a se traduit par

\phi(t) - \phi(a) = (t - a)\phi'(a) + o(t - a)

On obtient alors en rempla\ccant

f(\phi(t)) - f(\phi(a)) = (t - a)\phi'(a)f'(\phi(a)) + o(t - a)

ce qui montre que

lim_t\rightarrow~a,t\neq~a~f(\phi(t))
- f(\phi(a))\over t - a = \phi'(a)f'(\phi(a))

Théorème~8.2.5 Soit f : I \rightarrow~ \mathbb{R}~, a \in I tel que
f(a)\neq~0. Si f est dérivable au point a, il
existe \epsilon \textgreater{} 0 tel que f ne s'annule pas sur J = I\bigcap{]}a - \epsilon,a
+ \epsilon{[}. La fonction  1 \over f est dérivable au point
a et \left ( 1 \over f
\right )'(a) = - f'(a) \over
f(a)^2 .

Démonstration La fonction f étant continue au point a, il existe \epsilon
\textgreater{} 0 tel que t \in I\bigcap{]}a - \epsilon,a + \epsilon{[}\rigtharrow~f(t) -
f(a) \textless{} f(a)
\over 2 ~; on en déduit que t \in J \rigtharrow~
f(t)\neq~0. Pour t \in J
\diagdown\a\ on a  1 \over
t-a \left ( 1 \over f (t) - 1
\over f (a)\right ) = - 1
\over f(t)f(a)  f(t)-f(a) \over t-a
qui tend vers - f'(a) \over f(a)^2 quand t
tend vers a.

\paragraph{8.2.3 Dérivées d'ordre supérieur}

Définition~8.2.3 Soit I un intervalle de \mathbb{R}~, E un espace vectoriel
normé~et f : I \rightarrow~ E. Soit n ≥ 1. On dit que f est n fois dérivable au
point a \in I s'il existe \eta \textgreater{} 0 tel que f est n - 1 fois
dérivable sur I\bigcap{]}a - \eta,a + \eta{[} et si l'application f^(n-1)
est dérivable au point a~; on pose alors f^(n)(a) =
(f^(n-1))'(a). On dit que f est n fois dérivable sur I si
elle est n fois dérivable en tout point de I~; on dit que f est de
classe C^n si elle est n fois dérivable sur I et si
f^(n) est continue sur I~; on dit que f est C^\infty~ si
elle est de classe C^n pour tout n.

Remarque~8.2.4 Puisque toute fonction dérivable est continue, si f est n
fois dérivable, elle est de classe C^n-1.

Théorème~8.2.6 (Leibnitz). Soit I un intervalle de \mathbb{R}~, E,F,G trois
espaces vectoriels normés, f : I \rightarrow~ E, g : I \rightarrow~ F~; soit u : E \times F \rightarrow~ G une
application bilinéaire continue et h : I \rightarrow~ G,
t\mapsto~u(f(t),g(t)). Si f et g sont n fois
dérivables au point a, alors h est n fois dérivable au point a et

h^(n)(a) = \\sum
_p=0^nC_
n^pu(f^(p)(a),g^(n-p)(a))

Démonstration Par récurrence sur n~; le résultat a dé\\jmathmathà été vu pour n =
1~; supposons le vrai pour n - 1 et soit \epsilon \textgreater{} 0 tel que f et
g soient n - 1 fois dérivables sur I\bigcap{]}a - \eta,a + \eta{[}. L'hypothèse de
récurrence implique que h est n - 1 fois dérivable sur I\bigcap{]}a - \eta,a +
\eta{[} et que sa dérivée (n - 1)^\textième
est donnée par

h^(n-1)(t) = \\sum
_p=0^n-1C_
n-1^pu(f^(p)(t),g^(n-1-p)(t))

Mais toutes les applications f^(p), g^(n-1-p) sont
dérivables au point a~; il en est donc de même des applications
t\mapsto~u(f^(p)(t),g^(n-1-p)(t)),
et donc de h^(n-1). Donc h est n fois dérivable au point a et

\begin{align*} h^(n)(a)& =&
\sum _p=0^n-1C_
n-1^p(u(f^(p+1)(a),g^(n-1-p)(a))\%&
\\ & \text &
\quad + u(f^(p)(a),g^(n-p)(a)))
\%& \\ & =& \\sum
_p=1^nC_
n-1^p-1u(f^(p)(a),g^(n-p)(a)) \%&
\\ & \text &
\quad + \\sum
_p=0^n-1C_
n-1^pu(f^(p)(a),g^(n-p)(a)) \%&
\\ \end{align*}

en changeant dans la première somme p + 1 en p~; puis

\begin{align*} h^(n)(a)& =&
u(f(a),g^(n)(a)) \%& \\ &
\text & +\\sum
_p=1^n-1(C_ n-1^p-1 + C_
n-1^p)u(f^(p)(a),g^(n-p)(a))\%&
\\ & \text &
+u(f^(n)(a),g(a)) \%& \\ &
=& \sum _p=0^nC_
n^pu(f^(p)(a),g^(n-p)(a)) \%&
\\ \end{align*}

ce qui achève la récurrence.

Corollaire~8.2.7 Sous les mêmes hypothèses, si f et g sont de classe
C^n, h est de classe C^n.

Démonstration C'est clair d'après la formule ci dessus.

Théorème~8.2.8 Soit \phi : I \rightarrow~ \mathbb{R}~ et f : J \rightarrow~ E avec \phi(I) \subset~ J~; soit a \in I.
Si \phi est n fois dérivable au point a et si f est n fois dérivable au
point f(a), alors f \cdot \phi est n fois dérivable au point a.

Démonstration Par récurrence sur n~; le résultat a dé\\jmathmathà été vu pour n =
1~; supposons le vrai pour n - 1 et soit \eta \textgreater{} 0 tel que f \cdot
\phi soit dérivable sur I\bigcap{]}a - \eta,a + \eta{[} avec (f \cdot \phi)' = \phi'(f' \cdot \phi).
Comme f' et \phi sont n - 1 fois dérivables en a, l'hypothèse de récurrence
implique que f' \cdot \phi est n - 1 fois dérivable en a~; comme \phi' l'est
également, le théorème de Leibnitz appliqué au produit ordinaire assure
que (f \cdot \phi)' = \phi'(f' \cdot \phi) est n - 1 fois dérivable au point a, donc que
f \cdot \phi est n fois dérivable au point a.

Corollaire~8.2.9 Sous les mêmes hypothèses, si f et \phi sont de classe
C^n, f \cdot \phi est de classe C^n.

{[}
{[}
{[}
{[}

\end{document}
