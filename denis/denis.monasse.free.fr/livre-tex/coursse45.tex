\documentclass[]{article}
\usepackage[T1]{fontenc}
\usepackage{lmodern}
\usepackage{amssymb,amsmath}
\usepackage{ifxetex,ifluatex}
\usepackage{fixltx2e} % provides \textsubscript
% use upquote if available, for straight quotes in verbatim environments
\IfFileExists{upquote.sty}{\usepackage{upquote}}{}
\ifnum 0\ifxetex 1\fi\ifluatex 1\fi=0 % if pdftex
  \usepackage[utf8]{inputenc}
\else % if luatex or xelatex
  \ifxetex
    \usepackage{mathspec}
    \usepackage{xltxtra,xunicode}
  \else
    \usepackage{fontspec}
  \fi
  \defaultfontfeatures{Mapping=tex-text,Scale=MatchLowercase}
  \newcommand{\euro}{€}
\fi
% use microtype if available
\IfFileExists{microtype.sty}{\usepackage{microtype}}{}
\ifxetex
  \usepackage[setpagesize=false, % page size defined by xetex
              unicode=false, % unicode breaks when used with xetex
              xetex]{hyperref}
\else
  \usepackage[unicode=true]{hyperref}
\fi
\hypersetup{breaklinks=true,
            bookmarks=true,
            pdfauthor={},
            pdftitle={Derivee},
            colorlinks=true,
            citecolor=blue,
            urlcolor=blue,
            linkcolor=magenta,
            pdfborder={0 0 0}}
\urlstyle{same}  % don't use monospace font for urls
\setlength{\parindent}{0pt}
\setlength{\parskip}{6pt plus 2pt minus 1pt}
\setlength{\emergencystretch}{3em}  % prevent overfull lines
\setcounter{secnumdepth}{0}
 
/* start css.sty */
.cmr-5{font-size:50%;}
.cmr-7{font-size:70%;}
.cmmi-5{font-size:50%;font-style: italic;}
.cmmi-7{font-size:70%;font-style: italic;}
.cmmi-10{font-style: italic;}
.cmsy-5{font-size:50%;}
.cmsy-7{font-size:70%;}
.cmex-7{font-size:70%;}
.cmex-7x-x-71{font-size:49%;}
.msbm-7{font-size:70%;}
.cmtt-10{font-family: monospace;}
.cmti-10{ font-style: italic;}
.cmbx-10{ font-weight: bold;}
.cmr-17x-x-120{font-size:204%;}
.cmsl-10{font-style: oblique;}
.cmti-7x-x-71{font-size:49%; font-style: italic;}
.cmbxti-10{ font-weight: bold; font-style: italic;}
p.noindent { text-indent: 0em }
td p.noindent { text-indent: 0em; margin-top:0em; }
p.nopar { text-indent: 0em; }
p.indent{ text-indent: 1.5em }
@media print {div.crosslinks {visibility:hidden;}}
a img { border-top: 0; border-left: 0; border-right: 0; }
center { margin-top:1em; margin-bottom:1em; }
td center { margin-top:0em; margin-bottom:0em; }
.Canvas { position:relative; }
li p.indent { text-indent: 0em }
.enumerate1 {list-style-type:decimal;}
.enumerate2 {list-style-type:lower-alpha;}
.enumerate3 {list-style-type:lower-roman;}
.enumerate4 {list-style-type:upper-alpha;}
div.newtheorem { margin-bottom: 2em; margin-top: 2em;}
.obeylines-h,.obeylines-v {white-space: nowrap; }
div.obeylines-v p { margin-top:0; margin-bottom:0; }
.overline{ text-decoration:overline; }
.overline img{ border-top: 1px solid black; }
td.displaylines {text-align:center; white-space:nowrap;}
.centerline {text-align:center;}
.rightline {text-align:right;}
div.verbatim {font-family: monospace; white-space: nowrap; text-align:left; clear:both; }
.fbox {padding-left:3.0pt; padding-right:3.0pt; text-indent:0pt; border:solid black 0.4pt; }
div.fbox {display:table}
div.center div.fbox {text-align:center; clear:both; padding-left:3.0pt; padding-right:3.0pt; text-indent:0pt; border:solid black 0.4pt; }
div.minipage{width:100%;}
div.center, div.center div.center {text-align: center; margin-left:1em; margin-right:1em;}
div.center div {text-align: left;}
div.flushright, div.flushright div.flushright {text-align: right;}
div.flushright div {text-align: left;}
div.flushleft {text-align: left;}
.underline{ text-decoration:underline; }
.underline img{ border-bottom: 1px solid black; margin-bottom:1pt; }
.framebox-c, .framebox-l, .framebox-r { padding-left:3.0pt; padding-right:3.0pt; text-indent:0pt; border:solid black 0.4pt; }
.framebox-c {text-align:center;}
.framebox-l {text-align:left;}
.framebox-r {text-align:right;}
span.thank-mark{ vertical-align: super }
span.footnote-mark sup.textsuperscript, span.footnote-mark a sup.textsuperscript{ font-size:80%; }
div.tabular, div.center div.tabular {text-align: center; margin-top:0.5em; margin-bottom:0.5em; }
table.tabular td p{margin-top:0em;}
table.tabular {margin-left: auto; margin-right: auto;}
div.td00{ margin-left:0pt; margin-right:0pt; }
div.td01{ margin-left:0pt; margin-right:5pt; }
div.td10{ margin-left:5pt; margin-right:0pt; }
div.td11{ margin-left:5pt; margin-right:5pt; }
table[rules] {border-left:solid black 0.4pt; border-right:solid black 0.4pt; }
td.td00{ padding-left:0pt; padding-right:0pt; }
td.td01{ padding-left:0pt; padding-right:5pt; }
td.td10{ padding-left:5pt; padding-right:0pt; }
td.td11{ padding-left:5pt; padding-right:5pt; }
table[rules] {border-left:solid black 0.4pt; border-right:solid black 0.4pt; }
.hline hr, .cline hr{ height : 1px; margin:0px; }
.tabbing-right {text-align:right;}
span.TEX {letter-spacing: -0.125em; }
span.TEX span.E{ position:relative;top:0.5ex;left:-0.0417em;}
a span.TEX span.E {text-decoration: none; }
span.LATEX span.A{ position:relative; top:-0.5ex; left:-0.4em; font-size:85%;}
span.LATEX span.TEX{ position:relative; left: -0.4em; }
div.float img, div.float .caption {text-align:center;}
div.figure img, div.figure .caption {text-align:center;}
.marginpar {width:20%; float:right; text-align:left; margin-left:auto; margin-top:0.5em; font-size:85%; text-decoration:underline;}
.marginpar p{margin-top:0.4em; margin-bottom:0.4em;}
.equation td{text-align:center; vertical-align:middle; }
td.eq-no{ width:5%; }
table.equation { width:100%; } 
div.math-display, div.par-math-display{text-align:center;}
math .texttt { font-family: monospace; }
math .textit { font-style: italic; }
math .textsl { font-style: oblique; }
math .textsf { font-family: sans-serif; }
math .textbf { font-weight: bold; }
.partToc a, .partToc, .likepartToc a, .likepartToc {line-height: 200%; font-weight:bold; font-size:110%;}
.chapterToc a, .chapterToc, .likechapterToc a, .likechapterToc, .appendixToc a, .appendixToc {line-height: 200%; font-weight:bold;}
.index-item, .index-subitem, .index-subsubitem {display:block}
.caption td.id{font-weight: bold; white-space: nowrap; }
table.caption {text-align:center;}
h1.partHead{text-align: center}
p.bibitem { text-indent: -2em; margin-left: 2em; margin-top:0.6em; margin-bottom:0.6em; }
p.bibitem-p { text-indent: 0em; margin-left: 2em; margin-top:0.6em; margin-bottom:0.6em; }
.paragraphHead, .likeparagraphHead { margin-top:2em; font-weight: bold;}
.subparagraphHead, .likesubparagraphHead { font-weight: bold;}
.quote {margin-bottom:0.25em; margin-top:0.25em; margin-left:1em; margin-right:1em; text-align:justify;}
.verse{white-space:nowrap; margin-left:2em}
div.maketitle {text-align:center;}
h2.titleHead{text-align:center;}
div.maketitle{ margin-bottom: 2em; }
div.author, div.date {text-align:center;}
div.thanks{text-align:left; margin-left:10%; font-size:85%; font-style:italic; }
div.author{white-space: nowrap;}
.quotation {margin-bottom:0.25em; margin-top:0.25em; margin-left:1em; }
h1.partHead{text-align: center}
.sectionToc, .likesectionToc {margin-left:2em;}
.subsectionToc, .likesubsectionToc {margin-left:4em;}
.subsubsectionToc, .likesubsubsectionToc {margin-left:6em;}
.frenchb-nbsp{font-size:75%;}
.frenchb-thinspace{font-size:75%;}
.figure img.graphics {margin-left:10%;}
/* end css.sty */

\title{Derivee}
\author{}
\date{}

\begin{document}
\maketitle

\textbf{Warning: \href{http://www.math.union.edu/locate/jsMath}{jsMath}
requires JavaScript to process the mathematics on this page.\\ If your
browser supports JavaScript, be sure it is enabled.}

\begin{center}\rule{3in}{0.4pt}\end{center}

{[}\href{coursse46.html}{next}{]} {[}\href{coursse44.html}{prev}{]}
{[}\href{coursse44.html\#tailcoursse44.html}{prev-tail}{]}
{[}\hyperref[tailcoursse45.html]{tail}{]}
{[}\href{coursch9.html\#coursse45.html}{up}{]}

\subsubsection{8.2 Dérivée}

\paragraph{8.2.1 Notion de dérivée}

Définition~8.2.1 Soit I un intervalle de ℝ, E un espace vectoriel
normé~et f : I → E. On dit que f est dérivable en a ∈ I si existe
\{\textbackslash{}mathop\{lim\}\}\_\{x→a,x\textbackslash{}mathrel\{≠\}a\}\{
f(x)−f(a) \textbackslash{}over x−a\} ~; dans ce cas cette limite est
appelée la dérivée de f au point a et notée f'(a).

Remarque~8.2.1 Comme toute notion de limite, il s'agit d'une notion
locale~: f : I → E est dérivable au point a si et seulement si~sa
restriction à {]}a − η,a + η{[}∩I est dérivable au point a.

Proposition~8.2.1 Si f est dérivable au point a ∈ I elle est continue au
point a.

Démonstration On écrit pour x\textbackslash{}mathrel\{≠\}a, \{ f(x)−f(a)
\textbackslash{}over x−a\} = f'(a) + ε(x − a) avec
\{\textbackslash{}mathop\{lim\}\}\_\{h→0\}ε(h) = 0~; on a donc f(x) =
f(a) + (x − a)f'(a) + (x − a)ε(x − a) ce qui montre que
\{\textbackslash{}mathop\{lim\}\}\_\{x→a,x\textbackslash{}mathrel\{≠\}a\}f(x)
= f(a)~; donc f est continue au point a.

Définition~8.2.2 Soit I un intervalle de ℝ, E un espace vectoriel
normé~et f : I → E. On dit que f est dérivable si elle est dérivable en
tout point de I~; l'application f' : a\textbackslash{}mathrel\{↦\}f'(a)
est appelée la dérivée de f.

Remarque~8.2.2 On a donc~: f dérivable ⇒ f continue.

\paragraph{8.2.2 Opérations sur les dérivées}

Théorème~8.2.2 Soit I un intervalle de ℝ, E un espace vectoriel normé~et
f et g des applications de I dans E dérivables au point a~; si α et β
sont des scalaires, αf + βg est dérivable au point a et (αf + βg)'(a) =
αf'(a) + βg'(a).

Démonstration Il suffit de remarquer que \{ (αf+βg)(x)−(αf+βg)(a)
\textbackslash{}over x−a\} = α\{ f(x)−f(a) \textbackslash{}over x−a\} +
β\{ g(x)−g(a) \textbackslash{}over x−a\} et d'appliquer les théorèmes
sur les limites.

Théorème~8.2.3 Soit I un intervalle de ℝ, E,F,G trois espaces vectoriels
normés, f : I → E, g : I → F~; soit u : E × F → G une application
bilinéaire continue et h : I → G,
t\textbackslash{}mathrel\{↦\}u(f(t),g(t)). Si f et g sont dérivables au
point a, alors h est dérivable au point a et h'(a) = u(f'(a),g(a)) +
u(f(a),g'(a)).

Démonstration On vérifie immédiatement que \{ h(x)−h(a)
\textbackslash{}over x−a\} = u(\{ f(x)−f(a) \textbackslash{}over x−a\}
,g(x)) + u(f(a),\{ g(x)−g(a) \textbackslash{}over x−a\} ). Or
\{\textbackslash{}mathop\{lim\}\}\_\{x→a,x\textbackslash{}mathrel\{≠\}a\}\{
f(x)−f(a) \textbackslash{}over x−a\} = f'(a),
\{\textbackslash{}mathop\{lim\}\}\_\{x→a,x\textbackslash{}mathrel\{≠\}a\}g(x)
= g(a) et
\{\textbackslash{}mathop\{lim\}\}\_\{x→a,x\textbackslash{}mathrel\{≠\}a\}\{
g(x)−g(a) \textbackslash{}over x−a\} = g'(a). Comme u est continue, on a
\{\textbackslash{}mathop\{lim\}\}\_\{x→a,x\textbackslash{}mathrel\{≠\}a\}\{
h(x)−h(a) \textbackslash{}over x−a\} = u(f'(a),g(a)) + u(f(a),g'(a)).

Remarque~8.2.3 Ce théorème s'étend immédiatement au cas d'une
application p-linéaire continue u : \{E\}\_\{1\}
×\textbackslash{}mathrel\{⋯\} × \{E\}\_\{p\} dans G. Dans ce cas on a

h'(a) =\{ \textbackslash{}mathop\{∑ \}\}\_\{i=1\}\^{}\{p\}u(\{f\}\_\{
1\}(a),\textbackslash{}mathop\{\ldots{}\},\{f\}\_\{i−1\}(a),\{f\}\_\{i\}'(a),\{f\}\_\{i+1\}(a),\textbackslash{}mathop\{\ldots{}\},\{f\}\_\{p\}(a))

En particulier, dans le cas d'un déterminant on retiendra

\textbackslash{}begin\{eqnarray*\}
{[}\{\textbackslash{}mathop\{\textbackslash{}mathrm\{det\}\}
\}\_\{ℰ\}(\{f\}\_\{1\},\textbackslash{}mathop\{\textbackslash{}mathop\{\ldots{}\}\},\{f\}\_\{n\}){]}'(a)\&\&
\%\& \textbackslash{}\textbackslash{} \& =\& \{\textbackslash{}mathop\{∑
\}\}\_\{i=1\}\^{}\{n\}\{ \textbackslash{}mathrm\{det\} \}\_\{
ℰ\}(\{f\}\_\{1\}(a),\textbackslash{}mathop\{\ldots{}\},\{f\}\_\{i−1\}(a),\{f\}\_\{i\}'(a),\{f\}\_\{i+1\}(a),\textbackslash{}mathop\{\ldots{}\},\{f\}\_\{n\}(a))\%\&
\textbackslash{}\textbackslash{} \textbackslash{}end\{eqnarray*\}

Théorème~8.2.4 Soit φ : I → ℝ et f : J → E avec φ(I) ⊂ J~; soit a ∈ I.
Si φ est dérivable au point a et si f est dérivable au point f(a), alors
f ∘ φ est dérivable au point a et (f ∘ φ)'(a) = φ'(a)f'(φ(a)).

Démonstration En effet la dérivabilité de f au point φ(a) peut se
traduire par

f(x) − f(φ(a)) = (x − φ(a))f'(φ(a)) + (x − φ(a))ε(x)

avec \{\textbackslash{}mathop\{lim\}\}\_\{x→φ(a)\}ε(x) = 0. On a donc

f(φ(t)) − f(φ(a)) = (φ(t) − φ(a))f'(φ(a)) + (φ(t) − φ(a))ε(φ(t))

avec \{\textbackslash{}mathop\{lim\}\}\_\{t→a\}ε(φ(t)) = 0 puisque φ est
continue au point a.

De même la dérivabilité de φ au point a se traduit par

φ(t) − φ(a) = (t − a)φ'(a) + o(t − a)

On obtient alors en rempla\textbackslash{}c\{c\}ant

f(φ(t)) − f(φ(a)) = (t − a)φ'(a)f'(φ(a)) + o(t − a)

ce qui montre que

\{\textbackslash{}mathop\{lim\}\}\_\{t→a,t\textbackslash{}mathrel\{≠\}a\}\{f(φ(t))
− f(φ(a))\textbackslash{}over t − a\} = φ'(a)f'(φ(a))

Théorème~8.2.5 Soit f : I → ℝ, a ∈ I tel que
f(a)\textbackslash{}mathrel\{≠\}0. Si f est dérivable au point a, il
existe ε \textgreater{} 0 tel que f ne s'annule pas sur J = I∩{]}a − ε,a
+ ε{[}. La fonction \{ 1 \textbackslash{}over f\} est dérivable au point
a et \textbackslash{}left (\{ 1 \textbackslash{}over f\}
\textbackslash{}right )'(a) = −\{ f'(a) \textbackslash{}over
f\{(a)\}\^{}\{2\}\} .

Démonstration La fonction f étant continue au point a, il existe ε
\textgreater{} 0 tel que t ∈ I∩{]}a − ε,a + ε{[}⇒\textbar{}f(t) −
f(a)\textbar{} \textless{}\{ \textbar{}f(a)\textbar{}
\textbackslash{}over 2\} ~; on en déduit que t ∈ J ⇒
f(t)\textbackslash{}mathrel\{≠\}0. Pour t ∈ J
∖\textbackslash{}\{a\textbackslash{}\} on a \{ 1 \textbackslash{}over
t−a\} \textbackslash{}left (\{ 1 \textbackslash{}over f\} (t) −\{ 1
\textbackslash{}over f\} (a)\textbackslash{}right ) = −\{ 1
\textbackslash{}over f(t)f(a)\} \{ f(t)−f(a) \textbackslash{}over t−a\}
qui tend vers −\{ f'(a) \textbackslash{}over f\{(a)\}\^{}\{2\}\} quand t
tend vers a.

\paragraph{8.2.3 Dérivées d'ordre supérieur}

Définition~8.2.3 Soit I un intervalle de ℝ, E un espace vectoriel
normé~et f : I → E. Soit n ≥ 1. On dit que f est n fois dérivable au
point a ∈ I s'il existe η \textgreater{} 0 tel que f est n − 1 fois
dérivable sur I∩{]}a − η,a + η{[} et si l'application \{f\}\^{}\{(n−1)\}
est dérivable au point a~; on pose alors \{f\}\^{}\{(n)\}(a) =
(\{f\}\^{}\{(n−1)\})'(a). On dit que f est n fois dérivable sur I si
elle est n fois dérivable en tout point de I~; on dit que f est de
classe \{C\}\^{}\{n\} si elle est n fois dérivable sur I et si
\{f\}\^{}\{(n)\} est continue sur I~; on dit que f est \{C\}\^{}\{∞\} si
elle est de classe \{C\}\^{}\{n\} pour tout n.

Remarque~8.2.4 Puisque toute fonction dérivable est continue, si f est n
fois dérivable, elle est de classe \{C\}\^{}\{n−1\}.

Théorème~8.2.6 (Leibnitz). Soit I un intervalle de ℝ, E,F,G trois
espaces vectoriels normés, f : I → E, g : I → F~; soit u : E × F → G une
application bilinéaire continue et h : I → G,
t\textbackslash{}mathrel\{↦\}u(f(t),g(t)). Si f et g sont n fois
dérivables au point a, alors h est n fois dérivable au point a et

\{h\}\^{}\{(n)\}(a) =\{ \textbackslash{}mathop\{∑
\}\}\_\{p=0\}\^{}\{n\}\{C\}\_\{
n\}\^{}\{p\}u(\{f\}\^{}\{(p)\}(a),\{g\}\^{}\{(n−p)\}(a))

Démonstration Par récurrence sur n~; le résultat a déjà été vu pour n =
1~; supposons le vrai pour n − 1 et soit ε \textgreater{} 0 tel que f et
g soient n − 1 fois dérivables sur I∩{]}a − η,a + η{[}. L'hypothèse de
récurrence implique que h est n − 1 fois dérivable sur I∩{]}a − η,a +
η{[} et que sa dérivée \{(n − 1)\}\^{}\{\textbackslash{}text\{ième\}\}
est donnée par

\{h\}\^{}\{(n−1)\}(t) =\{ \textbackslash{}mathop\{∑
\}\}\_\{p=0\}\^{}\{n−1\}\{C\}\_\{
n−1\}\^{}\{p\}u(\{f\}\^{}\{(p)\}(t),\{g\}\^{}\{(n−1−p)\}(t))

Mais toutes les applications \{f\}\^{}\{(p)\}, \{g\}\^{}\{(n−1−p)\} sont
dérivables au point a~; il en est donc de même des applications
t\textbackslash{}mathrel\{↦\}u(\{f\}\^{}\{(p)\}(t),\{g\}\^{}\{(n−1−p)\}(t)),
et donc de \{h\}\^{}\{(n−1)\}. Donc h est n fois dérivable au point a et

\textbackslash{}begin\{eqnarray*\}\{ h\}\^{}\{(n)\}(a)\& =\&
\{\textbackslash{}mathop\{∑ \}\}\_\{p=0\}\^{}\{n−1\}\{C\}\_\{
n−1\}\^{}\{p\}(u(\{f\}\^{}\{(p+1)\}(a),\{g\}\^{}\{(n−1−p)\}(a))\%\&
\textbackslash{}\textbackslash{} \& \textbackslash{}text\{\} \&
\textbackslash{}quad + u(\{f\}\^{}\{(p)\}(a),\{g\}\^{}\{(n−p)\}(a)))
\%\& \textbackslash{}\textbackslash{} \& =\& \{\textbackslash{}mathop\{∑
\}\}\_\{p=1\}\^{}\{n\}\{C\}\_\{
n−1\}\^{}\{p−1\}u(\{f\}\^{}\{(p)\}(a),\{g\}\^{}\{(n−p)\}(a)) \%\&
\textbackslash{}\textbackslash{} \& \textbackslash{}text\{\} \&
\textbackslash{}quad +\{ \textbackslash{}mathop\{∑
\}\}\_\{p=0\}\^{}\{n−1\}\{C\}\_\{
n−1\}\^{}\{p\}u(\{f\}\^{}\{(p)\}(a),\{g\}\^{}\{(n−p)\}(a)) \%\&
\textbackslash{}\textbackslash{} \textbackslash{}end\{eqnarray*\}

en changeant dans la première somme p + 1 en p~; puis

\textbackslash{}begin\{eqnarray*\}\{ h\}\^{}\{(n)\}(a)\& =\&
u(f(a),\{g\}\^{}\{(n)\}(a)) \%\& \textbackslash{}\textbackslash{} \&
\textbackslash{}text\{\} \& +\{\textbackslash{}mathop\{∑
\}\}\_\{p=1\}\^{}\{n−1\}(\{C\}\_\{ n−1\}\^{}\{p−1\} + \{C\}\_\{
n−1\}\^{}\{p\})u(\{f\}\^{}\{(p)\}(a),\{g\}\^{}\{(n−p)\}(a))\%\&
\textbackslash{}\textbackslash{} \& \textbackslash{}text\{\} \&
+u(\{f\}\^{}\{(n)\}(a),g(a)) \%\& \textbackslash{}\textbackslash{} \&
=\& \{\textbackslash{}mathop\{∑ \}\}\_\{p=0\}\^{}\{n\}\{C\}\_\{
n\}\^{}\{p\}u(\{f\}\^{}\{(p)\}(a),\{g\}\^{}\{(n−p)\}(a)) \%\&
\textbackslash{}\textbackslash{} \textbackslash{}end\{eqnarray*\}

ce qui achève la récurrence.

Corollaire~8.2.7 Sous les mêmes hypothèses, si f et g sont de classe
\{C\}\^{}\{n\}, h est de classe \{C\}\^{}\{n\}.

Démonstration C'est clair d'après la formule ci dessus.

Théorème~8.2.8 Soit φ : I → ℝ et f : J → E avec φ(I) ⊂ J~; soit a ∈ I.
Si φ est n fois dérivable au point a et si f est n fois dérivable au
point f(a), alors f ∘ φ est n fois dérivable au point a.

Démonstration Par récurrence sur n~; le résultat a déjà été vu pour n =
1~; supposons le vrai pour n − 1 et soit η \textgreater{} 0 tel que f ∘
φ soit dérivable sur I∩{]}a − η,a + η{[} avec (f ∘ φ)' = φ'(f' ∘ φ).
Comme f' et φ sont n − 1 fois dérivables en a, l'hypothèse de récurrence
implique que f' ∘ φ est n − 1 fois dérivable en a~; comme φ' l'est
également, le théorème de Leibnitz appliqué au produit ordinaire assure
que (f ∘ φ)' = φ'(f' ∘ φ) est n − 1 fois dérivable au point a, donc que
f ∘ φ est n fois dérivable au point a.

Corollaire~8.2.9 Sous les mêmes hypothèses, si f et φ sont de classe
\{C\}\^{}\{n\}, f ∘ φ est de classe \{C\}\^{}\{n\}.

{[}\href{coursse46.html}{next}{]} {[}\href{coursse44.html}{prev}{]}
{[}\href{coursse44.html\#tailcoursse44.html}{prev-tail}{]}
{[}\href{coursse45.html}{front}{]}
{[}\href{coursch9.html\#coursse45.html}{up}{]}

\end{document}
