\documentclass[]{article}
\usepackage[T1]{fontenc}
\usepackage{lmodern}
\usepackage{amssymb,amsmath}
\usepackage{ifxetex,ifluatex}
\usepackage{fixltx2e} % provides \textsubscript
% use upquote if available, for straight quotes in verbatim environments
\IfFileExists{upquote.sty}{\usepackage{upquote}}{}
\ifnum 0\ifxetex 1\fi\ifluatex 1\fi=0 % if pdftex
  \usepackage[utf8]{inputenc}
\else % if luatex or xelatex
  \ifxetex
    \usepackage{mathspec}
    \usepackage{xltxtra,xunicode}
  \else
    \usepackage{fontspec}
  \fi
  \defaultfontfeatures{Mapping=tex-text,Scale=MatchLowercase}
  \newcommand{\euro}{€}
\fi
% use microtype if available
\IfFileExists{microtype.sty}{\usepackage{microtype}}{}
\ifxetex
  \usepackage[setpagesize=false, % page size defined by xetex
              unicode=false, % unicode breaks when used with xetex
              xetex]{hyperref}
\else
  \usepackage[unicode=true]{hyperref}
\fi
\hypersetup{breaklinks=true,
            bookmarks=true,
            pdfauthor={},
            pdftitle={Complements sur la conjugaison},
            colorlinks=true,
            citecolor=blue,
            urlcolor=blue,
            linkcolor=magenta,
            pdfborder={0 0 0}}
\urlstyle{same}  % don't use monospace font for urls
\setlength{\parindent}{0pt}
\setlength{\parskip}{6pt plus 2pt minus 1pt}
\setlength{\emergencystretch}{3em}  % prevent overfull lines
\setcounter{secnumdepth}{0}
 
/* start css.sty */
.cmr-5{font-size:50%;}
.cmr-7{font-size:70%;}
.cmmi-5{font-size:50%;font-style: italic;}
.cmmi-7{font-size:70%;font-style: italic;}
.cmmi-10{font-style: italic;}
.cmsy-5{font-size:50%;}
.cmsy-7{font-size:70%;}
.cmex-7{font-size:70%;}
.cmex-7x-x-71{font-size:49%;}
.msbm-7{font-size:70%;}
.cmtt-10{font-family: monospace;}
.cmti-10{ font-style: italic;}
.cmbx-10{ font-weight: bold;}
.cmr-17x-x-120{font-size:204%;}
.cmsl-10{font-style: oblique;}
.cmti-7x-x-71{font-size:49%; font-style: italic;}
.cmbxti-10{ font-weight: bold; font-style: italic;}
p.noindent { text-indent: 0em }
td p.noindent { text-indent: 0em; margin-top:0em; }
p.nopar { text-indent: 0em; }
p.indent{ text-indent: 1.5em }
@media print {div.crosslinks {visibility:hidden;}}
a img { border-top: 0; border-left: 0; border-right: 0; }
center { margin-top:1em; margin-bottom:1em; }
td center { margin-top:0em; margin-bottom:0em; }
.Canvas { position:relative; }
li p.indent { text-indent: 0em }
.enumerate1 {list-style-type:decimal;}
.enumerate2 {list-style-type:lower-alpha;}
.enumerate3 {list-style-type:lower-roman;}
.enumerate4 {list-style-type:upper-alpha;}
div.newtheorem { margin-bottom: 2em; margin-top: 2em;}
.obeylines-h,.obeylines-v {white-space: nowrap; }
div.obeylines-v p { margin-top:0; margin-bottom:0; }
.overline{ text-decoration:overline; }
.overline img{ border-top: 1px solid black; }
td.displaylines {text-align:center; white-space:nowrap;}
.centerline {text-align:center;}
.rightline {text-align:right;}
div.verbatim {font-family: monospace; white-space: nowrap; text-align:left; clear:both; }
.fbox {padding-left:3.0pt; padding-right:3.0pt; text-indent:0pt; border:solid black 0.4pt; }
div.fbox {display:table}
div.center div.fbox {text-align:center; clear:both; padding-left:3.0pt; padding-right:3.0pt; text-indent:0pt; border:solid black 0.4pt; }
div.minipage{width:100%;}
div.center, div.center div.center {text-align: center; margin-left:1em; margin-right:1em;}
div.center div {text-align: left;}
div.flushright, div.flushright div.flushright {text-align: right;}
div.flushright div {text-align: left;}
div.flushleft {text-align: left;}
.underline{ text-decoration:underline; }
.underline img{ border-bottom: 1px solid black; margin-bottom:1pt; }
.framebox-c, .framebox-l, .framebox-r { padding-left:3.0pt; padding-right:3.0pt; text-indent:0pt; border:solid black 0.4pt; }
.framebox-c {text-align:center;}
.framebox-l {text-align:left;}
.framebox-r {text-align:right;}
span.thank-mark{ vertical-align: super }
span.footnote-mark sup.textsuperscript, span.footnote-mark a sup.textsuperscript{ font-size:80%; }
div.tabular, div.center div.tabular {text-align: center; margin-top:0.5em; margin-bottom:0.5em; }
table.tabular td p{margin-top:0em;}
table.tabular {margin-left: auto; margin-right: auto;}
div.td00{ margin-left:0pt; margin-right:0pt; }
div.td01{ margin-left:0pt; margin-right:5pt; }
div.td10{ margin-left:5pt; margin-right:0pt; }
div.td11{ margin-left:5pt; margin-right:5pt; }
table[rules] {border-left:solid black 0.4pt; border-right:solid black 0.4pt; }
td.td00{ padding-left:0pt; padding-right:0pt; }
td.td01{ padding-left:0pt; padding-right:5pt; }
td.td10{ padding-left:5pt; padding-right:0pt; }
td.td11{ padding-left:5pt; padding-right:5pt; }
table[rules] {border-left:solid black 0.4pt; border-right:solid black 0.4pt; }
.hline hr, .cline hr{ height : 1px; margin:0px; }
.tabbing-right {text-align:right;}
span.TEX {letter-spacing: -0.125em; }
span.TEX span.E{ position:relative;top:0.5ex;left:-0.0417em;}
a span.TEX span.E {text-decoration: none; }
span.LATEX span.A{ position:relative; top:-0.5ex; left:-0.4em; font-size:85%;}
span.LATEX span.TEX{ position:relative; left: -0.4em; }
div.float img, div.float .caption {text-align:center;}
div.figure img, div.figure .caption {text-align:center;}
.marginpar {width:20%; float:right; text-align:left; margin-left:auto; margin-top:0.5em; font-size:85%; text-decoration:underline;}
.marginpar p{margin-top:0.4em; margin-bottom:0.4em;}
.equation td{text-align:center; vertical-align:middle; }
td.eq-no{ width:5%; }
table.equation { width:100%; } 
div.math-display, div.par-math-display{text-align:center;}
math .texttt { font-family: monospace; }
math .textit { font-style: italic; }
math .textsl { font-style: oblique; }
math .textsf { font-family: sans-serif; }
math .textbf { font-weight: bold; }
.partToc a, .partToc, .likepartToc a, .likepartToc {line-height: 200%; font-weight:bold; font-size:110%;}
.chapterToc a, .chapterToc, .likechapterToc a, .likechapterToc, .appendixToc a, .appendixToc {line-height: 200%; font-weight:bold;}
.index-item, .index-subitem, .index-subsubitem {display:block}
.caption td.id{font-weight: bold; white-space: nowrap; }
table.caption {text-align:center;}
h1.partHead{text-align: center}
p.bibitem { text-indent: -2em; margin-left: 2em; margin-top:0.6em; margin-bottom:0.6em; }
p.bibitem-p { text-indent: 0em; margin-left: 2em; margin-top:0.6em; margin-bottom:0.6em; }
.paragraphHead, .likeparagraphHead { margin-top:2em; font-weight: bold;}
.subparagraphHead, .likesubparagraphHead { font-weight: bold;}
.quote {margin-bottom:0.25em; margin-top:0.25em; margin-left:1em; margin-right:1em; text-align:justify;}
.verse{white-space:nowrap; margin-left:2em}
div.maketitle {text-align:center;}
h2.titleHead{text-align:center;}
div.maketitle{ margin-bottom: 2em; }
div.author, div.date {text-align:center;}
div.thanks{text-align:left; margin-left:10%; font-size:85%; font-style:italic; }
div.author{white-space: nowrap;}
.quotation {margin-bottom:0.25em; margin-top:0.25em; margin-left:1em; }
h1.partHead{text-align: center}
.sectionToc, .likesectionToc {margin-left:2em;}
.subsectionToc, .likesubsectionToc {margin-left:4em;}
.subsubsectionToc, .likesubsubsectionToc {margin-left:6em;}
.frenchb-nbsp{font-size:75%;}
.frenchb-thinspace{font-size:75%;}
.figure img.graphics {margin-left:10%;}
/* end css.sty */

\title{Complements sur la conjugaison}
\author{}
\date{}

\begin{document}
\maketitle

\textbf{Warning: \href{http://www.math.union.edu/locate/jsMath}{jsMath}
requires JavaScript to process the mathematics on this page.\\ If your
browser supports JavaScript, be sure it is enabled.}

\begin{center}\rule{3in}{0.4pt}\end{center}

{[}\href{coursse74.html}{next}{]}
{[}\hyperref[tailcoursse73.html]{tail}{]}
{[}\href{coursch14.html\#coursse73.html}{up}{]}

\subsubsection{13.1 Compléments sur la conjugaison}

\paragraph{13.1.1 Applications semi-linéaires}

Définition~13.1.1 Soit E et F deux ℂ-espaces vectoriels et u : E → F. On
dit que u est semi-linéaire si elle vérifie

\begin{itemize}
\itemsep1pt\parskip0pt\parsep0pt
\item
  (i) \textbackslash{}mathop\{∀\}x,y ∈ E, u(x + y) = u(x) + u(y)
\item
  (ii) \textbackslash{}mathop\{∀\}λ ∈ ℂ, \textbackslash{}mathop\{∀\}x ∈
  E, u(λx) = \textbackslash{}overline\{λ\}u(x).
\end{itemize}

Remarque~13.1.1 Soit E un ℂ-espace vectoriel . On munit E d'une autre
structure d'espace vectoriel, notée \textbackslash{}check\{E\} en posant
λ ∗ x = \textbackslash{}overline\{λ\}x. Une application semi-linéaire de
E dans F n'est autre qu'une application linéaire de E dans
\textbackslash{}check\{F\}. Ceci permet d'appliquer aux applications
semi-linéaires la plupart des résultats sur les applications linéaires
en tenant compte des résultats suivants dont la démonstration est
élémentaire~:

\begin{itemize}
\item
  a) une famille \{(\{x\}\_\{i\})\}\_\{i∈I\} d'éléments de E est libre
  (resp. génératrice, resp. base) dans \textbackslash{}check\{E\} si et
  seulement si~il en est de même dans E
\item
  b) \{\textbackslash{}mathop\{\textbackslash{}mathrm\{rg\}\}
  \}\_\{\textbackslash{}check\{E\}\}\{(\{x\}\_\{i\})\}\_\{i∈I\}
  =\{\textbackslash{}mathop\{ \textbackslash{}mathrm\{rg\}\}
  \}\_\{E\}\{(\{x\}\_\{i\})\}\_\{i∈I\}, \textbackslash{}mathop\{dim\}
  \textbackslash{}check\{E\} =\textbackslash{}mathop\{ dim\} E
\item
  c) F est un sous-espace vectoriel de \textbackslash{}check\{E\} si et
  seulement si~c'est un sous-espace vectoriel de E
\item
  d) le théorème du rang s'applique aux applications semi-linéaires~; en
  particulier, si u : E → F est semi-linéaire entre deux espaces de même
  dimension finie, alors u est injective si et seulement si~elle est
  surjective
\item
  e) si l'on définit A =\textbackslash{}mathop\{
  \textbackslash{}mathrm\{Mat\}\} (u,ℰ,ℱ) par u(\{e\}\_\{j\})
  =\{\textbackslash{}mathop\{ \textbackslash{}mathop\{∑ \}\}
  \}\_\{i\}\{a\}\_\{i,j\}\{f\}\_\{i\} (notations évidentes) alors

  y = u(x) \textbackslash{}mathrel\{⇔\} Y =
  A\textbackslash{}overline\{X\}
\item
  f) la composée de deux applications semi-linéaires n'est pas
  semi-linéaire, mais au contraire linéaire.
\end{itemize}

\paragraph{13.1.2 Matrices conjuguées et transconjuguées}

Définition~13.1.2 Soit A = \{(\{a\}\_\{i,j\})\}\_\{1≤i≤m,1≤j≤n\} ∈
\{M\}\_\{ℂ\}(m,n). On appelle matrice conjuguée de A la matrice
\textbackslash{}overline\{A\} =
\{(\textbackslash{}overline\{\{a\}\_\{i,j\}\})\}\_\{1≤i≤m,1≤j≤n\} ∈
\{M\}\_\{ℂ\}(m,n).

Proposition~13.1.1 L'application
A\textbackslash{}mathrel\{↦\}\textbackslash{}overline\{A\} est un
automorphisme semi-linéaire de \{M\}\_\{ℂ\}(m,n). On a
\textbackslash{}mathop\{\textbackslash{}mathrm\{rg\}\}\textbackslash{}overline\{A\}
=\textbackslash{}mathop\{ \textbackslash{}mathrm\{rg\}\}A. Si A ∈
\{M\}\_\{ℂ\}(m,n) et B ∈ \{M\}\_\{ℂ\}(n,p), alors
\textbackslash{}overline\{AB\} =
\textbackslash{}overline\{A\}\textbackslash{},\textbackslash{}overline\{B\}.
Dans le cadre des matrices carrées, on a
\textbackslash{}mathop\{\textbackslash{}mathrm\{det\}\}
\textbackslash{}overline\{A\} =
\textbackslash{}overline\{\textbackslash{}mathop\{\textbackslash{}mathrm\{det\}\}
A\},
\textbackslash{}mathop\{\textbackslash{}mathrm\{tr\}\}\textbackslash{}overline\{A\}
=
\textbackslash{}overline\{\textbackslash{}mathop\{\textbackslash{}mathrm\{tr\}\}A\},
\{χ\}\_\{\textbackslash{}overline\{A\}\}(X) =
\textbackslash{}overline\{\{χ\}\_\{A\}(X)\}, A est inversible si et
seulement si~\textbackslash{}overline\{A\} est inversible, et dans ce
cas \{(\textbackslash{}overline\{A\})\}\^{}\{−1\} =
\textbackslash{}overline\{\{A\}\^{}\{−1\}\}.

Démonstration Vérification élémentaire laissée au lecteur.

Définition~13.1.3 Soit A = \{(\{a\}\_\{i,j\})\}\_\{1≤i≤m,1≤j≤n\} ∈
\{M\}\_\{ℂ\}(m,n). On appelle matrice transconjuguée (ou matrice
adjointe) de A la matrice \{A\}\^{}\{∗\} \{=
\}\^{}\{t\}(\textbackslash{}overline\{A\}) =
\textbackslash{}overline\{\{\}\^{}\{t\}A\} =
\{(\textbackslash{}overline\{\{a\}\_\{j,i\}\})\}\_\{1≤i≤m,1≤j≤n\} ∈
\{M\}\_\{ℂ\}(n,m).

A partir des propriétés de A\{\textbackslash{}mathrel\{↦\}\}\^{}\{t\}A
et de A\textbackslash{}mathrel\{↦\}\textbackslash{}overline\{A\} on
déduit facilement les propriétés suivantes

Proposition~13.1.2 L'application
A\textbackslash{}mathrel\{↦\}\{A\}\^{}\{∗\} est un isomorphisme
semi-linéaire involutif de \{M\}\_\{ℂ\}(m,n) sur \{M\}\_\{ℂ\}(n,m). On a
\textbackslash{}mathop\{\textbackslash{}mathrm\{rg\}\}\{A\}\^{}\{∗\}
=\textbackslash{}mathop\{ \textbackslash{}mathrm\{rg\}\}A. Si A ∈
\{M\}\_\{ℂ\}(m,n) et B ∈ \{M\}\_\{ℂ\}(n,p), alors \{(AB)\}\^{}\{∗\} =
\{B\}\^{}\{∗\}\{A\}\^{}\{∗\}. Dans le cadre des matrices carrées, on a
\textbackslash{}mathop\{\textbackslash{}mathrm\{det\}\} \{A\}\^{}\{∗\} =
\textbackslash{}overline\{\textbackslash{}mathop\{\textbackslash{}mathrm\{det\}\}
A\},
\textbackslash{}mathop\{\textbackslash{}mathrm\{tr\}\}\{A\}\^{}\{∗\} =
\textbackslash{}overline\{\textbackslash{}mathop\{\textbackslash{}mathrm\{tr\}\}A\},
\{χ\}\_\{\{A\}\^{}\{∗\}\}(X) =
\textbackslash{}overline\{\{χ\}\_\{A\}(X)\}, A est inversible si et
seulement si~\{A\}\^{}\{∗\} est inversible, et dans ce cas
\{(\{A\}\^{}\{∗\})\}\^{}\{−1\} = \{(\{A\}\^{}\{−1\})\}\^{}\{∗\}.

Remarque~13.1.2 On prendra garde à la relation \{(λA)\}\^{}\{∗\} =
\textbackslash{}overline\{λ\}\{A\}\^{}\{∗\} en n'oubliant pas la
conjugaison.

\paragraph{13.1.3 Matrices hermitiennes, antihermitiennes}

Définition~13.1.4 Soit A ∈ \{M\}\_\{ℂ\}(n). on dit que A est hermitienne
(resp. antihermitienne) si \{A\}\^{}\{∗\} = A (resp. \{A\}\^{}\{∗\} =
−A).

Remarque~13.1.3 A = (\{a\}\_\{i,j\}) est hermitienne si et seulement
si~\textbackslash{}mathop\{∀\}i,j, \{a\}\_\{j,i\} =
\textbackslash{}overline\{\{a\}\_\{i,j\}\}. En particulier les
coefficients diagonaux \{a\}\_\{i,i\} doivent être réels

Théorème~13.1.3 Les ensembles des matrices hermitiennes et
antihermitiennes sont des ℝ-sous-espaces vectoriels (mais pas des
ℂ-sous-espaces vectoriels) de \{M\}\_\{ℂ\}(n). On a

A\textbackslash{}text\{ hermitienne \} \textbackslash{}mathrel\{⇔\}
iA\textbackslash{}text\{ antihermitienne\}

Si \{ℋ\}\_\{n\} désigne le ℝ-sous-espace vectoriel des matrices
hermitiennes, on a \{M\}\_\{ℂ\}(n) = \{ℋ\}\_\{n\} ⊕ i\{ℋ\}\_\{n\}.

Démonstration La vérification du premier point est élémentaire. Si on a
A = \{A\}\_\{1\} + i\{A\}\_\{2\} avec \{A\}\_\{1\} et \{A\}\_\{2\}
hermitiennes, alors \{A\}\^{}\{∗\} = \{A\}\_\{1\} − i\{A\}\_\{2\} ce qui
donne \{A\}\_\{1\} =\{ 1 \textbackslash{}over 2\} (A + \{A\}\^{}\{∗\})
et \{A\}\_\{2\} =\{ 1 \textbackslash{}over 2i\} (A − \{A\}\^{}\{∗\}) et
démontre déjà l'unicité de la décomposition. De plus la formule

A =\{ 1 \textbackslash{}over 2\} (A + \{A\}\^{}\{∗\}) + i\{ 1
\textbackslash{}over 2i\} (A − \{A\}\^{}\{∗\})

avec \{ 1 \textbackslash{}over 2\} (A + \{A\}\^{}\{∗\}) et \{ 1
\textbackslash{}over 2i\} (A − \{A\}\^{}\{∗\}) qui sont hermitiennes
(facile) montre l'existence de la décomposition.

Remarque~13.1.4 On voit donc que contrairement aux matrices symétriques
ou antisymétriques qui sont de nature différentes, il n'y a pas de
différence essentielle entre matrices hermitiennes ou antihermitiennes~:
on passe des unes aux autres par multiplication par i, ce qui permet de
limiter l'étude aux matrices hermitiennes. Pour une telle matrice, les
formules \textbackslash{}mathop\{\textbackslash{}mathrm\{det\}\}
\{A\}\^{}\{∗\} =
\textbackslash{}overline\{\textbackslash{}mathop\{\textbackslash{}mathrm\{det\}\}
A\},
\textbackslash{}mathop\{\textbackslash{}mathrm\{tr\}\}\{A\}\^{}\{∗\} =
\textbackslash{}overline\{\textbackslash{}mathop\{\textbackslash{}mathrm\{tr\}\}A\},
\{χ\}\_\{\{A\}\^{}\{∗\}\}(X) =
\textbackslash{}overline\{\{χ\}\_\{A\}(X)\} montrent que
\textbackslash{}mathop\{\textbackslash{}mathrm\{det\}\} A ∈ ℝ,
\textbackslash{}mathop\{\textbackslash{}mathrm\{tr\}\}A ∈ ℝ et que
\{χ\}\_\{A\}(X) ∈ ℝ{[}X{]}.

{[}\href{coursse74.html}{next}{]} {[}\href{coursse73.html}{front}{]}
{[}\href{coursch14.html\#coursse73.html}{up}{]}

\end{document}
