\documentclass[]{article}
\usepackage[T1]{fontenc}
\usepackage{lmodern}
\usepackage{amssymb,amsmath}
\usepackage{ifxetex,ifluatex}
\usepackage{fixltx2e} % provides \textsubscript
% use upquote if available, for straight quotes in verbatim environments
\IfFileExists{upquote.sty}{\usepackage{upquote}}{}
\ifnum 0\ifxetex 1\fi\ifluatex 1\fi=0 % if pdftex
  \usepackage[utf8]{inputenc}
\else % if luatex or xelatex
  \ifxetex
    \usepackage{mathspec}
    \usepackage{xltxtra,xunicode}
  \else
    \usepackage{fontspec}
  \fi
  \defaultfontfeatures{Mapping=tex-text,Scale=MatchLowercase}
  \newcommand{\euro}{€}
\fi
% use microtype if available
\IfFileExists{microtype.sty}{\usepackage{microtype}}{}
\ifxetex
  \usepackage[setpagesize=false, % page size defined by xetex
              unicode=false, % unicode breaks when used with xetex
              xetex]{hyperref}
\else
  \usepackage[unicode=true]{hyperref}
\fi
\hypersetup{breaklinks=true,
            bookmarks=true,
            pdfauthor={},
            pdftitle={Fonctions vectorielles d'une variable reelle},
            colorlinks=true,
            citecolor=blue,
            urlcolor=blue,
            linkcolor=magenta,
            pdfborder={0 0 0}}
\urlstyle{same}  % don't use monospace font for urls
\setlength{\parindent}{0pt}
\setlength{\parskip}{6pt plus 2pt minus 1pt}
\setlength{\emergencystretch}{3em}  % prevent overfull lines
\setcounter{secnumdepth}{0}
 
/* start css.sty */
.cmr-5{font-size:50%;}
.cmr-7{font-size:70%;}
.cmmi-5{font-size:50%;font-style: italic;}
.cmmi-7{font-size:70%;font-style: italic;}
.cmmi-10{font-style: italic;}
.cmsy-5{font-size:50%;}
.cmsy-7{font-size:70%;}
.cmex-7{font-size:70%;}
.cmex-7x-x-71{font-size:49%;}
.msbm-7{font-size:70%;}
.cmtt-10{font-family: monospace;}
.cmti-10{ font-style: italic;}
.cmbx-10{ font-weight: bold;}
.cmr-17x-x-120{font-size:204%;}
.cmsl-10{font-style: oblique;}
.cmti-7x-x-71{font-size:49%; font-style: italic;}
.cmbxti-10{ font-weight: bold; font-style: italic;}
p.noindent { text-indent: 0em }
td p.noindent { text-indent: 0em; margin-top:0em; }
p.nopar { text-indent: 0em; }
p.indent{ text-indent: 1.5em }
@media print {div.crosslinks {visibility:hidden;}}
a img { border-top: 0; border-left: 0; border-right: 0; }
center { margin-top:1em; margin-bottom:1em; }
td center { margin-top:0em; margin-bottom:0em; }
.Canvas { position:relative; }
li p.indent { text-indent: 0em }
.enumerate1 {list-style-type:decimal;}
.enumerate2 {list-style-type:lower-alpha;}
.enumerate3 {list-style-type:lower-roman;}
.enumerate4 {list-style-type:upper-alpha;}
div.newtheorem { margin-bottom: 2em; margin-top: 2em;}
.obeylines-h,.obeylines-v {white-space: nowrap; }
div.obeylines-v p { margin-top:0; margin-bottom:0; }
.overline{ text-decoration:overline; }
.overline img{ border-top: 1px solid black; }
td.displaylines {text-align:center; white-space:nowrap;}
.centerline {text-align:center;}
.rightline {text-align:right;}
div.verbatim {font-family: monospace; white-space: nowrap; text-align:left; clear:both; }
.fbox {padding-left:3.0pt; padding-right:3.0pt; text-indent:0pt; border:solid black 0.4pt; }
div.fbox {display:table}
div.center div.fbox {text-align:center; clear:both; padding-left:3.0pt; padding-right:3.0pt; text-indent:0pt; border:solid black 0.4pt; }
div.minipage{width:100%;}
div.center, div.center div.center {text-align: center; margin-left:1em; margin-right:1em;}
div.center div {text-align: left;}
div.flushright, div.flushright div.flushright {text-align: right;}
div.flushright div {text-align: left;}
div.flushleft {text-align: left;}
.underline{ text-decoration:underline; }
.underline img{ border-bottom: 1px solid black; margin-bottom:1pt; }
.framebox-c, .framebox-l, .framebox-r { padding-left:3.0pt; padding-right:3.0pt; text-indent:0pt; border:solid black 0.4pt; }
.framebox-c {text-align:center;}
.framebox-l {text-align:left;}
.framebox-r {text-align:right;}
span.thank-mark{ vertical-align: super }
span.footnote-mark sup.textsuperscript, span.footnote-mark a sup.textsuperscript{ font-size:80%; }
div.tabular, div.center div.tabular {text-align: center; margin-top:0.5em; margin-bottom:0.5em; }
table.tabular td p{margin-top:0em;}
table.tabular {margin-left: auto; margin-right: auto;}
div.td00{ margin-left:0pt; margin-right:0pt; }
div.td01{ margin-left:0pt; margin-right:5pt; }
div.td10{ margin-left:5pt; margin-right:0pt; }
div.td11{ margin-left:5pt; margin-right:5pt; }
table[rules] {border-left:solid black 0.4pt; border-right:solid black 0.4pt; }
td.td00{ padding-left:0pt; padding-right:0pt; }
td.td01{ padding-left:0pt; padding-right:5pt; }
td.td10{ padding-left:5pt; padding-right:0pt; }
td.td11{ padding-left:5pt; padding-right:5pt; }
table[rules] {border-left:solid black 0.4pt; border-right:solid black 0.4pt; }
.hline hr, .cline hr{ height : 1px; margin:0px; }
.tabbing-right {text-align:right;}
span.TEX {letter-spacing: -0.125em; }
span.TEX span.E{ position:relative;top:0.5ex;left:-0.0417em;}
a span.TEX span.E {text-decoration: none; }
span.LATEX span.A{ position:relative; top:-0.5ex; left:-0.4em; font-size:85%;}
span.LATEX span.TEX{ position:relative; left: -0.4em; }
div.float img, div.float .caption {text-align:center;}
div.figure img, div.figure .caption {text-align:center;}
.marginpar {width:20%; float:right; text-align:left; margin-left:auto; margin-top:0.5em; font-size:85%; text-decoration:underline;}
.marginpar p{margin-top:0.4em; margin-bottom:0.4em;}
.equation td{text-align:center; vertical-align:middle; }
td.eq-no{ width:5%; }
table.equation { width:100%; } 
div.math-display, div.par-math-display{text-align:center;}
math .texttt { font-family: monospace; }
math .textit { font-style: italic; }
math .textsl { font-style: oblique; }
math .textsf { font-family: sans-serif; }
math .textbf { font-weight: bold; }
.partToc a, .partToc, .likepartToc a, .likepartToc {line-height: 200%; font-weight:bold; font-size:110%;}
.chapterToc a, .chapterToc, .likechapterToc a, .likechapterToc, .appendixToc a, .appendixToc {line-height: 200%; font-weight:bold;}
.index-item, .index-subitem, .index-subsubitem {display:block}
.caption td.id{font-weight: bold; white-space: nowrap; }
table.caption {text-align:center;}
h1.partHead{text-align: center}
p.bibitem { text-indent: -2em; margin-left: 2em; margin-top:0.6em; margin-bottom:0.6em; }
p.bibitem-p { text-indent: 0em; margin-left: 2em; margin-top:0.6em; margin-bottom:0.6em; }
.paragraphHead, .likeparagraphHead { margin-top:2em; font-weight: bold;}
.subparagraphHead, .likesubparagraphHead { font-weight: bold;}
.quote {margin-bottom:0.25em; margin-top:0.25em; margin-left:1em; margin-right:1em; text-align:\\jmathmathustify;}
.verse{white-space:nowrap; margin-left:2em}
div.maketitle {text-align:center;}
h2.titleHead{text-align:center;}
div.maketitle{ margin-bottom: 2em; }
div.author, div.date {text-align:center;}
div.thanks{text-align:left; margin-left:10%; font-size:85%; font-style:italic; }
div.author{white-space: nowrap;}
.quotation {margin-bottom:0.25em; margin-top:0.25em; margin-left:1em; }
h1.partHead{text-align: center}
.sectionToc, .likesectionToc {margin-left:2em;}
.subsectionToc, .likesubsectionToc {margin-left:4em;}
.subsubsectionToc, .likesubsubsectionToc {margin-left:6em;}
.frenchb-nbsp{font-size:75%;}
.frenchb-thinspace{font-size:75%;}
.figure img.graphics {margin-left:10%;}
/* end css.sty */

\title{Fonctions vectorielles d'une variable reelle}
\author{}
\date{}

\begin{document}
\maketitle

\textbf{Warning: 
requires JavaScript to process the mathematics on this page.\\ If your
browser supports JavaScript, be sure it is enabled.}

\begin{center}\rule{3in}{0.4pt}\end{center}

{[}
{[}
{[}{]}
{[}

\subsubsection{8.4 Fonctions vectorielles d'une variable réelle}

\paragraph{8.4.1 Inégalité des accroissements finis}

Dans le cas d'une fonction vectorielle d'une variable réelle, le
théorème de Rolle et le théorème des accroissements finis ne sont plus
valables comme le montre l'exemple de la fonction f :
t\mapsto~e^it entre 0 et 2\pi~ (on a f(0) =
f(2\pi~) et cependant la dérivée f'(t) = ie^it ne s'annule pas).
On obtient uniquement une inégalité que l'on peut mettre sous une forme
plus générale

Théorème~8.4.1 (inégalité des accroissements finis) . Soit f : {[}a,b{]}
\rightarrow~ E et g : {[}a,b{]} \rightarrow~ \mathbb{R}~. On suppose que f et g sont continues sur
{[}a,b{]} et dérivables sur {]}a,b{[} avec \forall~~t
\in{]}a,b{[}, \f'(t)\ \leq
g'(t). Alors \f(b) -
f(a)\ \leq g(b) - g(a).

Démonstration (Première démonstration) Si on suppose en plus que f et g
sont de classe \mathcal{C}^1 sur {]}a,b{[}, on peut écrire pour a
\textless{} x \textless{} y \textless{} b,

\begin{align*} \f(y) -
f(x)& =&
\\int ~
_x^yf'(t) dt\ \%&
\\ & \leq& \int ~
_x^y\f'(t)\
dt \leq\int  _x^y~g'(t) dt = g(y) -
g(x)\%& \\
\end{align*}

Il ne reste plus qu'à faire tendre x vers a et y vers b pour obtenir
l'inégalité souhaitée.

Remarque~8.4.1 Attention~! L'utilisation inconsidérée de cette première
démonstration peut provoquer un cercle vicieux dans l'exposé~: la
formule f(y) - f(x) =\int ~
_x^yf'(t) dt fait appel de manière cachée au fait qu'une
fonction de dérivée nulle est constante, ce qui est en général vu comme
une conséquence de l'inégalité des accroissements finis~!

Démonstration (Deuxième démonstration) Dans le cas général, soit \epsilon
\textgreater{} 0, soit \phi_\epsilon(t) =\ f(t)
- f(a)\ - (g(t) - g(a)) - \epsilon(t - a) et
X_\epsilon = \t \in
{[}a,b{]}∣\phi_\epsilon(t) \leq
\epsilon\. On a X_\epsilon = \phi_\epsilon^-1({]}
-\infty~,\epsilon{]}) et comme \phi_\epsilon est continue, X_\epsilon est fermé
(image réciproque d'un fermé). De plus \phi_\epsilon(a) = 0, donc il
existe \eta \textgreater{} 0 tel que {[}a,a + \eta{]} \subset~ X_\epsilon. Soit c
= supX_\epsilon~. Supposons que c \textless{}
b. Comme c ≥ a + \eta, on a c \in{]}a,b{[} et donc f et g sont dérivables au
point c. On a \f'(c)\
= lim_t\rightarrow~c~\
f(t)-f(c) \over t-c \ et g'(c)
= lim_t\rightarrow~c~ g(t)-g(c)
\over t-c . Donc il existe \alpha~ \textgreater{} 0 tel que
pour t \in{]}c,c + \alpha~{[} on ait à la fois \
f(t)-f(c) \over t-c \
\leq\ f'(c)\ + \epsilon
\over 2 et  g(t)-g(c) \over t-c ≥
g'(c) - \epsilon \over 2 . Tenant compte de
\f'(c)\ \leq g'(c), on
obtient

\ f(t) - f(c) \over t - c
\ \leq\
f'(c)\ + \epsilon \over 2 \leq g'(c)
+ \epsilon \over 2 \leq g(t) - g(c) \over t -
c + \epsilon

soit encore \f(t) -
f(c)\ \leq g(t) - g(c) + \epsilon(t - c). Comme c \in
X_\epsilon, on a \f(c) -
f(a)\ \leq g(c) - g(a) + \epsilon(c - a) + \epsilon et donc pour
t \in{]}c,c + \alpha~{[}, \f(t) -
f(a)\ \leq\ f(t) -
f(c)\ +\ f(c) -
f(a)\ \leq g(t) - g(a) + \epsilon(t - a) + \epsilon, soit encore
\phi_\epsilon(t) \leq \epsilon. On a donc {]}c,c + \alpha~{[}\subset~ X_\epsilon ce qui
contredit la définition de c =\
supX_\epsilon. On a donc b = c \in X_\epsilon, soit encore
\f(b) - f(a)\ \leq g(b) -
g(a) + \epsilon(b - a) + \epsilon. En faisant tendre \epsilon vers 0, on trouve alors
l'inégalité \f(b) -
f(a)\ \leq g(b) - g(a).

En fait on utilisera la plupart du temps la version suivante du théorème
précédent

Corollaire~8.4.2 Soit f : {[}a,b{]} \rightarrow~ E, continue sur {[}a,b{]}
dérivable sur {]}a,b{[} telle que \forall~~t
\in{]}a,b{[}, \f'(t)\ \leq
M. Alors \f(b) - f(a)\
\leq M(b - a).

Démonstration Il suffit d'appliquer l'inégalité des accroissements finis
à g(t) = Mt pour laquelle on a g'(t) = M et g(b) - g(a) = M(b - a).

\paragraph{8.4.2 Applications de l'inégalité des accroissements finis}

Théorème~8.4.3 Soit f : I \rightarrow~ E, continue sur I, dérivable sur
I^o. Alors f est k-lipschitzienne si et seulement
si~\forall~t \in I^o~,
\f'(t)\ \leq k.

Démonstration Si f est k-lipschitzienne, on a pour a \in I^o et
t\neq~a, \ f(t)-f(a)
\over t-a \ \leq k d'où en
faisant tendre t vers a,
\f'(a)\ \leq k. La
condition est donc nécessaire. Réciproquement, supposons que
\forall~t \in I^o~,
\f'(t)\ \leq k, soit a,b
\in I tels que a \textless{} b. Alors {[}a,b{]} \subset~ I et {]}a,b{[}\subset~
I^o, on peut donc appliquer le corollaire de l'inégalité des
accroissements finis, \f(b) -
f(a)\ \leq k(b - a), ce qui montre que f est
k-lipschitzienne.

Remarque~8.4.2 Ce théorème peut permettre en particulier de caractériser
les applications contractantes.

Théorème~8.4.4 Soit E un espace vectoriel normé complet et f :{]}a,b{[}\rightarrow~
E dérivable, telle que la fonction f' admet une limite \ell au point a.
Alors f se prolonge en une application continue
\tildef : {[}a,b{[}\rightarrow~ E. L'application
\tildef est dérivable sur {[}a,b{[} et
\tildef'(a) = \ell.

Démonstration Il existe \eta_0 \textgreater{} 0 tel que
\forall~t \in{]}a,a + \eta_0~{[},
\f'(t)\
\leq\ \ell\ + 1. Donc f est
lipschitzienne sur {]}a,a + \eta_0{[}. En particulier si une suite
(x_n) de {]}a,b{[} admet la limite a, c'est une suite de
Cauchy, donc son image par f est encore une suite de Cauchy. Comme E est
complet, la suite (f(x_n)) est donc convergente. Pour toute
suite (x_n) de limite a, la suite (f(x_n)) admet une
limite, donc f admet une limite L au point a. Si l'on pose alors
\tildef(a) = L et \tildef(t) =
f(t) si t \in{]}a,b{[}, la fonction \tildef est donc
continue sur {[}a,b{[} et dérivable sur {]}a,b{[} avec
\tildef'(t) = f'(t). Montrons que
\tildef est dérivable au point a. Posons g(t)
=\tilde f(t) - \ellt et soit \epsilon \textgreater{} 0. Il
existe \eta \textgreater{} 0 tel que t \in{]}a,a +
\eta{[}\rigtharrow~\ g'(t)\
=\ f'(t) - \ell\
\textless{} \epsilon. On en déduit que g est \epsilon-lipschitzienne sur {[}a,a + \eta{]}
et en particulier pour t \in {[}a,a + \eta{]}, \g(t)
- g(a)\ \leq \epsilon(t - a) soit encore (après division
par t - a), \
\tildef(t)-\tildef(a)
\over t-a - \ell\ \leq \epsilon, ce qui
montre que \tildef est dérivable au point a et que sa
dérivée en ce point est \ell.

Remarque~8.4.3 Si f est continue sur {[}a,b{[}, dérivable sur {]}a,b{[}
et si f' admet la limite \ell au point a, on a évidemment L = f(a) et donc
\tildef = f. Travaillant séparément à gauche de a et
à droite de a, on obtient le corollaire suivant

Corollaire~8.4.5 Soit f : I \rightarrow~ E, a \in I. On suppose que f est continue
sur I, dérivable sur I \diagdown\a\ et que f'
admet une limite \ell au point a. Alors f est dérivable au point a et f'(a)
= \ell.

Remarque~8.4.4 Une récurrence évidente à partir du théorème ci dessus
montre que si f est continue sur {[}a,b{[}, n fois dérivable sur
{]}a,b{[} et si f^(n) admet une limite \ell au point a, alors
toutes les dérivées intermédiaires f^(k) admettent une limite
au point a~; ceci permet alors d'appliquer le corollaire ci dessus qui
garantira que f est n fois dérivable au point a (et que toutes les
dérivées f^(k) sont continues au point a). Par contre la même
méthode ne peut s'appliquer sur I
\diagdown\a\, rien ne garantissant que les
limites à droite et à gauche de f^(k) sont les mêmes~:
l'exemple de la fonction x \rightarrow~x dont la dérivée
seconde sur \mathbb{R}~ \diagdown\0\ est nulle fournit
un contre exemple évident.

\paragraph{8.4.3 Formules de Taylor}

Théorème~8.4.6 (inégalité de Taylor-Lagrange). Soit f : {[}a,b{]} \rightarrow~ E
(resp. f : {[}b,a{]} \rightarrow~ E) de classe C^n~; on suppose que f
est n + 1 fois dérivable sur {]}a,b{[} (resp. {]}a,b{[}) et que
\f^(n+1)(t)\
\leq M. Alors

\f(b) - f(a) -\\sum
_k=1^n f^(k)(a) \over k!
(b - a)^k\ \leq M
\over (n + 1)! (b - a)^n+1

Démonstration Posons \phi(t) = f(b) - f(t)
-\\sum ~
_k=1^n f^(k)(t) \over k!
(b - t)^k. Il est clair que \phi est continue sur {[}a,b{]},
dérivable sur {]}a,b{[} comme toutes les fonctions f^(k), 0 \leq
k \leq n. De plus

\begin{align*} \phi'(t)&& \%&
\\ & =& -f'(t)
-\sum _k=1^n~
f^(k+1)(t) \over k! (b - t)^k +
\sum _k=1^n f^(k)~(t)
\over (k - 1)! (b - t)^k-1 \%&
\\ & =& -f'(t)
-\sum _l=2^n+1~
f^(l)(t) \over (l - 1)! (b -
t)^l-1 + \sum _k=1^n~
f^(k)(t) \over (k - 1)! (b -
t)^k-1\%& \\ & =& -
f^(n+1)(t) \over n! (b - t)^n
\%& \\ \end{align*}

(tous les autres termes se détruisent deux à deux). On a donc
\\phi'(t)\ \leq M
(b-t)^n \over n! = \psi'(t) pour \psi(t) = -M
(b-t)^n+1 \over (n+1)! . L'inégalité des
accroissements finis assure que \\phi(b) -
\phi(a)\ \leq \psi(b) - \psi(a), soit encore
\\phi(a)\ \leq-\psi(a) ce qui
n'est autre que l'inégalité à démontrer.

Théorème~8.4.7 (formule de Taylor Young). Soit I un intervalle de \mathbb{R}~, a \in
I et f : I \rightarrow~ E, n fois dérivable au point a. Alors, au voisinage de a,

f(t) = f(a) + \sum _k=1^n~
f^(k)(a) \over k! (t - a)^k +
o((t - a)^n)

Démonstration On montre le résultat par récurrence sur n. Pour n = 1, il
s'agit seulement de la définition de la dérivée~: en posant \epsilon(t - a) =
f(t)-f(a) \over t-a - f'(a), on a f(t) = f(a) + (t -
a)f'(a) + (t - a)\epsilon(t - a) avec
lim_t\rightarrow~a~\epsilon(t - a) = 0. Supposons donc
le résultat vrai pour n - 1 et soit \eta_0 \textgreater{} 0 tel
que f soit n - 1 fois dérivable sur {]}a,-\eta_0,a +
\eta_0{[}\bigcapI. On peut alors appliquer notre hypothèse de récurrence
à la fonction f' sur {]}a,-\eta_0,a + \eta_0{[}\bigcapI puisque
celle ci est n - 1 fois dérivable au point a. On a donc f'(t) = f'(a)
+ \\sum ~
_k=1^n-1 f^(k+1)(a) \over k!
(t - a)^k + o((t - a)^n-1). Etant donné \epsilon
\textgreater{} 0, il existe donc \eta \textgreater{} 0 tel que, pour t
\in{]}a,-\eta,a + \eta{[}\bigcapI, \f'(t) - f'(a)
+ \\sum ~
_k=1^n-1 f^(k+1)(a) \over k!
(t - a)^k\ \leq \epsilont -
a^n-1. Pour t \in{]}a,a + \eta{[}, ceci s'écrit encore
\\phi'(t)\ \leq \psi'(t) avec
\phi(t) = f(t) - f(a) -\\\sum
 _k=1^n f^(k)(a) \over
k! (t - a)^k et \psi(t) = \epsilon(t-a)^n
\over n . L'inégalité des accroissements finis (dont
les conditions de validité sur {[}a,t{]} sont évidemment vérifiées)
assure que \\phi(t) -
\phi(a)\ \leq \psi(t) - \psi(a) soit encore

\f(t) - f(a) -\\sum
_k=1^n f^(k)(a) \over k!
(t - a)^k\ \leq \epsilon(t - a)^n
\over n

On a donc f(t) - f(a)
-\\sum ~
_k=1^n f^(k)(a) \over k!
(t - a)^k = o((t - a)^n) en a, à droite de a. On
montre de manière similaire avec \psi(t) = (-1)^n
\epsilon(t-a)^n \over n , le même résultat à gauche
de a.

Remarque~8.4.5 On prendra soin de ne pas confondre l'inégalité de Taylor
Lagrange (ou la formule de Taylor Lagrange pour les fonctions à valeurs
réelles) qui donne une estimation globale de la fonction f sur tout un
intervalle, avec la formule de Taylor Young qui donne un comportement
local de la fonction (en fait un développement limité).

On montre en intégration le résultat suivant (d'où l'on peut d'ailleurs
déduire facilement l'inégalité de Taylor Lagrange, mais avec des
conditions plus fortes de validité)~; la démonstration consiste
simplement en n intégrations par parties successives.

Théorème~8.4.8 (formule de Taylor avec reste intégral). Soit f : I \rightarrow~ E
de classe C^n+1. Alors, \forall~~a,b \in I,

f(b) = f(a) + \sum _k=1^n~
f^(k)(a) \over k! (b - a)^k +
\\int  ~
_a^b (b - t)^n \over n!
f^(n+1)(t) dt

{[}
{[}
{[}
{[}

\end{document}
