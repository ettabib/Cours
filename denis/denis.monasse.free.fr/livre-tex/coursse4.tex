
\subsubsection{1.4 Anneaux et corps}

\paragraph{1.4.1 Généralités sur les anneaux}

Définition~1.4.1 Un anneau est un triplet (A,+,.) tel que (A,+) est un
groupe abélien et dans lequel la loi de multiplication est associative,
possède un élément neutre 1_A et est distributive par rapport à
l'addition.

Remarque~1.4.1 Nous admettrons dans la suite que l'on puisse avoir
1_A = 0_A, auquel cas on obtient immédiatement que A =
\0_A\.

Définition~1.4.2 On dit qu'un élément x d'un anneau A (non réduit à
\0\) est inversible s'il possède un
inverse (nécessairement unique) pour la multiplication. L'ensemble
A^\times des éléments inversibles d'un anneau A forme un groupe
multiplicatif.

Définition~1.4.3 On dit qu'un anneau A est intègre s'il est commutatif
et si xy = 0 \rigtharrow~ x = 0\text ou y = 0.

Définition~1.4.4 On dit qu'une partie B de A en est un sous-anneau si
elle est stable pour les deux lois et si pour les lois induites, B est
un anneau de même élément unité que A.

Proposition~1.4.1 Une partie B de A en est un sous-anneau si et
seulement si elle vérifie

\begin{itemize}
\itemsep1pt\parskip0pt\parsep0pt
\item
  (i) 1_A \in B
\item
  (ii) \forall~~x,y \in B, x - y \in B
\item
  (iii) \forall~~x,y \in B, xy \in B
\end{itemize}

Démonstration Elémentaire.

\paragraph{1.4.2 Idéaux et quotients}

Définition~1.4.5 On dit qu'une partie I de A est un idéal à gauche
(resp. à droite) de A si c'est un sous-groupe additif de A et si
\forall~a \in A,\\forall~~x \in I, ax \in
I (resp. xa \in I).

Définition~1.4.6 Un idéal bilatère est une partie qui est à la fois un
idéal à gauche et à droite (exemples triviaux
\0\ et A).

Proposition~1.4.2 Une partie I de A en est un idéal à gauche si et
seulement si elle vérifie

\begin{itemize}
\itemsep1pt\parskip0pt\parsep0pt
\item
  (i) I\neq~\varnothing~
\item
  (ii) \forall~~x,y \in I, x - y \in I
\item
  (iii) \forall~a \in A,\\forall~~x \in
  I, ax \in I
\end{itemize}

Proposition~1.4.3 Soit I un idéal à gauche de l'anneau A. Alors les
propriétés suivantes sont équivalentes

\begin{itemize}
\itemsep1pt\parskip0pt\parsep0pt
\item
  (i) I = A
\item
  (ii) 1_A \in I
\item
  (iii) I \bigcap A^\times\neq~\varnothing~
\end{itemize}

Démonstration On a clairement (i) \rigtharrow~(iii). De plus (iii) \rigtharrow~(ii) puisque si
x \in I \bigcap A^\times, alors x^-1 \in A,x \in I \rigtharrow~ 1_A =
x^-1x \in I. Enfin (ii) \rigtharrow~(i) puisque, si (ii) est vérifié, a \in
A,1_A \in I \rigtharrow~ a = a1_A \in I, soit A \subset~ I et A = I.

Remarque~1.4.2 Soit I un idéal à gauche de l'anneau A. Puisque c'est un
sous-groupe additif, on peut parler de l'ensemble quotient A\diagupI qui est
un groupe additif. Mais si l'on veut munir A\diagupI d'une structure d'anneau
raisonnable, il faut supposer que I est un idéal bilatère.

Théorème~1.4.4 Soit I un idéal bilatère. On munit l'ensemble quotient
A\diagupI d'une structure d'anneau en posant

\pi~(a) + \pi~(b) = \pi~(a + b),\quad \pi~(a)\pi~(b) = \pi~(ab)

autrement dit (a + I) + (b + I) = (a + b) + I et (a + I)(b + I) = ab +
I.

Démonstration Le seul point qui ne résulte pas d'un calcul évident est
qu'on obtient bien une application en posant (a + I)(b + I) = ab + I,
c'est-à-dire que si a + I = a' + I et b + I = b' + I, alors ab + I =
a'b' + I. Mais dans ce cas, on a a' = a + i,b' = b + \\jmathmath (i,\\jmathmath \in I), et
donc a'b' = ab + ib + a\\jmathmath + i\\jmathmath. Puisque I est un idéal bilatère, on a ib
+ a\\jmathmath + i\\jmathmath \in I et donc ib + a\\jmathmath + i\\jmathmath + I = I (un sous-groupe est stable
par translation par ses éléments), soit a'b' + I = ab + (ib + a\\jmathmath + i\\jmathmath +
I) = ab + I.

\paragraph{1.4.3 Morphisme d'anneaux}

Définition~1.4.7 On dit que f : A \rightarrow~ B est un morphisme d'anneaux si on a

\begin{itemize}
\itemsep1pt\parskip0pt\parsep0pt
\item
  (i)f(1_A) = 1_B
\item
  (ii) \forall~~x,y \in A, f(x + y) = f(x) + f(y)
\item
  (iii) \forall~~x,y \in A, f(xy) = f(x)f(y)
\end{itemize}

Remarque~1.4.3 On a alors f(0) = 0, f(-x) = -f(x) (comme pour tout
morphisme de groupes) et de plus, si x est inversible, f(x) l'est
également et f(x^-1) = f(x)^-1.

Théorème~1.4.5 (factorisation canonique). Soit f : A \rightarrow~ B un morphisme
d'anneaux. Alors
\mathrmKer~f =
\x \in A∣f(x) =
0\ est un idéal bilatère de A et
\mathrmIm~f = f(A) est un
sous-anneau de B. Il existe une unique application
\overlinef :
A\diagup\mathrmKer~f
\rightarrow~\mathrmIm~f vérifiant
\forall~x \in A, \overlinef~(x
+ \mathrmKer~f) = f(x).
L'application \overlinef est un isomorphisme
d'anneaux.

Démonstration L'existence, l'unicité et la bi\\jmathmathectivité de
\overlinef résultent du théorème analogue sur les
groupes. Le fait que ce soit un morphisme d'anneaux résulte d'un calcul
élémentaire.

\paragraph{1.4.4 Corps}

Définition~1.4.8 Un corps est un anneau non réduit à
\0\ où tout élément non nul est
inversible.

Proposition~1.4.6 Soit K un corps. Alors les seuls idéaux de K sont
\0\ et K. En particulier tout
morphisme d'un corps dans un anneau non réduit à
\0\ est in\\jmathmathectif.

Démonstration En effet un idéal non nul doit contenir un élément non
nul, donc un élément inversible, et donc doit être égal au corps tout
entier. Le noyau d'un morphisme d'un corps dans un anneau étant un idéal
du corps, ce doit être soit \0\
(auquel cas le morphisme est in\\jmathmathectif), soit le corps tout entier. Mais
ceci est exclu puisqu'on doit avoir f(1_K) = 1_A.

Remarque~1.4.4 Tout corps commutatif est un anneau intègre. Inversement,
tout anneau intègre fini est un corps~: en effet si a \in A, l'application
x\mapsto~ax doit être in\\jmathmathective à cause de
l'intégrité de A, donc bi\\jmathmathective puisque A est fini~; il doit donc
exister b_1 tel que ab_1 = 1_A~; de même il
doit exister b_2 tel que b_2a = 1_A et alors
b_2 = b_21_A = b_2ab_1 =
1_Ab_1 = b_1 ce qui montre que b_1 =
b_2 est inverse de a.

\paragraph{1.4.5 Idéaux maximaux}

Définition~1.4.9 Soit A un anneau commutatif. On dit qu'un idéal I de A,
distinct de A, est maximal si les seuls idéaux contenant I sont I et A.

Proposition~1.4.7 Soit A un anneau commutatif et I un idéal de A. Alors
A est maximal si et seulement si A\diagupI est un corps.

Démonstration Supposons tout d'abord que A\diagupI est un corps. On a donc
A\diagupI\neq~\0\ et
donc I\neq~A. Soit \pi~ : A \rightarrow~ A\diagupI la pro\\jmathmathection
canonique de A sur A\diagupI définie par \pi~(x) = x + I. Soit J un idéal de A
contenant strictement I et soit a \in J \diagdown I. Alors \pi~(a) \in A\diagupI
\diagdown\0\ et donc a + I est un élément
inversible de A\diagupI. Il existe donc b \in A tel que (a + I)(b + I) =
1_A + I ce qui se traduit encore par ab = 1_A + i avec
i \in I. Mais alors i \in I \subset~ J et ab \in J (puisque a \in J), et donc
1_A = ab - i \in J. L'idéal J qui contient l'élément unité est
donc nécessairement égal à A, et donc I est un idéal maximal de A.

Inversement, supposons que I est un idéal maximal de A et soit a + I un
élément non nul de A\diagupI, si bien que a +
I\neq~0_A + I, soit
a∉I~; alors J = I + aA est un idéal de A qui
contient strictement I, c'est donc A. On a alors 1_A \in A = I +
aA et donc il existe b \in A et i \in I tel que 1_A = i + ab. Mais
alors (a + I)(b + I) = ab + I = 1_A + I ce qui montre que a + I
est un élément inversible de A\diagupI, qui est donc un corps.

Théorème~1.4.8 Soit A un anneau commutatif. Alors tout idéal de A
distinct de A est contenu dans un idéal maximal (non unique).

Démonstration Soit I un idéal distinct de A et soit X l'ensemble des
idéaux distincts de A et qui contiennent I, ordonné par l'inclusion.
Nous allons utiliser le théorème de Zorn pour montrer que X admet un
élément maximal (qui sera bien évidemment un idéal maximal de A
contenant I). Il nous suffit de montrer que toute partie Y totalement
ordonnée de X admet un ma\\jmathmathorant~; pour cela considérons J_Y
= \⋃  _J\inY~J
et montrons que J_Y est dans X (auquel cas il sera bien un
ma\\jmathmathorant tel qu'on le cherche).

Tout d'abord, J_Y est bien un idéal de A~:

\begin{itemize}
\itemsep1pt\parskip0pt\parsep0pt
\item
  si a_1,a_2 \in J_Y, il existe
  J_1,J_2 \inY tels que a_1 \in
  J_1,a_2 \in J_2~; mais comme Y est une partie
  totalement ordonnée, on a soit J_1 \subset~ J_2, soit
  J_2 \subset~ J_1~; supposons par exemple que J_1 \subset~
  J_2, alors a_1 et a_2 sont dans
  J_2 et donc a_1 - a_2 \in J_2 \subset~
  J_Y~; donc J_Y est un sous-groupe additif de A
\item
  si a \in J_Y et b \in A, il existe J \inY tel que a \in J et alors ab
  \in J \subset~ J_Y, ce qui montre que J_Y est bien un idéal
  de A.
\end{itemize}

Cet idéal contient bien évidemment I. Supposons qu'il est égal à A, on
aurait alors 1_A \in J_Y et donc il existerait J \inY tel
que 1_A \in J, soit J = A~; mais ceci contredit la définition de
X. On a donc bien J_Y\inX.

\paragraph{1.4.6 Idéaux et anneaux principaux}

Remarque~1.4.5 Soit A un anneau commutatif. On vérifie facilement que si
a \in A, l'idéal engendré par a est aA =
\ax∣x \in A\
c'est-à-dire encore l'ensemble des multiples de a.

Proposition~1.4.9 Soit A un anneau intègre. Alors aA = bA
\Leftrightarrow \exists~x \in
A^\times,b = ax.

Démonstration Supposons que aA = bA~; on a b \in bA et donc il existe x \in
A tel que b = ax et de même il existe y \in A tel que a = by~; on a donc a
= axy. Si a\neq~0, alors on a 1_A = xy
et donc x est un élément inversible qui vérifie b = ax~; si par contre a
= 0, on a aussi b = 0 et alors b = a1_A avec 1_A
inversible.

Inversement, si b = ax avec x inversible, on a xA = A (car x inversible)
et donc bA = axA = aA.

Remarque~1.4.6 Autrement dit, l'élément qui engendre l'idéal est unique
à la multiplication près par un élément inversible de l'anneau.

Définition~1.4.10 On dit que a et b sont associés, et on note a ∼ b,
s'il existe x \in A^\times tel que b = ax (c'est-à-dire lorsqu'ils
engendrent le même idéal). C'est bien évidemment une relation
d'équivalence.

Définition~1.4.11 Un idéal I de A est dit principal s'il est engendré
par un élément. L'anneau A est dit principal s'il est intègre et si tout
idéal de A est principal.

Théorème~1.4.10 (propriété de Noether). Soit A un anneau principal, et
soit (I_n)_n\in\mathbb{N}~ une suite croissante d'idéaux. Alors
cette suite est stationnaire, c'est à dire qu'il existe N \in \mathbb{N}~ tel que
\forall~n ≥ N, I_n = I_N~.

Démonstration Considérons en effet I =\
⋃  _n\in\mathbb{N}~I_n~. Montrons que
c'est un idéal de A.

\begin{itemize}
\itemsep1pt\parskip0pt\parsep0pt
\item
  il est évidemment non vide
\item
  si a_1,a_2 \in I, il existe n_1,n_2
  \in \mathbb{N}~ tels que a_1 \in I_n_1,a_2 \in
  I_n_2~; mais alors, si n =\
  max(n_1,n_2), on a a_1,a_2 \in
  I_n, soit a_1 - a_2 \in I_n \subset~ I~;
  donc I est un sous groupe additif de A
\item
  si a \in I et b \in A, il existe n \in \mathbb{N}~ tel que a \in I_n, alors ab
  \in I_n \subset~ I~; donc I est bien un idéal de A
\end{itemize}

Puisque A est principal, il existe a \in I tel que I = aA~; mais alors, il
existe N \in \mathbb{N}~ tel que a \in I_N et donc I = aA \subset~ I_N.
Alors, pour tout n ≥ N, on a I_N \subset~ I_n \subset~ I \subset~
I_N, et donc I_n = I_N.

On dit que a divise b et on note a∣b s'il
existe x \in A tel que b = ax (soit bA \subset~ aA)

Définition~1.4.12 Soit A un anneau intègre, a et b deux éléments de A.
Soit d \in A. On dit que d est un PGCD de a et b si on a

\forall~~x \in A,\quad
x∣a\text et
x∣b \mathrel\Leftrightarrow
x∣d

Soit m \in A. On dit que m est un PPCM de a et b si

\forall~~x \in A,\quad
a∣x\text et
b∣x \mathrel\Leftrightarrow
m∣x

Remarque~1.4.7 On vérifie immédiatement qu'un PGCD est défini à la
multiplication par un élément inversible près~; il en est de même d'un
PPCM.

Théorème~1.4.11 Soit A un anneau principal, a et b deux éléments de A.
Soit d \in A vérifiant aA + bA = dA. Alors d est un PGCD de a et b. Il est
caractérisé par la propriété de Bézout~:
d∣a,d\mathrel∣b et
\exists~u,v \in A, d = ua + vb.

Démonstration En effet d∣a
\Leftrightarrow aA \subset~ dA, d\mathrel∣b
\Leftrightarrow bA \subset~ dA et donc
d∣a et d\mathrel∣b est
équivalent à aA + bA \subset~ dA. De même \exists~u,v \in A, d
= ua + vb \Leftrightarrow dA \subset~ aA + bA. D'où la
caractérisation par Bézout. Le fait que d est un PGCD en découle
immédiatement.

De même

Théorème~1.4.12 Soit A un anneau principal, a et b deux éléments de A.
Soit m \in A vérifiant aA \bigcap bA = mA. Alors m est un PPCM de a et b.

Définition~1.4.13 Soit A un anneau, a et b deux éléments de A. On dit
que a et b sont premiers entre eux si 1 est un PGCD de a et b (autrement
dit si les seuls diviseurs communs à a et b sont les éléments
inversibles de l'anneau).

Théorème~1.4.13 (Bézout). Soit A un anneau principal, a et b dans A. On
a équivalence de (i) a et b sont premiers entre eux (ii) aA + bA = A
(iii) \exists~u,v \in A, ua + vb = 1.

Théorème~1.4.14 (Gauss). Soit A un anneau principal, a,b et c dans A. On
suppose que (i)~a∣bc (ii)~a et b sont
premiers entre eux. Alors a divise c.

Démonstration On écrit ua + vb = 1 d'où c = uac + vbc les deux termes de
la somme étant divisibles par a.

Corollaire~1.4.15 Soit A un anneau principal,
a_1,\\ldots,a_n~
des éléments deux à deux premiers entre eux. Soit b \in A. On suppose que
\forall~i, a_i\mathrel∣~b.
Alors
a_1\\ldotsa_n\mathrel∣~b.

Démonstration Récurrence facile. On suppose que
a_1\\ldotsa_k\mathrel∣~b,
on écrit b =
a_1\\ldotsa_k~c
et le théorème de Gauss nécessite que
a_k+1∣c, soit
a_1\\ldotsa_ka_k+1\mathrel∣~b.

Définition~1.4.14 Soit A un anneau. On dit qu'un élément non nul a de A
est irréductible s'il est non inversible et vérifie les trois propriétés
équivalentes suivantes (i) x∣a \rigtharrow~ x ∼ a ou x \in
A^\times (ii) a = bc \rigtharrow~ a ∼ b ou a ∼ c (iii) a = bc \rigtharrow~ b \in
A^\times ou c \in A^\times.

Théorème~1.4.16 Soit A un anneau principal et a \in A
\diagdown\0\ non inversible. Alors les
propriétés suivantes sont équivalentes (i) a est irréductible dans A
(ii) a∣bc \rigtharrow~ a\mathrel∣b ou
a∣c (iii) A\diagupaA est un anneau intègre (iv)
A\diagupaA est un corps.

Démonstration On a clairement (ii) \Leftrightarrow (iii),
(ii) \rigtharrow~(i) et (iv) \rigtharrow~(iii). Si a est irréductible, il est premier avec
tout élément qu'il ne divise pas~; soit X un élément non nul de A\diagupaA, X
= \pi~(x)~; puisque X\neq~0, a ne divise pas x, donc
est premier avec x~; d'après Bézout, il existe u,v tels que ua + vx = 1
et on a alors \pi~(u)\pi~(x) = 1, donc X = \pi~(x) est inversible dans A\diagupaA qui
est donc un corps, ce qui montre que (i) \rigtharrow~(iv) et achève la
démonstration.

Définition~1.4.15 On notera P un système de représentants des éléments
irréductibles de A à la multiplication par un élément inversible près.

On a alors le théorème suivant

Théorème~1.4.17 (décomposition en éléments irréductibles). Soit A un
anneau principal et a \in A, a\neq~0. Alors a
s'écrit de manière unique sous la forme a =
up_1^n_1\\ldotsp_k^n_k~
avec u inversible, k ≥ 0,
p_1,\\ldots,p_k~
\inP,
n_1,\\ldots,n_k~
\textgreater{} 0.

Démonstration L'existence découle immédiatement d'une double application
du lemme de Noether~: d'une part a doit être divisible par un élément
irréductible sinon on pourrait trouver une chaîne infinie de diviseurs
de a et donc une chaîne infinie strictement croissante d'idéaux~; enfin
le processus de division par des éléments irréductibles doit s'arrêter
au bout d'un nombre fini d'opérations pour la même raison~; quand il
s'achève, c'est que l'on est tombé sur un élément inversible et donc sur
une décomposition adéquate.

L'unicité provient bien évidemment du théorème de Gauss et d'une petite
récurrence sur m_1 +
\\ldots~ +
m_k~: si on a

up_1^m_1
\\ldotsp_k^m_k~
 = vq_1^n_1
\\ldotsq_l^n_l~


alors q_l doit diviser le terme de droite, donc l'un des
p_i (puisqu'ils sont irréductibles), donc être égal à ce
p_i que l'on peut supposer être p_k. Mais alors en
simplifiant par p_k (car A est intègre) on obtient

up_1^m_1
\\ldotsp_k^m_k-1~
= vq_ 1^n_1
\\ldotsq_l^n_l-1~

et d'après l'hypothèse de récurrence u = v, les p_i sont les
mêmes que les q_i et les exposants sont les mêmes. Le démarrage
de la récurrence avec m_1 +
\\ldots~ +
m_k = 0 est laissé au lecteur.

Définition~1.4.16 Pour p \inP on notera v_p(a) la puissance de p
qui intervient dans la décomposition en facteurs irréductibles de a.

Remarque~1.4.8 On a alors a∣b
\Leftrightarrow \forall~~p \inP,
v_p(a) \leq v_p(b) d'où

Proposition~1.4.18 Soit a,b \in A. Alors un PGCD de a et b est
\∏ ~
_p\inPp^min(v_p(a),v_p(b))~
et un PPCM est \∏ ~
_p\inPp^max(v_p(a),v_p(b))~.

\paragraph{1.4.7 Anneaux euclidiens}

Définition~1.4.17 Soit A un anneau. On appelle stathme euclidien sur A
toute application v:A \diagdown\0\ \rightarrow~ \mathbb{N}~
vérifiant a∣b \rigtharrow~ v(a) \leq v(b). On dit que A
intègre est un anneau euclidien s'il existe sur A un stathme euclidien v
sur A vérifiant

\forall~~(a,b) \in
A^2,b\neq~0,
\exists~q,r \in A,\quad a = bq + r

avec r = 0 ou v(r) \textless{} v(b) (division euclidienne)

Théorème~1.4.19 Tout anneau euclidien est principal.

Démonstration Soit I un idéal non réduit à
\0\ et b un élément de I tel que v(b)
soit minimal. Si a \in I, \exists~q,r \in
A,\quad a = bq + r avec r = 0 ou v(r) \textless{} v(b)~;
la dernière possibilité est exclue car r = a - bq \in I. Donc I \subset~ bA,
l'inclusion réciproque étant évidente.

Remarque~1.4.9 Algorithme d'Euclide. Dans un anneau euclidien on dispose
d'un algorithme pour déterminer un PGCD de a et b en construisant des
suites q_n et r_n de la manière suivante~:
r_0 = a, r_1 = b et pour n ≥ 1, r_n-1 =
q_nr_n + r_n+1 avec r_n+1 = 0
(auquel cas on stoppe l'algorithme) ou v(r_n+1) \textless{}
v(r_n). La construction s'arrête au bout d'un nombre fini
d'opérations par décroissance stricte de la suite d'entiers naturels
v(r_n)~; comme l'ensemble des diviseurs communs à r_n
et r_n+1 reste constant (invariant de boucle), le dernier reste
non nul est un PGCD de a et b.

Remarque~1.4.10 On verra en particulier que \mathbb{Z} est un anneau euclidien
pour v(a) = a et que K{[}X{]} est un anneau
euclidien pour v(P) = deg~ P.

\paragraph{1.4.8 L'anneau \mathbb{Z}. Caractéristique d'un anneau}

Théorème~1.4.20 Soit a \in \mathbb{Z} et b \in \mathbb{N}~,b\neq~0. Alors il existe un unique
couple (q,r) \in \mathbb{Z} \times \mathbb{N}~ tel que a = bq + r et 0 \leq r \textless{} b.

Démonstration Il suffit de considérer q =\
max\x∣a - bx ≥
0\ en remarquant que cet ensemble est ma\\jmathmathoré par a.

Corollaire~1.4.21 L'anneau \mathbb{Z} est un anneau euclidien, donc principal.

Remarque~1.4.11 On choisit pour représentants des éléments irréductibles
les nombres premiers.

Soit alors A un anneau d'élément unité 1_A et considérons
l'application f : \mathbb{Z} \rightarrow~ A définie par f(n) = n1_A =
\left \ \cases
1_A +
\\ldots~ +
1_A (n\text fois)&si n ≥ 0
\cr -1_A
-\\ldots~ -
1_A (-n\text fois)&si n \textless{} 0 
\right .. On vérifie immédiatement que f est un morphisme
d'anneaux dont le noyau est un idéal de \mathbb{Z}, donc de la forme m\mathbb{Z} pour un
unique m \in \mathbb{N}~.

Définition~1.4.18 L'entier m est appelée la caractéristique de A.
L'image de f est un sous-anneau de A isomorphe à \mathbb{Z}\diagupm\mathbb{Z}.

Proposition~1.4.22 Soit A un anneau intègre. Alors sa caractéristique
est soit 0 soit un nombre premier. S'il est de caractéristique 0, il
contient un sous-anneau isomorphe à \mathbb{Z}. S'il est de caractéristique p
premier, il contient un sous-corps isomorphe à \mathbb{Z}\diagupp\mathbb{Z}.

\paragraph{1.4.9 Théorème chinois, indicateur d'Euler}

Théorème~1.4.23 Soit A un anneau principal,
a_1,\\ldots,a_n~
des éléments de A deux à deux premiers entre eux. Alors, pour tous
x_1,\\ldots,x_n~
\in A, il existe un élément x \in A, unique à un multiple près de a =
a_1\\ldotsa_n~,
tel que

x ≡ x_1
(mod\,\,a_1),\quad
\\ldots~\quad
, x ≡ x_n (mod\,\,a_n)

(en notant x ≡ y (mod\,\,a) pour
a∣x - y)

Démonstration En ce qui concerne l'unicité, remarquons que si x et y
conviennent, alors x - y doit être divisible à la fois par
a_1,\\ldots,a_n~
et donc par leur produit puisqu'ils sont deux à deux premiers entre eux.

Montrons maintenant l'existence par récurrence sur n. Pour n = 1, il n'y
a rien à démontrer. Lorsque n = 2, prenons u_1 et u_2
tels que u_1a_1 + u_2a_2 = 1 et
posons x = x_2u_1a_1 +
x_1u_2a_2. On a alors

x ≡ x_2u_1a_1 ≡
x_2(u_1a_1 + u_2a_2) ≡
x_2 (mod\,\,a_2)

et de même x ≡ x_1
(mod\,\,a_1). Donc x convient.

Supposons maintenant le résultat démontré pour n - 1 et choisissons y
tel que

y ≡ x_1
(mod\,\,a_1),\quad
\\ldots~\quad
, y ≡ x_n-1
(mod\,\,a_n-1)

Comme
a_1\\ldotsa_n-1~
et a_n sont premiers entre eux, on peut trouver u et v tels que
ua_1\\ldotsa_n-1~
+ va_n = 1. Posons x =
x_nua_1\\ldotsa_n-1~
+ yva_n, on a alors pour i \leq n - 1

x ≡ yva_n ≡ y(va_n +
ua_1\\ldotsa_n-1~)
= y (mod\,\,a_i)

et

x ≡
x_nua_1\\ldotsa_n-1~
≡
x_n(ua_1\\ldotsa_n-1~
+ va_n) = x_n
(mod\,\,a_n)

ce qui montre que x convient.

Corollaire~1.4.24 Soit A un anneau principal,
a_1,\\ldots,a_n~
des éléments de A deux à deux premiers entre eux. Alors l'anneau
A\diagup(a_1\\ldotsa_n~)A
est isomorphe à l'anneau produit A\diagupa_1A
\times⋯ \times A\diagupa_nA.

Démonstration En effet le théorème ci dessus ne fait que traduire la
bi\\jmathmathectivité de l'application de
A\diagup(a_1\\ldotsa_n~)A
dans A\diagupa_1A \times⋯ \times A\diagupa_nA
définie par

x +
a_1\\ldotsa_nA\mathrel\mapsto~~(x
+
a_1A,\\ldots~,x
+ a_nA)

Or on vérifie immédiatement que cette application est un morphisme
d'anneaux.

Définition~1.4.19 A tout nombre entier naturel n ≥ 2, on associe son
indicateur d'Euler \phi(n) défini par

\begin{align*} \phi(n)& =&
\mathrmCard~\x
\in {[}1,n - 1{]}∣x ∧ n = 1\
\%& \\ & =&
Card\\overlinex~
\in
\mathbb{Z}\diagupn\mathbb{Z}∣\overlinex\text
engendre (\mathbb{Z}\diagupn\mathbb{Z},+)\\%&
\\ & =&
Card~\left
(\mathbb{Z}\diagupn\mathbb{Z}\right )^\times \%&
\\ \end{align*}

Un isomorphisme d'anneaux réalisant évidemment une bi\\jmathmathection entre les
éléments inversibles des deux anneaux et les éléments inversibles de
\mathbb{Z}\diagupa_1\mathbb{Z} \times⋯ \times \mathbb{Z}\diagupa_n\mathbb{Z} étant
exactement les n-uples
(x_1,\\ldots,x_n~)
dont toutes les composantes x_i sont inversibles, on en déduit
que

Théorème~1.4.25 Soit
a_1,\\ldots,a_n~
des éléments de \mathbb{Z} deux à deux premiers entre eux. Alors
\phi(a_1\\ldotsa_n~)
=
\phi(a_1)\\ldots\phi(a_n~).

Remarque~1.4.12 Ceci va nous permettre de calculer \phi(n) à l'aide d'une
décomposition en nombres premiers de n, soit n =
p_1^\alpha_1\\ldotsp_k^\alpha_k~.
En effet, si n est de la forme n = p^k où p est premier, les
éléments inférieurs à n non premiers avec n sont exactement
p,2p,3p,\\ldots,p^k-1~p,
et donc \phi(p^k) = p^k - p^k-1 =
p^k-1(p - 1). D'après le résultat précédent on a donc, si n =
p_1^\alpha_1\\ldotsp_k^\alpha_k~,

\phi(n) =
p_1^\alpha_1-1\\ldotsp_
k^\alpha_k-1(p_ 1 -
1)\\ldots(p_k~
- 1)
