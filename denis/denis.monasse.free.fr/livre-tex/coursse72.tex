\documentclass[]{article}
\usepackage[T1]{fontenc}
\usepackage{lmodern}
\usepackage{amssymb,amsmath}
\usepackage{ifxetex,ifluatex}
\usepackage{fixltx2e} % provides \textsubscript
% use upquote if available, for straight quotes in verbatim environments
\IfFileExists{upquote.sty}{\usepackage{upquote}}{}
\ifnum 0\ifxetex 1\fi\ifluatex 1\fi=0 % if pdftex
  \usepackage[utf8]{inputenc}
\else % if luatex or xelatex
  \ifxetex
    \usepackage{mathspec}
    \usepackage{xltxtra,xunicode}
  \else
    \usepackage{fontspec}
  \fi
  \defaultfontfeatures{Mapping=tex-text,Scale=MatchLowercase}
  \newcommand{\euro}{€}
\fi
% use microtype if available
\IfFileExists{microtype.sty}{\usepackage{microtype}}{}
\ifxetex
  \usepackage[setpagesize=false, % page size defined by xetex
              unicode=false, % unicode breaks when used with xetex
              xetex]{hyperref}
\else
  \usepackage[unicode=true]{hyperref}
\fi
\hypersetup{breaklinks=true,
            bookmarks=true,
            pdfauthor={},
            pdftitle={Endomorphismes d'un espace euclidien},
            colorlinks=true,
            citecolor=blue,
            urlcolor=blue,
            linkcolor=magenta,
            pdfborder={0 0 0}}
\urlstyle{same}  % don't use monospace font for urls
\setlength{\parindent}{0pt}
\setlength{\parskip}{6pt plus 2pt minus 1pt}
\setlength{\emergencystretch}{3em}  % prevent overfull lines
\setcounter{secnumdepth}{0}
 
/* start css.sty */
.cmr-5{font-size:50%;}
.cmr-7{font-size:70%;}
.cmmi-5{font-size:50%;font-style: italic;}
.cmmi-7{font-size:70%;font-style: italic;}
.cmmi-10{font-style: italic;}
.cmsy-5{font-size:50%;}
.cmsy-7{font-size:70%;}
.cmex-7{font-size:70%;}
.cmex-7x-x-71{font-size:49%;}
.msbm-7{font-size:70%;}
.cmtt-10{font-family: monospace;}
.cmti-10{ font-style: italic;}
.cmbx-10{ font-weight: bold;}
.cmr-17x-x-120{font-size:204%;}
.cmsl-10{font-style: oblique;}
.cmti-7x-x-71{font-size:49%; font-style: italic;}
.cmbxti-10{ font-weight: bold; font-style: italic;}
p.noindent { text-indent: 0em }
td p.noindent { text-indent: 0em; margin-top:0em; }
p.nopar { text-indent: 0em; }
p.indent{ text-indent: 1.5em }
@media print {div.crosslinks {visibility:hidden;}}
a img { border-top: 0; border-left: 0; border-right: 0; }
center { margin-top:1em; margin-bottom:1em; }
td center { margin-top:0em; margin-bottom:0em; }
.Canvas { position:relative; }
li p.indent { text-indent: 0em }
.enumerate1 {list-style-type:decimal;}
.enumerate2 {list-style-type:lower-alpha;}
.enumerate3 {list-style-type:lower-roman;}
.enumerate4 {list-style-type:upper-alpha;}
div.newtheorem { margin-bottom: 2em; margin-top: 2em;}
.obeylines-h,.obeylines-v {white-space: nowrap; }
div.obeylines-v p { margin-top:0; margin-bottom:0; }
.overline{ text-decoration:overline; }
.overline img{ border-top: 1px solid black; }
td.displaylines {text-align:center; white-space:nowrap;}
.centerline {text-align:center;}
.rightline {text-align:right;}
div.verbatim {font-family: monospace; white-space: nowrap; text-align:left; clear:both; }
.fbox {padding-left:3.0pt; padding-right:3.0pt; text-indent:0pt; border:solid black 0.4pt; }
div.fbox {display:table}
div.center div.fbox {text-align:center; clear:both; padding-left:3.0pt; padding-right:3.0pt; text-indent:0pt; border:solid black 0.4pt; }
div.minipage{width:100%;}
div.center, div.center div.center {text-align: center; margin-left:1em; margin-right:1em;}
div.center div {text-align: left;}
div.flushright, div.flushright div.flushright {text-align: right;}
div.flushright div {text-align: left;}
div.flushleft {text-align: left;}
.underline{ text-decoration:underline; }
.underline img{ border-bottom: 1px solid black; margin-bottom:1pt; }
.framebox-c, .framebox-l, .framebox-r { padding-left:3.0pt; padding-right:3.0pt; text-indent:0pt; border:solid black 0.4pt; }
.framebox-c {text-align:center;}
.framebox-l {text-align:left;}
.framebox-r {text-align:right;}
span.thank-mark{ vertical-align: super }
span.footnote-mark sup.textsuperscript, span.footnote-mark a sup.textsuperscript{ font-size:80%; }
div.tabular, div.center div.tabular {text-align: center; margin-top:0.5em; margin-bottom:0.5em; }
table.tabular td p{margin-top:0em;}
table.tabular {margin-left: auto; margin-right: auto;}
div.td00{ margin-left:0pt; margin-right:0pt; }
div.td01{ margin-left:0pt; margin-right:5pt; }
div.td10{ margin-left:5pt; margin-right:0pt; }
div.td11{ margin-left:5pt; margin-right:5pt; }
table[rules] {border-left:solid black 0.4pt; border-right:solid black 0.4pt; }
td.td00{ padding-left:0pt; padding-right:0pt; }
td.td01{ padding-left:0pt; padding-right:5pt; }
td.td10{ padding-left:5pt; padding-right:0pt; }
td.td11{ padding-left:5pt; padding-right:5pt; }
table[rules] {border-left:solid black 0.4pt; border-right:solid black 0.4pt; }
.hline hr, .cline hr{ height : 1px; margin:0px; }
.tabbing-right {text-align:right;}
span.TEX {letter-spacing: -0.125em; }
span.TEX span.E{ position:relative;top:0.5ex;left:-0.0417em;}
a span.TEX span.E {text-decoration: none; }
span.LATEX span.A{ position:relative; top:-0.5ex; left:-0.4em; font-size:85%;}
span.LATEX span.TEX{ position:relative; left: -0.4em; }
div.float img, div.float .caption {text-align:center;}
div.figure img, div.figure .caption {text-align:center;}
.marginpar {width:20%; float:right; text-align:left; margin-left:auto; margin-top:0.5em; font-size:85%; text-decoration:underline;}
.marginpar p{margin-top:0.4em; margin-bottom:0.4em;}
.equation td{text-align:center; vertical-align:middle; }
td.eq-no{ width:5%; }
table.equation { width:100%; } 
div.math-display, div.par-math-display{text-align:center;}
math .texttt { font-family: monospace; }
math .textit { font-style: italic; }
math .textsl { font-style: oblique; }
math .textsf { font-family: sans-serif; }
math .textbf { font-weight: bold; }
.partToc a, .partToc, .likepartToc a, .likepartToc {line-height: 200%; font-weight:bold; font-size:110%;}
.chapterToc a, .chapterToc, .likechapterToc a, .likechapterToc, .appendixToc a, .appendixToc {line-height: 200%; font-weight:bold;}
.index-item, .index-subitem, .index-subsubitem {display:block}
.caption td.id{font-weight: bold; white-space: nowrap; }
table.caption {text-align:center;}
h1.partHead{text-align: center}
p.bibitem { text-indent: -2em; margin-left: 2em; margin-top:0.6em; margin-bottom:0.6em; }
p.bibitem-p { text-indent: 0em; margin-left: 2em; margin-top:0.6em; margin-bottom:0.6em; }
.paragraphHead, .likeparagraphHead { margin-top:2em; font-weight: bold;}
.subparagraphHead, .likesubparagraphHead { font-weight: bold;}
.quote {margin-bottom:0.25em; margin-top:0.25em; margin-left:1em; margin-right:1em; text-align:justify;}
.verse{white-space:nowrap; margin-left:2em}
div.maketitle {text-align:center;}
h2.titleHead{text-align:center;}
div.maketitle{ margin-bottom: 2em; }
div.author, div.date {text-align:center;}
div.thanks{text-align:left; margin-left:10%; font-size:85%; font-style:italic; }
div.author{white-space: nowrap;}
.quotation {margin-bottom:0.25em; margin-top:0.25em; margin-left:1em; }
h1.partHead{text-align: center}
.sectionToc, .likesectionToc {margin-left:2em;}
.subsectionToc, .likesubsectionToc {margin-left:4em;}
.subsubsectionToc, .likesubsubsectionToc {margin-left:6em;}
.frenchb-nbsp{font-size:75%;}
.frenchb-thinspace{font-size:75%;}
.figure img.graphics {margin-left:10%;}
/* end css.sty */

\title{Endomorphismes d'un espace euclidien}
\author{}
\date{}

\begin{document}
\maketitle

\textbf{Warning: \href{http://www.math.union.edu/locate/jsMath}{jsMath}
requires JavaScript to process the mathematics on this page.\\ If your
browser supports JavaScript, be sure it is enabled.}

\begin{center}\rule{3in}{0.4pt}\end{center}

{[}\href{coursse71.html}{prev}{]}
{[}\href{coursse71.html\#tailcoursse71.html}{prev-tail}{]}
{[}\hyperref[tailcoursse72.html]{tail}{]}
{[}\href{coursch13.html\#coursse72.html}{up}{]}

\subsubsection{12.6 Endomorphismes d'un espace euclidien}

\paragraph{12.6.1 Droites et plans stables}

Nous utiliserons à deux reprises le lemme suivant

Lemme~12.6.1 Soit E un ℝ-espace vectoriel ~de dimension finie et u ∈
L(E). Alors u admet soit une droite stable, soit un plan stable.

Démonstration Soit P un polynôme normalisé annulateur de u et soit P =
\{P\}\_\{1\}\textbackslash{}mathop\{\textbackslash{}mathop\{\ldots{}\}\}\{P\}\_\{n\}
la décomposition de P en polynômes normalisés irréductibles sur ℝ. On a
0 = P(u) = \{P\}\_\{1\}(u) ∘\textbackslash{}mathrel\{⋯\} ∘
\{P\}\_\{n\}(u). Donc l'un des \{P\}\_\{i\}(u) est non injectif. Soit x
∈\textbackslash{}mathop\{\textbackslash{}mathrm\{Ker\}\}\{P\}\_\{i\}(u)
∖\textbackslash{}\{0\textbackslash{}\}. Deux cas sont possibles~:

\begin{itemize}
\itemsep1pt\parskip0pt\parsep0pt
\item
  \{P\}\_\{1\} est de degré 1, soit \{P\}\_\{1\}(X) = X − λ, alors (u −
  λ\textbackslash{}mathrm\{Id\})(x) = 0, x est vecteur propre de u et la
  droite ℝx est stable par u~;
\item
  \{P\}\_\{1\} est de degré 2, alors \{P\}\_\{1\}(X) = \{X\}\^{}\{2\} −
  aX − b et on a donc \{u\}\^{}\{2\}(x) = au(x) + bx~; le sous-espace
  \textbackslash{}mathop\{\textbackslash{}mathrm\{Vect\}\}(x,u(x)) est
  de dimension au plus 2 (en fait il est facile de vérifier qu'elle est
  égale à 2) et il est stable par u.
\end{itemize}

\paragraph{12.6.2 Réduction des endomorphismes symétriques}

Théorème~12.6.2 Soit E un espace euclidien et u un endomorphisme de E.
Alors on a équivalence de

\begin{itemize}
\itemsep1pt\parskip0pt\parsep0pt
\item
  (i) u est un endomorphisme symétrique
\item
  (ii) il existe une base orthonormée formée de vecteurs propres de u
\item
  (iii) il existe une base orthonormée ℰ telle que
  \textbackslash{}mathop\{\textbackslash{}mathrm\{Mat\}\} (u,ℰ) soit
  diagonale.
\end{itemize}

Démonstration (ii) et (iii) sont clairement équivalents. Si (iii) est
vérifiée, la matrice de u dans la base orthonormée ℰ est symétrique et
donc u est un endomorphisme symétrique. Il nous reste donc à montrer que
(i) ⇒(ii), ce que nous allons faire par récurrence sur n
=\textbackslash{}mathop\{ dim\} E. Montrons pour cela que u a un vecteur
propre. D'après le lemme ci dessus, u admet soit une droite, soit un
plan stable. Si u admet une droite stable, cette droite est engendrée
par un vecteur propre. Si u a un plan stable Π, soit u' l'endomorphisme
induit par u sur Π (c'est bien entendu un endomorphisme symétrique de
Π), ℰ = (\{e\}\_\{1\},\{e\}\_\{2\}) une base orthonormée de Π et
\textbackslash{}mathop\{\textbackslash{}mathrm\{Mat\}\} (u',ℰ) =
\textbackslash{}left
(\textbackslash{}matrix\{\textbackslash{},a\&b\textbackslash{}cr b
\&c\}\textbackslash{}right )~; alors \{χ\}\_\{u\}(X) = (X − a)(X − c) −
\{b\}\^{}\{2\} = \{X\}\^{}\{2\} − (a + b)X + ac − \{b\}\^{}\{2\} de
discriminant Δ = \{(a + c)\}\^{}\{2\} − 4(ac − \{b\}\^{}\{2\}) = \{(a −
c)\}\^{}\{2\} + 4\{b\}\^{}\{2\} ≥ 0~; donc u' a un vecteur propre dans Π
qui est également un vecteur propre de u dans E. Supposons donc que tout
endomorphisme symétrique d'un espace de dimension n − 1 admet une base
orthonormée de vecteurs propres. Soit \{e\}\_\{1\} un vecteur propre de
u (endomorphisme symétrique d'un espace euclidien de dimension n).
Quitte à remplacer \{e\}\_\{1\} par \{ \{e\}\_\{1\} \textbackslash{}over
\textbackslash{}\textbar{}\{e\}\_\{1\}\textbackslash{}\textbar{}\} , on
peut supposer que
\textbackslash{}\textbar{}\{e\}\_\{1\}\textbackslash{}\textbar{} = 1.
Soit H = \{e\}\_\{1\}\^{}\{⊥\}. Comme K\{e\}\_\{1\} est stable par u,
son orthogonal H est stable par \{u\}\^{}\{∗\} = u. La restriction u' de
u à H est un endomorphisme symétrique de H de dimension n − 1, donc
admet une base orthonormée
(\{e\}\_\{2\},\textbackslash{}mathop\{\textbackslash{}mathop\{\ldots{}\}\},\{e\}\_\{n\})
formée de vecteurs propres de u' (donc de u)~; alors
(\{e\}\_\{1\},\textbackslash{}mathop\{\textbackslash{}mathop\{\ldots{}\}\},\{e\}\_\{n\})
est une base orthonormée de E formée de vecteurs propres de u.

Corollaire~12.6.3 Soit E un espace euclidien et u un endomorphisme
symétrique de E~; alors E est somme directe orthogonale des sous-espaces
propres de u.

Démonstration Puisque u est diagonalisable, E est somme directe des
sous-espaces propres de u. Il suffit de montrer que ces sous-espaces
sont deux à deux orthogonaux. Mais, si x ∈ \{E\}\_\{u\}(λ) et y ∈
\{E\}\_\{u\}(μ) avec λ\textbackslash{}mathrel\{≠\}μ, on a

λ(x\textbackslash{}mathrel\{∣\}y) = (u(x)\textbackslash{}mathrel\{∣\}y)
= (x\textbackslash{}mathrel\{∣\}u(y)) =
(x\textbackslash{}mathrel\{∣\}μy) = μ(x\textbackslash{}mathrel\{∣\}y)

Comme λ\textbackslash{}mathrel\{≠\}μ, on a
(x\textbackslash{}mathrel\{∣\}y) = 0.

Remarque~12.6.1 Ceci permet une pratique simple de la réduction d'un
endomorphisme symétrique~; il suffit en effet de déterminer une base
orthonormée de chacun des sous-espaces propres de u et de réunir ces
bases~; on obtient une base orthonormée de E formée de vecteurs propres
de u.

Définition~12.6.1 Soit E un espace euclidien et u un endomorphisme
symétrique de E. On dit que u est un endomorphisme positif (resp. défini
positif) s'il vérifie les conditions équivalentes~:

\begin{itemize}
\itemsep1pt\parskip0pt\parsep0pt
\item
  (i) \textbackslash{}mathop\{∀\}x ∈
  E,(u(x)\textbackslash{}mathrel\{∣\}x) ≥ 0 (resp.
  \textbackslash{}mathop\{∀\}x\textbackslash{}mathrel\{≠\}0,(u(x)\textbackslash{}mathrel\{∣\}x)
  \textgreater{} 0)
\item
  (ii) L'application
  x\textbackslash{}mathrel\{↦\}(u(x)\textbackslash{}mathrel\{∣\}x) est
  une forme quadratique positive sur E (resp. définie positive)
\item
  (iii) Les valeurs propres de u sont positives (resp. strictement
  positives).
\end{itemize}

Démonstration En remarquant que l'application \{Q\}\_\{u\} :
x\textbackslash{}mathrel\{↦\}(u(x)\textbackslash{}mathrel\{∣\}x) est une
forme quadratique de forme polaire
(x,y)\textbackslash{}mathrel\{↦\}(u(x)\textbackslash{}mathrel\{∣\}y), on
a immédiatement l'équivalence de (i) et (ii). Soit λ une valeur propre
de u et x un vecteur propre associé~; on a alors
(u(x)\textbackslash{}mathrel\{∣\}x) = (λx\textbackslash{}mathrel\{∣\}x)
= λ\textbackslash{}\textbar{}\{x\textbackslash{}\textbar{}\}\^{}\{2\},
si bien que λ =\{ (u(x)\textbackslash{}mathrel\{∣\}x)
\textbackslash{}over
\textbackslash{}\textbar{}\{x\textbackslash{}\textbar{}\}\^{}\{2\}\} ~;
il en résulte que (i) ⇒(iii). Inversement, supposons (iii) vérifiée et
soit
(\{e\}\_\{1\},\textbackslash{}mathop\{\textbackslash{}mathop\{\ldots{}\}\},\{e\}\_\{n\})
une base orthonormée formée de vecteurs propres de u, u(\{e\}\_\{i\}) =
\{λ\}\_\{i\}\{e\}\_\{i\}. On a alors, si x =\{\textbackslash{}mathop\{
\textbackslash{}mathop\{∑ \}\} \}\_\{i\}\{x\}\_\{i\}\{e\}\_\{i\},

(u(x)\textbackslash{}mathrel\{∣\}x) = (\{\textbackslash{}mathop\{∑
\}\}\_\{i\}\{λ\}\_\{i\}\{x\}\_\{i\}\{e\}\_\{i\}\textbackslash{}mathrel\{∣\}\{\textbackslash{}mathop\{∑
\}\}\_\{i\}\{x\}\_\{i\}\{e\}\_\{i\}) =\{ \textbackslash{}mathop\{∑
\}\}\_\{i\}\{λ\}\_\{i\}\{x\}\_\{i\}\^{}\{2\} ≥ 0

(resp. \textgreater{} 0 si x\textbackslash{}mathrel\{≠\}0) si bien que
(iii) ⇒(i).

Théorème~12.6.4 (réduction simultanée des formes quadratiques). Soit E
un ℝ espace vectoriel de dimension finie, Φ une forme quadratique
définie positive, Ψ une forme quadratique sur E. Alors il existe une
base ℰ de E orthonormée pour Φ et orthogonale pour Ψ.

Démonstration On sait qu'il existe un unique endomorphisme u de E tel
que \textbackslash{}mathop\{∀\}x,y ∈ E, ψ(x,y) = φ(u(x),y). Comme ψ est
symétrique, u est un endomorphisme symétrique de l'espace euclidien
(E,Φ) et il existe une base ℰ =
(\{e\}\_\{1\},\textbackslash{}mathop\{\textbackslash{}mathop\{\ldots{}\}\},\{e\}\_\{n\})
orthonormée pour Φ formée de vecteurs propres de u~: u(\{e\}\_\{i\}) =
\{λ\}\_\{i\}\{e\}\_\{i\}. On a alors

ψ(\{e\}\_\{i\},\{e\}\_\{j\}) = φ(\{e\}\_\{i\},u(\{e\}\_\{j\})) =
φ(\{e\}\_\{i\},\{λ\}\_\{j\}\{e\}\_\{j\}) =
\{λ\}\_\{j\}\{δ\}\_\{i\}\^{}\{j\}

ce qui montre que ℰ est une base orthogonale pour ψ.

\paragraph{12.6.3 Normes d'endomorphismes}

Théorème~12.6.5 Soit E un espace euclidien et u ∈ L(E). Alors

\textbackslash{}\textbar{}u\textbackslash{}\textbar{}
=\{\textbackslash{}mathop\{
sup\}\}\_\{\textbackslash{}\textbar{}x\textbackslash{}\textbar{}≤1,\textbackslash{}\textbar{}y\textbackslash{}\textbar{}≤1\}\textbar{}(u(x)\textbackslash{}mathrel\{∣\}y)\textbar{}

Démonstration Supposons que
\textbackslash{}\textbar{}x\textbackslash{}\textbar{} ≤
1,\textbackslash{}\textbar{}y\textbackslash{}\textbar{} ≤ 1. On a alors
d'après l'inégalité de Schwarz

\textbar{}(u(x)\textbackslash{}mathrel\{∣\}y)\textbar{}≤\textbackslash{}\textbar{}
u(x)\textbackslash{}\textbar{}\textbackslash{}\textbar{}y\textbackslash{}\textbar{}
≤\textbackslash{}\textbar{} u\textbackslash{}\textbar{}

ce qui montre que \textbackslash{}\textbar{}u\textbackslash{}\textbar{}
≥\{\textbackslash{}mathop\{
sup\}\}\_\{\textbackslash{}\textbar{}x\textbackslash{}\textbar{}≤1,\textbackslash{}\textbar{}y\textbackslash{}\textbar{}≤1\}\textbar{}(u(x)\textbackslash{}mathrel\{∣\}y)\textbar{}.
Mais d'autre part, puisque la boule unité fermée est compacte, il existe
\{x\}\_\{0\} tel que
\textbackslash{}\textbar{}\{x\}\_\{0\}\textbackslash{}\textbar{} ≤ 1
avec \textbackslash{}\textbar{}u(\{x\}\_\{0\})\textbackslash{}\textbar{}
=\{\textbackslash{}mathop\{
sup\}\}\_\{\textbackslash{}\textbar{}x\textbackslash{}\textbar{}≤1\}\textbackslash{}\textbar{}u(x)\textbackslash{}\textbar{}
=\textbackslash{}\textbar{} u\textbackslash{}\textbar{}. Posons alors,
si u\textbackslash{}mathrel\{≠\}0, \{y\}\_\{0\} =\{ u(\{x\}\_\{0\})
\textbackslash{}over
\textbackslash{}\textbar{}u(\{x\}\_\{0\})\textbackslash{}\textbar{}\} .
On a \textbackslash{}\textbar{}\{y\}\_\{0\}\textbackslash{}\textbar{} =
1 et

\textbar{}(u(\{x\}\_\{0\})\textbackslash{}mathrel\{∣\}\{y\}\_\{0\})\textbar{}
=\{ (u(\{x\}\_\{0\})\textbackslash{}mathrel\{∣\}u(\{x\}\_\{0\}))
\textbackslash{}over
\textbackslash{}\textbar{}u(\{x\}\_\{0\})\textbackslash{}\textbar{}\}
=\textbackslash{}\textbar{} u(\{x\}\_\{0\})\textbackslash{}\textbar{}
=\textbackslash{}\textbar{} u\textbackslash{}\textbar{}

ce qui montre que \textbackslash{}\textbar{}u\textbackslash{}\textbar{}
≤\{\textbackslash{}mathop\{
sup\}\}\_\{\textbackslash{}\textbar{}x\textbackslash{}\textbar{}≤1,\textbackslash{}\textbar{}y\textbackslash{}\textbar{}≤1\}\textbar{}(u(x)\textbackslash{}mathrel\{∣\}y)\textbar{},
et donc l'égalité.

Corollaire~12.6.6 Soit E un espace euclidien et u ∈ L(E). Alors
\textbackslash{}\textbar{}u\textbackslash{}\textbar{}
=\textbackslash{}\textbar{} \{u\}\^{}\{∗\}\textbackslash{}\textbar{}.

Démonstration En effet

\textbackslash{}\textbar{}u\textbackslash{}\textbar{}
=\{\textbackslash{}mathop\{
sup\}\}\_\{\textbackslash{}\textbar{}x\textbackslash{}\textbar{}≤1,\textbackslash{}\textbar{}y\textbackslash{}\textbar{}≤1\}\textbar{}(u(x)\textbackslash{}mathrel\{∣\}y)\textbar{}
=\{\textbackslash{}mathop\{
sup\}\}\_\{\textbackslash{}\textbar{}x\textbackslash{}\textbar{}≤1,\textbackslash{}\textbar{}y\textbackslash{}\textbar{}≤1\}\textbar{}(x\textbackslash{}mathrel\{∣\}\{u\}\^{}\{∗\}(y))\textbar{}
=\textbackslash{}\textbar{} \{u\}\^{}\{∗\}\textbackslash{}\textbar{}

Théorème~12.6.7 Soit E un espace euclidien et u ∈ L(E) symétrique
positif. Alors

\textbackslash{}\textbar{}u\textbackslash{}\textbar{}
=\{\textbackslash{}mathop\{
sup\}\}\_\{\textbackslash{}\textbar{}x\textbackslash{}\textbar{}≤1\}(u(x)\textbackslash{}mathrel\{∣\}x)
=\{\textbackslash{}mathop\{
max\}\}\_\{λ∈\textbackslash{}mathop\{\textbackslash{}mathrm\{Sp\}\}(u)\}λ

Démonstration Supposons que
\textbackslash{}\textbar{}x\textbackslash{}\textbar{} ≤ 1. On a alors
d'après l'inégalité de Schwarz

(u(x)\textbackslash{}mathrel\{∣\}x) ≤\textbackslash{}\textbar{}
u(x)\textbackslash{}\textbar{}\textbackslash{}\textbar{}x\textbackslash{}\textbar{}
≤\textbackslash{}\textbar{} u\textbackslash{}\textbar{}

ce qui montre que \textbackslash{}\textbar{}u\textbackslash{}\textbar{}
≥\{\textbackslash{}mathop\{
sup\}\}\_\{\textbackslash{}\textbar{}x\textbackslash{}\textbar{}≤1\}(u(x)\textbackslash{}mathrel\{∣\}x).
De plus, soit k =\{\textbackslash{}mathop\{
sup\}\}\_\{\textbackslash{}\textbar{}x\textbackslash{}\textbar{}≤1\}(u(x)\textbackslash{}mathrel\{∣\}x),
si bien que \textbackslash{}mathop\{∀\}x ∈
E,(u(x)\textbackslash{}mathrel\{∣\}x) ≤
k\textbackslash{}\textbar{}\{x\textbackslash{}\textbar{}\}\^{}\{2\}, et
supposons que \textbackslash{}\textbar{}x\textbackslash{}\textbar{} ≤
1,\textbackslash{}\textbar{}y\textbackslash{}\textbar{} ≤ 1. On a alors,
puisque u est symétrique

\textbackslash{}begin\{eqnarray*\}
\textbar{}(u(x)\textbackslash{}mathrel\{∣\}y)\textbar{}\& =\&\{ 1
\textbackslash{}over 4\} \textbar{}(u(x +
y)\textbackslash{}mathrel\{∣\}x + y) − (u(x −
y)\textbackslash{}mathrel\{∣\}x − y)\textbar{}\%\&
\textbackslash{}\textbackslash{} \& ≤\&\{ k \textbackslash{}over 4\}
(\textbackslash{}\textbar{}x + \{y\textbackslash{}\textbar{}\}\^{}\{2\}
+\textbackslash{}\textbar{} x −
\{y\textbackslash{}\textbar{}\}\^{}\{2\}) =\{ k \textbackslash{}over 2\}
(\textbackslash{}\textbar{}\{x\textbackslash{}\textbar{}\}\^{}\{2\}
+\textbackslash{}\textbar{} \{y\textbackslash{}\textbar{}\}\^{}\{2\}) ≤
k \%\& \textbackslash{}\textbackslash{} \textbackslash{}end\{eqnarray*\}

et donc

\textbackslash{}\textbar{}u\textbackslash{}\textbar{}
=\{\textbackslash{}mathop\{
sup\}\}\_\{\textbackslash{}\textbar{}x\textbackslash{}\textbar{}≤1,\textbackslash{}\textbar{}y\textbackslash{}\textbar{}≤1\}\textbar{}(u(x)\textbackslash{}mathrel\{∣\}y)\textbar{}≤
k

et par conséquent

\textbackslash{}\textbar{}u\textbackslash{}\textbar{}
=\{\textbackslash{}mathop\{
sup\}\}\_\{\textbackslash{}\textbar{}x\textbackslash{}\textbar{}≤1\}(u(x)\textbackslash{}mathrel\{∣\}x)

Soit
(\{e\}\_\{1\},\textbackslash{}mathop\{\textbackslash{}mathop\{\ldots{}\}\},\{e\}\_\{n\})
une base orthonormée formée de vecteurs propres de u, u(\{e\}\_\{i\}) =
\{λ\}\_\{i\}\{e\}\_\{i\}. On peut supposer que \{λ\}\_\{1\} ≤
\{λ\}\_\{2\}
≤\textbackslash{}mathop\{\textbackslash{}mathop\{\ldots{}\}\} ≤
\{λ\}\_\{n\}. On a alors, si x =\{\textbackslash{}mathop\{
\textbackslash{}mathop\{∑ \}\} \}\_\{i\}\{x\}\_\{i\}\{e\}\_\{i\},

(u(x)\textbackslash{}mathrel\{∣\}x) = (\{\textbackslash{}mathop\{∑
\}\}\_\{i\}\{λ\}\_\{i\}\{x\}\_\{i\}\{e\}\_\{i\}\textbackslash{}mathrel\{∣\}\{\textbackslash{}mathop\{∑
\}\}\_\{i\}\{x\}\_\{i\}\{e\}\_\{i\}) =\{ \textbackslash{}mathop\{∑
\}\}\_\{i\}\{λ\}\_\{i\}\{x\}\_\{i\}\^{}\{2\} ≤ \{λ\}\_\{ n\}\{
\textbackslash{}mathop\{∑ \}\}\_\{i\}\{x\}\_\{i\}\^{}\{2\} ≤ \{λ\}\_\{
n\}

avec égalité si x = \{e\}\_\{n\}. Ceci montre que

\{\textbackslash{}mathop\{sup\}\}\_\{\textbackslash{}\textbar{}x\textbackslash{}\textbar{}≤1\}(u(x)\textbackslash{}mathrel\{∣\}x)
=\{\textbackslash{}mathop\{
max\}\}\_\{λ∈\textbackslash{}mathop\{\textbackslash{}mathrm\{Sp\}\}(u)\}λ

et achève la démonstration.

Corollaire~12.6.8 Soit E un espace euclidien et u ∈ L(E). Alors
\{u\}\^{}\{∗\}u est un endomorphisme symétrique positif et
\textbackslash{}\textbar{}\{u\}\^{}\{∗\}u\textbackslash{}\textbar{}
=\textbackslash{}\textbar{} \{u\textbackslash{}\textbar{}\}\^{}\{2\}.

Démonstration On a \{(\{u\}\^{}\{∗\}u)\}\^{}\{∗\} =
\{u\}\^{}\{∗\}\{u\}\^{}\{∗∗\} = \{u\}\^{}\{∗\}u donc \{u\}\^{}\{∗\}u est
symétrique. De plus (\{u\}\^{}\{∗\}u(x)\textbackslash{}mathrel\{∣\}x) =
(u(x)\textbackslash{}mathrel\{∣\}u(x)) =\textbackslash{}\textbar{}
u\{(x)\textbackslash{}\textbar{}\}\^{}\{2\} ≥ 0, ce qui montre que
\{u\}\^{}\{∗\}u est positif. On a alors

\textbackslash{}\textbar{}\{u\}\^{}\{∗\}\{u\textbackslash{}\textbar{}\}\^{}\{2\}
=\{\textbackslash{}mathop\{ sup\}\}\_\{\textbackslash{}\textbar{}
x\textbackslash{}\textbar{}≤1\}(\{u\}\^{}\{∗\}u(x)\textbackslash{}mathrel\{∣\}x)
=\{\textbackslash{}mathop\{ sup\}\}\_\{\textbackslash{}\textbar{}
x\textbackslash{}\textbar{}≤1\}\textbackslash{}\textbar{}u\{(x)\textbackslash{}\textbar{}\}\^{}\{2\}
=\textbackslash{}\textbar{} \{u\textbackslash{}\textbar{}\}\^{}\{2\}

Corollaire~12.6.9 \textbackslash{}\textbar{}u\textbackslash{}\textbar{}
est la racine carrée de la plus grande valeur propre de \{u\}\^{}\{∗\}u.

Démonstration Résulte immédiatement des résultats précédents.

\paragraph{12.6.4 Endomorphismes orthogonaux d'un plan euclidien}

Remarque~12.6.2 Soit E un ℝ-espace vectoriel ~de dimension finie. La
relation définie sur l'ensemble des bases de E par

ℰℛℰ'\textbackslash{}mathrel\{⇔\}
\textbackslash{}mathop\{\textbackslash{}mathrm\{det\}\}
\{P\}\_\{ℰ\}\^{}\{ℰ'\} \textgreater{} 0

est une relation d'équivalence pour laquelle il y a deux classes
d'équivalence appelées orientations de l'espace. Le choix d'une de ces
classes (les bases directes) oriente l'espace E.

Soit E un espace euclidien de dimension 2.

Théorème~12.6.10 Soit u ∈ O(E) et ℰ une base orthonormée de E.

\begin{itemize}
\itemsep1pt\parskip0pt\parsep0pt
\item
  (i) Si u ∈ SO(E), alors
  \textbackslash{}mathop\{\textbackslash{}mathrm\{Mat\}\} (u,ℰ) =
  \textbackslash{}left
  (\textbackslash{}matrix\{\textbackslash{},\textbackslash{}mathop\{cos\}
  θ\&−\textbackslash{}mathop\{sin\} θ\textbackslash{}cr
  \textbackslash{}mathop\{sin\} θ \&\textbackslash{}mathop\{cos\}
  θ\}\textbackslash{}right ) pour un θ ∈ ℝ∕2πℤ ne dépendant que de
  l'orientation de la base ℰ (un changement d'orientation changeant θ en
  − θ)~; le groupe SO(E) est commutatif, isomorphe au groupe (ℝ∕2πℤ,+)
\item
  (ii) Si \textbackslash{}mathop\{\textbackslash{}mathrm\{det\}\} u =
  −1, alors \textbackslash{}mathop\{\textbackslash{}mathrm\{Mat\}\}
  (u,ℰ) = \textbackslash{}left
  (\textbackslash{}matrix\{\textbackslash{},\textbackslash{}mathop\{cos\}
  θ\&\textbackslash{}mathop\{sin\} θ \textbackslash{}cr
  \textbackslash{}mathop\{sin\} θ\&−\textbackslash{}mathop\{cos\}
  θ\}\textbackslash{}right )~; u est une symétrie orthogonale par
  rapport à une droite.
\end{itemize}

Démonstration Posons ℰ = (\{e\}\_\{1\},\{e\}\_\{2\}) et u(\{e\}\_\{1\})
= a\{e\}\_\{1\} + b\{e\}\_\{2\}. On a \{a\}\^{}\{2\} + \{b\}\^{}\{2\}
=\textbackslash{}\textbar{}
u\{(\{e\}\_\{1\})\textbackslash{}\textbar{}\}\^{}\{2\}
=\textbackslash{}\textbar{}
\{e\{\}\_\{1\}\textbackslash{}\textbar{}\}\^{}\{2\} = 1, donc il existe
θ ∈ ℝ∕2πℤ tel que a =\textbackslash{}mathop\{ cos\} θ et b
=\textbackslash{}mathop\{ sin\} θ. On a u(\{e\}\_\{2\}) ∈
u\{(\{e\}\_\{1\})\}\^{}\{⊥\} = ℝ(−b\{e\}\_\{1\} + a\{e\}\_\{2\}). On en
déduit que u(\{e\}\_\{2\}) = λ(−\textbackslash{}mathop\{sin\}
θ\{e\}\_\{1\} +\textbackslash{}mathop\{ cos\} θ\{e\}\_\{2\})~; comme
\textbackslash{}\textbar{}u(\{e\}\_\{2\})\textbackslash{}\textbar{}
=\textbackslash{}\textbar{} \{e\}\_\{2\}\textbackslash{}\textbar{} = 1,
on doit avoir \{λ\}\^{}\{2\} = 1, soit λ = ±1. Donc la matrice de u dans
la base ℰ est de l'une des deux formes

\textbackslash{}left
(\textbackslash{}matrix\{\textbackslash{},\textbackslash{}mathop\{cos\}
θ\&−\textbackslash{}mathop\{sin\} θ \textbackslash{}cr
\textbackslash{}mathop\{sin\} θ\&\textbackslash{}mathop\{cos\} θ
\textbackslash{}cr \}\textbackslash{}right )\textbackslash{}text\{ ou
\}\textbackslash{}left
(\textbackslash{}matrix\{\textbackslash{},\textbackslash{}mathop\{cos\}
θ\&\textbackslash{}mathop\{sin\} θ \textbackslash{}cr
\textbackslash{}mathop\{sin\} θ\&−\textbackslash{}mathop\{cos\}
θ\}\textbackslash{}right )

Il est clair que le premier cas correspond à
\textbackslash{}mathop\{\textbackslash{}mathrm\{det\}\} u = 1 et le
second cas à \textbackslash{}mathop\{\textbackslash{}mathrm\{det\}\} u =
−1. Dans le second cas, on vérifie immédiatement que \{u\}\^{}\{2\} =\{
\textbackslash{}mathrm\{Id\}\}\_\{E\}, ce qui montre que u est une
symétrie (évidemment orthogonale). Comme
u\textbackslash{}mathrel\{≠\}\textbackslash{}mathrm\{Id\} et
u\textbackslash{}mathrel\{≠\} −\textbackslash{}mathrm\{Id\}, c'est
nécessairement une symétrie par rapport à une droite.

Dans le premier cas, on a
\textbackslash{}mathop\{\textbackslash{}mathrm\{tr\}\}u =
2\textbackslash{}mathop\{cos\} θ, ce qui montre que
\textbackslash{}mathop\{cos\} θ est indépendant du choix de la base ℰ,
et que donc θ ∈ ℝ∕2πℤ est déterminé au signe près. On vérifie
immédiatement que

\textbackslash{}begin\{eqnarray*\} \textbackslash{}left
(\textbackslash{}matrix\{\textbackslash{},\textbackslash{}mathop\{cos\}
θ\&−\textbackslash{}mathop\{sin\} θ \textbackslash{}cr
\textbackslash{}mathop\{sin\} θ\&\textbackslash{}mathop\{cos\} θ
\textbackslash{}cr \}\textbackslash{}right )\textbackslash{}left
(\textbackslash{}matrix\{\textbackslash{},\textbackslash{}mathop\{cos\}
θ'\&−\textbackslash{}mathop\{sin\} θ'\textbackslash{}cr
\textbackslash{}mathop\{sin\} θ' \&\textbackslash{}mathop\{cos\}
θ'\}\textbackslash{}right )\&\& \%\& \textbackslash{}\textbackslash{} \&
\textbackslash{}quad \& = \textbackslash{}left
(\textbackslash{}matrix\{\textbackslash{},\textbackslash{}mathop\{cos\}
(θ + θ')\&−\textbackslash{}mathop\{sin\} (θ + θ') \textbackslash{}cr
\textbackslash{}mathop\{sin\} (θ + θ')\&\textbackslash{}mathop\{cos\} (θ
+ θ')\}\textbackslash{}right )\%\& \textbackslash{}\textbackslash{}
\textbackslash{}end\{eqnarray*\}

ce qui montre que le groupe SO(2) = \textbackslash{}\{R(θ) =
\textbackslash{}left
(\textbackslash{}matrix\{\textbackslash{},\textbackslash{}mathop\{cos\}
θ\&−\textbackslash{}mathop\{sin\} θ\textbackslash{}cr
\textbackslash{}mathop\{sin\} θ \&\textbackslash{}mathop\{cos\}
θ\}\textbackslash{}right )\textbackslash{}mathrel\{∣\}θ ∈
ℝ∕2πℤ\textbackslash{}\} est commutatif et isomorphe à (ℝ∕2πℤ,+). Soit u
∈ SO(E)~; si ℰ et ℰ' sont deux bases orthonormées de même sens
(c'est-à-dire que
\textbackslash{}mathop\{\textbackslash{}mathrm\{det\}\}
\{P\}\_\{ℰ\}\^{}\{ℰ'\} \textgreater{} 0), alors P =
\{P\}\_\{ℰ\}\^{}\{ℰ'\}∈ SO(2), on a

\textbackslash{}mathop\{\textbackslash{}mathrm\{Mat\}\} (u,ℰ') =
\{P\}\^{}\{−1\}\textbackslash{}mathop\{ \textbackslash{}mathrm\{Mat\}\}
(u,ℰ)P =
\{P\}\^{}\{−1\}P\textbackslash{}mathop\{\textbackslash{}mathrm\{Mat\}\}
(u,ℰ) =\textbackslash{}mathop\{ \textbackslash{}mathrm\{Mat\}\} (u,ℰ)

puisque SO(2) est commutatif et que P et
\textbackslash{}mathop\{\textbackslash{}mathrm\{Mat\}\} (u,ℰ) sont
toutes deux dans SO(2). Donc θ ne dépend que de l'orientation de la base
ℰ. Si maintenant, ℰ et ℰ' sont deux bases orthonormées de sens contraire
(c'est-à-dire que
\textbackslash{}mathop\{\textbackslash{}mathrm\{det\}\}
\{P\}\_\{ℰ\}\^{}\{ℰ'\} \textless{} 0), alors
\textbackslash{}mathop\{\textbackslash{}mathrm\{Mat\}\} (u,ℰ)P est une
matrice orthogonale de déterminant − 1. Comme on l'a vu, son carré est
nécessairement l'identité de même que le carré de P, ce qui montre que
\textbackslash{}mathop\{\textbackslash{}mathrm\{Mat\}\} (u,ℰ') =
\{P\}\^{}\{−1\}\textbackslash{}mathop\{ \textbackslash{}mathrm\{Mat\}\}
(u,ℰ)P = P\textbackslash{}mathop\{\textbackslash{}mathrm\{Mat\}\} (u,ℰ)P
=\textbackslash{}mathop\{ \textbackslash{}mathrm\{Mat\}\}
\{(u,ℰ)\}\^{}\{−1\} = R(−θ) (si
\textbackslash{}mathop\{\textbackslash{}mathrm\{Mat\}\} (u,ℰ) = R(θ))~:
un changement d'orientation de la base change donc θ en − θ.

Définition~12.6.2 Soit E un plan euclidien orienté, u ∈ SO(E). On
appelle mesure de la rotation u l'unique élément θ de ℝ∕2πℤ tel que,
pour toute base orthonormée directe ℰ de E, on ait
\textbackslash{}mathop\{\textbackslash{}mathrm\{Mat\}\} (u,ℰ) =
\textbackslash{}left
(\textbackslash{}matrix\{\textbackslash{},\textbackslash{}mathop\{cos\}
θ\&−\textbackslash{}mathop\{sin\} θ\textbackslash{}cr
\textbackslash{}mathop\{sin\} θ \&\textbackslash{}mathop\{cos\}
θ\}\textbackslash{}right ).

\paragraph{12.6.5 Réduction des endomorphismes orthogonaux}

Théorème~12.6.11 Soit E un espace euclidien et u un endomorphisme
orthogonal de E. Alors il existe une base orthonormée ℰ de E telle que

\textbackslash{}mathop\{\textbackslash{}mathrm\{Mat\}\} (u,ℰ) =
\textbackslash{}left
(\textbackslash{}matrix\{\textbackslash{},\{I\}\_\{p\}\&0
\&\textbackslash{}mathop\{\textbackslash{}mathop\{\ldots{}\}\}
\&\textbackslash{}mathop\{\textbackslash{}mathop\{\ldots{}\}\}\&\textbackslash{}mathop\{\textbackslash{}mathop\{\ldots{}\}\}\&0
\textbackslash{}cr 0 \&−\{I\}\_\{q\}\&0
\&\textbackslash{}mathop\{\textbackslash{}mathop\{\ldots{}\}\}\&\textbackslash{}mathop\{\textbackslash{}mathop\{\ldots{}\}\}\&\textbackslash{}mathop\{\textbackslash{}mathop\{⋮\}\}
\textbackslash{}cr \textbackslash{}mathop\{\textbackslash{}mathop\{⋮\}\}
\&0
\&\{A\}\_\{1\}\&0\&\textbackslash{}mathop\{\textbackslash{}mathop\{\ldots{}\}\}\&\textbackslash{}mathop\{\textbackslash{}mathop\{⋮\}\}
\textbackslash{}cr \textbackslash{}mathop\{\textbackslash{}mathop\{⋮\}\}
\&\textbackslash{}mathop\{\textbackslash{}mathop\{\ldots{}\}\}
\&\textbackslash{}mathrel\{⋱\}
\&\textbackslash{}mathrel\{⋱\}\&\textbackslash{}mathrel\{⋱\}\&\textbackslash{}mathop\{\textbackslash{}mathop\{⋮\}\}
\textbackslash{}cr \textbackslash{}mathop\{\textbackslash{}mathop\{⋮\}\}
\&\textbackslash{}mathop\{\textbackslash{}mathop\{\ldots{}\}\}
\&\textbackslash{}mathop\{\textbackslash{}mathop\{\ldots{}\}\}
\&\textbackslash{}mathrel\{⋱\}\&\textbackslash{}mathrel\{⋱\}\&0
\textbackslash{}cr 0
\&\textbackslash{}mathop\{\textbackslash{}mathop\{\ldots{}\}\}
\&\textbackslash{}mathop\{\textbackslash{}mathop\{\ldots{}\}\}
\&\textbackslash{}mathop\{\textbackslash{}mathop\{\ldots{}\}\}\&0\&\{A\}\_\{s\}\}\textbackslash{}right
)

avec \{A\}\_\{i\} = \textbackslash{}left
(\textbackslash{}matrix\{\textbackslash{},\textbackslash{}mathop\{cos\}
\{θ\}\_\{i\}\&−\textbackslash{}mathop\{sin\} \{θ\}\_\{i\}
\textbackslash{}cr \textbackslash{}mathop\{sin\}
\{θ\}\_\{i\}\&\textbackslash{}mathop\{cos\} \{θ\}\_\{i\}
\}\textbackslash{}right ), \{θ\}\_\{i\} ∈ ℝ ∖ 2πℤ.

Démonstration Par récurrence sur n =\textbackslash{}mathop\{ dim\} E. Si
n = 1, alors u = ±\{\textbackslash{}mathrm\{Id\}\}\_\{E\} et le résultat
est évident. Supposons le donc démontré pour tout espace euclidien de
dimension strictement inférieure à n et soit E de dimension n, u ∈ O(E).
Si u admet une valeur propre λ, soit x un vecteur propre associé. On a
\textbar{}λ\textbar{}\textbackslash{}\textbar{}x\textbackslash{}\textbar{}
=\textbackslash{}\textbar{} u(x)\textbackslash{}\textbar{}
=\textbackslash{}\textbar{} x\textbackslash{}\textbar{}, d'où λ = ±1. La
droite ℝx est stable par u, donc H = \{(ℝx)\}\^{}\{⊥\} aussi. La
restriction v de u à H est un endomorphisme orthogonal de H et par
l'hypothèse de récurrence, il existe une base orthonormée
(\{e\}\_\{2\},\textbackslash{}mathop\{\textbackslash{}mathop\{\ldots{}\}\},\{e\}\_\{n\})
de H telle que la matrice de v dans cette base soit de la forme voulue.
Alors (\{ x \textbackslash{}over
\textbackslash{}\textbar{}x\textbackslash{}\textbar{}\}
,\{e\}\_\{2\},\textbackslash{}mathop\{\textbackslash{}mathop\{\ldots{}\}\},\{e\}\_\{n\})
est une base orthonormée de E et à une permutation près de cette base
(si λ = −1), la matrice de u dans cette base est de la forme voulue. Si
u n'a pas de valeur propre (réelle), soit Π un plan stable par u, dont
l'existence est garantie par un lemme précédent. L'endomorphisme de Π
induit par u est un endomorphisme orthogonal de Π sans valeur propre,
donc une rotation d'angle \{θ\}\_\{1\} ∈ ℝ ∖ πℤ. Soit
(\{e\}\_\{1\},\{e\}\_\{2\}) une base orthonormée de Π. Le sous-espace de
dimension n − 2, H = \{Π\}\^{}\{⊥\} est également stable par u et
l'endomorphisme v de H induit par u est un endomorphisme orthogonal de H
sans valeur propre. Par hypothèse de récurrence, il existe une base
orthonormée
(\{e\}\_\{3\},\textbackslash{}mathop\{\textbackslash{}mathop\{\ldots{}\}\},\{e\}\_\{n\})
de H telle que la matrice de v dans cette base soit de la forme voulue
(avec p = q = 0). Alors la matrice de u dans la base orthonormée
(\{e\}\_\{1\},\{e\}\_\{2\},\{e\}\_\{3\},\textbackslash{}mathop\{\textbackslash{}mathop\{\ldots{}\}\},\{e\}\_\{n\})
est de la forme voulue, ce qui achève la démonstration.

Exemple~12.6.1 Si \textbackslash{}mathop\{dim\} E = 3, on a les formes
réduites possibles (en tenant compte de
\textbackslash{}mathop\{\textbackslash{}mathrm\{det\}\} u =
\{(−1)\}\^{}\{q\} et de p + q + 2s = 3)

\begin{itemize}
\itemsep1pt\parskip0pt\parsep0pt
\item
  \textbackslash{}mathop\{\textbackslash{}mathrm\{det\}\} u = 1

  \begin{itemize}
  \itemsep1pt\parskip0pt\parsep0pt
  \item
    \textbackslash{}left
    (\textbackslash{}matrix\{\textbackslash{},1\&0\&0 \textbackslash{}cr
    0\&1\&0 \textbackslash{}cr 0\&0\&1\}\textbackslash{}right )
    (identité),
  \item
    \textbackslash{}left (\textbackslash{}matrix\{\textbackslash{},1\&0
    \&0 \textbackslash{}cr 0\&−1\&0 \textbackslash{}cr 0\&0
    \&−1\}\textbackslash{}right ) (retournement d'axe ℝ\{e\}\_\{1\}),
  \item
    \textbackslash{}left (\textbackslash{}matrix\{\textbackslash{},1\&0
    \&0 \textbackslash{}cr 0\&\textbackslash{}mathop\{cos\}
    θ\&−\textbackslash{}mathop\{sin\} θ \textbackslash{}cr
    0\&\textbackslash{}mathop\{sin\} θ\&\textbackslash{}mathop\{cos\} θ
    \}\textbackslash{}right ) (rotation d'axe ℝ\{e\}\_\{1\})
  \end{itemize}
\item
  \textbackslash{}mathop\{\textbackslash{}mathrm\{det\}\} u = −1

  \begin{itemize}
  \itemsep1pt\parskip0pt\parsep0pt
  \item
    \textbackslash{}left (\textbackslash{}matrix\{\textbackslash{},−1\&0
    \&0 \textbackslash{}cr 0 \&−1\&0 \textbackslash{}cr 0 \&0
    \&−1\}\textbackslash{}right )\textbackslash{}quad (
    −\{\textbackslash{}mathrm\{Id\}\}\_\{E\}),
  \item
    \textbackslash{}left
    (\textbackslash{}matrix\{\textbackslash{},1\&0\&0 \textbackslash{}cr
    0\&1\&0 \textbackslash{}cr 0\&0\&−1\}\textbackslash{}right )
    (symétrie par rapport au plan
    \textbackslash{}mathop\{\textbackslash{}mathrm\{Vect\}\}(\{e\}\_\{1\},\{e\}\_\{2\})),
  \item
    \textbackslash{}left (\textbackslash{}matrix\{\textbackslash{},−1\&0
    \&0 \textbackslash{}cr 0 \&\textbackslash{}mathop\{cos\}
    θ\&−\textbackslash{}mathop\{sin\} θ \textbackslash{}cr 0
    \&\textbackslash{}mathop\{sin\} θ\&\textbackslash{}mathop\{cos\} θ
    \}\textbackslash{}right ) (composée de la symétrie par rapport au
    plan
    \textbackslash{}mathop\{\textbackslash{}mathrm\{Vect\}\}(\{e\}\_\{2\},\{e\}\_\{3\})
    et d'une rotation d'axe ℝ\{e\}\_\{1\})
  \end{itemize}
\end{itemize}

\paragraph{12.6.6 Produit vectoriel, produit mixte}

Théorème~12.6.12 Soit E un espace euclidien orienté de dimension n.
L'application n linéaire alternée
\{\textbackslash{}mathop\{\textbackslash{}mathrm\{det\}\} \}\_\{ℰ\} est
indépendante du choix de la base orthonormée directe ℰ.

Démonstration Si ℰ' est une autre base orthonormée directe, la matrice
de passage \{P\}\_\{ℰ\}\^{}\{ℰ'\} est à la fois orthogonale et de
déterminant strictement positif, donc de déterminant 1. Or on a

\{\textbackslash{}mathop\{\textbackslash{}mathrm\{det\}\} \}\_\{ℰ\}
=\{\textbackslash{}mathop\{ \textbackslash{}mathrm\{det\}\}
\}\_\{ℰ\}(ℰ')\{\textbackslash{}mathop\{\textbackslash{}mathrm\{det\}\}
\}\_\{ℰ'\} =\textbackslash{}mathop\{ \textbackslash{}mathrm\{det\}\}
\{P\}\_\{ℰ\}\^{}\{ℰ'\}\{\textbackslash{}mathop\{\textbackslash{}mathrm\{det\}\}
\}\_\{ ℰ'\} =\{\textbackslash{}mathop\{ \textbackslash{}mathrm\{det\}\}
\}\_\{ℰ'\}

Définition~12.6.3 On notera
{[}\{x\}\_\{1\},\textbackslash{}mathop\{\textbackslash{}mathop\{\ldots{}\}\},\{x\}\_\{n\}{]}
=\{\textbackslash{}mathop\{ \textbackslash{}mathrm\{det\}\}
\}\_\{ℰ\}(\{x\}\_\{1\},\textbackslash{}mathop\{\textbackslash{}mathop\{\ldots{}\}\},\{x\}\_\{n\})
et on l'appellera le produit mixte des n vecteurs
\{x\}\_\{1\},\textbackslash{}mathop\{\textbackslash{}mathop\{\ldots{}\}\},\{x\}\_\{n\}.

Remarque~12.6.3 Il est clair qu'un changement d'orientation de l'espace
change le produit mixte en son opposé.

Théorème~12.6.13 Soit E un espace euclidien orienté. Alors, pour toute
famille
(\{x\}\_\{1\},\textbackslash{}mathop\{\textbackslash{}mathop\{\ldots{}\}\},\{x\}\_\{n\})
de E on a

\textbackslash{}mathop\{\textbackslash{}mathrm\{det\}\}
\textbackslash{}mathop\{Gram\}(\{x\}\_\{1\},\textbackslash{}mathop\{\textbackslash{}mathop\{\ldots{}\}\},\{x\}\_\{n\})
=
\{{[}\{x\}\_\{1\},\textbackslash{}mathop\{\textbackslash{}mathop\{\ldots{}\}\},\{x\}\_\{n\}{]}\}\^{}\{2\}

Démonstration Soit ℰ une base orthonormée directe et soit A la matrice
des coordonnées de
(\{x\}\_\{1\},\textbackslash{}mathop\{\textbackslash{}mathop\{\ldots{}\}\},\{x\}\_\{n\})
dans la base ℰ. On a alors
\textbackslash{}mathop\{\textbackslash{}mathrm\{det\}\} A =
{[}\{x\}\_\{1\},\textbackslash{}mathop\{\textbackslash{}mathop\{\ldots{}\}\},\{x\}\_\{n\}{]}.
D'autre part

(\{x\}\_\{i\}\textbackslash{}mathrel\{∣\}\{x\}\_\{j\}) =\{
\textbackslash{}mathop\{∑ \}\}\_\{k=1\}\^{}\{n\}\{a\}\_\{
k,i\}\{a\}\_\{k,j\} = \{\{(\}\^{}\{t\}AA)\}\_\{ i,j\}

si bien que
\textbackslash{}mathop\{Gram\}(\{x\}\_\{1\},\textbackslash{}mathop\{\textbackslash{}mathop\{\ldots{}\}\},\{x\}\_\{n\})
\{= \}\^{}\{t\}AA. On a donc

\textbackslash{}mathop\{\textbackslash{}mathrm\{det\}\}
\textbackslash{}mathop\{Gram\}(\{x\}\_\{1\},\textbackslash{}mathop\{\textbackslash{}mathop\{\ldots{}\}\},\{x\}\_\{n\})
=\{\textbackslash{}mathop\{ \textbackslash{}mathrm\{det\}\}
\}\^{}\{t\}AA =
\{(\textbackslash{}mathop\{\textbackslash{}mathrm\{det\}\} A)\}\^{}\{2\}
= \{{[}\{x\}\_\{
1\},\textbackslash{}mathop\{\textbackslash{}mathop\{\ldots{}\}\},\{x\}\_\{n\}{]}\}\^{}\{2\}

Théorème~12.6.14 (et définition). Soit E un espace euclidien orienté de
dimension n. Soit
\{x\}\_\{1\},\textbackslash{}mathop\{\textbackslash{}mathop\{\ldots{}\}\},\{x\}\_\{n−1\}
∈ E. Il existe un unique vecteur, appelé le produit vectoriel des n − 1
vecteurs
\{x\}\_\{1\},\textbackslash{}mathop\{\textbackslash{}mathop\{\ldots{}\}\},\{x\}\_\{n−1\}
et noté \{x\}\_\{1\}
∧\textbackslash{}mathop\{\textbackslash{}mathop\{\ldots{}\}\} ∧
\{x\}\_\{n−1\} tel que

\textbackslash{}mathop\{∀\}y ∈ E,
{[}\{x\}\_\{1\},\textbackslash{}mathop\{\textbackslash{}mathop\{\ldots{}\}\},\{x\}\_\{n−1\},y{]}
= (\{x\}\_\{1\}
∧\textbackslash{}mathop\{\textbackslash{}mathop\{\ldots{}\}\} ∧
\{x\}\_\{n−1\}\textbackslash{}mathrel\{∣\}y)

Démonstration L'application
y\textbackslash{}mathrel\{↦\}{[}\{x\}\_\{1\},\textbackslash{}mathop\{\textbackslash{}mathop\{\ldots{}\}\},\{x\}\_\{n−1\},y{]}
est une forme linéaire sur E, donc représentée par le produit scalaire
avec un unique vecteur.

Proposition~12.6.15

\begin{itemize}
\itemsep1pt\parskip0pt\parsep0pt
\item
  (i) \{x\}\_\{1\}
  ∧\textbackslash{}mathop\{\textbackslash{}mathop\{\ldots{}\}\} ∧
  \{x\}\_\{n−1\} = 0 \textbackslash{}mathrel\{⇔\}
  (\{x\}\_\{1\},\textbackslash{}mathop\{\textbackslash{}mathop\{\ldots{}\}\},\{x\}\_\{n−1\})
  est une famille liée
\item
  (ii) \textbackslash{}mathop\{∀\}i ∈ {[}1,n − 1{]}, \{x\}\_\{1\}
  ∧\textbackslash{}mathop\{\textbackslash{}mathop\{\ldots{}\}\} ∧
  \{x\}\_\{n−1\} ⊥ \{x\}\_\{i\}
\item
  (iii) si
  (\{x\}\_\{1\},\textbackslash{}mathop\{\textbackslash{}mathop\{\ldots{}\}\},\{x\}\_\{n−1\})
  est une famille libre, alors
  (\{x\}\_\{1\},\textbackslash{}mathop\{\textbackslash{}mathop\{\ldots{}\}\},\{x\}\_\{n−1\},\{x\}\_\{1\}
  ∧\textbackslash{}mathop\{\textbackslash{}mathop\{\ldots{}\}\} ∧
  \{x\}\_\{n−1\}) est une base directe de E
\item
  (iv) \textbackslash{}\textbar{}\{x\}\_\{1\}
  ∧\textbackslash{}mathop\{\textbackslash{}mathop\{\ldots{}\}\} ∧
  \{x\{\}\_\{n−1\}\textbackslash{}\textbar{}\}\^{}\{2\}
  =\textbackslash{}mathop\{ \textbackslash{}mathrm\{det\}\}
  \textbackslash{}mathop\{Gram\}(\{x\}\_\{1\},\textbackslash{}mathop\{\textbackslash{}mathop\{\ldots{}\}\},\{x\}\_\{n−1\}).
\end{itemize}

Démonstration (i) On a en effet

\textbackslash{}begin\{eqnarray*\}
(\{x\}\_\{1\},\textbackslash{}mathop\{\textbackslash{}mathop\{\ldots{}\}\},\{x\}\_\{n−1\})\textbackslash{}text\{
libre \}\& \textbackslash{}mathrel\{⇔\} \& \textbackslash{}mathop\{∃\}y
∈ E,
(\{x\}\_\{1\},\textbackslash{}mathop\{\textbackslash{}mathop\{\ldots{}\}\},\{x\}\_\{n−1\},y)\textbackslash{}text\{
base de \}E\%\& \textbackslash{}\textbackslash{} \&
\textbackslash{}mathrel\{⇔\} \& \textbackslash{}mathop\{∃\}y ∈ E,
{[}\{x\}\_\{1\},\textbackslash{}mathop\{\textbackslash{}mathop\{\ldots{}\}\},\{x\}\_\{n−1\},y{]}\textbackslash{}mathrel\{≠\}0
\%\& \textbackslash{}\textbackslash{} \& \textbackslash{}mathrel\{⇔\} \&
\textbackslash{}mathop\{∃\}y ∈ E, (\{x\}\_\{1\}
∧\textbackslash{}mathop\{\textbackslash{}mathop\{\ldots{}\}\} ∧
\{x\}\_\{n−1\}\textbackslash{}mathrel\{∣\}y)\textbackslash{}mathrel\{≠\}0
\%\& \textbackslash{}\textbackslash{} \& \textbackslash{}mathrel\{⇔\} \&
\{x\}\_\{1\}
∧\textbackslash{}mathop\{\textbackslash{}mathop\{\ldots{}\}\} ∧
\{x\}\_\{n−1\}\textbackslash{}mathrel\{≠\}0 \%\&
\textbackslash{}\textbackslash{} \textbackslash{}end\{eqnarray*\}

(ii) (\{x\}\_\{1\}
∧\textbackslash{}mathop\{\textbackslash{}mathop\{\ldots{}\}\} ∧
\{x\}\_\{n−1\}\textbackslash{}mathrel\{∣\}\{x\}\_\{i\}) =
{[}\{x\}\_\{1\},\textbackslash{}mathop\{\textbackslash{}mathop\{\ldots{}\}\},\{x\}\_\{i\},\textbackslash{}mathop\{\textbackslash{}mathop\{\ldots{}\}\},\{x\}\_\{n−1\},\{x\}\_\{i\}{]}
= 0

(iii) On a

\textbackslash{}begin\{eqnarray*\}
{[}\{x\}\_\{1\},\textbackslash{}mathop\{\textbackslash{}mathop\{\ldots{}\}\},\{x\}\_\{n−1\},\{x\}\_\{1\}
∧\textbackslash{}mathop\{\textbackslash{}mathop\{\ldots{}\}\} ∧
\{x\}\_\{n−1\}{]}\& =\& (\{x\}\_\{1\}
∧\textbackslash{}mathop\{\textbackslash{}mathop\{\ldots{}\}\} ∧
\{x\}\_\{n−1\}\textbackslash{}mathrel\{∣\}\{x\}\_\{1\}
∧\textbackslash{}mathop\{\textbackslash{}mathop\{\ldots{}\}\} ∧
\{x\}\_\{n−1\})\%\& \textbackslash{}\textbackslash{} \& =\&
\textbackslash{}\textbar{}\{x\}\_\{1\}
∧\textbackslash{}mathop\{\textbackslash{}mathop\{\ldots{}\}\} ∧
\{x\{\}\_\{n−1\}\textbackslash{}\textbar{}\}\^{}\{2\} \textgreater{} 0
\%\& \textbackslash{}\textbackslash{} \textbackslash{}end\{eqnarray*\}

(iv) Comme on vient de le voir,

\textbackslash{}begin\{eqnarray*\}
\textbackslash{}\textbar{}\{x\}\_\{1\}
∧\textbackslash{}mathop\{\textbackslash{}mathop\{\ldots{}\}\} ∧
\{x\{\}\_\{n−1\}\textbackslash{}\textbar{}\}\^{}\{4\}\&\& \%\&
\textbackslash{}\textbackslash{} \& =\&
\{{[}\{x\}\_\{1\},\textbackslash{}mathop\{\textbackslash{}mathop\{\ldots{}\}\},\{x\}\_\{n−1\},\{x\}\_\{1\}
∧\textbackslash{}mathop\{\textbackslash{}mathop\{\ldots{}\}\} ∧
\{x\}\_\{n−1\}{]}\}\^{}\{2\} \%\& \textbackslash{}\textbackslash{} \&
=\& \textbackslash{}mathop\{\textbackslash{}mathrm\{det\}\}
\textbackslash{}mathop\{Gram\}(\{x\}\_\{1\},\textbackslash{}mathop\{\textbackslash{}mathop\{\ldots{}\}\},\{x\}\_\{n−1\},\{x\}\_\{1\}
∧\textbackslash{}mathop\{\textbackslash{}mathop\{\ldots{}\}\} ∧
\{x\}\_\{n−1\}) \%\& \textbackslash{}\textbackslash{} \& =\&
\textbackslash{}left
\textbar{}\textbackslash{}matrix\{\textbackslash{},\textbackslash{}mathop\{Gram\}(\{x\}\_\{1\},\textbackslash{}mathop\{\textbackslash{}mathop\{\ldots{}\}\},\{x\}\_\{n−1\})\&0
\textbackslash{}cr 0 \&\textbackslash{}\textbar{}\{x\}\_\{1\}
∧\textbackslash{}mathop\{\textbackslash{}mathop\{\ldots{}\}\} ∧
\{x\{\}\_\{n−1\}\textbackslash{}\textbar{}\}\^{}\{2\}\}\textbackslash{}right
\textbar{} \%\& \textbackslash{}\textbackslash{} \& =\&
\textbackslash{}\textbar{}\{x\}\_\{1\}
∧\textbackslash{}mathop\{\textbackslash{}mathop\{\ldots{}\}\} ∧
\{x\{\}\_\{n−1\}\textbackslash{}\textbar{}\}\^{}\{2\}\textbackslash{}mathop\{
\textbackslash{}mathrm\{det\}\} \textbackslash{}mathop\{Gram\}(\{x\}\_\{
1\},\textbackslash{}mathop\{\textbackslash{}mathop\{\ldots{}\}\},\{x\}\_\{n−1\})\%\&
\textbackslash{}\textbackslash{} \textbackslash{}end\{eqnarray*\}

puisque (\{x\}\_\{1\}
∧\textbackslash{}mathop\{\textbackslash{}mathop\{\ldots{}\}\} ∧
\{x\}\_\{n−1\}\textbackslash{}mathrel\{∣\}\{x\}\_\{i\}) = 0. Si
(\{x\}\_\{1\},\textbackslash{}mathop\{\textbackslash{}mathop\{\ldots{}\}\},\{x\}\_\{n\})
est libre, on peut simplifier par \textbackslash{}\textbar{}\{x\}\_\{1\}
∧\textbackslash{}mathop\{\textbackslash{}mathop\{\ldots{}\}\} ∧
\{x\{\}\_\{n−1\}\textbackslash{}\textbar{}\}\^{}\{2\} et on obtient
\textbackslash{}\textbar{}\{x\}\_\{1\}
∧\textbackslash{}mathop\{\textbackslash{}mathop\{\ldots{}\}\} ∧
\{x\{\}\_\{n−1\}\textbackslash{}\textbar{}\}\^{}\{2\}
=\textbackslash{}mathop\{ \textbackslash{}mathrm\{det\}\}
\textbackslash{}mathop\{Gram\}(\{x\}\_\{1\},\textbackslash{}mathop\{\textbackslash{}mathop\{\ldots{}\}\},\{x\}\_\{n−1\}),
formule qui est encore exacte si la famille est liée puisque les deux
termes valent 0.

Remarque~12.6.4 Si
(\{x\}\_\{1\},\textbackslash{}mathop\{\textbackslash{}mathop\{\ldots{}\}\},\{x\}\_\{n−1\})
est une famille libre, (ii) définit la droite engendrée par \{x\}\_\{1\}
∧\textbackslash{}mathop\{\textbackslash{}mathop\{\ldots{}\}\} ∧
\{x\}\_\{n−1\}, (iii) définit son orientation sur cette droite et (iv)
définit sa norme, ce qui fournit une construction géométrique du produit
vectoriel~: c'est le vecteur orthogonal à l'hyperplan
\textbackslash{}mathop\{\textbackslash{}mathrm\{Vect\}\}(\{x\}\_\{1\},\textbackslash{}mathop\{\textbackslash{}mathop\{\ldots{}\}\},\{x\}\_\{n−1\}),
tel que la base
(\{x\}\_\{1\},\textbackslash{}mathop\{\textbackslash{}mathop\{\ldots{}\}\},\{x\}\_\{n−1\},\{x\}\_\{1\}
∧\textbackslash{}mathop\{\textbackslash{}mathop\{\ldots{}\}\} ∧
\{x\}\_\{n−1\}) soit une base directe de E et dont la norme est
\textbackslash{}sqrt\{\textbackslash{}mathop\{\textbackslash{}mathrm\{det
\}\} \textbackslash{}mathop\{ Gram\} (\{x\}\_\{1 \} ,
\textbackslash{}mathop\{\textbackslash{}mathop\{\ldots{}\}\} ,
\{x\}\_\{n−1 \} )\}.

Coordonnées du produit vectoriel

Soit ℰ une base orthonormée directe de E, \{x\}\_\{j\}
=\{\textbackslash{}mathop\{ \textbackslash{}mathop\{∑ \}\}
\}\_\{i=1\}\^{}\{n\}\{α\}\_\{i,j\}\{e\}\_\{i\}, y
=\{\textbackslash{}mathop\{ \textbackslash{}mathop\{∑ \}\}
\}\_\{i=1\}\^{}\{n\}\{y\}\_\{i\}\{e\}\_\{i\}. On a alors, en développant
le déterminant suivant la dernière colonne

\textbackslash{}begin\{eqnarray*\}
{[}\{x\}\_\{1\},\textbackslash{}mathop\{\textbackslash{}mathop\{\ldots{}\}\},\{x\}\_\{n−1\},y{]}\&
=\& \textbackslash{}left
\textbar{}\textbackslash{}matrix\{\textbackslash{},\{α\}\_\{1,1\}\&\textbackslash{}mathop\{\textbackslash{}mathop\{\ldots{}\}\}\&\{α\}\_\{1,n−1\}\&\{y\}\_\{1\}
\textbackslash{}cr
\textbackslash{}mathop\{\textbackslash{}mathop\{\ldots{}\}\}
\&\textbackslash{}mathop\{\textbackslash{}mathop\{\ldots{}\}\}\&\textbackslash{}mathop\{\textbackslash{}mathop\{\ldots{}\}\}
\&\textbackslash{}mathop\{\textbackslash{}mathop\{\ldots{}\}\}
\textbackslash{}cr
\{α\}\_\{n,1\}\&\textbackslash{}mathop\{\textbackslash{}mathop\{\ldots{}\}\}\&\{α\}\_\{n,n−1\}\&\{y\}\_\{n\}\}\textbackslash{}right
\textbar{} =\{ \textbackslash{}mathop\{∑ \}\}\_\{i=1\}\^{}\{n\}\{Δ\}\_\{
i\}\{y\}\_\{i\}\%\& \textbackslash{}\textbackslash{} \& =\&
(\{\textbackslash{}mathop\{∑ \}\}\_\{i=1\}\^{}\{n\}\{Δ\}\_\{
i\}\{e\}\_\{i\}\textbackslash{}mathrel\{∣\}y) \%\&
\textbackslash{}\textbackslash{} \textbackslash{}end\{eqnarray*\}

avec

\{Δ\}\_\{i\} = \{(−1)\}\^{}\{n+i\}\textbackslash{}left
\textbar{}\textbackslash{}matrix\{\textbackslash{},\{α\}\_\{1,1\}
\&\textbackslash{}mathop\{\textbackslash{}mathop\{\ldots{}\}\}\&\{α\}\_\{1,n−1\}
\textbackslash{}cr
\textbackslash{}mathop\{\textbackslash{}mathop\{\ldots{}\}\}
\&\textbackslash{}mathop\{\textbackslash{}mathop\{\ldots{}\}\}\&\textbackslash{}mathop\{\textbackslash{}mathop\{\ldots{}\}\}
\textbackslash{}cr
\{α\}\_\{i−1,1\}\&\textbackslash{}mathop\{\textbackslash{}mathop\{\ldots{}\}\}\&\{α\}\_\{i−1,n−1\}
\textbackslash{}cr
\{α\}\_\{i+1,1\}\&\textbackslash{}mathop\{\textbackslash{}mathop\{\ldots{}\}\}\&\{α\}\_\{i+1,n−1\}
\textbackslash{}cr
\textbackslash{}mathop\{\textbackslash{}mathop\{\ldots{}\}\}
\&\textbackslash{}mathop\{\textbackslash{}mathop\{\ldots{}\}\}\&\textbackslash{}mathop\{\textbackslash{}mathop\{\ldots{}\}\}
\textbackslash{}cr \{α\}\_\{n,1\}
\&\textbackslash{}mathop\{\textbackslash{}mathop\{\ldots{}\}\}\&\{α\}\_\{n,n−1\}
\}\textbackslash{}right \textbar{}

On en déduit que

\{x\}\_\{1\}
∧\textbackslash{}mathop\{\textbackslash{}mathop\{\ldots{}\}\} ∧
\{x\}\_\{n−1\} =\{ \textbackslash{}mathop\{∑
\}\}\_\{i=1\}\^{}\{n\}\{Δ\}\_\{ i\}\{e\}\_\{i\}

Produit vectoriel en dimension 3

On a alors \textbackslash{}\textbar{}\{x\}\_\{1\} ∧
\{x\{\}\_\{2\}\textbackslash{}\textbar{}\}\^{}\{2\}
=\textbackslash{}mathop\{ \textbackslash{}mathrm\{det\}\}
\textbackslash{}mathop\{Gram\}(\{x\}\_\{1\},\{x\}\_\{2\})
=\textbackslash{}\textbar{}
\{x\{\}\_\{1\}\textbackslash{}\textbar{}\}\^{}\{2\}\textbackslash{}\textbar{}\{x\{\}\_\{2\}\textbackslash{}\textbar{}\}\^{}\{2\}
− \{(\{x\}\_\{1\}\textbackslash{}mathrel\{∣\}\{x\}\_\{2\})\}\^{}\{2\}
=\textbackslash{}\textbar{}
\{x\{\}\_\{1\}\textbackslash{}\textbar{}\}\^{}\{2\}\textbackslash{}\textbar{}\{x\{\}\_\{2\}\textbackslash{}\textbar{}\}\^{}\{2\}(1
−\{\textbackslash{}mathop\{ cos\} \}\^{}\{2\}θ) où θ désigne l'angle non
orienté des vecteurs \{x\}\_\{1\} et \{x\}\_\{2\}. On a donc alors

\textbackslash{}\textbar{}\{x\}\_\{1\} ∧
\{x\}\_\{2\}\textbackslash{}\textbar{} =\textbackslash{}\textbar{}
\{x\}\_\{1\}\textbackslash{}\textbar{}\textbackslash{},\textbackslash{}\textbar{}\{x\}\_\{2\}\textbackslash{}\textbar{}\textbackslash{}mathop\{
sin\} θ

On a également le résultat important suivant

Théorème~12.6.16 Soit E un espace euclidien de dimension 3,
\{x\}\_\{1\},\{x\}\_\{2\},\{x\}\_\{3\} ∈ E. Alors

(\{x\}\_\{1\} ∧ \{x\}\_\{2\}) ∧ \{x\}\_\{3\} =
(\{x\}\_\{1\}\textbackslash{}mathrel\{∣\}\{x\}\_\{3\})\{x\}\_\{2\} −
(\{x\}\_\{2\}\textbackslash{}mathrel\{∣\}\{x\}\_\{3\})\{x\}\_\{1\}

Démonstration Si (\{x\}\_\{1\},\{x\}\_\{2\}) est liée, on a par exemple
\{x\}\_\{2\} = λ\{x\}\_\{1\} et on vérifie facilement que les deux
membres valent 0. Si (\{x\}\_\{1\},\{x\}\_\{2\}) est libre, alors
(\{x\}\_\{1\} ∧ \{x\}\_\{2\}) ∧ \{x\}\_\{3\} ∈ \{(\{x\}\_\{1\} ∧
\{x\}\_\{2\})\}\^{}\{⊥\} =\textbackslash{}mathop\{
\textbackslash{}mathrm\{Vect\}\}(\{x\}\_\{1\},\{x\}\_\{2\}), si bien
qu'a priori (\{x\}\_\{1\} ∧ \{x\}\_\{2\}) ∧ \{x\}\_\{3\} = λ\{x\}\_\{1\}
+ μ\{x\}\_\{2\}. Un calcul sur les coordonnées dans une base orthonormée
directe adéquate (par exemple (\{ \{x\}\_\{1\} \textbackslash{}over
\textbackslash{}\textbar{}\{x\}\_\{1\}\textbackslash{}\textbar{}\} ,\{
\{x\}\_\{1\}∧\{x\}\_\{2\} \textbackslash{}over
\textbackslash{}\textbar{}\{x\}\_\{1\}∧\{x\}\_\{2\}\textbackslash{}\textbar{}\}
,\textbackslash{}mathop\{\textbackslash{}mathop\{\ldots{}\}\})) fournit
les valeurs de λ et μ.

Remarque~12.6.5 On pourra utiliser le moyen mnémotechnique suivant~: le
produit scalaire affecté du signe + concerne les deux termes extrêmes de
l'expression (\{x\}\_\{1\} ∧ \{x\}\_\{2\}) ∧ \{x\}\_\{3\}.

Corollaire~12.6.17 Soit a\textbackslash{}mathrel\{≠\}0. L'équation x ∧ a
= b a une solution si et seulement si~a ⊥ b.

Démonstration Il est clair que la condition est nécessaire. Si elle est
vérifiée, cherchons x sous la forme \{x\}\_\{0\} = λa ∧ b. On a alors

\{x\}\_\{0\} ∧ a = λ(a ∧ b) ∧ a = λ(a\textbackslash{}mathrel\{∣\}a)b −
λ(a\textbackslash{}mathrel\{∣\}b)a =
λ\textbackslash{}\textbar{}\{a\textbackslash{}\textbar{}\}\^{}\{2\}b

Donc \{x\}\_\{0\} =\{ 1 \textbackslash{}over
\textbackslash{}\textbar{}\{a\textbackslash{}\textbar{}\}\^{}\{2\}\} a ∧
b est une solution.

Remarque~12.6.6 On a alors

\textbackslash{}begin\{eqnarray*\} x ∧ a = b\&
\textbackslash{}mathrel\{⇔\} \& x ∧ a = \{x\}\_\{0\} ∧ a
\textbackslash{}mathrel\{⇔\} (x − \{x\}\_\{0\}) ∧ a = 0\%\&
\textbackslash{}\textbackslash{} \& \textbackslash{}mathrel\{⇔\} \& x −
\{x\}\_\{0\} = λa \%\& \textbackslash{}\textbackslash{}
\textbackslash{}end\{eqnarray*\}

\paragraph{12.6.7 Angles}

On désigne par E un espace euclidien (de dimension finie), par O(E)
(resp. \{O\}\^{}\{+\}(E)) le groupe orthogonal (resp. le groupe des
rotations de E).

Notion générale d'angles d'objets

Soit X un ensemble de parties de E stable par O(E), c'est-à-dire que

\textbackslash{}mathop\{∀\}r ∈ O(E)\textbackslash{}quad
\textbackslash{}mathop\{∀\}A ∈ X\textbackslash{}quad r(A) ∈ X.

Exemple~: X peut être l'ensemble D(E) des droites de E, ou l'ensemble
\textbackslash{}tilde\{D\}(ℰ) des demi-droites de E, ou l'ensemble des
plans de E, ou l'ensemble des hyperplans de E.

Définition~12.6.4 On appelle angle non orienté (resp angle orienté)
d'éléments de X le quotient de X × X par la relation ℛ définie par

\textbackslash{}begin\{eqnarray*\}
(\{D\}\_\{1\},\{D\}\_\{2\})ℛ(\{D\}\_\{1\}',\{D\}\_\{2\}')
\textbackslash{}mathrel\{⇔\}\&\& \%\& \textbackslash{}\textbackslash{}
\& \& \textbackslash{}mathop\{∃\}r ∈ O(E) (\textbackslash{}text\{resp.
\}\{O\}\^{}\{+\}(E))\textbackslash{}quad r(\{D\}\_\{ 1\}) =
\{D\}\_\{1\}'\textbackslash{}text\{ et \}r(\{D\}\_\{2\}) =
\{D\}\_\{2\}'\%\& \textbackslash{}\textbackslash{}
\textbackslash{}end\{eqnarray*\}

On notera \textbackslash{}overline\{(\{D\}\_\{1\},\{D\}\_\{2\})\}(resp.
\textbackslash{}widehat\{(\{D\}\_\{1\},\{D\}\_\{2\})\}) la classe
d'équivalence du couple (\{D\}\_\{1\},\{D\}\_\{2\}) et on l'appellera
l'angle non orienté (resp. l'angle orienté) des objets \{D\}\_\{1\} et
\{D\}\_\{2\}.

Cela revient à définir les angles par les propriétés

\textbackslash{}overline\{(\{D\}\_\{1\},\{D\}\_\{2\})\} =
\textbackslash{}overline\{(\{D\}\_\{1\}',\{D\}\_\{2\}')\}
\textbackslash{}mathrel\{⇔\} \textbackslash{}mathop\{∃\}r ∈ O(E),
\textbackslash{}quad r(\{D\}\_\{1\}) = \{D\}\_\{1\}',r(\{D\}\_\{2\}) =
\{D\}\_\{2\}'

et

\textbackslash{}widehat\{(\{D\}\_\{1\},\{D\}\_\{2\})\}
=\textbackslash{}widehat\{ (\{D\}\_\{1\}',\{D\}\_\{2\}')
\textbackslash{}mathrel\{⇔\} \}\textbackslash{}mathop\{∃\}r ∈
\{O\}\^{}\{+\}(E), \textbackslash{}quad r(\{D\}\_\{ 1\}) =
\{D\}\_\{1\}',r(\{D\}\_\{2\}) = \{D\}\_\{2\}'

On s'intéressera par la suite uniquement au cas où X est l'ensemble D(E)
des droites de E ou l'ensemble \textbackslash{}tilde\{D\}(ℰ) des
demi-droites de E (angles de droites ou de demi droites).

Comparaison des angles orientés et non orientés

Théorème~12.6.18 Si \textbackslash{}mathop\{dim\} E ≥ 3 les notions
d'angles orientés ou non orientés coïncident aussi bien pour les droites
que pour les demi-droites, c'est-à-dire que

\textbackslash{}mathop\{∃\}r ∈ O(E)\textbackslash{}quad r(\{D\}\_\{1\})
= \{D\}\_\{1\}'\textbackslash{}text\{ et \}r(\{D\}\_\{2\}) =
\{D\}\_\{2\}'

si et seulement si

\textbackslash{}mathop\{∃\}r ∈ \{O\}\^{}\{+\}(E)\textbackslash{}quad
r(\{D\}\_\{ 1\}) = \{D\}\_\{1\}'\textbackslash{}text\{ et
\}r(\{D\}\_\{2\}) = \{D\}\_\{2\}'.

Démonstration L'implication '' ⇐'' est claire, et si la propriété de
gauche est vérifiée, soit r appartient à \{O\}\^{}\{+\}(E) et c'est
terminé, soit r appartient à \{O\}\^{}\{−\}(E), mais alors il suffit de
composer r par une symétrie s par rapport à un hyperplan contenant
\{D\}\_\{1\}' et \{D\}\_\{2\}' pour trouver un r' = s ∘ r ∈
\{O\}\^{}\{+\}(E) tel que r'(\{D\}\_\{1\}) = \{D\}\_\{1\}' et
r'(\{D\}\_\{2\}) = \{D\}\_\{2\}', car s laisse invariantes \{D\}\_\{1\}'
et \{D\}\_\{2\}'.

Mesure des angles non orientés de droites ou de demi-droites

Pour (\{D\}\_\{1\},\{D\}\_\{2\}) ∈D\{(E)\}\^{}\{2\}, on définit
φ(\{D\}\_\{1\},\{D\}\_\{2\}) de la manière suivante~: soit \{x\}\_\{1\}
un vecteur directeur de \{D\}\_\{1\} et \{x\}\_\{2\} un vecteur
directeur de \{D\}\_\{2\}, le réel \{
\textbar{}(\{x\}\_\{1\}\textbackslash{}mathrel\{∣\}\{x\}\_\{2\})\textbar{}
\textbackslash{}over
\textbackslash{}\textbar{}\{x\}\_\{1\}\textbackslash{}\textbar{}
\textbackslash{}\textbar{}\{x\}\_\{2\}\textbackslash{}\textbar{}\} ∈
{[}0,1{]} est indépendant du choix des vecteurs directeurs de
\{D\}\_\{1\} et \{D\}\_\{2\} (changer \{x\}\_\{1\} en λ\{x\}\_\{1\} et
\{x\}\_\{2\} en μ\{x\}\_\{2\} avec λ\textbackslash{}mathrel\{≠\}0 et
μ\textbackslash{}mathrel\{≠\}0 ne change pas sa valeur), on le définit
comme φ(\{D\}\_\{1\},\{D\}\_\{2\}).

Théorème~12.6.19 \textbackslash{}mathop\{∀\}(\{D\}\_\{1\},\{D\}\_\{2\})
∈D\{(E)\}\^{}\{2\}\textbackslash{}quad
\textbackslash{}overline\{(\{D\}\_\{1\},\{D\}\_\{2\})\} =
\textbackslash{}overline\{(\{D\}\_\{1\}',\{D\}\_\{2\}')\}\textbackslash{}quad
\textbackslash{}mathrel\{⇔\} \textbackslash{}quad
φ(\{D\}\_\{1\},\{D\}\_\{2\}) = φ(\{D\}\_\{1\}',\{D\}\_\{2\}').

Démonstration ( ⇒). Il suffit de remarquer que si r ∈ O(E) alors

\{
\textbar{}(r(\{x\}\_\{1\})\textbackslash{}mathrel\{∣\}r(\{x\}\_\{2\}))\textbar{}
\textbackslash{}over
\textbackslash{}\textbar{}r(\{x\}\_\{1\})\textbackslash{}\textbar{}
\textbackslash{}\textbar{}r(\{x\}\_\{2\})\textbackslash{}\textbar{}\}
=\{
\textbar{}(\{x\}\_\{1\}\textbackslash{}mathrel\{∣\}\{x\}\_\{2\})\textbar{}
\textbackslash{}over
\textbackslash{}\textbar{}\{x\}\_\{1\}\textbackslash{}\textbar{}
\textbackslash{}\textbar{}\{x\}\_\{2\}\textbackslash{}\textbar{}\} .

( ⇐). Supposons que φ(\{D\}\_\{1\},\{D\}\_\{2\}) =
φ(\{D\}\_\{1\}',\{D\}\_\{2\}')\textbackslash{}mathrel\{≠\}1. Soit
d'abord r ∈ O(E) qui envoie \{D\}\_\{1\} sur \{D\}\_\{1\}' et le plan
Vect(\{D\}\_\{1\},\{D\}\_\{2\}) sur le plan
Vect(\{D\}\_\{1\}',\{D\}\_\{2\}') (la construire en prenant des bonnes
bases orthonormées). Alors si on pose \{D\}\_\{3\} = r(\{D\}\_\{2\}), on
a φ(\{D\}\_\{1\}',\{D\}\_\{3\}) = φ(\{D\}\_\{1\},\{D\}\_\{2\}) =
φ(\{D\}\_\{1\}',\{D\}\_\{2\}'). Dans le plan
Vect(\{D\}\_\{1\}',\{D\}\_\{2\}') (qui contient les trois droites
\{D\}\_\{1\}', \{D\}\_\{2\}' et \{D\}\_\{3\}) ceci impose que soit
\{D\}\_\{3\} = \{D\}\_\{2\}' (et dans ce cas on a trouvé r tel que
r(\{D\}\_\{1\}) = \{D\}\_\{1\}' et r(\{D\}\_\{2\}) = \{D\}\_\{2\}'),
soit \{D\}\_\{3\} et \{D\}\_\{2\}' sont symétriques par rapport à
\{D\}\_\{1\}, auquel cas en composant r par la symétrie orthogonale par
rapport à l'hyperplan \{D\}\_\{1\}' ⊕ V
ect\{(\{D\}\_\{1\}',\{D\}\_\{2\}')\}\^{}\{⊥\} on trouve un r' tel que
r'(\{D\}\_\{1\}) = \{D\}\_\{1\}' et r'(\{D\}\_\{2\}) = \{D\}\_\{2\}'.

Si φ(\{D\}\_\{1\},\{D\}\_\{2\}) = φ(\{D\}\_\{1\}',\{D\}\_\{2\}') = 1, on
a \{D\}\_\{1\} = \{D\}\_\{2\}, \{D\}\_\{1\}' = \{D\}\_\{2\}' et il
suffit de choisir un r tel que r(\{D\}\_\{1\}) = \{D\}\_\{1\}'.

Définition~12.6.5 On appelle mesure de l'angle non orienté des droites
\{D\}\_\{1\} et \{D\}\_\{2\} l'unique réel θ ∈ {[}0,π∕2{]} tel que
\textbackslash{}mathop\{cos\} θ = φ(\{D\}\_\{1\},\{D\}\_\{2\}).

Le théorème précédent montre que deux angles non orientés de droites
sont égaux si et seulement si leurs mesures sont égales.

Pour les demi-droites on suit un plan analogue en posant cette fois
φ(\{D\}\_\{1\},\{D\}\_\{2\}) =\{
(\{x\}\_\{1\}\textbackslash{}mathrel\{∣\}\{x\}\_\{2\})
\textbackslash{}over
\textbackslash{}\textbar{}\{x\}\_\{1\}\textbackslash{}\textbar{}
\textbackslash{}\textbar{}\{x\}\_\{2\}\textbackslash{}\textbar{}\} ∈
{[}−1,1{]} qui ne dépend pas du choix des vecteurs directeurs des
demi-droites (car cette fois λ et μ sont nécessairement positifs). On a
le même théorème (avec une démonstration analogue) et on peut donc poser

Définition~12.6.6 On appelle mesure de l'angle non orienté des
demi-droites \{D\}\_\{1\} et \{D\}\_\{2\} l'unique réel θ ∈ {[}0,π{]}
tel que \textbackslash{}mathop\{cos\} θ = φ(\{D\}\_\{1\},\{D\}\_\{2\}).

Le théorème montre que deux angles non orientés de demi-droites sont
égaux si et seulement si leurs mesures sont égales.

Angles orientés de demi-droites dans le plan euclidien

On notera \textbackslash{}tilde\{A\}(ℰ) l'ensemble des angles orientés
de demi-droites du plan euclidien E.

Théorème~12.6.20 Soit D ∈\textbackslash{}tilde\{D\}(ℰ). Alors
l'application f : \{O\}\^{}\{+\}(E) →\textbackslash{}tilde\{A\}(ℰ),
r\textbackslash{}mathrel\{↦\}\textbackslash{}widehat\{(D,r(D))\} est une
bijection qui ne dépend pas du choix de D.

Démonstration Pour l'injectivité, si on a f(r) = f(r') c'est qu'il
existe r'' ∈ \{O\}\^{}\{+\}(E) tel que r''(D) = D et r'`∘ r(D) = r'(D).
Mais la première relation impose que r'' = \textbackslash{}mathrm\{Id\}
(une rotation du plan euclidien qui laisse invariante une demi-droite
est l'identité) et la deuxième que \{r\}\^{}\{−1\} ∘ r'(D) = D soit r' =
r pour la même raison. En ce qui concerne l'indépendance de D il suffit
de remarquer que si D' ∈\textbackslash{}tilde\{D\}(ℰ), il existe
\{r\}\_\{0\} ∈ \{O\}\^{}\{+\}(E) tel que D' = \{r\}\_\{0\}(D) et alors

\textbackslash{}widehat\{(D',r(D'))\} =\textbackslash{}widehat\{
(\{r\}\_\{0\}(D),r ∘ \{r\}\_\{0\}(D))\} =\textbackslash{}widehat\{
(\{r\}\_\{0\}(D),\{r\}\_\{0\} ∘ r(D))\} =\textbackslash{}widehat\{
(D,r(D))\}

(on voit ici le rôle essentiel de la commutativité de \{O\}\^{}\{+\}(E)
en dimension 2). La surjectivité provient alors du fait que tout angle
\textbackslash{}widehat\{(\{D\}\_\{1\},\{D\}\_\{2\})\} est de la forme
\textbackslash{}widehat\{(\{D\}\_\{1\},r(\{D\}\_\{1\}))\} = f(r) en
utilisant une rotation r qui envoie \{D\}\_\{1\} sur \{D\}\_\{2\}.

Cette bijection naturelle permet de transporter la structure de groupe
commutatif de \{O\}\^{}\{+\}(E) à \textbackslash{}tilde\{A\}(ℰ) en
posant~:

si \{θ\}\_\{1\} = f(\{r\}\_\{1\}) et \{θ\}\_\{2\} = f(\{r\}\_\{2\}) ,
alors on pose \{θ\}\_\{1\} + \{θ\}\_\{2\} = f(\{r\}\_\{1\} ∘
\{r\}\_\{2\}).

On définit de plus l'angle nul comme étant
f(\textbackslash{}mathrm\{Id\}) =\textbackslash{}widehat\{ (D,D)\},
l'angle plat comme étant f(−\textbackslash{}mathrm\{Id\})
=\textbackslash{}widehat\{ (D,−D)\}.

Théorème~12.6.21 (relation de Chasles)
\textbackslash{}widehat\{(\{D\}\_\{1\},\{D\}\_\{2\})\}
+\textbackslash{}widehat\{ (\{D\}\_\{2\},\{D\}\_\{3\})\}
=\textbackslash{}widehat\{ (\{D\}\_\{1\},\{D\}\_\{3\})\}.

Démonstration Soit \{r\}\_\{1\} ∈ \{O\}\^{}\{+\}(E) telle que
\{r\}\_\{1\}(\{D\}\_\{1\}) = \{D\}\_\{2\} et \{r\}\_\{2\} ∈
\{O\}\^{}\{+\}(E) telle que \{r\}\_\{2\}(\{D\}\_\{2\}) = \{D\}\_\{3\}.
Alors

\textbackslash{}widehat\{(\{D\}\_\{1\},\{D\}\_\{2\})\}
+\textbackslash{}widehat\{ (\{D\}\_\{2\},\{D\}\_\{3\})\} =
f(\{r\}\_\{1\}) + f(\{r\}\_\{2\}) = f(\{r\}\_\{2\} ∘ \{r\}\_\{1\})
=\textbackslash{}widehat\{ (\{D\}\_\{1\},\{r\}\_\{2\} ∘
\{r\}\_\{1\}(\{D\}\_\{1\}))\} =\textbackslash{}widehat\{
(\{D\}\_\{1\},\{D\}\_\{3\})\}

On retrouve alors sans difficulté toutes les propriétés des angles de
demi-droites. Comme de plus, si E est orienté, on connait un
isomorphisme de groupes abéliens entre ℝ∕2πℤ et \{O\}\^{}\{+\}(E), on en
déduit un isomorphisme entre \textbackslash{}tilde\{A\}(ℰ) et ℝ∕2πℤ qui
permet de mesurer les angles modulo 2π et d'effectuer les calculs (en
particulier les divisions par 2) dans ℝ∕2πℤ. On en déduit par exemple
facilement qu'un couple de demi-droites a deux bissectrices opposés.

Angles orientés de droites dans le plan euclidien

On notera A(E) l'ensemble des angles orientés de droites du plan
euclidien E. La seule différence est que f n'est plus injective, mais
que f(\{r\}\_\{1\}) = f(\{r\}\_\{2\}) \textbackslash{}mathrel\{⇔\}
\{r\}\_\{2\} = ±\{r\}\_\{1\}. On en déduit que f induit une bijection
\textbackslash{}tilde\{f\} de \{O\}\^{}\{+\}(E)∕\textbackslash{}\{
±\textbackslash{}mathrm\{Id\}\textbackslash{}\} sur A(E). On peut donc
encore transporter la structure de groupe de
\{O\}\^{}\{+\}(E)∕\textbackslash{}\{
±\textbackslash{}mathrm\{Id\}\textbackslash{}\} sur A(E) et on obtient
encore une relation de Chasles. Si E est orienté, on obtient un
isomorphisme de A(E) avec ℝ∕πℤ qui permet de mesurer les angles de
droites modulo π.

Exemple~12.6.2 L'application θ\textbackslash{}mathrel\{↦\}2θ est un
isomorphisme de groupes de ℝ∕πℤ sur ℝ∕2πℤ. On en déduit un isomorphisme
encore noté θ\textbackslash{}mathrel\{↦\}2θ de A(E) sur
\textbackslash{}tilde\{A\}(ℰ). Soit \{D\}\_\{1\} et \{D\}\_\{2\} deux
droites de E, \{s\}\_\{1\} et \{s\}\_\{2\} les symétries orthogonales
par rapport à ces droites. Montrer que \{s\}\_\{2\} ∘ \{s\}\_\{1\} est
la rotation d'angle (de demi-droites)
2\textbackslash{}widehat\{(\{D\}\_\{1\},\{D\}\_\{2\})\}.

{[}\href{coursse71.html}{prev}{]}
{[}\href{coursse71.html\#tailcoursse71.html}{prev-tail}{]}
{[}\href{coursse72.html}{front}{]}
{[}\href{coursch13.html\#coursse72.html}{up}{]}

\end{document}
