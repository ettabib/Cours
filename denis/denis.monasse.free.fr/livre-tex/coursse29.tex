\documentclass[]{article}
\usepackage[T1]{fontenc}
\usepackage{lmodern}
\usepackage{amssymb,amsmath}
\usepackage{ifxetex,ifluatex}
\usepackage{fixltx2e} % provides \textsubscript
% use upquote if available, for straight quotes in verbatim environments
\IfFileExists{upquote.sty}{\usepackage{upquote}}{}
\ifnum 0\ifxetex 1\fi\ifluatex 1\fi=0 % if pdftex
  \usepackage[utf8]{inputenc}
\else % if luatex or xelatex
  \ifxetex
    \usepackage{mathspec}
    \usepackage{xltxtra,xunicode}
  \else
    \usepackage{fontspec}
  \fi
  \defaultfontfeatures{Mapping=tex-text,Scale=MatchLowercase}
  \newcommand{\euro}{€}
\fi
% use microtype if available
\IfFileExists{microtype.sty}{\usepackage{microtype}}{}
\ifxetex
  \usepackage[setpagesize=false, % page size defined by xetex
              unicode=false, % unicode breaks when used with xetex
              xetex]{hyperref}
\else
  \usepackage[unicode=true]{hyperref}
\fi
\hypersetup{breaklinks=true,
            bookmarks=true,
            pdfauthor={},
            pdftitle={Espaces vectoriels normes de dimensions finies},
            colorlinks=true,
            citecolor=blue,
            urlcolor=blue,
            linkcolor=magenta,
            pdfborder={0 0 0}}
\urlstyle{same}  % don't use monospace font for urls
\setlength{\parindent}{0pt}
\setlength{\parskip}{6pt plus 2pt minus 1pt}
\setlength{\emergencystretch}{3em}  % prevent overfull lines
\setcounter{secnumdepth}{0}
 
/* start css.sty */
.cmr-5{font-size:50%;}
.cmr-7{font-size:70%;}
.cmmi-5{font-size:50%;font-style: italic;}
.cmmi-7{font-size:70%;font-style: italic;}
.cmmi-10{font-style: italic;}
.cmsy-5{font-size:50%;}
.cmsy-7{font-size:70%;}
.cmex-7{font-size:70%;}
.cmex-7x-x-71{font-size:49%;}
.msbm-7{font-size:70%;}
.cmtt-10{font-family: monospace;}
.cmti-10{ font-style: italic;}
.cmbx-10{ font-weight: bold;}
.cmr-17x-x-120{font-size:204%;}
.cmsl-10{font-style: oblique;}
.cmti-7x-x-71{font-size:49%; font-style: italic;}
.cmbxti-10{ font-weight: bold; font-style: italic;}
p.noindent { text-indent: 0em }
td p.noindent { text-indent: 0em; margin-top:0em; }
p.nopar { text-indent: 0em; }
p.indent{ text-indent: 1.5em }
@media print {div.crosslinks {visibility:hidden;}}
a img { border-top: 0; border-left: 0; border-right: 0; }
center { margin-top:1em; margin-bottom:1em; }
td center { margin-top:0em; margin-bottom:0em; }
.Canvas { position:relative; }
li p.indent { text-indent: 0em }
.enumerate1 {list-style-type:decimal;}
.enumerate2 {list-style-type:lower-alpha;}
.enumerate3 {list-style-type:lower-roman;}
.enumerate4 {list-style-type:upper-alpha;}
div.newtheorem { margin-bottom: 2em; margin-top: 2em;}
.obeylines-h,.obeylines-v {white-space: nowrap; }
div.obeylines-v p { margin-top:0; margin-bottom:0; }
.overline{ text-decoration:overline; }
.overline img{ border-top: 1px solid black; }
td.displaylines {text-align:center; white-space:nowrap;}
.centerline {text-align:center;}
.rightline {text-align:right;}
div.verbatim {font-family: monospace; white-space: nowrap; text-align:left; clear:both; }
.fbox {padding-left:3.0pt; padding-right:3.0pt; text-indent:0pt; border:solid black 0.4pt; }
div.fbox {display:table}
div.center div.fbox {text-align:center; clear:both; padding-left:3.0pt; padding-right:3.0pt; text-indent:0pt; border:solid black 0.4pt; }
div.minipage{width:100%;}
div.center, div.center div.center {text-align: center; margin-left:1em; margin-right:1em;}
div.center div {text-align: left;}
div.flushright, div.flushright div.flushright {text-align: right;}
div.flushright div {text-align: left;}
div.flushleft {text-align: left;}
.underline{ text-decoration:underline; }
.underline img{ border-bottom: 1px solid black; margin-bottom:1pt; }
.framebox-c, .framebox-l, .framebox-r { padding-left:3.0pt; padding-right:3.0pt; text-indent:0pt; border:solid black 0.4pt; }
.framebox-c {text-align:center;}
.framebox-l {text-align:left;}
.framebox-r {text-align:right;}
span.thank-mark{ vertical-align: super }
span.footnote-mark sup.textsuperscript, span.footnote-mark a sup.textsuperscript{ font-size:80%; }
div.tabular, div.center div.tabular {text-align: center; margin-top:0.5em; margin-bottom:0.5em; }
table.tabular td p{margin-top:0em;}
table.tabular {margin-left: auto; margin-right: auto;}
div.td00{ margin-left:0pt; margin-right:0pt; }
div.td01{ margin-left:0pt; margin-right:5pt; }
div.td10{ margin-left:5pt; margin-right:0pt; }
div.td11{ margin-left:5pt; margin-right:5pt; }
table[rules] {border-left:solid black 0.4pt; border-right:solid black 0.4pt; }
td.td00{ padding-left:0pt; padding-right:0pt; }
td.td01{ padding-left:0pt; padding-right:5pt; }
td.td10{ padding-left:5pt; padding-right:0pt; }
td.td11{ padding-left:5pt; padding-right:5pt; }
table[rules] {border-left:solid black 0.4pt; border-right:solid black 0.4pt; }
.hline hr, .cline hr{ height : 1px; margin:0px; }
.tabbing-right {text-align:right;}
span.TEX {letter-spacing: -0.125em; }
span.TEX span.E{ position:relative;top:0.5ex;left:-0.0417em;}
a span.TEX span.E {text-decoration: none; }
span.LATEX span.A{ position:relative; top:-0.5ex; left:-0.4em; font-size:85%;}
span.LATEX span.TEX{ position:relative; left: -0.4em; }
div.float img, div.float .caption {text-align:center;}
div.figure img, div.figure .caption {text-align:center;}
.marginpar {width:20%; float:right; text-align:left; margin-left:auto; margin-top:0.5em; font-size:85%; text-decoration:underline;}
.marginpar p{margin-top:0.4em; margin-bottom:0.4em;}
.equation td{text-align:center; vertical-align:middle; }
td.eq-no{ width:5%; }
table.equation { width:100%; } 
div.math-display, div.par-math-display{text-align:center;}
math .texttt { font-family: monospace; }
math .textit { font-style: italic; }
math .textsl { font-style: oblique; }
math .textsf { font-family: sans-serif; }
math .textbf { font-weight: bold; }
.partToc a, .partToc, .likepartToc a, .likepartToc {line-height: 200%; font-weight:bold; font-size:110%;}
.chapterToc a, .chapterToc, .likechapterToc a, .likechapterToc, .appendixToc a, .appendixToc {line-height: 200%; font-weight:bold;}
.index-item, .index-subitem, .index-subsubitem {display:block}
.caption td.id{font-weight: bold; white-space: nowrap; }
table.caption {text-align:center;}
h1.partHead{text-align: center}
p.bibitem { text-indent: -2em; margin-left: 2em; margin-top:0.6em; margin-bottom:0.6em; }
p.bibitem-p { text-indent: 0em; margin-left: 2em; margin-top:0.6em; margin-bottom:0.6em; }
.paragraphHead, .likeparagraphHead { margin-top:2em; font-weight: bold;}
.subparagraphHead, .likesubparagraphHead { font-weight: bold;}
.quote {margin-bottom:0.25em; margin-top:0.25em; margin-left:1em; margin-right:1em; text-align:justify;}
.verse{white-space:nowrap; margin-left:2em}
div.maketitle {text-align:center;}
h2.titleHead{text-align:center;}
div.maketitle{ margin-bottom: 2em; }
div.author, div.date {text-align:center;}
div.thanks{text-align:left; margin-left:10%; font-size:85%; font-style:italic; }
div.author{white-space: nowrap;}
.quotation {margin-bottom:0.25em; margin-top:0.25em; margin-left:1em; }
h1.partHead{text-align: center}
.sectionToc, .likesectionToc {margin-left:2em;}
.subsectionToc, .likesubsectionToc {margin-left:4em;}
.subsubsectionToc, .likesubsubsectionToc {margin-left:6em;}
.frenchb-nbsp{font-size:75%;}
.frenchb-thinspace{font-size:75%;}
.figure img.graphics {margin-left:10%;}
/* end css.sty */

\title{Espaces vectoriels normes de dimensions finies}
\author{}
\date{}

\begin{document}
\maketitle

\textbf{Warning: \href{http://www.math.union.edu/locate/jsMath}{jsMath}
requires JavaScript to process the mathematics on this page.\\ If your
browser supports JavaScript, be sure it is enabled.}

\begin{center}\rule{3in}{0.4pt}\end{center}

{[}\href{coursse30.html}{next}{]} {[}\href{coursse28.html}{prev}{]}
{[}\href{coursse28.html\#tailcoursse28.html}{prev-tail}{]}
{[}\hyperref[tailcoursse29.html]{tail}{]}
{[}\href{coursch6.html\#coursse29.html}{up}{]}

\subsubsection{5.3 Espaces vectoriels normés de dimensions finies}

\paragraph{5.3.1 Equivalence des normes}

Lemme~5.3.1 Toutes les normes sur \{ℝ\}\^{}\{n\} sont équivalentes.

Démonstration Posons
\textbackslash{}\textbar{}x\textbackslash{}\textbar{}
=\textbackslash{}mathop\{ max\}\textbar{}\{x\}\_\{i\}\textbar{} et
montrons que toute autre norme N est équivalente à cette norme. Soit
(\{e\}\_\{1\},\textbackslash{}mathop\{\textbackslash{}mathop\{\ldots{}\}\}\{e\}\_\{n\})
la base canonique de \{ℝ\}\^{}\{n\} et x ∈ \{ℝ\}\^{}\{n\}. On a N(x) =
N(\textbackslash{}mathop\{\textbackslash{}mathop\{∑ \}\}
\{x\}\_\{i\}\{e\}\_\{i\})
≤\textbackslash{}mathop\{\textbackslash{}mathop\{∑ \}\}
\textbar{}\{x\}\_\{i\}\textbar{}N(\{e\}\_\{i\})
≤\textbackslash{}mathop\{
max\}\textbar{}\{x\}\_\{i\}\textbar{}\{\textbackslash{}mathop\{\textbackslash{}mathop\{∑
\}\} \}\_\{i\}N(\{e\}\_\{i\}) =
β\textbackslash{}\textbar{}x\textbackslash{}\textbar{}. On en déduit que
\textbar{}N(x) − N(y)\textbar{}≤ N(x − y) ≤ β\textbackslash{}\textbar{}x
− y\textbackslash{}\textbar{} ce qui démontre que l'application N :
(\{ℝ\}\^{}\{n\},\textbackslash{}\textbar{}.\textbackslash{}\textbar{}) →
ℝ est continue. Soit S = \textbackslash{}\{x ∈
\{ℝ\}\^{}\{n\}\textbackslash{}mathrel\{∣\}\textbackslash{}\textbar{}x\textbackslash{}\textbar{}
= 1\textbackslash{}\}~; S est une partie compacte de
(\{ℝ\}\^{}\{n\},\textbackslash{}\textbar{}.\textbackslash{}\textbar{})
(fermée bornée), donc l'application N y atteint sa borne inférieure.
Soit α =\{\textbackslash{}mathop\{ inf\} \}\_\{x∈S\}N(x) =
N(\{x\}\_\{0\}). On a \{x\}\_\{0\}\textbackslash{}mathrel\{≠\}0 (car
\{x\}\_\{0\} ∈ S) donc α \textgreater{} 0. Alors, si x ∈ \{ℝ\}\^{}\{n\},
x\textbackslash{}mathrel\{≠\}0, on a \{ x \textbackslash{}over
\textbackslash{}\textbar{}x\textbackslash{}\textbar{}\} ∈ S soit N(\{ x
\textbackslash{}over
\textbackslash{}\textbar{}x\textbackslash{}\textbar{}\} ) ≥ α soit
encore N(x) ≥ α\textbackslash{}\textbar{}x\textbackslash{}\textbar{}. On
a donc trouvé α et β strictement positifs tels que
\textbackslash{}mathop\{∀\}x ∈ \{ℝ\}\^{}\{n\},
α\textbackslash{}\textbar{}x\textbackslash{}\textbar{} ≤ N(x) ≤
β\textbackslash{}\textbar{}x\textbackslash{}\textbar{}, ce qu'il fallait
démontrer.

Théorème~5.3.2 Sur un K-espace vectoriel normé~de dimension finie toutes
les normes sont équivalentes.

Démonstration Tout ℂ-espace vectoriel normé~étant aussi un ℝ-espace
vectoriel normé, il suffit de le montrer lorsque le corps de base est ℝ.
Soit \{N\}\_\{1\} et \{N\}\_\{2\} deux normes sur E~; soit ℰ =
(\{e\}\_\{1\},\textbackslash{}mathop\{\textbackslash{}mathop\{\ldots{}\}\},\{e\}\_\{n\})
une base de E et u : \{ℝ\}\^{}\{n\} → E définie par
u(\{x\}\_\{1\},\textbackslash{}mathop\{\textbackslash{}mathop\{\ldots{}\}\},\{x\}\_\{n\})
=\textbackslash{}mathop\{ \textbackslash{}mathop\{∑ \}\}
\{x\}\_\{i\}\{e\}\_\{i\} (u est un isomorphisme d'espaces vectoriels).
Alors \{N\}\_\{1\} ∘ u et \{N\}\_\{2\} ∘ u sont deux normes sur
\{ℝ\}\^{}\{n\} (facile), elles sont donc équivalentes, et donc il existe
α et β strictement positifs tels que \textbackslash{}mathop\{∀\}x ∈
\{ℝ\}\^{}\{n\}, α\{N\}\_\{1\}(u(x)) ≤ \{N\}\_\{2\}(u(x)) ≤
β\{N\}\_\{1\}(u(x)). Mais tout élément de E s'écrivant sous la forme
u(x), on a, \textbackslash{}mathop\{∀\}y ∈ E, α\{N\}\_\{1\}(y) ≤
\{N\}\_\{2\}(y) ≤ β\{N\}\_\{1\}(y), ce qu'il fallait démontrer.

\paragraph{5.3.2 Propriétés topologiques et métriques des espaces
vectoriels normés de dimension finie}

Remarque~5.3.1 Tout ℂ-espace vectoriel normé~étant aussi un ℝ-espace
vectoriel normé, il suffit de considérer le cas où le corps de base est
ℝ. Soit (E,\textbackslash{}\textbar{}.\textbackslash{}\textbar{}) un
espace vectoriel normé~de dimension finie, soit ℰ =
(\{e\}\_\{1\},\textbackslash{}mathop\{\textbackslash{}mathop\{\ldots{}\}\},\{e\}\_\{n\})
une base de E et u : \{ℝ\}\^{}\{n\} → E définie par
u(\{x\}\_\{1\},\textbackslash{}mathop\{\textbackslash{}mathop\{\ldots{}\}\},\{x\}\_\{n\})
=\textbackslash{}mathop\{ \textbackslash{}mathop\{∑ \}\}
\{x\}\_\{i\}\{e\}\_\{i\} (u est un isomorphisme d'espaces vectoriels).
Alors N :
x\textbackslash{}mathrel\{↦\}\textbackslash{}\textbar{}u(x)\textbackslash{}\textbar{}
est une norme sur \{ℝ\}\^{}\{n\} qui est équivalente à la norme
\textbackslash{}\textbar{}\{.\textbackslash{}\textbar{}\}\_\{∞\}~; de
plus l'application u : (\{ℝ\}\^{}\{n\},N) →
(E,\textbackslash{}\textbar{}.\textbackslash{}\textbar{}) est une
isométrie~; on en déduit que
(E,\textbackslash{}\textbar{}.\textbackslash{}\textbar{}) a, en tant
qu'espace vectoriel normé, les mêmes propriétés que
(\{ℝ\}\^{}\{n\},\textbackslash{}\textbar{}\{.\textbackslash{}\textbar{}\}\_\{∞\})
c'est-à-dire

Théorème~5.3.3 Tout espace vectoriel normé de dimension finie est
complet~; les parties compactes en sont les fermés bornés.

Corollaire~5.3.4 Tout sous-espace vectoriel de dimension finie d'un
espace vectoriel normé~est fermé.

Démonstration Muni de la restriction de la norme, il est complet, donc
fermé.

\paragraph{5.3.3 Continuité des applications linéaires}

Théorème~5.3.5 Soit E et F deux espaces vectoriels normés, E étant
supposé de dimension finie. Alors toute application linéaire de E dans F
est continue.

Démonstration Soit ℰ =
(\{e\}\_\{1\},\textbackslash{}mathop\{\textbackslash{}mathop\{\ldots{}\}\},\{e\}\_\{n\})
une base de E~; comme toute les normes sur E sont équivalentes, on peut
prendre la norme définie par
\textbackslash{}\textbar{}x\textbackslash{}\textbar{}
=\textbackslash{}mathop\{ sup\}\textbar{}\{x\}\_\{i\}\textbar{} si x
=\textbackslash{}mathop\{ \textbackslash{}mathop\{∑ \}\}
\{x\}\_\{i\}\{e\}\_\{i\}. On a alors

\textbackslash{}begin\{eqnarray*\}
\textbackslash{}\textbar{}u(x)\textbackslash{}\textbar{}\& =\&
\textbackslash{}\textbar{}\textbackslash{}mathop\{∑
\}\{x\}\_\{i\}u(\{e\}\_\{i\})\textbackslash{}\textbar{}
≤\textbackslash{}mathop\{∑
\}\textbar{}\{x\}\_\{i\}\textbar{}\textbackslash{},\textbackslash{}\textbar{}u(\{e\}\_\{i\})\textbackslash{}\textbar{}\%\&
\textbackslash{}\textbackslash{} \& ≤\&
\textbackslash{}\textbar{}x\textbackslash{}\textbar{}\textbackslash{}mathop\{∑
\}\textbackslash{}\textbar{}u(\{e\}\_\{i\})\textbackslash{}\textbar{} =
K\textbackslash{}\textbar{}x\textbackslash{}\textbar{} \%\&
\textbackslash{}\textbackslash{} \textbackslash{}end\{eqnarray*\}

Ceci montre la continuité de l'application linéaire u.

Remarque~5.3.2 Ce résultat s'étend sans difficulté aux applications
bilinéaires de \{E\}\_\{1\} × \{E\}\_\{2\} dans F à condition que
\{E\}\_\{1\} et \{E\}\_\{2\} soient de dimensions finies~; de même pour
des applications p-linéaires.

{[}\href{coursse30.html}{next}{]} {[}\href{coursse28.html}{prev}{]}
{[}\href{coursse28.html\#tailcoursse28.html}{prev-tail}{]}
{[}\href{coursse29.html}{front}{]}
{[}\href{coursch6.html\#coursse29.html}{up}{]}

\end{document}
