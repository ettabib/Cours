\documentclass[]{article}
\usepackage[T1]{fontenc}
\usepackage{lmodern}
\usepackage{amssymb,amsmath}
\usepackage{ifxetex,ifluatex}
\usepackage{fixltx2e} % provides \textsubscript
% use upquote if available, for straight quotes in verbatim environments
\IfFileExists{upquote.sty}{\usepackage{upquote}}{}
\ifnum 0\ifxetex 1\fi\ifluatex 1\fi=0 % if pdftex
  \usepackage[utf8]{inputenc}
\else % if luatex or xelatex
  \ifxetex
    \usepackage{mathspec}
    \usepackage{xltxtra,xunicode}
  \else
    \usepackage{fontspec}
  \fi
  \defaultfontfeatures{Mapping=tex-text,Scale=MatchLowercase}
  \newcommand{\euro}{€}
\fi
% use microtype if available
\IfFileExists{microtype.sty}{\usepackage{microtype}}{}
\ifxetex
  \usepackage[setpagesize=false, % page size defined by xetex
              unicode=false, % unicode breaks when used with xetex
              xetex]{hyperref}
\else
  \usepackage[unicode=true]{hyperref}
\fi
\hypersetup{breaklinks=true,
            bookmarks=true,
            pdfauthor={},
            pdftitle={Fonctions reelles d'une variable reelle},
            colorlinks=true,
            citecolor=blue,
            urlcolor=blue,
            linkcolor=magenta,
            pdfborder={0 0 0}}
\urlstyle{same}  % don't use monospace font for urls
\setlength{\parindent}{0pt}
\setlength{\parskip}{6pt plus 2pt minus 1pt}
\setlength{\emergencystretch}{3em}  % prevent overfull lines
\setcounter{secnumdepth}{0}
 
/* start css.sty */
.cmr-5{font-size:50%;}
.cmr-7{font-size:70%;}
.cmmi-5{font-size:50%;font-style: italic;}
.cmmi-7{font-size:70%;font-style: italic;}
.cmmi-10{font-style: italic;}
.cmsy-5{font-size:50%;}
.cmsy-7{font-size:70%;}
.cmex-7{font-size:70%;}
.cmex-7x-x-71{font-size:49%;}
.msbm-7{font-size:70%;}
.cmtt-10{font-family: monospace;}
.cmti-10{ font-style: italic;}
.cmbx-10{ font-weight: bold;}
.cmr-17x-x-120{font-size:204%;}
.cmsl-10{font-style: oblique;}
.cmti-7x-x-71{font-size:49%; font-style: italic;}
.cmbxti-10{ font-weight: bold; font-style: italic;}
p.noindent { text-indent: 0em }
td p.noindent { text-indent: 0em; margin-top:0em; }
p.nopar { text-indent: 0em; }
p.indent{ text-indent: 1.5em }
@media print {div.crosslinks {visibility:hidden;}}
a img { border-top: 0; border-left: 0; border-right: 0; }
center { margin-top:1em; margin-bottom:1em; }
td center { margin-top:0em; margin-bottom:0em; }
.Canvas { position:relative; }
li p.indent { text-indent: 0em }
.enumerate1 {list-style-type:decimal;}
.enumerate2 {list-style-type:lower-alpha;}
.enumerate3 {list-style-type:lower-roman;}
.enumerate4 {list-style-type:upper-alpha;}
div.newtheorem { margin-bottom: 2em; margin-top: 2em;}
.obeylines-h,.obeylines-v {white-space: nowrap; }
div.obeylines-v p { margin-top:0; margin-bottom:0; }
.overline{ text-decoration:overline; }
.overline img{ border-top: 1px solid black; }
td.displaylines {text-align:center; white-space:nowrap;}
.centerline {text-align:center;}
.rightline {text-align:right;}
div.verbatim {font-family: monospace; white-space: nowrap; text-align:left; clear:both; }
.fbox {padding-left:3.0pt; padding-right:3.0pt; text-indent:0pt; border:solid black 0.4pt; }
div.fbox {display:table}
div.center div.fbox {text-align:center; clear:both; padding-left:3.0pt; padding-right:3.0pt; text-indent:0pt; border:solid black 0.4pt; }
div.minipage{width:100%;}
div.center, div.center div.center {text-align: center; margin-left:1em; margin-right:1em;}
div.center div {text-align: left;}
div.flushright, div.flushright div.flushright {text-align: right;}
div.flushright div {text-align: left;}
div.flushleft {text-align: left;}
.underline{ text-decoration:underline; }
.underline img{ border-bottom: 1px solid black; margin-bottom:1pt; }
.framebox-c, .framebox-l, .framebox-r { padding-left:3.0pt; padding-right:3.0pt; text-indent:0pt; border:solid black 0.4pt; }
.framebox-c {text-align:center;}
.framebox-l {text-align:left;}
.framebox-r {text-align:right;}
span.thank-mark{ vertical-align: super }
span.footnote-mark sup.textsuperscript, span.footnote-mark a sup.textsuperscript{ font-size:80%; }
div.tabular, div.center div.tabular {text-align: center; margin-top:0.5em; margin-bottom:0.5em; }
table.tabular td p{margin-top:0em;}
table.tabular {margin-left: auto; margin-right: auto;}
div.td00{ margin-left:0pt; margin-right:0pt; }
div.td01{ margin-left:0pt; margin-right:5pt; }
div.td10{ margin-left:5pt; margin-right:0pt; }
div.td11{ margin-left:5pt; margin-right:5pt; }
table[rules] {border-left:solid black 0.4pt; border-right:solid black 0.4pt; }
td.td00{ padding-left:0pt; padding-right:0pt; }
td.td01{ padding-left:0pt; padding-right:5pt; }
td.td10{ padding-left:5pt; padding-right:0pt; }
td.td11{ padding-left:5pt; padding-right:5pt; }
table[rules] {border-left:solid black 0.4pt; border-right:solid black 0.4pt; }
.hline hr, .cline hr{ height : 1px; margin:0px; }
.tabbing-right {text-align:right;}
span.TEX {letter-spacing: -0.125em; }
span.TEX span.E{ position:relative;top:0.5ex;left:-0.0417em;}
a span.TEX span.E {text-decoration: none; }
span.LATEX span.A{ position:relative; top:-0.5ex; left:-0.4em; font-size:85%;}
span.LATEX span.TEX{ position:relative; left: -0.4em; }
div.float img, div.float .caption {text-align:center;}
div.figure img, div.figure .caption {text-align:center;}
.marginpar {width:20%; float:right; text-align:left; margin-left:auto; margin-top:0.5em; font-size:85%; text-decoration:underline;}
.marginpar p{margin-top:0.4em; margin-bottom:0.4em;}
.equation td{text-align:center; vertical-align:middle; }
td.eq-no{ width:5%; }
table.equation { width:100%; } 
div.math-display, div.par-math-display{text-align:center;}
math .texttt { font-family: monospace; }
math .textit { font-style: italic; }
math .textsl { font-style: oblique; }
math .textsf { font-family: sans-serif; }
math .textbf { font-weight: bold; }
.partToc a, .partToc, .likepartToc a, .likepartToc {line-height: 200%; font-weight:bold; font-size:110%;}
.chapterToc a, .chapterToc, .likechapterToc a, .likechapterToc, .appendixToc a, .appendixToc {line-height: 200%; font-weight:bold;}
.index-item, .index-subitem, .index-subsubitem {display:block}
.caption td.id{font-weight: bold; white-space: nowrap; }
table.caption {text-align:center;}
h1.partHead{text-align: center}
p.bibitem { text-indent: -2em; margin-left: 2em; margin-top:0.6em; margin-bottom:0.6em; }
p.bibitem-p { text-indent: 0em; margin-left: 2em; margin-top:0.6em; margin-bottom:0.6em; }
.paragraphHead, .likeparagraphHead { margin-top:2em; font-weight: bold;}
.subparagraphHead, .likesubparagraphHead { font-weight: bold;}
.quote {margin-bottom:0.25em; margin-top:0.25em; margin-left:1em; margin-right:1em; text-align:\jmathustify;}
.verse{white-space:nowrap; margin-left:2em}
div.maketitle {text-align:center;}
h2.titleHead{text-align:center;}
div.maketitle{ margin-bottom: 2em; }
div.author, div.date {text-align:center;}
div.thanks{text-align:left; margin-left:10%; font-size:85%; font-style:italic; }
div.author{white-space: nowrap;}
.quotation {margin-bottom:0.25em; margin-top:0.25em; margin-left:1em; }
h1.partHead{text-align: center}
.sectionToc, .likesectionToc {margin-left:2em;}
.subsectionToc, .likesubsectionToc {margin-left:4em;}
.subsubsectionToc, .likesubsubsectionToc {margin-left:6em;}
.frenchb-nbsp{font-size:75%;}
.frenchb-thinspace{font-size:75%;}
.figure img.graphics {margin-left:10%;}
/* end css.sty */

\title{Fonctions reelles d'une variable reelle}
\author{}
\date{}

\begin{document}
\maketitle

\textbf{Warning: 
requires JavaScript to process the mathematics on this page.\\ If your
browser supports JavaScript, be sure it is enabled.}

\begin{center}\rule{3in}{0.4pt}\end{center}

{[}
{[}
{[}{]}
{[}

\subsubsection{8.3 Fonctions réelles d'une variable réelle}

\paragraph{8.3.1 Théorème de Rolle, formule des accroissements finis}

Lemme~8.3.1 Soit f : I \rightarrow~ \mathbb{R}~~; si f admet en c \in I^o un
extremum local et si f est dérivable au point c, alors f'(c) = 0.

Démonstration Supposons par exemple que f a en c un maximum local. Pour
c - \eta \textless{} x \textless{} c, on a  f(x)-f(c)
\over x-c ≥ 0 d'où en faisant tendre x vers c, f'(c) ≥
0. Pour c \textless{} x \textless{} c + \eta, on a  f(x)-f(c)
\over x-c \leq 0 d'où en faisant tendre x vers c, f'(c) \leq
0. On a donc f'(c) = 0.

Théorème~8.3.2 (Rolle). Soit f : {[}a,b{]} \rightarrow~ \mathbb{R}~, continue sur {[}a,b{]},
dérivable sur {]}a,b{[} telle que f(a) = f(b). Alors il existe c
\in{]}a,b{[} tel que f'(c) = 0.

Démonstration Si f est constante sur {[}a,b{]}, n'importe quel c
\in{]}a,b{[} convient. Sinon, par exemple, il existe x \in {[}a,b{]} tel que
f(x) \textgreater{} f(a) = f(b). La fonction f est continue sur le
compact {[}a,b{]} donc elle est bornée et atteint ses bornes. Soit c \in
{[}a,b{]} tel que f(c) =\
sup\f(t)∣t \in
{[}a,b{]}\. On a f(c) ≥ f(x) \textgreater{} f(a) =
f(b), donc c \in{]}a,b{[}. Mais alors, le lemme ci dessus garantit que
f'(c) = 0.

Corollaire~8.3.3 (formule des accroissements finis). Soit f : {[}a,b{]}
\rightarrow~ \mathbb{R}~, continue sur {[}a,b{]}, dérivable sur {]}a,b{[}. Alors il existe c
\in{]}a,b{[} tel que f(b) - f(a) = (b - a)f'(c).

Démonstration On applique le théorème de Rolle à g(t) = f(t) -
f(b)-f(a) \over b-a (t - a). On a g(b) = g(a) = f(a), g
est, comme f, continue sur {[}a,b{]} et dérivable sur {]}a,b{[}. Donc il
existe c \in{]}a,b{[} tel que g'(c) = 0~; mais g'(c) = f'(c) - f(b)-f(a)
\over b-a d'où le résultat.

\paragraph{8.3.2 Monotonie et dérivation}

Théorème~8.3.4 Soit I un intervalle de \mathbb{R}~, f : I \rightarrow~ \mathbb{R}~ continue sur I et
dérivable sur I^o. Alors (i) f est constante sur I si et
seulement si~\forall~t \in I^o~, f'(t) = 0
(ii) f est croissante sur I si et seulement
si~\forall~t \in I^o~, f'(t) ≥ 0 (iii) f est
décroissante sur I si et seulement si~\forall~~t \in
I^o, f'(t) \leq 0

Démonstration La définition de la dérivée f'(t)
=\
lim\_x\rightarrow~t,x\neq~t f(x)-f(t)
\over x-t montre clairement que les conditions sont
nécessaires (prendre x \textgreater{} t et faire tendre x vers t).
Inversement, si x,y \in I avec x \textless{} y, f est continue sur
{[}x,y{]} \subset~ I et dérivable sur {]}x,y{[}\subset~ I^o et donc la
formule des accroissements finis assure qu'il existe z \in{]}x,y{[}\subset~
I^o tel que f(y) - f(x) = (y - x)f'(z), ce qui montre
immédiatement que les conditions sont suffisantes.

Corollaire~8.3.5 Soit I un intervalle de \mathbb{R}~, f : I \rightarrow~ \mathbb{R}~ continue sur I et
dérivable sur I^o. Alors on a équivalence de (i) f est
strictement croissante (ii) \forall~~t \in
I^o, f'(t) ≥ 0 et \t \in
I^o∣f'(t) = 0\
est d'intérieur vide.

Démonstration (i) \rigtharrow~(ii) Si f est strictement croissante, alors
\forall~t \in I^o~, f'(t) ≥ 0~; supposons que
\t \in I^o∣f'(t) =
0\ n'est pas d'intérieur vide~; alors il contient un
segment {[}a,b{]} avec a \textless{} b~; mais alors d'après le théorème
précédent, f est constante sur {[}a,b{]} ce qui contredit la stricte
monotonie de f.

(ii) \rigtharrow~(i) On sait que si \forall~t \in I^o~,
f'(t) ≥ 0, f est croissante~; supposons qu'elle n'est pas strictement
croissante~; alors il existe a,b \in I tels que a \textless{} b et f(a) =
f(b)~; en conséquence f est constante sur {]}a,b{[}\subset~ I^o et
donc \forall~~t \in{]}a,b{[}, f'(t) = 0~; donc
l'intervalle ouvert {]}a,b{[} est contenu dans l'intérieur de
\t \in I^o∣f'(t) =
0\, c'est absurde.

\paragraph{8.3.3 Difféomorphismes}

Théorème~8.3.6 Soit I et J deux intervalles de \mathbb{R}~ et f : I \rightarrow~ J un
homéomorphisme. Soit a \in I un point où f est dérivable. Alors
f^-1 est dérivable au point f(a) si et seulement
si~f'(a)\neq~0. Dans ce cas,
(f^-1)'(f(a)) = 1 \over f'(a) .

Démonstration Posons g = f^-1. On a g \cdot f =
\mathrmId\_I. Si f est dérivable au point a et
g dérivable au point f(a), le théorème de dérivation des fonctions
composées assure que 1 = (\mathrmId\_I)'(a) =
(g \cdot f)'(a) = g'(f(a))f'(a), donc f'(a)\neq~0 et
g'(f(a)) = 1 \over f'(a) . Inversement supposons que
f'(a)\neq~0. On a alors
lim\_t\rightarrow~a,t\neq~a~
t-a \over f(t)-f(a) = 1 \over f'(a)
. Appliquons le théorème de composition des limites en posant t = g(u)
(avec a = g(f(a))), en remarquant que u\neq~f(a)
\rigtharrow~ g(u)\neq~a. On a donc, puisque g est continue
au point f(a),

lim\_u\rightarrow~f(a),u\neq~f(a)~
g(u) - g(f(a)) \over u - f(a) = 1
\over f'(a)

Donc g est dérivable au point f(a).

Définition~8.3.1 Soit I et J deux intervalles de \mathbb{R}~~; on dit que f : I \rightarrow~
J est un difféomorphisme de classe C^n (n ≥ 1) si f est
bi\jmathective et f et f^-1 sont de classe C^n.

Théorème~8.3.7 Soit n ≥ 1, f : I \rightarrow~ \mathbb{R}~. On a équivalence de (i) f est un
C^n difféomorphisme de I sur f(I) (ii) f est de classe
C^n et f' ne s'annule pas.

Démonstration (i) \rigtharrow~(ii) est clair d'après le théorème précédent.
Inversement, supposons que f est de classe C^n et que f' ne
s'annule pas. Alors f' garde un signe constant (elle est continue), et
donc f est strictement monotone. Donc f définit un homéomorphisme de I
sur J = f(I). Le théorème précédent assure que f^-1 est
dérivable sur I et que (f^-1)' = 1 \over
f'\cdotf^-1 ce qui garantit dé\jmathà la continuité de
(f^-1)'. Supposons alors que f^-1 est de classe
C^k avec k \textless{} n. Comme f' est de classe
C^k, f' \cdot f^-1 est de classe C^k~; il
en est donc de même de  1 \over f'\cdotf^-1 ,
donc de (f^-1)' et donc f^-1 est de classe
C^k+1~; par récurrence, on en déduit que f^-1 est
de classe C^n.

\paragraph{8.3.4 Formule de Taylor Lagrange}

Théorème~8.3.8 (Taylor-Lagrange). Soit f : {[}a,b{]} \rightarrow~ \mathbb{R}~ de classe
C^n sur {[}a,b{]} et n + 1 fois dérivable sur {]}a,b{[}.
Alors il existe c \in{]}a,b{[} tel que

f(b) = f(a) + \sum \_k=1^n~
f^(k)(a) \over k! (b - a)^k +
f^(n+1)(c) \over (n + 1)! (b -
a)^n+1

Démonstration Posons \phi(t) = f(b) - f(t)
-\\sum ~
\_k=1^n f^(k)(t) \over k!
(b - t)^k - \lambda~(b - t)^n+1 où \lambda~ est choisi de telle
sorte que \phi(a) = 0 (c'est évidemment possible). Il est clair que \phi est
continue sur {[}a,b{]}, dérivable sur {]}a,b{[} comme toutes les
fonctions f^(k), 0 \leq k \leq n. De plus

\begin{align*} \phi'(t)& =& -f'(t)
-\sum \_k=1^n~
f^(k+1)(t) \over k! (b - t)^k
\%& \\ & \text &
+\sum \_k=1^n~
f^(k)(t) \over (k - 1)! (b -
t)^k-1 + \lambda~(n + 1)(b - t)^n\%&
\\ & =& -f'(t)
-\sum \_l=2^n+1~
f^(l)(t) \over (l - 1)! (b -
t)^l-1 \%& \\ &
\text & +\\sum
\_k=1^n f^(k)(t) \over (k -
1)! (b - t)^k-1 + \lambda~(n + 1)(b - t)^n\%&
\\ & =& (b -
t)^n\left ((n + 1)\lambda~ - f^(n+1)(t)
\over n! \right ) \%&
\\ \end{align*}

(tous les autres termes se détruisent deux à deux). D'après le théorème
de Rolle, il existe c \in{]}a,b{[} tel que \phi'(c) = 0, soit (b -
c)^n\left ((n + 1)\lambda~ - f^(n+1)(c)
\over n! \right ) = 0. Comme b -
c\neq~0, on a \lambda~ = f^(n+1)(c)
\over (n+1)! . En écrivant que \phi(a) = 0, on obtient
alors la formule souhaitée.

Remarque~8.3.1 Pour n = 0, on trouve comme cas particulier la formule
des accroissements finis. La même formule est encore valable si on prend
f : {[}b,a{]} \rightarrow~ \mathbb{R}~.

\paragraph{8.3.5 Extensions du théorème des accroissements finis}

Théorème~8.3.9 Soit f,g : {[}a,b{]} \rightarrow~ \mathbb{R}~ continues sur {[}a,b{]},
dérivables sur {]}a,b{[}. Alors, il existe c \in{]}a,b{[} tel que
\left
\textbar{}\matrix\,f(b) - f(a)&f'(c)
\cr g(b) - g(a)&g'(c)\right \textbar{}
= 0.

Démonstration Posons

\phi(t) = \left
\textbar{}\matrix\,f(b) - f(a)&f(t) -
f(a) \cr g(b) - g(a)&g(t) - g(a)\right
\textbar{}

La fonction \phi est continue sur {[}a,b{]}, dérivable sur {]}a,b{[} avec
\phi'(t) = \left
\textbar{}\matrix\,f(b) - f(a)&f'(t)
\cr g(b) - g(a)&g'(t)\right \textbar{}.
Comme \phi(a) = \phi(b) = 0, le théorème de Rolle garantit l'existence d'un c
\in{]}a,b{[} tel que \phi'(c) = 0.

Corollaire~8.3.10 (règle de L'Hôpital). Soit f,g : I \rightarrow~ \mathbb{R}~ continues sur
I, dérivables sur I \diagdown\a\. On suppose
qu'il existe \eta \textgreater{} 0 tel que g' ne s'annule pas sur {]}a -
\eta,a + \eta{[}\diagdown\a\ et que  f'
\over g' a une limite \ell au point a. Alors  f(t)-f(a)
\over g(t)-g(a) admet la même limite au point a.

Démonstration Le théorème de Rolle garantit dé\jmathà que g(t) - g(a) ne
s'annule pas sur {]}a - \eta,a +
\eta{[}\diagdown\a\. De plus le théorème
précédent montre que pour t \in{]}a - \eta,a +
\eta{[}\diagdown\a\, il existe c\_t
\in{]}a,t{[} (ou {]}t,a{[}) tel que \left
\textbar{}\matrix\,f(t) -
f(a)&f'(c\_t) \cr g(t) -
g(a)&g'(c\_t)\right \textbar{} = 0 soit 
f(t)-f(a) \over g(t)-g(a) = f'(c\_t)
\over g'(c\_t) . Quand t tend vers a, il en est
de même de c\_t et le théorème de composition des limites donne

lim\_t\rightarrow~a,t\neq~a~
f(t) - f(a) \over g(t) - g(a) = \ell

\paragraph{8.3.6 Fonctions convexes de classe \mathcal{C}^1}

Définition~8.3.2 Soit I un intervalle de \mathbb{R}~ et f : I \rightarrow~ \mathbb{R}~ une fonction de
classe \mathcal{C}^1. On dit que f est convexe si f' est croissante.

Remarque~8.3.2 Si f est de classe C^2, f est convexe si et
seulement si~f'' est positive.

Théorème~8.3.11 Soit I un intervalle de \mathbb{R}~ et f : I \rightarrow~ \mathbb{R}~ une fonction de
classe \mathcal{C}^1 convexe. Alors (i) \forall~~a,b
\in I, \forall~~t \in {[}0,1{]}, f(ta + (1 - t)b) \leq tf(a)
+ (1 - t)f(b) (ii) \Gamma = \(x,y) \in
\mathbb{R}~^2∣x \in I\text et
y ≥ f(x)\ est une partie convexe de \mathbb{R}~^2
(iii) \forall~~a,b \in I, f(b) ≥ f(a) + (b - a)f'(a)
(iv) si a \in I, l'application I \diagdown\a\
dans \mathbb{R}~, t\mapsto~p\_a(t) = f(t)-f(a)
\over t-a est croissante (v)
\forall~~a,b,c \in I, a \textless{} b \textless{} c \rigtharrow~
f(b)-f(a) \over b-a \leq f(c)-f(a) \over
c-a \leq f(c)-f(b) \over c-b

Démonstration (i) On peut évidemment supposer a \textless{} b. D'après
le théorème des accroissements finis, il existe c \in{]}a,b{[} tel que
f(b) - f(a) = (b - a)f'(c). Posons c = t\_0a + (1 -
t\_0)b. Soit \phi(t) = tf(a) + (1 - t)f(b) - f(ta + (1 - t)b) pour
t \in {[}0,1{]}. Alors \phi est de classe \mathcal{C}^1 et \phi'(t) = f(a) -
f(b) - (a - b)f'(ta + (1 - t)b) = (b - a)(f'(ta + (1 - t)b) -
f'(t\_0a + (1 - t\_0)b). Comme f' est croissante et
t\mapsto~ta + (1 - t)b est décroissante, la composée
est décroissante et donc on a le tableau de variation

\begin{center}\rule{3in}{0.4pt}\end{center}

\begin{center}\rule{3in}{0.4pt}\end{center}

\begin{center}\rule{3in}{0.4pt}\end{center}

\begin{center}\rule{3in}{0.4pt}\end{center}

\begin{center}\rule{3in}{0.4pt}\end{center}

\begin{center}\rule{3in}{0.4pt}\end{center}

t

0

t\_0

1

\begin{center}\rule{3in}{0.4pt}\end{center}

\begin{center}\rule{3in}{0.4pt}\end{center}

\begin{center}\rule{3in}{0.4pt}\end{center}

\begin{center}\rule{3in}{0.4pt}\end{center}

\begin{center}\rule{3in}{0.4pt}\end{center}

\begin{center}\rule{3in}{0.4pt}\end{center}

\phi'(t)

+

0

-

\begin{center}\rule{3in}{0.4pt}\end{center}

\begin{center}\rule{3in}{0.4pt}\end{center}

\begin{center}\rule{3in}{0.4pt}\end{center}

\begin{center}\rule{3in}{0.4pt}\end{center}

\begin{center}\rule{3in}{0.4pt}\end{center}

\begin{center}\rule{3in}{0.4pt}\end{center}

\phi(t)

0

\nearrow

\searrow

0

\begin{center}\rule{3in}{0.4pt}\end{center}

\begin{center}\rule{3in}{0.4pt}\end{center}

\begin{center}\rule{3in}{0.4pt}\end{center}

\begin{center}\rule{3in}{0.4pt}\end{center}

\begin{center}\rule{3in}{0.4pt}\end{center}

\begin{center}\rule{3in}{0.4pt}\end{center}

ce qui montre que la fonction \phi est positive sur {[}0,1{]}.

(ii) Soit (x\_1,y\_1) et (x\_2,y\_2)
dans \Gamma et t \in {[}0,1{]}. On a

ty\_1 + (1 - t)y\_2 ≥ tf(x\_1) + (1 -
t)f(x\_2) ≥ f(tx\_1 + (1 - t)x\_2)

donc t(x\_1,y\_1) + (1 - t)(x\_2,y\_2) \in
\Gamma. Donc \Gamma est convexe.

(iii) Posons \phi(t) = f(t) - f(a) - (t - a)f'(a). La fonction \phi est de
classe \mathcal{C}^1 et \phi'(t) = f'(t) - f'(a). Comme f' est croissante,
on a le tableau de variation

\begin{center}\rule{3in}{0.4pt}\end{center}

\begin{center}\rule{3in}{0.4pt}\end{center}

\begin{center}\rule{3in}{0.4pt}\end{center}

\begin{center}\rule{3in}{0.4pt}\end{center}

\begin{center}\rule{3in}{0.4pt}\end{center}

\begin{center}\rule{3in}{0.4pt}\end{center}

t

a

\begin{center}\rule{3in}{0.4pt}\end{center}

\begin{center}\rule{3in}{0.4pt}\end{center}

\begin{center}\rule{3in}{0.4pt}\end{center}

\begin{center}\rule{3in}{0.4pt}\end{center}

\begin{center}\rule{3in}{0.4pt}\end{center}

\begin{center}\rule{3in}{0.4pt}\end{center}

\phi'(t)

+

0

-

\begin{center}\rule{3in}{0.4pt}\end{center}

\begin{center}\rule{3in}{0.4pt}\end{center}

\begin{center}\rule{3in}{0.4pt}\end{center}

\begin{center}\rule{3in}{0.4pt}\end{center}

\begin{center}\rule{3in}{0.4pt}\end{center}

\begin{center}\rule{3in}{0.4pt}\end{center}

\phi(t)

\searrow

0

\nearrow

\begin{center}\rule{3in}{0.4pt}\end{center}

\begin{center}\rule{3in}{0.4pt}\end{center}

\begin{center}\rule{3in}{0.4pt}\end{center}

\begin{center}\rule{3in}{0.4pt}\end{center}

\begin{center}\rule{3in}{0.4pt}\end{center}

\begin{center}\rule{3in}{0.4pt}\end{center}

ce qui montre que la fonction \phi est positive sur I.

(iv) Posons p\_a(t) = f(t)-f(a) \over t-a si
t\neq~a et p\_a(a) = f'(a). La fonction
p\_a est continue sur I, dérivable sur I
\diagdown\a\ et p\_a'(t) =
f(a)-f(t)-(a-t)f'(t) \over (t-a)^2 ≥ 0
d'après (iii). On en déduit que p\_a est croissante.

(v) D'après (iv), on a p\_a(b) \leq p\_a(c) =
p\_c(a) \leq p\_c(b) ce qui est le résultat souhaité.

Théorème~8.3.12 Soit f : I \rightarrow~ \mathbb{R}~ de classe \mathcal{C}^1 convexe. Alors,
pour tout
(x\_1,\\ldots,x\_n~)
\in I^n, pour toute famille
(\alpha~\_1,\\ldots,\alpha~\_n~)
\in (\mathbb{R}~^+)^n telle que \alpha~\_1 +
\\ldots~ +
\alpha~\_n = 1, on a

f(\sum \_i=1^n\alpha~~\_
ix\_i) \leq\\sum
\_i=1^n\alpha~\_ if(x\_i)

Démonstration Par récurrence sur n. Si n = 2, on a \alpha~\_2 = 1 -
\alpha~\_1 et \alpha~\_1 \in {[}0,1{]}. L'inégalité se réduit à
l'assertion (i) du théorème précédent. Supposons le résultat vrai pour n
- 1 et montrons le pour n. Si \alpha~\_n = 0, on est immédiatement
ramené au cas n - 1. On peut donc supposer
\alpha~\_n\neq~0. Si \alpha~\_n = 1, alors
tous les autres \alpha~\_i sont nuls et l'inégalité est triviale. On
peut donc supposer \alpha~\_n \in{]}0,1{[}. On écrit alors
\\sum ~
\_i=1^n\alpha~\_ix\_i = \alpha~\_nx\_n
+ (1 - \alpha~\_n)y avec y =
\alpha~\_1x\_1+\\ldots+\alpha~\_n-1x\_n-1~
\over
\alpha~\_1+\\ldots+\alpha~\_n-1~
= \beta~\_1x\_1 +
\\ldots\beta~\_n-1x\_n-1~
\in I. On a alors \beta~\_i ≥ 0 et
\\sum ~
\_i=1^n-1\beta~\_i = 1. On peut donc écrire (par
l'hypothèse de récurrence) f(y)
\leq\\sum ~
\_i=1^n-1\beta~\_if(x\_i) soit

\begin{align*} f(\\sum
\_i=1^n\alpha~\_ ix\_i)& =&
f(\alpha~\_nx\_n + (1 - \alpha~\_n)y) \leq
\alpha~\_nf(x\_n) + (1 - \alpha~\_n)f(y) \%&
\\ & \leq& \alpha~\_nf(x\_n) + (1
- \alpha~\_n)\\sum
\_i=1^n-1\beta~\_ if(x\_i) =
\sum \_i=1^n\alpha~~\_
if(x\_i)\%& \\
\end{align*}

puisque (1 - \alpha~\_n)\beta~\_i = \alpha~\_i.

Corollaire~8.3.13 (inégalité de Hölder). Soit p,q \in \mathbb{R}~^+∗ tels
que  1 \over p + 1 \over q = 1.
Pour toute famille
a\_1,\\ldots,a\_n,b\_1,\\\ldots,b\_n~
de réels positifs, on a

\sum \_i=1^na~\_
ib\_i \leq\left (\\sum
\_i=1^na\_ i^p\right
)^1\diagupp\left (\\sum
\_i=1^nb\_ i^q\right
)^1\diagupq

Démonstration Posons A = \left
(\\sum ~
\_i=1^na\_i^p\right
)^1\diagupp, B = \left
(\\sum ~
\_i=1^nb\_i^q\right
)^1\diagupq. La fonction exponentielle étant convexe sur \mathbb{R}~, on a
\forall~s,t \in \mathbb{R}~, e~^ s \over
p + t \over q  \leq 1 \over p
e^s + 1 \over q e^t. Si
a\_i et b\_i sont non nuls, en appliquant ceci à s =
plog  a\_i \over A~
et t = qlog  b\_i~
\over B , on obtient  a\_i
\over A  b\_i \over B \leq 1
\over p  a\_i^p \over
A^p + 1 \over q  b\_i^q
\over B^q , inégalité qui reste vrai si
a\_ib\_i = 0~; en sommant de i = 1 \jmathusque n on obtient

 1 \over AB \\sum
\_i=1^na\_ ib\_i \leq 1
\over pA^p  \\sum
\_i=1^na\_ i^p + 1 \over
qB^q  \\sum
\_i=1^nb\_ i^q = 1 \over
p + 1 \over q = 1

soit \\sum ~
\_i=1^na\_ib\_i \leq AB ce qu'on voulait
démontrer.

Corollaire~8.3.14 (inégalité de Minkowski). Soit p ≥ 1. Pour toute
famille
a\_1,\\ldots,a\_n,b\_1,\\\ldots,b\_n~
de réels positifs, on a

 \left (\\sum
\_i=1^n(a\_ i +
b\_i)^p\right )^1\diagupp
\leq\left (\\sum
\_i=1^na\_ i^p\right
)^1\diagupp + \left (\\sum
\_i=1^nb\_ i^p\right
)^1\diagupp

Démonstration C'est évident si p = 1~; si p \textgreater{} 1,
définissons q par la condition  1 \over p + 1
\over q = 1~; on écrit (a\_i +
b\_i)^p = a\_i(a\_i +
b\_i)^p-1 + b\_i(a\_i +
b\_i)^p-1 et on applique deux fois l'inégalité de
Hölder. On obtient alors

\begin{align*} \\sum
\_i=1^n(a\_ i + b\_i)^p&
\leq& \left (\\sum
\_i=1^na\_ i^p\right
)^1\diagupp\left (\\sum
\_i=1^n(a\_ i +
b\_i)^(p-1)q\right )^1\diagupq \%&
\\ & \text &
+\left (\\sum
\_i=1^nb\_ i^p\right
)^1\diagupp\left (\\sum
\_i=1^n(a\_ i +
b\_i)^(p-1)q\right )^1\diagupq\%&
\\ \end{align*}

Mais (p - 1)q = p et l'inégalité ci dessus s'écrit donc après mise en
facteur

\begin{align*} \left
(\sum \_i=1^n(a\_ i~ +
b\_i)^p\right )^1\diagupp&& \%&
\\ & \leq& \left
(\left (\\sum
\_i=1^na\_ i^p\right
)^1\diagupp + \left (\\sum
\_i=1^nb\_ i^p\right
)^1\diagupp\right )\left
(\sum \_i=1^n(a\_ i~ +
b\_i)^p\right )^1\diagupq\%&
\\ \end{align*}

Si \\sum ~
\_i=1^n(a\_i + b\_i)^p = 0,
l'inégalité cherchée est évidente~; sinon, on peut diviser les deux
membres par \left
(\\sum ~
\_i=1^n(a\_i +
b\_i)^p\right )^1\diagupq et on
obtient (en tenant compte de 1 - 1 \over p = 1
\over q )

 \left (\\sum
\_i=1^n(a\_ i +
b\_i)^p\right )^1\diagupp
\leq\left (\\sum
\_i=1^na\_ i^p\right
)^1\diagupp + \left (\\sum
\_i=1^nb\_ i^p\right
)^1\diagupp

{[}
{[}
{[}
{[}

\end{document}
