\documentclass[]{article}
\usepackage[T1]{fontenc}
\usepackage{lmodern}
\usepackage{amssymb,amsmath}
\usepackage{ifxetex,ifluatex}
\usepackage{fixltx2e} % provides \textsubscript
% use upquote if available, for straight quotes in verbatim environments
\IfFileExists{upquote.sty}{\usepackage{upquote}}{}
\ifnum 0\ifxetex 1\fi\ifluatex 1\fi=0 % if pdftex
  \usepackage[utf8]{inputenc}
\else % if luatex or xelatex
  \ifxetex
    \usepackage{mathspec}
    \usepackage{xltxtra,xunicode}
  \else
    \usepackage{fontspec}
  \fi
  \defaultfontfeatures{Mapping=tex-text,Scale=MatchLowercase}
  \newcommand{\euro}{€}
\fi
% use microtype if available
\IfFileExists{microtype.sty}{\usepackage{microtype}}{}
\ifxetex
  \usepackage[setpagesize=false, % page size defined by xetex
              unicode=false, % unicode breaks when used with xetex
              xetex]{hyperref}
\else
  \usepackage[unicode=true]{hyperref}
\fi
\hypersetup{breaklinks=true,
            bookmarks=true,
            pdfauthor={},
            pdftitle={Somme d'une serie enti`ere},
            colorlinks=true,
            citecolor=blue,
            urlcolor=blue,
            linkcolor=magenta,
            pdfborder={0 0 0}}
\urlstyle{same}  % don't use monospace font for urls
\setlength{\parindent}{0pt}
\setlength{\parskip}{6pt plus 2pt minus 1pt}
\setlength{\emergencystretch}{3em}  % prevent overfull lines
\setcounter{secnumdepth}{0}
 
/* start css.sty */
.cmr-5{font-size:50%;}
.cmr-7{font-size:70%;}
.cmmi-5{font-size:50%;font-style: italic;}
.cmmi-7{font-size:70%;font-style: italic;}
.cmmi-10{font-style: italic;}
.cmsy-5{font-size:50%;}
.cmsy-7{font-size:70%;}
.cmex-7{font-size:70%;}
.cmex-7x-x-71{font-size:49%;}
.msbm-7{font-size:70%;}
.cmtt-10{font-family: monospace;}
.cmti-10{ font-style: italic;}
.cmbx-10{ font-weight: bold;}
.cmr-17x-x-120{font-size:204%;}
.cmsl-10{font-style: oblique;}
.cmti-7x-x-71{font-size:49%; font-style: italic;}
.cmbxti-10{ font-weight: bold; font-style: italic;}
p.noindent { text-indent: 0em }
td p.noindent { text-indent: 0em; margin-top:0em; }
p.nopar { text-indent: 0em; }
p.indent{ text-indent: 1.5em }
@media print {div.crosslinks {visibility:hidden;}}
a img { border-top: 0; border-left: 0; border-right: 0; }
center { margin-top:1em; margin-bottom:1em; }
td center { margin-top:0em; margin-bottom:0em; }
.Canvas { position:relative; }
li p.indent { text-indent: 0em }
.enumerate1 {list-style-type:decimal;}
.enumerate2 {list-style-type:lower-alpha;}
.enumerate3 {list-style-type:lower-roman;}
.enumerate4 {list-style-type:upper-alpha;}
div.newtheorem { margin-bottom: 2em; margin-top: 2em;}
.obeylines-h,.obeylines-v {white-space: nowrap; }
div.obeylines-v p { margin-top:0; margin-bottom:0; }
.overline{ text-decoration:overline; }
.overline img{ border-top: 1px solid black; }
td.displaylines {text-align:center; white-space:nowrap;}
.centerline {text-align:center;}
.rightline {text-align:right;}
div.verbatim {font-family: monospace; white-space: nowrap; text-align:left; clear:both; }
.fbox {padding-left:3.0pt; padding-right:3.0pt; text-indent:0pt; border:solid black 0.4pt; }
div.fbox {display:table}
div.center div.fbox {text-align:center; clear:both; padding-left:3.0pt; padding-right:3.0pt; text-indent:0pt; border:solid black 0.4pt; }
div.minipage{width:100%;}
div.center, div.center div.center {text-align: center; margin-left:1em; margin-right:1em;}
div.center div {text-align: left;}
div.flushright, div.flushright div.flushright {text-align: right;}
div.flushright div {text-align: left;}
div.flushleft {text-align: left;}
.underline{ text-decoration:underline; }
.underline img{ border-bottom: 1px solid black; margin-bottom:1pt; }
.framebox-c, .framebox-l, .framebox-r { padding-left:3.0pt; padding-right:3.0pt; text-indent:0pt; border:solid black 0.4pt; }
.framebox-c {text-align:center;}
.framebox-l {text-align:left;}
.framebox-r {text-align:right;}
span.thank-mark{ vertical-align: super }
span.footnote-mark sup.textsuperscript, span.footnote-mark a sup.textsuperscript{ font-size:80%; }
div.tabular, div.center div.tabular {text-align: center; margin-top:0.5em; margin-bottom:0.5em; }
table.tabular td p{margin-top:0em;}
table.tabular {margin-left: auto; margin-right: auto;}
div.td00{ margin-left:0pt; margin-right:0pt; }
div.td01{ margin-left:0pt; margin-right:5pt; }
div.td10{ margin-left:5pt; margin-right:0pt; }
div.td11{ margin-left:5pt; margin-right:5pt; }
table[rules] {border-left:solid black 0.4pt; border-right:solid black 0.4pt; }
td.td00{ padding-left:0pt; padding-right:0pt; }
td.td01{ padding-left:0pt; padding-right:5pt; }
td.td10{ padding-left:5pt; padding-right:0pt; }
td.td11{ padding-left:5pt; padding-right:5pt; }
table[rules] {border-left:solid black 0.4pt; border-right:solid black 0.4pt; }
.hline hr, .cline hr{ height : 1px; margin:0px; }
.tabbing-right {text-align:right;}
span.TEX {letter-spacing: -0.125em; }
span.TEX span.E{ position:relative;top:0.5ex;left:-0.0417em;}
a span.TEX span.E {text-decoration: none; }
span.LATEX span.A{ position:relative; top:-0.5ex; left:-0.4em; font-size:85%;}
span.LATEX span.TEX{ position:relative; left: -0.4em; }
div.float img, div.float .caption {text-align:center;}
div.figure img, div.figure .caption {text-align:center;}
.marginpar {width:20%; float:right; text-align:left; margin-left:auto; margin-top:0.5em; font-size:85%; text-decoration:underline;}
.marginpar p{margin-top:0.4em; margin-bottom:0.4em;}
.equation td{text-align:center; vertical-align:middle; }
td.eq-no{ width:5%; }
table.equation { width:100%; } 
div.math-display, div.par-math-display{text-align:center;}
math .texttt { font-family: monospace; }
math .textit { font-style: italic; }
math .textsl { font-style: oblique; }
math .textsf { font-family: sans-serif; }
math .textbf { font-weight: bold; }
.partToc a, .partToc, .likepartToc a, .likepartToc {line-height: 200%; font-weight:bold; font-size:110%;}
.chapterToc a, .chapterToc, .likechapterToc a, .likechapterToc, .appendixToc a, .appendixToc {line-height: 200%; font-weight:bold;}
.index-item, .index-subitem, .index-subsubitem {display:block}
.caption td.id{font-weight: bold; white-space: nowrap; }
table.caption {text-align:center;}
h1.partHead{text-align: center}
p.bibitem { text-indent: -2em; margin-left: 2em; margin-top:0.6em; margin-bottom:0.6em; }
p.bibitem-p { text-indent: 0em; margin-left: 2em; margin-top:0.6em; margin-bottom:0.6em; }
.paragraphHead, .likeparagraphHead { margin-top:2em; font-weight: bold;}
.subparagraphHead, .likesubparagraphHead { font-weight: bold;}
.quote {margin-bottom:0.25em; margin-top:0.25em; margin-left:1em; margin-right:1em; text-align:justify;}
.verse{white-space:nowrap; margin-left:2em}
div.maketitle {text-align:center;}
h2.titleHead{text-align:center;}
div.maketitle{ margin-bottom: 2em; }
div.author, div.date {text-align:center;}
div.thanks{text-align:left; margin-left:10%; font-size:85%; font-style:italic; }
div.author{white-space: nowrap;}
.quotation {margin-bottom:0.25em; margin-top:0.25em; margin-left:1em; }
h1.partHead{text-align: center}
.sectionToc, .likesectionToc {margin-left:2em;}
.subsectionToc, .likesubsectionToc {margin-left:4em;}
.subsubsectionToc, .likesubsubsectionToc {margin-left:6em;}
.frenchb-nbsp{font-size:75%;}
.frenchb-thinspace{font-size:75%;}
.figure img.graphics {margin-left:10%;}
/* end css.sty */

\title{Somme d'une serie enti`ere}
\author{}
\date{}

\begin{document}
\maketitle

\textbf{Warning: 
requires JavaScript to process the mathematics on this page.\\ If your
browser supports JavaScript, be sure it is enabled.}

\begin{center}\rule{3in}{0.4pt}\end{center}

[
[
[]
[

\subsubsection{11.2 Somme d'une série entière}

\paragraph{11.2.1 Etude sur le disque ouvert de convergence (domaine
complexe)}

Théorème~11.2.1 (continuité de la somme). Soit
\\sum ~
a_nz^n une série entière à coefficients dans E, de
rayon de convergence R > 0. Alors la fonction S :
z\mapsto~\\\sum
 _n=0^+\infty~a_nz^n est continue sur le
disque D(0,R) = \z \in
K∣z <
R\.

Démonstration On a vu en effet que la série convergeait normalement sur
D'(0,r) pour r < R, donc S est continue sur un tel D'(0,r) et
donc finalement sur D(0,R).

Corollaire~11.2.2 (principe des zéros isolés). Soit
\\sum ~
a_nz^n une série entière non nulle à coefficients
dans E, de rayon de convergence R > 0. Alors, il existe \eta
> 0 tel que la fonction S :
z\mapsto~\\\sum
 _n=0^+\infty~a_nz^n ne s'annule pas sur
D(0,\eta) \diagdown\0\.

Démonstration Soit en effet p =\
min\k \in
\mathbb{N}~∣a_k\mathrel\neq~0\.
On a alors S(z) =\ \\sum
 _n=p^+\infty~a_nz^n =
z^p \\sum ~
_n=p^+\infty~a_nz^n-p =
z^p \\sum ~
_n=0^+\infty~a_n+pz^n. Mais la série entière
\\sum ~
_na_n+pz^n a même rayon de convergence que la
série entière \\sum ~
a_nz^n (facile) et sa somme définit donc une
fonction s continue sur D(0,R) avec s(0) =
a_p\neq~0. Donc il existe \eta
> 0 tel que z < \eta \rigtharrow~
s(z)\neq~0. Mais alors, pour 0 <
z < \eta, on a S(z) =
z^ps(z)\neq~0, ce que l'on voulait
démontrer.

Corollaire~11.2.3 (principe d'identification). Soit
\\sum ~
a_nz^n et et
\\sum ~
b_nz^n deux séries entières à coefficients dans E,
de rayons de convergence non nuls, de sommes S_1 et
S_2. Alors on a équivalence de (i) \forall~~n
\in \mathbb{N}~, a_n = b_n (ii) il existe \eta > 0 tel
que \forall~z \in D(0,\eta), S_1~(z) =
S_2(z) (iii) il existe une suite (z_n) de K formée
d'éléments distincts telle que limz_n~
= 0 et \forall~n \in \mathbb{N}~, S_1(z_n~) =
S_2(z_n)

Démonstration Il suffit d'appliquer le principe des zéros isolés à la
série entière \\sum ~
(a_n - b_n)z^n dont la somme est
S_1 - S_2 dans le disque
D(0,min(R_1,R_2~)).

Remarque~11.2.1 Le corollaire précédent qui garantit l'unicité du
développement en série entière d'une fonction est très souvent utilisé~;
il permet en particulier de travailler par identification. Il laisse
penser qu'il doit être possible de récupérer les valeurs des
coefficients a_n à partir de la somme S de la série. En fait,
dans une première approche, les techniques sont très différentes suivant
que le corps de base est \mathbb{C} ou \mathbb{R}~.

Dans le cadre complexe, on a le théorème suivant qui relie les
coefficients du développement en série entière à la somme de la fonction

Théorème~11.2.4 (formules de Cauchy). Soit E un \mathbb{C}-espace vectoriel normé
complet, \\sum ~
a_nz^n une série entière à coefficients dans E, de
rayon de convergence R > 0, de somme S. Alors, pour tout r
< R, on a

\forall~n \in \mathbb{N}~, a_n~ = 1
\over 2\pi~r^n \int ~
_0^2\pi~S(re^i\theta)e^-in\theta d\theta

Démonstration Puisque r < R, la série
\\sum ~
\a_n\r^n
est convergente.

On a S(re^i\theta)e^-in\theta =\
\sum ~
_p=0^+\infty~a_pr^pe^i(p-n)\theta.
Mais l'inégalité
\a_nr^pe^i(p-n)\theta\
\leq\
a_p\r^p montre que la série
converge normalement par rapport à \theta. On en déduit que

\begin{align*} \int ~
_0^2\pi~S(re^i\theta)e^-in\theta d\theta& =&
\int  _0^2\pi~~
\sum _p=0^+\infty~a_
pr^pe^i(p-n)\theta d\theta\%&
\\ & =& \\sum
_p=0^+\infty~a_ pr^p
\\int  ~
_0^2\pi~e^i(p-n)\theta d\theta\%&
\\ & =& 2\pi~a_nr^n
\%& \\ \end{align*}

car \int  _0^2\pi~e^ik\theta~
d\theta = \left \ \cases 0
&si k\neq~0 \cr 2\pi~&si k = 0 
\right .. On obtient donc la formule ci dessus.

Théorème~11.2.5 Soit \\\sum
 a_nz_n une série entière de rayon de convergence
R et de somme S(z). Soit z_0 \in \mathbb{C} tel que
z_0 < R. Alors la fonction
S(z_0 + u) est développable en série entière de u dans le
disque ouvert u < R
-z_0, ce que l'on traduit par~: la somme
d'une série entière est analytique dans son disque ouvert de
convergence.

Démonstration Puisque z_0 +
u\leqz_0 + u
< R, on peut écrire

\begin{align*} S(z_0 + u)& =&
\sum _n=0^+\infty~a_
n(z_0 + u)^n \%& \\
& =& \\sum
_n=0^+\infty~\left (\\sum
_m=0^nC_ n^ma_
nz_0^n-mu^m\right )\%&
\\ \end{align*}

On considère alors la famille (x_m,n)_m,n\in\mathbb{N}~ définie
par

 x_m,n = \left \
\cases
C_n^ma_nz_0^n-mu^m&si
m \leq n \cr 0 &si m > n 
\right .

On a

\sum _m=0^+\infty~x_
m,n = \\sum
_m=0^nC_ n^ma_
nz_0^n-mu^m = a_
n(z_0 +
u)^n

qui est une série convergente puisque la série
\\sum ~
_na_n(z_0
+ u)^n converge (une série entière converge
absolument dans son disque ouvert de convergence). Ceci montre que la
famille (x_m,n)_m,n\in\mathbb{N}~ est sommable. On peut donc
appliquer d'interversion des sommations et on a

\begin{align*} S(z_0 + u)& =&
\sum _n=0^+\infty~~\left
(\sum _m=0^+\infty~x_
m,n\right ) = \\sum
_m=0^+\infty~\left (\\sum
_n=0^+\infty~x_ m,n\right )\%&
\\ & =& \\sum
_m=0^+\infty~u^m\left
(\sum _n=m^+\infty~C_
n^ma_ nz_0^n-m\right )
\%& \\ \end{align*}

ce qui montre le résultat.

\paragraph{11.2.2 Etude sur le disque ouvert de convergence (domaine
réel)}

Avant de regarder le cas réel, nous allons démontrer le lemme suivant

Lemme~11.2.6 Soit \\sum ~
a_nz^n une série entière à coefficients dans le
K-espace vectoriel normé E, de rayon de convergence R. Soit F \in K(X) une
fraction rationnelle non nulle et N \in \mathbb{N}~ tel que F n'ait pas de pôle
entier supérieur à N. Alors la série entière
\\sum ~
F(n)a_nz^n a encore pour rayon de convergence R.

Démonstration Soit z \in K tel que z < R et
soit r tel que z < r < R. La
suite
\a_n\r^n
est donc bornée, par exemple majorée par M. On a alors, pour n ≥ N,
\F(n)a_nz^n\
\leq MF(n)\left (
z \over r \right
)^n ∼ \lambda~n^d\left (
z \over r \right
)^n (où d est le degré de la fraction rationnelle, différence
entre le degré de son numérateur et celui de son dénominateur, si bien
que F(t)∼ \lambda~t^d au voisinage de + \infty~) qui
tend vers 0 quand n tend vers + \infty~. On a donc R' ≥ R. Mais on a aussi
a_n = 1 \over F (n)\left
(F(n)a_n\right ) si bien que les suites
(a_n) et (F(n)a_n) jouent ici un rôle parfaitement
symétrique. On a donc aussi R ≥ R', soit R = R'.

On va en déduire le théorème suivant

Théorème~11.2.7 Soit E un \mathbb{R}~-espace vectoriel normé complet,
\\sum ~
a_nt^n une série entière à coefficients dans E, de
rayon de convergence R > 0, de somme S. Alors la fonction S
est de classe C^\infty~ sur ] - R,R[ et
\forall~p \in \mathbb{N}~, \\forall~~t \in] -
R,R[

\begin{align*} S^(p)(t)& =&
\sum _n=p^+\infty~~n(n -
1)\ldots(n - p + 1)a_
pt^n-p\%& \\ & =&
\sum _n=0^+\infty~~ (n + p)!
\over n! a_n+pt^n \%&
\\ \end{align*}

Démonstration Les deux formules se déduisent l'une de l'autre par un
changement d'indice (le changement de n - p en n). Il suffit donc de
montrer la première. Mais le lemme précédent assure que la série entière
\\sum  _n≥p~n(n
- 1)\\ldots~(n - p +
1)a_pt^n a même rayon de convergence R que la série
de départ. Il en est donc de même de la série entière
\\sum  _n≥p~n(n
- 1)\\ldots~(n - p +
1)a_pt^n-p et cette série converge donc normalement
sur [-r,r] pour r < R. Montrons donc le résultat par
récurrence sur p. Pour p = 0, il n'y a rien à montrer. Supposons le
résultat vrai pour p avec \forall~~t \in] - R,R[,
S^(p)(t) =\
\sum  _n=p^+\infty~~n(n -
1)\\ldots~(n - p +
1)a_pt^n-p. La série dérivée
\\sum ~
_n≥p+1n(n -
1)\\ldots~(n - p +
1)(n - p)a_pt^n-p-1 converge normalement sur
[-r,+r] et le théorème de dérivation des séries de fonctions nous
garantit que S^(p) est de classe \mathcal{C}^1 (donc S de
classe C^p+1) sur [-r,r] avec
\forall~t \in [-r,r], S^(p)~(t)
= \\sum ~
_n=p^+\infty~n(n -
1)\\ldots~(n - p +
1)(n - p)a_pt^n-p-1~; mais comme r est quelconque (r
< R), S est de classe C^p+1 sur ] - R,R[ et la
formule ci-dessus y reste valable, ce qui achève la récurrence.

Corollaire~11.2.8 Soit E un \mathbb{R}~-espace vectoriel normé complet,
\\sum ~
a_nt^n une série entière à coefficients dans E, de
rayon de convergence R > 0, de somme S. Alors

\forall~n \in \mathbb{N}~, a_n = S^(n)~(0)
\over n!

Démonstration Faire t = 0 dans la formule précédente.

Remarque~11.2.2 Les coefficients a_n sont donc les mêmes que
ceux qui apparaissent dans un développement limité en 0 de la fonction
S.

Le même argument de convergence normale sur [-r,r] \subset~] - R,R[
montrera le théorème suivant

Théorème~11.2.9 Soit E un \mathbb{R}~-espace vectoriel normé complet,
\\sum ~
a_nt^n une série entière à coefficients dans E, de
rayon de convergence R > 0, de somme S. Alors

\forall~~t \in] - R,R[, \\int
 _0^tS(u) du = \\sum
_n=0^+\infty~a_ n t^n+1
\over n + 1

\paragraph{11.2.3 Etude sur le cercle de convergence}

On a vu qu'en un point du cercle de convergence, la série pouvait aussi
bien diverger que converger. Si la série converge, la question de la
continuité de la somme en ce point se pose immédiatement. En fait, on
peut montrer que sur \mathbb{C}, la somme peut très bien être discontinue en un
tel point, mais qu'il s'agit en fait d'une discontinuité tangentielle~:
il se peut que S(z) ne tende pas vers S(z_0) quand z tend vers
z_0 tangentiellement au cercle de convergence. Pour nous il
nous suffira de savoir que S(z) tend vers S(z_0) quand z tend
vers z_0 suivant un rayon, ce que garantit le théorème suivant

Théorème~11.2.10 (Abel) Soit
\\sum ~
a_nz^n une série entière à coefficients dans E, de
rayon de convergence R > 0 et S :
z\mapsto~\\\sum
 _n=0^+\infty~a_nz^n continue sur le
disque D(0,R) = \z \in
K∣z <
R\. Soit z_0 \in K tel que
z_0 = R et la série
\\sum ~
a_nz_0^n converge. Alors

\sum _n=0^+\infty~a_
nz_0^n = lim_
t\rightarrow~1^-S(tz_0)

Démonstration On considère la série de fonctions
\\sum ~
a_nz_0^nt^n, qui converge sur
[0,1]. Nous allons démontrer sa convergence uniforme~; ceci
garantira la continuité de sa somme au point 1, ce qui n'est autre
l'assertion à démontrer.

Premier cas~: la série \\\sum
 a_nz_0^n converge absolument. Alors on a
\forall~~t \in [0,1],
\a_nz_0^nt^n\
\leq\
a_nz_0^n\, série
convergente indépendante de t. Donc la série converge normalement.

Deuxième cas~: le critère des séries alternées s'applique à la série
\\sum ~
a_nz_0^n, autrement dit
a_nz_0^n = (-1)^nb_n avec
(b_n) qui tend vers 0 en décroissant. Alors
a_nz_0^nt^n =
(-1)^nb_nt^n, avec
t\mapsto~b_nt^n qui tend
uniformément vers 0 en décroissant. Le critère de convergence uniforme
des séries alternées garantit la convergence uniforme de la série.

Cas général~: nous allons montrer que la série de fonctions vérifie le
critère de Cauchy uniforme. Pour cela posons R_n
= \\sum ~
_k=n^+\infty~a_kz_0^k. On a alors

\begin{align*} \\sum
_n=p^qa_ nz_0^nt^n&
=& \sum _n=p^q(R_ n~ -
R_n+1)t^n = \\sum
_n=p^qR_ nt^n
-\sum _n=p^qR_
n+1t^n\%& \\ & =&
\sum _n=p^qR_
nt^n -\\sum
_n=p+1^q+1R_ nt^n-1 \%&
\\ & =& R_pt^p -
R_ q+1t^q -\\sum
_n=p+1^qR_ n(t^n-1 - t^n)
\%& \\ \end{align*}

On a limR_n~ = 0 (reste d'une série
convergente). Soit \epsilon > 0~; il existe N \in \mathbb{N}~ (indépendant de
t) tel que n ≥ N \rigtharrow~\
R_n\ < \epsilon
\over 2 . Alors, en tenant compte de t^p ≥
0, t^q ≥ 0 et t^n-1 - t^n ≥ 0, on a
\forall~~t \in [0,1],

\\\sum
_n=p^qa_
nz_0^nt^n\ \leq \epsilon
\over 2 (t^p + t^q +
\sum _n=p+1^q(t^n-1~ -
t^n)) = 2t^p \epsilon \over 2 \leq \epsilon

ce qui montre que la série vérifie le critère de Cauchy uniforme, donc
est uniformément convergente.

Remarque~11.2.3 Une des premières utilités du théorème précédent est de
calculer la somme de certaines séries numériques du type
\\sum ~
a_nz_0^n~; il arrive en effet fréquemment que
la somme de la série entière
\\sum ~
a_nz^n soit facile à calculer pour
z < R (par exemple par dérivation ou par
résolution d'une certaine équation différentielle). Il suffit alors de
passer à la limite pour calculer la somme de la série.

Exemple~11.2.1 On cherche à calculer la somme de la série alternée
\\sum ~
_n=1^+\infty~ (-1)^n-1 \over n .
Pour t < 1, on pose f(t)
= \\sum ~
_n=1^+\infty~ (-1)^n-1 \over n
t^n~; on sait que f est C^\infty~ sur ] - 1,1[ et
que f'(t) = \\sum ~
_n=1^+\infty~(-1)^n-1t^n-1 = 1
\over 1+t . Comme f(0) = 0, on a f(t)
= log~ (1 + t). Le théorème précédent assure
que \\sum ~
_n=1^+\infty~ (-1)^n-1 \over n
=\
lim_t\rightarrow~1^-log~ (1 + t)
= log~ 2. Suivant le même principe, le lecteur
montrera que \\sum ~
_n=0^+\infty~ (-1)^n \over 2n+1
= \mathrmarctg~ 1 = \pi~
\over 4 , en introduisant la série entière
\\sum ~
_n=0^+\infty~ (-1)^n \over 2n+1
t^2n+1.

[
[
[
[

\end{document}
