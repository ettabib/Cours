\documentclass[]{article}
\usepackage[T1]{fontenc}
\usepackage{lmodern}
\usepackage{amssymb,amsmath}
\usepackage{ifxetex,ifluatex}
\usepackage{fixltx2e} % provides \textsubscript
% use upquote if available, for straight quotes in verbatim environments
\IfFileExists{upquote.sty}{\usepackage{upquote}}{}
\ifnum 0\ifxetex 1\fi\ifluatex 1\fi=0 % if pdftex
  \usepackage[utf8]{inputenc}
\else % if luatex or xelatex
  \ifxetex
    \usepackage{mathspec}
    \usepackage{xltxtra,xunicode}
  \else
    \usepackage{fontspec}
  \fi
  \defaultfontfeatures{Mapping=tex-text,Scale=MatchLowercase}
  \newcommand{\euro}{€}
\fi
% use microtype if available
\IfFileExists{microtype.sty}{\usepackage{microtype}}{}
\ifxetex
  \usepackage[setpagesize=false, % page size defined by xetex
              unicode=false, % unicode breaks when used with xetex
              xetex]{hyperref}
\else
  \usepackage[unicode=true]{hyperref}
\fi
\hypersetup{breaklinks=true,
            bookmarks=true,
            pdfauthor={},
            pdftitle={Suites},
            colorlinks=true,
            citecolor=blue,
            urlcolor=blue,
            linkcolor=magenta,
            pdfborder={0 0 0}}
\urlstyle{same}  % don't use monospace font for urls
\setlength{\parindent}{0pt}
\setlength{\parskip}{6pt plus 2pt minus 1pt}
\setlength{\emergencystretch}{3em}  % prevent overfull lines
\setcounter{secnumdepth}{0}
 
/* start css.sty */
.cmr-5{font-size:50%;}
.cmr-7{font-size:70%;}
.cmmi-5{font-size:50%;font-style: italic;}
.cmmi-7{font-size:70%;font-style: italic;}
.cmmi-10{font-style: italic;}
.cmsy-5{font-size:50%;}
.cmsy-7{font-size:70%;}
.cmex-7{font-size:70%;}
.cmex-7x-x-71{font-size:49%;}
.msbm-7{font-size:70%;}
.cmtt-10{font-family: monospace;}
.cmti-10{ font-style: italic;}
.cmbx-10{ font-weight: bold;}
.cmr-17x-x-120{font-size:204%;}
.cmsl-10{font-style: oblique;}
.cmti-7x-x-71{font-size:49%; font-style: italic;}
.cmbxti-10{ font-weight: bold; font-style: italic;}
p.noindent { text-indent: 0em }
td p.noindent { text-indent: 0em; margin-top:0em; }
p.nopar { text-indent: 0em; }
p.indent{ text-indent: 1.5em }
@media print {div.crosslinks {visibility:hidden;}}
a img { border-top: 0; border-left: 0; border-right: 0; }
center { margin-top:1em; margin-bottom:1em; }
td center { margin-top:0em; margin-bottom:0em; }
.Canvas { position:relative; }
li p.indent { text-indent: 0em }
.enumerate1 {list-style-type:decimal;}
.enumerate2 {list-style-type:lower-alpha;}
.enumerate3 {list-style-type:lower-roman;}
.enumerate4 {list-style-type:upper-alpha;}
div.newtheorem { margin-bottom: 2em; margin-top: 2em;}
.obeylines-h,.obeylines-v {white-space: nowrap; }
div.obeylines-v p { margin-top:0; margin-bottom:0; }
.overline{ text-decoration:overline; }
.overline img{ border-top: 1px solid black; }
td.displaylines {text-align:center; white-space:nowrap;}
.centerline {text-align:center;}
.rightline {text-align:right;}
div.verbatim {font-family: monospace; white-space: nowrap; text-align:left; clear:both; }
.fbox {padding-left:3.0pt; padding-right:3.0pt; text-indent:0pt; border:solid black 0.4pt; }
div.fbox {display:table}
div.center div.fbox {text-align:center; clear:both; padding-left:3.0pt; padding-right:3.0pt; text-indent:0pt; border:solid black 0.4pt; }
div.minipage{width:100%;}
div.center, div.center div.center {text-align: center; margin-left:1em; margin-right:1em;}
div.center div {text-align: left;}
div.flushright, div.flushright div.flushright {text-align: right;}
div.flushright div {text-align: left;}
div.flushleft {text-align: left;}
.underline{ text-decoration:underline; }
.underline img{ border-bottom: 1px solid black; margin-bottom:1pt; }
.framebox-c, .framebox-l, .framebox-r { padding-left:3.0pt; padding-right:3.0pt; text-indent:0pt; border:solid black 0.4pt; }
.framebox-c {text-align:center;}
.framebox-l {text-align:left;}
.framebox-r {text-align:right;}
span.thank-mark{ vertical-align: super }
span.footnote-mark sup.textsuperscript, span.footnote-mark a sup.textsuperscript{ font-size:80%; }
div.tabular, div.center div.tabular {text-align: center; margin-top:0.5em; margin-bottom:0.5em; }
table.tabular td p{margin-top:0em;}
table.tabular {margin-left: auto; margin-right: auto;}
div.td00{ margin-left:0pt; margin-right:0pt; }
div.td01{ margin-left:0pt; margin-right:5pt; }
div.td10{ margin-left:5pt; margin-right:0pt; }
div.td11{ margin-left:5pt; margin-right:5pt; }
table[rules] {border-left:solid black 0.4pt; border-right:solid black 0.4pt; }
td.td00{ padding-left:0pt; padding-right:0pt; }
td.td01{ padding-left:0pt; padding-right:5pt; }
td.td10{ padding-left:5pt; padding-right:0pt; }
td.td11{ padding-left:5pt; padding-right:5pt; }
table[rules] {border-left:solid black 0.4pt; border-right:solid black 0.4pt; }
.hline hr, .cline hr{ height : 1px; margin:0px; }
.tabbing-right {text-align:right;}
span.TEX {letter-spacing: -0.125em; }
span.TEX span.E{ position:relative;top:0.5ex;left:-0.0417em;}
a span.TEX span.E {text-decoration: none; }
span.LATEX span.A{ position:relative; top:-0.5ex; left:-0.4em; font-size:85%;}
span.LATEX span.TEX{ position:relative; left: -0.4em; }
div.float img, div.float .caption {text-align:center;}
div.figure img, div.figure .caption {text-align:center;}
.marginpar {width:20%; float:right; text-align:left; margin-left:auto; margin-top:0.5em; font-size:85%; text-decoration:underline;}
.marginpar p{margin-top:0.4em; margin-bottom:0.4em;}
.equation td{text-align:center; vertical-align:middle; }
td.eq-no{ width:5%; }
table.equation { width:100%; } 
div.math-display, div.par-math-display{text-align:center;}
math .texttt { font-family: monospace; }
math .textit { font-style: italic; }
math .textsl { font-style: oblique; }
math .textsf { font-family: sans-serif; }
math .textbf { font-weight: bold; }
.partToc a, .partToc, .likepartToc a, .likepartToc {line-height: 200%; font-weight:bold; font-size:110%;}
.chapterToc a, .chapterToc, .likechapterToc a, .likechapterToc, .appendixToc a, .appendixToc {line-height: 200%; font-weight:bold;}
.index-item, .index-subitem, .index-subsubitem {display:block}
.caption td.id{font-weight: bold; white-space: nowrap; }
table.caption {text-align:center;}
h1.partHead{text-align: center}
p.bibitem { text-indent: -2em; margin-left: 2em; margin-top:0.6em; margin-bottom:0.6em; }
p.bibitem-p { text-indent: 0em; margin-left: 2em; margin-top:0.6em; margin-bottom:0.6em; }
.paragraphHead, .likeparagraphHead { margin-top:2em; font-weight: bold;}
.subparagraphHead, .likesubparagraphHead { font-weight: bold;}
.quote {margin-bottom:0.25em; margin-top:0.25em; margin-left:1em; margin-right:1em; text-align:justify;}
.verse{white-space:nowrap; margin-left:2em}
div.maketitle {text-align:center;}
h2.titleHead{text-align:center;}
div.maketitle{ margin-bottom: 2em; }
div.author, div.date {text-align:center;}
div.thanks{text-align:left; margin-left:10%; font-size:85%; font-style:italic; }
div.author{white-space: nowrap;}
.quotation {margin-bottom:0.25em; margin-top:0.25em; margin-left:1em; }
h1.partHead{text-align: center}
.sectionToc, .likesectionToc {margin-left:2em;}
.subsectionToc, .likesubsectionToc {margin-left:4em;}
.subsubsectionToc, .likesubsubsectionToc {margin-left:6em;}
.frenchb-nbsp{font-size:75%;}
.frenchb-thinspace{font-size:75%;}
.figure img.graphics {margin-left:10%;}
/* end css.sty */

\title{Suites}
\author{}
\date{}

\begin{document}
\maketitle

\textbf{Warning: 
requires JavaScript to process the mathematics on this page.\\ If your
browser supports JavaScript, be sure it is enabled.}

\begin{center}\rule{3in}{0.4pt}\end{center}

[
[
[]
[

\subsubsection{4.3 Suites}

\paragraph{4.3.1 Suites convergentes, limites}

Définition~4.3.1 Soit E un espace métrique et
(x_n)_n\in\mathbb{N}~ une suite de E. On dit qu'elle est
convergente s'il existe \ell \in E vérifiant les conditions équivalentes

\begin{itemize}
\itemsep1pt\parskip0pt\parsep0pt
\item
  (i) \forall~V \in V (\ell), \\exists~N
  \in \mathbb{N}~, n ≥ N \rigtharrow~ x_n \in V
\item
  (ii) \forall~~\epsilon > 0,
  \existsN \in \mathbb{N}~, n ≥ N \rigtharrow~ d(x_n~,\ell)
  < \epsilon
\end{itemize}

Une suite qui n'est pas convergente est dite divergente.

Démonstration La propriété (ii) ne fait que traduire (i) pour V =
B(\ell,\epsilon). Or toute boule est un voisinage et tout voisinage contient une
boule. Donc (i) et (ii) sont équivalents.

Proposition~4.3.1 Si la suite (x_n)_n\in\mathbb{N}~ est
convergente, l'élément \ell de E est unique~; on l'appelle la limite de la
suite. On note \ell = limx_n~.

Démonstration Si \ell et \ell' conviennent avec \ell\neq~\ell', il existe d'après la
propriété de séparation U ouvert contenant \ell et V ouvert contenant \ell'
tels que U \bigcap V = \varnothing~. Mais
\existsN,\quad n ≥ N \rigtharrow~ x_n~ \in
U et \existsN', n ≥ N' \rigtharrow~ x_n~ \in V . Pour n
≥ max(N,N'), on a x_n~ \in U \bigcap V . C'est
absurde.

Remarque~4.3.1 On prendra garde à ne pas introduire le symbole
limx_n~ de manière opératoire avant
d'avoir démontré son existence. On remarquera d'autre part que les
notions de convergence et de limites sont purement topologiques
puisqu'on peut les exprimer en terme de voisinages~; elles sont donc
inchangées par changement de distance topologiquement équivalente.

\paragraph{4.3.2 Sous suites, valeurs d'adhérences}

Définition~4.3.2 Soit (x_n)_n\in\mathbb{N}~ une suite d'éléments
de E et soit \phi : \mathbb{N}~ \rightarrow~ \mathbb{N}~ strictement croissante. On dit alors que la suite
(x_\phi(n))_n\in\mathbb{N}~ est une sous suite de la suite
(x_n)_n\in\mathbb{N}~.

Théorème~4.3.2 Si la suite (x_n)_n\in\mathbb{N}~ est convergente
de limite \ell, alors toute sous suite converge et a la même limite.

C'est une conséquence du lemme suivant qui se démontre à l'aide d'une
récurrence facile.

Lemme~4.3.3 Soit \phi : \mathbb{N}~ \rightarrow~ \mathbb{N}~ strictement croissante. Alors
\forall~~n \in \mathbb{N}~, \phi(n) ≥ n.

Démonstration (du théorème) Soit V voisinage de \ell et N \in \mathbb{N}~ tel que n ≥ N
\rigtharrow~ x_n \in V . Alors pour n ≥ N on a \phi(n) ≥ n ≥ N donc
x_\phi(n) \in V . Donc \ell est encore limite de la suite
(x_\phi(n))_n\in\mathbb{N}~.

Définition~4.3.3 Soit (x_n)_n\in\mathbb{N}~ une suite d'éléments
de E et a \in E. On dit que a est valeur d'adhérence de la suite si on a
les conditions équivalentes

\begin{itemize}
\itemsep1pt\parskip0pt\parsep0pt
\item
  (i) \forall~V \in V (a), \\forall~~N
  \in \mathbb{N}~, \exists~n ≥ N,\quad
  x_n \in V
\item
  (i)' \forall~~\epsilon > 0,
  \forall~N \in \mathbb{N}~, \\exists~n ≥
  N,\quad d(x_n,a) < \epsilon
\item
  (ii) \forall~~V \in V (a), \n \in
  \mathbb{N}~∣x_n \in V \ est
  infini.
\item
  (ii)' \forall~~\epsilon > 0,
  \n \in \mathbb{N}~∣d(x_n,a)
  < \epsilon\ est infini.
\item
  (iii) a est limite d'une sous suite (x_\phi(n)) de la suite
  (x_n).
\end{itemize}

Démonstration (i)' n'est qu'une reformulation de (i) en termes de
boules, comme (ii)' vis à vis de (ii). Si \n \in
\mathbb{N}~∣x_n \in V \ est
fini, il existe N \in \mathbb{N}~ tel que n ≥ N \rigtharrow~
x_n∉V et donc (i) n'est pas vérifié.
Ceci montre que (i) \rigtharrow~(ii). De même, si \n \in
\mathbb{N}~∣x_n \in V \ est
infini, il contient des éléments d'indices arbitrairement grands, donc
(ii) \rigtharrow~(i). Si a est limite de la sous suite (x_\phi(n)) et V est
un voisinage de a, il existe N \in \mathbb{N}~ tel que n ≥ N \rigtharrow~ x_\phi(n) \in V .
Donc \n \in \mathbb{N}~∣x_n \in V
\ contient \phi([N,+\infty~[), il est donc infini, soit
(iii) \rigtharrow~(ii). Montrons maintenant que (i)' \rigtharrow~(iii), ce qui achèvera la
démonstration. On construit \phi(n) par récurrence de la manière suivante~:
on prend \epsilon = 1 \over n+1 et N = \left
\ \cases 0 &si n = 0
\cr \phi(n - 1) + 1&si n ≥ 1 \cr 
\right .~; il existe alors p ≥ \phi(n - 1) + 1 tel que
d(a,x_p) < 1 \over n+1 ~; on pose
\phi(n) = p. On construit ainsi une fonction strictement croissante de \mathbb{N}~
dans \mathbb{N}~ vérifiant d(a,x_\phi(n)) < 1
\over n+1 . D'où une sous suite de limite a.

Remarque~4.3.2 On a donc vu qu'une suite convergente admet une unique
valeur d'adhérence, sa limite. Il est clair qu'une valeur d'adhérence
d'une sous suite est encore une valeur d'adhérence de la suite.

\paragraph{4.3.3 Caractérisation des fermés d'un espace métrique}

Théorème~4.3.4 Soit E un espace métrique, A une partie de E et a \in E.
Alors a est dans l'adhérence de A si et seulement si~a est limite (dans
E) d'une suite d'éléments de A.

Démonstration Si a est limite d'une suite (a_n)_n\in\mathbb{N}~
d'éléments de A, soit V un voisinage de a. Alors
\existsN \in \mathbb{N}~, n ≥ N \rigtharrow~ a_n~ \in V . En
particulier a_n \in V \bigcap A qui est donc non vide. Donc a
appartient à \overlineA. Inversement, soit a
\in\overlineA. Alors, pour tout \epsilon > 0, A \bigcap
B(a,\epsilon)\neq~\varnothing~. Pour \epsilon = 1 \over
n+1 on peut donc trouver a_n \in A tel que d(a,a_n)
< 1 \over n+1 . Donc a est limite de la
suite (a_n) d'éléments de A.

Corollaire~4.3.5 Soit E un espace métrique. Une partie A de E est fermée
si et seulement si~toute suite d'éléments de A qui converge dans E a une
limite qui appartient à A.

Démonstration Ceci traduit simplement l'inclusion
\overlineA \subset~ A.

[
[
[
[

\end{document}
