\documentclass[]{article}
\usepackage[T1]{fontenc}
\usepackage{lmodern}
\usepackage{amssymb,amsmath}
\usepackage{ifxetex,ifluatex}
\usepackage{fixltx2e} % provides \textsubscript
% use upquote if available, for straight quotes in verbatim environments
\IfFileExists{upquote.sty}{\usepackage{upquote}}{}
\ifnum 0\ifxetex 1\fi\ifluatex 1\fi=0 % if pdftex
  \usepackage[utf8]{inputenc}
\else % if luatex or xelatex
  \ifxetex
    \usepackage{mathspec}
    \usepackage{xltxtra,xunicode}
  \else
    \usepackage{fontspec}
  \fi
  \defaultfontfeatures{Mapping=tex-text,Scale=MatchLowercase}
  \newcommand{\euro}{€}
\fi
% use microtype if available
\IfFileExists{microtype.sty}{\usepackage{microtype}}{}
\ifxetex
  \usepackage[setpagesize=false, % page size defined by xetex
              unicode=false, % unicode breaks when used with xetex
              xetex]{hyperref}
\else
  \usepackage[unicode=true]{hyperref}
\fi
\hypersetup{breaklinks=true,
            bookmarks=true,
            pdfauthor={},
            pdftitle={Primitives et integrales},
            colorlinks=true,
            citecolor=blue,
            urlcolor=blue,
            linkcolor=magenta,
            pdfborder={0 0 0}}
\urlstyle{same}  % don't use monospace font for urls
\setlength{\parindent}{0pt}
\setlength{\parskip}{6pt plus 2pt minus 1pt}
\setlength{\emergencystretch}{3em}  % prevent overfull lines
\setcounter{secnumdepth}{0}
 
/* start css.sty */
.cmr-5{font-size:50%;}
.cmr-7{font-size:70%;}
.cmmi-5{font-size:50%;font-style: italic;}
.cmmi-7{font-size:70%;font-style: italic;}
.cmmi-10{font-style: italic;}
.cmsy-5{font-size:50%;}
.cmsy-7{font-size:70%;}
.cmex-7{font-size:70%;}
.cmex-7x-x-71{font-size:49%;}
.msbm-7{font-size:70%;}
.cmtt-10{font-family: monospace;}
.cmti-10{ font-style: italic;}
.cmbx-10{ font-weight: bold;}
.cmr-17x-x-120{font-size:204%;}
.cmsl-10{font-style: oblique;}
.cmti-7x-x-71{font-size:49%; font-style: italic;}
.cmbxti-10{ font-weight: bold; font-style: italic;}
p.noindent { text-indent: 0em }
td p.noindent { text-indent: 0em; margin-top:0em; }
p.nopar { text-indent: 0em; }
p.indent{ text-indent: 1.5em }
@media print {div.crosslinks {visibility:hidden;}}
a img { border-top: 0; border-left: 0; border-right: 0; }
center { margin-top:1em; margin-bottom:1em; }
td center { margin-top:0em; margin-bottom:0em; }
.Canvas { position:relative; }
li p.indent { text-indent: 0em }
.enumerate1 {list-style-type:decimal;}
.enumerate2 {list-style-type:lower-alpha;}
.enumerate3 {list-style-type:lower-roman;}
.enumerate4 {list-style-type:upper-alpha;}
div.newtheorem { margin-bottom: 2em; margin-top: 2em;}
.obeylines-h,.obeylines-v {white-space: nowrap; }
div.obeylines-v p { margin-top:0; margin-bottom:0; }
.overline{ text-decoration:overline; }
.overline img{ border-top: 1px solid black; }
td.displaylines {text-align:center; white-space:nowrap;}
.centerline {text-align:center;}
.rightline {text-align:right;}
div.verbatim {font-family: monospace; white-space: nowrap; text-align:left; clear:both; }
.fbox {padding-left:3.0pt; padding-right:3.0pt; text-indent:0pt; border:solid black 0.4pt; }
div.fbox {display:table}
div.center div.fbox {text-align:center; clear:both; padding-left:3.0pt; padding-right:3.0pt; text-indent:0pt; border:solid black 0.4pt; }
div.minipage{width:100%;}
div.center, div.center div.center {text-align: center; margin-left:1em; margin-right:1em;}
div.center div {text-align: left;}
div.flushright, div.flushright div.flushright {text-align: right;}
div.flushright div {text-align: left;}
div.flushleft {text-align: left;}
.underline{ text-decoration:underline; }
.underline img{ border-bottom: 1px solid black; margin-bottom:1pt; }
.framebox-c, .framebox-l, .framebox-r { padding-left:3.0pt; padding-right:3.0pt; text-indent:0pt; border:solid black 0.4pt; }
.framebox-c {text-align:center;}
.framebox-l {text-align:left;}
.framebox-r {text-align:right;}
span.thank-mark{ vertical-align: super }
span.footnote-mark sup.textsuperscript, span.footnote-mark a sup.textsuperscript{ font-size:80%; }
div.tabular, div.center div.tabular {text-align: center; margin-top:0.5em; margin-bottom:0.5em; }
table.tabular td p{margin-top:0em;}
table.tabular {margin-left: auto; margin-right: auto;}
div.td00{ margin-left:0pt; margin-right:0pt; }
div.td01{ margin-left:0pt; margin-right:5pt; }
div.td10{ margin-left:5pt; margin-right:0pt; }
div.td11{ margin-left:5pt; margin-right:5pt; }
table[rules] {border-left:solid black 0.4pt; border-right:solid black 0.4pt; }
td.td00{ padding-left:0pt; padding-right:0pt; }
td.td01{ padding-left:0pt; padding-right:5pt; }
td.td10{ padding-left:5pt; padding-right:0pt; }
td.td11{ padding-left:5pt; padding-right:5pt; }
table[rules] {border-left:solid black 0.4pt; border-right:solid black 0.4pt; }
.hline hr, .cline hr{ height : 1px; margin:0px; }
.tabbing-right {text-align:right;}
span.TEX {letter-spacing: -0.125em; }
span.TEX span.E{ position:relative;top:0.5ex;left:-0.0417em;}
a span.TEX span.E {text-decoration: none; }
span.LATEX span.A{ position:relative; top:-0.5ex; left:-0.4em; font-size:85%;}
span.LATEX span.TEX{ position:relative; left: -0.4em; }
div.float img, div.float .caption {text-align:center;}
div.figure img, div.figure .caption {text-align:center;}
.marginpar {width:20%; float:right; text-align:left; margin-left:auto; margin-top:0.5em; font-size:85%; text-decoration:underline;}
.marginpar p{margin-top:0.4em; margin-bottom:0.4em;}
.equation td{text-align:center; vertical-align:middle; }
td.eq-no{ width:5%; }
table.equation { width:100%; } 
div.math-display, div.par-math-display{text-align:center;}
math .texttt { font-family: monospace; }
math .textit { font-style: italic; }
math .textsl { font-style: oblique; }
math .textsf { font-family: sans-serif; }
math .textbf { font-weight: bold; }
.partToc a, .partToc, .likepartToc a, .likepartToc {line-height: 200%; font-weight:bold; font-size:110%;}
.chapterToc a, .chapterToc, .likechapterToc a, .likechapterToc, .appendixToc a, .appendixToc {line-height: 200%; font-weight:bold;}
.index-item, .index-subitem, .index-subsubitem {display:block}
.caption td.id{font-weight: bold; white-space: nowrap; }
table.caption {text-align:center;}
h1.partHead{text-align: center}
p.bibitem { text-indent: -2em; margin-left: 2em; margin-top:0.6em; margin-bottom:0.6em; }
p.bibitem-p { text-indent: 0em; margin-left: 2em; margin-top:0.6em; margin-bottom:0.6em; }
.paragraphHead, .likeparagraphHead { margin-top:2em; font-weight: bold;}
.subparagraphHead, .likesubparagraphHead { font-weight: bold;}
.quote {margin-bottom:0.25em; margin-top:0.25em; margin-left:1em; margin-right:1em; text-align:justify;}
.verse{white-space:nowrap; margin-left:2em}
div.maketitle {text-align:center;}
h2.titleHead{text-align:center;}
div.maketitle{ margin-bottom: 2em; }
div.author, div.date {text-align:center;}
div.thanks{text-align:left; margin-left:10%; font-size:85%; font-style:italic; }
div.author{white-space: nowrap;}
.quotation {margin-bottom:0.25em; margin-top:0.25em; margin-left:1em; }
h1.partHead{text-align: center}
.sectionToc, .likesectionToc {margin-left:2em;}
.subsectionToc, .likesubsectionToc {margin-left:4em;}
.subsubsectionToc, .likesubsubsectionToc {margin-left:6em;}
.frenchb-nbsp{font-size:75%;}
.frenchb-thinspace{font-size:75%;}
.figure img.graphics {margin-left:10%;}
/* end css.sty */

\title{Primitives et integrales}
\author{}
\date{}

\begin{document}
\maketitle

\textbf{Warning: \href{http://www.math.union.edu/locate/jsMath}{jsMath}
requires JavaScript to process the mathematics on this page.\\ If your
browser supports JavaScript, be sure it is enabled.}

\begin{center}\rule{3in}{0.4pt}\end{center}

{[}\href{coursse53.html}{next}{]} {[}\href{coursse51.html}{prev}{]}
{[}\href{coursse51.html\#tailcoursse51.html}{prev-tail}{]}
{[}\hyperref[tailcoursse52.html]{tail}{]}
{[}\href{coursch10.html\#coursse52.html}{up}{]}

\subsubsection{9.3 Primitives et intégrales}

\paragraph{9.3.1 Continuité et dérivabilité par rapport à une borne}

Théorème~9.3.1 Soit I un intervalle de ℝ, f : I → E réglée, a ∈ I. Pour
t ∈ I, on pose F(t) =\{\textbackslash{}mathop\{∫ \} \}\_\{a\}\^{}\{t\}f.
Alors l'application F est continue sur I~; elle est dérivable en tout
point \{t\}\_\{0\} où f est continue et on a alors F'(\{t\}\_\{0\}) =
f(\{t\}\_\{0\}).

Démonstration Soit \{t\}\_\{0\} ∈ I. Supposons tout d'abord que
\{t\}\_\{0\} n'est pas une extrémité de I et soit η \textgreater{} 0 tel
que {[}\{t\}\_\{0\} − η,\{t\}\_\{0\} + η{]} ⊂ I. Alors f est réglée sur
{[}\{t\}\_\{0\} − η,\{t\}\_\{0\} + η{]} donc bornée par une constante M
≥ 0. Pour t ∈ {[}\{t\}\_\{0\} − η,\{t\}\_\{0\} + η{]}, on a alors
\textbackslash{}\textbar{}F(t) −
F(\{t\}\_\{0\})\textbackslash{}\textbar{}
=\textbackslash{}\textbar{}\{\textbackslash{}mathop\{∫ \}
\}\_\{\{t\}\_\{0\}\}\^{}\{t\}f\textbackslash{}\textbar{} ≤ M\textbar{}t
− \{t\}\_\{0\}\textbar{} ce qui montre que F est continue au point
\{t\}\_\{0\}. On montre le résultat d'une manière similaire si
\{t\}\_\{0\} est une extrémité de I en introduisant selon le cas
{[}\{t\}\_\{0\},\{t\}\_\{0\} + η{]} ou {[}\{t\}\_\{0\} −
η,\{t\}\_\{0\}{]}.

Supposons maintenant que f est continue au point \{t\}\_\{0\}~; soit ε
\textgreater{} 0 et soit η \textgreater{} 0 tel que \textbar{}t −
\{t\}\_\{0\}\textbar{} \textless{} η ⇒\textbackslash{}\textbar{} f(t) −
f(\{t\}\_\{0\})\textbackslash{}\textbar{} ≤ ε. Pour \textbar{}t −
\{t\}\_\{0\}\textbar{} \textless{} η, on a

\textbackslash{}begin\{eqnarray*\} \textbackslash{}\textbar{}F(t) −
F(\{t\}\_\{0\}) − (t −
\{t\}\_\{0\})f(\{t\}\_\{0\})\textbackslash{}\textbar{}
=\textbackslash{}\textbar{}\{\textbackslash{}mathop\{∫ \}
\}\_\{\{t\}\_\{0\}\}\^{}\{t\}f − (t − \{t\}\_\{
0\})f(\{t\}\_\{0\})\textbackslash{}\textbar{}\&\&\%\&
\textbackslash{}\textbackslash{} \& =\&
\textbackslash{}\textbar{}\{\textbackslash{}mathop\{∫ \}
\}\_\{\{t\}\_\{0\}\}\^{}\{t\}(f − f(\{t\}\_\{
0\}))\textbackslash{}\textbar{} ≤\textbackslash{}mathop\{ sgn\}(t −
\{t\}\_\{0\})\{\textbackslash{}mathop\{∫ \}
\}\_\{\{t\}\_\{0\}\}\^{}\{t\}\textbackslash{}\textbar{}f − f(\{t\}\_\{
0\})\textbackslash{}\textbar{}\%\& \textbackslash{}\textbackslash{} \&
≤\& ε\textbar{}t − \{t\}\_\{0\}\textbar{} \%\&
\textbackslash{}\textbackslash{} \textbackslash{}end\{eqnarray*\}

ce qui peut encore s'écrire \textbackslash{}\textbar{}\{
F(t)−F(\{t\}\_\{0\}) \textbackslash{}over t−\{t\}\_\{0\}\} −
f(\{t\}\_\{0\})\textbackslash{}\textbar{} ≤ ε. Ceci montre que F est
dérivable au point \{t\}\_\{0\} et que F'(\{t\}\_\{0\}) =
f(\{t\}\_\{0\}).

Remarque~9.3.1 De la même fa\textbackslash{}c\{c\}on, on montre que la
continuité à gauche de f implique la dérivabilité à gauche de F~; le
même résultat est évidemment encore valable à droite.

\paragraph{9.3.2 Primitives}

Définition~9.3.1 Soit f : I → E une application~; on dit que F : I → E
est une primitive de f si F est dérivable et F' = f.

En remarquant que F' = G' \textbackslash{}mathrel\{⇔\} F − G est
constante sur l'intervalle I, on obtient immédiatement

Proposition~9.3.2 Si f : I → E admet une primitive F, elle en admet une
infinité qui sont exactement les t\textbackslash{}mathrel\{↦\}F(t) + k
où k ∈ E.

Remarque~9.3.2 On a vu que si F' admet une limite en un point
\{t\}\_\{0\}, alors nécessairement F' était continue au point
\{t\}\_\{0\}~; ceci permet d'exhiber facilement une fonction qui n'admet
pas de primitive (toute fonction qui admet une limite en un point sans
que ce soit la valeur de cette fonction en ce point)~; ceci montre
d'autre part qu'une fonction réglée qui admet une primitive est
nécessairement continue, puisqu'elle doit admettre en tout point une
limite à gauche et une limite à droite, qui ne peuvent être que la
valeur de la fonction en ce point (étudier séparément ce qui se passe à
gauche et à droite du point).

Théorème~9.3.3 Soit f : I → E une fonction continue~; alors f admet des
primitives sur I. Si F est une telle primitive, on a
\textbackslash{}mathop\{∀\}a,b ∈ I, \{\textbackslash{}mathop\{∫ \}
\}\_\{a\}\^{}\{b\}f = F(b) − F(a) =\{ \textbackslash{}left
{[}F(t)\textbackslash{}right {]}\}\_\{a\}\^{}\{b\}.

Démonstration Soit α ∈ I et posons G(t) =\{\textbackslash{}mathop\{∫ \}
\}\_\{α\}\^{}\{t\}f~; puisque f est continue, G est dérivable et G' = f.
Donc G est une primitive de f. Si F est une autre primitive de f, on a F
= G + k et donc

\{\textbackslash{}mathop\{∫ \} \}\_\{a\}\^{}\{b\}f
=\{\textbackslash{}mathop\{∫ \} \}\_\{α\}\^{}\{b\}f
−\{\textbackslash{}mathop\{∫ \} \}\_\{α\}\^{}\{a\}f = G(b) − G(a) = F(b)
− F(a)

Remarque~9.3.3 Ce dernier résultat est un des moyens les plus simples et
les plus généraux de calcul d'intégrales~; il ramène le calcul d'une
intégrale à la recherche d'une primitive de la fonction f.

\paragraph{9.3.3 Changement de variable, intégration par parties}

En ce qui concerne le changement de variable, on est confronté à un
choix~: soit autoriser des fonctions très générales (des fonctions
réglées) et se limiter à des changements de variables très particuliers
(mais néanmoins fort utiles)~; soit restreindre le champ d'application
aux fonctions continues et autoriser des changements de variables plus
généraux (par exemple de classe \{C\}\^{}\{1\}).

Le premier théorème se montre trivialement pour les fonctions en
escalier et ensuite par un simple passage à la limite pour les fonctions
réglées sur un segment.

Théorème~9.3.4 Soit f : {[}a,b{]} → E une fonction réglée. (i) Soit T ∈
ℝ et g : {[}a − T,b − T{]} → E, t\textbackslash{}mathrel\{↦\}f(t + T).
Alors g est réglée et \{\textbackslash{}mathop\{∫ \}
\}\_\{a−T\}\^{}\{b−T\}g =\{\textbackslash{}mathop\{∫ \}
\}\_\{a\}\^{}\{b\}f. (ii) Soit λ ∈ \{ℝ\}\^{}\{∗\} et soit g :
t\textbackslash{}mathrel\{↦\}f(λt) définie sur {[}\{ a
\textbackslash{}over λ\} ,\{ b \textbackslash{}over λ\} {]} si λ
\textgreater{} 0 et sur {[}\{ b \textbackslash{}over λ\} ,\{ a
\textbackslash{}over λ\} {]} si λ \textless{} 0. Alors g est réglée et
(avec la convention de Chasles) \{\textbackslash{}mathop\{∫ \}
\}\_\{a∕λ\}\^{}\{b∕λ\}g =\{ 1 \textbackslash{}over λ\}
\{\textbackslash{}mathop\{∫ \} \}\_\{a\}\^{}\{b\}f.

Théorème~9.3.5 (changement de variables). Soit f : I → E continue et
soit φ : J → I de classe \{C\}\^{}\{1\} (où I et J sont deux intervalles
de ℝ). Alors,

\textbackslash{}mathop\{∀\}a,b ∈ J, \{\textbackslash{}mathop\{∫ \}
\}\_\{a\}\^{}\{b\}f ∘ φ φ' =\{\textbackslash{}mathop\{∫ \}
\}\_\{φ(a)\}\^{}\{φ(b)\}f

Démonstration Soit F une primitive de f sur I, alors (F ∘ φ)' = f ∘ φ φ'
et donc F ∘ φ est une primitive de f ∘ φ φ' sur J~; on a donc
\{\textbackslash{}mathop\{∫ \} \}\_\{a\}\^{}\{b\}f ∘ φ φ' = F ∘ φ(b) − F
∘ φ(a) =\{\textbackslash{}mathop\{∫ \} \}\_\{φ(a)\}\^{}\{φ(b)\}f.

Remarque~9.3.4 Notation On introduira la notation différentielle
\{\textbackslash{}mathop\{∫ \} \}\_\{a\}\^{}\{b\}f
=\{\textbackslash{}mathop\{∫ \} \}\_\{a\}\^{}\{b\}f(t) dt où t est une
variable muette. De cette manière, faire le changement de variables t =
φ(u) dans l'intégrale \{\textbackslash{}mathop\{∫ \}
\}\_\{φ(a)\}\^{}\{φ(b)\}f(t) dt c'est faire varier u de a à b (pour que
t varie de φ(a) à φ(b)) et remplacer f(t) par (f ∘ φ)(u) et dt par φ'(u)
du si bien que la formule ci dessus s'écrit \{\textbackslash{}mathop\{∫
\} \}\_\{a\}\^{}\{b\}(f ∘ φ)(u)φ'(u) du =\{\textbackslash{}mathop\{∫ \}
\}\_\{φ(a)\}\^{}\{φ(b)\}f(t) dt. De même, faire le changement de
variable inverse t = φ(u) dans l'intégrale \{\textbackslash{}mathop\{∫
\} \}\_\{a\}\^{}\{b\}f ∘ φ(u)φ'(u) du c'est faire varier t de φ(a) à
φ(b) (puisque u varie de a à b), remplacer f(φ(u)) par f(t) et φ'(u) du
par dt.

Exemple~9.3.1 Les deux sens du théorème de changement de variables sont
utiles comme nous allons le voir sur deux exemples~:
\{\textbackslash{}mathop\{∫ \}
\}\_\{0\}\^{}\{x\}t\textbackslash{}mathop\{sin\} (\{t\}\^{}\{2\}) dt =\{
1 \textbackslash{}over 2\} \{\textbackslash{}mathop\{∫ \}
\}\_\{0\}\^{}\{\{x\}\^{}\{2\} \}\textbackslash{}mathop\{ sin\} (u) du
=\{ 1−\textbackslash{}mathop\{cos\} \{x\}\^{}\{2\} \textbackslash{}over
2\} en posant u = \{t\}\^{}\{2\}~; de même \{\textbackslash{}mathop\{∫
\} \}\_\{0\}\^{}\{1\}\textbackslash{}sqrt\{1 − \{x\}\^{}\{2\}\} dx
=\{\textbackslash{}mathop\{∫ \}
\}\_\{0\}\^{}\{π∕2\}\textbackslash{}sqrt\{1 −\{\textbackslash{}mathop\{
sin\} \}\^{}\{2 \} t\}\textbackslash{}mathop\{cos\} t dt
=\{\textbackslash{}mathop\{∫ \}
\}\_\{0\}\^{}\{π∕2\}\{\textbackslash{}mathop\{ cos\} \}\^{}\{2\}t dt
=\{\textbackslash{}mathop\{∫ \} \}\_\{0\}\^{}\{π∕2\}\{
1+\textbackslash{}mathop\{cos\} (2t) \textbackslash{}over 2\} dt =\{ π
\textbackslash{}over 4\} en posant x =\textbackslash{}mathop\{ cos\} t.

Théorème~9.3.6 (intégration par parties). Soit f,g : {[}a,b{]} → ℂ de
classe \{C\}\^{}\{1\}~; alors

\{\textbackslash{}mathop\{∫ \} \}\_\{a\}\^{}\{b\}f(t)g'(t) dt =\{
\textbackslash{}left {[}f(t)g(t)\textbackslash{}right {]}\}\_\{
a\}\^{}\{b\} −\{\textbackslash{}mathop\{∫ \} \}\_\{a\}\^{}\{b\}f'(t)g(t)
dt

Démonstration Il suffit de remarquer que fg est une primitive de la
fonction continue f'g + fg' et que en conséquence
\{\textbackslash{}mathop\{∫ \} \}\_\{a\}\^{}\{b\}(fg' + f'g) =\{
\textbackslash{}left {[}f(t)g(t)\textbackslash{}right
{]}\}\_\{a\}\^{}\{b\}.

Remarque~9.3.5 Le résultat s'étend à n'importe quelle application
bilinéaire continue φ (produit scalaire, produit vectoriel, produit
matriciel) et on obtient la formule

\{\textbackslash{}mathop\{∫ \} \}\_\{a\}\^{}\{b\}φ(f(t),g'(t)) dt =\{
\textbackslash{}left {[}φ(f(t),g(t))\textbackslash{}right {]}\}\_\{
a\}\^{}\{b\} −\{\textbackslash{}mathop\{∫ \}
\}\_\{a\}\^{}\{b\}φ(f'(t),g(t)) dt

avec la même démonstration.

Corollaire~9.3.7 (formule de Taylor avec reste intégral). Soit f : I → E
de classe \{C\}\^{}\{n+1\} et a ∈ I. Alors \textbackslash{}mathop\{∀\}b
∈ I,

f(b) = f(a) +\{ \textbackslash{}mathop\{∑ \}\}\_\{k=1\}\^{}\{n\}\{
\{f\}\^{}\{(k)\}(a) \textbackslash{}over k!\} \{(b − a)\}\^{}\{k\} +\{
\textbackslash{}mathop\{\textbackslash{}mathop\{∫ \} \}
\}\_\{a\}\^{}\{b\}\{ \{(b − t)\}\^{}\{n\} \textbackslash{}over n!\}
\{f\}\^{}\{(n+1)\}(t) dt

Démonstration Par récurrence sur n. Si n = 0, il s'agit simplement de la
formule f(b) = f(a) +\{\textbackslash{}mathop\{∫ \}
\}\_\{a\}\^{}\{b\}f'(t) dt pour f de classe \{C\}\^{}\{1\}. De plus une
intégration par parties (en intégrant \{ \{(b−t)\}\^{}\{n−1\}
\textbackslash{}over (n−1)!\} et en dérivant \{f\}\^{}\{(n)\}(t)) donne

\textbackslash{}begin\{eqnarray*\} \{\textbackslash{}mathop\{∫ \}
\}\_\{a\}\^{}\{b\}\{ \{(b − t)\}\^{}\{n−1\} \textbackslash{}over (n −
1)!\} \{f\}\^{}\{(n)\}(t) dt\&\& \%\& \textbackslash{}\textbackslash{}
\& =\&\{ \textbackslash{}left {[}−\{ \{(b − t)\}\^{}\{n\}
\textbackslash{}over n!\} \{f\}\^{}\{(n)\}(t)\textbackslash{}right
{]}\}\_\{ a\}\^{}\{b\} +\{\textbackslash{}mathop\{∫ \}
\}\_\{a\}\^{}\{b\}\{ \{(b − t)\}\^{}\{n\} \textbackslash{}over n!\}
\{f\}\^{}\{(n+1)\}(t) dt\%\& \textbackslash{}\textbackslash{} \& =\&\{
\{f\}\^{}\{(n)\}(a) \textbackslash{}over n!\} \{(b − a)\}\^{}\{n\}
+\{\textbackslash{}mathop\{∫ \} \}\_\{a\}\^{}\{b\}\{ \{(b −
t)\}\^{}\{n\} \textbackslash{}over n!\} \{f\}\^{}\{(n+1)\}(t) dt \%\&
\textbackslash{}\textbackslash{} \textbackslash{}end\{eqnarray*\}

ce qui permet de passer de n − 1 à n.

\paragraph{9.3.4 Deuxième formule de la moyenne}

Théorème~9.3.8 (deuxième formule de la moyenne). Soit f,g : {[}a,b{]} →
ℝ. On suppose que f est de classe \{C\}\^{}\{1\}, positive et
décroissante et que g est continue. Alors, il existe c ∈ {[}a,b{]} tel
que \{\textbackslash{}mathop\{∫ \} \}\_\{a\}\^{}\{b\}fg =
f(a)\{\textbackslash{}mathop\{∫ \} \}\_\{a\}\^{}\{c\}g.

Démonstration Posons G(x) =\{\textbackslash{}mathop\{∫ \}
\}\_\{a\}\^{}\{x\}g. On sait que G est continue. L'image de {[}a,b{]}
par G est à la fois connexe et compact dans ℝ, c'est donc un segment de
ℝ. Soit G({[}a,b{]}) = {[}m,M{]}. On a \textbackslash{}mathop\{∀\}t ∈
{[}a,b{]},m ≤ G(t) ≤ M. Supposons démontré que mf(a)
≤\{\textbackslash{}mathop\{∫ \} \}\_\{a\}\^{}\{b\}fg ≤ Mf(a). Alors soit
f(a) = 0, auquel cas \{\textbackslash{}mathop\{∫ \} \}\_\{a\}\^{}\{b\}fg
= 0 et n'importe quel c convient, soit f(a)\textbackslash{}mathrel\{≠\}0
et donc \{ 1 \textbackslash{}over f(a)\} \{\textbackslash{}mathop\{∫ \}
\}\_\{a\}\^{}\{b\}fg ∈ {[}m,M{]} et donc \textbackslash{}mathop\{∃\}c ∈
{[}a,b{]},\{ 1 \textbackslash{}over f(a)\} \{\textbackslash{}mathop\{∫
\} \}\_\{a\}\^{}\{b\}fg = G(c), ce que l'on voulait démontrer. Nous
allons donc montrer que mf(a) ≤\{\textbackslash{}mathop\{∫ \}
\}\_\{a\}\^{}\{b\}fg ≤ Mf(a).

On peut faire une intégration par parties et on a (en tenant compte de
G(a) = 0)

\textbackslash{}begin\{eqnarray*\} \{\textbackslash{}mathop\{∫ \}
\}\_\{a\}\^{}\{b\}fg\& =\& \{\textbackslash{}mathop\{∫ \}
\}\_\{a\}\^{}\{b\}fG' =\{ \textbackslash{}left
{[}f(t)G(t)\textbackslash{}right {]}\}\_\{ a\}\^{}\{b\}
−\{\textbackslash{}mathop\{∫ \} \}\_\{a\}\^{}\{b\}f'(t)G(t) dt\%\&
\textbackslash{}\textbackslash{} \& =\& f(b)G(b)
+\{\textbackslash{}mathop\{∫ \} \}\_\{a\}\^{}\{b\}(−f'(t))G(t) dt \%\&
\textbackslash{}\textbackslash{} \textbackslash{}end\{eqnarray*\}

Comme − f' est positive, on peut appliquer la première formule de la
moyenne et il existe d ∈ {[}a,b{]} tel que \{\textbackslash{}mathop\{∫
\} \}\_\{a\}\^{}\{b\}(−f'(t))G(t) dt = G(d)\{\textbackslash{}mathop\{∫
\} \}\_\{a\}\^{}\{b\}(−f'(t)) dt = (f(a) − f(b))G(d). On a donc
\{\textbackslash{}mathop\{∫ \} \}\_\{a\}\^{}\{b\}fg = f(b)G(b) + (f(a) −
f(b))G(d). Mais m ≤ G(b) ≤ M, m ≤ G(d) ≤ M, f(b) ≥ 0 et f(a) − f(b) ≥ 0.
On a donc

mf(a) = f(b)m + (f(a) − f(b))m ≤\{\textbackslash{}mathop\{∫ \}
\}\_\{a\}\^{}\{b\}fg ≤ f(b)M + (f(a) − f(b))M = Mf(a)

ce qui achève la démonstration.

{[}\href{coursse53.html}{next}{]} {[}\href{coursse51.html}{prev}{]}
{[}\href{coursse51.html\#tailcoursse51.html}{prev-tail}{]}
{[}\href{coursse52.html}{front}{]}
{[}\href{coursch10.html\#coursse52.html}{up}{]}

\end{document}
