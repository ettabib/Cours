\documentclass[]{article}
\usepackage[T1]{fontenc}
\usepackage{lmodern}
\usepackage{amssymb,amsmath}
\usepackage{ifxetex,ifluatex}
\usepackage{fixltx2e} % provides \textsubscript
% use upquote if available, for straight quotes in verbatim environments
\IfFileExists{upquote.sty}{\usepackage{upquote}}{}
\ifnum 0\ifxetex 1\fi\ifluatex 1\fi=0 % if pdftex
  \usepackage[utf8]{inputenc}
\else % if luatex or xelatex
  \ifxetex
    \usepackage{mathspec}
    \usepackage{xltxtra,xunicode}
  \else
    \usepackage{fontspec}
  \fi
  \defaultfontfeatures{Mapping=tex-text,Scale=MatchLowercase}
  \newcommand{\euro}{€}
\fi
% use microtype if available
\IfFileExists{microtype.sty}{\usepackage{microtype}}{}
\ifxetex
  \usepackage[setpagesize=false, % page size defined by xetex
              unicode=false, % unicode breaks when used with xetex
              xetex]{hyperref}
\else
  \usepackage[unicode=true]{hyperref}
\fi
\hypersetup{breaklinks=true,
            bookmarks=true,
            pdfauthor={},
            pdftitle={Recherches de primitives},
            colorlinks=true,
            citecolor=blue,
            urlcolor=blue,
            linkcolor=magenta,
            pdfborder={0 0 0}}
\urlstyle{same}  % don't use monospace font for urls
\setlength{\parindent}{0pt}
\setlength{\parskip}{6pt plus 2pt minus 1pt}
\setlength{\emergencystretch}{3em}  % prevent overfull lines
\setcounter{secnumdepth}{0}
 
/* start css.sty */
.cmr-5{font-size:50%;}
.cmr-7{font-size:70%;}
.cmmi-5{font-size:50%;font-style: italic;}
.cmmi-7{font-size:70%;font-style: italic;}
.cmmi-10{font-style: italic;}
.cmsy-5{font-size:50%;}
.cmsy-7{font-size:70%;}
.cmex-7{font-size:70%;}
.cmex-7x-x-71{font-size:49%;}
.msbm-7{font-size:70%;}
.cmtt-10{font-family: monospace;}
.cmti-10{ font-style: italic;}
.cmbx-10{ font-weight: bold;}
.cmr-17x-x-120{font-size:204%;}
.cmsl-10{font-style: oblique;}
.cmti-7x-x-71{font-size:49%; font-style: italic;}
.cmbxti-10{ font-weight: bold; font-style: italic;}
p.noindent { text-indent: 0em }
td p.noindent { text-indent: 0em; margin-top:0em; }
p.nopar { text-indent: 0em; }
p.indent{ text-indent: 1.5em }
@media print {div.crosslinks {visibility:hidden;}}
a img { border-top: 0; border-left: 0; border-right: 0; }
center { margin-top:1em; margin-bottom:1em; }
td center { margin-top:0em; margin-bottom:0em; }
.Canvas { position:relative; }
li p.indent { text-indent: 0em }
.enumerate1 {list-style-type:decimal;}
.enumerate2 {list-style-type:lower-alpha;}
.enumerate3 {list-style-type:lower-roman;}
.enumerate4 {list-style-type:upper-alpha;}
div.newtheorem { margin-bottom: 2em; margin-top: 2em;}
.obeylines-h,.obeylines-v {white-space: nowrap; }
div.obeylines-v p { margin-top:0; margin-bottom:0; }
.overline{ text-decoration:overline; }
.overline img{ border-top: 1px solid black; }
td.displaylines {text-align:center; white-space:nowrap;}
.centerline {text-align:center;}
.rightline {text-align:right;}
div.verbatim {font-family: monospace; white-space: nowrap; text-align:left; clear:both; }
.fbox {padding-left:3.0pt; padding-right:3.0pt; text-indent:0pt; border:solid black 0.4pt; }
div.fbox {display:table}
div.center div.fbox {text-align:center; clear:both; padding-left:3.0pt; padding-right:3.0pt; text-indent:0pt; border:solid black 0.4pt; }
div.minipage{width:100%;}
div.center, div.center div.center {text-align: center; margin-left:1em; margin-right:1em;}
div.center div {text-align: left;}
div.flushright, div.flushright div.flushright {text-align: right;}
div.flushright div {text-align: left;}
div.flushleft {text-align: left;}
.underline{ text-decoration:underline; }
.underline img{ border-bottom: 1px solid black; margin-bottom:1pt; }
.framebox-c, .framebox-l, .framebox-r { padding-left:3.0pt; padding-right:3.0pt; text-indent:0pt; border:solid black 0.4pt; }
.framebox-c {text-align:center;}
.framebox-l {text-align:left;}
.framebox-r {text-align:right;}
span.thank-mark{ vertical-align: super }
span.footnote-mark sup.textsuperscript, span.footnote-mark a sup.textsuperscript{ font-size:80%; }
div.tabular, div.center div.tabular {text-align: center; margin-top:0.5em; margin-bottom:0.5em; }
table.tabular td p{margin-top:0em;}
table.tabular {margin-left: auto; margin-right: auto;}
div.td00{ margin-left:0pt; margin-right:0pt; }
div.td01{ margin-left:0pt; margin-right:5pt; }
div.td10{ margin-left:5pt; margin-right:0pt; }
div.td11{ margin-left:5pt; margin-right:5pt; }
table[rules] {border-left:solid black 0.4pt; border-right:solid black 0.4pt; }
td.td00{ padding-left:0pt; padding-right:0pt; }
td.td01{ padding-left:0pt; padding-right:5pt; }
td.td10{ padding-left:5pt; padding-right:0pt; }
td.td11{ padding-left:5pt; padding-right:5pt; }
table[rules] {border-left:solid black 0.4pt; border-right:solid black 0.4pt; }
.hline hr, .cline hr{ height : 1px; margin:0px; }
.tabbing-right {text-align:right;}
span.TEX {letter-spacing: -0.125em; }
span.TEX span.E{ position:relative;top:0.5ex;left:-0.0417em;}
a span.TEX span.E {text-decoration: none; }
span.LATEX span.A{ position:relative; top:-0.5ex; left:-0.4em; font-size:85%;}
span.LATEX span.TEX{ position:relative; left: -0.4em; }
div.float img, div.float .caption {text-align:center;}
div.figure img, div.figure .caption {text-align:center;}
.marginpar {width:20%; float:right; text-align:left; margin-left:auto; margin-top:0.5em; font-size:85%; text-decoration:underline;}
.marginpar p{margin-top:0.4em; margin-bottom:0.4em;}
.equation td{text-align:center; vertical-align:middle; }
td.eq-no{ width:5%; }
table.equation { width:100%; } 
div.math-display, div.par-math-display{text-align:center;}
math .texttt { font-family: monospace; }
math .textit { font-style: italic; }
math .textsl { font-style: oblique; }
math .textsf { font-family: sans-serif; }
math .textbf { font-weight: bold; }
.partToc a, .partToc, .likepartToc a, .likepartToc {line-height: 200%; font-weight:bold; font-size:110%;}
.chapterToc a, .chapterToc, .likechapterToc a, .likechapterToc, .appendixToc a, .appendixToc {line-height: 200%; font-weight:bold;}
.index-item, .index-subitem, .index-subsubitem {display:block}
.caption td.id{font-weight: bold; white-space: nowrap; }
table.caption {text-align:center;}
h1.partHead{text-align: center}
p.bibitem { text-indent: -2em; margin-left: 2em; margin-top:0.6em; margin-bottom:0.6em; }
p.bibitem-p { text-indent: 0em; margin-left: 2em; margin-top:0.6em; margin-bottom:0.6em; }
.paragraphHead, .likeparagraphHead { margin-top:2em; font-weight: bold;}
.subparagraphHead, .likesubparagraphHead { font-weight: bold;}
.quote {margin-bottom:0.25em; margin-top:0.25em; margin-left:1em; margin-right:1em; text-align:\jmathustify;}
.verse{white-space:nowrap; margin-left:2em}
div.maketitle {text-align:center;}
h2.titleHead{text-align:center;}
div.maketitle{ margin-bottom: 2em; }
div.author, div.date {text-align:center;}
div.thanks{text-align:left; margin-left:10%; font-size:85%; font-style:italic; }
div.author{white-space: nowrap;}
.quotation {margin-bottom:0.25em; margin-top:0.25em; margin-left:1em; }
h1.partHead{text-align: center}
.sectionToc, .likesectionToc {margin-left:2em;}
.subsectionToc, .likesubsectionToc {margin-left:4em;}
.subsubsectionToc, .likesubsubsectionToc {margin-left:6em;}
.frenchb-nbsp{font-size:75%;}
.frenchb-thinspace{font-size:75%;}
.figure img.graphics {margin-left:10%;}
/* end css.sty */

\title{Recherches de primitives}
\author{}
\date{}

\begin{document}
\maketitle

\textbf{Warning: 
requires JavaScript to process the mathematics on this page.\\ If your
browser supports JavaScript, be sure it is enabled.}

\begin{center}\rule{3in}{0.4pt}\end{center}

{[}
{[}
{[}{]}
{[}

\subsubsection{9.4 Recherches de primitives}

\paragraph{9.4.1 Position du problème}

Soit f une fonction de \mathbb{R}~ vers \mathbb{R}~ ou \mathbb{C}. On cherche à déterminer des
intervalles (maximaux) I sur lesquels f est continue et sur un tel
intervalle, une primitive F de f. La notation F(t)
=\int ~ f(t) dt + k, t \in I signifiera~: f est
continue sur I et F est une primitive de f sur I

Remarque~9.4.1 On prendra garde que dans cette notation, et
contrairement à la notation différentielle des intégrales, la variable t
n'est pas muette. C'est bien le même t qui figure dans F(t) et dans
\int ~ f(t) dt

\paragraph{9.4.2 Techniques usuelles}

Si F est une primitive de f sur I et si G est une primitive de g sur I,
alors \alpha~F + \beta~G est une primitive de \alpha~f + \beta~g sur I ce qu'on écrira

\int ~ (\alpha~f(t) + \beta~g(t)) dt =
\alpha~\int  f(t) dt + \beta~\\int ~
g(t) dt, t \in I

Sur le même modèle on écrira le théorème de changement de variables avec
\phi : I \rightarrow~ J de classe \mathcal{C}^1

\int ~ f(\phi(t))\phi'(t) dt =\\int
 f(u) du, u = \phi(t), t \in I

et le théorème d'intégrations par parties pour deux fonctions f et g de
classe \mathcal{C}^1

\int ~ f(t)g'(t) dt = f(t)g(t)
-\int ~ f'(t)g(t) dt, t \in I

théorèmes dont la démonstration est évidente.

\paragraph{9.4.3 Primitives usuelles}

On posera I\_n ={]} - \pi~ \over 2 + n\pi~, \pi~
\over 2 + n\pi~{[} et J\_n ={]}n\pi~,(n + 1)\pi~{[} pour
n \in \mathbb{N}~.

\array \int ~
cos t dt =\ sin~ t +
k, t \in \mathbb{R}~; &\int  \sin~ t
dt = -cos~ t + k, t \in \mathbb{R}~ \cr
\int ~  dt \over
cos ^2t~ =\
\mathrmtg t + k, t \in I\_n;
&\int ~  dt \over
sin ^2t~ =
-\mathrmcotg~ t + k, t \in
J\_n \cr \int ~  dt
\over cos t~
= log~ \left
\textbar{}\mathrmtg~ ( t
\over 2 + \pi~ \over 4
)\right \textbar{} + k, t \in
I\_n;&\int ~  dt \over
sin t =\ log~
\left
\textbar{}\mathrmtg~  t
\over 2 \right \textbar{}, t \in
J\_n \cr \int ~
\mathrmtg~ t dt =
-log~ \left
\textbar{}cos~ t\right
\textbar{} + k, t \in I\_n; &\int ~
\mathrmcotg~ t dt
= log~ \left
\textbar{}sin~ t\right
\textbar{}, t \in J\_n \cr
\int ~
\mathrmch~ t dt
= \mathrmsh~ t + k, t \in \mathbb{R}~;
&\int ~
\mathrmsh~ t dt
= \mathrmch~ t + k, t \in \mathbb{R}~
\cr \int ~  dt
\over
\mathrmch ^2t~
= \mathrmth~ t + k, t \in \mathbb{R}~;
&\int ~  dt \over
\mathrmsh ^2t~
= -coth~ t + k, t \in{]}
-\infty~,0{[}\text ou t \in{]}0,+\infty~{[} \cr
\int ~  dt \over
\mathrmch t~ =
2\mathrmarctg~
e^t + k, t \in \mathbb{R}~; &\int~  dt
\over
\mathrmsh t~
= log~ \left
\textbar{}\mathrmth~  t
\over 2 \right \textbar{}, t \in{]}
-\infty~,0{[}\text ou t \in{]}0,+\infty~{[} \cr
\int ~
\mathrmth~ t dt
= log~
\mathrmch~ t + k, t \in \mathbb{R}~;
&\int  \coth~ t dt
= log~ \left
\textbar{}\mathrmsh~
t\right \textbar{}, t \in{]} -\infty~,0{[}\text
ou t \in{]}0,+\infty~{[} \cr \int ~
t^\alpha~ dt = t^\alpha~+1 \over \alpha~+1 + k,
(\alpha~\neq~ - 1) &\int ~  dt
\over t = log~
\textbar{}t\textbar{} + k, t \in{]} -\infty~,0{[}\text ou t
\in{]}0,+\infty~{[} 

\begin{align*} \int ~  dt
\over t^2 + a^2 & =& 1
\over a
\mathrmarctg~  t
\over a , t \in \mathbb{R}~ \%& \\
\int   dt \over a^2~ -
t^2 & =& 1 \over 2a
log~ \left \textbar{} t + a
\over t - a \right \textbar{} = 1
\over a arg~
\mathrmth~  t
\over a ,t \in{]}
-\textbar{}a\textbar{},\textbar{}a\textbar{}{[}\text
pour la dernière expression  \%& \\
\int ~  dt \over
\sqrta^2  - t^2 & =&
arcsin~  t \over
\textbar{}a\textbar{} + k,t \in{]}
-\textbar{}a\textbar{},\textbar{}a\textbar{}{[} \%&
\\ \int ~  dt
\over \sqrtt^2  +
a^2 & =& arg~
\mathrmsh~  t
\over \textbar{}a\textbar{} + k
= log (t + \sqrtt~^2
 + a^2) + k', t \in \mathbb{R}~ \%& \\
\int ~  dt \over
\sqrtt^2  - a^2 & =&
log~ \left \textbar{}t +
\sqrtt^2  -
a^2\right \textbar{} + k =
\left \ \cases
arg~
\mathrmch~  t
\over \textbar{}a\textbar{} + k&si t
\in{]}\textbar{}a\textbar{},+\infty~{[} \cr
-arg~
\mathrmch~ 
\textbar{}t\textbar{} \over \textbar{}a\textbar{} +
k&si t \in{]} -\infty~,-\textbar{}a\textbar{}{[} \cr 
\right .\%&\\
\end{align*}

\paragraph{9.4.4 Fractions rationnelles}

On rappelle le résultat suivant

Théorème~9.4.1 Soit R(X) = A(X) \over B(X) une
fraction rationnelle à coefficients complexes, B(X) =
b\∏ ~
\_i=1^k(X - a\_i)^m\_i la
décomposition du dénominateur en facteurs du premier degré. Alors R(X)
s'écrit de manière unique sous la forme

R(X) = E(X) + \\sum
\_i=1^k\left ( \alpha~\_i,1
\over X - a\_i +
\ldots + \alpha~\_i,m\_i~
\over (X - a\_i)^m\_i
\right )

Démonstration E(X) est évidemment le quotient de la division euclidienne
de A(X) par B(X).

On montre que si A(X),B\_1(X),B\_2(X) sont trois
polynômes tels que B\_1(X) et B\_2(X) sont premiers
entre eux, alors il existe des polynômes U(X) et V (X) tels que

 A(X) \over B\_1(X)B\_2(X) = U(X)
\over B\_1(X) + V (X) \over
B\_2(X)

en effet puisque B\_1 et B\_2 sont premiers entre eux,
on a \mathbb{C}{[}X{]} = B\_1(X)\mathbb{C}{[}X{]} + B\_2(X)\mathbb{C}{[}X{]}, donc
A(X) peut s'écrire sous la forme A(X) = U(X)B\_2(X) + V
(X)B\_1(X) et en divisant par B\_1(X)B\_2(X) on
obtient la décomposition souhaitée. De plus, si un couple (U,V )
convient, il est clair que tout couple (U - B\_1Q,V +
B\_2Q) convient. En rempla\ccant U par le
reste de sa division euclidienne par B\_1, on peut donc supposer
que deg~ U \textless{}\
deg B\_1~; on voit alors immédiatement que si
deg~ A \textless{}\
deg B\_1B\_2, on a aussi deg~
V \textless{} deg B\_2~ (l'ensemble des
fractions rationnelles dont le degré du numérateur est strictement
inférieur au degré du numérateur est une sous algèbre de \mathbb{C}(X)). Une
récurrence évidente permet donc d'écrire

 A(X) \over B(X) = E(X) + \\sum
\_i=1^k A\_i(X) \over (X -
a\_i)^m\_i

avec deg A\_i~ \textless{}
m\_i. On écrit alors la formule de Taylor pour le polynôme
A\_i au point a\_i, soit A\_i(X) =
\alpha~\_i,m\_i + \alpha~\_i,m\_i-1(X -
a\_i) +
\\ldots~ +
\alpha~\_i,1(X - a\_i)^m\_i-1 (car
deg A\_i \leq m\_i~ - 1) d'où la
décomposition souhaitée. L'unicité de la décomposition découle
immédiatement du lemme suivant

Lemme~9.4.2 Le polynôme \alpha~\_i,1X +
\\ldots~ +
\alpha~\_i,m\_iX^m\_i est l'unique polynôme
P(X) sans terme constant tel que  A(X) \over B(X) -
P( 1 \over X-a\_i ) n'admette pas
a\_i comme pôle.

Démonstration Il est clair que ce polynôme convient. Si P\_1 et
P\_2 sont deux tels polynômes, alors (P\_1 -
P\_2)( 1 \over X-a\_i ) =
\left ( A(X) \over B(X) -
P\_2( 1 \over X-a\_i
)\right ) -\left ( A(X)
\over B(X) - P\_1( 1 \over
X-a\_i )\right ) est la différence de deux
fractions rationnelles qui n'admettent pas le pôle a\_i donc
c'est une fraction rationnelle qui n'admet pas le pôle a\_i.
Ceci n'est possible que si P\_1 - P\_2 est constant,
mais comme P\_1 et P\_2 sont sans terme constant, on a
P\_1 = P\_2.

Méthode de calcul E(X) est le quotient de la division euclidienne de
A(X) par B(X). En ce qui concerne les parties polaires  \alpha~\_i,1
\over X-a\_i +
\\ldots~ +
\alpha~\_i,m\_i \over
(X-a\_i)^m\_i on peut procéder de la
manière suivante~:

\begin{itemize}
\item
  si m\_i = 1 (pôle simple) on peut poser B(X) = (X -
  a\_i)B\_1(X)~; en multipliant les deux membres de la
  décomposition par X - a\_i et en substituant a\_i à X,
  on obtient (en remarquant que B'(X) = B\_1(X) + (X -
  a\_i)B\_1'(X))

  \alpha~\_i,1 = A(a\_i) \over
  B\_1(a\_i) = A(a\_i) \over
  B'(a\_i)
\item
  si m\_i \textgreater{} 1, on écrit  A(X+a\_i)
  \over B(X+a\_i) = P(X) \over
  X^m\_iQ(X) avec
  Q(0)\neq~0. On effectue la division suivant les
  puissances croissantes de P par Q à l'ordre m\_i, d'où P(X) =
  S(X)Q(X) + X^m\_iT(X) avec
  deg S \leq m\_i~ - 1. On obtient alors
   P(X) \over X^m\_iQ(X) = S(X)
  \over X^m\_i + T(X)
  \over Q(X) = \alpha~\_i,1 \over
  X + \\ldots~ +
  \alpha~\_i,m\_i \over
  X^m\_i + T(X) \over Q(X) et
  donc

   A(X) \over B(X) = \alpha~\_i,1
  \over X - a\_i +
  \\ldots~ +
  \alpha~\_i,m\_i \over (X -
  a\_i)^m\_i + T(X - a\_i)
  \over Q(X - a\_i)

  Comme  T(X-a\_i) \over Q(X-a\_i)
  n'admet pas a\_i comme pôle, c'est que l'on a déterminé la
  partie polaire relative au pôle a\_i.
\end{itemize}

Pour chercher une primitive d'une fraction rationnelle  A(X)
\over B(X) dont on connaît la décomposition en éléments
simples

R(X) = E(X) + \\sum
\_i=1^k\left ( \alpha~\_i,1
\over X - a\_i +
\ldots + \alpha~\_i,m\_i~
\over (X - a\_i)^m\_i
\right )

il suffit donc de savoir chercher une primitive du polynôme E(X) (ce qui
est élémentaire) et de chacun des éléments simples  1
\over (X-a\_i)^k .

Théorème~9.4.3 (i) Une primitive de t\mapsto~ 1
\over (t-a)^k ,
k\neq~1, est - 1 \over k-1 
1 \over (t-a)^k-1 (ii) Une primitive de
t\mapsto~ 1 \over t-a est
log~ \textbar{}t - a\textbar{} si a \in \mathbb{R}~,
log~ \textbar{}t - a\textbar{} +
i\mathrmarctg~ ( t-\alpha~
\over \beta~ ) si a = \alpha~ + i\beta~ \in \mathbb{C} \diagdown \mathbb{R}~.

Démonstration Le premier point et le deuxième sont évidents~; si a = \alpha~ +
i\beta~ \in \mathbb{C} \diagdown \mathbb{R}~, on écrit  1 \over t-a = 1
\over t-\alpha~-i\beta~ = t-\alpha~ \over
(t-\alpha~)^2+\beta~^2 + i \beta~ \over
(t-\alpha~)^2+\beta~^2 dont une primitive est  1
\over 2  log~ ((t -
\alpha~)^2 + \beta~^2) +
i\mathrmarctg~ ( t-\alpha~
\over \beta~ ).

\paragraph{9.4.5 Fractions rationnelles en sinus et cosinus}

On cherche une primitive d'une fonction du type f :
t\mapsto~R(cos~
t,sin~ t) où R est une fraction rationnelle.

Dans le cas où R est un polynôme, la linéarisation de f(t) en utilisant
les formules de trigonométrie et en particulier
cos t = e^it+e^-it~
\over 2 , sin~ t =
e^it-e^-it \over 2i permettra de
calculer une primitive.

Pour une fraction rationnelle, nous utiliserons à plusieurs reprises le
lemme suivant

Lemme~9.4.4 Soit R(X,Y ) une fraction rationnelle à deux variables.
Alors il existe deux fractions rationnelles R\_1 et R\_2
à deux variables telles que R(X,Y ) = R\_1(X^2,Y ) +
XR\_2(X^2,Y )

Démonstration On écrit, en séparant au dénominateur, les puissances
paires de X des puissances impaires,

\begin{align*} R(X,Y )& =& A(X,Y )
\over B\_1(X^2,Y ) +
XB\_2(X^2,Y ) \%& \\
& =& A(X,Y )(B\_1(X^2,Y ) -
XB\_2(X^2,Y )) \over
B\_1(X^2,Y )^2 -
X^2B\_2(X^2,Y )^2 \%&
\\ & =& C(X,Y ) \over
D(X^2,Y ) = C\_1(X^2,Y ) +
XC\_2(X^2,Y ) \over D(X^2,Y
) \%& \\ & =&
R\_1(X^2,Y ) + XR\_ 2(X^2,Y ) \%&
\\ \end{align*}

En appliquant ce lemme, nous constatons que nous pouvons écrire

\begin{align*} f(t)& =&
R\_1(cos~
^2t,sin~ t) +\
cos tR\_ 2(cos~
^2t,sin~ t) \%&
\\ & =& R\_1(1
- sin~
^2t,sin~ t) +\
cos t R\_ 2(1 - sin~
^2t,sin~ t)\%&
\\ & =&
f\_1(sin~ t) +\
cos t f\_2(sin~ t) \%&
\\ \end{align*}

où f\_1 et f\_2 sont des fractions rationnelles à une
variable. Si f\_1 = 0, on a alors

\int  f(t) dt =\\int ~
f\_2(sin~
t)cos t dt =\\int ~
f\_2(u) du

avec u = sin~ t. On est donc ramené à la
recherche d'une primitive de fraction rationnelle, ce que nous savons
faire. Or on constate facilement que, puisque
cos (\pi~ - t) = -\cos~ t
et sin (\pi~ - t) =\ sin~
t, on a f\_1 = 0 \Leftrightarrow
\forall~~t \in \mathbb{R}~, f(\pi~ - t) = -f(t).

De même on peut écrire f(t) = f\_3(cos~
t) + sin~
tf\_4(cos~ t) (en intervertissant le
rôle du sinus et du cosinus, ou en changeant t en  \pi~
\over 2 - t) et si f\_3 = 0, on a
\int  f(t) dt =\\int ~
f\_4(cos~
t)sin t dt = -\\int ~
f\_4(u) du avec u = cos~ t. Or comme ci
dessus, f\_3 = 0 \Leftrightarrow
\forall~~t \in \mathbb{R}~, f(-t) = -f(t).

Mais on peut encore écrire f(t) = R(cos~
t,sin t) = R(\cos~
t,\mathrmtg~
tcos t) = S(\cos~
t,\mathrmtg~ t) et en
appliquant de nouveau le lemme, f(t) =
R\_3(cos~
^2t,\mathrmtg~ t)
+ cos~
tR\_4(cos~
^2t,\mathrmtg~ t).
Mais cos ^2~t = 1
\over
1+\mathrmtg~
^2t ce qui permet d'écrire f(t) =
f\_5(\mathrmtg~ t)
+ cos~
tf\_6(\mathrmtg~ t).
Alors, si f\_6 = 0, le changement de variables u
= \mathrmtg~ t pour t \in{]}
- \pi~ \over 2 + n\pi~, \pi~ \over 2 +
n\pi~{[}, conduira à \int ~ f(t) dt
=\int ~
f\_5(\mathrmtg~ t)
dt =\int   f\_4~(u) \over
1+u^2 du, c'est-à-dire encore à une primitive de fraction
rationnelle. Or f\_6 = 0 \Leftrightarrow
\forall~~t \in \mathbb{R}~, f(t + \pi~) = f(t).

Dans tous les autres cas, le changement de variable u
= \mathrmtg~  t
\over 2 , t \in{]}(2n - 1)\pi~,(2n + 1)\pi~{[} conduit à

\int  R(\cos~
t,sin t) dt =\\int ~ R(
1 - u^2 \over 1 + u^2 , 2u
\over 1 + u^2 ) 2du \over 1
+ u^2

c'est-à-dire encore à une primitive de fraction rationnelle.

On déduit de cette étude que

Proposition~9.4.5 Soit f(t) une fraction rationnelle en
sin t et \cos~ t

\begin{itemize}
\itemsep1pt\parskip0pt\parsep0pt
\item
  (i) si \forall~~t \in \mathbb{R}~, f(\pi~ - t) = -f(t), le
  changement de variable u = sin~ t conduit à
  la recherche d'une primitive de fraction rationnelle
\item
  (ii) si \forall~~t \in \mathbb{R}~, f(-t) = -f(t), le changement
  de variable u = cos~ t conduit à la recherche
  d'une primitive de fraction rationnelle
\item
  (iii) si \forall~~t \in \mathbb{R}~, f(t + \pi~) = f(t), le
  changement de variable u =\
  \mathrmtg t, t \in{]} - \pi~ \over
  2 + n\pi~, \pi~ \over 2 + n\pi~{[}, conduit à la recherche
  d'une primitive de fraction rationnelle
\item
  (iv) dans tous les autres cas, le changement de variable u
  = \mathrmtg~  t
  \over 2 , t \in{]}(2n - 1)\pi~,(2n + 1)\pi~{[}, conduit à la
  recherche d'une primitive de fraction rationnelle.
\end{itemize}

Remarque~9.4.2 Les règles (i),(ii) et (iii) doivent tou\jmathours être
utilisées de préférence à la règle (iv) car elles conduisent à une
fraction rationnelle dont les degrés des numérateurs et dénominateurs
sont plus petits que dans la règle (iv). Le lecteur prendra garde à ne
pas appliquer les règles (iii) et (iv) en dehors de leurs intervalles de
validité respectifs (t \in{]} - \pi~ \over 2 + n\pi~, \pi~
\over 2 + n\pi~{[} ou t \in{]}(2n - 1)\pi~,(2n + 1)\pi~{[}) sous
peine d'erreurs difficilement décelables.

\paragraph{9.4.6 Fractions rationnelles en sinus et cosinus
hyperboliques}

On cherche une primitive d'une fonction du type f :
t\mapsto~R(\mathrmch~
t,\mathrmsh~ t) où R est une
fraction rationnelle.

Une première méthode est de rechercher le changement de variable que
l'on ferait pour calculer une primitive de g(t) =
R(cos t,\sin~ t)
(c'est-à-dire en transformant toutes les fonctions hyperboliques en
leurs analogues circulaires) et de faire le changement de variable
analogue u = \mathrmsh~ t, u
= \mathrmch~ t, u
= \mathrmth~ t ou u
= \mathrmth~  t
\over 2 .

Une deuxième méthode est de remarquer que f(t) est de la forme
S(e^t) où S est une fraction rationnelle à une variable. Le
changement de variable u = e^t conduit alors à
\int  f(t) dt =\\int ~
S(e^t) dt =\int ~  S(u)
\over u du c'est-à-dire encore à une primitive de
fraction rationnelle.

\paragraph{9.4.7 Intégrales abéliennes}

On cherche une primitive d'une fonction du type g :
x\mapsto~R(x,f(x)) où R est une fraction rationnelle
et f une fonction telle que la courbe d'équation y = f(x) puisse être
paramétrée par x = \phi(t),y = \psi(t) où \phi et \psi sont des fractions
rationnelles (où éventuellement des fonctions trigonométriques).

On a alors \int ~ g(x) dx
=\int  R(x,f(x)) dx =\\int ~
R(\phi(t),\psi(t))\phi'(t) dt par le changement de variable x = \phi(t) ce qui
conduit donc à une primitive de fractions rationnelles~; le paramètre t
doit varier de telle sorte que y = f(x) \Leftrightarrow x =
\phi(t), y = \psi(t)

Le cas le plus important est le cas des intégrales abéliennes où f est
une fonction algébrique~; autrement dit où la courbe y = f(x) est une
partie d'une courbe algébrique \Gamma d'équation P(x,y) = 0 où P est un
polynôme à deux variables. Une telle courbe, paramétrable par deux
fractions rationnelles x = \phi(t),y = \psi(t) est appelée une courbe
unicursale.

Remarque~9.4.3 L'exemple le plus simple de courbe algébrique non
unicursale est une courbe elliptique d'équation y^2 =
x^3 + px + q~; c'est ainsi que le calcul des primitives du
type \int  R(x,\sqrtx~^3
 + px + q) dx ne relèvera pas en général de la théorie précédente.

Nous allons étudier tout particulièrement deux exemples de fonctions
algébriques f.

Premier exemple~: f(x) = \rootn
\ofax+b \over cx+d  avec ad -
bc\neq~0. La courbe \Gamma est alors la courbe (cx +
d)y^n - (ax + b) = 0. On peut la paramétrer en posant y = t
auquel cas on obtient x = dt^n-b \over
-ct^n+a ~; d'où dx = nt^n-1 ad-bc
\over (ct^n-a)^2 . On obtient
donc

\int ~ R(x,\rootn
\ofax + b \over cx + d ) dx
=\int  R( dt^n~ - b
\over -ct^n + a ,t)nt^n-1 ad -
bc \over (ct^n - a)^2 dt

en posant t = \rootn \ofax+b
\over cx+d  ce qui conduit à la recherche d'une
primitive de fraction rationnelle.

Deuxième exemple~: f(x) = \sqrtax^2  + bx +
c avec a\neq~0 (sinon on retombe sur l'exemple
précédent avec n = 2, c = 0 et d = 1). La courbe \Gamma est alors la courbe
d'équation y^2 = ax^2 + bx + c, il s'agit soit
d'une ellipse (si a \textless{} 0) soit d'une hyperbole (si a
\textgreater{} 0). Bien entendu on doit se limiter à la portion de cette
conique située dans le demi plan supérieur~: y ≥ 0. Introduisons \Delta =
b^2 - 4ac que l'on peut manifestement supposer non nul, car
sinon ax^2 + bx + c est un carré parfait.

Premier cas~: a \textless{} 0~; on peut se limiter cas où \Delta
\textgreater{} 0 car sinon \forall~~x \in \mathbb{R}~,
ax^2 + bx + c \textless{} 0 et la fonction n'est \jmathamais
définie. On écrit ax^2 + bx + c = a(x - \alpha~)(x - \beta~) = a((x -
p)^2 - q^2) en introduisant d'une part les racines \alpha~
et \beta~ du trinome, d'autre part sa forme canonique. La fonction f est
définie sur {[}\alpha~,\beta~{]}.

Une première manière de paramétrer \Gamma est d'écrire son équation sous la
forme (x - p)^2 + y^2 \over
\textbar{}a\textbar{} = q^2 ce qui conduit au paramétrage x
- p = qcos~ t, y =
q\sqrt\textbar{}a\textbar{}sin~
t et donc à \int ~
R(x,\sqrtax^2  + bx + c) dx
=\int  R(p + q\cos~
t,q\sqrt\textbar{}a\textbar{}sin~
t)(-qsin~ t) dt, fraction rationnelle en
sin et \cos~ ~; le
paramètre t varie dans {[}0,\pi~{]} de telle manière que y ≥ 0.

Une deuxième manière est de couper l'ellipse \Gamma par une droite variable
passant par un point de l'ellipse, par exemple le point (\alpha~,0). On pose
donc y = t(x - \alpha~). Ceci conduit à y^2 = t^2(x -
\alpha~)^2 = a(x - \alpha~)(x - \beta~), soit t^2(x - \alpha~) = a(x - \beta~),
soit x = \alpha~t^2-a\beta~ \over t^2-a ,
puis y = t(x - \alpha~) = at(\beta~-\alpha~) \over t^2-a ~;
on obtient ainsi un paramétrage unicursal de \Gamma et on aboutit à une
recherche de primitive de fraction rationnelle~; le paramètre t varie de
telle sorte que y ≥ 0, soit t ≥ 0.

Deuxième cas~: a \textgreater{} 0, \Delta \textless{} 0. La fonction f(x) =
\sqrtax^2  + bx + c est définie sur \mathbb{R}~. On
écrit ax^2 + bx + c = a((x - p)^2 +
q^2) en introduisant sa forme canonique.

Une première manière de paramétrer \Gamma est d'écrire son équation sous la
forme  y^2 \over a - (x - p)^2
= q^2 ce qui conduit au paramétrage x - p =
q\mathrmsh~ t, y =
q\sqrta\mathrmch~
t et donc à \int ~
R(x,\sqrtax^2  + bx + c) dx
=\int ~ R(p +
q\mathrmsh~
t,q\sqrta\mathrmch~
t)(q\mathrmch~ t) dt,
fraction rationnelle en
\mathrmsh~ et
\mathrmch~ ~; le paramètre t
varie dans \mathbb{R}~.

Une deuxième manière est de couper l'hyperbole \Gamma par une droite variable
parallèle à l'une de ses asymptotes (de telles droites ne coupant \Gamma
qu'en un seul point), par exemple y = \sqrtax + t. On
a alors y^2 = (\sqrtax + t)^2 =
ax^2 + bx + c soit 2tx\sqrta +
t^2 = bx + c soit encore x = c-t^2
\over 2t\sqrta-b puis y =
\sqrtax + t =
\\ldots~~; on
aboutit à une recherche de primitive de fraction rationnelle~; le
paramètre t varie de telle sorte que y ≥ 0.

Troisième cas~: a \textgreater{} 0, \Delta \textgreater{} 0. On écrit
ax^2 + bx + c = a(x - \alpha~)(x - \beta~) = a((x - p)^2 -
q^2) en introduisant d'une part les racines \alpha~ et \beta~ du
trinome, d'autre part sa forme dite canonique. La fonction f est définie
sur {]} -\infty~,\alpha~{]} et sur {[}\beta~,+\infty~{[}.

Une première manière de paramétrer \Gamma est d'écrire son équation sous la
forme (x - p)^2 - y^2 \over a =
q^2 ce qui conduit au paramétrage x - p =
q\epsilon\mathrmch~ t, y =
q\sqrta\mathrmsh~
t, avec \epsilon = ±1 = sgn~(x - p), et donc à
\int  R(x,\sqrtax^2 ~
+ bx + c) dx =\int ~ R(p +
q\epsilon\mathrmch~
t,q\sqrta\mathrmsh~
t)(\epsilonq\mathrmsh~ t) dt,
fraction rationnelle en
\mathrmsh~ et
\mathrmch~ ~; le paramètre t
varie dans {[}0,+\infty~{[} de telle manière que y ≥ 0.

Une deuxième manière est de couper l'hyperbole \Gamma par une droite variable
passant par un point de l'hyperbole, par exemple le point (\alpha~,0). On pose
donc y = t(x - \alpha~). Ceci conduit à y^2 = t^2(x -
\alpha~)^2 = a(x - \alpha~)(x - \beta~), soit t^2(x - \alpha~) = a(x - \beta~),
soit x = \alpha~t^2-a\beta~ \over t^2-a ,
puis y = t(x - \alpha~) = at(\beta~-\alpha~) \over t^2-a ~;
on obtient ainsi un paramétrage unicursal de \Gamma et on aboutit à une
recherche de primitive de fraction rationnelle~; le paramètre t varie de
telle sorte que y ≥ 0.

Une troisième manière est de couper l'hyperbole \Gamma par une droite
variable parallèle à l'une de ses asymptotes (de telles droites ne
coupant \Gamma qu'en un seul point), par exemple y =
\sqrtax + t. On a alors y^2 =
(\sqrtax + t)^2 = ax^2 + bx + c
soit 2tx\sqrta + t^2 = bx + c soit encore
x = c-t^2 \over
2t\sqrta-b puis y = \sqrtax + t
= \\ldots~~; on
aboutit à une recherche de primitive de fraction rationnelle~; le
paramètre t varie de telle sorte que y ≥ 0.

{[}
{[}
{[}
{[}

\end{document}
