\documentclass[]{article}
\usepackage[T1]{fontenc}
\usepackage{lmodern}
\usepackage{amssymb,amsmath}
\usepackage{ifxetex,ifluatex}
\usepackage{fixltx2e} % provides \textsubscript
% use upquote if available, for straight quotes in verbatim environments
\IfFileExists{upquote.sty}{\usepackage{upquote}}{}
\ifnum 0\ifxetex 1\fi\ifluatex 1\fi=0 % if pdftex
  \usepackage[utf8]{inputenc}
\else % if luatex or xelatex
  \ifxetex
    \usepackage{mathspec}
    \usepackage{xltxtra,xunicode}
  \else
    \usepackage{fontspec}
  \fi
  \defaultfontfeatures{Mapping=tex-text,Scale=MatchLowercase}
  \newcommand{\euro}{€}
\fi
% use microtype if available
\IfFileExists{microtype.sty}{\usepackage{microtype}}{}
\ifxetex
  \usepackage[setpagesize=false, % page size defined by xetex
              unicode=false, % unicode breaks when used with xetex
              xetex]{hyperref}
\else
  \usepackage[unicode=true]{hyperref}
\fi
\hypersetup{breaklinks=true,
            bookmarks=true,
            pdfauthor={},
            pdftitle={Recherches de primitives},
            colorlinks=true,
            citecolor=blue,
            urlcolor=blue,
            linkcolor=magenta,
            pdfborder={0 0 0}}
\urlstyle{same}  % don't use monospace font for urls
\setlength{\parindent}{0pt}
\setlength{\parskip}{6pt plus 2pt minus 1pt}
\setlength{\emergencystretch}{3em}  % prevent overfull lines
\setcounter{secnumdepth}{0}
 
/* start css.sty */
.cmr-5{font-size:50%;}
.cmr-7{font-size:70%;}
.cmmi-5{font-size:50%;font-style: italic;}
.cmmi-7{font-size:70%;font-style: italic;}
.cmmi-10{font-style: italic;}
.cmsy-5{font-size:50%;}
.cmsy-7{font-size:70%;}
.cmex-7{font-size:70%;}
.cmex-7x-x-71{font-size:49%;}
.msbm-7{font-size:70%;}
.cmtt-10{font-family: monospace;}
.cmti-10{ font-style: italic;}
.cmbx-10{ font-weight: bold;}
.cmr-17x-x-120{font-size:204%;}
.cmsl-10{font-style: oblique;}
.cmti-7x-x-71{font-size:49%; font-style: italic;}
.cmbxti-10{ font-weight: bold; font-style: italic;}
p.noindent { text-indent: 0em }
td p.noindent { text-indent: 0em; margin-top:0em; }
p.nopar { text-indent: 0em; }
p.indent{ text-indent: 1.5em }
@media print {div.crosslinks {visibility:hidden;}}
a img { border-top: 0; border-left: 0; border-right: 0; }
center { margin-top:1em; margin-bottom:1em; }
td center { margin-top:0em; margin-bottom:0em; }
.Canvas { position:relative; }
li p.indent { text-indent: 0em }
.enumerate1 {list-style-type:decimal;}
.enumerate2 {list-style-type:lower-alpha;}
.enumerate3 {list-style-type:lower-roman;}
.enumerate4 {list-style-type:upper-alpha;}
div.newtheorem { margin-bottom: 2em; margin-top: 2em;}
.obeylines-h,.obeylines-v {white-space: nowrap; }
div.obeylines-v p { margin-top:0; margin-bottom:0; }
.overline{ text-decoration:overline; }
.overline img{ border-top: 1px solid black; }
td.displaylines {text-align:center; white-space:nowrap;}
.centerline {text-align:center;}
.rightline {text-align:right;}
div.verbatim {font-family: monospace; white-space: nowrap; text-align:left; clear:both; }
.fbox {padding-left:3.0pt; padding-right:3.0pt; text-indent:0pt; border:solid black 0.4pt; }
div.fbox {display:table}
div.center div.fbox {text-align:center; clear:both; padding-left:3.0pt; padding-right:3.0pt; text-indent:0pt; border:solid black 0.4pt; }
div.minipage{width:100%;}
div.center, div.center div.center {text-align: center; margin-left:1em; margin-right:1em;}
div.center div {text-align: left;}
div.flushright, div.flushright div.flushright {text-align: right;}
div.flushright div {text-align: left;}
div.flushleft {text-align: left;}
.underline{ text-decoration:underline; }
.underline img{ border-bottom: 1px solid black; margin-bottom:1pt; }
.framebox-c, .framebox-l, .framebox-r { padding-left:3.0pt; padding-right:3.0pt; text-indent:0pt; border:solid black 0.4pt; }
.framebox-c {text-align:center;}
.framebox-l {text-align:left;}
.framebox-r {text-align:right;}
span.thank-mark{ vertical-align: super }
span.footnote-mark sup.textsuperscript, span.footnote-mark a sup.textsuperscript{ font-size:80%; }
div.tabular, div.center div.tabular {text-align: center; margin-top:0.5em; margin-bottom:0.5em; }
table.tabular td p{margin-top:0em;}
table.tabular {margin-left: auto; margin-right: auto;}
div.td00{ margin-left:0pt; margin-right:0pt; }
div.td01{ margin-left:0pt; margin-right:5pt; }
div.td10{ margin-left:5pt; margin-right:0pt; }
div.td11{ margin-left:5pt; margin-right:5pt; }
table[rules] {border-left:solid black 0.4pt; border-right:solid black 0.4pt; }
td.td00{ padding-left:0pt; padding-right:0pt; }
td.td01{ padding-left:0pt; padding-right:5pt; }
td.td10{ padding-left:5pt; padding-right:0pt; }
td.td11{ padding-left:5pt; padding-right:5pt; }
table[rules] {border-left:solid black 0.4pt; border-right:solid black 0.4pt; }
.hline hr, .cline hr{ height : 1px; margin:0px; }
.tabbing-right {text-align:right;}
span.TEX {letter-spacing: -0.125em; }
span.TEX span.E{ position:relative;top:0.5ex;left:-0.0417em;}
a span.TEX span.E {text-decoration: none; }
span.LATEX span.A{ position:relative; top:-0.5ex; left:-0.4em; font-size:85%;}
span.LATEX span.TEX{ position:relative; left: -0.4em; }
div.float img, div.float .caption {text-align:center;}
div.figure img, div.figure .caption {text-align:center;}
.marginpar {width:20%; float:right; text-align:left; margin-left:auto; margin-top:0.5em; font-size:85%; text-decoration:underline;}
.marginpar p{margin-top:0.4em; margin-bottom:0.4em;}
.equation td{text-align:center; vertical-align:middle; }
td.eq-no{ width:5%; }
table.equation { width:100%; } 
div.math-display, div.par-math-display{text-align:center;}
math .texttt { font-family: monospace; }
math .textit { font-style: italic; }
math .textsl { font-style: oblique; }
math .textsf { font-family: sans-serif; }
math .textbf { font-weight: bold; }
.partToc a, .partToc, .likepartToc a, .likepartToc {line-height: 200%; font-weight:bold; font-size:110%;}
.chapterToc a, .chapterToc, .likechapterToc a, .likechapterToc, .appendixToc a, .appendixToc {line-height: 200%; font-weight:bold;}
.index-item, .index-subitem, .index-subsubitem {display:block}
.caption td.id{font-weight: bold; white-space: nowrap; }
table.caption {text-align:center;}
h1.partHead{text-align: center}
p.bibitem { text-indent: -2em; margin-left: 2em; margin-top:0.6em; margin-bottom:0.6em; }
p.bibitem-p { text-indent: 0em; margin-left: 2em; margin-top:0.6em; margin-bottom:0.6em; }
.paragraphHead, .likeparagraphHead { margin-top:2em; font-weight: bold;}
.subparagraphHead, .likesubparagraphHead { font-weight: bold;}
.quote {margin-bottom:0.25em; margin-top:0.25em; margin-left:1em; margin-right:1em; text-align:justify;}
.verse{white-space:nowrap; margin-left:2em}
div.maketitle {text-align:center;}
h2.titleHead{text-align:center;}
div.maketitle{ margin-bottom: 2em; }
div.author, div.date {text-align:center;}
div.thanks{text-align:left; margin-left:10%; font-size:85%; font-style:italic; }
div.author{white-space: nowrap;}
.quotation {margin-bottom:0.25em; margin-top:0.25em; margin-left:1em; }
h1.partHead{text-align: center}
.sectionToc, .likesectionToc {margin-left:2em;}
.subsectionToc, .likesubsectionToc {margin-left:4em;}
.subsubsectionToc, .likesubsubsectionToc {margin-left:6em;}
.frenchb-nbsp{font-size:75%;}
.frenchb-thinspace{font-size:75%;}
.figure img.graphics {margin-left:10%;}
/* end css.sty */

\title{Recherches de primitives}
\author{}
\date{}

\begin{document}
\maketitle

\textbf{Warning: \href{http://www.math.union.edu/locate/jsMath}{jsMath}
requires JavaScript to process the mathematics on this page.\\ If your
browser supports JavaScript, be sure it is enabled.}

\begin{center}\rule{3in}{0.4pt}\end{center}

{[}\href{coursse54.html}{next}{]} {[}\href{coursse52.html}{prev}{]}
{[}\href{coursse52.html\#tailcoursse52.html}{prev-tail}{]}
{[}\hyperref[tailcoursse53.html]{tail}{]}
{[}\href{coursch10.html\#coursse53.html}{up}{]}

\subsubsection{9.4 Recherches de primitives}

\paragraph{9.4.1 Position du problème}

Soit f une fonction de ℝ vers ℝ ou ℂ. On cherche à déterminer des
intervalles (maximaux) I sur lesquels f est continue et sur un tel
intervalle, une primitive F de f. La notation F(t)
=\textbackslash{}mathop\{∫ \} f(t) dt + k, t ∈ I signifiera~: f est
continue sur I et F est une primitive de f sur I

Remarque~9.4.1 On prendra garde que dans cette notation, et
contrairement à la notation différentielle des intégrales, la variable t
n'est pas muette. C'est bien le même t qui figure dans F(t) et dans
\textbackslash{}mathop\{∫ \} f(t) dt

\paragraph{9.4.2 Techniques usuelles}

Si F est une primitive de f sur I et si G est une primitive de g sur I,
alors αF + βG est une primitive de αf + βg sur I ce qu'on écrira

\textbackslash{}mathop\{∫ \} (αf(t) + βg(t)) dt =
α\textbackslash{}mathop\{∫ \} f(t) dt + β\textbackslash{}mathop\{∫ \}
g(t) dt, t ∈ I

Sur le même modèle on écrira le théorème de changement de variables avec
φ : I → J de classe \{C\}\^{}\{1\}

\textbackslash{}mathop\{∫ \} f(φ(t))φ'(t) dt =\textbackslash{}mathop\{∫
\} f(u) du, u = φ(t), t ∈ I

et le théorème d'intégrations par parties pour deux fonctions f et g de
classe \{C\}\^{}\{1\}

\textbackslash{}mathop\{∫ \} f(t)g'(t) dt = f(t)g(t)
−\textbackslash{}mathop\{∫ \} f'(t)g(t) dt, t ∈ I

théorèmes dont la démonstration est évidente.

\paragraph{9.4.3 Primitives usuelles}

On posera \{I\}\_\{n\} ={]} −\{ π \textbackslash{}over 2\} + nπ,\{ π
\textbackslash{}over 2\} + nπ{[} et \{J\}\_\{n\} ={]}nπ,(n + 1)π{[} pour
n ∈ ℕ.

\textbackslash{}array\{ \textbackslash{}mathop\{∫ \}
\textbackslash{}mathop\{cos\} t dt =\textbackslash{}mathop\{ sin\} t +
k, t ∈ ℝ; \&\textbackslash{}mathop\{∫ \} \textbackslash{}mathop\{sin\} t
dt = −\textbackslash{}mathop\{cos\} t + k, t ∈ ℝ \textbackslash{}cr
\textbackslash{}mathop\{∫ \} \{ dt \textbackslash{}over
\{\textbackslash{}mathop\{cos\} \}\^{}\{2\}t\} =\textbackslash{}mathop\{
\textbackslash{}mathrm\{tg\}\} t + k, t ∈ \{I\}\_\{n\};
\&\textbackslash{}mathop\{∫ \} \{ dt \textbackslash{}over
\{\textbackslash{}mathop\{sin\} \}\^{}\{2\}t\} =
−\textbackslash{}mathop\{\textbackslash{}mathrm\{cotg\}\} t + k, t ∈
\{J\}\_\{n\} \textbackslash{}cr \textbackslash{}mathop\{∫ \} \{ dt
\textbackslash{}over \textbackslash{}mathop\{cos\} t\}
=\textbackslash{}mathop\{ log\} \textbackslash{}left
\textbar{}\textbackslash{}mathop\{\textbackslash{}mathrm\{tg\}\} (\{ t
\textbackslash{}over 2\} +\{ π \textbackslash{}over 4\}
)\textbackslash{}right \textbar{} + k, t ∈
\{I\}\_\{n\};\&\textbackslash{}mathop\{∫ \} \{ dt \textbackslash{}over
\textbackslash{}mathop\{sin\} t\} =\textbackslash{}mathop\{ log\}
\textbackslash{}left
\textbar{}\textbackslash{}mathop\{\textbackslash{}mathrm\{tg\}\} \{ t
\textbackslash{}over 2\} \textbackslash{}right \textbar{}, t ∈
\{J\}\_\{n\} \textbackslash{}cr \textbackslash{}mathop\{∫ \}
\textbackslash{}mathop\{\textbackslash{}mathrm\{tg\}\} t dt =
−\textbackslash{}mathop\{log\} \textbackslash{}left
\textbar{}\textbackslash{}mathop\{cos\} t\textbackslash{}right
\textbar{} + k, t ∈ \{I\}\_\{n\}; \&\textbackslash{}mathop\{∫ \}
\textbackslash{}mathop\{\textbackslash{}mathrm\{cotg\}\} t dt
=\textbackslash{}mathop\{ log\} \textbackslash{}left
\textbar{}\textbackslash{}mathop\{sin\} t\textbackslash{}right
\textbar{}, t ∈ \{J\}\_\{n\} \textbackslash{}cr
\textbackslash{}mathop\{∫ \}
\textbackslash{}mathop\{\textbackslash{}mathrm\{ch\}\} t dt
=\textbackslash{}mathop\{ \textbackslash{}mathrm\{sh\}\} t + k, t ∈ ℝ;
\&\textbackslash{}mathop\{∫ \}
\textbackslash{}mathop\{\textbackslash{}mathrm\{sh\}\} t dt
=\textbackslash{}mathop\{ \textbackslash{}mathrm\{ch\}\} t + k, t ∈ ℝ
\textbackslash{}cr \textbackslash{}mathop\{∫ \} \{ dt
\textbackslash{}over
\{\textbackslash{}mathop\{\textbackslash{}mathrm\{ch\}\} \}\^{}\{2\}t\}
=\textbackslash{}mathop\{ \textbackslash{}mathrm\{th\}\} t + k, t ∈ ℝ;
\&\textbackslash{}mathop\{∫ \} \{ dt \textbackslash{}over
\{\textbackslash{}mathop\{\textbackslash{}mathrm\{sh\}\} \}\^{}\{2\}t\}
= −\textbackslash{}mathop\{coth\} t + k, t ∈{]}
−∞,0{[}\textbackslash{}text\{ ou \}t ∈{]}0,+∞{[} \textbackslash{}cr
\textbackslash{}mathop\{∫ \} \{ dt \textbackslash{}over
\textbackslash{}mathop\{\textbackslash{}mathrm\{ch\}\} t\} =
2\textbackslash{}mathop\{\textbackslash{}mathrm\{arctg\}\}
\{e\}\^{}\{t\} + k, t ∈ ℝ; \&\textbackslash{}mathop\{∫\} \{ dt
\textbackslash{}over
\textbackslash{}mathop\{\textbackslash{}mathrm\{sh\}\} t\}
=\textbackslash{}mathop\{ log\} \textbackslash{}left
\textbar{}\textbackslash{}mathop\{\textbackslash{}mathrm\{th\}\} \{ t
\textbackslash{}over 2\} \textbackslash{}right \textbar{}, t ∈{]}
−∞,0{[}\textbackslash{}text\{ ou \}t ∈{]}0,+∞{[} \textbackslash{}cr
\textbackslash{}mathop\{∫ \}
\textbackslash{}mathop\{\textbackslash{}mathrm\{th\}\} t dt
=\textbackslash{}mathop\{ log\}
\textbackslash{}mathop\{\textbackslash{}mathrm\{ch\}\} t + k, t ∈ ℝ;
\&\textbackslash{}mathop\{∫ \} \textbackslash{}mathop\{coth\} t dt
=\textbackslash{}mathop\{ log\} \textbackslash{}left
\textbar{}\textbackslash{}mathop\{\textbackslash{}mathrm\{sh\}\}
t\textbackslash{}right \textbar{}, t ∈{]} −∞,0{[}\textbackslash{}text\{
ou \}t ∈{]}0,+∞{[} \textbackslash{}cr \textbackslash{}mathop\{∫ \}
\{t\}\^{}\{α\} dt =\{ \{t\}\^{}\{α+1\} \textbackslash{}over α+1\} + k,
(α\textbackslash{}mathrel\{≠\} − 1) \&\textbackslash{}mathop\{∫ \} \{ dt
\textbackslash{}over t\} =\textbackslash{}mathop\{ log\}
\textbar{}t\textbar{} + k, t ∈{]} −∞,0{[}\textbackslash{}text\{ ou \}t
∈{]}0,+∞{[} \}

\textbackslash{}begin\{eqnarray*\} \textbackslash{}mathop\{∫ \} \{ dt
\textbackslash{}over \{t\}\^{}\{2\} + \{a\}\^{}\{2\}\} \& =\&\{ 1
\textbackslash{}over a\}
\textbackslash{}mathop\{\textbackslash{}mathrm\{arctg\}\} \{ t
\textbackslash{}over a\} , t ∈ ℝ \%\& \textbackslash{}\textbackslash{}
\textbackslash{}mathop\{∫ \} \{ dt \textbackslash{}over \{a\}\^{}\{2\} −
\{t\}\^{}\{2\}\} \& =\&\{ 1 \textbackslash{}over 2a\}
\textbackslash{}mathop\{log\} \textbackslash{}left \textbar{}\{ t + a
\textbackslash{}over t − a\} \textbackslash{}right \textbar{} =\{ 1
\textbackslash{}over a\} \textbackslash{}mathop\{arg\}
\textbackslash{}mathop\{\textbackslash{}mathrm\{th\}\} \{ t
\textbackslash{}over a\} ,t ∈{]}
−\textbar{}a\textbar{},\textbar{}a\textbar{}{[}\textbackslash{}text\{
pour la dernière expression \} \%\& \textbackslash{}\textbackslash{}
\textbackslash{}mathop\{∫ \} \{ dt \textbackslash{}over
\textbackslash{}sqrt\{\{a\}\^{}\{2 \} − \{t\}\^{}\{2\}\}\} \& =\&
\textbackslash{}mathop\{arcsin\} \{ t \textbackslash{}over
\textbar{}a\textbar{}\} + k,t ∈{]}
−\textbar{}a\textbar{},\textbar{}a\textbar{}{[} \%\&
\textbackslash{}\textbackslash{} \textbackslash{}mathop\{∫ \} \{ dt
\textbackslash{}over \textbackslash{}sqrt\{\{t\}\^{}\{2 \} +
\{a\}\^{}\{2\}\}\} \& =\& \textbackslash{}mathop\{arg\}
\textbackslash{}mathop\{\textbackslash{}mathrm\{sh\}\} \{ t
\textbackslash{}over \textbar{}a\textbar{}\} + k
=\textbackslash{}mathop\{ log\} (t + \textbackslash{}sqrt\{\{t\}\^{}\{2
\} + \{a\}\^{}\{2\}\}) + k', t ∈ ℝ \%\& \textbackslash{}\textbackslash{}
\textbackslash{}mathop\{∫ \} \{ dt \textbackslash{}over
\textbackslash{}sqrt\{\{t\}\^{}\{2 \} − \{a\}\^{}\{2\}\}\} \& =\&
\textbackslash{}mathop\{log\} \textbackslash{}left \textbar{}t +
\textbackslash{}sqrt\{\{t\}\^{}\{2 \} −
\{a\}\^{}\{2\}\}\textbackslash{}right \textbar{} + k =
\textbackslash{}left \textbackslash{}\{ \textbackslash{}cases\{
\textbackslash{}mathop\{arg\}
\textbackslash{}mathop\{\textbackslash{}mathrm\{ch\}\} \{ t
\textbackslash{}over \textbar{}a\textbar{}\} + k\&si t
∈{]}\textbar{}a\textbar{},+∞{[} \textbackslash{}cr
−\textbackslash{}mathop\{arg\}
\textbackslash{}mathop\{\textbackslash{}mathrm\{ch\}\} \{
\textbar{}t\textbar{} \textbackslash{}over \textbar{}a\textbar{}\} +
k\&si t ∈{]} −∞,−\textbar{}a\textbar{}{[} \textbackslash{}cr \}
\textbackslash{}right .\%\&\textbackslash{}\textbackslash{}
\textbackslash{}end\{eqnarray*\}

\paragraph{9.4.4 Fractions rationnelles}

On rappelle le résultat suivant

Théorème~9.4.1 Soit R(X) =\{ A(X) \textbackslash{}over B(X)\} une
fraction rationnelle à coefficients complexes, B(X) =
b\{\textbackslash{}mathop\{\textbackslash{}mathop\{∏ \}\}
\}\_\{i=1\}\^{}\{k\}\{(X − \{a\}\_\{i\})\}\^{}\{\{m\}\_\{i\}\} la
décomposition du dénominateur en facteurs du premier degré. Alors R(X)
s'écrit de manière unique sous la forme

R(X) = E(X) +\{ \textbackslash{}mathop\{∑
\}\}\_\{i=1\}\^{}\{k\}\textbackslash{}left (\{ \{α\}\_\{i,1\}
\textbackslash{}over X − \{a\}\_\{i\}\} +
\textbackslash{}mathop\{\ldots{}\} +\{ \{α\}\_\{i,\{m\}\_\{i\}\}
\textbackslash{}over \{(X − \{a\}\_\{i\})\}\^{}\{\{m\}\_\{i\}\}\}
\textbackslash{}right )

Démonstration E(X) est évidemment le quotient de la division euclidienne
de A(X) par B(X).

On montre que si A(X),\{B\}\_\{1\}(X),\{B\}\_\{2\}(X) sont trois
polynômes tels que \{B\}\_\{1\}(X) et \{B\}\_\{2\}(X) sont premiers
entre eux, alors il existe des polynômes U(X) et V (X) tels que

\{ A(X) \textbackslash{}over \{B\}\_\{1\}(X)\{B\}\_\{2\}(X)\} =\{ U(X)
\textbackslash{}over \{B\}\_\{1\}(X)\} +\{ V (X) \textbackslash{}over
\{B\}\_\{2\}(X)\}

en effet puisque \{B\}\_\{1\} et \{B\}\_\{2\} sont premiers entre eux,
on a ℂ{[}X{]} = \{B\}\_\{1\}(X)ℂ{[}X{]} + \{B\}\_\{2\}(X)ℂ{[}X{]}, donc
A(X) peut s'écrire sous la forme A(X) = U(X)\{B\}\_\{2\}(X) + V
(X)\{B\}\_\{1\}(X) et en divisant par \{B\}\_\{1\}(X)\{B\}\_\{2\}(X) on
obtient la décomposition souhaitée. De plus, si un couple (U,V )
convient, il est clair que tout couple (U − \{B\}\_\{1\}Q,V +
\{B\}\_\{2\}Q) convient. En rempla\textbackslash{}c\{c\}ant U par le
reste de sa division euclidienne par \{B\}\_\{1\}, on peut donc supposer
que \textbackslash{}mathop\{deg\} U \textless{}\textbackslash{}mathop\{
deg\} \{B\}\_\{1\}~; on voit alors immédiatement que si
\textbackslash{}mathop\{deg\} A \textless{}\textbackslash{}mathop\{
deg\} \{B\}\_\{1\}\{B\}\_\{2\}, on a aussi \textbackslash{}mathop\{deg\}
V \textless{}\textbackslash{}mathop\{ deg\} \{B\}\_\{2\} (l'ensemble des
fractions rationnelles dont le degré du numérateur est strictement
inférieur au degré du numérateur est une sous algèbre de ℂ(X)). Une
récurrence évidente permet donc d'écrire

\{ A(X) \textbackslash{}over B(X)\} = E(X) +\{ \textbackslash{}mathop\{∑
\}\}\_\{i=1\}\^{}\{k\}\{ \{A\}\_\{i\}(X) \textbackslash{}over \{(X −
\{a\}\_\{i\})\}\^{}\{\{m\}\_\{i\}\}\}

avec \textbackslash{}mathop\{deg\} \{A\}\_\{i\} \textless{}
\{m\}\_\{i\}. On écrit alors la formule de Taylor pour le polynôme
\{A\}\_\{i\} au point \{a\}\_\{i\}, soit \{A\}\_\{i\}(X) =
\{α\}\_\{i,\{m\}\_\{i\}\} + \{α\}\_\{i,\{m\}\_\{i\}−1\}(X −
\{a\}\_\{i\}) +
\textbackslash{}mathop\{\textbackslash{}mathop\{\ldots{}\}\} +
\{α\}\_\{i,1\}\{(X − \{a\}\_\{i\})\}\^{}\{\{m\}\_\{i\}−1\} (car
\textbackslash{}mathop\{deg\} \{A\}\_\{i\} ≤ \{m\}\_\{i\} − 1) d'où la
décomposition souhaitée. L'unicité de la décomposition découle
immédiatement du lemme suivant

Lemme~9.4.2 Le polynôme \{α\}\_\{i,1\}X +
\textbackslash{}mathop\{\textbackslash{}mathop\{\ldots{}\}\} +
\{α\}\_\{i,\{m\}\_\{i\}\}\{X\}\^{}\{\{m\}\_\{i\}\} est l'unique polynôme
P(X) sans terme constant tel que \{ A(X) \textbackslash{}over B(X)\} −
P(\{ 1 \textbackslash{}over X−\{a\}\_\{i\}\} ) n'admette pas
\{a\}\_\{i\} comme pôle.

Démonstration Il est clair que ce polynôme convient. Si \{P\}\_\{1\} et
\{P\}\_\{2\} sont deux tels polynômes, alors (\{P\}\_\{1\} −
\{P\}\_\{2\})(\{ 1 \textbackslash{}over X−\{a\}\_\{i\}\} ) =
\textbackslash{}left (\{ A(X) \textbackslash{}over B(X)\} −
\{P\}\_\{2\}(\{ 1 \textbackslash{}over X−\{a\}\_\{i\}\}
)\textbackslash{}right ) −\textbackslash{}left (\{ A(X)
\textbackslash{}over B(X)\} − \{P\}\_\{1\}(\{ 1 \textbackslash{}over
X−\{a\}\_\{i\}\} )\textbackslash{}right ) est la différence de deux
fractions rationnelles qui n'admettent pas le pôle \{a\}\_\{i\} donc
c'est une fraction rationnelle qui n'admet pas le pôle \{a\}\_\{i\}.
Ceci n'est possible que si \{P\}\_\{1\} − \{P\}\_\{2\} est constant,
mais comme \{P\}\_\{1\} et \{P\}\_\{2\} sont sans terme constant, on a
\{P\}\_\{1\} = \{P\}\_\{2\}.

Méthode de calcul E(X) est le quotient de la division euclidienne de
A(X) par B(X). En ce qui concerne les parties polaires \{ \{α\}\_\{i,1\}
\textbackslash{}over X−\{a\}\_\{i\}\} +
\textbackslash{}mathop\{\textbackslash{}mathop\{\ldots{}\}\} +\{
\{α\}\_\{i,\{m\}\_\{i\}\} \textbackslash{}over
\{(X−\{a\}\_\{i\})\}\^{}\{\{m\}\_\{i\}\}\} on peut procéder de la
manière suivante~:

\begin{itemize}
\item
  si \{m\}\_\{i\} = 1 (pôle simple) on peut poser B(X) = (X −
  \{a\}\_\{i\})\{B\}\_\{1\}(X)~; en multipliant les deux membres de la
  décomposition par X − \{a\}\_\{i\} et en substituant \{a\}\_\{i\} à X,
  on obtient (en remarquant que B'(X) = \{B\}\_\{1\}(X) + (X −
  \{a\}\_\{i\})\{B\}\_\{1\}'(X))

  \{α\}\_\{i,1\} =\{ A(\{a\}\_\{i\}) \textbackslash{}over
  \{B\}\_\{1\}(\{a\}\_\{i\})\} =\{ A(\{a\}\_\{i\}) \textbackslash{}over
  B'(\{a\}\_\{i\})\}
\item
  si \{m\}\_\{i\} \textgreater{} 1, on écrit \{ A(X+\{a\}\_\{i\})
  \textbackslash{}over B(X+\{a\}\_\{i\})\} =\{ P(X) \textbackslash{}over
  \{X\}\^{}\{\{m\}\_\{i\}\}Q(X)\} avec
  Q(0)\textbackslash{}mathrel\{≠\}0. On effectue la division suivant les
  puissances croissantes de P par Q à l'ordre \{m\}\_\{i\}, d'où P(X) =
  S(X)Q(X) + \{X\}\^{}\{\{m\}\_\{i\}\}T(X) avec
  \textbackslash{}mathop\{deg\} S ≤ \{m\}\_\{i\} − 1. On obtient alors
  \{ P(X) \textbackslash{}over \{X\}\^{}\{\{m\}\_\{i\}\}Q(X)\} =\{ S(X)
  \textbackslash{}over \{X\}\^{}\{\{m\}\_\{i\}\}\} +\{ T(X)
  \textbackslash{}over Q(X)\} =\{ \{α\}\_\{i,1\} \textbackslash{}over
  X\} + \textbackslash{}mathop\{\textbackslash{}mathop\{\ldots{}\}\} +\{
  \{α\}\_\{i,\{m\}\_\{i\}\} \textbackslash{}over
  \{X\}\^{}\{\{m\}\_\{i\}\}\} +\{ T(X) \textbackslash{}over Q(X)\} et
  donc

  \{ A(X) \textbackslash{}over B(X)\} =\{ \{α\}\_\{i,1\}
  \textbackslash{}over X − \{a\}\_\{i\}\} +
  \textbackslash{}mathop\{\textbackslash{}mathop\{\ldots{}\}\} +\{
  \{α\}\_\{i,\{m\}\_\{i\}\} \textbackslash{}over \{(X −
  \{a\}\_\{i\})\}\^{}\{\{m\}\_\{i\}\}\} +\{ T(X − \{a\}\_\{i\})
  \textbackslash{}over Q(X − \{a\}\_\{i\})\}

  Comme \{ T(X−\{a\}\_\{i\}) \textbackslash{}over Q(X−\{a\}\_\{i\})\}
  n'admet pas \{a\}\_\{i\} comme pôle, c'est que l'on a déterminé la
  partie polaire relative au pôle \{a\}\_\{i\}.
\end{itemize}

Pour chercher une primitive d'une fraction rationnelle \{ A(X)
\textbackslash{}over B(X)\} dont on connaît la décomposition en éléments
simples

R(X) = E(X) +\{ \textbackslash{}mathop\{∑
\}\}\_\{i=1\}\^{}\{k\}\textbackslash{}left (\{ \{α\}\_\{i,1\}
\textbackslash{}over X − \{a\}\_\{i\}\} +
\textbackslash{}mathop\{\ldots{}\} +\{ \{α\}\_\{i,\{m\}\_\{i\}\}
\textbackslash{}over \{(X − \{a\}\_\{i\})\}\^{}\{\{m\}\_\{i\}\}\}
\textbackslash{}right )

il suffit donc de savoir chercher une primitive du polynôme E(X) (ce qui
est élémentaire) et de chacun des éléments simples \{ 1
\textbackslash{}over \{(X−\{a\}\_\{i\})\}\^{}\{k\}\} .

Théorème~9.4.3 (i) Une primitive de t\textbackslash{}mathrel\{↦\}\{ 1
\textbackslash{}over \{(t−a)\}\^{}\{k\}\} ,
k\textbackslash{}mathrel\{≠\}1, est −\{ 1 \textbackslash{}over k−1\} \{
1 \textbackslash{}over \{(t−a)\}\^{}\{k−1\}\} (ii) Une primitive de
t\textbackslash{}mathrel\{↦\}\{ 1 \textbackslash{}over t−a\} est
\textbackslash{}mathop\{log\} \textbar{}t − a\textbar{} si a ∈ ℝ,
\textbackslash{}mathop\{log\} \textbar{}t − a\textbar{} +
i\textbackslash{}mathop\{\textbackslash{}mathrm\{arctg\}\} (\{ t−α
\textbackslash{}over β\} ) si a = α + iβ ∈ ℂ ∖ ℝ.

Démonstration Le premier point et le deuxième sont évidents~; si a = α +
iβ ∈ ℂ ∖ ℝ, on écrit \{ 1 \textbackslash{}over t−a\} =\{ 1
\textbackslash{}over t−α−iβ\} =\{ t−α \textbackslash{}over
\{(t−α)\}\^{}\{2\}+\{β\}\^{}\{2\}\} + i\{ β \textbackslash{}over
\{(t−α)\}\^{}\{2\}+\{β\}\^{}\{2\}\} dont une primitive est \{ 1
\textbackslash{}over 2\} \textbackslash{}mathop\{ log\} (\{(t −
α)\}\^{}\{2\} + \{β\}\^{}\{2\}) +
i\textbackslash{}mathop\{\textbackslash{}mathrm\{arctg\}\} (\{ t−α
\textbackslash{}over β\} ).

\paragraph{9.4.5 Fractions rationnelles en sinus et cosinus}

On cherche une primitive d'une fonction du type f :
t\textbackslash{}mathrel\{↦\}R(\textbackslash{}mathop\{cos\}
t,\textbackslash{}mathop\{sin\} t) où R est une fraction rationnelle.

Dans le cas où R est un polynôme, la linéarisation de f(t) en utilisant
les formules de trigonométrie et en particulier
\textbackslash{}mathop\{cos\} t =\{ \{e\}\^{}\{it\}+\{e\}\^{}\{−it\}
\textbackslash{}over 2\} , \textbackslash{}mathop\{sin\} t =\{
\{e\}\^{}\{it\}−\{e\}\^{}\{−it\} \textbackslash{}over 2i\} permettra de
calculer une primitive.

Pour une fraction rationnelle, nous utiliserons à plusieurs reprises le
lemme suivant

Lemme~9.4.4 Soit R(X,Y ) une fraction rationnelle à deux variables.
Alors il existe deux fractions rationnelles \{R\}\_\{1\} et \{R\}\_\{2\}
à deux variables telles que R(X,Y ) = \{R\}\_\{1\}(\{X\}\^{}\{2\},Y ) +
X\{R\}\_\{2\}(\{X\}\^{}\{2\},Y )

Démonstration On écrit, en séparant au dénominateur, les puissances
paires de X des puissances impaires,

\textbackslash{}begin\{eqnarray*\} R(X,Y )\& =\&\{ A(X,Y )
\textbackslash{}over \{B\}\_\{1\}(\{X\}\^{}\{2\},Y ) +
X\{B\}\_\{2\}(\{X\}\^{}\{2\},Y )\} \%\& \textbackslash{}\textbackslash{}
\& =\&\{ A(X,Y )(\{B\}\_\{1\}(\{X\}\^{}\{2\},Y ) −
X\{B\}\_\{2\}(\{X\}\^{}\{2\},Y )) \textbackslash{}over
\{B\}\_\{1\}\{(\{X\}\^{}\{2\},Y )\}\^{}\{2\} −
\{X\}\^{}\{2\}\{B\}\_\{2\}\{(\{X\}\^{}\{2\},Y )\}\^{}\{2\}\} \%\&
\textbackslash{}\textbackslash{} \& =\&\{ C(X,Y ) \textbackslash{}over
D(\{X\}\^{}\{2\},Y )\} =\{ \{C\}\_\{1\}(\{X\}\^{}\{2\},Y ) +
X\{C\}\_\{2\}(\{X\}\^{}\{2\},Y ) \textbackslash{}over D(\{X\}\^{}\{2\},Y
)\} \%\& \textbackslash{}\textbackslash{} \& =\&
\{R\}\_\{1\}(\{X\}\^{}\{2\},Y ) + X\{R\}\_\{ 2\}(\{X\}\^{}\{2\},Y ) \%\&
\textbackslash{}\textbackslash{} \textbackslash{}end\{eqnarray*\}

En appliquant ce lemme, nous constatons que nous pouvons écrire

\textbackslash{}begin\{eqnarray*\} f(t)\& =\&
\{R\}\_\{1\}(\{\textbackslash{}mathop\{cos\}
\}\^{}\{2\}t,\textbackslash{}mathop\{sin\} t) +\textbackslash{}mathop\{
cos\} t\{R\}\_\{ 2\}(\{\textbackslash{}mathop\{cos\}
\}\^{}\{2\}t,\textbackslash{}mathop\{sin\} t) \%\&
\textbackslash{}\textbackslash{} \& =\& \{R\}\_\{1\}(1
−\{\textbackslash{}mathop\{ sin\}
\}\^{}\{2\}t,\textbackslash{}mathop\{sin\} t) +\textbackslash{}mathop\{
cos\} t \{R\}\_\{ 2\}(1 −\{\textbackslash{}mathop\{ sin\}
\}\^{}\{2\}t,\textbackslash{}mathop\{sin\} t)\%\&
\textbackslash{}\textbackslash{} \& =\&
\{f\}\_\{1\}(\textbackslash{}mathop\{sin\} t) +\textbackslash{}mathop\{
cos\} t \{f\}\_\{2\}(\textbackslash{}mathop\{sin\} t) \%\&
\textbackslash{}\textbackslash{} \textbackslash{}end\{eqnarray*\}

où \{f\}\_\{1\} et \{f\}\_\{2\} sont des fractions rationnelles à une
variable. Si \{f\}\_\{1\} = 0, on a alors

\textbackslash{}mathop\{∫ \} f(t) dt =\textbackslash{}mathop\{∫ \}
\{f\}\_\{2\}(\textbackslash{}mathop\{sin\}
t)\textbackslash{}mathop\{cos\} t dt =\textbackslash{}mathop\{∫ \}
\{f\}\_\{2\}(u) du

avec u =\textbackslash{}mathop\{ sin\} t. On est donc ramené à la
recherche d'une primitive de fraction rationnelle, ce que nous savons
faire. Or on constate facilement que, puisque
\textbackslash{}mathop\{cos\} (π − t) = −\textbackslash{}mathop\{cos\} t
et \textbackslash{}mathop\{sin\} (π − t) =\textbackslash{}mathop\{ sin\}
t, on a \{f\}\_\{1\} = 0 \textbackslash{}mathrel\{⇔\}
\textbackslash{}mathop\{∀\}t ∈ ℝ, f(π − t) = −f(t).

De même on peut écrire f(t) = \{f\}\_\{3\}(\textbackslash{}mathop\{cos\}
t) +\textbackslash{}mathop\{ sin\}
t\{f\}\_\{4\}(\textbackslash{}mathop\{cos\} t) (en intervertissant le
rôle du sinus et du cosinus, ou en changeant t en \{ π
\textbackslash{}over 2\} − t) et si \{f\}\_\{3\} = 0, on a
\textbackslash{}mathop\{∫ \} f(t) dt =\textbackslash{}mathop\{∫ \}
\{f\}\_\{4\}(\textbackslash{}mathop\{cos\}
t)\textbackslash{}mathop\{sin\} t dt = −\textbackslash{}mathop\{∫ \}
\{f\}\_\{4\}(u) du avec u =\textbackslash{}mathop\{ cos\} t. Or comme ci
dessus, \{f\}\_\{3\} = 0 \textbackslash{}mathrel\{⇔\}
\textbackslash{}mathop\{∀\}t ∈ ℝ, f(−t) = −f(t).

Mais on peut encore écrire f(t) = R(\textbackslash{}mathop\{cos\}
t,\textbackslash{}mathop\{sin\} t) = R(\textbackslash{}mathop\{cos\}
t,\textbackslash{}mathop\{\textbackslash{}mathrm\{tg\}\}
t\textbackslash{}mathop\{cos\} t) = S(\textbackslash{}mathop\{cos\}
t,\textbackslash{}mathop\{\textbackslash{}mathrm\{tg\}\} t) et en
appliquant de nouveau le lemme, f(t) =
\{R\}\_\{3\}(\{\textbackslash{}mathop\{cos\}
\}\^{}\{2\}t,\textbackslash{}mathop\{\textbackslash{}mathrm\{tg\}\} t)
+\textbackslash{}mathop\{ cos\}
t\{R\}\_\{4\}(\{\textbackslash{}mathop\{cos\}
\}\^{}\{2\}t,\textbackslash{}mathop\{\textbackslash{}mathrm\{tg\}\} t).
Mais \{\textbackslash{}mathop\{cos\} \}\^{}\{2\}t =\{ 1
\textbackslash{}over
1+\{\textbackslash{}mathop\{\textbackslash{}mathrm\{tg\}\}
\}\^{}\{2\}t\} ce qui permet d'écrire f(t) =
\{f\}\_\{5\}(\textbackslash{}mathop\{\textbackslash{}mathrm\{tg\}\} t)
+\textbackslash{}mathop\{ cos\}
t\{f\}\_\{6\}(\textbackslash{}mathop\{\textbackslash{}mathrm\{tg\}\} t).
Alors, si \{f\}\_\{6\} = 0, le changement de variables u
=\textbackslash{}mathop\{ \textbackslash{}mathrm\{tg\}\} t pour t ∈{]}
−\{ π \textbackslash{}over 2\} + nπ,\{ π \textbackslash{}over 2\} +
nπ{[}, conduira à \textbackslash{}mathop\{∫ \} f(t) dt
=\textbackslash{}mathop\{∫ \}
\{f\}\_\{5\}(\textbackslash{}mathop\{\textbackslash{}mathrm\{tg\}\} t)
dt =\textbackslash{}mathop\{∫ \} \{ \{f\}\_\{4\}(u) \textbackslash{}over
1+\{u\}\^{}\{2\}\} du, c'est-à-dire encore à une primitive de fraction
rationnelle. Or \{f\}\_\{6\} = 0 \textbackslash{}mathrel\{⇔\}
\textbackslash{}mathop\{∀\}t ∈ ℝ, f(t + π) = f(t).

Dans tous les autres cas, le changement de variable u
=\textbackslash{}mathop\{ \textbackslash{}mathrm\{tg\}\} \{ t
\textbackslash{}over 2\} , t ∈{]}(2n − 1)π,(2n + 1)π{[} conduit à

\textbackslash{}mathop\{∫ \} R(\textbackslash{}mathop\{cos\}
t,\textbackslash{}mathop\{sin\} t) dt =\textbackslash{}mathop\{∫ \} R(\{
1 − \{u\}\^{}\{2\} \textbackslash{}over 1 + \{u\}\^{}\{2\}\} ,\{ 2u
\textbackslash{}over 1 + \{u\}\^{}\{2\}\} )\{ 2du \textbackslash{}over 1
+ \{u\}\^{}\{2\}\}

c'est-à-dire encore à une primitive de fraction rationnelle.

On déduit de cette étude que

Proposition~9.4.5 Soit f(t) une fraction rationnelle en
\textbackslash{}mathop\{sin\} t et \textbackslash{}mathop\{cos\} t

\begin{itemize}
\itemsep1pt\parskip0pt\parsep0pt
\item
  (i) si \textbackslash{}mathop\{∀\}t ∈ ℝ, f(π − t) = −f(t), le
  changement de variable u =\textbackslash{}mathop\{ sin\} t conduit à
  la recherche d'une primitive de fraction rationnelle
\item
  (ii) si \textbackslash{}mathop\{∀\}t ∈ ℝ, f(−t) = −f(t), le changement
  de variable u =\textbackslash{}mathop\{ cos\} t conduit à la recherche
  d'une primitive de fraction rationnelle
\item
  (iii) si \textbackslash{}mathop\{∀\}t ∈ ℝ, f(t + π) = f(t), le
  changement de variable u =\textbackslash{}mathop\{
  \textbackslash{}mathrm\{tg\}\} t, t ∈{]} −\{ π \textbackslash{}over
  2\} + nπ,\{ π \textbackslash{}over 2\} + nπ{[}, conduit à la recherche
  d'une primitive de fraction rationnelle
\item
  (iv) dans tous les autres cas, le changement de variable u
  =\textbackslash{}mathop\{ \textbackslash{}mathrm\{tg\}\} \{ t
  \textbackslash{}over 2\} , t ∈{]}(2n − 1)π,(2n + 1)π{[}, conduit à la
  recherche d'une primitive de fraction rationnelle.
\end{itemize}

Remarque~9.4.2 Les règles (i),(ii) et (iii) doivent toujours être
utilisées de préférence à la règle (iv) car elles conduisent à une
fraction rationnelle dont les degrés des numérateurs et dénominateurs
sont plus petits que dans la règle (iv). Le lecteur prendra garde à ne
pas appliquer les règles (iii) et (iv) en dehors de leurs intervalles de
validité respectifs (t ∈{]} −\{ π \textbackslash{}over 2\} + nπ,\{ π
\textbackslash{}over 2\} + nπ{[} ou t ∈{]}(2n − 1)π,(2n + 1)π{[}) sous
peine d'erreurs difficilement décelables.

\paragraph{9.4.6 Fractions rationnelles en sinus et cosinus
hyperboliques}

On cherche une primitive d'une fonction du type f :
t\textbackslash{}mathrel\{↦\}R(\textbackslash{}mathop\{\textbackslash{}mathrm\{ch\}\}
t,\textbackslash{}mathop\{\textbackslash{}mathrm\{sh\}\} t) où R est une
fraction rationnelle.

Une première méthode est de rechercher le changement de variable que
l'on ferait pour calculer une primitive de g(t) =
R(\textbackslash{}mathop\{cos\} t,\textbackslash{}mathop\{sin\} t)
(c'est-à-dire en transformant toutes les fonctions hyperboliques en
leurs analogues circulaires) et de faire le changement de variable
analogue u =\textbackslash{}mathop\{ \textbackslash{}mathrm\{sh\}\} t, u
=\textbackslash{}mathop\{ \textbackslash{}mathrm\{ch\}\} t, u
=\textbackslash{}mathop\{ \textbackslash{}mathrm\{th\}\} t ou u
=\textbackslash{}mathop\{ \textbackslash{}mathrm\{th\}\} \{ t
\textbackslash{}over 2\} .

Une deuxième méthode est de remarquer que f(t) est de la forme
S(\{e\}\^{}\{t\}) où S est une fraction rationnelle à une variable. Le
changement de variable u = \{e\}\^{}\{t\} conduit alors à
\textbackslash{}mathop\{∫ \} f(t) dt =\textbackslash{}mathop\{∫ \}
S(\{e\}\^{}\{t\}) dt =\textbackslash{}mathop\{∫ \} \{ S(u)
\textbackslash{}over u\} du c'est-à-dire encore à une primitive de
fraction rationnelle.

\paragraph{9.4.7 Intégrales abéliennes}

On cherche une primitive d'une fonction du type g :
x\textbackslash{}mathrel\{↦\}R(x,f(x)) où R est une fraction rationnelle
et f une fonction telle que la courbe d'équation y = f(x) puisse être
paramétrée par x = φ(t),y = ψ(t) où φ et ψ sont des fractions
rationnelles (où éventuellement des fonctions trigonométriques).

On a alors \textbackslash{}mathop\{∫ \} g(x) dx
=\textbackslash{}mathop\{∫ \} R(x,f(x)) dx =\textbackslash{}mathop\{∫ \}
R(φ(t),ψ(t))φ'(t) dt par le changement de variable x = φ(t) ce qui
conduit donc à une primitive de fractions rationnelles~; le paramètre t
doit varier de telle sorte que y = f(x) \textbackslash{}mathrel\{⇔\} x =
φ(t), y = ψ(t)

Le cas le plus important est le cas des intégrales abéliennes où f est
une fonction algébrique~; autrement dit où la courbe y = f(x) est une
partie d'une courbe algébrique Γ d'équation P(x,y) = 0 où P est un
polynôme à deux variables. Une telle courbe, paramétrable par deux
fractions rationnelles x = φ(t),y = ψ(t) est appelée une courbe
unicursale.

Remarque~9.4.3 L'exemple le plus simple de courbe algébrique non
unicursale est une courbe elliptique d'équation \{y\}\^{}\{2\} =
\{x\}\^{}\{3\} + px + q~; c'est ainsi que le calcul des primitives du
type \textbackslash{}mathop\{∫ \} R(x,\textbackslash{}sqrt\{\{x\}\^{}\{3
\} + px + q\}) dx ne relèvera pas en général de la théorie précédente.

Nous allons étudier tout particulièrement deux exemples de fonctions
algébriques f.

Premier exemple~: f(x) = \textbackslash{}root\{n\{
\}\textbackslash{}of\{ax+b \textbackslash{}over cx+d\} \} avec ad −
bc\textbackslash{}mathrel\{≠\}0. La courbe Γ est alors la courbe (cx +
d)\{y\}\^{}\{n\} − (ax + b) = 0. On peut la paramétrer en posant y = t
auquel cas on obtient x =\{ d\{t\}\^{}\{n\}−b \textbackslash{}over
−c\{t\}\^{}\{n\}+a\} ~; d'où dx = n\{t\}\^{}\{n−1\}\{ ad−bc
\textbackslash{}over \{(c\{t\}\^{}\{n\}−a)\}\^{}\{2\}\} . On obtient
donc

\textbackslash{}mathop\{∫ \} R(x,\textbackslash{}root\{n\{
\}\textbackslash{}of\{ax + b \textbackslash{}over cx + d\} \}) dx
=\textbackslash{}mathop\{∫ \} R(\{ d\{t\}\^{}\{n\} − b
\textbackslash{}over −c\{t\}\^{}\{n\} + a\} ,t)n\{t\}\^{}\{n−1\}\{ ad −
bc \textbackslash{}over \{(c\{t\}\^{}\{n\} − a)\}\^{}\{2\}\} dt

en posant t = \textbackslash{}root\{n\{ \}\textbackslash{}of\{ax+b
\textbackslash{}over cx+d\} \} ce qui conduit à la recherche d'une
primitive de fraction rationnelle.

Deuxième exemple~: f(x) = \textbackslash{}sqrt\{a\{x\}\^{}\{2 \} + bx +
c\} avec a\textbackslash{}mathrel\{≠\}0 (sinon on retombe sur l'exemple
précédent avec n = 2, c = 0 et d = 1). La courbe Γ est alors la courbe
d'équation \{y\}\^{}\{2\} = a\{x\}\^{}\{2\} + bx + c, il s'agit soit
d'une ellipse (si a \textless{} 0) soit d'une hyperbole (si a
\textgreater{} 0). Bien entendu on doit se limiter à la portion de cette
conique située dans le demi plan supérieur~: y ≥ 0. Introduisons Δ =
\{b\}\^{}\{2\} − 4ac que l'on peut manifestement supposer non nul, car
sinon a\{x\}\^{}\{2\} + bx + c est un carré parfait.

Premier cas~: a \textless{} 0~; on peut se limiter cas où Δ
\textgreater{} 0 car sinon \textbackslash{}mathop\{∀\}x ∈ ℝ,
a\{x\}\^{}\{2\} + bx + c \textless{} 0 et la fonction n'est jamais
définie. On écrit a\{x\}\^{}\{2\} + bx + c = a(x − α)(x − β) = a(\{(x −
p)\}\^{}\{2\} − \{q\}\^{}\{2\}) en introduisant d'une part les racines α
et β du trinome, d'autre part sa forme canonique. La fonction f est
définie sur {[}α,β{]}.

Une première manière de paramétrer Γ est d'écrire son équation sous la
forme \{(x − p)\}\^{}\{2\} +\{ \{y\}\^{}\{2\} \textbackslash{}over
\textbar{}a\textbar{}\} = \{q\}\^{}\{2\} ce qui conduit au paramétrage x
− p = q\textbackslash{}mathop\{cos\} t, y =
q\textbackslash{}sqrt\{\textbar{}a\textbar{}\}\textbackslash{}mathop\{sin\}
t et donc à \textbackslash{}mathop\{∫ \}
R(x,\textbackslash{}sqrt\{a\{x\}\^{}\{2 \} + bx + c\}) dx
=\textbackslash{}mathop\{∫ \} R(p + q\textbackslash{}mathop\{cos\}
t,q\textbackslash{}sqrt\{\textbar{}a\textbar{}\}\textbackslash{}mathop\{sin\}
t)(−q\textbackslash{}mathop\{sin\} t) dt, fraction rationnelle en
\textbackslash{}mathop\{sin\} et \textbackslash{}mathop\{cos\} ~; le
paramètre t varie dans {[}0,π{]} de telle manière que y ≥ 0.

Une deuxième manière est de couper l'ellipse Γ par une droite variable
passant par un point de l'ellipse, par exemple le point (α,0). On pose
donc y = t(x − α). Ceci conduit à \{y\}\^{}\{2\} = \{t\}\^{}\{2\}\{(x −
α)\}\^{}\{2\} = a(x − α)(x − β), soit \{t\}\^{}\{2\}(x − α) = a(x − β),
soit x =\{ α\{t\}\^{}\{2\}−aβ \textbackslash{}over \{t\}\^{}\{2\}−a\} ,
puis y = t(x − α) =\{ at(β−α) \textbackslash{}over \{t\}\^{}\{2\}−a\} ~;
on obtient ainsi un paramétrage unicursal de Γ et on aboutit à une
recherche de primitive de fraction rationnelle~; le paramètre t varie de
telle sorte que y ≥ 0, soit t ≥ 0.

Deuxième cas~: a \textgreater{} 0, Δ \textless{} 0. La fonction f(x) =
\textbackslash{}sqrt\{a\{x\}\^{}\{2 \} + bx + c\} est définie sur ℝ. On
écrit a\{x\}\^{}\{2\} + bx + c = a(\{(x − p)\}\^{}\{2\} +
\{q\}\^{}\{2\}) en introduisant sa forme canonique.

Une première manière de paramétrer Γ est d'écrire son équation sous la
forme \{ \{y\}\^{}\{2\} \textbackslash{}over a\} − \{(x − p)\}\^{}\{2\}
= \{q\}\^{}\{2\} ce qui conduit au paramétrage x − p =
q\textbackslash{}mathop\{\textbackslash{}mathrm\{sh\}\} t, y =
q\textbackslash{}sqrt\{a\}\textbackslash{}mathop\{\textbackslash{}mathrm\{ch\}\}
t et donc à \textbackslash{}mathop\{∫ \}
R(x,\textbackslash{}sqrt\{a\{x\}\^{}\{2 \} + bx + c\}) dx
=\textbackslash{}mathop\{∫ \} R(p +
q\textbackslash{}mathop\{\textbackslash{}mathrm\{sh\}\}
t,q\textbackslash{}sqrt\{a\}\textbackslash{}mathop\{\textbackslash{}mathrm\{ch\}\}
t)(q\textbackslash{}mathop\{\textbackslash{}mathrm\{ch\}\} t) dt,
fraction rationnelle en
\textbackslash{}mathop\{\textbackslash{}mathrm\{sh\}\} et
\textbackslash{}mathop\{\textbackslash{}mathrm\{ch\}\} ~; le paramètre t
varie dans ℝ.

Une deuxième manière est de couper l'hyperbole Γ par une droite variable
parallèle à l'une de ses asymptotes (de telles droites ne coupant Γ
qu'en un seul point), par exemple y = \textbackslash{}sqrt\{a\}x + t. On
a alors \{y\}\^{}\{2\} = \{(\textbackslash{}sqrt\{a\}x + t)\}\^{}\{2\} =
a\{x\}\^{}\{2\} + bx + c soit 2tx\textbackslash{}sqrt\{a\} +
\{t\}\^{}\{2\} = bx + c soit encore x =\{ c−\{t\}\^{}\{2\}
\textbackslash{}over 2t\textbackslash{}sqrt\{a\}−b\} puis y =
\textbackslash{}sqrt\{a\}x + t =
\textbackslash{}mathop\{\textbackslash{}mathop\{\ldots{}\}\}~; on
aboutit à une recherche de primitive de fraction rationnelle~; le
paramètre t varie de telle sorte que y ≥ 0.

Troisième cas~: a \textgreater{} 0, Δ \textgreater{} 0. On écrit
a\{x\}\^{}\{2\} + bx + c = a(x − α)(x − β) = a(\{(x − p)\}\^{}\{2\} −
\{q\}\^{}\{2\}) en introduisant d'une part les racines α et β du
trinome, d'autre part sa forme dite canonique. La fonction f est définie
sur {]} −∞,α{]} et sur {[}β,+∞{[}.

Une première manière de paramétrer Γ est d'écrire son équation sous la
forme \{(x − p)\}\^{}\{2\} −\{ \{y\}\^{}\{2\} \textbackslash{}over a\} =
\{q\}\^{}\{2\} ce qui conduit au paramétrage x − p =
qε\textbackslash{}mathop\{\textbackslash{}mathrm\{ch\}\} t, y =
q\textbackslash{}sqrt\{a\}\textbackslash{}mathop\{\textbackslash{}mathrm\{sh\}\}
t, avec ε = ±1 =\textbackslash{}mathop\{ sgn\}(x − p), et donc à
\textbackslash{}mathop\{∫ \} R(x,\textbackslash{}sqrt\{a\{x\}\^{}\{2 \}
+ bx + c\}) dx =\textbackslash{}mathop\{∫ \} R(p +
qε\textbackslash{}mathop\{\textbackslash{}mathrm\{ch\}\}
t,q\textbackslash{}sqrt\{a\}\textbackslash{}mathop\{\textbackslash{}mathrm\{sh\}\}
t)(εq\textbackslash{}mathop\{\textbackslash{}mathrm\{sh\}\} t) dt,
fraction rationnelle en
\textbackslash{}mathop\{\textbackslash{}mathrm\{sh\}\} et
\textbackslash{}mathop\{\textbackslash{}mathrm\{ch\}\} ~; le paramètre t
varie dans {[}0,+∞{[} de telle manière que y ≥ 0.

Une deuxième manière est de couper l'hyperbole Γ par une droite variable
passant par un point de l'hyperbole, par exemple le point (α,0). On pose
donc y = t(x − α). Ceci conduit à \{y\}\^{}\{2\} = \{t\}\^{}\{2\}\{(x −
α)\}\^{}\{2\} = a(x − α)(x − β), soit \{t\}\^{}\{2\}(x − α) = a(x − β),
soit x =\{ α\{t\}\^{}\{2\}−aβ \textbackslash{}over \{t\}\^{}\{2\}−a\} ,
puis y = t(x − α) =\{ at(β−α) \textbackslash{}over \{t\}\^{}\{2\}−a\} ~;
on obtient ainsi un paramétrage unicursal de Γ et on aboutit à une
recherche de primitive de fraction rationnelle~; le paramètre t varie de
telle sorte que y ≥ 0.

Une troisième manière est de couper l'hyperbole Γ par une droite
variable parallèle à l'une de ses asymptotes (de telles droites ne
coupant Γ qu'en un seul point), par exemple y =
\textbackslash{}sqrt\{a\}x + t. On a alors \{y\}\^{}\{2\} =
\{(\textbackslash{}sqrt\{a\}x + t)\}\^{}\{2\} = a\{x\}\^{}\{2\} + bx + c
soit 2tx\textbackslash{}sqrt\{a\} + \{t\}\^{}\{2\} = bx + c soit encore
x =\{ c−\{t\}\^{}\{2\} \textbackslash{}over
2t\textbackslash{}sqrt\{a\}−b\} puis y = \textbackslash{}sqrt\{a\}x + t
= \textbackslash{}mathop\{\textbackslash{}mathop\{\ldots{}\}\}~; on
aboutit à une recherche de primitive de fraction rationnelle~; le
paramètre t varie de telle sorte que y ≥ 0.

{[}\href{coursse54.html}{next}{]} {[}\href{coursse52.html}{prev}{]}
{[}\href{coursse52.html\#tailcoursse52.html}{prev-tail}{]}
{[}\href{coursse53.html}{front}{]}
{[}\href{coursch10.html\#coursse53.html}{up}{]}

\end{document}
