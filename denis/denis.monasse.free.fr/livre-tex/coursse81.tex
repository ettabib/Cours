\documentclass[]{article}
\usepackage[T1]{fontenc}
\usepackage{lmodern}
\usepackage{amssymb,amsmath}
\usepackage{ifxetex,ifluatex}
\usepackage{fixltx2e} % provides \textsubscript
% use upquote if available, for straight quotes in verbatim environments
\IfFileExists{upquote.sty}{\usepackage{upquote}}{}
\ifnum 0\ifxetex 1\fi\ifluatex 1\fi=0 % if pdftex
  \usepackage[utf8]{inputenc}
\else % if luatex or xelatex
  \ifxetex
    \usepackage{mathspec}
    \usepackage{xltxtra,xunicode}
  \else
    \usepackage{fontspec}
  \fi
  \defaultfontfeatures{Mapping=tex-text,Scale=MatchLowercase}
  \newcommand{\euro}{€}
\fi
% use microtype if available
\IfFileExists{microtype.sty}{\usepackage{microtype}}{}
\ifxetex
  \usepackage[setpagesize=false, % page size defined by xetex
              unicode=false, % unicode breaks when used with xetex
              xetex]{hyperref}
\else
  \usepackage[unicode=true]{hyperref}
\fi
\hypersetup{breaklinks=true,
            bookmarks=true,
            pdfauthor={},
            pdftitle={Produit de convolution},
            colorlinks=true,
            citecolor=blue,
            urlcolor=blue,
            linkcolor=magenta,
            pdfborder={0 0 0}}
\urlstyle{same}  % don't use monospace font for urls
\setlength{\parindent}{0pt}
\setlength{\parskip}{6pt plus 2pt minus 1pt}
\setlength{\emergencystretch}{3em}  % prevent overfull lines
\setcounter{secnumdepth}{0}
 
/* start css.sty */
.cmr-5{font-size:50%;}
.cmr-7{font-size:70%;}
.cmmi-5{font-size:50%;font-style: italic;}
.cmmi-7{font-size:70%;font-style: italic;}
.cmmi-10{font-style: italic;}
.cmsy-5{font-size:50%;}
.cmsy-7{font-size:70%;}
.cmex-7{font-size:70%;}
.cmex-7x-x-71{font-size:49%;}
.msbm-7{font-size:70%;}
.cmtt-10{font-family: monospace;}
.cmti-10{ font-style: italic;}
.cmbx-10{ font-weight: bold;}
.cmr-17x-x-120{font-size:204%;}
.cmsl-10{font-style: oblique;}
.cmti-7x-x-71{font-size:49%; font-style: italic;}
.cmbxti-10{ font-weight: bold; font-style: italic;}
p.noindent { text-indent: 0em }
td p.noindent { text-indent: 0em; margin-top:0em; }
p.nopar { text-indent: 0em; }
p.indent{ text-indent: 1.5em }
@media print {div.crosslinks {visibility:hidden;}}
a img { border-top: 0; border-left: 0; border-right: 0; }
center { margin-top:1em; margin-bottom:1em; }
td center { margin-top:0em; margin-bottom:0em; }
.Canvas { position:relative; }
li p.indent { text-indent: 0em }
.enumerate1 {list-style-type:decimal;}
.enumerate2 {list-style-type:lower-alpha;}
.enumerate3 {list-style-type:lower-roman;}
.enumerate4 {list-style-type:upper-alpha;}
div.newtheorem { margin-bottom: 2em; margin-top: 2em;}
.obeylines-h,.obeylines-v {white-space: nowrap; }
div.obeylines-v p { margin-top:0; margin-bottom:0; }
.overline{ text-decoration:overline; }
.overline img{ border-top: 1px solid black; }
td.displaylines {text-align:center; white-space:nowrap;}
.centerline {text-align:center;}
.rightline {text-align:right;}
div.verbatim {font-family: monospace; white-space: nowrap; text-align:left; clear:both; }
.fbox {padding-left:3.0pt; padding-right:3.0pt; text-indent:0pt; border:solid black 0.4pt; }
div.fbox {display:table}
div.center div.fbox {text-align:center; clear:both; padding-left:3.0pt; padding-right:3.0pt; text-indent:0pt; border:solid black 0.4pt; }
div.minipage{width:100%;}
div.center, div.center div.center {text-align: center; margin-left:1em; margin-right:1em;}
div.center div {text-align: left;}
div.flushright, div.flushright div.flushright {text-align: right;}
div.flushright div {text-align: left;}
div.flushleft {text-align: left;}
.underline{ text-decoration:underline; }
.underline img{ border-bottom: 1px solid black; margin-bottom:1pt; }
.framebox-c, .framebox-l, .framebox-r { padding-left:3.0pt; padding-right:3.0pt; text-indent:0pt; border:solid black 0.4pt; }
.framebox-c {text-align:center;}
.framebox-l {text-align:left;}
.framebox-r {text-align:right;}
span.thank-mark{ vertical-align: super }
span.footnote-mark sup.textsuperscript, span.footnote-mark a sup.textsuperscript{ font-size:80%; }
div.tabular, div.center div.tabular {text-align: center; margin-top:0.5em; margin-bottom:0.5em; }
table.tabular td p{margin-top:0em;}
table.tabular {margin-left: auto; margin-right: auto;}
div.td00{ margin-left:0pt; margin-right:0pt; }
div.td01{ margin-left:0pt; margin-right:5pt; }
div.td10{ margin-left:5pt; margin-right:0pt; }
div.td11{ margin-left:5pt; margin-right:5pt; }
table[rules] {border-left:solid black 0.4pt; border-right:solid black 0.4pt; }
td.td00{ padding-left:0pt; padding-right:0pt; }
td.td01{ padding-left:0pt; padding-right:5pt; }
td.td10{ padding-left:5pt; padding-right:0pt; }
td.td11{ padding-left:5pt; padding-right:5pt; }
table[rules] {border-left:solid black 0.4pt; border-right:solid black 0.4pt; }
.hline hr, .cline hr{ height : 1px; margin:0px; }
.tabbing-right {text-align:right;}
span.TEX {letter-spacing: -0.125em; }
span.TEX span.E{ position:relative;top:0.5ex;left:-0.0417em;}
a span.TEX span.E {text-decoration: none; }
span.LATEX span.A{ position:relative; top:-0.5ex; left:-0.4em; font-size:85%;}
span.LATEX span.TEX{ position:relative; left: -0.4em; }
div.float img, div.float .caption {text-align:center;}
div.figure img, div.figure .caption {text-align:center;}
.marginpar {width:20%; float:right; text-align:left; margin-left:auto; margin-top:0.5em; font-size:85%; text-decoration:underline;}
.marginpar p{margin-top:0.4em; margin-bottom:0.4em;}
.equation td{text-align:center; vertical-align:middle; }
td.eq-no{ width:5%; }
table.equation { width:100%; } 
div.math-display, div.par-math-display{text-align:center;}
math .texttt { font-family: monospace; }
math .textit { font-style: italic; }
math .textsl { font-style: oblique; }
math .textsf { font-family: sans-serif; }
math .textbf { font-weight: bold; }
.partToc a, .partToc, .likepartToc a, .likepartToc {line-height: 200%; font-weight:bold; font-size:110%;}
.chapterToc a, .chapterToc, .likechapterToc a, .likechapterToc, .appendixToc a, .appendixToc {line-height: 200%; font-weight:bold;}
.index-item, .index-subitem, .index-subsubitem {display:block}
.caption td.id{font-weight: bold; white-space: nowrap; }
table.caption {text-align:center;}
h1.partHead{text-align: center}
p.bibitem { text-indent: -2em; margin-left: 2em; margin-top:0.6em; margin-bottom:0.6em; }
p.bibitem-p { text-indent: 0em; margin-left: 2em; margin-top:0.6em; margin-bottom:0.6em; }
.paragraphHead, .likeparagraphHead { margin-top:2em; font-weight: bold;}
.subparagraphHead, .likesubparagraphHead { font-weight: bold;}
.quote {margin-bottom:0.25em; margin-top:0.25em; margin-left:1em; margin-right:1em; text-align:\jmathustify;}
.verse{white-space:nowrap; margin-left:2em}
div.maketitle {text-align:center;}
h2.titleHead{text-align:center;}
div.maketitle{ margin-bottom: 2em; }
div.author, div.date {text-align:center;}
div.thanks{text-align:left; margin-left:10%; font-size:85%; font-style:italic; }
div.author{white-space: nowrap;}
.quotation {margin-bottom:0.25em; margin-top:0.25em; margin-left:1em; }
h1.partHead{text-align: center}
.sectionToc, .likesectionToc {margin-left:2em;}
.subsectionToc, .likesubsectionToc {margin-left:4em;}
.subsubsectionToc, .likesubsubsectionToc {margin-left:6em;}
.frenchb-nbsp{font-size:75%;}
.frenchb-thinspace{font-size:75%;}
.figure img.graphics {margin-left:10%;}
/* end css.sty */

\title{Produit de convolution}
\author{}
\date{}

\begin{document}
\maketitle

\textbf{Warning: 
requires JavaScript to process the mathematics on this page.\\ If your
browser supports JavaScript, be sure it is enabled.}

\begin{center}\rule{3in}{0.4pt}\end{center}

{[}
{[}
{[}{]}
{[}

\subsubsection{14.5 Produit de convolution}

\paragraph{14.5.1 Convolution de fonctions périodiques}

Définition~14.5.1 Soit f,g : \mathbb{R}~ \rightarrow~ \mathbb{C} continues par morceaux et périodiques
de période 2\pi~. On définit le produit de convolution de f et g comme la
fonction f ∗ g : \mathbb{R}~ \rightarrow~ \mathbb{C} définie par

\forall~~x \in \mathbb{R}~, f ∗ g(x) = 1 \over
2\pi~ \int  \_0^2\pi~~f(t)g(x - t) dt

Théorème~14.5.1

\begin{itemize}
\itemsep1pt\parskip0pt\parsep0pt
\item
  (i) la fonction f ∗ g est continue et périodique de période 2\pi~
\item
  (ii) l'application (f,g)\mapsto~f ∗ g est
  bilinéaire
\item
  (iii) le produit de convolution est commutatif~: g ∗ f = f ∗ g
\item
  (iv) le produit de convolution est associatif~: (f ∗ g) ∗ h = f ∗ (g ∗
  h)
\end{itemize}

Démonstration (i) On a f ∗ g(x + 2\pi~) = 1 \over 2\pi~
\int  \_0^2\pi~~f(t)g(x + 2\pi~ - t) dt
= 1 \over 2\pi~ \int ~
\_0^2\pi~f(t)g(x - t) dt = f ∗ g(x) puisque g est périodique
de période 2\pi~. Montrons la continuité de f ∗ g. Supposons tout d'abord
que f est en escalier et soit a\_0 = 0 \leq a\_1
\leq\\ldots~ \leq
a\_p = 2\pi~ une subdivision de {[}0,2\pi~{]} adaptée à f, si bien que
\forall~t \in{]}a\_i-1,a\_i~{[}, f(t) =
\lambda~\_i~; on a alors

\begin{align*} \int ~
\_0^2\pi~f(t)g(x - t) dt& =& \\sum
\_i=1^p\lambda~\_ i
\\int  ~
\_a\_i-1^a\_i g(x - t) dt \%&
\\ & =& -\\sum
\_i=1^p\lambda~\_ i
\\int  ~
\_x-a\_i-1^x-a\_i g(u) du\%&
\\ \end{align*}

en faisant le changement de variable u = x - t. Comme une intégrale de
fonction réglée dépend de fa\ccon continue des bornes
d'intégration, l'application
x\mapsto~\int ~
\_x-a\_i-1^x-a\_ig(u) du est continue et
donc f ∗ g est continue. Si maintenant f est continue par morceaux, soit
(f\_n) une suite d'applications en escalier qui converge
uniformément vers f. On a alors

\begin{align*} \textbar{}f ∗ g(x) - f\_n ∗
g(x)& =& \left \textbar{} 1 \over 2\pi~
\int  \_0^2\pi~(f(t) - f~\_
n(t))g(x - t) dt\right \textbar{}\%&
\\ & \leq& 1 \over 2\pi~
\int  \_0^2\pi~~\textbar{}f(t) -
f\_ n(t)\textbar{}\,\textbar{}g(x - t)\textbar{}
dt \%& \\ & \leq&
\\textbar{}f -
f\_n\\textbar{}\_\infty~\\textbar{}g\\textbar{}\_\infty~
\%& \\ \end{align*}

ce qui montre que la suite (f\_n ∗ g) converge uniformément vers
f ∗ g. Comme ces applications sont continues, il en est de même de f ∗
g.

(ii) est évident

(iii) On a, en faisant le changement de variable u = x - t et en
remarquant que la fonction intégrée étant périodique de période 2\pi~, son
intégrale sur le segment de longueur 2\pi~, {[}x - 2\pi~,x{]} est égale à
l'intégrale sur {[}0,2\pi~{]}

\begin{align*} f ∗ g(x)& =& 1
\over 2\pi~ \int ~
\_0^2\pi~f(t)g(x - t) dt \%& \\
& =& - 1 \over 2\pi~ \int ~
\_x^x-2\pi~f(x - u)g(u) du \%&
\\ & =& 1 \over 2\pi~
\int  \_x-2\pi~^x~f(x - u)g(u) du \%&
\\ & =& 1 \over 2\pi~
\int  \_0^2\pi~~f(x - u)g(u) du = g ∗
f(x)\%& \\
\end{align*}

Ceci démontre la commutativité.

(iv) On a

\begin{align*} (f ∗ g) ∗ h(x)& =& 1
\over 2\pi~ \int ~
\_0^2\pi~f ∗ g(t)h(x - t) dt \%&
\\ & =& 1 \over
4\pi~^2 \int  \_0^2\pi~~h(x
- t)\left (\int ~
\_0^2\pi~f(u)g(t - u) du\right ) dt\%&
\\ & =& 1 \over
4\pi~^2 \int ~
\_0^2\pi~\left (\int ~
\_0^2\pi~f(u)g(t - u)h(x - t) du\right ) dt\%&
\\ \end{align*}

Si f, g et h sont continues, le théorème de Fubini permet d'intervertir
les deux signes d'intégration et on obtient

\begin{align*} (f ∗ g) ∗ h(x)&& \%&
\\ & =& 1 \over
4\pi~^2 \int ~
\_0^2\pi~\left (\int ~
\_0^2\pi~f(u)g(t - u)h(x - t) dt\right ) du
\%& \\ & =& 1 \over
4\pi~^2 \int ~
\_0^2\pi~f(u)\left (\\int
 \_0^2\pi~g(t - u)h(x - t) dt\right ) du \%&
\\ & =& 1 \over
4\pi~^2 \int ~
\_0^2\pi~f(u)\left (\\int
 \_-u^2\pi~-ug(v)h(x - u - v) dv\right )
du\%& \\ & =& 1 \over
4\pi~^2 \int ~
\_0^2\pi~f(u)\left (\\int
 \_0^2\pi~g(v)h(x - u - v) dv\right ) du \%&
\\ & =& 1 \over 2\pi~
\int  \_0^2\pi~~f(u) g ∗ h(x - u) du =
f ∗ (g ∗ h)(x) \%& \\
\end{align*}

en faisant le changement de variable v = t - u, soit t = v + u, dans
l'intégrale interne et en utilisant le fait que la fonction est
périodique de période 2\pi~. Si f et g sont seulement continues par
morceaux, il suffit d'utiliser des subdivisions adaptées et de découper
les intégrales suivant ces subdivisions.

Théorème~14.5.2 Soit f et g des applications de \mathbb{R}~ dans \mathbb{C} périodiques de
période 2\pi~. On suppose que f est continue par morceaux et que g est de
classe C^k. Alors f ∗ g est de classe C^k et (f
∗ g)^(k) = f ∗ (g^(k)).

Démonstration Une récurrence évidente permet d'obtenir le résultat pour
k quelconque à partir de k = 1. Quitte à utiliser une subdivision de
{[}0,2\pi~{]} et à découper l'intégrale, il suffit de montrer que
x\mapsto~\int ~
\_a^bf(t)g(x - t) dt est de classe \mathcal{C}^1 lorsque f
est continue sur {[}a,b{]} et g de classe \mathcal{C}^1 sur \mathbb{R}~. Mais
l'application (x,t)\mapsto~f(t)g(x - t) admet une
dérivée partielle par rapport à x égale à  \partial~ \over \partial~x
(f(t)g(x - t)) = f(t)g'(x - t) qui est une fonction continue du couple
(x,t). Le théorème de dérivation des intégrales dépendant d'un paramètre
montre que x\mapsto~\int ~
\_a^bf(t)g(x - t) dt est de classe \mathcal{C}^1 et que

(f ∗ g)'(x) =\int  \_a^b~ \partial~
\over \partial~x (f(t)g(x - t)) dt =\\int
 \_a^bf(t)g'(x - t) dt

Ceci montre que f ∗ g est de classe \mathcal{C}^1 et que (f ∗ g)' = f ∗
(g').

\paragraph{14.5.2 Produit de convolution et séries de Fourier}

Théorème~14.5.3 Soit f et g des applications de \mathbb{R}~ dans \mathbb{C} périodiques de
période 2\pi~, continues par morceaux. Alors \forall~~n \in
ℤ, c\_n(f ∗ g) = c\_n(f)c\_n(g).

Démonstration On a pour f continue par morceaux,

\begin{align*} f ∗ e\_n(x)& =& 1
\over 2\pi~ \int ~
\_0^2\pi~f(t)e^in(x-t) dt = 1
\over 2\pi~ e^inx\int ~
\_0^2\pi~f(t)e^-int dt\%&
\\ & =& c\_n(f)e^inx
\%& \\ \end{align*}

On en déduit que

\begin{align*} c\_n(f ∗ g)& =& (f ∗ g) ∗
e\_n(0) = f ∗ (g ∗ e\_n)(0) \%&
\\ & =& f ∗
(c\_n(g)e\_n)(0) = c\_n(g)(f ∗ e\_n)(0)
= c\_n(g)c\_n(f)\%& \\
\end{align*}

Remarque~14.5.1 Pour une fonction g donnée, l'application
f\mapsto~f ∗ g se traduit donc comme un filtre sur
le signal f~: l'amplitude c\_n(f) de l'harmonique de f
correspondant à la fréquence n est multipliée par le coefficient
c\_n(g). Comme les c\_n(g) tendent vers 0 quand
\textbar{}n\textbar{} tend vers + \infty~, on voit qu'il ne peut exister
d'élément neutre pour le produit de convolution, c'est-à-dire de
fonction \epsilon telle que \forall~~f \inC, f ∗ \epsilon = f.

{[}
{[}
{[}
{[}

\end{document}
