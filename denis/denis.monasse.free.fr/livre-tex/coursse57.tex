\documentclass[]{article}
\usepackage[T1]{fontenc}
\usepackage{lmodern}
\usepackage{amssymb,amsmath}
\usepackage{ifxetex,ifluatex}
\usepackage{fixltx2e} % provides \textsubscript
% use upquote if available, for straight quotes in verbatim environments
\IfFileExists{upquote.sty}{\usepackage{upquote}}{}
\ifnum 0\ifxetex 1\fi\ifluatex 1\fi=0 % if pdftex
  \usepackage[utf8]{inputenc}
\else % if luatex or xelatex
  \ifxetex
    \usepackage{mathspec}
    \usepackage{xltxtra,xunicode}
  \else
    \usepackage{fontspec}
  \fi
  \defaultfontfeatures{Mapping=tex-text,Scale=MatchLowercase}
  \newcommand{\euro}{€}
\fi
% use microtype if available
\IfFileExists{microtype.sty}{\usepackage{microtype}}{}
\ifxetex
  \usepackage[setpagesize=false, % page size defined by xetex
              unicode=false, % unicode breaks when used with xetex
              xetex]{hyperref}
\else
  \usepackage[unicode=true]{hyperref}
\fi
\hypersetup{breaklinks=true,
            bookmarks=true,
            pdfauthor={},
            pdftitle={Generalites sur les integrales impropres},
            colorlinks=true,
            citecolor=blue,
            urlcolor=blue,
            linkcolor=magenta,
            pdfborder={0 0 0}}
\urlstyle{same}  % don't use monospace font for urls
\setlength{\parindent}{0pt}
\setlength{\parskip}{6pt plus 2pt minus 1pt}
\setlength{\emergencystretch}{3em}  % prevent overfull lines
\setcounter{secnumdepth}{0}
 
/* start css.sty */
.cmr-5{font-size:50%;}
.cmr-7{font-size:70%;}
.cmmi-5{font-size:50%;font-style: italic;}
.cmmi-7{font-size:70%;font-style: italic;}
.cmmi-10{font-style: italic;}
.cmsy-5{font-size:50%;}
.cmsy-7{font-size:70%;}
.cmex-7{font-size:70%;}
.cmex-7x-x-71{font-size:49%;}
.msbm-7{font-size:70%;}
.cmtt-10{font-family: monospace;}
.cmti-10{ font-style: italic;}
.cmbx-10{ font-weight: bold;}
.cmr-17x-x-120{font-size:204%;}
.cmsl-10{font-style: oblique;}
.cmti-7x-x-71{font-size:49%; font-style: italic;}
.cmbxti-10{ font-weight: bold; font-style: italic;}
p.noindent { text-indent: 0em }
td p.noindent { text-indent: 0em; margin-top:0em; }
p.nopar { text-indent: 0em; }
p.indent{ text-indent: 1.5em }
@media print {div.crosslinks {visibility:hidden;}}
a img { border-top: 0; border-left: 0; border-right: 0; }
center { margin-top:1em; margin-bottom:1em; }
td center { margin-top:0em; margin-bottom:0em; }
.Canvas { position:relative; }
li p.indent { text-indent: 0em }
.enumerate1 {list-style-type:decimal;}
.enumerate2 {list-style-type:lower-alpha;}
.enumerate3 {list-style-type:lower-roman;}
.enumerate4 {list-style-type:upper-alpha;}
div.newtheorem { margin-bottom: 2em; margin-top: 2em;}
.obeylines-h,.obeylines-v {white-space: nowrap; }
div.obeylines-v p { margin-top:0; margin-bottom:0; }
.overline{ text-decoration:overline; }
.overline img{ border-top: 1px solid black; }
td.displaylines {text-align:center; white-space:nowrap;}
.centerline {text-align:center;}
.rightline {text-align:right;}
div.verbatim {font-family: monospace; white-space: nowrap; text-align:left; clear:both; }
.fbox {padding-left:3.0pt; padding-right:3.0pt; text-indent:0pt; border:solid black 0.4pt; }
div.fbox {display:table}
div.center div.fbox {text-align:center; clear:both; padding-left:3.0pt; padding-right:3.0pt; text-indent:0pt; border:solid black 0.4pt; }
div.minipage{width:100%;}
div.center, div.center div.center {text-align: center; margin-left:1em; margin-right:1em;}
div.center div {text-align: left;}
div.flushright, div.flushright div.flushright {text-align: right;}
div.flushright div {text-align: left;}
div.flushleft {text-align: left;}
.underline{ text-decoration:underline; }
.underline img{ border-bottom: 1px solid black; margin-bottom:1pt; }
.framebox-c, .framebox-l, .framebox-r { padding-left:3.0pt; padding-right:3.0pt; text-indent:0pt; border:solid black 0.4pt; }
.framebox-c {text-align:center;}
.framebox-l {text-align:left;}
.framebox-r {text-align:right;}
span.thank-mark{ vertical-align: super }
span.footnote-mark sup.textsuperscript, span.footnote-mark a sup.textsuperscript{ font-size:80%; }
div.tabular, div.center div.tabular {text-align: center; margin-top:0.5em; margin-bottom:0.5em; }
table.tabular td p{margin-top:0em;}
table.tabular {margin-left: auto; margin-right: auto;}
div.td00{ margin-left:0pt; margin-right:0pt; }
div.td01{ margin-left:0pt; margin-right:5pt; }
div.td10{ margin-left:5pt; margin-right:0pt; }
div.td11{ margin-left:5pt; margin-right:5pt; }
table[rules] {border-left:solid black 0.4pt; border-right:solid black 0.4pt; }
td.td00{ padding-left:0pt; padding-right:0pt; }
td.td01{ padding-left:0pt; padding-right:5pt; }
td.td10{ padding-left:5pt; padding-right:0pt; }
td.td11{ padding-left:5pt; padding-right:5pt; }
table[rules] {border-left:solid black 0.4pt; border-right:solid black 0.4pt; }
.hline hr, .cline hr{ height : 1px; margin:0px; }
.tabbing-right {text-align:right;}
span.TEX {letter-spacing: -0.125em; }
span.TEX span.E{ position:relative;top:0.5ex;left:-0.0417em;}
a span.TEX span.E {text-decoration: none; }
span.LATEX span.A{ position:relative; top:-0.5ex; left:-0.4em; font-size:85%;}
span.LATEX span.TEX{ position:relative; left: -0.4em; }
div.float img, div.float .caption {text-align:center;}
div.figure img, div.figure .caption {text-align:center;}
.marginpar {width:20%; float:right; text-align:left; margin-left:auto; margin-top:0.5em; font-size:85%; text-decoration:underline;}
.marginpar p{margin-top:0.4em; margin-bottom:0.4em;}
.equation td{text-align:center; vertical-align:middle; }
td.eq-no{ width:5%; }
table.equation { width:100%; } 
div.math-display, div.par-math-display{text-align:center;}
math .texttt { font-family: monospace; }
math .textit { font-style: italic; }
math .textsl { font-style: oblique; }
math .textsf { font-family: sans-serif; }
math .textbf { font-weight: bold; }
.partToc a, .partToc, .likepartToc a, .likepartToc {line-height: 200%; font-weight:bold; font-size:110%;}
.chapterToc a, .chapterToc, .likechapterToc a, .likechapterToc, .appendixToc a, .appendixToc {line-height: 200%; font-weight:bold;}
.index-item, .index-subitem, .index-subsubitem {display:block}
.caption td.id{font-weight: bold; white-space: nowrap; }
table.caption {text-align:center;}
h1.partHead{text-align: center}
p.bibitem { text-indent: -2em; margin-left: 2em; margin-top:0.6em; margin-bottom:0.6em; }
p.bibitem-p { text-indent: 0em; margin-left: 2em; margin-top:0.6em; margin-bottom:0.6em; }
.paragraphHead, .likeparagraphHead { margin-top:2em; font-weight: bold;}
.subparagraphHead, .likesubparagraphHead { font-weight: bold;}
.quote {margin-bottom:0.25em; margin-top:0.25em; margin-left:1em; margin-right:1em; text-align:\\jmathmathustify;}
.verse{white-space:nowrap; margin-left:2em}
div.maketitle {text-align:center;}
h2.titleHead{text-align:center;}
div.maketitle{ margin-bottom: 2em; }
div.author, div.date {text-align:center;}
div.thanks{text-align:left; margin-left:10%; font-size:85%; font-style:italic; }
div.author{white-space: nowrap;}
.quotation {margin-bottom:0.25em; margin-top:0.25em; margin-left:1em; }
h1.partHead{text-align: center}
.sectionToc, .likesectionToc {margin-left:2em;}
.subsectionToc, .likesubsectionToc {margin-left:4em;}
.subsubsectionToc, .likesubsubsectionToc {margin-left:6em;}
.frenchb-nbsp{font-size:75%;}
.frenchb-thinspace{font-size:75%;}
.figure img.graphics {margin-left:10%;}
/* end css.sty */

\title{Generalites sur les integrales impropres}
\author{}
\date{}

\begin{document}
\maketitle

\textbf{Warning: 
requires JavaScript to process the mathematics on this page.\\ If your
browser supports JavaScript, be sure it is enabled.}

\begin{center}\rule{3in}{0.4pt}\end{center}

{[}
{[}
{[}{]}
{[}

\subsubsection{9.8 Généralités sur les intégrales impropres}

Remarque~9.8.1 Ce paragraphe et ceux qui suivent, qui correspondent aux
anciens programmes de classes préparatoires, font double emploi avec les
paragraphes sur les fonctions intégrables sur un intervalle. Ils ont été
maintenus ici par souci de compatibilité avec certains enseignements
universitaires.

Définition~9.8.1 On dit qu'une fonction est réglée sur un intervalle I
si elle est réglée sur tout segment inclus dans I.

\paragraph{9.8.1 Notion d'intégrale impropre}

Définition~9.8.2 Soit -\infty~ \textless{} a \textless{} b \leq +\infty~ et f :
{[}a,b{[}\rightarrow~ E réglée. On dit que l'intégrale \\int
 _a^bf(t) dt converge si existe
lim_x\rightarrow~b,x\textless{}b~\\int
 _a^xf(t) dt. Dans ce cas on pose
\int  _a^b~f(t) dt
=\
lim_x\rightarrow~b,x\textless{}b\int ~
_a^xf(t) dt. On a une notion similaire avec -\infty~\leq a
\textless{} b \textless{} +\infty~ et f :{]}a,b{]} \rightarrow~ E réglée.

Remarque~9.8.2 Si l'intégrale ne converge pas, elle est dite divergente.
Si b \textless{} +\infty~ et si f est la restriction à {[}a,b{[} d'une
fonction réglée sur {[}a,b{]}, alors l'application
x\mapsto~\int ~
_a^xf(t) dt est continue au point b~; l'intégrale impropre
est donc convergente et la valeur de l'intégrale impropre est donc la
valeur de l'intégrale, si bien qu'il n'y a pas d'ambiguïté dans la
notation \int  _a^b~f(t) dt~; dans
ce cas nous parlerons d'une intégrale faussement impropre. Un exemple
typique est celui de \int  _0^1~
sin t \over t~ dt qui est a
priori impropre en 0, mais qui est la restriction à {]}0,1{]} de la
fonction continue f(t) = \left \
\cases  sin~ t
\over t &si t\neq~0
\cr 1 &si t = 0 \cr 
\right ..

Proposition~9.8.1 Soit f : {[}a,b{[}\rightarrow~ E une fonction réglée et c \in
{[}a,b{[}. Alors l'intégrale \int ~
_a^bf(t) dt converge si et seulement si~l'intégrale
\int  _c^b~f(t) dt converge.

Démonstration On a \int  _a^x~f(t)
dt =\int  _a^c~f(t) dt
+\int  _c^x~f(t) dt ce qui montre
que \int  _a^x~f(t) dt a une
limite en b si et seulement si~\int ~
_c^xf(t) dt en a une.

Remarque~9.8.3 Cette propriété montre que si f : {[}a,b{[}\rightarrow~ E est une
fonction réglée, la convergence de \int ~
_a^bf(t) dt ne dépend que de la restriction de f à un
voisinage de b~; il s'agit donc d'une notion locale en b.

\paragraph{9.8.2 Intégrales plusieurs fois impropres}

Définition~9.8.3 Soit -\infty~\leq a \textless{} b \leq +\infty~ et f :{]}a,b{[}\rightarrow~ E
réglée. On dit que l'intégrale \int ~
_a^bf(t) dt converge si on a les conditions équivalentes
(i) il existe c \in{]}a,b{[} tel que les deux intégrales impropres
\int  _a^c~f(t) dt et
\int  _c^b~f(t) dt convergent (ii)
pour tout c \in{]}a,b{[}, les deux intégrales impropres
\int  _a^c~f(t) dt et
\int  _c^b~f(t) dt convergent
(iii) l'application
(x,y)\mapsto~\int ~
_x^yf(t) dt admet une limite quand x tend vers a et y tend
vers b indépendamment l'un de l'autre. On pose alors
\int  _a^b~f(t) dt
=\int  _a^c~f(t) dt
+\int  _c^b~f(t) dt
=\
lim_x\rightarrow~a,y\rightarrow~b\int ~
_x^yf(t) dt.

Démonstration Découle immédiatement de la relation de Chasles.

Remarque~9.8.4 L'intégrale \int ~
_-\infty~^+\infty~t dt diverge alors que
lim_x\rightarrow~+\infty~~\\int
 _-x^xt dt = 0. Il est donc impératif dans (iii)
d'introduire deux variables x et y et de les faire varier
indépendamment.

Définition~9.8.4 Soit -\infty~\leq a_0 \textless{} a_1
\textless{} \\ldots~
\textless{} a_n \leq +\infty~ et f
:{]}a_0,a_n{[}\diagdown\a_1,\\ldots,a_n-1\~
\rightarrow~ E une fonction réglée. On dit que l'intégrale
\int ~
_a_0^a_nf(t) dt converge si chacune des
intégrales \int ~
_a_i-1^a_if(t) dt converge. On pose
alors \int  _a^b~f(t) dt
= \\sum ~
_i=1^n\int ~
_a_i-1^a_if(t) dt

Avec ces définitions, toutes les questions concernant des intégrales
plusieurs fois impropres se ramènent à des problèmes sur les intégrales
une fois impropres.

\paragraph{9.8.3 Opérations sur les intégrales impropres}

Théorème~9.8.2 L'ensemble des fonctions réglées de {[}a,b{[} dans E
telles que l'intégrale impropre \int ~
_a^bf(t) dt converge est un sous espace vectoriel ~de
l'ensemble des fonctions réglées de {[}a,b{[} dans E. L'application
f\mapsto~\int ~
_a^bf(t) dt est linéaire de cet espace vectoriel dans E.

Démonstration Il suffit d'écrire \int ~
_a^x(\alpha~f(t) + \beta~g(t)) dt = \alpha~\int ~
_a^xf(t) dt + \beta~\int ~
_a^xg(t) dt et d'utiliser les théorèmes sur les limites.

Théorème~9.8.3 Soit u : E \rightarrow~ F une application linéaire continue, f :
{[}a,b{[}\rightarrow~ E réglée telle que \int ~
_a^bf(t) dt converge. Alors \int ~
_a^bu(f(t)) dt converge et \int ~
_a^bu(f(t)) dt = u(\int ~
_a^bf(t) dt).

Démonstration Il suffit d'écrire \int ~
_a^xu(f(t)) dt = u(\int ~
_a^xf(t) dt) et d'utiliser la continuité de u.

Corollaire~9.8.4 Soit E un espace vectoriel normé~de dimension finie,
(e_1,\\ldots,e_n~)
une base de E, f : {[}a,b{[}\rightarrow~ E réglée. On écrit f(t) =
f_1(t)e_1 +
\\ldots~ +
f_n(t)e_n. Alors \int ~
_a^bf(t) dt converge si et seulement si~chacune des
intégrales \int ~
_a^bf_i(t) dt converge et alors

\int  _a^b~f(t) dt =
(\int  _a^bf_ 1~(t)
dt)e_1 +
\\ldots~ +
(\int  _a^bf_ n~(t)
dt)e_n

Démonstration Soit u : K^n \rightarrow~ E,
(x_1,\\ldots,x_n)\mapsto~x_1e_1~
+ \\ldots~ +
x_ne_n. L'application linéaire u est un isomorphisme
d'espaces vectoriels et puisque les espaces sont de dimension finie, u
et u^-1 sont continues. Il suffit alors d'appliquer le
théorème précédent en remarquant que f = u \cdot
(f_1,\\ldots,f_n~)
et que
(f_1,\\ldots,f_n~)
= u^-1 \cdot f.

Changement de variables Soit \phi : {[}a,b{[}\rightarrow~ {[}\alpha~,\beta~{[} de classe
\mathcal{C}^1 telle que lim_u\rightarrow~b~\phi(u)
= \beta~. Soit f : {[}\alpha~,\beta~{[}\rightarrow~ E continue. Pour x \in {[}a,b{[}, on peut alors
écrire \int  _a^x~f(\phi(u))\phi'(u) du
=\int  _\phi(a)^\phi(x)~f(t) dt. On en
déduit par le théorème de composition des limites, que si l'intégrale
\int  _\alpha~^\beta~~f(t) dt converge, alors
l'intégrale \int ~
_a^bf(\phi(u))\phi'(u) du converge et que dans ce cas

\int  _a^b~f(\phi(u))\phi'(u) du
=\int  _\phi(a)^\beta~~f(t) dt

Inversement, si l'on suppose que \phi est un homéomorphisme de {[}a,b{[}
sur {[}\alpha~,\beta~{[}, on a, pour y \in {[}\alpha~,\beta~{[}, \int ~
_\alpha~^yf(t) dt =\int ~
_\phi^-1(\alpha~)^\phi^-1(y) f(\phi(u))\phi'(u) du et
alors la convergence de \int ~
_a^bf(\phi(u))\phi'(u) du implique celle de
\int  _\alpha~^\beta~~f(t) dt et l'égalité
ci-dessus. On retiendra en particulier

Théorème~9.8.5 Soit \phi : {[}a,b{[}\rightarrow~ {[}\alpha~,\beta~{[} un homéomorphisme. On
suppose que \phi est de classe \mathcal{C}^1 et que
lim_u\rightarrow~b~\phi(u) = \beta~. Soit f : {[}\alpha~,\beta~{[}\rightarrow~
E continue. Alors les deux intégrales impropres
\int  _a^b~f(\phi(u))\phi'(u) du et
\int  _\alpha~^\beta~~f(t) dt sont de même
nature (convergentes ou divergentes) et on a l'égalité

\int  _a^b~f(\phi(u))\phi'(u) du
=\int  _\phi(a)^\beta~~f(t) dt

Intégration par parties Soit f,g : {[}a,b{[}\rightarrow~ \mathbb{C} de classe
\mathcal{C}^1. Pour x \in {[}a,b{[}, on peut alors faire une intégration
par parties et écrire

\int  _a^x~f(t)g'(t) dt =
\left {[}f(t)g(t)\right {]}_
a^x -\int  _a^x~f'(t)g(t)
dt

Si deux des trois termes qui dépendent de x admettent une limite en b,
alors le troisième aussi et on a alors

\int  _a^b~f(t)g'(t) dt
= lim_ x\rightarrow~b~(f(x)g(x)) - f(a)g(a)
-\int  _a^b~f'(t)g(t) dt

que l'on écrit encore

\int  _a^b~f(t)g'(t) dt =
\left {[}f(t)g(t)\right {]}_
a^b -\int  _a^b~f'(t)g(t)
dt

Sinon, on conserve les intégrales partielles de a à x \\jmathmathusqu'à pouvoir
lever les indéterminations éventuelles.

Remarque~9.8.5 Le lecteur devra faire preuve d'une grande prudence~: une
intégration par parties peut facilement faire passer d'une intégrale
convergente à une intégrale divergente, en particulier avec des
fonctions comme le logarithme.

\paragraph{9.8.4 Intégrales et séries~: intégration par paquets}

Théorème~9.8.6 Soit f : {[}a,b{[}\rightarrow~ E réglée, (b_n) une suite
strictement croissante de {[}a,b{[} de limite b. On pose pour n ≥ 1,
x_n =\int ~
_b_n-1^b_nf(t) dt. Alors

\begin{itemize}
\itemsep1pt\parskip0pt\parsep0pt
\item
  (i) si l'intégrale \int ~
  _a^bf(t) dt converge, la série
  \\sum ~
  _n≥1x_n converge
\item
  (ii) la réciproque est exacte dans les deux cas suivants

  \begin{itemize}
  \itemsep1pt\parskip0pt\parsep0pt
  \item
    (a) la suite (b_n - b_n-1) est bornée et
    lim_t\rightarrow~b~f(t) = 0
  \item
    (b) E = \mathbb{R}~ et la fonction f est de signe constant sur chaque
    intervalle {[}b_n-1,b_n{]}.
  \end{itemize}
\end{itemize}

Démonstration (i) On a
\\sum ~
_n=1^Nx_n =\int ~
_b_0^b_Nf(t) dt = F(b_N) avec
F(x) =\int  _b_0^x~f(t)
dt. Puisque l'intégrale converge, la fonction F a une limite au point
b~; le théorème de composition des limites assure alors l'existence de
lim_N\rightarrow~+\infty~F(b_N~), donc la
convergence de la série~; on a d'ailleurs
\\sum ~
_n=1^+\infty~x_n =\int ~
_b_0^bf(t) dt.

(ii)(a) Soit x \textgreater{} b_0 et soit p l'unique entier tel
que b_p-1 \leq x \textless{} b_p. On a alors

\sum _n=1^px_ n~
-\\int  ~
_b_0^xf(t) dt =
\\int  ~
_x^b_p f(t) dt

Soit alors \epsilon \textgreater{} 0, K \textgreater{} 0 tel que
\forall~n \in \mathbb{N}~, b_n - b_n-1~ \leq K, c \in
{[}a,b{[} tel que t \in {[}c,b{[}\rigtharrow~\
f(t)\ \textless{} \epsilon \over 2K
. Pour x \textgreater{} c, on a alors
\\\\sum
 _n=1^px_n -\int ~
_b_0^xf(t) dt\
\leq\int ~
_x^b_p\f(t)\
dt \leq K \epsilon \over 2K = \epsilon \over 2 .
Soit S = \\sum ~
_n=1^+\infty~x_n et N \in \mathbb{N}~ tel que n \rigtharrow~ N
\rigtharrow~\ S
-\\sum ~
_k=1^nx_k\ \textless{}
\epsilon \over 2 . Pour x ≥ x_N, on a p ≥ N et donc
pour x \textgreater{} max(c,b_N~) on a

\begin{align*} \S
-\int  _b_0^x~f(t)
dt& \leq& \S
-\sum _k=1^px_
k\ +\
\sum _n=1^px_ n~
-\\int  ~
_b_0^xf(t) dt\\%&
\\ & \textless{}& \epsilon
\over 2 + \epsilon \over 2 = \epsilon \%&
\\ \end{align*}

ce qui montre la convergence de l'intégrale.

(ii)(b) La démonstration est similaire. Mais on écrit, en utilisant le
fait que f est de signe constant sur {[}b_p-1,b_p{]}

\begin{align*} \\sum
_n=1^px_ n
-\\int  ~
_b_0^xf(t) dt& =&
\int  _x^b_p~
f(t) dt =\int ~
_x^b_p f(t) dt \%&
\\ & \leq& \int ~
_b_p-1^b_p f(t) dt =
\int ~
_b_p-1^b_p f(t) dt\%&
\\ & =& x_p
\%& \\ \end{align*}

Puisque la série converge, limx_n~ = 0
et donc, il existe M tel que n ≥ M \rigtharrow~x_n
\textless{} \epsilon \over 2 . Soit N \in \mathbb{N}~ tel que n ≥ N
\rigtharrow~S -\\sum ~
_k=1^nx_k \textless{} \epsilon
\over 2 . Pour x ≥\
max(x_N,x_M), on a p ≥\
max(N,M) et donc

S -\int ~
_b_0^xf(t) dt\leqS
-\sum _k=1^px_
k + \\sum
_n=1^px_ n
-\\int  ~
_b_0^xf(t) dt \textless{} \epsilon
\over 2 + \epsilon \over 2 = \epsilon

ce qui montre la convergence de l'intégrale.

{[}
{[}
{[}
{[}

\end{document}
