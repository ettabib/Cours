\documentclass[]{article}
\usepackage[T1]{fontenc}
\usepackage{lmodern}
\usepackage{amssymb,amsmath}
\usepackage{ifxetex,ifluatex}
\usepackage{fixltx2e} % provides \textsubscript
% use upquote if available, for straight quotes in verbatim environments
\IfFileExists{upquote.sty}{\usepackage{upquote}}{}
\ifnum 0\ifxetex 1\fi\ifluatex 1\fi=0 % if pdftex
  \usepackage[utf8]{inputenc}
\else % if luatex or xelatex
  \ifxetex
    \usepackage{mathspec}
    \usepackage{xltxtra,xunicode}
  \else
    \usepackage{fontspec}
  \fi
  \defaultfontfeatures{Mapping=tex-text,Scale=MatchLowercase}
  \newcommand{\euro}{€}
\fi
% use microtype if available
\IfFileExists{microtype.sty}{\usepackage{microtype}}{}
\ifxetex
  \usepackage[setpagesize=false, % page size defined by xetex
              unicode=false, % unicode breaks when used with xetex
              xetex]{hyperref}
\else
  \usepackage[unicode=true]{hyperref}
\fi
\hypersetup{breaklinks=true,
            bookmarks=true,
            pdfauthor={},
            pdftitle={Convergence des suites},
            colorlinks=true,
            citecolor=blue,
            urlcolor=blue,
            linkcolor=magenta,
            pdfborder={0 0 0}}
\urlstyle{same}  % don't use monospace font for urls
\setlength{\parindent}{0pt}
\setlength{\parskip}{6pt plus 2pt minus 1pt}
\setlength{\emergencystretch}{3em}  % prevent overfull lines
\setcounter{secnumdepth}{0}
 
/* start css.sty */
.cmr-5{font-size:50%;}
.cmr-7{font-size:70%;}
.cmmi-5{font-size:50%;font-style: italic;}
.cmmi-7{font-size:70%;font-style: italic;}
.cmmi-10{font-style: italic;}
.cmsy-5{font-size:50%;}
.cmsy-7{font-size:70%;}
.cmex-7{font-size:70%;}
.cmex-7x-x-71{font-size:49%;}
.msbm-7{font-size:70%;}
.cmtt-10{font-family: monospace;}
.cmti-10{ font-style: italic;}
.cmbx-10{ font-weight: bold;}
.cmr-17x-x-120{font-size:204%;}
.cmsl-10{font-style: oblique;}
.cmti-7x-x-71{font-size:49%; font-style: italic;}
.cmbxti-10{ font-weight: bold; font-style: italic;}
p.noindent { text-indent: 0em }
td p.noindent { text-indent: 0em; margin-top:0em; }
p.nopar { text-indent: 0em; }
p.indent{ text-indent: 1.5em }
@media print {div.crosslinks {visibility:hidden;}}
a img { border-top: 0; border-left: 0; border-right: 0; }
center { margin-top:1em; margin-bottom:1em; }
td center { margin-top:0em; margin-bottom:0em; }
.Canvas { position:relative; }
li p.indent { text-indent: 0em }
.enumerate1 {list-style-type:decimal;}
.enumerate2 {list-style-type:lower-alpha;}
.enumerate3 {list-style-type:lower-roman;}
.enumerate4 {list-style-type:upper-alpha;}
div.newtheorem { margin-bottom: 2em; margin-top: 2em;}
.obeylines-h,.obeylines-v {white-space: nowrap; }
div.obeylines-v p { margin-top:0; margin-bottom:0; }
.overline{ text-decoration:overline; }
.overline img{ border-top: 1px solid black; }
td.displaylines {text-align:center; white-space:nowrap;}
.centerline {text-align:center;}
.rightline {text-align:right;}
div.verbatim {font-family: monospace; white-space: nowrap; text-align:left; clear:both; }
.fbox {padding-left:3.0pt; padding-right:3.0pt; text-indent:0pt; border:solid black 0.4pt; }
div.fbox {display:table}
div.center div.fbox {text-align:center; clear:both; padding-left:3.0pt; padding-right:3.0pt; text-indent:0pt; border:solid black 0.4pt; }
div.minipage{width:100%;}
div.center, div.center div.center {text-align: center; margin-left:1em; margin-right:1em;}
div.center div {text-align: left;}
div.flushright, div.flushright div.flushright {text-align: right;}
div.flushright div {text-align: left;}
div.flushleft {text-align: left;}
.underline{ text-decoration:underline; }
.underline img{ border-bottom: 1px solid black; margin-bottom:1pt; }
.framebox-c, .framebox-l, .framebox-r { padding-left:3.0pt; padding-right:3.0pt; text-indent:0pt; border:solid black 0.4pt; }
.framebox-c {text-align:center;}
.framebox-l {text-align:left;}
.framebox-r {text-align:right;}
span.thank-mark{ vertical-align: super }
span.footnote-mark sup.textsuperscript, span.footnote-mark a sup.textsuperscript{ font-size:80%; }
div.tabular, div.center div.tabular {text-align: center; margin-top:0.5em; margin-bottom:0.5em; }
table.tabular td p{margin-top:0em;}
table.tabular {margin-left: auto; margin-right: auto;}
div.td00{ margin-left:0pt; margin-right:0pt; }
div.td01{ margin-left:0pt; margin-right:5pt; }
div.td10{ margin-left:5pt; margin-right:0pt; }
div.td11{ margin-left:5pt; margin-right:5pt; }
table[rules] {border-left:solid black 0.4pt; border-right:solid black 0.4pt; }
td.td00{ padding-left:0pt; padding-right:0pt; }
td.td01{ padding-left:0pt; padding-right:5pt; }
td.td10{ padding-left:5pt; padding-right:0pt; }
td.td11{ padding-left:5pt; padding-right:5pt; }
table[rules] {border-left:solid black 0.4pt; border-right:solid black 0.4pt; }
.hline hr, .cline hr{ height : 1px; margin:0px; }
.tabbing-right {text-align:right;}
span.TEX {letter-spacing: -0.125em; }
span.TEX span.E{ position:relative;top:0.5ex;left:-0.0417em;}
a span.TEX span.E {text-decoration: none; }
span.LATEX span.A{ position:relative; top:-0.5ex; left:-0.4em; font-size:85%;}
span.LATEX span.TEX{ position:relative; left: -0.4em; }
div.float img, div.float .caption {text-align:center;}
div.figure img, div.figure .caption {text-align:center;}
.marginpar {width:20%; float:right; text-align:left; margin-left:auto; margin-top:0.5em; font-size:85%; text-decoration:underline;}
.marginpar p{margin-top:0.4em; margin-bottom:0.4em;}
.equation td{text-align:center; vertical-align:middle; }
td.eq-no{ width:5%; }
table.equation { width:100%; } 
div.math-display, div.par-math-display{text-align:center;}
math .texttt { font-family: monospace; }
math .textit { font-style: italic; }
math .textsl { font-style: oblique; }
math .textsf { font-family: sans-serif; }
math .textbf { font-weight: bold; }
.partToc a, .partToc, .likepartToc a, .likepartToc {line-height: 200%; font-weight:bold; font-size:110%;}
.chapterToc a, .chapterToc, .likechapterToc a, .likechapterToc, .appendixToc a, .appendixToc {line-height: 200%; font-weight:bold;}
.index-item, .index-subitem, .index-subsubitem {display:block}
.caption td.id{font-weight: bold; white-space: nowrap; }
table.caption {text-align:center;}
h1.partHead{text-align: center}
p.bibitem { text-indent: -2em; margin-left: 2em; margin-top:0.6em; margin-bottom:0.6em; }
p.bibitem-p { text-indent: 0em; margin-left: 2em; margin-top:0.6em; margin-bottom:0.6em; }
.paragraphHead, .likeparagraphHead { margin-top:2em; font-weight: bold;}
.subparagraphHead, .likesubparagraphHead { font-weight: bold;}
.quote {margin-bottom:0.25em; margin-top:0.25em; margin-left:1em; margin-right:1em; text-align:\jmathustify;}
.verse{white-space:nowrap; margin-left:2em}
div.maketitle {text-align:center;}
h2.titleHead{text-align:center;}
div.maketitle{ margin-bottom: 2em; }
div.author, div.date {text-align:center;}
div.thanks{text-align:left; margin-left:10%; font-size:85%; font-style:italic; }
div.author{white-space: nowrap;}
.quotation {margin-bottom:0.25em; margin-top:0.25em; margin-left:1em; }
h1.partHead{text-align: center}
.sectionToc, .likesectionToc {margin-left:2em;}
.subsectionToc, .likesubsectionToc {margin-left:4em;}
.subsubsectionToc, .likesubsubsectionToc {margin-left:6em;}
.frenchb-nbsp{font-size:75%;}
.frenchb-thinspace{font-size:75%;}
.figure img.graphics {margin-left:10%;}
/* end css.sty */

\title{Convergence des suites}
\author{}
\date{}

\begin{document}
\maketitle

\textbf{Warning: 
requires JavaScript to process the mathematics on this page.\\ If your
browser supports JavaScript, be sure it is enabled.}

\begin{center}\rule{3in}{0.4pt}\end{center}

{[}
{[}{]}
{[}

\subsubsection{7.1 Convergence des suites}

\paragraph{7.1.1 Monotonie (suites à termes réels)}

Théorème~7.1.1 Soit (x\_n) une suite croissante de nombres
réels. Alors la suite est convergente si et seulement si~elle est
ma\jmathorée. Dans ce cas on a limx\_n~
= supx\_n~.

Démonstration On sait dé\jmathà que toute suite convergente est bornée, donc
ma\jmathorée. Inversement, si la suite est ma\jmathorée, soit l
= supx\_n~ et \epsilon \textgreater{} 0. Par
définition de la borne supérieure, il existe n\_0 tel que l - \epsilon
\textless{} x\_n\_0 \leq l. Pour n ≥ n\_0, on a l -
\epsilon \textless{} x\_n\_0 \leq x\_n \leq l ce qui montre
que la suite converge vers l.

Remarque~7.1.1 On a un résultat analogue avec les suites décroissantes
et minorées.

Corollaire~7.1.2 Soit (a\_n) et (b\_n) deux suites de
nombres réels vérifiant

\begin{itemize}
\itemsep1pt\parskip0pt\parsep0pt
\item
  (i) (a\_n) est croissante et (b\_n) décroissante
\item
  (ii) \forall~n \in \mathbb{N}~, a\_n \leq b\_n~
\item
  (iii) lim(b\_n - a\_n~) = 0
\end{itemize}

Alors les suites (a\_n) et (b\_n) convergent et ont la
même limite \ell qui vérifie

\forall~n \in \mathbb{N}~, a\_n \leq \ell \leq b\_n~

On dit que deux telles suites sont ad\jmathacentes.

Démonstration On remarque que \forall~~p,q,
a\_p \leq b\_q~; en effet si p \leq q on a a\_p \leq
a\_q \leq b\_q et si p \textgreater{} q, on a a\_p
\leq b\_p \leq a\_q. La suite (a\_n) est croissante
ma\jmathorée par b\_0, donc converge. De même la suite (b\_n)
converge et la propriété (iii) implique qu'elles ont la même limite.

Exemple~7.1.1 Posons u\_n = 1 + 1 \over 2 +
\\ldots~ + 1
\over n - log~ n. On a
u\_n+1 - u\_n = 1 \over n+1
- log (n + 1) -\ log~
n = 1 \over n+1 -\int ~
\_n^n+1 dt \over t
=\int  \_n^n+1~( 1
\over n+1 - 1 \over t ) dt \leq 0.
Posons v\_n = 1 + 1 \over 2 +
\\ldots~ + 1
\over n-1 - log~ n. On a de
même v\_n+1 - v\_n =\int ~
\_n^n+1( 1 \over n - 1
\over t ) dt ≥ 0. On a donc (u\_n)
décroissante, (v\_n) croissante, v\_n \leq u\_n,
lim(v\_n - u\_n~) = 0. Donc les
suites convergent. Soit \gamma leur limite commune (la constante d'Euler). On
a donc 1 + 1 \over 2 +
\\ldots~ + 1
\over n = log~ n + \gamma +
\epsilon\_n avec lim\epsilon\_n~ = 0.

\paragraph{7.1.2 Critère de Cauchy}

Dans un espace métrique complet, une suite converge si et seulement
si~elle vérifie le critère de Cauchy. Cela peut servir aussi bien comme
critère de convergence (exemple des suites x\_n+1 =
f(x\_n) où f est contractante) que comme critère de divergence.

Exemple~7.1.2 Posons x\_n = 1 + 1 \over 2 +
\\ldots~ + 1
\over n . On a x\_2n - x\_n = 1
\over n+1 +
\\ldots~ + 1
\over 2n ≥ n \times 1 \over 2n = 1
\over 2 . La suite ne vérifie donc pas le critère de
Cauchy (bien que lim(x\_n+1~ -
x\_n) = 0), donc elle ne converge pas.

\paragraph{7.1.3 Valeurs d'adhérences, limites inférieures et
supérieures}

Proposition~7.1.3 Soit E un espace métrique et (x\_n) une suite
de E. L'ensemble de ses valeurs d'adhérences est fermé dans E.

Démonstration On a vu dans le chapitre sur les compacts que l'ensemble X
des valeurs d'adhérences de la suite (x\_n) est
\⋂ ~
\_N\in\mathbb{N}~\overlineX\_N avec X\_N =
\x\_n∣n ≥
N\. Comme intersection de fermés, c'est un fermé. On
peut aussi le redémontrer directement. Soit x
\in\overlineX et V \in V (x). Soit U ouvert tel que x \in U
\subset~ V . On a U \bigcap X\neq~\varnothing~. Soit y \in U \bigcap X. Comme U
est ouvert, U est un voisinage de la valeur d'adhérence y et donc
\n \in \mathbb{N}~∣x\_n \in
U\ est infini~; il en est de même a fortiori de
\n \in \mathbb{N}~∣x\_n \in V
\, donc x est encore valeur d'adhérence de la suite.

Théorème~7.1.4 Soit E un espace métrique compact et (x\_n) une
suite de E.

\begin{itemize}
\itemsep1pt\parskip0pt\parsep0pt
\item
  (i) La suite a au moins une valeur d'adhérence
\item
  (ii) La suite converge si et seulement si~elle a une unique valeur
  d'adhérence.
\end{itemize}

Démonstration L'affirmation (i) n'est autre que la définition d'un
compact.

(ii) La condition est évidemment nécessaire. Supposons la remplie et
soit \ell cette unique valeur d'adhérence. Supposons que \ell n'est pas limite
de la suite. Ceci signifie qu'il existe U ouvert contenant \ell tel que
\forall~N \in \mathbb{N}~, \\exists~n ≥ N tel
que x\_n∉U. On construit ainsi
facilement une sous suite (x\_\phi(n)) telle que
\forall~n, x\_\phi(n)~ \in E \diagdown U (prendre \phi(n) le
plus petit entier supérieur à N = \phi(n - 1) + 1 vérifiant la condition).
Comme E \diagdown U est fermé dans un compact, c'est un compact et la suite
(x\_\phi(n)) doit avoir une valeur d'adhérence \ell' \in E \diagdown U. Mais
alors la suite (x\_n) a deux valeurs d'adhérences
\ell\neq~\ell'. C'est absurde.

Soit donc (x\_n) une suite de \overline\mathbb{R}~.
Soit X l'ensemble de ses valeurs d'adhérences. C'est un fermé non vide
de \overline\mathbb{R}~, donc il contient sa borne supérieure
et sa borne inférieure.

Définition~7.1.1 Soit (x\_n) une suite de
\overline\mathbb{R}~. Soit X l'ensemble de ses valeurs
d'adhérences. On pose limsupx\_n~
= maxX et \liminf~
x\_n = min~X. La suite converge (dans
\overline\mathbb{R}~) si et seulement
si~limsupx\_n~
= liminf x\_n~.

Théorème~7.1.5 Soit (x\_n) une suite de
\overline\mathbb{R}~ et \ell \in\overline\mathbb{R}~. On a
équivalence de

\begin{itemize}
\itemsep1pt\parskip0pt\parsep0pt
\item
  (i) \ell = limsupx\_n~
\item
  (ii) \ell est valeur d'adhérence de la suite et
  \forall~~c \textgreater{} \ell, \n \in
  \mathbb{N}~∣x\_n ≥ c\ est
  fini.
\item
  (iii) \ell =\
  lim\_p\rightarrow~+\infty~(sup\_n≥px\_n~)
\end{itemize}

Démonstration (i) \rigtharrow~(ii) Soit \ell =\
limsupx\_n. Alors \ell est valeur d'adhérence de la suite. Si
\n \in \mathbb{N}~∣x\_n ≥
c\ est infini, on peut construire une sous suite dans
{[}c,+\infty~{]} qui est compact~; cette suite doit admettre une valeur
d'adhérence \ell' \in {[}c,+\infty~{]}. On a donc \ell' \in X avec \ell'
\textgreater{} sup~X. C'est absurde.

(ii) \rigtharrow~(iii) Remarquons que la suite y\_p
= sup\_n≥px\_n~ est
décroissante, donc convergente dans \overline\mathbb{R}~. Soit
\ell' sa limite. Soit c \textgreater{} \ell. Il existe N \in \mathbb{N}~ tel que n ≥ N \rigtharrow~
x\_n \textless{} c. Donc pour n ≥ N, on a y\_n \leq c et
donc \ell' \leq c. Comme c est quelconque ( \textgreater{} \ell), on a \ell' \leq \ell.
Mais d'autre part on sait que \ell est valeur d'adhérence de la suite
(x\_n) d'où \ell = limx\_\phi(n)~
\leq limy\_\phi(n)~ = \ell'. Donc \ell = \ell'.

(iii) \rigtharrow~(i) Posons tou\jmathours y\_p =\
sup\_n≥px\_n. Si \ell' est une valeur d'adhérence de la
suite (x\_n), on a \ell' =\
limx\_\phi(n) \leq limy\_\phi(n)~ = \ell,
donc limsupx\_n~ \leq \ell. Mais d'autre
part, soit U un ouvert contenant \ell, on peut trouver un N tel que p ≥ N \rigtharrow~
y\_p \in U. Pour un tel p, comme U \in V (y\_p), on peut
trouver un n ≥ p tel que x\_n \in U. Ceci montre que \ell est valeur
d'adhérence de la suite (x\_n) soit \ell \leq\
limsupx\_n et donc l'égalité.

Proposition~7.1.6

\begin{itemize}
\itemsep1pt\parskip0pt\parsep0pt
\item
  (i) limsup(u\_n + v\_n~)
  \leq limsupu\_n~
  + limsupv\_n~ (avec égalité si l'une
  des suites est convergente)
\item
  (ii) si (u\_n) et (v\_n) sont deux suites positives,
  limsup(u\_nv\_n~)
  \leq limsupu\_n~\
  limsupv\_n (avec égalité si l'une des suites est
  convergente)
\item
  (iii) si \lambda~ \textgreater{} 0,
  limsup(\lambda~x\_n~) =
  \lambda~limsupx\_n~
\item
  (iv) si f est continue,
  f(limsupx\_n~)
  \leq limsupf(x\_n~) (avec égalité si f
  est croissante)
\end{itemize}

Démonstration (i) On pose \ell =\
limsupu\_n, v = limsupv\_n~.
Soit \epsilon \textgreater{} 0. Il existe N \in \mathbb{N}~ tel que n ≥ N \rigtharrow~ u\_n
\textless{} \ell + \epsilon. De même, il existe N' tel que n ≥ N' \rigtharrow~ v\_n
\textless{} \ell' + \epsilon. Alors n ≥ max~(N,N') \rigtharrow~
u\_n + v\_n \textless{} \ell + \ell' + 2\epsilon, ce qui montre que
limsup(u\_n + v\_n~) \leq \ell + \ell'.
Si la suite u\_n converge, on a par exemple \ell'
= limv\_\phi(n)~, d'où \ell + \ell'
= lim(u\_\phi(n) + v\_\phi(n)~) est
encore valeur d'adhérence de la suite (u\_n + v\_n)~;
donc limsup(u\_n + v\_n~) = \ell +
\ell'. La démonstration de (ii) est tout à fait similaire.

(iii) est tout à fait élémentaire.

(iv) soit \ell = limsupx\_n~. On a \ell
= limx\_\phi(n)~, donc f(\ell)
= limf(x\_\phi(n)~) est valeur d'adhérence
de la suite (f(x\_n)). On en déduit que f(\ell)
\leq limsupf(x\_n~). Supposons maintenant
f croissante et supposons que f(\ell) \textless{}\
limsupf(x\_n) = \ell'. Soit \alpha~ tel que f(\ell) \textless{} \alpha~
\textless{} \ell'. Le réel \ell' est valeur d'adhérence de la suite
f(x\_n), donc on peut trouver N tel que f(x\_N)
\textgreater{} \alpha~(\textgreater{} f(\ell)). Le théorème des valeurs
intermédiaires assure qu'il existe a tel que \alpha~ = f(a). Comme f est
croissante, on a a \textgreater{} \ell. On a \ell =\
limf(x\_\phi(n)) donc il existe N' tel que n ≥ N' \rigtharrow~
f(x\_\phi(n)) \textgreater{} \alpha~ = f(a). Mais alors n ≥ N' \rigtharrow~
x\_\phi(n) \textgreater{} a \textgreater{} \ell. Ceci contredit le
fait qu'il n'y a qu'un nombre fini de n tels que x\_n
\textgreater{} a. On a donc f(\ell) = \ell'.

Remarque~7.1.2 L'exemple u\_n = (-1)^n, v\_n
= -u\_n montre que l'on n'a pas généralement d'égalité dans (i).
En ce qui concerne (iii), si \lambda~ \textless{} 0 on a évidemment
limsup(\lambda~x\_n~) =
\lambda~liminf x\_n~. De même pour (iv), si f
est décroissante, on a f(limsupx\_n~)
= liminf f(x\_n~), ce qui montre qu'en
général on n'a pas d'égalité dans (iv).

Les résultats concernant la limite inférieure sont tout à fait
similaires, les inégalités changeant de sens

Exemple~7.1.3 Soit f :{]}0,+\infty~{[}\rightarrow~{]}0,+\infty~{[} continue croissante~; on
suppose que l'équation f(x) = x \over 2 a une unique
solution \ell, que x \textless{} \ell \rigtharrow~ f(x) \textgreater{} x
\over 2 et x \textgreater{} \ell \rigtharrow~ f(x) \textless{} x
\over 2 ~; on considère la suite (x\_n) définie
par x\_n+1 = f(x\_n) + f(x\_n-1). On vérifie
facilement que si a =\
min(\ell,x\_0,x\_1), b =\
max(\ell,x\_0,x\_1), alors \forall~~n \in
\mathbb{N}~, x\_n \in {[}a,b{]}. Posons M =\
limsupx\_n et m = liminf~
x\_n. On a alors M =\
limsup(f(x\_n-1) + f(x\_n-2))
\leq limsupf(x\_n-1~)
+ limsupf(x\_n-2~) = 2f(M). On en
déduit que M \leq \ell. On montre de même que m ≥ \ell d'où m = M = \ell et la suite
converge.

\paragraph{7.1.4 Récurrences d'ordre 1}

Soit D une partie de \mathbb{R}~ et f : D \rightarrow~ \mathbb{R}~ une fonction continue. On considère
x\_0 \in D et la suite (x\_n) définie par récurrence par
x\_n+1 = f(x\_n). On note D' =
\x\_0 \in
D∣(x\_n)\_n\in\mathbb{N}~\text
est définie \ (on montre facilement que D'
= \⋃ ~
\_A\subset~D,f(A)\subset~AA). On remarque immédiatement que D contient tous les
points fixes de f.

Proposition~7.1.7 Si la suite (x\_n) converge vers un point \ell \in
D, alors f(\ell) = \ell.

Démonstration On a alors \ell = limx\_n+1~
= limf(x\_n~) =
f(limx\_n~) = f(\ell) par continuité de f
au point \ell.

Proposition~7.1.8 Soit \ell \in D^o tel que f(\ell) = \ell et supposons
f dérivable au point \ell.

\begin{itemize}
\itemsep1pt\parskip0pt\parsep0pt
\item
  (i) Si \textbar{}f'(\ell)\textbar{} \textless{} 1 (point fixe attractif),
  il existe un \eta \textgreater{} 0 tel que

  \begin{itemize}
  \itemsep1pt\parskip0pt\parsep0pt
  \item
    (a) f({]}\ell - \eta,\ell + \eta{[}) \subset~{]}\ell - \eta,\ell + \eta{[}\subset~ D'
  \item
    (b) \left (\existsn\_0~ \in
    \mathbb{N}~, x\_n\_0 \in{]}\ell - \eta,\ell + \eta{[}\right )
    \rigtharrow~ limx\_n~ = \ell
  \end{itemize}
\item
  (ii) Si \textbar{}f'(\ell)\textbar{} \textgreater{} 1 (point fixe
  répulsif) et si limx\_n~ = \ell, alors
  la suite est stationnaire en \ell.
\end{itemize}

Démonstration (i) Soit k tel que \textbar{}f'(\ell)\textbar{} \textless{} k
\textless{} 1. Comme

lim\_x\rightarrow~\ell,x\neq~\ell~\left
\textbar{} f(x) - f(\ell) \over x - \ell
\right \textbar{} =\
lim\_x\rightarrow~\ell,x\neq~\ell\left
\textbar{} f(x) - \ell \over x - \ell \right
\textbar{} = \textbar{}f'(\ell)\textbar{} \textless{} k

il existe \eta \textgreater{} 0 tel que \textbar{}x - \ell\textbar{}
\textless{} \eta \rigtharrow~\textbar{}f(x) - \ell\textbar{}\leq k\textbar{}x - \ell\textbar{}.
On a alors évidemment f({]}\ell - \eta,\ell + \eta{[}) \subset~{]}\ell - \eta,\ell + \eta{[}\subset~ D'. Soit
n\_0 tel que x\_n\_0 \in{]}\ell - \eta,\ell + \eta{[}. Alors
pour tout n ≥ n\_0 on a x\_n \in{]}\ell - \eta,\ell + \eta{[} et
\textbar{}x\_n+1 - \ell\textbar{}\leq k\textbar{}x\_n -
\ell\textbar{}. On a alors \textbar{}x\_n - \ell\textbar{}\leq
k^n-n\_0\textbar{}x\_n\_ 0 -
\ell\textbar{} ce qui montre que limx\_n~
= \ell.

(ii) Une méthode similaire montre que si \textbar{}f'(\ell)\textbar{}
\textgreater{} k \textgreater{} 1, alors il existe \eta \textgreater{} 0
tel que \textbar{}x - \ell\textbar{} \textless{} \eta \rigtharrow~\textbar{}f(x) -
\ell\textbar{}≥ k\textbar{}x - \ell\textbar{}. Si
limx\_n = \ell, il existe n\_0~
tel que n ≥ n\_0 \rigtharrow~\textbar{}x\_n - \ell\textbar{}
\textless{} \eta. On a alors \textbar{}x\_n+1 - \ell\textbar{}≥
k\textbar{}x\_n - \ell\textbar{}, soit encore
\textbar{}x\_n - \ell\textbar{}≥
k^n-n\_0\textbar{}x\_n\_ 0 -
\ell\textbar{} avec k \textgreater{} 1. Ce n'est compatible avec le fait
que x\_n - \ell tend vers 0 que si x\_n\_0 - \ell = 0,
et la suite est alors stationnaire.

Les deux propositions précédentes permettent de conclure dans un certain
nombre de cas. Une étude plus fine relève en général de propriétés de
monotonie de la fonction f.

Proposition~7.1.9 Soit I un intervalle stable par f sur lequel f est
monotone. On suppose qu'il existe n\_0 \in \mathbb{N}~ tel que
x\_n\_0 \in I. Alors \forall~~n ≥
n\_0, x\_n \in I et de plus

\begin{itemize}
\itemsep1pt\parskip0pt\parsep0pt
\item
  (i) si f est croissante sur I, la suite
  (x\_n)\_n≥n\_0 est monotone (le sens étant
  déterminé par le signe de x\_n\_0+1 -
  x\_n\_0 = f(x\_n\_0) -
  x\_n\_0)
\item
  (ii) si f est décroissante sur I, les deux sous suites (x\_2n)
  et (x\_2n+1) sont monotones et de sens contraire à partir de
  l'indice n\_0.
\end{itemize}

Démonstration Supposons f croissante et par exemple
f(x\_n\_0) = x\_n\_0+1 \leq
x\_n\_0, alors x\_n \leq x\_n-1 \rigtharrow~
f(x\_n) \leq f(x\_n-1) \rigtharrow~ x\_n+1 \leq x\_n ce
qui montre par récurrence que \forall~~n ≥
n\_0, x\_n+1 \leq x\_n et la suite est décroissante
à partir de n\_0. De même, si f(x\_n\_0) =
x\_n\_0+1 ≥ x\_n\_0, la suite est
croissante à partir de n\_0. Supposons maintenant f décroissante
sur I et f(I) \subset~ I. Alors f \cdot f est croissante sur I et donc les deux
sous suites (x\_2n) et (x\_2n+1) sont monotones, car
elles vérifient la relation y\_n+1 = f \cdot f(y\_n). De
plus elles sont de sens contraire car x\_2n+3 - x\_2n+1
= f(x\_2n+2) - f(x\_2n) et f est décroissante.

Remarque~7.1.3 Supposons que l'on est dans la situation de la
proposition avec f croissante~; soit \ell \in I tel que f(\ell) = \ell. On constate
immédiatement que le signe de \ell - x\_n = f(\ell) -
f(x\_n-1) est constant, si bien que \ell fournit soit un ma\jmathorant,
soit un minorant de la suite.

{[}
{[}

\end{document}
