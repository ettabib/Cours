\documentclass[]{article}
\usepackage[T1]{fontenc}
\usepackage{lmodern}
\usepackage{amssymb,amsmath}
\usepackage{ifxetex,ifluatex}
\usepackage{fixltx2e} % provides \textsubscript
% use upquote if available, for straight quotes in verbatim environments
\IfFileExists{upquote.sty}{\usepackage{upquote}}{}
\ifnum 0\ifxetex 1\fi\ifluatex 1\fi=0 % if pdftex
  \usepackage[utf8]{inputenc}
\else % if luatex or xelatex
  \ifxetex
    \usepackage{mathspec}
    \usepackage{xltxtra,xunicode}
  \else
    \usepackage{fontspec}
  \fi
  \defaultfontfeatures{Mapping=tex-text,Scale=MatchLowercase}
  \newcommand{\euro}{€}
\fi
% use microtype if available
\IfFileExists{microtype.sty}{\usepackage{microtype}}{}
\ifxetex
  \usepackage[setpagesize=false, % page size defined by xetex
              unicode=false, % unicode breaks when used with xetex
              xetex]{hyperref}
\else
  \usepackage[unicode=true]{hyperref}
\fi
\hypersetup{breaklinks=true,
            bookmarks=true,
            pdfauthor={},
            pdftitle={Convergence des suites},
            colorlinks=true,
            citecolor=blue,
            urlcolor=blue,
            linkcolor=magenta,
            pdfborder={0 0 0}}
\urlstyle{same}  % don't use monospace font for urls
\setlength{\parindent}{0pt}
\setlength{\parskip}{6pt plus 2pt minus 1pt}
\setlength{\emergencystretch}{3em}  % prevent overfull lines
\setcounter{secnumdepth}{0}
 
/* start css.sty */
.cmr-5{font-size:50%;}
.cmr-7{font-size:70%;}
.cmmi-5{font-size:50%;font-style: italic;}
.cmmi-7{font-size:70%;font-style: italic;}
.cmmi-10{font-style: italic;}
.cmsy-5{font-size:50%;}
.cmsy-7{font-size:70%;}
.cmex-7{font-size:70%;}
.cmex-7x-x-71{font-size:49%;}
.msbm-7{font-size:70%;}
.cmtt-10{font-family: monospace;}
.cmti-10{ font-style: italic;}
.cmbx-10{ font-weight: bold;}
.cmr-17x-x-120{font-size:204%;}
.cmsl-10{font-style: oblique;}
.cmti-7x-x-71{font-size:49%; font-style: italic;}
.cmbxti-10{ font-weight: bold; font-style: italic;}
p.noindent { text-indent: 0em }
td p.noindent { text-indent: 0em; margin-top:0em; }
p.nopar { text-indent: 0em; }
p.indent{ text-indent: 1.5em }
@media print {div.crosslinks {visibility:hidden;}}
a img { border-top: 0; border-left: 0; border-right: 0; }
center { margin-top:1em; margin-bottom:1em; }
td center { margin-top:0em; margin-bottom:0em; }
.Canvas { position:relative; }
li p.indent { text-indent: 0em }
.enumerate1 {list-style-type:decimal;}
.enumerate2 {list-style-type:lower-alpha;}
.enumerate3 {list-style-type:lower-roman;}
.enumerate4 {list-style-type:upper-alpha;}
div.newtheorem { margin-bottom: 2em; margin-top: 2em;}
.obeylines-h,.obeylines-v {white-space: nowrap; }
div.obeylines-v p { margin-top:0; margin-bottom:0; }
.overline{ text-decoration:overline; }
.overline img{ border-top: 1px solid black; }
td.displaylines {text-align:center; white-space:nowrap;}
.centerline {text-align:center;}
.rightline {text-align:right;}
div.verbatim {font-family: monospace; white-space: nowrap; text-align:left; clear:both; }
.fbox {padding-left:3.0pt; padding-right:3.0pt; text-indent:0pt; border:solid black 0.4pt; }
div.fbox {display:table}
div.center div.fbox {text-align:center; clear:both; padding-left:3.0pt; padding-right:3.0pt; text-indent:0pt; border:solid black 0.4pt; }
div.minipage{width:100%;}
div.center, div.center div.center {text-align: center; margin-left:1em; margin-right:1em;}
div.center div {text-align: left;}
div.flushright, div.flushright div.flushright {text-align: right;}
div.flushright div {text-align: left;}
div.flushleft {text-align: left;}
.underline{ text-decoration:underline; }
.underline img{ border-bottom: 1px solid black; margin-bottom:1pt; }
.framebox-c, .framebox-l, .framebox-r { padding-left:3.0pt; padding-right:3.0pt; text-indent:0pt; border:solid black 0.4pt; }
.framebox-c {text-align:center;}
.framebox-l {text-align:left;}
.framebox-r {text-align:right;}
span.thank-mark{ vertical-align: super }
span.footnote-mark sup.textsuperscript, span.footnote-mark a sup.textsuperscript{ font-size:80%; }
div.tabular, div.center div.tabular {text-align: center; margin-top:0.5em; margin-bottom:0.5em; }
table.tabular td p{margin-top:0em;}
table.tabular {margin-left: auto; margin-right: auto;}
div.td00{ margin-left:0pt; margin-right:0pt; }
div.td01{ margin-left:0pt; margin-right:5pt; }
div.td10{ margin-left:5pt; margin-right:0pt; }
div.td11{ margin-left:5pt; margin-right:5pt; }
table[rules] {border-left:solid black 0.4pt; border-right:solid black 0.4pt; }
td.td00{ padding-left:0pt; padding-right:0pt; }
td.td01{ padding-left:0pt; padding-right:5pt; }
td.td10{ padding-left:5pt; padding-right:0pt; }
td.td11{ padding-left:5pt; padding-right:5pt; }
table[rules] {border-left:solid black 0.4pt; border-right:solid black 0.4pt; }
.hline hr, .cline hr{ height : 1px; margin:0px; }
.tabbing-right {text-align:right;}
span.TEX {letter-spacing: -0.125em; }
span.TEX span.E{ position:relative;top:0.5ex;left:-0.0417em;}
a span.TEX span.E {text-decoration: none; }
span.LATEX span.A{ position:relative; top:-0.5ex; left:-0.4em; font-size:85%;}
span.LATEX span.TEX{ position:relative; left: -0.4em; }
div.float img, div.float .caption {text-align:center;}
div.figure img, div.figure .caption {text-align:center;}
.marginpar {width:20%; float:right; text-align:left; margin-left:auto; margin-top:0.5em; font-size:85%; text-decoration:underline;}
.marginpar p{margin-top:0.4em; margin-bottom:0.4em;}
.equation td{text-align:center; vertical-align:middle; }
td.eq-no{ width:5%; }
table.equation { width:100%; } 
div.math-display, div.par-math-display{text-align:center;}
math .texttt { font-family: monospace; }
math .textit { font-style: italic; }
math .textsl { font-style: oblique; }
math .textsf { font-family: sans-serif; }
math .textbf { font-weight: bold; }
.partToc a, .partToc, .likepartToc a, .likepartToc {line-height: 200%; font-weight:bold; font-size:110%;}
.chapterToc a, .chapterToc, .likechapterToc a, .likechapterToc, .appendixToc a, .appendixToc {line-height: 200%; font-weight:bold;}
.index-item, .index-subitem, .index-subsubitem {display:block}
.caption td.id{font-weight: bold; white-space: nowrap; }
table.caption {text-align:center;}
h1.partHead{text-align: center}
p.bibitem { text-indent: -2em; margin-left: 2em; margin-top:0.6em; margin-bottom:0.6em; }
p.bibitem-p { text-indent: 0em; margin-left: 2em; margin-top:0.6em; margin-bottom:0.6em; }
.paragraphHead, .likeparagraphHead { margin-top:2em; font-weight: bold;}
.subparagraphHead, .likesubparagraphHead { font-weight: bold;}
.quote {margin-bottom:0.25em; margin-top:0.25em; margin-left:1em; margin-right:1em; text-align:justify;}
.verse{white-space:nowrap; margin-left:2em}
div.maketitle {text-align:center;}
h2.titleHead{text-align:center;}
div.maketitle{ margin-bottom: 2em; }
div.author, div.date {text-align:center;}
div.thanks{text-align:left; margin-left:10%; font-size:85%; font-style:italic; }
div.author{white-space: nowrap;}
.quotation {margin-bottom:0.25em; margin-top:0.25em; margin-left:1em; }
h1.partHead{text-align: center}
.sectionToc, .likesectionToc {margin-left:2em;}
.subsectionToc, .likesubsectionToc {margin-left:4em;}
.subsubsectionToc, .likesubsubsectionToc {margin-left:6em;}
.frenchb-nbsp{font-size:75%;}
.frenchb-thinspace{font-size:75%;}
.figure img.graphics {margin-left:10%;}
/* end css.sty */

\title{Convergence des suites}
\author{}
\date{}

\begin{document}
\maketitle

\textbf{Warning: 
requires JavaScript to process the mathematics on this page.\\ If your
browser supports JavaScript, be sure it is enabled.}

\begin{center}\rule{3in}{0.4pt}\end{center}

[
[]
[

\subsubsection{7.1 Convergence des suites}

\paragraph{7.1.1 Monotonie (suites à termes réels)}

Théorème~7.1.1 Soit (x_n) une suite croissante de nombres
réels. Alors la suite est convergente si et seulement si~elle est
majorée. Dans ce cas on a limx_n~
= supx_n~.

Démonstration On sait déjà que toute suite convergente est bornée, donc
majorée. Inversement, si la suite est majorée, soit l
= supx_n~ et \epsilon > 0. Par
définition de la borne supérieure, il existe n_0 tel que l - \epsilon
< x_n_0 \leq l. Pour n ≥ n_0, on a l -
\epsilon < x_n_0 \leq x_n \leq l ce qui montre
que la suite converge vers l.

Remarque~7.1.1 On a un résultat analogue avec les suites décroissantes
et minorées.

Corollaire~7.1.2 Soit (a_n) et (b_n) deux suites de
nombres réels vérifiant

\begin{itemize}
\itemsep1pt\parskip0pt\parsep0pt
\item
  (i) (a_n) est croissante et (b_n) décroissante
\item
  (ii) \forall~n \in \mathbb{N}~, a_n \leq b_n~
\item
  (iii) lim(b_n - a_n~) = 0
\end{itemize}

Alors les suites (a_n) et (b_n) convergent et ont la
même limite \ell qui vérifie

\forall~n \in \mathbb{N}~, a_n \leq \ell \leq b_n~

On dit que deux telles suites sont adjacentes.

Démonstration On remarque que \forall~~p,q,
a_p \leq b_q~; en effet si p \leq q on a a_p \leq
a_q \leq b_q et si p > q, on a a_p
\leq b_p \leq a_q. La suite (a_n) est croissante
majorée par b_0, donc converge. De même la suite (b_n)
converge et la propriété (iii) implique qu'elles ont la même limite.

Exemple~7.1.1 Posons u_n = 1 + 1 \over 2 +
\\ldots~ + 1
\over n - log~ n. On a
u_n+1 - u_n = 1 \over n+1
- log (n + 1) -\ log~
n = 1 \over n+1 -\int ~
_n^n+1 dt \over t
=\int  _n^n+1~( 1
\over n+1 - 1 \over t ) dt \leq 0.
Posons v_n = 1 + 1 \over 2 +
\\ldots~ + 1
\over n-1 - log~ n. On a de
même v_n+1 - v_n =\int ~
_n^n+1( 1 \over n - 1
\over t ) dt ≥ 0. On a donc (u_n)
décroissante, (v_n) croissante, v_n \leq u_n,
lim(v_n - u_n~) = 0. Donc les
suites convergent. Soit \gamma leur limite commune (la constante d'Euler). On
a donc 1 + 1 \over 2 +
\\ldots~ + 1
\over n = log~ n + \gamma +
\epsilon_n avec lim\epsilon_n~ = 0.

\paragraph{7.1.2 Critère de Cauchy}

Dans un espace métrique complet, une suite converge si et seulement
si~elle vérifie le critère de Cauchy. Cela peut servir aussi bien comme
critère de convergence (exemple des suites x_n+1 =
f(x_n) où f est contractante) que comme critère de divergence.

Exemple~7.1.2 Posons x_n = 1 + 1 \over 2 +
\\ldots~ + 1
\over n . On a x_2n - x_n = 1
\over n+1 +
\\ldots~ + 1
\over 2n ≥ n \times 1 \over 2n = 1
\over 2 . La suite ne vérifie donc pas le critère de
Cauchy (bien que lim(x_n+1~ -
x_n) = 0), donc elle ne converge pas.

\paragraph{7.1.3 Valeurs d'adhérences, limites inférieures et
supérieures}

Proposition~7.1.3 Soit E un espace métrique et (x_n) une suite
de E. L'ensemble de ses valeurs d'adhérences est fermé dans E.

Démonstration On a vu dans le chapitre sur les compacts que l'ensemble X
des valeurs d'adhérences de la suite (x_n) est
\⋂ ~
_N\in\mathbb{N}~\overlineX_N avec X_N =
\x_n∣n ≥
N\. Comme intersection de fermés, c'est un fermé. On
peut aussi le redémontrer directement. Soit x
\in\overlineX et V \in V (x). Soit U ouvert tel que x \in U
\subset~ V . On a U \bigcap X\neq~\varnothing~. Soit y \in U \bigcap X. Comme U
est ouvert, U est un voisinage de la valeur d'adhérence y et donc
\n \in \mathbb{N}~∣x_n \in
U\ est infini~; il en est de même a fortiori de
\n \in \mathbb{N}~∣x_n \in V
\, donc x est encore valeur d'adhérence de la suite.

Théorème~7.1.4 Soit E un espace métrique compact et (x_n) une
suite de E.

\begin{itemize}
\itemsep1pt\parskip0pt\parsep0pt
\item
  (i) La suite a au moins une valeur d'adhérence
\item
  (ii) La suite converge si et seulement si~elle a une unique valeur
  d'adhérence.
\end{itemize}

Démonstration L'affirmation (i) n'est autre que la définition d'un
compact.

(ii) La condition est évidemment nécessaire. Supposons la remplie et
soit \ell cette unique valeur d'adhérence. Supposons que \ell n'est pas limite
de la suite. Ceci signifie qu'il existe U ouvert contenant \ell tel que
\forall~N \in \mathbb{N}~, \\exists~n ≥ N tel
que x_n∉U. On construit ainsi
facilement une sous suite (x_\phi(n)) telle que
\forall~n, x_\phi(n)~ \in E \diagdown U (prendre \phi(n) le
plus petit entier supérieur à N = \phi(n - 1) + 1 vérifiant la condition).
Comme E \diagdown U est fermé dans un compact, c'est un compact et la suite
(x_\phi(n)) doit avoir une valeur d'adhérence \ell' \in E \diagdown U. Mais
alors la suite (x_n) a deux valeurs d'adhérences
\ell\neq~\ell'. C'est absurde.

Soit donc (x_n) une suite de \overline\mathbb{R}~.
Soit X l'ensemble de ses valeurs d'adhérences. C'est un fermé non vide
de \overline\mathbb{R}~, donc il contient sa borne supérieure
et sa borne inférieure.

Définition~7.1.1 Soit (x_n) une suite de
\overline\mathbb{R}~. Soit X l'ensemble de ses valeurs
d'adhérences. On pose limsupx_n~
= maxX et \liminf~
x_n = min~X. La suite converge (dans
\overline\mathbb{R}~) si et seulement
si~limsupx_n~
= liminf x_n~.

Théorème~7.1.5 Soit (x_n) une suite de
\overline\mathbb{R}~ et \ell \in\overline\mathbb{R}~. On a
équivalence de

\begin{itemize}
\itemsep1pt\parskip0pt\parsep0pt
\item
  (i) \ell = limsupx_n~
\item
  (ii) \ell est valeur d'adhérence de la suite et
  \forall~~c > \ell, \n \in
  \mathbb{N}~∣x_n ≥ c\ est
  fini.
\item
  (iii) \ell =\
  lim_p\rightarrow~+\infty~(sup_n≥px_n~)
\end{itemize}

Démonstration (i) \rigtharrow~(ii) Soit \ell =\
limsupx_n. Alors \ell est valeur d'adhérence de la suite. Si
\n \in \mathbb{N}~∣x_n ≥
c\ est infini, on peut construire une sous suite dans
[c,+\infty~] qui est compact~; cette suite doit admettre une valeur
d'adhérence \ell' \in [c,+\infty~]. On a donc \ell' \in X avec \ell'
> sup~X. C'est absurde.

(ii) \rigtharrow~(iii) Remarquons que la suite y_p
= sup_n≥px_n~ est
décroissante, donc convergente dans \overline\mathbb{R}~. Soit
\ell' sa limite. Soit c > \ell. Il existe N \in \mathbb{N}~ tel que n ≥ N \rigtharrow~
x_n < c. Donc pour n ≥ N, on a y_n \leq c et
donc \ell' \leq c. Comme c est quelconque ( > \ell), on a \ell' \leq \ell.
Mais d'autre part on sait que \ell est valeur d'adhérence de la suite
(x_n) d'où \ell = limx_\phi(n)~
\leq limy_\phi(n)~ = \ell'. Donc \ell = \ell'.

(iii) \rigtharrow~(i) Posons toujours y_p =\
sup_n≥px_n. Si \ell' est une valeur d'adhérence de la
suite (x_n), on a \ell' =\
limx_\phi(n) \leq limy_\phi(n)~ = \ell,
donc limsupx_n~ \leq \ell. Mais d'autre
part, soit U un ouvert contenant \ell, on peut trouver un N tel que p ≥ N \rigtharrow~
y_p \in U. Pour un tel p, comme U \in V (y_p), on peut
trouver un n ≥ p tel que x_n \in U. Ceci montre que \ell est valeur
d'adhérence de la suite (x_n) soit \ell \leq\
limsupx_n et donc l'égalité.

Proposition~7.1.6

\begin{itemize}
\itemsep1pt\parskip0pt\parsep0pt
\item
  (i) limsup(u_n + v_n~)
  \leq limsupu_n~
  + limsupv_n~ (avec égalité si l'une
  des suites est convergente)
\item
  (ii) si (u_n) et (v_n) sont deux suites positives,
  limsup(u_nv_n~)
  \leq limsupu_n~\
  limsupv_n (avec égalité si l'une des suites est
  convergente)
\item
  (iii) si \lambda~ > 0,
  limsup(\lambda~x_n~) =
  \lambda~limsupx_n~
\item
  (iv) si f est continue,
  f(limsupx_n~)
  \leq limsupf(x_n~) (avec égalité si f
  est croissante)
\end{itemize}

Démonstration (i) On pose \ell =\
limsupu_n, v = limsupv_n~.
Soit \epsilon > 0. Il existe N \in \mathbb{N}~ tel que n ≥ N \rigtharrow~ u_n
< \ell + \epsilon. De même, il existe N' tel que n ≥ N' \rigtharrow~ v_n
< \ell' + \epsilon. Alors n ≥ max~(N,N') \rigtharrow~
u_n + v_n < \ell + \ell' + 2\epsilon, ce qui montre que
limsup(u_n + v_n~) \leq \ell + \ell'.
Si la suite u_n converge, on a par exemple \ell'
= limv_\phi(n)~, d'où \ell + \ell'
= lim(u_\phi(n) + v_\phi(n)~) est
encore valeur d'adhérence de la suite (u_n + v_n)~;
donc limsup(u_n + v_n~) = \ell +
\ell'. La démonstration de (ii) est tout à fait similaire.

(iii) est tout à fait élémentaire.

(iv) soit \ell = limsupx_n~. On a \ell
= limx_\phi(n)~, donc f(\ell)
= limf(x_\phi(n)~) est valeur d'adhérence
de la suite (f(x_n)). On en déduit que f(\ell)
\leq limsupf(x_n~). Supposons maintenant
f croissante et supposons que f(\ell) <\
limsupf(x_n) = \ell'. Soit \alpha~ tel que f(\ell) < \alpha~
< \ell'. Le réel \ell' est valeur d'adhérence de la suite
f(x_n), donc on peut trouver N tel que f(x_N)
> \alpha~(> f(\ell)). Le théorème des valeurs
intermédiaires assure qu'il existe a tel que \alpha~ = f(a). Comme f est
croissante, on a a > \ell. On a \ell =\
limf(x_\phi(n)) donc il existe N' tel que n ≥ N' \rigtharrow~
f(x_\phi(n)) > \alpha~ = f(a). Mais alors n ≥ N' \rigtharrow~
x_\phi(n) > a > \ell. Ceci contredit le
fait qu'il n'y a qu'un nombre fini de n tels que x_n
> a. On a donc f(\ell) = \ell'.

Remarque~7.1.2 L'exemple u_n = (-1)^n, v_n
= -u_n montre que l'on n'a pas généralement d'égalité dans (i).
En ce qui concerne (iii), si \lambda~ < 0 on a évidemment
limsup(\lambda~x_n~) =
\lambda~liminf x_n~. De même pour (iv), si f
est décroissante, on a f(limsupx_n~)
= liminf f(x_n~), ce qui montre qu'en
général on n'a pas d'égalité dans (iv).

Les résultats concernant la limite inférieure sont tout à fait
similaires, les inégalités changeant de sens

Exemple~7.1.3 Soit f :]0,+\infty~[\rightarrow~]0,+\infty~[ continue croissante~; on
suppose que l'équation f(x) = x \over 2 a une unique
solution \ell, que x < \ell \rigtharrow~ f(x) > x
\over 2 et x > \ell \rigtharrow~ f(x) < x
\over 2 ~; on considère la suite (x_n) définie
par x_n+1 = f(x_n) + f(x_n-1). On vérifie
facilement que si a =\
min(\ell,x_0,x_1), b =\
max(\ell,x_0,x_1), alors \forall~~n \in
\mathbb{N}~, x_n \in [a,b]. Posons M =\
limsupx_n et m = liminf~
x_n. On a alors M =\
limsup(f(x_n-1) + f(x_n-2))
\leq limsupf(x_n-1~)
+ limsupf(x_n-2~) = 2f(M). On en
déduit que M \leq \ell. On montre de même que m ≥ \ell d'où m = M = \ell et la suite
converge.

\paragraph{7.1.4 Récurrences d'ordre 1}

Soit D une partie de \mathbb{R}~ et f : D \rightarrow~ \mathbb{R}~ une fonction continue. On considère
x_0 \in D et la suite (x_n) définie par récurrence par
x_n+1 = f(x_n). On note D' =
\x_0 \in
D∣(x_n)_n\in\mathbb{N}~\text
est définie \ (on montre facilement que D'
= \⋃ ~
_A\subset~D,f(A)\subset~AA). On remarque immédiatement que D contient tous les
points fixes de f.

Proposition~7.1.7 Si la suite (x_n) converge vers un point \ell \in
D, alors f(\ell) = \ell.

Démonstration On a alors \ell = limx_n+1~
= limf(x_n~) =
f(limx_n~) = f(\ell) par continuité de f
au point \ell.

Proposition~7.1.8 Soit \ell \in D^o tel que f(\ell) = \ell et supposons
f dérivable au point \ell.

\begin{itemize}
\itemsep1pt\parskip0pt\parsep0pt
\item
  (i) Si f'(\ell) < 1 (point fixe attractif),
  il existe un \eta > 0 tel que

  \begin{itemize}
  \itemsep1pt\parskip0pt\parsep0pt
  \item
    (a) f(]\ell - \eta,\ell + \eta[) \subset~]\ell - \eta,\ell + \eta[\subset~ D'
  \item
    (b) \left (\existsn_0~ \in
    \mathbb{N}~, x_n_0 \in]\ell - \eta,\ell + \eta[\right )
    \rigtharrow~ limx_n~ = \ell
  \end{itemize}
\item
  (ii) Si f'(\ell) > 1 (point fixe
  répulsif) et si limx_n~ = \ell, alors
  la suite est stationnaire en \ell.
\end{itemize}

Démonstration (i) Soit k tel que f'(\ell) < k
< 1. Comme

lim_x\rightarrow~\ell,x\neq~\ell~\left
 f(x) - f(\ell) \over x - \ell
\right  =\
lim_x\rightarrow~\ell,x\neq~\ell\left
 f(x) - \ell \over x - \ell \right
 = f'(\ell) < k

il existe \eta > 0 tel que x - \ell
< \eta \rigtharrow~f(x) - \ell\leq kx - \ell.
On a alors évidemment f(]\ell - \eta,\ell + \eta[) \subset~]\ell - \eta,\ell + \eta[\subset~ D'. Soit
n_0 tel que x_n_0 \in]\ell - \eta,\ell + \eta[. Alors
pour tout n ≥ n_0 on a x_n \in]\ell - \eta,\ell + \eta[ et
x_n+1 - \ell\leq kx_n -
\ell. On a alors x_n - \ell\leq
k^n-n_0x_n_ 0 -
\ell ce qui montre que limx_n~
= \ell.

(ii) Une méthode similaire montre que si f'(\ell)
> k > 1, alors il existe \eta > 0
tel que x - \ell < \eta \rigtharrow~f(x) -
\ell≥ kx - \ell. Si
limx_n = \ell, il existe n_0~
tel que n ≥ n_0 \rigtharrow~x_n - \ell
< \eta. On a alors x_n+1 - \ell≥
kx_n - \ell, soit encore
x_n - \ell≥
k^n-n_0x_n_ 0 -
\ell avec k > 1. Ce n'est compatible avec le fait
que x_n - \ell tend vers 0 que si x_n_0 - \ell = 0,
et la suite est alors stationnaire.

Les deux propositions précédentes permettent de conclure dans un certain
nombre de cas. Une étude plus fine relève en général de propriétés de
monotonie de la fonction f.

Proposition~7.1.9 Soit I un intervalle stable par f sur lequel f est
monotone. On suppose qu'il existe n_0 \in \mathbb{N}~ tel que
x_n_0 \in I. Alors \forall~~n ≥
n_0, x_n \in I et de plus

\begin{itemize}
\itemsep1pt\parskip0pt\parsep0pt
\item
  (i) si f est croissante sur I, la suite
  (x_n)_n≥n_0 est monotone (le sens étant
  déterminé par le signe de x_n_0+1 -
  x_n_0 = f(x_n_0) -
  x_n_0)
\item
  (ii) si f est décroissante sur I, les deux sous suites (x_2n)
  et (x_2n+1) sont monotones et de sens contraire à partir de
  l'indice n_0.
\end{itemize}

Démonstration Supposons f croissante et par exemple
f(x_n_0) = x_n_0+1 \leq
x_n_0, alors x_n \leq x_n-1 \rigtharrow~
f(x_n) \leq f(x_n-1) \rigtharrow~ x_n+1 \leq x_n ce
qui montre par récurrence que \forall~~n ≥
n_0, x_n+1 \leq x_n et la suite est décroissante
à partir de n_0. De même, si f(x_n_0) =
x_n_0+1 ≥ x_n_0, la suite est
croissante à partir de n_0. Supposons maintenant f décroissante
sur I et f(I) \subset~ I. Alors f \cdot f est croissante sur I et donc les deux
sous suites (x_2n) et (x_2n+1) sont monotones, car
elles vérifient la relation y_n+1 = f \cdot f(y_n). De
plus elles sont de sens contraire car x_2n+3 - x_2n+1
= f(x_2n+2) - f(x_2n) et f est décroissante.

Remarque~7.1.3 Supposons que l'on est dans la situation de la
proposition avec f croissante~; soit \ell \in I tel que f(\ell) = \ell. On constate
immédiatement que le signe de \ell - x_n = f(\ell) -
f(x_n-1) est constant, si bien que \ell fournit soit un majorant,
soit un minorant de la suite.

[
[

\end{document}
