\documentclass[]{article}
\usepackage[T1]{fontenc}
\usepackage{lmodern}
\usepackage{amssymb,amsmath}
\usepackage{ifxetex,ifluatex}
\usepackage{fixltx2e} % provides \textsubscript
% use upquote if available, for straight quotes in verbatim environments
\IfFileExists{upquote.sty}{\usepackage{upquote}}{}
\ifnum 0\ifxetex 1\fi\ifluatex 1\fi=0 % if pdftex
  \usepackage[utf8]{inputenc}
\else % if luatex or xelatex
  \ifxetex
    \usepackage{mathspec}
    \usepackage{xltxtra,xunicode}
  \else
    \usepackage{fontspec}
  \fi
  \defaultfontfeatures{Mapping=tex-text,Scale=MatchLowercase}
  \newcommand{\euro}{€}
\fi
% use microtype if available
\IfFileExists{microtype.sty}{\usepackage{microtype}}{}
\ifxetex
  \usepackage[setpagesize=false, % page size defined by xetex
              unicode=false, % unicode breaks when used with xetex
              xetex]{hyperref}
\else
  \usepackage[unicode=true]{hyperref}
\fi
\hypersetup{breaklinks=true,
            bookmarks=true,
            pdfauthor={},
            pdftitle={Generalites sur les espaces vectoriels},
            colorlinks=true,
            citecolor=blue,
            urlcolor=blue,
            linkcolor=magenta,
            pdfborder={0 0 0}}
\urlstyle{same}  % don't use monospace font for urls
\setlength{\parindent}{0pt}
\setlength{\parskip}{6pt plus 2pt minus 1pt}
\setlength{\emergencystretch}{3em}  % prevent overfull lines
\setcounter{secnumdepth}{0}
 
/* start css.sty */
.cmr-5{font-size:50%;}
.cmr-7{font-size:70%;}
.cmmi-5{font-size:50%;font-style: italic;}
.cmmi-7{font-size:70%;font-style: italic;}
.cmmi-10{font-style: italic;}
.cmsy-5{font-size:50%;}
.cmsy-7{font-size:70%;}
.cmex-7{font-size:70%;}
.cmex-7x-x-71{font-size:49%;}
.msbm-7{font-size:70%;}
.cmtt-10{font-family: monospace;}
.cmti-10{ font-style: italic;}
.cmbx-10{ font-weight: bold;}
.cmr-17x-x-120{font-size:204%;}
.cmsl-10{font-style: oblique;}
.cmti-7x-x-71{font-size:49%; font-style: italic;}
.cmbxti-10{ font-weight: bold; font-style: italic;}
p.noindent { text-indent: 0em }
td p.noindent { text-indent: 0em; margin-top:0em; }
p.nopar { text-indent: 0em; }
p.indent{ text-indent: 1.5em }
@media print {div.crosslinks {visibility:hidden;}}
a img { border-top: 0; border-left: 0; border-right: 0; }
center { margin-top:1em; margin-bottom:1em; }
td center { margin-top:0em; margin-bottom:0em; }
.Canvas { position:relative; }
li p.indent { text-indent: 0em }
.enumerate1 {list-style-type:decimal;}
.enumerate2 {list-style-type:lower-alpha;}
.enumerate3 {list-style-type:lower-roman;}
.enumerate4 {list-style-type:upper-alpha;}
div.newtheorem { margin-bottom: 2em; margin-top: 2em;}
.obeylines-h,.obeylines-v {white-space: nowrap; }
div.obeylines-v p { margin-top:0; margin-bottom:0; }
.overline{ text-decoration:overline; }
.overline img{ border-top: 1px solid black; }
td.displaylines {text-align:center; white-space:nowrap;}
.centerline {text-align:center;}
.rightline {text-align:right;}
div.verbatim {font-family: monospace; white-space: nowrap; text-align:left; clear:both; }
.fbox {padding-left:3.0pt; padding-right:3.0pt; text-indent:0pt; border:solid black 0.4pt; }
div.fbox {display:table}
div.center div.fbox {text-align:center; clear:both; padding-left:3.0pt; padding-right:3.0pt; text-indent:0pt; border:solid black 0.4pt; }
div.minipage{width:100%;}
div.center, div.center div.center {text-align: center; margin-left:1em; margin-right:1em;}
div.center div {text-align: left;}
div.flushright, div.flushright div.flushright {text-align: right;}
div.flushright div {text-align: left;}
div.flushleft {text-align: left;}
.underline{ text-decoration:underline; }
.underline img{ border-bottom: 1px solid black; margin-bottom:1pt; }
.framebox-c, .framebox-l, .framebox-r { padding-left:3.0pt; padding-right:3.0pt; text-indent:0pt; border:solid black 0.4pt; }
.framebox-c {text-align:center;}
.framebox-l {text-align:left;}
.framebox-r {text-align:right;}
span.thank-mark{ vertical-align: super }
span.footnote-mark sup.textsuperscript, span.footnote-mark a sup.textsuperscript{ font-size:80%; }
div.tabular, div.center div.tabular {text-align: center; margin-top:0.5em; margin-bottom:0.5em; }
table.tabular td p{margin-top:0em;}
table.tabular {margin-left: auto; margin-right: auto;}
div.td00{ margin-left:0pt; margin-right:0pt; }
div.td01{ margin-left:0pt; margin-right:5pt; }
div.td10{ margin-left:5pt; margin-right:0pt; }
div.td11{ margin-left:5pt; margin-right:5pt; }
table[rules] {border-left:solid black 0.4pt; border-right:solid black 0.4pt; }
td.td00{ padding-left:0pt; padding-right:0pt; }
td.td01{ padding-left:0pt; padding-right:5pt; }
td.td10{ padding-left:5pt; padding-right:0pt; }
td.td11{ padding-left:5pt; padding-right:5pt; }
table[rules] {border-left:solid black 0.4pt; border-right:solid black 0.4pt; }
.hline hr, .cline hr{ height : 1px; margin:0px; }
.tabbing-right {text-align:right;}
span.TEX {letter-spacing: -0.125em; }
span.TEX span.E{ position:relative;top:0.5ex;left:-0.0417em;}
a span.TEX span.E {text-decoration: none; }
span.LATEX span.A{ position:relative; top:-0.5ex; left:-0.4em; font-size:85%;}
span.LATEX span.TEX{ position:relative; left: -0.4em; }
div.float img, div.float .caption {text-align:center;}
div.figure img, div.figure .caption {text-align:center;}
.marginpar {width:20%; float:right; text-align:left; margin-left:auto; margin-top:0.5em; font-size:85%; text-decoration:underline;}
.marginpar p{margin-top:0.4em; margin-bottom:0.4em;}
.equation td{text-align:center; vertical-align:middle; }
td.eq-no{ width:5%; }
table.equation { width:100%; } 
div.math-display, div.par-math-display{text-align:center;}
math .texttt { font-family: monospace; }
math .textit { font-style: italic; }
math .textsl { font-style: oblique; }
math .textsf { font-family: sans-serif; }
math .textbf { font-weight: bold; }
.partToc a, .partToc, .likepartToc a, .likepartToc {line-height: 200%; font-weight:bold; font-size:110%;}
.chapterToc a, .chapterToc, .likechapterToc a, .likechapterToc, .appendixToc a, .appendixToc {line-height: 200%; font-weight:bold;}
.index-item, .index-subitem, .index-subsubitem {display:block}
.caption td.id{font-weight: bold; white-space: nowrap; }
table.caption {text-align:center;}
h1.partHead{text-align: center}
p.bibitem { text-indent: -2em; margin-left: 2em; margin-top:0.6em; margin-bottom:0.6em; }
p.bibitem-p { text-indent: 0em; margin-left: 2em; margin-top:0.6em; margin-bottom:0.6em; }
.paragraphHead, .likeparagraphHead { margin-top:2em; font-weight: bold;}
.subparagraphHead, .likesubparagraphHead { font-weight: bold;}
.quote {margin-bottom:0.25em; margin-top:0.25em; margin-left:1em; margin-right:1em; text-align:justify;}
.verse{white-space:nowrap; margin-left:2em}
div.maketitle {text-align:center;}
h2.titleHead{text-align:center;}
div.maketitle{ margin-bottom: 2em; }
div.author, div.date {text-align:center;}
div.thanks{text-align:left; margin-left:10%; font-size:85%; font-style:italic; }
div.author{white-space: nowrap;}
.quotation {margin-bottom:0.25em; margin-top:0.25em; margin-left:1em; }
h1.partHead{text-align: center}
.sectionToc, .likesectionToc {margin-left:2em;}
.subsectionToc, .likesubsectionToc {margin-left:4em;}
.subsubsectionToc, .likesubsubsectionToc {margin-left:6em;}
.frenchb-nbsp{font-size:75%;}
.frenchb-thinspace{font-size:75%;}
.figure img.graphics {margin-left:10%;}
/* end css.sty */

\title{Generalites sur les espaces vectoriels}
\author{}
\date{}

\begin{document}
\maketitle

\textbf{Warning: \href{http://www.math.union.edu/locate/jsMath}{jsMath}
requires JavaScript to process the mathematics on this page.\\ If your
browser supports JavaScript, be sure it is enabled.}

\begin{center}\rule{3in}{0.4pt}\end{center}

{[}\href{coursse8.html}{next}{]}
{[}\hyperref[tailcoursse7.html]{tail}{]}
{[}\href{coursch3.html\#coursse7.html}{up}{]}

\subsubsection{2.1 Généralités sur les espaces vectoriels}

\paragraph{2.1.1 Notion de K-espace vectoriel}

Définition~2.1.1 On appelle K-espace vectoriel un triplet (E,+,.) où
(E,+) est un groupe abélien, . une loi externe à domaine d'opérateurs K,
doublement distributive par rapport à l'addition dans K et dans E,
vérifiant \textbackslash{}mathop\{∀\}x ∈ E, \{1\}\_\{K\}x = x et
\textbackslash{}mathop\{∀\}α,β ∈ K, \textbackslash{}mathop\{∀\}x ∈ E,
α(βx) = (αβ)x.

Exemples fondamentaux~: si L est un sur-corps de K, L est naturellement
un K-espace vectoriel .

Soit n ∈ ℕ~; \{K\}\^{}\{n\} muni des lois
(\{x\}\_\{1\},\textbackslash{}mathop\{\textbackslash{}mathop\{\ldots{}\}\},\{x\}\_\{n\})
+
(\{y\}\_\{1\},\textbackslash{}mathop\{\textbackslash{}mathop\{\ldots{}\}\},\{y\}\_\{n\})
= (\{x\}\_\{1\} +
\{y\}\_\{1\},\textbackslash{}mathop\{\textbackslash{}mathop\{\ldots{}\}\},\{x\}\_\{n\}
+ \{y\}\_\{n\}) et
λ(\{x\}\_\{1\},\textbackslash{}mathop\{\textbackslash{}mathop\{\ldots{}\}\},\{x\}\_\{n\})
=
(λ\{x\}\_\{1\},\textbackslash{}mathop\{\textbackslash{}mathop\{\ldots{}\}\},λ\{x\}\_\{n\})est
un K-espace vectoriel. Plus généralement, si I est un ensemble,

\{K\}\^{}\{(I)\} = \textbackslash{}\{\{(\{a\}\_\{
i\})\}\_\{i∈I\}\textbackslash{}mathrel\{∣\}\textbackslash{}mathop\{∀\}i,
\{a\}\_\{i\} ∈ K\textbackslash{}text\{ et nombre fini de
\$\{a\}\_\{i\}\$ non nuls\}\textbackslash{}\}

est un K-espace vectoriel pour les lois (\{a\}\_\{i\}) + (\{b\}\_\{i\})
= (\{a\}\_\{i\} + \{b\}\_\{i\}) et λ(\{a\}\_\{i\}) = (λ\{a\}\_\{i\}).

\paragraph{2.1.2 Notion de sous-espace vectoriel}

Remarque~2.1.1 Un sous-espace vectoriel est une partie stable aussi bien
par la loi interne que par la loi externe et qui est muni par les lois
induites d'une structure d'espace vectoriel. On vérifie immédiatement
que c'est équivalent à la définition suivante que l'on retiendra~:

Définition~2.1.2 On appelle sous-espace vectoriel de l'espace vectoriel
E une partie F de E telle que

\begin{itemize}
\itemsep1pt\parskip0pt\parsep0pt
\item
  (i) F\textbackslash{}mathrel\{≠\}∅
\item
  (ii) \textbackslash{}mathop\{∀\}α,β ∈ K,
  \textbackslash{}mathop\{∀\}x,y ∈ F, αx + βy ∈ F.
\end{itemize}

Remarque~2.1.2 L'intersection d'une famille quelconque de sous-espaces
vectoriels en est encore un ce qui conduit à la définition suivante

Définition~2.1.3 L'ensemble des sous-espaces vectoriels contenant une
partie A de E admet un plus petit élément appelé le sous-espace
vectoriel engendré par A et noté
\textbackslash{}mathop\{\textbackslash{}mathrm\{Vect\}\}(A). On a
\textbackslash{}mathop\{\textbackslash{}mathrm\{Vect\}\}(A)
=\{\textbackslash{}mathop\{ \textbackslash{}mathop\{⋂ \}\} \}\_\{\{ A⊂F
\textbackslash{}atop F\textbackslash{}text\{sev \}\} \}F

Définition~2.1.4 On appelle sous-espace vectoriel engendré par la
famille \{(\{x\}\_\{i\})\}\_\{i∈I\} le plus petit sous-espace vectoriel
contenant tous les vecteurs de la famille, et on le note
\textbackslash{}mathop\{\textbackslash{}mathrm\{Vect\}\}(\{x\}\_\{i\},i
∈ I). On a

\textbackslash{}begin\{eqnarray*\}
\textbackslash{}mathop\{\textbackslash{}mathrm\{Vect\}\}(\{x\}\_\{i\},i
∈ I)\& =\&
\textbackslash{}mathop\{\textbackslash{}mathrm\{Vect\}\}(\{\textbackslash{}mathop\{⋃
\}\}\_\{i∈I\}\textbackslash{}\{\{x\}\_\{i\}\textbackslash{}\}) \%\&
\textbackslash{}\textbackslash{} \& =\&
\textbackslash{}\{\{\textbackslash{}mathop\{∑
\}\}\_\{i∈I\}\{α\}\_\{i\}\{x\}\_\{i\}\textbackslash{}mathrel\{∣\}(\{α\}\_\{i\})
∈ \{K\}\^{}\{(I)\}\textbackslash{}\}\%\&
\textbackslash{}\textbackslash{} \textbackslash{}end\{eqnarray*\}

(ensemble des combinaisons linéaires de la famille (\{x\}\_\{i\})).

\paragraph{2.1.3 Produits, quotients}

Définition~2.1.5 (espace produit) Si E et F sont deux espaces vectoriels
, l'espace E × F est muni d'une structure d'espace vectoriel en posant
(\{x\}\_\{1\},\{y\}\_\{1\}) + (\{x\}\_\{2\},\{y\}\_\{2\}) =
(\{x\}\_\{1\} + \{x\}\_\{2\},\{y\}\_\{1\} + \{y\}\_\{2\}), λ(x,y) =
(λx,λy).

Définition~2.1.6 (espace quotient) Soit E un espace vectoriel et F un
sous-espace vectoriel de E. La relation ''xℛy
\textbackslash{}mathrel\{⇔\} x − y ∈ F'' est une relation d'équivalence
sur E. La classe d'un élément x de E est x + F. Il existe sur E∕F une
unique structure d'espace vectoriel telle que la projection π : E → E∕F
vérifie \textbackslash{}mathop\{∀\}α,β ∈ K,
\textbackslash{}mathop\{∀\}x,y ∈ E, π(αx + βy) = απ(x) + βπ(y).

Démonstration La relation d'équivalence et la caractérisation de la
classe d'équivalence proviennent du même résultat sur les groupes
additifs. La loi d'espace vectoriel sur E∕F doit être définie de telle
sorte que α(x + F) + β(y + F) = (αx + βy) + F~; il suffit donc de
vérifier que si x + F = x' + F et y + F = y' + F, alors (αx + βy) + F =
(αx' + βy') + F. Or les deux premières relations signifient que x − x' ∈
F et y − y' ∈ F. On a donc (αx + βy) − (αx' + βy') = α(x − x') + β(y −
y') ∈ F, soit encore (αx + βy) + F = (αx' + βy') + F. Ceci définit
parfaitement une structure d'espace vectoriel sur E∕F (vérification
facile) et on a bien π(αx + βy) = απ(x) + βπ(y).

\paragraph{2.1.4 Applications linéaires}

Proposition~2.1.1 Soit E et F deux espaces vectoriels . On appelle
application linéaire une application f : E → F telle que
\textbackslash{}mathop\{∀\}α,β ∈ K, \textbackslash{}mathop\{∀\}x,y ∈ E,
f(αx + βy) = αf(x) + βf(y).

Notation~: L(E,F) l'ensemble des applications linéaires de E dans F.

Proposition~2.1.2 L'ensemble L(E,F) est muni d'une structure de K espace
vectoriel en posant (f + g)(x) = f(x) + g(x) et (λf)(x) = λf(x).

Proposition~2.1.3 Soit f ∈ L(E,F). L'image par f de tout sous-espace
vectoriel de E est un sous-espace vectoriel de F. L'image réciproque de
tout sous-espace vectoriel de F est un sous-espace vectoriel de E.

Remarque~2.1.3 En particulier
\textbackslash{}mathop\{\textbackslash{}mathrm\{Ker\}\}f =
\{f\}\^{}\{−1\}(\textbackslash{}\{0\textbackslash{}\}) et
\textbackslash{}mathop\{\textbackslash{}mathrm\{Im\}\}f = f(E) sont des
sous-espaces vectoriels respectivement de E et F.

Théorème~2.1.4 Soit f ∈ L(E,F). L'application f est injective si et
seulement si \textbackslash{}mathop\{\textbackslash{}mathrm\{Ker\}\}f =
\textbackslash{}\{0\textbackslash{}\}.

Démonstration C'est une traduction du résultat sur les groupes.

Théorème~2.1.5 Soit f ∈ L(E,F). Il existe une unique application
\textbackslash{}overline\{f\} :
E∕\textbackslash{}mathop\{\textbackslash{}mathrm\{Ker\}\}f
→\textbackslash{}mathop\{\textbackslash{}mathrm\{Im\}\}f vérifiant
\textbackslash{}mathop\{∀\}x ∈ E, \textbackslash{}overline\{f\}(π(x)) =
f(x) (où π désigne la projection canonique de E sur
E∕\textbackslash{}mathop\{\textbackslash{}mathrm\{Ker\}\}f).
L'application \textbackslash{}overline\{f\} est un isomorphisme
d'espaces vectoriels .

Démonstration Analogue au résultat similaire sur les groupes.

\paragraph{2.1.5 Somme de sous-espaces}

Soit E un K-espace vectoriel et
\{F\}\_\{1\},\textbackslash{}mathop\{\textbackslash{}mathop\{\ldots{}\}\},\{F\}\_\{k\}
des sous-espaces vectoriels de E. Soit f : \{F\}\_\{1\}
×\textbackslash{}mathrel\{⋯\} × \{F\}\_\{k\} → E définie par
f(\{x\}\_\{1\},\textbackslash{}mathop\{\textbackslash{}mathop\{\ldots{}\}\},\{x\}\_\{k\})
= \{x\}\_\{1\} +
\textbackslash{}mathop\{\textbackslash{}mathop\{\ldots{}\}\} +
\{x\}\_\{k\}. On vérifie facilement que f est linéaire.

Définition~2.1.7 On appelle somme des sous-espaces vectoriels
\{F\}\_\{1\},\textbackslash{}mathop\{\textbackslash{}mathop\{\ldots{}\}\},\{F\}\_\{k\}
le sous-espace vectoriel \{F\}\_\{1\} + \textbackslash{}mathrel\{⋯\} +
\{F\}\_\{k\} =\textbackslash{}mathop\{ \textbackslash{}mathrm\{Im\}\}f =
\textbackslash{}\{\{x\}\_\{1\} +
\textbackslash{}mathop\{\textbackslash{}mathop\{\ldots{}\}\} +
\{x\}\_\{k\}\textbackslash{}mathrel\{∣\}\textbackslash{}mathop\{∀\}i,
\{x\}\_\{i\} ∈ \{F\}\_\{i\}\textbackslash{}\}. On dit que
\{F\}\_\{1\},\textbackslash{}mathop\{\textbackslash{}mathop\{\ldots{}\}\},\{F\}\_\{k\}
sont en somme directe si f est injective (c'est-à-dire si l'écriture
d'un x sous la forme x = \{x\}\_\{1\} +
\textbackslash{}mathop\{\textbackslash{}mathop\{\ldots{}\}\} +
\{x\}\_\{k\} lorsqu'elle existe, est unique). Dans ce cas on écrit la
somme sous la forme \{F\}\_\{1\} ⊕\textbackslash{}mathrel\{⋯\} ⊕
\{F\}\_\{k\}.

Théorème~2.1.6 Les sous-espaces
\{F\}\_\{1\},\textbackslash{}mathop\{\textbackslash{}mathop\{\ldots{}\}\},\{F\}\_\{k\}
sont en somme directe si et seulement si

\{x\}\_\{1\} +
\textbackslash{}mathop\{\textbackslash{}mathop\{\ldots{}\}\} +
\{x\}\_\{k\} = 0 ⇒ \{x\}\_\{1\} =
0,\textbackslash{}mathop\{\textbackslash{}mathop\{\ldots{}\}\},\{x\}\_\{k\}
= 0

Démonstration Ceci traduit simplement que f est injective si et
seulement si son noyau est réduit à
\textbackslash{}\{0\textbackslash{}\}.

Remarque~2.1.4 Il n'existe pas d'autre caractérisation correcte et
réellement utile de la somme directe dans le cas où k ≥ 3. Par contre,
si k = 2 on a

Théorème~2.1.7 Les sous-espaces \{F\}\_\{1\} et \{F\}\_\{2\} sont en
somme directe si et seulement si \{F\}\_\{1\} ∩ \{F\}\_\{2\} =
\textbackslash{}\{0\textbackslash{}\}.

Démonstration On a en effet

\{x\}\_\{1\} + \{x\}\_\{2\} = 0 \textbackslash{}mathrel\{⇔\}
\{x\}\_\{1\} = −\{x\}\_\{2\} ∈ \{F\}\_\{1\} ∩ \{F\}\_\{2\}

Définition~2.1.8 On dit que deux sous-espaces vectoriels F et G de
l'espace vectoriel E sont supplémentaires s'ils vérifient les trois
propriétés équivalentes

\begin{itemize}
\itemsep1pt\parskip0pt\parsep0pt
\item
  (i) E = F ⊕ G
\item
  (ii) E = F + G et F ∩ G = \textbackslash{}\{0\textbackslash{}\}
\item
  (iii) Tout élément x de E s'écrit de manière unique sous la forme x =
  y + z avec y ∈ F et z ∈ G.
\end{itemize}

On dit que y est la projection de x sur F parallèlement à G~: y =
\{π\}\_\{F∥G\}(x).

Proposition~2.1.8 Si F et G sont supplémentaires, \{π\}\_\{F∥G\} est une
application linéaire de E dans E et on a \{π\}\_\{F∥G\} + \{π\}\_\{G∥F\}
=\{ \textbackslash{}mathrm\{Id\}\}\_\{E\}.

Le théorème suivant peut rendre de grands services

Théorème~2.1.9 Soit f : E → F une application linéaire et V un
supplémentaire de
\textbackslash{}mathop\{\textbackslash{}mathrm\{Ker\}\}f dans E. Alors
la restriction de f à V , \{f\}\_\{\{\textbar{}\}\_\{V \}\}, induit un
isomorphisme de V sur
\textbackslash{}mathop\{\textbackslash{}mathrm\{Im\}\}f.

Démonstration Soit f' la restriction de f à V ~; on a
\textbackslash{}mathop\{\textbackslash{}mathrm\{Ker\}\}f'
=\textbackslash{}mathop\{ \textbackslash{}mathrm\{Ker\}\}f ∩ V =
\textbackslash{}\{0\textbackslash{}\} ce qui montre que f' est
injective. De plus, si y
∈\textbackslash{}mathop\{\textbackslash{}mathrm\{Im\}\}f, il existe x ∈
E tel que y = f(x). Cet élément x peut s'écrire x = \{x\}\_\{1\} +
\{x\}\_\{2\} avec \{x\}\_\{1\} ∈ V,\{x\}\_\{2\}
∈\textbackslash{}mathop\{\textbackslash{}mathrm\{Ker\}\}f, d'où y = f(x)
= f(\{x\}\_\{1\}) + f(\{x\}\_\{2\}) = f(\{x\}\_\{1\}) = f'(\{x\}\_\{1\})
ce qui montre que f' est surjective de V sur
\textbackslash{}mathop\{\textbackslash{}mathrm\{Im\}\}f.

Remarque~2.1.5 Appelons g l'isomorphisme réciproque~; on a alors f ∘ g
=\{
\textbackslash{}mathrm\{Id\}\}\_\{\textbackslash{}mathop\{\textbackslash{}mathrm\{Im\}\}
f\} et g ∘ f = \{π\}\_\{V
∥\textbackslash{}mathop\{\textbackslash{}mathrm\{Ker\}\} f\}. Le
résultat suivant n'est qu'un cas particulier utile du théorème énoncé~:

Proposition~2.1.10 Si F et G sont supplémentaires, soit π la projection
canonique de E sur E∕F. Alors la restriction de π à G est un
isomorphisme de G sur E∕F

Remarque~2.1.6 Contrairement au quotient E∕F qui est unique, un
supplémentaire G ne l'est pas~; par contre deux supplémentaires d'un
même sous-espace vectoriel sont isomorphes (puisqu'ils sont tous deux
isomorphes à E∕F).

\paragraph{2.1.6 Algèbres}

Définition~2.1.9 On appelle K-algèbre un quadruplet (A,+,∗,.) tel que
(A,+,∗) est un anneau, (A,+,.) est un K-espace vectoriel avec

\textbackslash{}mathop\{∀\}λ ∈ K, \textbackslash{}mathop\{∀\}x,y ∈ A,
(λx) ∗ y = x ∗ (λy) = λ(x ∗ y)

(avec la distributivité, ces propriétés traduisent la bilinéarité du
produit ∗).

Remarque~2.1.7 Notions évidentes~: sous-algèbres, morphisme d'algèbres.

Exemple~2.1.1 L'ensemble L(E) = L(E,E) des endomorphismes de E est une
K-algèbre, la multiplication étant la composition. Le groupe des
éléments inversibles (automorphismes de E) est noté GL(E).

\paragraph{2.1.7 Familles libres, génératrices. Bases}

Soit E un espace vectoriel, X = \{(\{x\}\_\{i\})\}\_\{i∈I\} une famille
de vecteurs de E. A cette famille on peut associer une application
linéaire \{f\}\_\{X\} : \{K\}\^{}\{(I)\} → E par
f(\{(\{α\}\_\{i\})\}\_\{i∈I\}) =\{\textbackslash{}mathop\{
\textbackslash{}mathop\{∑ \}\} \}\_\{i∈I\}\{α\}\_\{i\}\{x\}\_\{i\}.

Définition~2.1.10 On dit que la famille X est (i) libre si \{f\}\_\{X\}
est injective (ii) génératrice si \{f\}\_\{X\} est surjective (iii) une
base de E si \{f\}\_\{X\} est bijective.

Proposition~2.1.11 La famille X est (i) libre si et seulement si
\textbackslash{}mathop\{∀\}(\{α\}\_\{i\}) ∈ \{K\}\^{}\{(I)\},
\textbackslash{}mathop\{\textbackslash{}mathop\{∑ \}\}
\{α\}\_\{i\}\{x\}\_\{i\} = 0 ⇒\textbackslash{}mathop\{∀\}i ∈ I,
\{α\}\_\{i\} = 0 (ii) génératrice si et seulement si tout élément x de E
s'écrit sous la forme x =\textbackslash{}mathop\{
\textbackslash{}mathop\{∑ \}\} \{α\}\_\{i\}\{x\}\_\{i\} (iii) une base
si et seulement si tout élément x de E s'écrit de manière unique sous la
forme x =\textbackslash{}mathop\{ \textbackslash{}mathop\{∑ \}\}
\{α\}\_\{i\}\{x\}\_\{i\} (on dit alors que les \{α\}\_\{i\} sont les
coordonnées de x dans la base X).

Démonstration Seul (i) n'est pas totalement évident. Il traduit que
\{f\}\_\{X\} est injective si et seulement si son noyau est réduit à la
famille nulle. On remarque alors facilement qu'une famille est libre si
et seulement si toute sous-famille finie est libre.

Définition~2.1.11 On dit qu'une famille est liée lorsqu'elle n'est pas
libre.

Proposition~2.1.12 Toute sous-famille d'une famille libre est libre,
toute surfamille d'une famille génératrice est génératrice. L'image par
une application linéaire injective d'une famille libre est libre.
L'image par une application linéaire surjective d'une famille
génératrice est génératrice. L'image par une application linéaire d'une
famille liée est liée. L'image par un isomorphisme d'une base est une
base.

Démonstration Elémentaire

\paragraph{2.1.8 Théorèmes fondamentaux}

Théorème~2.1.13 Une famille \{(\{x\}\_\{i\})\}\_\{i∈I\} est liée si et
seulement si il existe \{i\}\_\{0\} ∈ I tel que \{x\}\_\{\{i\}\_\{0\}\}
soit combinaison linéaire de la famille
\{(\{x\}\_\{i\})\}\_\{i∈I∖\textbackslash{}\{\{i\}\_\{0\}\textbackslash{}\}\}

Démonstration Si \{x\}\_\{\{i\}\_\{0\}\} =\{\textbackslash{}mathop\{
\textbackslash{}mathop\{∑ \}\}
\}\_\{i∈I∖\textbackslash{}\{\{i\}\_\{0\}\textbackslash{}\}\}\{α\}\_\{i\}\{x\}\_\{i\},
on a \{\textbackslash{}mathop\{\textbackslash{}mathop\{∑ \}\}
\}\_\{i∈I\}\{α\}\_\{i\}\{x\}\_\{i\} = 0 en posant
\{α\}\_\{\{i\}\_\{0\}\} = −1 et la famille est donc liée. En ce qui
concerne la réciproque, on écrit
\{\textbackslash{}mathop\{\textbackslash{}mathop\{∑ \}\}
\}\_\{i∈I\}\{α\}\_\{i\}\{x\}\_\{i\} = 0 avec par exemple
\{α\}\_\{\{i\}\_\{0\}\}\textbackslash{}mathrel\{≠\}0. Alors
\{x\}\_\{\{i\}\_\{0\}\} =\{\textbackslash{}mathop\{
\textbackslash{}mathop\{∑ \}\}
\}\_\{i∈I∖\textbackslash{}\{\{i\}\_\{0\}\textbackslash{}\}\}(−\{
\{α\}\_\{i\} \textbackslash{}over \{α\}\_\{\{i\}\_\{0\}\}\}
)\{x\}\_\{i\}. En adaptant de fa\textbackslash{}c\{c\}on évidente la
démonstration on a

Théorème~2.1.14 Soit \{(\{x\}\_\{i\})\}\_\{i∈I\} une famille liée. On
suppose que la famille
\{(\{x\}\_\{i\})\}\_\{i∈I∖\textbackslash{}\{\{i\}\_\{0\}\textbackslash{}\}\}
est libre. Alors \{x\}\_\{\{i\}\_\{0\}\} est combinaison linéaire de la
famille
\{(\{x\}\_\{i\})\}\_\{i∈I∖\textbackslash{}\{\{i\}\_\{0\}\textbackslash{}\}\}

Démonstration En effet le fait que la famille
\{(\{x\}\_\{i\})\}\_\{i∈I∖\textbackslash{}\{\{i\}\_\{0\}\textbackslash{}\}\}
soit libre implique que nécessairement
\{α\}\_\{\{i\}\_\{0\}\}\textbackslash{}mathrel\{≠\}0.

Théorème~2.1.15 Soit E et F deux K-espaces vectoriels et ℰ =
\{(\{e\}\_\{i\})\}\_\{i∈I\} une base de E. Pour toute famille
\{(\{b\}\_\{i\})\}\_\{i∈I\} d'éléments de F indexée par I, il existe une
unique application linéaire f : E → F vérifiant

\textbackslash{}mathop\{∀\}i ∈ I, f(\{e\}\_\{i\}) = \{b\}\_\{i\}

Démonstration L'application f est bien évidemment définie par
f(\textbackslash{}mathop\{\textbackslash{}mathop\{∑ \}\}
\{x\}\_\{i\}\{e\}\_\{i\}) =\textbackslash{}mathop\{
\textbackslash{}mathop\{∑ \}\} \{x\}\_\{i\}\{b\}\_\{i\}. On vérifie
facilement qu'elle est linéaire.

Remarque~2.1.8 Les deux théorèmes suivants découlent simplement de la
relation

\{\textbackslash{}mathop\{∑ \}\}\_\{i∈I\}\{α\}\_\{i\}\{e\}\_\{i\} =\{
\textbackslash{}mathop\{∑
\}\}\_\{j=1\}\^{}\{p\}\{\textbackslash{}underbrace\{
\{\textbackslash{}mathop\{∑
\}\}\_\{i∈\{I\}\_\{j\}\}\{α\}\_\{i\}\{e\}\_\{i\}\} \}\_\{∈\{E\}\_\{j\}\}

et des caractérisations d'une base et d'une somme directe~:

Théorème~2.1.16 Soit E un K-espace vectoriel , ℰ =
\{(\{e\}\_\{i\})\}\_\{i∈I\} une base de E, I = \{I\}\_\{1\}
∪\textbackslash{}mathop\{\textbackslash{}mathop\{\ldots{}\}\} ∪
\{I\}\_\{p\} une partition de I, \{E\}\_\{j\} =\textbackslash{}mathop\{
\textbackslash{}mathrm\{Vect\}\}(\{e\}\_\{i\}, i ∈ \{I\}\_\{j\}). Alors
E = \{E\}\_\{1\} ⊕\textbackslash{}mathrel\{⋯\} ⊕ \{E\}\_\{p\}.

Théorème~2.1.17 Soit E un K-espace vectoriel , E = \{E\}\_\{1\}
⊕\textbackslash{}mathrel\{⋯\} ⊕ \{E\}\_\{p\} une décomposition en somme
directe. Pour j ∈ {[}1,p{]}, soit \{ℰ\}\_\{j\} =
\{(\{e\}\_\{i\})\}\_\{i∈\{I\}\_\{j\}\} une base de \{E\}\_\{j\} (les
ensembles \{I\}\_\{j\} sont disjoints). Alors la famille \{ℰ\}\_\{1\}
∪\textbackslash{}mathop\{\textbackslash{}mathop\{\ldots{}\}\}
∪\{ℰ\}\_\{p\} est une base de E (dite adaptée à la décomposition en
somme directe).

{[}\href{coursse8.html}{next}{]} {[}\href{coursse7.html}{front}{]}
{[}\href{coursch3.html\#coursse7.html}{up}{]}

\end{document}
