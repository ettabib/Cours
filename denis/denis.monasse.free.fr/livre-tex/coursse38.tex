\documentclass[]{article}
\usepackage[T1]{fontenc}
\usepackage{lmodern}
\usepackage{amssymb,amsmath}
\usepackage{ifxetex,ifluatex}
\usepackage{fixltx2e} % provides \textsubscript
% use upquote if available, for straight quotes in verbatim environments
\IfFileExists{upquote.sty}{\usepackage{upquote}}{}
\ifnum 0\ifxetex 1\fi\ifluatex 1\fi=0 % if pdftex
  \usepackage[utf8]{inputenc}
\else % if luatex or xelatex
  \ifxetex
    \usepackage{mathspec}
    \usepackage{xltxtra,xunicode}
  \else
    \usepackage{fontspec}
  \fi
  \defaultfontfeatures{Mapping=tex-text,Scale=MatchLowercase}
  \newcommand{\euro}{€}
\fi
% use microtype if available
\IfFileExists{microtype.sty}{\usepackage{microtype}}{}
\ifxetex
  \usepackage[setpagesize=false, % page size defined by xetex
              unicode=false, % unicode breaks when used with xetex
              xetex]{hyperref}
\else
  \usepackage[unicode=true]{hyperref}
\fi
\hypersetup{breaklinks=true,
            bookmarks=true,
            pdfauthor={},
            pdftitle={Series absolument convergentes},
            colorlinks=true,
            citecolor=blue,
            urlcolor=blue,
            linkcolor=magenta,
            pdfborder={0 0 0}}
\urlstyle{same}  % don't use monospace font for urls
\setlength{\parindent}{0pt}
\setlength{\parskip}{6pt plus 2pt minus 1pt}
\setlength{\emergencystretch}{3em}  % prevent overfull lines
\setcounter{secnumdepth}{0}
 
/* start css.sty */
.cmr-5{font-size:50%;}
.cmr-7{font-size:70%;}
.cmmi-5{font-size:50%;font-style: italic;}
.cmmi-7{font-size:70%;font-style: italic;}
.cmmi-10{font-style: italic;}
.cmsy-5{font-size:50%;}
.cmsy-7{font-size:70%;}
.cmex-7{font-size:70%;}
.cmex-7x-x-71{font-size:49%;}
.msbm-7{font-size:70%;}
.cmtt-10{font-family: monospace;}
.cmti-10{ font-style: italic;}
.cmbx-10{ font-weight: bold;}
.cmr-17x-x-120{font-size:204%;}
.cmsl-10{font-style: oblique;}
.cmti-7x-x-71{font-size:49%; font-style: italic;}
.cmbxti-10{ font-weight: bold; font-style: italic;}
p.noindent { text-indent: 0em }
td p.noindent { text-indent: 0em; margin-top:0em; }
p.nopar { text-indent: 0em; }
p.indent{ text-indent: 1.5em }
@media print {div.crosslinks {visibility:hidden;}}
a img { border-top: 0; border-left: 0; border-right: 0; }
center { margin-top:1em; margin-bottom:1em; }
td center { margin-top:0em; margin-bottom:0em; }
.Canvas { position:relative; }
li p.indent { text-indent: 0em }
.enumerate1 {list-style-type:decimal;}
.enumerate2 {list-style-type:lower-alpha;}
.enumerate3 {list-style-type:lower-roman;}
.enumerate4 {list-style-type:upper-alpha;}
div.newtheorem { margin-bottom: 2em; margin-top: 2em;}
.obeylines-h,.obeylines-v {white-space: nowrap; }
div.obeylines-v p { margin-top:0; margin-bottom:0; }
.overline{ text-decoration:overline; }
.overline img{ border-top: 1px solid black; }
td.displaylines {text-align:center; white-space:nowrap;}
.centerline {text-align:center;}
.rightline {text-align:right;}
div.verbatim {font-family: monospace; white-space: nowrap; text-align:left; clear:both; }
.fbox {padding-left:3.0pt; padding-right:3.0pt; text-indent:0pt; border:solid black 0.4pt; }
div.fbox {display:table}
div.center div.fbox {text-align:center; clear:both; padding-left:3.0pt; padding-right:3.0pt; text-indent:0pt; border:solid black 0.4pt; }
div.minipage{width:100%;}
div.center, div.center div.center {text-align: center; margin-left:1em; margin-right:1em;}
div.center div {text-align: left;}
div.flushright, div.flushright div.flushright {text-align: right;}
div.flushright div {text-align: left;}
div.flushleft {text-align: left;}
.underline{ text-decoration:underline; }
.underline img{ border-bottom: 1px solid black; margin-bottom:1pt; }
.framebox-c, .framebox-l, .framebox-r { padding-left:3.0pt; padding-right:3.0pt; text-indent:0pt; border:solid black 0.4pt; }
.framebox-c {text-align:center;}
.framebox-l {text-align:left;}
.framebox-r {text-align:right;}
span.thank-mark{ vertical-align: super }
span.footnote-mark sup.textsuperscript, span.footnote-mark a sup.textsuperscript{ font-size:80%; }
div.tabular, div.center div.tabular {text-align: center; margin-top:0.5em; margin-bottom:0.5em; }
table.tabular td p{margin-top:0em;}
table.tabular {margin-left: auto; margin-right: auto;}
div.td00{ margin-left:0pt; margin-right:0pt; }
div.td01{ margin-left:0pt; margin-right:5pt; }
div.td10{ margin-left:5pt; margin-right:0pt; }
div.td11{ margin-left:5pt; margin-right:5pt; }
table[rules] {border-left:solid black 0.4pt; border-right:solid black 0.4pt; }
td.td00{ padding-left:0pt; padding-right:0pt; }
td.td01{ padding-left:0pt; padding-right:5pt; }
td.td10{ padding-left:5pt; padding-right:0pt; }
td.td11{ padding-left:5pt; padding-right:5pt; }
table[rules] {border-left:solid black 0.4pt; border-right:solid black 0.4pt; }
.hline hr, .cline hr{ height : 1px; margin:0px; }
.tabbing-right {text-align:right;}
span.TEX {letter-spacing: -0.125em; }
span.TEX span.E{ position:relative;top:0.5ex;left:-0.0417em;}
a span.TEX span.E {text-decoration: none; }
span.LATEX span.A{ position:relative; top:-0.5ex; left:-0.4em; font-size:85%;}
span.LATEX span.TEX{ position:relative; left: -0.4em; }
div.float img, div.float .caption {text-align:center;}
div.figure img, div.figure .caption {text-align:center;}
.marginpar {width:20%; float:right; text-align:left; margin-left:auto; margin-top:0.5em; font-size:85%; text-decoration:underline;}
.marginpar p{margin-top:0.4em; margin-bottom:0.4em;}
.equation td{text-align:center; vertical-align:middle; }
td.eq-no{ width:5%; }
table.equation { width:100%; } 
div.math-display, div.par-math-display{text-align:center;}
math .texttt { font-family: monospace; }
math .textit { font-style: italic; }
math .textsl { font-style: oblique; }
math .textsf { font-family: sans-serif; }
math .textbf { font-weight: bold; }
.partToc a, .partToc, .likepartToc a, .likepartToc {line-height: 200%; font-weight:bold; font-size:110%;}
.chapterToc a, .chapterToc, .likechapterToc a, .likechapterToc, .appendixToc a, .appendixToc {line-height: 200%; font-weight:bold;}
.index-item, .index-subitem, .index-subsubitem {display:block}
.caption td.id{font-weight: bold; white-space: nowrap; }
table.caption {text-align:center;}
h1.partHead{text-align: center}
p.bibitem { text-indent: -2em; margin-left: 2em; margin-top:0.6em; margin-bottom:0.6em; }
p.bibitem-p { text-indent: 0em; margin-left: 2em; margin-top:0.6em; margin-bottom:0.6em; }
.paragraphHead, .likeparagraphHead { margin-top:2em; font-weight: bold;}
.subparagraphHead, .likesubparagraphHead { font-weight: bold;}
.quote {margin-bottom:0.25em; margin-top:0.25em; margin-left:1em; margin-right:1em; text-align:justify;}
.verse{white-space:nowrap; margin-left:2em}
div.maketitle {text-align:center;}
h2.titleHead{text-align:center;}
div.maketitle{ margin-bottom: 2em; }
div.author, div.date {text-align:center;}
div.thanks{text-align:left; margin-left:10%; font-size:85%; font-style:italic; }
div.author{white-space: nowrap;}
.quotation {margin-bottom:0.25em; margin-top:0.25em; margin-left:1em; }
h1.partHead{text-align: center}
.sectionToc, .likesectionToc {margin-left:2em;}
.subsectionToc, .likesubsectionToc {margin-left:4em;}
.subsubsectionToc, .likesubsubsectionToc {margin-left:6em;}
.frenchb-nbsp{font-size:75%;}
.frenchb-thinspace{font-size:75%;}
.figure img.graphics {margin-left:10%;}
/* end css.sty */

\title{Series absolument convergentes}
\author{}
\date{}

\begin{document}
\maketitle

\textbf{Warning: \href{http://www.math.union.edu/locate/jsMath}{jsMath}
requires JavaScript to process the mathematics on this page.\\ If your
browser supports JavaScript, be sure it is enabled.}

\begin{center}\rule{3in}{0.4pt}\end{center}

{[}\href{coursse39.html}{next}{]} {[}\href{coursse37.html}{prev}{]}
{[}\href{coursse37.html\#tailcoursse37.html}{prev-tail}{]}
{[}\hyperref[tailcoursse38.html]{tail}{]}
{[}\href{coursch8.html\#coursse38.html}{up}{]}

\subsubsection{7.4 Séries absolument convergentes}

\paragraph{7.4.1 Notion de convergence absolue}

Définition~7.4.1 Soit E un espace vectoriel normé. On dit que la série
\textbackslash{}mathop\{\textbackslash{}mathop\{∑ \}\} \{x\}\_\{n\} est
absolument convergente si la série à termes réels positifs
\textbackslash{}mathop\{\textbackslash{}mathop\{∑ \}\}
\textbackslash{}\textbar{}\{x\}\_\{n\}\textbackslash{}\textbar{}
converge.

Théorème~7.4.1 Soit E un espace vectoriel normé~complet. Alors toute
série absolument convergente à terme général dans E est convergente.

Démonstration On a
\textbackslash{}\textbar{}\{\textbackslash{}mathop\{\textbackslash{}mathop\{∑
\}\} \}\_\{n=p\}\^{}\{q\}\{x\}\_\{n\}\textbackslash{}\textbar{}
≤\{\textbackslash{}mathop\{\textbackslash{}mathop\{∑ \}\}
\}\_\{n=p\}\^{}\{q\}\textbackslash{}\textbar{}\{x\}\_\{n\}\textbackslash{}\textbar{}.
Si la série \textbackslash{}mathop\{\textbackslash{}mathop\{∑ \}\}
\textbackslash{}\textbar{}\{x\}\_\{n\}\textbackslash{}\textbar{}
converge, elle vérifie le critère de Cauchy, il en est donc de même de
la série \textbackslash{}mathop\{\textbackslash{}mathop\{∑ \}\}
\{x\}\_\{n\} et donc celle-ci converge.

Remarque~7.4.1 L'avantage est bien entendu de ramener l'étude à celle
d'une série à termes réels positifs.

\paragraph{7.4.2 Critères de convergence absolue}

Théorème~7.4.2 Soit \textbackslash{}mathop\{\textbackslash{}mathop\{∑
\}\} \{x\}\_\{n\} et \textbackslash{}mathop\{\textbackslash{}mathop\{∑
\}\} \{y\}\_\{n\} deux séries telles que \{x\}\_\{n\} = O(\{y\}\_\{n\})
et \textbackslash{}mathop\{\textbackslash{}mathop\{∑ \}\} \{y\}\_\{n\}
est absolument convergente. Alors
\textbackslash{}mathop\{\textbackslash{}mathop\{∑ \}\} \{x\}\_\{n\}
converge absolument.

Démonstration On a \{x\}\_\{n\} = O(\{y\}\_\{n\})
\textbackslash{}mathrel\{⇔\}
\textbackslash{}\textbar{}\{x\}\_\{n\}\textbackslash{}\textbar{} =
O(\textbackslash{}\textbar{}\{y\}\_\{n\}\textbackslash{}\textbar{}) et
il suffit d'appliquer le théorème de comparaison pour les séries à
termes réels positifs.

Remarque~7.4.2 Le théorème ci-dessus reste valable même si les termes
généraux \{x\}\_\{n\} et \{y\}\_\{n\} ne sont pas dans le même espace
vectoriel normé. En particulier, la série étalon
\textbackslash{}mathop\{\textbackslash{}mathop\{∑ \}\} \{y\}\_\{n\} sera
le plus souvent une série à termes réels positifs.

En ce qui concerne les équivalents, on a un résultat plus fort

Théorème~7.4.3 Soit \textbackslash{}mathop\{\textbackslash{}mathop\{∑
\}\} \{x\}\_\{n\} une série à terme général dans l'espace vectoriel
normé~E et \textbackslash{}mathop\{\textbackslash{}mathop\{∑ \}\}
\{y\}\_\{n\} une série à termes réels positifs. On suppose qu'il existe
ℓ ∈ E ∖\textbackslash{}\{0\textbackslash{}\} tel que \{x\}\_\{n\} ∼
ℓ\{y\}\_\{n\}. Alors les deux séries sont simultanément convergentes ou
divergentes.

Démonstration Si \textbackslash{}mathop\{\textbackslash{}mathop\{∑ \}\}
\{y\}\_\{n\} converge, on a \{x\}\_\{n\} = O(\{y\}\_\{n\}) et
\{y\}\_\{n\} ≥ 0, donc la série
\textbackslash{}mathop\{\textbackslash{}mathop\{∑ \}\} \{x\}\_\{n\} est
absolument convergente. Inversement, supposons que la série
\textbackslash{}mathop\{\textbackslash{}mathop\{∑ \}\} \{x\}\_\{n\}
converge. Puisque \{x\}\_\{n\} − ℓ\{y\}\_\{n\} = o(ℓ\{y\}\_\{n\}), il
existe N ∈ ℕ tel que n ≥ N ⇒\textbackslash{}\textbar{} \{x\}\_\{n\} −
ℓ\{y\}\_\{n\}\textbackslash{}\textbar{} ≤\{ 1 \textbackslash{}over 2\}
\textbackslash{}\textbar{}ℓ\{y\}\_\{n\}\textbackslash{}\textbar{} =\{ 1
\textbackslash{}over 2\}
\textbackslash{}\textbar{}ℓ\textbackslash{}\textbar{}\{y\}\_\{n\}. En
sommant on obtient
\textbackslash{}\textbar{}\{\textbackslash{}mathop\{\textbackslash{}mathop\{∑
\}\} \}\_\{n=p\}\^{}\{q\}\{x\}\_\{n\} −
ℓ\{\textbackslash{}mathop\{\textbackslash{}mathop\{∑ \}\}
\}\_\{n=p\}\^{}\{q\}\{y\}\_\{n\}\textbackslash{}\textbar{} ≤\{ 1
\textbackslash{}over 2\}
\textbackslash{}\textbar{}ℓ\textbackslash{}\textbar{}\{\textbackslash{}mathop\{\textbackslash{}mathop\{∑
\}\} \}\_\{n=p\}\^{}\{q\}\{y\}\_\{n\}. On en déduit

\textbackslash{}begin\{eqnarray*\}
\textbackslash{}\textbar{}ℓ\textbackslash{}\textbar{}\{\textbackslash{}mathop\{∑
\}\}\_\{n=p\}\^{}\{q\}\{y\}\_\{ n\}\& =\&
\textbackslash{}\textbar{}ℓ\{\textbackslash{}mathop\{∑
\}\}\_\{n=p\}\^{}\{q\}\{y\}\_\{ n\}\textbackslash{}\textbar{}
≤\textbackslash{}\textbar{} ℓ\{\textbackslash{}mathop\{∑
\}\}\_\{n=p\}\^{}\{q\}\{y\}\_\{ n\} −\{\textbackslash{}mathop\{∑
\}\}\_\{n=p\}\^{}\{q\}\{x\}\_\{ n\}\textbackslash{}\textbar{}
+\textbackslash{}\textbar{}\{ \textbackslash{}mathop\{∑
\}\}\_\{n=p\}\^{}\{q\}\{x\}\_\{ n\}\textbackslash{}\textbar{}\%\&
\textbackslash{}\textbackslash{} \& ≤\&\{ 1 \textbackslash{}over 2\}
\textbackslash{}\textbar{}ℓ\textbackslash{}\textbar{}\{\textbackslash{}mathop\{∑
\}\}\_\{n=p\}\^{}\{q\}\{y\}\_\{ n\} +\textbackslash{}\textbar{}\{
\textbackslash{}mathop\{∑ \}\}\_\{n=p\}\^{}\{q\}\{x\}\_\{
n\}\textbackslash{}\textbar{} \%\& \textbackslash{}\textbackslash{}
\textbackslash{}end\{eqnarray*\}

d'où en définitive \{\textbackslash{}mathop\{\textbackslash{}mathop\{∑
\}\} \}\_\{n=p\}\^{}\{q\}\{y\}\_\{n\} ≤\{ 2 \textbackslash{}over
\textbackslash{}\textbar{}ℓ\textbackslash{}\textbar{}\}
\textbackslash{}\textbar{}\{\textbackslash{}mathop\{
\textbackslash{}mathop\{∑ \}\}
\}\_\{n=p\}\^{}\{q\}\{x\}\_\{n\}\textbackslash{}\textbar{}. La série
\textbackslash{}mathop\{\textbackslash{}mathop\{∑ \}\} \{x\}\_\{n\}
converge, donc vérifie le critère de Cauchy. Il en est donc de même de
la série \textbackslash{}mathop\{\textbackslash{}mathop\{∑ \}\}
\{y\}\_\{n\}, qui est par suite convergente.

\paragraph{7.4.3 Règles classiques}

Il suffit maintenant d'appliquer ces résultats à des séries étalons,
comme les séries de Riemann ou les séries géométriques.

Lemme~7.4.4 Soit a un nombre complexe. La série
\textbackslash{}mathop\{\textbackslash{}mathop\{∑ \}\} \{a\}\^{}\{n\}
converge si et seulement si~\textbar{}a\textbar{} \textless{} 1.

Démonstration La condition est évidemment nécessaire puisque le terme
général doit tendre vers 0. Supposons la vérifiée. On a
\{\textbackslash{}mathop\{\textbackslash{}mathop\{∑ \}\}
\}\_\{p=0\}\^{}\{n\}\{a\}\^{}\{p\} =\{ 1−\{a\}\^{}\{n+1\}
\textbackslash{}over 1−a\} qui admet la limite \{ 1 \textbackslash{}over
1−a\} . Donc la série converge.

Théorème~7.4.5 (règle de d'Alembert). Soit E un espace vectoriel normé
complet. Soit \textbackslash{}mathop\{\textbackslash{}mathop\{∑ \}\}
\{x\}\_\{n\} une série à termes dans E telle que pour tout n ∈ ℕ,
\{x\}\_\{n\}\textbackslash{}mathrel\{≠\}0 et telle que la suite (\{
\textbackslash{}\textbar{}\{x\}\_\{n+1\}\textbackslash{}\textbar{}
\textbackslash{}over
\textbackslash{}\textbar{}\{x\}\_\{n\}\textbackslash{}\textbar{}\} )
admet une limite ℓ ∈ ℝ ∪\textbackslash{}\{ + ∞\textbackslash{}\}. Alors

\begin{itemize}
\itemsep1pt\parskip0pt\parsep0pt
\item
  (i) si ℓ \textless{} 1, la série converge absolument
\item
  (ii) si ℓ \textgreater{} 1, la série diverge
\end{itemize}

Démonstration (i) Si ℓ \textless{} 1, soit ρ tel que ℓ \textless{} ρ
\textless{} 1~; il existe N ∈ ℕ tel que n ≥ N ⇒\{
\textbackslash{}\textbar{}\{x\}\_\{n+1\}\textbackslash{}\textbar{}
\textbackslash{}over
\textbackslash{}\textbar{}\{x\}\_\{n\}\textbackslash{}\textbar{}\} ≤ ρ
soit \textbackslash{}\textbar{}\{x\}\_\{n+1\}\textbackslash{}\textbar{}
≤ ρ\textbackslash{}\textbar{}\{x\}\_\{n\}\textbackslash{}\textbar{}. On
a donc alors par récurrence
\textbackslash{}\textbar{}\{x\}\_\{n\}\textbackslash{}\textbar{} ≤
\{ρ\}\^{}\{n−N\}\textbackslash{}\textbar{}\{x\}\_\{N\}\textbackslash{}\textbar{}
= O(\{ρ\}\^{}\{n\}). Comme la série
\textbackslash{}mathop\{\textbackslash{}mathop\{∑ \}\} \{ρ\}\^{}\{n\}
converge, la série \textbackslash{}mathop\{\textbackslash{}mathop\{∑
\}\} \{x\}\_\{n\} converge absolument.

(ii) Si ℓ \textgreater{} 1, il existe N ∈ ℕ tel que n ≥ N ⇒\{
\textbackslash{}\textbar{}\{x\}\_\{n+1\}\textbackslash{}\textbar{}
\textbackslash{}over
\textbackslash{}\textbar{}\{x\}\_\{n\}\textbackslash{}\textbar{}\}
\textgreater{} 1 soit
\textbackslash{}\textbar{}\{x\}\_\{n+1\}\textbackslash{}\textbar{}
\textgreater{}\textbackslash{}\textbar{}
\{x\}\_\{n\}\textbackslash{}\textbar{}. On a donc alors par récurrence
\textbackslash{}\textbar{}\{x\}\_\{n\}\textbackslash{}\textbar{}
\textgreater{}\textbackslash{}\textbar{}
\{x\}\_\{N\}\textbackslash{}\textbar{}. La suite (\{x\}\_\{n\}) ne peut
donc pas avoir 0 pour limite et la série diverge.

Remarque~7.4.3 Si ℓ = 1 on ne peut rien conclure comme le montre
l'exemple des séries de Riemann. Lorsque la règle de d'Alembert
s'applique, elle conduit à des convergences rapides (de type
exponentielle) ou des divergences grossières (le terme général ne tend
pas vers 0).

Théorème~7.4.6 (règle de Riemann). Soit E un espace vectoriel normé.
Soit \textbackslash{}mathop\{\textbackslash{}mathop\{∑ \}\} \{x\}\_\{n\}
une série à termes dans E.

\begin{itemize}
\itemsep1pt\parskip0pt\parsep0pt
\item
  (i) S'il existe α \textgreater{} 1 tel que \{x\}\_\{n\} = O(\{ 1
  \textbackslash{}over \{n\}\^{}\{α\}\} ), alors la série converge
  absolument
\item
  (ii) S'il existe α ∈ ℝ et ℓ ∈ E ∖\textbackslash{}\{0\textbackslash{}\}
  tels que \{x\}\_\{n\} ∼\{ ℓ \textbackslash{}over \{n\}\^{}\{α\}\}
  alors la série converge absolument si α \textgreater{} 1 et diverge si
  α ≤ 1.
\item
  (iii) Si E = ℝ et \{x\}\_\{n\} ≥ 0, et s'il existe α ≤ 1 et ℓ
  \textgreater{} 0 (y compris + ∞) tel que
  \textbackslash{}mathop\{lim\}\{n\}\^{}\{α\}\{x\}\_\{n\} = ℓ, alors la
  série diverge.
\end{itemize}

Démonstration (i) et (ii) résultent de ce qui précède. Pour (iii), il
suffit de remarquer que les hypothèses impliquent que \{ 1
\textbackslash{}over \{n\}\^{}\{α\}\} = O(\{x\}\_\{n\}). Comme α ≤ 1, la
série \textbackslash{}mathop\{\textbackslash{}mathop\{∑ \}\} \{ 1
\textbackslash{}over \{n\}\^{}\{α\}\} diverge et donc aussi la série
\textbackslash{}mathop\{\textbackslash{}mathop\{∑ \}\} \{x\}\_\{n\}.

\paragraph{7.4.4 Règles complémentaires}

Théorème~7.4.7 (règle de Cauchy). Soit E un espace vectoriel normé
complet. Soit \textbackslash{}mathop\{\textbackslash{}mathop\{∑ \}\}
\{x\}\_\{n\} une série à termes dans E telle que la suite
\textbackslash{}left
(\textbackslash{}root\{n\}\textbackslash{}of\{\textbackslash{}\textbar{}\{x\}\_\{n\}\textbackslash{}\textbar{}\}\textbackslash{}right
) admet une limite ℓ ∈ ℝ ∪\textbackslash{}\{ + ∞\textbackslash{}\}.
Alors

\begin{itemize}
\itemsep1pt\parskip0pt\parsep0pt
\item
  (i) si ℓ \textless{} 1, la série converge absolument
\item
  (ii) si ℓ \textgreater{} 1, la série diverge
\end{itemize}

Démonstration (i) Si ℓ \textless{} 1, soit ρ tel que ℓ \textless{} ρ
\textless{} 1~; il existe N ∈ ℕ tel que n ≥ N
⇒\textbackslash{}root\{n\}\textbackslash{}of\{\textbackslash{}\textbar{}\{x\}\_\{n\}\textbackslash{}\textbar{}\}
≤ ρ soit
\textbackslash{}\textbar{}\{x\}\_\{n\}\textbackslash{}\textbar{} ≤
\{ρ\}\^{}\{n\}. Comme la série
\textbackslash{}mathop\{\textbackslash{}mathop\{∑ \}\} \{ρ\}\^{}\{n\}
converge, la série \textbackslash{}mathop\{\textbackslash{}mathop\{∑
\}\} \{x\}\_\{n\} converge absolument.

(ii) Si ℓ \textgreater{} 1, il existe N ∈ ℕ tel que n ≥ N
⇒\textbackslash{}root\{n\}\textbackslash{}of\{\textbackslash{}\textbar{}\{x\}\_\{n\}\textbackslash{}\textbar{}\}
\textgreater{} 1 soit
\textbackslash{}\textbar{}\{x\}\_\{n\}\textbackslash{}\textbar{}
\textgreater{} 1. La suite (\{x\}\_\{n\}) ne peut donc pas avoir 0 pour
limite et la série diverge.

Théorème~7.4.8 (règle de Duhamel). Soit
\textbackslash{}mathop\{\textbackslash{}mathop\{∑ \}\} \{x\}\_\{n\} une
série à termes dans \{ℝ\}\^{}\{+\} telle que pour tout n ∈ ℕ,
\{x\}\_\{n\}\textbackslash{}mathrel\{≠\}0 et telle que \{ \{x\}\_\{n+1\}
\textbackslash{}over \{x\}\_\{n\}\} = 1 −\{ λ \textbackslash{}over n\} +
o(\{ 1 \textbackslash{}over n\} ) . Alors

\begin{itemize}
\itemsep1pt\parskip0pt\parsep0pt
\item
  (i) si λ \textgreater{} 1, la série converge
\item
  (ii) si λ \textless{} 1, la série diverge
\end{itemize}

Démonstration Posons \{y\}\_\{n\} =\{ 1 \textbackslash{}over
\{n\}\^{}\{α\}\} . On a \{ \{y\}\_\{n+1\} \textbackslash{}over
\{y\}\_\{n\}\} = 1 −\{ α \textbackslash{}over n\} + o(\{ 1
\textbackslash{}over n\} ). On en déduit que si
α\textbackslash{}mathrel\{≠\}λ, \{ \{x\}\_\{n+1\} \textbackslash{}over
\{x\}\_\{n\}\} −\{ \{y\}\_\{n+1\} \textbackslash{}over \{y\}\_\{n\}\}
∼\{ α−λ \textbackslash{}over n\} est pour n assez grand du signe de α −
λ. Si λ \textless{} 1, soit α tel que λ \textless{} α \textless{} 1. On
a donc pour n ≥ N, \{ \{x\}\_\{n+1\} \textbackslash{}over \{x\}\_\{n\}\}
≥\{ \{y\}\_\{n+1\} \textbackslash{}over \{y\}\_\{n\}\} et comme la série
\textbackslash{}mathop\{\textbackslash{}mathop\{∑ \}\} \{y\}\_\{n\}
diverge (car α \textless{} 1), la série
\textbackslash{}mathop\{\textbackslash{}mathop\{∑ \}\} \{x\}\_\{n\}
diverge. Si λ \textgreater{} 1, soit α tel que λ \textgreater{} α
\textgreater{} 1. On a donc pour n ≥ N, \{ \{x\}\_\{n+1\}
\textbackslash{}over \{x\}\_\{n\}\} ≤\{ \{y\}\_\{n+1\}
\textbackslash{}over \{y\}\_\{n\}\} et comme la série
\textbackslash{}mathop\{\textbackslash{}mathop\{∑ \}\} \{y\}\_\{n\}
converge (car α \textgreater{} 1), la série
\textbackslash{}mathop\{\textbackslash{}mathop\{∑ \}\} \{x\}\_\{n\}
converge.

\paragraph{7.4.5 Comparaison à une intégrale}

Théorème~7.4.9 Soit f : {[}0,+∞{[}→ ℂ de classe \{C\}\^{}\{1\} telle que
f' soit intégrable sur {[}0,+∞{[}. Posons \{w\}\_\{n\}
=\{\textbackslash{}mathop\{∫ \} \}\_\{n−1\}\^{}\{n\}f(t) dt − f(n).
Alors la série \textbackslash{}mathop\{\textbackslash{}mathop\{∑ \}\}
\{w\}\_\{n\} est absolument convergente.

Démonstration On a par une intégration par parties

\textbackslash{}begin\{eqnarray*\} \{\textbackslash{}mathop\{∫ \}
\}\_\{n−1\}\^{}\{n\}(t − n + 1)f'(t) dt\& =\&\{ \textbackslash{}left
{[}(t − n + 1)f(t)\textbackslash{}right {]}\}\_\{ n−1\}\^{}\{n\}
−\{\textbackslash{}mathop\{∫ \} \}\_\{n−1\}\^{}\{n\}f(t) dt\%\&
\textbackslash{}\textbackslash{} \& =\& −\{w\}\_\{n\} \%\&
\textbackslash{}\textbackslash{} \textbackslash{}end\{eqnarray*\}

On en déduit que

\textbar{}\{w\}\_\{n\}\textbar{}≤\{\textbackslash{}mathop\{∫ \}
\}\_\{n−1\}\^{}\{n\}(t − n + 1)\textbar{}f'(t)\textbar{} dt
≤\{\textbackslash{}mathop\{∫ \}
\}\_\{n−1\}\^{}\{n\}\textbar{}f'(t)\textbar{} dt

et donc

\{\textbackslash{}mathop\{∑ \}\}\_\{p=1\}\^{}\{n\}\textbar{}\{w\}\_\{
p\}\textbar{}≤\{\textbackslash{}mathop\{\textbackslash{}mathop\{∫ \} \}
\}\_\{0\}\^{}\{n\}\textbar{}f'(t)\textbar{} dt
≤\{\textbackslash{}mathop\{\textbackslash{}mathop\{∫ \} \}
\}\_\{0\}\^{}\{+∞\}\textbar{}f'(t)\textbar{} dt

ce qui montre la convergence de la série à termes positifs
\textbackslash{}mathop\{\textbackslash{}mathop\{∑ \}\}
\textbar{}\{w\}\_\{n\}\textbar{} et donc la convergence absolue de la
série.

Corollaire~7.4.10 Soit f : {[}0,+∞{[}→ ℂ de classe \{C\}\^{}\{1\} telle
que f et f' soient intégrables sur {[}0,+∞{[}. Alors la série
\textbackslash{}mathop\{\textbackslash{}mathop\{∑ \}\} f(n) est
absolument convergente.

Démonstration En effet la série
\textbackslash{}mathop\{\textbackslash{}mathop\{∑ \}\}
\{\textbackslash{}mathop\{∫ \}
\}\_\{n−1\}\^{}\{n\}\textbar{}f(t)\textbar{} dt est convergente car

\{\textbackslash{}mathop\{∑ \}\}\_\{p=1\}\^{}\{n\}\{
\textbackslash{}mathop\{\textbackslash{}mathop\{∫ \} \}
\}\_\{n−1\}\^{}\{n\}\textbar{}f(t)\textbar{} dt =\{
\textbackslash{}mathop\{\textbackslash{}mathop\{∫ \} \}
\}\_\{0\}\^{}\{n\}\textbar{}f(t)\textbar{} dt
≤\{\textbackslash{}mathop\{\textbackslash{}mathop\{∫ \} \}
\}\_\{0\}\^{}\{+∞\}\textbar{}f(t)\textbar{} dt

et comme \textbackslash{}left \textbar{}\{\textbackslash{}mathop\{∫ \}
\}\_\{n−1\}\^{}\{n\}f(t) dt\textbackslash{}right
\textbar{}≤\{\textbackslash{}mathop\{∫ \}
\}\_\{n−1\}\^{}\{n\}\textbar{}f(t)\textbar{} dt, la série
\textbackslash{}mathop\{\textbackslash{}mathop\{∑ \}\}
\{\textbackslash{}mathop\{∫ \} \}\_\{n−1\}\^{}\{n\}f(t) dt est
absolument convergente. Comme
\textbackslash{}mathop\{\textbackslash{}mathop\{∑ \}\} \{w\}\_\{n\} est
également absolument convergente, il en est de même de la série
\textbackslash{}mathop\{\textbackslash{}mathop\{∑ \}\} f(n).

{[}\href{coursse39.html}{next}{]} {[}\href{coursse37.html}{prev}{]}
{[}\href{coursse37.html\#tailcoursse37.html}{prev-tail}{]}
{[}\href{coursse38.html}{front}{]}
{[}\href{coursch8.html\#coursse38.html}{up}{]}

\end{document}
