\section{Cardinaux et entiers naturels}

\subsection{Notion de cardinal}

\begin{de}
\index{cardinal!définition}
On dit que deux ensembles $E$ et $F$ ont même cardinal s'il existe une bijection de $E$ sur $F$. On notera $\operatorname{Card}(E)$ ou encore $|E|$ le cardinal d'un ensemble $E$.
\end{de}

\begin{rem}
\index{cardinal!relation d'équivalence}
La relation "il existe une bijection de $E$ sur $F$" est une relation d'équivalence ; les cardinaux sont en quelque sorte les classes d'équivalence pour cette relation (pas tout à fait puisque l'ensemble de tous les ensembles n'existe pas).
\end{rem}

\begin{de}
\index{cardinal!opérations}
On définit des opérations sur les cardinaux en posant :
\begin{align*}
\operatorname{Card}(A) + \operatorname{Card}(B) &= \operatorname{Card}(A \times \{0\} \cup B \times \{1\}) \\
\operatorname{Card}(A) \cdot \operatorname{Card}(B) &= \operatorname{Card}(A \times B)
\end{align*}
\end{de}

\begin{rem}
Cette définition est justifiée par le fait que si on a $\operatorname{Card}(A) = \operatorname{Card}(A')$ et $\operatorname{Card}(B) = \operatorname{Card}(B')$, on a aussi
\[ \operatorname{Card}(A \times \{0\} \cup B \times \{1\}) = \operatorname{Card}(A' \times \{0\} \cup B' \times \{1\}) \]
et $\operatorname{Card}(A \times B) = \operatorname{Card}(A' \times B')$, comme on le vérifie facilement en construisant les bijections appropriées.
\end{rem}

\begin{de}
\index{cardinal!ordre}
On pose $\operatorname{Card}(A) \leq \operatorname{Card}(B)$ s'il existe une injection de $A$ dans $B$.
\end{de}

\begin{thm}[Théorème de Cantor-Bernstein]
\index{théorème!Cantor-Bernstein}
S'il existe une injection de $A$ dans $B$ et une injection de $B$ dans $A$, alors il existe une bijection de $A$ sur $B$.
\end{thm}

\subsection{Les entiers naturels}

\begin{de}
\index{ensemble!fini}
Un ensemble $A$ est fini si $\operatorname{Card}(A) < \operatorname{Card}(A) + 1$ (c'est équivalent à : il n'existe pas de bijection de $A$ sur une partie stricte de $A$).
\end{de}

\begin{prop}
\index{entiers naturels}
L'ensemble des cardinaux finis forme un ensemble totalement ordonné appelé ensemble des entiers naturels et noté $\mathbb{N}$.
\end{prop}

\begin{axiom}[de Peano]
\index{axiomes de Peano}
$\mathbb{N}$ est un ensemble infini où toute partie non vide a un plus petit élément et où toute partie non vide majorée a un plus grand élément.
\end{axiom}

\begin{thm}[Principe de récurrence forte]
\index{récurrence!forte}
Soit $P(n)$ une propriété qui peut être vraie ou fausse pour tout entier naturel $n$. On suppose que $P(n_0)$ est vraie et que si $P(n)$ est vraie, alors $P(n + 1)$ est vraie. Alors $P(n)$ est vraie pour tout $n \geq n_0$.
\end{thm}

\begin{thm}[Principe de récurrence faible]
\index{récurrence!faible}
On suppose que $P(n_0)$ est vraie et que si $P(n_0 + 1), P(n_0 + 2), \ldots, P(n)$ sont vraies, alors $P(n + 1)$ est vraie. Alors $P(n)$ est vraie pour tout $n \geq n_0$.
\end{thm}

\begin{proof}
Soit $X$ l'ensemble des $n \geq n_0$ tels que $P(n)$ soit fausse et supposons que $X$ est non vide ; alors il admet un plus petit élément $n_1 \in X$. Comme $n_0 \notin X$, on a $n_1 > n_0$ ; mais alors $n_0, \ldots, n_1 - 1 \notin X$ et donc $P(n_0), \ldots, P(n_1 - 1)$ sont vraies ; il en est donc de même de $P(n_1)$, soit $n_1 \notin X$. C'est absurde. Donc $X = \varnothing$, et par conséquent, $P(n)$ est vraie pour tout $n \geq n_0$.
\end{proof}