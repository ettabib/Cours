\documentclass[]{article}
\usepackage[T1]{fontenc}
\usepackage{lmodern}
\usepackage{amssymb,amsmath}
\usepackage{ifxetex,ifluatex}
\usepackage{fixltx2e} % provides \textsubscript
% use upquote if available, for straight quotes in verbatim environments
\IfFileExists{upquote.sty}{\usepackage{upquote}}{}
\ifnum 0\ifxetex 1\fi\ifluatex 1\fi=0 % if pdftex
  \usepackage[utf8]{inputenc}
\else % if luatex or xelatex
  \ifxetex
    \usepackage{mathspec}
    \usepackage{xltxtra,xunicode}
  \else
    \usepackage{fontspec}
  \fi
  \defaultfontfeatures{Mapping=tex-text,Scale=MatchLowercase}
  \newcommand{\euro}{€}
\fi
% use microtype if available
\IfFileExists{microtype.sty}{\usepackage{microtype}}{}
\ifxetex
  \usepackage[setpagesize=false, % page size defined by xetex
              unicode=false, % unicode breaks when used with xetex
              xetex]{hyperref}
\else
  \usepackage[unicode=true]{hyperref}
\fi
\hypersetup{breaklinks=true,
            bookmarks=true,
            pdfauthor={},
            pdftitle={Cardinaux et entiers naturels},
            colorlinks=true,
            citecolor=blue,
            urlcolor=blue,
            linkcolor=magenta,
            pdfborder={0 0 0}}
\urlstyle{same}  % don't use monospace font for urls
\setlength{\parindent}{0pt}
\setlength{\parskip}{6pt plus 2pt minus 1pt}
\setlength{\emergencystretch}{3em}  % prevent overfull lines
\setcounter{secnumdepth}{0}
 
/* start css.sty */
.cmr-5{font-size:50%;}
.cmr-7{font-size:70%;}
.cmmi-5{font-size:50%;font-style: italic;}
.cmmi-7{font-size:70%;font-style: italic;}
.cmmi-10{font-style: italic;}
.cmsy-5{font-size:50%;}
.cmsy-7{font-size:70%;}
.cmex-7{font-size:70%;}
.cmex-7x-x-71{font-size:49%;}
.msbm-7{font-size:70%;}
.cmtt-10{font-family: monospace;}
.cmti-10{ font-style: italic;}
.cmbx-10{ font-weight: bold;}
.cmr-17x-x-120{font-size:204%;}
.cmsl-10{font-style: oblique;}
.cmti-7x-x-71{font-size:49%; font-style: italic;}
.cmbxti-10{ font-weight: bold; font-style: italic;}
p.noindent { text-indent: 0em }
td p.noindent { text-indent: 0em; margin-top:0em; }
p.nopar { text-indent: 0em; }
p.indent{ text-indent: 1.5em }
@media print {div.crosslinks {visibility:hidden;}}
a img { border-top: 0; border-left: 0; border-right: 0; }
center { margin-top:1em; margin-bottom:1em; }
td center { margin-top:0em; margin-bottom:0em; }
.Canvas { position:relative; }
li p.indent { text-indent: 0em }
.enumerate1 {list-style-type:decimal;}
.enumerate2 {list-style-type:lower-alpha;}
.enumerate3 {list-style-type:lower-roman;}
.enumerate4 {list-style-type:upper-alpha;}
div.newtheorem { margin-bottom: 2em; margin-top: 2em;}
.obeylines-h,.obeylines-v {white-space: nowrap; }
div.obeylines-v p { margin-top:0; margin-bottom:0; }
.overline{ text-decoration:overline; }
.overline img{ border-top: 1px solid black; }
td.displaylines {text-align:center; white-space:nowrap;}
.centerline {text-align:center;}
.rightline {text-align:right;}
div.verbatim {font-family: monospace; white-space: nowrap; text-align:left; clear:both; }
.fbox {padding-left:3.0pt; padding-right:3.0pt; text-indent:0pt; border:solid black 0.4pt; }
div.fbox {display:table}
div.center div.fbox {text-align:center; clear:both; padding-left:3.0pt; padding-right:3.0pt; text-indent:0pt; border:solid black 0.4pt; }
div.minipage{width:100%;}
div.center, div.center div.center {text-align: center; margin-left:1em; margin-right:1em;}
div.center div {text-align: left;}
div.flushright, div.flushright div.flushright {text-align: right;}
div.flushright div {text-align: left;}
div.flushleft {text-align: left;}
.underline{ text-decoration:underline; }
.underline img{ border-bottom: 1px solid black; margin-bottom:1pt; }
.framebox-c, .framebox-l, .framebox-r { padding-left:3.0pt; padding-right:3.0pt; text-indent:0pt; border:solid black 0.4pt; }
.framebox-c {text-align:center;}
.framebox-l {text-align:left;}
.framebox-r {text-align:right;}
span.thank-mark{ vertical-align: super }
span.footnote-mark sup.textsuperscript, span.footnote-mark a sup.textsuperscript{ font-size:80%; }
div.tabular, div.center div.tabular {text-align: center; margin-top:0.5em; margin-bottom:0.5em; }
table.tabular td p{margin-top:0em;}
table.tabular {margin-left: auto; margin-right: auto;}
div.td00{ margin-left:0pt; margin-right:0pt; }
div.td01{ margin-left:0pt; margin-right:5pt; }
div.td10{ margin-left:5pt; margin-right:0pt; }
div.td11{ margin-left:5pt; margin-right:5pt; }
table[rules] {border-left:solid black 0.4pt; border-right:solid black 0.4pt; }
td.td00{ padding-left:0pt; padding-right:0pt; }
td.td01{ padding-left:0pt; padding-right:5pt; }
td.td10{ padding-left:5pt; padding-right:0pt; }
td.td11{ padding-left:5pt; padding-right:5pt; }
table[rules] {border-left:solid black 0.4pt; border-right:solid black 0.4pt; }
.hline hr, .cline hr{ height : 1px; margin:0px; }
.tabbing-right {text-align:right;}
span.TEX {letter-spacing: -0.125em; }
span.TEX span.E{ position:relative;top:0.5ex;left:-0.0417em;}
a span.TEX span.E {text-decoration: none; }
span.LATEX span.A{ position:relative; top:-0.5ex; left:-0.4em; font-size:85%;}
span.LATEX span.TEX{ position:relative; left: -0.4em; }
div.float img, div.float .caption {text-align:center;}
div.figure img, div.figure .caption {text-align:center;}
.marginpar {width:20%; float:right; text-align:left; margin-left:auto; margin-top:0.5em; font-size:85%; text-decoration:underline;}
.marginpar p{margin-top:0.4em; margin-bottom:0.4em;}
.equation td{text-align:center; vertical-align:middle; }
td.eq-no{ width:5%; }
table.equation { width:100%; } 
div.math-display, div.par-math-display{text-align:center;}
math .texttt { font-family: monospace; }
math .textit { font-style: italic; }
math .textsl { font-style: oblique; }
math .textsf { font-family: sans-serif; }
math .textbf { font-weight: bold; }
.partToc a, .partToc, .likepartToc a, .likepartToc {line-height: 200%; font-weight:bold; font-size:110%;}
.chapterToc a, .chapterToc, .likechapterToc a, .likechapterToc, .appendixToc a, .appendixToc {line-height: 200%; font-weight:bold;}
.index-item, .index-subitem, .index-subsubitem {display:block}
.caption td.id{font-weight: bold; white-space: nowrap; }
table.caption {text-align:center;}
h1.partHead{text-align: center}
p.bibitem { text-indent: -2em; margin-left: 2em; margin-top:0.6em; margin-bottom:0.6em; }
p.bibitem-p { text-indent: 0em; margin-left: 2em; margin-top:0.6em; margin-bottom:0.6em; }
.paragraphHead, .likeparagraphHead { margin-top:2em; font-weight: bold;}
.subparagraphHead, .likesubparagraphHead { font-weight: bold;}
.quote {margin-bottom:0.25em; margin-top:0.25em; margin-left:1em; margin-right:1em; text-align:justify;}
.verse{white-space:nowrap; margin-left:2em}
div.maketitle {text-align:center;}
h2.titleHead{text-align:center;}
div.maketitle{ margin-bottom: 2em; }
div.author, div.date {text-align:center;}
div.thanks{text-align:left; margin-left:10%; font-size:85%; font-style:italic; }
div.author{white-space: nowrap;}
.quotation {margin-bottom:0.25em; margin-top:0.25em; margin-left:1em; }
h1.partHead{text-align: center}
.sectionToc, .likesectionToc {margin-left:2em;}
.subsectionToc, .likesubsectionToc {margin-left:4em;}
.subsubsectionToc, .likesubsubsectionToc {margin-left:6em;}
.frenchb-nbsp{font-size:75%;}
.frenchb-thinspace{font-size:75%;}
.figure img.graphics {margin-left:10%;}
/* end css.sty */

\title{Cardinaux et entiers naturels}
\author{}
\date{}

\begin{document}
\maketitle

\textbf{Warning: \href{http://www.math.union.edu/locate/jsMath}{jsMath}
requires JavaScript to process the mathematics on this page.\\ If your
browser supports JavaScript, be sure it is enabled.}

\begin{center}\rule{3in}{0.4pt}\end{center}

{[}\href{coursse3.html}{next}{]} {[}\href{coursse1.html}{prev}{]}
{[}\href{coursse1.html\#tailcoursse1.html}{prev-tail}{]}
{[}\hyperref[tailcoursse2.html]{tail}{]}
{[}\href{coursch2.html\#coursse2.html}{up}{]}

\subsubsection{1.2 Cardinaux et entiers naturels}

\paragraph{1.2.1 Notion de cardinal}

Définition~1.2.1 On dit que deux ensembles E et F ont même cardinal s'il
existe une bijection de E sur F. On notera
\textbackslash{}mathop\{Card\}E ou encore \textbar{}E\textbar{} le
cardinal d'un ensemble E.

Remarque~1.2.1 La relation il existe une bijection de E sur F est bien
entendu une relation d'équivalence~; les cardinaux sont en quelque sorte
les classes d'équivalence pour cette relation (pas tout à fait puisque
l'ensemble de tous les ensembles n'existe pas).

Définition~1.2.2 On définit alors des opérations sur les cardinaux en
posant

\textbackslash{}begin\{eqnarray*\} \textbackslash{}mathop\{Card\}A
+\textbackslash{}mathop\{ Card\}B\& =\& \textbackslash{}mathop\{Card\}A
×\textbackslash{}\{0\textbackslash{}\} ∪ B
×\textbackslash{}\{1\textbackslash{}\}\%\&
\textbackslash{}\textbackslash{}
\textbackslash{}mathop\{Card\}A.\textbackslash{}mathop\{Card\}B\& =\&
\textbackslash{}mathop\{Card\}A × B \%\&
\textbackslash{}\textbackslash{} \textbackslash{}end\{eqnarray*\}

Remarque~1.2.2 Cette définition est justifiée par le fait que si on a
\textbackslash{}mathop\{Card\}A =\textbackslash{}mathop\{ Card\}A' et
\textbackslash{}mathop\{Card\}B =\textbackslash{}mathop\{ Card\}B', on a
aussi

\textbackslash{}mathop\{Card\}A ×\textbackslash{}\{0\textbackslash{}\} ∪
B ×\textbackslash{}\{1\textbackslash{}\} =\textbackslash{}mathop\{
Card\}A' ×\textbackslash{}\{0\textbackslash{}\} ∪ B'
×\textbackslash{}\{1\textbackslash{}\}

et \textbackslash{}mathop\{Card\}A × B =\textbackslash{}mathop\{
Card\}A' × B', comme on le vérifie facilement en construisant les
bijections appropriées.

Définition~1.2.3 On pose \textbackslash{}mathop\{Card\}A
≤\textbackslash{}mathop\{ Card\}B s'il existe une injection de A dans B.

On admettra que c'est une relation d'ordre total sur les cardinaux~; les
seuls points non évidents sont l'antisymétrie et la totalité~:
l'antisymétrie constitue le théorème de Cantor-Bernstein qui dit que
s'il existe une injection de A dans B et une injection de B dans A,
alors il existe une bijection de A sur B~; la totalité résulte assez
facilement de l'axiome de Zorn.

\paragraph{1.2.2 Les entiers naturels}

On dira qu'un ensemble A est fini si \textbackslash{}mathop\{Card\}A
\textless{}\textbackslash{}mathop\{ Card\}A + 1 (c'est équivalent à~: il
n'existe pas de bijection de A sur une partie stricte de A). L'ensemble
des cardinaux finis forme alors un ensemble totalement ordonné appelé
ensemble des entiers naturels et noté ℕ. Il vérifie les propriétés
suivantes qui le caractérisent à un isomorphime près d'ensembles
ordonnés

Axiome~1.2.1 (de Peano) ℕ est un ensemble infini où toute partie non
vide a un plus petit élément et où toute partie non vide majorée a un
plus grand élément

On en déduit immédiatement l'existence d'un successeur de tout élément a
de ℕ et on montre en théorie des cardinaux que ce n'est autre que a + 1.

L'existence d'un plus petit élément pour toute partie non vide conduit
immédiatement aux deux résultats suivants~:

Théorème~1.2.1 (Principe de récurrence forte) Soit P(n) une propriété
qui peut être vraie ou fausse pour tout entier naturel n. On suppose que
P(\{n\}\_\{0\}) est vraie et que si P(n) est vraie, alors P(n + 1) est
vraie. Alors P(n) est vraie pour tout n ≥ \{n\}\_\{0\}.

Démonstration Soit en effet X l'ensemble des n ≥ \{n\}\_\{0\} tels que
P(n) soit fausse et supposons que X est non vide~; alors il admet un
plus petit élément \{n\}\_\{1\} ∈X. Comme
\{n\}\_\{0\}\textbackslash{}mathrel\{∉\}X, on a \{n\}\_\{1\}
\textgreater{} \{n\}\_\{0\}~; mais alors \{n\}\_\{1\} −
1\textbackslash{}mathrel\{∉\}X et \{n\}\_\{1\} − 1 ≥ \{n\}\_\{0\}~; donc
P(\{n\}\_\{1\} − 1) est vraie, et il en est de même de P((\{n\}\_\{1\} −
1) + 1) = P(\{n\}\_\{1\}), soit
\{n\}\_\{1\}\textbackslash{}mathrel\{∉\}X. C'est absurde. Donc X = ∅, et
par conséquent, P(n) est vraie pour tout n ≥ \{n\}\_\{0\}.

Théorème~1.2.2 (Principe de récurrence faible) On suppose que
P(\{n\}\_\{0\}) est vraie et que si P(\{n\}\_\{0\} + 1),P(\{n\}\_\{0\} +
2),\textbackslash{}mathop\{\textbackslash{}mathop\{\ldots{}\}\},P(n)
sont vraies, alors P(n + 1) est vraie. Alors P(n) est vraie pour tout n
≥ \{n\}\_\{0\}.

Démonstration Soit en effet X l'ensemble des n ≥ \{n\}\_\{0\} tels que
P(n) soit fausse et supposons que X est non vide~; alors il admet un
plus petit élément \{n\}\_\{1\} ∈X. Comme
\{n\}\_\{0\}\textbackslash{}mathrel\{∉\}X, on a \{n\}\_\{1\}
\textgreater{} \{n\}\_\{0\}~; mais alors
\{n\}\_\{0\},\textbackslash{}mathop\{\textbackslash{}mathop\{\ldots{}\}\},\{n\}\_\{1\}
− 1\textbackslash{}mathrel\{∉\}X et donc
P(\{n\}\_\{0\}),\textbackslash{}mathop\{\textbackslash{}mathop\{\ldots{}\}\},P(\{n\}\_\{1\}
− 1) sont vraies~; il en est donc de même de P(\{n\}\_\{1\}), soit
\{n\}\_\{1\}\textbackslash{}mathrel\{∉\}X. C'est absurde. Donc X = ∅, et
par conséquent, P(n) est vraie pour tout n ≥ \{n\}\_\{0\}.

{[}\href{coursse3.html}{next}{]} {[}\href{coursse1.html}{prev}{]}
{[}\href{coursse1.html\#tailcoursse1.html}{prev-tail}{]}
{[}\href{coursse2.html}{front}{]}
{[}\href{coursch2.html\#coursse2.html}{up}{]}

\end{document}
