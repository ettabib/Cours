
\subsubsection{1.2 Cardinaux et entiers naturels}

\paragraph{1.2.1 Notion de cardinal}

Définition~1.2.1 On dit que deux ensembles E et F ont même cardinal s'il
existe une bi\jmathection de E sur F. On notera
Card~E ou encore \textbar{}E\textbar{} le
cardinal d'un ensemble E.

Remarque~1.2.1 La relation il existe une bi\jmathection de E sur F est bien
entendu une relation d'équivalence~; les cardinaux sont en quelque sorte
les classes d'équivalence pour cette relation (pas tout à fait puisque
l'ensemble de tous les ensembles n'existe pas).

Définition~1.2.2 On définit alors des opérations sur les cardinaux en
posant

\begin{align*} Card~A
+ CardB& =& \Card~A
\times\0\ \cup B
\times\1\\%&
\\
CardA.\Card~B& =&
Card~A \times B \%&
\\ \end{align*}

Remarque~1.2.2 Cette définition est \jmathustifiée par le fait que si on a
CardA =\ Card~A' et
CardB =\ Card~B', on a
aussi

CardA \times\0\~ \cup
B \times\1\ =\
CardA' \times\0\ \cup B'
\times\1\

et Card~A \times B =\
CardA' \times B', comme on le vérifie facilement en construisant les
bi\jmathections appropriées.

Définition~1.2.3 On pose Card~A
\leq Card~B s'il existe une in\jmathection de A dans B.

On admettra que c'est une relation d'ordre total sur les cardinaux~; les
seuls points non évidents sont l'antisymétrie et la totalité~:
l'antisymétrie constitue le théorème de Cantor-Bernstein qui dit que
s'il existe une in\jmathection de A dans B et une in\jmathection de B dans A,
alors il existe une bi\jmathection de A sur B~; la totalité résulte assez
facilement de l'axiome de Zorn.

\paragraph{1.2.2 Les entiers naturels}

On dira qu'un ensemble A est fini si Card~A
\textless{} Card~A + 1 (c'est équivalent à~: il
n'existe pas de bi\jmathection de A sur une partie stricte de A). L'ensemble
des cardinaux finis forme alors un ensemble totalement ordonné appelé
ensemble des entiers naturels et noté \mathbb{N}~. Il vérifie les propriétés
suivantes qui le caractérisent à un isomorphime près d'ensembles
ordonnés

Axiome~1.2.1 (de Peano) \mathbb{N}~ est un ensemble infini où toute partie non
vide a un plus petit élément et où toute partie non vide ma\jmathorée a un
plus grand élément

On en déduit immédiatement l'existence d'un successeur de tout élément a
de \mathbb{N}~ et on montre en théorie des cardinaux que ce n'est autre que a + 1.

L'existence d'un plus petit élément pour toute partie non vide conduit
immédiatement aux deux résultats suivants~:

Théorème~1.2.1 (Principe de récurrence forte) Soit P(n) une propriété
qui peut être vraie ou fausse pour tout entier naturel n. On suppose que
P(n\_0) est vraie et que si P(n) est vraie, alors P(n + 1) est
vraie. Alors P(n) est vraie pour tout n ≥ n\_0.

Démonstration Soit en effet X l'ensemble des n ≥ n\_0 tels que
P(n) soit fausse et supposons que X est non vide~; alors il admet un
plus petit élément n\_1 \inX. Comme
n\_0∉X, on a n\_1
\textgreater{} n\_0~; mais alors n\_1 -
1∉X et n\_1 - 1 ≥ n\_0~; donc
P(n\_1 - 1) est vraie, et il en est de même de P((n\_1 -
1) + 1) = P(n\_1), soit
n\_1∉X. C'est absurde. Donc X = \varnothing~, et
par conséquent, P(n) est vraie pour tout n ≥ n\_0.

Théorème~1.2.2 (Principe de récurrence faible) On suppose que
P(n\_0) est vraie et que si P(n\_0 + 1),P(n\_0 +
2),\\ldots~,P(n)
sont vraies, alors P(n + 1) est vraie. Alors P(n) est vraie pour tout n
≥ n\_0.

Démonstration Soit en effet X l'ensemble des n ≥ n\_0 tels que
P(n) soit fausse et supposons que X est non vide~; alors il admet un
plus petit élément n\_1 \inX. Comme
n\_0∉X, on a n\_1
\textgreater{} n\_0~; mais alors
n\_0,\\ldots,n\_1~
- 1∉X et donc
P(n\_0),\\ldots,P(n\_1~
- 1) sont vraies~; il en est donc de même de P(n\_1), soit
n\_1∉X. C'est absurde. Donc X = \varnothing~, et
par conséquent, P(n) est vraie pour tout n ≥ n\_0.
