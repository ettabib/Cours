\section{Rang}

\subsection{Rang d'une famille de vecteurs}

\begin{de}
\index{rang!famille de vecteurs}
Soit $(x_i)_{i\in I}$ une famille de vecteurs. On appelle rang de cette famille, noté $\operatorname{rg}(x_i)_{i\in I}$, la dimension de $\operatorname{Vect}(x_i, i\in I)$, qui est égale à $\sup\{|J|, (x_i)_{i\in J} \text{ libre}\}$.
\end{de}

\begin{proof}
L'égalité provient évidemment du théorème de la base incomplète qui garantit que l'on peut extraire de la famille $(x_i)$ une base du sous-espace $\operatorname{Vect}(x_i)$.
\end{proof}

Une application linéaire transformant une famille liée en une famille liée, on a :

\begin{thm}
\index{rang!inégalité}
Soit $u \in L(E,F)$ et $(x_i)_{i\in I}$ une famille de $E$. Alors $\operatorname{rg}(u(x_i))_{i\in I} \leq \operatorname{rg}(x_i)_{i\in I}$.
\end{thm}

\begin{rem}
Recherche pratique : voir le rang d'une matrice.
\end{rem}

\subsection{Rang d'une application linéaire}

\begin{de}
\index{rang!application linéaire}
Soit $u \in L(E,F)$. On appelle rang de $u$, noté $\operatorname{rg}(u)$, la dimension de $\operatorname{Im}(u)$. C'est un élément de $\mathbb{N} \cup \{+\infty\}$.
\end{de}

\begin{rem}
Recherche pratique : Si $(e_i)_{i\in I}$ est une base de $E$, $(u(e_i))_{i\in I}$ est une famille génératrice de $\operatorname{Im}(u)$ et donc $\operatorname{rg}(u) = \operatorname{rg}(u(e_i))_{i\in I}$.
\end{rem}

\begin{thm}
\index{théorème!rang-noyau}
Soit $u \in L(E,F)$. 
\begin{enumerate}
\item Si $\dim E < +\infty$, alors $u$ est de rang fini, $\operatorname{rg}(u) = \dim E - \dim \operatorname{Ker}(u)$; on a $\operatorname{rg}(u) = \dim E$ si et seulement si $u$ est injective.
\item Si $\dim F < +\infty$, alors $u$ est de rang fini, $\operatorname{rg}(u) \leq \dim F$; on a $\operatorname{rg}(u) = \dim F$ si et seulement si $u$ est surjective.
\end{enumerate}
\end{thm}

\begin{proof}
Découle immédiatement des résultats sur la dimension.
\end{proof}

\begin{rem}
On a donc dans tous les cas $\operatorname{rg}(u) \leq \min(\dim E, \dim F)$.
\end{rem}

\begin{prop}
\index{rang!composition}
Soit $u \in L(E,F), v \in L(F,G)$. Alors $\operatorname{rg}(v \circ u) \leq \min(\operatorname{rg}(v), \operatorname{rg}(u))$.
Si $u$ est bijectif, $\operatorname{rg}(v \circ u) = \operatorname{rg}(v)$. 
Si $v$ est bijectif, $\operatorname{rg}(v \circ u) = \operatorname{rg}(u)$.
\end{prop}

\begin{proof}
Il suffit de remarquer que $\operatorname{Im}(v \circ u) = v(u(E))$.
\end{proof}

\begin{thm}
\index{théorème!caractérisation bijection}
On suppose ici $\dim E = \dim F < +\infty$, $u \in L(E,F)$. Les propriétés suivantes sont équivalentes :
\begin{enumerate}
\item $u$ injective
\item $u$ surjective
\item $u$ bijective
\item $\exists v \in L(F,E), v \circ u = \operatorname{Id}_E$
\item $\exists v \in L(F,E), u \circ v = \operatorname{Id}_F$
\end{enumerate}
\end{thm}

\begin{proof}
L'équivalence entre (i), (ii) et (iii) est une compilation des résultats précédents. De plus (iii) $\Rightarrow$ (iv) et (v), (iv) $\Rightarrow$ (i) et (v) $\Rightarrow$ (ii), ce qui boucle la boucle.
\end{proof}