\documentclass[]{article}
\usepackage[T1]{fontenc}
\usepackage{lmodern}
\usepackage{amssymb,amsmath}
\usepackage{ifxetex,ifluatex}
\usepackage{fixltx2e} % provides \textsubscript
% use upquote if available, for straight quotes in verbatim environments
\IfFileExists{upquote.sty}{\usepackage{upquote}}{}
\ifnum 0\ifxetex 1\fi\ifluatex 1\fi=0 % if pdftex
  \usepackage[utf8]{inputenc}
\else % if luatex or xelatex
  \ifxetex
    \usepackage{mathspec}
    \usepackage{xltxtra,xunicode}
  \else
    \usepackage{fontspec}
  \fi
  \defaultfontfeatures{Mapping=tex-text,Scale=MatchLowercase}
  \newcommand{\euro}{€}
\fi
% use microtype if available
\IfFileExists{microtype.sty}{\usepackage{microtype}}{}
\ifxetex
  \usepackage[setpagesize=false, % page size defined by xetex
              unicode=false, % unicode breaks when used with xetex
              xetex]{hyperref}
\else
  \usepackage[unicode=true]{hyperref}
\fi
\hypersetup{breaklinks=true,
            bookmarks=true,
            pdfauthor={},
            pdftitle={Rang},
            colorlinks=true,
            citecolor=blue,
            urlcolor=blue,
            linkcolor=magenta,
            pdfborder={0 0 0}}
\urlstyle{same}  % don't use monospace font for urls
\setlength{\parindent}{0pt}
\setlength{\parskip}{6pt plus 2pt minus 1pt}
\setlength{\emergencystretch}{3em}  % prevent overfull lines
\setcounter{secnumdepth}{0}
 
/* start css.sty */
.cmr-5{font-size:50%;}
.cmr-7{font-size:70%;}
.cmmi-5{font-size:50%;font-style: italic;}
.cmmi-7{font-size:70%;font-style: italic;}
.cmmi-10{font-style: italic;}
.cmsy-5{font-size:50%;}
.cmsy-7{font-size:70%;}
.cmex-7{font-size:70%;}
.cmex-7x-x-71{font-size:49%;}
.msbm-7{font-size:70%;}
.cmtt-10{font-family: monospace;}
.cmti-10{ font-style: italic;}
.cmbx-10{ font-weight: bold;}
.cmr-17x-x-120{font-size:204%;}
.cmsl-10{font-style: oblique;}
.cmti-7x-x-71{font-size:49%; font-style: italic;}
.cmbxti-10{ font-weight: bold; font-style: italic;}
p.noindent { text-indent: 0em }
td p.noindent { text-indent: 0em; margin-top:0em; }
p.nopar { text-indent: 0em; }
p.indent{ text-indent: 1.5em }
@media print {div.crosslinks {visibility:hidden;}}
a img { border-top: 0; border-left: 0; border-right: 0; }
center { margin-top:1em; margin-bottom:1em; }
td center { margin-top:0em; margin-bottom:0em; }
.Canvas { position:relative; }
li p.indent { text-indent: 0em }
.enumerate1 {list-style-type:decimal;}
.enumerate2 {list-style-type:lower-alpha;}
.enumerate3 {list-style-type:lower-roman;}
.enumerate4 {list-style-type:upper-alpha;}
div.newtheorem { margin-bottom: 2em; margin-top: 2em;}
.obeylines-h,.obeylines-v {white-space: nowrap; }
div.obeylines-v p { margin-top:0; margin-bottom:0; }
.overline{ text-decoration:overline; }
.overline img{ border-top: 1px solid black; }
td.displaylines {text-align:center; white-space:nowrap;}
.centerline {text-align:center;}
.rightline {text-align:right;}
div.verbatim {font-family: monospace; white-space: nowrap; text-align:left; clear:both; }
.fbox {padding-left:3.0pt; padding-right:3.0pt; text-indent:0pt; border:solid black 0.4pt; }
div.fbox {display:table}
div.center div.fbox {text-align:center; clear:both; padding-left:3.0pt; padding-right:3.0pt; text-indent:0pt; border:solid black 0.4pt; }
div.minipage{width:100%;}
div.center, div.center div.center {text-align: center; margin-left:1em; margin-right:1em;}
div.center div {text-align: left;}
div.flushright, div.flushright div.flushright {text-align: right;}
div.flushright div {text-align: left;}
div.flushleft {text-align: left;}
.underline{ text-decoration:underline; }
.underline img{ border-bottom: 1px solid black; margin-bottom:1pt; }
.framebox-c, .framebox-l, .framebox-r { padding-left:3.0pt; padding-right:3.0pt; text-indent:0pt; border:solid black 0.4pt; }
.framebox-c {text-align:center;}
.framebox-l {text-align:left;}
.framebox-r {text-align:right;}
span.thank-mark{ vertical-align: super }
span.footnote-mark sup.textsuperscript, span.footnote-mark a sup.textsuperscript{ font-size:80%; }
div.tabular, div.center div.tabular {text-align: center; margin-top:0.5em; margin-bottom:0.5em; }
table.tabular td p{margin-top:0em;}
table.tabular {margin-left: auto; margin-right: auto;}
div.td00{ margin-left:0pt; margin-right:0pt; }
div.td01{ margin-left:0pt; margin-right:5pt; }
div.td10{ margin-left:5pt; margin-right:0pt; }
div.td11{ margin-left:5pt; margin-right:5pt; }
table[rules] {border-left:solid black 0.4pt; border-right:solid black 0.4pt; }
td.td00{ padding-left:0pt; padding-right:0pt; }
td.td01{ padding-left:0pt; padding-right:5pt; }
td.td10{ padding-left:5pt; padding-right:0pt; }
td.td11{ padding-left:5pt; padding-right:5pt; }
table[rules] {border-left:solid black 0.4pt; border-right:solid black 0.4pt; }
.hline hr, .cline hr{ height : 1px; margin:0px; }
.tabbing-right {text-align:right;}
span.TEX {letter-spacing: -0.125em; }
span.TEX span.E{ position:relative;top:0.5ex;left:-0.0417em;}
a span.TEX span.E {text-decoration: none; }
span.LATEX span.A{ position:relative; top:-0.5ex; left:-0.4em; font-size:85%;}
span.LATEX span.TEX{ position:relative; left: -0.4em; }
div.float img, div.float .caption {text-align:center;}
div.figure img, div.figure .caption {text-align:center;}
.marginpar {width:20%; float:right; text-align:left; margin-left:auto; margin-top:0.5em; font-size:85%; text-decoration:underline;}
.marginpar p{margin-top:0.4em; margin-bottom:0.4em;}
.equation td{text-align:center; vertical-align:middle; }
td.eq-no{ width:5%; }
table.equation { width:100%; } 
div.math-display, div.par-math-display{text-align:center;}
math .texttt { font-family: monospace; }
math .textit { font-style: italic; }
math .textsl { font-style: oblique; }
math .textsf { font-family: sans-serif; }
math .textbf { font-weight: bold; }
.partToc a, .partToc, .likepartToc a, .likepartToc {line-height: 200%; font-weight:bold; font-size:110%;}
.chapterToc a, .chapterToc, .likechapterToc a, .likechapterToc, .appendixToc a, .appendixToc {line-height: 200%; font-weight:bold;}
.index-item, .index-subitem, .index-subsubitem {display:block}
.caption td.id{font-weight: bold; white-space: nowrap; }
table.caption {text-align:center;}
h1.partHead{text-align: center}
p.bibitem { text-indent: -2em; margin-left: 2em; margin-top:0.6em; margin-bottom:0.6em; }
p.bibitem-p { text-indent: 0em; margin-left: 2em; margin-top:0.6em; margin-bottom:0.6em; }
.paragraphHead, .likeparagraphHead { margin-top:2em; font-weight: bold;}
.subparagraphHead, .likesubparagraphHead { font-weight: bold;}
.quote {margin-bottom:0.25em; margin-top:0.25em; margin-left:1em; margin-right:1em; text-align:justify;}
.verse{white-space:nowrap; margin-left:2em}
div.maketitle {text-align:center;}
h2.titleHead{text-align:center;}
div.maketitle{ margin-bottom: 2em; }
div.author, div.date {text-align:center;}
div.thanks{text-align:left; margin-left:10%; font-size:85%; font-style:italic; }
div.author{white-space: nowrap;}
.quotation {margin-bottom:0.25em; margin-top:0.25em; margin-left:1em; }
h1.partHead{text-align: center}
.sectionToc, .likesectionToc {margin-left:2em;}
.subsectionToc, .likesubsectionToc {margin-left:4em;}
.subsubsectionToc, .likesubsubsectionToc {margin-left:6em;}
.frenchb-nbsp{font-size:75%;}
.frenchb-thinspace{font-size:75%;}
.figure img.graphics {margin-left:10%;}
/* end css.sty */

\title{Rang}
\author{}
\date{}

\begin{document}
\maketitle

\textbf{Warning: \href{http://www.math.union.edu/locate/jsMath}{jsMath}
requires JavaScript to process the mathematics on this page.\\ If your
browser supports JavaScript, be sure it is enabled.}

\begin{center}\rule{3in}{0.4pt}\end{center}

{[}\href{coursse10.html}{next}{]} {[}\href{coursse8.html}{prev}{]}
{[}\href{coursse8.html\#tailcoursse8.html}{prev-tail}{]}
{[}\hyperref[tailcoursse9.html]{tail}{]}
{[}\href{coursch3.html\#coursse9.html}{up}{]}

\subsubsection{2.3 Rang}

\paragraph{2.3.1 Rang d'une famille de vecteurs}

Définition~2.3.1
\textbackslash{}mathop\{\textbackslash{}mathrm\{rg\}\}\{(\{x\}\_\{i\})\}\_\{i∈I\}
=\textbackslash{}mathop\{ dim\}
\textbackslash{}mathop\{\textbackslash{}mathrm\{Vect\}\}(\{x\}\_\{i\},i
∈ I) =\textbackslash{}mathop\{
sup\}\textbackslash{}\{\textbar{}J\textbar{}\textbackslash{}mathrel\{∣\}\{(\{x\}\_\{i\})\}\_\{i∈J\}\textbackslash{}text\{
libre \}\textbackslash{}\}

Démonstration L'égalité provient évidemment du théorème de la base
incomplète qui garantit que l'on peut extraire de la famille
(\{x\}\_\{i\}) une base du sous-espace
\textbackslash{}mathop\{\textbackslash{}mathrm\{Vect\}\}(\{x\}\_\{i\}).

Une application linéaire transformant une famille liée en une famille
liée, on a~:

Théorème~2.3.1 Soit u ∈ L(E,F) et \{(\{x\}\_\{i\})\}\_\{i∈I\} une
famille de E. Alors
\textbackslash{}mathop\{\textbackslash{}mathrm\{rg\}\}\{(u(\{x\}\_\{i\}))\}\_\{i∈I\}
≤\textbackslash{}mathop\{\textbackslash{}mathrm\{rg\}\}\{(\{x\}\_\{i\})\}\_\{i∈I\}.

Recherche pratique~: voir le rang d'une matrice.

\paragraph{2.3.2 Rang d'une application linéaire}

Définition~2.3.2 Soit u ∈ L(E,F). Alors
\textbackslash{}mathop\{\textbackslash{}mathrm\{rg\}\}u
=\textbackslash{}mathop\{ dim\}
\textbackslash{}mathop\{\textbackslash{}mathrm\{Im\}\}u ∈ ℕ
∪\textbackslash{}\{ + ∞\textbackslash{}\}.

Remarque~2.3.1 Recherche pratique~: Si \{(\{e\}\_\{i\})\}\_\{i∈I\} est
une base de E, \{(u(\{e\}\_\{i\}))\}\_\{i∈I\} est une famille
génératrice de \textbackslash{}mathop\{\textbackslash{}mathrm\{Im\}\}u
et donc \textbackslash{}mathop\{\textbackslash{}mathrm\{rg\}\}u
=\textbackslash{}mathop\{
\textbackslash{}mathrm\{rg\}\}\{(u(\{e\}\_\{i\}))\}\_\{i∈I\}.

Théorème~2.3.2 Soit u ∈ L(E,F). (i) Si \textbackslash{}mathop\{dim\} E
\textless{} +∞, alors u est de rang fini,
\textbackslash{}mathop\{\textbackslash{}mathrm\{rg\}\}u
=\textbackslash{}mathop\{ dim\} E −\textbackslash{}mathop\{ dim\}
\textbackslash{}mathop\{\textbackslash{}mathrm\{Ker\}\}u~; on a
\textbackslash{}mathop\{\textbackslash{}mathrm\{rg\}\}u
=\textbackslash{}mathop\{ dim\} E si et seulement si u est injectif (ii)
Si \textbackslash{}mathop\{dim\} F \textless{} +∞, alors u est de rang
fini, \textbackslash{}mathop\{\textbackslash{}mathrm\{rg\}\}u
≤\textbackslash{}mathop\{ dim\} F~; on a
\textbackslash{}mathop\{\textbackslash{}mathrm\{rg\}\}u
=\textbackslash{}mathop\{ dim\} F si et seulement si u est surjectif

Démonstration Découle immédiatement des résultats sur la dimension.

Remarque~2.3.2 On a donc dans tous les cas
\textbackslash{}mathop\{\textbackslash{}mathrm\{rg\}\}u
≤\textbackslash{}mathop\{ min\}(\textbackslash{}mathop\{dim\}
E,\textbackslash{}mathop\{dim\} F)

Proposition~2.3.3 Soit u ∈ L(E,F),v ∈ L(F,G). Alors
\textbackslash{}mathop\{\textbackslash{}mathrm\{rg\}\}v ∘ u
≤\textbackslash{}mathop\{
min\}(\textbackslash{}mathop\{\textbackslash{}mathrm\{rg\}\}v,\textbackslash{}mathop\{\textbackslash{}mathrm\{rg\}\}u).
Si u est bijectif,
\textbackslash{}mathop\{\textbackslash{}mathrm\{rg\}\}v ∘ u
=\textbackslash{}mathop\{ \textbackslash{}mathrm\{rg\}\}v. Si v est
bijectif, \textbackslash{}mathop\{\textbackslash{}mathrm\{rg\}\}v ∘ u
=\textbackslash{}mathop\{ \textbackslash{}mathrm\{rg\}\}u.

Démonstration Il suffit de remarquer que
\textbackslash{}mathop\{\textbackslash{}mathrm\{Im\}\}v ∘ u = v(u(E)).

Théorème~2.3.4 On suppose ici \textbackslash{}mathop\{dim\} E
=\textbackslash{}mathop\{ dim\} F \textless{} +∞, u ∈ L(E,F). On a
équivalence de

\begin{itemize}
\itemsep1pt\parskip0pt\parsep0pt
\item
  (i) u injective
\item
  (ii) u surjective
\item
  (iii) u bijective
\item
  (iv) \textbackslash{}mathop\{∃\}v ∈ L(F,E), v ∘ u =\{
  \textbackslash{}mathrm\{Id\}\}\_\{E\}
\item
  (v) \textbackslash{}mathop\{∃\}v ∈ L(F,E), u ∘ v =\{
  \textbackslash{}mathrm\{Id\}\}\_\{F\}
\end{itemize}

Démonstration L'équivalence entre (i), (ii) et (iii) est une compilation
des résultats précédents. De plus (iii) ⇒ (iv) et (v), (iv) ⇒ (i) et (v)
⇒(ii), ce qui boucle la boucle.

{[}\href{coursse10.html}{next}{]} {[}\href{coursse8.html}{prev}{]}
{[}\href{coursse8.html\#tailcoursse8.html}{prev-tail}{]}
{[}\href{coursse9.html}{front}{]}
{[}\href{coursch3.html\#coursse9.html}{up}{]}

\end{document}
