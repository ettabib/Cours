\documentclass[]{article}
\usepackage[T1]{fontenc}
\usepackage{lmodern}
\usepackage{amssymb,amsmath}
\usepackage{ifxetex,ifluatex}
\usepackage{fixltx2e} % provides \textsubscript
% use upquote if available, for straight quotes in verbatim environments
\IfFileExists{upquote.sty}{\usepackage{upquote}}{}
\ifnum 0\ifxetex 1\fi\ifluatex 1\fi=0 % if pdftex
  \usepackage[utf8]{inputenc}
\else % if luatex or xelatex
  \ifxetex
    \usepackage{mathspec}
    \usepackage{xltxtra,xunicode}
  \else
    \usepackage{fontspec}
  \fi
  \defaultfontfeatures{Mapping=tex-text,Scale=MatchLowercase}
  \newcommand{\euro}{€}
\fi
% use microtype if available
\IfFileExists{microtype.sty}{\usepackage{microtype}}{}
\ifxetex
  \usepackage[setpagesize=false, % page size defined by xetex
              unicode=false, % unicode breaks when used with xetex
              xetex]{hyperref}
\else
  \usepackage[unicode=true]{hyperref}
\fi
\hypersetup{breaklinks=true,
            bookmarks=true,
            pdfauthor={},
            pdftitle={Table des mati`eres},
            colorlinks=true,
            citecolor=blue,
            urlcolor=blue,
            linkcolor=magenta,
            pdfborder={0 0 0}}
\urlstyle{same}  % don't use monospace font for urls
\setlength{\parindent}{0pt}
\setlength{\parskip}{6pt plus 2pt minus 1pt}
\setlength{\emergencystretch}{3em}  % prevent overfull lines
\setcounter{secnumdepth}{0}
 
/* start css.sty */
.cmr-5{font-size:50%;}
.cmr-7{font-size:70%;}
.cmmi-5{font-size:50%;font-style: italic;}
.cmmi-7{font-size:70%;font-style: italic;}
.cmmi-10{font-style: italic;}
.cmsy-5{font-size:50%;}
.cmsy-7{font-size:70%;}
.cmex-7{font-size:70%;}
.cmex-7x-x-71{font-size:49%;}
.msbm-7{font-size:70%;}
.cmtt-10{font-family: monospace;}
.cmti-10{ font-style: italic;}
.cmbx-10{ font-weight: bold;}
.cmr-17x-x-120{font-size:204%;}
.cmsl-10{font-style: oblique;}
.cmti-7x-x-71{font-size:49%; font-style: italic;}
.cmbxti-10{ font-weight: bold; font-style: italic;}
p.noindent { text-indent: 0em }
td p.noindent { text-indent: 0em; margin-top:0em; }
p.nopar { text-indent: 0em; }
p.indent{ text-indent: 1.5em }
@media print {div.crosslinks {visibility:hidden;}}
a img { border-top: 0; border-left: 0; border-right: 0; }
center { margin-top:1em; margin-bottom:1em; }
td center { margin-top:0em; margin-bottom:0em; }
.Canvas { position:relative; }
li p.indent { text-indent: 0em }
.enumerate1 {list-style-type:decimal;}
.enumerate2 {list-style-type:lower-alpha;}
.enumerate3 {list-style-type:lower-roman;}
.enumerate4 {list-style-type:upper-alpha;}
div.newtheorem { margin-bottom: 2em; margin-top: 2em;}
.obeylines-h,.obeylines-v {white-space: nowrap; }
div.obeylines-v p { margin-top:0; margin-bottom:0; }
.overline{ text-decoration:overline; }
.overline img{ border-top: 1px solid black; }
td.displaylines {text-align:center; white-space:nowrap;}
.centerline {text-align:center;}
.rightline {text-align:right;}
div.verbatim {font-family: monospace; white-space: nowrap; text-align:left; clear:both; }
.fbox {padding-left:3.0pt; padding-right:3.0pt; text-indent:0pt; border:solid black 0.4pt; }
div.fbox {display:table}
div.center div.fbox {text-align:center; clear:both; padding-left:3.0pt; padding-right:3.0pt; text-indent:0pt; border:solid black 0.4pt; }
div.minipage{width:100%;}
div.center, div.center div.center {text-align: center; margin-left:1em; margin-right:1em;}
div.center div {text-align: left;}
div.flushright, div.flushright div.flushright {text-align: right;}
div.flushright div {text-align: left;}
div.flushleft {text-align: left;}
.underline{ text-decoration:underline; }
.underline img{ border-bottom: 1px solid black; margin-bottom:1pt; }
.framebox-c, .framebox-l, .framebox-r { padding-left:3.0pt; padding-right:3.0pt; text-indent:0pt; border:solid black 0.4pt; }
.framebox-c {text-align:center;}
.framebox-l {text-align:left;}
.framebox-r {text-align:right;}
span.thank-mark{ vertical-align: super }
span.footnote-mark sup.textsuperscript, span.footnote-mark a sup.textsuperscript{ font-size:80%; }
div.tabular, div.center div.tabular {text-align: center; margin-top:0.5em; margin-bottom:0.5em; }
table.tabular td p{margin-top:0em;}
table.tabular {margin-left: auto; margin-right: auto;}
div.td00{ margin-left:0pt; margin-right:0pt; }
div.td01{ margin-left:0pt; margin-right:5pt; }
div.td10{ margin-left:5pt; margin-right:0pt; }
div.td11{ margin-left:5pt; margin-right:5pt; }
table[rules] {border-left:solid black 0.4pt; border-right:solid black 0.4pt; }
td.td00{ padding-left:0pt; padding-right:0pt; }
td.td01{ padding-left:0pt; padding-right:5pt; }
td.td10{ padding-left:5pt; padding-right:0pt; }
td.td11{ padding-left:5pt; padding-right:5pt; }
table[rules] {border-left:solid black 0.4pt; border-right:solid black 0.4pt; }
.hline hr, .cline hr{ height : 1px; margin:0px; }
.tabbing-right {text-align:right;}
span.TEX {letter-spacing: -0.125em; }
span.TEX span.E{ position:relative;top:0.5ex;left:-0.0417em;}
a span.TEX span.E {text-decoration: none; }
span.LATEX span.A{ position:relative; top:-0.5ex; left:-0.4em; font-size:85%;}
span.LATEX span.TEX{ position:relative; left: -0.4em; }
div.float img, div.float .caption {text-align:center;}
div.figure img, div.figure .caption {text-align:center;}
.marginpar {width:20%; float:right; text-align:left; margin-left:auto; margin-top:0.5em; font-size:85%; text-decoration:underline;}
.marginpar p{margin-top:0.4em; margin-bottom:0.4em;}
.equation td{text-align:center; vertical-align:middle; }
td.eq-no{ width:5%; }
table.equation { width:100%; } 
div.math-display, div.par-math-display{text-align:center;}
math .texttt { font-family: monospace; }
math .textit { font-style: italic; }
math .textsl { font-style: oblique; }
math .textsf { font-family: sans-serif; }
math .textbf { font-weight: bold; }
.partToc a, .partToc, .likepartToc a, .likepartToc {line-height: 200%; font-weight:bold; font-size:110%;}
.chapterToc a, .chapterToc, .likechapterToc a, .likechapterToc, .appendixToc a, .appendixToc {line-height: 200%; font-weight:bold;}
.index-item, .index-subitem, .index-subsubitem {display:block}
.caption td.id{font-weight: bold; white-space: nowrap; }
table.caption {text-align:center;}
h1.partHead{text-align: center}
p.bibitem { text-indent: -2em; margin-left: 2em; margin-top:0.6em; margin-bottom:0.6em; }
p.bibitem-p { text-indent: 0em; margin-left: 2em; margin-top:0.6em; margin-bottom:0.6em; }
.paragraphHead, .likeparagraphHead { margin-top:2em; font-weight: bold;}
.subparagraphHead, .likesubparagraphHead { font-weight: bold;}
.quote {margin-bottom:0.25em; margin-top:0.25em; margin-left:1em; margin-right:1em; text-align:\\jmathmathustify;}
.verse{white-space:nowrap; margin-left:2em}
div.maketitle {text-align:center;}
h2.titleHead{text-align:center;}
div.maketitle{ margin-bottom: 2em; }
div.author, div.date {text-align:center;}
div.thanks{text-align:left; margin-left:10%; font-size:85%; font-style:italic; }
div.author{white-space: nowrap;}
.quotation {margin-bottom:0.25em; margin-top:0.25em; margin-left:1em; }
h1.partHead{text-align: center}
.sectionToc, .likesectionToc {margin-left:2em;}
.subsectionToc, .likesubsectionToc {margin-left:4em;}
.subsubsectionToc, .likesubsubsectionToc {margin-left:6em;}
.frenchb-nbsp{font-size:75%;}
.frenchb-thinspace{font-size:75%;}
.figure img.graphics {margin-left:10%;}
/* end css.sty */

\title{Table des mati`eres}
\author{}
\date{}

\begin{document}
\maketitle

\textbf{Warning: 
requires JavaScript to process the mathematics on this page.\\ If your
browser supports JavaScript, be sure it is enabled.}

\begin{center}\rule{3in}{0.4pt}\end{center}

{[}
{[}
{[}{]}
{[}

\subsection{Table des matières}

 \\ 1
 \\ ~1.1
 \\ ~~1.1.1
 \\ ~~1.1.2
 \\ ~~1.1.3
 \\ ~~1.1.4
 \\ ~1.2
 \\
~~1.2.1  \\
~~1.2.2  \\
~1.3  \\ ~~1.3.1

\\ ~~1.3.2  \\ ~~1.3.3
 \\
~~1.3.4  \\
~~1.3.5  \\ ~~1.3.6
 \\ ~~1.3.7
 \\ ~~1.3.8
 \\ ~~1.3.9
 \\
~~1.3.10 {Groupe des permutations
d'un ensemble fini} \\ ~1.4 {Anneaux et
corps} \\ ~~1.4.1 {Généralités sur
les anneaux} \\ ~~1.4.2 {Idéaux et
quotients} \\ ~~1.4.3 {Morphisme
d'anneaux} \\ ~~1.4.4  \\
~~1.4.5  \\ ~~1.4.6
 \\
~~1.4.7  \\
~~1.4.8 {L'anneau \mathbb{Z}. Caractéristique
d'un anneau} \\ ~~1.4.9 {Théorème
chinois, indicateur d'Euler} \\ ~1.5
 \\ ~~1.5.1
{L'anneau des séries formelles à
coefficients dans A} \\ ~~1.5.2
{L'anneau des polynômes à
coefficients dans A} \\ ~~1.5.3
 \\
~~1.5.4  \\ ~~1.5.5
 \\
~~1.5.6 {Formule de Taylor.
Multiplicité d'une racine} \\ ~~1.5.7
 \\
~~1.5.8  \\
~~1.5.9 {Division suivant les
puissances croissantes} \\ ~1.6
 \\
~~1.6.1  \\ ~~1.6.2
{Dérivées partielles, formule de
Taylor} \\ ~~1.6.3 {Degré total,
polynômes homogènes} \\ ~~1.6.4
 \\ 2
 \\ ~2.1
{Généralités sur les espaces
vectoriels} \\ ~~2.1.1 {Notion de
K-espace vectoriel} \\ ~~2.1.2
 \\
~~2.1.3  \\
~~2.1.4  \\
~~2.1.5  \\
~~2.1.6  \\ ~~2.1.7
{Familles libres, génératrices.
Bases} \\ ~~2.1.8 {Théorèmes
fondamentaux} \\ ~2.2 {Bases et
dimension} \\ ~~2.2.1 {Existence de
bases} \\ ~~2.2.2 {Espaces
vectoriels de dimension finie. Dimension} \\ ~~2.2.3
 \\ ~2.3
 \\ ~~2.3.1
 \\
~~2.3.2 {Rang d'une application
linéaire} \\ ~2.4 {Dualité: approche
restreinte} \\ ~~2.4.1 {Formes
linéaires, dual, formes coordonnées} \\ ~~2.4.2
{Base duale d'un espace vectoriel
de dimension finie} \\ ~~2.4.3
 \\ ~~2.4.4
 \\ ~~2.4.5
 \\ ~~2.4.6
{Application: polynômes
d'interpolation de Lagrange} \\ ~2.5
 \\
~~2.5.1 {Notion de dual.
Orthogonalité} \\ ~~2.5.2
 \\ ~~2.5.3
 \\ ~~2.5.4
 \\ ~~2.5.5
 \\
~2.6  \\ ~~2.6.1
 \\ ~~2.6.2
 \\ ~~2.6.3
 \\ ~~2.6.4
 \\ ~~2.6.5
 \\ ~~2.6.6
 \\ ~~2.6.7
 \\
~2.7  \\ ~~2.7.1
 \\ ~~2.7.2
{Déterminant d'une famille de
vecteurs} \\ ~~2.7.3 {Déterminant
d'un endomorphisme} \\ ~~2.7.4
 \\
~~2.7.5 {Application des
déterminants à la recherche du rang} \\ ~~2.7.6
 \\
~2.8  \\ ~~2.8.1
 \\ ~~2.8.2
 \\ ~~2.8.3
 \\
~~2.8.4  \\ 3
 \\ ~3.1

\\ ~~3.1.1 
\\ ~~3.1.2 {Valeurs propres,
vecteurs propres} \\ ~~3.1.3
 \\
~~3.1.4 {Endomorphismes
diagonalisables} \\ ~~3.1.5
 \\
~~3.1.6 {Endomorphismes et
matrices trigonalisables} \\ ~3.2
 \\
~~3.2.1  \\ ~~3.2.2
{Idéal annulateur. Polynôme
minimal} \\ ~~3.2.3 {Théorème de
Cayley-Hamilton} \\ ~~3.2.4
{Polynôme annulateur et
trigonalisation} \\ ~~3.2.5
 \\
~~3.2.6 {Sous-espaces
caractéristiques} \\ ~~3.2.7
{Application: récurrences
linéaires d'ordre 2} \\ ~3.3 {A
propos de Jordan} \\ ~~3.3.1
 \\
~~3.3.2  \\ ~~3.3.3
{Réduction des endomorphismes
nilpotents} \\ ~~3.3.4 {Première
démonstration} \\ ~~3.3.5
 \\
~~3.3.6  \\ 4
 \\
~4.1 {Eléments de topologie
générale} \\ ~~4.1.1 {Espaces
topologiques} \\ ~~4.1.2 {La
topologie de \mathbb{R}~} \\ ~~4.1.3 {Fermés
et voisinages} \\ ~~4.1.4

\\ ~~4.1.5  \\
~4.2  \\ ~~4.2.1
 \\ ~~4.2.2
{Topologie définie par une
distance} \\ ~~4.2.3 {Points
isolés, points d'accumulation} \\ ~~4.2.4
 \\
~~4.2.5 
\\ ~~4.2.6 {La droite numérique
achevée} \\ ~4.3  \\ ~~4.3.1
 \\
~~4.3.2 {Sous suites, valeurs
d'adhérences} \\ ~~4.3.3
{Caractérisation des fermés d'un
espace métrique} \\ ~4.4 {Limites de
fonctions} \\ ~~4.4.1 {Notion de
limite suivant une partie} \\ ~~4.4.2
 \\
~~4.4.3 
\\ ~~4.4.4  \\
~4.5  \\ ~~4.5.1
 \\
~~4.5.2 
\\ ~~4.5.3  \\
~4.6  \\
~~4.6.1 {Applications uniformément
continues} \\ ~~4.6.2
 \\
~4.7  \\ ~~4.7.1
 \\ ~~4.7.2
 \\ ~~4.7.3

\\ ~4.8 {Espaces et parties
compactes} \\ ~~4.8.1 {Propriété
de Bolzano-Weierstrass} \\ ~~4.8.2
 \\
~~4.8.3 {Compacts de \mathbb{R}~ et
\mathbb{R}~^n} \\ ~4.9 {Espaces et
parties connexes} \\ ~~4.9.1
 \\ ~~4.9.2
 \\
~~4.9.3  \\ ~~4.9.4
 \\ 5
 \\ ~5.1
 \\
~~5.1.1 {Norme et distance
associée} \\ ~~5.1.2 {Convexes,
connexes} \\ ~~5.1.3 {Continuité
des opérations algébriques} \\ ~5.2

\\ ~~5.2.1 {Caractérisations et
normes des applications linéaires continues} \\ ~~5.2.2
{L'espace vectoriel normé des
applications linéaires continues de E dans F} \\ ~~5.2.3
 \\
~~5.2.4 {Caractérisation des
applications bilinéaires continues} \\ ~5.3
{Espaces vectoriels normés de
dimensions finies} \\ ~~5.3.1
 \\
~~5.3.2 {Propriétés topologiques
et métriques des espaces vectoriels normés de dimension finie} \\
~~5.3.3 {Continuité des
applications linéaires} \\ ~5.4
{Compléments: le théorème de Baire
et ses conséquences} \\ ~~5.4.1
 \\ ~~5.4.2
 \\ ~5.5
{Compléments: convexité dans les
espaces vectoriels normés} \\ ~~5.5.1
 \\ ~~5.5.2

\\ ~~5.5.3 {Hahn-Banach (version
géométrique)} \\ ~~5.5.4
{L'enveloppe convexe: Carathéodory
et Krein Millman} \\ 6 {Comparaison des
fonctions} \\ ~6.1 {Relations de
comparaison} \\ ~~6.1.1
 \\ ~~6.1.2
 \\
~~6.1.3  \\ ~~6.1.4
 \\ ~6.2
 \\ ~~6.2.1

\\ ~~6.2.2 {Opérations sur les
développements limités} \\ ~~6.2.3
{Développements limités
classiques} \\ ~6.3 {Développements
asymptotiques} \\ ~~6.3.1
{Echelles de comparaison, parties
principales} \\ ~~6.3.2
 \\
~~6.3.3 {Opérations sur les
développements asymptotiques} \\ 7
 \\ ~7.1
 \\ ~~7.1.1
{Monotonie (suites à termes
réels)} \\ ~~7.1.2 {Critère de
Cauchy} \\ ~~7.1.3 {Valeurs
d'adhérences, limites inférieures et supérieures} \\ ~~7.1.4
 \\ ~7.2
 \\
~~7.2.1  \\
~~7.2.2 {Terme général, critère de
Cauchy} \\ ~7.3 {Séries à termes
réels positifs} \\ ~~7.3.1
{Convergence des séries à termes
réels positifs} \\ ~~7.3.2
{Comparaison des séries à termes
réels positifs} \\ ~~7.3.3 {Séries
de Riemann et de Bertrand} \\ ~~7.3.4
 \\
~7.4 {Séries absolument
convergentes} \\ ~~7.4.1 {Notion
de convergence absolue} \\ ~~7.4.2

\\ ~~7.4.3  \\
~~7.4.4 
\\ ~~7.4.5 {Comparaison à une
intégrale} \\ ~7.5 {Séries
semi-convergentes} \\ ~~7.5.1
 \\ ~~7.5.2
{Etude de séries
semi-convergentes} \\ ~7.6
 \\
~~7.6.1 
\\ ~~7.6.2 
\\ ~~7.6.3 {Modification de
l'ordre des termes} \\ ~~7.6.4
 \\ ~7.7
 \\ ~7.8
 \\ ~7.9
{Compléments: développements
asymptotiques, analyse numérique} \\ ~~7.9.1
{Calcul approché de la somme d'une
série} \\ ~~7.9.2 {Accélération de
la convergence} \\ 8 {Fonctions d'une
variable réelle} \\ ~8.1 {Monotonie,
continuité} \\ ~~8.1.1 {Limites et
monotonie} \\ ~~8.1.2 {Continuité
et monotonie} \\ ~8.2  \\
~~8.2.1  \\
~~8.2.2 {Opérations sur les
dérivées} \\ ~~8.2.3 {Dérivées
d'ordre supérieur} \\ ~8.3
{Fonctions réelles d'une variable
réelle} \\ ~~8.3.1 {Théorème de
Rolle, formule des accroissements finis} \\ ~~8.3.2
 \\
~~8.3.3  \\
~~8.3.4 {Formule de Taylor
Lagrange} \\ ~~8.3.5 {Extensions
du théorème des accroissements finis} \\ ~~8.3.6
{Fonctions convexes de classe
\mathcal{C}^1} \\ ~8.4 {Fonctions
vectorielles d'une variable réelle} \\ ~~8.4.1
{Inégalité des accroissements
finis} \\ ~~8.4.2 {Applications de
l'inégalité des accroissements finis} \\ ~~8.4.3
 \\ ~8.5
 \\ ~~8.5.1
{Fonctions circulaires
réciproques} \\ ~~8.5.2 {Fonctions
hyperboliques directes} \\ ~~8.5.3
{Fonctions hyperboliques
réciproques} \\ ~8.6 {Analyse
numérique des fonctions d'une variable} \\ ~~8.6.1
{Interpolation linéaire,
interpolation polynomiale} \\ ~~8.6.2
 \\ ~~8.6.3
{Recherche des zéros d'une
fonction} \\ 9  \\ ~9.1
{Subdivisions, approximation des
fonctions} \\ ~~9.1.1
 \\
~~9.1.2 {Propriétés liées aux
subdivisions} \\ ~~9.1.3
 \\
~9.2 {Intégrale des fonctions
réglées sur un segment} \\ ~~9.2.1
{Intégrale des applications en
escalier} \\ ~~9.2.2 {Intégrale
des fonctions réglées} \\ ~~9.2.3
 \\ ~~9.2.4
 \\ ~~9.2.5
 \\ ~9.3
 \\
~~9.3.1 {Continuité et
dérivabilité par rapport à une borne} \\ ~~9.3.2
 \\ ~~9.3.3
{Changement de variable,
intégration par parties} \\ ~~9.3.4

\\ ~9.4 
\\ ~~9.4.1 
\\ ~~9.4.2 
\\ ~~9.4.3 
\\ ~~9.4.4 {Fractions
rationnelles} \\ ~~9.4.5
{Fractions rationnelles en sinus
et cosinus} \\ ~~9.4.6 {Fractions
rationnelles en sinus et cosinus hyperboliques} \\ ~~9.4.7
 \\ ~9.5
{Intégration sur un intervalle
quelconque: fonctions à valeurs réelles positives} \\ ~~9.5.1
{Fonctions intégrables à valeurs
réelles positives} \\ ~~9.5.2
 \\ ~~9.5.3
 \\ ~9.6
{Intégration sur un intervalle
quelconque: fonctions à valeurs complexes} \\ ~~9.6.1
{Fonctions à valeurs complexes
intégrables} \\ ~~9.6.2
{Décomposition des fonctions à
valeurs complexes} \\ ~~9.6.3
{Convention et relation de
Chasles} \\ ~~9.6.4 {Règles de
comparaison} \\ ~~9.6.5 {Espaces
de fonctions continues} \\ ~~9.6.6
 \\
~9.7 {Développements asymptotiques
et analyse numérique} \\ ~~9.7.1
 \\
~~9.7.2 {Calcul approché
d'intégrales} \\ ~~9.7.3 {La
méthode de Laplace} \\ ~9.8
{Généralités sur les intégrales
impropres} \\ ~~9.8.1 {Notion
d'intégrale impropre} \\ ~~9.8.2
{Intégrales plusieurs fois
impropres} \\ ~~9.8.3 {Opérations
sur les intégrales impropres} \\ ~~9.8.4
{Intégrales et séries: intégration
par paquets} \\ ~9.9 {Intégrale des
fonctions réelles positives} \\ ~~9.9.1
{Critère de convergence des
fonctions réelles positives} \\ ~~9.9.2
 \\ ~~9.9.3
 \\ ~9.10
{Convergence absolue,
semi-convergence} \\ ~~9.10.1
{Critère de Cauchy pour les
intégrales} \\ ~~9.10.2
 \\ ~~9.10.3
 \\
~~9.10.4  \\ 10
 \\
~10.1  \\
~~10.1.1 {Convergence simple,
convergence uniforme} \\ ~~10.1.2
{Plan d'étude d'une suite de
fonctions} \\ ~~10.1.3 {Critère
de Cauchy uniforme} \\ ~~10.1.4
{Fonctions bornées, norme de la
convergence uniforme} \\ ~~10.1.5
 \\
~~10.1.6 {Propriétés de la
convergence uniforme} \\ ~~10.1.7
{Suites de fonctions intégrables
sur un intervalle} \\ ~10.2 {Séries
de fonctions} \\ ~~10.2.1

\\ ~~10.2.2 {Critères
supplémentaires de convergence uniforme} \\ ~~10.2.3
{Propriétés de la convergence
uniforme} \\ ~~10.2.4 {Séries de
fonctions intégrables sur un intervalle} \\ ~10.3
{Intégrales dépendant d'un
paramètre} \\ ~~10.3.1 {Position
du problème} \\ ~~10.3.2
 \\ ~~10.3.3
 \\ ~~10.3.4
{Théorème de Fubini sur un
produit de segments} \\ ~~10.3.5
{Intégrales sur un pavé ou un
rectangle} \\ ~~10.3.6 {Théorème
de Fubini sur un produit d'intervalles} \\ ~~10.3.7
 \\ ~~10.3.8
 \\ 11
 \\ ~11.1

\\ ~~11.1.1 {Notion de série
entière} \\ ~~11.1.2 {Rayon de
convergence} \\ ~~11.1.3
{Recherche du rayon de
convergence} \\ ~~11.1.4
{Opérations sur les séries
entières} \\ ~11.2 {Somme d'une
série entière} \\ ~~11.2.1 {Etude
sur le disque ouvert de convergence (domaine complexe)} \\ ~~11.2.2
{Etude sur le disque ouvert de
convergence (domaine réel)} \\ ~~11.2.3
{Etude sur le cercle de
convergence} \\ ~11.3

\\ ~~11.3.1 {Problème local,
problème global} \\ ~~11.3.2
 \\
~~11.3.3 {Fonction exponentielle.
Fonctions trigonométriques} \\ ~~11.3.4

\\ ~~11.3.5 {Fonctions
classiques} \\ ~~11.3.6 {Méthodes
de sommation} \\ ~11.4 {Application
aux endomorphismes continus et aux matrices} \\ ~~11.4.1
{Calcul fonctionnel et premières
applications} \\ ~~11.4.2
{Exponentielle d'un endomorphisme
ou d'une matrice} \\ ~~11.4.3
{Application aux systèmes
différentiels homogènes à coefficients constants} \\ 12
 \\ ~12.1
 \\ ~~12.1.1
 \\ ~~12.1.2
{Formes bilinéaires symétriques,
antisymétriques} \\ ~~12.1.3

\\ ~~12.1.4 {Changements de
bases, discriminant} \\ ~~12.1.5
 \\ ~~12.1.6
 \\
~~12.1.7  \\ ~12.2
 \\ ~~12.2.1
 \\
~~12.2.2 {Formes quadratiques en
dimension finie} \\ ~~12.2.3
{Matrices et déterminants de
Gram} \\ ~12.3 {Réduction des
formes quadratiques en dimension finie} \\ ~~12.3.1

\\ ~~12.3.2 {Décomposition en
carrés. Algorithme de Gauss} \\ ~12.4
 \\
~~12.4.1 {Formes positives,
négatives} \\ ~~12.4.2 {Bases de
Sylvester. Signature} \\ ~~12.4.3
 \\ ~~12.4.4
 \\
~~12.4.5  \\
~~12.4.6 {Algorithme de
Gram-Schmidt} \\ ~~12.4.7
{Application: polynômes
orthogonaux} \\ ~12.5
{Endomorphismes et formes
quadratiques} \\ ~~12.5.1 {Notion
d'ad\\jmathmathoint} \\ ~~12.5.2 {Ad\\jmathmathoint
en dimension finie} \\ ~~12.5.3
{Endomorphismes symétriques et
formes quadratiques} \\ ~~12.5.4
 \\ ~~12.5.5
 \\ ~12.6
{Endomorphismes d'un espace
euclidien} \\ ~~12.6.1 {Droites
et plans stables} \\ ~~12.6.2
{Réduction des endomorphismes
symétriques} \\ ~~12.6.3 {Normes
d'endomorphismes} \\ ~~12.6.4
{Endomorphismes orthogonaux d'un
plan euclidien} \\ ~~12.6.5
{Réduction des endomorphismes
orthogonaux} \\ ~~12.6.6 {Produit
vectoriel, produit mixte} \\ ~~12.6.7
 \\ 13
 \\ ~13.1
 \\
~~13.1.1 {Applications
semi-linéaires} \\ ~~13.1.2
{Matrices con\\jmathmathuguées et
transcon\\jmathmathuguées} \\ ~~13.1.3
{Matrices hermitiennes,
antihermitiennes} \\ ~13.2 {Formes
sesquilinéaires} \\ ~~13.2.1
 \\ ~~13.2.2
{Formes sesquilinéaires
hermitiennes, antihermitiennes} \\ ~~13.2.3
{Matrice d'une forme
sesquilinéaire} \\ ~~13.2.4
 \\
~~13.2.5  \\
~~13.2.6 
\\ ~13.3 {Formes quadratiques
hermitiennes} \\ ~~13.3.1 {Notion
de forme quadratique hermitienne} \\ ~~13.3.2
{Formes quadratiques hermitiennes
en dimension finie} \\ ~~13.3.3
{Formes quadratiques hermitiennes
définies positives} \\ ~~13.3.4
 \\ ~13.4
{Endomorphismes d'un espace
hermitien} \\ ~~13.4.1 {Notion
d'ad\\jmathmathoint} \\ ~~13.4.2
 \\
~~13.4.3  \\
~~13.4.4  \\
~~13.4.5 {Réduction des
endomorphismes normaux} \\ ~~13.4.6

\\ 14  \\ ~14.1
{Introduction: transformée de
Fourier sur les groupes abéliens finis} \\ ~~14.1.1
{Caractères des groupes abéliens
finis} \\ ~~14.1.2 {Transformée
de Fourier sur un groupe abélien fini} \\ ~14.2
 \\
~~14.2.1 
\\ ~~14.2.2  \\
~~14.2.3 {Un cas de convergence
normale} \\ ~14.3 {Série de Fourier
d'une fonction} \\ ~~14.3.1 {Les
espaces C et D} \\ ~~14.3.2
{Coefficients de Fourier d'une
fonction continue par morceaux} \\ ~~14.3.3
{Inégalité de Bessel et théorème
de Riemann-Lebesgue} \\ ~~14.3.4
 \\
~~14.3.5 {Coefficients de Fourier
des fonctions de classe C^k} \\ ~~14.3.6
 \\
~14.4 {Fonctions périodiques de
période T} \\ ~14.5 {Produit de
convolution} \\ ~~14.5.1
{Convolution de fonctions
périodiques} \\ ~~14.5.2 {Produit
de convolution et séries de Fourier} \\ 15
 \\ ~15.1
 \\ ~~15.1.1
 \\
~~15.1.2 {Composition des
dérivées partielles} \\ ~~15.1.3
{Théorème des accroissements
finis et applications} \\ ~~15.1.4
{Dérivées partielles
successives} \\ ~~15.1.5
 \\ ~~15.1.6
{Application aux extremums de
fonctions de plusieurs variables} \\ ~15.2
 \\ ~~15.2.1

\\ ~~15.2.2 {Exemples
d'applications différentiables} \\ ~~15.2.3
{Opérations sur les
différentielles} \\ ~~15.2.4
{Différentielle et dérivées
partielles} \\ ~~15.2.5
{Matrices \\jmathmathacobiennes,
\\jmathmathacobiens} \\ ~~15.2.6
{Inégalité des accroissements
finis} \\ ~15.3 {Formes
différentielles} \\ ~~15.3.1
{Rappels sur les formes
linéaires alternées} \\ ~~15.3.2

\\ ~~15.3.3 {Notion de gradient
d'une fonction} \\ ~~15.3.4
{Invariance de la
différentielle} \\ ~~15.3.5
 \\
~~15.3.6 
\\ ~15.4 {Fonctions implicites et
inversion locale} \\ ~~15.4.1
{Position du problème des
fonctions implicites} \\ ~~15.4.2
{Théorème des fonctions
implicites} \\ ~~15.4.3
{Applications du théorème des
fonctions implicites} \\ ~~15.4.4
{Difféomorphismes et inversion
locale} \\ 16 {Equations
différentielles} \\ ~16.1 {Notions
générales} \\ ~~16.1.1
{Solutions d'une équation
différentielle} \\ ~~16.1.2
 \\ ~~16.1.3
 \\
~~16.1.4 {Equivalence avec une
équation intégrale} \\ ~~16.1.5
 \\ ~16.2
 \\
~~16.2.1 {Unicité de solutions,
solutions maximales} \\ ~16.3
{Equations différentielles
linéaires d'ordre 1} \\ ~~16.3.1
 \\ ~~16.3.2
{Equation différentielle
linéaire scalaire d'ordre 1} \\ ~~16.3.3
{Théorie de Cauchy-Lipschitz
pour les équations linéaires} \\ ~~16.3.4
{Structure des solutions de
l'équation homogène} \\ ~~16.3.5
{Méthode de variation des
constantes} \\ ~~16.3.6
{Systèmes différentiels à
coefficients constants} \\ ~16.4
{Equation différentielle linéaire
d'ordre n} \\ ~~16.4.1
 \\ ~~16.4.2
 \\
~~16.4.3 {Structure des
solutions de l'équation homogène. Wronskien} \\ ~~16.4.4
{Méthode de variation des
constantes} \\ ~~16.4.5 {Méthode
d'abaissement du degré} \\ ~~16.4.6
{Equation homogène à
coefficients constants} \\ ~~16.4.7
{Equation linéaire à
coefficients constants} \\ ~~16.4.8
 \\ ~16.5
{Equations différentielles non
linéaires} \\ ~~16.5.1 {Théorie
de Cauchy-Lipschitz} \\ ~~16.5.2
{Application aux équations
d'ordre n} \\ ~~16.5.3 {Systèmes
différentiels autonomes d'ordre 1} \\ ~~16.5.4
{Equations différentielles et
formes différentielles} \\ ~~16.5.5
{Equations aux différentielles
totales} \\ ~~16.5.6 {Equations
à variables séparables} \\ ~~16.5.7
{Equations se ramenant à des
équations à variables séparables} \\ ~~16.5.8
 \\ ~16.6
{Analyse numérique des équations
différentielles} \\ ~~16.6.1
 \\ ~~16.6.2
 \\
~~16.6.3 {Equations
différentielles d'ordre supérieur} \\ 17
 \\ ~17.1
{Généralités sur les espaces
affines} \\ ~~17.1.1 {Notion
d'espace affine} \\ ~~17.1.2

\\ ~~17.1.3 
\\ ~~17.1.4 {Parallélisme,
intersection} \\ ~~17.1.5
 \\
~~17.1.6 {Utilisation de repères
affines} \\ ~~17.1.7 {Formes
affines et sous-espaces affines} \\ ~17.2
 \\ ~~17.2.1
 \\
~~17.2.2 {Associativité des
barycentres} \\ ~~17.2.3
{Barycentres, sous-espaces
affines, applications affines} \\ ~~17.2.4
 \\
~17.3 
\\ ~~17.3.1 {Notion d'espace
affine euclidien} \\ ~~17.3.2
{Formule de Leibnitz et
applications} \\ ~~17.3.3
 \\ ~~17.3.4
{Forme réduite d'une isométrie
affine} \\ ~~17.3.5 {Distance à
un sous-espace affine} \\ ~~17.3.6
{Distance de deux sous-espaces
affines} \\ ~17.4 {Cercles,
sphères, triangle} \\ ~~17.4.1
 \\
~~17.4.2  \\
~~17.4.3 {Eléments de géométrie
du triangle} \\ 18  \\
~18.1  \\
~~18.1.1  \\
~~18.1.2 {Equivalence des arcs
paramétrés} \\ ~~18.1.3
 \\ ~~18.1.4
 \\
~~18.1.5 {Plan osculateur,
concavité} \\ ~~18.1.6 {Etude
locale des arcs plans} \\ ~~18.1.7
 \\ ~~18.1.8
{Plan d'étude d'un arc plan en
paramétriques} \\ ~~18.1.9
 \\ ~~18.1.10
 \\ ~18.2
 \\ ~~18.2.1
 \\
~~18.2.2 {Arcs en coordonnées
polaires: étude locale} \\ ~~18.2.3
{Branches infinies et phénomènes
asymptotiques} \\ ~~18.2.4 {Plan
d'étude d'un arc plan en polaires} \\ ~~18.2.5
{Equations polaires
remarquables} \\ ~18.3 {Problèmes
classiques sur les courbes} \\ ~~18.3.1
 \\
~~18.3.2 
\\ ~~18.3.3 {Podaire d'une
courbe} \\ ~~18.3.4 {Conchoïdes
d'une courbe} \\ ~18.4 {Etude
métrique des arcs} \\ ~~18.4.1
 \\ ~~18.4.2

\\ ~~18.4.3 {Abscisses
curvilignes} \\ ~~18.4.4
{Introduction à la méthode du
repère mobile} \\ ~~18.4.5
{Repère de Frénet et courbure
des arcs d'un plan euclidien orienté} \\ ~~18.4.6
{Centre de courbure, cercle
osculateur} \\ ~~18.4.7
 \\
~~18.4.8 {Equations
intrinsèques} \\ ~~18.4.9
 \\
19  \\ ~19.1
 \\ ~~19.1.1
{Notion de nappe paramétrée.
Equivalence} \\ ~~19.1.2
 \\ ~~19.1.3
{Plan tangent à une nappe
paramétrée, vecteur normal} \\ ~~19.1.4
{Points réguliers et nappes
cartésiennes} \\ ~~19.1.5
{Intersection de nappes
paramétrées} \\ ~~19.1.6
{Intersection d'une nappe et de
son plan tangent} \\ ~19.2
 \\ ~~19.2.1
 \\
~~19.2.2 {Plan tangent à une
nappe réglée} \\ ~~19.2.3
{Nappes cylindriques. Nappes
coniques} \\ ~19.3 {Equations de
surfaces} \\ ~~19.3.1 {Surfaces
cartésiennes et nappes paramétrées} \\ ~~19.3.2
 \\ ~~19.3.3
 \\ ~~19.3.4
 \\
~19.4  \\ ~~19.4.1
 \\
~~19.4.2 {Réduction des
quadriques} \\ ~~19.4.3
{Classification des quadriques
en dimension 2 et 3} \\ ~~19.4.4
{Quadriques réglées, quadriques
de révolution} \\ 20 {Intégrales
curvilignes, intégrales multiples} \\ ~20.1
 \\
~~20.1.1 {Formes
différentielles sur un arc paramétré} \\ ~~20.1.2
{Intégrale d'une forme
différentielle sur un arc} \\ ~~20.1.3
{Formes différentielles exactes
et champs de gradients} \\ ~20.2
 \\
~~20.2.1 {Pavés et
subdivisions. Ensembles négligeables} \\ ~~20.2.2
{Intégrales multiples sur un
pavé de \mathbb{R}~^n} \\ ~~20.2.3
{Intégrales multiples sur une
partie de \mathbb{R}~^n} \\ ~~20.2.4
{Mesure d'un sous-ensemble
borné de \mathbb{R}~^n} \\ ~20.3
{Calcul des intégrales doubles et
triples} \\ ~~20.3.1 {Théorème
de Fubini sur une partie de \mathbb{R}~^2} \\ ~~20.3.2
{Théorème de Fubini sur une
partie de \mathbb{R}~^3} \\ ~~20.3.3
{Théorème de changement de
variables dans les intégrales multiples} \\ ~~20.3.4
 \\
~20.4 {Introduction aux
intégrales de surface}

{[}
{[}
{[}
{[}

\end{document}
