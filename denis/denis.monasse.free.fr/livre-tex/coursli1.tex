\documentclass[]{article}
\usepackage[T1]{fontenc}
\usepackage{lmodern}
\usepackage{amssymb,amsmath}
\usepackage{ifxetex,ifluatex}
\usepackage{fixltx2e} % provides \textsubscript
% use upquote if available, for straight quotes in verbatim environments
\IfFileExists{upquote.sty}{\usepackage{upquote}}{}
\ifnum 0\ifxetex 1\fi\ifluatex 1\fi=0 % if pdftex
  \usepackage[utf8]{inputenc}
\else % if luatex or xelatex
  \ifxetex
    \usepackage{mathspec}
    \usepackage{xltxtra,xunicode}
  \else
    \usepackage{fontspec}
  \fi
  \defaultfontfeatures{Mapping=tex-text,Scale=MatchLowercase}
  \newcommand{\euro}{€}
\fi
% use microtype if available
\IfFileExists{microtype.sty}{\usepackage{microtype}}{}
\ifxetex
  \usepackage[setpagesize=false, % page size defined by xetex
              unicode=false, % unicode breaks when used with xetex
              xetex]{hyperref}
\else
  \usepackage[unicode=true]{hyperref}
\fi
\hypersetup{breaklinks=true,
            bookmarks=true,
            pdfauthor={},
            pdftitle={Table des mati`eres},
            colorlinks=true,
            citecolor=blue,
            urlcolor=blue,
            linkcolor=magenta,
            pdfborder={0 0 0}}
\urlstyle{same}  % don't use monospace font for urls
\setlength{\parindent}{0pt}
\setlength{\parskip}{6pt plus 2pt minus 1pt}
\setlength{\emergencystretch}{3em}  % prevent overfull lines
\setcounter{secnumdepth}{0}
 
/* start css.sty */
.cmr-5{font-size:50%;}
.cmr-7{font-size:70%;}
.cmmi-5{font-size:50%;font-style: italic;}
.cmmi-7{font-size:70%;font-style: italic;}
.cmmi-10{font-style: italic;}
.cmsy-5{font-size:50%;}
.cmsy-7{font-size:70%;}
.cmex-7{font-size:70%;}
.cmex-7x-x-71{font-size:49%;}
.msbm-7{font-size:70%;}
.cmtt-10{font-family: monospace;}
.cmti-10{ font-style: italic;}
.cmbx-10{ font-weight: bold;}
.cmr-17x-x-120{font-size:204%;}
.cmsl-10{font-style: oblique;}
.cmti-7x-x-71{font-size:49%; font-style: italic;}
.cmbxti-10{ font-weight: bold; font-style: italic;}
p.noindent { text-indent: 0em }
td p.noindent { text-indent: 0em; margin-top:0em; }
p.nopar { text-indent: 0em; }
p.indent{ text-indent: 1.5em }
@media print {div.crosslinks {visibility:hidden;}}
a img { border-top: 0; border-left: 0; border-right: 0; }
center { margin-top:1em; margin-bottom:1em; }
td center { margin-top:0em; margin-bottom:0em; }
.Canvas { position:relative; }
li p.indent { text-indent: 0em }
.enumerate1 {list-style-type:decimal;}
.enumerate2 {list-style-type:lower-alpha;}
.enumerate3 {list-style-type:lower-roman;}
.enumerate4 {list-style-type:upper-alpha;}
div.newtheorem { margin-bottom: 2em; margin-top: 2em;}
.obeylines-h,.obeylines-v {white-space: nowrap; }
div.obeylines-v p { margin-top:0; margin-bottom:0; }
.overline{ text-decoration:overline; }
.overline img{ border-top: 1px solid black; }
td.displaylines {text-align:center; white-space:nowrap;}
.centerline {text-align:center;}
.rightline {text-align:right;}
div.verbatim {font-family: monospace; white-space: nowrap; text-align:left; clear:both; }
.fbox {padding-left:3.0pt; padding-right:3.0pt; text-indent:0pt; border:solid black 0.4pt; }
div.fbox {display:table}
div.center div.fbox {text-align:center; clear:both; padding-left:3.0pt; padding-right:3.0pt; text-indent:0pt; border:solid black 0.4pt; }
div.minipage{width:100%;}
div.center, div.center div.center {text-align: center; margin-left:1em; margin-right:1em;}
div.center div {text-align: left;}
div.flushright, div.flushright div.flushright {text-align: right;}
div.flushright div {text-align: left;}
div.flushleft {text-align: left;}
.underline{ text-decoration:underline; }
.underline img{ border-bottom: 1px solid black; margin-bottom:1pt; }
.framebox-c, .framebox-l, .framebox-r { padding-left:3.0pt; padding-right:3.0pt; text-indent:0pt; border:solid black 0.4pt; }
.framebox-c {text-align:center;}
.framebox-l {text-align:left;}
.framebox-r {text-align:right;}
span.thank-mark{ vertical-align: super }
span.footnote-mark sup.textsuperscript, span.footnote-mark a sup.textsuperscript{ font-size:80%; }
div.tabular, div.center div.tabular {text-align: center; margin-top:0.5em; margin-bottom:0.5em; }
table.tabular td p{margin-top:0em;}
table.tabular {margin-left: auto; margin-right: auto;}
div.td00{ margin-left:0pt; margin-right:0pt; }
div.td01{ margin-left:0pt; margin-right:5pt; }
div.td10{ margin-left:5pt; margin-right:0pt; }
div.td11{ margin-left:5pt; margin-right:5pt; }
table[rules] {border-left:solid black 0.4pt; border-right:solid black 0.4pt; }
td.td00{ padding-left:0pt; padding-right:0pt; }
td.td01{ padding-left:0pt; padding-right:5pt; }
td.td10{ padding-left:5pt; padding-right:0pt; }
td.td11{ padding-left:5pt; padding-right:5pt; }
table[rules] {border-left:solid black 0.4pt; border-right:solid black 0.4pt; }
.hline hr, .cline hr{ height : 1px; margin:0px; }
.tabbing-right {text-align:right;}
span.TEX {letter-spacing: -0.125em; }
span.TEX span.E{ position:relative;top:0.5ex;left:-0.0417em;}
a span.TEX span.E {text-decoration: none; }
span.LATEX span.A{ position:relative; top:-0.5ex; left:-0.4em; font-size:85%;}
span.LATEX span.TEX{ position:relative; left: -0.4em; }
div.float img, div.float .caption {text-align:center;}
div.figure img, div.figure .caption {text-align:center;}
.marginpar {width:20%; float:right; text-align:left; margin-left:auto; margin-top:0.5em; font-size:85%; text-decoration:underline;}
.marginpar p{margin-top:0.4em; margin-bottom:0.4em;}
.equation td{text-align:center; vertical-align:middle; }
td.eq-no{ width:5%; }
table.equation { width:100%; } 
div.math-display, div.par-math-display{text-align:center;}
math .texttt { font-family: monospace; }
math .textit { font-style: italic; }
math .textsl { font-style: oblique; }
math .textsf { font-family: sans-serif; }
math .textbf { font-weight: bold; }
.partToc a, .partToc, .likepartToc a, .likepartToc {line-height: 200%; font-weight:bold; font-size:110%;}
.chapterToc a, .chapterToc, .likechapterToc a, .likechapterToc, .appendixToc a, .appendixToc {line-height: 200%; font-weight:bold;}
.index-item, .index-subitem, .index-subsubitem {display:block}
.caption td.id{font-weight: bold; white-space: nowrap; }
table.caption {text-align:center;}
h1.partHead{text-align: center}
p.bibitem { text-indent: -2em; margin-left: 2em; margin-top:0.6em; margin-bottom:0.6em; }
p.bibitem-p { text-indent: 0em; margin-left: 2em; margin-top:0.6em; margin-bottom:0.6em; }
.paragraphHead, .likeparagraphHead { margin-top:2em; font-weight: bold;}
.subparagraphHead, .likesubparagraphHead { font-weight: bold;}
.quote {margin-bottom:0.25em; margin-top:0.25em; margin-left:1em; margin-right:1em; text-align:justify;}
.verse{white-space:nowrap; margin-left:2em}
div.maketitle {text-align:center;}
h2.titleHead{text-align:center;}
div.maketitle{ margin-bottom: 2em; }
div.author, div.date {text-align:center;}
div.thanks{text-align:left; margin-left:10%; font-size:85%; font-style:italic; }
div.author{white-space: nowrap;}
.quotation {margin-bottom:0.25em; margin-top:0.25em; margin-left:1em; }
h1.partHead{text-align: center}
.sectionToc, .likesectionToc {margin-left:2em;}
.subsectionToc, .likesubsectionToc {margin-left:4em;}
.subsubsectionToc, .likesubsubsectionToc {margin-left:6em;}
.frenchb-nbsp{font-size:75%;}
.frenchb-thinspace{font-size:75%;}
.figure img.graphics {margin-left:10%;}
/* end css.sty */

\title{Table des mati`eres}
\author{}
\date{}

\begin{document}
\maketitle

\textbf{Warning: \href{http://www.math.union.edu/locate/jsMath}{jsMath}
requires JavaScript to process the mathematics on this page.\\ If your
browser supports JavaScript, be sure it is enabled.}

\begin{center}\rule{3in}{0.4pt}\end{center}

{[}\href{coursch2.html}{next}{]} {[}\href{coursch1.html}{prev}{]}
{[}\href{coursch1.html\#tailcoursch1.html}{prev-tail}{]}
{[}\hyperref[tailcoursli1.html]{tail}{]}
{[}\href{cours.html\#coursli1.html}{up}{]}

\subsection{Table des matières}

\href{coursch1.html\#x2-1000}{Avant-propos} \\ 1
\href{coursch2.html\#x4-60001}{Ensembles et structures} \\ ~1.1
\href{coursse1.html\#x5-70001.1}{Ensembles et relations} \\ ~~1.1.1
\href{coursse1.html\#x5-80001.1.1}{Relations d'équivalences} \\ ~~1.1.2
\href{coursse1.html\#x5-90001.1.2}{Relations d'ordre} \\ ~~1.1.3
\href{coursse1.html\#x5-100001.1.3}{Eléments extrémaux} \\ ~~1.1.4
\href{coursse1.html\#x5-110001.1.4}{L'axiome de Zorn} \\ ~1.2
\href{coursse2.html\#x6-120001.2}{Cardinaux et entiers naturels} \\
~~1.2.1 \href{coursse2.html\#x6-130001.2.1}{Notion de cardinal} \\
~~1.2.2 \href{coursse2.html\#x6-140001.2.2}{Les entiers naturels} \\
~1.3 \href{coursse3.html\#x7-150001.3}{Groupes} \\ ~~1.3.1
\href{coursse3.html\#x7-160001.3.1}{Définitions et première propriété}
\\ ~~1.3.2 \href{coursse3.html\#x7-170001.3.2}{Sous-groupes} \\ ~~1.3.3
\href{coursse3.html\#x7-180001.3.3}{Quotient par un sous-groupe} \\
~~1.3.4 \href{coursse3.html\#x7-190001.3.4}{Morphisme de groupes} \\
~~1.3.5 \href{coursse3.html\#x7-200001.3.5}{Le groupe ℤ} \\ ~~1.3.6
\href{coursse3.html\#x7-210001.3.6}{Ordre d'un élément} \\ ~~1.3.7
\href{coursse3.html\#x7-220001.3.7}{Groupes finis} \\ ~~1.3.8
\href{coursse3.html\#x7-230001.3.8}{Groupes cycliques} \\ ~~1.3.9
\href{coursse3.html\#x7-240001.3.9}{Groupe opérant sur un ensemble} \\
~~1.3.10 \href{coursse3.html\#x7-250001.3.10}{Groupe des permutations
d'un ensemble fini} \\ ~1.4 \href{coursse4.html\#x8-260001.4}{Anneaux et
corps} \\ ~~1.4.1 \href{coursse4.html\#x8-270001.4.1}{Généralités sur
les anneaux} \\ ~~1.4.2 \href{coursse4.html\#x8-280001.4.2}{Idéaux et
quotients} \\ ~~1.4.3 \href{coursse4.html\#x8-290001.4.3}{Morphisme
d'anneaux} \\ ~~1.4.4 \href{coursse4.html\#x8-300001.4.4}{Corps} \\
~~1.4.5 \href{coursse4.html\#x8-310001.4.5}{Idéaux maximaux} \\ ~~1.4.6
\href{coursse4.html\#x8-320001.4.6}{Idéaux et anneaux principaux} \\
~~1.4.7 \href{coursse4.html\#x8-330001.4.7}{Anneaux euclidiens} \\
~~1.4.8 \href{coursse4.html\#x8-340001.4.8}{L'anneau ℤ. Caractéristique
d'un anneau} \\ ~~1.4.9 \href{coursse4.html\#x8-350001.4.9}{Théorème
chinois, indicateur d'Euler} \\ ~1.5
\href{coursse5.html\#x9-360001.5}{Polynômes à une variable} \\ ~~1.5.1
\href{coursse5.html\#x9-370001.5.1}{L'anneau des séries formelles à
coefficients dans A} \\ ~~1.5.2
\href{coursse5.html\#x9-380001.5.2}{L'anneau des polynômes à
coefficients dans A} \\ ~~1.5.3
\href{coursse5.html\#x9-390001.5.3}{Division euclidienne et racines} \\
~~1.5.4 \href{coursse5.html\#x9-400001.5.4}{Dérivation} \\ ~~1.5.5
\href{coursse5.html\#x9-410001.5.5}{L'anneau principal K{[}X{]}} \\
~~1.5.6 \href{coursse5.html\#x9-420001.5.6}{Formule de Taylor.
Multiplicité d'une racine} \\ ~~1.5.7
\href{coursse5.html\#x9-430001.5.7}{Racines et extensions de corps} \\
~~1.5.8 \href{coursse5.html\#x9-440001.5.8}{Polynômes sur ℂ et ℝ} \\
~~1.5.9 \href{coursse5.html\#x9-450001.5.9}{Division suivant les
puissances croissantes} \\ ~1.6
\href{coursse6.html\#x10-460001.6}{Polynômes à plusieurs variables} \\
~~1.6.1 \href{coursse6.html\#x10-470001.6.1}{Généralités} \\ ~~1.6.2
\href{coursse6.html\#x10-480001.6.2}{Dérivées partielles, formule de
Taylor} \\ ~~1.6.3 \href{coursse6.html\#x10-490001.6.3}{Degré total,
polynômes homogènes} \\ ~~1.6.4
\href{coursse6.html\#x10-500001.6.4}{Polynômes symétriques} \\ 2
\href{coursch3.html\#x11-510002}{Algèbre linéaire élémentaire} \\ ~2.1
\href{coursse7.html\#x12-520002.1}{Généralités sur les espaces
vectoriels} \\ ~~2.1.1 \href{coursse7.html\#x12-530002.1.1}{Notion de
K-espace vectoriel} \\ ~~2.1.2
\href{coursse7.html\#x12-540002.1.2}{Notion de sous-espace vectoriel} \\
~~2.1.3 \href{coursse7.html\#x12-550002.1.3}{Produits, quotients} \\
~~2.1.4 \href{coursse7.html\#x12-560002.1.4}{Applications linéaires} \\
~~2.1.5 \href{coursse7.html\#x12-570002.1.5}{Somme de sous-espaces} \\
~~2.1.6 \href{coursse7.html\#x12-580002.1.6}{Algèbres} \\ ~~2.1.7
\href{coursse7.html\#x12-590002.1.7}{Familles libres, génératrices.
Bases} \\ ~~2.1.8 \href{coursse7.html\#x12-600002.1.8}{Théorèmes
fondamentaux} \\ ~2.2 \href{coursse8.html\#x13-610002.2}{Bases et
dimension} \\ ~~2.2.1 \href{coursse8.html\#x13-620002.2.1}{Existence de
bases} \\ ~~2.2.2 \href{coursse8.html\#x13-630002.2.2}{Espaces
vectoriels de dimension finie. Dimension} \\ ~~2.2.3
\href{coursse8.html\#x13-640002.2.3}{Résultats sur la dimension} \\ ~2.3
\href{coursse9.html\#x14-650002.3}{Rang} \\ ~~2.3.1
\href{coursse9.html\#x14-660002.3.1}{Rang d'une famille de vecteurs} \\
~~2.3.2 \href{coursse9.html\#x14-670002.3.2}{Rang d'une application
linéaire} \\ ~2.4 \href{coursse10.html\#x15-680002.4}{Dualité: approche
restreinte} \\ ~~2.4.1 \href{coursse10.html\#x15-690002.4.1}{Formes
linéaires, dual, formes coordonnées} \\ ~~2.4.2
\href{coursse10.html\#x15-700002.4.2}{Base duale d'un espace vectoriel
de dimension finie} \\ ~~2.4.3
\href{coursse10.html\#x15-710002.4.3}{Orthogonalité 1} \\ ~~2.4.4
\href{coursse10.html\#x15-720002.4.4}{Hyperplans} \\ ~~2.4.5
\href{coursse10.html\#x15-730002.4.5}{Orthogonalité 2} \\ ~~2.4.6
\href{coursse10.html\#x15-740002.4.6}{Application: polynômes
d'interpolation de Lagrange} \\ ~2.5
\href{coursse11.html\#x16-750002.5}{Dualité: approche générale} \\
~~2.5.1 \href{coursse11.html\#x16-760002.5.1}{Notion de dual.
Orthogonalité} \\ ~~2.5.2
\href{coursse11.html\#x16-770002.5.2}{Hyperplans} \\ ~~2.5.3
\href{coursse11.html\#x16-780002.5.3}{Bidual} \\ ~~2.5.4
\href{coursse11.html\#x16-790002.5.4}{Transposée} \\ ~~2.5.5
\href{coursse11.html\#x16-800002.5.5}{Dualité en dimension finie} \\
~2.6 \href{coursse12.html\#x17-810002.6}{Matrices} \\ ~~2.6.1
\href{coursse12.html\#x17-820002.6.1}{Généralités} \\ ~~2.6.2
\href{coursse12.html\#x17-830002.6.2}{Matrices carrées} \\ ~~2.6.3
\href{coursse12.html\#x17-840002.6.3}{Transposée} \\ ~~2.6.4
\href{coursse12.html\#x17-850002.6.4}{Rang d'une matrice} \\ ~~2.6.5
\href{coursse12.html\#x17-860002.6.5}{La méthode du pivot} \\ ~~2.6.6
\href{coursse12.html\#x17-870002.6.6}{Changement de bases} \\ ~~2.6.7
\href{coursse12.html\#x17-880002.6.7}{Produit des matrices par blocs} \\
~2.7 \href{coursse13.html\#x18-890002.7}{Déterminants} \\ ~~2.7.1
\href{coursse13.html\#x18-900002.7.1}{Formes p-linéaires} \\ ~~2.7.2
\href{coursse13.html\#x18-910002.7.2}{Déterminant d'une famille de
vecteurs} \\ ~~2.7.3 \href{coursse13.html\#x18-920002.7.3}{Déterminant
d'un endomorphisme} \\ ~~2.7.4
\href{coursse13.html\#x18-930002.7.4}{Déterminant d'une matrice} \\
~~2.7.5 \href{coursse13.html\#x18-940002.7.5}{Application des
déterminants à la recherche du rang} \\ ~~2.7.6
\href{coursse13.html\#x18-950002.7.6}{Formes p-linéaires alternées} \\
~2.8 \href{coursse14.html\#x19-960002.8}{Systèmes linéaires} \\ ~~2.8.1
\href{coursse14.html\#x19-970002.8.1}{Position du problème} \\ ~~2.8.2
\href{coursse14.html\#x19-980002.8.2}{Systèmes de Cramer} \\ ~~2.8.3
\href{coursse14.html\#x19-990002.8.3}{Théorème de Rouché-Fontené} \\
~~2.8.4 \href{coursse14.html\#x19-1000002.8.4}{Méthode du pivot} \\ 3
\href{coursch4.html\#x20-1010003}{Réduction des endomorphismes} \\ ~3.1
\href{coursse15.html\#x21-1020003.1}{Valeurs propres. Vecteurs propres}
\\ ~~3.1.1 \href{coursse15.html\#x21-1030003.1.1}{Sous-espaces stables}
\\ ~~3.1.2 \href{coursse15.html\#x21-1040003.1.2}{Valeurs propres,
vecteurs propres} \\ ~~3.1.3
\href{coursse15.html\#x21-1050003.1.3}{Polynôme caractéristique} \\
~~3.1.4 \href{coursse15.html\#x21-1060003.1.4}{Endomorphismes
diagonalisables} \\ ~~3.1.5
\href{coursse15.html\#x21-1070003.1.5}{Matrices diagonalisables} \\
~~3.1.6 \href{coursse15.html\#x21-1080003.1.6}{Endomorphismes et
matrices trigonalisables} \\ ~3.2
\href{coursse16.html\#x22-1090003.2}{Polynômes d'endomorphismes} \\
~~3.2.1 \href{coursse16.html\#x22-1100003.2.1}{Généralités} \\ ~~3.2.2
\href{coursse16.html\#x22-1110003.2.2}{Idéal annulateur. Polynôme
minimal} \\ ~~3.2.3 \href{coursse16.html\#x22-1120003.2.3}{Théorème de
Cayley-Hamilton} \\ ~~3.2.4
\href{coursse16.html\#x22-1130003.2.4}{Polynôme annulateur et
trigonalisation} \\ ~~3.2.5
\href{coursse16.html\#x22-1140003.2.5}{Décomposition des noyaux} \\
~~3.2.6 \href{coursse16.html\#x22-1150003.2.6}{Sous-espaces
caractéristiques} \\ ~~3.2.7
\href{coursse16.html\#x22-1160003.2.7}{Application: récurrences
linéaires d'ordre 2} \\ ~3.3 \href{coursse17.html\#x23-1170003.3}{A
propos de Jordan} \\ ~~3.3.1
\href{coursse17.html\#x23-1180003.3.1}{Décomposition de Jordan} \\
~~3.3.2 \href{coursse17.html\#x23-1190003.3.2}{Applications} \\ ~~3.3.3
\href{coursse17.html\#x23-1200003.3.3}{Réduction des endomorphismes
nilpotents} \\ ~~3.3.4 \href{coursse17.html\#x23-1210003.3.4}{Première
démonstration} \\ ~~3.3.5
\href{coursse17.html\#x23-1220003.3.5}{Deuxième démonstration} \\
~~3.3.6 \href{coursse17.html\#x23-1230003.3.6}{Réduction de Jordan} \\ 4
\href{coursch5.html\#x24-1240004}{Topologie des espaces métriques} \\
~4.1 \href{coursse18.html\#x25-1250004.1}{Eléments de topologie
générale} \\ ~~4.1.1 \href{coursse18.html\#x25-1260004.1.1}{Espaces
topologiques} \\ ~~4.1.2 \href{coursse18.html\#x25-1270004.1.2}{La
topologie de ℝ} \\ ~~4.1.3 \href{coursse18.html\#x25-1280004.1.3}{Fermés
et voisinages} \\ ~~4.1.4
\href{coursse18.html\#x25-1290004.1.4}{Intérieur, adhérence, frontière}
\\ ~~4.1.5 \href{coursse18.html\#x25-1300004.1.5}{Topologie induite} \\
~4.2 \href{coursse19.html\#x26-1310004.2}{Espaces métriques} \\ ~~4.2.1
\href{coursse19.html\#x26-1320004.2.1}{Distances} \\ ~~4.2.2
\href{coursse19.html\#x26-1330004.2.2}{Topologie définie par une
distance} \\ ~~4.2.3 \href{coursse19.html\#x26-1340004.2.3}{Points
isolés, points d'accumulation} \\ ~~4.2.4
\href{coursse19.html\#x26-1350004.2.4}{Propriété de séparation} \\
~~4.2.5 \href{coursse19.html\#x26-1360004.2.5}{Changement de distances}
\\ ~~4.2.6 \href{coursse19.html\#x26-1370004.2.6}{La droite numérique
achevée} \\ ~4.3 \href{coursse20.html\#x27-1380004.3}{Suites} \\ ~~4.3.1
\href{coursse20.html\#x27-1390004.3.1}{Suites convergentes, limites} \\
~~4.3.2 \href{coursse20.html\#x27-1400004.3.2}{Sous suites, valeurs
d'adhérences} \\ ~~4.3.3
\href{coursse20.html\#x27-1410004.3.3}{Caractérisation des fermés d'un
espace métrique} \\ ~4.4 \href{coursse21.html\#x28-1420004.4}{Limites de
fonctions} \\ ~~4.4.1 \href{coursse21.html\#x28-1430004.4.1}{Notion de
limite suivant une partie} \\ ~~4.4.2
\href{coursse21.html\#x28-1440004.4.2}{Propriétés élémentaires} \\
~~4.4.3 \href{coursse21.html\#x28-1450004.4.3}{Composition des limites}
\\ ~~4.4.4 \href{coursse21.html\#x28-1460004.4.4}{Limites et suites} \\
~4.5 \href{coursse22.html\#x29-1470004.5}{Continuité} \\ ~~4.5.1
\href{coursse22.html\#x29-1480004.5.1}{Continuité en un point} \\
~~4.5.2 \href{coursse22.html\#x29-1490004.5.2}{Continuité sur un espace}
\\ ~~4.5.3 \href{coursse22.html\#x29-1500004.5.3}{Homéomorphismes} \\
~4.6 \href{coursse23.html\#x30-1510004.6}{Continuité uniforme} \\
~~4.6.1 \href{coursse23.html\#x30-1520004.6.1}{Applications uniformément
continues} \\ ~~4.6.2
\href{coursse23.html\#x30-1530004.6.2}{Applications lipschitziennes} \\
~4.7 \href{coursse24.html\#x31-1540004.7}{Espaces complets} \\ ~~4.7.1
\href{coursse24.html\#x31-1550004.7.1}{Suites de Cauchy} \\ ~~4.7.2
\href{coursse24.html\#x31-1560004.7.2}{Espaces complets} \\ ~~4.7.3
\href{coursse24.html\#x31-1570004.7.3}{Propriétés des espaces complets}
\\ ~4.8 \href{coursse25.html\#x32-1580004.8}{Espaces et parties
compactes} \\ ~~4.8.1 \href{coursse25.html\#x32-1590004.8.1}{Propriété
de Bolzano-Weierstrass} \\ ~~4.8.2
\href{coursse25.html\#x32-1600004.8.2}{Propriété de Borel Lebesgue} \\
~~4.8.3 \href{coursse25.html\#x32-1610004.8.3}{Compacts de ℝ et
\{ℝ\}\^{}\{n\}} \\ ~4.9 \href{coursse26.html\#x33-1620004.9}{Espaces et
parties connexes} \\ ~~4.9.1
\href{coursse26.html\#x33-1630004.9.1}{Notion de connexe} \\ ~~4.9.2
\href{coursse26.html\#x33-1640004.9.2}{Propriétés des connexes} \\
~~4.9.3 \href{coursse26.html\#x33-1650004.9.3}{Connexes de ℝ} \\ ~~4.9.4
\href{coursse26.html\#x33-1660004.9.4}{Connexité par arcs} \\ 5
\href{coursch6.html\#x34-1670005}{Espaces vectoriels normés} \\ ~5.1
\href{coursse27.html\#x35-1680005.1}{Notion d'espace vectoriel normé} \\
~~5.1.1 \href{coursse27.html\#x35-1690005.1.1}{Norme et distance
associée} \\ ~~5.1.2 \href{coursse27.html\#x35-1700005.1.2}{Convexes,
connexes} \\ ~~5.1.3 \href{coursse27.html\#x35-1710005.1.3}{Continuité
des opérations algébriques} \\ ~5.2
\href{coursse28.html\#x36-1720005.2}{Applications linéaires continues}
\\ ~~5.2.1 \href{coursse28.html\#x36-1730005.2.1}{Caractérisations et
normes des applications linéaires continues} \\ ~~5.2.2
\href{coursse28.html\#x36-1740005.2.2}{L'espace vectoriel normé des
applications linéaires continues de E dans F} \\ ~~5.2.3
\href{coursse28.html\#x36-1750005.2.3}{Equivalence des normes} \\
~~5.2.4 \href{coursse28.html\#x36-1760005.2.4}{Caractérisation des
applications bilinéaires continues} \\ ~5.3
\href{coursse29.html\#x37-1770005.3}{Espaces vectoriels normés de
dimensions finies} \\ ~~5.3.1
\href{coursse29.html\#x37-1780005.3.1}{Equivalence des normes} \\
~~5.3.2 \href{coursse29.html\#x37-1790005.3.2}{Propriétés topologiques
et métriques des espaces vectoriels normés de dimension finie} \\
~~5.3.3 \href{coursse29.html\#x37-1800005.3.3}{Continuité des
applications linéaires} \\ ~5.4
\href{coursse30.html\#x38-1810005.4}{Compléments: le théorème de Baire
et ses conséquences} \\ ~~5.4.1
\href{coursse30.html\#x38-1820005.4.1}{Le théorème de Baire} \\ ~~5.4.2
\href{coursse30.html\#x38-1830005.4.2}{Les grands théorèmes} \\ ~5.5
\href{coursse31.html\#x39-1840005.5}{Compléments: convexité dans les
espaces vectoriels normés} \\ ~~5.5.1
\href{coursse31.html\#x39-1850005.5.1}{Jauge d'un convexe} \\ ~~5.5.2
\href{coursse31.html\#x39-1860005.5.2}{Projection sur un convexe fermé}
\\ ~~5.5.3 \href{coursse31.html\#x39-1870005.5.3}{Hahn-Banach (version
géométrique)} \\ ~~5.5.4
\href{coursse31.html\#x39-1880005.5.4}{L'enveloppe convexe: Carathéodory
et Krein Millman} \\ 6 \href{coursch7.html\#x40-1890006}{Comparaison des
fonctions} \\ ~6.1 \href{coursse32.html\#x41-1900006.1}{Relations de
comparaison} \\ ~~6.1.1
\href{coursse32.html\#x41-1910006.1.1}{Notations} \\ ~~6.1.2
\href{coursse32.html\#x41-1920006.1.2}{Domination, prépondérance} \\
~~6.1.3 \href{coursse32.html\#x41-1930006.1.3}{Equivalence} \\ ~~6.1.4
\href{coursse32.html\#x41-1940006.1.4}{Changement de variables} \\ ~6.2
\href{coursse33.html\#x42-1950006.2}{Développements limités} \\ ~~6.2.1
\href{coursse33.html\#x42-1960006.2.1}{Notion de développement limité}
\\ ~~6.2.2 \href{coursse33.html\#x42-1970006.2.2}{Opérations sur les
développements limités} \\ ~~6.2.3
\href{coursse33.html\#x42-1980006.2.3}{Développements limités
classiques} \\ ~6.3 \href{coursse34.html\#x43-1990006.3}{Développements
asymptotiques} \\ ~~6.3.1
\href{coursse34.html\#x43-2000006.3.1}{Echelles de comparaison, parties
principales} \\ ~~6.3.2
\href{coursse34.html\#x43-2010006.3.2}{Développements asymptotiques} \\
~~6.3.3 \href{coursse34.html\#x43-2020006.3.3}{Opérations sur les
développements asymptotiques} \\ 7
\href{coursch8.html\#x44-2030007}{Suites et séries} \\ ~7.1
\href{coursse35.html\#x45-2040007.1}{Convergence des suites} \\ ~~7.1.1
\href{coursse35.html\#x45-2050007.1.1}{Monotonie (suites à termes
réels)} \\ ~~7.1.2 \href{coursse35.html\#x45-2060007.1.2}{Critère de
Cauchy} \\ ~~7.1.3 \href{coursse35.html\#x45-2070007.1.3}{Valeurs
d'adhérences, limites inférieures et supérieures} \\ ~~7.1.4
\href{coursse35.html\#x45-2080007.1.4}{Récurrences d'ordre 1} \\ ~7.2
\href{coursse36.html\#x46-2090007.2}{Généralités sur les séries} \\
~~7.2.1 \href{coursse36.html\#x46-2100007.2.1}{Notion de série} \\
~~7.2.2 \href{coursse36.html\#x46-2110007.2.2}{Terme général, critère de
Cauchy} \\ ~7.3 \href{coursse37.html\#x47-2120007.3}{Séries à termes
réels positifs} \\ ~~7.3.1
\href{coursse37.html\#x47-2130007.3.1}{Convergence des séries à termes
réels positifs} \\ ~~7.3.2
\href{coursse37.html\#x47-2140007.3.2}{Comparaison des séries à termes
réels positifs} \\ ~~7.3.3 \href{coursse37.html\#x47-2150007.3.3}{Séries
de Riemann et de Bertrand} \\ ~~7.3.4
\href{coursse37.html\#x47-2160007.3.4}{Comparaison à des intégrales} \\
~7.4 \href{coursse38.html\#x48-2170007.4}{Séries absolument
convergentes} \\ ~~7.4.1 \href{coursse38.html\#x48-2180007.4.1}{Notion
de convergence absolue} \\ ~~7.4.2
\href{coursse38.html\#x48-2190007.4.2}{Critères de convergence absolue}
\\ ~~7.4.3 \href{coursse38.html\#x48-2200007.4.3}{Règles classiques} \\
~~7.4.4 \href{coursse38.html\#x48-2210007.4.4}{Règles complémentaires}
\\ ~~7.4.5 \href{coursse38.html\#x48-2220007.4.5}{Comparaison à une
intégrale} \\ ~7.5 \href{coursse39.html\#x49-2230007.5}{Séries
semi-convergentes} \\ ~~7.5.1
\href{coursse39.html\#x49-2240007.5.1}{Séries alternées} \\ ~~7.5.2
\href{coursse39.html\#x49-2250007.5.2}{Etude de séries
semi-convergentes} \\ ~7.6
\href{coursse40.html\#x50-2260007.6}{Opérations sur les séries} \\
~~7.6.1 \href{coursse40.html\#x50-2270007.6.1}{Combinaisons linéaires}
\\ ~~7.6.2 \href{coursse40.html\#x50-2280007.6.2}{Sommation par paquets}
\\ ~~7.6.3 \href{coursse40.html\#x50-2290007.6.3}{Modification de
l'ordre des termes} \\ ~~7.6.4
\href{coursse40.html\#x50-2300007.6.4}{Produit de Cauchy} \\ ~7.7
\href{coursse41.html\#x51-2310007.7}{Séries doubles} \\ ~7.8
\href{coursse42.html\#x52-2320007.8}{Espaces de suites} \\ ~7.9
\href{coursse43.html\#x53-2330007.9}{Compléments: développements
asymptotiques, analyse numérique} \\ ~~7.9.1
\href{coursse43.html\#x53-2340007.9.1}{Calcul approché de la somme d'une
série} \\ ~~7.9.2 \href{coursse43.html\#x53-2350007.9.2}{Accélération de
la convergence} \\ 8 \href{coursch9.html\#x54-2360008}{Fonctions d'une
variable réelle} \\ ~8.1 \href{coursse44.html\#x55-2370008.1}{Monotonie,
continuité} \\ ~~8.1.1 \href{coursse44.html\#x55-2380008.1.1}{Limites et
monotonie} \\ ~~8.1.2 \href{coursse44.html\#x55-2390008.1.2}{Continuité
et monotonie} \\ ~8.2 \href{coursse45.html\#x56-2400008.2}{Dérivée} \\
~~8.2.1 \href{coursse45.html\#x56-2410008.2.1}{Notion de dérivée} \\
~~8.2.2 \href{coursse45.html\#x56-2420008.2.2}{Opérations sur les
dérivées} \\ ~~8.2.3 \href{coursse45.html\#x56-2430008.2.3}{Dérivées
d'ordre supérieur} \\ ~8.3
\href{coursse46.html\#x57-2440008.3}{Fonctions réelles d'une variable
réelle} \\ ~~8.3.1 \href{coursse46.html\#x57-2450008.3.1}{Théorème de
Rolle, formule des accroissements finis} \\ ~~8.3.2
\href{coursse46.html\#x57-2460008.3.2}{Monotonie et dérivation} \\
~~8.3.3 \href{coursse46.html\#x57-2470008.3.3}{Difféomorphismes} \\
~~8.3.4 \href{coursse46.html\#x57-2480008.3.4}{Formule de Taylor
Lagrange} \\ ~~8.3.5 \href{coursse46.html\#x57-2490008.3.5}{Extensions
du théorème des accroissements finis} \\ ~~8.3.6
\href{coursse46.html\#x57-2500008.3.6}{Fonctions convexes de classe
\{C\}\^{}\{1\}} \\ ~8.4 \href{coursse47.html\#x58-2510008.4}{Fonctions
vectorielles d'une variable réelle} \\ ~~8.4.1
\href{coursse47.html\#x58-2520008.4.1}{Inégalité des accroissements
finis} \\ ~~8.4.2 \href{coursse47.html\#x58-2530008.4.2}{Applications de
l'inégalité des accroissements finis} \\ ~~8.4.3
\href{coursse47.html\#x58-2540008.4.3}{Formules de Taylor} \\ ~8.5
\href{coursse48.html\#x59-2550008.5}{Fonctions classiques} \\ ~~8.5.1
\href{coursse48.html\#x59-2560008.5.1}{Fonctions circulaires
réciproques} \\ ~~8.5.2 \href{coursse48.html\#x59-2570008.5.2}{Fonctions
hyperboliques directes} \\ ~~8.5.3
\href{coursse48.html\#x59-2580008.5.3}{Fonctions hyperboliques
réciproques} \\ ~8.6 \href{coursse49.html\#x60-2590008.6}{Analyse
numérique des fonctions d'une variable} \\ ~~8.6.1
\href{coursse49.html\#x60-2600008.6.1}{Interpolation linéaire,
interpolation polynomiale} \\ ~~8.6.2
\href{coursse49.html\#x60-2610008.6.2}{Dérivation numérique} \\ ~~8.6.3
\href{coursse49.html\#x60-2620008.6.3}{Recherche des zéros d'une
fonction} \\ 9 \href{coursch10.html\#x61-2630009}{Intégration} \\ ~9.1
\href{coursse50.html\#x62-2640009.1}{Subdivisions, approximation des
fonctions} \\ ~~9.1.1
\href{coursse50.html\#x62-2650009.1.1}{Subdivisions d'un segment} \\
~~9.1.2 \href{coursse50.html\#x62-2660009.1.2}{Propriétés liées aux
subdivisions} \\ ~~9.1.3
\href{coursse50.html\#x62-2670009.1.3}{Approximation des fonctions} \\
~9.2 \href{coursse51.html\#x63-2680009.2}{Intégrale des fonctions
réglées sur un segment} \\ ~~9.2.1
\href{coursse51.html\#x63-2690009.2.1}{Intégrale des applications en
escalier} \\ ~~9.2.2 \href{coursse51.html\#x63-2700009.2.2}{Intégrale
des fonctions réglées} \\ ~~9.2.3
\href{coursse51.html\#x63-2710009.2.3}{Convention de Chasles} \\ ~~9.2.4
\href{coursse51.html\#x63-2720009.2.4}{Sommes de Riemann} \\ ~~9.2.5
\href{coursse51.html\#x63-2730009.2.5}{Sommes de Darboux} \\ ~9.3
\href{coursse52.html\#x64-2740009.3}{Primitives et intégrales} \\
~~9.3.1 \href{coursse52.html\#x64-2750009.3.1}{Continuité et
dérivabilité par rapport à une borne} \\ ~~9.3.2
\href{coursse52.html\#x64-2760009.3.2}{Primitives} \\ ~~9.3.3
\href{coursse52.html\#x64-2770009.3.3}{Changement de variable,
intégration par parties} \\ ~~9.3.4
\href{coursse52.html\#x64-2780009.3.4}{Deuxième formule de la moyenne}
\\ ~9.4 \href{coursse53.html\#x65-2790009.4}{Recherches de primitives}
\\ ~~9.4.1 \href{coursse53.html\#x65-2800009.4.1}{Position du problème}
\\ ~~9.4.2 \href{coursse53.html\#x65-2810009.4.2}{Techniques usuelles}
\\ ~~9.4.3 \href{coursse53.html\#x65-2820009.4.3}{Primitives usuelles}
\\ ~~9.4.4 \href{coursse53.html\#x65-2830009.4.4}{Fractions
rationnelles} \\ ~~9.4.5
\href{coursse53.html\#x65-2840009.4.5}{Fractions rationnelles en sinus
et cosinus} \\ ~~9.4.6 \href{coursse53.html\#x65-2850009.4.6}{Fractions
rationnelles en sinus et cosinus hyperboliques} \\ ~~9.4.7
\href{coursse53.html\#x65-2860009.4.7}{Intégrales abéliennes} \\ ~9.5
\href{coursse54.html\#x66-2870009.5}{Intégration sur un intervalle
quelconque: fonctions à valeurs réelles positives} \\ ~~9.5.1
\href{coursse54.html\#x66-2880009.5.1}{Fonctions intégrables à valeurs
réelles positives} \\ ~~9.5.2
\href{coursse54.html\#x66-2890009.5.2}{Règles de comparaison} \\ ~~9.5.3
\href{coursse54.html\#x66-2900009.5.3}{Exemples fondamentaux} \\ ~9.6
\href{coursse55.html\#x67-2910009.6}{Intégration sur un intervalle
quelconque: fonctions à valeurs complexes} \\ ~~9.6.1
\href{coursse55.html\#x67-2920009.6.1}{Fonctions à valeurs complexes
intégrables} \\ ~~9.6.2
\href{coursse55.html\#x67-2930009.6.2}{Décomposition des fonctions à
valeurs complexes} \\ ~~9.6.3
\href{coursse55.html\#x67-2940009.6.3}{Convention et relation de
Chasles} \\ ~~9.6.4 \href{coursse55.html\#x67-2950009.6.4}{Règles de
comparaison} \\ ~~9.6.5 \href{coursse55.html\#x67-2960009.6.5}{Espaces
de fonctions continues} \\ ~~9.6.6
\href{coursse55.html\#x67-2970009.6.6}{Notion d'intégrale impropre} \\
~9.7 \href{coursse56.html\#x68-2980009.7}{Développements asymptotiques
et analyse numérique} \\ ~~9.7.1
\href{coursse56.html\#x68-2990009.7.1}{La formule d'Euler-Mac Laurin} \\
~~9.7.2 \href{coursse56.html\#x68-3000009.7.2}{Calcul approché
d'intégrales} \\ ~~9.7.3 \href{coursse56.html\#x68-3010009.7.3}{La
méthode de Laplace} \\ ~9.8
\href{coursse57.html\#x69-3020009.8}{Généralités sur les intégrales
impropres} \\ ~~9.8.1 \href{coursse57.html\#x69-3030009.8.1}{Notion
d'intégrale impropre} \\ ~~9.8.2
\href{coursse57.html\#x69-3040009.8.2}{Intégrales plusieurs fois
impropres} \\ ~~9.8.3 \href{coursse57.html\#x69-3050009.8.3}{Opérations
sur les intégrales impropres} \\ ~~9.8.4
\href{coursse57.html\#x69-3060009.8.4}{Intégrales et séries: intégration
par paquets} \\ ~9.9 \href{coursse58.html\#x70-3070009.9}{Intégrale des
fonctions réelles positives} \\ ~~9.9.1
\href{coursse58.html\#x70-3080009.9.1}{Critère de convergence des
fonctions réelles positives} \\ ~~9.9.2
\href{coursse58.html\#x70-3090009.9.2}{Règles de comparaison} \\ ~~9.9.3
\href{coursse58.html\#x70-3100009.9.3}{Exemples fondamentaux} \\ ~9.10
\href{coursse59.html\#x71-3110009.10}{Convergence absolue,
semi-convergence} \\ ~~9.10.1
\href{coursse59.html\#x71-3120009.10.1}{Critère de Cauchy pour les
intégrales} \\ ~~9.10.2
\href{coursse59.html\#x71-3130009.10.2}{Convergence absolue} \\ ~~9.10.3
\href{coursse59.html\#x71-3140009.10.3}{Règles de convergence} \\
~~9.10.4 \href{coursse59.html\#x71-3150009.10.4}{Semi-convergence} \\ 10
\href{coursch11.html\#x72-31600010}{Suites et séries de fonctions} \\
~10.1 \href{coursse60.html\#x73-31700010.1}{Suites de fonctions} \\
~~10.1.1 \href{coursse60.html\#x73-31800010.1.1}{Convergence simple,
convergence uniforme} \\ ~~10.1.2
\href{coursse60.html\#x73-31900010.1.2}{Plan d'étude d'une suite de
fonctions} \\ ~~10.1.3 \href{coursse60.html\#x73-32000010.1.3}{Critère
de Cauchy uniforme} \\ ~~10.1.4
\href{coursse60.html\#x73-32100010.1.4}{Fonctions bornées, norme de la
convergence uniforme} \\ ~~10.1.5
\href{coursse60.html\#x73-32200010.1.5}{Opérations sur les fonctions} \\
~~10.1.6 \href{coursse60.html\#x73-32300010.1.6}{Propriétés de la
convergence uniforme} \\ ~~10.1.7
\href{coursse60.html\#x73-32400010.1.7}{Suites de fonctions intégrables
sur un intervalle} \\ ~10.2 \href{coursse61.html\#x74-32500010.2}{Séries
de fonctions} \\ ~~10.2.1
\href{coursse61.html\#x74-32600010.2.1}{Différents modes de convergence}
\\ ~~10.2.2 \href{coursse61.html\#x74-32700010.2.2}{Critères
supplémentaires de convergence uniforme} \\ ~~10.2.3
\href{coursse61.html\#x74-32800010.2.3}{Propriétés de la convergence
uniforme} \\ ~~10.2.4 \href{coursse61.html\#x74-32900010.2.4}{Séries de
fonctions intégrables sur un intervalle} \\ ~10.3
\href{coursse62.html\#x75-33000010.3}{Intégrales dépendant d'un
paramètre} \\ ~~10.3.1 \href{coursse62.html\#x75-33100010.3.1}{Position
du problème} \\ ~~10.3.2
\href{coursse62.html\#x75-33200010.3.2}{Continuité} \\ ~~10.3.3
\href{coursse62.html\#x75-33300010.3.3}{Dérivabilité} \\ ~~10.3.4
\href{coursse62.html\#x75-33400010.3.4}{Théorème de Fubini sur un
produit de segments} \\ ~~10.3.5
\href{coursse62.html\#x75-33500010.3.5}{Intégrales sur un pavé ou un
rectangle} \\ ~~10.3.6 \href{coursse62.html\#x75-33600010.3.6}{Théorème
de Fubini sur un produit d'intervalles} \\ ~~10.3.7
\href{coursse62.html\#x75-33700010.3.7}{La fonction Γ} \\ ~~10.3.8
\href{coursse62.html\#x75-33800010.3.8}{Méthodes directes} \\ 11
\href{coursch12.html\#x76-33900011}{Séries entières} \\ ~11.1
\href{coursse63.html\#x77-34000011.1}{Convergence des séries entières}
\\ ~~11.1.1 \href{coursse63.html\#x77-34100011.1.1}{Notion de série
entière} \\ ~~11.1.2 \href{coursse63.html\#x77-34200011.1.2}{Rayon de
convergence} \\ ~~11.1.3
\href{coursse63.html\#x77-34300011.1.3}{Recherche du rayon de
convergence} \\ ~~11.1.4
\href{coursse63.html\#x77-34400011.1.4}{Opérations sur les séries
entières} \\ ~11.2 \href{coursse64.html\#x78-34500011.2}{Somme d'une
série entière} \\ ~~11.2.1 \href{coursse64.html\#x78-34600011.2.1}{Etude
sur le disque ouvert de convergence (domaine complexe)} \\ ~~11.2.2
\href{coursse64.html\#x78-34700011.2.2}{Etude sur le disque ouvert de
convergence (domaine réel)} \\ ~~11.2.3
\href{coursse64.html\#x78-34800011.2.3}{Etude sur le cercle de
convergence} \\ ~11.3
\href{coursse65.html\#x79-34900011.3}{Développements en séries entières}
\\ ~~11.3.1 \href{coursse65.html\#x79-35000011.3.1}{Problème local,
problème global} \\ ~~11.3.2
\href{coursse65.html\#x79-35100011.3.2}{Méthodes de développement} \\
~~11.3.3 \href{coursse65.html\#x79-35200011.3.3}{Fonction exponentielle.
Fonctions trigonométriques} \\ ~~11.3.4
\href{coursse65.html\#x79-35300011.3.4}{Nombres complexes de module 1}
\\ ~~11.3.5 \href{coursse65.html\#x79-35400011.3.5}{Fonctions
classiques} \\ ~~11.3.6 \href{coursse65.html\#x79-35500011.3.6}{Méthodes
de sommation} \\ ~11.4 \href{coursse66.html\#x80-35800011.4}{Application
aux endomorphismes continus et aux matrices} \\ ~~11.4.1
\href{coursse66.html\#x80-35900011.4.1}{Calcul fonctionnel et premières
applications} \\ ~~11.4.2
\href{coursse66.html\#x80-36000011.4.2}{Exponentielle d'un endomorphisme
ou d'une matrice} \\ ~~11.4.3
\href{coursse66.html\#x80-36100011.4.3}{Application aux systèmes
différentiels homogènes à coefficients constants} \\ 12
\href{coursch13.html\#x81-36200012}{Formes quadratiques} \\ ~12.1
\href{coursse67.html\#x82-36300012.1}{Formes bilinéaires} \\ ~~12.1.1
\href{coursse67.html\#x82-36400012.1.1}{Généralités} \\ ~~12.1.2
\href{coursse67.html\#x82-36500012.1.2}{Formes bilinéaires symétriques,
antisymétriques} \\ ~~12.1.3
\href{coursse67.html\#x82-36600012.1.3}{Matrice d'une forme bilinéaire}
\\ ~~12.1.4 \href{coursse67.html\#x82-36700012.1.4}{Changements de
bases, discriminant} \\ ~~12.1.5
\href{coursse67.html\#x82-36800012.1.5}{Orthogonalité} \\ ~~12.1.6
\href{coursse67.html\#x82-36900012.1.6}{Formes non dégénérées} \\
~~12.1.7 \href{coursse67.html\#x82-37000012.1.7}{Isotropie} \\ ~12.2
\href{coursse68.html\#x83-37100012.2}{Formes quadratiques} \\ ~~12.2.1
\href{coursse68.html\#x83-37200012.2.1}{Notion de forme quadratique} \\
~~12.2.2 \href{coursse68.html\#x83-37300012.2.2}{Formes quadratiques en
dimension finie} \\ ~~12.2.3
\href{coursse68.html\#x83-37400012.2.3}{Matrices et déterminants de
Gram} \\ ~12.3 \href{coursse69.html\#x84-37500012.3}{Réduction des
formes quadratiques en dimension finie} \\ ~~12.3.1
\href{coursse69.html\#x84-37600012.3.1}{Familles et bases orthogonales}
\\ ~~12.3.2 \href{coursse69.html\#x84-37700012.3.2}{Décomposition en
carrés. Algorithme de Gauss} \\ ~12.4
\href{coursse70.html\#x85-37800012.4}{Formes quadratiques réelles} \\
~~12.4.1 \href{coursse70.html\#x85-37900012.4.1}{Formes positives,
négatives} \\ ~~12.4.2 \href{coursse70.html\#x85-38000012.4.2}{Bases de
Sylvester. Signature} \\ ~~12.4.3
\href{coursse70.html\#x85-38100012.4.3}{Inégalités} \\ ~~12.4.4
\href{coursse70.html\#x85-38200012.4.4}{Espaces préhilbertiens réels} \\
~~12.4.5 \href{coursse70.html\#x85-38300012.4.5}{Espaces euclidiens} \\
~~12.4.6 \href{coursse70.html\#x85-38400012.4.6}{Algorithme de
Gram-Schmidt} \\ ~~12.4.7
\href{coursse70.html\#x85-38500012.4.7}{Application: polynômes
orthogonaux} \\ ~12.5
\href{coursse71.html\#x86-38600012.5}{Endomorphismes et formes
quadratiques} \\ ~~12.5.1 \href{coursse71.html\#x86-38700012.5.1}{Notion
d'adjoint} \\ ~~12.5.2 \href{coursse71.html\#x86-38800012.5.2}{Adjoint
en dimension finie} \\ ~~12.5.3
\href{coursse71.html\#x86-38900012.5.3}{Endomorphismes symétriques et
formes quadratiques} \\ ~~12.5.4
\href{coursse71.html\#x86-39000012.5.4}{Groupe orthogonal} \\ ~~12.5.5
\href{coursse71.html\#x86-39100012.5.5}{Matrices orthogonales} \\ ~12.6
\href{coursse72.html\#x87-39200012.6}{Endomorphismes d'un espace
euclidien} \\ ~~12.6.1 \href{coursse72.html\#x87-39300012.6.1}{Droites
et plans stables} \\ ~~12.6.2
\href{coursse72.html\#x87-39400012.6.2}{Réduction des endomorphismes
symétriques} \\ ~~12.6.3 \href{coursse72.html\#x87-39500012.6.3}{Normes
d'endomorphismes} \\ ~~12.6.4
\href{coursse72.html\#x87-39600012.6.4}{Endomorphismes orthogonaux d'un
plan euclidien} \\ ~~12.6.5
\href{coursse72.html\#x87-39700012.6.5}{Réduction des endomorphismes
orthogonaux} \\ ~~12.6.6 \href{coursse72.html\#x87-39800012.6.6}{Produit
vectoriel, produit mixte} \\ ~~12.6.7
\href{coursse72.html\#x87-39900012.6.7}{Angles} \\ 13
\href{coursch14.html\#x88-40000013}{Formes hermitiennes} \\ ~13.1
\href{coursse73.html\#x89-40100013.1}{Compléments sur la conjugaison} \\
~~13.1.1 \href{coursse73.html\#x89-40200013.1.1}{Applications
semi-linéaires} \\ ~~13.1.2
\href{coursse73.html\#x89-40300013.1.2}{Matrices conjuguées et
transconjuguées} \\ ~~13.1.3
\href{coursse73.html\#x89-40400013.1.3}{Matrices hermitiennes,
antihermitiennes} \\ ~13.2 \href{coursse74.html\#x90-40500013.2}{Formes
sesquilinéaires} \\ ~~13.2.1
\href{coursse74.html\#x90-40600013.2.1}{Généralités} \\ ~~13.2.2
\href{coursse74.html\#x90-40700013.2.2}{Formes sesquilinéaires
hermitiennes, antihermitiennes} \\ ~~13.2.3
\href{coursse74.html\#x90-40800013.2.3}{Matrice d'une forme
sesquilinéaire} \\ ~~13.2.4
\href{coursse74.html\#x90-40900013.2.4}{Changements de bases} \\
~~13.2.5 \href{coursse74.html\#x90-41000013.2.5}{Orthogonalité} \\
~~13.2.6 \href{coursse74.html\#x90-41100013.2.6}{Formes non dégénérées}
\\ ~13.3 \href{coursse75.html\#x91-41200013.3}{Formes quadratiques
hermitiennes} \\ ~~13.3.1 \href{coursse75.html\#x91-41300013.3.1}{Notion
de forme quadratique hermitienne} \\ ~~13.3.2
\href{coursse75.html\#x91-41400013.3.2}{Formes quadratiques hermitiennes
en dimension finie} \\ ~~13.3.3
\href{coursse75.html\#x91-41500013.3.3}{Formes quadratiques hermitiennes
définies positives} \\ ~~13.3.4
\href{coursse75.html\#x91-41600013.3.4}{Espaces hermitiens} \\ ~13.4
\href{coursse76.html\#x92-41700013.4}{Endomorphismes d'un espace
hermitien} \\ ~~13.4.1 \href{coursse76.html\#x92-41800013.4.1}{Notion
d'adjoint} \\ ~~13.4.2
\href{coursse76.html\#x92-41900013.4.2}{Endomorphismes hermitiens} \\
~~13.4.3 \href{coursse76.html\#x92-42000013.4.3}{Groupe unitaire} \\
~~13.4.4 \href{coursse76.html\#x92-42100013.4.4}{Matrices unitaires} \\
~~13.4.5 \href{coursse76.html\#x92-42200013.4.5}{Réduction des
endomorphismes normaux} \\ ~~13.4.6
\href{coursse76.html\#x92-42300013.4.6}{Réduction des matrices normales}
\\ 14 \href{coursch15.html\#x93-42400014}{Séries de Fourier} \\ ~14.1
\href{coursse77.html\#x94-42500014.1}{Introduction: transformée de
Fourier sur les groupes abéliens finis} \\ ~~14.1.1
\href{coursse77.html\#x94-42600014.1.1}{Caractères des groupes abéliens
finis} \\ ~~14.1.2 \href{coursse77.html\#x94-42700014.1.2}{Transformée
de Fourier sur un groupe abélien fini} \\ ~14.2
\href{coursse78.html\#x95-42800014.2}{Séries trigonométriques} \\
~~14.2.1 \href{coursse78.html\#x95-42900014.2.1}{Rappels d'intégration}
\\ ~~14.2.2 \href{coursse78.html\#x95-43000014.2.2}{Généralités} \\
~~14.2.3 \href{coursse78.html\#x95-43100014.2.3}{Un cas de convergence
normale} \\ ~14.3 \href{coursse79.html\#x96-43200014.3}{Série de Fourier
d'une fonction} \\ ~~14.3.1 \href{coursse79.html\#x96-43300014.3.1}{Les
espaces C et D} \\ ~~14.3.2
\href{coursse79.html\#x96-43400014.3.2}{Coefficients de Fourier d'une
fonction continue par morceaux} \\ ~~14.3.3
\href{coursse79.html\#x96-43500014.3.3}{Inégalité de Bessel et théorème
de Riemann-Lebesgue} \\ ~~14.3.4
\href{coursse79.html\#x96-43600014.3.4}{Les théorèmes de Dirichlet} \\
~~14.3.5 \href{coursse79.html\#x96-43700014.3.5}{Coefficients de Fourier
des fonctions de classe \{C\}\^{}\{k\}} \\ ~~14.3.6
\href{coursse79.html\#x96-43800014.3.6}{Le théorème de Parseval} \\
~14.4 \href{coursse80.html\#x97-43900014.4}{Fonctions périodiques de
période T} \\ ~14.5 \href{coursse81.html\#x98-44000014.5}{Produit de
convolution} \\ ~~14.5.1
\href{coursse81.html\#x98-44100014.5.1}{Convolution de fonctions
périodiques} \\ ~~14.5.2 \href{coursse81.html\#x98-44200014.5.2}{Produit
de convolution et séries de Fourier} \\ 15
\href{coursch16.html\#x99-44300015}{Calcul différentiel} \\ ~15.1
\href{coursse82.html\#x100-44400015.1}{Dérivées partielles} \\ ~~15.1.1
\href{coursse82.html\#x100-44500015.1.1}{Notion de dérivée partielle} \\
~~15.1.2 \href{coursse82.html\#x100-44600015.1.2}{Composition des
dérivées partielles} \\ ~~15.1.3
\href{coursse82.html\#x100-44700015.1.3}{Théorème des accroissements
finis et applications} \\ ~~15.1.4
\href{coursse82.html\#x100-44800015.1.4}{Dérivées partielles
successives} \\ ~~15.1.5
\href{coursse82.html\#x100-44900015.1.5}{Formules de Taylor} \\ ~~15.1.6
\href{coursse82.html\#x100-45000015.1.6}{Application aux extremums de
fonctions de plusieurs variables} \\ ~15.2
\href{coursse83.html\#x101-45100015.2}{Différentielle} \\ ~~15.2.1
\href{coursse83.html\#x101-45200015.2.1}{Applications différentiables}
\\ ~~15.2.2 \href{coursse83.html\#x101-45300015.2.2}{Exemples
d'applications différentiables} \\ ~~15.2.3
\href{coursse83.html\#x101-45400015.2.3}{Opérations sur les
différentielles} \\ ~~15.2.4
\href{coursse83.html\#x101-45500015.2.4}{Différentielle et dérivées
partielles} \\ ~~15.2.5
\href{coursse83.html\#x101-45600015.2.5}{Matrices jacobiennes,
jacobiens} \\ ~~15.2.6
\href{coursse83.html\#x101-45700015.2.6}{Inégalité des accroissements
finis} \\ ~15.3 \href{coursse84.html\#x102-45800015.3}{Formes
différentielles} \\ ~~15.3.1
\href{coursse84.html\#x102-45900015.3.1}{Rappels sur les formes
linéaires alternées} \\ ~~15.3.2
\href{coursse84.html\#x102-46000015.3.2}{Notion de forme différentielle}
\\ ~~15.3.3 \href{coursse84.html\#x102-46100015.3.3}{Notion de gradient
d'une fonction} \\ ~~15.3.4
\href{coursse84.html\#x102-46200015.3.4}{Invariance de la
différentielle} \\ ~~15.3.5
\href{coursse84.html\#x102-46300015.3.5}{Différentielle extérieure} \\
~~15.3.6 \href{coursse84.html\#x102-46400015.3.6}{Théorème de Poincaré}
\\ ~15.4 \href{coursse85.html\#x103-46500015.4}{Fonctions implicites et
inversion locale} \\ ~~15.4.1
\href{coursse85.html\#x103-46600015.4.1}{Position du problème des
fonctions implicites} \\ ~~15.4.2
\href{coursse85.html\#x103-46700015.4.2}{Théorème des fonctions
implicites} \\ ~~15.4.3
\href{coursse85.html\#x103-46800015.4.3}{Applications du théorème des
fonctions implicites} \\ ~~15.4.4
\href{coursse85.html\#x103-46900015.4.4}{Difféomorphismes et inversion
locale} \\ 16 \href{coursch17.html\#x104-47000016}{Equations
différentielles} \\ ~16.1 \href{coursse86.html\#x105-47100016.1}{Notions
générales} \\ ~~16.1.1
\href{coursse86.html\#x105-47200016.1.1}{Solutions d'une équation
différentielle} \\ ~~16.1.2
\href{coursse86.html\#x105-47300016.1.2}{Type de problèmes} \\ ~~16.1.3
\href{coursse86.html\#x105-47400016.1.3}{Réduction à l'ordre 1} \\
~~16.1.4 \href{coursse86.html\#x105-47500016.1.4}{Equivalence avec une
équation intégrale} \\ ~~16.1.5
\href{coursse86.html\#x105-47600016.1.5}{Le lemme de Gronwall} \\ ~16.2
\href{coursse87.html\#x106-47700016.2}{Théorie de Cauchy-Lipschitz} \\
~~16.2.1 \href{coursse87.html\#x106-47800016.2.1}{Unicité de solutions,
solutions maximales} \\ ~16.3
\href{coursse88.html\#x107-47900016.3}{Equations différentielles
linéaires d'ordre 1} \\ ~~16.3.1
\href{coursse88.html\#x107-48000016.3.1}{Généralités} \\ ~~16.3.2
\href{coursse88.html\#x107-48100016.3.2}{Equation différentielle
linéaire scalaire d'ordre 1} \\ ~~16.3.3
\href{coursse88.html\#x107-48200016.3.3}{Théorie de Cauchy-Lipschitz
pour les équations linéaires} \\ ~~16.3.4
\href{coursse88.html\#x107-48300016.3.4}{Structure des solutions de
l'équation homogène} \\ ~~16.3.5
\href{coursse88.html\#x107-48400016.3.5}{Méthode de variation des
constantes} \\ ~~16.3.6
\href{coursse88.html\#x107-48500016.3.6}{Systèmes différentiels à
coefficients constants} \\ ~16.4
\href{coursse89.html\#x108-48600016.4}{Equation différentielle linéaire
d'ordre n} \\ ~~16.4.1
\href{coursse89.html\#x108-48700016.4.1}{Généralités} \\ ~~16.4.2
\href{coursse89.html\#x108-48800016.4.2}{Théorie de Cauchy-Lipschitz} \\
~~16.4.3 \href{coursse89.html\#x108-48900016.4.3}{Structure des
solutions de l'équation homogène. Wronskien} \\ ~~16.4.4
\href{coursse89.html\#x108-49000016.4.4}{Méthode de variation des
constantes} \\ ~~16.4.5 \href{coursse89.html\#x108-49100016.4.5}{Méthode
d'abaissement du degré} \\ ~~16.4.6
\href{coursse89.html\#x108-49200016.4.6}{Equation homogène à
coefficients constants} \\ ~~16.4.7
\href{coursse89.html\#x108-49300016.4.7}{Equation linéaire à
coefficients constants} \\ ~~16.4.8
\href{coursse89.html\#x108-49400016.4.8}{Equations d'Euler} \\ ~16.5
\href{coursse90.html\#x109-49500016.5}{Equations différentielles non
linéaires} \\ ~~16.5.1 \href{coursse90.html\#x109-49600016.5.1}{Théorie
de Cauchy-Lipschitz} \\ ~~16.5.2
\href{coursse90.html\#x109-49700016.5.2}{Application aux équations
d'ordre n} \\ ~~16.5.3 \href{coursse90.html\#x109-49800016.5.3}{Systèmes
différentiels autonomes d'ordre 1} \\ ~~16.5.4
\href{coursse90.html\#x109-49900016.5.4}{Equations différentielles et
formes différentielles} \\ ~~16.5.5
\href{coursse90.html\#x109-50000016.5.5}{Equations aux différentielles
totales} \\ ~~16.5.6 \href{coursse90.html\#x109-50100016.5.6}{Equations
à variables séparables} \\ ~~16.5.7
\href{coursse90.html\#x109-50200016.5.7}{Equations se ramenant à des
équations à variables séparables} \\ ~~16.5.8
\href{coursse90.html\#x109-50300016.5.8}{Equation de Riccati} \\ ~16.6
\href{coursse91.html\#x110-50400016.6}{Analyse numérique des équations
différentielles} \\ ~~16.6.1
\href{coursse91.html\#x110-50500016.6.1}{Méthode d'Euler} \\ ~~16.6.2
\href{coursse91.html\#x110-50600016.6.2}{Méthode de Runge et Kutta} \\
~~16.6.3 \href{coursse91.html\#x110-50700016.6.3}{Equations
différentielles d'ordre supérieur} \\ 17
\href{coursch18.html\#x111-50800017}{Espaces affines} \\ ~17.1
\href{coursse92.html\#x112-50900017.1}{Généralités sur les espaces
affines} \\ ~~17.1.1 \href{coursse92.html\#x112-51000017.1.1}{Notion
d'espace affine} \\ ~~17.1.2
\href{coursse92.html\#x112-51100017.1.2}{Repères affines, bases affines}
\\ ~~17.1.3 \href{coursse92.html\#x112-51200017.1.3}{Sous-espace affine}
\\ ~~17.1.4 \href{coursse92.html\#x112-51300017.1.4}{Parallélisme,
intersection} \\ ~~17.1.5
\href{coursse92.html\#x112-51400017.1.5}{Applications affines} \\
~~17.1.6 \href{coursse92.html\#x112-51500017.1.6}{Utilisation de repères
affines} \\ ~~17.1.7 \href{coursse92.html\#x112-51600017.1.7}{Formes
affines et sous-espaces affines} \\ ~17.2
\href{coursse93.html\#x113-51700017.2}{Barycentres} \\ ~~17.2.1
\href{coursse93.html\#x113-51800017.2.1}{Notion de barycentres} \\
~~17.2.2 \href{coursse93.html\#x113-51900017.2.2}{Associativité des
barycentres} \\ ~~17.2.3
\href{coursse93.html\#x113-52000017.2.3}{Barycentres, sous-espaces
affines, applications affines} \\ ~~17.2.4
\href{coursse93.html\#x113-52100017.2.4}{Barycentres et convexité} \\
~17.3 \href{coursse94.html\#x114-52200017.3}{Espaces affines euclidiens}
\\ ~~17.3.1 \href{coursse94.html\#x114-52300017.3.1}{Notion d'espace
affine euclidien} \\ ~~17.3.2
\href{coursse94.html\#x114-52400017.3.2}{Formule de Leibnitz et
applications} \\ ~~17.3.3
\href{coursse94.html\#x114-52500017.3.3}{Isométries affines} \\ ~~17.3.4
\href{coursse94.html\#x114-52600017.3.4}{Forme réduite d'une isométrie
affine} \\ ~~17.3.5 \href{coursse94.html\#x114-52700017.3.5}{Distance à
un sous-espace affine} \\ ~~17.3.6
\href{coursse94.html\#x114-52800017.3.6}{Distance de deux sous-espaces
affines} \\ ~17.4 \href{coursse95.html\#x115-52900017.4}{Cercles,
sphères, triangle} \\ ~~17.4.1
\href{coursse95.html\#x115-53000017.4.1}{Généralités sur les sphères} \\
~~17.4.2 \href{coursse95.html\#x115-53100017.4.2}{Cercles et angles} \\
~~17.4.3 \href{coursse95.html\#x115-53200017.4.3}{Eléments de géométrie
du triangle} \\ 18 \href{coursch19.html\#x116-53300018}{Courbes} \\
~18.1 \href{coursse96.html\#x117-53400018.1}{Arcs paramétrés} \\
~~18.1.1 \href{coursse96.html\#x117-53500018.1.1}{Vocabulaire} \\
~~18.1.2 \href{coursse96.html\#x117-53600018.1.2}{Equivalence des arcs
paramétrés} \\ ~~18.1.3
\href{coursse96.html\#x117-53700018.1.3}{Orientation} \\ ~~18.1.4
\href{coursse96.html\#x117-53800018.1.4}{Tangente à un arc paramétré} \\
~~18.1.5 \href{coursse96.html\#x117-53900018.1.5}{Plan osculateur,
concavité} \\ ~~18.1.6 \href{coursse96.html\#x117-54000018.1.6}{Etude
locale des arcs plans} \\ ~~18.1.7
\href{coursse96.html\#x117-54100018.1.7}{Branches infinies} \\ ~~18.1.8
\href{coursse96.html\#x117-54200018.1.8}{Plan d'étude d'un arc plan en
paramétriques} \\ ~~18.1.9
\href{coursse96.html\#x117-54300018.1.9}{Notion de contact} \\ ~~18.1.10
\href{coursse96.html\#x117-54400018.1.10}{Enveloppes} \\ ~18.2
\href{coursse97.html\#x118-54500018.2}{Arcs en polaires} \\ ~~18.2.1
\href{coursse97.html\#x118-54600018.2.1}{Coordonnées polaires} \\
~~18.2.2 \href{coursse97.html\#x118-54700018.2.2}{Arcs en coordonnées
polaires: étude locale} \\ ~~18.2.3
\href{coursse97.html\#x118-54800018.2.3}{Branches infinies et phénomènes
asymptotiques} \\ ~~18.2.4 \href{coursse97.html\#x118-54900018.2.4}{Plan
d'étude d'un arc plan en polaires} \\ ~~18.2.5
\href{coursse97.html\#x118-55000018.2.5}{Equations polaires
remarquables} \\ ~18.3 \href{coursse98.html\#x119-55100018.3}{Problèmes
classiques sur les courbes} \\ ~~18.3.1
\href{coursse98.html\#x119-55200018.3.1}{Trajectoires orthogonales} \\
~~18.3.2 \href{coursse98.html\#x119-55300018.3.2}{Inverse d'une courbe}
\\ ~~18.3.3 \href{coursse98.html\#x119-55400018.3.3}{Podaire d'une
courbe} \\ ~~18.3.4 \href{coursse98.html\#x119-55500018.3.4}{Conchoïdes
d'une courbe} \\ ~18.4 \href{coursse99.html\#x120-55600018.4}{Etude
métrique des arcs} \\ ~~18.4.1
\href{coursse99.html\#x120-55700018.4.1}{Arcs rectifiables} \\ ~~18.4.2
\href{coursse99.html\#x120-55800018.4.2}{Arcs de classe \{C\}\^{}\{1\}}
\\ ~~18.4.3 \href{coursse99.html\#x120-55900018.4.3}{Abscisses
curvilignes} \\ ~~18.4.4
\href{coursse99.html\#x120-56000018.4.4}{Introduction à la méthode du
repère mobile} \\ ~~18.4.5
\href{coursse99.html\#x120-56100018.4.5}{Repère de Frénet et courbure
des arcs d'un plan euclidien orienté} \\ ~~18.4.6
\href{coursse99.html\#x120-56200018.4.6}{Centre de courbure, cercle
osculateur} \\ ~~18.4.7
\href{coursse99.html\#x120-56300018.4.7}{Développée, développantes} \\
~~18.4.8 \href{coursse99.html\#x120-56400018.4.8}{Equations
intrinsèques} \\ ~~18.4.9
\href{coursse99.html\#x120-56500018.4.9}{Courbure des arcs gauches} \\
19 \href{coursch20.html\#x121-56600019}{Surfaces} \\ ~19.1
\href{coursse100.html\#x122-56700019.1}{Nappes paramétrées} \\ ~~19.1.1
\href{coursse100.html\#x122-56800019.1.1}{Notion de nappe paramétrée.
Equivalence} \\ ~~19.1.2
\href{coursse100.html\#x122-56900019.1.2}{Orientation} \\ ~~19.1.3
\href{coursse100.html\#x122-57000019.1.3}{Plan tangent à une nappe
paramétrée, vecteur normal} \\ ~~19.1.4
\href{coursse100.html\#x122-57100019.1.4}{Points réguliers et nappes
cartésiennes} \\ ~~19.1.5
\href{coursse100.html\#x122-57200019.1.5}{Intersection de nappes
paramétrées} \\ ~~19.1.6
\href{coursse100.html\#x122-57300019.1.6}{Intersection d'une nappe et de
son plan tangent} \\ ~19.2
\href{coursse101.html\#x123-57400019.2}{Nappes réglées} \\ ~~19.2.1
\href{coursse101.html\#x123-57500019.2.1}{Notion de nappe réglée} \\
~~19.2.2 \href{coursse101.html\#x123-57600019.2.2}{Plan tangent à une
nappe réglée} \\ ~~19.2.3
\href{coursse101.html\#x123-57700019.2.3}{Nappes cylindriques. Nappes
coniques} \\ ~19.3 \href{coursse102.html\#x124-57800019.3}{Equations de
surfaces} \\ ~~19.3.1 \href{coursse102.html\#x124-57900019.3.1}{Surfaces
cartésiennes et nappes paramétrées} \\ ~~19.3.2
\href{coursse102.html\#x124-58000019.3.2}{Cylindres} \\ ~~19.3.3
\href{coursse102.html\#x124-58100019.3.3}{Cônes} \\ ~~19.3.4
\href{coursse102.html\#x124-58200019.3.4}{Surfaces de révolution} \\
~19.4 \href{coursse103.html\#x125-58300019.4}{Quadriques} \\ ~~19.4.1
\href{coursse103.html\#x125-58400019.4.1}{Notion de quadrique} \\
~~19.4.2 \href{coursse103.html\#x125-58500019.4.2}{Réduction des
quadriques} \\ ~~19.4.3
\href{coursse103.html\#x125-58600019.4.3}{Classification des quadriques
en dimension 2 et 3} \\ ~~19.4.4
\href{coursse103.html\#x125-58700019.4.4}{Quadriques réglées, quadriques
de révolution} \\ 20 \href{coursch21.html\#x126-58800020}{Intégrales
curvilignes, intégrales multiples} \\ ~20.1
\href{coursse104.html\#x127-58900020.1}{Intégrales curvilignes} \\
~~20.1.1 \href{coursse104.html\#x127-59000020.1.1}{Formes
différentielles sur un arc paramétré} \\ ~~20.1.2
\href{coursse104.html\#x127-59100020.1.2}{Intégrale d'une forme
différentielle sur un arc} \\ ~~20.1.3
\href{coursse104.html\#x127-59200020.1.3}{Formes différentielles exactes
et champs de gradients} \\ ~20.2
\href{coursse105.html\#x128-59300020.2}{Intégrales multiples} \\
~~20.2.1 \href{coursse105.html\#x128-59400020.2.1}{Pavés et
subdivisions. Ensembles négligeables} \\ ~~20.2.2
\href{coursse105.html\#x128-59500020.2.2}{Intégrales multiples sur un
pavé de \{ℝ\}\^{}\{n\}} \\ ~~20.2.3
\href{coursse105.html\#x128-59600020.2.3}{Intégrales multiples sur une
partie de \{ℝ\}\^{}\{n\}} \\ ~~20.2.4
\href{coursse105.html\#x128-59700020.2.4}{Mesure d'un sous-ensemble
borné de \{ℝ\}\^{}\{n\}} \\ ~20.3
\href{coursse106.html\#x129-59800020.3}{Calcul des intégrales doubles et
triples} \\ ~~20.3.1 \href{coursse106.html\#x129-59900020.3.1}{Théorème
de Fubini sur une partie de \{ℝ\}\^{}\{2\}} \\ ~~20.3.2
\href{coursse106.html\#x129-60000020.3.2}{Théorème de Fubini sur une
partie de \{ℝ\}\^{}\{3\}} \\ ~~20.3.3
\href{coursse106.html\#x129-60100020.3.3}{Théorème de changement de
variables dans les intégrales multiples} \\ ~~20.3.4
\href{coursse106.html\#x129-60200020.3.4}{Théorème de Green-Riemann} \\
~20.4 \href{coursse107.html\#x130-60300020.4}{Introduction aux
intégrales de surface}

{[}\href{coursch2.html}{next}{]} {[}\href{coursch1.html}{prev}{]}
{[}\href{coursch1.html\#tailcoursch1.html}{prev-tail}{]}
{[}\href{coursli1.html}{front}{]}
{[}\href{cours.html\#coursli1.html}{up}{]}

\end{document}
