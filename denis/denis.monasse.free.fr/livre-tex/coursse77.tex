\section{Dérivées partielles}

\subsection{Notion de dérivée partielle}

\begin{de}
Soit $E$ et $F$ deux espaces vectoriels normés. Soit $U$ un ouvert de $E$, $f:U\to F$, $a\in U$. Soit $v\in E\setminus{0}$. On dit que $f$ admet au point $a$ une dérivée partielle suivant le vecteur $v$ si l’application $t\mapsto f(a+tv)$ (définie sur un voisinage de 0) est dérivable au point 0.
\end{de}

\begin{rem}
L’existence de la dérivée partielle en $a$ suivant le vecteur $v$ est donc équivalente à l’existence de $\lim_{t\to 0}\frac{f(a+tv)-f(a)}{t}=\partial_v f(a)$. Remarquons que si $v'=\lambda v$, $\lambda\neq 0$, alors $\frac{f(a+tv')-f(a)}{t}=\lambda\frac{f(a+uv)-f(a)}{u}$ avec $u=\lambda t$ ce qui montre que $f$ admet en $a$ une dérivée partielle selon $v$ si et seulement si $f$ admet une dérivée partielle suivant $\lambda v$ et qu’alors $\partial_{\lambda v} f(a)=\lambda \partial_v f(a)$.
\end{rem}

\begin{rem}
L’existence de dérivée partielle suivant tout vecteur n’implique donc pas la continuité.
\end{rem}

\begin{prop}
On a les propriétés évidentes de la dérivation de $t\mapsto f(a+tv)$ à savoir (i) si $f$ et $g$ admettent en $a$ une dérivée partielle suivant le vecteur $v$, il en est de même de $\alpha f+\beta g$ et $\partial_v(\alpha f+\beta g)(a)=\alpha \partial_v f(a)+\beta \partial_v g(a)$. (ii) si $f$ et $g$ (à valeurs scalaires) admettent en $a$ une dérivée partielle suivant le vecteur $v$, il en est de même de $fg$ et $\partial_v(fg)(a)=g(a)\partial_v f(a)+f(a)\partial_v g(a)$.
\end{prop}

\begin{rem}
Par contre, on n’a pas de théorème général de composition des dérivées partielles.
\end{rem}

\begin{de}
Soit $E$ un espace vectoriel normé de dimension finie, $\mathcal{E}=(e_1,\ldots,e_n)$ une base de $E$, $F$ un espace vectoriel normé. Soit $U$ un ouvert de $E$, $f:U\to F$, $a\in U$. On dit que $f$ admet au point $a$ une dérivée partielle d’indice $i$ (suivant la base $\mathcal{E}$) si elle admet une dérivée partielle suivant le vecteur $e_i$. On note alors $\frac{\partial f}{\partial x_i}(a)=\partial_{e_i} f(a)$.
\end{de}

\begin{rem}
On retrouve bien la notion habituelle de dérivée partielle : dérivée suivant la variable $x_i$, toutes les autres étant considérées comme constantes.
\end{rem}

\subsection{Composition des dérivées partielles}

\begin{de}
Soit $U$ un ouvert de $\mathbb{R}^n$, $f:U\to F$. On dit que $f$ est de classe $C^1$ au point $a$ si, sur un certain voisinage $V$ de $a$, $f$ admet des dérivées partielles de tout indice $i\in[1,n]$ et si ces dérivées partielles $x\mapsto \frac{\partial f}{\partial x_i}(x)$ sont continues au point $a$.
\end{de}

\begin{lem}
Soit $F$ un espace vectoriel de dimension finie, $V$ un ouvert de $\mathbb{R}^n$, $f:V\to F$. Soit $I$ un intervalle de $\mathbb{R}$, $t\in I$ et $\varphi=(\varphi_1,\ldots,\varphi_n):I\to V$. On suppose que $\varphi$ est dérivable au point $t$ et que $f$ est de classe $C^1$ au point $\varphi(t)$. Alors $f\circ \varphi$ est dérivable au point $t$ et $(f\circ \varphi)'(t)=\sum_{i=1}^n \frac{\partial f}{\partial x_i}(\varphi(t))\varphi'_i(t)$.
\end{lem}

\begin{thm}
Soit $F$ un espace vectoriel de dimension finie, $U$ un ouvert de $\mathbb{R}^n$, $f:U\to F$. Soit $E$ un espace vectoriel normé, $V$ un ouvert de $E$ et $g=(g_1,\ldots,g_n):V\to U\subset\mathbb{R}^n$. Soit $a\in V$ et $v\in E\setminus{0}$. Si $g$ admet en $a$ une dérivée partielle suivant le vecteur $v$ et si $f$ est de classe $C^1$ au point $a$, alors $f\circ g$ admet en $a$ une dérivée partielle suivant le vecteur $v$ et on a
[\partial_v(f\circ g)(a)=\sum_{i=1}^n \frac{\partial f}{\partial x_i}(g(a))\partial_v g_i(a).]
\end{thm}

\begin{rem}
On en déduit immédiatement que la composée de deux applications de classe $C^1$ est encore de classe $C^1$.
\end{rem}

\begin{rem}
On voit que dans ce cas $v\mapsto \partial_v f(a)$ est linéaire. Cette remarque nous conduira à la définition de la différentielle dans la section suivante.
\end{rem}

\subsection{Théorème des accroissements finis et applications}

\begin{thm}
Soit $U$ un ouvert de $\mathbb{R}^n$, $f:U\to\mathbb{R}$ de classe $C^1$. Soit $a\in U$ et $h\in\mathbb{R}^n$ tel que $[a,a+h]\subset U$. Alors, il existe $\theta\in]0,1[$ tel que
[f(a+h)-f(a)=\sum_{i=1}^n h_i \frac{\partial f}{\partial x_i}(a+\theta h).]
\end{thm}

\begin{rem}
On utilisera le plus souvent cette formule pour $k=1$; dans cas d’une fonction définie sur un ouvert de $\mathbb{R}^2$ on obtiendra par exemple
[f(a+h)=f(a)+h_1\frac{\partial f}{\partial x_1}(a)+h_2\frac{\partial f}{\partial x_2}(a)+h_1^2\int_0^1(1-t)\frac{\partial^2 f}{\partial x_1^2}(a+th)dt+h_2^2\int_0^1(1-t)\frac{\partial^2 f}{\partial x_2^2}(a+th)dt+2h_1h_2\int_0^1(1-t)\frac{\partial^2 f}{\partial x_1 \partial x_2}(a+th)dt.]
\end{rem}

\begin{thm}
(formule de Taylor avec reste intégral). Soit $U$ un ouvert de $\mathbb{R}^n$ et $f:U\to E$ de classe $C^{k+1}$. Soit $a\in U$ et $h\in\mathbb{R}^n$ tel que $[a,a+h]\subset U$. Alors
[f(a+h)+\int_0^1(1-t)^k\frac{k!}{(h_1\frac{\partial}{\partial x_1}+\ldots+h_n\frac{\partial}{\partial x_n})(k+1)\ast f(a+th)}dt=f(a)+\sum_{p=1}^k \frac{1}{p!}(h_1\frac{\partial}{\partial x_1}+\ldots+h_n\frac{\partial}{\partial x_n})^p\ast f(a).]
\end{thm}

\begin{thm}
(formule de Taylor-Lagrange). Soit $U$ un ouvert de $\mathbb{R}^n$ et $f:U\to\mathbb{R}$ de classe $C^{k+1}$. Soit $a\in U$ et $h\in\mathbb{R}^n$ tel que $[a,a+h]\subset U$. Alors, il existe $\theta\in]0,1[$ tel que
[f(a+h)=f(a)+\sum_{p=1}^k \frac{1}{p!}(h_1\frac{\partial}{\partial x_1}+\ldots+h_n\frac{\partial}{\partial x_n})^p\ast f(a)+\frac{1}{(k+1)!}(h_1\frac{\partial}{\partial x_1}+\ldots+h_n\frac{\partial}{\partial x_n})(k+1)\ast f(a+\theta h).]
\end{thm}

\begin{thm}
(formule de Taylor-Young). Soit $U$ un ouvert de $\mathbb{R}^n$ et $f:U\to E$ (espace vectoriel normé de dimension finie) de classe $C^k$. Soit $a\in U$. Alors, quand $h$ tend vers 0 on a
[f(a+h)=f(a)+\sum_{p=1}^k \frac{1}{p!}(h_1\frac{\partial}{\partial x_1}+\ldots+h_n\frac{\partial}{\partial x_n})^p\ast f(a)+o(|h|^k).]
\end{thm}

\subsection{Application aux extremums de fonctions de plusieurs variables}

\begin{prop}
Soit $U$ un ouvert de $\mathbb{R}^n$ et $f:U\to\mathbb{R}$ de classe $C^1$. Soit $a\in U$. Si $f$ admet en $a$ un extremum local, on a $\forall i\in[1,n],\frac{\partial f}{\partial x_i}(a)=0$.
\end{prop}

\begin{de}
Soit $U$ un ouvert de $\mathbb{R}^n$ et $f:U\to\mathbb{R}$ de classe $C^2$. Soit $a\in U$. On appelle différentielle seconde au point $a$ la forme quadratique sur $\mathbb{R}^n$,
[h=(h_1,\ldots,h_n)\mapsto (h_1\frac{\partial}{\partial x_1}+\ldots+h_n\frac{\partial}{\partial x_n})^2\ast f(a)=\sum_{i=1}^n h_i^2 \frac{\partial^2 f}{\partial x_i^2}(a)+2\sum_{i<j} h_i h_j \frac{\partial^2 f}{\partial x_i \partial x_j}(a).]
\end{de}

\begin{thm}
Soit $U$ un ouvert de $\mathbb{R}^n$ et $f:U\to\mathbb{R}$ de classe $C^2$. Soit $a\in U$ tel que $\forall i\in[1,n],\frac{\partial f}{\partial x_i}(a)=0$ et soit $\Phi$ la forme quadratique différentielle seconde au point $a$. Alors (i) si $\Phi$ est définie positive, c’est-à-dire si $h\neq 0\Rightarrow \Phi(h)>0$, alors $f$ admet en $a$ un minimum local strict (ii) si $\Phi$ est définie négative, c’est-à-dire si $h\neq 0\Rightarrow \Phi(h)<0$, alors $f$ admet en $a$ un maximum local strict (iii) si $\Phi$ n’est ni positive ni négative, alors $f$ n’admet pas d’extremum en $a$ (on dit dans ce cas que $a$ est un point selle ou point col de $a$, par analogie avec une selle de cheval ou un col de montagne).
\end{thm}

\begin{rem}
Dans le cas où $\Phi$ est soit positive, soit négative, mais non définie (c’est-à-dire que $\Phi(h)$ peut être nul sans que $h$ soit nul), on ne peut pas conclure en général et il faut utiliser une formule de Taylor à un ordre supérieur.
\end{rem}

\begin{rem}
Le lecteur comparera les surfaces $z=f(x,y)$ ainsi que lignes de niveau de ces surfaces dans les trois exemples ci dessous (correspondant respectivement à un minimum local, un point selle et un point de type (iv)).
\end{rem}
