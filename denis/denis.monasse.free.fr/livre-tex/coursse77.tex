\subsubsection{14.1 Introduction : transformée de Fourier sur les groupes abéliens finis}

Ce paragraphe sert simplement d'introduction à la suite du chapitre.
Lors d'une première lecture il peut être sauté sans inconvénient.

\paragraph{14.1.1 Caractères des groupes abéliens finis}

Définition~14.1.1 Soit $(G,\cdot)$ un groupe abélien fini. On appelle
caractère de $G$ tout morphisme de groupe $\chi$ de $G$ dans $(\mathbb{C}^*, \cdot)$.
On note $\hat{G}$ l'ensemble des caractères de $G$.

Remarque~14.1.1 On vérifie immédiatement que $\hat{G}$
est lui-même muni d'une structure de groupe en posant $(\chi \chi')(x) =
\chi(x)\chi'(x)$.

Proposition~14.1.1 Soit $\chi \in \hat{G}$. Alors
$\forall x \in G, |\chi(x)| = 1$.

Démonstration Puisque $G$ est un groupe fini, tout élément est d'ordre
fini, et donc il existe $n \in \mathbb{N}$ tel que $x^n = e$. On a donc $1 =
\chi(e) = \chi(x^n) = \chi(x)^n$ ce qui montre que $\chi(x)$ est
une racine de l'unité donc de module 1.

Proposition~14.1.2 $\hat{G}$ est un groupe abélien fini.

Démonstration On sait que $\forall x \in G,
x^{|G|} = e$. Le même raisonnement que ci-dessus monte que $\chi(x)$ est une racine $|G|$-ième de
l'unité. Donc $\hat{G}$ est un sous-ensemble de
l'ensemble des applications de $G$ dans le groupe fini
$\Gamma_{|G|}$ des racines $|G|$-ièmes
de l'unité, donc il est fini.

Lemme~14.1.3 Soit $G$ un groupe abélien fini et $H$ un sous-groupe de $G$.
Soit $\psi$ un caractère de $H$. Alors il existe un caractère $\chi$ de $G$ dont la
restriction à $H$ est $\psi$.

Démonstration Pour des raisons de cardinal, il existe un sous-groupe $K$
maximal auquel $\psi$ admet un prolongement $\phi \in \hat{K}$.
Nous allons montrer par l'absurde que $K = G$, ce qui démontrera le lemme.
Supposons donc que $K \neq G$ et soit $x \in G \setminus K$.
L'ensemble des $n \in \mathbb{Z}$ tel que $x^n \in K$ est un sous-groupe de $\mathbb{Z}$,
donc de la forme $d\mathbb{Z}$ pour un $d > 0$ (car
$x^{|G|} = e \in K$) et
$d \neq 1$ (car $x \notin K$).
Soit $\omega \in \mathbb{C}$ tel que $\omega^d = \phi(x^d)$. Soit $K' = \{x^n k | n \in \mathbb{Z}, k \in K\}$ le sous-groupe de $G$ engendré par $K$ et $x$ et
prolongeons $\phi$ à $K'$ en posant $\phi'(x^n k) = \omega^n \phi(k)$.
Montrons tout d'abord que $\phi'$ est bien définie. Si l'on a $x^n k
= x^m k'$, on a $x^{n-m} = k' k^{-1} \in K$ et
donc $d$ divise $n - m$. Posons $n - m = dq$ si bien que $x^{dq} =
k' k^{-1}$ ; on a alors $\phi(k')\phi(k)^{-1} =
\phi(x^{dq}) = \phi(x^d)^q =
(\omega^d)^q = \omega^{dq} = \omega^{n-m}$, si
bien que $\omega^n \phi(k) = \omega^m \phi(k')$ ce qui montre que $\phi'$
est bien définie. Il est élémentaire de vérifier que $\phi'$ est encore un
morphisme de groupes et il est clair qu'il prolonge $\phi$ et donc qu'il
prolonge $\psi$. Mais ceci contredit alors la maximalité de $K$. On a donc $K =
G$ et donc $\psi$ se prolonge à $G$ tout entier.

Théorème~14.1.4 Soit $x \in G$, $x \neq e$. Alors il
existe $\chi \in \hat{G}$ tel que
$\chi(x) \neq 1$.

Démonstration Soit $d$ l'ordre de $x$, $\omega = \exp(\frac{2i\pi}{d})$, $H$ le sous-groupe engendré par $x$.
L'application $x^n \mapsto \omega^n$
est bien définie (car si $x^n = x^p$, alors $d$ divise
$n - p$ et donc $\omega^n = \omega^p$) et c'est un caractère de
$H$ (facile). Donc il existe $\chi \in \hat{G}$ qui prolonge $\psi$.
On a en particulier $\chi(x) = \omega \neq 1$.

Définition~14.1.2 Si $f$ est une application de $G$ dans $\mathbb{C}$, on notera
$\int_G f = \sum_{x \in G} f(x)$. En posant alors
$(f|g) = \frac{1}{|G|} \int_G \overline{f}g$, on munit l'espace $E$ des
applications de $G$ dans $\mathbb{C}$ d'une structure d'espace hermitien.

Proposition~14.1.5

\begin{itemize}
\itemsep1pt\parskip0pt\parsep0pt
\item
  (i) Soit $\chi \in \hat{G}$. Alors
  $\int_G \chi = \begin{cases} 
0 & \text{si } \chi \neq 1 \\
|G| & \text{si } \chi = 1
\end{cases}$.
\item
  (ii) Soit $x \in G$. Alors
  $\sum_{\chi \in \hat{G}} \chi(x) = \begin{cases}
0 & \text{si } x \neq e \\
|\hat{G}| & \text{si } x = e
\end{cases}$.
\end{itemize}

Démonstration (i) Supposons que $\chi \neq 1$ et soit $x
\in G$ tel que $\chi(x) \neq 1$. On a alors

$\chi(x) \sum_{g \in G} \chi(g) =
\sum_{g \in G} \chi(xg) =
\sum_{g \in G} \chi(g)$

puisque $g \mapsto xg$ est une bijection de $G$ sur lui-même. Comme $\chi(x) \neq 1$, on en déduit que
$\sum_{g \in G} \chi(g) = 0$. Si par contre, $\chi = 1$, on a
$\sum_{g \in G} \chi(g) = |G|$.

(ii) Si $x \neq e$, soit $\phi \in \hat{G}$ tel que $\phi(x) \neq 1$. On a alors

$\phi(x) \sum_{\chi \in \hat{G}} \chi(x)
= \sum_{\chi \in \hat{G}}(\phi\chi)(x) =
\sum_{\chi \in \hat{G}} \chi(x)$

puisque $\chi \mapsto \phi\chi$ est une bijection de
$\hat{G}$ sur lui-même. Comme
$\phi(x) \neq 1$, on a
$\sum_{\chi \in \hat{G}} \chi(x) = 0$. Si par contre, $x = e$, on a
pour tout $\chi$, $\chi(x) = 1$ et donc
$\sum_{\chi \in \hat{G}} \chi(x) = |\hat{G}|$.

Corollaire~14.1.6 $|G| = |\hat{G}|$.

Démonstration Soit $S = \sum_{\chi \in \hat{G}} \sum_{x \in G} \chi(x)$. On a (en utilisant le
symbole de Kronecker $\delta_a^b = \begin{cases}
1 & \text{si } a = b \\
0 & \text{si } a \neq b
\end{cases}$ et en notant $1$ le caractère constant égal à 1) 
$S = \sum_{\chi \in \hat{G}} |G| \delta_\chi^1 = |G|$. 

Mais on a aussi $S = \sum_{x \in G} \sum_{\chi \in \hat{G}} \chi(x)
= \sum_{x \in G} |\hat{G}| \delta_x^e
= |\hat{G}|$ d'où le résultat.

Corollaire~14.1.7 $\hat{G}$ est une famille orthonormée
de $E$.

Démonstration On a $(\chi | \phi) = \frac{1}{|G|} \int_G \overline{\chi}\phi = 0$ si
$\overline{\chi}\phi \neq 1$, soit
$\chi \neq \phi$ (puisque $\overline{\chi}(g) = \frac{1}{\chi(g)}$ comme nombre complexe de module 1). Si
par contre, $\chi = \phi$, on a $\overline{\chi}\phi =
|\chi|^2 = 1$ et donc
$(\chi|\phi) = 1$.

\paragraph{14.1.2 Transformée de Fourier sur un groupe abélien fini}

Définition~14.1.3 Soit $G$ un groupe abélien fini et soit $f : G \rightarrow \mathbb{C}$. On
définit la transformée de Fourier de $f$ comme étant l'application
$\hat{f} : \hat{G} \rightarrow \mathbb{C}$ définie par

$\forall \chi \in \hat{G},
\hat{f}(\chi) = (\chi|f) =
\frac{1}{|G|} \int_G f \overline{\chi}$

Théorème~14.1.8 (cf Dirichlet). Soit $f : G \rightarrow \mathbb{C}$. Alors,

$\forall x \in G, f(x) = \sum_{\chi \in \hat{G}} \hat{f}(\chi)\chi(x)$

Démonstration On a

\begin{align*}
\sum_{\chi \in \hat{G}} \hat{f}(\chi)\chi(x) &= \frac{1}{|G|}
\sum_{\chi \in \hat{G}} \sum_{y \in G} f(y) \overline{\chi}(y) \chi(x) \\
&= \frac{1}{|G|} \sum_{y \in G} f(y) \sum_{\chi \in \hat{G}} \chi(xy^{-1})
\end{align*}

puisque $\overline{\chi}(y) = \frac{1}{\chi(y)} = \chi(y^{-1})$. Mais, d'après un résultat précédent
$\sum_{\chi \in \hat{G}} \chi(xy^{-1}) = 0$ si
$xy^{-1} \neq e$, soit
$y \neq x$ et
$\sum_{\chi \in \hat{G}} \chi(xy^{-1}) =
|\hat{G}| = |G|$ si
$y = x$. D'où ne persiste dans la somme que le terme pour $y = x$ et donc
$\sum_{\chi \in \hat{G}} \hat{f}(\chi)\chi(x) = f(x)$.

Corollaire~14.1.9 $\hat{G}$ est une base orthonormée de $E$.

Démonstration Le théorème précédent montre que c'est une famille
génératrice et on a vu précédemment que c'est une famille orthonormée,
donc libre.

Théorème~14.1.10 (cf Parseval-Plancherel). Soit $f : G \rightarrow \mathbb{C}$. Alors

$\|f\|^2 =
(f|f) = \sum_{\chi \in \hat{G}} |\hat{f}(\chi)|^2$

Démonstration On a vu précédemment que les $\hat{f}(\chi)$
sont les coordonnées de $f$ dans la base orthonormée
$\hat{G}$ de $E$ et le carré de la norme de $f$ dans $E$ est la
somme des modules des carrés des coordonnées dans une base orthonormée.