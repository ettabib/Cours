\documentclass[]{article}
\usepackage[T1]{fontenc}
\usepackage{lmodern}
\usepackage{amssymb,amsmath}
\usepackage{ifxetex,ifluatex}
\usepackage{fixltx2e} % provides \textsubscript
% use upquote if available, for straight quotes in verbatim environments
\IfFileExists{upquote.sty}{\usepackage{upquote}}{}
\ifnum 0\ifxetex 1\fi\ifluatex 1\fi=0 % if pdftex
  \usepackage[utf8]{inputenc}
\else % if luatex or xelatex
  \ifxetex
    \usepackage{mathspec}
    \usepackage{xltxtra,xunicode}
  \else
    \usepackage{fontspec}
  \fi
  \defaultfontfeatures{Mapping=tex-text,Scale=MatchLowercase}
  \newcommand{\euro}{€}
\fi
% use microtype if available
\IfFileExists{microtype.sty}{\usepackage{microtype}}{}
\ifxetex
  \usepackage[setpagesize=false, % page size defined by xetex
              unicode=false, % unicode breaks when used with xetex
              xetex]{hyperref}
\else
  \usepackage[unicode=true]{hyperref}
\fi
\hypersetup{breaklinks=true,
            bookmarks=true,
            pdfauthor={},
            pdftitle={Operations sur les series},
            colorlinks=true,
            citecolor=blue,
            urlcolor=blue,
            linkcolor=magenta,
            pdfborder={0 0 0}}
\urlstyle{same}  % don't use monospace font for urls
\setlength{\parindent}{0pt}
\setlength{\parskip}{6pt plus 2pt minus 1pt}
\setlength{\emergencystretch}{3em}  % prevent overfull lines
\setcounter{secnumdepth}{0}
 
/* start css.sty */
.cmr-5{font-size:50%;}
.cmr-7{font-size:70%;}
.cmmi-5{font-size:50%;font-style: italic;}
.cmmi-7{font-size:70%;font-style: italic;}
.cmmi-10{font-style: italic;}
.cmsy-5{font-size:50%;}
.cmsy-7{font-size:70%;}
.cmex-7{font-size:70%;}
.cmex-7x-x-71{font-size:49%;}
.msbm-7{font-size:70%;}
.cmtt-10{font-family: monospace;}
.cmti-10{ font-style: italic;}
.cmbx-10{ font-weight: bold;}
.cmr-17x-x-120{font-size:204%;}
.cmsl-10{font-style: oblique;}
.cmti-7x-x-71{font-size:49%; font-style: italic;}
.cmbxti-10{ font-weight: bold; font-style: italic;}
p.noindent { text-indent: 0em }
td p.noindent { text-indent: 0em; margin-top:0em; }
p.nopar { text-indent: 0em; }
p.indent{ text-indent: 1.5em }
@media print {div.crosslinks {visibility:hidden;}}
a img { border-top: 0; border-left: 0; border-right: 0; }
center { margin-top:1em; margin-bottom:1em; }
td center { margin-top:0em; margin-bottom:0em; }
.Canvas { position:relative; }
li p.indent { text-indent: 0em }
.enumerate1 {list-style-type:decimal;}
.enumerate2 {list-style-type:lower-alpha;}
.enumerate3 {list-style-type:lower-roman;}
.enumerate4 {list-style-type:upper-alpha;}
div.newtheorem { margin-bottom: 2em; margin-top: 2em;}
.obeylines-h,.obeylines-v {white-space: nowrap; }
div.obeylines-v p { margin-top:0; margin-bottom:0; }
.overline{ text-decoration:overline; }
.overline img{ border-top: 1px solid black; }
td.displaylines {text-align:center; white-space:nowrap;}
.centerline {text-align:center;}
.rightline {text-align:right;}
div.verbatim {font-family: monospace; white-space: nowrap; text-align:left; clear:both; }
.fbox {padding-left:3.0pt; padding-right:3.0pt; text-indent:0pt; border:solid black 0.4pt; }
div.fbox {display:table}
div.center div.fbox {text-align:center; clear:both; padding-left:3.0pt; padding-right:3.0pt; text-indent:0pt; border:solid black 0.4pt; }
div.minipage{width:100%;}
div.center, div.center div.center {text-align: center; margin-left:1em; margin-right:1em;}
div.center div {text-align: left;}
div.flushright, div.flushright div.flushright {text-align: right;}
div.flushright div {text-align: left;}
div.flushleft {text-align: left;}
.underline{ text-decoration:underline; }
.underline img{ border-bottom: 1px solid black; margin-bottom:1pt; }
.framebox-c, .framebox-l, .framebox-r { padding-left:3.0pt; padding-right:3.0pt; text-indent:0pt; border:solid black 0.4pt; }
.framebox-c {text-align:center;}
.framebox-l {text-align:left;}
.framebox-r {text-align:right;}
span.thank-mark{ vertical-align: super }
span.footnote-mark sup.textsuperscript, span.footnote-mark a sup.textsuperscript{ font-size:80%; }
div.tabular, div.center div.tabular {text-align: center; margin-top:0.5em; margin-bottom:0.5em; }
table.tabular td p{margin-top:0em;}
table.tabular {margin-left: auto; margin-right: auto;}
div.td00{ margin-left:0pt; margin-right:0pt; }
div.td01{ margin-left:0pt; margin-right:5pt; }
div.td10{ margin-left:5pt; margin-right:0pt; }
div.td11{ margin-left:5pt; margin-right:5pt; }
table[rules] {border-left:solid black 0.4pt; border-right:solid black 0.4pt; }
td.td00{ padding-left:0pt; padding-right:0pt; }
td.td01{ padding-left:0pt; padding-right:5pt; }
td.td10{ padding-left:5pt; padding-right:0pt; }
td.td11{ padding-left:5pt; padding-right:5pt; }
table[rules] {border-left:solid black 0.4pt; border-right:solid black 0.4pt; }
.hline hr, .cline hr{ height : 1px; margin:0px; }
.tabbing-right {text-align:right;}
span.TEX {letter-spacing: -0.125em; }
span.TEX span.E{ position:relative;top:0.5ex;left:-0.0417em;}
a span.TEX span.E {text-decoration: none; }
span.LATEX span.A{ position:relative; top:-0.5ex; left:-0.4em; font-size:85%;}
span.LATEX span.TEX{ position:relative; left: -0.4em; }
div.float img, div.float .caption {text-align:center;}
div.figure img, div.figure .caption {text-align:center;}
.marginpar {width:20%; float:right; text-align:left; margin-left:auto; margin-top:0.5em; font-size:85%; text-decoration:underline;}
.marginpar p{margin-top:0.4em; margin-bottom:0.4em;}
.equation td{text-align:center; vertical-align:middle; }
td.eq-no{ width:5%; }
table.equation { width:100%; } 
div.math-display, div.par-math-display{text-align:center;}
math .texttt { font-family: monospace; }
math .textit { font-style: italic; }
math .textsl { font-style: oblique; }
math .textsf { font-family: sans-serif; }
math .textbf { font-weight: bold; }
.partToc a, .partToc, .likepartToc a, .likepartToc {line-height: 200%; font-weight:bold; font-size:110%;}
.chapterToc a, .chapterToc, .likechapterToc a, .likechapterToc, .appendixToc a, .appendixToc {line-height: 200%; font-weight:bold;}
.index-item, .index-subitem, .index-subsubitem {display:block}
.caption td.id{font-weight: bold; white-space: nowrap; }
table.caption {text-align:center;}
h1.partHead{text-align: center}
p.bibitem { text-indent: -2em; margin-left: 2em; margin-top:0.6em; margin-bottom:0.6em; }
p.bibitem-p { text-indent: 0em; margin-left: 2em; margin-top:0.6em; margin-bottom:0.6em; }
.paragraphHead, .likeparagraphHead { margin-top:2em; font-weight: bold;}
.subparagraphHead, .likesubparagraphHead { font-weight: bold;}
.quote {margin-bottom:0.25em; margin-top:0.25em; margin-left:1em; margin-right:1em; text-align:justify;}
.verse{white-space:nowrap; margin-left:2em}
div.maketitle {text-align:center;}
h2.titleHead{text-align:center;}
div.maketitle{ margin-bottom: 2em; }
div.author, div.date {text-align:center;}
div.thanks{text-align:left; margin-left:10%; font-size:85%; font-style:italic; }
div.author{white-space: nowrap;}
.quotation {margin-bottom:0.25em; margin-top:0.25em; margin-left:1em; }
h1.partHead{text-align: center}
.sectionToc, .likesectionToc {margin-left:2em;}
.subsectionToc, .likesubsectionToc {margin-left:4em;}
.subsubsectionToc, .likesubsubsectionToc {margin-left:6em;}
.frenchb-nbsp{font-size:75%;}
.frenchb-thinspace{font-size:75%;}
.figure img.graphics {margin-left:10%;}
/* end css.sty */

\title{Operations sur les series}
\author{}
\date{}

\begin{document}
\maketitle

\textbf{Warning: \href{http://www.math.union.edu/locate/jsMath}{jsMath}
requires JavaScript to process the mathematics on this page.\\ If your
browser supports JavaScript, be sure it is enabled.}

\begin{center}\rule{3in}{0.4pt}\end{center}

{[}\href{coursse41.html}{next}{]} {[}\href{coursse39.html}{prev}{]}
{[}\href{coursse39.html\#tailcoursse39.html}{prev-tail}{]}
{[}\hyperref[tailcoursse40.html]{tail}{]}
{[}\href{coursch8.html\#coursse40.html}{up}{]}

\subsubsection{7.6 Opérations sur les séries}

\paragraph{7.6.1 Combinaisons linéaires}

Proposition~7.6.1 Soit E un espace vectoriel normé,
\textbackslash{}mathop\{\textbackslash{}mathop\{∑ \}\} \{a\}\_\{n\} et
\textbackslash{}mathop\{\textbackslash{}mathop\{∑ \}\} \{b\}\_\{n\} deux
séries à termes dans E, α et β deux scalaires. Si
\textbackslash{}mathop\{\textbackslash{}mathop\{∑ \}\} \{a\}\_\{n\} et
\textbackslash{}mathop\{\textbackslash{}mathop\{∑ \}\} \{b\}\_\{n\} sont
convergentes (resp. absolument convergentes), il en est de même de la
série \textbackslash{}mathop\{\textbackslash{}mathop\{∑ \}\}
(α\{a\}\_\{n\} + β\{b\}\_\{n\}) et alors

\{\textbackslash{}mathop\{∑ \}\}\_\{n=0\}\^{}\{+∞\}(α\{a\}\_\{ n\} +
β\{b\}\_\{n\}) = α\{\textbackslash{}mathop\{∑
\}\}\_\{n=0\}\^{}\{+∞\}\{a\}\_\{ n\} + β\{\textbackslash{}mathop\{∑
\}\}\_\{n=0\}\^{}\{+∞\}\{b\}\_\{ n\}

Démonstration Le résultat a déjà été vu pour la convergence~; pour la
convergence absolue, il résulte de
\textbackslash{}\textbar{}α\{a\}\_\{n\} +
β\{b\}\_\{n\}\textbackslash{}\textbar{}
≤\textbar{}α\textbar{}\textbackslash{},\textbackslash{}\textbar{}\{a\}\_\{n\}\textbackslash{}\textbar{}
+
\textbar{}β\textbar{}\textbackslash{},\textbackslash{}\textbar{}\{b\}\_\{n\}\textbackslash{}\textbar{}

Corollaire~7.6.2 Soit (\{z\}\_\{n\}) une suite de nombres complexes,
\{z\}\_\{n\} = \{x\}\_\{n\} + i\{y\}\_\{n\}, \{x\}\_\{n\},\{y\}\_\{n\} ∈
ℝ. Alors la série \textbackslash{}mathop\{\textbackslash{}mathop\{∑ \}\}
\{z\}\_\{n\} est convergente (resp. absolument convergente) si et
seulement si~les deux séries
\textbackslash{}mathop\{\textbackslash{}mathop\{∑ \}\} \{x\}\_\{n\} et
\textbackslash{}mathop\{\textbackslash{}mathop\{∑ \}\} \{y\}\_\{n\} le
sont.

Démonstration Le sens direct résulte de \{x\}\_\{n\} =\{ 1
\textbackslash{}over 2\} (\{z\}\_\{n\} +
\textbackslash{}overline\{\{z\}\_\{n\}\}) et \{y\}\_\{n\} =\{ 1
\textbackslash{}over 2i\} (\{z\}\_\{n\}
−\textbackslash{}overline\{\{z\}\_\{n\}\}). La réciproque est évidente.

\paragraph{7.6.2 Sommation par paquets}

Théorème~7.6.3 (Sommation par paquets) Soit E un espace vectoriel normé,
\textbackslash{}mathop\{\textbackslash{}mathop\{∑ \}\} \{x\}\_\{n\} une
série à termes dans E, φ une application strictement croissante de ℕ
dans ℕ. On pose \{y\}\_\{0\} =\{\textbackslash{}mathop\{
\textbackslash{}mathop\{∑ \}\} \}\_\{k=0\}\^{}\{φ(0)\}\{x\}\_\{k\} et
pour n ≥ 1, \{y\}\_\{n\} =\{\textbackslash{}mathop\{
\textbackslash{}mathop\{∑ \}\}
\}\_\{k=φ(n−1)+1\}\^{}\{φ(n)\}\{x\}\_\{k\}. Alors

\begin{itemize}
\itemsep1pt\parskip0pt\parsep0pt
\item
  (i) si la série \textbackslash{}mathop\{\textbackslash{}mathop\{∑ \}\}
  \{x\}\_\{n\} converge, la série
  \textbackslash{}mathop\{\textbackslash{}mathop\{∑ \}\} \{y\}\_\{n\}
  converge et a même somme
\item
  (ii) la réciproque est vraie dans les deux cas suivants

  \begin{itemize}
  \itemsep1pt\parskip0pt\parsep0pt
  \item
    (a) la suite \{x\}\_\{n\} tend vers 0 et la suite φ(n + 1) − φ(n)
    (la taille des ''paquets'') est majorée
  \item
    (b) E = ℝ et à l'intérieur de chaque ''paquet'' (k ∈ {[}φ(n − 1) +
    1,φ(n){]}), tous les \{x\}\_\{k\}, sont de même signe.
  \end{itemize}
\end{itemize}

Démonstration On a d'abord

\{S\}\_\{n\}(y) =\{ \textbackslash{}mathop\{∑
\}\}\_\{p=0\}\^{}\{n\}(\{\textbackslash{}mathop\{∑
\}\}\_\{k=φ(n−1)+1\}\^{}\{φ(n)\}\{x\}\_\{ k\}) =\{
\textbackslash{}mathop\{∑ \}\}\_\{k=0\}\^{}\{φ(n)\}\{x\}\_\{ k\} =
\{S\}\_\{φ(n)\}(x)

(en convenant que φ(−1) = −1). La suite \{S\}\_\{n\}(y) est donc une
sous suite de la suite \{S\}\_\{n\}(x), ce qui montre l'assertion (i).

(ii.a) Soit S =\{\textbackslash{}mathop\{ \textbackslash{}mathop\{∑ \}\}
\}\_\{n=0\}\^{}\{+∞\}\{y\}\_\{n\} et K tel que
\textbackslash{}mathop\{∀\}n, φ(n + 1) − φ(n) ≤ K. Soit n ∈ ℕ et p
l'unique entier tel que φ(p − 1) \textless{} n ≤ φ(p). On a alors

\{S\}\_\{p\}(y) − \{S\}\_\{n\}(x) = \{S\}\_\{φ(p)\}(x) − \{S\}\_\{n\}(x)
=\{ \textbackslash{}mathop\{∑ \}\}\_\{k=n+1\}\^{}\{φ(p)\}\{x\}\_\{ k\}

Soit alors ε \textgreater{} 0 et N ∈ ℕ tel que n ≥ N
⇒\textbackslash{}\textbar{} \{x\}\_\{n\}\textbackslash{}\textbar{}
\textless{}\{ ε \textbackslash{}over 2K\} . Alors pour n ≥ N, on a
\textbackslash{}\textbar{}\{S\}\_\{p\}(y) −
\{S\}\_\{n\}(x)\textbackslash{}\textbar{}
≤\{\textbackslash{}mathop\{\textbackslash{}mathop\{∑ \}\}
\}\_\{k=n+1\}\^{}\{φ(p)\}\textbackslash{}\textbar{}\{x\}\_\{k\}\textbackslash{}\textbar{}
≤ (φ(p) − n)\{ ε \textbackslash{}over 2K\} ≤\{ ε \textbackslash{}over
2\} . Mais il existe N' tel que q ≥ N' ⇒\textbackslash{}\textbar{} S −
\{S\}\_\{q\}(y)\textbackslash{}\textbar{} \textless{}\{ ε
\textbackslash{}over 2\} . Si on choisit n ≥\textbackslash{}mathop\{
max\}(N,φ(N')), on a p ≥ N' et donc

\textbackslash{}\textbar{}S − \{S\}\_\{n\}(x)\textbackslash{}\textbar{}
≤\textbackslash{}\textbar{} S −
\{S\}\_\{p\}(y)\textbackslash{}\textbar{} +\textbackslash{}\textbar{}
\{S\}\_\{p\}(y) − \{S\}\_\{n\}(x)\textbackslash{}\textbar{} \textless{}
ε

ce qui montre que la série
\textbackslash{}mathop\{\textbackslash{}mathop\{∑ \}\} \{x\}\_\{n\}
converge.

(ii.b) La démonstration est similaire mais on remarque que

\textbackslash{}begin\{eqnarray*\} \textbar{}\{S\}\_\{p\}(y) −
\{S\}\_\{n\}(x)\textbar{}\& =\& \textbar{}\{\textbackslash{}mathop\{∑
\}\}\_\{k=n+1\}\^{}\{φ(p)\}\{x\}\_\{ k\}\textbar{} =\{
\textbackslash{}mathop\{∑ \}\}\_\{k=n+1\}\^{}\{φ(p)\}\textbar{}\{x\}\_\{
k\}\textbar{} \%\& \textbackslash{}\textbackslash{} \& ≤\&
\{\textbackslash{}mathop\{∑
\}\}\_\{k=φ(p−1)+1\}\^{}\{φ(p)\}\textbar{}\{x\}\_\{ k\}\textbar{} =
\textbar{}\{\textbackslash{}mathop\{∑
\}\}\_\{k=φ(p−1)+1\}\^{}\{φ(p)\}\{x\}\_\{ k\}\textbar{}\%\&
\textbackslash{}\textbackslash{} \& =\& \textbar{}\{y\}\_\{p\}\textbar{}
\%\& \textbackslash{}\textbackslash{} \textbackslash{}end\{eqnarray*\}

(car tous les \{x\}\_\{k\} sont de même signe). Comme la série
\textbackslash{}mathop\{\textbackslash{}mathop\{∑ \}\} \{y\}\_\{q\}
converge, pour q ≥ N on a \textbar{}\{y\}\_\{q\}\textbar{} \textless{}\{
ε \textbackslash{}over 2\} . Alors pour n ≥ φ(N), on a p ≥ N et donc
\textbar{}\{S\}\_\{p\}(y) −
\{S\}\_\{n\}(x)\textbar{}≤\textbar{}\{y\}\_\{p\}\textbar{} \textless{}\{
ε \textbackslash{}over 2\} . On achève alors la démonstration comme dans
le cas précédent.

Remarque~7.6.1 La réciproque de (i) n'est pas valable en toute
généralité comme le montre l'exemple de la série
\textbackslash{}mathop\{\textbackslash{}mathop\{∑ \}\} \{(−1)\}\^{}\{n\}
et de φ(n) = 2n. On a alors \{y\}\_\{n\} = 0, la série
\textbackslash{}mathop\{\textbackslash{}mathop\{∑ \}\} \{y\}\_\{n\}
converge alors que la série
\textbackslash{}mathop\{\textbackslash{}mathop\{∑ \}\} \{x\}\_\{n\} est
divergente. La réciproque (ii.b) est particulièrement intéressante pour
le cas de séries de nombres réels qui ne sont pas de signe constant~; en
regroupant ensemble les termes consécutifs de même signe, on obtient une
série de même nature que la série initiale et dont les termes sont
alternés en signe.

\paragraph{7.6.3 Modification de l'ordre des termes}

Nous allons ici étudier l'effet d'une permutation sur les termes d'une
série convergente. Pour cela nous aurons besoin du lemme suivant.

Théorème~7.6.4 Soit \textbackslash{}mathop\{\textbackslash{}mathop\{∑
\}\} \{x\}\_\{n\} une série à termes réels ou complexes absolument
convergente et soit σ : ℕ → ℕ bijective, une permutation de ℕ. Alors la
série \textbackslash{}mathop\{\textbackslash{}mathop\{∑ \}\}
\{x\}\_\{σ(n)\} est absolument convergente et
\{\textbackslash{}mathop\{\textbackslash{}mathop\{∑ \}\}
\}\_\{n=0\}\^{}\{+∞\}\{x\}\_\{σ(n)\} =\{\textbackslash{}mathop\{
\textbackslash{}mathop\{∑ \}\} \}\_\{n=0\}\^{}\{+∞\}\{x\}\_\{n\}.

Démonstration Premier cas~: série à termes réels positifs. Pour n ∈ ℕ,
soit \{N\}\_\{n\} le plus grand élément de σ({[}0,n{]}). On a alors

\{\textbackslash{}mathop\{∑ \}\}\_\{k=0\}\^{}\{n\}\{x\}\_\{ σ(k)\}
≤\{\textbackslash{}mathop\{∑ \}\}\_\{p=0\}\^{}\{\{N\}\_\{n\}
\}\{x\}\_\{p\} ≤\{\textbackslash{}mathop\{∑
\}\}\_\{p=0\}\^{}\{+∞\}\{x\}\_\{ p\}

ce qui montre que la série à termes réels positifs
\textbackslash{}mathop\{\textbackslash{}mathop\{∑ \}\} \{x\}\_\{σ(k)\}
converge et que \{\textbackslash{}mathop\{\textbackslash{}mathop\{∑ \}\}
\}\_\{n=0\}\^{}\{+∞\}\{x\}\_\{σ(n)\}
≤\{\textbackslash{}mathop\{\textbackslash{}mathop\{∑ \}\}
\}\_\{n=0\}\^{}\{+∞\}\{x\}\_\{n\}. Mais les deux séries jouent un rôle
symétrique puisque \{x\}\_\{n\} = \{x\}\_\{\{σ\}\^{}\{−1\}(σ(n))\}, et
donc on a aussi \{\textbackslash{}mathop\{\textbackslash{}mathop\{∑ \}\}
\}\_\{n=0\}\^{}\{+∞\}\{x\}\_\{n\}
≤\{\textbackslash{}mathop\{\textbackslash{}mathop\{∑ \}\}
\}\_\{n=0\}\^{}\{+∞\}\{x\}\_\{σ(n)\} ce qui nous donne l'égalité.

Deuxième cas~: séries à termes réels On introduit, comme d'habitude,
pour x ∈ ℝ, \{x\}\^{}\{+\} =\textbackslash{}mathop\{ max\}(x,0) ∈
\{ℝ\}\^{}\{+\} et \{x\}\^{}\{−\} =\textbackslash{}mathop\{ max\}(−x,0) ∈
\{ℝ\}\^{}\{+\} si bien que x = \{x\}\^{}\{+\} − \{x\}\^{}\{−\},
\textbar{}x\textbar{} = \{x\}\^{}\{+\} + \{x\}\^{}\{−\}. On a alors 0 ≤
\{x\}\_\{n\}\^{}\{+\} ≤\textbar{}\{x\}\_\{n\}\textbar{} et 0 ≤
\{x\}\_\{n\}\^{}\{−\}≤\textbar{}\{x\}\_\{n\}\textbar{}, ce qui montre
que les deux séries à termes positifs
\textbackslash{}mathop\{\textbackslash{}mathop\{∑ \}\}
\{x\}\_\{n\}\^{}\{+\} et
\textbackslash{}mathop\{\textbackslash{}mathop\{∑ \}\}
\{x\}\_\{n\}\^{}\{−\} sont convergentes. D'après le premier cas de la
démonstration, les deux séries
\textbackslash{}mathop\{\textbackslash{}mathop\{∑ \}\}
\{x\}\_\{σ(n)\}\^{}\{+\} et
\textbackslash{}mathop\{\textbackslash{}mathop\{∑ \}\}
\{x\}\_\{σ(n)\}\^{}\{−\} sont convergentes et on a

\{\textbackslash{}mathop\{∑ \}\}\_\{n=0\}\^{}\{+∞\}\{x\}\_\{
σ(n)\}\^{}\{+\} =\{ \textbackslash{}mathop\{∑
\}\}\_\{n=0\}\^{}\{+∞\}\{x\}\_\{ n\}\^{}\{+\},\textbackslash{}quad
\{\textbackslash{}mathop\{∑ \}\}\_\{n=0\}\^{}\{+∞\}\{x\}\_\{
σ(n)\}\^{}\{−\} =\{ \textbackslash{}mathop\{∑
\}\}\_\{n=0\}\^{}\{+∞\}\{x\}\_\{ n\}\^{}\{−\}

Comme \textbar{}\{x\}\_\{σ(n)\}\textbar{} = \{x\}\_\{σ(n)\}\^{}\{+\} +
\{x\}\_\{σ(n)\}\^{}\{−\}, la série
\textbackslash{}mathop\{\textbackslash{}mathop\{∑ \}\}
\textbar{}\{x\}\_\{σ(n)\}\textbar{} converge, donc la série
\textbackslash{}mathop\{\textbackslash{}mathop\{∑ \}\} \{x\}\_\{σ(n)\}
est absolument convergente, et comme \{x\}\_\{σ(n)\} =
\{x\}\_\{σ(n)\}\^{}\{+\} − \{x\}\_\{σ(n)\}\^{}\{−\}, on a

\{\textbackslash{}mathop\{∑ \}\}\_\{n=0\}\^{}\{+∞\}\{x\}\_\{ σ(n)\} =\{
\textbackslash{}mathop\{∑ \}\}\_\{n=0\}\^{}\{+∞\}\{x\}\_\{
σ(n)\}\^{}\{+\}−\{\textbackslash{}mathop\{∑
\}\}\_\{n=0\}\^{}\{+∞\}\{x\}\_\{ σ(n)\}\^{}\{−\} =\{
\textbackslash{}mathop\{∑ \}\}\_\{n=0\}\^{}\{+∞\}\{x\}\_\{
n\}\^{}\{+\}−\{\textbackslash{}mathop\{∑
\}\}\_\{n=0\}\^{}\{+∞\}\{x\}\_\{ n\}\^{}\{−\} =\{
\textbackslash{}mathop\{∑ \}\}\_\{n=0\}\^{}\{+∞\}\{x\}\_\{ n\}

Troisième cas~: séries à termes complexes On travaille de la même
fa\textbackslash{}c\{c\}on avec les parties réelles et parties
imaginaires. On a 0
≤\textbar{}\textbackslash{}mathop\{\textbackslash{}mathrm\{Re\}\}(\{x\}\_\{n\})\textbar{}≤\textbar{}\{x\}\_\{n\}\textbar{}
et 0
≤\textbar{}\textbackslash{}mathop\{\textbackslash{}mathrm\{Im\}\}(\{x\}\_\{n\})\textbar{}≤\textbar{}\{x\}\_\{n\}\textbar{},
ce qui montre que les deux séries
\textbackslash{}mathop\{\textbackslash{}mathop\{∑ \}\}
\textbackslash{}mathop\{\textbackslash{}mathrm\{Re\}\}(\{x\}\_\{n\}) et
\textbackslash{}mathop\{\textbackslash{}mathop\{∑ \}\}
\textbackslash{}mathop\{\textbackslash{}mathrm\{Im\}\}(\{x\}\_\{n\})
sont absolument convergentes. D'après le deuxième cas de la
démonstration, les deux séries
\textbackslash{}mathop\{\textbackslash{}mathop\{∑ \}\}
\textbackslash{}mathop\{\textbackslash{}mathrm\{Re\}\}(\{x\}\_\{σ(n)\})
et \textbackslash{}mathop\{\textbackslash{}mathop\{∑ \}\}
\textbackslash{}mathop\{\textbackslash{}mathrm\{Im\}\}(\{x\}\_\{σ(n)\})
sont absolument convergentes et on a

\{\textbackslash{}mathop\{∑
\}\}\_\{n=0\}\^{}\{+∞\}\textbackslash{}mathrm\{Re\}(\{x\}\_\{ σ(n)\})
=\{ \textbackslash{}mathop\{∑
\}\}\_\{n=0\}\^{}\{+∞\}\textbackslash{}mathrm\{Re\}(\{x\}\_\{
n\}),\textbackslash{}quad \{\textbackslash{}mathop\{∑
\}\}\_\{n=0\}\^{}\{+∞\}\textbackslash{}mathrm\{Im\}(\{x\}\_\{ σ(n)\})
=\{ \textbackslash{}mathop\{∑
\}\}\_\{n=0\}\^{}\{+∞\}\textbackslash{}mathrm\{Im\}(\{x\}\_\{ n\})

Comme
\textbar{}\{x\}\_\{σ(n)\}\textbar{}≤\textbar{}\textbackslash{}mathop\{\textbackslash{}mathrm\{Re\}\}(\{x\}\_\{σ(n)\})\textbar{}
+
\textbar{}\textbackslash{}mathop\{\textbackslash{}mathrm\{Im\}\}(\{x\}\_\{σ(n)\})\textbar{},
la série \textbackslash{}mathop\{\textbackslash{}mathop\{∑ \}\}
\textbar{}\{x\}\_\{σ(n)\}\textbar{} converge, donc la série
\textbackslash{}mathop\{\textbackslash{}mathop\{∑ \}\} \{x\}\_\{σ(n)\}
est absolument convergente, et comme \{x\}\_\{σ(n)\}
=\textbackslash{}mathop\{
\textbackslash{}mathrm\{Re\}\}(\{x\}\_\{σ(n)\}) +
i\textbackslash{}mathop\{\textbackslash{}mathrm\{Re\}\}(\{x\}\_\{σ(n)\}),
on a

\{\textbackslash{}mathop\{∑ \}\}\_\{n=0\}\^{}\{+∞\}\{x\}\_\{ σ(n)\} =\{
\textbackslash{}mathop\{∑
\}\}\_\{n=0\}\^{}\{+∞\}\textbackslash{}mathrm\{Re\}(\{x\}\_\{
σ(n)\})+i\{\textbackslash{}mathop\{∑
\}\}\_\{n=0\}\^{}\{+∞\}\textbackslash{}mathrm\{Re\}(\{x\}\_\{ σ(n)\})
=\{ \textbackslash{}mathop\{∑
\}\}\_\{n=0\}\^{}\{+∞\}\textbackslash{}mathrm\{Re\}(\{x\}\_\{
n\})+i\{\textbackslash{}mathop\{∑
\}\}\_\{n=0\}\^{}\{+∞\}\textbackslash{}mathrm\{Im\}(\{x\}\_\{ n\}) =\{
\textbackslash{}mathop\{∑ \}\}\_\{n=0\}\^{}\{+∞\}\{x\}\_\{ n\}

Remarque~7.6.2 La condition de convergence absolue est indispensable à
la validité du théorème. Considérons la série semi convergente
\textbackslash{}mathop\{\textbackslash{}mathop\{∑ \}\} \{x\}\_\{n\} avec
\{x\}\_\{n\} =\{ \{(−1)\}\^{}\{n−1\} \textbackslash{}over n\} et soit S
sa somme (on peut montrer que S =\textbackslash{}mathop\{ log\} 2). Soit
φ : \{ℕ\}\^{}\{∗\}→ \{ℕ\}\^{}\{∗\} définie par φ(3k + 1) = 2k + 1, φ(3k
+ 2) = 4k + 2 et φ(3k + 3) = 4k + 4. On vérifie facilement que φ est une
bijection de ℕ dans ℕ (la bijection réciproque est définie par des
congruences modulo 4). Sommons alors par paquets de 3 la série
\textbackslash{}mathop\{\textbackslash{}mathop\{∑ \}\} \{x\}\_\{φ(n)\}.
On a

\textbackslash{}begin\{eqnarray*\}\{ x\}\_\{φ(3k+1)\} +
\{x\}\_\{φ(3k+2)\} + \{x\}\_\{φ(3k+3)\}\&\& \%\&
\textbackslash{}\textbackslash{} \& =\&\{ 1 \textbackslash{}over 2k +
1\} −\{ 1 \textbackslash{}over 4k + 2\} −\{ 1 \textbackslash{}over 4k +
4\} =\{ 1 \textbackslash{}over 4k + 2\} −\{ 1 \textbackslash{}over 4k +
4\} \%\& \textbackslash{}\textbackslash{} \& =\&\{ 1
\textbackslash{}over 2\} \textbackslash{}left (\{x\}\_\{2k+1\} +
\{x\}\_\{2k+2\}\textbackslash{}right ) \%\&
\textbackslash{}\textbackslash{} \textbackslash{}end\{eqnarray*\}

Ceci montre (réciproque du théorème de sommation par paquets, la taille
des paquets étant bornée et le terme général tendant vers 0) que la
nouvelle série converge encore, mais que sa somme est la moitié de la
somme de la série initiale.

\paragraph{7.6.4 Produit de Cauchy}

Définition~7.6.1 Soit \textbackslash{}mathop\{\textbackslash{}mathop\{∑
\}\} \{a\}\_\{n\} et \textbackslash{}mathop\{\textbackslash{}mathop\{∑
\}\} \{b\}\_\{n\} deux séries à termes réels ou complexes. On appelle
produit de Cauchy (ou produit de convolution) des deux séries, la série
\textbackslash{}mathop\{\textbackslash{}mathop\{∑ \}\} \{c\}\_\{n\} avec

\textbackslash{}mathop\{∀\}n ∈ ℕ, \{c\}\_\{n\} =\{
\textbackslash{}mathop\{∑ \}\}\_\{k=0\}\^{}\{n\}\{a\}\_\{
k\}\{b\}\_\{n−k\} =\{ \textbackslash{}mathop\{∑
\}\}\_\{p+q=n\}\{a\}\_\{p\}\{b\}\_\{q\}

Théorème~7.6.5 Soit \textbackslash{}mathop\{\textbackslash{}mathop\{∑
\}\} \{a\}\_\{n\} et \textbackslash{}mathop\{\textbackslash{}mathop\{∑
\}\} \{b\}\_\{n\} deux séries à termes réels ou complexes, absolument
convergentes. Alors leur produit de Cauchy
\textbackslash{}mathop\{\textbackslash{}mathop\{∑ \}\} \{c\}\_\{n\} est
une série absolument convergente et on a

\{\textbackslash{}mathop\{∑ \}\}\_\{n=0\}\^{}\{+∞\}\{c\}\_\{ n\} =
\textbackslash{}left (\{\textbackslash{}mathop\{∑
\}\}\_\{n=0\}\^{}\{+∞\}\{a\}\_\{ n\}\textbackslash{}right
)\textbackslash{}left (\{\textbackslash{}mathop\{∑
\}\}\_\{n=0\}\^{}\{+∞\}\{b\}\_\{ n\}\textbackslash{}right )

Démonstration Cas particulier~: les deux séries sont à termes réels
positifs. Notons \{K\}\_\{n\} = {[}0,n{]} × {[}0,n{]} ⊂ \{ℕ\}\^{}\{2\}
et \{T\}\_\{n\} = \textbackslash{}\{(p,q) ∈
\{ℕ\}\^{}\{2\}\textbackslash{}mathrel\{∣\}p + q ≤ n\textbackslash{}\}.
On a évidemment \{T\}\_\{n\} ⊂ \{K\}\_\{n\} ⊂ \{T\}\_\{2n\}. On a alors

\textbackslash{}begin\{eqnarray*\} \{\textbackslash{}mathop\{∑
\}\}\_\{k=0\}\^{}\{n\}\{c\}\_\{ k\}\& =\& \{\textbackslash{}mathop\{∑
\}\}\_\{k=0\}\^{}\{n\}\{ \textbackslash{}mathop\{∑
\}\}\_\{p+q=k\}\{a\}\_\{p\}\{b\}\_\{q\} =\{ \textbackslash{}mathop\{∑
\}\}\_\{(p,q)∈\{T\}\_\{n\}\}\{a\}\_\{p\}\{b\}\_\{q\}
≤\{\textbackslash{}mathop\{∑
\}\}\_\{(p,q)∈\{K\}\_\{n\}\}\{a\}\_\{p\}\{b\}\_\{q\}\%\&
\textbackslash{}\textbackslash{} \& =\& \{\textbackslash{}mathop\{∑
\}\}\_\{p=0\}\^{}\{n\}\{a\}\_\{ p\}\{ \textbackslash{}mathop\{∑
\}\}\_\{q=0\}\^{}\{n\}\{b\}\_\{ q\} ≤\{\textbackslash{}mathop\{∑
\}\}\_\{p=0\}\^{}\{+∞\}\{a\}\_\{ p\}\{ \textbackslash{}mathop\{∑
\}\}\_\{q=0\}\^{}\{+∞\}\{b\}\_\{ q\} \%\&
\textbackslash{}\textbackslash{} \textbackslash{}end\{eqnarray*\}

La série \textbackslash{}mathop\{\textbackslash{}mathop\{∑ \}\}
\{c\}\_\{n\} est une série à termes réels positifs dont les sommes
partielles sont majorées, donc elle converge. De plus les inclusions
\{T\}\_\{n\} ⊂ \{K\}\_\{n\} ⊂ \{T\}\_\{2n\} se traduisent par
\{S\}\_\{n\}(c) ≤ \{S\}\_\{n\}(a)\{S\}\_\{n\}(b) ≤ \{S\}\_\{2n\}(c) et
en faisant tendre n vers + ∞, on obtient S(c) = S(a)S(b) ce qui est la
formule souhaitée.

Cas général Posons \{a\}\_\{n\}' = \textbar{}\{a\}\_\{n\}\textbar{},
\{b\}\_\{n\}' = \textbar{}\{b\}\_\{n\}\textbar{} et \{c\}\_\{n\}'
=\{\textbackslash{}mathop\{ \textbackslash{}mathop\{∑ \}\}
\}\_\{p+q=n\}\textbar{}\{a\}\_\{p\}\textbar{}\textbar{}\{b\}\_\{q\}\textbar{}
leur produit de Cauchy, et désignons par
\{S\}\_\{n\}(a'),\{S\}\_\{n\}(b') et \{S\}\_\{n\}(c') les sommes
partielles d'indice n de ces trois séries. Puisque les séries
\textbackslash{}mathop\{\textbackslash{}mathop\{∑ \}\} \{a\}\_\{n\}' et
\textbackslash{}mathop\{\textbackslash{}mathop\{∑ \}\} \{b\}\_\{n\}'
sont convergentes, le cas particulier ci dessus montre que la série
\textbackslash{}mathop\{\textbackslash{}mathop\{∑ \}\} \{c\}\_\{n\}' est
convergente et que sa somme est le produit des sommes de ces deux
séries. Mais, comme \textbar{}\{c\}\_\{n\}\textbar{}≤ \{c\}\_\{n\}', on
en déduit la convergence absolue de la série
\textbackslash{}mathop\{\textbackslash{}mathop\{∑ \}\} \{c\}\_\{n\}. On
a alors

\textbackslash{}begin\{eqnarray*\} \textbackslash{}left
\textbar{}\{S\}\_\{n\}(a)\{S\}\_\{n\}(b) −
\{S\}\_\{n\}(c)\textbackslash{}right \textbar{}\& =\&
\textbackslash{}left \textbar{}\{\textbackslash{}mathop\{∑
\}\}\_\{(p,q)∈\{K\}\_\{n\}\}\{a\}\_\{p\}\{b\}\_\{q\}
−\{\textbackslash{}mathop\{∑
\}\}\_\{(p,q)∈\{T\}\_\{n\}\}\{a\}\_\{p\}\{b\}\_\{q\}\textbackslash{}right
\textbar{} = \textbackslash{}left \textbar{}\{\textbackslash{}mathop\{∑
\}\}\_\{(p,q)∈\{K\}\_\{n\}∖\{T\}\_\{n\}\}\{a\}\_\{p\}\{b\}\_\{q\}\textbackslash{}right
\textbar{} \%\& \textbackslash{}\textbackslash{} \& ≤\&
\{\textbackslash{}mathop\{∑
\}\}\_\{(p,q)∈\{K\}\_\{n\}∖\{T\}\_\{n\}\}\textbar{}\{a\}\_\{p\}\textbar{}\textbar{}\{b\}\_\{q\}\textbar{}
=\{ \textbackslash{}mathop\{∑
\}\}\_\{(p,q)∈\{K\}\_\{n\}\}\textbar{}\{a\}\_\{p\}\textbar{}\textbar{}\{b\}\_\{q\}\textbar{}−\{\textbackslash{}mathop\{∑
\}\}\_\{(p,q)∈\{T\}\_\{n\}\}\textbar{}\{a\}\_\{p\}\textbar{}\textbar{}\{b\}\_\{q\}\textbar{}
= \{S\}\_\{n\}(a')\{S\}\_\{n\}(b') −
\{S\}\_\{n\}(c')\%\&\textbackslash{}\textbackslash{}
\textbackslash{}end\{eqnarray*\}

Puisque la somme de la série
\textbackslash{}mathop\{\textbackslash{}mathop\{∑ \}\} \{c\}\_\{n\}' est
le produit des sommes des deux séries
\textbackslash{}mathop\{\textbackslash{}mathop\{∑ \}\} \{a\}\_\{n\}' et
\textbackslash{}mathop\{\textbackslash{}mathop\{∑ \}\} \{b\}\_\{n\}', on
a
\{\textbackslash{}mathop\{lim\}\}\_\{n→+∞\}(\{S\}\_\{n\}(a')\{S\}\_\{n\}(b')
− \{S\}\_\{n\}(c')) = 0 et donc par la majoration ci-dessus
\{\textbackslash{}mathop\{lim\}\}\_\{n→+∞\}(\{S\}\_\{n\}(a)\{S\}\_\{n\}(b)
− \{S\}\_\{n\}(c)) = 0, ce qui montre que la somme de la série
\textbackslash{}mathop\{\textbackslash{}mathop\{∑ \}\} \{c\}\_\{n\} est
le produit des sommes des deux séries
\textbackslash{}mathop\{\textbackslash{}mathop\{∑ \}\} \{a\}\_\{n\} et
\textbackslash{}mathop\{\textbackslash{}mathop\{∑ \}\} \{b\}\_\{n\} et
achève la démonstration.

Remarque~7.6.3 On aurait pu passer aussi du cas réel positif au cas
complexe en utilisant, comme dans le théorème de permutation des termes,
les parties positives \{x\}\^{}\{+\} et \{x\}\^{}\{−\} d'un réel x, puis
les parties réelle et imaginaire d'un nombre complexe, mais la
démonstration n'aurait pas pu se généraliser comme nous le ferons
ci-dessous au cas d'une application bilinéaire plus générale.

Remarque~7.6.4 Le théorème ci dessus n'est plus valable pour des séries
convergentes~: posons \{a\}\_\{n\} = \{b\}\_\{n\} =\{ \{(−1)\}\^{}\{n\}
\textbackslash{}over \textbackslash{}sqrt\{n+1\}\} . On a
\textbar{}\{c\}\_\{n\}\textbar{} =\{\textbackslash{}mathop\{
\textbackslash{}mathop\{∑ \}\} \}\_\{k=0\}\^{}\{n\}\{ 1
\textbackslash{}over \textbackslash{}sqrt\{(k+1)(n−k+1)\}\} . Mais pour
k ∈ {[}0,n{]}, (k + 1)(n − k + 1) ≤ \{(\{ n \textbackslash{}over 2\} +
1)\}\^{}\{2\} (facile). Donc \textbar{}\{c\}\_\{n\}\textbar{}≥\{ n+1
\textbackslash{}over \{ n \textbackslash{}over 2\} +1\} qui tend vers
2~; donc la suite (\{c\}\_\{n\}) ne tend pas vers 0 et la série
\textbackslash{}mathop\{\textbackslash{}mathop\{∑ \}\} \{c\}\_\{n\}
diverge.

On a une généralisation du théorème précédent sous la forme suivante qui
nous sera utile quand nous considérerons des séries d'endomorphismes.

Théorème~7.6.6 Soit E, F et G sont trois espaces vectoriels normés, u :
E × F → G une application bilinéaire continue,
\textbackslash{}mathop\{\textbackslash{}mathop\{∑ \}\} \{a\}\_\{n\} une
série à termes dans E absolument convergente,
\textbackslash{}mathop\{\textbackslash{}mathop\{∑ \}\} \{b\}\_\{n\} une
série à termes dans F absolument convergente, et si l'on pose
\{c\}\_\{n\} =\{\textbackslash{}mathop\{ \textbackslash{}mathop\{∑ \}\}
\}\_\{p+q=n\}u(\{a\}\_\{p\},\{b\}\_\{q\}), alors la série
\textbackslash{}mathop\{\textbackslash{}mathop\{∑ \}\} \{c\}\_\{n\} est
absolument convergente et on a

\{\textbackslash{}mathop\{∑ \}\}\_\{n=0\}\^{}\{+∞\}\{c\}\_\{ n\} =
u\textbackslash{}left (\{\textbackslash{}mathop\{∑
\}\}\_\{n=0\}\^{}\{+∞\}\{a\}\_\{ n\},\{\textbackslash{}mathop\{∑
\}\}\_\{n=0\}\^{}\{+∞\}\{b\}\_\{ n\}\textbackslash{}right )

Démonstration La démonstration est tout à fait similaire~: utiliser
l'existence d'un réel positif K tel que
\textbackslash{}\textbar{}u(x,y)\textbackslash{}\textbar{} ≤
K\textbackslash{}\textbar{}x\textbackslash{}\textbar{}
\textbackslash{}\textbar{}y\textbackslash{}\textbar{} pour montrer que
\textbackslash{}left \textbar{}\{S\}\_\{n\}(a)\{S\}\_\{n\}(b) −
\{S\}\_\{n\}(c)\textbackslash{}right \textbar{}≤ K\textbackslash{}left
(\{S\}\_\{n\}(a')\{S\}\_\{n\}(b') −
\{S\}\_\{n\}(c')\textbackslash{}right ) en posant \{a\}\_\{n\}'
=\textbackslash{}\textbar{} \{a\}\_\{n\}\textbackslash{}\textbar{},
\{b\}\_\{n\}' =\textbackslash{}\textbar{}
\{b\}\_\{n\}\textbackslash{}\textbar{} et \{c\}\_\{n\}'
=\{\textbackslash{}mathop\{ \textbackslash{}mathop\{∑ \}\}
\}\_\{p+q=n\}\textbackslash{}\textbar{}\{a\}\_\{p\}\textbackslash{}\textbar{}\textbackslash{}\textbar{}\{b\}\_\{q\}\textbackslash{}\textbar{}

{[}\href{coursse41.html}{next}{]} {[}\href{coursse39.html}{prev}{]}
{[}\href{coursse39.html\#tailcoursse39.html}{prev-tail}{]}
{[}\href{coursse40.html}{front}{]}
{[}\href{coursch8.html\#coursse40.html}{up}{]}

\end{document}
