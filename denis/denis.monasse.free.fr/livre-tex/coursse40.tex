\documentclass[]{article}
\usepackage[T1]{fontenc}
\usepackage{lmodern}
\usepackage{amssymb,amsmath}
\usepackage{ifxetex,ifluatex}
\usepackage{fixltx2e} % provides \textsubscript
% use upquote if available, for straight quotes in verbatim environments
\IfFileExists{upquote.sty}{\usepackage{upquote}}{}
\ifnum 0\ifxetex 1\fi\ifluatex 1\fi=0 % if pdftex
  \usepackage[utf8]{inputenc}
\else % if luatex or xelatex
  \ifxetex
    \usepackage{mathspec}
    \usepackage{xltxtra,xunicode}
  \else
    \usepackage{fontspec}
  \fi
  \defaultfontfeatures{Mapping=tex-text,Scale=MatchLowercase}
  \newcommand{\euro}{€}
\fi
% use microtype if available
\IfFileExists{microtype.sty}{\usepackage{microtype}}{}
\ifxetex
  \usepackage[setpagesize=false, % page size defined by xetex
              unicode=false, % unicode breaks when used with xetex
              xetex]{hyperref}
\else
  \usepackage[unicode=true]{hyperref}
\fi
\hypersetup{breaklinks=true,
            bookmarks=true,
            pdfauthor={},
            pdftitle={Operations sur les series},
            colorlinks=true,
            citecolor=blue,
            urlcolor=blue,
            linkcolor=magenta,
            pdfborder={0 0 0}}
\urlstyle{same}  % don't use monospace font for urls
\setlength{\parindent}{0pt}
\setlength{\parskip}{6pt plus 2pt minus 1pt}
\setlength{\emergencystretch}{3em}  % prevent overfull lines
\setcounter{secnumdepth}{0}
 
/* start css.sty */
.cmr-5{font-size:50%;}
.cmr-7{font-size:70%;}
.cmmi-5{font-size:50%;font-style: italic;}
.cmmi-7{font-size:70%;font-style: italic;}
.cmmi-10{font-style: italic;}
.cmsy-5{font-size:50%;}
.cmsy-7{font-size:70%;}
.cmex-7{font-size:70%;}
.cmex-7x-x-71{font-size:49%;}
.msbm-7{font-size:70%;}
.cmtt-10{font-family: monospace;}
.cmti-10{ font-style: italic;}
.cmbx-10{ font-weight: bold;}
.cmr-17x-x-120{font-size:204%;}
.cmsl-10{font-style: oblique;}
.cmti-7x-x-71{font-size:49%; font-style: italic;}
.cmbxti-10{ font-weight: bold; font-style: italic;}
p.noindent { text-indent: 0em }
td p.noindent { text-indent: 0em; margin-top:0em; }
p.nopar { text-indent: 0em; }
p.indent{ text-indent: 1.5em }
@media print {div.crosslinks {visibility:hidden;}}
a img { border-top: 0; border-left: 0; border-right: 0; }
center { margin-top:1em; margin-bottom:1em; }
td center { margin-top:0em; margin-bottom:0em; }
.Canvas { position:relative; }
li p.indent { text-indent: 0em }
.enumerate1 {list-style-type:decimal;}
.enumerate2 {list-style-type:lower-alpha;}
.enumerate3 {list-style-type:lower-roman;}
.enumerate4 {list-style-type:upper-alpha;}
div.newtheorem { margin-bottom: 2em; margin-top: 2em;}
.obeylines-h,.obeylines-v {white-space: nowrap; }
div.obeylines-v p { margin-top:0; margin-bottom:0; }
.overline{ text-decoration:overline; }
.overline img{ border-top: 1px solid black; }
td.displaylines {text-align:center; white-space:nowrap;}
.centerline {text-align:center;}
.rightline {text-align:right;}
div.verbatim {font-family: monospace; white-space: nowrap; text-align:left; clear:both; }
.fbox {padding-left:3.0pt; padding-right:3.0pt; text-indent:0pt; border:solid black 0.4pt; }
div.fbox {display:table}
div.center div.fbox {text-align:center; clear:both; padding-left:3.0pt; padding-right:3.0pt; text-indent:0pt; border:solid black 0.4pt; }
div.minipage{width:100%;}
div.center, div.center div.center {text-align: center; margin-left:1em; margin-right:1em;}
div.center div {text-align: left;}
div.flushright, div.flushright div.flushright {text-align: right;}
div.flushright div {text-align: left;}
div.flushleft {text-align: left;}
.underline{ text-decoration:underline; }
.underline img{ border-bottom: 1px solid black; margin-bottom:1pt; }
.framebox-c, .framebox-l, .framebox-r { padding-left:3.0pt; padding-right:3.0pt; text-indent:0pt; border:solid black 0.4pt; }
.framebox-c {text-align:center;}
.framebox-l {text-align:left;}
.framebox-r {text-align:right;}
span.thank-mark{ vertical-align: super }
span.footnote-mark sup.textsuperscript, span.footnote-mark a sup.textsuperscript{ font-size:80%; }
div.tabular, div.center div.tabular {text-align: center; margin-top:0.5em; margin-bottom:0.5em; }
table.tabular td p{margin-top:0em;}
table.tabular {margin-left: auto; margin-right: auto;}
div.td00{ margin-left:0pt; margin-right:0pt; }
div.td01{ margin-left:0pt; margin-right:5pt; }
div.td10{ margin-left:5pt; margin-right:0pt; }
div.td11{ margin-left:5pt; margin-right:5pt; }
table[rules] {border-left:solid black 0.4pt; border-right:solid black 0.4pt; }
td.td00{ padding-left:0pt; padding-right:0pt; }
td.td01{ padding-left:0pt; padding-right:5pt; }
td.td10{ padding-left:5pt; padding-right:0pt; }
td.td11{ padding-left:5pt; padding-right:5pt; }
table[rules] {border-left:solid black 0.4pt; border-right:solid black 0.4pt; }
.hline hr, .cline hr{ height : 1px; margin:0px; }
.tabbing-right {text-align:right;}
span.TEX {letter-spacing: -0.125em; }
span.TEX span.E{ position:relative;top:0.5ex;left:-0.0417em;}
a span.TEX span.E {text-decoration: none; }
span.LATEX span.A{ position:relative; top:-0.5ex; left:-0.4em; font-size:85%;}
span.LATEX span.TEX{ position:relative; left: -0.4em; }
div.float img, div.float .caption {text-align:center;}
div.figure img, div.figure .caption {text-align:center;}
.marginpar {width:20%; float:right; text-align:left; margin-left:auto; margin-top:0.5em; font-size:85%; text-decoration:underline;}
.marginpar p{margin-top:0.4em; margin-bottom:0.4em;}
.equation td{text-align:center; vertical-align:middle; }
td.eq-no{ width:5%; }
table.equation { width:100%; } 
div.math-display, div.par-math-display{text-align:center;}
math .texttt { font-family: monospace; }
math .textit { font-style: italic; }
math .textsl { font-style: oblique; }
math .textsf { font-family: sans-serif; }
math .textbf { font-weight: bold; }
.partToc a, .partToc, .likepartToc a, .likepartToc {line-height: 200%; font-weight:bold; font-size:110%;}
.chapterToc a, .chapterToc, .likechapterToc a, .likechapterToc, .appendixToc a, .appendixToc {line-height: 200%; font-weight:bold;}
.index-item, .index-subitem, .index-subsubitem {display:block}
.caption td.id{font-weight: bold; white-space: nowrap; }
table.caption {text-align:center;}
h1.partHead{text-align: center}
p.bibitem { text-indent: -2em; margin-left: 2em; margin-top:0.6em; margin-bottom:0.6em; }
p.bibitem-p { text-indent: 0em; margin-left: 2em; margin-top:0.6em; margin-bottom:0.6em; }
.paragraphHead, .likeparagraphHead { margin-top:2em; font-weight: bold;}
.subparagraphHead, .likesubparagraphHead { font-weight: bold;}
.quote {margin-bottom:0.25em; margin-top:0.25em; margin-left:1em; margin-right:1em; text-align:justify;}
.verse{white-space:nowrap; margin-left:2em}
div.maketitle {text-align:center;}
h2.titleHead{text-align:center;}
div.maketitle{ margin-bottom: 2em; }
div.author, div.date {text-align:center;}
div.thanks{text-align:left; margin-left:10%; font-size:85%; font-style:italic; }
div.author{white-space: nowrap;}
.quotation {margin-bottom:0.25em; margin-top:0.25em; margin-left:1em; }
h1.partHead{text-align: center}
.sectionToc, .likesectionToc {margin-left:2em;}
.subsectionToc, .likesubsectionToc {margin-left:4em;}
.subsubsectionToc, .likesubsubsectionToc {margin-left:6em;}
.frenchb-nbsp{font-size:75%;}
.frenchb-thinspace{font-size:75%;}
.figure img.graphics {margin-left:10%;}
/* end css.sty */

\title{Operations sur les series}
\author{}
\date{}

\begin{document}
\maketitle

\textbf{Warning: 
requires JavaScript to process the mathematics on this page.\\ If your
browser supports JavaScript, be sure it is enabled.}

\begin{center}\rule{3in}{0.4pt}\end{center}

[
[
[]
[

\subsubsection{7.6 Opérations sur les séries}

\paragraph{7.6.1 Combinaisons linéaires}

Proposition~7.6.1 Soit E un espace vectoriel normé,
\\sum  a_n~ et
\\sum  b_n~ deux
séries à termes dans E, \alpha~ et \beta~ deux scalaires. Si
\\sum  a_n~ et
\\sum  b_n~ sont
convergentes (resp. absolument convergentes), il en est de même de la
série \\sum ~
(\alpha~a_n + \beta~b_n) et alors

\sum _n=0^+\infty~(\alpha~a_ n~ +
\beta~b_n) = \alpha~\\sum
_n=0^+\infty~a_ n + \beta~\\sum
_n=0^+\infty~b_ n

Démonstration Le résultat a déjà été vu pour la convergence~; pour la
convergence absolue, il résulte de
\\alpha~a_n +
\beta~b_n\
\leq\alpha~\,\a_n\
+
\beta~\,\b_n\

Corollaire~7.6.2 Soit (z_n) une suite de nombres complexes,
z_n = x_n + iy_n, x_n,y_n \in
\mathbb{R}~. Alors la série \\sum ~
z_n est convergente (resp. absolument convergente) si et
seulement si~les deux séries
\\sum  x_n~ et
\\sum  y_n~ le
sont.

Démonstration Le sens direct résulte de x_n = 1
\over 2 (z_n +
\overlinez_n) et y_n = 1
\over 2i (z_n
-\overlinez_n). La réciproque est évidente.

\paragraph{7.6.2 Sommation par paquets}

Théorème~7.6.3 (Sommation par paquets) Soit E un espace vectoriel normé,
\\sum  x_n~ une
série à termes dans E, \phi une application strictement croissante de \mathbb{N}~
dans \mathbb{N}~. On pose y_0 =\
\sum  _k=0^\phi(0)x_k~ et
pour n ≥ 1, y_n =\
\sum ~
_k=\phi(n-1)+1^\phi(n)x_k. Alors

\begin{itemize}
\itemsep1pt\parskip0pt\parsep0pt
\item
  (i) si la série \\sum ~
  x_n converge, la série
  \\sum  y_n~
  converge et a même somme
\item
  (ii) la réciproque est vraie dans les deux cas suivants

  \begin{itemize}
  \itemsep1pt\parskip0pt\parsep0pt
  \item
    (a) la suite x_n tend vers 0 et la suite \phi(n + 1) - \phi(n)
    (la taille des ''paquets'') est majorée
  \item
    (b) E = \mathbb{R}~ et à l'intérieur de chaque ''paquet'' (k \in [\phi(n - 1) +
    1,\phi(n)]), tous les x_k, sont de même signe.
  \end{itemize}
\end{itemize}

Démonstration On a d'abord

S_n(y) = \\sum
_p=0^n(\\sum
_k=\phi(n-1)+1^\phi(n)x_ k) =
\sum _k=0^\phi(n)x_ k~ =
S_\phi(n)(x)

(en convenant que \phi(-1) = -1). La suite S_n(y) est donc une
sous suite de la suite S_n(x), ce qui montre l'assertion (i).

(ii.a) Soit S = \\sum ~
_n=0^+\infty~y_n et K tel que
\forall~~n, \phi(n + 1) - \phi(n) \leq K. Soit n \in \mathbb{N}~ et p
l'unique entier tel que \phi(p - 1) < n \leq \phi(p). On a alors

S_p(y) - S_n(x) = S_\phi(p)(x) - S_n(x)
= \sum _k=n+1^\phi(p)x_ k~

Soit alors \epsilon > 0 et N \in \mathbb{N}~ tel que n ≥ N
\rigtharrow~\ x_n\
< \epsilon \over 2K . Alors pour n ≥ N, on a
\S_p(y) -
S_n(x)\
\leq\\sum ~
_k=n+1^\phi(p)\x_k\
\leq (\phi(p) - n) \epsilon \over 2K \leq \epsilon \over
2 . Mais il existe N' tel que q ≥ N' \rigtharrow~\ S -
S_q(y)\ < \epsilon
\over 2 . Si on choisit n ≥\
max(N,\phi(N')), on a p ≥ N' et donc

\S - S_n(x)\
\leq\ S -
S_p(y)\ +\
S_p(y) - S_n(x)\ <
\epsilon

ce qui montre que la série
\\sum  x_n~
converge.

(ii.b) La démonstration est similaire mais on remarque que

\begin{align*} S_p(y) -
S_n(x)& =& \\sum
_k=n+1^\phi(p)x_ k =
\sum _k=n+1^\phi(p)x_
k \%& \\ & \leq&
\\sum
_k=\phi(p-1)+1^\phi(p)x_ k =
\\sum
_k=\phi(p-1)+1^\phi(p)x_ k\%&
\\ & =& y_p
\%& \\ \end{align*}

(car tous les x_k sont de même signe). Comme la série
\\sum  y_q~
converge, pour q ≥ N on a y_q <
\epsilon \over 2 . Alors pour n ≥ \phi(N), on a p ≥ N et donc
S_p(y) -
S_n(x)\leqy_p <
\epsilon \over 2 . On achève alors la démonstration comme dans
le cas précédent.

Remarque~7.6.1 La réciproque de (i) n'est pas valable en toute
généralité comme le montre l'exemple de la série
\\sum  (-1)^n~
et de \phi(n) = 2n. On a alors y_n = 0, la série
\\sum  y_n~
converge alors que la série
\\sum  x_n~ est
divergente. La réciproque (ii.b) est particulièrement intéressante pour
le cas de séries de nombres réels qui ne sont pas de signe constant~; en
regroupant ensemble les termes consécutifs de même signe, on obtient une
série de même nature que la série initiale et dont les termes sont
alternés en signe.

\paragraph{7.6.3 Modification de l'ordre des termes}

Nous allons ici étudier l'effet d'une permutation sur les termes d'une
série convergente. Pour cela nous aurons besoin du lemme suivant.

Théorème~7.6.4 Soit \\\sum
 x_n une série à termes réels ou complexes absolument
convergente et soit \sigma : \mathbb{N}~ \rightarrow~ \mathbb{N}~ bijective, une permutation de \mathbb{N}~. Alors la
série \\sum ~
x_\sigma(n) est absolument convergente et
\\sum ~
_n=0^+\infty~x_\sigma(n) =\
\sum  _n=0^+\infty~x_n~.

Démonstration Premier cas~: série à termes réels positifs. Pour n \in \mathbb{N}~,
soit N_n le plus grand élément de \sigma([0,n]). On a alors

\sum _k=0^nx_ \sigma(k)~
\leq\sum _p=0^N_n~
x_p \leq\\sum
_p=0^+\infty~x_ p

ce qui montre que la série à termes réels positifs
\\sum  x_\sigma(k)~
converge et que \\sum ~
_n=0^+\infty~x_\sigma(n)
\leq\\sum ~
_n=0^+\infty~x_n. Mais les deux séries jouent un rôle
symétrique puisque x_n = x_\sigma^-1(\sigma(n)), et
donc on a aussi \\sum ~
_n=0^+\infty~x_n
\leq\\sum ~
_n=0^+\infty~x_\sigma(n) ce qui nous donne l'égalité.

Deuxième cas~: séries à termes réels On introduit, comme d'habitude,
pour x \in \mathbb{R}~, x^+ = max~(x,0) \in
\mathbb{R}~^+ et x^- = max~(-x,0) \in
\mathbb{R}~^+ si bien que x = x^+ - x^-,
x = x^+ + x^-. On a alors 0 \leq
x_n^+ \leqx_n et 0 \leq
x_n^-\leqx_n, ce qui montre
que les deux séries à termes positifs
\\sum ~
x_n^+ et
\\sum ~
x_n^- sont convergentes. D'après le premier cas de la
démonstration, les deux séries
\\sum ~
x_\sigma(n)^+ et
\\sum ~
x_\sigma(n)^- sont convergentes et on a

\sum _n=0^+\infty~x_
\sigma(n)^+ = \\sum
_n=0^+\infty~x_ n^+,\quad
\sum _n=0^+\infty~x_
\sigma(n)^- = \\sum
_n=0^+\infty~x_ n^-

Comme x_\sigma(n) = x_\sigma(n)^+ +
x_\sigma(n)^-, la série
\\sum ~
x_\sigma(n) converge, donc la série
\\sum  x_\sigma(n)~
est absolument convergente, et comme x_\sigma(n) =
x_\sigma(n)^+ - x_\sigma(n)^-, on a

\sum _n=0^+\infty~x_ \sigma(n)~ =
\sum _n=0^+\infty~x_
\sigma(n)^+-\\sum
_n=0^+\infty~x_ \sigma(n)^- =
\sum _n=0^+\infty~x_
n^+-\\sum
_n=0^+\infty~x_ n^- =
\sum _n=0^+\infty~x_ n~

Troisième cas~: séries à termes complexes On travaille de la même
fa\ccon avec les parties réelles et parties
imaginaires. On a 0
\leq\mathrmRe(x_n)\leqx_n~
et 0
\leq\mathrmIm(x_n)\leqx_n~,
ce qui montre que les deux séries
\\sum ~
\mathrmRe(x_n~) et
\\sum ~
\mathrmIm(x_n~)
sont absolument convergentes. D'après le deuxième cas de la
démonstration, les deux séries
\\sum ~
\mathrmRe(x_\sigma(n)~)
et \\sum ~
\mathrmIm(x_\sigma(n)~)
sont absolument convergentes et on a

\\sum
_n=0^+\infty~\mathrmRe(x_ \sigma(n))
= \\sum
_n=0^+\infty~\mathrmRe(x_
n),\quad \\sum
_n=0^+\infty~\mathrmIm(x_ \sigma(n))
= \\sum
_n=0^+\infty~\mathrmIm(x_ n)

Comme
x_\sigma(n)\leq\mathrmRe(x_\sigma(n)~)
+
\mathrmIm(x_\sigma(n)~),
la série \\sum ~
x_\sigma(n) converge, donc la série
\\sum  x_\sigma(n)~
est absolument convergente, et comme x_\sigma(n)
=\
\mathrmRe(x_\sigma(n)) +
i\mathrmRe(x_\sigma(n)~),
on a

\sum _n=0^+\infty~x_ \sigma(n)~ =
\\sum
_n=0^+\infty~\mathrmRe(x_
\sigma(n))+i\\sum
_n=0^+\infty~\mathrmRe(x_ \sigma(n))
= \\sum
_n=0^+\infty~\mathrmRe(x_
n)+i\\sum
_n=0^+\infty~\mathrmIm(x_ n) =
\sum _n=0^+\infty~x_ n~

Remarque~7.6.2 La condition de convergence absolue est indispensable à
la validité du théorème. Considérons la série semi convergente
\\sum  x_n~ avec
x_n = (-1)^n-1 \over n et soit S
sa somme (on peut montrer que S = log~ 2). Soit
\phi : \mathbb{N}~^∗\rightarrow~ \mathbb{N}~^∗ définie par \phi(3k + 1) = 2k + 1, \phi(3k
+ 2) = 4k + 2 et \phi(3k + 3) = 4k + 4. On vérifie facilement que \phi est une
bijection de \mathbb{N}~ dans \mathbb{N}~ (la bijection réciproque est définie par des
congruences modulo 4). Sommons alors par paquets de 3 la série
\\sum  x_\phi(n)~.
On a

\begin{align*} x_\phi(3k+1) +
x_\phi(3k+2) + x_\phi(3k+3)&& \%&
\\ & =& 1 \over 2k +
1 - 1 \over 4k + 2 - 1 \over 4k +
4 = 1 \over 4k + 2 - 1 \over 4k +
4 \%& \\ & =& 1
\over 2 \left (x_2k+1 +
x_2k+2\right ) \%&
\\ \end{align*}

Ceci montre (réciproque du théorème de sommation par paquets, la taille
des paquets étant bornée et le terme général tendant vers 0) que la
nouvelle série converge encore, mais que sa somme est la moitié de la
somme de la série initiale.

\paragraph{7.6.4 Produit de Cauchy}

Définition~7.6.1 Soit \\\sum
 a_n et \\\sum
 b_n deux séries à termes réels ou complexes. On appelle
produit de Cauchy (ou produit de convolution) des deux séries, la série
\\sum  c_n~ avec

\forall~n \in \mathbb{N}~, c_n~ =
\sum _k=0^na_
kb_n-k = \\sum
_p+q=na_pb_q

Théorème~7.6.5 Soit \\\sum
 a_n et \\\sum
 b_n deux séries à termes réels ou complexes, absolument
convergentes. Alors leur produit de Cauchy
\\sum  c_n~ est
une série absolument convergente et on a

\sum _n=0^+\infty~c_ n~ =
\left (\\sum
_n=0^+\infty~a_ n\right
)\left (\\sum
_n=0^+\infty~b_ n\right )

Démonstration Cas particulier~: les deux séries sont à termes réels
positifs. Notons K_n = [0,n] \times [0,n] \subset~ \mathbb{N}~^2
et T_n = \(p,q) \in
\mathbb{N}~^2∣p + q \leq n\.
On a évidemment T_n \subset~ K_n \subset~ T_2n. On a alors

\begin{align*} \\sum
_k=0^nc_ k& =& \\sum
_k=0^n \\sum
_p+q=ka_pb_q = \\sum
_(p,q)\inT_na_pb_q
\leq\\sum
_(p,q)\inK_na_pb_q\%&
\\ & =& \\sum
_p=0^na_ p \\sum
_q=0^nb_ q \leq\\sum
_p=0^+\infty~a_ p \\sum
_q=0^+\infty~b_ q \%&
\\ \end{align*}

La série \\sum ~
c_n est une série à termes réels positifs dont les sommes
partielles sont majorées, donc elle converge. De plus les inclusions
T_n \subset~ K_n \subset~ T_2n se traduisent par
S_n(c) \leq S_n(a)S_n(b) \leq S_2n(c) et
en faisant tendre n vers + \infty~, on obtient S(c) = S(a)S(b) ce qui est la
formule souhaitée.

Cas général Posons a_n' = a_n,
b_n' = b_n et c_n'
= \\sum ~
_p+q=na_pb_q
leur produit de Cauchy, et désignons par
S_n(a'),S_n(b') et S_n(c') les sommes
partielles d'indice n de ces trois séries. Puisque les séries
\\sum  a_n~' et
\\sum  b_n~'
sont convergentes, le cas particulier ci dessus montre que la série
\\sum  c_n~' est
convergente et que sa somme est le produit des sommes de ces deux
séries. Mais, comme c_n\leq c_n', on
en déduit la convergence absolue de la série
\\sum  c_n~. On
a alors

\begin{align*} \left
S_n(a)S_n(b) -
S_n(c)\right & =&
\left \\sum
_(p,q)\inK_na_pb_q
-\\sum
_(p,q)\inT_na_pb_q\right
 = \left \\sum
_(p,q)\inK_n\diagdownT_na_pb_q\right
 \%& \\ & \leq&
\\sum
_(p,q)\inK_n\diagdownT_na_pb_q
= \\sum
_(p,q)\inK_na_pb_q-\\sum
_(p,q)\inT_na_pb_q
= S_n(a')S_n(b') -
S_n(c')\%&\\
\end{align*}

Puisque la somme de la série
\\sum  c_n~' est
le produit des sommes des deux séries
\\sum  a_n~' et
\\sum  b_n~', on
a
lim_n\rightarrow~+\infty~(S_n(a')S_n~(b')
- S_n(c')) = 0 et donc par la majoration ci-dessus
lim_n\rightarrow~+\infty~(S_n(a)S_n~(b)
- S_n(c)) = 0, ce qui montre que la somme de la série
\\sum  c_n~ est
le produit des sommes des deux séries
\\sum  a_n~ et
\\sum  b_n~ et
achève la démonstration.

Remarque~7.6.3 On aurait pu passer aussi du cas réel positif au cas
complexe en utilisant, comme dans le théorème de permutation des termes,
les parties positives x^+ et x^- d'un réel x, puis
les parties réelle et imaginaire d'un nombre complexe, mais la
démonstration n'aurait pas pu se généraliser comme nous le ferons
ci-dessous au cas d'une application bilinéaire plus générale.

Remarque~7.6.4 Le théorème ci dessus n'est plus valable pour des séries
convergentes~: posons a_n = b_n = (-1)^n
\over \sqrtn+1 . On a
c_n =\
\sum  _k=0^n~ 1
\over \sqrt(k+1)(n-k+1) . Mais pour
k \in [0,n], (k + 1)(n - k + 1) \leq ( n \over 2 +
1)^2 (facile). Donc c_n≥ n+1
\over  n \over 2 +1 qui tend vers
2~; donc la suite (c_n) ne tend pas vers 0 et la série
\\sum  c_n~
diverge.

On a une généralisation du théorème précédent sous la forme suivante qui
nous sera utile quand nous considérerons des séries d'endomorphismes.

Théorème~7.6.6 Soit E, F et G sont trois espaces vectoriels normés, u :
E \times F \rightarrow~ G une application bilinéaire continue,
\\sum  a_n~ une
série à termes dans E absolument convergente,
\\sum  b_n~ une
série à termes dans F absolument convergente, et si l'on pose
c_n = \\sum ~
_p+q=nu(a_p,b_q), alors la série
\\sum  c_n~ est
absolument convergente et on a

\sum _n=0^+\infty~c_ n~ =
u\left (\\sum
_n=0^+\infty~a_ n,\\sum
_n=0^+\infty~b_ n\right )

Démonstration La démonstration est tout à fait similaire~: utiliser
l'existence d'un réel positif K tel que
\u(x,y)\ \leq
K\x\
\y\ pour montrer que
\left S_n(a)S_n(b) -
S_n(c)\right \leq K\left
(S_n(a')S_n(b') -
S_n(c')\right ) en posant a_n'
=\ a_n\,
b_n' =\
b_n\ et c_n'
= \\sum ~
_p+q=n\a_p\\b_q\

[
[
[
[

\end{document}
