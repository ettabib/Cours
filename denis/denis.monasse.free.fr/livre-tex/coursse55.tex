\documentclass[]{article}
\usepackage[T1]{fontenc}
\usepackage{lmodern}
\usepackage{amssymb,amsmath}
\usepackage{ifxetex,ifluatex}
\usepackage{fixltx2e} % provides \textsubscript
% use upquote if available, for straight quotes in verbatim environments
\IfFileExists{upquote.sty}{\usepackage{upquote}}{}
\ifnum 0\ifxetex 1\fi\ifluatex 1\fi=0 % if pdftex
  \usepackage[utf8]{inputenc}
\else % if luatex or xelatex
  \ifxetex
    \usepackage{mathspec}
    \usepackage{xltxtra,xunicode}
  \else
    \usepackage{fontspec}
  \fi
  \defaultfontfeatures{Mapping=tex-text,Scale=MatchLowercase}
  \newcommand{\euro}{€}
\fi
% use microtype if available
\IfFileExists{microtype.sty}{\usepackage{microtype}}{}
\ifxetex
  \usepackage[setpagesize=false, % page size defined by xetex
              unicode=false, % unicode breaks when used with xetex
              xetex]{hyperref}
\else
  \usepackage[unicode=true]{hyperref}
\fi
\hypersetup{breaklinks=true,
            bookmarks=true,
            pdfauthor={},
            pdftitle={Integration sur un intervalle quelconque : fonctions `a valeurs complexes},
            colorlinks=true,
            citecolor=blue,
            urlcolor=blue,
            linkcolor=magenta,
            pdfborder={0 0 0}}
\urlstyle{same}  % don't use monospace font for urls
\setlength{\parindent}{0pt}
\setlength{\parskip}{6pt plus 2pt minus 1pt}
\setlength{\emergencystretch}{3em}  % prevent overfull lines
\setcounter{secnumdepth}{0}
 
/* start css.sty */
.cmr-5{font-size:50%;}
.cmr-7{font-size:70%;}
.cmmi-5{font-size:50%;font-style: italic;}
.cmmi-7{font-size:70%;font-style: italic;}
.cmmi-10{font-style: italic;}
.cmsy-5{font-size:50%;}
.cmsy-7{font-size:70%;}
.cmex-7{font-size:70%;}
.cmex-7x-x-71{font-size:49%;}
.msbm-7{font-size:70%;}
.cmtt-10{font-family: monospace;}
.cmti-10{ font-style: italic;}
.cmbx-10{ font-weight: bold;}
.cmr-17x-x-120{font-size:204%;}
.cmsl-10{font-style: oblique;}
.cmti-7x-x-71{font-size:49%; font-style: italic;}
.cmbxti-10{ font-weight: bold; font-style: italic;}
p.noindent { text-indent: 0em }
td p.noindent { text-indent: 0em; margin-top:0em; }
p.nopar { text-indent: 0em; }
p.indent{ text-indent: 1.5em }
@media print {div.crosslinks {visibility:hidden;}}
a img { border-top: 0; border-left: 0; border-right: 0; }
center { margin-top:1em; margin-bottom:1em; }
td center { margin-top:0em; margin-bottom:0em; }
.Canvas { position:relative; }
li p.indent { text-indent: 0em }
.enumerate1 {list-style-type:decimal;}
.enumerate2 {list-style-type:lower-alpha;}
.enumerate3 {list-style-type:lower-roman;}
.enumerate4 {list-style-type:upper-alpha;}
div.newtheorem { margin-bottom: 2em; margin-top: 2em;}
.obeylines-h,.obeylines-v {white-space: nowrap; }
div.obeylines-v p { margin-top:0; margin-bottom:0; }
.overline{ text-decoration:overline; }
.overline img{ border-top: 1px solid black; }
td.displaylines {text-align:center; white-space:nowrap;}
.centerline {text-align:center;}
.rightline {text-align:right;}
div.verbatim {font-family: monospace; white-space: nowrap; text-align:left; clear:both; }
.fbox {padding-left:3.0pt; padding-right:3.0pt; text-indent:0pt; border:solid black 0.4pt; }
div.fbox {display:table}
div.center div.fbox {text-align:center; clear:both; padding-left:3.0pt; padding-right:3.0pt; text-indent:0pt; border:solid black 0.4pt; }
div.minipage{width:100%;}
div.center, div.center div.center {text-align: center; margin-left:1em; margin-right:1em;}
div.center div {text-align: left;}
div.flushright, div.flushright div.flushright {text-align: right;}
div.flushright div {text-align: left;}
div.flushleft {text-align: left;}
.underline{ text-decoration:underline; }
.underline img{ border-bottom: 1px solid black; margin-bottom:1pt; }
.framebox-c, .framebox-l, .framebox-r { padding-left:3.0pt; padding-right:3.0pt; text-indent:0pt; border:solid black 0.4pt; }
.framebox-c {text-align:center;}
.framebox-l {text-align:left;}
.framebox-r {text-align:right;}
span.thank-mark{ vertical-align: super }
span.footnote-mark sup.textsuperscript, span.footnote-mark a sup.textsuperscript{ font-size:80%; }
div.tabular, div.center div.tabular {text-align: center; margin-top:0.5em; margin-bottom:0.5em; }
table.tabular td p{margin-top:0em;}
table.tabular {margin-left: auto; margin-right: auto;}
div.td00{ margin-left:0pt; margin-right:0pt; }
div.td01{ margin-left:0pt; margin-right:5pt; }
div.td10{ margin-left:5pt; margin-right:0pt; }
div.td11{ margin-left:5pt; margin-right:5pt; }
table[rules] {border-left:solid black 0.4pt; border-right:solid black 0.4pt; }
td.td00{ padding-left:0pt; padding-right:0pt; }
td.td01{ padding-left:0pt; padding-right:5pt; }
td.td10{ padding-left:5pt; padding-right:0pt; }
td.td11{ padding-left:5pt; padding-right:5pt; }
table[rules] {border-left:solid black 0.4pt; border-right:solid black 0.4pt; }
.hline hr, .cline hr{ height : 1px; margin:0px; }
.tabbing-right {text-align:right;}
span.TEX {letter-spacing: -0.125em; }
span.TEX span.E{ position:relative;top:0.5ex;left:-0.0417em;}
a span.TEX span.E {text-decoration: none; }
span.LATEX span.A{ position:relative; top:-0.5ex; left:-0.4em; font-size:85%;}
span.LATEX span.TEX{ position:relative; left: -0.4em; }
div.float img, div.float .caption {text-align:center;}
div.figure img, div.figure .caption {text-align:center;}
.marginpar {width:20%; float:right; text-align:left; margin-left:auto; margin-top:0.5em; font-size:85%; text-decoration:underline;}
.marginpar p{margin-top:0.4em; margin-bottom:0.4em;}
.equation td{text-align:center; vertical-align:middle; }
td.eq-no{ width:5%; }
table.equation { width:100%; } 
div.math-display, div.par-math-display{text-align:center;}
math .texttt { font-family: monospace; }
math .textit { font-style: italic; }
math .textsl { font-style: oblique; }
math .textsf { font-family: sans-serif; }
math .textbf { font-weight: bold; }
.partToc a, .partToc, .likepartToc a, .likepartToc {line-height: 200%; font-weight:bold; font-size:110%;}
.chapterToc a, .chapterToc, .likechapterToc a, .likechapterToc, .appendixToc a, .appendixToc {line-height: 200%; font-weight:bold;}
.index-item, .index-subitem, .index-subsubitem {display:block}
.caption td.id{font-weight: bold; white-space: nowrap; }
table.caption {text-align:center;}
h1.partHead{text-align: center}
p.bibitem { text-indent: -2em; margin-left: 2em; margin-top:0.6em; margin-bottom:0.6em; }
p.bibitem-p { text-indent: 0em; margin-left: 2em; margin-top:0.6em; margin-bottom:0.6em; }
.paragraphHead, .likeparagraphHead { margin-top:2em; font-weight: bold;}
.subparagraphHead, .likesubparagraphHead { font-weight: bold;}
.quote {margin-bottom:0.25em; margin-top:0.25em; margin-left:1em; margin-right:1em; text-align:justify;}
.verse{white-space:nowrap; margin-left:2em}
div.maketitle {text-align:center;}
h2.titleHead{text-align:center;}
div.maketitle{ margin-bottom: 2em; }
div.author, div.date {text-align:center;}
div.thanks{text-align:left; margin-left:10%; font-size:85%; font-style:italic; }
div.author{white-space: nowrap;}
.quotation {margin-bottom:0.25em; margin-top:0.25em; margin-left:1em; }
h1.partHead{text-align: center}
.sectionToc, .likesectionToc {margin-left:2em;}
.subsectionToc, .likesubsectionToc {margin-left:4em;}
.subsubsectionToc, .likesubsubsectionToc {margin-left:6em;}
.frenchb-nbsp{font-size:75%;}
.frenchb-thinspace{font-size:75%;}
.figure img.graphics {margin-left:10%;}
/* end css.sty */

\title{Integration sur un intervalle quelconque : fonctions `a valeurs
complexes}
\author{}
\date{}

\begin{document}
\maketitle

\textbf{Warning: \href{http://www.math.union.edu/locate/jsMath}{jsMath}
requires JavaScript to process the mathematics on this page.\\ If your
browser supports JavaScript, be sure it is enabled.}

\begin{center}\rule{3in}{0.4pt}\end{center}

{[}\href{coursse56.html}{next}{]} {[}\href{coursse54.html}{prev}{]}
{[}\href{coursse54.html\#tailcoursse54.html}{prev-tail}{]}
{[}\hyperref[tailcoursse55.html]{tail}{]}
{[}\href{coursch10.html\#coursse55.html}{up}{]}

\subsubsection{9.6 Intégration sur un intervalle quelconque~: fonctions
à valeurs complexes}

\paragraph{9.6.1 Fonctions à valeurs complexes intégrables}

Définition~9.6.1 Soit I un intervalle de ℝ et f : I → ℂ continue par
morceaux. On dit que f est intégrable sur I si la fonction à valeurs
réelles positives \textbar{}f\textbar{} est intégrable sur I.

Théorème~9.6.1 Soit f : I → ℂ continue par morceaux et intégrable. Alors
pour toute suite (\{J\}\_\{n\}) croissante de segments contenus dans I
de réunion I, la suite \{(\{\textbackslash{}mathop\{∫ \}
\}\_\{\{J\}\_\{n\}\}f)\}\_\{n∈ℕ\} est convergente. Sa limite est
indépendante de la suite (\{J\}\_\{n\}) et notée
\{\textbackslash{}mathop\{∫ \} \}\_\{I\}f.

Démonstration Soit q \textgreater{} p. On a

\textbackslash{}left \textbar{}\{\textbackslash{}mathop\{∫ \}
\}\_\{\{J\}\_\{q\}\}f −\{\textbackslash{}mathop\{∫ \}
\}\_\{\{J\}\_\{p\}\}f\textbackslash{}right \textbar{} =
\textbackslash{}left \textbar{}\{\textbackslash{}mathop\{∫ \}
\}\_\{\{J\}\_\{q\}∖\{J\}\_\{p\}\}f\textbackslash{}right
\textbar{}≤\{\textbackslash{}mathop\{∫ \}
\}\_\{\{J\}\_\{q\}∖\{J\}\_\{p\}\}\textbar{}f\textbar{}
=\{\textbackslash{}mathop\{∫ \}
\}\_\{\{J\}\_\{q\}\}\textbar{}f\textbar{}−\{\textbackslash{}mathop\{∫ \}
\}\_\{\{J\}\_\{p\}\}\textbar{}f\textbar{}

(avec un tout petit abus d'écriture en notant \{J\}\_\{q\} ∖
\{J\}\_\{p\} = {[}\{a\}\_\{q\},\{a\}\_\{p\}{]} ∪
{[}\{b\}\_\{p\},\{b\}\_\{q\}{]} la réunion de deux segments disjoints ).
Comme \textbar{}f\textbar{} est intégrable, la suite
(\{\textbackslash{}mathop\{∫ \}
\}\_\{\{J\}\_\{n\}\}\textbar{}f\textbar{}) converge, donc c'est une
suite de Cauchy, et par conséquent il en est de même de la suite
(\{\textbackslash{}mathop\{∫ \} \}\_\{\{J\}\_\{n\}\}f) qui est donc
convergente. Si (\{K\}\_\{n\}) est une autre suite de segments vérifiant
les mêmes propriétés, deux cas se présentent. Si
\textbackslash{}mathop\{∀\}n, \{J\}\_\{n\} ⊂ \{K\}\_\{n\}, alors

\textbackslash{}left \textbar{}\{\textbackslash{}mathop\{∫ \}
\}\_\{\{K\}\_\{n\}\}f −\{\textbackslash{}mathop\{∫ \}
\}\_\{\{J\}\_\{n\}\}f\textbackslash{}right \textbar{} =
\textbackslash{}left \textbar{}\{\textbackslash{}mathop\{∫ \}
\}\_\{\{K\}\_\{n\}∖\{J\}\_\{n\}\}f\textbackslash{}right
\textbar{}≤\{\textbackslash{}mathop\{∫ \}
\}\_\{\{K\}\_\{n\}∖\{J\}\_\{n\}\}\textbar{}f\textbar{}
=\{\textbackslash{}mathop\{∫ \}
\}\_\{\{K\}\_\{n\}\}\textbar{}f\textbar{}−\{\textbackslash{}mathop\{∫ \}
\}\_\{\{J\}\_\{n\}\}\textbar{}f\textbar{}

Mais les deux suites (\{\textbackslash{}mathop\{∫ \}
\}\_\{\{J\}\_\{n\}\}\textbar{}f\textbar{}) et
(\{\textbackslash{}mathop\{∫ \}
\}\_\{\{K\}\_\{n\}\}\textbar{}f\textbar{}) ont la même limite et donc
leur différence tend vers 0. Il en est donc de même de la différence des
suites (\{\textbackslash{}mathop\{∫ \} \}\_\{\{J\}\_\{n\}\}f) et
(\{\textbackslash{}mathop\{∫ \} \}\_\{\{K\}\_\{n\}\}f), qui, étant
convergentes, ont donc la même limite. Si \{J\}\_\{n\} n'est pas
forcément inclus dans \{K\}\_\{n\} il suffit d'écrire

\textbackslash{}mathop\{lim\}\{\textbackslash{}mathop\{∫ \}
\}\_\{\{J\}\_\{n\}\}f =\textbackslash{}mathop\{
lim\}\{\textbackslash{}mathop\{∫ \} \}\_\{\{J\}\_\{n\}∪\{K\}\_\{n\}\}f
=\textbackslash{}mathop\{ lim\}\{\textbackslash{}mathop\{∫ \}
\}\_\{\{K\}\_\{n\}\}f

Corollaire~9.6.2 Soit f : I → ℂ continue par morceaux et intégrable.
Alors \textbackslash{}left \textbar{}\{\textbackslash{}mathop\{∫ \}
\}\_\{I\}f\textbackslash{}right \textbar{}≤\{\textbackslash{}mathop\{∫
\} \}\_\{I\}\textbar{}f\textbar{}.

Démonstration Il suffit de passer à la limite à partir de l'inégalité
\textbackslash{}left \textbar{}\{\textbackslash{}mathop\{∫ \}
\}\_\{\{J\}\_\{n\}\}f\textbackslash{}right \textbar{}
≤\{\textbackslash{}mathop\{∫ \}
\}\_\{\{J\}\_\{n\}\}\textbar{}f\textbar{}.

Théorème~9.6.3 Soit −∞ \textless{} a \textless{} b ≤ +∞ et f :
{[}a,b{[}→ ℂ intégrable. Alors la fonction
x\textbackslash{}mathrel\{↦\}\{\textbackslash{}mathop\{∫ \}
\}\_\{a\}\^{}\{x\}f(t) dt admet la limite \{\textbackslash{}mathop\{∫ \}
\}\_\{{[}a,b{[}\}f au point b.

Démonstration Soit a \textless{} x \textless{} y \textless{} b~; on a

\textbackslash{}left \textbar{}\{\textbackslash{}mathop\{∫ \}
\}\_\{a\}\^{}\{y\}f −\{\textbackslash{}mathop\{∫ \}
\}\_\{a\}\^{}\{x\}f\textbackslash{}right \textbar{} =
\textbackslash{}left \textbar{}\{\textbackslash{}mathop\{∫ \}
\}\_\{x\}\^{}\{y\}f\textbackslash{}right
\textbar{}≤\{\textbackslash{}mathop\{∫ \}
\}\_\{x\}\^{}\{y\}\textbar{}f\textbar{} =\{\textbackslash{}mathop\{∫ \}
\}\_\{a\}\^{}\{y\}\textbar{}f\textbar{}−\{\textbackslash{}mathop\{∫ \}
\}\_\{a\}\^{}\{x\}\textbar{}f\textbar{}

Comme l'application
x\textbackslash{}mathrel\{↦\}\{\textbackslash{}mathop\{∫ \}
\}\_\{a\}\^{}\{x\}\textbar{}f\textbar{} admet une limite au point b,
elle vérifie le critère de Cauchy~: pour tout ε \textgreater{} 0, il
existe c ∈ {[}a,b{[} tel que c \textless{} x \textless{} y \textless{} b
⇒\textbackslash{}left \textbar{}\{\textbackslash{}mathop\{∫ \}
\}\_\{a\}\^{}\{y\}\textbar{}f\textbar{}−\{\textbackslash{}mathop\{∫ \}
\}\_\{a\}\^{}\{x\}\textbar{}f\textbar{}\textbackslash{}right \textbar{}
\textless{} ε~; alors l'inégalité ci dessus montre que
x\textbackslash{}mathrel\{↦\}\{\textbackslash{}mathop\{∫ \}
\}\_\{a\}\^{}\{x\}f(t) dt vérifie également ce critère de Cauchy, donc
admet une limite au point b. Soit alors \{b\}\_\{n\} une suite
croissante de limite b. On a

\{\textbackslash{}mathop\{lim\}\}\_\{x→b\}\{\textbackslash{}mathop\{∫ \}
\}\_\{a\}\^{}\{x\}f(t) dt =\{\textbackslash{}mathop\{ lim\}\}\_\{
n→+∞\}\{\textbackslash{}mathop\{∫ \} \}\_\{a\}\^{}\{\{b\}\_\{n\} \}f(t)
dt =\{\textbackslash{}mathop\{
lim\}\}\_\{n→+∞\}\{\textbackslash{}mathop\{∫ \}
\}\_\{{[}a,\{b\}\_\{n\}{]}\}f =\{\textbackslash{}mathop\{∫ \}
\}\_\{{[}a,b{[}\}f

Proposition~9.6.4 Soit I un intervalle de ℝ, f : I → ℂ continue par
morceaux, intégrable sur I. Alors f est intégrable sur tout intervalle
I' inclus dans I.

Démonstration En effet l'intégrabilité de f équivaut à celle de
\textbar{}f\textbar{}.

Proposition~9.6.5 Soit f : I → ℂ et φ : I → \{ℝ\}\^{}\{+\} continues par
morceaux telles que 0 ≤\textbar{}f\textbar{}≤ φ. Si φ est intégrable sur
I il en est de même de f et \textbackslash{}left
\textbar{}\{\textbackslash{}mathop\{∫ \} \}\_\{I\}f\textbackslash{}right
\textbar{}≤\{\textbackslash{}mathop\{∫ \} \}\_\{I\}φ.

Démonstration Evident d'après les définitions.

Corollaire~9.6.6 Soit I un intervalle borné de ℝ et soit f : I → ℂ
continue par morceaux et bornée. Alors f est intégrable sur I.

Démonstration Appliquer la proposition précédente avec φ constante
majorant \textbar{}f\textbar{}.

Proposition~9.6.7 Soit I = {[}a,b{]} un segment de ℝ, f : I → ℂ continue
par morceaux. Alors f est intégrable sur I et
\{\textbackslash{}mathop\{∫ \} \}\_\{I\}f =\{\textbackslash{}mathop\{∫
\} \}\_\{a\}\^{}\{b\}f. De plus f est intégrable sur {]}a,b{[},
{[}a,b{[} et {]}a,b{]}, toutes ces intégrales étant égales.

Démonstration La fonction \textbar{}f\textbar{} est positive et continue
par morceaux, donc intégrable sur {[}a,b{]}. Donc f l'est également. On
sait alors que \textbar{}f\textbar{} est intégrable sur tout intervalle
inclus dans I et en particulier sur {]}a,b{[}, {[}a,b{[} et {]}a,b{]}~;
il en est donc de même pour f. De plus, si \{a\}\_\{n\} = a +\{ 1
\textbackslash{}over n\} et \{b\}\_\{n\} = b −\{ 1 \textbackslash{}over
n\} , \{J\}\_\{n\} = {[}\{a\}\_\{n\},\{b\}\_\{n\}{]} est une suite
croissante de segments dont la réunion est {]}a,b{[}, donc

\{\textbackslash{}mathop\{∫ \} \}\_\{{]}a,b{[}\}f
=\textbackslash{}mathop\{ lim\}\{\textbackslash{}mathop\{∫ \}
\}\_\{\{a\}\_\{n\}\}\^{}\{\{b\}\_\{n\} \}f =\{\textbackslash{}mathop\{∫
\} \}\_\{a\}\^{}\{b\}f

par continuité de l'intégrale par rapport à ses bornes. On fait une
démonstration similaire pour {[}a,b{[} avec {[}a,\{b\}\_\{n\}{]} et
{]}a,b{]} avec {[}\{a\}\_\{n\},b{]}. Pour {[}a,b{]}, on prend
\{a\}\_\{n\} = a et \{b\}\_\{n\} = b.

Théorème~9.6.8 Soit f,g : I → ℂ continues par morceaux, soit α,β ∈ ℂ. Si
f et g sont intégrables sur I, il en est de même de αf + βg et on a

\{\textbackslash{}mathop\{∫ \} \}\_\{I\}(αf + βg) =
α\{\textbackslash{}mathop\{∫ \} \}\_\{I\}f +
β\{\textbackslash{}mathop\{∫ \} \}\_\{I\}g

Autrement dit, l'ensemble des applications de I dans ℂ qui sont
intégrables sur I est un sous-espace vectoriel de l'espace vectoriel des
applications de I dans ℂ et l'application
f\textbackslash{}mathrel\{↦\}\{\textbackslash{}mathop\{∫ \} \}\_\{I\}f
est linéaire.

Démonstration L'intégrabilité est évidente à partir de l'inégalité
\textbar{}αf + βg\textbar{}≤\textbar{}α\textbar{}\textbar{}f\textbar{} +
\textbar{}β\textbar{}\textbar{}g\textbar{} et du fait que
\textbar{}f\textbar{} et \textbar{}g\textbar{} étant intégrables, il en
est de même de \textbar{}α\textbar{}\textbar{}f\textbar{} +
\textbar{}β\textbar{}\textbar{}g\textbar{}. Pour les égalités, il suffit
de prendre une suite (\{J\}\_\{n\}) croissante de segments de réunion I
et de passer à la limite dans les formules

\{\textbackslash{}mathop\{∫ \} \}\_\{\{J\}\_\{n\}\}(αf + βg) =
α\{\textbackslash{}mathop\{∫ \} \}\_\{\{J\}\_\{n\}\}f +
β\{\textbackslash{}mathop\{∫ \} \}\_\{\{J\}\_\{n\}\}g

Proposition~9.6.9 Soit I un intervalle de ℝ, f : I → ℝ continue par
morceaux. Soit a ∈ \{I\}\^{}\{o\}. Alors f est intégrable sur I si et
seulement si elle est intégrable sur I∩{]} −∞,a{]} et sur I ∩
{[}a,+∞{[}. Dans ce cas,

\{\textbackslash{}mathop\{∫ \} \}\_\{I\}f =\{\textbackslash{}mathop\{∫
\} \}\_\{I∩{]}−∞,a{]}\}f +\{\textbackslash{}mathop\{∫ \}
\}\_\{I∩{[}a,+∞{[}\}f

Démonstration Le résultat similaire déjà démontré pour
\textbar{}f\textbar{} démontre l'équivalence entre les diverses
intégrabilités. Soit alors \{J\}\_\{n\} =
{[}\{a\}\_\{n\},\{b\}\_\{n\}{]} une suite croissante de segments de
réunion I. Pour n assez grand, on a \{a\}\_\{n\} ≤ a ≤ \{b\}\_\{n\} car
a est dans l'intérieur de I. Mais ({[}\{a\}\_\{n\},a{]}) est une suite
croissante de segments de réunion I∩{]} −∞,a{]} et
({[}a,\{b\}\_\{n\}{]}) est une suite croissante de segments de réunion I
∩ {[}a,+∞{[}. On peut donc passer à la limite dans la formule
\{\textbackslash{}mathop\{∫ \} \}\_\{{[}\{a\}\_\{n\},\{b\}\_\{n\}{]}\}f
=\{\textbackslash{}mathop\{∫ \} \}\_\{{[}\{a\}\_\{n\},a{]}\}f
+\{\textbackslash{}mathop\{∫ \} \}\_\{{[}a,\{b\}\_\{n\}{]}\}f, et on
obtient

\{\textbackslash{}mathop\{∫ \} \}\_\{I\}f =\{\textbackslash{}mathop\{∫
\} \}\_\{I∩{]}−∞,a{]}\}f +\{\textbackslash{}mathop\{∫ \}
\}\_\{I∩{[}a,+∞{[}\}

\paragraph{9.6.2 Décomposition des fonctions à valeurs complexes}

Soit x ∈ ℝ. On pose \{x\}\^{}\{+\} =\textbackslash{}mathop\{ max\}(x,0)
et \{x\}\^{}\{−\} =\textbackslash{}mathop\{ max\}(−x,0). On a
\{x\}\^{}\{+\},\{x\}\^{}\{−\}∈ \{ℝ\}\^{}\{+\}, x = \{x\}\^{}\{+\} −
\{x\}\^{}\{−\}, \textbar{}x\textbar{} = \{x\}\^{}\{+\} + \{x\}\^{}\{−\},
\{x\}\^{}\{+\} =\{ 1 \textbackslash{}over 2\} (\textbar{}x\textbar{} +
x) et \{x\}\^{}\{−\} =\{ 1 \textbackslash{}over 2\}
(\textbar{}x\textbar{}− x).

Remarque~9.6.1 Si f : I → ℝ, on peut ainsi lui associer des fonctions
\{f\}\^{}\{+\} et \{f\}\^{}\{−\} à valeurs dans \{ℝ\}\^{}\{+\}. On a
\{f\}\^{}\{+\},\{f\}\^{}\{−\}∈ \{ℝ\}\^{}\{+\}, f = \{f\}\^{}\{+\} −
\{f\}\^{}\{−\}, \textbar{}f\textbar{} = \{f\}\^{}\{+\} + \{f\}\^{}\{−\},
\{f\}\^{}\{+\} =\{ 1 \textbackslash{}over 2\} (\textbar{}f\textbar{} +
f) et \{f\}\^{}\{−\} =\{ 1 \textbackslash{}over 2\}
(\textbar{}f\textbar{}− f). Ces deux dernières formules montrent
clairement que si f est continue par morceaux, il en est de même de
\{f\}\^{}\{+\} et \{f\}\^{}\{−\}.

Théorème~9.6.10 Soit f : I → ℝ continue par morceaux. Alors f est
intégrable sur I si et seulement si les fonctions (à valeurs réelles
positives) \{f\}\^{}\{+\} et \{f\}\^{}\{−\} le sont. Dans ce cas

\{\textbackslash{}mathop\{∫ \} \}\_\{I\}f =\{\textbackslash{}mathop\{∫
\} \}\_\{I\}\{f\}\^{}\{+\} −\{\textbackslash{}mathop\{∫ \}
\}\_\{I\}\{f\}\^{}\{−\}\textbackslash{}text\{ et
\}\{\textbackslash{}mathop\{∫ \} \}\_\{I\}\textbar{}f\textbar{}
=\{\textbackslash{}mathop\{∫ \} \}\_\{I\}\{f\}\^{}\{+\}
+\{\textbackslash{}mathop\{∫ \} \}\_\{I\}\{f\}\^{}\{−\}

Démonstration Si f est intégrable sur I, il en est de même pour
\textbar{}f\textbar{} et donc pour \{f\}\^{}\{+\} et \{f\}\^{}\{−\}
puisque 0 ≤ \{f\}\^{}\{+\} ≤\textbar{}f\textbar{} et 0 ≤
\{f\}\^{}\{−\}≤\textbar{}f\textbar{}. Inversement, si \{f\}\^{}\{+\} et
\{f\}\^{}\{−\} sont intégrables, leur différence f l'est également. Les
formules proviennent de la linéarité de l'intégrale.

Théorème~9.6.11 Soit f : I → ℂ continue par morceaux. Alors f est
intégrable sur I si et seulement si les fonctions (à valeurs réelles)
\textbackslash{}mathop\{\textbackslash{}mathrm\{Re\}\}f et
\textbackslash{}mathop\{\textbackslash{}mathrm\{Im\}\}f le sont. Dans ce
cas

\{\textbackslash{}mathop\{∫ \} \}\_\{I\}f =\{\textbackslash{}mathop\{∫
\} \}\_\{I\}\textbackslash{}mathop\{ \textbackslash{}mathrm\{Re\}\}f +
i\{\textbackslash{}mathop\{∫ \} \}\_\{I\}\textbackslash{}mathop\{
\textbackslash{}mathrm\{Im\}\}f,\textbackslash{}quad
\{\textbackslash{}mathop\{∫ \} \}\_\{I\}\textbackslash{}overline\{f\} =
\textbackslash{}overline\{\{\textbackslash{}mathop\{∫ \} \}\_\{I\}f\}

Démonstration Si f est intégrable sur I, il en est de même pour
\textbar{}f\textbar{} et donc pour
\textbackslash{}mathop\{\textbackslash{}mathrm\{Re\}\}f et
\textbackslash{}mathop\{\textbackslash{}mathrm\{Im\}\}f puisque 0
≤\textbar{}\textbackslash{}mathop\{\textbackslash{}mathrm\{Re\}\}f\textbar{}≤\textbar{}f\textbar{}
et 0
≤\textbar{}\textbackslash{}mathop\{\textbackslash{}mathrm\{Im\}\}f\textbar{}≤\textbar{}f\textbar{}.
Inversement, si \textbackslash{}mathop\{\textbackslash{}mathrm\{Re\}\}f
et \textbackslash{}mathop\{\textbackslash{}mathrm\{Im\}\}f sont
intégrables, alors f =\textbackslash{}mathop\{
\textbackslash{}mathrm\{Re\}\}f +
i\textbackslash{}mathop\{\textbackslash{}mathrm\{Im\}\}f l'est
également. Les formules proviennent de la linéarité de l'intégrale.

Remarque~9.6.2 La combinaison de ces deux théorèmes peut permettre de
ramener un problème sur des fonctions à valeurs complexes à des
problèmes sur des fonctions à valeurs réelles positives.

\paragraph{9.6.3 Convention et relation de Chasles}

Définition~9.6.2 Soit I un intervalle de ℝ, f : I → ℂ continue par
morceaux et intégrable. Soit a,b ∈\textbackslash{}overline\{I\}. Alors
on posera

\{\textbackslash{}mathop\{∫ \} \}\_\{a\}\^{}\{b\}f(t) dt =
\textbackslash{}left \textbackslash{}\{ \textbackslash{}cases\{
\{\textbackslash{}mathop\{∫ \} \}\_\{{]}a,b{[}\}f \&si a \textless{} b
\textbackslash{}cr 0 \&si a = b \textbackslash{}cr
−\{\textbackslash{}mathop\{∫ \} \}\_\{{]}b,a{[}\}f\&si b \textless{} a
\} \textbackslash{}right .

La définition a bien un sens puisque f est intégrable sur {]}a,b{[}⊂ I
ou {]}b,a{[}⊂ I suivant le cas.

Théorème~9.6.12 Soit I un intervalle de ℝ, f : I → ℂ continue par
morceaux et intégrable. Soit a,b,c ∈\textbackslash{}overline\{I\}. Alors
on a

\{\textbackslash{}mathop\{∫ \} \}\_\{a\}\^{}\{c\}f
=\{\textbackslash{}mathop\{∫ \} \}\_\{a\}\^{}\{b\}f
+\{\textbackslash{}mathop\{∫ \} \}\_\{b\}\^{}\{c\}f

Démonstration Etudier toutes les positions relatives de a,b et c.

\paragraph{9.6.4 Règles de comparaison}

Théorème~9.6.13 Soit f : {[}a,b{[}→ ℂ continue par morceaux et g :
{[}a,b{[}→ \{ℝ\}\^{}\{+\} continue par morceaux, positive et intégrable.
On suppose qu'au voisinage de b on a f = O(g) (resp. f = o(g)). Alors f
est intégrable sur {[}a,b{[} et \{\textbackslash{}mathop\{∫ \}
\}\_\{{[}x,b{[}\}f(t) dt = O(\{\textbackslash{}mathop\{∫ \}
\}\_\{{[}x,b{[}\}g(t) dt) (resp. \{\textbackslash{}mathop\{∫ \}
\}\_\{{[}x,b{[}\}f(t) dt = o(\{\textbackslash{}mathop\{∫ \}
\}\_\{{[}x,b{[}\}g(t) dt))

Démonstration On a en effet \textbar{}f\textbar{} = O(g) (resp.
\textbar{}f\textbar{} = o(g)) et \textbackslash{}left
\textbar{}\{\textbackslash{}mathop\{∫ \}
\}\_\{{[}x,b{[}\}f\textbackslash{}right
\textbar{}≤\{\textbackslash{}mathop\{∫ \}
\}\_\{{[}x,b{[}\}\textbar{}f\textbar{}. Il suffit donc d'appliquer le
théorème de comparaison à \textbar{}f\textbar{} et g.

Remarque~9.6.3 Il suffit pour appliquer le théorème précédent que la
condition de positivité de g soit vérifiée dans un voisinage de b.

\paragraph{9.6.5 Espaces de fonctions continues}

Théorème~9.6.14 Soit I un intervalle de ℝ. L'ensemble des fonctions
continues et intégrables sur I à valeurs complexes est un sous-espace
vectoriel de l'espace C(I, ℂ). L'application
f\textbackslash{}mathrel\{↦\}\textbackslash{}\textbar{}\{f\textbackslash{}\textbar{}\}\_\{1\}
=\{\textbackslash{}mathop\{∫ \} \}\_\{I\}\textbar{}f\textbar{} est une
norme sur cet espace (appelée la norme de la convergence en moyenne).

Démonstration Vérification immédiate à partir des résultats précédents.

Théorème~9.6.15 Soit I un intervalle de ℝ. L'ensemble des fonctions
continues à valeurs complexes dont le carré est intégrable sur I est un
sous-espace vectoriel de l'espace C(I, ℂ). L'application
(f,g)\textbackslash{}mathrel\{↦\}(f\textbackslash{}mathrel\{∣\}g)
=\{\textbackslash{}mathop\{∫ \} \}\_\{I\}\textbackslash{}overline\{f\}g
est un produit scalaire hermitien sur cet espace. En particulier,
l'application
f\textbackslash{}mathrel\{↦\}\textbackslash{}\textbar{}\{f\textbackslash{}\textbar{}\}\_\{2\}
= \{(f\textbackslash{}mathrel\{∣\}f)\}\^{}\{1∕2\} est une norme sur cet
espace et on a l'inégalité de Cauchy-Schwarz
\textbar{}(f\textbackslash{}mathrel\{∣\}g)\textbar{}≤\textbackslash{}\textbar{}
\{f\textbackslash{}\textbar{}\}\_\{2\}\textbackslash{}\textbar{}\{g\textbackslash{}\textbar{}\}\_\{2\}.

Démonstration Il est clair que si f est de carré intégrable, il en est
de même de αf pour α ∈ ℂ. De plus l'inégalité élémentaire \textbar{}f +
g\{\textbar{}\}\^{}\{2\} ≤ 2\textbar{}f\{\textbar{}\}\^{}\{2\} +
2\textbar{}g\{\textbar{}\}\^{}\{2\} montre que si f et g sont de carré
intégrables, il en est de même de f + g. Comme de surcroît il existe des
fonctions de carré intégrables (par exemple la fonction nulle),
celles-ci forment un sous-espace vectoriel de C(I, ℂ). L'inégalité
élémentaire \textbar{}\textbackslash{}overline\{f\}g\textbar{}≤\{ 1
\textbackslash{}over 2\} \textbar{}f\{\textbar{}\}\^{}\{2\} +\{ 1
\textbackslash{}over 2\} \textbar{}g\{\textbar{}\}\^{}\{2\} montre que
si f et g sont de carré intégrables, \textbackslash{}overline\{g\}f est
intégrable ce qui permet de définir (f\textbackslash{}mathrel\{∣\}g)
=\{\textbackslash{}mathop\{∫ \} \}\_\{I\}\textbackslash{}overline\{f\}g.
L'application est visiblement sesquilinéaire hermitienne, on a
(f\textbackslash{}mathrel\{∣\}f) =\{\textbackslash{}mathop\{∫ \}
\}\_\{I\}\textbar{}f\{\textbar{}\}\^{}\{2\} ≥ 0 avec égalité si et
seulement si \textbar{}f\{\textbar{}\}\^{}\{2\} = 0, soit f = 0, puisque
f est continue. Les autres affirmations sont des conséquences des
résultats sur les produits scalaires hermitiens.

\paragraph{9.6.6 Notion d'intégrale impropre}

Définition~9.6.3 Soit −∞ \textless{} a \textless{} b ≤ +∞ et f :
{[}a,b{[}→ E continue par morceaux. On dit que l'intégrale
\{\textbackslash{}mathop\{∫ \} \}\_\{a\}\^{}\{b\}f(t) dt converge si
existe
\{\textbackslash{}mathop\{lim\}\}\_\{x→b,x\textless{}b\}\{\textbackslash{}mathop\{∫
\} \}\_\{a\}\^{}\{x\}f(t) dt. Dans ce cas on pose
\{\textbackslash{}mathop\{∫ \} \}\_\{a\}\^{}\{b\}f(t) dt
=\{\textbackslash{}mathop\{
lim\}\}\_\{x→b,x\textless{}b\}\{\textbackslash{}mathop\{∫ \}
\}\_\{a\}\^{}\{x\}f(t) dt.

On a une notion similaire avec −∞≤ a \textless{} b \textless{} +∞ et f
:{]}a,b{]} → E continue par morceaux.

Remarque~9.6.4 Si l'intégrale ne converge pas, elle est dite divergente.
Si b \textless{} +∞ et si f est la restriction à {[}a,b{[} d'une
fonction réglée sur {[}a,b{]}, alors l'application
x\textbackslash{}mathrel\{↦\}\{\textbackslash{}mathop\{∫ \}
\}\_\{a\}\^{}\{x\}f(t) dt est continue au point b~; l'intégrale impropre
est donc convergente et la valeur de l'intégrale impropre est donc la
valeur de l'intégrale, si bien qu'il n'y a pas d'ambiguïté dans la
notation \{\textbackslash{}mathop\{∫ \} \}\_\{a\}\^{}\{b\}f(t) dt~; dans
ce cas nous parlerons d'une intégrale faussement impropre. Un exemple
typique est celui de \{\textbackslash{}mathop\{∫ \} \}\_\{0\}\^{}\{1\}\{
\textbackslash{}mathop\{sin\} t \textbackslash{}over t\} dt qui est a
priori impropre en 0, mais qui est la restriction à {]}0,1{]} de la
fonction continue f(t) = \textbackslash{}left \textbackslash{}\{
\textbackslash{}cases\{ \{ \textbackslash{}mathop\{sin\} t
\textbackslash{}over t\} \&si t\textbackslash{}mathrel\{≠\}0
\textbackslash{}cr 1 \&si t = 0 \textbackslash{}cr \}
\textbackslash{}right ..

Proposition~9.6.16 Soit f : {[}a,b{[}→ E une fonction continue par
morceaux et c ∈ {[}a,b{[}. Alors l'intégrale \{\textbackslash{}mathop\{∫
\} \}\_\{a\}\^{}\{b\}f(t) dt converge si et seulement si~l'intégrale
\{\textbackslash{}mathop\{∫ \} \}\_\{c\}\^{}\{b\}f(t) dt converge.

Démonstration On a \{\textbackslash{}mathop\{∫ \} \}\_\{a\}\^{}\{x\}f(t)
dt =\{\textbackslash{}mathop\{∫ \} \}\_\{a\}\^{}\{c\}f(t) dt
+\{\textbackslash{}mathop\{∫ \} \}\_\{c\}\^{}\{x\}f(t) dt ce qui montre
que \{\textbackslash{}mathop\{∫ \} \}\_\{a\}\^{}\{x\}f(t) dt a une
limite en b si et seulement si~\{\textbackslash{}mathop\{∫ \}
\}\_\{c\}\^{}\{x\}f(t) dt en a une.

Remarque~9.6.5 Cette propriété montre que si f : {[}a,b{[}→ E est une
fonction continue par morceaux, la convergence de
\{\textbackslash{}mathop\{∫ \} \}\_\{a\}\^{}\{b\}f(t) dt ne dépend que
de la restriction de f à un voisinage de b~; il s'agit donc d'une notion
locale en b.

Théorème~9.6.17 Si f est intégrable sur {[}a,b{[}, alors
\{\textbackslash{}mathop\{∫ \} \}\_\{a\}\^{}\{b\}f(t) dt converge. Mais
la réciproque est fausse dans le cas général (mais vraie pour les
fonctions à valeurs dans \{ℝ\}\^{}\{+\}).

Démonstration On a vu que si f est intégrable sur {[}a,b{[}, alors
x\textbackslash{}mathrel\{↦\}\{\textbackslash{}mathop\{∫ \}
\}\_\{a\}\^{}\{x\}f(t) dt admet la limite \{\textbackslash{}mathop\{∫ \}
\}\_\{I\}f au point b. L'exemple suivant montre que la réciproque est
fausse.

Exemple~9.6.1 Etude de l'intégrale \{\textbackslash{}mathop\{∫ \}
\}\_\{1\}\^{}\{+∞\}\{ \textbackslash{}mathop\{sin\} t
\textbackslash{}over \{t\}\^{}\{α\}\} dt pour α \textgreater{} 0. On a
\{ \textbackslash{}mathop\{sin\} t \textbackslash{}over \{t\}\^{}\{α\}\}
= O(\{ 1 \textbackslash{}over \{t\}\^{}\{α\}\} ), donc si α
\textgreater{} 1 la fonction est intégrable.

Si 0 \textless{} α ≤ 1, on a après intégration par parties

\{\textbackslash{}mathop\{∫ \} \}\_\{1\}\^{}\{x\}\{
\textbackslash{}mathop\{sin\} t \textbackslash{}over \{t\}\^{}\{α\}\} dt
=\textbackslash{}mathop\{ cos\} 1 −\{ \textbackslash{}mathop\{cos\} x
\textbackslash{}over \{x\}\^{}\{α\}\} +\{\textbackslash{}mathop\{∫ \}
\}\_\{1\}\^{}\{x\}\{ \textbackslash{}mathop\{cos\} t
\textbackslash{}over \{t\}\^{}\{α+1\}\} dt

Mais \{\textbackslash{}mathop\{lim\}\}\_\{x→+∞\}\{
\textbackslash{}mathop\{cos\} x \textbackslash{}over \{x\}\^{}\{α\}\} =
0 et la fonction t\textbackslash{}mathrel\{↦\}\{
\textbackslash{}mathop\{cos\} t \textbackslash{}over \{t\}\^{}\{α+1\}\}
est intégrable puisque \{ \textbackslash{}mathop\{cos\} t
\textbackslash{}over \{t\}\^{}\{α+1\}\} = O(\{ 1 \textbackslash{}over
\{t\}\^{}\{α+1\}\} ). On en déduit que le terme de droite de l'égalité
ci dessus a une limite en + ∞, et donc le terme de gauche aussi. En
conséquence, l'intégrale impropre \{\textbackslash{}mathop\{∫ \}
\}\_\{1\}\^{}\{+∞\}\{ \textbackslash{}mathop\{sin\} t
\textbackslash{}over \{t\}\^{}\{α\}\} dt converge. Montrons que la
fonction n'est pas intégrable~; on a

\textbackslash{}begin\{eqnarray*\} \{\textbackslash{}mathop\{∫ \}
\}\_\{1\}\^{}\{x\}\{ \textbar{}\textbackslash{}mathop\{sin\} t\textbar{}
\textbackslash{}over \{t\}\^{}\{α\}\} \& ≥\& \{\textbackslash{}mathop\{∫
\} \}\_\{1\}\^{}\{x\}\{ \{\textbackslash{}mathop\{sin\} \}\^{}\{2\}t
\textbackslash{}over \{t\}\^{}\{α\}\} dt \%\&
\textbackslash{}\textbackslash{} \& =\&\{ 1 \textbackslash{}over 2\}
\{\textbackslash{}mathop\{∫ \} \}\_\{1\}\^{}\{x\}\{ 1
−\textbackslash{}mathop\{ cos\} (2t) \textbackslash{}over
\{t\}\^{}\{α\}\} dt \%\& \textbackslash{}\textbackslash{} \& =\&\{ 1
\textbackslash{}over 2\} \{\textbackslash{}mathop\{∫ \}
\}\_\{1\}\^{}\{x\}\{ 1 \textbackslash{}over \{t\}\^{}\{α\}\} dt −\{ 1
\textbackslash{}over 2\} \{\textbackslash{}mathop\{∫ \}
\}\_\{1\}\^{}\{x\}\{ \textbackslash{}mathop\{cos\} (2t)
\textbackslash{}over \{t\}\^{}\{α\}\} dt\%\&
\textbackslash{}\textbackslash{} \textbackslash{}end\{eqnarray*\}

Mais l'intégrale \{\textbackslash{}mathop\{∫ \} \}\_\{1\}\^{}\{x\}\{ 1
\textbackslash{}over \{t\}\^{}\{α\}\} dt admet pour limite + ∞ (car α ≤
1), alors que l'intégrale \{\textbackslash{}mathop\{∫ \}
\}\_\{1\}\^{}\{x\}\{ \textbackslash{}mathop\{cos\} (2t)
\textbackslash{}over \{t\}\^{}\{α\}\} dt converge (même méthode
d'intégration par parties). On en déduit que
\{\textbackslash{}mathop\{lim\}\}\_\{x→+∞\}\{\textbackslash{}mathop\{∫
\} \}\_\{1\}\^{}\{x\}\{ \{\textbackslash{}mathop\{sin\} \}\^{}\{2\}t
\textbackslash{}over \{t\}\^{}\{α\}\} dt = +∞ et donc aussi
\{\textbackslash{}mathop\{lim\}\}\_\{x→+∞\}\{\textbackslash{}mathop\{∫
\} \}\_\{1\}\^{}\{x\}\{ \textbar{}\textbackslash{}mathop\{ sin\}
t\textbar{} \textbackslash{}over \{t\}\^{}\{α\}\} dt = +∞.

{[}\href{coursse56.html}{next}{]} {[}\href{coursse54.html}{prev}{]}
{[}\href{coursse54.html\#tailcoursse54.html}{prev-tail}{]}
{[}\href{coursse55.html}{front}{]}
{[}\href{coursch10.html\#coursse55.html}{up}{]}

\end{document}
