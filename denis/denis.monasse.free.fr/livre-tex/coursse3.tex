\section{Groupes}
\index{groupe}

\subsection{Définitions et premières propriétés}

\begin{de}
\index{groupe!définition}
\index{élément neutre}
\index{inverse!élément}
On dit qu'un couple $(G,∗)$ d'un ensemble $G$ et d'une loi interne $∗$ sur $G$ est un groupe si la loi est associative, possède un élément neutre et si tout élément a un inverse, autrement dit
\begin{itemize}
\itemsep1pt\parskip0pt\parsep0pt
\item $\forall x,y,z \in G, x ∗ (y ∗ z) = (x ∗ y) ∗ z$
\item il existe un élément neutre $e_G \in G$ tel que $\forall x \in G, e_G ∗ x = x ∗ e_G = x$
\item pour tout $x \in G$, il existe un inverse $y \in G$ tel que $x ∗ y = y ∗ x = e_G$
\end{itemize}
\end{de}

\begin{rem}
\index{groupe!abélien}
\index{groupe!commutatif}
On montre alors que l'élément neutre est unique, de même que l'inverse d'un élément. On dit que le groupe est abélien ou commutatif si la loi est commutative ($x ∗ y = y ∗ x$). Enfin un groupe possède la propriété essentielle suivante
\end{rem}

\begin{prop}
Pour tout $a \in G$, les applications $x \mapsto a ∗ x$ et $x \mapsto x ∗ a$ sont des bijections de $G$ dans $G$.
\end{prop}

\subsection{Sous-groupes}
\index{sous-groupe}

\begin{de}
\index{conjugaison!éléments conjugués}
Soit $G$ un groupe. On dit que deux éléments $x$ et $y$ de $G$ sont conjugués s'il existe $g \in G$ tel que $y = gxg^{-1}$.
\end{de}

\begin{rem}
\index{conjugaison!classe de conjugaison}
On montre facilement qu'il s'agit d'une relation d'équivalence sur $G$, dont les classes d'équivalence sont appelées les classes de conjugaison.
\end{rem}

\begin{de}
\index{sous-groupe!distingué}
On dit qu'un sous-groupe $H$ de $G$ est distingué dans $G$ si
$\forall x \in G, \forall h \in H, xhx^{-1} \in H$
(autrement dit $H$ est stable par conjugaison par tous les éléments de $G$).
\end{de}

\section{Groupes}
\index{groupe}

\subsection{Définitions et premières propriétés}

\begin{de}
\index{groupe!définition}
\index{élément neutre}
\index{inverse!élément}
On dit qu'un couple $(G,∗)$ d'un ensemble $G$ et d'une loi interne $∗$ sur $G$ est un groupe si la loi est associative, possède un élément neutre et si tout élément a un inverse, autrement dit
\begin{itemize}
\itemsep1pt\parskip0pt\parsep0pt
\item $\forall x,y,z \in G, x ∗ (y ∗ z) = (x ∗ y) ∗ z$
\item il existe un élément neutre $e_G \in G$ tel que $\forall x \in G, e_G ∗ x = x ∗ e_G = x$
\item pour tout $x \in G$, il existe un inverse $y \in G$ tel que $x ∗ y = y ∗ x = e_G$
\end{itemize}
\end{de}

\begin{rem}
\index{groupe!abélien}
\index{groupe!commutatif}
On montre alors que l'élément neutre est unique, de même que l'inverse d'un élément. On dit que le groupe est abélien ou commutatif si la loi est commutative ($x ∗ y = y ∗ x$). Enfin un groupe possède la propriété essentielle suivante
\end{rem}

\begin{prop}
Pour tout $a \in G$, les applications $x \mapsto a ∗ x$ et $x \mapsto x ∗ a$ sont des bijections de $G$ dans $G$.
\end{prop}

\subsection{Sous-groupes}
\index{sous-groupe}

\begin{de}
\index{conjugaison!éléments conjugués}
Soit $G$ un groupe. On dit que deux éléments $x$ et $y$ de $G$ sont conjugués s'il existe $g \in G$ tel que $y = gxg^{-1}$.
\end{de}

\begin{rem}
\index{conjugaison!classe de conjugaison}
On montre facilement qu'il s'agit d'une relation d'équivalence sur $G$, dont les classes d'équivalence sont appelées les classes de conjugaison.
\end{rem}

[...]

\begin{de}
\index{sous-groupe!distingué}
On dit qu'un sous-groupe $H$ de $G$ est distingué dans $G$ si
$\forall x \in G, \forall h \in H, xhx^{-1} \in H$
(autrement dit $H$ est stable par conjugaison par tous les éléments de $G$).
\end{de}

\subsection{Sous-groupes}
\index{sous-groupe!caractérisation}

\begin{prop}
(Caractérisation 1) Une partie $H$ de $G$ en est un sous-groupe si et seulement si elle vérifie
\begin{itemize}
\item $H \neq \varnothing$
\item $\forall x,y \in H, xy \in H$
\item $\forall x \in H, x^{-1} \in H$
\end{itemize}
\end{prop}

\begin{prop}
(Caractérisation 2) Une partie $H$ de $G$ en est un sous-groupe si et seulement si elle vérifie
\begin{itemize}
\item $H \neq \varnothing$
\item $\forall x,y \in H, xy^{-1} \in H$
\end{itemize}
\end{prop}

\begin{prop}
\index{groupe!engendré}
Soit $A$ une partie de $G$. Alors l'ensemble des sous-groupes de $G$ qui contiennent $A$ admet un plus petit élément (pour l'inclusion). On l'appelle le groupe engendré par la partie $A$ et on le note $\text{Groupe}(A)$ ou $\langle A \rangle$. On a les deux caractérisations suivantes:

\begin{align*} 
\text{Groupe}(A) &= \bigcap_{H \text{ sous-groupe de } G \atop A \subset H} H \\
&= \{x_1 \ldots x_k | k \geq 0 \text{ et } x_i \in A \cup A^{-1}\}
\end{align*}
\end{prop}

\subsection{Quotient par un sous-groupe}
\index{groupe!quotient}

\begin{thm}
\index{relation d'équivalence}
La relation $\mathcal{R}$ définie par $x \mathcal{R} y \Leftrightarrow x^{-1}y \in H$ est une relation d'équivalence sur $G$. Si $x$ appartient à $G$, la classe d'équivalence de $x$ est $xH = \{xh | h \in H\}$ (en particulier la classe de $e$ est $H$). L'ensemble quotient $G/\mathcal{R}$ est noté $G/H$.
\end{thm}

\begin{thm}
La relation $\mathcal{R}'$ définie par $x \mathcal{R}' y \Leftrightarrow yx^{-1} \in H$ est une relation d'équivalence sur $G$. Si $x$ appartient à $G$, la classe d'équivalence de $x$ est $Hx = \{hx | h \in H\}$ (en particulier la classe de $e$ est $H$). L'ensemble quotient $G/\mathcal{R}'$ est noté $H \setminus G$.
\end{thm}

\begin{rem}
Bien évidemment, si le groupe $G$ est commutatif, les deux relations sont confondues ainsi que les deux ensembles $G/H$ et $H \setminus G$.
\end{rem}

\begin{thm}
\index{groupe!quotient commutatif}
Soit $G$ un groupe commutatif (noté additivement) et $H$ un sous-groupe de $G$. On définit alors une loi de groupe sur $G/H$ en posant:
$(x + H) + (y + H) = (x + y) + H$

Le groupe ainsi obtenu est appelé groupe quotient du groupe commutatif $G$ par le sous-groupe $H$.
\end{thm}

\begin{thm}
\index{groupe!quotient non commutatif}
Soit $H$ un sous-groupe de $G$. Alors les relations d'équivalence $\mathcal{R}$ et $\mathcal{R}'$ définies ci-dessus coïncident si et seulement si $H$ est distingué dans $G$. Dans ce cas on a $\forall x \in G, xH = Hx$. L'ensemble $G/H$ est muni d'une structure de groupe en posant $xH \cdot yH = xyH$.
\end{thm}