
\subsubsection{1.3 Groupes}

\paragraph{1.3.1 Définitions et première propriété}

Définition~1.3.1 On dit qu'un couple (G,∗) d'un ensemble G et d'une loi
interne ∗ sur G est un groupe si la loi est associative, possède un
élément neutre et si tout élément a un inverse, autrement dit

\begin{itemize}
\itemsep1pt\parskip0pt\parsep0pt
\item
  \forall~~x,y,z \in G,x ∗ (y ∗ z) = (x ∗ y) ∗ z
\item
  il existe un élément neutre e_G \in G tel que
  \forall~x \in G, e_G ∗ x = x ∗ e_G~
  = x
\item
  pour tout x \in G, il existe un inverse y \in G tel que x ∗ y = y ∗ x =
  e_G
\end{itemize}

Remarque~1.3.1 On montre alors que l'élément neutre est unique, de même
que l'inverse d'un élément. On dit que le groupe est abélien ou
commutatif si la loi est commutative (x ∗ y = y ∗ x). Enfin un groupe
possède la propriété essentielle suivante

Proposition~1.3.1 Pour tout a \in G, les applications
x\mapsto~a ∗ x et x\mathrel\mapsto~x ∗ a
sont des bi\\jmathmathections de G dans G.

Démonstration En effet, si a' désigne l'inverse de a, a ∗ x = a ∗ y \rigtharrow~ a'
∗ a ∗ x = a' ∗ a ∗ y \rigtharrow~ x = y.

Notations habituelles Les groupes sont en général notés
multiplicativement (xy) ou additivement (x + y). Les conventions
suivantes sont alors utilisées

\begin{center}\rule{3in}{0.4pt}\end{center}

\begin{center}\rule{3in}{0.4pt}\end{center}

\begin{center}\rule{3in}{0.4pt}\end{center}

Notation

multiplicative

additive

\begin{center}\rule{3in}{0.4pt}\end{center}

\begin{center}\rule{3in}{0.4pt}\end{center}

\begin{center}\rule{3in}{0.4pt}\end{center}

x ∗ y

xy

x + y

élément neutre

e_G ou e

0_G ou 0

inverse

x^-1

- x

puissance

x^n

nx

\begin{center}\rule{3in}{0.4pt}\end{center}

\begin{center}\rule{3in}{0.4pt}\end{center}

\begin{center}\rule{3in}{0.4pt}\end{center}

Définition~1.3.2 Soit G un groupe. On dit que deux éléments x et y de G
sont con\\jmathmathugués s'il existe g \in G tel que y = gxg^-1.

Remarque~1.3.2 On montre facilement qu'il s'agit d'une relation
d'équivalence sur G, dont les classes d'équivalence sont appelées les
classes de con\\jmathmathugaison.

\paragraph{1.3.2 Sous-groupes}

On dit qu'une partie H de G en est un sous-groupe si elle est stable
pour la loi interne et est munie d'une structure de groupe pour la loi
induite. On montre alors facilement que l'élément neutre de H doit être
celui de G, de même que l'inverse d'un élément dans G doit être le même
que l'inverse dans H. On aboutit alors aux deux caractérisations
suivantes~:

Proposition~1.3.2 (Caractérisation 1) Une partie H de G en est un
sous-groupe si et seulement si elle vérifie (i)
H\neq~\varnothing~, (ii)\forall~~x,y \in H,
xy \in H, (iii) \forall~x \in H, x^-1~ \in H.

Démonstration Les conditions sont bien entendu nécessaires. Elles sont
également suffisantes car si H vérifie ces conditions, il contient un
élément x donc aussi x^-1, donc aussi e_G =
xx^-1. Comme de plus H est stable pour la loi de groupe,
c'est un sous-groupe de G.

Proposition~1.3.3 (Caractérisation 2) Une partie H de G en est un
sous-groupe si et seulement si elle vérifie (i)
H\neq~\varnothing~, (ii)\forall~~x,y \in H,
xy^-1 \in H.

Démonstration Les conditions sont bien entendu nécessaires. Elles sont
également suffisantes car si H vérifie ces conditions, il contient un
élément x, donc aussi e_G = xx^-1 (prendre y = x).
Mais alors e_G,x \in G \rigtharrow~ x^-1 =
e_Gx^-1 \in G et donc x,y \in G \rigtharrow~ x,y^-1 \in G
\rigtharrow~ xy = x(y^-1)^-1 \in G ce qui ramène à la première
caractérisation.

Exemple~1.3.1 \e\ et G sont bien
évidemment des sous-groupes de G (dits triviaux). De même Z(G) =
\x \in
G∣\forall~~y \in G, xy =
yx\ est un sous-groupe de G appelé le centre de G.

Remarque~1.3.3 On vérifie immédiatement que toute intersection de
sous-groupes est encore un sous groupe à l'aide de la première
caractérisation et du fait que tout sous-groupe contient e_G
(ce qui garantit que l'intersection est non vide). On obtient donc la
proposition suivante

Proposition~1.3.4 Soit A une partie de G. Alors l'ensemble des
sous-groupes de G qui contiennent A admet un plus petit élément (pour
l'inclusion). On l'appelle le groupe engendré par la partie A et on le
note Groupe~(A) ou \langle
A\rangle . On a les deux caractérisations suivantes

\begin{align*} Groupe~(A)&
=& ⋂ _ H\text
sous-groupe de G \atop A\subset~H H \%&
\\ & =&
\x_1\\ldotsx_k\mathrel∣~k
≥ 0\text et x_i \in A \cup
A^-1\\%& \\
\end{align*}

Démonstration La première caractérisation est évidente, puisque
\⋂  _
H\text sous-groupe de G \atop A\subset~H
H est un sous-groupe de G (comme intersection de sous-groupes de G),
contenant A et inclus dans tout sous-groupe de G contenant A.

En ce qui concerne la seconde (plus constructive), posons H =
\x_1\\ldotsx_k\mathrel∣~k
≥ 0\text et x_i \in A \cup
A^-1\. L'une quelconque des caractérisations
des sous-groupes montre que H est un sous-groupe de G~; il contient bien
évidemment A. De plus, si H' est un sous-groupe de G contenant A, il
doit contenir tous les éléments de A, tous leurs inverses, et tous les
produits d'éléments de A et de leurs inverses, donc il doit contenir H.
Donc H est bien le plus petit sous-groupe de G contenant A.

\paragraph{1.3.3 Quotient par un sous-groupe}

Considérons H un sous-groupe de G. Un calcul élémentaire montre le
résultat suivant

Théorème~1.3.5 La relation \mathcal{R} définie par x \mathcal{R} y
\Leftrightarrow x^-1y \in H est une relation
d'équivalence sur G. Si x appartient à G, la classe d'équivalence de x
est xH = \xh∣h \in
H\ (en particulier la classe de e est H). L'ensemble
quotient G\diagup\mathcal{R} est noté G\diagupH.

Démonstration On a

\begin{itemize}
\itemsep1pt\parskip0pt\parsep0pt
\item
  e_G = x^-1x \in H \rigtharrow~ x\mathcal{R}x (réflexivité)
\item
  x\mathcal{R}y \rigtharrow~ x^-1y \in H \rigtharrow~ (x^-1y)^-1 \in H \rigtharrow~
  y^-1x \in H \rigtharrow~ y\mathcal{R}x (symétrie)
\item
  x\mathcal{R}y et y\mathcal{R}z \rigtharrow~ x^-1y,y^-1z \in H \rigtharrow~ x^-1z
  = x^-1yy^-1z \in H \rigtharrow~ x\mathcal{R}z (transitivité)
\end{itemize}

On a de même

Théorème~1.3.6 La relation \mathcal{R}' définie par x \mathcal{R}'y
\Leftrightarrow yx^-1 \in H est une relation
d'équivalence sur G. Si x appartient à G, la classe d'équivalence de x
est Hx = \hx∣h \in
H\ (en particulier la classe de e est H). L'ensemble
quotient G\diagup\mathcal{R}' est noté H\diagdownG.

Remarque~1.3.4 Bien évidemment, si le groupe G est commutatif, les deux
relations sont confondues ainsi que les deux ensembles G\diagupH et H\diagdownG.

Lorsque le groupe est noté additivement, on a x \mathcal{R} y
\Leftrightarrow x - y \in H et C_\mathcal{R}(x) = x + H.

Théorème~1.3.7 Soit G un groupe commutatif (noté additivement) et H un
sous-groupe de G. On définit alors une loi de groupe sur G\diagupH en posant

(x + H) + (y + H) = (x + y) + H

Le groupe ainsi obtenu est appelé groupe quotient du groupe commutatif G
par le sous-groupe H.

Démonstration Le principal point est de vérifier que l'on définit bien
une application, c'est à dire que si x + H = x' + H et y + H = y' + H,
alors x + y + H = x' + y' + H. Mais dans ce cas, il existe h et h' dans
H tels que x' = x + h et y' = y + h', d'où

\begin{align*} (x' + y') + H& =& (x + h + y + h')
+ H \%& \\ & =& (x + y) + (h + h' + H)
= (x + y) + H\%& \\
\end{align*}

puisque h + h' \in H et que \forall~~h \in H, h + H = H.
(On remarquera le rôle essentiel \\jmathmathoué par la commutativité du groupe G
lors de ce calcul.)

A partir de là, dans G\diagupH, l'associativité est évidente, l'élément neutre
est 0 + H = H et l'opposé de x + H est (-x) + H.

Lorsque le groupe n'est pas commutatif, on est amené à introduire les
notions suivantes~:

Définition~1.3.3 On dit qu'un sous-groupe H de G est distingué dans G si

\forall~x \in G,\\forall~~h \in H,
xhx^-1 \in H

(autrement dit H est stable par con\\jmathmathugaison par tous les éléments de G).

Exemple~1.3.2 \e\,G,Z(G) sont des
sous-groupes distingués de G.

Dans un groupe abélien, tout sous-groupe est distingué.

On a alors

Théorème~1.3.8 Soit H un sous-groupe de G. Alors les relations
d'équivalences \mathcal{R} et \mathcal{R}' définies ci dessus coïncident si et seulement si
H est distingué dans G. Dans ce cas on a \forall~~x \in
G, xH = Hx. L'ensemble G\diagupH est muni d'une structure de groupe en posant
xH.yH = xyH.

Démonstration Les deux relations d'équivalences sont égales si et
seulement si leurs classes d'équivalences sont égales. On a donc

\begin{align*} \mathcal{R} = \mathcal{R}'& \Leftrightarrow
& \forall~~x \in G, xH = Hx \%&
\\ & \Leftrightarrow &
\forall~x \in G, xHx^-1~ = H \%&
\\ & \Leftrightarrow &
\forall~x \in G, xHx^-1~ \subset~
H\text et H \subset~ xHx^-1\%&
\\ & \Leftrightarrow &
\forall~x \in G, xHx^-1~ \subset~
H\text et x^-1Hx \subset~ H\%&
\\ & \Leftrightarrow &
\forall~x \in G, xHx^-1~ \subset~ H \%&
\\ \end{align*}

(car x \in G \rigtharrow~ x^-1 \in G). Or ceci équivaut au fait que H soit
distingué dans G.

En ce qui concerne la structure de groupe sur G\diagupH, le seul point non
évident est le fait que l'on définit bien une application en posant
xH.yH = xyH, c'est-à-dire que si xH = x'H et yH = y'H, alors x'y'H =
xyH~; mais dans ce cas il existe h,k \in H tels que x' = xh, y' = yk, d'où
x'y' = xhyk = xy(y^-1hy)k. Mais, comme H est distingué dans
G, y \in G,h \in H \rigtharrow~ y^-1hy \in H et donc k' = (y^-1hy)k
\in H, soit k'H = H (car H est invariant par ses propres translations). On
a donc x'y'H = xyk'H = xyH, ce qu'il fallait démontrer.

\paragraph{1.3.4 Morphisme de groupes}

Définition~1.3.4 On dit que f:G \rightarrow~ G' est un morphisme de groupes si
\forall~~x,y \in G, f(xy) = f(x)f(y).

Exemple~1.3.3 Soit (G,+) un groupe commutatif et H un sous-groupe .
Alors l'application \pi~:G \rightarrow~ G\diagupH définie par \pi~(x) = x + H est par
définition un morphisme de groupes.

Proposition~1.3.9 Soit f:G \rightarrow~ G' un morphisme de groupes. Alors
f(e_G) = e_G' et \forall~~x \in G,
f(x^-1) = f(x)^-1. Pour tout sous-groupe H de G,
f(H) est un sous-groupe de G'. Pour tout sous-groupe (resp. sous-groupe
distingué) H' de G', f^-1(H') est un sous-groupe (resp.
sous-groupe distingué) de G.

Démonstration Vérification laissée au lecteur.

Théorème~1.3.10 Soit f:G \rightarrow~ G' un morphisme de groupes. Alors
\mathrmKer~f =
\x \in G∣f(x) =
e_G'\ est un sous-groupe distingué de G appelé
le noyau de f et \mathrmIm~f
= f(G) est un sous-groupe de G' appelé l'image de f.

On a le résultat essentiel suivant

Théorème~1.3.11 Soit f:G \rightarrow~ G' un morphisme de groupes. Alors f est
in\\jmathmathectif si et seulement si
\mathrmKer~f =
\e_G\.

Démonstration La vérification est immédiate puisque

\begin{align*} f(x) = f(y)&
\Leftrightarrow & f(x)f(y)^-1 = e_ G'
\Leftrightarrow f(xy^-1) = e_ G'\%&
\\ & \Leftrightarrow &
xy^-1
\in\mathrmKer~f \%&
\\ \end{align*}

Théorème~1.3.12 (factorisation canonique). Soit f:G \rightarrow~ G' un morphisme de
groupes, G étant commutatif et noté additivement. Alors il existe une
unique application \overlinef :
G\diagup\mathrmKer~f
\rightarrow~\mathrmIm~f vérifiant
\forall~x \in G, \overlinef~(x
+ \mathrmKer~f) = f(x).
L'application \overlinef est un isomorphisme de
groupes.

Démonstration L'unicité est claire puisque tout élément de
G\diagup\mathrmKer~f peut s'écrire
sous la forme x +\
\mathrmKerf. Par contre l'existence de
\overlinef n'est pas évidente car l'écriture d'un
élément de G\diagup\mathrmKer~f
sous la forme x +\
\mathrmKerf n'est pas unique. Il nous faut donc
montrer que x + \mathrmKer~f
= y + \mathrmKer~f \rigtharrow~ f(x) =
f(y), mais on a

\begin{align*} x +\
\mathrmKerf = y +\
\mathrmKerf& \Leftrightarrow & x - y
\in\mathrmKer~f \%&
\\ & \Leftrightarrow & f(x
- y) = 0 \%& \\ &
\Leftrightarrow & f(x) - f(y) = 0\%&
\\ & \Leftrightarrow & f(x)
= f(y) \%& \\
\end{align*}

L'in\\jmathmathectivité en résulte également, puisqu'en lisant les implications de
droite à gauche on obtient

\begin{align*} \overlinef(x
+ \mathrmKer~f) =
\overlinef(y +\
\mathrmKerf)& \rigtharrow~& f(x) = f(y) \%&
\\ & \rigtharrow~& x +\
\mathrmKerf = y +\
\mathrmKerf\%& \\
\end{align*}

La sur\\jmathmathectivité est évidente et le fait que
\overlinef soit un morphisme de groupes nécessite une
vérification élémentaire.

Remarque~1.3.5 On a donc le diagramme suivant

\matrix\,G &f
\rightarrow~~&G' \cr ↓ \pi~& &↑ i
\cr
G\diagup\mathrmKerf&\overlinef~
\rightarrow~&\\mathrmImf~

où i désigne la restriction de l'identité
i:\mathrmIm~f \rightarrow~ G,
y\mapsto~y

\paragraph{1.3.5 Le groupe \mathbb{Z}}

Théorème~1.3.13 (division euclidienne) Soit n \in \mathbb{Z},a \in \mathbb{N}~
\diagdown\0\~; alors il existe un unique
couple (q,r) \in \mathbb{Z}^2 vérifiant

\begin{itemize}
\itemsep1pt\parskip0pt\parsep0pt
\item
  n = aq + r
\item
  0 \leq r \textless{} a
\end{itemize}

Démonstration Si on a n = aq + r = aq' + r', on a a(q - q') = r' - r
avec r' - r \in-a,a~; comme q - q' est un entier, ceci nécessite r' - r =
0, soit r' = r et donc q' = q, d'où l'unicité.

En ce qui concerne l'existence, on considère X = \p \in
\mathbb{Z}∣n - ap ≥ 0\~; X n'est pas
vide car il contient 0 si n ≥ 0 et n si n \textless{} 0~; de plus, il
est ma\\jmathmathoré par n si n ≥ 0 et par 0 si n \textless{} 0 (facile). Il admet
donc un plus grand élément, noté q. On a donc n - aq ≥ 0 et n - a(q + 1)
\textless{} 0, soit encore, en posant r = n - aq, 0 \leq r \textless{} q,
ce que l'on voulait.

Théorème~1.3.14 Les sous-groupes du groupe (\mathbb{Z},+) sont exactement les m\mathbb{Z}
pour m \in \mathbb{N}~.

Démonstration Il est clair que les m\mathbb{Z} pour m \in \mathbb{N}~ sont des sous-groupes
de \mathbb{Z}. Inversement, soit H un sous-groupe de \mathbb{Z}. Si H =
\0\, alors H = 0\mathbb{Z}~; sinon, il contient
un élément x\neq~0 et comme il contient également
- x, on peut supposer x \textgreater{} 0~; soit donc a
= min~\y \textgreater{}
0∣y \in H\, on a a \in H et a
\textgreater{} 0. Comme a \in H et que H est un sous-groupe additif, on a
a\mathbb{Z} \subset~ H. Inversement, soit x \in H~; on peut écrire x = aq + r avec 0 \leq r
\textless{} a~; on a r = x - aq et comme x \in H,aq \in a\mathbb{Z} \subset~ H, on a r \in H.
Comme r \textless{} a = min~\y
\textgreater{} 0∣y \in H\, on
a forcément r = 0, et donc x \in a\mathbb{Z}~; donc H \subset~ a\mathbb{Z} et donc H = a\mathbb{Z}.

Remarque~1.3.6 Ceci nous permettra de parler du groupe quotient \mathbb{Z}\diagupm\mathbb{Z}. On
notera \overlinex pour la classe x + m\mathbb{Z} d'un élément
x de \mathbb{Z} (m étant fixé). On a donc \overlinex =
\overliney \Leftrightarrow
m∣x - y.

\paragraph{1.3.6 Ordre d'un élément}

Définition~1.3.5 Soit G un groupe et x \in G. On dispose alors d'un
morphisme de groupes f_x : \mathbb{Z} \rightarrow~ G défini par f_x(n) =
x^n. Son noyau est un sous-groupe de \mathbb{Z}, donc de la forme
\mathrmKerf_x~ =
m_x\mathbb{Z}, m_x \in \mathbb{N}~. Deux cas sont alors possibles

\begin{itemize}
\itemsep1pt\parskip0pt\parsep0pt
\item
  Premier cas~: m_x = 0, autrement dit f_x est
  in\\jmathmathectif. On dit alors que x est d'ordre infini (ce n'est évidemment
  possible que si G est infini, puisque \mathbb{Z} l'est).
\item
  Deuxième cas~: m_x \textgreater{} 0, autrement dit
  f_x n'est pas in\\jmathmathectif. On dit alors que x est d'ordre fini
  et on appelle m_x l'ordre de x.
\end{itemize}

Proposition~1.3.15 On a aussi m_x =\
min\n \textgreater{}
0∣x^n = e\ et
m_x est le cardinal du sous-groupe \langle
x\rangle de G engendré par x.

Remarque~1.3.7 On a pour un élément x d'ordre fini m,
\langle x\rangle =
\e,x,x^2,\\ldots,x^m-1\~,
l'application de \mathbb{Z}\diagupn\mathbb{Z} dans \langle
x\rangle ,
\overlinek\mapsto~x^k
est un isomorphisme de groupes.

\paragraph{1.3.7 Groupes finis}

Théorème~1.3.16 (de Lagrange) Soit G un groupe fini et H un sous-groupe
de G. Alors on a

Card~G =\
CardH.Card~G\diagupH

et en particulier Card~H divise
Card~G.

Démonstration En effet les éléments de G\diagupH sont les classes
d'équivalence xH~; elles sont toutes de cardinal
Card~H (car l'application
h\mapsto~xh est in\\jmathmathective), elles forment une
partition de G et sont au nombre de Card~G\diagupH.

Corollaire~1.3.17 Soit G un groupe fini. Alors tout élément de G est
d'ordre fini divisant le cardinal de G. En particulier,
\forall~x \in G, x^G~ =
e.

Démonstration On a vu en effet que l'ordre de x est le cardinal du
sous-groupe de G engendré par x.

Corollaire~1.3.18 (Fermat). Soit p un nombre premier. On a
\forall~~x \in \mathbb{Z}\diagupp\mathbb{Z}
\diagdown\0\, x^p-1 = 1 et
\forall~x \in \mathbb{Z}\diagupp\mathbb{Z}, x^p~ = x.

Démonstration \mathbb{Z}\diagupp\mathbb{Z} \diagdown\0\ est en effet
un groupe pour la multiplication, de cardinal p - 1, et donc, si x \in
\mathbb{Z}\diagupp\mathbb{Z} \diagdown\0\, x^p-1 = 1.

\paragraph{1.3.8 Groupes cycliques}

Définition~1.3.6 On dit qu'un groupe G est cyclique s'il est fini et
engendré par un élément a.

Remarque~1.3.8 Le cardinal du groupe est égal à l'ordre m d'un de ses
générateurs. Tout groupe cyclique est isomorphe à un groupe (\mathbb{Z}\diagupm\mathbb{Z},+). Un
tel groupe est donc abélien.

Proposition~1.3.19 Soit G un groupe cyclique de cardinal m engendré par
un élément a. Alors les générateurs de G sont exactement les
a^k avec 1 \leq k \textless{} m, k premier avec m.

Démonstration En effet, soit x un générateur de G, alors x s'écrit sous
la forme x = a^k. De plus, il doit exister \ell tel que a =
x^\ell = a^k\ell. On a donc a^k\ell-1 = e, et
donc m divise k\ell - 1~; il existe donc n \in \mathbb{Z} tel que k\ell - 1 = mn, soit k\ell
+ mn = 1~; d'après le théorème de Bézout, k et m sont premiers entre
eux.

En remontant les implications précédentes, on voit que si k et m sont
premiers entre eux, alors il existe \ell tel que a = x^\ell~; tout
élément de G s'écrivant sous la forme a^q s'écrira également
sous la forme x^\ellq et donc x est un générateur de G.

Remarque~1.3.9 En notation additive, si le groupe cyclique est engendré
par a, ses générateurs sont les ka où k et m sont premiers entre eux. En
particulier, les générateurs de (\mathbb{Z}\diagupm\mathbb{Z},+) sont les
k\overline1 = \overlinek avec k et
m premiers entre eux.

Proposition~1.3.20 Tout sous-groupe d'un groupe cyclique est cyclique,
tout quotient d'un groupe cyclique est cyclique. Tout groupe de cardinal
premier est cyclique, engendré par n'importe lequel de ses éléments
distinct de l'élément neutre.

Démonstration Si G est un groupe cyclique engendré par a et H un
sous-groupe de G, alors G\diagupH est clairement engendré par \pi~(a) où \pi~ est la
pro\\jmathmathection canonique de G sur G\diagupH, puisque \pi~(x) = \pi~(a^k) =
\pi~(a)^k~; donc G\diagupH est cyclique. Montrons que H l'est
également, et pour cela soit n =\
min\k \textgreater{}
0∣a^k \in H\
(l'ensemble est non vide car il contient m puisque a^m = e \in
H). Soit alors x \in H~; on peut écrire x = a^k. Effectuons la
division euclidienne de k par n~; on peut écrire k = nq + r avec 0 \leq r
\textless{} n. On a alors a^r = a^k-nq =
a^ka^-nq = a^kx^-q, ce qui
implique que a^r \in H et donc nécessite r = 0 (puisque r
\textless{} n = min~\k
\textgreater{} 0∣a^k \in
H\)~; on a donc x = a^nq =
(a^n)^q ce qui montre que a^n est un
générateur de H.

\paragraph{1.3.9 Groupe opérant sur un ensemble}

Soit G un groupe, X un ensemble et \phi : G \times X \rightarrow~ X une application, notée
\phi(g,x) = g.x.

Définition~1.3.7 On dit que \phi est une loi de groupe opérant sur un
ensemble si on a les deux propriétés (i)\forall~~x \in
X,e.x = x (ii)\forall~~g,g' \in
G,\forall~~x \in X, g.(g'.x) = (gg').x.

Remarque~1.3.10 Notons \sigma_g : X \rightarrow~ X,
x\mapsto~g.x. On voit que \sigma est un morphisme de
groupes de G dans le groupe Perm~(X) des
permutations de l'ensemble X. Inversement la donnée d'un tel morphisme
de groupes définit une loi de groupe opérant sur un ensemble.

Définition~1.3.8 Soit x \in X On note Stab~(x) =
\g \in G∣g.x =
x\ \subset~ G et Orb~(x) =
\g.x∣g \in G\
\subset~ X appelés stabilisateur et orbite de x.

Théorème~1.3.21 L'application G\diagupStab~(x)
\rightarrow~ Orb~(x),
gStab(x)\mathrel\mapsto~~g.x est
bien définie et c'est une bi\\jmathmathection. La relation sur X définie par x\mathcal{R}y
\Leftrightarrow \exists~g \in G,y = g.x
est une relation d'équivalence sur X dont les classes d'équivalence sont
exactement les orbites~; deux orbites sont donc soit dis\\jmathmathointes, soit
égales.

Démonstration En effet on a

\begin{align*} gStab~(x) =
g'Stab(x)& \mathrel\Leftrightarrow~ &
g^-1g' \in Stab~(x)\%&
\\ & \Leftrightarrow &
g^-1g'.x = x \%& \\ &
\Leftrightarrow & g'.x = g.x \%&
\\ \end{align*}

ce qui montre à la fois que l'application est bien définie et qu'elle
est in\\jmathmathective, sa sur\\jmathmathectivité étant évidente.

Le fait que la relation soit une relation d'équivalence est immédiat,
ainsi que le fait que les classes d'équivalence en soient les orbites.

Corollaire~1.3.22 (formule des classes). Supposons l'ensemble X fini et
soit Y un ensemble de représentants des orbites (c'est-à-dire que Y
contient un et un seul élément de chaque orbite). Alors

Card~(X) = \\sum
_y\inY  Card(G\diagupStab(y))

Démonstration En effet les Orb~(y), y \in Y
forment une partition de X et chaque orbite est en bi\\jmathmathection avec le
quotient G\diagupStab~(y) correspondant.

\paragraph{1.3.10 Groupe des permutations d'un ensemble fini}

Définition~1.3.9 Soit X un ensemble fini. On appelle permutation de X
toute bi\\jmathmathection de X dans X. L'ensemble S_X des permutations de
X est muni par la composition d'une structure de groupe fini. Si
Card~X = n alors
CardS_X~ = n!.

Remarque~1.3.11 On vérifie immédiatement que si deux ensembles ont le
même cardinal, leurs groupes des permutations sont isomorphes. Par la
suite on considérera S_n = S_{[}1,n{]}.

Définition~1.3.10 Soit \sigma \inS_n. On appelle orbite de \sigma toute
orbite pour l'action du groupe \langle
\sigma\rangle sur {[}1,n{]}, autrement dit x et y sont dans la
même orbite si et seulement si \exists~k \in \mathbb{Z}, y =
\sigma^k(x). On dit que \sigma est un cycle si \sigma a une seule orbite non
réduite à un élément (appelée le support du cycle).

Remarque~1.3.12 Soit \sigma un cycle et x dans son support. Alors le support
de \sigma est exactement

\x,\sigma(x),\sigma^2(x),\\ldots,\sigma^k-1(x)\~

et \sigma^k(x) = x. \sigma agit sur ce support par permutation
circulaire et laisse fixe tous les autres éléments.

Proposition~1.3.23 L'ordre d'un cycle est égal au cardinal de son
support. Deux cycles sont con\\jmathmathugués si et seulement si ils ont même
ordre. Deux cycles de supports dis\\jmathmathoints commutent.

Démonstration Si le support de x est X =
\x,\sigma(x),\sigma^2(x),
\\ldots,\sigma^k-1(x)\~,
il est clair que \sigma^k = \mathrmId (vérifier
que \sigma^k(y) = y pour tous les éléments de l'orbite, pour les
autres cela résulte de \sigma(y) = y). De plus, si p \textless{} k, on a
\sigma^p(x)\neq~x et donc
\sigma^p\neq~\mathrmId
ce qui montre bien que k est l'ordre de \sigma.

Il est clair que deux éléments con\\jmathmathugués ont même ordre~; inversement
considérons deux cycles \sigma_1 et \sigma_2 de même ordre k.
L'un a une orbite
\x,\sigma_1(x),\sigma_1^2(x),
\\ldots,\sigma_1^k-1(x)\~
et l'autre une orbite
\y,\sigma_2(y),\sigma_2^2(y),
\\ldots,\sigma_2^k-1(y)\~.
Si l'on prend une permutation \tau vérifiant

\tau(y) = x,\tau(\sigma_2(y)) =
\sigma_1(x),\\ldots,\tau(\sigma_2^k-1~(y))
= \sigma_ 1^k-1(x)

(l'existence d'une telle permutation est évidente), on vérifie que \tau \cdot
\sigma_2 = \sigma_1 \cdot \tau, c'est-à-dire que \sigma_2 =
\tau^-1 \cdot \sigma_1 \cdot \tau.

On vérifie immédiatement que deux cycles de supports dis\\jmathmathoints
commutent.

Théorème~1.3.24 Toute permutation \sigma s'écrit, de manière unique à l'ordre
près, comme produit de cycles de supports deux à deux dis\\jmathmathoints (donc
ces cycles commutent deux à deux). Les supports de ces cycles sont
exactement les orbites de \sigma non réduites à un élément.

Démonstration On note
A_1,\\ldots,A_k~
les orbites de x non réduites à un élément et on définit \sigma_i
par \sigma_i__A_ i =
\sigma__A_i et \sigma_i(x) = x pour
x∉A_i. Alors \sigma_i est un
cycle de support A_i et on vérifie que \sigma =
\sigma_1\\ldots\sigma_k~,
ce qui montre l'existence de la décomposition. Pour l'unicité, il suffit
de remarquer que si
\sigma_1,\\ldots,\sigma_k~
sont des cycles de supports deux à deux dis\\jmathmathoints, les orbites de \sigma =
\sigma_1\\ldots\sigma_k~
sont exactement les orbites A_i des \sigma_i et que
\sigma_i__A_ i =
\sigma__A_i.

Remarque~1.3.13 La démonstration précédente suppose implicitement que la
permutation n'est pas l'identité. Par convention, l'identité sera
considérée comme le produit de zéro cycles (c'est une convention
habituelle en mathématiques de considérer qu'une somme vide est nulle et
qu'un produit vide est l'élément unité).

Définition~1.3.11 On appelle transposition tout cycle d'ordre 2.

Remarque~1.3.14 Une transposition est définie par son orbite
\x,y\. On a \sigma(x) = y, \sigma(y) = x et \sigma(k)
= k pour
k∉\x,y\. On
notera \sigma = \tau_x,y.

Théorème~1.3.25 Toute permutation est produit de transpositions.

Démonstration Par récurrence sur n. Si n = 2, la permutation est soit
l'identité qui est produit de 0 transpositions, soit la transposition
qui échange 1 et 2. Supposons donc le résultat vrai pour n - 1 et soit \sigma
une permutation de
\1,\\ldots,n\~.
Si \sigma(n) = n, alors la restriction \sigma' de \sigma à
\1,\\ldots~,n
- 1\ est une permutation de
\1,\\ldots~,n
- 1\, et donc est produit de transpositions \sigma' =
\tau_1'\\ldots\tau_k~'.
On prolonge les \tau_i' en \tau_i en posant \tau_i(n)
= n et on obtient des transpositions de
\1,\\ldots,n\~
qui vérifient \sigma =
\tau_1\\ldots\tau_k~.
Si \sigma(n) = p\neq~n, on pose \sigma_1 =
\tau_p,n\sigma~; alors \sigma_1(n) = n et d'après le premier cas
étudié, on peut écrire \tau_p,n\sigma = \sigma_1 =
\tau_1\\ldots\tau_k~,
d'où (une transposition étant involutive) \sigma =
\tau_p,n\tau_1\\ldots\tau_k~.

Remarque~1.3.15 Une telle décomposition n'est en aucune
fa\ccon unique.

Définition~1.3.12 Soit \sigma une permutation. On appelle signature de \sigma le
signe \epsilon(\sigma) de l'expression
\∏  _
\i,\\jmathmath\ \atop
i\neq~\\jmathmath  \sigma(\\jmathmath)-\sigma(i) \over
\\jmathmath-i .

Remarque~1.3.16 La définition, qui comporte une indexation sur les
paires \i,\\jmathmath\ et non sur les couples
(i,\\jmathmath) est \\jmathmathustifiée par le fait que  \sigma(\\jmathmath)-\sigma(i) \over
\\jmathmath-i = \sigma(i)-\sigma(\\jmathmath) \over i-\\jmathmath (qui ne dépend donc pas
de l'ordre entre i et \\jmathmath).

Proposition~1.3.26 L'application \epsilon : S_n \rightarrow~\-
1,1\ est un morphisme de groupes multiplicatifs. La
signature d'un cycle d'ordre k est (-1)^k-1 (en particulier
les transpositions sont de signature -1).

Démonstration On a

\begin{align*} \∏
_ \i,\\jmathmath\ \atop
i\neq~\\jmathmath  \sigma\tau(\\jmathmath) - \sigma\tau(i) \over
\\jmathmath - i & =& ∏ _
\i,\\jmathmath\ \atop
i\neq~\\jmathmath  \sigma\tau(\\jmathmath) - \sigma\tau(i) \over
\tau(\\jmathmath) - \tau(i)  \tau(\\jmathmath) - \tau(i) \over \\jmathmath - i \%&
\\ & =& \∏
_ \i,\\jmathmath\ \atop
i\neq~\\jmathmath  \sigma\tau(\\jmathmath) - \sigma\tau(i) \over
\tau(\\jmathmath) - \tau(i) ∏ _
\i,\\jmathmath\ \atop
i\neq~\\jmathmath  \tau(\\jmathmath) - \tau(i) \over \\jmathmath
- i \%& \\
\end{align*}

Mais l'application
\i,\\jmathmath\\mapsto~\\tau(i),\tau(\\jmathmath)\
est une bi\\jmathmathection de l'ensemble des paires dans lui même, et donc en
posant i' = \tau(i),\\jmathmath' = \tau(\\jmathmath), on a

∏ _
\i,\\jmathmath\ \atop
i\neq~\\jmathmath  \sigma\tau(\\jmathmath) - \sigma\tau(i) \over
\\jmathmath - i = ∏ _
\i',\\jmathmath'\ \atop
i'\neq~\\jmathmath'  \sigma(\\jmathmath') - \sigma(i')
\over \\jmathmath' - i' ∏ _
\i,\\jmathmath\ \atop
i\neq~\\jmathmath  \tau(\\jmathmath) - \tau(i) \over \\jmathmath
- i

et en prenant les signes \epsilon(\sigma\tau) = \epsilon(\sigma)\epsilon(\tau) ce qui montre que \sigma est un
morphisme de groupes multiplicatifs.

On en déduit que deux permutations con\\jmathmathuguées ont la même signature~:
\epsilon(\tau\sigma\tau^-1) = \epsilon(\tau)\epsilon(\sigma)\epsilon(\tau)^-1 = \epsilon(\sigma) puisque
\ - 1,1\ est un groupe commutatif.
Comme deux cycles de même ordre sont con\\jmathmathugués, pour calculer la
signature d'un cycle d'ordre k, il suffit de calculer la signature du
cycle \sigma d'ordre k qui effectue une permutation circulaire sur
(1,2,\\ldots~,k) et
qui laisse les autres éléments fixes~; mais pour ce cycle, les paires
\i,\\jmathmath\ telles que i \textless{} \\jmathmath et
\sigma(i) \textgreater{} \sigma(\\jmathmath) (c'est-à-dire telles que  \sigma(\\jmathmath)-\sigma(i)
\over \\jmathmath-i \textless{} 0) sont exactement les paires
\1,k\,\2,k\,\\ldots~,\k
- 1,k\ et il y en a k - 1~; la signature est donc
(-1)^k-1 et en particulier, les transpositions sont de
signature - 1.

Corollaire~1.3.27 Dans une décomposition de \sigma en produit de
transpositions, la parité du nombre de ces transpositions est fixée~:
paire si \epsilon(\sigma) = +1, impaire si \epsilon(\sigma) = -1.
