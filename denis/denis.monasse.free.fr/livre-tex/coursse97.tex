\documentclass[]{article}
\usepackage[T1]{fontenc}
\usepackage{lmodern}
\usepackage{amssymb,amsmath}
\usepackage{ifxetex,ifluatex}
\usepackage{fixltx2e} % provides \textsubscript
% use upquote if available, for straight quotes in verbatim environments
\IfFileExists{upquote.sty}{\usepackage{upquote}}{}
\ifnum 0\ifxetex 1\fi\ifluatex 1\fi=0 % if pdftex
  \usepackage[utf8]{inputenc}
\else % if luatex or xelatex
  \ifxetex
    \usepackage{mathspec}
    \usepackage{xltxtra,xunicode}
  \else
    \usepackage{fontspec}
  \fi
  \defaultfontfeatures{Mapping=tex-text,Scale=MatchLowercase}
  \newcommand{\euro}{€}
\fi
% use microtype if available
\IfFileExists{microtype.sty}{\usepackage{microtype}}{}
\ifxetex
  \usepackage[setpagesize=false, % page size defined by xetex
              unicode=false, % unicode breaks when used with xetex
              xetex]{hyperref}
\else
  \usepackage[unicode=true]{hyperref}
\fi
\hypersetup{breaklinks=true,
            bookmarks=true,
            pdfauthor={},
            pdftitle={Arcs en polaires},
            colorlinks=true,
            citecolor=blue,
            urlcolor=blue,
            linkcolor=magenta,
            pdfborder={0 0 0}}
\urlstyle{same}  % don't use monospace font for urls
\setlength{\parindent}{0pt}
\setlength{\parskip}{6pt plus 2pt minus 1pt}
\setlength{\emergencystretch}{3em}  % prevent overfull lines
\setcounter{secnumdepth}{0}
 
/* start css.sty */
.cmr-5{font-size:50%;}
.cmr-7{font-size:70%;}
.cmmi-5{font-size:50%;font-style: italic;}
.cmmi-7{font-size:70%;font-style: italic;}
.cmmi-10{font-style: italic;}
.cmsy-5{font-size:50%;}
.cmsy-7{font-size:70%;}
.cmex-7{font-size:70%;}
.cmex-7x-x-71{font-size:49%;}
.msbm-7{font-size:70%;}
.cmtt-10{font-family: monospace;}
.cmti-10{ font-style: italic;}
.cmbx-10{ font-weight: bold;}
.cmr-17x-x-120{font-size:204%;}
.cmsl-10{font-style: oblique;}
.cmti-7x-x-71{font-size:49%; font-style: italic;}
.cmbxti-10{ font-weight: bold; font-style: italic;}
p.noindent { text-indent: 0em }
td p.noindent { text-indent: 0em; margin-top:0em; }
p.nopar { text-indent: 0em; }
p.indent{ text-indent: 1.5em }
@media print {div.crosslinks {visibility:hidden;}}
a img { border-top: 0; border-left: 0; border-right: 0; }
center { margin-top:1em; margin-bottom:1em; }
td center { margin-top:0em; margin-bottom:0em; }
.Canvas { position:relative; }
li p.indent { text-indent: 0em }
.enumerate1 {list-style-type:decimal;}
.enumerate2 {list-style-type:lower-alpha;}
.enumerate3 {list-style-type:lower-roman;}
.enumerate4 {list-style-type:upper-alpha;}
div.newtheorem { margin-bottom: 2em; margin-top: 2em;}
.obeylines-h,.obeylines-v {white-space: nowrap; }
div.obeylines-v p { margin-top:0; margin-bottom:0; }
.overline{ text-decoration:overline; }
.overline img{ border-top: 1px solid black; }
td.displaylines {text-align:center; white-space:nowrap;}
.centerline {text-align:center;}
.rightline {text-align:right;}
div.verbatim {font-family: monospace; white-space: nowrap; text-align:left; clear:both; }
.fbox {padding-left:3.0pt; padding-right:3.0pt; text-indent:0pt; border:solid black 0.4pt; }
div.fbox {display:table}
div.center div.fbox {text-align:center; clear:both; padding-left:3.0pt; padding-right:3.0pt; text-indent:0pt; border:solid black 0.4pt; }
div.minipage{width:100%;}
div.center, div.center div.center {text-align: center; margin-left:1em; margin-right:1em;}
div.center div {text-align: left;}
div.flushright, div.flushright div.flushright {text-align: right;}
div.flushright div {text-align: left;}
div.flushleft {text-align: left;}
.underline{ text-decoration:underline; }
.underline img{ border-bottom: 1px solid black; margin-bottom:1pt; }
.framebox-c, .framebox-l, .framebox-r { padding-left:3.0pt; padding-right:3.0pt; text-indent:0pt; border:solid black 0.4pt; }
.framebox-c {text-align:center;}
.framebox-l {text-align:left;}
.framebox-r {text-align:right;}
span.thank-mark{ vertical-align: super }
span.footnote-mark sup.textsuperscript, span.footnote-mark a sup.textsuperscript{ font-size:80%; }
div.tabular, div.center div.tabular {text-align: center; margin-top:0.5em; margin-bottom:0.5em; }
table.tabular td p{margin-top:0em;}
table.tabular {margin-left: auto; margin-right: auto;}
div.td00{ margin-left:0pt; margin-right:0pt; }
div.td01{ margin-left:0pt; margin-right:5pt; }
div.td10{ margin-left:5pt; margin-right:0pt; }
div.td11{ margin-left:5pt; margin-right:5pt; }
table[rules] {border-left:solid black 0.4pt; border-right:solid black 0.4pt; }
td.td00{ padding-left:0pt; padding-right:0pt; }
td.td01{ padding-left:0pt; padding-right:5pt; }
td.td10{ padding-left:5pt; padding-right:0pt; }
td.td11{ padding-left:5pt; padding-right:5pt; }
table[rules] {border-left:solid black 0.4pt; border-right:solid black 0.4pt; }
.hline hr, .cline hr{ height : 1px; margin:0px; }
.tabbing-right {text-align:right;}
span.TEX {letter-spacing: -0.125em; }
span.TEX span.E{ position:relative;top:0.5ex;left:-0.0417em;}
a span.TEX span.E {text-decoration: none; }
span.LATEX span.A{ position:relative; top:-0.5ex; left:-0.4em; font-size:85%;}
span.LATEX span.TEX{ position:relative; left: -0.4em; }
div.float img, div.float .caption {text-align:center;}
div.figure img, div.figure .caption {text-align:center;}
.marginpar {width:20%; float:right; text-align:left; margin-left:auto; margin-top:0.5em; font-size:85%; text-decoration:underline;}
.marginpar p{margin-top:0.4em; margin-bottom:0.4em;}
.equation td{text-align:center; vertical-align:middle; }
td.eq-no{ width:5%; }
table.equation { width:100%; } 
div.math-display, div.par-math-display{text-align:center;}
math .texttt { font-family: monospace; }
math .textit { font-style: italic; }
math .textsl { font-style: oblique; }
math .textsf { font-family: sans-serif; }
math .textbf { font-weight: bold; }
.partToc a, .partToc, .likepartToc a, .likepartToc {line-height: 200%; font-weight:bold; font-size:110%;}
.chapterToc a, .chapterToc, .likechapterToc a, .likechapterToc, .appendixToc a, .appendixToc {line-height: 200%; font-weight:bold;}
.index-item, .index-subitem, .index-subsubitem {display:block}
.caption td.id{font-weight: bold; white-space: nowrap; }
table.caption {text-align:center;}
h1.partHead{text-align: center}
p.bibitem { text-indent: -2em; margin-left: 2em; margin-top:0.6em; margin-bottom:0.6em; }
p.bibitem-p { text-indent: 0em; margin-left: 2em; margin-top:0.6em; margin-bottom:0.6em; }
.paragraphHead, .likeparagraphHead { margin-top:2em; font-weight: bold;}
.subparagraphHead, .likesubparagraphHead { font-weight: bold;}
.quote {margin-bottom:0.25em; margin-top:0.25em; margin-left:1em; margin-right:1em; text-align:\jmathustify;}
.verse{white-space:nowrap; margin-left:2em}
div.maketitle {text-align:center;}
h2.titleHead{text-align:center;}
div.maketitle{ margin-bottom: 2em; }
div.author, div.date {text-align:center;}
div.thanks{text-align:left; margin-left:10%; font-size:85%; font-style:italic; }
div.author{white-space: nowrap;}
.quotation {margin-bottom:0.25em; margin-top:0.25em; margin-left:1em; }
h1.partHead{text-align: center}
.sectionToc, .likesectionToc {margin-left:2em;}
.subsectionToc, .likesubsectionToc {margin-left:4em;}
.subsubsectionToc, .likesubsubsectionToc {margin-left:6em;}
.frenchb-nbsp{font-size:75%;}
.frenchb-thinspace{font-size:75%;}
.figure img.graphics {margin-left:10%;}
/* end css.sty */

\title{Arcs en polaires}
\author{}
\date{}

\begin{document}
\maketitle

\textbf{Warning: 
requires JavaScript to process the mathematics on this page.\\ If your
browser supports JavaScript, be sure it is enabled.}

\begin{center}\rule{3in}{0.4pt}\end{center}

{[}
{[}
{[}{]}
{[}

\subsubsection{18.2 Arcs en polaires}

\paragraph{18.2.1 Coordonnées polaires}

Soit E un plan euclidien rapporté à un repère orthonormé
(O,\vec\imath,\vecȷ). On notera
\vecu(\theta) le vecteur cos~
(\theta)\vec\imath + sin~
(\theta)\vecȷ. On vérifie immédiatement le résultat
suivant

Proposition~18.2.1 Pour tout \theta \in \mathbb{R}~,
(\vecu(\theta),\vecu'(\theta)) est une base
orthonormée de E de même sens que
(\vec\imath,\vecȷ). On a 
d^n \over d\theta^n
\vecu(\theta) =\vec u(\theta + n \pi~
\over 2 ).

On dispose ainsi d'une application P : \mathbb{R}~^2 \rightarrow~ \mathbb{R}~ définie par
P(\rho,\theta) = \rho\vecu(\theta). Le résultat suivant est tout à
fait élémentaire

Lemme~18.2.2 On a

\begin{align*} P(\rho,\theta)& =& P(\rho',\theta')
\Leftrightarrow \left (\rho = \rho' =
0\right ) \%& \\ & &
\text ou \left (\rho =
\rho'\text et \theta = \theta' + 2k\pi~\right ) \%&
\\ & & \text ou
\left (\rho = -\rho'\text et \theta = \theta' + (2k
+ 1)\pi~\right )\%& \\
\end{align*}

pour un k \in ℤ.

Définition~18.2.1 On dit que (\rho,\theta) est un système de coordonnées
polaires de M \in E si M = P(\rho,\theta).

\paragraph{18.2.2 Arcs en coordonnées polaires~: étude locale}

Définition~18.2.2 Soit I un intervalle de \mathbb{R}~ et f : I \rightarrow~ \mathbb{R}~ de classe
C^k. On appelle arc en coordonnées polaires défini par
l'équation \rho = f(\theta), \theta \in I l'arc paramétré (I,F) où F(\theta) = O +
f(\theta)\vecu(\theta) = O + f(\theta)cos~
(\theta)\vec\imath + f(\theta)sin~
(\theta)\vecȷ.

Remarque~18.2.1 Par abus d'écriture on s'autorisera parfois la notation
\rho^(n) à la place de f^(n)(\theta).

La formule de Leibnitz nous fournit immédiatement

F^(n)(\theta) = \\sum
\_\jmath=0^nC\_
n^\jmathf^(\jmath)(\theta)\vecu^(n-\jmath)(\theta)
= \sum \_\jmath=0^nC~\_
n^\jmathf^(\jmath)(\theta)\vecu(\theta + (n - \jmath) \pi~
\over 2 )

En particulier F'(\theta) = f'(\theta)\vecu(\theta) +
f(\theta)\vecu'(\theta) et F'`(\theta) = (f'`(\theta) -
f(\theta))\vecu(\theta) + 2f'(\theta)\vecu'(\theta).

Etude en un point dont l'image n'est pas l'origine

Si f(\theta)\neq~0, on a
F'(\theta)\neq~0 et donc le point est régulier~; le
point est birégulier si et seulement si le produit mixte
{[}F'(\theta),F'`(\theta){]} = \left
\textbar{}\matrix\,f'(\theta)&f'`(\theta) - f(\theta)
\cr f(\theta)&2f'(\theta) \right \textbar{} =
f(\theta)^2 + 2f'(\theta)^2 - f(\theta)f''(\theta) est différent de 0.

Remarquons que le vecteur F'(\theta) n'est pas colinéaire au vecteur
\overrightarrowOF(\theta) ce qui montre que la tangente ne
passe \jmathamais par l'origine. On en déduit que l'origine appartient soit
au demi plan de concavité, soit au demi plan opposé. Elle appartient au
demi plan de concavité si et seulement si~le vecteur
\overrightarrowF(\theta)O a une coordonnée positive
suivant F''(\theta) dans la base (F'(\theta),F''(\theta))~; mais si on écrit - F(\theta)
=\overrightarrow F(\theta)O = \lambda~F'(\theta) + \muF''(\theta), on a
immédiatement {[}F(\theta),F'(\theta){]} = \mu{[}F'(\theta),F''(\theta){]} soit encore
f(\theta)^2 = \mu(f(\theta)^2 + 2f'(\theta)^2 -
f(\theta)f''(\theta)). On en déduit que \mu est du même signe que f(\theta)^2
+ 2f'(\theta)^2 - f(\theta)f''(\theta).

Si l'on appelle \alpha~ l'angle du vecteur F'(\theta) avec
\vecu(\theta), on a
\mathrmtg~ \alpha~ = f(\theta)
\over f'(\theta) . On peut donc résumer en

Théorème~18.2.3 Soit \Gamma l'arc en polaire défini par l'équation \rho = f(\theta),
\theta \in I. Soit \theta un point de I dont l'image n'est pas l'origine
(c'est-à-dire que f(\theta)\neq~0). Alors (i) le point
est régulier, c'est donc soit un point banal, soit un point d'inflexion
(ii) le point est birégulier si et seulement si~\rho^2 +
2\rho'^2 - \rho\rho''\neq~0 (iii) la tangente
ne passe pas par l'origine du repère~; l'origine appartient au demi plan
de concavité si et seulement si~\rho^2 + 2\rho'^2 - \rho\rho''
\textgreater{} 0~; la tangente fait un angle \alpha~ avec le rayon vecteur, où
\alpha~ est donné par

\mathrmtg~ \alpha~ = \rho
\over \rho'

Remarque~18.2.2 On vérifie facilement que lorsque \rho^2 +
2\rho'^2 - \rho\rho'' s'annule en changeant de signe on a un point
d'inflexion. On pourra également remarquer que si l'on pose \phi = 1
\over \rho , l'expression \rho^2 +
2\rho'^2 - \rho\rho'' est du même signe que \phi(\phi + \phi'') ce qui permet
dans certains cas d'alléger les calculs.

Etude en un point dont l'image est l'origine

Si f(\theta) = 0, on voit immédiatement que le point est non totalement
singulier si et seulement si~il existe n tel que
f^(n)(\theta)\neq~0. Supposons donc que
f(\theta) = f'(\theta) =
\\ldots~ =
f^(p-1)(\theta) = 0 et que
f^(p)(\theta)\neq~0. La formule de Leibnitz
écrite ci dessus montre que F'(\theta) =
\\ldots~ =
F^(p-1)(\theta) = 0, que F^(p)(\theta) =
f^(p)(\theta)\vecu(\theta)\neq~0
et que F^(p+1)(\theta) =
f^(p+1)(\theta)\vecu(\theta) + (p +
1)f^(p)(\theta)\vecu'(\theta). On voit donc tout
d'abord que le vecteur \vecu(\theta) est un vecteur
directeur de la tangente qui fait donc un angle \theta avec l'axe Ox = O +
\mathbb{R}~\vec\imath et que d'autre part les vecteurs
F^(p)(\theta) et F^(p+1)(\theta) forment une famille libre.
L'entier q qui intervient dans la classification locale des points est
donc tou\jmathours égal à p + 1, ce qui montre que le point est un point
banal si p est impair et un point de rebroussement de première espèce si
p est pair. En résumé

Théorème~18.2.4 Soit \Gamma l'arc en polaire défini par l'équation \rho = f(\theta),
\theta \in I. Soit \theta un point de I dont l'image est l'origine (c'est-à-dire que
f(\theta) = 0). Soit p tel que f(\theta) = f'(\theta) =
\\ldots~ =
f^(p-1)(\theta) = 0 et
f^(p)(\theta)\neq~0~; alors (i) la tangente
au point \theta est la droite passant par l'origine et faisant l'angle \theta avec
l'axe Ox (ii) le point est un point banal si p est impair et un point de
rebroussement de première espèce si p est pair.

\paragraph{18.2.3 Branches infinies et phénomènes asymptotiques}

Soit \Gamma l'arc en polaire défini par l'équation \rho = f(\theta), \theta \in I. Soit \alpha~ \in
\mathbb{R}~ \cup\-\infty~,+\infty~\ une extrémité de I. On a
\\textbar{}F(\theta)\\textbar{} =
\textbar{}f(\theta)\textbar{}, on a donc une branche infinie si et seulement
si~lim\_\theta\rightarrow~\alpha~~\textbar{}f(\theta)\textbar{} =
+\infty~. Dans ce cas, on a  F(\theta) \over
\\textbar{}F(\theta)\\textbar{}
= sgn(f(\theta))\vecu~(\theta) qui
admet une limite en \alpha~ si et seulement si \alpha~ est fini.

Si \alpha~ = ±\infty~, il n'y a pas de direction asymptotique~; le point F(\theta)
s'éloigne indéfiniment en tournant autour de l'origine~; on dit que la
courbe présente une branche spirale.

Si \alpha~ \in \mathbb{R}~, la droite \mathbb{R}~\vecu(\alpha~) est direction
asymptotique. Le point d'intersection de la droite passant par F(\theta) et
parallèle à la droite \mathbb{R}~\vecu(\alpha~) avec la droite affine
O + \mathbb{R}~\vecu'(\alpha~) a pour ordonnée dans le repère
(\vecu(\alpha~),\vecu'(\alpha~)) le nombre
f(\theta)sin~ (\theta - \alpha~). On en déduit que l'arc admet
une asymptote si et seulement si~f(\theta)sin~ (\theta -
\alpha~) admet une limite \ell quand \theta tend vers \alpha~ dans I~; dans ce cas
l'asymptote est la droite d'équation Y = \ell dans le repère
(O,\vecu(\alpha~),\vecu'(\alpha~)).

Remarque~18.2.3 Dans le cas où \alpha~ \in \pi~ \over 2 ℤ, il y
a de gros risques de confusions entre le repère mobile
(O,\vecu(\alpha~),\vecu'(\alpha~)) et le
repère fixe (0,\vec\imath,\vecȷ). Il
est de beaucoup préférable (i) si \alpha~ \in \pi~ℤ de regarder si y(\theta) =
\rhosin~ \theta admet une limite \ell~; dans ce cas la
droite d'équation y = \ell dans le repère fixe est asymptote (ii) si \alpha~ \in
\pi~ \over 2 + \pi~ℤ de regarder si x(\theta) =
\rhocos~ \theta admet une limite \ell~; dans ce cas la
droite d'équation x = \ell dans le repère fixe est asymptote.

Autres phénomènes asymptotiques On peut également remarquer que si \alpha~ =
±\infty~ est une borne de \alpha~ et si \rho = f(\theta) a une limite r, quand \theta tend vers \alpha~

\begin{itemize}
\itemsep1pt\parskip0pt\parsep0pt
\item
  (i) si r = 0, le point F(\theta) s'enroule autour de l'origine~; on dit que
  l'origine est un point asymptote de l'arc
\item
  (ii) si r\neq~0, le point F(\theta) s'enroule autour
  du cercle de centre O de rayon \textbar{}r\textbar{}~; on dit que ce
  cercle est un cercle asymptote.
\end{itemize}

\paragraph{18.2.4 Plan d'étude d'un arc plan en polaires}

Soit f une fonction de \mathbb{R}~ vers \mathbb{R}~. On considère l'arc défini par
l'équation \rho = f(\theta), \theta \in \mathbb{R}~.

Première étape~: domaine de définition. On détermine le domaine de
définition de f~; c'est en général une réunion finie d'intervalles deux
à deux dis\jmathoints, si bien que la courbe étudiée sera une réunion finie
d'arcs paramétrés.

Deuxième étape~: réduction du domaine d'étude. On recherche les
applications \sigma : D\rightarrow~D tel que pour \theta \inD, F(\sigma(\theta)) =
f(\sigma(\theta))\vecu(\sigma(\theta)) se déduise par une transformation
géométrique simple S de F(\theta) = f(\theta)\vecu(\theta). Soit \Delta
un domaine fondamental pour \sigma.

On recherchera principalement des transformations \theta du type

\begin{itemize}
\itemsep1pt\parskip0pt\parsep0pt
\item
  (i) \sigma(\theta) = \theta + T avec alors \Delta = {[}a,a + T{]} \bigcapD
\item
  (ii) \sigma(\theta) = \omega - \theta avec alors \Delta = {[} \omega \over 2
  ,+\infty~{[}\bigcapD
\end{itemize}

On aura alors

\begin{itemize}
\itemsep1pt\parskip0pt\parsep0pt
\item
  (i) Si f(\theta + T) = f(\theta), alors le point F(\theta + T) se déduit du point
  F(\theta) par la rotation de centre O et d'angle T~; trois cas sont alors à
  examiner

  \begin{itemize}
  \itemsep1pt\parskip0pt\parsep0pt
  \item
    a) si T \in 2\pi~ℤ, on étudie sur \Delta = {[}a,a + T{]} \bigcapD et c'est terminé
  \item
    b) si T \in \pi~ℚ, T = 2\pi~ p \over q , on étudie sur \Delta
    = {[}a,a + T{]} \bigcapD et on complète par q - 1 rotations d'angle T,
    2T,\\ldots~,(q -
    1)T
  \item
    c) si T∉\pi~ℚ, on étudie sur \Delta = {[}a,a +
    T{]} \bigcapD et on complète par une infinité de rotations d'angle kT, k \in
    ℤ
  \end{itemize}
\item
  (ii) Si f(\theta + T \over 2 ) = -f(\theta), alors le point
  F(\theta + T \over 2 ) se déduit du point F(\theta) par la
  rotation de centre O et d'angle \pi~ + T \over 2 ~; on
  étudie sur \Delta = {[}a,a + T \over 2 {]} \bigcapD, la
  discussion est ensuite similaire (on a bien entendu dans ce cas
  également f(\theta + T) = f(\theta))~; dans le cas particulier ou f(\theta + \pi~) =
  -f(\theta), on obtient toute la courbe en faisant varier \theta dans {[}a,a +
  \pi~{]}.
\item
  (iii) Si f(\omega - \theta) = f(\theta), alors le point F(\omega - \theta) se déduit du point
  F(\theta) par la symétrie orthogonale par rapport à la droite
  D\_\omega\diagup2 qui fait l'angle \omega\diagup2 avec l'axe Ox~; on étudie sur \Delta =
  {[} \omega \over 2 ,+\infty~{[}\bigcapD, et on complète par la
  symétrie par rapport à la droite D\_\omega\diagup2 (remarquer que \omega = 0
  correspond à une symétrie par rapport à Ox et \omega = \pi~ à une symétrie par
  rapport à Oy).
\item
  (iv) Si f(\omega - \theta) = -f(\theta), alors le point F(\omega - \theta) se déduit du point
  F(\theta) par la symétrie orthogonale par rapport à la droite
  D\_(\omega+\pi~)\diagup2 qui fait l'angle (\omega + \pi~)\diagup2 avec l'axe Ox~; on
  étudie sur \Delta = {[} \omega \over 2 ,+\infty~{[}\bigcapD, et on
  complète par la symétrie par rapport à la droite D\_(\omega+\pi~)\diagup2
  (remarquer que \omega = 0 correspond à une symétrie par rapport à Oy et \omega =
  \pi~ à une symétrie par rapport à Ox).
\end{itemize}

Troisième étape~: signe de \rho. On étudie le signe de \rho = f(\theta)~; au
passage on repère les points dont l'image est l'origine~; on peut
immédiatement placer les tangentes en ces points~: elles font l'angle
correspondant avec l'axe Ox

Quatrième étape~: étude des branches infinies. L'arc admet en \alpha~
\in\overlineD une branche infinie si et seulement si
lim\_\theta\rightarrow~\alpha~~\textbar{}f(\theta)\textbar{} = +\infty~.
On reproduit la discussion dé\jmathà faite. On peut également détecter
d'autres phénomènes asymptotiques comme point ou cercle asymptotes.

Une étude complémentaire de signe peut parfois préciser la position du
point F(\theta) par rapport à une asymptote, ce qui peut permettre de
préciser un tracé.

Cinquième étape~: ébauche de tracé. En s'aidant d'une calculatrice ou
d'un ordinateur, on peut calculer un certain nombre de points
supplémentaires en plus des points remarquables~; ceci, en plus des
points et des tangentes remarquables, du signe de \rho et de l'étude des
branches infinies permet en général une ébauche convaincante du tracé.

Etapes facultatives

Si la question est posée ou si l'ébauche du tracé suggère la nécessité
de certaines précisions, on peut procéder à quelques étapes
supplémentaires

Sixième étape~: variations de \rho = f(\theta). On étudie le signe de la dérivée
\rho' = f'(\theta). Au passage on repère des points remarquables où
f'(\theta\_0) = 0,f(\theta\_0)\neq~0~:
points où la tangente est orthogonale au rayon vecteur F(\theta\_0)

Septième étape~: détermination des points non biréguliers et étude de la
concavité. On étudie le signe de \rho^2 + 2\rho'^2 -
\rho\rho'' ou encore de \phi(\phi + \phi'') avec \phi = 1 \over \rho ~;
l'annulation correspond aux points non biréguliers, la positivité aux
points où l'origine appartient au demi plan de concavité

Huitième étape~: détermination des points multiples. Il s'agit de
résoudre l'équation F(\theta) = F(\theta') c'est-à-dire encore les systèmes f(\theta) =
f(\theta + 2k\pi~), k \in ℤ^∗, et f(\theta) = -f(\theta + (2k + 1)\pi~), k \in ℤ.

\paragraph{18.2.5 Equations polaires remarquables}

Equation polaire d'une droite ne passant pas par l'origine

Une telle droite a dans un repère orthonormé d'origine O une équation du
type xcos \theta\_0~ +
ysin \theta\_0~ - h = 0 avec
h\neq~0 (où \vecn
= cos \theta\_0\vec\imath~
+ sin \theta\_0\vecȷ~ est
un vecteur normal à la droite et où h désigne la distance de O à la
droite, orientée par le choix de \vecn). En reportant
x = \rhocos~ \theta et y =
\rhosin~ \theta, on obtient l'équation

\rho = h \over cos~ (\theta -
\theta\_0)

Inversement il est clair qu'une telle équation définit une droite ne
passant pas par O.

Remarque~18.2.4 Il suffit de faire varier \theta dans un intervalle de
longueur \pi~ pour avoir toute la droite.

Equation polaire d'un cercle passant par l'origine

Un tel cercle a dans un repère orthonormé d'origine O une équation du
type x^2 + y^2 - 2\alpha~x - 2\beta~y = 0 où le point \Omega de
coordonnées (\alpha~,\beta~) est le centre du cercle. Posons \alpha~ =
Rcos \theta\_0~ et \beta~ =
Rsin \theta\_0~ (où R est bien évidemment le
rayon du cercle puisque celui ci passe par O). En reportant x =
\rhocos \theta et y = \rho\sin~
\theta, on obtient l'équation

\rho = 2Rcos (\theta - \theta\_0~)

Inversement il est clair qu'une telle équation définit un cercle passant
par O.

Remarque~18.2.5 Il suffit de faire varier \theta dans un intervalle de
longueur \pi~ pour avoir tout le cercle.

Equation polaire d'une conique ayant l'origine pour foyer

Soit e son excentricité et D : xcos~
\theta\_0 + ysin \theta\_0~ - h = 0 la
directrice correspondant au foyer O. La conique est alors
\m \in E∣d(m,F) =
ed(m,D)\. Elle est donc définie par l'équation
x^2 + y^2 = e(xcos~
\theta\_0 + ysin \theta\_0~ -
h)^2. En portant x = \rhocos~ \theta et y =
\rhosin \theta, on obtient l'équation \rho^2~ =
e^2(\rhocos (\theta - \theta\_0~) - h)
soit encore \rho = ±e(\rhocos (\theta - \theta\_0~) - h
ou encore \rho = ±eh \over 1±e\
cos (\theta-\theta\_0) .Remarquons alors que les deux courbes
d'équations polaires \rho = eh \over
1+e cos (\theta-\theta\_0)~ et \rho = - eh
\over 1-e cos~
(\theta-\theta\_0) se déduisent l'une de l'autre par le changement de
(\rho,\theta) en (-\rho,\theta + \pi~) ce qui redonne le même point géométrique. Elles ont
donc la même image. Donc la conique est entièrement définie par l'une
des deux équations soit par exemple

\rho = p \over 1 + ecos~ (\theta -
\theta\_0)

où \theta décrit un intervalle de longueur 2\pi~ et p = eh. En remontant les
calculs on vérifie immédiatement qu'inversement une telle équation
polaire définit une conique de foyer O.

{[}
{[}
{[}
{[}

\end{document}
