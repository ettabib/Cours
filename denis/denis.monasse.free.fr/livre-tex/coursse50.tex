\section{Subdivisions, approximation des fonctions}

\subsection{Subdivisions d'un segment}

\begin{de}
Soit $[a,b]$ un segment de $\mathbb{R}$. On appelle subdivision de $[a,b]$ toute famille $\sigma = (a_i)_{0 \leq i \leq n}$ telle que $a_0 = a$, $a_n = b$ et $\forall i \in [1,n], a_{i-1} < a_i$. On appelle pas de la subdivision $\sigma$ le nombre réel $\delta(\sigma) = \min_{i \in [1,n]}(a_i - a_{i-1})$.
\index{subdivision}
\index{pas d'une subdivision}
\end{de}

On notera $\mathrm{Pt}(\sigma) = \{a_i \mid 0 \leq i \leq n\}$ et $S([a,b])$ l'ensemble des subdivisions de $[a,b]$. On définit une relation d'ordre partielle sur $S(I)$ en disant que $\sigma'$ est plus fine que $\sigma$ si $\mathrm{Pt}(\sigma) \subset \mathrm{Pt}(\sigma')$ (on a alors clairement $\delta(\sigma') \leq \delta(\sigma)$). On notera $\sigma \cup \sigma'$ la subdivision définie par $\mathrm{Pt}(\sigma \cup \sigma') = \mathrm{Pt}(\sigma) \cup \mathrm{Pt}(\sigma')$. Elle est plus fine que $\sigma$ et que $\sigma'$.

\subsection{Propriétés liées aux subdivisions}

\begin{de}
On dit que $f : [a,b] \to E$ est une fonction en escalier (resp. affine par morceaux) s'il existe une subdivision $\sigma = (a_i)_{0 \leq i \leq n}$ de $[a,b]$ (que l'on dira adaptée à $f$) telle que $f$ soit constante (resp. affine) sur chacun des intervalles ouverts $]a_{i-1},a_i[$.
\index{fonction en escalier}
\index{fonction affine par morceaux}
\end{de}

\begin{rem}
Il est clair que toute fonction affine par morceaux est bornée et que toute fonction en escalier est affine par morceaux.
\end{rem}

\begin{de}
On dit que $f : [a,b] \to E$ est une fonction de classe $C^k$ par morceaux s'il existe une subdivision $\sigma = (a_i)_{0 \leq i \leq n}$ de $[a,b]$ (que l'on dira adaptée à $f$) telle que l'on ait les conditions équivalentes:
\begin{enumerate}
  \item pour chaque $i \in [1,n]$, il existe une fonction $f_i : [a_{i-1},a_i] \to E$ de classe $C^k$ telle que $\forall t \in ]a_{i-1},a_i[, f(t) = f_i(t)$
  \item la fonction $f$ est de classe $C^k$ sur $[a,b] \setminus \{a_0,\ldots,a_n\}$ et $f,f',\ldots,f^{(k)}$ admettent des limites à gauche et à droite en tous les points $a_i$ où cela a un sens.
\end{enumerate}
\index{fonction de classe $C^k$ par morceaux}
\end{de}

\begin{proof}
$(i) \Rightarrow (ii)$ est clair puisque dans ce cas $\lim_{t \to a_i^+} f^{(p)}(t) = \lim_{t \to a_i^+} f_{i+1}^{(p)}(t) = f_{i+1}^{(p)}(a_i)$ et $\lim_{t \to a_i^-} f^{(p)}(t) = \lim_{t \to a_i^-} f_i^{(p)}(t) = f_i^{(p)}(a_i)$.

En ce qui concerne $(ii) \Rightarrow (i)$ posons
\[
f_i(t) = \begin{cases}
f(t) &\text{si } a_{i-1} < t < a_i \\
f(a_{i-1}^+) &\text{si } t = a_{i-1} \\
f(a_i^-) &\text{si } t = a_i
\end{cases}
\]
alors la fonction $f_i$ est continue sur $[a_{i-1},a_i]$, de classe $C^k$ sur $]a_{i-1},a_i[$ et toutes les dérivées $f_i^{(p)} = f^{(p)}$ admettent des limites aux points $a_{i-1}$ et $a_i$. On a vu dans le chapitre sur les fonctions d'une variable qu'une telle fonction était de classe $C^k$.
\end{proof}

\begin{rem}
Une fonction de classe $C^k$ par morceaux n'est pas nécessairement continue (en particulier $f(a_i)$ peut être distinct de la limite à gauche et de la limite à droite au point $a_i$). Cependant, comme chacune des $f_i$ est continue sur un compact donc bornée, et que les $a_i$ sont en nombre fini, une fonction de classe $C^k$ par morceaux est bornée. Remarquons également qu'une fonction affine par morceaux (et a fortiori une fonction en escalier) est de classe $C^\infty$ par morceaux.
\end{rem}

\begin{rem}
Si $f$ est en escalier, ou affine par morceaux, ou $C^k$ par morceaux et si $\sigma \in S([a,b])$ est adaptée à $f$, alors toute subdivision plus fine est encore adaptée à $f$. En particulier si $\sigma$ est adaptée à $f$ et $\sigma'$ adaptée à $g$, alors $\sigma \cup \sigma'$ est adaptée à la fois à $f$ et à $g$, d'où l'on déduit immédiatement la proposition suivante:
\end{rem}

\begin{prop}
L'ensemble des applications en escalier (resp. affine par morceaux, resp. $C^k$ par morceaux) est un sous-espace vectoriel de l'ensemble des applications de $[a,b]$ dans $E$.
\end{prop}

\begin{de}[Extension]
Si $I$ est un intervalle de $\mathbb{R}$, on dira que $f : I \to E$ est en escalier (resp. affine par morceaux, resp. $C^k$ par morceaux) si sa restriction à tout segment $[a,b]$ contenu dans $I$ est en escalier (resp. affine par morceaux, resp. $C^k$ par morceaux).
\end{de}

\subsection{Approximation des fonctions}

Soit $[a,b]$ un segment de $\mathbb{R}$ et $\mathcal{B}([a,b],E)$ l'espace vectoriel des fonctions bornées de $[a,b]$ dans $E$. Pour $f \in \mathcal{B}([a,b],E)$, on notera $\|f\|_\infty = \sup_{x \in [a,b]} \|f(x)\|$.

\begin{de}
On dit que $f : [a,b] \to E$ est réglée si elle vérifie les conditions équivalentes:
\begin{enumerate}
  \item pour tout $\epsilon > 0$, il existe $\phi : [a,b] \to E$ en escalier telle que $\sup_{t \in [a,b]} \|f(t) - \phi(t)\| \leq \epsilon$
  \item il existe une suite $\phi_n$ de fonctions en escalier de $[a,b]$ dans $E$ telle que $\lim_{n \to +\infty} \left(\sup_{t \in [a,b]} \|f(t) - \phi_n(t)\|\right) = 0$.
\end{enumerate}
L'ensemble des fonctions réglées de $[a,b]$ dans $E$ est un sous-espace vectoriel de $\mathcal{B}([a,b],E)$.
\index{fonction réglée}
\end{de}

\begin{proof}
Les deux définitions sont clairement équivalentes (prendre $\epsilon = \frac{1}{n}$ pour $(i) \Rightarrow (ii)$). La définition $(i)$ implique évidemment que $f - \phi$ est bornée et comme $\phi$ l'est également, une fonction réglée sur un segment est nécessairement bornée. Autrement dit, le sous-espace vectoriel des fonctions réglées de $[a,b]$ dans $E$ (car il est évidemment stable par combinaisons linéaires) n'est autre que l'adhérence de l'espace vectoriel des fonctions en escalier pour la norme $\|\cdot\|_\infty$.
\end{proof}

\begin{thm}
Toute fonction continue par morceaux est réglée.
\end{thm}

\begin{proof}
Commençons par le démontrer pour une fonction continue sur un segment. Une telle fonction est uniformément continue. Soit $\epsilon > 0$. Il existe $\eta > 0$ tel que $\forall t,t' \in [a,b], |t - t'| < \eta \Rightarrow \|f(t) - f(t')\| < \epsilon$. Soit alors $\sigma = (a_i)_{0 \leq i \leq n}$ une subdivision de pas plus petit que $\eta$ et définissons $\phi : [a,b] \to E$ par $\phi(a_i) = f(a_i)$ pour $i \in [0,n]$ et $\phi(t) = f(\frac{a_{i-1}+a_i}{2})$ si $t \in ]a_{i-1},a_i[$ avec $i \in [1,n]$. Alors $\|f(t) - \phi(t)\|$ vaut $0$ si $t$ est l'un des $a_i$ et $\|f(t) - f(\frac{a_{i-1}+a_i}{2})\| \leq \epsilon$ si $t \in ]a_{i-1},a_i[$ puisque alors $|t - \frac{a_{i-1}+a_i}{2}| < \delta(\sigma) < \eta$. On a bien une fonction $\phi$ en escalier telle que $\sup_{t \in [a,b]} \|f(t) - \phi(t)\| \leq \epsilon$.

Si maintenant $f$ est continue par morceaux, soit $\sigma = (a_i)_{0 \leq i \leq n}$ une subdivision adaptée à $f$ et soit $f_i : [a_{i-1},a_i] \to E$ de classe $C^0$ telle que $\forall t \in ]a_{i-1},a_i[, f(t) = f_i(t)$. On peut appliquer le cas précédent à $f_i$ et trouver $\phi_i$ en escalier telle que $\sup_{t \in [a_{i-1},a_i]} \|f_i(t) - \phi_i(t)\| \leq \epsilon$. On définit alors une fonction en escalier $\phi : [a,b] \to E$ par $\phi(a_i) = f(a_i)$ et $\phi(t) = \phi_i(t)$ si $t \in ]a_{i-1},a_i[$. Alors on a $\sup_{t \in [a,b]} \|f(t) - \phi(t)\| \leq \max_{i \in [1,n]} \left(\sup_{t \in [a_{i-1},a_i]} \|f_i(t) - \phi_i(t)\|\right) \leq \epsilon$, ce qui montre que $f$ est réglée.
\end{proof}

En fait, on peut montrer le résultat plus général suivant (que nous n'utiliserons pas par la suite)

\begin{thm}
Une fonction $f : [a,b] \to E$ (espace vectoriel normé complet) est réglée si et seulement si elle admet en tout point de $[a,b]$ (où cela a un sens) une limite à gauche et une limite à droite.
\end{thm}

\begin{proof}
Supposons tout d'abord $f$ réglée; soit $x_0$ un point de $[a,b]$ et montrons que, si $x_0 \neq b$, $f$ a une limite à droite au point $x_0$ à l'aide du critère de Cauchy pour les fonctions. Soit donc $\epsilon > 0$ et $\phi$ en escalier telle que $\forall t \in [a,b], \|f(t) - \phi(t)\| < \frac{\epsilon}{3}$. Soit $(a_i)$ une subdivision adaptée à $\phi$ et soit $\eta > 0$ tel que $]x_0,x_0 + \eta[ \subset ]a_{i-1},a_i[$. Pour $t,t' \in ]x_0,x_0 + \eta[$, on a $\|f(t) - f(t')\| = \|(f(t) - \phi(t)) + (\phi(t') - f(t'))\|$ (car $\phi(t) = \phi(t')$), soit $\|f(t) - f(t')\| \leq \frac{\epsilon}{3} + \frac{\epsilon}{3} < \epsilon$. La fonction $f$ vérifie donc le critère de Cauchy en $x_0$ à droite, et donc elle admet une limite à droite.

Inversement, supposons que $f$ admette en tout point de $[a,b]$ une limite à gauche et une limite à droite et soit $\epsilon > 0$. Alors, pour tout $x \in [a,b]$, il existe $\eta_x > 0$ tel que $\forall t,t' \in ]x,x + \eta_x[, \|f(t) - f(t')\| < \epsilon$ et $\forall t,t' \in ]x - \eta_x,x[, \|f(t) - f(t')\| < \epsilon$ (critère de Cauchy comme condition nécessaire d'existence des limites). On a alors $[a,b] \subset \bigcup_{x \in [a,b]} ]x - \eta_x,x + \eta_x[$ (recouvrement de $[a,b]$ par des ouverts). D'après le théorème de Borel-Lebesgue, on peut trouver $x_1,\ldots,x_p$ tels que $[a,b] \subset ]x_1 - \eta_{x_1},x_1 + \eta_{x_1}[ \cup \ldots \cup ]x_p - \eta_{x_p},x_p + \eta_{x_p}[$. Soit alors $(a_i)_{0 \leq i \leq n}$ une subdivision de $[a,b]$ telle que, pour tout $i \in [1,n]$, il existe $j \in [$[1,p]$ tel que $]a_{i-1},a_i[ \subset ]x_j - \eta_{x_j},x_j[$ ou $]a_{i-1},a_i[ \subset ]x_j,x_j + \eta_{x_j}[$. On a alors $\forall t,t' \in ]a_{i-1},a_i[, \|f(t) - f(t')\| < \epsilon$. On définit alors une fonction $\phi : [a,b] \to E$ par $\phi(a_i) = f(a_i)$ et $\phi(t) = f(\frac{a_{i-1}+a_i}{2})$ si $t \in ]a_{i-1},a_i[$ avec $i \in [1,n]$. Alors on a $\forall t \in [a,b], \|f(t) - \phi(t)\| \leq \epsilon$. Donc $f$ est réglée.
\end{proof}

\begin{rem}
En particulier, on retrouve que les fonctions continues par morceaux, mais aussi les fonctions monotones, sont réglées.
\end{rem}

Enfin, pour terminer sur le problème de l'approximation des fonctions nous donnerons les résultats suivants permettant d'approcher une fonction continue soit par une fonction continue et affine par morceaux, soit par une fonction polynomiale, soit par un polynôme trigonométrique.

\begin{thm}
Soit $f : [a,b] \to \mathbb{R}$ continue. Alors, pour tout $\epsilon > 0$, il existe une fonction $\phi : [a,b] \to E$ continue et affine par morceaux telle que $\forall t \in [a,b], |f(t) - \phi(t)| < \epsilon$.
\end{thm}

\begin{proof}
Une telle fonction est uniformément continue. Soit $\epsilon > 0$. Il existe $\eta > 0$ tel que $\forall t,t' \in [a,b], |t - t'| < \eta \Rightarrow |f(t) - f(t')| < \frac{\epsilon}{2}$. Soit alors $\sigma = (a_i)_{0 \leq i \leq n}$ une subdivision de pas plus petit que $\eta$ et définissons $\phi : [a,b] \to E$ par $\phi(a_i) = f(a_i)$ pour $i \in [0,n]$, $\phi(t) = f(a_{i-1}) + \frac{f(a_i)-f(a_{i-1})}{a_i-a_{i-1}}(t - a_{i-1})$ si $t \in ]a_{i-1},a_i[$. Alors $\phi$ est clairement affine par morceaux et continue. Alors $|f(t) - \phi(t)|$ vaut $0$ si $t$ est l'un des $a_i$ et si $t \in ]a_{i-1},a_i[$, on a (en tenant compte de $f(a_{i-1}) = \phi(a_{i-1})$)

\begin{align*}
|f(t) - \phi(t)| &\leq |f(t) - f(a_{i-1})| + |\phi(a_{i-1}) - \phi(t)| \\
&\leq |f(t) - f(a_{i-1})| + |\phi(a_{i-1}) - \phi(a_i)| \\
&= |f(t) - f(a_{i-1})| + |f(a_{i-1}) - f(a_i)| \\
&< 2\frac{\epsilon}{2} = \epsilon
\end{align*}

puisque, $\phi$ étant affine sur $[a_{i-1},a_i]$, on a $|\phi(a_{i-1}) - \phi(t)| \leq |\phi(a_{i-1}) - \phi(a_i)|$. Ceci termine la démonstration.
\end{proof}

\begin{thm}[Premier théorème de Weierstrass]
Soit $f : [a,b] \to \mathbb{C}$ continue. Alors, pour tout $\epsilon > 0$, il existe un polynôme $P \in \mathbb{C}[X]$ tel que $\forall t \in [a,b], |f(t) - P(t)| < \epsilon$.
\end{thm}

\begin{proof}
Ce résultat pourra être admis. Nous en donnerons cependant une démonstration qui construit effectivement un tel polynôme (appelé polynôme de Bernstein, de tels polynômes jouent un rôle important en infographie). Il suffit évidemment de montrer ce résultat lorsque $a = 0$ et $b = 1$ (on fait ensuite un changement de variable affine qui transforme $[0,1]$ en $[a,b]$). Pour cela on part des identités élémentaires suivantes (la première est la formule du binôme, les deux autres s'en déduisent en dérivant par rapport à $u$)

\begin{align*}
(u + v)^n &= \sum_{k=0}^n C_n^k u^k v^{n-k} \\
n(u + v)^{n-1} &= \sum_{k=1}^n C_n^k k u^{k-1} v^{n-k} \\
n(n - 1)(u + v)^{n-2} &= \sum_{k=2}^n C_n^k k(k - 1) u^{k-2} v^{n-k}
\end{align*}

Changeant $u$ en $x$ et $v$ en $1 - x$, on en déduit que

\begin{align*}
1 &= \sum_{k=0}^n C_n^k x^k (1 - x)^{n-k} \\
n &= \sum_{k=1}^n C_n^k k x^{k-1} (1 - x)^{n-k} \\
n(n - 1) &= \sum_{k=2}^n C_n^k k(k - 1) x^{k-2} (1 - x)^{n-k}
\end{align*}

Soit $a \in \mathbb{R}$. Écrivons alors que

$\left(\frac{a - k}{n}\right)^2 = a^2 + (\frac{1}{n^2} - \frac{2a}{n})k + \frac{1}{n^2}k(k - 1)$

On en déduit que

\begin{align*}
\sum_{k=0}^n C_n^k \left(\frac{a - k}{n}\right)^2 x^k (1 - x)^{n-k} &= a^2 \sum_{k=0}^n C_n^k x^k (1 - x)^{n-k} + (\frac{1}{n^2} - \frac{2a}{n}) \sum_{k=0}^n C_n^k k x^k (1 - x)^{n-k} \\
&\quad + \frac{1}{n^2} \sum_{k=0}^n C_n^k k(k - 1) x^k (1 - x)^{n-k} \\
&= a^2 \sum_{k=0}^n C_n^k x^k (1 - x)^{n-k} + (\frac{1}{n^2} - \frac{2a}{n}) \sum_{k=1}^n C_n^k k x^k (1 - x)^{n-k} \\
&\quad + \frac{1}{n^2} \sum_{k=2}^n C_n^k k(k - 1) x^k (1 - x)^{n-k} \\
&= a^2 \sum_{k=0}^n C_n^k x^k (1 - x)^{n-k} + (\frac{1}{n^2} - \frac{2a}{n}) x \sum_{k=1}^n C_n^k k x^{k-1} (1 - x)^{n-k} \\
&\quad + \frac{1}{n^2} x^2 \sum_{k=2}^n C_n^k k(k - 1) x^{k-2} (1 - x)^{n-k} \\
&= a^2 + (\frac{1}{n^2} - \frac{2a}{n}) x n + \frac{1}{n^2} x^2 n(n - 1) = (x - a)^2 + \frac{x(1 - x)}{n}
\end{align*}

puis en remplaçant $a$ par $x$

$\sum_{k=0}^n C_n^k \left(\frac{x - k}{n}\right)^2 x^k (1 - x)^{n-k} = \frac{x(1 - x)}{n}$

Soit $\delta > 0$. On a donc, pour $x \in [0,1]$,

\begin{align*}
\delta^2 \sum_{|\frac{x - k}{n}| \geq \delta} C_n^k x^k (1 - x)^{n-k} &\leq \sum_{|\frac{x - k}{n}| \geq \delta} C_n^k \left(\frac{x - k}{n}\right)^2 x^k (1 - x)^{n-k} \\
&\leq \sum_{k=0}^n C_n^k \left(\frac{x - k}{n}\right)^2 x^k (1 - x)^{n-k} \\
&= \frac{x(1 - x)}{n} \leq \frac{1}{4n}
\end{align*}

soit encore

$\sum_{|\frac{x - k}{n}| \geq \delta} C_n^k x^k (1 - x)^{n-k} \leq \frac{1}{4n\delta^2}$

Soit $f : [0,1] \to \mathbb{C}$ continue et posons $B_n(x) = \sum_{k=0}^n f(\frac{k}{n}) C_n^k x^k (1 - x)^{n-k}$ (polynôme en $x$ de degré inférieur ou égal à $n$). Écrivons

$f(x) = f(x) 1 = \sum_{k=0}^n f(x) C_n^k x^k (1 - x)^{n-k}$

On a alors

\begin{align*}
|f(x) - B_n(x)| &= \left|\sum_{k=0}^n (f(x) - f(\frac{k}{n})) C_n^k x^k (1 - x)^{n-k}\right| \\
&\leq \sum_{k=0}^n \left|f(x) - f(\frac{k}{n})\right| C_n^k x^k (1 - x)^{n-k}
\end{align*}

Soit $\epsilon > 0$, puisque $f$ est continue sur $[0,1]$, elle est uniformément continue et donc il existe $\delta > 0$ tel que, pour tout couple $t,t'$ vérifiant $|t - t'| < \delta$, on ait $|f(t) - f(t')| < \frac{\epsilon}{2}$. On a alors en majorant suivant les cas $\left|f(x) - f(\frac{k}{n})\right|$ par $\frac{\epsilon}{2}$ ou par $2\|f\|_\infty$

\begin{align*}
|f(x) - B_n(x)| &\leq \sum_{|\frac{x - k}{n}| < \delta} \left|f(x) - f(\frac{k}{n})\right| C_n^k x^k (1 - x)^{n-k} \\
&\quad + \sum_{|\frac{x - k}{n}| \geq \delta} \left|f(x) - f(\frac{k}{n})\right| C_n^k x^k (1 - x)^{n-k} \\
&\leq \frac{\epsilon}{2} \sum_{|\frac{x - k}{n}| < \delta} C_n^k x^k (1 - x)^{n-k} \\
&\quad + 2\|f\|_\infty \sum_{|\frac{x - k}{n}| \geq \delta} C_n^k x^k (1 - x)^{n-k} \\
&\leq \frac{\epsilon}{2} \sum_{k=0}^n C_n^k x^k (1 - x)^{n-k} \\
&\quad + 2\|f\|_\infty \sum_{|\frac{x - k}{n}| \geq \delta} C_n^k x^k (1 - x)^{n-k} \\
&\leq \frac{\epsilon}{2} + 2\|f\|_\infty \frac{1}{4n\delta^2}
\end{align*}

Prenons alors $n$ assez grand pour que $2\|f\|_\infty \frac{1}{4n\delta^2} < \frac{\epsilon}{2}$. On a alors, pour tout $x \in [0,1]$, $|f(x) - B_n(x)| < \epsilon$, ce qui achève la démonstration.
\end{proof}

\begin{de}
On appelle polynôme trigonométrique de coefficients $a_p \in \mathbb{C}, p = -N,\ldots,N$ la fonction périodique de période $2\pi$ de $\mathbb{R}$ dans $\mathbb{C}$, $t \mapsto \sum_{p=-N}^N a_p e^{ipt}$.
\index{polynôme trigonométrique}
\end{de}

\begin{rem}
Les coefficients $a_p$ sont uniquement déterminés puisque les fonctions $t \mapsto e^{ipt}$ forment une famille libre (ce sont par exemple des vecteurs propres de l'opérateur de dérivation).
\end{rem}

On vérifie immédiatement que les polynômes trigonométriques forment une sous-algèbre de l'algèbre des fonctions de classe $C^\infty$ de $\mathbb{R}$ dans $\mathbb{C}$.

\begin{thm}[Deuxième théorème de Weierstrass]
Soit $f : \mathbb{R} \to \mathbb{C}$ continue, périodique de période $2\pi$. Alors, pour tout $\epsilon > 0$, il existe un polynôme trigonométrique $g : \mathbb{R} \to \mathbb{C}$ tel que $\forall t \in \mathbb{R}, |f(t) - g(t)| < \epsilon$.
\end{thm}

\begin{proof}
Ce théorème sera admis : c'est une conséquence facile du théorème de Dirichlet qui dit qu'une fonction continue, de classe $$\mathcal{C}^1$ par morceaux, périodique de période $2\pi$ est somme de sa série de Fourier qui converge normalement, donc peut être approchée à $\frac{\epsilon}{2}$ près par un polynôme trigonométrique. Il suffit donc d'approcher notre fonction continue à $\frac{\epsilon}{2}$ près par une fonction continue, de classe $\mathcal{C}^1$ par morceaux, périodique de période $2\pi$, et pour cela d'approcher la restriction de $f$ à $[0,2\pi]$ par une fonction continue affine par morceaux prenant la même valeur en $0$ et $2\pi$, ce qui se fait comme ci-dessus.
\end{proof}
