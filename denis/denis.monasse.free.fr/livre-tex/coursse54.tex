\documentclass[]{article}
\usepackage[T1]{fontenc}
\usepackage{lmodern}
\usepackage{amssymb,amsmath}
\usepackage{ifxetex,ifluatex}
\usepackage{fixltx2e} % provides \textsubscript
% use upquote if available, for straight quotes in verbatim environments
\IfFileExists{upquote.sty}{\usepackage{upquote}}{}
\ifnum 0\ifxetex 1\fi\ifluatex 1\fi=0 % if pdftex
  \usepackage[utf8]{inputenc}
\else % if luatex or xelatex
  \ifxetex
    \usepackage{mathspec}
    \usepackage{xltxtra,xunicode}
  \else
    \usepackage{fontspec}
  \fi
  \defaultfontfeatures{Mapping=tex-text,Scale=MatchLowercase}
  \newcommand{\euro}{€}
\fi
% use microtype if available
\IfFileExists{microtype.sty}{\usepackage{microtype}}{}
\ifxetex
  \usepackage[setpagesize=false, % page size defined by xetex
              unicode=false, % unicode breaks when used with xetex
              xetex]{hyperref}
\else
  \usepackage[unicode=true]{hyperref}
\fi
\hypersetup{breaklinks=true,
            bookmarks=true,
            pdfauthor={},
            pdftitle={Integration sur un intervalle quelconque : fonctions `a valeurs reelles positives},
            colorlinks=true,
            citecolor=blue,
            urlcolor=blue,
            linkcolor=magenta,
            pdfborder={0 0 0}}
\urlstyle{same}  % don't use monospace font for urls
\setlength{\parindent}{0pt}
\setlength{\parskip}{6pt plus 2pt minus 1pt}
\setlength{\emergencystretch}{3em}  % prevent overfull lines
\setcounter{secnumdepth}{0}
 
/* start css.sty */
.cmr-5{font-size:50%;}
.cmr-7{font-size:70%;}
.cmmi-5{font-size:50%;font-style: italic;}
.cmmi-7{font-size:70%;font-style: italic;}
.cmmi-10{font-style: italic;}
.cmsy-5{font-size:50%;}
.cmsy-7{font-size:70%;}
.cmex-7{font-size:70%;}
.cmex-7x-x-71{font-size:49%;}
.msbm-7{font-size:70%;}
.cmtt-10{font-family: monospace;}
.cmti-10{ font-style: italic;}
.cmbx-10{ font-weight: bold;}
.cmr-17x-x-120{font-size:204%;}
.cmsl-10{font-style: oblique;}
.cmti-7x-x-71{font-size:49%; font-style: italic;}
.cmbxti-10{ font-weight: bold; font-style: italic;}
p.noindent { text-indent: 0em }
td p.noindent { text-indent: 0em; margin-top:0em; }
p.nopar { text-indent: 0em; }
p.indent{ text-indent: 1.5em }
@media print {div.crosslinks {visibility:hidden;}}
a img { border-top: 0; border-left: 0; border-right: 0; }
center { margin-top:1em; margin-bottom:1em; }
td center { margin-top:0em; margin-bottom:0em; }
.Canvas { position:relative; }
li p.indent { text-indent: 0em }
.enumerate1 {list-style-type:decimal;}
.enumerate2 {list-style-type:lower-alpha;}
.enumerate3 {list-style-type:lower-roman;}
.enumerate4 {list-style-type:upper-alpha;}
div.newtheorem { margin-bottom: 2em; margin-top: 2em;}
.obeylines-h,.obeylines-v {white-space: nowrap; }
div.obeylines-v p { margin-top:0; margin-bottom:0; }
.overline{ text-decoration:overline; }
.overline img{ border-top: 1px solid black; }
td.displaylines {text-align:center; white-space:nowrap;}
.centerline {text-align:center;}
.rightline {text-align:right;}
div.verbatim {font-family: monospace; white-space: nowrap; text-align:left; clear:both; }
.fbox {padding-left:3.0pt; padding-right:3.0pt; text-indent:0pt; border:solid black 0.4pt; }
div.fbox {display:table}
div.center div.fbox {text-align:center; clear:both; padding-left:3.0pt; padding-right:3.0pt; text-indent:0pt; border:solid black 0.4pt; }
div.minipage{width:100%;}
div.center, div.center div.center {text-align: center; margin-left:1em; margin-right:1em;}
div.center div {text-align: left;}
div.flushright, div.flushright div.flushright {text-align: right;}
div.flushright div {text-align: left;}
div.flushleft {text-align: left;}
.underline{ text-decoration:underline; }
.underline img{ border-bottom: 1px solid black; margin-bottom:1pt; }
.framebox-c, .framebox-l, .framebox-r { padding-left:3.0pt; padding-right:3.0pt; text-indent:0pt; border:solid black 0.4pt; }
.framebox-c {text-align:center;}
.framebox-l {text-align:left;}
.framebox-r {text-align:right;}
span.thank-mark{ vertical-align: super }
span.footnote-mark sup.textsuperscript, span.footnote-mark a sup.textsuperscript{ font-size:80%; }
div.tabular, div.center div.tabular {text-align: center; margin-top:0.5em; margin-bottom:0.5em; }
table.tabular td p{margin-top:0em;}
table.tabular {margin-left: auto; margin-right: auto;}
div.td00{ margin-left:0pt; margin-right:0pt; }
div.td01{ margin-left:0pt; margin-right:5pt; }
div.td10{ margin-left:5pt; margin-right:0pt; }
div.td11{ margin-left:5pt; margin-right:5pt; }
table[rules] {border-left:solid black 0.4pt; border-right:solid black 0.4pt; }
td.td00{ padding-left:0pt; padding-right:0pt; }
td.td01{ padding-left:0pt; padding-right:5pt; }
td.td10{ padding-left:5pt; padding-right:0pt; }
td.td11{ padding-left:5pt; padding-right:5pt; }
table[rules] {border-left:solid black 0.4pt; border-right:solid black 0.4pt; }
.hline hr, .cline hr{ height : 1px; margin:0px; }
.tabbing-right {text-align:right;}
span.TEX {letter-spacing: -0.125em; }
span.TEX span.E{ position:relative;top:0.5ex;left:-0.0417em;}
a span.TEX span.E {text-decoration: none; }
span.LATEX span.A{ position:relative; top:-0.5ex; left:-0.4em; font-size:85%;}
span.LATEX span.TEX{ position:relative; left: -0.4em; }
div.float img, div.float .caption {text-align:center;}
div.figure img, div.figure .caption {text-align:center;}
.marginpar {width:20%; float:right; text-align:left; margin-left:auto; margin-top:0.5em; font-size:85%; text-decoration:underline;}
.marginpar p{margin-top:0.4em; margin-bottom:0.4em;}
.equation td{text-align:center; vertical-align:middle; }
td.eq-no{ width:5%; }
table.equation { width:100%; } 
div.math-display, div.par-math-display{text-align:center;}
math .texttt { font-family: monospace; }
math .textit { font-style: italic; }
math .textsl { font-style: oblique; }
math .textsf { font-family: sans-serif; }
math .textbf { font-weight: bold; }
.partToc a, .partToc, .likepartToc a, .likepartToc {line-height: 200%; font-weight:bold; font-size:110%;}
.chapterToc a, .chapterToc, .likechapterToc a, .likechapterToc, .appendixToc a, .appendixToc {line-height: 200%; font-weight:bold;}
.index-item, .index-subitem, .index-subsubitem {display:block}
.caption td.id{font-weight: bold; white-space: nowrap; }
table.caption {text-align:center;}
h1.partHead{text-align: center}
p.bibitem { text-indent: -2em; margin-left: 2em; margin-top:0.6em; margin-bottom:0.6em; }
p.bibitem-p { text-indent: 0em; margin-left: 2em; margin-top:0.6em; margin-bottom:0.6em; }
.paragraphHead, .likeparagraphHead { margin-top:2em; font-weight: bold;}
.subparagraphHead, .likesubparagraphHead { font-weight: bold;}
.quote {margin-bottom:0.25em; margin-top:0.25em; margin-left:1em; margin-right:1em; text-align:justify;}
.verse{white-space:nowrap; margin-left:2em}
div.maketitle {text-align:center;}
h2.titleHead{text-align:center;}
div.maketitle{ margin-bottom: 2em; }
div.author, div.date {text-align:center;}
div.thanks{text-align:left; margin-left:10%; font-size:85%; font-style:italic; }
div.author{white-space: nowrap;}
.quotation {margin-bottom:0.25em; margin-top:0.25em; margin-left:1em; }
h1.partHead{text-align: center}
.sectionToc, .likesectionToc {margin-left:2em;}
.subsectionToc, .likesubsectionToc {margin-left:4em;}
.subsubsectionToc, .likesubsubsectionToc {margin-left:6em;}
.frenchb-nbsp{font-size:75%;}
.frenchb-thinspace{font-size:75%;}
.figure img.graphics {margin-left:10%;}
/* end css.sty */

\title{Integration sur un intervalle quelconque : fonctions `a valeurs reelles
positives}
\author{}
\date{}

\begin{document}
\maketitle

\textbf{Warning: 
requires JavaScript to process the mathematics on this page.\\ If your
browser supports JavaScript, be sure it is enabled.}

\begin{center}\rule{3in}{0.4pt}\end{center}

[
[
[]
[

\subsubsection{9.5 Intégration sur un intervalle quelconque~: fonctions
à valeurs réelles positives}

\paragraph{9.5.1 Fonctions intégrables à valeurs réelles positives}

Définition~9.5.1 Soit I un intervalle de \mathbb{R}~, f : I \rightarrow~ \mathbb{R}~ positive et
continue par morceaux. On dit que f est intégrable sur I s'il existe une
constante M ≥ 0 telle que, pour tout segment [a,b] \subset~ I, on ait
\int  _a^b~f \leq M . On note alors
\int  _I~f =\
sup_[a,b]\subset~I\int ~
_a^bf.

Proposition~9.5.1 Soit I un intervalle de \mathbb{R}~, f : I \rightarrow~ \mathbb{R}~ positive et
continue par morceaux, intégrable sur I. Alors f est intégrable sur tout
intervalle I' inclus dans I et \int ~
_I'f \leq\int  _I~f.

Démonstration En effet tout segment inclus dans I' est également un
segment inclus dans I, donc le même M convient comme majorant.

Proposition~9.5.2 Soit f,g : I \rightarrow~ \mathbb{R}~ positives et continues par morceaux
telles que 0 \leq f \leq g. Si g est intégrable sur I il en est de même de f
et \int  _I~f
\leq\int  _I~g.

Démonstration Evident d'après la définition.

Proposition~9.5.3 Soit I un intervalle de \mathbb{R}~, f : I \rightarrow~ \mathbb{R}~ positive et
continue par morceaux. Alors f est intégrable sur I si et seulement si
il existe une suite ([a_n,b_n])_n\in\mathbb{N}~
croissante de segments contenus dans I, dont la réunion est égale à I,
et une constante positive M telle que \forall~~n \in \mathbb{N}~,
\int  _a_n^b_n~f
\leq M. Dans ce cas, on a

\int  _I~f =\
sup_n\int ~
_a_n^b_n f =\
lim_n\rightarrow~+\infty~\int ~
_a_n^b_n f

Démonstration La condition est bien évidemment nécessaire~: prendre
n'importe quelle suite ([a_n,b_n]) vérifiant les
conditions voulues. Inversement supposons qu'il existe une telle suite
([a_n,b_n]) et une constante M ≥ 0. Soit J =
[a,b] un segment inclus dans I et posons J_n =
[a_n,b_n]. Si b = sup~I,
alors supI \in\cupJ_n~ et donc il existe N
\in \mathbb{N}~ tel que supI \in J_N~ auquel cas
supI \in J_n~ pour tout n ≥ N. Si par
contre, b < sup~I
= limb_n~, alors il existe N \in \mathbb{N}~ tel
que n ≥ N \rigtharrow~ b_n > b. Dans les deux cas il existe N
\in \mathbb{N}~ tel que n ≥ N \rigtharrow~ b_n ≥ b. De même, il existe N' \in \mathbb{N}~ tel que
n ≥ N' \rigtharrow~ a_n \leq a. Soit n = max~(N,N'),
on a alors J = [a,b] \subset~ [a_n,b_n] =
J_n et donc

\int  _a^b~f
\leq\int  _a_n^b_n~
f \leq M

ce qui montre que f est intégrable sur I.

La démonstration précédente montre clairement dans sa première partie
que sup_n~\\int
 _a_n^b_nf \leq\
sup_[a,b]\subset~I\int ~
_a^bf =\int  _I~f et dans
sa deuxième partie que
sup_[a,b]\subset~I~\\int
 _a^bf \leq\
sup_n\int ~
_a_n^b_nf, et donc l'égalité
\int  _I~f =\
sup_n\int ~
_a_n^b_nf. Mais comme la suite
\left (\int ~
_a_n^b_nf\right
)_n\in\mathbb{N}~ est croissante majorée, sa borne supérieure est aussi sa
limite.

Proposition~9.5.4 Soit I = [a,b] un segment de \mathbb{R}~, f : I \rightarrow~ \mathbb{R}~ positive
et continue par morceaux. Alors f est intégrable sur I et
\int  _I~f =\\int
 _a^bf. De plus f est intégrable sur ]a,b[,
[a,b[ et ]a,b], toutes ces intégrales étant égales.

Démonstration Si J = [c,d] est un segment inclus dans [a,b], on
a \int  _c^d~f
\leq\int  _a^b~f, donc f est
intégrable et \int  _I~f
\leq\int  _a^b~f. Mais d'autre part,
[a,b] est lui même un segment inclus dans I, donc
\int  _a^b~f
\leq\int  _I~f, et donc l'égalité. On sait
alors que f est intégrable sur tout intervalle inclus dans I et en
particulier sur ]a,b[, [a,b[ et ]a,b]. De plus, si
a_n = a + 1 \over n et b_n = b -
1 \over n , J_n =
[a_n,b_n] est une suite croissante de segments
dont la réunion est ]a,b[, donc

\int  _]a,b[~f
= lim\\int ~
_a_n^b_n f =\\int
 _a^bf =\int ~
_[a,b]f

par continuité de l'intégrale par rapport à ses bornes. Comme on a
]a,b[\subset~ [a,b[\subset~ [a,b], on a aussi \\int
 _]a,b[f \leq\int  _[a,b[~f
\leq\int  _[a,b]~f, d'où l'égalité des
trois nombres. Il en est de même de \int ~
_]a,b]f.

Proposition~9.5.5 Soit f : I \rightarrow~ \mathbb{R}~ continue positive intégrable, telle que
\int  _I~f = 0. Alors f = 0.

Démonstration Pour tout segment J \subset~ I, on a 0
\leq\int  _J~f \leq\\int
 _If = 0, donc \int  _J~f = 0 ce
qui implique que f est nulle sur J. La fonction f est donc nulle sur
tout segment inclus dans I, donc elle est nulle.

Proposition~9.5.6 Soit f,g : I \rightarrow~ \mathbb{R}~ positives et continues par morceaux,
soit \alpha~ \in \mathbb{R}~^+. Si f et g sont intégrables sur I, il en est de
même de f + g et de \alpha~f et on a

\int  _I~(f + g)
=\int  _I~f +\\int
 _Ig\text et \int ~
_I(\alpha~f) = \alpha~\int  _I~f

Démonstration L'intégrabilité est évidente à partir de la définition.
Pour les égalités, il suffit de prendre une suite (J_n)
croissante de segments de réunion I et de passer à la limite dans les
formules

\int  _J_n~(f + g)
=\int  _J_n~f
+\int ~
_J_ng\text et
\int  _J_n~(\alpha~f) =
\alpha~\int  _J_n~f

Proposition~9.5.7 Soit I un intervalle de \mathbb{R}~, f : I \rightarrow~ \mathbb{R}~ positive et
continue par morceaux. Soit a \in I^o. Alors f est intégrable
sur I si et seulement si elle est intégrable sur I\bigcap] -\infty~,a] et sur I
\bigcap [a,+\infty~[. Dans ce cas,

\int  _I~f =\\int
 _I\bigcap]-\infty~,a]f +\int ~
_I\bigcap[a,+\infty~[

Démonstration Si f est intégrable sur I, elle est intégrable sur tout
sous intervalle de I et donc sur I\bigcap] -\infty~,a] et sur I \bigcap [a,+\infty~[.
Inversement, si f est intégrable sur ces deux sous intervalles, soit
M_1 et M_2 les majorants des intégrales sur les sous
segments de I\bigcap] -\infty~,a] et I \bigcap [a,+\infty~[. Si J est un segment inclus
dans I on a

\int  _J~f \leq\left
\ \cases M_1 &si
supJ \leq a \cr M_1~ +
M_2&si a \in J \cr M_2 &si a
\leq inf J ~ \right .

et dans tous les cas \int  _J~f \leq
M_1 + M_2. Donc f est intégrable sur I. Soit alors
J_n = [a_n,b_n] une suite croissante de
segments de réunion I. Pour n assez grand, on a a_n \leq a \leq
b_n car a est dans l'intérieur de I. Mais
([a_n,a]) est une suite croissante de segments de réunion
I\bigcap] -\infty~,a] et ([a,b_n]) est une suite croissante de
segments de réunion I \bigcap [a,+\infty~[. On peut donc passer à la limite dans
la formule \int ~
_[a_n,b_n]f =\int ~
_[a_n,a]f +\int ~
_[a,b_n]f, et on obtient

\int  _I~f =\\int
 _I\bigcap]-\infty~,a]f +\int ~
_I\bigcap[a,+\infty~[

Proposition~9.5.8 Soit -\infty~ < a < b \leq +\infty~, et f :
[a,b[\rightarrow~ \mathbb{R}~ positive et continue par morceaux. Pour x \in [a,b[,
posons F(x) =\int  _a^x~f(t) dt.
Alors f est intégrable sur [a,b[ si et seulement si F admet une
limite au point b. Dans ce cas, \int ~
_[a,b[f = lim_x\rightarrow~b~F(x) -
F(a)

Démonstration Soit b_n une suite croissante de [a,b[ de
limite b. Alors [a,b_n] est une suite croissante de
segments dont la réunion est [a,b[. Donc f est intégrable si et
seulement si la suite \int ~
_a^b_nf = F(b_n) - F(a) admet une
limite, donc si et seulement si la suite (F(b_n)) est
convergente. Mais comme F est croissante, ceci équivaut à l'existence de
la limite de F en b.

Remarque~9.5.1 Si f n'est pas intégrable sur [a,b[, alors F, qui est
croissante, admet + \infty~ comme limite au point b.

Remarque~9.5.2 De même, si -\infty~\leq a < b < +\infty~, et f
:]a,b] \rightarrow~ \mathbb{R}~ positive et continue par morceaux. Pour x \in]a,b],
posons F(x) =\int  _x^b~f(t) dt.
Alors f est intégrable sur ]a,b] si et seulement si F (qui est cette
fois décroissante) admet une limite au point a. Dans ce cas,
\int  _]a,b]~f = F(b)
- lim_x\rightarrow~a~F(x)

\paragraph{9.5.2 Règles de comparaison}

Théorème~9.5.9 Soit f,g : [a,b[\rightarrow~ \mathbb{R}~ continues par morceaux positives.
On suppose qu'au voisinage de b on a f = O(g) (resp. f = o(g)). Alors
(i) si g est intégrable sur [a,b[, il en est de même de f et
\int  _[x,b[~f(t) dt =
O(\int  _[x,b[~g(t) dt) (resp.
\int  _[x,b[~f(t) dt =
o(\int  _[x,b[~g(t) dt)) (ii) si f
n'est pas intégrable sur [a,b[, g ne l'est pas non plus et
\int  _a^x~f(t) dt =
O(\int  _a^x~g(t) dt) (resp.
\int  _a^x~f(t) dt =
o(\int  _a^x~g(t) dt))

Démonstration Les convergences et divergences découlent immédiatement de
l'inégalité 0 \leq f \leq Kg qui est vraie sur [c,b[ et du fait que f et g
sont intégrables sur [a,c] (car continues par morceaux sur ce
segment). De plus f = o(g) \rigtharrow~ f = O(g). En ce qui concerne la comparaison
des restes ou des intégrales partielles, la démonstration est tout à
fait similaire à celle du théorème analogue sur les séries. Nous allons
la faire dans le cas f = o(g), la démonstration étant analogue pour f =
O(g) en changeant \epsilon en K ou en 2K.

(i) Supposons f = o(g) et g intégrable. Soit \epsilon > 0. Il
existe c \in [a,b[ tel que t ≥ c \rigtharrow~ 0 \leq f(t) \leq \epsilong(t). Alors pour x ≥ c,
on a (en intégrant l'inégalité de x à b), 0 \leq\\int
 _[x,b[f(t) dt \leq \epsilon\int ~
_[x,b[g(t) dt et donc \int ~
_[x,b[f(t) dt = o(\int ~
_[x,b[g(t) dt).

(ii) Supposons f = o(g) et f non intégrable sur [a,b[. Soit \epsilon
> 0. Il existe c \in [a,b[ tel que t ≥ c \rigtharrow~ 0 \leq f(t) \leq \epsilon
\over 2 g(t). Alors pour x ≥ c, on a (en intégrant
l'inégalité de c à x), \int ~
_c^xf(t) dt \leq \epsilon \over 2
\int  _c^x~g(t) dt, soit encore à
l'aide de la relation de Chasles

0 \leq\int  _a^x~f(t) dt \leq \epsilon
\over 2 \int ~
_a^xg(t) dt + \left
(\int  _a^c~f(t) dt - \epsilon
\over 2 \int ~
_a^cg(t) dt\right )

Mais comme on sait que g n'est pas intégrable sur [a,b[ et que g ≥
0, on a
lim_x\rightarrow~b\\int ~
_a^xg(t) dt = +\infty~. Donc il existe c' \in [a,b[ tel que x
≥ c' \rigtharrow~ \epsilon \over 2 \int ~
_a^xg(t) dt >\int ~
_a^cf(t) dt - \epsilon \over 2
\int  _a^c~g(t) dt. Alors, pour x
≥ max~(c,c'), on a

0 \leq\int  _a^x~f(t) dt \leq \epsilon
\over 2 \int ~
_a^xg(t) dt + \epsilon \over 2
\int  _a^x~g(t) dt =
\epsilon\int  _a^x~g(t) dt

et donc \int  _a^x~f(t) dt =
o(\int  _a^x~g(t) dt).

Remarque~9.5.3 Il suffit pour appliquer le théorème précédent que la
condition de positivité de f et g soit vérifiée dans un voisinage de b.

Théorème~9.5.10 Soit f,g : [a,b[\rightarrow~ \mathbb{R}~ continues par morceaux. On
suppose que g est positive et que au voisinage de b, on a f ∼ g. Alors f
et g sont simultanément intégrables ou non intégrables sur [a,b[.
Plus précisément (i) Si g est intégrable sur [a,b[, alors f
également et \int  _[x,b[~f(t) dt
∼\int  _[x,b[~g(t) dt (ii) Si g est
non intégrable sur [a,b[, alors f également et
\int  _a^x~f(t) dt
∼\int  _a^x~g(t) dt.

Démonstration Puisque f(t) ∼ g(t), il existe c \in [a,b[ tel que x
> c \rigtharrow~ 1 \over 2 g(t) \leq f(t) \leq 3
\over 2 g(t) ce qui montre que f est positive au
voisinage de b et que l'on a à la fois f = O(g) et g = O(f). Le théorème
précédent assure alors que f est intégrable sur [a,b[ si et
seulement si~g l'est. Pla\ccons nous dans le cas
d'intégrabilité. On a f - g = o(g), on en déduit que
f - g est intégrable et que
\int  _[x,b[~f(t) -
g(t) dt = o(\int ~
_[x,b[g(t) dt). Mais bien évidemment \left
\int  _[x,b[~f(t) dt
-\int  _[x,b[~g(t)
dt\right \leq\int ~
_[x,b[f(t) - g(t) dt. On a donc
\int  _[x,b[~f(t) dt
-\int  _[x,b[~g(t) dt =
o(\int  _[x,b[~g(t) dt) et donc
\int  _[x,b[~f(t) dt
∼\int  _[x,b[~g(t) dt. Dans le cas de
non intégrabilité, deux cas se présentent. Si f - g
est non intégrable, le théorème précédent assure que
\int  _a^x~f(t) -
g(t) dt = o(\int ~
_a^xg(t) dt)~; si par contre elle est intégrable,
\int  _a^x~f(t) -
g(t) dt admet une limite finie en b alors que
\int  _a^x~g(t) dt tend vers + \infty~
et on a donc encore \int ~
_a^xf(t) - g(t) dt =
o(\int  _a^x~g(t) dt). L'inégalité
\left \int ~
_a^xf(t) dt -\int ~
_a^xg(t) dt\right
\leq\int ~
_a^xf(t) - g(t) dt donne alors
\int  _a^x~f(t) dt
-\int  _a^x~g(t) dt =
o(\int  _a^x~g(t) dt) et donc
\int  _a^x~f(t) dt
∼\int  _a^x~g(t) dt.

\paragraph{9.5.3 Exemples fondamentaux}

L'idée générale est d'obtenir une famille de fonctions étalons.

Proposition~9.5.11 La fonction
t\mapsto~t^\alpha~ est intégrable sur
[a,+\infty~[ (avec a > 0) si et seulement si~\alpha~ >
1.

Démonstration On a

\int  _1^x~ dt
\over t^\alpha~ = \left
\ \cases  1 \over
\alpha~-1 (1 - x^1-\alpha~)&si \alpha~\neq~1
\cr \cr log~ x
&si \alpha~ = 1 \cr  \right .

qui admet une limite finie en + \infty~ si et seulement si~\alpha~ > 1.

Exemple~9.5.1 Intégrales de Bertrand \int ~
_e^+\infty~ dt \over
t^\alpha~(log t)^\beta~~ . Si \alpha~
> 1, soit \gamma tel que 1 < \alpha~ < \gamma. On a
alors  1 \over
t^\alpha~(log t)^\beta~~ = o( 1
\over t^\gamma ) et donc
t\mapsto~ 1 \over
t^\alpha~(log t)^\beta~~ est
intégrable sur [e,+\infty~[. Si \alpha~ < 1, soit \gamma tel que \alpha~
< \gamma < 1~; on a alors  1 \over
t^\gamma = o( 1 \over
t^\alpha~(log t)^\beta~~ ) et
comme t\mapsto~ 1 \over
t^\gamma n'est pas intégrable sur [e,+\infty~[,
t\mapsto~ 1 \over
t^\alpha~(log t)^\beta~~ n'est pas
intégrable sur [e,+\infty~[. Si \alpha~ = 1, on a par le changement de variables
u = log~ t,

\begin{align*} \int ~
_e^x dt \over
t(log t)^\beta~~ & =&
\int ~
_1^log x~ du
\over u^\beta~ \%&
\\ & =& \left
\ \cases  1 \over
\beta~-1 (1 - (log x)^1-\beta~~)&si
\alpha~\neq~1 \cr \cr
log \log~ x &si \alpha~ = 1
 \right .\%&\\
\end{align*}

qui admet une limite en + \infty~ si et seulement si~\beta~ > 1. En
définitive t\mapsto~ 1 \over
t^\alpha~(log t)^\beta~~ est
intégrable sur [e,+\infty~[ si et seulement si~\alpha~ > 1 ou (\alpha~ =
1 et \beta~ > 1).

Proposition~9.5.12 La fonction
t\mapsto~t^\alpha~ est intégrable sur ]0,a]
(avec a > 0) si et seulement si~\alpha~ < 1.

Démonstration On a

\int  _x^a~ dt
\over t^\alpha~ = \left
\ \cases  1 \over
1-\alpha~ (a^1-\alpha~ - x^1-\alpha~)&si
\alpha~\neq~1 \cr \cr
log a -\ log~ x&si \alpha~
= 1  \right .

qui admet une limite au point 0 si et seulement si~\alpha~ < 1.

Exemple~9.5.2 Intégrales de Bertrand \int ~
_0^1\diagupet^\alpha~log~
t^\beta~ dt. Si \alpha~ > -1, soit \gamma tel que \alpha~
> \gamma > -1. On a alors en 0,
t^\alpha~log~
t^\beta~ = o(t^\gamma) (car 
t^\alpha~ log~
t^\beta~ \over t^\gamma =
t^\alpha~-\gammalog~
t^\beta~ tend vers 0 quand t tend vers 0) et comme
t\mapsto~t^\gamma est intégrable sur
]0,1\diagupe], il en est de même de
t\mapsto~t^\alpha~log~
t^\beta~. Si \alpha~ < -1, soit \gamma tel que \alpha~
< \gamma < -1. Alors t^\gamma =
o(t^\alpha~log~
t^\beta~) et comme
t\mapsto~t^\gamma n'est pas intégrable sur
]0,1\diagupe], il en est de même de
t\mapsto~t^\alpha~log~
t^\beta~. Si \alpha~ = -1, le changement de variables u =
-log~ t conduit à

\int  _x^1\diagupe~
log t^\beta~~
\over t dt =\int ~
_1^- log xu^\beta~~ du

qui admet une limite quand x tend vers 0 si et seulement si~\beta~
< -1. En définitive,
t\mapsto~t^\alpha~log~
t^\beta~ est intégrable sur [0,1\diagupe[ si et seulement
si~\alpha~ > -1 ou (\alpha~ = -1 et \beta~ < -1).

[
[
[
[

\end{document}
