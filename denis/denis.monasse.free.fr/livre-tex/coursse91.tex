\documentclass[]{article}
\usepackage[T1]{fontenc}
\usepackage{lmodern}
\usepackage{amssymb,amsmath}
\usepackage{ifxetex,ifluatex}
\usepackage{fixltx2e} % provides \textsubscript
% use upquote if available, for straight quotes in verbatim environments
\IfFileExists{upquote.sty}{\usepackage{upquote}}{}
\ifnum 0\ifxetex 1\fi\ifluatex 1\fi=0 % if pdftex
  \usepackage[utf8]{inputenc}
\else % if luatex or xelatex
  \ifxetex
    \usepackage{mathspec}
    \usepackage{xltxtra,xunicode}
  \else
    \usepackage{fontspec}
  \fi
  \defaultfontfeatures{Mapping=tex-text,Scale=MatchLowercase}
  \newcommand{\euro}{€}
\fi
% use microtype if available
\IfFileExists{microtype.sty}{\usepackage{microtype}}{}
\ifxetex
  \usepackage[setpagesize=false, % page size defined by xetex
              unicode=false, % unicode breaks when used with xetex
              xetex]{hyperref}
\else
  \usepackage[unicode=true]{hyperref}
\fi
\hypersetup{breaklinks=true,
            bookmarks=true,
            pdfauthor={},
            pdftitle={Analyse numerique des equations differentielles},
            colorlinks=true,
            citecolor=blue,
            urlcolor=blue,
            linkcolor=magenta,
            pdfborder={0 0 0}}
\urlstyle{same}  % don't use monospace font for urls
\setlength{\parindent}{0pt}
\setlength{\parskip}{6pt plus 2pt minus 1pt}
\setlength{\emergencystretch}{3em}  % prevent overfull lines
\setcounter{secnumdepth}{0}
 
/* start css.sty */
.cmr-5{font-size:50%;}
.cmr-7{font-size:70%;}
.cmmi-5{font-size:50%;font-style: italic;}
.cmmi-7{font-size:70%;font-style: italic;}
.cmmi-10{font-style: italic;}
.cmsy-5{font-size:50%;}
.cmsy-7{font-size:70%;}
.cmex-7{font-size:70%;}
.cmex-7x-x-71{font-size:49%;}
.msbm-7{font-size:70%;}
.cmtt-10{font-family: monospace;}
.cmti-10{ font-style: italic;}
.cmbx-10{ font-weight: bold;}
.cmr-17x-x-120{font-size:204%;}
.cmsl-10{font-style: oblique;}
.cmti-7x-x-71{font-size:49%; font-style: italic;}
.cmbxti-10{ font-weight: bold; font-style: italic;}
p.noindent { text-indent: 0em }
td p.noindent { text-indent: 0em; margin-top:0em; }
p.nopar { text-indent: 0em; }
p.indent{ text-indent: 1.5em }
@media print {div.crosslinks {visibility:hidden;}}
a img { border-top: 0; border-left: 0; border-right: 0; }
center { margin-top:1em; margin-bottom:1em; }
td center { margin-top:0em; margin-bottom:0em; }
.Canvas { position:relative; }
li p.indent { text-indent: 0em }
.enumerate1 {list-style-type:decimal;}
.enumerate2 {list-style-type:lower-alpha;}
.enumerate3 {list-style-type:lower-roman;}
.enumerate4 {list-style-type:upper-alpha;}
div.newtheorem { margin-bottom: 2em; margin-top: 2em;}
.obeylines-h,.obeylines-v {white-space: nowrap; }
div.obeylines-v p { margin-top:0; margin-bottom:0; }
.overline{ text-decoration:overline; }
.overline img{ border-top: 1px solid black; }
td.displaylines {text-align:center; white-space:nowrap;}
.centerline {text-align:center;}
.rightline {text-align:right;}
div.verbatim {font-family: monospace; white-space: nowrap; text-align:left; clear:both; }
.fbox {padding-left:3.0pt; padding-right:3.0pt; text-indent:0pt; border:solid black 0.4pt; }
div.fbox {display:table}
div.center div.fbox {text-align:center; clear:both; padding-left:3.0pt; padding-right:3.0pt; text-indent:0pt; border:solid black 0.4pt; }
div.minipage{width:100%;}
div.center, div.center div.center {text-align: center; margin-left:1em; margin-right:1em;}
div.center div {text-align: left;}
div.flushright, div.flushright div.flushright {text-align: right;}
div.flushright div {text-align: left;}
div.flushleft {text-align: left;}
.underline{ text-decoration:underline; }
.underline img{ border-bottom: 1px solid black; margin-bottom:1pt; }
.framebox-c, .framebox-l, .framebox-r { padding-left:3.0pt; padding-right:3.0pt; text-indent:0pt; border:solid black 0.4pt; }
.framebox-c {text-align:center;}
.framebox-l {text-align:left;}
.framebox-r {text-align:right;}
span.thank-mark{ vertical-align: super }
span.footnote-mark sup.textsuperscript, span.footnote-mark a sup.textsuperscript{ font-size:80%; }
div.tabular, div.center div.tabular {text-align: center; margin-top:0.5em; margin-bottom:0.5em; }
table.tabular td p{margin-top:0em;}
table.tabular {margin-left: auto; margin-right: auto;}
div.td00{ margin-left:0pt; margin-right:0pt; }
div.td01{ margin-left:0pt; margin-right:5pt; }
div.td10{ margin-left:5pt; margin-right:0pt; }
div.td11{ margin-left:5pt; margin-right:5pt; }
table[rules] {border-left:solid black 0.4pt; border-right:solid black 0.4pt; }
td.td00{ padding-left:0pt; padding-right:0pt; }
td.td01{ padding-left:0pt; padding-right:5pt; }
td.td10{ padding-left:5pt; padding-right:0pt; }
td.td11{ padding-left:5pt; padding-right:5pt; }
table[rules] {border-left:solid black 0.4pt; border-right:solid black 0.4pt; }
.hline hr, .cline hr{ height : 1px; margin:0px; }
.tabbing-right {text-align:right;}
span.TEX {letter-spacing: -0.125em; }
span.TEX span.E{ position:relative;top:0.5ex;left:-0.0417em;}
a span.TEX span.E {text-decoration: none; }
span.LATEX span.A{ position:relative; top:-0.5ex; left:-0.4em; font-size:85%;}
span.LATEX span.TEX{ position:relative; left: -0.4em; }
div.float img, div.float .caption {text-align:center;}
div.figure img, div.figure .caption {text-align:center;}
.marginpar {width:20%; float:right; text-align:left; margin-left:auto; margin-top:0.5em; font-size:85%; text-decoration:underline;}
.marginpar p{margin-top:0.4em; margin-bottom:0.4em;}
.equation td{text-align:center; vertical-align:middle; }
td.eq-no{ width:5%; }
table.equation { width:100%; } 
div.math-display, div.par-math-display{text-align:center;}
math .texttt { font-family: monospace; }
math .textit { font-style: italic; }
math .textsl { font-style: oblique; }
math .textsf { font-family: sans-serif; }
math .textbf { font-weight: bold; }
.partToc a, .partToc, .likepartToc a, .likepartToc {line-height: 200%; font-weight:bold; font-size:110%;}
.chapterToc a, .chapterToc, .likechapterToc a, .likechapterToc, .appendixToc a, .appendixToc {line-height: 200%; font-weight:bold;}
.index-item, .index-subitem, .index-subsubitem {display:block}
.caption td.id{font-weight: bold; white-space: nowrap; }
table.caption {text-align:center;}
h1.partHead{text-align: center}
p.bibitem { text-indent: -2em; margin-left: 2em; margin-top:0.6em; margin-bottom:0.6em; }
p.bibitem-p { text-indent: 0em; margin-left: 2em; margin-top:0.6em; margin-bottom:0.6em; }
.paragraphHead, .likeparagraphHead { margin-top:2em; font-weight: bold;}
.subparagraphHead, .likesubparagraphHead { font-weight: bold;}
.quote {margin-bottom:0.25em; margin-top:0.25em; margin-left:1em; margin-right:1em; text-align:justify;}
.verse{white-space:nowrap; margin-left:2em}
div.maketitle {text-align:center;}
h2.titleHead{text-align:center;}
div.maketitle{ margin-bottom: 2em; }
div.author, div.date {text-align:center;}
div.thanks{text-align:left; margin-left:10%; font-size:85%; font-style:italic; }
div.author{white-space: nowrap;}
.quotation {margin-bottom:0.25em; margin-top:0.25em; margin-left:1em; }
h1.partHead{text-align: center}
.sectionToc, .likesectionToc {margin-left:2em;}
.subsectionToc, .likesubsectionToc {margin-left:4em;}
.subsubsectionToc, .likesubsubsectionToc {margin-left:6em;}
.frenchb-nbsp{font-size:75%;}
.frenchb-thinspace{font-size:75%;}
.figure img.graphics {margin-left:10%;}
/* end css.sty */

\title{Analyse numerique des equations differentielles}
\author{}
\date{}

\begin{document}
\maketitle

\textbf{Warning: \href{http://www.math.union.edu/locate/jsMath}{jsMath}
requires JavaScript to process the mathematics on this page.\\ If your
browser supports JavaScript, be sure it is enabled.}

\begin{center}\rule{3in}{0.4pt}\end{center}

{[}\href{coursse90.html}{prev}{]}
{[}\href{coursse90.html\#tailcoursse90.html}{prev-tail}{]}
{[}\hyperref[tailcoursse91.html]{tail}{]}
{[}\href{coursch17.html\#coursse91.html}{up}{]}

\subsubsection{16.6 Analyse numérique des équations différentielles}

\paragraph{16.6.1 Méthode d'Euler}

Soit f : J × E → E de classe \{C\}\^{}\{1\} , soit \{t\}\_\{o\} dans J ,
\{y\}\_\{o\} dans E et φ : \{J\}\_\{0\} → E la solution maximale
vérifiant la condition φ(\{t\}\_\{o\}) = \{y\}\_\{o\}. \{J\}\_\{0\} est
un intervalle contenu dans J et contenant \{t\}\_\{o\}, et dire que la
solution est maximale, c'est dire que φ ne peut se prolonger en une
solution de l'équation différentielle sur un intervalle strictement plus
grand que \{J\}\_\{0\}. Notre but est de trouver une approximation de la
fonction φ .

Pour cela soit h un nombre réel suffisamment petit et t dans
\{J\}\_\{0\} tel que t + h appartienne encore à \{J\}\_\{0\}. Alors φ(t
+ h) est peu différent de φ(t) + hφ'(t). Mais φ'(t) = f(t,φ(t)). Donc
φ(t + h) est peu différent de φ(t) + hf(t,φ(t)) = y + f(t,y) si y =
φ(t). Ceci nous amène à définir pour un nombre réel h donné

\begin{itemize}
\itemsep1pt\parskip0pt\parsep0pt
\item
  (i) une suite \{(\{t\}\_\{i\})\}\_\{i∈ℤ\} par \{t\}\_\{i\} =
  \{t\}\_\{o\} + ih
\item
  (ii) une suite \{(\{y\}\_\{i\})\}\_\{i∈ℤ\} par \{y\}\_\{i+1\} =
  \{y\}\_\{i\} + hf(\{t\}\_\{i\},\{y\}\_\{i\}) pour i ≥ 0,
  \{y\}\_\{i−1\} = \{y\}\_\{i\} − hf(\{t\}\_\{i\},\{y\}\_\{i\}) pour i ≤
  0
\item
  (iii) une fonction \{φ\}\_\{h\} prenant aux points \{t\}\_\{i\} la
  valeur \{y\}\_\{i\} et affine sur chacun des intervalles
  {[}\{t\}\_\{i\},\{t\}\_\{i+1\}{]}
\end{itemize}

Nous espérons bien entendu que la fonction \{φ\}\_\{h\} ainsi définie
sera une approximation de φ pour h petit. Nous allons montrer que c'est
effectivement le cas, tout au moins sur un segment {[}a,b{]} contenu
dans \{J\}\_\{0\} et dans le cas simple où E = ℝ (bien que le résultat
reste valable si E est un espace vectoriel normé).

Remarquons tout d'abord que puisque f est de classe \{C\}\^{}\{1\} et
que φ'(t) = f(t,φ(t)) , φ' est dérivable et

\textbackslash{}begin\{eqnarray*\} φ'`(t)\& =\&\{ ∂f
\textbackslash{}over ∂t\} (t,φ(t)) + φ'(t)\{ ∂f \textbackslash{}over
∂y\} (t,φ(t)) \%\& \textbackslash{}\textbackslash{} \& =\&\{ ∂f
\textbackslash{}over ∂t\} (t,φ(t)) + f(t,φ(t))\{ ∂f \textbackslash{}over
∂y\} (t,φ(t))\%\& \textbackslash{}\textbackslash{}
\textbackslash{}end\{eqnarray*\}

Soit α \textgreater{} 0 et soit K = \textbackslash{}\{(t,y) ∈ {[}a,b{]}
× E\textbackslash{}mathrel\{∣\}φ(t) − α ≤ y ≤ φ(t) +
α\textbackslash{}\}.

K est un compact qui contient le graphe de φ. La fonction
(t,y)\textbackslash{}mathrel\{↦\}\{ ∂f \textbackslash{}over ∂t\} (t,y) +
f(t,y)\{ ∂f \textbackslash{}over ∂y\} (t,y) est continue sur ce compact,
donc bornée. Soit

M =\{\textbackslash{}mathop\{ sup\}\}\_\{(t,y)∈K\}\textbar{}\{ ∂f
\textbackslash{}over ∂t\} (t,y) + f(t,y)\{ ∂f \textbackslash{}over ∂y\}
(t,y)\textbar{}

Alors on a pour tout t dans {[}a,b{]} \textbar{}φ''(t)\textbar{}≤ M. La
formule de Taylor-Lagrange nous donne alors φ(t + h) = φ(t) + hφ'(t) +\{
\{h\}\^{}\{2\} \textbackslash{}over 2\} φ''(ξ), d'où

\textbackslash{}left \textbar{}\{ φ(t + h) − φ(t) \textbackslash{}over
h\} − f(t,φ(t))\textbackslash{}right \textbar{}≤ M\{
\textbar{}h\textbar{} \textbackslash{}over 2\}

D'autre part la fonction (t,y)\textbackslash{}mathrel\{↦\}\{ ∂f
\textbackslash{}over ∂y\} (t,y) est continue sur K donc bornée. Soit A
=\{\textbackslash{}mathop\{ sup\}\}\_\{(t,y)∈K\}\textbar{}\{ ∂f
\textbackslash{}over ∂y\} (t,y)\textbar{}. Alors on a si (t,y) ∈ K et
(t,y') ∈ K, f(t,y) − f(t,y') = (y − y')\{ ∂f \textbackslash{}over ∂y\}
(t,z) pour un certain z appartenant à {[}y,y'{]}, donc \textbar{}f(t,y)
− f(t,y')\textbar{}≤ A\textbar{}y − y'\textbar{} .

Définissons alors une suite \{(\{t\}\_\{i\})\}\_\{i∈ℕ\} par \{t\}\_\{i\}
= \{t\}\_\{o\} + ih, une suite \{(\{y\}\_\{i\})\}\_\{i∈ℕ\} par
\{y\}\_\{i+1\} = \{y\}\_\{i\} + hf(\{t\}\_\{i\},\{y\}\_\{i\}). Nous
allons mesurer l'erreur \{e\}\_\{i\} = \textbar{}\{y\}\_\{i\} −
φ(\{t\}\_\{i\})\textbar{}. Comme nous ne sommes malheureusement pas sûrs
que les couples (\{t\}\_\{i\},\{y\}\_\{i\}) appartiennent à K nous
allons définir une fonction g : {[}a,b{]} × ℝ → ℝ par

g(x,y) = \textbackslash{}left \textbackslash{}\{ \textbackslash{}cases\{
f(t,y) \&si φ(t) − α ≤ y ≤ φ(t) + α \textbackslash{}cr f(t,φ(t) − α)\&si
y \textless{} φ(t) − α \textbackslash{}cr f(t,φ(t) + α)\&si y
\textgreater{} φ(t) + α)) \} \textbackslash{}right .

et une suite \{(\{z\}\_\{i\})\}\_\{i∈ℕ\} par \{z\}\_\{i+1\} =
\{z\}\_\{i\} + hg(\{t\}\_\{i\},\{z\}\_\{i\}). Il est clair que
\{z\}\_\{i\} = \{y\}\_\{i\} tant que (\{t\}\_\{i\},\{y\}\_\{i\})
appartient à K. Posons \{ε\}\_\{i\} = \textbar{}\{z\}\_\{i\} −
φ(\{t\}\_\{i\})\textbar{}. La fonction g est continue sur {[}a,b{]} × ℝ
et vérifie \textbar{}g(t,y) − g(t,y')\textbar{}≤ A\textbar{}y −
y'\textbar{} pour tout t ∈ {[}a,b{]} et tous y,y' ∈ ℝ d'après (2). On a
alors

\textbackslash{}begin\{eqnarray*\}\{ ε\}\_\{i+1\}\& =\&
\textbar{}\{z\}\_\{i+1\} − φ(\{t\}\_\{i+1\})\textbar{} =
\textbar{}\{z\}\_\{i\} + hg(\{t\}\_\{i\},\{z\}\_\{i\}) − φ(\{t\}\_\{i\}
+ h)\textbar{}\%\& \textbackslash{}\textbackslash{} \& ≤\&
\textbar{}\{z\}\_\{i\} − φ(\{t\}\_\{i\})\textbar{} +
\textbar{}h\textbar{}\textbar{}g(\{t\}\_\{i\},\{z\}\_\{i\}) −
g(\{t\}\_\{i\},φ(\{t\}\_\{i\}))\textbar{} \%\&
\textbackslash{}\textbackslash{} \& \& +\textbar{}φ(\{t\}\_\{i\}) +
hg(\{t\}\_\{i\},φ(\{t\}\_\{i\})) − φ(\{t\}\_\{i\} + h)\textbar{} \%\&
\textbackslash{}\textbackslash{} \& ≤\& \textbar{}\{z\}\_\{i\} −
φ(\{t\}\_\{i\})\textbar{} + A\textbar{}h\textbar{}\textbar{}\{z\}\_\{i\}
− φ(\{t\}\_\{i\})\textbar{} \%\& \textbackslash{}\textbackslash{} \& \&
+\textbar{}h\textbar{}\textbackslash{}left \textbar{}\{ φ(\{t\}\_\{i\} +
h) − φ(\{t\}\_\{i\}) \textbackslash{}over h\} −
f(\{t\}\_\{i\},φ(\{t\}\_\{i\}))\textbackslash{}right \textbar{} \%\&
\textbackslash{}\textbackslash{} \textbackslash{}end\{eqnarray*\}

car g(t,φ(t)) = f(t,φ(t)). On obtient donc en utilisant (1)
\{ε\}\_\{i+1\} ≤ (1 + A\textbar{}h\textbar{})\{ε\}\_\{i\} + M\{
\textbar{}h\{\textbar{}\}\^{}\{2\} \textbackslash{}over 2\} . Comme
\{ε\}\_\{o\} = \{e\}\_\{o\} = 0, on a donc par récurrence

\textbackslash{}begin\{eqnarray*\}\{ ε\}\_\{i\}\& ≤\& M\{
\textbar{}h\{\textbar{}\}\^{}\{2\} \textbackslash{}over 2\} (1 + (1 +
A\textbar{}h\textbar{}) + \ldots{} + \{(1 +
A\textbar{}h\textbar{})\}\^{}\{i−1\})\%\&
\textbackslash{}\textbackslash{} \& =\& M\{
\textbar{}h\{\textbar{}\}\^{}\{2\} \textbackslash{}over 2\} \{ \{(1 +
A\textbar{}h\textbar{})\}\^{}\{i\} − 1 \textbackslash{}over
\textbar{}h\textbar{}A\} \%\& \textbackslash{}\textbackslash{}
\textbackslash{}end\{eqnarray*\}

soit, puisque 1 + x ≤ \{e\}\^{}\{x\},

\textbackslash{}begin\{eqnarray*\}\{ ε\}\_\{i\}\& ≤ M\&\{
\textbar{}h\textbar{} \textbackslash{}over 2\} \{
\{e\}\^{}\{Ai\textbar{}h\textbar{}\}− 1 \textbackslash{}over A\}
≤\textbar{}h\textbar{}\{ M \textbackslash{}over 2A\}
(\{e\}\^{}\{A\textbar{}t−ti\textbar{}\}− 1)\%\&
\textbackslash{}\textbackslash{} \& ≤ \& \textbar{}h\textbar{}\{ M
\textbackslash{}over 2A\} (\{e\}\^{}\{A(b−a)\} − 1) \%\&
\textbackslash{}\textbackslash{} \textbackslash{}end\{eqnarray*\}

On voit donc que pour h assez petit, on a pour tout i tel que
\{t\}\_\{i\} appartienne à {[}a,b{]}, \{ε\}\_\{i\} ≤ α, donc
\{z\}\_\{i\} = \{y\}\_\{i\}, et donc \{ε\}\_\{i\} = \{e\}\_\{i\}, avec
une erreur \{e\}\_\{i\} ≤\textbar{}h\textbar{}\{ M \textbackslash{}over
2A\} (\{e\}\^{}\{A(b−a)\} − 1).

Soit maintenant x ∈{]}\{t\}\_\{i\},\{t\}\_\{i+1\}{[}. Considérons la
fonction affine g qui vérifie g(\{t\}\_\{i\}) = φ(\{t\}\_\{i\}) et
g(\{t\}\_\{i+1\}) = φ(\{t\}\_\{i+1\}). Soit h(t) = φ(t) − g(t) − μ\{
(t−\{t\}\_\{i\})(t−\{t\}\_\{i+1\}) \textbackslash{}over 2\} où μ est
choisi de telle sorte que h(x) = 0. Deux applications du théorème de
Rolle à la fonction h qui s'annule en \{t\}\_\{i\}, x et \{t\}\_\{i+1\}
montrent qu'il existe ξ tel que h''(ξ) = 0. Or h'`(ξ) = φ''(ξ) − μ
puisque g'' = 0. On a donc en écrivant que h(x) = 0,

φ(x) − g(x) = φ''(ξ)\{ (x − \{t\}\_\{i\})(x − \{t\}\_\{i+1\})
\textbackslash{}over 2\}

soit

\textbar{}φ(x) − g(x)\textbar{}≤ M\{ \{(\{t\}\_\{i+1\} −
\{t\}\_\{i\})\}\^{}\{2\} \textbackslash{}over 8\} = M\{ \{h\}\^{}\{2\}
\textbackslash{}over 8\}

puisque l'on a vu que pour tout t dans {[}a,b{]},
\textbar{}φ''(t)\textbar{}≤ M. Or g et \{φ\}\_\{h\} sont affines sur
{[}\{t\}\_\{i\},\{t\}\_\{i+1\}{]} et donc

\textbackslash{}begin\{eqnarray*\} \textbar{}g(x) −
\{φ\}\_\{h\}(x)\textbar{}\& ≤\&
\textbackslash{}mathop\{max\}(\textbar{}g(\{t\}\_\{i\}) −
\{φ\}\_\{h\}(\{t\}\_\{i\})\textbar{},\textbar{}g(\{t\}\_\{i+1\}) −
\{φ\}\_\{h\}(\{t\}\_\{i+1\})\textbar{})\%\&
\textbackslash{}\textbackslash{} \& =\&
\textbackslash{}mathop\{max\}(\{e\}\_\{i\},\{e\}\_\{i+1\})
≤\textbar{}h\textbar{}\{ M \textbackslash{}over 2A\}
(\{e\}\^{}\{A(b−a)\} − 1) \%\& \textbackslash{}\textbackslash{}
\textbackslash{}end\{eqnarray*\}

On en déduit donc que

\textbackslash{}mathop\{∀\}x ∈ {[}a,b{]}, \textbar{}φ(x) −
\{φ\}\_\{h\}(x)\textbar{}≤\textbar{}h\textbar{}\{ M \textbackslash{}over
2A\} (\{e\}\^{}\{A(b−a)\} − 1) + M\{ \{h\}\^{}\{2\} \textbackslash{}over
8\}

(en fait on n'a vu cette majoration que sur {[}\{t\}\_\{o\},b{]}, mais
il suffit de changer h en − h pour avoir le même résultat sur
{[}a,\{t\}\_\{o\}{]}, c'est pour cela que volontairement nous avons
laissé les valeurs absolues partout). Les fonctions \{φ\}\_\{h\}
convergent donc uniformément vers φ sur {[}a,b{]} quand h tend vers 0,
avec une majoration du type

\textbackslash{}mathop\{∀\}x ∈ {[}a,b{]},\textbar{}φ(x) −
\{φ\}\_\{h\}(x)\textbar{}≤ B\textbar{}h\textbar{}

Dans la pratique il faut faire attention aux accumulations d'erreurs
d'arrondis. L'erreur sur le calcul de \{y\}\_\{n\} est de l'ordre de nε,
où ε est la précision de calcul de l'ordinateur. On en déduit que
l'erreur sur le calcul de \{φ\}\_\{h\}(x) est de l'ordre de grandeur de
\{ b−a \textbackslash{}over h\} ε. Soit une erreur totale du type Bh +\{
b−a \textbackslash{}over h\} ε. Une étude simple de cette fonction
montre que l'erreur totale est minimale pour des fonctions usuelles
quand h est de l'ordre de \textbackslash{}sqrt\{ε\}. On préférera
prendre une valeur de h un peu trop grande, plutôt que trop petite. Il
faut se méfier également du temps de calcul qui croit rapidement si l'on
prend des valeurs de h trop petites.

\paragraph{16.6.2 Méthode de Runge et Kutta}

La méthode est inspirée de la même idée que celle de la méthode d'Euler,
mais on améliore l'approximation faite. Dans la méthode d'Euler nous
prenions pour approximation de φ(t + h) l'expression φ(t) + hφ'(t) =
φ(t) + hf(t,φ(t)). Ici on prendra comme approximation de φ(t + h)
l'expression φ(t) + hk(t,h) où k(t,h) est défini en posant

\textbackslash{}begin\{eqnarray*\}\{ k\}\_\{1\}(t,h)\& =\& f(t,φ(t)),
\%\& \textbackslash{}\textbackslash{} \{k\}\_\{2\}(t,h)\& =\& f(t +\{ h
\textbackslash{}over 2\} ,φ(t) +\{ h \textbackslash{}over 2\}
\{k\}\_\{1\}(t,h)),\%\& \textbackslash{}\textbackslash{}
\{k\}\_\{3\}(t,h)\& =\& f(t +\{ h \textbackslash{}over 2\} ,φ(t) +\{ h
\textbackslash{}over 2\} \{k\}\_\{2\}(t,h)),\%\&
\textbackslash{}\textbackslash{} \{k\}\_\{4\}(t,h)\& =\& f(t + h,φ(t) +
h\{k\}\_\{3\}(t,h)) \%\& \textbackslash{}\textbackslash{}
\textbackslash{}end\{eqnarray*\}

et enfin

k(t,h) =\{ 1 \textbackslash{}over 6\} (\{k\}\_\{1\}(t,h) +
2\{k\}\_\{2\}(t,h) + 2\{k\}\_\{3\}(t,h) + \{k\}\_\{4\}(t,h)).

On définit donc notre suite \{y\}\_\{i\} par la relation de récurrence
\{y\}\_\{i+1\} = \{y\}\_\{i\} + hk(i) où l'on a posé

\textbackslash{}begin\{eqnarray*\}\{ k\}\_\{1\}(i)\& =\&
f(\{t\}\_\{i\},\{y\}\_\{i\}), \%\& \textbackslash{}\textbackslash{}
\{k\}\_\{2\}(i)\& =\& f(\{t\}\_\{i\} +\{ h \textbackslash{}over 2\}
,\{y\}\_\{i\} +\{ h \textbackslash{}over 2\} \{k\}\_\{1\}(i)), \%\&
\textbackslash{}\textbackslash{} \{k\}\_\{3\}(i)\& =\& f(\{t\}\_\{i\}
+\{ h \textbackslash{}over 2\} ,\{y\}\_\{i\} +\{ h \textbackslash{}over
2\} \{k\}\_\{2\}(i)), \%\& \textbackslash{}\textbackslash{}
\{k\}\_\{4\}(i)\& =\& f(\{t\}\_\{i\} + h,\{y\}\_\{i\} +
h\{k\}\_\{3\}(i)), \%\& \textbackslash{}\textbackslash{} k(i)\& =\&\{ 1
\textbackslash{}over 6\} (\{k\}\_\{1\}(i) + 2\{k\}\_\{2\}(i) +
2\{k\}\_\{3\}(i) + \{k\}\_\{4\}(i))\%\& \textbackslash{}\textbackslash{}
\textbackslash{}end\{eqnarray*\}

pour i ≥ 0 , pour i \textless{} 0 on change h en − h.

On montre alors par un calcul pénible (à base de formule de Taylor) que

\textbackslash{}left \textbar{}\{ φ(t+h)−φ(t) \textbackslash{}over h\} −
k(t,h)\textbackslash{}right \textbar{}≤ M\{
\textbar{}h\{\textbar{}\}\^{}\{4\} \textbackslash{}over 2\} et la même
démonstration que dans la méthode d'Euler montre qu'il existe une
constante C telle que

\textbackslash{}mathop\{∀\}x ∈ {[}a,b{]}, \textbar{}φ(x) −
\{φ\}\_\{h\}(x)\textbar{}≤ C\textbar{}h\{\textbar{}\}\^{}\{4\} +
D\textbar{}h\{\textbar{}\}\^{}\{2\}

Le terme en \{h\}\^{}\{2\} provient en fait de l'interpolation linéaire.
Aux points \{t\}\_\{i\} l'erreur est en fait en
C\textbar{}h\{\textbar{}\}\^{}\{4\}. On obtient ainsi une convergence
beaucoup plus rapide que dans la méthode d'Euler. L'étude de
l'accumulation des erreurs montre que le meilleur h possible (pour les
points \{t\}\_\{i\}) est de l'ordre de
\textbackslash{}root\{5\}\textbackslash{}of\{ε\} où ε est la précision
de l'ordinateur. On pourra par exemple prendre un h de l'ordre de
1\{0\}\^{}\{−2\} ou 1\{0\}\^{}\{−3\}.

\paragraph{16.6.3 Equations différentielles d'ordre supérieur}

Il suffit de rappeler qu'une équation différentielle d'ordre p du type
\{y\}\^{}\{(p)\} = f(t,y,y',\ldots{},\{y\}\^{}\{(p−1)\}) se ramène à un
système différentiel

\textbackslash{}left \textbackslash{}\{\textbackslash{}array\{
\{y\}\_\{1\}' \& = \{y\}\_\{2\} \textbackslash{}cr
\&\textbackslash{}mathop\{\textbackslash{}mathop\{\ldots{}\}\}\textbackslash{}cr
\{y\}\_\{ p−1\}'\& = \{y\}\_\{p\} \textbackslash{}cr \{y\}\_\{p\}' \& =
f(t,\{y\}\_\{1\},\{y\}\_\{2\},\ldots{},\{y\}\_\{p\}) \}
\textbackslash{}right .

en posant \{y\}\_\{1\} = y,\{y\}\_\{2\} = y',\ldots{},\{y\}\_\{p\} =
\{y\}\^{}\{(p−1)\}. On appliquera donc l'une des deux méthodes
précédentes à ce système.

{[}\href{coursse90.html}{prev}{]}
{[}\href{coursse90.html\#tailcoursse90.html}{prev-tail}{]}
{[}\href{coursse91.html}{front}{]}
{[}\href{coursch17.html\#coursse91.html}{up}{]}

\end{document}
