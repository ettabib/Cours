\documentclass[]{article}
\usepackage[T1]{fontenc}
\usepackage{lmodern}
\usepackage{amssymb,amsmath}
\usepackage{ifxetex,ifluatex}
\usepackage{fixltx2e} % provides \textsubscript
% use upquote if available, for straight quotes in verbatim environments
\IfFileExists{upquote.sty}{\usepackage{upquote}}{}
\ifnum 0\ifxetex 1\fi\ifluatex 1\fi=0 % if pdftex
  \usepackage[utf8]{inputenc}
\else % if luatex or xelatex
  \ifxetex
    \usepackage{mathspec}
    \usepackage{xltxtra,xunicode}
  \else
    \usepackage{fontspec}
  \fi
  \defaultfontfeatures{Mapping=tex-text,Scale=MatchLowercase}
  \newcommand{\euro}{€}
\fi
% use microtype if available
\IfFileExists{microtype.sty}{\usepackage{microtype}}{}
\ifxetex
  \usepackage[setpagesize=false, % page size defined by xetex
              unicode=false, % unicode breaks when used with xetex
              xetex]{hyperref}
\else
  \usepackage[unicode=true]{hyperref}
\fi
\hypersetup{breaklinks=true,
            bookmarks=true,
            pdfauthor={},
            pdftitle={Introduction aux integrales de surface},
            colorlinks=true,
            citecolor=blue,
            urlcolor=blue,
            linkcolor=magenta,
            pdfborder={0 0 0}}
\urlstyle{same}  % don't use monospace font for urls
\setlength{\parindent}{0pt}
\setlength{\parskip}{6pt plus 2pt minus 1pt}
\setlength{\emergencystretch}{3em}  % prevent overfull lines
\setcounter{secnumdepth}{0}
 
/* start css.sty */
.cmr-5{font-size:50%;}
.cmr-7{font-size:70%;}
.cmmi-5{font-size:50%;font-style: italic;}
.cmmi-7{font-size:70%;font-style: italic;}
.cmmi-10{font-style: italic;}
.cmsy-5{font-size:50%;}
.cmsy-7{font-size:70%;}
.cmex-7{font-size:70%;}
.cmex-7x-x-71{font-size:49%;}
.msbm-7{font-size:70%;}
.cmtt-10{font-family: monospace;}
.cmti-10{ font-style: italic;}
.cmbx-10{ font-weight: bold;}
.cmr-17x-x-120{font-size:204%;}
.cmsl-10{font-style: oblique;}
.cmti-7x-x-71{font-size:49%; font-style: italic;}
.cmbxti-10{ font-weight: bold; font-style: italic;}
p.noindent { text-indent: 0em }
td p.noindent { text-indent: 0em; margin-top:0em; }
p.nopar { text-indent: 0em; }
p.indent{ text-indent: 1.5em }
@media print {div.crosslinks {visibility:hidden;}}
a img { border-top: 0; border-left: 0; border-right: 0; }
center { margin-top:1em; margin-bottom:1em; }
td center { margin-top:0em; margin-bottom:0em; }
.Canvas { position:relative; }
li p.indent { text-indent: 0em }
.enumerate1 {list-style-type:decimal;}
.enumerate2 {list-style-type:lower-alpha;}
.enumerate3 {list-style-type:lower-roman;}
.enumerate4 {list-style-type:upper-alpha;}
div.newtheorem { margin-bottom: 2em; margin-top: 2em;}
.obeylines-h,.obeylines-v {white-space: nowrap; }
div.obeylines-v p { margin-top:0; margin-bottom:0; }
.overline{ text-decoration:overline; }
.overline img{ border-top: 1px solid black; }
td.displaylines {text-align:center; white-space:nowrap;}
.centerline {text-align:center;}
.rightline {text-align:right;}
div.verbatim {font-family: monospace; white-space: nowrap; text-align:left; clear:both; }
.fbox {padding-left:3.0pt; padding-right:3.0pt; text-indent:0pt; border:solid black 0.4pt; }
div.fbox {display:table}
div.center div.fbox {text-align:center; clear:both; padding-left:3.0pt; padding-right:3.0pt; text-indent:0pt; border:solid black 0.4pt; }
div.minipage{width:100%;}
div.center, div.center div.center {text-align: center; margin-left:1em; margin-right:1em;}
div.center div {text-align: left;}
div.flushright, div.flushright div.flushright {text-align: right;}
div.flushright div {text-align: left;}
div.flushleft {text-align: left;}
.underline{ text-decoration:underline; }
.underline img{ border-bottom: 1px solid black; margin-bottom:1pt; }
.framebox-c, .framebox-l, .framebox-r { padding-left:3.0pt; padding-right:3.0pt; text-indent:0pt; border:solid black 0.4pt; }
.framebox-c {text-align:center;}
.framebox-l {text-align:left;}
.framebox-r {text-align:right;}
span.thank-mark{ vertical-align: super }
span.footnote-mark sup.textsuperscript, span.footnote-mark a sup.textsuperscript{ font-size:80%; }
div.tabular, div.center div.tabular {text-align: center; margin-top:0.5em; margin-bottom:0.5em; }
table.tabular td p{margin-top:0em;}
table.tabular {margin-left: auto; margin-right: auto;}
div.td00{ margin-left:0pt; margin-right:0pt; }
div.td01{ margin-left:0pt; margin-right:5pt; }
div.td10{ margin-left:5pt; margin-right:0pt; }
div.td11{ margin-left:5pt; margin-right:5pt; }
table[rules] {border-left:solid black 0.4pt; border-right:solid black 0.4pt; }
td.td00{ padding-left:0pt; padding-right:0pt; }
td.td01{ padding-left:0pt; padding-right:5pt; }
td.td10{ padding-left:5pt; padding-right:0pt; }
td.td11{ padding-left:5pt; padding-right:5pt; }
table[rules] {border-left:solid black 0.4pt; border-right:solid black 0.4pt; }
.hline hr, .cline hr{ height : 1px; margin:0px; }
.tabbing-right {text-align:right;}
span.TEX {letter-spacing: -0.125em; }
span.TEX span.E{ position:relative;top:0.5ex;left:-0.0417em;}
a span.TEX span.E {text-decoration: none; }
span.LATEX span.A{ position:relative; top:-0.5ex; left:-0.4em; font-size:85%;}
span.LATEX span.TEX{ position:relative; left: -0.4em; }
div.float img, div.float .caption {text-align:center;}
div.figure img, div.figure .caption {text-align:center;}
.marginpar {width:20%; float:right; text-align:left; margin-left:auto; margin-top:0.5em; font-size:85%; text-decoration:underline;}
.marginpar p{margin-top:0.4em; margin-bottom:0.4em;}
.equation td{text-align:center; vertical-align:middle; }
td.eq-no{ width:5%; }
table.equation { width:100%; } 
div.math-display, div.par-math-display{text-align:center;}
math .texttt { font-family: monospace; }
math .textit { font-style: italic; }
math .textsl { font-style: oblique; }
math .textsf { font-family: sans-serif; }
math .textbf { font-weight: bold; }
.partToc a, .partToc, .likepartToc a, .likepartToc {line-height: 200%; font-weight:bold; font-size:110%;}
.chapterToc a, .chapterToc, .likechapterToc a, .likechapterToc, .appendixToc a, .appendixToc {line-height: 200%; font-weight:bold;}
.index-item, .index-subitem, .index-subsubitem {display:block}
.caption td.id{font-weight: bold; white-space: nowrap; }
table.caption {text-align:center;}
h1.partHead{text-align: center}
p.bibitem { text-indent: -2em; margin-left: 2em; margin-top:0.6em; margin-bottom:0.6em; }
p.bibitem-p { text-indent: 0em; margin-left: 2em; margin-top:0.6em; margin-bottom:0.6em; }
.subsectionHead, .likesubsectionHead { margin-top:2em; font-weight: bold;}
.sectionHead, .likesectionHead { font-weight: bold;}
.quote {margin-bottom:0.25em; margin-top:0.25em; margin-left:1em; margin-right:1em; text-align:justify;}
.verse{white-space:nowrap; margin-left:2em}
div.maketitle {text-align:center;}
h2.titleHead{text-align:center;}
div.maketitle{ margin-bottom: 2em; }
div.author, div.date {text-align:center;}
div.thanks{text-align:left; margin-left:10%; font-size:85%; font-style:italic; }
div.author{white-space: nowrap;}
.quotation {margin-bottom:0.25em; margin-top:0.25em; margin-left:1em; }
h1.partHead{text-align: center}
.sectionToc, .likesectionToc {margin-left:2em;}
.subsectionToc, .likesubsectionToc {margin-left:4em;}
.sectionToc, .likesectionToc {margin-left:6em;}
.frenchb-nbsp{font-size:75%;}
.frenchb-thinspace{font-size:75%;}
.figure img.graphics {margin-left:10%;}
/* end css.sty */

\title{Introduction aux integrales de surface}
\author{}
\date{}

\begin{document}
\maketitle

\textbf{Warning: 
requires JavaScript to process the mathematics on this page.\\ If your
browser supports JavaScript, be sure it is enabled.}

\begin{center}\rule{3in}{0.4pt}\end{center}

[
[
[]
[

\section{20.4 Introduction aux intégrales de surface}

Définition~20.4.1 Soit \Sigma = (D,F) une nappe paramétrée de classe
\mathcal{C}^1de \mathbb{R}~^3, où D est un compact de \mathbb{R}~^2
de frontière négligeable. Soit f une fonction définie et continue sur
l'image de \Sigma et à valeurs dans l'espace vectoriel normé E. On appelle
intégrale de f le long de \Sigma et on note \int ~
\int  _\Sigma~f(m) d\sigma l'élément de E

\int  \\int ~
_\Sigmaf(m) d\sigma =\int ~
\int  _D~f(F(u,v))
\ \partial~F \over \partial~u (u,v) ∧ \partial~F
\over \partial~v (u,v)\ du dv

En particulier, on appelle aire de \Sigma le nombre réel positif

m(\Sigma) =\int  \\int ~
_\Sigma d\sigma =\int ~ \\int
 _D\ \partial~F \over \partial~u
(u,v) ∧ \partial~F \over \partial~v (u,v)\
du dv

Le principal résultat sur ces intégrales de surface est l'invariance par
changement de paramétrage admissible

Théorème~20.4.1 Soit \Sigma_1 = (D_1,F_1) et
\Sigma_2 = (D_2,F_2) deux nappes paramétrées de
classe \mathcal{C}^1 équivalentes, où D_1 et D_2 sont
des compacts de \mathbb{R}~^2 de frontières négligeables. Soit f une
fonction définie sur l'image de \Sigma_1 et \Sigma_2, à valeurs
dans l'espace vectoriel normé E. Alors

\int  \\int ~
_\Sigma_1f(m) d\sigma =\int ~
\int  _\Sigma_2~f(m) d\sigma

En particulier, l'aire de la nappe est invariante par changement de
paramétrage.

Démonstration Soit \theta : D_1 \rightarrow~ D_2 un difféomorphisme de
l'intérieur de D_1 sur l'intérieur de D_2 vérifiant
F_1 = F_2 \cdot \theta. Un calcul fait dans le chapitre sur les
nappes paramétrées montre que (si on note
(u,v)\mapsto~F_1(u,v) et
(\lambda~,\mu)\mapsto~F_2(\lambda~,\mu))

 \partial~F_1 \over \partial~u (u,v) ∧ \partial~F_1
\over \partial~v (u,v) = j_\theta(u,v) \partial~F_2
\over \partial~\lambda~ (\theta(u,v)) ∧ \partial~F_2
\over \partial~\mu (\theta(u,v))

On en déduit que

\begin{align*} \int ~
\int  _\Sigma_1~f(m) d\sigma&& \%&
\\ & =& \int ~
\int  _D_1f(F_1~(u,v))
\ \partial~F \over \partial~u (u,v) ∧ \partial~F
\over \partial~v (u,v)\ du dv \%&
\\ & =& \int ~
\int  _D_1f(F_2~ \cdot
\theta(u,v)) \ \partial~F_2 \over
\partial~\lambda~ (\theta(u,v)) ∧ \partial~F_2 \over \partial~\mu
(\theta(u,v))\ j_\theta(u,v)
du dv\%& \\ & =&
\int  \\int ~
_D_2f(F_2(\lambda~,\mu)) \
\partial~F_2 \over \partial~\lambda~ (\lambda~,\mu) ∧ \partial~F_2
\over \partial~\mu (\lambda~,\mu)\ d\lambda~ d\mu \%&
\\ \end{align*}

par le théorème de changement de variables dans les intégrales doubles.

Remarque~20.4.1 Le lecteur attentif aura remarqué que nous avons modifié
légèrement les définitions d'une nappe paramétrée et de l'équivalence de
deux nappes paramétrées, de fa\ccon à ce que cela
nous arrange. Nous réclamons toute son indulgence pour ces modifications
de détail.

Comme cas particulier, cherchons l'aire d'une nappe de révolution d'axe
Oz. Soit \Gamma une méridienne de cette nappe, paramétrée en coordonnées
cylindriques par r = \phi(t) et z = \psi(t), t \in [a,b]. Un paramétrage de
la nappe est alors F(t,\theta) = O + \phi(t)\vecu(\theta) +
\psi(t)\veck, (t,\theta) \in [a,b] \times [0,2\pi~] si bien que
 \partial~F \over \partial~t (t,\theta) = \phi'(t)\vecu(\theta)
+ \psi'(t)\veck et  \partial~F \over \partial~\theta (t,\theta)
= \phi(t)\vecu'(\theta). On a donc

 \partial~F \over \partial~t (t,\theta) ∧ \partial~F \over \partial~\theta
(t,\theta) = \phi(t)\left (\phi'(t)\veck -
\psi(t)\vecu(\theta)\right )

et donc

\ \partial~F \over \partial~t (t,\theta) ∧ \partial~F
\over \partial~\theta (t,\theta)\ =
\phi(t)\sqrt\phi'(t)^2  +
\psi'(t)^2

On en déduit que

\begin{align*} m(\Sigma)& =& \\int
 \int ~
_[a,b]\times[0,2\pi~]\phi(t)\sqrt\phi'(t)^2
 + \psi'(t)^2 dt d\theta\%& \\ &
=& 2\pi~\int ~
_a^b\phi(t)\sqrt\phi'(t)^2
 + \psi'(t)^2 dt \%& \\ &
=& 2\pi~\int  _\Gamma~r ds
\%& \\ \end{align*}

en notant r = \phi(t) et ds = \sqrt\phi'(t)^2  +
\psi'(t)^2 dt la différentielle de l'abscisse curviligne sur
\Gamma. On obtient donc

Proposition~20.4.2 Soit \Sigma la nappe de révolution engendrée par la
rotation de la méridienne \Gamma autour de la droite D. Soit ds la
différentielle de l'abscisse curviligne de \Gamma et r la distance d'un point
de \Gamma à la droite D. Alors l'aire de la nappe est égale à
2\pi~\int  _\Gamma~r ds.

[
[
[
[

\end{document}
