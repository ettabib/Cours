\documentclass[]{article}
\usepackage[T1]{fontenc}
\usepackage{lmodern}
\usepackage{amssymb,amsmath}
\usepackage{ifxetex,ifluatex}
\usepackage{fixltx2e} % provides \textsubscript
% use upquote if available, for straight quotes in verbatim environments
\IfFileExists{upquote.sty}{\usepackage{upquote}}{}
\ifnum 0\ifxetex 1\fi\ifluatex 1\fi=0 % if pdftex
  \usepackage[utf8]{inputenc}
\else % if luatex or xelatex
  \ifxetex
    \usepackage{mathspec}
    \usepackage{xltxtra,xunicode}
  \else
    \usepackage{fontspec}
  \fi
  \defaultfontfeatures{Mapping=tex-text,Scale=MatchLowercase}
  \newcommand{\euro}{€}
\fi
% use microtype if available
\IfFileExists{microtype.sty}{\usepackage{microtype}}{}
\ifxetex
  \usepackage[setpagesize=false, % page size defined by xetex
              unicode=false, % unicode breaks when used with xetex
              xetex]{hyperref}
\else
  \usepackage[unicode=true]{hyperref}
\fi
\hypersetup{breaklinks=true,
            bookmarks=true,
            pdfauthor={},
            pdftitle={Developpements asymptotiques},
            colorlinks=true,
            citecolor=blue,
            urlcolor=blue,
            linkcolor=magenta,
            pdfborder={0 0 0}}
\urlstyle{same}  % don't use monospace font for urls
\setlength{\parindent}{0pt}
\setlength{\parskip}{6pt plus 2pt minus 1pt}
\setlength{\emergencystretch}{3em}  % prevent overfull lines
\setcounter{secnumdepth}{0}
 
/* start css.sty */
.cmr-5{font-size:50%;}
.cmr-7{font-size:70%;}
.cmmi-5{font-size:50%;font-style: italic;}
.cmmi-7{font-size:70%;font-style: italic;}
.cmmi-10{font-style: italic;}
.cmsy-5{font-size:50%;}
.cmsy-7{font-size:70%;}
.cmex-7{font-size:70%;}
.cmex-7x-x-71{font-size:49%;}
.msbm-7{font-size:70%;}
.cmtt-10{font-family: monospace;}
.cmti-10{ font-style: italic;}
.cmbx-10{ font-weight: bold;}
.cmr-17x-x-120{font-size:204%;}
.cmsl-10{font-style: oblique;}
.cmti-7x-x-71{font-size:49%; font-style: italic;}
.cmbxti-10{ font-weight: bold; font-style: italic;}
p.noindent { text-indent: 0em }
td p.noindent { text-indent: 0em; margin-top:0em; }
p.nopar { text-indent: 0em; }
p.indent{ text-indent: 1.5em }
@media print {div.crosslinks {visibility:hidden;}}
a img { border-top: 0; border-left: 0; border-right: 0; }
center { margin-top:1em; margin-bottom:1em; }
td center { margin-top:0em; margin-bottom:0em; }
.Canvas { position:relative; }
li p.indent { text-indent: 0em }
.enumerate1 {list-style-type:decimal;}
.enumerate2 {list-style-type:lower-alpha;}
.enumerate3 {list-style-type:lower-roman;}
.enumerate4 {list-style-type:upper-alpha;}
div.newtheorem { margin-bottom: 2em; margin-top: 2em;}
.obeylines-h,.obeylines-v {white-space: nowrap; }
div.obeylines-v p { margin-top:0; margin-bottom:0; }
.overline{ text-decoration:overline; }
.overline img{ border-top: 1px solid black; }
td.displaylines {text-align:center; white-space:nowrap;}
.centerline {text-align:center;}
.rightline {text-align:right;}
div.verbatim {font-family: monospace; white-space: nowrap; text-align:left; clear:both; }
.fbox {padding-left:3.0pt; padding-right:3.0pt; text-indent:0pt; border:solid black 0.4pt; }
div.fbox {display:table}
div.center div.fbox {text-align:center; clear:both; padding-left:3.0pt; padding-right:3.0pt; text-indent:0pt; border:solid black 0.4pt; }
div.minipage{width:100%;}
div.center, div.center div.center {text-align: center; margin-left:1em; margin-right:1em;}
div.center div {text-align: left;}
div.flushright, div.flushright div.flushright {text-align: right;}
div.flushright div {text-align: left;}
div.flushleft {text-align: left;}
.underline{ text-decoration:underline; }
.underline img{ border-bottom: 1px solid black; margin-bottom:1pt; }
.framebox-c, .framebox-l, .framebox-r { padding-left:3.0pt; padding-right:3.0pt; text-indent:0pt; border:solid black 0.4pt; }
.framebox-c {text-align:center;}
.framebox-l {text-align:left;}
.framebox-r {text-align:right;}
span.thank-mark{ vertical-align: super }
span.footnote-mark sup.textsuperscript, span.footnote-mark a sup.textsuperscript{ font-size:80%; }
div.tabular, div.center div.tabular {text-align: center; margin-top:0.5em; margin-bottom:0.5em; }
table.tabular td p{margin-top:0em;}
table.tabular {margin-left: auto; margin-right: auto;}
div.td00{ margin-left:0pt; margin-right:0pt; }
div.td01{ margin-left:0pt; margin-right:5pt; }
div.td10{ margin-left:5pt; margin-right:0pt; }
div.td11{ margin-left:5pt; margin-right:5pt; }
table[rules] {border-left:solid black 0.4pt; border-right:solid black 0.4pt; }
td.td00{ padding-left:0pt; padding-right:0pt; }
td.td01{ padding-left:0pt; padding-right:5pt; }
td.td10{ padding-left:5pt; padding-right:0pt; }
td.td11{ padding-left:5pt; padding-right:5pt; }
table[rules] {border-left:solid black 0.4pt; border-right:solid black 0.4pt; }
.hline hr, .cline hr{ height : 1px; margin:0px; }
.tabbing-right {text-align:right;}
span.TEX {letter-spacing: -0.125em; }
span.TEX span.E{ position:relative;top:0.5ex;left:-0.0417em;}
a span.TEX span.E {text-decoration: none; }
span.LATEX span.A{ position:relative; top:-0.5ex; left:-0.4em; font-size:85%;}
span.LATEX span.TEX{ position:relative; left: -0.4em; }
div.float img, div.float .caption {text-align:center;}
div.figure img, div.figure .caption {text-align:center;}
.marginpar {width:20%; float:right; text-align:left; margin-left:auto; margin-top:0.5em; font-size:85%; text-decoration:underline;}
.marginpar p{margin-top:0.4em; margin-bottom:0.4em;}
.equation td{text-align:center; vertical-align:middle; }
td.eq-no{ width:5%; }
table.equation { width:100%; } 
div.math-display, div.par-math-display{text-align:center;}
math .texttt { font-family: monospace; }
math .textit { font-style: italic; }
math .textsl { font-style: oblique; }
math .textsf { font-family: sans-serif; }
math .textbf { font-weight: bold; }
.partToc a, .partToc, .likepartToc a, .likepartToc {line-height: 200%; font-weight:bold; font-size:110%;}
.chapterToc a, .chapterToc, .likechapterToc a, .likechapterToc, .appendixToc a, .appendixToc {line-height: 200%; font-weight:bold;}
.index-item, .index-subitem, .index-subsubitem {display:block}
.caption td.id{font-weight: bold; white-space: nowrap; }
table.caption {text-align:center;}
h1.partHead{text-align: center}
p.bibitem { text-indent: -2em; margin-left: 2em; margin-top:0.6em; margin-bottom:0.6em; }
p.bibitem-p { text-indent: 0em; margin-left: 2em; margin-top:0.6em; margin-bottom:0.6em; }
.subsectionHead, .likesubsectionHead { margin-top:2em; font-weight: bold;}
.sectionHead, .likesectionHead { font-weight: bold;}
.quote {margin-bottom:0.25em; margin-top:0.25em; margin-left:1em; margin-right:1em; text-align:justify;}
.verse{white-space:nowrap; margin-left:2em}
div.maketitle {text-align:center;}
h2.titleHead{text-align:center;}
div.maketitle{ margin-bottom: 2em; }
div.author, div.date {text-align:center;}
div.thanks{text-align:left; margin-left:10%; font-size:85%; font-style:italic; }
div.author{white-space: nowrap;}
.quotation {margin-bottom:0.25em; margin-top:0.25em; margin-left:1em; }
h1.partHead{text-align: center}
.sectionToc, .likesectionToc {margin-left:2em;}
.subsectionToc, .likesubsectionToc {margin-left:4em;}
.sectionToc, .likesectionToc {margin-left:6em;}
.frenchb-nbsp{font-size:75%;}
.frenchb-thinspace{font-size:75%;}
.figure img.graphics {margin-left:10%;}
/* end css.sty */

\title{Developpements asymptotiques}
\author{}
\date{}

\begin{document}
\maketitle

\textbf{Warning: 
requires JavaScript to process the mathematics on this page.\\ If your
browser supports JavaScript, be sure it is enabled.}

\begin{center}\rule{3in}{0.4pt}\end{center}

[
[
[]
[

\section{6.3 Développements asymptotiques}

\subsection{6.3.1 Echelles de comparaison, parties principales}

Définition~6.3.1 On appelle échelle de comparaison en a suivant A toute
famille (\phi_i)_i\inI de fonctions de ℱ_a,A(\mathbb{R}~)
vérifiant

\begin{itemize}
\itemsep1pt\parskip0pt\parsep0pt
\item
  (i) \forall~i \in I, \\forall~~V \in V
  (a), \exists~t \in V \bigcap A,
  \phi_i(t)\neq~0~; autrement dit, aucune
  des \phi_i n'est identiquement nulle au voisinage de a suivant A
\item
  (ii) si i\neq~j, l'une des deux fonctions
  \phi_i ou \phi_j est négligeable devant l'autre
\end{itemize}

Remarque~6.3.1 On obtient une relation d'ordre strict sur I en posant i
< j \Leftrightarrow \phi_j =
o(\phi_i).

Exemple~6.3.1 Au voisinage d'un point a \in \mathbb{R}~, on a plusieurs échelles de
comparaison classiques

\begin{itemize}
\item
  a) la famille des t\mapsto~(t - a)^n, n
  \in \mathbb{N}~ (l'échelle qui conduit aux développements limités)
\item
  b) la famille des t\mapsto~(t - a)^n, n
  \in \mathbb{Z}
\item
  c) la famille des t\mapsto~t -
  a^\alpha~, \alpha~ \in \mathbb{R}~
\item
  d) la famille des t\mapsto~t -
  a^\alpha~log~
  t^\beta~, \alpha~,\beta~ \in \mathbb{R}~~: on a alors

  (\alpha~,\beta~) < (\alpha~',\beta~') \Leftrightarrow
  \bigl (\alpha~ < \alpha~'\text ou (\alpha~
  = \alpha~'\text et \beta~ >
  \beta~')\bigr )
\end{itemize}

Exemple~6.3.2 Au voisinage de a = +\infty~, on a plusieurs échelles de
comparaison classiques

\begin{itemize}
\item
  a) la famille des t\mapsto~t^n, n \in \mathbb{Z}
  (l'ordre obtenu est l'inverse de l'ordre naturel)
\item
  b) la famille des t\mapsto~t^\alpha~, \alpha~ \in \mathbb{R}~
  (l'ordre obtenu est l'inverse de l'ordre naturel)
\item
  c) la famille des
  t\mapsto~t^\alpha~(log~
  t)^\beta~, \alpha~,\beta~ \in \mathbb{R}~~: on a alors

  (\alpha~,\beta~) < (\alpha~',\beta~') \Leftrightarrow
  \bigl (\alpha~ > \alpha~'\text ou
  (\alpha~ = \alpha~'\text et \beta~ >
  \beta~')\bigr )
\item
  c) la famille des
  t\mapsto~e^P(t)t^\alpha~(log~
  t)^\beta~, P \in \mathbb{R}~[X],\alpha~,\beta~ \in \mathbb{R}~~: on a alors

  (P,\alpha~,\beta~) < (Q,\alpha~',\beta~') \Leftrightarrow
  \left \\cases
  lim_t\rightarrow~+\infty~~(Q(t) - P(t)) = +\infty~
  \cr \cr \textou &
  \cr P = Q\text et \alpha~ >
  \alpha~' \cr \textou & \cr
  P = Q\text et \alpha~ = \alpha~'\text et \beta~
  > \beta~'  \right .
\end{itemize}

Définition~6.3.2 Soit (\phi_i)_i\inI une échelle de
comparaison en a suivant A et f \inℱ_a,A(E). On dit que f admet
une partie principale suivant l'échelle de comparaison s'il existe i \in I
et a_i \in E \diagdown\0\ tels que f(t)
∼ a_i\phi_i(t). Une telle partie principale si elle
existe est unique.

Démonstration Si on a f(t) ∼ a_i\phi_i(t) ∼
b_j\phi_j(t), on a nécessairement i = j car sinon une des
deux fonctions serait négligeable devant l'autre. On a alors
(a_i - b_i)\phi_i = o(\phi_i) ce qui n'est
possible que si a_i = b_i.

\subsection{6.3.2 Développements asymptotiques}

Définition~6.3.3 Soit (\phi_i)_i\inI une échelle de
comparaison en a suivant A et f \inℱ_a,A(E). On dit que f admet
un développement asymptotique à la précision \phi_j suivant
l'échelle de comparaison s'il existe i_0 <
i_1 <
\\ldots~ <
i_p \leq j et
a_0,a_1,\\ldots,a_p~
\in E \diagdown\0\ tels que

f(t) = a_0\phi_i_0(t) +
\\ldots~ +
a_p\phi_i_p(t) + o(\phi_j(t))

Remarque~6.3.2 Un tel développement est nécessairement unique puisque
a_0\phi_i_0(t) est nécessairement la partie
principale de f(t), a_1\phi_i_1(t) celle de f(t)
- a_0\phi_i_0(t) et ainsi de suite jusqu'à
a_p\phi_i_p(t) qui doit être la partie
principale de f(t) - a_0\phi_i_0(t)
-\\ldots~ -
a_p-1\phi_i_p-1(t).

\subsection{6.3.3 Opérations sur les développements asymptotiques}

Il est clair que si f et g admettent des développements asymptotiques à
la précision \phi_j et \phi_j', alors \alpha~f + \beta~g admet un
développement asymptotique à la précision
\phi_min(j,j')~ obtenu de la manière
évidente (additionner les deux et supprimer les termes non
significatifs).

Si l'échelle de comparaison est stable par produit, en faisant le
produit de deux développements asymptotiques on obtient un développement
asymptotique du produit des deux fonctions, à une précision à évaluer
suivant les cas. Ceci peut permettre également de composer
développements asymptotiques et développements limités.

De plus, les théorèmes de comparaison des intégrales impropres peuvent
permettre d'intégrer des développements asymptotiques~:

à condition que la fonction g soit positive au voisinage de a et que son
intégrale converge au point a,

f = o(g) \rigtharrow~\int  _a^x~f =
o(\int  _a^x~g)

Exemple~6.3.3 ~: on a pour x > 0 au voisinage du point 0,

\begin{align*} d \over dx
(arcsin~ (1 - x))& =& - 1
\over \sqrt2x - x^2 = -
1 \over \sqrt2x  1
\over \sqrt1 - x \over
2  \%& \\ & =& - 1
\over \sqrt2x (1 + x
\over 4 + x^2 \over 16 +
o(x^2)) \%& \\ & =& -
\sqrt2 \over
2\sqrtx - \sqrt2
\over 8 \sqrtx -
\sqrt2 \over 32 x^3\diagup2 +
o(x^3\diagup2)\%& \\
\end{align*}

En intégrant de 0 à x on va obtenir, en tenant compte de \phi(t) =
o(t^3\diagup2) \rigtharrow~\int ~
_0^x\phi(t) dt = o(\int ~
_0^xt^3\diagup2 dt)

arcsin (1 - x) = \pi~ \over 2~
-\sqrt2x - \sqrt2
\over 12 x^3\diagup2 - \sqrt2
\over 80 x^5\diagup2 + o(x^5\diagup2)

[
[
[
[

\end{document}
