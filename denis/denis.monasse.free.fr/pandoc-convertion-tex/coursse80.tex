\section{14.4 Fonctions périodiques de période T}

Remarque~14.4.1 Remarquons que si f est périodique de période T, alors
\tildef définie par $\tildef(t) =
f( T \over 2\pi~ t)$ est périodique de période $2\pi~$ et l'on
$a f(x) =\tilde f( 2\pi~ \over T x)$ .
Ceci permet d'adapter tous les résultats précédents aux fonctions de
période T.

On pose

\begin{align*} (f∣g)&
=& 1 \over T \int ~
_0^T\overlinef(t)g(t) dt = 1
\over T \int ~
_a^a+T\overlinef(t)g(t) dt\%&
\\
\f_2^2&
=& (f∣f) = 1 \over T
\int ~
_0^Tf(t)^2 dt \%&
\\ e_n(t)& =&
e^2i\pi~nt\diagupT \%& \\
\end{align*}

Alors $(e_n)_n \in \mathbb{Z}$ est une famille orthonormée de C. On
définit les coefficients de Fourier de $f \in C $par

\begin{align*} \forall~~n \in
\mathbb{Z},\quad c_n(f)& =&
(e_n∣f) = 1 \over
T \int ~
_0^Tf(t)e^-2\pi~int\diagupT dt\%&
\\ \forall~~n ≥
0,\quad a_n(f)& =& 2 \over
T \int ~
_0^Tf(t)cos~  2\pi~nt
\over T dt \%& \\
\forall~n ≥ 1,\quad b_n~(f)&
=& 2 \over T \int ~
_0^Tf(t)sin~  2\pi~nt
\over T dt \%& \\
\end{align*}

la série de Fourier de f par

\begin{align*} c_0(f)& +&
\\sum
_n≥1(c_n(f)e^2\pi~inx\diagupT + c_
-n(f)e^-2\pi~inx\diagupT) \%& \\ &
=& a_0(f) \over 2 +
\\sum
_n≥1(a_n(f)\cos  2\pi~nx
\over T + b_n(f)\sin 2\pi~nx
\over T )\%& \\
\end{align*}

et on a les théorèmes

\begin{thm}[Bessel]
 Soit $f : \mathbb{R}~ \rightarrow~ \mathbb{C}$ continue par morceaux et
périodique de période T. Alors la série
\[
c_0(f)^2
+ \\sum ~
_n≥1(c_n(f)^2 +
c_-n(f)^2)
\]
 est convergente et on
a
\[
c_0(f)^2 +
\sum _n=1^+\infty~(c_
n(f)^2 + c_
-n(f)^2) \leq\
f_ 2^2
\]
\end{thm}
Théorème~14.4.2 (Dirichlet). Soit $f :  \mathbb{R}~ \rightarrow~ \mathbb{C} $ de
classe $\mathcal{C}^1 $ par
morceaux et périodique de période T. Alors la série de Fourier de f
converge sur $\mathbb{R}~ et \forall~~x \in \mathbb{R}$~,

\begin{align*} f(x^+) +
f(x^-) \over 2 && \%&
\\ & =& c_0(f) +
\sum _n=1^+\infty~(c_
n(f)e^2\pi~inx\diagupT + c_ -n(f)e^-2\pi~inx\diagupT)
\%& \\ & =& a_0(f)
\over 2 + \\sum
_n=1^+\infty~(a_ n(f)\cos  2\pi~nx
\over T + b_n(f)\sin  2\pi~nx
\over T )\%& \\
\end{align*}

\begin{thm}
  (Dirichlet). Soit $f : \mathbb{R}~ \rightarrow~ \mathbb{C}$ périodique de période Tde
classe $\mathcal{C}^1$ par morceaux et continue. Alors la série
\[
c_0(f) +\
\sum ~
_n≥1(c_n(f) +
c_-n(f))
\] converge, la série de Fourier de f
converge normalement sur $\mathbb{R}$ et on a $\forall~~x \in \mathbb{R}$,

\begin{align*} f(x)& =& c_0(f) +
\sum _n=1^+\infty~(c_
n(f)e^2\pi~inx\diagupT + c_ -n(f)e^-2\pi~inx\diagupT)
\%& \\ & =& a_0(f)
\over 2 + \\sum
_n=1^+\infty~(a_ n(f)\cos  2\pi~nx
\over T + b_n(f)\sin  2\pi~nx
\over T )\%& \\
\end{align*}

\end{thm}
(autrement dit f est somme de sa série de Fourier).

\begin{thm}
  Soit f : \mathbb{R}~ \rightarrow~ \mathbb{C} périodique de période T de classe
C^k. Alors
\[
\forall~n \in \mathbb{Z}, c_n~(f) =
\left ( 2\pi~in \over T
\right )^kc_ n(f^(k))
\]
et, quand n tend vers $+ \infty~$, $c_n(f) = o( 1
\over n^k )$.
\end{thm}

\begin{thm}[Parseval-Plancherel]
 . Soit $f : \mathbb{R}~ \rightarrow~ \mathbb{C} $ périodique de
période T et continue par morceaux. Alors
\begin{align*}
\\f_2^2& =& 1 \over T \int ~
_0^Tf(t)^2 dt \%&
\\ & =&
c_0(f)^2 +
\sum _n=1^+\infty~(c_
n(f)^2 + c_
-n(f)^2) \%& \\ &
=& a_0(f)^2
\over 4 + 1 \over 2
\sum _n=1^+\infty~(a_
n(f)^2 + b_
n(f)^2)\%& \\
\end{align*}

\end{thm}
