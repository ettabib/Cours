\documentclass[]{article}
\usepackage[T1]{fontenc}
\usepackage{lmodern}
\usepackage{amssymb,amsmath}
\usepackage{ifxetex,ifluatex}
\usepackage{fixltx2e} % provides \textsubscript
% use upquote if available, for straight quotes in verbatim environments
\IfFileExists{upquote.sty}{\usepackage{upquote}}{}
\ifnum 0\ifxetex 1\fi\ifluatex 1\fi=0 % if pdftex
  \usepackage[utf8]{inputenc}
\else % if luatex or xelatex
  \ifxetex
    \usepackage{mathspec}
    \usepackage{xltxtra,xunicode}
  \else
    \usepackage{fontspec}
  \fi
  \defaultfontfeatures{Mapping=tex-text,Scale=MatchLowercase}
  \newcommand{\euro}{€}
\fi
% use microtype if available
\IfFileExists{microtype.sty}{\usepackage{microtype}}{}
\ifxetex
  \usepackage[setpagesize=false, % page size defined by xetex
              unicode=false, % unicode breaks when used with xetex
              xetex]{hyperref}
\else
  \usepackage[unicode=true]{hyperref}
\fi
\hypersetup{breaklinks=true,
            bookmarks=true,
            pdfauthor={},
            pdftitle={Serie de Fourier d'une fonction},
            colorlinks=true,
            citecolor=blue,
            urlcolor=blue,
            linkcolor=magenta,
            pdfborder={0 0 0}}
\urlstyle{same}  % don't use monospace font for urls
\setlength{\parindent}{0pt}
\setlength{\parskip}{6pt plus 2pt minus 1pt}
\setlength{\emergencystretch}{3em}  % prevent overfull lines
\setcounter{secnumdepth}{0}
 
/* start css.sty */
.cmr-5{font-size:50%;}
.cmr-7{font-size:70%;}
.cmmi-5{font-size:50%;font-style: italic;}
.cmmi-7{font-size:70%;font-style: italic;}
.cmmi-10{font-style: italic;}
.cmsy-5{font-size:50%;}
.cmsy-7{font-size:70%;}
.cmex-7{font-size:70%;}
.cmex-7x-x-71{font-size:49%;}
.msbm-7{font-size:70%;}
.cmtt-10{font-family: monospace;}
.cmti-10{ font-style: italic;}
.cmbx-10{ font-weight: bold;}
.cmr-17x-x-120{font-size:204%;}
.cmsl-10{font-style: oblique;}
.cmti-7x-x-71{font-size:49%; font-style: italic;}
.cmbxti-10{ font-weight: bold; font-style: italic;}
p.noindent { text-indent: 0em }
td p.noindent { text-indent: 0em; margin-top:0em; }
p.nopar { text-indent: 0em; }
p.indent{ text-indent: 1.5em }
@media print {div.crosslinks {visibility:hidden;}}
a img { border-top: 0; border-left: 0; border-right: 0; }
center { margin-top:1em; margin-bottom:1em; }
td center { margin-top:0em; margin-bottom:0em; }
.Canvas { position:relative; }
li p.indent { text-indent: 0em }
.enumerate1 {list-style-type:decimal;}
.enumerate2 {list-style-type:lower-alpha;}
.enumerate3 {list-style-type:lower-roman;}
.enumerate4 {list-style-type:upper-alpha;}
div.newtheorem { margin-bottom: 2em; margin-top: 2em;}
.obeylines-h,.obeylines-v {white-space: nowrap; }
div.obeylines-v p { margin-top:0; margin-bottom:0; }
.overline{ text-decoration:overline; }
.overline img{ border-top: 1px solid black; }
td.displaylines {text-align:center; white-space:nowrap;}
.centerline {text-align:center;}
.rightline {text-align:right;}
div.verbatim {font-family: monospace; white-space: nowrap; text-align:left; clear:both; }
.fbox {padding-left:3.0pt; padding-right:3.0pt; text-indent:0pt; border:solid black 0.4pt; }
div.fbox {display:table}
div.center div.fbox {text-align:center; clear:both; padding-left:3.0pt; padding-right:3.0pt; text-indent:0pt; border:solid black 0.4pt; }
div.minipage{width:100%;}
div.center, div.center div.center {text-align: center; margin-left:1em; margin-right:1em;}
div.center div {text-align: left;}
div.flushright, div.flushright div.flushright {text-align: right;}
div.flushright div {text-align: left;}
div.flushleft {text-align: left;}
.underline{ text-decoration:underline; }
.underline img{ border-bottom: 1px solid black; margin-bottom:1pt; }
.framebox-c, .framebox-l, .framebox-r { padding-left:3.0pt; padding-right:3.0pt; text-indent:0pt; border:solid black 0.4pt; }
.framebox-c {text-align:center;}
.framebox-l {text-align:left;}
.framebox-r {text-align:right;}
span.thank-mark{ vertical-align: super }
span.footnote-mark sup.textsuperscript, span.footnote-mark a sup.textsuperscript{ font-size:80%; }
div.tabular, div.center div.tabular {text-align: center; margin-top:0.5em; margin-bottom:0.5em; }
table.tabular td p{margin-top:0em;}
table.tabular {margin-left: auto; margin-right: auto;}
div.td00{ margin-left:0pt; margin-right:0pt; }
div.td01{ margin-left:0pt; margin-right:5pt; }
div.td10{ margin-left:5pt; margin-right:0pt; }
div.td11{ margin-left:5pt; margin-right:5pt; }
table[rules] {border-left:solid black 0.4pt; border-right:solid black 0.4pt; }
td.td00{ padding-left:0pt; padding-right:0pt; }
td.td01{ padding-left:0pt; padding-right:5pt; }
td.td10{ padding-left:5pt; padding-right:0pt; }
td.td11{ padding-left:5pt; padding-right:5pt; }
table[rules] {border-left:solid black 0.4pt; border-right:solid black 0.4pt; }
.hline hr, .cline hr{ height : 1px; margin:0px; }
.tabbing-right {text-align:right;}
span.TEX {letter-spacing: -0.125em; }
span.TEX span.E{ position:relative;top:0.5ex;left:-0.0417em;}
a span.TEX span.E {text-decoration: none; }
span.LATEX span.A{ position:relative; top:-0.5ex; left:-0.4em; font-size:85%;}
span.LATEX span.TEX{ position:relative; left: -0.4em; }
div.float img, div.float .caption {text-align:center;}
div.figure img, div.figure .caption {text-align:center;}
.marginpar {width:20%; float:right; text-align:left; margin-left:auto; margin-top:0.5em; font-size:85%; text-decoration:underline;}
.marginpar p{margin-top:0.4em; margin-bottom:0.4em;}
.equation td{text-align:center; vertical-align:middle; }
td.eq-no{ width:5%; }
table.equation { width:100%; } 
div.math-display, div.par-math-display{text-align:center;}
math .texttt { font-family: monospace; }
math .textit { font-style: italic; }
math .textsl { font-style: oblique; }
math .textsf { font-family: sans-serif; }
math .textbf { font-weight: bold; }
.partToc a, .partToc, .likepartToc a, .likepartToc {line-height: 200%; font-weight:bold; font-size:110%;}
.chapterToc a, .chapterToc, .likechapterToc a, .likechapterToc, .appendixToc a, .appendixToc {line-height: 200%; font-weight:bold;}
.index-item, .index-subitem, .index-subsubitem {display:block}
.caption td.id{font-weight: bold; white-space: nowrap; }
table.caption {text-align:center;}
h1.partHead{text-align: center}
p.bibitem { text-indent: -2em; margin-left: 2em; margin-top:0.6em; margin-bottom:0.6em; }
p.bibitem-p { text-indent: 0em; margin-left: 2em; margin-top:0.6em; margin-bottom:0.6em; }
.subsectionHead, .likesubsectionHead { margin-top:2em; font-weight: bold;}
.sectionHead, .likesectionHead { font-weight: bold;}
.quote {margin-bottom:0.25em; margin-top:0.25em; margin-left:1em; margin-right:1em; text-align:justify;}
.verse{white-space:nowrap; margin-left:2em}
div.maketitle {text-align:center;}
h2.titleHead{text-align:center;}
div.maketitle{ margin-bottom: 2em; }
div.author, div.date {text-align:center;}
div.thanks{text-align:left; margin-left:10%; font-size:85%; font-style:italic; }
div.author{white-space: nowrap;}
.quotation {margin-bottom:0.25em; margin-top:0.25em; margin-left:1em; }
h1.partHead{text-align: center}
.sectionToc, .likesectionToc {margin-left:2em;}
.subsectionToc, .likesubsectionToc {margin-left:4em;}
.sectionToc, .likesectionToc {margin-left:6em;}
.frenchb-nbsp{font-size:75%;}
.frenchb-thinspace{font-size:75%;}
.figure img.graphics {margin-left:10%;}
/* end css.sty */

\title{Serie de Fourier d'une fonction}
\author{}
\date{}

\begin{document}
\maketitle

\textbf{Warning: 
requires JavaScript to process the mathematics on this page.\\ If your
browser supports JavaScript, be sure it is enabled.}

\begin{center}\rule{3in}{0.4pt}\end{center}

[
[
[]
[

\section{14.3 Série de Fourier d'une fonction}

\subsection{14.3.1 Les espaces C et D}

Définition~14.3.1 On considère l'espace vectoriel C des fonctions de \mathbb{R}~
dans \mathbb{C}, continues par morceaux et périodiques de période 2\pi~. On
désignera par D le sous-espace vectoriel des applications f : \mathbb{R}~ \rightarrow~ \mathbb{C},
continues par morceaux, périodiques de période 2\pi~ et vérifiant
\forall~~x \in \mathbb{R}~, f(x) =
f(x^+)+f(x^-) \over 2 (où
f(x^+) et f(x^-) désignent respectivement les
limites à gauche et à droite de f au point x). Pour f,g \inC, on posera
(f∣g) = 1 \over 2\pi~
\int ~
_0^2\pi~\overlinef(t)g(t) dt,
\f_2 =
\sqrt(f∣ f) et
e_n : t\mapsto~e^int.

Théorème~14.3.1 L'application
(f,g)\mapsto~(f\mathrel∣g) est
une forme hermitienne positive sur C dont la restriction à D est définie
positive. La famille (e_n)_n\in\mathbb{Z} est une famille
orthonormée de C. Pour toute f \inC, on a
\f_2
\leq\ f_\infty~
(norme de la convergence uniforme).

Démonstration Le caractère sesquilinéaire et la symétrie hermitienne
sont évidents. Si f \inC, on a (f∣f) = 1
\over 2\pi~ \int ~
_0^2\pi~f(t)^2 dt ≥ 0. La
nullité de (f∣f) nécessite que f soit nulle
en tout point de [0,2\pi~] où elle est continue, soit sur [0,2\pi~]
privé d'un nombre fini de points. Si f est dans D, alors en chacun de
ces points on a f(x^+) = f(x^-) = 0 (car il existe
tout un intervalle ouvert à gauche de x sur lequel f est nul, et de même
à droite) et donc f(x) = 0, par conséquent f est la fonction nulle sur
[0,2\pi~], donc sur \mathbb{R}~.

Remarque~14.3.1 On prendra garde que si f est seulement continue par
morceaux,
\f_2 = 0
n'implique pas f = 0.

\subsection{14.3.2 Coefficients de Fourier d'une fonction continue par
morceaux}

Définition~14.3.2 Soit f : \mathbb{R}~ \rightarrow~ \mathbb{C} continue par morceaux et périodique de
période 2\pi~. On définit les coefficients de Fourier de la fonction f par

\begin{align*} \forall~~n \in
\mathbb{Z},\quad c_n(f)& =&
(e_n∣f) = 1 \over
2\pi~ \int ~
_0^2\pi~f(t)e^-int dt\%&
\\ \forall~~n ≥
0,\quad a_n(f)& =& 1 \over
\pi~ \int ~
_0^2\pi~f(t)cos~ nt dt \%&
\\ \forall~~n ≥
1,\quad b_n(f)& =& 1 \over
\pi~ \int ~
_0^2\pi~f(t)sin~ nt dt \%&
\\ \end{align*}

Remarque~14.3.2 Les fonctions intégrées étant périodiques de période 2\pi~,
on a aussi pour tout a \in \mathbb{R}~, c_n(f) = 1 \over
2\pi~ \int ~
_a^a+2\pi~f(t)e^-int dt, a_n(f) = 1
\over \pi~ \int ~
_a^a+2\pi~f(t)cos~ nt dt,
b_n(f) = 1 \over \pi~
\int ~
_a^a+2\pi~f(t)sin~ nt dt et en
particulier c_n(f) = 1 \over 2\pi~
\int  _-\pi~^\pi~f(t)e^-int~
dt, a_n(f) = 1 \over \pi~
\int ~
_-\pi~^\pi~f(t)cos~ nt dt,
b_n(f) = 1 \over \pi~
\int ~
_-\pi~^\pi~f(t)sin~ nt dt

Proposition~14.3.2 On a les relations suivantes

\begin{align*} c_0(f)& =&
a_0(f) \over 2 \%&
\\ \forall~~n ≥
1,\quad c_n(f)& =& a_n(f) -
ib_n(f) \over 2 ,\quad
c_-n(f) = a_n(f) + ib_n(f)
\over 2 \%& \\
\forall~n ≥ 1,\quad a_n~(f)&
=& c_n(f) + c_-n(f),\quad b_n
= i(c_n(f) - c_-n(f)) \%&
\\ \end{align*}

Démonstration Elémentaire

Proposition~14.3.3 Soit f : \mathbb{R}~ \rightarrow~ \mathbb{C} continue par morceaux et périodique de
période 2\pi~. Si f est à valeurs réelles, on a a_n(f) \in \mathbb{R}~,
b_n(f) \in \mathbb{R}~ et c_-n(f) =
\overlinec_n(f). Si f est paire (resp.
impaire) on a b_n(f) = 0 (resp. a_n(f) = 0).

Démonstration Si f est à valeurs réelles, il en est de même de
x\mapsto~f(x)cos~ nx et de
x\mapsto~f(x)sin~ nx ce qui
montre que a_n(f) et b_n(f) sont réels~; de plus
f(x)e^inx = \overlinef(x)e^-inx
ce qui montre que c_-n(f) =
\overlinec_n(f). Si f est paire, on a
b_n(f) = 1 \over 2\pi~
\int ~
_-\pi~^\pi~f(x)sin~ nx dx = 0 puisque
la fonction f(x)sin~ nx est impaire. Le
raisonnement est similaire si f est impaire avec les a_n(f).

Définition~14.3.3 Soit f : \mathbb{R}~ \rightarrow~ \mathbb{C} continue par morceaux et périodique de
période 2\pi~. On appelle série de Fourier de la fonction f la série
trigonométrique

\begin{align*} c_0(f) +
\sum _n≥1(c_n(f)e^inx~
+ c_ -n(f)e^-inx)&& \%&
\\ & & = a_0(f)
\over 2 + \\sum
_n≥1(a_n(f)\cos nx +
b_n(f)\sin nx)\%&
\\ \end{align*}

Définition~14.3.4 Pour n ≥ 1, on posera (sommes partielles de la série
de Fourier)

\begin{align*} S_n(f)(x)& =&
c_0(f) + \\sum
_p=1^n(c_ p(f)e^ipx + c_
-p(f)e^-ipx) \%& \\ & =&
a_0(f) \over 2 + \\sum
_p=1^n(a_ p(f)\cos px +
b_p(f)\sin px)\%&
\\ \end{align*}

\subsection{14.3.3 Inégalité de Bessel et théorème de Riemann-Lebesgue}

Définition~14.3.5 Pour N ≥ 1, on posera T_N
=\
\mathrmVect(e_-N,e_-N+1,\\ldots,e_-1,e_0,e_1,\\\ldots,e_N-1,e_N~)
(espace vectoriel des polynômes trigonométriques de degré inférieur ou
égal à N.

Remarque~14.3.3 On a également

T_N = \x\mapsto~
a_0 \over 2 + \\sum
_p=1^N(a_ p \cos px +
b_p \sin px)\

Par définition même
(e_-N,e_-N+1,\\ldots,e_-1,e_0,e_1,\\\ldots,e_N-1,e_N~)
est une base orthonormée de T_N.

Lemme~14.3.4 Soit f : \mathbb{R}~ \rightarrow~ \mathbb{C} continue par morceaux et périodique de
période 2\pi~. Alors
\S_N(f)_2^2
= \\sum ~
_k=-N^Nc_k(f)^2.

Démonstration
c_-N(f),\\ldots,c_0(f),\\\ldots,c_N~(f)
sont les coordonnées de S_N(f) dans la base orthonormée
(e_-N,e_-N+1,\\ldots,e_-1,e_0,e_1,\\\ldots,e_N-1,e_N~)~;
la norme au carré de S_N(f) est donc la somme des carrés des
modules de ces coordonnées~; d'où le résultat.

Lemme~14.3.5 Soit f : \mathbb{R}~ \rightarrow~ \mathbb{C} continue par morceaux et périodique de
période 2\pi~. Alors S_N(f) est la projection orthogonale de f sur
le sous-espace vectoriel T_N.

Démonstration Puisque S_N(f) appartient à T_N, il
suffit de montrer que f - S_N(f) \bot T_N ou encore que
\forall~~n \in [-N,N],
(e_n∣f - S_N(f)) = 0, ou
encore que \forall~~n \in [-N,N],
(e_n∣f) =
(e_n∣S_N(f)). Mais
(e_n∣S_N(f)) est la
coordonnée suivant e_n de S_N(f) (puisque la base est
orthonormée), c'est donc c_n(f) =
(e_n∣f) par définition, ce qui
montre le résultat.

Théorème~14.3.6 (Bessel). Soit f : \mathbb{R}~ \rightarrow~ \mathbb{C} continue par morceaux et
périodique de période 2\pi~. Alors la série
c_0(f)^2
+ \\sum ~
_n≥1(c_n(f)^2 +
c_-n(f)^2) est convergente et on
a

c_0(f)^2 +
\sum _n=1^+\infty~(c_
n(f)^2 + c_
-n(f)^2) \leq\
f_ 2^2

Démonstration Puisque S_N(f) est la projection orthogonale de f
sur T_N, on a f = S_N(f) + (f - S_N(f)) avec
S_N(f) \bot f - S_N(f). Le théorème de Pythagore assure
que
\f_2^2
=\
S_N(f)_2^2
+\ f -
S_N(f)_2^2, d'où
encore d'après le lemme 1

c_0(f)^2 +
\sum _n=1^N(c_
n(f)^2 + c_
-n(f)^2) =\ S_
N(f)_2^2
\leq\ f_
2^2

La série à termes positifs
c_0(f)^2
+ \\sum ~
_n≥1(c_n(f)^2 +
c_-n(f)^2) a ses sommes
partielles majorées par
\f_2^2,
donc elle converge et sa somme est majorée par
\f_2^2,
ce qui achève la démonstration.

Remarque~14.3.4 Un calcul élémentaire montre que pour n ≥ 1,

c_n(f)^2 + c_
-n(f)^2 = 1 \over 2
(a_n(f)^2 + b_
n(f)^2)

ce qui montre que les séries
\\sum ~
a_n(f)^2 et
\\sum ~
b_n(f)^2 convergent et que (en
tenant compte de a_0(f) = c_0(f)
\over 2 )

 a_0(f)^2 \over
4 + 1 \over 2 \\sum
_n=1^+\infty~(a_
n(f)^2 + b_
n(f)^2) \leq\
f\^2

Théorème~14.3.7 (Riemann-Lebesgue). Soit f : \mathbb{R}~ \rightarrow~ \mathbb{C} continue par morceaux
et périodique de période 2\pi~. Alors

lim_n\rightarrow~±\infty~c_n~(f)
= lim_n\rightarrow~+\infty~a_n~(f)
= lim_n\rightarrow~+\infty~b_n~(f) = 0

Démonstration Puisque les séries
\\sum ~
_n≥1(c_n(f)^2 +
c_-n(f)^2),
\\sum ~
a_n(f)^2 et
\\sum ~
b_n(f)^2 sont convergentes,
leurs termes généraux admettent la limite 0, ce qui montre le résultat.

\subsection{14.3.4 Les théorèmes de Dirichlet}

Nous aurons besoin par la suite du lemme suivant

Lemme~14.3.8 Pour tout entier n ≥ 1 et pour
t∉2\pi~\mathbb{Z},
\\sum ~
_k=-n^ne^ikt = sin~
(2n+1) t \over 2 \over
sin  t \over 2 ~ .

Démonstration On a en effet

\begin{align*} \\sum
_k=-n^ne^ikt& =& e^-int
\sum _k=0^2ne^ikt~ =
e^-int e^(2n+1)it - 1 \over
e^it - 1 \%& \\ & =&
e^(n+1)it - e^-int \over
e^it - 1 = e^(n+ 1 \over 2
)it - e^-(n+ 1 \over 2 )it
\over e^i t \over 2  -
e^-i t \over 2  \%&
\\ \end{align*}

en multipliant numérateur et dénominateur par e^-it\diagup2. On en
déduit immédiatement la formule souhaitée.

Théorème~14.3.9 (Dirichlet). Soit f : \mathbb{R}~ \rightarrow~ \mathbb{C} de classe \mathcal{C}^1 par
morceaux et périodique de période 2\pi~. Alors la série de Fourier de f
converge sur \mathbb{R}~ et

\forall~x \in \mathbb{R}~, f(x^+~) +
f(x^-) \over 2 = c_0(f) +
\sum _n=1^+\infty~(c_
n(f)e^inx + c_ -n(f)e^-inx)

Démonstration On a

\begin{align*} S_n(f)(x)& =& 1
\over 2\pi~ \\sum
_k=-n^ne^inx
\\int  ~
_0^2\pi~f(t)e^-int dt\%&
\\ & =& 1 \over 2\pi~
\int ~
_0^2\pi~f(t)\left (\\sum
_k=-n^ne^in(x-t)\right )
dt\%& \\ & =& 1 \over
2\pi~ \int  _0^2\pi~~f(t)
sin (2n + 1) x-t \over 2~
\over sin~  x-t
\over 2  dt \%& \\
\end{align*}

Faisons le changement de variable t = x + u, on obtient

\begin{align*} S_n(f)(x)& =& 1
\over 2\pi~ \int ~
_-x^2\pi~-xf(x + u) sin~ (2n +
1) u \over 2 \over
sin  u \over 2 ~ du\%&
\\ & =& 1 \over 2\pi~
\int  _-\pi~^\pi~~f(x + u)
sin (2n + 1) u \over 2~
\over sin~  u
\over 2  du \%& \\
\end{align*}

puisque la fonction intégrée est périodique de période 2\pi~ et que donc
son intégrale sur tout intervalle de longueur 2\pi~ est la même. Coupons
l'intégrale en deux, l'une de - \pi~ à 0, l'autre de 0 à \pi~. Dans la
première faisons le changement de variable u = -2v et dans la seconde le
changement de variable u = 2v. On obtient

\begin{align*} S_n(f)(x)& =& 1
\over \pi~ \int ~
_0^\pi~\diagup2f(x - 2v) sin~ (2n + 1)v
\over sin v~ dv \%&
\\ & \text & + 1
\over \pi~ \int ~
_0^\pi~\diagup2f(x + 2v) sin~ (2n + 1)v
\over sin v~ dv \%&
\\ & =& 1 \over \pi~
\int  _0^\pi~\diagup2~(f(x + 2v) + f(x -
2v)) sin~ (2n + 1)v \over
sin v~ dv\%&
\\ \end{align*}

Appliquons le résultat précédent à la fonction constante f_0 :
x\mapsto~1. On a bien entendu
S_n(f_0)(x) = 1 puisque c_0(f_0) = 1
et c_n(f_0) = 0 pour n\neq~0~;
on obtient

1 = 2 \over \pi~ \int ~
_0^\pi~\diagup2 sin~ (2n + 1)v
\over sin v~ dv

On en déduit que

\begin{align*} S_n(f)(x) -
f(x^+) + f(x^-) \over 2 =&&
\%& \\ & & 1 \over \pi~
\int  _0^\pi~\diagup2~ f(x + 2v) -
f(x^+) + f(x - 2v) - f(x_ -) \over
sin v \sin~ (2n +
1)v dv\%& \\
\end{align*}

Considérons la fonction g périodique de période 2\pi~ définie par

\begin{align*} g(v)& =& f(x + 2v) -
f(x^+) + f(x - 2v) - f(x_-) \over
sin v \text pour ~v
\in]0, \pi~ \over 2 ]\%&
\\ g(0)& =& 2(f'(x^+) -
f'(x^-)) \%& \\ g(v)& =&
0\text pour v \in] \pi~ \over 2
,2\pi~[ \%& \\
\end{align*}

Comme la fonction \tildef définie par
\tildef(x) = f(x^+) et
\tildef(t) = f(t) pour t > x est
dérivable à droite au point x (puisque f est de classe \mathcal{C}^1
par morceaux), on a, quand v tend vers 0 par valeurs supérieures,

\begin{align*} f(x + 2v) - f(x^+))
\over sin v~ &
∼_v\rightarrow~0,v>0& f(x + 2v) - f(x^+)
\over v \%& \\ & = &
2 \tildef(x + 2v) -\tilde f(x)
\over 2v \%& \\
\end{align*}

de limite 2f'(x^+). De même on a

lim_v\rightarrow~0,v>0~ f(x - 2v)
- f(x_-) \over sin v~
= -2f'(x^-)

ce qui montre que g est continue à droite au point 0. On en déduit
immédiatement que g est continue par morceaux. Mais alors

\begin{align*} S_n(f)(x) -
f(x^+) + f(x^-) \over 2 && \%&
\\ & =& 1 \over \pi~
\int ~
_0^2\pi~g(v)sin~ (2n + 1)v dv =
b_ 2n+1(g)\%& \\
\end{align*}

D'après le théorème de Riemann-Lebesgue, cette expression tend vers 0
quand n tend vers + \infty~, ce qui montre à la fois la convergence de la
série et donne la valeur de sa somme.

Lemme~14.3.10 Soit f : \mathbb{R}~ \rightarrow~ \mathbb{C} périodique de période 2\pi~de classe
\mathcal{C}^1 par morceaux et continue. Alors
\forall~n \in \mathbb{Z}, c_n(f') = inc_n~(f)
(où f' désigne la fonction de D égale à la dérivée de f sauf en un
nombre fini de points modulo 2\pi~).

Démonstration Soit \sigma = (a_i)_0\leqi\leqp une subdivision de
[0,2\pi~] adaptée à f. En tout point de [0,2\pi~]
\diagdown\a_0,\\ldots,a_p\~,
f'(t) est la dérivée de f et on pose f'(a_i) = 1
\over 2 (f'(a_i^+) + f'(a_
i^-)), si bien que f' \inD. Une intégration par parties donne, si
[a,b] \subset~]a_i-1,a_i[,

\begin{align*} \int ~
_a^bf'(t)e^-int dt& =& \left
[f(t)e^-int\right ]_ a^b +
in\int  _a^bf(t)e^-int~
dt \%& \\ & =& f(b)e^-inb -
f(a)e^-ina + in\int ~
_a^bf(t)e^-int dt\%&
\\ \end{align*}

En faisant tendre a vers a_i-1 et b vers a_i, en
tenant compte de la continuité de f aux points a_i-1 et
a_i on obtient

\begin{align*} \int ~
_a_i-1^a_i f'(t)e^-int dt&
=& f(a_ i)e^-ina_i  -
f(a_i-1)e^-ina_i-1 \%&
\\ & \text &
+in\int ~
_a_i-1^a_i f(t)e^-int dt \%&
\\ \end{align*}

et en sommant

\begin{align*} \int ~
_0^2\pi~f'(t)e^-int dt&& \%&
\\ & =& \\sum
_i=1^p
\\int  ~
_a_i-1^a_i f'(t)e^-int dt
\%& \\ & =& \\sum
_i=1^p\left (f(a_
i)e^-ina_i  -
f(a_i-1)e^-ina_i-1 \right )
+ in\\int  ~
_a_i-1^a_i f(t)e^-int dt\%&
\\ & =&
f(a_p)e^-ina_p  -
f(a_0)e^-ina_0  +
in\int  _a_0^a_p~
f(t)e^-int dt \%& \\ & =&
in\int ~
_0^2\pi~f(t)e^-int dt \%&
\\ \end{align*}

puisque a_0 = 0, a_p = 2\pi~,
f(a_p)e^-ina_p = f(2\pi~)e^-in2\pi~ =
f(2\pi~) = f(0) = f(a_0)e^-ina_0. En divisant
par 2\pi~, on obtient c_n(f') = inc_n(f).

Théorème~14.3.11 (Dirichlet). Soit f : \mathbb{R}~ \rightarrow~ \mathbb{C} périodique de période 2\pi~ de
classe \mathcal{C}^1 par morceaux et continue. Alors la série
c_0(f) +\
\sum ~
_n≥1(c_n(f) +
c_-n(f)) converge, la série de Fourier de f
converge normalement sur \mathbb{R}~ et on a

\forall~x \in \mathbb{R}~, f(x) = c_0~(f) +
\sum _n=1^+\infty~(c_
n(f)e^inx + c_ -n(f)e^-inx)

(autrement dit f est somme de sa série de Fourier).

Démonstration Pour a et b réels on a ab \leq 1 \over 2
(a^2 + b^2)~; on en déduit que si
n\neq~0, on a 0
\leqc_n(f) = \left 
c_n(f') \over in \right
\leq 1 \over 2
(c_n(f)^2 + 1
\over n^2 ). D'après le théorème de Bessel,
la série \\sum ~
_n≥0c_n(f')^2 converge et
d'après la théorie des séries de Riemann la série
\\sum ~  1
\over n^2 converge. On en déduit que la
série \\sum ~
_n≥1c_n(f) converge. On montre de la
même fa\ccon que la série
\\sum ~
_n≥1c_-n(f) converge, d'où la
convergence de la série
\\sum ~
_n≥1(c_n(f) +
c_-n(f)). La convergence normale de la
série de Fourier en résulte immédiatement puisque

\forall~~x \in \mathbb{R}~,
c_n(f)e^inx + c_
-n(f)e^-inx\leqc_ n(f)
+ c_-n(f)

qui est une série convergente indépendante de x. La formule résulte du
premier théorème de Dirichlet en remarquant que si f est continue, f(x)
= f(x^+)+f(x^-) \over 2 .

\subsection{14.3.5 Coefficients de Fourier des fonctions de classe
C^k}

Théorème~14.3.12 Soit f : \mathbb{R}~ \rightarrow~ \mathbb{C} périodique de période 2\pi~ de classe
C^k. Alors

\forall~n \in \mathbb{Z}, c_n~(f) =
(in)^kc_ n(f^(k))

et, quand n tend vers + \infty~, c_n(f) = o( 1
\over n^k ).

Démonstration On a vu que c_n(f') = inc_n(f) et il
suffit de faire une récurrence évidente sur k pour obtenir
c_n(f) = (in)^kc_n(f^(k)). Comme
le théorème de Riemann-Lebesgue assure que
lim_n\rightarrow~+\infty~c_n(f^(k)~)
= 0, on a c_n(f) = o( 1 \over n^k
).

Remarque~14.3.5 Autrement dit, plus la fonction est régulière, plus vite
les coefficients de Fourier tendent vers 0 à l'infini. Si f est de
classe C^\infty~, on a pour tout k \in \mathbb{N}~,
lim_n\rightarrow~+\infty~n^kc_n~(f)
= 0 (typiquement les coefficients de Fourier seront à décroissance
exponentielle).

\subsection{14.3.6 Le théorème de Parseval}

Lemme~14.3.13 Soit f : \mathbb{R}~ \rightarrow~ \mathbb{C} périodique de période 2\pi~ et continue par
morceaux. Alors, pour tout \epsilon > 0, il existe g : \mathbb{R}~ \rightarrow~ \mathbb{C}
périodique de période 2\pi~, de classe \mathcal{C}^1 par morceaux et
continue telle que \f -
g_2 < \epsilon.

Démonstration Supposons tout d'abord que f est en escalier et soit 0 =
a_0 < a_1 <
\\ldots~ <
a_p = 2\pi~ une subdivision de [0,2\pi~] adaptée à f avec f(t) =
\lambda_i pour t \in]a_i-1,a_i[. Soit \delta le pas de
la subdivision. Pour  2 \over n < \eta
définissons une fonction g_n par

\begin{itemize}
\itemsep1pt\parskip0pt\parsep0pt
\item
  (i) \forall~~i \in [0,p],
  g_n(a_i) = 0
\item
  (ii) \forall~~i \in [1,p],
  \forall~t \in [a_i-1~ + 1
  \over n ,a_i - 1 \over n
  ], g_n(t) = \lambda_i
\item
  (iii) g_n est affine sur chacun des intervalles
  [a_i-1,a_i-1 + 1 \over n ] et
  [a_i - 1 \over n ,a_i].
\end{itemize}

Il est clair que g_n est continue, affine par morceaux. Comme
de plus g_n(0) = g_n(2\pi~) = 0 elle se prolonge en une
application continue et périodique de période 2\pi~ sur \mathbb{R}~. Puisque
g_n est affine par morceaux, elle est a fortiori de classe
\mathcal{C}^1 par morceaux. On a

\begin{align*} \int ~
_a_i-1^a_i f(t) -
g_n(t)^2 dt& =&
\int ~
_a_i-1^a_i-1+ 1 \over n
f(t) - g_n(t)^2 dt + \%&
\\ & \text &
\int  _a_i~- 1
\over n ^a_i f(t) -
g_n(t)^2 dt \%&
\\ \end{align*}

Mais on a g(t) = n\lambda_i(t - a_i) pour t \in
[a_i-1,a_i-1 + 1 \over n ] et
g(t) = -n\lambda_i(t - a_i) pour t \in [a_i - 1
\over n ,a_i]. On a donc

\begin{align*} \int ~
_a_i-1^a_i f(t) -
g_n(t)^2 dt&& \%&
\\ & =&
\lambda_i^2\left
(\int ~
_a_i-1^a_i-1+ 1 \over n
(1 - n(t - a_i-1))^2 dt\right .
\%& \\ & \text &
\quad \quad + \left
.\int  _a_i~- 1
\over n ^a_i (1 + n(t -
a_i))^2 dt\right ) \%&
\\ & =&
\lambda_i^2\left
(\int  _0~^ 1 \over
n (1 - nu)^2 dt +\int  _-
1 \over n ^0(1 + nu)^2
dt\right )\%& \\ & =&
\lambda_i^2 \over 3n
\left (\left [-(1 -
nu)^3\right ]_ 0^ 1
\over n  + \left [(1 +
nu)^3\right ]_- 1 \over
n ^0\right ) \%&
\\ & =&
2\lambda_i^2 \over 3n
\%& \\ \end{align*}

soit encore

2\pi~\f -
g_n_2^2
=\int  _0^2\pi~~f(t) -
g_ n(t)^2 dt = 2 \over
3n \\sum
_i=1^p\lambda_ i^2

On en déduit que
lim_n\rightarrow~+\infty~~\f -
g_n_2 = 0 et que donc on
peut trouver un n tel que  2 \over n < \eta
avec \f -
g_n_2 < \epsilon.

Supposons maintenant que f est continue par morceaux. Sa restriction à
[0,2\pi~] est réglée et donc on peut trouver \phi en escalier sur
[0,2\pi~[ (et que l'on prolonge par périodicité) telle que
\f - \phi_\infty~
< \epsilon \over 2 . On a alors

\begin{align*} \f -
\phi_2^2& =& 1
\over 2\pi~ \int ~
_0^2\pi~f(t) - \phi(t)^2 dt \leq 1
\over 2\pi~ \int ~
_0^2\pi~\f -
\phi_ \infty~^2 dt\%&
\\ & =& \f -
\phi_\infty~^2 \%&
\\ \end{align*}

soit encore \f -
\phi_2 \leq\ f -
\phi_\infty~ < \epsilon
\over 2 . Mais d'autre part, comme \phi est en escalier,
on sait qu'on peut trouver g continue et affine par morceaux telle que
\\phi - g_2
< \epsilon \over 2 . On a alors
\f - g_2
\leq\ f - \phi_2
+\ \phi - g_2
< \epsilon ce qui démontre le lemme.

Théorème~14.3.14 (Parseval-Plancherel). Soit f : \mathbb{R}~ \rightarrow~ \mathbb{C} périodique de
période 2\pi~ et continue par morceaux. Alors

\begin{align*}
\f_2^2&
=& 1 \over 2\pi~ \int ~
_0^2\pi~f(t)^2 dt \%&
\\ & =&
c_0(f)^2 +
\sum _n=1^+\infty~(c_
n(f)^2 + c_
-n(f)^2) \%& \\ &
=& a_0(f)^2
\over 4 + 1 \over 2
\sum _n=1^+\infty~(a_
n(f)^2 + b_
n(f)^2)\%& \\
\end{align*}

Démonstration On sait que
c_0(f)^2
+ \\sum ~
_n=1^N(c_n(f)^2 +
c_-n(f)^2)
=\
S_N(f)_2^2. D'autre
part, à l'aide du théorème de Pythagore et puisque S_N(f) est
la projection orthogonale de f sur le sous-espace T_N des
polynômes trigonométriques de degré au plus N, on a
\f_2^2
=\
S_N(f)_2^2
+\ f -
S_N(f)_2^2. Le
résultat à démontrer est donc équivalent à
lim_N\rightarrow~+\infty~\S_N(f)_2^2~
=\
f_2^2, soit encore à
lim_N\rightarrow~+\infty~~\f -
S_N(f)_2 = 0.

Supposons tout d'abord que f est \mathcal{C}^1 par morceaux et
continue. On sait que la série de Fourier de f converge normalement,
donc uniformément vers f. On a donc
lim_N\rightarrow~+\infty~~\f -
S_N(f)_\infty~ = 0, mais comme ci
dessus, on a \f -
S_N(f)_2
\leq\ f -
S_N(f)_\infty~ ce qui montre que
lim_N\rightarrow~+\infty~~\f -
S_N(f)_2 = 0.

Si maintenant f est seulement continue par morceaux, soit \epsilon
> 0 et g : \mathbb{R}~ \rightarrow~ \mathbb{C} périodique de période 2\pi~, de classe
\mathcal{C}^1 par morceaux et continue telle que
\f - g_2
< \epsilon \over 2 . D'après le premier cas, on a
lim_N\rightarrow~+\infty~~\g -
S_N(g)_2 = 0 et donc il
existe N_0 \in \mathbb{N}~ tel que N ≥ N_0
\rigtharrow~\ g -
S_N(g)_2 < \epsilon
\over 2 . Mais comme S_N(g) \in T_N et
que S_N(f) est la projection orthogonale de f sur T_N,
on a \f -
S_N(f)_2
\leq\ f -
S_N(g)_2 soit encore, pour N
≥ N_0,

\begin{align*} \f -
S_N(f)_2& \leq&
\f -
S_N(g)_2
\leq\ f - g_2
+\ g -
S_N(g)_2\%&
\\ & <& \epsilon
\over 2 + \epsilon \over 2 = \epsilon \%&
\\ \end{align*}

ce qui démontre le résultat. La deuxième formule résulte d'un calcul
précédent qui montre que

\begin{align*}
c_0(f)^2& +&
\sum _n=1^+\infty~(c_
n(f)^2 + c_
-n(f)^2) \%& \\ &
=& a_0(f)^2
\over 4 + 1 \over 2
\sum _n=1^+\infty~(a_
n(f)^2 + b_
n(f)^2)\%& \\
\end{align*}

Corollaire~14.3.15 (injectivité de la transformation de Fourier). Soit f
et g deux fonctions de D telles que \forall~~n \in \mathbb{N}~,
c_n(f) = c_n(g). Alors f = g.

Démonstration On a \forall~n \in \mathbb{N}~, c_n~(f - g)
= 0, soit encore d'après le théorème de Parseval,
\f - g_2 =
0. Comme f - g appartient à D sur laquelle le produit scalaire est
défini positif, on a f - g = 0.

Remarque~14.3.6 Si on suppose seulement que f et g sont continues par
morceaux, on obtient seulement que f et g coïncident sauf en un nombre
fini de points (sur un intervalle de longueur 2\pi~).

[
[
[
[

\end{document}
