\documentclass[]{article}
\usepackage[T1]{fontenc}
\usepackage{lmodern}
\usepackage{amssymb,amsmath}
\usepackage{ifxetex,ifluatex}
\usepackage{fixltx2e} % provides \textsubscript
% use upquote if available, for straight quotes in verbatim environments
\IfFileExists{upquote.sty}{\usepackage{upquote}}{}
\ifnum 0\ifxetex 1\fi\ifluatex 1\fi=0 % if pdftex
  \usepackage[utf8]{inputenc}
\else % if luatex or xelatex
  \ifxetex
    \usepackage{mathspec}
    \usepackage{xltxtra,xunicode}
  \else
    \usepackage{fontspec}
  \fi
  \defaultfontfeatures{Mapping=tex-text,Scale=MatchLowercase}
  \newcommand{\euro}{€}
\fi
% use microtype if available
\IfFileExists{microtype.sty}{\usepackage{microtype}}{}
\ifxetex
  \usepackage[setpagesize=false, % page size defined by xetex
              unicode=false, % unicode breaks when used with xetex
              xetex]{hyperref}
\else
  \usepackage[unicode=true]{hyperref}
\fi
\hypersetup{breaklinks=true,
            bookmarks=true,
            pdfauthor={},
            pdftitle={Continuite},
            colorlinks=true,
            citecolor=blue,
            urlcolor=blue,
            linkcolor=magenta,
            pdfborder={0 0 0}}
\urlstyle{same}  % don't use monospace font for urls
\setlength{\parindent}{0pt}
\setlength{\parskip}{6pt plus 2pt minus 1pt}
\setlength{\emergencystretch}{3em}  % prevent overfull lines
\setcounter{secnumdepth}{0}
 
/* start css.sty */
.cmr-5{font-size:50%;}
.cmr-7{font-size:70%;}
.cmmi-5{font-size:50%;font-style: italic;}
.cmmi-7{font-size:70%;font-style: italic;}
.cmmi-10{font-style: italic;}
.cmsy-5{font-size:50%;}
.cmsy-7{font-size:70%;}
.cmex-7{font-size:70%;}
.cmex-7x-x-71{font-size:49%;}
.msbm-7{font-size:70%;}
.cmtt-10{font-family: monospace;}
.cmti-10{ font-style: italic;}
.cmbx-10{ font-weight: bold;}
.cmr-17x-x-120{font-size:204%;}
.cmsl-10{font-style: oblique;}
.cmti-7x-x-71{font-size:49%; font-style: italic;}
.cmbxti-10{ font-weight: bold; font-style: italic;}
p.noindent { text-indent: 0em }
td p.noindent { text-indent: 0em; margin-top:0em; }
p.nopar { text-indent: 0em; }
p.indent{ text-indent: 1.5em }
@media print {div.crosslinks {visibility:hidden;}}
a img { border-top: 0; border-left: 0; border-right: 0; }
center { margin-top:1em; margin-bottom:1em; }
td center { margin-top:0em; margin-bottom:0em; }
.Canvas { position:relative; }
li p.indent { text-indent: 0em }
.enumerate1 {list-style-type:decimal;}
.enumerate2 {list-style-type:lower-alpha;}
.enumerate3 {list-style-type:lower-roman;}
.enumerate4 {list-style-type:upper-alpha;}
div.newtheorem { margin-bottom: 2em; margin-top: 2em;}
.obeylines-h,.obeylines-v {white-space: nowrap; }
div.obeylines-v p { margin-top:0; margin-bottom:0; }
.overline{ text-decoration:overline; }
.overline img{ border-top: 1px solid black; }
td.displaylines {text-align:center; white-space:nowrap;}
.centerline {text-align:center;}
.rightline {text-align:right;}
div.verbatim {font-family: monospace; white-space: nowrap; text-align:left; clear:both; }
.fbox {padding-left:3.0pt; padding-right:3.0pt; text-indent:0pt; border:solid black 0.4pt; }
div.fbox {display:table}
div.center div.fbox {text-align:center; clear:both; padding-left:3.0pt; padding-right:3.0pt; text-indent:0pt; border:solid black 0.4pt; }
div.minipage{width:100%;}
div.center, div.center div.center {text-align: center; margin-left:1em; margin-right:1em;}
div.center div {text-align: left;}
div.flushright, div.flushright div.flushright {text-align: right;}
div.flushright div {text-align: left;}
div.flushleft {text-align: left;}
.underline{ text-decoration:underline; }
.underline img{ border-bottom: 1px solid black; margin-bottom:1pt; }
.framebox-c, .framebox-l, .framebox-r { padding-left:3.0pt; padding-right:3.0pt; text-indent:0pt; border:solid black 0.4pt; }
.framebox-c {text-align:center;}
.framebox-l {text-align:left;}
.framebox-r {text-align:right;}
span.thank-mark{ vertical-align: super }
span.footnote-mark sup.textsuperscript, span.footnote-mark a sup.textsuperscript{ font-size:80%; }
div.tabular, div.center div.tabular {text-align: center; margin-top:0.5em; margin-bottom:0.5em; }
table.tabular td p{margin-top:0em;}
table.tabular {margin-left: auto; margin-right: auto;}
div.td00{ margin-left:0pt; margin-right:0pt; }
div.td01{ margin-left:0pt; margin-right:5pt; }
div.td10{ margin-left:5pt; margin-right:0pt; }
div.td11{ margin-left:5pt; margin-right:5pt; }
table[rules] {border-left:solid black 0.4pt; border-right:solid black 0.4pt; }
td.td00{ padding-left:0pt; padding-right:0pt; }
td.td01{ padding-left:0pt; padding-right:5pt; }
td.td10{ padding-left:5pt; padding-right:0pt; }
td.td11{ padding-left:5pt; padding-right:5pt; }
table[rules] {border-left:solid black 0.4pt; border-right:solid black 0.4pt; }
.hline hr, .cline hr{ height : 1px; margin:0px; }
.tabbing-right {text-align:right;}
span.TEX {letter-spacing: -0.125em; }
span.TEX span.E{ position:relative;top:0.5ex;left:-0.0417em;}
a span.TEX span.E {text-decoration: none; }
span.LATEX span.A{ position:relative; top:-0.5ex; left:-0.4em; font-size:85%;}
span.LATEX span.TEX{ position:relative; left: -0.4em; }
div.float img, div.float .caption {text-align:center;}
div.figure img, div.figure .caption {text-align:center;}
.marginpar {width:20%; float:right; text-align:left; margin-left:auto; margin-top:0.5em; font-size:85%; text-decoration:underline;}
.marginpar p{margin-top:0.4em; margin-bottom:0.4em;}
.equation td{text-align:center; vertical-align:middle; }
td.eq-no{ width:5%; }
table.equation { width:100%; } 
div.math-display, div.par-math-display{text-align:center;}
math .texttt { font-family: monospace; }
math .textit { font-style: italic; }
math .textsl { font-style: oblique; }
math .textsf { font-family: sans-serif; }
math .textbf { font-weight: bold; }
.partToc a, .partToc, .likepartToc a, .likepartToc {line-height: 200%; font-weight:bold; font-size:110%;}
.chapterToc a, .chapterToc, .likechapterToc a, .likechapterToc, .appendixToc a, .appendixToc {line-height: 200%; font-weight:bold;}
.index-item, .index-subitem, .index-subsubitem {display:block}
.caption td.id{font-weight: bold; white-space: nowrap; }
table.caption {text-align:center;}
h1.partHead{text-align: center}
p.bibitem { text-indent: -2em; margin-left: 2em; margin-top:0.6em; margin-bottom:0.6em; }
p.bibitem-p { text-indent: 0em; margin-left: 2em; margin-top:0.6em; margin-bottom:0.6em; }
.subsectionHead, .likesubsectionHead { margin-top:2em; font-weight: bold;}
.sectionHead, .likesectionHead { font-weight: bold;}
.quote {margin-bottom:0.25em; margin-top:0.25em; margin-left:1em; margin-right:1em; text-align:justify;}
.verse{white-space:nowrap; margin-left:2em}
div.maketitle {text-align:center;}
h2.titleHead{text-align:center;}
div.maketitle{ margin-bottom: 2em; }
div.author, div.date {text-align:center;}
div.thanks{text-align:left; margin-left:10%; font-size:85%; font-style:italic; }
div.author{white-space: nowrap;}
.quotation {margin-bottom:0.25em; margin-top:0.25em; margin-left:1em; }
h1.partHead{text-align: center}
.sectionToc, .likesectionToc {margin-left:2em;}
.subsectionToc, .likesubsectionToc {margin-left:4em;}
.sectionToc, .likesectionToc {margin-left:6em;}
.frenchb-nbsp{font-size:75%;}
.frenchb-thinspace{font-size:75%;}
.figure img.graphics {margin-left:10%;}
/* end css.sty */

\title{Continuite}
\author{}
\date{}

\begin{document}
\maketitle

\textbf{Warning: 
requires JavaScript to process the mathematics on this page.\\ If your
browser supports JavaScript, be sure it is enabled.}

\begin{center}\rule{3in}{0.4pt}\end{center}

[
[
[]
[

\section{4.5 Continuité}

\subsection{4.5.1 Continuité en un point}

Définition~4.5.1 Soit f une fonction de E vers F et a
\in Def~ (f). On dit que f est continue au point
a si elle vérifie les conditions équivalentes

\begin{itemize}
\itemsep1pt\parskip0pt\parsep0pt
\item
  (i) lim_x\rightarrow~a~f(x) = f(a)
\item
  (ii) \forall~~V \in V (f(a)),
  \exists~U \in V (a), f(U) \subset~ V
\item
  (iii) \forall~~\epsilon > 0,
  \exists~\eta > 0, d(x,a) < \eta \rigtharrow~
  d(f(x),f(a)) < \epsilon
\end{itemize}

Remarque~4.5.1 On a bien entendu toutes les propriétés des limites, en
particulier

Proposition~4.5.1 La continuité est une notion locale~: si U_0
est un ouvert contenant a, f est continue au point a si et seulement
si~f_U_0 est continue au point a.

Proposition~4.5.2 Si f est une fonction de E vers F_1
\times⋯ \times F_k, f =
(f_1,\\ldots,f_k~),
alors f est continue au point a si et seulement si~chacune des
f_i est continue au point a.

Proposition~4.5.3 Si f est continue au point a et g continue au point
f(a), alors g \cdot f est continue au point a (on suppose que
\mathrmIm~f
\subset~ Def~ (g)).

Théorème~4.5.4 f est continue au point a si et seulement si~pour toute
suite (a_n) de Def~ (f) de limite a,
la suite (f(a_n)) admet f(a) pour limite.

\subsection{4.5.2 Continuité sur un espace}

Définition~4.5.2 Soit E et F deux espaces métriques. On dit que f : E \rightarrow~
F est continue si elle est continue en tout point de E.

Remarque~4.5.2 On a donc toutes les propriétés des limites et des
fonctions continues~; en particulier, la composée de deux applications
continues est continue.

Théorème~4.5.5 Soit E et F deux espaces métriques et f : E \rightarrow~ F. On a
équivalence de

\begin{itemize}
\itemsep1pt\parskip0pt\parsep0pt
\item
  (i) f est continue (sur E)
\item
  (ii) pour tout ouvert V de F, f^-1(V ) est un ouvert de E
\item
  (iii) pour tout fermé K de F, f^-1(K) est un fermé de E
\end{itemize}

Démonstration (ii) et (iii) sont équivalents puisque pour toute partie B
de F on a f^-1(cB) = cf^-1(B).

((i) \rigtharrow~(ii)) Supposons que f est continue et soit V un ouvert de F et a \in
f^-1(V ). On a f(a) \in V et V est un voisinage de f(a). Donc
il existe U \in V (a) tel que f(U) \subset~ V , soit U \subset~ f^-1(V ) et
donc f^-1(V ) est un voisinage de a. Puisque
f^-1(V ) est voisinage de tous ses points il est ouvert.

(ii) \rigtharrow~ (i)Inversement, supposons que l'image réciproque de tout ouvert
est un ouvert et soit a \in E et V \in V (f(a)). Il existe V _0
ouvert tel que f(a) \in V _0 \subset~ V . Alors U_0 =
f^-1(V _0) est un ouvert contenant a et on a
f(U_0) \subset~ V _0 \subset~ V . Donc f est continue en a.

Théorème~4.5.6 Soit E et F deux espaces métriques, f,g : E \rightarrow~ F deux
applications continues. Alors \x \in
E∣f(x) = g(x)\ est fermé
dans E.

Démonstration Soit \phi : E \rightarrow~ F \times F,
x\mapsto~(f(x),g(x)). Comme f et g sont continues, \phi
est continue. Or \x \in E∣f(x)
= g(x)\ = \phi^-1(\Delta) où \Delta =
\(y,y)∣y \in
F\. Or on sait, d'après la propriété de séparation que
\Delta est un fermé de F \times F. Son image réciproque par \phi est donc un fermé de
E.

Corollaire~4.5.7 Soit E et F deux espaces métriques, f,g : E \rightarrow~ F deux
applications continues. On suppose qu'il existe une partie A de E, dense
dans E telle que \forall~~x \in A, f(x) = g(x). Alors f
= g.

Démonstration \x \in E∣f(x) =
g(x)\ est un fermé contenant A donc
\overlineA = E~; donc f = g.

\subsection{4.5.3 Homéomorphismes}

Définition~4.5.3 Soit E et F deux espaces métriques. on dit que f : E \rightarrow~
F est un homéomorphisme si f est bijective et si f et f^-1
sont continues.

Remarque~4.5.3 Deux espaces homéomorphes ont exactement les mêmes
propriétés topologiques (toutes celles qui peuvent s'exprimer sans faire
intervenir de distances, uniquement en termes d'ouverts, de fermés et de
voisinages).

Exemple~4.5.1 Soit f : [0,2\pi~[\rightarrow~ U = \z \in
\mathbb{C}∣z =
1\, t\mapsto~e^it. Alors
f est continue bijective, mais sa réciproque n'est pas continue au point
1 (faire tendre z vers 1 par parties imaginaires négatives,
f^-1(z) tend vers 2\pi~\neq~0 =
f^-1(1)).

[
[
[
[

\end{document}
