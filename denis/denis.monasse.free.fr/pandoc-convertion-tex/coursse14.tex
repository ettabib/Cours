\documentclass[]{article}
\usepackage[T1]{fontenc}
\usepackage{lmodern}
\usepackage{amssymb,amsmath}
\usepackage{ifxetex,ifluatex}
\usepackage{fixltx2e} % provides \textsubscript
% use upquote if available, for straight quotes in verbatim environments
\IfFileExists{upquote.sty}{\usepackage{upquote}}{}
\ifnum 0\ifxetex 1\fi\ifluatex 1\fi=0 % if pdftex
  \usepackage[utf8]{inputenc}
\else % if luatex or xelatex
  \ifxetex
    \usepackage{mathspec}
    \usepackage{xltxtra,xunicode}
  \else
    \usepackage{fontspec}
  \fi
  \defaultfontfeatures{Mapping=tex-text,Scale=MatchLowercase}
  \newcommand{\euro}{€}
\fi
% use microtype if available
\IfFileExists{microtype.sty}{\usepackage{microtype}}{}
\ifxetex
  \usepackage[setpagesize=false, % page size defined by xetex
              unicode=false, % unicode breaks when used with xetex
              xetex]{hyperref}
\else
  \usepackage[unicode=true]{hyperref}
\fi
\hypersetup{breaklinks=true,
            bookmarks=true,
            pdfauthor={},
            pdftitle={Syst`emes lineaires},
            colorlinks=true,
            citecolor=blue,
            urlcolor=blue,
            linkcolor=magenta,
            pdfborder={0 0 0}}
\urlstyle{same}  % don't use monospace font for urls
\setlength{\parindent}{0pt}
\setlength{\parskip}{6pt plus 2pt minus 1pt}
\setlength{\emergencystretch}{3em}  % prevent overfull lines
\setcounter{secnumdepth}{0}
 
/* start css.sty */
.cmr-5{font-size:50%;}
.cmr-7{font-size:70%;}
.cmmi-5{font-size:50%;font-style: italic;}
.cmmi-7{font-size:70%;font-style: italic;}
.cmmi-10{font-style: italic;}
.cmsy-5{font-size:50%;}
.cmsy-7{font-size:70%;}
.cmex-7{font-size:70%;}
.cmex-7x-x-71{font-size:49%;}
.msbm-7{font-size:70%;}
.cmtt-10{font-family: monospace;}
.cmti-10{ font-style: italic;}
.cmbx-10{ font-weight: bold;}
.cmr-17x-x-120{font-size:204%;}
.cmsl-10{font-style: oblique;}
.cmti-7x-x-71{font-size:49%; font-style: italic;}
.cmbxti-10{ font-weight: bold; font-style: italic;}
p.noindent { text-indent: 0em }
td p.noindent { text-indent: 0em; margin-top:0em; }
p.nopar { text-indent: 0em; }
p.indent{ text-indent: 1.5em }
@media print {div.crosslinks {visibility:hidden;}}
a img { border-top: 0; border-left: 0; border-right: 0; }
center { margin-top:1em; margin-bottom:1em; }
td center { margin-top:0em; margin-bottom:0em; }
.Canvas { position:relative; }
li p.indent { text-indent: 0em }
.enumerate1 {list-style-type:decimal;}
.enumerate2 {list-style-type:lower-alpha;}
.enumerate3 {list-style-type:lower-roman;}
.enumerate4 {list-style-type:upper-alpha;}
div.newtheorem { margin-bottom: 2em; margin-top: 2em;}
.obeylines-h,.obeylines-v {white-space: nowrap; }
div.obeylines-v p { margin-top:0; margin-bottom:0; }
.overline{ text-decoration:overline; }
.overline img{ border-top: 1px solid black; }
td.displaylines {text-align:center; white-space:nowrap;}
.centerline {text-align:center;}
.rightline {text-align:right;}
div.verbatim {font-family: monospace; white-space: nowrap; text-align:left; clear:both; }
.fbox {padding-left:3.0pt; padding-right:3.0pt; text-indent:0pt; border:solid black 0.4pt; }
div.fbox {display:table}
div.center div.fbox {text-align:center; clear:both; padding-left:3.0pt; padding-right:3.0pt; text-indent:0pt; border:solid black 0.4pt; }
div.minipage{width:100%;}
div.center, div.center div.center {text-align: center; margin-left:1em; margin-right:1em;}
div.center div {text-align: left;}
div.flushright, div.flushright div.flushright {text-align: right;}
div.flushright div {text-align: left;}
div.flushleft {text-align: left;}
.underline{ text-decoration:underline; }
.underline img{ border-bottom: 1px solid black; margin-bottom:1pt; }
.framebox-c, .framebox-l, .framebox-r { padding-left:3.0pt; padding-right:3.0pt; text-indent:0pt; border:solid black 0.4pt; }
.framebox-c {text-align:center;}
.framebox-l {text-align:left;}
.framebox-r {text-align:right;}
span.thank-mark{ vertical-align: super }
span.footnote-mark sup.textsuperscript, span.footnote-mark a sup.textsuperscript{ font-size:80%; }
div.tabular, div.center div.tabular {text-align: center; margin-top:0.5em; margin-bottom:0.5em; }
table.tabular td p{margin-top:0em;}
table.tabular {margin-left: auto; margin-right: auto;}
div.td00{ margin-left:0pt; margin-right:0pt; }
div.td01{ margin-left:0pt; margin-right:5pt; }
div.td10{ margin-left:5pt; margin-right:0pt; }
div.td11{ margin-left:5pt; margin-right:5pt; }
table[rules] {border-left:solid black 0.4pt; border-right:solid black 0.4pt; }
td.td00{ padding-left:0pt; padding-right:0pt; }
td.td01{ padding-left:0pt; padding-right:5pt; }
td.td10{ padding-left:5pt; padding-right:0pt; }
td.td11{ padding-left:5pt; padding-right:5pt; }
table[rules] {border-left:solid black 0.4pt; border-right:solid black 0.4pt; }
.hline hr, .cline hr{ height : 1px; margin:0px; }
.tabbing-right {text-align:right;}
span.TEX {letter-spacing: -0.125em; }
span.TEX span.E{ position:relative;top:0.5ex;left:-0.0417em;}
a span.TEX span.E {text-decoration: none; }
span.LATEX span.A{ position:relative; top:-0.5ex; left:-0.4em; font-size:85%;}
span.LATEX span.TEX{ position:relative; left: -0.4em; }
div.float img, div.float .caption {text-align:center;}
div.figure img, div.figure .caption {text-align:center;}
.marginpar {width:20%; float:right; text-align:left; margin-left:auto; margin-top:0.5em; font-size:85%; text-decoration:underline;}
.marginpar p{margin-top:0.4em; margin-bottom:0.4em;}
.equation td{text-align:center; vertical-align:middle; }
td.eq-no{ width:5%; }
table.equation { width:100%; } 
div.math-display, div.par-math-display{text-align:center;}
math .texttt { font-family: monospace; }
math .textit { font-style: italic; }
math .textsl { font-style: oblique; }
math .textsf { font-family: sans-serif; }
math .textbf { font-weight: bold; }
.partToc a, .partToc, .likepartToc a, .likepartToc {line-height: 200%; font-weight:bold; font-size:110%;}
.chapterToc a, .chapterToc, .likechapterToc a, .likechapterToc, .appendixToc a, .appendixToc {line-height: 200%; font-weight:bold;}
.index-item, .index-subitem, .index-subsubitem {display:block}
.caption td.id{font-weight: bold; white-space: nowrap; }
table.caption {text-align:center;}
h1.partHead{text-align: center}
p.bibitem { text-indent: -2em; margin-left: 2em; margin-top:0.6em; margin-bottom:0.6em; }
p.bibitem-p { text-indent: 0em; margin-left: 2em; margin-top:0.6em; margin-bottom:0.6em; }
.subsectionHead, .likesubsectionHead { margin-top:2em; font-weight: bold;}
.sectionHead, .likesectionHead { font-weight: bold;}
.quote {margin-bottom:0.25em; margin-top:0.25em; margin-left:1em; margin-right:1em; text-align:justify;}
.verse{white-space:nowrap; margin-left:2em}
div.maketitle {text-align:center;}
h2.titleHead{text-align:center;}
div.maketitle{ margin-bottom: 2em; }
div.author, div.date {text-align:center;}
div.thanks{text-align:left; margin-left:10%; font-size:85%; font-style:italic; }
div.author{white-space: nowrap;}
.quotation {margin-bottom:0.25em; margin-top:0.25em; margin-left:1em; }
h1.partHead{text-align: center}
.sectionToc, .likesectionToc {margin-left:2em;}
.subsectionToc, .likesubsectionToc {margin-left:4em;}
.sectionToc, .likesectionToc {margin-left:6em;}
.frenchb-nbsp{font-size:75%;}
.frenchb-thinspace{font-size:75%;}
.figure img.graphics {margin-left:10%;}
/* end css.sty */

\title{Syst`emes lineaires}
\author{}
\date{}

\begin{document}
\maketitle

\textbf{Warning: 
requires JavaScript to process the mathematics on this page.\\ If your
browser supports JavaScript, be sure it is enabled.}

\begin{center}\rule{3in}{0.4pt}\end{center}

[
[
[]
[

\section{2.8 Systèmes linéaires}

\subsection{2.8.1 Position du problème}

Soit A = (a_i,j)_1\leqi\leqm,1\leqj\leqn \in M_K(m,n) et
b_1,\\ldots,b_m~
\in K. On considère le système d'équations aux inconnues
x_1,\\ldots,x_n~
\in K

(L)\quad \left
\\matrix\,a_1,1x_1
+ \\ldots~ +
a_1,nx_n = b_1 \cr
\\ldots~
\cr a_m,1x_1 +
\\ldots~ +
a_m,nx_n = b_m\right .

On a diverses interprétations possibles d'un tel système (en notant
\textCan_n la base canonique de
K^n)~:

\begin{itemize}
\itemsep1pt\parskip0pt\parsep0pt
\item
  Interprétation matricielle~: AX = B si X = \left
  (\matrix\,x_1
  \cr
  \\ldots~
  \cr x_n\right ) et B =
  \left
  (\matrix\,b_1
  \cr
  \\ldots~
  \cr b_m\right ).
\item
  Interprétation linéaire~: u(x) = b si u : K^n \rightarrow~
  K^m est telle que
  \mathrmMat~
  (u,\textCan_n,\textCan_m)
  et si x et b admettent
  (x_1,\\ldots,x_n~)
  et
  (b_1,\\ldots,b_m~)
  comme coordonnées dans les bases canoniques
  \textCan_n et
  \textCan_m.
\item
  Interprétation vectorielle~: b = x_1c_1 +
  \\ldots~ +
  x_nc_n, si
  c_1,\\ldots,c_n~
  \in K^m sont les vecteurs colonnes de la matrice A
\item
  Interprétation duale~: f_1(x) =
  b_1,\\ldots,f_m~(x)
  = b_m si f_j désigne la forme linéaire sur
  K^n,
  (x_1,\\ldots,x_n)\mapsto~a_j,1x_1~
  + \\ldots~ +
  a_j,nx_n.
\end{itemize}

Remarque~2.8.1 En remarquant que A est la matrice de u dans les bases
canoniques, mais aussi la matrice des coordonnées de
(c_1,\\ldots,c_n~)
dans la base canonique de K^m et la transposée de la matrice
de
(f_1,\\ldots,f_m~)
dans la base duale de la base canonique de K^n, on obtient

Théorème~2.8.1 Avec les notations ci-dessus, on a
\mathrmrg~A
= \mathrmrg~u
=\
\mathrmrg(c_1,\\ldots,c_n~)
=\
\mathrmrg(f_1,\\ldots,f_m~).
Cette valeur commune est appelée le rang du système.

\subsection{2.8.2 Systèmes de Cramer}

Définition~2.8.1 On appelle système de Cramer un système d'équations
linéaires vérifiant les conditions équivalentes (qui toutes impliquent
que m = n)

\begin{itemize}
\itemsep1pt\parskip0pt\parsep0pt
\item
  (i) A est une matrice inversible
\item
  (ii) u est un isomorphisme de K^n sur K^m
\item
  (iii)
  (c_1,\\ldots,c_n~)
  est une base de K^m
\item
  (iv)
  (f_1,\\ldots,f_m~)
  est une base de (K^n)^∗
\end{itemize}

Théorème~2.8.2 (Résolution d'un système de Cramer) Un système de Cramer
admet une solution unique donnée par

\begin{itemize}
\item
  (i) Interprétation matricielle~: X = A^-1B
\item
  (ii) Interprétation linéaire~: x = u^-1(b)
\item
  (iii) Interprétation vectorielle~:

  x_j =
  \mathrm{det}~
  (c_1,\\ldots,c_j-1,b,c_j+1,\\\ldots,c_n~)
  \over
  \mathrm{det}~
  (c_1,\\ldots,c_n)~

  (formules de Cramer)
\item
  (iv) Soit
  (v_1,\\ldots,v_n~)
  la base duale de
  (f_1,\\ldots,f_n~),
  alors x = b_1v_1 +
  \\ldots~ +
  b_nv_n.
\end{itemize}

Démonstration Tout est évident sauf le (iii). Mais on a b =
x_1c_1 +
\\ldots~ +
x_nc_n, soit

\begin{align*}
\mathrm{det}~
(c_1,\\ldots,c_j-1,b,c_j+1,\\\ldots,c_n~)&&
\%& \\ & =&
\mathrm{det}~
(c_1,\\ldots,c_j-1,x_1c_1~
+ \\ldots~ +
x_nc_n,c_j+1,\\ldots,c_n~)\%&
\\ & =&
x_j \mathrm{det}~
(c_1,\\ldots,c_n~)
\%& \\ \end{align*}

en développant suivant la j-ième colonne (tous les autres déterminants
sont nuls car contenant deux fois la même colonne c_i).

\subsection{2.8.3 Théorème de Rouché-Fontené}

Définition~2.8.2 On associe au système (L) le système dit homogène

(H)\quad \left
\\matrix\,a_1,1x_1
+ \\ldots~ +
a_1,nx_n = 0 \cr
\\ldots~
\cr a_m,1x_1 +
\\ldots~ +
a_m,nx_n = 0\right .

On appelle S_L l'ensemble des solutions du système linéaire,
S_H celui du système homogène.

Proposition~2.8.3 S_H est un sous-espace vectoriel de
K^n de dimension égale à n
-\mathrmrgL. S_L~
est soit vide, soit un sous-espace affine de direction S_H~;
dans ce cas on obtient la solution générale de (L) en ajoutant à une
solution particulière, la solution générale de (H).

Démonstration On a S_H =\
\mathrmKeru d'où le fait qu'il est un sous-espace
vectoriel~; sa dimension est donnée par le théorème du rang. De plus, si
v \in S_L, on a x \in S_L \Leftrightarrow
u(x) = b = u(v) \Leftrightarrow u(x - v) = 0
\Leftrightarrow x - v \in S_H.

Définition~2.8.3 Soit P = (a_i,j)_i\inI,j\inJ une
sous-matrice principale de la matrice du système (L). On appelle
déterminants caractéristiques du système associés à la sous-matrice
principale P les m - r déterminants

\Delta_i_0 =\
\mathrm{det} \left
(\matrix\,(a_i,j)_i\inI,j\inJ&(b_i)_i\inI
\cr (a_i_0,j)_j\inJ
&b_i_0 \right )

avec i_0 \in [1,m] \diagdown I

Théorème~2.8.4 (Rouché-Fontené). Soit P une sous-matrice principale de
la matrice du système (L). Alors le système a des solutions si et
seulement si les m - r déterminants caractéristiques associés à P sont
nuls. Dans ce cas, l'ensemble des solutions de (L) est le même que
l'ensemble des solutions du système L' obtenu en éliminant les équations
non principales. On résout ce système (L'), en donnant des valeurs
arbitraires aux inconnues non principales et en déterminant les valeurs
correspondantes des inconnues principales par la résolution du système
de Cramer (de matrice P) ainsi obtenu.

Démonstration Etudions tout d'abord la compatibilité du système. On a

\begin{align*}
S_L\neq~\varnothing~& \mathrel\Leftrightarrow
& b
\in\mathrmVect(c_1,\\\ldots,c_n~)
\%& \\ & \Leftrightarrow &
dim~
\mathrmVect(c_1,\\\ldots,c_n~,b)
= dim~
\mathrmVect(c_1,\\\ldots,c_n~)\%&
\\ & \Leftrightarrow &
\mathrmrg~\left
(\matrix\,A&B\right )
= \mathrmrg~A \%&
\\ \end{align*}

Soit donc P = (a_i,j)_i\inI,j\inJ une sous-matrice
principale de A (avec I = J =
r). Le système a des solutions si et seulement si P est encore une
sous-matrice principale de \left
(\matrix\,A&B\right ),
c'est-à-dire si et seulement si toutes les sous matrices bordantes de P
dans \left
(\matrix\,A&B\right )
sont non inversibles~; mais ces matrices bordantes sont de deux types~:
soit des matrices bordantes dans A qui sont forcément non inversibles,
soit des matrices de la forme \left
(\matrix\,(a_i,j)_i\inI,j\inJ&(b_i)_i\inI
\cr (a_i_0,j)_j\inJ
&b_i_0 \right ), avec i_0 \in
[1,m] \diagdown I. On a noté \Delta_i_0 le déterminant d'une
telle matrice~: les \Delta_i_0, i \in [1,m] \diagdown I sont les
déterminants caractéristiques du système (il y en a m - r). Le système a
des solutions si et seulement si ces déterminants caractéristiques sont
tous nuls.

Supposons la condition réalisée, soit (L') le système
\left
\\matrix\,a_i,1x_1
+ \\ldots~ +
a_i,nx_n = b_i\right
.,\quad i \in I (système d'équations principales associées
à P). On a clairement S_L \subset~ S_L'. Mais les deux
systèmes ont même rang, si bien que dim~
S_L = dim S_L'~. On a donc
S_L = S_L'. Or le système L' se résout facilement en
l'écrivant sous la forme

\sum _j\inJa_i,jx_j~ =
b_i -\\sum
_j∉Ja_i,jx_j,\quad
i \in I

On donne des valeurs arbitraires aux inconnues x_j,
j∉J (inconnues non principales associées à la
matrice P)~; on obtient alors un système de Cramer de matrice P qui
permet de déterminer les x_j, j \in J (les inconnues principales
associées à P).

\subsection{2.8.4 Méthode du pivot}

Appliquons la méthode du pivot sur les lignes de la matrice A en
effectuant les opérations correspondantes sur la matrice B. On obtient
un nouveau système du type

\left
\\matrix\,\alpha_1,i_1x_i_1
+ \alpha_1,i_1+1x_i_1+1 +
\quad \quad
\\ldots~\quad
\quad + \alpha_1,nx_n = \beta_1
\cr \alpha_2,i_2x_i_2 +
\alpha_2,i_2+1x_i_2+1 +
\quad
\\ldots~\quad
+ \alpha_2,nx_n = \beta_2 \cr
\\ldots~
\cr \alpha_r,i_rx_i_r +
\alpha_1,i_r+1x_i_r+1 +
\\ldots~ +
\alpha_r,nx_n = \beta_r \cr 0 =
\beta_r+1 \cr
\\ldots~
\cr 0 = \beta_m\right .

avec i_1 < i_2 <
\\ldots~ <
i_r, \alpha_k,i_k\neq~0
pour k \in [1,r]

Un tel système se résout immédiatement~: il a des solutions si et
seulement si \beta_r+1 =
\\ldots~ =
\beta_m = 0. Dans ce cas il est équivalent au système formé des r
premières équations. Celui-ci se résout en donnant des valeurs
arbitraires aux x_j pour
j∉\i_1,\\ldots,i_r\~
et en calculant les x_i_k par résolution d'un système
triangulaire supérieur.

Application~: recherche de l'inverse d'une matrice~: AX = Y
\Leftrightarrow X = A^-1Y .

[
[
[
[

\end{document}
