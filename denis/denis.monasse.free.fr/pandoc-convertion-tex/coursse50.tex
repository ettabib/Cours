\documentclass[]{article}
\usepackage[T1]{fontenc}
\usepackage{lmodern}
\usepackage{amssymb,amsmath}
\usepackage{ifxetex,ifluatex}
\usepackage{fixltx2e} % provides \textsubscript
% use upquote if available, for straight quotes in verbatim environments
\IfFileExists{upquote.sty}{\usepackage{upquote}}{}
\ifnum 0\ifxetex 1\fi\ifluatex 1\fi=0 % if pdftex
  \usepackage[utf8]{inputenc}
\else % if luatex or xelatex
  \ifxetex
    \usepackage{mathspec}
    \usepackage{xltxtra,xunicode}
  \else
    \usepackage{fontspec}
  \fi
  \defaultfontfeatures{Mapping=tex-text,Scale=MatchLowercase}
  \newcommand{\euro}{€}
\fi
% use microtype if available
\IfFileExists{microtype.sty}{\usepackage{microtype}}{}
\ifxetex
  \usepackage[setpagesize=false, % page size defined by xetex
              unicode=false, % unicode breaks when used with xetex
              xetex]{hyperref}
\else
  \usepackage[unicode=true]{hyperref}
\fi
\hypersetup{breaklinks=true,
            bookmarks=true,
            pdfauthor={},
            pdftitle={Subdivisions, approximation des fonctions},
            colorlinks=true,
            citecolor=blue,
            urlcolor=blue,
            linkcolor=magenta,
            pdfborder={0 0 0}}
\urlstyle{same}  % don't use monospace font for urls
\setlength{\parindent}{0pt}
\setlength{\parskip}{6pt plus 2pt minus 1pt}
\setlength{\emergencystretch}{3em}  % prevent overfull lines
\setcounter{secnumdepth}{0}
 
/* start css.sty */
.cmr-5{font-size:50%;}
.cmr-7{font-size:70%;}
.cmmi-5{font-size:50%;font-style: italic;}
.cmmi-7{font-size:70%;font-style: italic;}
.cmmi-10{font-style: italic;}
.cmsy-5{font-size:50%;}
.cmsy-7{font-size:70%;}
.cmex-7{font-size:70%;}
.cmex-7x-x-71{font-size:49%;}
.msbm-7{font-size:70%;}
.cmtt-10{font-family: monospace;}
.cmti-10{ font-style: italic;}
.cmbx-10{ font-weight: bold;}
.cmr-17x-x-120{font-size:204%;}
.cmsl-10{font-style: oblique;}
.cmti-7x-x-71{font-size:49%; font-style: italic;}
.cmbxti-10{ font-weight: bold; font-style: italic;}
p.noindent { text-indent: 0em }
td p.noindent { text-indent: 0em; margin-top:0em; }
p.nopar { text-indent: 0em; }
p.indent{ text-indent: 1.5em }
@media print {div.crosslinks {visibility:hidden;}}
a img { border-top: 0; border-left: 0; border-right: 0; }
center { margin-top:1em; margin-bottom:1em; }
td center { margin-top:0em; margin-bottom:0em; }
.Canvas { position:relative; }
li p.indent { text-indent: 0em }
.enumerate1 {list-style-type:decimal;}
.enumerate2 {list-style-type:lower-alpha;}
.enumerate3 {list-style-type:lower-roman;}
.enumerate4 {list-style-type:upper-alpha;}
div.newtheorem { margin-bottom: 2em; margin-top: 2em;}
.obeylines-h,.obeylines-v {white-space: nowrap; }
div.obeylines-v p { margin-top:0; margin-bottom:0; }
.overline{ text-decoration:overline; }
.overline img{ border-top: 1px solid black; }
td.displaylines {text-align:center; white-space:nowrap;}
.centerline {text-align:center;}
.rightline {text-align:right;}
div.verbatim {font-family: monospace; white-space: nowrap; text-align:left; clear:both; }
.fbox {padding-left:3.0pt; padding-right:3.0pt; text-indent:0pt; border:solid black 0.4pt; }
div.fbox {display:table}
div.center div.fbox {text-align:center; clear:both; padding-left:3.0pt; padding-right:3.0pt; text-indent:0pt; border:solid black 0.4pt; }
div.minipage{width:100%;}
div.center, div.center div.center {text-align: center; margin-left:1em; margin-right:1em;}
div.center div {text-align: left;}
div.flushright, div.flushright div.flushright {text-align: right;}
div.flushright div {text-align: left;}
div.flushleft {text-align: left;}
.underline{ text-decoration:underline; }
.underline img{ border-bottom: 1px solid black; margin-bottom:1pt; }
.framebox-c, .framebox-l, .framebox-r { padding-left:3.0pt; padding-right:3.0pt; text-indent:0pt; border:solid black 0.4pt; }
.framebox-c {text-align:center;}
.framebox-l {text-align:left;}
.framebox-r {text-align:right;}
span.thank-mark{ vertical-align: super }
span.footnote-mark sup.textsuperscript, span.footnote-mark a sup.textsuperscript{ font-size:80%; }
div.tabular, div.center div.tabular {text-align: center; margin-top:0.5em; margin-bottom:0.5em; }
table.tabular td p{margin-top:0em;}
table.tabular {margin-left: auto; margin-right: auto;}
div.td00{ margin-left:0pt; margin-right:0pt; }
div.td01{ margin-left:0pt; margin-right:5pt; }
div.td10{ margin-left:5pt; margin-right:0pt; }
div.td11{ margin-left:5pt; margin-right:5pt; }
table[rules] {border-left:solid black 0.4pt; border-right:solid black 0.4pt; }
td.td00{ padding-left:0pt; padding-right:0pt; }
td.td01{ padding-left:0pt; padding-right:5pt; }
td.td10{ padding-left:5pt; padding-right:0pt; }
td.td11{ padding-left:5pt; padding-right:5pt; }
table[rules] {border-left:solid black 0.4pt; border-right:solid black 0.4pt; }
.hline hr, .cline hr{ height : 1px; margin:0px; }
.tabbing-right {text-align:right;}
span.TEX {letter-spacing: -0.125em; }
span.TEX span.E{ position:relative;top:0.5ex;left:-0.0417em;}
a span.TEX span.E {text-decoration: none; }
span.LATEX span.A{ position:relative; top:-0.5ex; left:-0.4em; font-size:85%;}
span.LATEX span.TEX{ position:relative; left: -0.4em; }
div.float img, div.float .caption {text-align:center;}
div.figure img, div.figure .caption {text-align:center;}
.marginpar {width:20%; float:right; text-align:left; margin-left:auto; margin-top:0.5em; font-size:85%; text-decoration:underline;}
.marginpar p{margin-top:0.4em; margin-bottom:0.4em;}
.equation td{text-align:center; vertical-align:middle; }
td.eq-no{ width:5%; }
table.equation { width:100%; } 
div.math-display, div.par-math-display{text-align:center;}
math .texttt { font-family: monospace; }
math .textit { font-style: italic; }
math .textsl { font-style: oblique; }
math .textsf { font-family: sans-serif; }
math .textbf { font-weight: bold; }
.partToc a, .partToc, .likepartToc a, .likepartToc {line-height: 200%; font-weight:bold; font-size:110%;}
.chapterToc a, .chapterToc, .likechapterToc a, .likechapterToc, .appendixToc a, .appendixToc {line-height: 200%; font-weight:bold;}
.index-item, .index-subitem, .index-subsubitem {display:block}
.caption td.id{font-weight: bold; white-space: nowrap; }
table.caption {text-align:center;}
h1.partHead{text-align: center}
p.bibitem { text-indent: -2em; margin-left: 2em; margin-top:0.6em; margin-bottom:0.6em; }
p.bibitem-p { text-indent: 0em; margin-left: 2em; margin-top:0.6em; margin-bottom:0.6em; }
.subsectionHead, .likesubsectionHead { margin-top:2em; font-weight: bold;}
.sectionHead, .likesectionHead { font-weight: bold;}
.quote {margin-bottom:0.25em; margin-top:0.25em; margin-left:1em; margin-right:1em; text-align:justify;}
.verse{white-space:nowrap; margin-left:2em}
div.maketitle {text-align:center;}
h2.titleHead{text-align:center;}
div.maketitle{ margin-bottom: 2em; }
div.author, div.date {text-align:center;}
div.thanks{text-align:left; margin-left:10%; font-size:85%; font-style:italic; }
div.author{white-space: nowrap;}
.quotation {margin-bottom:0.25em; margin-top:0.25em; margin-left:1em; }
h1.partHead{text-align: center}
.sectionToc, .likesectionToc {margin-left:2em;}
.subsectionToc, .likesubsectionToc {margin-left:4em;}
.sectionToc, .likesectionToc {margin-left:6em;}
.frenchb-nbsp{font-size:75%;}
.frenchb-thinspace{font-size:75%;}
.figure img.graphics {margin-left:10%;}
/* end css.sty */

\title{Subdivisions, approximation des fonctions}
\author{}
\date{}

\begin{document}
\maketitle

\textbf{Warning: 
requires JavaScript to process the mathematics on this page.\\ If your
browser supports JavaScript, be sure it is enabled.}

\begin{center}\rule{3in}{0.4pt}\end{center}

[
[]
[

\section{9.1 Subdivisions, approximation des fonctions}

\subsection{9.1.1 Subdivisions d'un segment}

Définition~9.1.1 Soit [a,b] un segment de \mathbb{R}~~; on appelle subdivision
de [a,b] toute famille \sigma = (a_i)_0\leqi\leqn telle que
a_0 = a, a_n = b et \forall~~i \in
[1,n], a_i-1 < a_i. On appelle pas de la
subdivision \sigma le nombre réel \delta(\sigma) =\
min_i\in[1,n](a_i - a_i-1).

On notera \mathrmPt~(\sigma) =
\a_i∣0 \leq i \leq
n\ et S([a,b]) l'ensemble des subdivisions de
[a,b]. On définit une relation d'ordre partielle sur S(I) en disant
que \sigma' est plus fine que \sigma si
\mathrmPt~(\sigma)
\subset~\mathrmPt~(\sigma') on a alors
clairement \delta(\sigma') \leq \delta(\sigma)). On notera \sigma \cup \sigma' la subdivision définie par
\mathrmPt~(\sigma \cup \sigma')
= \mathrmPt~(\sigma)
\cup\mathrmPt~(\sigma'). Elle est
plus fine que \sigma et que \sigma'.

\subsection{9.1.2 Propriétés liées aux subdivisions}

Définition~9.1.2 On dit que f : [a,b] \rightarrow~ E est une fonction en
escalier (resp. affine par morceaux) s'il existe une subdivision \sigma =
(a_i)_0\leqi\leqn de [a,b] (que l'on dira adaptée à f)
telle que f soit constante (resp. affine) sur chacun des intervalles
ouverts ]a_i-1,a_i[.

Remarque~9.1.1 Il est clair que toute fonction affine par morceaux est
bornée et que toute fonction en escalier est affine par morceaux.

Définition~9.1.3 On dit que f : [a,b] \rightarrow~ E est une fonction de classe
C^k par morceaux s'il existe une subdivision \sigma =
(a_i)_0\leqi\leqn de [a,b] (que l'on dira adaptée à f)
telle que l'on ait les conditions équivalentes (i) pour chaque i \in
[1,n], il existe une fonction f_i :
[a_i-1,a_i] \rightarrow~ E de classe C^k telle que
\forall~t \in]a_i-1,a_i~[, f(t) =
f_i(t) (ii) la fonction f est de classe C^k sur
[a,b]
\diagdown\a_0,\\ldots,a_n\~
et
f,f',\\ldots,f^(k)~
admettent des limites à gauche et à droite en tous les points
a_i où cela a un sens.

Démonstration (i) \rigtharrow~(ii) est clair puisque dans ce cas
lim_t\rightarrow~a_i^+f^(p)~(t)
=\
lim_t\rightarrow~a_i^+f_i+1^(p)(t) =
f_i+1^(p)(a_i) et
lim_t\rightarrow~a_i^-f^(p)~(t)
=\
lim_t\rightarrow~a_i^-f_i^(p)(t) =
f_i^(p)(a_i).

En ce qui concerne (ii) \rigtharrow~(i) posons

 f_i(t) = \left \
\cases f(t) &si a_i-1 < t
< a_i \cr
f(a_i-1^+)&si t = a_i-1 \cr
f(a_i^-) &si t = a_i  \right
.

alors la fonction f_i est continue sur
[a_i-1,a_i], de classe C^k sur
]a_i-1,a_i[ et toutes les dérivées
f_i^(p) = f^(p) admettent des limites aux
points a_i-1 et a_i. On a vu dans le chapitre sur les
fonctions d'une variable qu'une telle fonction était de classe
C^k.

Remarque~9.1.2 Une fonction de classe C^k par morceaux n'est
pas nécessairement continue (en particulier f(a_i) peut être
distinct de la limite à gauche et de la limite à droite au point
a_i)~; cependant, comme chacune des f_i est continue
sur un compact donc bornée, et que les a_i sont en nombre fini,
une fonction de classe C^k par morceaux est bornée~;
remarquons également qu'une fonction affine par morceaux (et a fortiori
une fonction en escalier) est de classe C^\infty~ par morceaux.

Remarque~9.1.3 Si f est en escalier, ou affine par morceaux, ou
C^k par morceaux et si \sigma \inS([a,b]) est adaptée à f, alors
toute subdivision plus fine est encore adaptée à f~; en particulier si \sigma
est adaptée à f et \sigma' adaptée à g, alors \sigma \cup \sigma' est adaptée à la fois à
f et à g, d'où l'on déduit immédiatement la proposition suivante~:

Proposition~9.1.1 L'ensemble des applications en escalier (resp. affine
par morceaux, resp. C^k par morceaux) est un sous-espace
vectoriel ~de l'ensemble des applications de [a,b] dans E.

Définition~9.1.4 (Extension).Si I est un intervalle de \mathbb{R}~, on dira que f
: I \rightarrow~ E est en escalier (resp. affine par morceaux, resp. C^k
par morceaux) si sa restriction à tout segment [a,b] contenu dans I
est en escalier (resp. affine par morceaux, resp. C^k par
morceaux).

\subsection{9.1.3 Approximation des fonctions}

Soit [a,b] un segment de \mathbb{R}~ et ℬ([a,b],E) l'espace vectoriel des
fonctions bornées de [a,b] dans E. Pour f \inℬ([a,b],E), on notera
\f\\infty~
=\
sup_x\in[a,b]\f(x)\.

Définition~9.1.5 On dit que f : [a,b] \rightarrow~ E est réglée si elle vérifie
les conditions équivalentes (i) pour tout \epsilon > 0, il existe
\phi : [a,b] \rightarrow~ E en escalier telle que
sup_t\in[a,b]~\f(t)
- \phi(t)\ \leq \epsilon (ii) il existe une suite
\phi_n de fonctions en escalier de [a,b] dans E telle que
lim_n\rightarrow~+\infty~~\left
(sup_t\in[a,b]~\f(t)
- \phi_n(t)\\right ) = 0.
L'ensemble des fonctions réglées de [a,b] dans E est un sous-espace
vectoriel de ℬ([a,b],E).

Démonstration Les deux définitions sont clairement équivalentes (prendre
\epsilon = 1\diagupn pour (i) \rigtharrow~(ii)). La définition (i) implique évidemment que f - \phi
est bornée et comme \phi l'est également, une fonction réglée sur un
segment est nécessairement bornée. Autrement dit, le sous-espace
vectoriel des fonctions réglées de [a,b] dans E (car il est
évidemment stable par combinaisons linéaires) n'est autre que
l'adhérence de l'espace vectoriel des fonctions en escalier pour la
norme \.\\infty~.

Théorème~9.1.2 Toute fonction continue par morceaux est réglée.

Démonstration Commen\ccons par le démontrer pour une
fonction continue sur un segment. Une telle fonction est uniformément
continue. Soit \epsilon > 0. Il existe \eta > 0 tel que
\forall~~t,t' \in [a,b], t - t'
< \eta \rigtharrow~\ f(t) -
f(t')\ < \epsilon. Soit alors \sigma =
(a_i)_0\leqi\leqn une subdivision de pas plus petit que \eta et
définissons \phi : [a,b] \rightarrow~ E par \phi(a_i) = f(a_i) pour
i \in [0,n] et \phi(t) = f( a_i-1+a_i
\over 2 ) si t \in]a_i-1,a_i[ avec
i \in [1,n]. Alors \f(t) -
\phi(t)\ vaut 0 si t est l'un des a_i et
\f(t) - f( a_i-1+a_i
\over 2 )\ \leq \epsilon si t
\in]a_i-1,a_i[ puisque alors t -
a_i-1+a_i \over 2 
< \delta(\sigma) < \eta. On a bien une fonction \phi en escalier
telle que
sup_t\in[a,b]~\f(t)
- \phi(t)\ \leq \epsilon.

Si maintenant f est continue par morceaux, soit \sigma =
(a_i)_0\leqi\leqn une subdivision adaptée à f et soit
f_i : [a_i-1,a_i] \rightarrow~ E de classe
C^o telle que \forall~~t
\in]a_i-1,a_i[, f(t) = f_i(t). On peut
appliquer le cas précédent à f_i et trouver \phi_i en
escalier telle que
sup_t\in[a_i-1,a_i]\f_i~(t)
- \phi_i(t)\ \leq \epsilon. On définit alors une
fonction en escalier \phi : [a,b] \rightarrow~ E par \phi(a_i) =
f(a_i) et \phi(t) = \phi_i(t) si t
\in]a_i-1,a_i[. Alors on a
sup_t\in[a,b]~\f(t)
- \phi(t)\ \leq\
max_i\in[1,n]\left
(sup_t\in[a_i-1,a_i]\f_i~(t)
- \phi_i(t)\\right ) \leq \epsilon,
ce qui montre que f est réglée.

En fait, on peut montrer le résultat plus général suivant (que nous
n'utiliserons pas par la suite)

Théorème~9.1.3 Une fonction f : [a,b] \rightarrow~ E (espace vectoriel
normé~complet) est réglée si et seulement si~elle admet en tout point de
[a,b] (où cela a un sens) une limite à gauche et une limite à
droite.

Démonstration Supposons tout d'abord f réglée~; soit x_o un
point de [a,b] et montrons que, si
x_o\neq~b, f a une limite à droite au
point x_o à l'aide du critère de Cauchy pour les fonctions.
Soit donc \epsilon > 0 et \phi en escalier telle que
\forall~~t \in [a,b], \f(t)
- \phi(t)\ < \epsilon \over
3 . Soit (a_i) une subdivision adaptée à \phi et soit \eta
> 0 tel que ]x_o,x_o +
\eta[\subset~]a_i-1,a_i[. Pour t,t'
\in]x_o,x_o + \eta[, on a
\f(t) - f(t')\
=\ (f(t) - \phi(t)) + (\phi(t') -
f(t'))\ (car \phi(t) = \phi(t')), soit
\f(t) - f(t')\ \leq \epsilon
\over 3 + \epsilon \over 3 < \epsilon.
La fonction f vérifie donc le critère de Cauchy en x_o à
droite, et donc elle admet une limite à droite.

Inversement, supposons que f admette en tout point de [a,b] une
limite à gauche et une limite à droite et soit \epsilon > 0.
Alors, pour tout x \in [a,b], il existe \eta_x > 0
tel que \forall~t,t' \in]x,x + \eta_x~[,
\f(t) - f(t')\
< \epsilon et \forall~~t,t' \in]x -
\eta_x,x[, \f(t) -
f(t')\ < \epsilon (critère de Cauchy comme
condition nécessaire d'existence des limites). On a alors [a,b]
\subset~\⋃ ~
_x\in[a,b]]x - \eta_x,x + \eta_x[ (recouvrement
de [a,b] par des ouverts). D'après le théorème de Borel Lebesgue, on
peut trouver
x_1,\\ldots,x_p~
tels que [a,b] \subset~]x_1 -
\eta_x_1,x_1 +
\eta_x_1[\cup\\ldots\cup]x_p~
- \eta_x_p,x_p + \eta_x_p[.
Soit alors (a_i)_0\leqi\leqn une subdivision de [a,b]
telle que, pour tout i \in [1,n], il existe j \in [1,p] tel que
]a_i-1,a_i[\subset~]x_j -
\eta_x_j,x_j[ ou
]a_i-1,a_i[\subset~]x_j,x_j +
\eta_x_j[. On a alors \forall~~t,t'
\in]a_i-1,a_i[, \f(t) -
f(t')\ < \epsilon. On définit alors une
fonction \phi : [a,b] \rightarrow~ E par \phi(a_i) = f(a_i) et \phi(t)
= f( a_i-1+a_i \over 2 ) si t
\in]a_i-1,a_i[ avec i \in [1,n]. Alors on a
\forall~~t \in [a,b], \f(t)
- \phi(t)\ \leq \epsilon. Donc f est réglée.

Remarque~9.1.4 En particulier, on retrouve que les fonctions continues
par morceaux, mais aussi les fonctions monotones, sont réglées.

Enfin, pour terminer sur le problème de l'approximation des fonctions
nous donnerons les résultat suivants permettant d'approcher une fonction
continue soit par une fonction continue et affine par morceaux, soit par
une fonction polynomiale, soit par un polynôme trigonométrique.

Théorème~9.1.4 Soit f : [a,b] \rightarrow~ \mathbb{R}~ continue. Alors, pour tout \epsilon
> 0, il existe une fonction \phi : [a,b] \rightarrow~ E continue et
affine par morceaux telle que \forall~~t \in [a,b],
\f(t) - \phi(t)\
< \epsilon.

Démonstration Une telle fonction est uniformément continue. Soit \epsilon
> 0. Il existe \eta > 0 tel que
\forall~~t,t' \in [a,b], t - t'
< \eta \rigtharrow~\ f(t) -
f(t')\ < \epsilon \over 2
. Soit alors \sigma = (a_i)_0\leqi\leqn une subdivision de pas
plus petit que \eta et définissons \phi : [a,b] \rightarrow~ E par \phi(a_i) =
f(a_i) pour i \in [0,n], \phi(t) = f(a_i-1) +
f(a_i)-f(a_i-1 \over
a_i-a_i-1 (t - a_i-1) si t
\in]a_i-1,a_i[. Alors \phi est clairement affine par
morceaux et continue. Alors \f(t) -
\phi(t)\ vaut 0 si t est l'un des a_i et
si t \in]a_i-1,a_i[, on a (en tenant compte de
f(a_i-1) = \phi(a_i-1))

\begin{align*} \f(t) -
\phi(t)& \leq& \f(t) -
f(a_i-1)\ +\
\phi(a_i-1) - \phi(t)\ \%&
\\ & \leq& \f(t) -
f(a_i-1)\ +\
\phi(a_i-1) - \phi(a_i)\\%&
\\ & =& \f(t) -
f(a_i-1)\ +\
f(a_i-1) - f(a_i)\\%&
\\ & <& 2 \epsilon
\over 2 = \epsilon \%& \\
\end{align*}

puisque, \phi étant affine sur [a_i-1,a_i], on a
\\phi(a_i-1) -
\phi(t)\ \leq\
\phi(a_i-1) - \phi(a_i)\. Ceci
termine la démonstration.

Théorème~9.1.5 (premier théorème de Weierstrass). Soit f : [a,b] \rightarrow~ \mathbb{C}
continue. Alors, pour tout \epsilon > 0, il existe un polynôme P \in
\mathbb{C}[X] tel que \forall~~t \in [a,b],
\f(t) - P(t)\
< \epsilon.

Démonstration Ce résultat pourra être admis. Nous en donnerons cependant
une démonstration qui construit effectivement un tel polynôme (appelé
polynôme de Bernstein, de tels polynômes jouent un rôle important en
infographie). Il suffit évidemment de montrer ce résultat lorsque a = 0
et b = 1 (on fait ensuite un changement de variable affine qui
transforme [0,1] en [a,b]). Pour cela on part des identités
élémentaires suivantes (la première est la formule du binôme, les deux
autres s'en déduisent en dérivant par rapport à u)

\begin{align*} (u + v)^n& =&
\sum _k=0^nC_
n^ku^kv^n-k \%&
\\ n(u + v)^n-1& =&
\sum _k=1^nC_
n^kku^k-1v^n-k \%&
\\ n(n - 1)(u + v)^n-2& =&
\sum _k=2^nC_
n^kk(k - 1)u^k-2v^n-k\%&
\\ \end{align*}

Changeant u en x et v en 1 - x, on en déduit que

\begin{align*} 1& =& \\sum
_k=0^nC_ n^kx^k(1 -
x)^n-k \%& \\ n& =&
\sum _k=1^nC_
n^kkx^k-1(1 - x)^n-k \%&
\\ n(n - 1)& =&
\sum _k=2^nC_
n^kk(k - 1)x^k-2(1 - x)^n-k\%&
\\ \end{align*}

Soit a \in \mathbb{R}~. Ecrivons alors que

 \left (a - k \over n
\right )^2 = a^2 + ( 1
\over n^2 - 2 a \over n
)k + 1 \over n^2 k(k - 1)

On en déduit que

\begin{align*} \\sum
_k=0^nC_ n^k\left (a
- k \over n \right
)^2x^k(1 - x)^n-k&& \%&
\\ & =& a^2
\sum _k=0^nC_
n^kx^k(1 - x)^n-k + ( 1
\over n^2 - 2 a \over n
)\sum _k=0^nC_
n^kkx^k(1 - x)^n-k \%&
\\ & \text & + 1
\over n^2  \\sum
_k=0^nC_ n^kk(k - 1)x^k(1 -
x)^n-k \%& \\ & =&
a^2 \\sum
_k=0^nC_ n^kx^k(1 -
x)^n-k + ( 1 \over n^2 - 2 a
\over n )\\sum
_k=1^nC_ n^kkx^k(1 -
x)^n-k \%& \\ &
\text & + 1 \over n^2
 \sum _k=2^nC_
n^kk(k - 1)x^k(1 - x)^n-k \%&
\\ & =& a^2
\sum _k=0^nC_
n^kx^k(1 - x)^n-k + ( 1
\over n^2 - 2 a \over n
)x\sum _k=1^nC_
n^kkx^k-1(1 - x)^n-k\%&
\\ & \text & + 1
\over n^2 x^2
\sum _k=2^nC_
n^kk(k - 1)x^k-2(1 - x)^n-k \%&
\\ & =& a^2 + ( 1
\over n^2 - 2 a \over n
)xn + 1 \over n^2 x^2n(n - 1) =
(x - a)^2 + x(1 - x) \over n \%&
\\ \end{align*}

puis en rempla\ccant a par x

\sum _k=0^nC_
n^k\left (x - k \over n
\right )^2x^k(1 - x)^n-k
= x(1 - x) \over n

Soit \delta > 0. On a donc, pour x \in [0,1],

\begin{align*} \delta^2
\sum _\left x- k
\over n \right
≥\deltaC_n^kx^k(1 -
x)^n-k&& \%& \\ & \leq&
\sum _\left x-
k \over n \right
≥\deltaC_n^k\left (x - k
\over n \right
)^2x^k(1 - x)^n-k\%&
\\ & \leq& \\sum
_k=0^nC_ n^k\left (x
- k \over n \right
)^2x^k(1 - x)^n-k \%&
\\ & =& x(1 - x) \over
n \leq 1 \over 4n \%&
\\ \end{align*}

soit encore

\sum _\left x-
k \over n \right
≥\deltaC_n^kx^k(1 - x)^n-k
\leq 1 \over 4n\delta^2

Soit f : [0,1] \rightarrow~ \mathbb{C} continue et posons B_n(x)
= \\sum ~
_k=0^nf( k \over n
)C_n^kx^k(1 - x)^n-k (polynôme en
x de degré inférieur ou égal à n). Ecrivons

f(x) = f(x)1 = \\sum
_k=0^nf(x)C_ n^kx^k(1 -
x)^n-k

On a alors

\begin{align*} f(x) -
B_n(x)& =& \left
\sum _k=0^n~(f(x) - f(
k \over n ))C_n^kx^k(1 -
x)^n-k\right \%&
\\ & \leq& \\sum
_k=0^n\left f(x) - f( k
\over n )\right
C_n^kx^k(1 - x)^n-k \%&
\\ \end{align*}

Soit \epsilon > 0, puisque f est continue sur [0,1], elle est
uniformément continue et donc il existe \delta > 0 tel que, pour
tout couple t,t' vérifiant t - t' < \delta, on
ait f(t) - f(t') < \epsilon
\over 2 . On a alors en majorant suivant les cas
\left f(x) - f( k \over n
)\right  par  \epsilon \over 2 ou
par 2\f_\infty~

\begin{align*} f(x) -
B_n(x)& \leq& \\sum
_\left x- k \over n
\right <\delta\left
f(x) - f( k \over n )\right
C_n^kx^k(1 - x)^n-k \%&
\\ & \text &
+\sum _\left x-
k \over n \right
≥\delta\left f(x) - f( k
\over n )\right
C_n^kx^k(1 - x)^n-k\%&
\\ & \leq& \epsilon \over 2
\sum _\left x-
k \over n \right
<\deltaC_n^kx^k(1 -
x)^n-k \%& \\ &
\text &
+2\f_\infty~\\sum
_\left x- k \over n
\right
≥\deltaC_n^kx^k(1 - x)^n-k
\%& \\ & \leq& \epsilon \over
2 \sum _k=0^nC_
n^kx^k(1 - x)^n-k \%&
\\ & \text &
+2\f_\infty~\\sum
_\left x- k \over n
\right
≥\deltaC_n^kx^k(1 - x)^n-k
\%& \\ & \leq& \epsilon \over
2 +
2\f_\infty~ 1
\over 4n\delta^2 \%&
\\ \end{align*}

Prenons alors n assez grand pour que
2\f_\infty~ 1
\over 4n\delta^2 < \epsilon
\over 2 . On a alors, pour tout x \in [0,1],
f(x) - B_n(x) < \epsilon, ce qui achève
la démonstration.

Définition~9.1.6 On appelle polynôme trigonométrique de coefficients
a_p \in \mathbb{C}, p =
-N,\\ldots~,N la
fonction périodique de période 2\pi~ de \mathbb{R}~ dans \mathbb{C},
t\mapsto~\\\sum
 _p=-N^Na_pe^ipt.

Remarque~9.1.5 Les coefficients a_p sont uniquement déterminés
puisque les fonctions t\mapsto~e^ipt
forment une famille libre (ce sont par exemple des vecteurs propres de
l'opérateur de dérivation).

On vérifie immédiatement que les polynômes trigonométriques forment une
sous algèbre de l'algèbre des fonctions de classe C^\infty~ de \mathbb{R}~
dans \mathbb{C}.

Théorème~9.1.6 (deuxième théorème de Weierstrass). Soit f : \mathbb{R}~ \rightarrow~ \mathbb{C}
continue, périodique de période 2\pi~. Alors, pour tout \epsilon > 0,
il existe un polynôme trigonométrique g : \mathbb{R}~ \rightarrow~ \mathbb{C} tel que
\forall~~t \in \mathbb{R}~, \f(t) -
g(t)\ < \epsilon.

Démonstration Ce théorème sera admis~: c'est une conséquence facile du
théorème de Dirichlet qui dit qu'une fonction continue, de classe
\mathcal{C}^1 par morceaux, périodique de période 2\pi~ est somme de sa
série de Fourier qui converge normalement, donc peut être approchée à 
\epsilon \over 2 près par un polynôme trigonométrique. Il
suffit donc d'approcher notre fonction continue à  \epsilon
\over 2 près par une fonction continue, de classe
\mathcal{C}^1 par morceaux, périodique de période 2\pi~, et pour cela
d'approcher la restriction de f à [0,2\pi~] par une fonction continue
affine par morceaux prenant la même valeur en 0 et 2\pi~, ce qui se fait
comme ci-dessus.

[
[

\end{document}
