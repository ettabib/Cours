\documentclass[]{article}
\usepackage[T1]{fontenc}
\usepackage{lmodern}
\usepackage{amssymb,amsmath}
\usepackage{ifxetex,ifluatex}
\usepackage{fixltx2e} % provides \textsubscript
% use upquote if available, for straight quotes in verbatim environments
\IfFileExists{upquote.sty}{\usepackage{upquote}}{}
\ifnum 0\ifxetex 1\fi\ifluatex 1\fi=0 % if pdftex
  \usepackage[utf8]{inputenc}
\else % if luatex or xelatex
  \ifxetex
    \usepackage{mathspec}
    \usepackage{xltxtra,xunicode}
  \else
    \usepackage{fontspec}
  \fi
  \defaultfontfeatures{Mapping=tex-text,Scale=MatchLowercase}
  \newcommand{\euro}{€}
\fi
% use microtype if available
\IfFileExists{microtype.sty}{\usepackage{microtype}}{}
\ifxetex
  \usepackage[setpagesize=false, % page size defined by xetex
              unicode=false, % unicode breaks when used with xetex
              xetex]{hyperref}
\else
  \usepackage[unicode=true]{hyperref}
\fi
\hypersetup{breaklinks=true,
            bookmarks=true,
            pdfauthor={},
            pdftitle={Analyse numerique des fonctions d'une variable},
            colorlinks=true,
            citecolor=blue,
            urlcolor=blue,
            linkcolor=magenta,
            pdfborder={0 0 0}}
\urlstyle{same}  % don't use monospace font for urls
\setlength{\parindent}{0pt}
\setlength{\parskip}{6pt plus 2pt minus 1pt}
\setlength{\emergencystretch}{3em}  % prevent overfull lines
\setcounter{secnumdepth}{0}
 
/* start css.sty */
.cmr-5{font-size:50%;}
.cmr-7{font-size:70%;}
.cmmi-5{font-size:50%;font-style: italic;}
.cmmi-7{font-size:70%;font-style: italic;}
.cmmi-10{font-style: italic;}
.cmsy-5{font-size:50%;}
.cmsy-7{font-size:70%;}
.cmex-7{font-size:70%;}
.cmex-7x-x-71{font-size:49%;}
.msbm-7{font-size:70%;}
.cmtt-10{font-family: monospace;}
.cmti-10{ font-style: italic;}
.cmbx-10{ font-weight: bold;}
.cmr-17x-x-120{font-size:204%;}
.cmsl-10{font-style: oblique;}
.cmti-7x-x-71{font-size:49%; font-style: italic;}
.cmbxti-10{ font-weight: bold; font-style: italic;}
p.noindent { text-indent: 0em }
td p.noindent { text-indent: 0em; margin-top:0em; }
p.nopar { text-indent: 0em; }
p.indent{ text-indent: 1.5em }
@media print {div.crosslinks {visibility:hidden;}}
a img { border-top: 0; border-left: 0; border-right: 0; }
center { margin-top:1em; margin-bottom:1em; }
td center { margin-top:0em; margin-bottom:0em; }
.Canvas { position:relative; }
li p.indent { text-indent: 0em }
.enumerate1 {list-style-type:decimal;}
.enumerate2 {list-style-type:lower-alpha;}
.enumerate3 {list-style-type:lower-roman;}
.enumerate4 {list-style-type:upper-alpha;}
div.newtheorem { margin-bottom: 2em; margin-top: 2em;}
.obeylines-h,.obeylines-v {white-space: nowrap; }
div.obeylines-v p { margin-top:0; margin-bottom:0; }
.overline{ text-decoration:overline; }
.overline img{ border-top: 1px solid black; }
td.displaylines {text-align:center; white-space:nowrap;}
.centerline {text-align:center;}
.rightline {text-align:right;}
div.verbatim {font-family: monospace; white-space: nowrap; text-align:left; clear:both; }
.fbox {padding-left:3.0pt; padding-right:3.0pt; text-indent:0pt; border:solid black 0.4pt; }
div.fbox {display:table}
div.center div.fbox {text-align:center; clear:both; padding-left:3.0pt; padding-right:3.0pt; text-indent:0pt; border:solid black 0.4pt; }
div.minipage{width:100%;}
div.center, div.center div.center {text-align: center; margin-left:1em; margin-right:1em;}
div.center div {text-align: left;}
div.flushright, div.flushright div.flushright {text-align: right;}
div.flushright div {text-align: left;}
div.flushleft {text-align: left;}
.underline{ text-decoration:underline; }
.underline img{ border-bottom: 1px solid black; margin-bottom:1pt; }
.framebox-c, .framebox-l, .framebox-r { padding-left:3.0pt; padding-right:3.0pt; text-indent:0pt; border:solid black 0.4pt; }
.framebox-c {text-align:center;}
.framebox-l {text-align:left;}
.framebox-r {text-align:right;}
span.thank-mark{ vertical-align: super }
span.footnote-mark sup.textsuperscript, span.footnote-mark a sup.textsuperscript{ font-size:80%; }
div.tabular, div.center div.tabular {text-align: center; margin-top:0.5em; margin-bottom:0.5em; }
table.tabular td p{margin-top:0em;}
table.tabular {margin-left: auto; margin-right: auto;}
div.td00{ margin-left:0pt; margin-right:0pt; }
div.td01{ margin-left:0pt; margin-right:5pt; }
div.td10{ margin-left:5pt; margin-right:0pt; }
div.td11{ margin-left:5pt; margin-right:5pt; }
table[rules] {border-left:solid black 0.4pt; border-right:solid black 0.4pt; }
td.td00{ padding-left:0pt; padding-right:0pt; }
td.td01{ padding-left:0pt; padding-right:5pt; }
td.td10{ padding-left:5pt; padding-right:0pt; }
td.td11{ padding-left:5pt; padding-right:5pt; }
table[rules] {border-left:solid black 0.4pt; border-right:solid black 0.4pt; }
.hline hr, .cline hr{ height : 1px; margin:0px; }
.tabbing-right {text-align:right;}
span.TEX {letter-spacing: -0.125em; }
span.TEX span.E{ position:relative;top:0.5ex;left:-0.0417em;}
a span.TEX span.E {text-decoration: none; }
span.LATEX span.A{ position:relative; top:-0.5ex; left:-0.4em; font-size:85%;}
span.LATEX span.TEX{ position:relative; left: -0.4em; }
div.float img, div.float .caption {text-align:center;}
div.figure img, div.figure .caption {text-align:center;}
.marginpar {width:20%; float:right; text-align:left; margin-left:auto; margin-top:0.5em; font-size:85%; text-decoration:underline;}
.marginpar p{margin-top:0.4em; margin-bottom:0.4em;}
.equation td{text-align:center; vertical-align:middle; }
td.eq-no{ width:5%; }
table.equation { width:100%; } 
div.math-display, div.par-math-display{text-align:center;}
math .texttt { font-family: monospace; }
math .textit { font-style: italic; }
math .textsl { font-style: oblique; }
math .textsf { font-family: sans-serif; }
math .textbf { font-weight: bold; }
.partToc a, .partToc, .likepartToc a, .likepartToc {line-height: 200%; font-weight:bold; font-size:110%;}
.chapterToc a, .chapterToc, .likechapterToc a, .likechapterToc, .appendixToc a, .appendixToc {line-height: 200%; font-weight:bold;}
.index-item, .index-subitem, .index-subsubitem {display:block}
.caption td.id{font-weight: bold; white-space: nowrap; }
table.caption {text-align:center;}
h1.partHead{text-align: center}
p.bibitem { text-indent: -2em; margin-left: 2em; margin-top:0.6em; margin-bottom:0.6em; }
p.bibitem-p { text-indent: 0em; margin-left: 2em; margin-top:0.6em; margin-bottom:0.6em; }
.subsectionHead, .likesubsectionHead { margin-top:2em; font-weight: bold;}
.sectionHead, .likesectionHead { font-weight: bold;}
.quote {margin-bottom:0.25em; margin-top:0.25em; margin-left:1em; margin-right:1em; text-align:justify;}
.verse{white-space:nowrap; margin-left:2em}
div.maketitle {text-align:center;}
h2.titleHead{text-align:center;}
div.maketitle{ margin-bottom: 2em; }
div.author, div.date {text-align:center;}
div.thanks{text-align:left; margin-left:10%; font-size:85%; font-style:italic; }
div.author{white-space: nowrap;}
.quotation {margin-bottom:0.25em; margin-top:0.25em; margin-left:1em; }
h1.partHead{text-align: center}
.sectionToc, .likesectionToc {margin-left:2em;}
.subsectionToc, .likesubsectionToc {margin-left:4em;}
.sectionToc, .likesectionToc {margin-left:6em;}
.frenchb-nbsp{font-size:75%;}
.frenchb-thinspace{font-size:75%;}
.figure img.graphics {margin-left:10%;}
/* end css.sty */

\title{Analyse numerique des fonctions d'une variable}
\author{}
\date{}

\begin{document}
\maketitle

\textbf{Warning: 
requires JavaScript to process the mathematics on this page.\\ If your
browser supports JavaScript, be sure it is enabled.}

\begin{center}\rule{3in}{0.4pt}\end{center}

[
[
[]
[

\section{8.6 Analyse numérique des fonctions d'une variable}

\subsection{8.6.1 Interpolation linéaire, interpolation polynomiale}

Considérons f une fonction de classe C^n sur l'intervalle
[a,b], soit x_1 <
\\ldots~ <
x_n des points de [a,b] et considérons l'unique polynôme P
\in \mathbb{R}_n-1[X] vérifiant P(x_i) = f(x_i) pour
1 \leq i \leq n (polynôme d'interpolation de Lagrange).

Lemme~8.6.1 Pour tout x \in [a,b], \exists~\zeta
\in]a,b[, f(x) - P(x) =
(x-x_1)\\ldots(x-x_n~)
\over n! f^(n)(\zeta).

Démonstration Si x est l'un des x_i, n'importe quel \zeta convient.
On peut donc supposer que f n'est pas l'un des x_i. Considérons
alors g : t\mapsto~f(t) - P(t) - \lambda~
(t-x_1)\\ldots(t-x_n~)
\over n! où \lambda~ est choisi de telle sorte que g(x) = 0
(c'est évidemment possible). Alors g est de classe C^n et a n
+ 1 zéros distincts sur l'intervalle [a,b]. Des applications
répétées du théorème de Rolle assurent que la fonction g' a n zéros
distincts sur l'intervalle ]a,b[ (un entre deux zéros de g au sens
strict), puis que g'' a n - 1 zéros distincts jusque g^(n)
qui a au moins un zéro. Soit donc \zeta tel que g^(n)(\zeta) = 0. On
a donc 0 = g^(n)(\zeta) = f^(n)(\zeta) - \lambda~ car
P^(n) = 0 (vu que deg~ P \leq n - 1) et
 d^n \over dt^n ((t -
x_1)\\ldots~(t
- x_n)) = n!. on a donc \lambda~ = f^(n)(\zeta) et en écrivant
que g(x) = 0, on obtient f(x) - P(x) =
(x-x_1)\\ldots(x-x_n~)
\over n! f^(n)(\zeta).

Considérons le cas particulier où n = 2 et où l'on prend x_1 =
a et x_2 = b~; c'est le cas de l'interpolation linéaire entre a
et b où l'on remplace la courbe y = f(x) par la corde joignant les
points (a,f(a)) et (b,f(b)). On a alors P(t) = f(a) + f(b)-f(a)
\over b-a (t - a). Si on appelle M_2
=\
sup_t\in[a,b]f''(t), on obtient

Proposition~8.6.2 (erreur dans une interpolation linéaire). Soit f :
[a,b] \rightarrow~ \mathbb{R}~ de classe C^2, M_2
=\
sup_t\in[a,b]f''(t). Alors

\forall~~t \in [a,b], f(t)
-\left (f(a) + f(b) - f(a) \over b - a
(t - a)\right )\leq M_2 (b -
a)^2 \over 8

Démonstration On a en effet

\begin{align*} f(x) - P(x)&
=& (x - a)(b - x) \over 2
f''(\zeta)\leq M_2 (x - a)(b - x)
\over 2 \%& \\ & \leq&
M_2 (b - a)^2 \over 8 \%&
\\ \end{align*}

car (x - a)(b - x) \leq (b-a)^2 \over 4 pour
x \in [a,b].

\subsection{8.6.2 Dérivation numérique}

Nous nous limiterons au calcul approché des dérivées d'ordre 1 et 2
d'une fonction numérique. Pour la dérivée d'ordre 1 nous utiliserons la
formule f(x + h) = f(x) + hf'(x) + h\epsilon(h) avec
lim_h\rightarrow~0~\epsilon(h) = 0. On en déduit que
f'(x) est peu différent de \Delta_hf(x) = f(x+h)-f(x)
\over h quand h est petit. Mathématiquement, plus h est
petit, plus \Delta_hf(x) est proche de f'(x). Mais qu'en est-il de
la valeur calculée \overline\Delta_hf(x))~? Le
calcul de f(x + h) - f(x) se fait avec une erreur absolue de l'ordre de
2\deltaf(x), où \delta est la précision de l'instrument de calcul (
10^-16 par exemple). On a donc \Delta_hf(x)
-\overline\Delta_hf(x))\leq
2\deltaf(x) \over h , qui tend vers + \infty~
quand h tend vers 0. Il faut donc trouver un compromis pour la valeur de
h à choisir. Supposons f de classe C^2. Alors on a f(x + h) =
f(x) + hf'(x) + h^2 \over 2 f''(x) +
h^2\epsilon(h) et donc f'(x) - \Delta_hf(x)
est peu différent de  h \over 2
f''(x). On a donc f'(x)
-\overline\Delta_hf(x))\leq h
\over 2 f''(x) +
2\deltaf(x) \over h . Le deuxième membre
est une fonction de h qui est minimale pour h =
2\sqrt \deltaf(x) \over
f''(x)  et qui vaut alors
2\sqrt\deltaf(x)f''(x). Avec les
fonctions usuelles, on retiendra qu'il faut choisir un h de l'ordre de
\sqrt\delta (plutôt un peu trop grand, qu'un peu trop
petit) et que l'on obtient alors une erreur de l'ordre de
\sqrt \delta. Ainsi on choisira par exemple h =
10^-8 et on pourra espérer avoir 7 ou 8 chiffres
significatifs dans le calcul de la dérivée (ce qui est le plus souvent
largement suffisant, par exemple pour une étude de fonction).

Une autre méthode de calcul de la dérivée qui donne des valeurs un peu
plus précises (mais peut poser des problèmes de définition de la
fonction aux bornes de l'intervalle) est de prendre comme valeur
approchée de la dérivée l'expression  f(x+h)-f(x-h)
\over 2h (dérivée symétrique). L'erreur est alors en 
h^2 \over 6
f^(3)(x) + 2\deltaf(x)
\over h , elle est minimale pour un h de l'ordre de
\root3\of\delta et elle est alors de
l'ordre de \delta^2\diagup3 (prendre h = 10^-5 pour obtenir
environ 9 à 10 chiffres significatifs).

Pour le calcul de la dérivée seconde, reprenons la formule de
Taylor-Young f(x + h) = f(x) + hf'(x) + h^2
\over 2 f''(x) + h^2\epsilon(h). On obtient alors
lim_h\rightarrow~0~ f(x+h)+f(x-h)-2f(x)
\over h^2 = f''(x). On utilisera comme
valeur approchée de f''(x) l'expression \Delta_h^(2)f(x) =
f(x+h)+f(x-h)-2f(x) \over h^2 . Utilisons la
même méthode pour évaluer l'erreur entre la valeur calculée de
\Delta_h^(2)f(x) et f''(x). Supposons f de classe
C^4. On a la formule de Taylor Young à l'ordre 4, f(x + h) =
f(x) + hf'(x) + h^2 \over 2 f''(x) +
h^3 \over 6 f^(3)(x) +
h^4 \over 24 f^(4)(x) +
h^4\epsilon(h) d'où \Delta_h^(2)f(x) -
f''(x) est peu différent de  h^2
\over 12 f^(4)(x). On
obtient donc une majoration du type
\overline\Delta_h^(2)f(x) -
f''(x)\leq h^2 \over 12
f^(4)(x) + 3\delta \over
h^2 f(x). L'erreur est minimale pour
une valeur de h de l'ordre de
\root4\of\delta. On choisira donc un h
de l'ordre de 10^-4 pour obtenir un résultat optimal pour les
fonctions usuelles.

\subsection{8.6.3 Recherche des zéros d'une fonction}

On suppose dans toutes les méthodes qui suivent que l'on a effectué la
séparation des zéros de la fonction f, c'est-à-dire que l'on a trouvé un
intervalle [a,b] sur lequel f est strictement monotone avec f(a)f(b)
< 0. On suppose également que f est suffisamment dérivable. On
sait alors que f a un unique zéro r sur l'intervalle [a,b].

Méthode de dichotomie

Soit c un point de ]a,b[. Alors, soit f(a) et f(c) sont de signe
contraire, auquel cas r \in]a,c[, soit f(a) et f(c) sont de même signe
et dans ce cas r \in]c,a[. En prenant c = a+b \over
2 et en itérant le procédé, on construit une suite de segments
emboîtés [a_n,b_n] tels que
\forall~n \in \mathbb{N}~, r \in [a_n,b_n~] et
b_n - a_n = 1 \over 2
(b_n-1 - a_n-1), soit b_n - a_n =
1 \over 2^n (b - a). Le théorème des
segments emboîtés garantit alors que
\⋂ ~
_n\in\mathbb{N}~[a_n,b_n] = r. Si l'on prend
a_n comme valeur approchée de r par exemple, on a r -
a_n\leq 1 \over 2^n (b -
a).

Méthode de Lagrange

La méthode de Lagrange consiste à assimiler sur le segment [a,b] la
courbe y = f(x) à la droite passant par les points (a,f(a)) et (b,f(b)),
c'est-à-dire à approcher f par la fonction P(x) = f(a) + f(b)-f(a)
\over b-a (x - a) et à prendre comme valeur approchée
de r le réel \barr vérifiant
P(\barr) = 0.

Etudions l'erreur commise dans cette méthode. Soit P(x) = f(a) +
f(b)-f(a) \over b-a (x - a) (interpolation linéaire).
La majoration de l'erreur dans une interpolation linéaire nous garantit
que, en posant M_2 =\
sup_t\in[a,b]f''(t), on a f(r)
- P(r)\leq M_2 (b-a)^2 \over
8 . Mais f(r) = 0 et  P(r) \over
r-\barr = P(r)-P(\barr)
\over r-\barr = f(b)-f(a)
\over b-a = f'(c) pour un c \in]a,b[. On en déduit
que \barr - r =
\left  P(r) \over f'(c)
\right \leq M_2(b-a)^2
\over 8m_1 si l'on pose m_1
= inf~
_t\in[a,b]f'(t) (que nous supposerons
strictement positif). D'où la majoration de l'erreur dans la méthode de
Lagrange

\barr - r\leq M_2(b -
a)^2 \over 8m_1

On pourra par exemple combiner la méthode de dichotomie et la méthode de
Lagrange pour trouver rapidement une bonne approximation de r.
Remarquons de plus que si on suppose en outre que f est de concavité
constante sur [a,b], alors on connait le signe de r
-\bar r. En effet

(f convexe croissante, ou f concave décroissante))
\rigtharrow~\bar r < r

(f convexe décroissante, ou f concave croissante))
\rigtharrow~\bar r > r

Méthode de Newton

La méthode de Newton consiste à assimiler la courbe y = f(x) à la
tangente en un point c \in [a,b]. Cette tangente a pour équation y -
f(c) = f'(c)(x - c). Elle coupe donc l'axe des x au point d'abscisse r'
= c - f(c) \over f'(c) .

Cherchons à majorer l'erreur commise r - r'. On a
d'après la formule de Taylor Lagrange 0 = f(r) = f(c) + (r - c)f'(c) +
(r-c)^2 \over 2 f''(x) pour un certain x
\in]r,c[. De plus on a f(c) + (r' - c)f'(c) = 0. En soustrayant membre
à membre les deux égalités on trouve (r' - r)f'(c) -
(r-c)^2 \over 2 f''(x) = 0, soit r' - r =
(r-c)^2 \over 2  f'`(x)
\over f'(c) et donc (avec les notations données ci
dessus dans la méthode de Lagrange)

r' - r\leq M_2(b - a)^2
\over 2m_1

Remarquons que la majoration de l'erreur obtenue est 4 fois plus grande
que dans la méthode de Lagrange. Ce n'est évidemment pas que la méthode
de Newton soit moins bonne que la méthode de Lagrange, c'est simplement
la majoration de l'erreur qui est un peu moins fine. Remarquons que si
l'on suppose en outre que f est de concavité constante sur [a,b],
alors on connait le signe de r' - r. En effet

(f convexe croissante, ou f concave décroissante)) \rigtharrow~ r' > r

(f convexe décroissante, ou f concave croissante)) \rigtharrow~ r' < r

Les inégalités sont donc en sens contraire de celles obtenues par la
méthode de Lagrange. Dans le cas où f est monotone et de concavité
constante sur [a,b], la combinaison de la méthode de Lagrange et de
la méthode de Newton fournit un encadrement de r, ce qui est meilleur
qu'une majoration d'erreur.

Dans la pratique on ne s'arrête pas après avoir appliqué une fois la
méthode de Newton, mais au contraire on applique de nouveau la méthode
de Newton, mais cette fois ci au point r'. Cela revient à construire une
suite (x_n) par récurrence par x_o = c et
x_n+1 = x_n - f(x_n) \over
f'(x_n) . Les questions qui se posent naturellement sont

\begin{itemize}
\itemsep1pt\parskip0pt\parsep0pt
\item
  (i) est ce que tous les x_n sont dans [a,b]~? ,
\item
  (ii) est ce que la suite (x_n) converge~?
\item
  (iii) dans ce cas, sa limite est-elle r~?
\end{itemize}

Il est clair que dans la mesure où les réponses aux questions (i) et
(ii) sont oui, la réponse à la question (iii) est aussi oui, puisque si
l'on appelle L la limite de la suite, on doit avoir L = L - f(L)
\over f'(L) , soit f(L) = 0. Il nous reste donc à
répondre aux deux premières questions.

Nous nous placerons sous les hypothèses suivantes~: f est de classe
C^2 sur [a,b], f' ne s'annule pas sur [a,b] et f''
est de signe constant sur [a,b] (donc f est strictement monotone et
de concavité constante). Dans un premier temps nous supposerons pour
faire les raisonnements que f est croissante convexe sur [a,b].
Prenons x_o = c > r (par exemple c = b). On va
alors montrer que \forall~n, x_n~ \in [r,b]
et que la suite (x_n) est décroissante, ce qui permettra de
répondre positivement aux questions (i) et (ii). Pour cela considérons
la fonction g définie par g(x) = x - f(x) \over f'(x)
= xf'(x)-f(x) \over f'(x) . On a g'(x) = f(x)f'`(x)
\over f'(x)^2 ≥ 0 donc g est croissante sur
[a,b]. Supposons que x_n \in [r,b]. Comme g(r) = r, on a
x_n+1 = g(x_n) ≥ g(r) = r. De plus f étant strictement
croissante, elle est positive sur [r,b] et donc x_n+1 =
x_n - f(x_n) \over f'(x_n)
\leq x_n, d'où r \leq x_n+1 \leq x_n \leq b. On en déduit
que les réponses aux questions (i) et (ii) sont positives et fournissent
donc un moyen d'approximation de r. Remarquons qu'il est fondamental
pour cela d'avoir choisi un c > r, car g est décroissante
sur [a,r] et donc si x_o = c < r, x_1 =
g(x_o) ≥ g(r) = r, mais plus rien ne garantit que x_1
appartient toujours à [a,b]. Dans les cas de monotonie ou de
concavité différents on a les conclusions suivantes

(f convexe croissante ou f concave décroissante)~: choisir x_o
> r~; la suite (x_n) est décroissante et converge
vers r

(f convexe décroissante ou f concave croissante)~: choisir x_o
< r~; la suite (x_n) est croissante et converge vers
r

Il nous reste à savoir avec quelle vitesse la suite converge. On a g'(r)
= 0. Puisque g est continue, si l'on se donne K < 1, il existe
h > 0 tel que \forall~~x \in [r,r + h],
g'(x) < K. Alors soit N tel que n
> N \rigtharrow~ x_n \in [r,r + h]. On a alors pour n
> N, x_n+1 - r =
g(x_n) - g(r)\leq Kx_n -
r, d'où pour n > N, x_n -
r\leq K^n-Nx_N - r. On en
déduit que la suite (x_n - r) est négligeable devant la suite
K^n pour tout K < 1, et on a donc une convergence
extrêmement rapide dès que l'on se rapproche de r. On peut d'ailleurs
trouver un équivalent de x_n - r dans le
cas où g''(r)\neq~0 et montrer que
x_n - r∼ K^2^n  avec
un K < 1. On a donc une convergence très rapide (de type
hyperexponentiel)~: en gros le nombre de décimales double à chaque
itération dans la limite de la précision de la machine.

Méthode des approximations successives

Elle consiste à transformer une équation du type f(x) = 0 en une
équation du type g(x) = x à laquelle on essayera d'appliquer une méthode
de point fixe~: on construit une suite x_n+1 =
g(x_n)~; si cette suite converge, elle converge vers un point
fixe de g. Le lecteur pourra remarquer que la méthode de Newton en est
un cas particulier avec g(x) = x - f(x) \over f'(x) .

[
[
[
[

\end{document}
