\documentclass[]{article}
\usepackage[T1]{fontenc}
\usepackage{lmodern}
\usepackage{amssymb,amsmath}
\usepackage{ifxetex,ifluatex}
\usepackage{fixltx2e} % provides \textsubscript
% use upquote if available, for straight quotes in verbatim environments
\IfFileExists{upquote.sty}{\usepackage{upquote}}{}
\ifnum 0\ifxetex 1\fi\ifluatex 1\fi=0 % if pdftex
  \usepackage[utf8]{inputenc}
\else % if luatex or xelatex
  \ifxetex
    \usepackage{mathspec}
    \usepackage{xltxtra,xunicode}
  \else
    \usepackage{fontspec}
  \fi
  \defaultfontfeatures{Mapping=tex-text,Scale=MatchLowercase}
  \newcommand{\euro}{€}
\fi
% use microtype if available
\IfFileExists{microtype.sty}{\usepackage{microtype}}{}
\ifxetex
  \usepackage[setpagesize=false, % page size defined by xetex
              unicode=false, % unicode breaks when used with xetex
              xetex]{hyperref}
\else
  \usepackage[unicode=true]{hyperref}
\fi
\hypersetup{breaklinks=true,
            bookmarks=true,
            pdfauthor={},
            pdftitle={Endomorphismes d'un espace hermitien},
            colorlinks=true,
            citecolor=blue,
            urlcolor=blue,
            linkcolor=magenta,
            pdfborder={0 0 0}}
\urlstyle{same}  % don't use monospace font for urls
\setlength{\parindent}{0pt}
\setlength{\parskip}{6pt plus 2pt minus 1pt}
\setlength{\emergencystretch}{3em}  % prevent overfull lines
\setcounter{secnumdepth}{0}
 
/* start css.sty */
.cmr-5{font-size:50%;}
.cmr-7{font-size:70%;}
.cmmi-5{font-size:50%;font-style: italic;}
.cmmi-7{font-size:70%;font-style: italic;}
.cmmi-10{font-style: italic;}
.cmsy-5{font-size:50%;}
.cmsy-7{font-size:70%;}
.cmex-7{font-size:70%;}
.cmex-7x-x-71{font-size:49%;}
.msbm-7{font-size:70%;}
.cmtt-10{font-family: monospace;}
.cmti-10{ font-style: italic;}
.cmbx-10{ font-weight: bold;}
.cmr-17x-x-120{font-size:204%;}
.cmsl-10{font-style: oblique;}
.cmti-7x-x-71{font-size:49%; font-style: italic;}
.cmbxti-10{ font-weight: bold; font-style: italic;}
p.noindent { text-indent: 0em }
td p.noindent { text-indent: 0em; margin-top:0em; }
p.nopar { text-indent: 0em; }
p.indent{ text-indent: 1.5em }
@media print {div.crosslinks {visibility:hidden;}}
a img { border-top: 0; border-left: 0; border-right: 0; }
center { margin-top:1em; margin-bottom:1em; }
td center { margin-top:0em; margin-bottom:0em; }
.Canvas { position:relative; }
li p.indent { text-indent: 0em }
.enumerate1 {list-style-type:decimal;}
.enumerate2 {list-style-type:lower-alpha;}
.enumerate3 {list-style-type:lower-roman;}
.enumerate4 {list-style-type:upper-alpha;}
div.newtheorem { margin-bottom: 2em; margin-top: 2em;}
.obeylines-h,.obeylines-v {white-space: nowrap; }
div.obeylines-v p { margin-top:0; margin-bottom:0; }
.overline{ text-decoration:overline; }
.overline img{ border-top: 1px solid black; }
td.displaylines {text-align:center; white-space:nowrap;}
.centerline {text-align:center;}
.rightline {text-align:right;}
div.verbatim {font-family: monospace; white-space: nowrap; text-align:left; clear:both; }
.fbox {padding-left:3.0pt; padding-right:3.0pt; text-indent:0pt; border:solid black 0.4pt; }
div.fbox {display:table}
div.center div.fbox {text-align:center; clear:both; padding-left:3.0pt; padding-right:3.0pt; text-indent:0pt; border:solid black 0.4pt; }
div.minipage{width:100%;}
div.center, div.center div.center {text-align: center; margin-left:1em; margin-right:1em;}
div.center div {text-align: left;}
div.flushright, div.flushright div.flushright {text-align: right;}
div.flushright div {text-align: left;}
div.flushleft {text-align: left;}
.underline{ text-decoration:underline; }
.underline img{ border-bottom: 1px solid black; margin-bottom:1pt; }
.framebox-c, .framebox-l, .framebox-r { padding-left:3.0pt; padding-right:3.0pt; text-indent:0pt; border:solid black 0.4pt; }
.framebox-c {text-align:center;}
.framebox-l {text-align:left;}
.framebox-r {text-align:right;}
span.thank-mark{ vertical-align: super }
span.footnote-mark sup.textsuperscript, span.footnote-mark a sup.textsuperscript{ font-size:80%; }
div.tabular, div.center div.tabular {text-align: center; margin-top:0.5em; margin-bottom:0.5em; }
table.tabular td p{margin-top:0em;}
table.tabular {margin-left: auto; margin-right: auto;}
div.td00{ margin-left:0pt; margin-right:0pt; }
div.td01{ margin-left:0pt; margin-right:5pt; }
div.td10{ margin-left:5pt; margin-right:0pt; }
div.td11{ margin-left:5pt; margin-right:5pt; }
table[rules] {border-left:solid black 0.4pt; border-right:solid black 0.4pt; }
td.td00{ padding-left:0pt; padding-right:0pt; }
td.td01{ padding-left:0pt; padding-right:5pt; }
td.td10{ padding-left:5pt; padding-right:0pt; }
td.td11{ padding-left:5pt; padding-right:5pt; }
table[rules] {border-left:solid black 0.4pt; border-right:solid black 0.4pt; }
.hline hr, .cline hr{ height : 1px; margin:0px; }
.tabbing-right {text-align:right;}
span.TEX {letter-spacing: -0.125em; }
span.TEX span.E{ position:relative;top:0.5ex;left:-0.0417em;}
a span.TEX span.E {text-decoration: none; }
span.LATEX span.A{ position:relative; top:-0.5ex; left:-0.4em; font-size:85%;}
span.LATEX span.TEX{ position:relative; left: -0.4em; }
div.float img, div.float .caption {text-align:center;}
div.figure img, div.figure .caption {text-align:center;}
.marginpar {width:20%; float:right; text-align:left; margin-left:auto; margin-top:0.5em; font-size:85%; text-decoration:underline;}
.marginpar p{margin-top:0.4em; margin-bottom:0.4em;}
.equation td{text-align:center; vertical-align:middle; }
td.eq-no{ width:5%; }
table.equation { width:100%; } 
div.math-display, div.par-math-display{text-align:center;}
math .texttt { font-family: monospace; }
math .textit { font-style: italic; }
math .textsl { font-style: oblique; }
math .textsf { font-family: sans-serif; }
math .textbf { font-weight: bold; }
.partToc a, .partToc, .likepartToc a, .likepartToc {line-height: 200%; font-weight:bold; font-size:110%;}
.chapterToc a, .chapterToc, .likechapterToc a, .likechapterToc, .appendixToc a, .appendixToc {line-height: 200%; font-weight:bold;}
.index-item, .index-subitem, .index-subsubitem {display:block}
.caption td.id{font-weight: bold; white-space: nowrap; }
table.caption {text-align:center;}
h1.partHead{text-align: center}
p.bibitem { text-indent: -2em; margin-left: 2em; margin-top:0.6em; margin-bottom:0.6em; }
p.bibitem-p { text-indent: 0em; margin-left: 2em; margin-top:0.6em; margin-bottom:0.6em; }
.subsectionHead, .likesubsectionHead { margin-top:2em; font-weight: bold;}
.sectionHead, .likesectionHead { font-weight: bold;}
.quote {margin-bottom:0.25em; margin-top:0.25em; margin-left:1em; margin-right:1em; text-align:justify;}
.verse{white-space:nowrap; margin-left:2em}
div.maketitle {text-align:center;}
h2.titleHead{text-align:center;}
div.maketitle{ margin-bottom: 2em; }
div.author, div.date {text-align:center;}
div.thanks{text-align:left; margin-left:10%; font-size:85%; font-style:italic; }
div.author{white-space: nowrap;}
.quotation {margin-bottom:0.25em; margin-top:0.25em; margin-left:1em; }
h1.partHead{text-align: center}
.sectionToc, .likesectionToc {margin-left:2em;}
.subsectionToc, .likesubsectionToc {margin-left:4em;}
.sectionToc, .likesectionToc {margin-left:6em;}
.frenchb-nbsp{font-size:75%;}
.frenchb-thinspace{font-size:75%;}
.figure img.graphics {margin-left:10%;}
/* end css.sty */

\title{Endomorphismes d'un espace hermitien}
\author{}
\date{}

\begin{document}
\maketitle

\textbf{Warning: 
requires JavaScript to process the mathematics on this page.\\ If your
browser supports JavaScript, be sure it is enabled.}

\begin{center}\rule{3in}{0.4pt}\end{center}

[
[
[]
[

\section{13.4 Endomorphismes d'un espace hermitien}

\subsection{13.4.1 Notion d'adjoint}

Soit E un espace préhilbertien complexe

Définition~13.4.1 Soit E un espace préhilbertien complexe. Soit u,v \in
L(E). On dit que u et v sont des endomorphismes adjoints si

\forall~x,y \in E, (u(x)\mathrel∣~y)
= (x∣v(y))

Remarque~13.4.1 La symétrie hermitienne du produit scalaire montre
clairement que u et v jouent des rôles symétriques, donc que u est
adjoint de v si et seulement si~v est adjoint de u.

Théorème~13.4.1 Soit E un espace hermitien. Tout endomorphisme de E
admet un unique adjoint u^∗. Si u \in L(E), \mathcal{E} une base de E, \Omega
= \mathrmMat~ (\phi,\mathcal{E}) et A
= \mathrmMat~ (u,\mathcal{E}), alors

\mathrmMat~
(u^∗,\mathcal{E}) = \Omega^-1A^∗\Omega

Démonstration Soit \mathcal{E} une base de E et \Omega =\
\mathrmMat (\phi,\mathcal{E}). Comme \phi est non dégénérée, la
matrice \Omega est inversible. Soit u,v \in L(E), A =\
\mathrmMat (u,\mathcal{E}) et B =\
\mathrmMat (v,\mathcal{E}). Si x,y \in E, on a
(u(x)∣y) = (AX)^∗\OmegaY =
X^∗A^∗\OmegaY et (x∣v(y)) =
X^∗\OmegaBY . L'unicité de la matrice de la forme sesquilinéaire
(x,y)\mapsto~(u(x)\mathrel∣y)
montre que

\begin{align*} \forall~~x,y \in E,
(u(x)∣y) =
(x∣v(y))&& \%&
\\ & \Leftrightarrow &
A^∗\Omega = \OmegaB \Leftrightarrow B =
\Omega^-1A^∗\Omega\%& \\
\end{align*}

ce qui montre à la fois l'existence et l'unicité de l'adjoint et la
formule voulue.

Proposition~13.4.2 Soit E un espace hermitien. L'application
u\mapsto~u^∗ est un endomorphisme
semi-linéaire involutif de L(E). Si u,v \in L(E), alors u \cdot v aussi et
(u \cdot v)^∗ = v^∗\cdot u^∗. Si u \in L(E) est
inversible, alors u^∗ est inversible et
(u^-1)^∗ = (u^∗)^-1.

Démonstration On a déjà vu que la relation u et v sont adjoints était
symétrique, donc si u \in L(E), u^∗ aussi et u^∗∗ =
u. Si u,v \in L(E), \alpha~,\beta~ \in \mathbb{C}, on a

\begin{align*} ((\alpha~u +
\beta~v)(x)∣y)& =& (\alpha~u(x) +
\beta~v(x)∣y) \%&
\\ & =&
\overline\alpha~(u(x)∣y) +
\overline\beta~(v(x)∣y) \%&
\\ & =&
\overline\alpha~(x∣u^∗(y))
+
\overline\beta~(x∣v^∗(y))\%&
\\ & =&
(x∣(\overline\alpha~u^∗
+ \overline\beta~v^∗)(y)) \%&
\\ \end{align*}

ce qui montre que (\alpha~u + \beta~v)^∗ =
\overline\alpha~u^∗ +
\overline\beta~v^∗ et donc la semilinéarité de
u\mapsto~u^∗. Si u,v \in L(E), on a

(u \cdot v(x)∣y) =
(v(x)∣u^∗(y)) =
(x∣v^∗\cdot u^∗(y))

ce qui montre que u \cdot v admet v^∗\cdot u^∗ comme
adjoint.

Si u est inversible, on a u^-1 \cdot u =
\mathrmId_E d'où (u^-1 \cdot
u)^∗ = \mathrmId_E^∗, soit
u^∗\cdot (u^-1)^∗ =
\mathrmId_E. De même u \cdot u^-1 =
\mathrmId_E donne par passage à l'adjoint
(u^-1)^∗\cdot u^∗ =
\mathrmId_E. Ceci montre que u^∗
est inversible et que (u^-1)^∗ =
(u^∗)^-1

Proposition~13.4.3 Soit E un espace hermitien, u \in L(E). Alors

\begin{itemize}
\itemsep1pt\parskip0pt\parsep0pt
\item
  (i) \mathrm{det}~
  u^∗ =
  \overline\mathrm{det}~
  u,
  \mathrm{tr}u^∗~ =
  \overline\mathrm{tr}u~,
  \chi_u^∗ = \overline\chi_u
\item
  (ii)
  \mathrmKeru^∗~
  =
  (\mathrmImu)^\bot~,
  \mathrmImu^∗~ =
  (\mathrmKeru)^\bot~
\item
  (iii)
  \mathrmKeru^∗~u
  = \mathrmKer~u et
  \mathrmImu^∗~u
  = \mathrmImu^∗~
\end{itemize}

Démonstration (i) Soit \mathcal{E} une base de E, \Omega =\
\mathrmMat (\phi,\mathcal{E}) et A =\
\mathrmMat (u,\mathcal{E}), alors
\mathrmMat~
(u^∗,\mathcal{E}) = \Omega^-1A^∗\Omega. On a donc
\mathrm{det} u^∗~
= \mathrm{det}~
\Omega^-1A^∗\Omega =\
\mathrm{det} A^∗ =
\overline\mathrm{det}~
A =
\overline\mathrm{det}~
u. La démonstration est la même pour la trace et pour le polynôme
caractéristique.

(ii) On a

\begin{align*} x
\in\mathrmKeru^∗~&
\Leftrightarrow & u^∗(x) = 0
\Leftrightarrow \forall~~y \in E,
(u^∗(x)∣y) = 0 \%&
\\ & \Leftrightarrow &
\forall~y \in E, (x\mathrel∣~u(y)) =
0 \Leftrightarrow x \in
(\mathrmImu)^\bot~\%&
\\ \end{align*}

En appliquant ce résultat à u^∗ on obtient,
\mathrmKer~u =
(\mathrmImu^∗)^\bot~
et en prenant l'orthogonal,
\mathrmImu^∗~ =
(\mathrmKeru)^\bot~

(iii) On a visiblement u(x) = 0 \rigtharrow~ u^∗u(x) = 0, donc
\mathrmKer~u
\subset~\mathrmKeru^∗~u~;
mais d'autre part, si x
\in\mathrmKeru^∗~u,
on a

\u(x)\^2 =
(u(x)∣u(x)) =
(u^∗u(x)∣x) =
(0∣x) = 0

et donc u(x) = 0, soit
\mathrmKeru^∗~u
\subset~\mathrmKer~u et l'égalité.
On en déduit alors que

\mathrmImu^∗~u =
(\mathrmKer(u^∗u)^∗)^\bot~
=
(\mathrmKeru^∗u)^\bot~
=
(\mathrmKeru)^\bot~
= \mathrmImu^∗~

Une des propriétés essentielles de l'adjoint que nous utiliserons de
fa\ccon systématique pour la réduction des
endomorphismes est la suivante

Théorème~13.4.4 Soit u \in L(E). Soit F un sous-espace de E stable par u~;
alors F^\bot est stable par u^∗.

Démonstration Soit x \in F^\bot. Si y \in F, on a
\phi(u^∗(x),y) = \phi(x,u(y)) = 0 puisque u(y) \in F et x \in
F^\bot. Donc u^∗(x) \in F^\bot et
F^\bot est stable par u^∗.

\subsection{13.4.2 Endomorphismes hermitiens}

Définition~13.4.2 Soit E un espace hermitien, u \in L(E). On dit que u est
hermitien (ou autoadjoint) s'il vérifie les conditions équivalentes

\begin{itemize}
\itemsep1pt\parskip0pt\parsep0pt
\item
  (i) u^∗ = u
\item
  (ii) \forall~~x,y \in E,
  (u(x)∣y) =
  (x∣u(y))
\end{itemize}

Remarque~13.4.2 Si la base \mathcal{E} est orthonormée, alors
\mathrmMat~ ((
∣ ),\mathcal{E}) = I_n et
\mathrmMat~
(u^∗,\mathcal{E}) =\
\mathrmMat (u,\mathcal{E})^∗~; en particulier

Théorème~13.4.5 Soit \mathcal{E} une base orthonormée de E~; alors u est hermitien
si et seulement
si~\mathrmMat~ (u,\mathcal{E}) est une
matrice hermitienne.

Proposition~13.4.6 L'ensemble H(E) des endomorphismes hermitiens est un
\mathbb{R}~-sous-espace vectoriel de L(E) (mais pas un \mathbb{C} sous-espace vectoriel).
On a L(E) = H(E) \oplus~ iH(E)

Démonstration L'endomorphisme de L^∗(E),
u\mapsto~u^∗ étant \mathbb{R}~ linéaire et
involutif, l'espace L(E) est somme directe du sous-espace propre associé
à la valeur propre 1 (les endomorphismes hermitiens) et du sous-espace
propre associé à la valeur propre -1 (les endomorphismes antihermitiens,
qui ne sont autre que les endomorphismes hermitiens multipliés par i).

\subsection{13.4.3 Groupe unitaire}

Soit E un espace hermitien

Définition~13.4.3 On dit que u \in L(E) est un endomorphisme unitaire si
on a les propriétés équivalentes

\begin{itemize}
\itemsep1pt\parskip0pt\parsep0pt
\item
  (i) \forall~~x \in E,
  \u(x)\
  =\ x\
\item
  (ii) \forall~~x,y \in E,
  (u(x)∣u(y)) =
  (x∣y)
\item
  (iii) u est inversible et u^-1 = u^∗
\item
  (iv) u \cdot u^∗ = \mathrmId_E
\item
  (v) u^∗\cdot u = \mathrmId_E
\end{itemize}

Démonstration (ii) \rigtharrow~(i) est évident (faire y = x). (i) \rigtharrow~(ii) provient de
l'identité de polarisation et de la linéarité de u. Pour un
endomorphisme d'un espace vectoriel de dimension finie, on sait que
l'inversibilité est équivalente à l'inversibilité à gauche ou à droite.
On a donc (iii) \Leftrightarrow (iv)
\Leftrightarrow (v). Supposons (ii) vérifié. Alors \phi(x,y) =
\phi(u(x),u(y)) = \phi(x,u^∗\cdot u(y)), ce qui montre (puisque \phi est
non dégénérée) que u^∗\cdot u =
\mathrmId_E~; donc (ii) \rigtharrow~(v). De même (v)
\rigtharrow~(ii) puisque \phi(u(x),u(y)) = \phi(x,u^∗\cdot u(y)).

Théorème~13.4.7 L'ensemble U(E) des endomorphismes unitaires de E est un
sous-groupe de (GL(E),\cdot). Pour tout endomorphisme unitaire u de E, on a
\mathrm{det}~
u = 1. L'ensemble SU(E) des endomorphismes unitaires de
déterminant 1 est un sous-groupe distingué de U(E).

Démonstration On a clairement \mathrmId_E \in
U(E). La définition (i) montre évidemment que si u et v sont unitaires,
il en est de même de u \cdot v. De plus, soit u \in U(E)~; on a
\u^-1(x)\
=\
u(u^-1(x))\
=\ x\ ce qui montre
que u^-1 \in U(E). Donc U(E) est un sous-groupe de (GL(E),\cdot).
On a alors 1 = \mathrm{det}~
\mathrmId_E =\
\mathrm{det} (u^∗\cdot u)
= \mathrm{det}~
u^∗\mathrm{det}~ u
= \mathrm{det}~
u^2, soit
\mathrm{det}~
u = 1. L'application de U(E) dans le groupe multiplicatif des
nombres complexes de module 1,
u\mapsto~\mathrm{det}~
u est un morphisme de groupes~; son noyau SU(E) est donc un sous groupe
distingué.

Théorème~13.4.8 Soit u \in L(E).

\begin{itemize}
\itemsep1pt\parskip0pt\parsep0pt
\item
  (i) Si u est unitaire, il envoie toute base orthonormée sur une base
  orthonormée.
\item
  (ii) Inversement, s'il existe une base orthonormée \mathcal{E} de E telle que
  u(\mathcal{E}) soit encore orthonormée, alors u est un endomorphisme unitaire.
\end{itemize}

Démonstration (i) On a
(u(e_i)∣u(e_j)) =
(e_i∣e_j) =
\delta_i^j.

(ii) Soit x = \\sum ~
x_ie_i \in E. On a
\x\^2
= \\sum ~
x_i^2. Mais on a aussi u(x)
= \\sum ~
x_iu(e_i) et comme u(\mathcal{E}) est orthonormée,
\u(x)\^2
= \\sum ~
x_i^2~; on a donc
\forall~~x \in E,
\u(x)\
=\ x\.

Théorème~13.4.9 Soit u un endomorphisme unitaire et F un sous-espace de
E stable par u. Alors F^\bot est stable par u.

Démonstration On a u(F) \subset~ F et comme u est inversible, on a
dim u(F) =\ dim~ F. On
a donc u(F) = F. Soit donc x \in F^\bot et y \in F~; il existe z \in F
tel que u(z) = y, d'où (u(x)∣y) =
(u(x)∣u(z)) =
(x∣z) = 0, et donc u(x) \in F^\bot.

\subsection{13.4.4 Matrices unitaires}

Proposition~13.4.10 Soit E un espace hermitien. Soit u \in L(E), \mathcal{E} une
base de E, \Omega = \mathrmMat~
(( ∣ ),\mathcal{E}) et A =\
\mathrmMat (u,\mathcal{E}). Alors u est un endomorphisme
unitaire si et seulement si~A^∗\OmegaA = \Omega.

Démonstration On a \phi(u(x),u(y)) = (AX)^∗\Omega(AY ) =
X^∗A^∗\OmegaAY . L'unicité de la matrice d'une forme
bilinéaire montre que

\forall~~x,y \in E,
(u(x)∣u(y)) =
(x∣y) \mathrel\Leftrightarrow
A^∗\OmegaA = \Omega

En particulier, si \mathcal{E} est une base orthonormée de E, u est un
endomorphisme unitaire si et seulement si~A^∗A =
I_n. Ceci conduit à la définition suivante

Définition~13.4.4 Soit A \in M_\mathbb{C}(n). On dit que A est une matrice
unitaire si elle vérifie les conditions équivalentes

\begin{itemize}
\itemsep1pt\parskip0pt\parsep0pt
\item
  (i) A est inversible et A^-1 = A^∗
\item
  (ii) A^∗A = I_n
\item
  (iii) AA^∗ = I_n
\end{itemize}

Théorème~13.4.11 L'ensemble U(n) des matrices carrées unitaires d'ordre
n est un sous-groupe de (GL_\mathbb{C}(n),.). Pour toute matrice
unitaire A, on a
\mathrm{det}~
A = 1. L'ensemble SU(n) des matrices unitaires de déterminant
1 est un sous-groupe distingué de U(n) .

Démonstration On a clairement I_n \in U(n). La définition (i)
montre évidemment que si A et B sont unitaires, il en est de même de AB.
De plus, soit A \in U(n)~; on a
A^-1(A^-1)^∗ =
A^-1(A^∗)^∗ = A^-1A =
I_n ce qui montre que A^-1 \in U(n). Donc U(n) est un
sous-groupe de (GL_\mathbb{C}(n),.). On a alors 1
= \mathrm{det} I_n~
= \mathrm{det}~
(A^∗A) =
\mathrm{det}~
A^2, soit
\mathrm{det}~
A = 1. L'application de U(n) dans le groupe multiplicatif des
nombres complexes de module 1,
A\mapsto~\mathrm{det}~
A est un morphisme de groupes multiplicatifs~; son noyau SU(n) est donc
un sous-groupe distingué.

Dans ce subsectione, on munira \mathbb{C}^n de la forme sesquilinéaire
hermitienne naturelle (qui rend la base canonique orthonormée),
c'est-à-dire que l'on posera (x∣y)
= \\sum ~
_i=1^n\overlinex_iy_i

Théorème~13.4.12 Une matrice A \in M_\mathbb{C}(n) est unitaire si et
seulement si~ses vecteurs colonnes (resp. lignes) forment une base
orthonormée de \mathbb{C}^n.

Démonstration Soit
(c_1,\\ldots,c_n~)
les vecteurs colonnes de A,
(l_1,\\ldots,l_n~)
ses vecteurs lignes. On a

\begin{align*} A \in U(n)&
\Leftrightarrow & A^∗A = I_ n
\Leftrightarrow \forall~~i,j,
(A^∗A)_ i,j = \delta_i^j \%&
\\ & \Leftrightarrow &
\forall~~i,j, \\sum
_k=1^n\overlinea_
k,ia_k,j = \delta_i^j
\Leftrightarrow \forall~i,j, (c_
i∣c_j) = \delta_i^j\%&
\\ \end{align*}

De la même fa\ccon, en traduisant la relation
AA^∗ = I_n, on obtiendrait
(l_i∣l_j) =
\delta_i^j.

Théorème~13.4.13 Soit E un espace hermitien. Soit \mathcal{E} une base orthonormée
de E, \mathcal{E}' une base de E. Alors on a équivalence de

\begin{itemize}
\itemsep1pt\parskip0pt\parsep0pt
\item
  (i) \mathcal{E}' est orthonormée
\item
  (ii) la matrice P_\mathcal{E}^\mathcal{E}' de passage de la base \mathcal{E} à la
  base \mathcal{E}' est unitaire.
\end{itemize}

Démonstration On sait que P_\mathcal{E}^\mathcal{E}'
= \mathrmMat~ (u,\mathcal{E}) où u est
l'endomorphisme de E défini par \forall~~i,
u(e_i) = e_i'. Or d'après les résultats du subsectione
précédent, u est un endomorphisme unitaire si et seulement si~\mathcal{E}' est
orthonormée~; mais d'autre part, comme \mathcal{E} est orthonormée, u est unitaire
si et seulement
si~\mathrmMat~ (u,\mathcal{E}) est une
matrice unitaire, d'où l'équivalence entre (i) et (ii).

\subsection{13.4.5 Réduction des endomorphismes normaux}

Définition~13.4.5 Soit E un espace hermitien et u \in L(E). On dit que u
est un endomorphisme normal si

u^∗u = uu^∗

Lemme~13.4.14 Soit u un endomorphisme normal. Alors
\mathrmKeru^∗~
= \mathrmKer~u.

Démonstration On a

\begin{align*} x
\in\mathrmKeru^∗~&
\Leftrightarrow &
(u^∗(x)∣u^∗(x)) = 0
\Leftrightarrow
(uu^∗(x)∣x) = 0\%&
\\ & \Leftrightarrow &
(u^∗u(x)∣x) = 0
\Leftrightarrow (u(x)\mathrel∣u(x)) = 0
\%& \\ & \Leftrightarrow &
x \in\mathrmKer~u \%&
\\ \end{align*}

Lemme~13.4.15 2. Soit u un endomorphisme normal. Alors, pour tout \lambda~ \in \mathbb{C},
\mathrmKer(u^∗-\overline\lambda~\mathrmId_E~)
= \mathrmKer~(u -
\lambda~\mathrmId_E).

Démonstration Il suffit de remarquer que u -
\lambda~\mathrmId est encore normal (élémentaire) et de lui
appliquer le lemme précédent en remarquant que
u^∗-\overline\lambda~\mathrmId_E
= (u - \lambda~\mathrmId_E)^∗

Théorème~13.4.16 Soit u un endomorphisme d'un espace hermitien. On a
équivalence de

\begin{itemize}
\itemsep1pt\parskip0pt\parsep0pt
\item
  (i) u est normal
\item
  (ii) u est diagonalisable dans une base orthonormée.
\end{itemize}

Démonstration (ii) \rigtharrow~(i) Soit \mathcal{E} une base orthonormée de diagonalisation
de u. Alors \mathrmMat~
(u,\mathcal{E}) =\
diag(\lambda_1,\\ldots,\lambda_n~).
Comme \mathcal{E} est orthonormée, on a
\mathrmMat~
(u^∗,\mathcal{E}) =\
\mathrmMat (u,\mathcal{E})^∗
=\
diag(\overline\lambda_1,\\ldots,\overline\lambda_n~).
Les deux matrices diagonales commutant, on a uu^∗ =
u^∗u, donc u est normal.

(i) \rigtharrow~(ii) Montrons le résultat par récurrence sur
dim~ E, le résultat étant évident si
dim~ E = 1. Supposons que u est normal. Comme \mathbb{C}
est algébriquement clos, u admet une valeur propre \lambda~. Comme
\mathrmKer(u^∗-\overline\lambda~\mathrmId_E~)
= \mathrmKer~(u -
\lambda~\mathrmId_E), E_u(\lambda~)
= \mathrmKer~(u -
\lambda~\mathrmId_E) est stable par u^∗
et donc E_u(\lambda~)^\bot est stable par u^∗∗ = u.
Mais comme E_u(\lambda~) est stable par u, le sous-espace
E_u(\lambda~)^\bot est stable par u^∗. Soit v =
u__ E_u(\lambda~)^\bot. La
relation (v(x)∣y) =
(u(x)∣y) =
(x∣u^∗(y)) pour x,y \in
E_u(\lambda~)^\bot montre que v^∗ =
u__ E_u(\lambda~)^\bot^∗,
donc v^∗v = vv^∗ et donc v est un endomorphisme
normal de E_u(\lambda~)^\bot. Par hypothèse de récurrence, il
existe une base orthonormée de E_u(\lambda~)^\bot formée de
vecteurs propres de v donc de u. Comme E = E_u(\lambda~) \bot \oplus~
E_u(\lambda~)^\bot, si on réunit cette base avec une base
orthonormée de E_u(\lambda~), on obtient une base orthonormée de E
formée évidemment de vecteurs propres de u, ce qui achève la
démonstration.

Remarque~13.4.3 Soit \mathcal{E} une telle base. Alors
\mathrmMat~ (u,\mathcal{E})
=\
diag(\lambda_1,\\ldots,\lambda_n~).
L'endomorphisme u est hermitien si et seulement si~sa matrice dans la
base orthonormée \mathcal{E} est hermitienne, c'est-à-dire si et seulement
si~\forall~i, \lambda_i~ \in \mathbb{R}~~; de même u est
unitaire si et seulement si~sa matrice dans la base orthonormée \mathcal{E} est
unitaire, c'est-à-dire si et seulement si~\forall~~i,
\lambda_i = 1. Comme il est clair que tout
endomorphisme hermitien ou unitaire est normal on obtient les deux
corollaires

Corollaire~13.4.17 Soit u un endomorphisme d'un espace hermitien. On a
équivalence de

\begin{itemize}
\itemsep1pt\parskip0pt\parsep0pt
\item
  (i) u est hermitien
\item
  (ii) u est diagonalisable dans une base orthonormée et
  \mathrm{Sp}~(u) \subset~ \mathbb{R}~
\end{itemize}

Corollaire~13.4.18 Soit u un endomorphisme d'un espace hermitien. On a
équivalence de

\begin{itemize}
\itemsep1pt\parskip0pt\parsep0pt
\item
  (i) u est unitaire
\item
  (ii) u est diagonalisable dans une base orthonormée et
  \mathrm{Sp}~(u) \subset~ U
  (ensemble des nombres complexes de module 1)
\end{itemize}

\subsection{13.4.6 Réduction des matrices normales}

En traduisant le subsectione précédent en terme de matrices (en utilisant
le produit hermitien canonique sur \mathbb{C}^2 défini par
(x∣y) =\
\sum ~
_i\overlinex_iy_i) on
obtient la définition et les résultats suivants.

Définition~13.4.6 Soit A \in M_\mathbb{C}(n). On dit que A est une matrice
normale si

A^∗A = AA^∗

Théorème~13.4.19 Soit A \in M_\mathbb{C}(n). On a équivalence de

\begin{itemize}
\itemsep1pt\parskip0pt\parsep0pt
\item
  (i) A est normal
\item
  (ii) Il existe P unitaire telle que P^-1AP =
  P^∗AP soit diagonale.
\end{itemize}

Corollaire~13.4.20 Soit A \in M_\mathbb{C}(n). On a équivalence de

\begin{itemize}
\itemsep1pt\parskip0pt\parsep0pt
\item
  (i) A est hermitienne
\item
  (ii) Il existe P unitaire telle que P^-1AP =
  P^∗AP soit diagonale réelle
\end{itemize}

Corollaire~13.4.21 Soit A \in M_\mathbb{C}(n). On a équivalence de

\begin{itemize}
\itemsep1pt\parskip0pt\parsep0pt
\item
  (i) A est unitaire
\item
  (ii) Il existe P unitaire telle que P^-1AP =
  P^∗AP soit diagonale à éléments diagonaux dans U (ensemble
  des nombres complexes de module 1)
\end{itemize}

[
[
[
[

\end{document}
