
\section{16.5 Equations différentielles non linéaires}

\subsection{16.5.1 Théorie de Cauchy-Lipschitz}

Définition~16.5.1 Soit E un espace vectoriel normé de dimension finie, U
un ouvert de \mathbb{R}~ \times E et F : U \rightarrow~ E,
(t,y)\mapsto~F(t,y). On dira que F est localement
lipschitzienne par rapport à la variable y si, pour tout
(t_0,y_0) \in U, il existe \eta > 0 et r
> 0 et une constante L ≥ 0 telle que

\begin{align*} \forall~~t \in
[t_0 - \eta,t_0 + \eta],
\forall~y_1,y_2~ \in
B'(y_0,r),& & \%& \\
\F(t,y_1) -
F(t,y_2)\ \leq
L\y_1 -
y_2& & \%&
\\ \end{align*}

L'ensemble C = [t_0 - \eta,t_0 + \eta] \times
B'(y_0,r) sera appelé un cylindre de sécurité associé à la
constante L.

Théorème~16.5.1 Soit E un espace vectoriel normé de dimension finie, U
un ouvert de \mathbb{R}~ \times E et f : U \rightarrow~ E, (t,y)\mapsto~f(t,y)
de classe \mathcal{C}^1. Alors f est localement lipschitzienne par
rapport à la variable y.

Démonstration Notons \partial_yf(t,y) la différentielle au point y de
l'application z\mapsto~f(t,z) de E dans E. Cette
application est composée de z\mapsto~(t,z) dont la
différentielle en tout point est l'application j :
h\mapsto~(0,h) et de l'application f. Si bien que
\partial_yf(t,y).h = df(t,y).(0,h) = df(t,y) \cdot j(h). On en déduit que
l'application (t,y)\mapsto~\partial_yf(t,y) =
df(t,y) \cdot j est continue (comme l'application
(t,y)\mapsto~df(t,y)). Soit alors \eta > 0
et r > 0 tels que [t_0 - \eta,t_0 + \eta] \times
B'(0,r) \subset~ U. L'application
(t,y)\mapsto~\partial_yf(t,y) est continue sur le
compact [t_0 - \eta,t_0 + \eta] \times B'(0,r), donc elle y
est bornée. Il existe donc L ≥ 0 tel que
\forall~(t,y) \in [t_0 - \eta,t_0~ +
\eta] \times B'(0,r),
\\partial_yf(t,y)\ \leq
L. Mais alors, soit t \in [t_0 - \eta,t_0 + \eta],
y_1,y_2 \in B'(y_0,r)~; l'inégalité des
accroissements finis appliquée à la fonction
y\mapsto~f(t,y) sur le segment
[y_1,y_2] \subset~ B'(y_0,r) assure que

\f(t,y_1) -
f(t,y_2)\ \leq\
y_1 -
y_2\\
sup_y\in[y_1,y_2]\\partial_yf(t,y)\
\leq L\y_1 -
y_2\

ce qui démontre le résultat.

Théorème~16.5.2 (Cauchy Lipschitz, unicité). Soit E un espace vectoriel
normé de dimension finie, U un ouvert de \mathbb{R}~ \times E et F : U \rightarrow~ E,
(t,y)\mapsto~F(t,y) localement lipschitzienne par
rapport à la variable y. Alors F vérifie la condition d'unicité au
problème de Cauchy-Lipschitz~:

soit (I,\phi) et (J,\psi) deux solutions de l'équation différentielle y' =
F(t,y) qui coïncident au point t_0 \in I \bigcap J . Alors \phi et \psi
coïncident sur I \bigcap J.

Démonstration I \bigcap J est un intervalle. En particulier I \bigcap J est connexe.
Soit X = \t \in I \bigcap J∣\phi(t) =
\psi(t)\. Comme \phi et \psi sont continues sur I \bigcap J, X = (\phi
- \psi)^-1(\0\) est un fermé de
I \bigcap J (image réciproque d'un fermé par une application continue). Soit
alors t_1 \in I \bigcap J et y_1 = \phi(t_1) =
\psi(t_1). On a donc d'après un théorème précédent
\forall~t \in I \bigcap J, \phi(t) = y_1~
+\int  _t_1^t~F(u,\phi(u))
du et \psi(t) = y_1 +\int ~
_t_1^tF(u,\psi(u)) du. On en déduit que \phi(t) - \psi(t)
=\int  _t_1^t~(F(u,\phi(u))
- F(u,\psi(u))) du, et donc, pour t ≥ t_1,

\\phi(t) - \psi(t)\
\leq\int ~
_t_1^t\F(u,\phi(u)) -
F(u,\psi(u))\ du

Puisque F est localement lipschitzienne, il existe \eta > 0 et
r > 0 et une constante L ≥ 0 telle que

\begin{align*} \forall~~t \in
[t_1 - \eta,t_1 + \eta],
\forall~z_1,z_2~ \in
B'(y_1,r),& & \%& \\
\F(t,z_1) -
F(t,z_2)\ \leq
L\z_1 -
z_2& & \%&
\\ \end{align*}

Comme \phi et \psi sont continues au point t_1, il existe \eta'
> 0 tel que t - t_1\leq \eta' \rigtharrow~
\phi(t),\psi(t) \in B(y_1,r). Pour t - t_1
< \alpha~ = min~(\eta,\eta'), on a donc
\F(t,\phi(t)) -
F(t,\psi(t))\ \leq L\\phi(t) -
\psi(t)\, et donc, en appliquant l'inégalité ci
dessus \\phi(t) - \psi(t)\
\leq\int ~
_t_1^tL\\phi(u) -
\psi(u)\ du. On peut donc appliquer à la fonction
\\phi - \psi\ le lemme de
Gronwall avec c = 0, g(t) = L qui montre que \\phi
- \psi\ est nulle sur [t_1,t_1
+ \alpha~[\bigcapI \bigcap J. On montre de manière similaire que
\\phi - \psi\ est nulle sur
]t_1 - \alpha~,t_1]\, \bigcap I \bigcap J. Donc, si X
contient t_1, il contient ]t_1 - \alpha~,t_1 +
\alpha~[\, \bigcap I \bigcap J ce qui montre que X est un ouvert de I \bigcap
J. En résumé X est une partie ouverte et fermée, non vide (car
t_0 \in X) de I \bigcap J qui est connexe. Par définition de la
connexité, on a X = I \bigcap J, et donc \phi et \psi coïncident sur I \bigcap J.

Théorème~16.5.3 (Cauchy Lipschitz, existence locale). Soit E un espace
vectoriel normé de dimension finie, U un ouvert de \mathbb{R}~ \times E et F : U \rightarrow~ E,
(t,y)\mapsto~F(t,y) continue et localement
lipschitzienne par rapport à la variable y. Alors F vérifie la condition
d'existence au problème de Cauchy-Lipschitz. Soit
(t_0,y_0) \in U, \alpha~ > 0 et r >
0 tels que C = [t_0 - \alpha~,t_0 + \alpha~] \times
B'(y_0,r) \subset~ U soit un cylindre de sécurité pour F associé à la
constante L, M =\
sup_(t,y)\inC\F(t,y)\,
\eta = min(\alpha~, r \over M~ )~;
alors il existe une solution \phi de l'équation différentielle y' = F(t,y)
définie sur ]t_0 - \eta,t_0 + \eta[ et vérifiant
\phi(t_0) = y_0.

Démonstration Remarquons tout d'abord que l'existence de M résulte de la
compacité du cylindre de sécurité C et de la continuité de F. Nous
savons d'autre part que \phi est une solution de l'équation différentielle
y' = F(t,y) vérifiant \phi(t_0) = y_0 si et seulement si
\phi est une fonction continue vérifiant \phi(t) = y_0
+\int  _t_0^t~F(u,\phi(u))
du, autrement dit si \phi est point fixe de l'application
\psi\mapsto~\Gamma(\psi), où l'on définit \Gamma(\psi)(t) =
y_0 +\int ~
_t_0^tF(u,\psi(u)) du. Nous inspirant de la
démonstration du théorème du point fixe, nous allons rechercher \phi par
une méthode d'approximation successive. Posons donc, pour t
\in]t_0 - \eta,t_0 + \eta[, \phi_0(t) =
y_0~; supposons maintenant que \phi_n est définie de
telle sorte que \forall~t \in]t_0~ -
\eta,t_0 + \eta[, \phi_n(t) \in B'(y_0,r) et posons
alors \phi_n+1(t) = y_0 +\int ~
_t_0^tF(u,\phi_n(u)) du (ceci a bien un
sens car \forall~u \in [t_0~,t],
(u,\phi_n(u)) \in C \subset~ U). On a alors

\begin{align*}
\\phi_n+1(t) -
y_0& =&
\\int ~
_t_0^tF(u,\phi_ n(u))
du\ \%& \\ &
\leq& t -
t_0sup_(u,y)\inC~\F(u,y)\
\leq \etaM \leq r\%& \\
\end{align*}

Ceci montre qu'en posant \phi_0(t) = y_0 et pour n ≥ 0,
\phi_n+1(t) = y_0 +\int ~
_t_0^tF(u,\phi_n(u)) du, on définit bien
une suite d'applications continues de ]t_0 - \eta,t_0 +
\eta[ dans B'(y_0,r).

Montrons maintenant par récurrence sur n que

\begin{align*} \forall~~n \in \mathbb{N}~,
\forall~t \in]t_0 - \eta,t_0~ + \eta[,
\\phi_n+1(t) -
\phi_n(t)\ \leq M \over L
 L^n+1t - t_0^n+1
\over (n + 1)! & & \%&
\\ \end{align*}

Pour n = 1, on a

\begin{align*}
\\phi_1(t) -
\phi_0(t)\ =\
\phi_1(t) - y_0\
=\\int ~
_t_0^tF(u,y_ 0)
du\ \leq Mt - t_0&
& \%& \\
\end{align*}

ce qui est bien la formule voulue. Si maintenant l'inégalité est
vérifiée pour n - 1, on a alors pour t \in [t_0,t_0 +
\eta[

\begin{align*}
\\phi_n+1(t) -
\phi_n(t)& =&
\\int ~
_t_0^t\left (F(u,\phi_ n(u))
- F(u,\phi_n-1(u))\right )
du\\%& \\ & \leq&
\int ~
_t_0^t\F(u,\phi_
n(u)) - F(u,\phi_n-1(u))\ du \%&
\\ & \leq& \int ~
_t_0^tL\\phi_ n(u)
- \phi_n-1(u)\ du \%&
\\ & \leq& L\int ~
_t_0^t M \over L 
L^nu - t_0^n
\over n! du \%& \\ &
=& M \over L  L^n+1t -
t_0^n+1 \over (n + 1)!
\%& \\ \end{align*}

Un calcul similaire conduit à la même inégalité pour t \in]t_0
- \eta,t_0].

On en déduit donc que

\forall~n \in \mathbb{N}~, \\forall~~t
\in]t_0 - \eta,t_0 + \eta[,
\\phi_n+1(t) -
\phi_n(t)\ \leq M \over L
 L^n+1\eta^n+1 \over (n + 1)!

Alors pour q > p, on a donc

\begin{align*} \forall~~t
\in]t_0 - \eta,t_0 + \eta[&& \%&
\\
\\phi_q(t) -
\phi_p(t)& \leq&
\\sum
_n=p^q-1\\phi_ n+1(t) -
\phi_n(t)\\%&
\\ & \leq& M \over L
\sum _n=p^q-1~
L^n+1\eta^n+1 \over (n + 1)! \%&
\\ & \leq& M \over L
\sum _n=p^+\infty~~
L^n+1\eta^n+1 \over (n + 1)! \%&
\\ \end{align*}

Comme la série \\sum ~
_n L^n+1\eta^n+1 \over
(n+1)! est une série convergente (exponentielle d'un nombre réel), son
reste tend vers 0~; étant donné \epsilon > 0, il existe N \in \mathbb{N}~ tel
que p ≥ N \rigtharrow~ M \over L \
\sum  _n=p^+\infty~~
L^n+1\eta^n+1 \over (n+1)!
< \epsilon. Alors

q > p ≥ N \rigtharrow~\forall~t \in]t_0~ -
\eta,t_0 + \eta[, \\phi_q(t) -
\phi_p(t)\ < \epsilon

La suite (\phi_n) vérifie donc la critère de Cauchy uniforme. En
conséquence, elle converge uniformément vers une fonction \phi
:]t_0 - \eta,t_0 + \eta[\rightarrow~ B'(y_0,r) qui est
elle même continue.

L'inégalité \F(u,\phi(u)) -
F(u,\phi_n(u))\ \leq
L\\phi(u) -
\phi_n(u)\, montre que la suite
F(u,\phi_n(u)) converge uniformément vers F(u,\phi(u))~; ceci nous
permet de passer à la limite sous le signe d'intégration et d'obtenir

\begin{align*} y_0
+\int  _t_0^t~F(u,\phi(u))
du&& \%& \\ & =& y_0
+ lim_n\rightarrow~+\infty~~\\int
 _t_0^tF(u,\phi_ n(u)) du\%&
\\ & =&
lim_n\rightarrow~+\infty~\phi_n+1~(t) = \phi(t) \%&
\\ \end{align*}

Comme \phi est continue, ceci montre que \phi est la solution cherchée de
l'équation y' = F(t,y) vérifiant \phi(t_0) = y_0.

Théorème~16.5.4 (Cauchy Lipschitz~: existence et unicité d'une solution
maximale). Soit E un espace vectoriel normé de dimension finie, U un
ouvert de \mathbb{R}~ \times E et F : U \rightarrow~ E, (t,y)\mapsto~F(t,y)
continue et localement lipschitzienne par rapport à la variable y. Soit
(t_0,y_0) \in U~; alors il existe une unique solution
maximale (I_0,\phi_0) de l'équation différentielle y' =
F(t,y) qui vérifie \phi_0(t_0) = y_0.
L'intervalle I_0 est ouvert. Pour toute solution (J,\psi) de
l'équation différentielle vérifiant \psi(t_0) = y_0, on
a~:

\text\$J \subset~ I_0\$ et \$\psi\$ est la restriction
de \$\phi_0\$ à \$J\$.

Démonstration La fonction F vérifie les conditions d'existence et
d'unicité au problème de Cauchy-Lipschitz et on sait que ces conditions
impliquent l'existence et l'unicité d'une solution maximale vérifiant
une condition initiale donnée. On sait d'autre part que cette solution
maximale est définie sur un intervalle ouvert et que c'est un plus grand
élément de l'ensemble des solutions vérifiant la condition initiale.

Remarque~16.5.1 L'intervalle I_0 de définition d'une solution
maximale est difficilement contrôlable a priori. Il dépend bien entendu
de l'équation différentielle, mais aussi de la condition initiale
imposée, quelle que soit la régularité de la fonction F. Considérons par
exemple l'équation différentielle scalaire (c'est-à-dire que les
solutions sont à valeurs réelles) du premier ordre y' = -y^2.
La fonction F(t,y) = -y^2 est bien entendu de classe
\mathcal{C}^1 donc localement lipschitzienne si bien que les résultats
précédents s'appliquent. Comme la fonction nulle définie sur \mathbb{R}~ est
solution, toute solution (I,\phi) qui s'annule en un point doit coïncider
sur I \bigcap \mathbb{R}~ = I avec la fonction nulle, donc être la fonction nulle. On en
déduit qu'une solution (I,\phi) non identiquement nulle ne doit pas
s'annuler et doit donc vérifier  \phi'(t) \over
\phi(t)^2 = -1 soit encore  d \over dt
\left ( 1 \over \phi(t)
\right ) = 1, ou encore  1 \over \phi(t)
= t + \lambda~. On en déduit que toute solution est du type
(I,t\mapsto~ 1 \over t+\lambda~ ).
Cherchons une solution vérifiant la condition initiale \phi(t_0) =
y_0 avec y_0\neq~0 (puisque
seule la solution nulle peut s'annuler). On obtient y_0 = 1
\over t_0+\lambda~ soit encore \lambda~ = 1
\over y_0 - t_0 si bien que toute
solution vérifiant la condition initiale en question est de la forme
(I,t\mapsto~ 1 \over
t-t_0+ 1 \over y_0  ), la
continuité de la fonction imposant que t_0 - 1
\over y_0 ∉I. A
partir de là, il est immédiat de déterminer les solutions maximales
vérifiant la condition initiale y(t_0) = y_0~: il
suffit de déterminer un intervalle maximal contenant t_0 et ne
contenant pas t_0 - 1 \over y_0 .
(i) si y_0 = 0, la solution maximale est (]
-\infty~,+\infty~[,t\mapsto~0) (ii) si y_0
> 0, la solution maximale est (]t_0 - 1
\over y_0
,+\infty~[,t\mapsto~ 1 \over
t-t_0+ 1 \over y_0  ) (iii) si
y_0 < 0, la solution maximale est (]
-\infty~,t_0 - 1 \over y_0
[,t\mapsto~ 1 \over
t-t_0+ 1 \over y_0  )

Bien que la fonction F(t,y) = -y^2 soit parfaitement
régulière sur \mathbb{R}~^2 tout entier, les solutions maximales
présentent des asymptotes verticales et ne sont pas (sauf la solution
nulle), définies sur \mathbb{R}~ tout entier.

Remarque~16.5.2 La condition imposée à F d'être localement
lipschitzienne (par exemple de classe \mathcal{C}^1) est essentielle.
Si on considère par exemple l'équation différentielle y' =
2\sqrty, le lecteur vérifiera
sans difficulté que la fonction nulle et la fonction
t\mapsto~tt sont toutes deux des
solutions maximales définies sur \mathbb{R}~ et vérifiant la même condition
initiale y(0) = 0 si bien que l'unicité d'une solution maximale à
condition initiale donnée n'est pas vérifiée.

\subsection{16.5.2 Application aux équations d'ordre n}

Par la technique de réduction à l'ordre 1, on aboutit aux résultats
suivants

Théorème~16.5.5 (Cauchy Lipschitz, unicité). Soit E un espace vectoriel
normé de dimension finie, U un ouvert de \mathbb{R}~ \times E^n et f : U \rightarrow~
E,
(t,y_0,\\ldots,y_n-1)\mapsto~f(t,y_0,\\\ldots,y_n-1~)
de classe \mathcal{C}^1. Alors f vérifie la condition d'unicité au
problème de Cauchy-Lipschitz~: soit (I,\phi) et (J,\psi) deux solutions de
l'équation différentielle y^(n) =
f(t,y,y',\\ldots,y^(n-1)~)
qui vérifient \phi(t_0) =
\psi(t_0),\\ldots,\phi^(n-1)(t_0~)
= \psi^(n-1)(t_0). Alors \phi et \psi coïncident sur I \bigcap J.

Théorème~16.5.6 (Cauchy Lipschitz, existence locale). Soit E un espace
vectoriel normé de dimension finie, U un ouvert de \mathbb{R}~ \times E^n et
f : U \rightarrow~ E de classe \mathcal{C}^1. Alors f vérifie la condition
d'existence au problème de Cauchy-Lipschitz. Soit
(t_0,y_0,\\ldots,y_n-1~)
\in U, alors il existe \eta > 0 et une solution \phi de l'équation
différentielle y^(n) =
f(t,y,y',\\ldots,y^(n-1)~)
définie sur ]t_0 - \eta,t_0 + \eta[ et vérifiant
\phi(t_0) =
y_0,\\ldots,\phi^(n-1)(t_0~)
= y_n-1.

Théorème~16.5.7 (Cauchy Lipschitz, existence et unicité d'une solution
maximale). Soit E un espace vectoriel normé de dimension finie, U un
ouvert de \mathbb{R}~ \times E^n et f : U \rightarrow~ E de classe \mathcal{C}^1. Soit
(t_0,y_0,\\ldots,y_n-1~)
\in U~; alors il existe une unique solution maximale
(I_0,\phi_0) de l'équation différentielle
y^(n) =
f(t,y,y',\\ldots,y^(n-1)~)
qui vérifie \phi(t_0) =
y_0,\\ldots,\phi^(n-1)(t_0~)
= y_n-1. L'intervalle I_0 est ouvert. Pour toute
solution (J,\psi) de l'équation différentielle vérifiant \psi(t_0) =
y_0,\\ldots,\psi^(n-1)(t_0~)
= y_n-1, on a~:

\text\$J \subset~ I_0\$ et \$\psi\$ est la restriction
de \$\phi_0\$ à \$J\$.

\subsection{16.5.3 Systèmes différentiels autonomes d'ordre 1}

Définition~16.5.2 Soit E un espace vectoriel normé, U un ouvert de
E^n et F : U \rightarrow~ E. On dit que l'équation différentielle
d'ordre n, y^(n) =
F(y,y',\\ldots,y^(n-1)~)
(indépendante du temps t) est une équation différentielle autonome.

Proposition~16.5.8 (invariance par translation de l'ensemble des
solutions). Soit (I,\phi) une solution de l'équation différentielle
autonome y^(n) =
F(y,y',\\ldots,y^(n-1)~),
et soit T \in \mathbb{R}~. Alors le couple (I_T,\phi_T), où
\phi_T(t) = \phi(t + T) et I_T est le translaté de I par le
nombre réel - T, est encore une solution de l'équation.

Démonstration En effet, pour t \in I_T, on a t + T \in I et donc

\phi^(n)(t + T) = F(\phi(t + T),\phi'(t +
T),\\ldots,\phi^(n-1)~(t
+ T))

c'est-à-dire \phi_T^(n)(t) =
F(\phi_T(t),\phi_T'(t),\\ldots,\phi_T^(n-1)~(t))

Par la suite, nous nous intéresserons tout particulièrement au cas d'une
équation autonome d'ordre 1 à valeurs dans \mathbb{R}~^2. Dans ce cas,
en introduisant les deux composantes y_1 et y_2 de la
fonction inconnue y et les deux composantes f et g de la fonction F, on
obtient un système différentiel autonome d'ordre 1

\left \ \cases
y_1' = f(y_1,y_2)& \cr
y_2' = g(y_1,y_2)&\\ 
\right .

que l'on pourra encore écrire après un changement de notation

\left \ \cases x' =
f(x,y)& \cr y' = g(x,y)&\\  \right .

Le théorème de Cauchy-Lipschitz pour un tel système peut encore s'écrire
sous la forme

Théorème~16.5.9 Soit U un ouvert de \mathbb{R}~^2, f et g deux
applications de classe \mathcal{C}^1 de U dans \mathbb{R}~ et S le système
différentiel autonome \left \
\cases x' = f(x,y)& \cr y' = g(x,y)&\\
 \right .. Alors (i) (unicité) si deux solutions
(I,(\phi_1,\psi_1)) et (J,(\phi_2,\psi_2))
coïncident en point t_0 \in I \bigcap J, elles coïncident sur I \bigcap J
(ii) (existence locale) pour tout t_0 \in \mathbb{R}~ et tout couple
(x_0,y_0) \in U, il existe \eta > 0 et une
solution (]t_0 - \eta,t_0 + \eta[,(\phi,\psi)) du système
différentiel (S) vérifiant \phi(t_0) =
x_0,\psi(t_0) = y_0 (iii) (solutions maximales)
pour tout t_0 \in \mathbb{R}~ et tout couple (x_0,y_0) \in
U, il existe une unique solution maximale (I,(\phi,\psi)) du système
différentiel (S) vérifiant \phi(t_0) =
x_0,\psi(t_0) = y_0~; I est un intervalle
ouvert.

Comme nous l'avons vu ci dessus, si (I,(\phi,\psi)) est une solution du
système différentiel autonome (S) et si T \in \mathbb{R}~^∗, alors
(I_T,(\phi_T,\psi_T)) est encore une solution du
système différentiel autonome (S). Il est clair que l'une des deux
solutions est maximale si et seulement si l'autre l'est~; dans ce cas,
s'il existe t_0 \in I \bigcap I_T tel que \phi(t_0 + T)
= \phi(t_0) et \psi(t_0 + T) = \psi(t_0), on doit
avoir (I,(\phi,\psi)) = (I_T,(\phi_T,\psi_T)), ce qui
implique que I = \mathbb{R}~ et que \phi et \psi sont périodiques de période T. On en
déduit~:

Théorème~16.5.10 Soit U un ouvert de \mathbb{R}~^2, f et g deux
applications de classe \mathcal{C}^1 de U dans \mathbb{R}~ et (I,(\phi,\psi)) une
solution maximale du système différentiel autonome \left
\ \cases x' = f(x,y)&
\cr y' = g(x,y)&  \right .. S'il existe
t_1,t_2 \in I distincts tels que \phi(t_1) =
\phi(t_2) et \psi(t_1) = \psi(t_2), alors I = \mathbb{R}~ et \phi
et \psi sont périodiques de période t_2 - t_1.

Parmi les solutions d'un système différentiel autonome, on peut
rechercher les constantes t\mapsto~(a,b)~; il est
clair qu'une telle constante est solution si et seulement si f(a,b) =
g(a,b) = 0. Ceci conduit à la définition~:

Définition~16.5.3 On appelle position d'équilibre d'un système autonome
\left \ \cases x' =
f(x,y)& \cr y' = g(x,y)&  \right .
tout couple (a,b) tel que f(a,b) = g(a,b) = 0.

Si (a,b) est une telle position d'équilibre,
(\mathbb{R}~,(t\mapsto~a,t\mathrel\mapsto~b)) est
clairement une solution maximale. Si f et g sont de classe
\mathcal{C}^1, on en déduit que toute solution (I,(\phi,\psi)) qui passe par
cette position d'équilibre, c'est-à-dire pour laquelle il existe
t_0 \in I tel que \phi(t_0) = a,\psi(t_0) = b, est
constante.

Interprétation géométrique des systèmes autonomes~: soit U un ouvert de
\mathbb{R}~^2 et V un champ de vecteurs sur U, c'est-à-dire une
application de V dans \mathbb{R}~^2. Pour (x,y) \in U, on peut écrire V
(x,y) = (f(x,y),g(x,y)). Alors les solutions (I,(\phi,\psi)) du système
autonome \left \ \cases
x' = f(x,y)& \cr y' = g(x,y)&  \right
.sont exactement les arcs paramétrés (I,\Phi) qui admettent en chaque point
m = \Phi(t) de l'image de l'arc un vecteur tangent \Phi'(t) = V (\Phi(t)), valeur
du champ de vecteurs V au point m. Un tel arc paramétré est appelé une
courbe intégrale du champ de vecteurs V .

\subsection{16.5.4 Equations différentielles et formes différentielles}

Définition~16.5.4 Soit \omega = a(t,y)dt + b(t,y)dy une forme différentielle
sur un ouvert U de \mathbb{R}~^2. On appelle solution de l'équation \omega =
0 tout couple (I,(f_1,f_2)) d'un intervalle I de \mathbb{R}~ et
d'une application (f_1,f_2) : I \rightarrow~ \mathbb{R}~^2 de
classe \mathcal{C}^1 tel que (i) \forall~~u \in I,
(f_1(u),f_2(u)) \in U (ii) \forall~~u
\in I, a(f_1(u),f_2(u))f_1'(u) +
b(f_1(u),f_2(u))f_2'(u) = 0

Formellement, (f_1,f_2) est solution de l'équation \omega =
0 si en rempla\ccant t par f_1(u), dt par
f_1'(u) du, y par f_2(u) et dy par f_2'(u)
du, on obtient la forme différentielle nulle. Nous allons à l'aide du
théorème suivant relier les solutions de l'équation a(t,y)dt + b(t,y)dy
= 0 aux solutions de l'équation différentielle a(t,y) + b(t,y) dy
\over dt = 0, obtenue formellement par division par dt
de la forme différentielle.

Théorème~16.5.11 Soit \omega = a(t,y)dt + b(t,y)dy une forme différentielle
sur un ouvert U de \mathbb{R}~^2. (i) Pour toute solution (I,f) de
l'équation différentielle a(t,y) + b(t,y) dy \over dt
= 0, le couple (I,t\mapsto~(t,f(t))) est une
solution de l'équation \omega = 0. (ii) Inversement si
(J,(f_1,f_2)) est une solution de l'équation \omega = 0
telle que \forall~~u \in
J,f_1'(u)\neq~0, alors f_1 est
un difféomorphisme de classe \mathcal{C}^1 de J sur un intervalle I =
f_1(J) de \mathbb{R}~ et le couple (I,f_2 \cdot
f_1^-1) est une solution de l'équation différentielle
a(t,y) + b(t,y) dy \over dt = 0.

Démonstration (i) On a en effet f_1(t) = t et f_2(t) =
f(t), si bien que

\begin{align*}
a(f_1(t),f_2(t))f_1'(t) +
b(f_1(t),f_2(t))f_2'(t)& & \%&
\\ = a(t,f(t)) + b(t,f(t))f'(t) = 0& &
\%& \\ \end{align*}

(ii) Comme f_1' ne s'annule pas sur J et qu'elle est continue,
cette fonction dérivée est de signe constant. Donc f_1 est
strictement monotone et c'est donc un homéomorphisme de J sur un
intervalle I = f_1(J). Comme f_1' ne s'annule pas,
l'application f_1^-1 est elle aussi de classe
\mathcal{C}^1 et donc f_1 est un difféomorphisme. De plus si f
= f_2 \cdot f_1^-1, on a

f'(t) = \left
(f_1^-1\right )'(t)f_
2'(f_1^-1(t)) =
f_2'(f_1^-1(t)) \over
f_1'(f_1^-1(t))

si bien qu'en posant u = f_1^-1(t) et donc t =
f_1(u), f(t) = f_2(u), on a

\begin{align*} a(t,f(t)) + b(t,f(t))f'(t)&& \%&
\\ & =&
a(f_1(u),f_2(u)) +
b(f_1(u),f_2(u)) f_2'(u)
\over f_1'(u) \%&
\\ & =&
a(f_1(u),f_2(u))f_1'(u) +
b(f_1(u),f_2(u))f_2'(u) \over
f_1'(u) = 0\%& \\
\end{align*}

donc (I,f) est solution de l'équation différentielle.

Remarque~16.5.3 Le résultat précédent signifie qu'il est équivalent de
résoudre l'équation différentielle ou de rechercher les solutions de
l'équation \omega = 0 tels que f_1' ne s'annule pas~; autrement dit
les graphes des solutions de l'équation différentielle (c'est-à-dire les
courbes intégrales de l'équation différentielle) sont paramétrés par les
solutions de l'équation \omega = 0. Ceci va nous permettre, plutôt que de
résoudre l'équation différentielle, de résoudre l'équation \omega = 0 et de
rechercher, parmi les arcs paramétrés solutions, ceux qui paramètrent
des graphes d'applications de classe \mathcal{C}^1, c'est-à-dire ceux
tels que f_1' ne s'annule pas.

\subsection{16.5.5 Equations aux différentielles totales}

Théorème~16.5.12 Soit U un ouvert de \mathbb{R}~^2 et F : U \rightarrow~ \mathbb{R}~ de
classe \mathcal{C}^1, \omega = dF = \partial~F \over \partial~t (t,y) dt
+ \partial~F \over \partial~y (t,y) dy. Alors les solutions de
l'équation \omega = 0 sont les applications (f_1,f_2) de
classe \mathcal{C}^1 d'un intervalle I de \mathbb{R}~ dans U telles que
l'application
u\mapsto~F(f_1(u),f_2(u)) soit
constante.

Démonstration On a en effet

\begin{align*} d \over du
F(f_1(u),f_2(u))&& \%&
\\ & =& \partial~F \over \partial~t
(f_1(u),f_2(u)) f_1'(u) + \partial~F
\over \partial~y (f_1(u),f_2(u))
f_2'(u)\%& \\
\end{align*}

qui vaut 0 si et seulement si~f = (f_1,f_2) est
solution de l'équation \omega = 0.

On en déduit que si la forme différentielle a(t,y) dt + b(t,y) dy est la
différentielle d'une fonction F sur U, alors les solutions de l'équation
différentielle a(t,y) + b(t,y)y' = 0 sont les couples (I,f) d'un
intervalle I de \mathbb{R}~ et d'une application f : I \rightarrow~ \mathbb{R}~ tel que l'application
t\mapsto~F(t,f(t)) soit constante. Autrement dit,
les solutions de l'équation différentielle sont définies implicitement
par l'équation F(t,y) = k où k est une constante. Une telle équation
sera dite équation aux différentielles totales~; une telle équation est
donc résoluble de manière implicite (ou même graphique~: il suffit de
tracer les courbes de niveau de la fonction F et d'en rechercher les
parties qui sont des graphes d'applications de classe \mathcal{C}^1)~;
dans certains cas, ceci peut conduire à une résolution explicite.

Exemple~16.5.1 Soit à résoudre l'équation différentielle y + (t - y)y' =
0. Sur l'ouvert U défini par t - y\neq~0, on a y'
= y \over y-t si bien que la théorie de Cauchy
Lipschitz s'applique. La forme différentielle y dt + (t - y) dy est
exacte puisque  \partial~y \over \partial~y = \partial~(t-y)
\over \partial~t . On constate sans difficulté qu'à un facteur
2 près, c'est la différentielle de la fonction F(t,y) = 2ty -
y^2. Autrement dit, les solutions de l'équation
différentielle sont les couples (I,f) tels que f(t)^2 -
2tf(t) = k. Cherchons les solutions telles que f(t_0) =
y_0. Alors nécessairement f(t)^2 - 2tf(t) =
y_0^2 - 2t_0y_0, si bien que nous
pouvons résoudre l'équation du second degré à l'inconnue f(t). Le
discriminant (réduit) est égal à \Delta' = t^2 +
(y_0^2 - 2t_0y_0) et on a donc

f(t) = t ±\sqrtt^2  + y_0 ^2
 - 2t_0  y_0

Cherchons donc les solutions dont le graphe est contenu dans U
c'est-à-dire telles que f(t) - t ne s'annule pas. Alors f(t) - t =
±\sqrtt^2  + y_0 ^2  -
2t_0  y_0 ne s'annule pas, et est donc de signe
constant. On en déduit que le signe ± ne dépend pas de t. De plus, on a
f(t_0) = t_0 ±\sqrtt_0
^2  + y_0 ^2  - 2t_0  y_0 =
t_0 ±y_0 - t_0 ce qui
montre que ce signe est égal à celui de y_0 - t_0, que
nous noterons \epsilon_0. On a donc \forall~~t \in I,
f(t) = t + \epsilon_0\sqrtt^2  +
y_0 ^2  - 2t_0  y_0. Un intervalle
maximal de définition de cette solution dépend évidemment du signe de
l'expression y_0^2 - 2t_0y_0 =
y_0(y_0 - 2t_0). Si cette expression est
positive, la solution est définie sur \mathbb{R}~ tout entier. Si cette expression
est négative, la solution est définie sur celui des intervalles ]
-\infty~,-\sqrtt_0 ^2  + 2t_0 
y_0[ ou ]\sqrtt_0 ^2  +
2t_0  y_0,+\infty~[ qui contient t_0 (ce qui
dépend bien entendu du signe de t_0). La figure ci dessous
représente un aper\ccu des courbes intégrales de
l'équation différentielle avec les deux droites séparatrices
y_0 = 0 et y_0 = 2t_0. On remarquera que par
les points vérifiant y_0 = t_0 passent deux courbes
intégrales (mais qui en fait ne sont pas dérivables en ce point)~: la
moitié supérieure de la branche d'hyperbole et la moitié inférieure~;
les autres courbes intégrales correspondent à des solutions définies sur
\mathbb{R}~ tout entier.

\includegraphics{cours10x.png}

\subsection{16.5.6 Equations à variables séparables}

Définition~16.5.5 On appelle équation à variables séparées une équation
du type a(t) = b(y)y', où a : I \rightarrow~ \mathbb{R}~ et b : J \rightarrow~ \mathbb{R}~ sont continues.

La forme différentielle correspondante est la forme différentielle \omega =
a(t) dt - b(y) dy. Si A désigne une primitive de a sur I et B une
primitive de b sur J, on a \omega = dF avec F(t,y) = A(t) - B(y). Une
solution (I_0,f) vérifiant f(t_0) = y_0 doit
donc vérifier A(t) - B(f(t)) = A(t_0) - B(y_0), où
encore en termes d'intégrales \int ~
_t_0^ta(u) du =\int ~
_y_0^f(t)b(v) dv. Si b = B' ne s'annule pas, B
est un difféomorphisme de J sur B(J) et la solution (I_0,f) est
donnée par f(t) = B^-1\left (A(t) -
A(t_0) + B(y_0)\right ).

Remarque~16.5.4 On appelle équation à variables séparables une équation
se ramenant par produits et quotients à une équation à variables
séparées. On prendra garde au fait que la séparation des variables est
une opération à risque. La multiplication par un facteur pouvant
s'annuler risque de faire apparaître de fausses solutions de l'équation
différentielle. Quant à la division par des facteurs pouvant s'annuler,
elle risque au contraire de faire disparaître des solutions annulant ces
facteurs.

\subsection{16.5.7 Equations se ramenant à des équations à variables
séparables}

Certaines équations sont connues comme pouvant se ramener à des
équations à variables séparables, ce sont les équations incomplètes
(c'est-à-dire où l'un des termes t ou y n'apparaît pas) et les équations
homogènes (c'est à dire les équations ne dépendant que de y' et de  y
\over t ).

Equations sous formes normales

Nous nous intéresserons tout d'abord aux équations sous forme normale.
Les équations incomplètes sont alors de deux types~: soit y' = F(t) soit
y' = F(y). La première se ramène à un simple calcul de primitive. La
seconde est une équation à variables séparables~: la division par F(y)
sépare les deux variables~; cependant cette division n'est possible que
pour des solutions t\mapsto~y(t) telles que F(y(t))
ne s'annule pas. Cherchons tout d'abord ces solutions~: l'équation
s'écrit alors  dy \over F(y) = dt ce qui conduit pour
une solution vérifiant \phi(t_0) = y_0 à
\int  _y_0^\phi(t)~ du
\over F(u) = t - t_0~; on obtient donc ainsi t
en fonction de y = \phi(t) et il ne reste plus qu'à inverser la fonction
obtenue. Inversement, si F(y_0) = 0, il est clair que la
fonction constante t\mapsto~y_0 est
solution. Dans le cas où le théorème de Cauchy-Lipschitz s'applique (par
exemple si F est de classe \mathcal{C}^1), on est certain d'avoir
obtenu ainsi toutes les solutions.

Remarque~16.5.5 Les équations incomplètes du type y' = F(y) présentent
une certaine invariance par translation~: si \phi en est une solution,
toutes les t\mapsto~\phi(t + T) en sont également
solutions.

Une équation homogène sous forme normale se présente nécessairement sous
la forme y' = F( y \over t ), t variant soit dans ]
-\infty~,0[, soit dans ]0,+\infty~[. Faisons le changement de fonction
inconnue y = tz. On obtient alors tz' + z = F(z) soit encore tz' = F(z)
- z. Il s'agit d'une équation à variables séparables. La séparation des
variables conduisant à l'équation  dz \over F(z)-z
= dt \over t nécessite une division par t qui de
toute fa\ccon ne s'annule pas sur les intervalles
considérés, mais aussi une division par F(z) - z qui lui peut s'annuler.
Les solutions t\mapsto~z(t) telles que F(z(t)) -
z(t) ne s'annulent pas sont données par \int ~
_z_0^z du \over F(u)-u
= log~ \left  t
\over t_0 \right ,
avec z(t) = y(t) \over t et z_0 =
y_0 \over t_0 . Par contre, si
F(z_0) - z_0 = 0, il est clair que la fonction
constante t\mapsto~z_0 est solution de tz'
= F(z) - z et donc t\mapsto~tz_0 est
solution de y' = F( y \over t ). Dans le cas où le
théorème de Cauchy-Lipschitz s'applique (par exemple si F est de classe
\mathcal{C}^1), on est certain d'avoir obtenu ainsi toutes les
solutions.

Remarque~16.5.6 Les équations homogènes du type y' = F( y
\over t ) présentent une invariance par homothétie~: si
\phi en est une solution, toutes les
t\mapsto~\lambda~\phi\left ( t
\over \lambda~ \right ) en sont également
solutions.

Equations sous formes non normales

Les équations incomplètes se présentent sous la forme F(t,y') = 0 ou
F(y,y') = 0, les équations homogènes sous la forme F( y
\over t ,y') = 0. Dans tous les cas, on supposera que
l'on sait paramétrer la courbe F(X,Y ) = 0 par X = \phi(u) et Y = \psi(u). On
cherchera alors à paramétrer les courbes intégrales de l'équation
différentielle en fonction de u ce qui peut se faire simplement en
termes de formes différentielles.

\begin{itemize}
\itemsep1pt\parskip0pt\parsep0pt
\item
  F(t,y') = 0. On écrit t = \phi(u),  dy \over dt =
  \psi(u)~; ceci conduit à dy = \psi(u) dt = \psi(u)\phi'(u) du. Un simple calcul de
  primitive conduit à y = \Psi(u) et les courbes intégrales sont
  paramétrées par t = \phi(u), y = \Psi(u).
\item
  F(y,y') = 0. On écrit y = \phi(u),  dy \over dt =
  \psi(u)~; ceci conduit à dy = \phi'(u) du = \psi(u) dt qui est une équation à
  variable séparable entre u et t~; si \psi(u) ne s'annule pas, un calcul
  de primitive conduit à t = \Phi(u) et les courbes intégrales sont
  paramétrées par t = \Phi(u), y = \psi(u). Si \phi(u_0) = 0, on obtient
  également comme solution la fonction constante
  t\mapsto~\phi(u_0).
\item
  F( y \over t ,y') = 0. On écrit  y
  \over t = \phi(u),  dy \over dt =
  \psi(u)~; ceci conduit à y = t\phi(u), d'où dy = \phi(u) dt + t\phi'(u) du = \psi(u)
  dt qui est une équation à variable séparable entre u et t que l'on
  peut écrire (\psi(u) - \phi(u)) dt = t\phi'(u) du. Comme t ne s'annule pas, si
  \psi - \phi ne s'annule pas, un calcul de primitive conduit à t = \Phi(u) et
  les courbes intégrales sont paramétrées par t = \Phi(u), y = t\psi(u) =
  \Phi(u)\psi(u). Si \psi(u_0) - \phi(u_0) = 0, on obtient
  également comme solution la fonction affine y = t\phi(u_0).
\end{itemize}

Remarque~16.5.7 Le lecteur vérifiera sans difficulté que l'ensemble des
courbes intégrales d'une équation incomplète est invariant par
translation parallèlement à l'un des axes et que l'ensemble des courbes
intégrales d'une équation homogène est invariant par homothétie de
centre O.

\subsection{16.5.8 Equation de Riccati}

On considère une équation différentielle du type y' = a(t)y^2
+ b(t)y + c(t), où b et c sont des applications continues de I dans \mathbb{R}~ et
a une application de classe \mathcal{C}^1 de I dans \mathbb{R}~ ne s'annulant
pas. Faisons le changement de fonction inconnue z
= exp (-\\int ~ a(t)y(t)
dt) soit encore a(t)y = - z' \over z où z est une
fonction de classe C^2 ne s'annulant pas. On obtient alors y'
= - 1 \over a(t)  z'`\over z + 1
\over a(t)  z'^2 \over
z^2 + a'(t) \over a(t)^2 
z' \over z si bien que l'équation devient

\begin{align*} - 1 \over a(t) 
z'`\over z + 1 \over a(t) 
z'^2 \over z^2 + a'(t)
\over a(t)^2  z' \over z
= 1 \over a(t)  z'^2
\over z^2 - b(t) \over
a(t)  z' \over z + c(t)& & \%&
\\ \end{align*}

soit encore après multiplication par a(t)z,

z'`-\left ( a'(t) \over a(t) +
b(t)\right )z' + a(t)c(t)z = 0

qui est une équation différentielle homogène d'ordre 2.

Inversement, étant donnée une équation homogène d'ordre 2, z'`+ p(t)z' +
q(t)z = 0, p et q étant des fonctions continues de I dans \mathbb{R}~, si l'on
recherche les solutions ne s'annulant pas sous la forme z =
e^u, on a z' = u'e^u et z'`= (u'`+
u'^2)e^u et donc, en posant y = u',

\begin{align*} z'`+ p(t)z' + q(t)z = 0&
\Leftrightarrow & u'`+ u'^2 + p(t)u' + q(t) =
0\%& \\ & \Leftrightarrow
& y' = -y^2 + p(t)y + q(t) \%&
\\ \end{align*}

qui est une équation de Riccati.

On voit donc en définitive que la résolution des équations de Riccati y'
= a(t)y^2 + b(t)y + c(t) est équivalente à la recherche des
solutions ne s'annulant pas des équations différentielles homogènes
d'ordre 2.
