\documentclass[]{article}
\usepackage[T1]{fontenc}
\usepackage{lmodern}
\usepackage{amssymb,amsmath}
\usepackage{ifxetex,ifluatex}
\usepackage{fixltx2e} % provides \textsubscript
% use upquote if available, for straight quotes in verbatim environments
\IfFileExists{upquote.sty}{\usepackage{upquote}}{}
\ifnum 0\ifxetex 1\fi\ifluatex 1\fi=0 % if pdftex
  \usepackage[utf8]{inputenc}
\else % if luatex or xelatex
  \ifxetex
    \usepackage{mathspec}
    \usepackage{xltxtra,xunicode}
  \else
    \usepackage{fontspec}
  \fi
  \defaultfontfeatures{Mapping=tex-text,Scale=MatchLowercase}
  \newcommand{\euro}{€}
\fi
% use microtype if available
\IfFileExists{microtype.sty}{\usepackage{microtype}}{}
\ifxetex
  \usepackage[setpagesize=false, % page size defined by xetex
              unicode=false, % unicode breaks when used with xetex
              xetex]{hyperref}
\else
  \usepackage[unicode=true]{hyperref}
\fi
\hypersetup{breaklinks=true,
            bookmarks=true,
            pdfauthor={},
            pdftitle={Series absolument convergentes},
            colorlinks=true,
            citecolor=blue,
            urlcolor=blue,
            linkcolor=magenta,
            pdfborder={0 0 0}}
\urlstyle{same}  % don't use monospace font for urls
\setlength{\parindent}{0pt}
\setlength{\parskip}{6pt plus 2pt minus 1pt}
\setlength{\emergencystretch}{3em}  % prevent overfull lines
\setcounter{secnumdepth}{0}
 
/* start css.sty */
.cmr-5{font-size:50%;}
.cmr-7{font-size:70%;}
.cmmi-5{font-size:50%;font-style: italic;}
.cmmi-7{font-size:70%;font-style: italic;}
.cmmi-10{font-style: italic;}
.cmsy-5{font-size:50%;}
.cmsy-7{font-size:70%;}
.cmex-7{font-size:70%;}
.cmex-7x-x-71{font-size:49%;}
.msbm-7{font-size:70%;}
.cmtt-10{font-family: monospace;}
.cmti-10{ font-style: italic;}
.cmbx-10{ font-weight: bold;}
.cmr-17x-x-120{font-size:204%;}
.cmsl-10{font-style: oblique;}
.cmti-7x-x-71{font-size:49%; font-style: italic;}
.cmbxti-10{ font-weight: bold; font-style: italic;}
p.noindent { text-indent: 0em }
td p.noindent { text-indent: 0em; margin-top:0em; }
p.nopar { text-indent: 0em; }
p.indent{ text-indent: 1.5em }
@media print {div.crosslinks {visibility:hidden;}}
a img { border-top: 0; border-left: 0; border-right: 0; }
center { margin-top:1em; margin-bottom:1em; }
td center { margin-top:0em; margin-bottom:0em; }
.Canvas { position:relative; }
li p.indent { text-indent: 0em }
.enumerate1 {list-style-type:decimal;}
.enumerate2 {list-style-type:lower-alpha;}
.enumerate3 {list-style-type:lower-roman;}
.enumerate4 {list-style-type:upper-alpha;}
div.newtheorem { margin-bottom: 2em; margin-top: 2em;}
.obeylines-h,.obeylines-v {white-space: nowrap; }
div.obeylines-v p { margin-top:0; margin-bottom:0; }
.overline{ text-decoration:overline; }
.overline img{ border-top: 1px solid black; }
td.displaylines {text-align:center; white-space:nowrap;}
.centerline {text-align:center;}
.rightline {text-align:right;}
div.verbatim {font-family: monospace; white-space: nowrap; text-align:left; clear:both; }
.fbox {padding-left:3.0pt; padding-right:3.0pt; text-indent:0pt; border:solid black 0.4pt; }
div.fbox {display:table}
div.center div.fbox {text-align:center; clear:both; padding-left:3.0pt; padding-right:3.0pt; text-indent:0pt; border:solid black 0.4pt; }
div.minipage{width:100%;}
div.center, div.center div.center {text-align: center; margin-left:1em; margin-right:1em;}
div.center div {text-align: left;}
div.flushright, div.flushright div.flushright {text-align: right;}
div.flushright div {text-align: left;}
div.flushleft {text-align: left;}
.underline{ text-decoration:underline; }
.underline img{ border-bottom: 1px solid black; margin-bottom:1pt; }
.framebox-c, .framebox-l, .framebox-r { padding-left:3.0pt; padding-right:3.0pt; text-indent:0pt; border:solid black 0.4pt; }
.framebox-c {text-align:center;}
.framebox-l {text-align:left;}
.framebox-r {text-align:right;}
span.thank-mark{ vertical-align: super }
span.footnote-mark sup.textsuperscript, span.footnote-mark a sup.textsuperscript{ font-size:80%; }
div.tabular, div.center div.tabular {text-align: center; margin-top:0.5em; margin-bottom:0.5em; }
table.tabular td p{margin-top:0em;}
table.tabular {margin-left: auto; margin-right: auto;}
div.td00{ margin-left:0pt; margin-right:0pt; }
div.td01{ margin-left:0pt; margin-right:5pt; }
div.td10{ margin-left:5pt; margin-right:0pt; }
div.td11{ margin-left:5pt; margin-right:5pt; }
table[rules] {border-left:solid black 0.4pt; border-right:solid black 0.4pt; }
td.td00{ padding-left:0pt; padding-right:0pt; }
td.td01{ padding-left:0pt; padding-right:5pt; }
td.td10{ padding-left:5pt; padding-right:0pt; }
td.td11{ padding-left:5pt; padding-right:5pt; }
table[rules] {border-left:solid black 0.4pt; border-right:solid black 0.4pt; }
.hline hr, .cline hr{ height : 1px; margin:0px; }
.tabbing-right {text-align:right;}
span.TEX {letter-spacing: -0.125em; }
span.TEX span.E{ position:relative;top:0.5ex;left:-0.0417em;}
a span.TEX span.E {text-decoration: none; }
span.LATEX span.A{ position:relative; top:-0.5ex; left:-0.4em; font-size:85%;}
span.LATEX span.TEX{ position:relative; left: -0.4em; }
div.float img, div.float .caption {text-align:center;}
div.figure img, div.figure .caption {text-align:center;}
.marginpar {width:20%; float:right; text-align:left; margin-left:auto; margin-top:0.5em; font-size:85%; text-decoration:underline;}
.marginpar p{margin-top:0.4em; margin-bottom:0.4em;}
.equation td{text-align:center; vertical-align:middle; }
td.eq-no{ width:5%; }
table.equation { width:100%; } 
div.math-display, div.par-math-display{text-align:center;}
math .texttt { font-family: monospace; }
math .textit { font-style: italic; }
math .textsl { font-style: oblique; }
math .textsf { font-family: sans-serif; }
math .textbf { font-weight: bold; }
.partToc a, .partToc, .likepartToc a, .likepartToc {line-height: 200%; font-weight:bold; font-size:110%;}
.chapterToc a, .chapterToc, .likechapterToc a, .likechapterToc, .appendixToc a, .appendixToc {line-height: 200%; font-weight:bold;}
.index-item, .index-subitem, .index-subsubitem {display:block}
.caption td.id{font-weight: bold; white-space: nowrap; }
table.caption {text-align:center;}
h1.partHead{text-align: center}
p.bibitem { text-indent: -2em; margin-left: 2em; margin-top:0.6em; margin-bottom:0.6em; }
p.bibitem-p { text-indent: 0em; margin-left: 2em; margin-top:0.6em; margin-bottom:0.6em; }
.subsectionHead, .likesubsectionHead { margin-top:2em; font-weight: bold;}
.sectionHead, .likesectionHead { font-weight: bold;}
.quote {margin-bottom:0.25em; margin-top:0.25em; margin-left:1em; margin-right:1em; text-align:justify;}
.verse{white-space:nowrap; margin-left:2em}
div.maketitle {text-align:center;}
h2.titleHead{text-align:center;}
div.maketitle{ margin-bottom: 2em; }
div.author, div.date {text-align:center;}
div.thanks{text-align:left; margin-left:10%; font-size:85%; font-style:italic; }
div.author{white-space: nowrap;}
.quotation {margin-bottom:0.25em; margin-top:0.25em; margin-left:1em; }
h1.partHead{text-align: center}
.sectionToc, .likesectionToc {margin-left:2em;}
.subsectionToc, .likesubsectionToc {margin-left:4em;}
.sectionToc, .likesectionToc {margin-left:6em;}
.frenchb-nbsp{font-size:75%;}
.frenchb-thinspace{font-size:75%;}
.figure img.graphics {margin-left:10%;}
/* end css.sty */

\title{Series absolument convergentes}
\author{}
\date{}

\begin{document}
\maketitle

\textbf{Warning: 
requires JavaScript to process the mathematics on this page.\\ If your
browser supports JavaScript, be sure it is enabled.}

\begin{center}\rule{3in}{0.4pt}\end{center}

[
[
[]
[

\section{7.4 Séries absolument convergentes}

\subsection{7.4.1 Notion de convergence absolue}

Définition~7.4.1 Soit E un espace vectoriel normé. On dit que la série
\\sum  x_n~ est
absolument convergente si la série à termes réels positifs
\\sum ~
\x_n\
converge.

Théorème~7.4.1 Soit E un espace vectoriel normé~complet. Alors toute
série absolument convergente à terme général dans E est convergente.

Démonstration On a
\\\\sum
 _n=p^qx_n\
\leq\\sum ~
_n=p^q\x_n\.
Si la série \\sum ~
\x_n\
converge, elle vérifie le critère de Cauchy, il en est donc de même de
la série \\sum ~
x_n et donc celle-ci converge.

Remarque~7.4.1 L'avantage est bien entendu de ramener l'étude à celle
d'une série à termes réels positifs.

\subsection{7.4.2 Critères de convergence absolue}

Théorème~7.4.2 Soit \\\sum
 x_n et \\\sum
 y_n deux séries telles que x_n = O(y_n)
et \\sum  y_n~
est absolument convergente. Alors
\\sum  x_n~
converge absolument.

Démonstration On a x_n = O(y_n)
\Leftrightarrow
\x_n\ =
O(\y_n\) et
il suffit d'appliquer le théorème de comparaison pour les séries à
termes réels positifs.

Remarque~7.4.2 Le théorème ci-dessus reste valable même si les termes
généraux x_n et y_n ne sont pas dans le même espace
vectoriel normé. En particulier, la série étalon
\\sum  y_n~ sera
le plus souvent une série à termes réels positifs.

En ce qui concerne les équivalents, on a un résultat plus fort

Théorème~7.4.3 Soit \\\sum
 x_n une série à terme général dans l'espace vectoriel
normé~E et \\sum ~
y_n une série à termes réels positifs. On suppose qu'il existe
\ell \in E \diagdown\0\ tel que x_n ∼
\elly_n. Alors les deux séries sont simultanément convergentes ou
divergentes.

Démonstration Si \\sum ~
y_n converge, on a x_n = O(y_n) et
y_n ≥ 0, donc la série
\\sum  x_n~ est
absolument convergente. Inversement, supposons que la série
\\sum  x_n~
converge. Puisque x_n - \elly_n = o(\elly_n), il
existe N \in \mathbb{N}~ tel que n ≥ N \rigtharrow~\ x_n -
\elly_n\ \leq 1 \over 2
\\elly_n\ = 1
\over 2
\\ell\y_n. En
sommant on obtient
\\\\sum
 _n=p^qx_n -
\ell\\sum ~
_n=p^qy_n\ \leq 1
\over 2
\\ell\\\\sum
 _n=p^qy_n. On en déduit

\begin{align*}
\\ell\\\sum
_n=p^qy_ n& =&
\\ell\\sum
_n=p^qy_ n\
\leq\ \ell\\sum
_n=p^qy_ n -\\sum
_n=p^qx_ n\
+\ \\sum
_n=p^qx_ n\\%&
\\ & \leq& 1 \over 2
\\ell\\\sum
_n=p^qy_ n +\
\sum _n=p^qx_
n\ \%& \\
\end{align*}

d'où en définitive \\\sum
 _n=p^qy_n \leq 2 \over
\\ell\
\\
\sum ~
_n=p^qx_n\. La série
\\sum  x_n~
converge, donc vérifie le critère de Cauchy. Il en est donc de même de
la série \\sum ~
y_n, qui est par suite convergente.

\subsection{7.4.3 Règles classiques}

Il suffit maintenant d'appliquer ces résultats à des séries étalons,
comme les séries de Riemann ou les séries géométriques.

Lemme~7.4.4 Soit a un nombre complexe. La série
\\sum  a^n~
converge si et seulement si~a < 1.

Démonstration La condition est évidemment nécessaire puisque le terme
général doit tendre vers 0. Supposons la vérifiée. On a
\\sum ~
_p=0^na^p = 1-a^n+1
\over 1-a qui admet la limite  1 \over
1-a . Donc la série converge.

Théorème~7.4.5 (règle de d'Alembert). Soit E un espace vectoriel normé
complet. Soit \\sum ~
x_n une série à termes dans E telle que pour tout n \in \mathbb{N}~,
x_n\neq~0 et telle que la suite (
\x_n+1\
\over
\x_n\ )
admet une limite \ell \in \mathbb{R}~ \cup\ + \infty~\. Alors

\begin{itemize}
\itemsep1pt\parskip0pt\parsep0pt
\item
  (i) si \ell < 1, la série converge absolument
\item
  (ii) si \ell > 1, la série diverge
\end{itemize}

Démonstration (i) Si \ell < 1, soit \rho tel que \ell < \rho
< 1~; il existe N \in \mathbb{N}~ tel que n ≥ N \rigtharrow~
\x_n+1\
\over
\x_n\ \leq \rho
soit \x_n+1\
\leq \rho\x_n\. On
a donc alors par récurrence
\x_n\ \leq
\rho^n-N\x_N\
= O(\rho^n). Comme la série
\\sum  \rho^n~
converge, la série \\\sum
 x_n converge absolument.

(ii) Si \ell > 1, il existe N \in \mathbb{N}~ tel que n ≥ N \rigtharrow~
\x_n+1\
\over
\x_n\
> 1 soit
\x_n+1\
>\
x_n\. On a donc alors par récurrence
\x_n\
>\
x_N\. La suite (x_n) ne peut
donc pas avoir 0 pour limite et la série diverge.

Remarque~7.4.3 Si \ell = 1 on ne peut rien conclure comme le montre
l'exemple des séries de Riemann. Lorsque la règle de d'Alembert
s'applique, elle conduit à des convergences rapides (de type
exponentielle) ou des divergences grossières (le terme général ne tend
pas vers 0).

Théorème~7.4.6 (règle de Riemann). Soit E un espace vectoriel normé.
Soit \\sum  x_n~
une série à termes dans E.

\begin{itemize}
\itemsep1pt\parskip0pt\parsep0pt
\item
  (i) S'il existe \alpha~ > 1 tel que x_n = O( 1
  \over n^\alpha~ ), alors la série converge
  absolument
\item
  (ii) S'il existe \alpha~ \in \mathbb{R}~ et \ell \in E \diagdown\0\
  tels que x_n ∼ \ell \over n^\alpha~
  alors la série converge absolument si \alpha~ > 1 et diverge si
  \alpha~ \leq 1.
\item
  (iii) Si E = \mathbb{R}~ et x_n ≥ 0, et s'il existe \alpha~ \leq 1 et \ell
  > 0 (y compris + \infty~) tel que
  limn^\alpha~x_n~ = \ell, alors la
  série diverge.
\end{itemize}

Démonstration (i) et (ii) résultent de ce qui précède. Pour (iii), il
suffit de remarquer que les hypothèses impliquent que  1
\over n^\alpha~ = O(x_n). Comme \alpha~ \leq 1, la
série \\sum ~  1
\over n^\alpha~ diverge et donc aussi la série
\\sum  x_n~.

\subsection{7.4.4 Règles complémentaires}

Théorème~7.4.7 (règle de Cauchy). Soit E un espace vectoriel normé
complet. Soit \\sum ~
x_n une série à termes dans E telle que la suite
\left
(\rootn\of\x_n\\right
) admet une limite \ell \in \mathbb{R}~ \cup\ + \infty~\.
Alors

\begin{itemize}
\itemsep1pt\parskip0pt\parsep0pt
\item
  (i) si \ell < 1, la série converge absolument
\item
  (ii) si \ell > 1, la série diverge
\end{itemize}

Démonstration (i) Si \ell < 1, soit \rho tel que \ell < \rho
< 1~; il existe N \in \mathbb{N}~ tel que n ≥ N
\rigtharrow~\rootn\of\x_n\
\leq \rho soit
\x_n\ \leq
\rho^n. Comme la série
\\sum  \rho^n~
converge, la série \\\sum
 x_n converge absolument.

(ii) Si \ell > 1, il existe N \in \mathbb{N}~ tel que n ≥ N
\rigtharrow~\rootn\of\x_n\
> 1 soit
\x_n\
> 1. La suite (x_n) ne peut donc pas avoir 0 pour
limite et la série diverge.

Théorème~7.4.8 (règle de Duhamel). Soit
\\sum  x_n~ une
série à termes dans \mathbb{R}~^+ telle que pour tout n \in \mathbb{N}~,
x_n\neq~0 et telle que  x_n+1
\over x_n = 1 - \lambda~ \over n +
o( 1 \over n ) . Alors

\begin{itemize}
\itemsep1pt\parskip0pt\parsep0pt
\item
  (i) si \lambda~ > 1, la série converge
\item
  (ii) si \lambda~ < 1, la série diverge
\end{itemize}

Démonstration Posons y_n = 1 \over
n^\alpha~ . On a  y_n+1 \over
y_n = 1 - \alpha~ \over n + o( 1
\over n ). On en déduit que si
\alpha~\neq~\lambda~,  x_n+1 \over
x_n - y_n+1 \over y_n
∼ \alpha~-\lambda~ \over n est pour n assez grand du signe de \alpha~ -
\lambda~. Si \lambda~ < 1, soit \alpha~ tel que \lambda~ < \alpha~ < 1. On
a donc pour n ≥ N,  x_n+1 \over x_n
≥ y_n+1 \over y_n et comme la série
\\sum  y_n~
diverge (car \alpha~ < 1), la série
\\sum  x_n~
diverge. Si \lambda~ > 1, soit \alpha~ tel que \lambda~ > \alpha~
> 1. On a donc pour n ≥ N,  x_n+1
\over x_n \leq y_n+1
\over y_n et comme la série
\\sum  y_n~
converge (car \alpha~ > 1), la série
\\sum  x_n~
converge.

\subsection{7.4.5 Comparaison à une intégrale}

Théorème~7.4.9 Soit f : [0,+\infty~[\rightarrow~ \mathbb{C} de classe \mathcal{C}^1 telle que
f' soit intégrable sur [0,+\infty~[. Posons w_n
=\int  _n-1^n~f(t) dt - f(n).
Alors la série \\sum ~
w_n est absolument convergente.

Démonstration On a par une intégration par parties

\begin{align*} \int ~
_n-1^n(t - n + 1)f'(t) dt& =& \left
[(t - n + 1)f(t)\right ]_ n-1^n
-\int  _n-1^n~f(t) dt\%&
\\ & =& -w_n \%&
\\ \end{align*}

On en déduit que

w_n\leq\int ~
_n-1^n(t - n + 1)f'(t) dt
\leq\int ~
_n-1^nf'(t) dt

et donc

\sum _p=1^nw_
p\leq\\int  ~
_0^nf'(t) dt
\leq\\int  ~
_0^+\infty~f'(t) dt

ce qui montre la convergence de la série à termes positifs
\\sum ~
w_n et donc la convergence absolue de la
série.

Corollaire~7.4.10 Soit f : [0,+\infty~[\rightarrow~ \mathbb{C} de classe \mathcal{C}^1 telle
que f et f' soient intégrables sur [0,+\infty~[. Alors la série
\\sum ~ f(n) est
absolument convergente.

Démonstration En effet la série
\\sum ~
\int ~
_n-1^nf(t) dt est convergente car

\sum _p=1^n~
\\int  ~
_n-1^nf(t) dt =
\\int  ~
_0^nf(t) dt
\leq\\int  ~
_0^+\infty~f(t) dt

et comme \left \int ~
_n-1^nf(t) dt\right
\leq\int ~
_n-1^nf(t) dt, la série
\\sum ~
\int  _n-1^n~f(t) dt est
absolument convergente. Comme
\\sum  w_n~ est
également absolument convergente, il en est de même de la série
\\sum ~ f(n).

[
[
[
[

\end{document}
