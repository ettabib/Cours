\documentclass[]{article}
\usepackage[T1]{fontenc}
\usepackage{lmodern}
\usepackage{amssymb,amsmath}
\usepackage{ifxetex,ifluatex}
\usepackage{fixltx2e} % provides \textsubscript
% use upquote if available, for straight quotes in verbatim environments
\IfFileExists{upquote.sty}{\usepackage{upquote}}{}
\ifnum 0\ifxetex 1\fi\ifluatex 1\fi=0 % if pdftex
  \usepackage[utf8]{inputenc}
\else % if luatex or xelatex
  \ifxetex
    \usepackage{mathspec}
    \usepackage{xltxtra,xunicode}
  \else
    \usepackage{fontspec}
  \fi
  \defaultfontfeatures{Mapping=tex-text,Scale=MatchLowercase}
  \newcommand{\euro}{€}
\fi
% use microtype if available
\IfFileExists{microtype.sty}{\usepackage{microtype}}{}
\ifxetex
  \usepackage[setpagesize=false, % page size defined by xetex
              unicode=false, % unicode breaks when used with xetex
              xetex]{hyperref}
\else
  \usepackage[unicode=true]{hyperref}
\fi
\hypersetup{breaklinks=true,
            bookmarks=true,
            pdfauthor={},
            pdftitle={A propos de Jordan},
            colorlinks=true,
            citecolor=blue,
            urlcolor=blue,
            linkcolor=magenta,
            pdfborder={0 0 0}}
\urlstyle{same}  % don't use monospace font for urls
\setlength{\parindent}{0pt}
\setlength{\parskip}{6pt plus 2pt minus 1pt}
\setlength{\emergencystretch}{3em}  % prevent overfull lines
\setcounter{secnumdepth}{0}
 
/* start css.sty */
.cmr-5{font-size:50%;}
.cmr-7{font-size:70%;}
.cmmi-5{font-size:50%;font-style: italic;}
.cmmi-7{font-size:70%;font-style: italic;}
.cmmi-10{font-style: italic;}
.cmsy-5{font-size:50%;}
.cmsy-7{font-size:70%;}
.cmex-7{font-size:70%;}
.cmex-7x-x-71{font-size:49%;}
.msbm-7{font-size:70%;}
.cmtt-10{font-family: monospace;}
.cmti-10{ font-style: italic;}
.cmbx-10{ font-weight: bold;}
.cmr-17x-x-120{font-size:204%;}
.cmsl-10{font-style: oblique;}
.cmti-7x-x-71{font-size:49%; font-style: italic;}
.cmbxti-10{ font-weight: bold; font-style: italic;}
p.noindent { text-indent: 0em }
td p.noindent { text-indent: 0em; margin-top:0em; }
p.nopar { text-indent: 0em; }
p.indent{ text-indent: 1.5em }
@media print {div.crosslinks {visibility:hidden;}}
a img { border-top: 0; border-left: 0; border-right: 0; }
center { margin-top:1em; margin-bottom:1em; }
td center { margin-top:0em; margin-bottom:0em; }
.Canvas { position:relative; }
li p.indent { text-indent: 0em }
.enumerate1 {list-style-type:decimal;}
.enumerate2 {list-style-type:lower-alpha;}
.enumerate3 {list-style-type:lower-roman;}
.enumerate4 {list-style-type:upper-alpha;}
div.newtheorem { margin-bottom: 2em; margin-top: 2em;}
.obeylines-h,.obeylines-v {white-space: nowrap; }
div.obeylines-v p { margin-top:0; margin-bottom:0; }
.overline{ text-decoration:overline; }
.overline img{ border-top: 1px solid black; }
td.displaylines {text-align:center; white-space:nowrap;}
.centerline {text-align:center;}
.rightline {text-align:right;}
div.verbatim {font-family: monospace; white-space: nowrap; text-align:left; clear:both; }
.fbox {padding-left:3.0pt; padding-right:3.0pt; text-indent:0pt; border:solid black 0.4pt; }
div.fbox {display:table}
div.center div.fbox {text-align:center; clear:both; padding-left:3.0pt; padding-right:3.0pt; text-indent:0pt; border:solid black 0.4pt; }
div.minipage{width:100%;}
div.center, div.center div.center {text-align: center; margin-left:1em; margin-right:1em;}
div.center div {text-align: left;}
div.flushright, div.flushright div.flushright {text-align: right;}
div.flushright div {text-align: left;}
div.flushleft {text-align: left;}
.underline{ text-decoration:underline; }
.underline img{ border-bottom: 1px solid black; margin-bottom:1pt; }
.framebox-c, .framebox-l, .framebox-r { padding-left:3.0pt; padding-right:3.0pt; text-indent:0pt; border:solid black 0.4pt; }
.framebox-c {text-align:center;}
.framebox-l {text-align:left;}
.framebox-r {text-align:right;}
span.thank-mark{ vertical-align: super }
span.footnote-mark sup.textsuperscript, span.footnote-mark a sup.textsuperscript{ font-size:80%; }
div.tabular, div.center div.tabular {text-align: center; margin-top:0.5em; margin-bottom:0.5em; }
table.tabular td p{margin-top:0em;}
table.tabular {margin-left: auto; margin-right: auto;}
div.td00{ margin-left:0pt; margin-right:0pt; }
div.td01{ margin-left:0pt; margin-right:5pt; }
div.td10{ margin-left:5pt; margin-right:0pt; }
div.td11{ margin-left:5pt; margin-right:5pt; }
table[rules] {border-left:solid black 0.4pt; border-right:solid black 0.4pt; }
td.td00{ padding-left:0pt; padding-right:0pt; }
td.td01{ padding-left:0pt; padding-right:5pt; }
td.td10{ padding-left:5pt; padding-right:0pt; }
td.td11{ padding-left:5pt; padding-right:5pt; }
table[rules] {border-left:solid black 0.4pt; border-right:solid black 0.4pt; }
.hline hr, .cline hr{ height : 1px; margin:0px; }
.tabbing-right {text-align:right;}
span.TEX {letter-spacing: -0.125em; }
span.TEX span.E{ position:relative;top:0.5ex;left:-0.0417em;}
a span.TEX span.E {text-decoration: none; }
span.LATEX span.A{ position:relative; top:-0.5ex; left:-0.4em; font-size:85%;}
span.LATEX span.TEX{ position:relative; left: -0.4em; }
div.float img, div.float .caption {text-align:center;}
div.figure img, div.figure .caption {text-align:center;}
.marginpar {width:20%; float:right; text-align:left; margin-left:auto; margin-top:0.5em; font-size:85%; text-decoration:underline;}
.marginpar p{margin-top:0.4em; margin-bottom:0.4em;}
.equation td{text-align:center; vertical-align:middle; }
td.eq-no{ width:5%; }
table.equation { width:100%; } 
div.math-display, div.par-math-display{text-align:center;}
math .texttt { font-family: monospace; }
math .textit { font-style: italic; }
math .textsl { font-style: oblique; }
math .textsf { font-family: sans-serif; }
math .textbf { font-weight: bold; }
.partToc a, .partToc, .likepartToc a, .likepartToc {line-height: 200%; font-weight:bold; font-size:110%;}
.chapterToc a, .chapterToc, .likechapterToc a, .likechapterToc, .appendixToc a, .appendixToc {line-height: 200%; font-weight:bold;}
.index-item, .index-subitem, .index-subsubitem {display:block}
.caption td.id{font-weight: bold; white-space: nowrap; }
table.caption {text-align:center;}
h1.partHead{text-align: center}
p.bibitem { text-indent: -2em; margin-left: 2em; margin-top:0.6em; margin-bottom:0.6em; }
p.bibitem-p { text-indent: 0em; margin-left: 2em; margin-top:0.6em; margin-bottom:0.6em; }
.subsectionHead, .likesubsectionHead { margin-top:2em; font-weight: bold;}
.sectionHead, .likesectionHead { font-weight: bold;}
.quote {margin-bottom:0.25em; margin-top:0.25em; margin-left:1em; margin-right:1em; text-align:justify;}
.verse{white-space:nowrap; margin-left:2em}
div.maketitle {text-align:center;}
h2.titleHead{text-align:center;}
div.maketitle{ margin-bottom: 2em; }
div.author, div.date {text-align:center;}
div.thanks{text-align:left; margin-left:10%; font-size:85%; font-style:italic; }
div.author{white-space: nowrap;}
.quotation {margin-bottom:0.25em; margin-top:0.25em; margin-left:1em; }
h1.partHead{text-align: center}
.sectionToc, .likesectionToc {margin-left:2em;}
.subsectionToc, .likesubsectionToc {margin-left:4em;}
.sectionToc, .likesectionToc {margin-left:6em;}
.frenchb-nbsp{font-size:75%;}
.frenchb-thinspace{font-size:75%;}
.figure img.graphics {margin-left:10%;}
/* end css.sty */

\title{A propos de Jordan}
\author{}
\date{}

\begin{document}
\maketitle

\textbf{Warning: 
requires JavaScript to process the mathematics on this page.\\ If your
browser supports JavaScript, be sure it is enabled.}

\begin{center}\rule{3in}{0.4pt}\end{center}

[
[
[]
[

\section{3.3 A propos de Jordan}

\subsection{3.3.1 Décomposition de Jordan}

Soit E un K-espace vectoriel de dimension finie et u \in L(E) dont le
polynôme caractéristique est scindé sur K,
E_1,\\ldots,E_k~
les sous-espaces caractéristiques de u associés aux valeurs propres
\lambda_1,\\ldots,\lambda_k~.
Soit u_i la restriction de u à E_i et n_i =
u_i -
\lambda_i\mathrmId_E_i. Avec les
notations précédentes, on a n_i^r_i = 0, donc
n_i est nilpotent. Soit d : E \rightarrow~ E définie par d(x_1 +
\\ldots~ +
x_k) = \lambda_1x_1 +
\\ldots~ +
\lambda_kx_k et n : E \rightarrow~ E défini par n(x_1 +
\\ldots~ +
x_k) = \\sum ~
_in_i(x_i). L'endomorphisme d est
diagonalisable (ses sous-espaces propres sont les E_i), n est
nilpotent (n^max(r_i)~ = 0)
et on a u = d + n. De plus, si d_i =
\lambda_i\mathrmId_E_i désigne
la restriction de d à E_i, on a d_i \cdot n_i =
n_i \cdot d_i et on en déduit donc que d \cdot n = d \cdot n.

Théorème~3.3.1 (décomposition de Jordan). Soit E un K-espace vectoriel
de dimension finie et u \in L(E) dont le polynôme caractéristique est
scindé sur K. Alors u s'écrit de manière unique sous la forme u = d + n
avec d diagonalisable, n nilpotent et d \cdot n = n \cdot d.

Démonstration L'existence de la décomposition vient d'être démontrée.
Soit u = d' + n' une autre décomposition vérifiant les conditions
imposées. Alors d' et n' commutent à u, donc à tous les P(u) et donc
laissent stables leurs noyaux. En particulier ils laissent stables les
sous-espaces caractéristiques de u. Soit d_i' et n_i'
les restrictions de d' et n' à E_i. n_i' est bien
entendu nilpotent. De plus, puisque d' est diagonalisable, il existe P
scindé à racines simples tel que P(d') = 0 et on a encore
P(d_i') = 0 ce qui montre que d_i' est diagonalisable.
On a d_i + n_i = d_i' + n_i' soit
encore \lambda_i\mathrmId + n_i =
d_i' + n_i' ou encore
\lambda_i\mathrmId_E_i -
d_i' = n_i' - n_i. Comme n_i'
commute à
\lambda_i\mathrmId_E_i,d_i'
et n_i', il commute à n_i et donc n_i -
n_i' est encore nilpotent. De plus
\lambda_i\mathrmId - d_i' est clairement
diagonalisable. Un endomorphisme à la fois diagonalisable et nilpotent
est nul puisqu'il doit avoir une matrice nulle dans une certaine base,
donc d_i = d_i' et n_i = n_i'.
Alors, d et d' coïncident sur des espaces dont la somme est E, donc ils
sont égaux. Ceci implique alors que n = n'. D'où l'unicité de la
décomposition.

\subsection{3.3.2 Applications}

Puissances d'un endomorphisme

Soit E un K-espace vectoriel de dimension finie et u \in L(E) dont le
polynôme caractéristique est scindé sur K,
E_1,\\ldots,E_k~
les sous-espaces caractéristiques de u associés aux valeurs propres
\lambda_1,\\ldots,\lambda_k~.
Soit u_i la restriction de u à E_i et n_i =
u_i -
\lambda_i\mathrmId_E_i.
L'endomorphisme n_i est nilpotent d'indice de nilpotence
r_i. On a u_i =
\lambda_i\mathrmId_E_i +
n_i et donc si q \in \mathbb{N}~

u_i^q = \\sum
_p=0^qC_ q^p\lambda_
i^q-pn_ i^p = \\sum
_p=0^min(q,r_i-1)C_
q^p\lambda_ i^q-pn_ i^p

Supposons désormais que \lambda_i\neq~0. On
obtient

\begin{align*} u_i^q& =& \lambda_
i^q \\sum
_p=0^min(q,r_i-1) q(q -
1)\ldots~(q - p + 1) \over
p! \lambda_i^-pn_ i^p\%&
\\ & =& \lambda_i^q
\sum _p=0^r_i-1~ q(q -
1)\ldots~(q - p + 1) \over
p! \lambda_i^-pn_ i^p \%&
\\ \end{align*}

(puisque 
q(q-1)\\ldots~(q-p+1)
\over p! = 0 si p > q). Posons alors, pour
t \in K

\sum _p=0^r_i-1~ t(t -
1)\ldots~(t - p + 1) \over
p! \lambda_i^-pn_ i^p =
\sum _p=0^r_i-1w_
i,pt^p

avec w_i,p \in L(E_i)~; c'est une fonction polynomiale
en t à valeurs dans L(E_i) de degré inférieur ou égal à
r_i - 1. On obtient

\forall~~q \in \mathbb{N}~,\quad
u_i^q = \lambda_ i^q
\sum _p=0^r_i-1w_
i,pq^p

Supposons maintenant que u est inversible, si bien que
\forall~~i \in [1,k],
\lambda_i\neq~0. Soit \pi_i(x) la
projection sur E_i parallèlement à
\\oplus~ ~
_j\neq~iE_j. On a
\\sum ~
_i\pi_i = \mathrmId_E si bien
que u = u \cdot (\\sum ~
_i\pi_i) =\
\sum  _iu_i \cdot \pi_i~.

Lemme~3.3.2 \forall~q \in \mathbb{N}~, u^q~
= \\sum ~
u_i^q \cdot \pi_i.

Démonstration Evident par récurrence sur q en remarquant que les
E_i sont stables par u et les u_i.

On en déduit donc que \forall~~q \in
\mathbb{N}~,\quad u^q =\
\sum ~
_i=1^k\lambda_i^q\
\sum ~
_p=0^r_i-1q^pw_i,p \cdot
\pi_i, d'où le théorème (en remarquant que r_i \leq
m_i)

Théorème~3.3.3 Soit u \in L(E) inversible dont le polynôme caractéristique
est scindé sur K,
\lambda_1,\\ldots,\lambda_k~
ses valeurs propres de multiplicités respectives
m_1,\\ldots,m_k~.
Alors il existe une famille
(v_i,p)_1\leqi\leqk,0\leqp\leqm_i-1 d'endomorphismes de E
tels que

\forall~q \in \mathbb{N}~,\quad u^q~ =
\sum _i=1^k\lambda_ i^q~
\\sum
_p=0^m_i-1q^pv_ i,p

Remarque~3.3.1 Bien entendu, on a un résultat similaire pour les
matrices inversibles

Théorème~3.3.4 Soit A \in M_K(n) inversible dont le polynôme
caractéristique est scindé sur K,
\lambda_1,\\ldots,\lambda_k~
ses valeurs propres de multiplicités respectives
m_1,\\ldots,m_k~.
Alors il existe une famille
(B_i,p)_1\leqi\leqk,0\leqp\leqm_i-1 de matrices carrées
d'ordre n telles que

\forall~q \in \mathbb{N}~,\quad A^q~ =
\sum _i=1^k\lambda_ i^q~
\\sum
_p=0^m_i-1q^pB_ i,p

Suites à récurrence linéaire

Remarque~3.3.2 Soit p \in \mathbb{N}~,
a_0,\\ldots,a_p-1~
une famille d'éléments de K et

V = \(u_n)_n\in\mathbb{N}~ \in
K^\mathbb{N}~∣\forall~~n \in
\mathbb{N}~, u_ n+p = a_p-1u_n+p-1 +
\\ldots~ +
a_0u_n\

V est un sous-espace vectoriel de K^\mathbb{N}~. Il est clair que la
donnée de
u_0,\\ldots,u_p-1~
détermine parfaitement un élément de V et on a donc

Théorème~3.3.5 L'application V \rightarrow~ K^p,
(u_n)_n\in\mathbb{N}~\mapsto~(u_0,\\ldots,u_p-1~)
est un isomorphisme d'espaces vectoriels. On a en particulier
dim~ V = p.

Remarque~3.3.3 Il est clair que l'on peut se limiter à étudier le cas où
a_0\neq~0 sinon notre récurrence
linéaire d'ordre p se réduit à une récurrence linéaire d'ordre k \leq p
valable pour n ≥ n_0.

Soit (u_n)_n\in\mathbb{N}~ \in V et considérons la suite (V
_n) définie par V _n = \left
(\matrix\,u_n
\cr u_n+1 \cr
\⋮~ \cr
u_n+p-1\right ) \in K^p. On a
clairement V _n+1 = AV _n avec

A = \left (\matrix\,0 &1
&0&\\ldots~&0
\cr &⋱
&⋱&
&\⋮~
\cr &
&⋱&\mathrel⋱&\⋮~
\cr 0
&\\ldots~
&\\ldots~&0&1
\cr
a_0&a_1&\\ldots&\\\ldots&a_p-1~\right
) \in M_K(p)

et donc V _n = A^nV _0. On a
\mathrm{det}~ A =
(-1)^n-1a_0\neq~0 et donc la
matrice est inversible. On peut donc appliquer le résultat précédent.
Soit \chi le polynôme caractéristique de la matrice A (encore appelé
polynôme caractéristique de la récurrence linéaire). Un calcul simple
donne

Lemme~3.3.6 On a \chi(X) = X^p - a_p-1X^p-1
-\\ldots~ -
a_0. Pour \lambda~ \in K^∗, on a

\chi(\lambda~) = 0 \Leftrightarrow (\lambda~^n)_ n\in\mathbb{N}~ \in V

Soit
\lambda_1,\\ldots,\lambda_k~
les racines de \chi de multiplicités respectives
m_1,\\ldots,m_k~.
On sait qu'il existe une famille
(B_i,q)_1\leqi\leqk,0\leqq\leqm_i-1 de matrices carrées
d'ordre p telles que

\forall~n \in \mathbb{N}~,\quad A^n~ =
\sum _i=1^k\lambda_ i^n~
\\sum
_q=0^m_i-1n^qB_ i,q

On a donc en particulier A^nV _0
= \\sum ~
_i=1^k\lambda_i^n\
\sum ~
_q=0^m_i-1n^qB_i,qV
_0 et en prenant la première coordonnée,

u_n = \\sum
_i=1^k\lambda_ i^n \\sum
_q=0^m_i-1\alpha_ i,qn^q

Soit alors W le sous-espace de K^\mathbb{N}~ engendré par les suites
(\lambda_i^nn^q)_1\leqi\leqk,0\leqq\leqm_i-1.
On a dim~ W
\leq\\sum  m_i~ = p
et V \subset~ W avec dim~ V = p. On en déduit que V =
W et que la famille
(\lambda_i^nn^q)_1\leqi\leqk,0\leqq\leqm_i-1
est une base de V . On a donc le théorème suivant

Théorème~3.3.7 Soit p \in \mathbb{N}~,
a_0,\\ldots,a_p-1~
une famille d'éléments de K avec
a_0\neq~0, et V l'espace des suites
vérifiant la récurrence linéaire \forall~~n \in \mathbb{N}~,
u_n+p = a_p-1u_n+p-1 +
\\ldots~ +
a_0u_n\. Soit \chi(X) = X^p -
a_p-1X^p-1
-\\ldots~ -
a_0 le polynôme caractéristique de la récurrence linéaire
(obtenu en recherchant des solutions particulières de la forme
u_n = \lambda~^n),
\lambda_1,\\ldots,\lambda_k~
les racines de \chi de multiplicités respectives
m_1,\\ldots,m_k~.
Alors la famille
(\lambda_i^nn^q)_1\leqi\leqk,0\leqq\leqm_i-1
est une base de V . Les solutions de la récurrence linéaire sont
exactement les suites qui s'écrivent sous la forme

u_n = \\sum
_i=1^k\lambda_ i^nP_
i(n),\quad P_i \in K[X], deg P_i \leq
m_i - 1

Retour aux puissances d'un endomorphisme

Soit u \in L(E) et soit P(X) = X^p -
a_p-1X^p-1
-\\ldots~ -
a_0 un polynôme qui annule u (par exemple le polynôme
caractéristique). On a immédiatement

Lemme~3.3.8 (u^n)_0\leqn\leqp-1 est une famille
génératrice de
\mathrmVect(u^n~,n
\in \mathbb{N}~).

On peut donc chercher à exprimer u^n sous la forme
u^n = \alpha_n^(p-1)u^p-1 +
\\ldots~ +
\alpha_n^(0)\mathrmId.

Théorème~3.3.9 Soit
(\alpha_n^(p-1))_n\in\mathbb{N}~,\\ldots,(\alpha_n^0)_n\in\mathbb{N}~~
les suites solutions de la récurrence linéaire \alpha_n+p =
a_p-1\alpha_n+p-1 +
\\ldots~ +
a_0\alpha_n vérifiant

\forall~~i \in [0,p - 1],
\forall~~j \in [0,p - 1],\quad
\alpha_i^(j) = \delta_ i^j

Alors

\forall~n \in \mathbb{N}~,\quad u^n~ =
\alpha_ n^(p-1)u^p-1 +
\\ldots + \alpha~_
n^(0)\mathrmId

Démonstration Par récurrence sur n. C'est manifestement vérifié si n \leq p
- 1. De plus, si n ≥ p la relation u^p =
a_p-1u^p-1 +
\\ldots~ +
a_0\mathrmId donne u^n =
a_p-1u^n-1 +
\\ldots~ +
a_0u^n-p soit par l'hypothèse de récurrence

\begin{align*} u^n& =&
\sum _i=1^pa_
p-iu^n-i = \\sum
_i=1^pa_ p-i \\sum
_j=0^p-1\alpha_ n-i^(j)u^j \%&
\\ & =& \\sum
_j=0^p-1(\\sum
_i=1^pa_
p-i\alpha_n-i^(j))u^j =
\sum _j=0^p-1\alpha~_
n^(j)u^j\%& \\
\end{align*}

d'après la relation vérifiée par les (\alpha_n^(j)). Ceci
achève la démonstration.

\subsection{3.3.3 Réduction des endomorphismes nilpotents}

Définition~3.3.1 Soit E un K-espace vectoriel et u \in L(E). On dit que u
est nilpotent d'indice de nilpotence r si u^r = 0 et
u^r-1\neq~0.

Remarque~3.3.4 Remarquons que la seule valeur propre d'un endomorphisme
nilpotent est 0, car si u(x) = \lambda~x, on a 0 = u^r(x) =
\lambda~^rx.

Proposition~3.3.10 Soit E un K-espace vectoriel de dimension n et u \in
L(E). Alors u est nilpotent si et seulement si \chi_u(X) =
X^n.

Démonstration ( \rigtharrow~) Supposons que u est nilpotent. Comme u est annulé par
le polynôme scindé X^r (si u^r = 0), u est
trigonalisable. Mais u admet comme seule valeur propre 0. On en déduit
que \chi_u(X) = X^n. Pour la réciproque, on peut par
exemple utiliser le théorème de Cayley Hamilton, ou trigonaliser u.

Remarque~3.3.5 On en déduit que l'indice de nilpotence r est inférieur
ou égal à n. On a bien entendu \mu_u(X) = X^r.

Définition~3.3.2 Soit p ≥ 1. On appelle matrice élémentaire de Jordan
d'ordre p la matrice

J_p = \left
(\matrix\,0&1&0&\\ldots~&0
\cr
\⋮&⋱&\mathrel⋱&\mathrel⋱&\\⋮~
\cr
\⋮~&
&⋱&\mathrel⋱&\⋮~
\cr
0&\\ldots&\\\ldots~&0&1
\cr
0&\\ldots&\\\ldots&\\\ldots&0~\right
)

Soit E un K-espace vectoriel de dimension n et u \in L(E) nilpotent
d'indice de nilpotence r. Supposons par exemple que r = n. On a donc
u^n-1\neq~0 avec u^n = 0.
Soit a \in E tel que u^n-1(a)\neq~0 et
posons e_i = u^n-i(a) pour 1 \leq i \leq n. Montrons que
(e_1,\\ldots,e_n~)
est une base de E. Il suffit de montrer que c'est une famille libre.
Pour cela supposons que \lambda_1e_1 +
\\ldots~ +
\lambda_ne_n = 0, soit encore

\lambda_1u^n-1(a) +
\\ldots + \lambda~_
n-1u(a) + \lambda_na = 0

Appliquons aux deux membres u^n-1 en tenant compte de
u^n(a) =
\\ldots~ =
u^2n-2(a) = 0~; on obtient \lambda_nu^n-1(a) =
0 soit \lambda_n = 0. Supposons montré que \lambda_n =
\lambda_n-1 =
\\ldots~ =
\lambda_n-k+1 = 0 si bien que l'on a

\lambda_1u^n-1(a) +
\\ldots + \lambda~_
n-k-1u^k+1(a) + \lambda_ n-ku^k(a) = 0

Appliquons aux deux membres u^n-k-1 en tenant compte de
u^n(a) =
\\ldots~ =
u^2n-k-2(a) = 0~; on obtient
\lambda_n-ku^n-1(a) = 0 soit \lambda_n-k = 0. Par
récurrence, on a bien \forall~i, \lambda_i~ = 0.
Donc
(e_1,\\ldots,e_n~)
est une base de E. Dans cette base, la matrice de u est clairement
J_n~: on a u(e_i) = e_i-1 si i ≥ 2 et
u(e_1) = 0. Ce cas particulier est à la base du résultat
suivant

Théorème~3.3.11 Soit E un K-espace vectoriel de dimension n et u \in L(E)
nilpotent. Alors il existe une base \mathcal{E} de E telle que la matrice de u
dans la base \mathcal{E} soit un tableau diagonal de matrices élémentaires de
Jordan

\mathrmMat~ (u,\mathcal{E})
=\
\mathrmdiag(J_p_1,\\ldots,J_p_k~)

Démonstration Elle va faire l'objet des deux sections suivantes

\subsection{3.3.4 Première démonstration}

Par récurrence sur n = dim~ E. Le résultat est
évident pour n = 1. Supposons le vrai pour tous les endomorphismes
nilpotents d'espaces de dimensions inférieures ou égales à n - 1. Soit r
l'indice de nilpotence de u. Si r = n, on a déjà vu que le résultat
était vrai (avec une seule matrice élémentaire de Jordan). On peut donc
supposer que r < n. Puisque
u^r-1\neq~0, soit a \in E tel que
u^r-1(a)\neq~0. Comme précédemment la
famille \mathcal{E}_1 =
(u^r-1(a),\\ldots~,u(a),a)
est libre et il est clair que le sous-espace F =\
\mathrmVect(u^r-1(a),\\ldots~,u(a),a)
est stable par u (chaque vecteur est décalé d'un cran vers la gauche,
sauf le premier qui est annulé par u). On a
\mathrmMat~
(u_F,\mathcal{E}_1) = J_r.

Puisque u^r-1(a)\neq~0 on peut trouver
une forme linéaire f telle que
f(u^r-1(a))\neq~0. Soit

\begin{align*} G& =& \⋂
_k=0^r-1 \mathrmKerf \cdot u^k
\%& \\ & =& \x \in
E∣f(x) = f(u(x)) =
\\ldots~ =
f(u^r-1(x) = 0\\%&
\\ \end{align*}

Lemme~3.3.12 G est un supplémentaire de F stable par u.

Démonstration La stabilité par u est claire, car si x \in G on a

\begin{align*} f(u(x)) = 0,f(u(u(x))) =
0,f(u^r-2(u(x)) = f(u^r-1(x) = 0,& & \%&
\\ f(u^r-1(u(x))) =
f(u^r(x)) = f(0) = 0& & \%&
\\ \end{align*}

Montrons que F \bigcap G = \0\. Pour cela
soit x = \lambda_1u^r-1(a) +
\\ldots~ +
\lambda_r-1u(a) + \lambda_ra \in F et supposons que x appartienne à
G. On a 0 = f(u^r-1(x)) = \lambda_1f(u^2r-2(a))
+ \\ldots~ +
\lambda_r-1f(u^r(a)) + \lambda_rf(u^r-1(a))
et tenant compte de u^r(a) =
\\ldots~ =
u^2r-2(a) = 0 on obtient \lambda_rf(u^r-1(a)) =
0 soit \lambda_r = 0. Comme précédemment une récurrence descendante
montre que \lambda_r = \lambda_r-1 =
\\ldots~ =
\lambda_1 = 0 soit x = 0. Donc F et G sont en somme directe. Mais G =
\bigcap_k=0^r-1\
\mathrmKerf \cdot u^k, et donc

dim~ G = n
-\mathrmrg~(f \cdot
u^k, 0 \leq k \leq r - 1) ≥ n - r = dim~ E
- dim~ F

On a donc E = F \oplus~ G.

(Fin de la démonstration) On peut maintenant terminer la démonstration
du théorème. En appliquant notre hypothèse de récurrence à
l'endomorphisme nilpotent u_G de G, on peut trouver
une base de G telle que
\mathrmMat~
(u_G,\mathcal{E}_2) =\
\mathrmdiag(J_p_2,\\ldots,J_p_k~).
Alors \mathcal{E} = \mathcal{E}_1 \cup\mathcal{E}_2 est une base de E dans laquelle
\mathrmMat~ (u,\mathcal{E})
=\
\mathrmdiag(J_r,J_p_2,\\ldots,J_p_k~),
ce qui achève la démonstration.

\subsection{3.3.5 Deuxième démonstration}

Posons V _i =\
\mathrmKeru^i.

Lemme~3.3.13 On a \0\ = V _0
\subset~ V _1
\subset~\\ldots~ \subset~ V
_r = E avec une suite strictement croissante.

Démonstration Les inclusions sont claires. Supposons que V _i =
V _i+1 pour i \leq r - 1. Soit x \in E. On a 0 = u^r(x) =
u^i+1(u^r-i-1(x)) donc u^r-i-1(x) \in V
_i+1 = V _i et donc u^r-1(x) =
u^i(u^r-i-1(x)) = 0. On aurait donc
u^r-1 = 0 ce qui est exclu.

Soit W_1 un supplémentaire de V _r-1 dans E = V
_r.

Lemme~3.3.14 On peut construire une suite de sous-espaces
W_2,\\ldots,W_r~
de E vérifiant

\begin{itemize}
\itemsep1pt\parskip0pt\parsep0pt
\item
  (i) \forall~~k \in [1,r],\quad V
  _r-k+1 = V _r-k \oplus~ W_k
\item
  (ii) \forall~~k \in [2,r],\quad
  u(W_k-1) \subset~ W_k
\item
  (iii) \forall~~k \in [1,r -
  1],\quad u_W_k est
  injective
\item
  On a alors E = W_1 \oplus~⋯ \oplus~
  W_r.
\end{itemize}

Démonstration On va construire W_k par récurrence sur k. Pour
ce qui concerne k = 1, il suffit de montrer que
u_W_1 est injective. Mais si x \in
W_1 \diagdown\0\, on a
x∉V _r-1, donc
u^r-1(x)\neq~0 et donc
u(x)\neq~0. Supposons donc
W_1,\\ldots,W_k-1~
construits. Soit x \in W_k-1
\diagdown\0\. On a
x∉V _r-k+1, donc
u^r-k+1(x)\neq~0, soit
u^r-k(u(x))\neq~0 et donc
u(x)∉V _r-k. On a ainsi
u(W_k-1) \bigcap V _r-k =
\0\. Mais d'autre part x \in
W_k-1 \subset~ V _r-k+2 et donc u(x) \in V _r-k+1. On
a donc u(W_k-1) \subset~ V _r-k+1, V _r-k \subset~ V
_r-k+1 avec u(W_k-1) \bigcap V _r-k =
\0\. On peut donc trouver un
supplémentaire W_k de V _r-k dans V _r-k+1
tel que u(W_k-1) \subset~ W_k. Alors, si x \in W_k
\diagdown\0\, x∉V
_r-k, soit u^r-k(x)\neq~0 et
donc si k < r, u(x)\neq~0. Ceci montre
bien que u_W_k est injective. On a donc bien
construit notre suite W_k. Il est clair par récurrence que V
_k = W_r-k+1
\oplus~\\ldots~ \oplus~
W_r et donc E = V _r = W_1
\oplus~⋯ \oplus~ W_r.

Soit alors maintenant (e_i,1)_1\leqi\leqs_1 une
base de W_1. Comme u_W_1 est
injective, (e_i,2 =
u(e_i,1))_1\leqi\leqs_1 est une base de
u(W_1) que l'on peut compléter en une base
(e_i,2)_1\leqi\leqm_2 de W_2. Une
récurrence immédiate nous permet de construire des bases
(e_i,k)_1\leqi\leqm_k des W_k telles que
pour k \leq r - 1, et 1 \leq i \leq m_k, u(e_i,k) =
e_i,k+1. On a e_i,r \in W_r \subset~ V _1
= \mathrmKer~u, donc
u(e_i,r) = 0. On obtient ainsi une base (e_i,j) de E.
Si on ordonne cette base en posant que (i,j) < (i',j')
\Leftrightarrow j > j'\text
ou (j = j'\text et i < i'), la matrice de
u est un tableau diagonal de matrices de Jordan.

\subsection{3.3.6 Réduction de Jordan}

Soit E un K-espace vectoriel de dimension finie et u \in L(E) dont le
polynôme caractéristique est scindé sur K,
E_1,\\ldots,E_k~
les sous-espaces caractéristiques de u associés aux valeurs propres
\lambda_1,\\ldots,\lambda_k~.
Soit u_i la restriction de u à E_i et n_i =
u_i -
\lambda_i\mathrmId_E_i. Avec les
notations précédentes, on a n_i^r_i = 0, donc
n_i est nilpotent. On peut donc trouver une base \mathcal{E}_i
de E_i dans laquelle la matrice de n_i est
\mathrmdiag(J_p_1,\\\ldots,J_p_k~)
et alors la matrice de u_i dans cette base est
\mathrmdiag(J_p_1(\lambda_i),\\\ldots,J_p_k(\lambda_i~))
avec

J_p(\lambda~) = \left
(\matrix\,\lambda~&1&0&\\ldots~&0
\cr
0&\lambda~&1&\\ldots~&0
\cr
\⋮~&
&⋱&\mathrel⋱&\⋮~
\cr
0&\\ldots&\\\ldots~&\lambda~&1
\cr
0&\\ldots&\\\ldots&0&\lambda~~\right
) = \lambda~I_p + J_p \in M_K(p)

En réunissant ces bases on obtient

Théorème~3.3.15 Soit E un K-espace vectoriel de dimension finie et u \in
L(E) dont le polynôme caractéristique est scindé sur K. Alors il existe
une base \mathcal{E} de E, des scalaires
\mu_1,\\ldots,\mu_l~
(non nécessairement distincts) et des entiers
n_1,\\ldots,n_l~
tels que \mathrmMat~ (u,\mathcal{E})
=\
\mathrmdiag(J_n_1(\mu_1),\\ldots,J_n_l(\mu_l~)).

[
[
[
[

\end{document}
