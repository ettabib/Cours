\documentclass[]{article}
\usepackage[T1]{fontenc}
\usepackage{lmodern}
\usepackage{amssymb,amsmath}
\usepackage{ifxetex,ifluatex}
\usepackage{fixltx2e} % provides \textsubscript
% use upquote if available, for straight quotes in verbatim environments
\IfFileExists{upquote.sty}{\usepackage{upquote}}{}
\ifnum 0\ifxetex 1\fi\ifluatex 1\fi=0 % if pdftex
  \usepackage[utf8]{inputenc}
\else % if luatex or xelatex
  \ifxetex
    \usepackage{mathspec}
    \usepackage{xltxtra,xunicode}
  \else
    \usepackage{fontspec}
  \fi
  \defaultfontfeatures{Mapping=tex-text,Scale=MatchLowercase}
  \newcommand{\euro}{€}
\fi
% use microtype if available
\IfFileExists{microtype.sty}{\usepackage{microtype}}{}
\ifxetex
  \usepackage[setpagesize=false, % page size defined by xetex
              unicode=false, % unicode breaks when used with xetex
              xetex]{hyperref}
\else
  \usepackage[unicode=true]{hyperref}
\fi
\hypersetup{breaklinks=true,
            bookmarks=true,
            pdfauthor={},
            pdftitle={Nappes parametrees},
            colorlinks=true,
            citecolor=blue,
            urlcolor=blue,
            linkcolor=magenta,
            pdfborder={0 0 0}}
\urlstyle{same}  % don't use monospace font for urls
\setlength{\parindent}{0pt}
\setlength{\parskip}{6pt plus 2pt minus 1pt}
\setlength{\emergencystretch}{3em}  % prevent overfull lines
\setcounter{secnumdepth}{0}
 
/* start css.sty */
.cmr-5{font-size:50%;}
.cmr-7{font-size:70%;}
.cmmi-5{font-size:50%;font-style: italic;}
.cmmi-7{font-size:70%;font-style: italic;}
.cmmi-10{font-style: italic;}
.cmsy-5{font-size:50%;}
.cmsy-7{font-size:70%;}
.cmex-7{font-size:70%;}
.cmex-7x-x-71{font-size:49%;}
.msbm-7{font-size:70%;}
.cmtt-10{font-family: monospace;}
.cmti-10{ font-style: italic;}
.cmbx-10{ font-weight: bold;}
.cmr-17x-x-120{font-size:204%;}
.cmsl-10{font-style: oblique;}
.cmti-7x-x-71{font-size:49%; font-style: italic;}
.cmbxti-10{ font-weight: bold; font-style: italic;}
p.noindent { text-indent: 0em }
td p.noindent { text-indent: 0em; margin-top:0em; }
p.nopar { text-indent: 0em; }
p.indent{ text-indent: 1.5em }
@media print {div.crosslinks {visibility:hidden;}}
a img { border-top: 0; border-left: 0; border-right: 0; }
center { margin-top:1em; margin-bottom:1em; }
td center { margin-top:0em; margin-bottom:0em; }
.Canvas { position:relative; }
li p.indent { text-indent: 0em }
.enumerate1 {list-style-type:decimal;}
.enumerate2 {list-style-type:lower-alpha;}
.enumerate3 {list-style-type:lower-roman;}
.enumerate4 {list-style-type:upper-alpha;}
div.newtheorem { margin-bottom: 2em; margin-top: 2em;}
.obeylines-h,.obeylines-v {white-space: nowrap; }
div.obeylines-v p { margin-top:0; margin-bottom:0; }
.overline{ text-decoration:overline; }
.overline img{ border-top: 1px solid black; }
td.displaylines {text-align:center; white-space:nowrap;}
.centerline {text-align:center;}
.rightline {text-align:right;}
div.verbatim {font-family: monospace; white-space: nowrap; text-align:left; clear:both; }
.fbox {padding-left:3.0pt; padding-right:3.0pt; text-indent:0pt; border:solid black 0.4pt; }
div.fbox {display:table}
div.center div.fbox {text-align:center; clear:both; padding-left:3.0pt; padding-right:3.0pt; text-indent:0pt; border:solid black 0.4pt; }
div.minipage{width:100%;}
div.center, div.center div.center {text-align: center; margin-left:1em; margin-right:1em;}
div.center div {text-align: left;}
div.flushright, div.flushright div.flushright {text-align: right;}
div.flushright div {text-align: left;}
div.flushleft {text-align: left;}
.underline{ text-decoration:underline; }
.underline img{ border-bottom: 1px solid black; margin-bottom:1pt; }
.framebox-c, .framebox-l, .framebox-r { padding-left:3.0pt; padding-right:3.0pt; text-indent:0pt; border:solid black 0.4pt; }
.framebox-c {text-align:center;}
.framebox-l {text-align:left;}
.framebox-r {text-align:right;}
span.thank-mark{ vertical-align: super }
span.footnote-mark sup.textsuperscript, span.footnote-mark a sup.textsuperscript{ font-size:80%; }
div.tabular, div.center div.tabular {text-align: center; margin-top:0.5em; margin-bottom:0.5em; }
table.tabular td p{margin-top:0em;}
table.tabular {margin-left: auto; margin-right: auto;}
div.td00{ margin-left:0pt; margin-right:0pt; }
div.td01{ margin-left:0pt; margin-right:5pt; }
div.td10{ margin-left:5pt; margin-right:0pt; }
div.td11{ margin-left:5pt; margin-right:5pt; }
table[rules] {border-left:solid black 0.4pt; border-right:solid black 0.4pt; }
td.td00{ padding-left:0pt; padding-right:0pt; }
td.td01{ padding-left:0pt; padding-right:5pt; }
td.td10{ padding-left:5pt; padding-right:0pt; }
td.td11{ padding-left:5pt; padding-right:5pt; }
table[rules] {border-left:solid black 0.4pt; border-right:solid black 0.4pt; }
.hline hr, .cline hr{ height : 1px; margin:0px; }
.tabbing-right {text-align:right;}
span.TEX {letter-spacing: -0.125em; }
span.TEX span.E{ position:relative;top:0.5ex;left:-0.0417em;}
a span.TEX span.E {text-decoration: none; }
span.LATEX span.A{ position:relative; top:-0.5ex; left:-0.4em; font-size:85%;}
span.LATEX span.TEX{ position:relative; left: -0.4em; }
div.float img, div.float .caption {text-align:center;}
div.figure img, div.figure .caption {text-align:center;}
.marginpar {width:20%; float:right; text-align:left; margin-left:auto; margin-top:0.5em; font-size:85%; text-decoration:underline;}
.marginpar p{margin-top:0.4em; margin-bottom:0.4em;}
.equation td{text-align:center; vertical-align:middle; }
td.eq-no{ width:5%; }
table.equation { width:100%; } 
div.math-display, div.par-math-display{text-align:center;}
math .texttt { font-family: monospace; }
math .textit { font-style: italic; }
math .textsl { font-style: oblique; }
math .textsf { font-family: sans-serif; }
math .textbf { font-weight: bold; }
.partToc a, .partToc, .likepartToc a, .likepartToc {line-height: 200%; font-weight:bold; font-size:110%;}
.chapterToc a, .chapterToc, .likechapterToc a, .likechapterToc, .appendixToc a, .appendixToc {line-height: 200%; font-weight:bold;}
.index-item, .index-subitem, .index-subsubitem {display:block}
.caption td.id{font-weight: bold; white-space: nowrap; }
table.caption {text-align:center;}
h1.partHead{text-align: center}
p.bibitem { text-indent: -2em; margin-left: 2em; margin-top:0.6em; margin-bottom:0.6em; }
p.bibitem-p { text-indent: 0em; margin-left: 2em; margin-top:0.6em; margin-bottom:0.6em; }
.subsectionHead, .likesubsectionHead { margin-top:2em; font-weight: bold;}
.sectionHead, .likesectionHead { font-weight: bold;}
.quote {margin-bottom:0.25em; margin-top:0.25em; margin-left:1em; margin-right:1em; text-align:justify;}
.verse{white-space:nowrap; margin-left:2em}
div.maketitle {text-align:center;}
h2.titleHead{text-align:center;}
div.maketitle{ margin-bottom: 2em; }
div.author, div.date {text-align:center;}
div.thanks{text-align:left; margin-left:10%; font-size:85%; font-style:italic; }
div.author{white-space: nowrap;}
.quotation {margin-bottom:0.25em; margin-top:0.25em; margin-left:1em; }
h1.partHead{text-align: center}
.sectionToc, .likesectionToc {margin-left:2em;}
.subsectionToc, .likesubsectionToc {margin-left:4em;}
.sectionToc, .likesectionToc {margin-left:6em;}
.frenchb-nbsp{font-size:75%;}
.frenchb-thinspace{font-size:75%;}
.figure img.graphics {margin-left:10%;}
/* end css.sty */

\title{Nappes parametrees}
\author{}
\date{}

\begin{document}
\maketitle

\textbf{Warning: 
requires JavaScript to process the mathematics on this page.\\ If your
browser supports JavaScript, be sure it is enabled.}

\begin{center}\rule{3in}{0.4pt}\end{center}

[
[]
[

\section{19.1 Nappes paramétrées}

\subsection{19.1.1 Notion de nappe paramétrée. Equivalence}

Définition~19.1.1 On appelle nappe paramétrée de classe C^k
de E tout couple (D,F) d'un ouvert D de \mathbb{R}~^2 et d'une
application F : D \rightarrow~ E de classe C^k, notée
(u,v)\mapsto~F(u,v)

Remarque~19.1.1 Vocabulaire associé. Soit \Sigma = (D,F) une nappe paramétrée
de classe C^k de E

\begin{itemize}
\itemsep1pt\parskip0pt\parsep0pt
\item
  (i) On appelle point de la nappe \Sigma un couple
  (u_0,v_0) \in D
\item
  (ii) On appelle image ou support de \Sigma la partie F(D) de E
\item
  (iii) On appelle multiplicité d'un point m de l'image de \Sigma le cardinal
  de l'ensemble F^-1(\m\)
  (éventuellement infinie)~; on dit qu'un point de l'image est simple
  s'il est de multiplicité 1, sinon on dit qu'il est multiple~; on dit
  que la nappe est simple si tout point de l'image est de multiplicité
  1, c'est-à-dire si F est injective.
\item
  (iv) On dit qu'un point (u_0,v_0) de la nappe \Sigma est
  régulier si la famille ( \partial~F \over \partial~u
  (u_0,v_0), \partial~F \over \partial~v
  (u_0,v_0)) est libre~; on dit que la nappe est
  régulière si tout point de la nappe est régulier~; un point non
  régulier est dit singulier.
\end{itemize}

Exemple~19.1.1 Soit
(O,\vec\imath,\vecȷ,\veck)
un repère affine, D un ouvert de \mathbb{R}~^2 et f une application de
D dans \mathbb{R}~. La nappe paramétrée F : (x,y)\mapsto~O +
x\vec\imath + y\vecȷ +
f(x,y)\veck sera appelée une nappe cartésienne. Pour
une telle nappe on a  \partial~F \over \partial~x
(x_0,y_0) = \vec\imath + \partial~f
\over \partial~x
(x_0,y_0)\veck et  \partial~F
\over \partial~y (x_0,y_0) =
\vecȷ + \partial~f \over \partial~x
(x_0,y_0)\veck qui sont évidemment
linéairement indépendants. Une nappe cartésienne est donc régulière.
Nous verrons un peu plus loin une réciproque partielle à ce résultat.

Définition~19.1.2 (D,F) et (\Delta,G) deux nappes paramétrées de classe
C^k. On dit que ces deux nappes sont
C^k-équivalentes s'il existe un difféomorphisme \theta : D \rightarrow~ \Delta de
classe C^k tel que F = G \cdot \theta.

Remarque~19.1.2 On dira qu'un tel difféomorphisme est un changement de
paramétrage admissible. L'étude des nappes paramétrées concerne
essentiellement l'étude des propriétés des arcs qui sont invariantes par
équivalence. L'application \theta étant bijective on voit immédiatement que

Proposition~19.1.1 Soit (D,F) et (\Delta,G) deux nappes paramétrées de classe
C^k qui sont C^k-équivalentes. Alors les deux
nappes ont la même image. Tous les points de l'image ont la même
multiplicité pour les deux nappes. En particulier un point de l'image
est simple pour l'un si et seulement si~il est simple pour l'autre.

Proposition~19.1.2 Soit (D,F) et (\Delta,G) deux nappes paramétrées de classe
C^k qui sont C^k-équivalentes, \theta : D \rightarrow~ \Delta un
difféomorphisme de classe C^k tel que F = G \cdot \theta. Alors
(u_0,v_0) est un point régulier de (D,F) si et
seulement si~\theta(u_0,v_0) est un point régulier de
(J,G). En particulier, (D,F) est régulière si et seulement si~(J,G)
l'est.

Démonstration Supposons que F = G \cdot \theta. Notons
(u,v)\mapsto~F(u,v) et
(\lambda~,\mu)\mapsto~G(\lambda~,\mu) les deux nappes paramétrées
équivalentes, et \theta(u,v) = (\theta_1(u,v),\theta_2(u,v)) le
changement de paramétrage admissible. On a donc F(u,v) =
G(\theta_1(u,v),\theta_2(u,v)) d'où l'on déduit

\begin{align*} \partial~F \over \partial~u
(u_0,v_0)& =& \partial~\theta_1 \over
\partial~u (u_0,v_0) \partial~G \over \partial~\lambda~
(\theta_1(u_0,v_0),\theta_2(u_0,v_0))
\%& \\ & & + \partial~\theta_2
\over \partial~u (u_0,v_0) \partial~G
\over \partial~\mu
(\theta_1(u_0,v_0),\theta_2(u_0,v_0)),\%&
\\  \partial~F \over \partial~v
(u_0,v_0)& =& \partial~\theta_1 \over
\partial~v (u_0,v_0) \partial~G \over \partial~\lambda~
(\theta_1(u_0,v_0),\theta_2(u_0,v_0))
\%& \\ & & + \partial~\theta_2
\over \partial~v (u_0,v_0) \partial~G
\over \partial~\mu
(\theta_1(u_0,v_0),\theta_2(u_0,v_0))
\%& \\ \end{align*}

Faisons le produit vectoriel en notant (\lambda_0,\mu_0) =
(\theta_1(u_0,v_0),\theta_2(u_0,v_0))~;
on a

\begin{align*} \partial~F \over \partial~u
(u_0,v_0) ∧ \partial~F \over \partial~v
(u_0,v_0)&& \%& \\ &
=& \partial~\theta_1 \over \partial~u
(u_0,v_0) \partial~\theta_2 \over \partial~v
(u_0,v_0) \partial~G \over \partial~\lambda~
(\lambda_0,\mu_0) ∧ \partial~G \over \partial~\mu
(\lambda_0,\mu_0) \%& \\ & &
+ \partial~\theta_2 \over \partial~u
(u_0,v_0) \partial~\theta_1 \over \partial~v
(u_0,v_0) \partial~G \over \partial~\mu
(\lambda_0,\mu_0) ∧ \partial~G \over \partial~\lambda~
(\lambda_0,\mu_0) \%& \\ & =&
\left ( \partial~\theta_1 \over \partial~u
(u_0,v_0) \partial~\theta_2 \over \partial~v
(u_0,v_0) - \partial~\theta_2 \over \partial~u
(u_0,v_0) \partial~\theta_1 \over \partial~v
(u_0,v_0)\right )\%&
\\ & & \partial~G \over \partial~\lambda~
(\lambda_0,\mu_0) ∧ \partial~G \over \partial~\mu
(\lambda_0,\mu_0) \%& \\ & =&
j_\theta(u_0,v_0) \partial~G \over \partial~\lambda~
(\lambda_0,\mu_0) ∧ \partial~G \over \partial~\mu
(\lambda_0,\mu_0) \%& \\
\end{align*}

où l'on désigne par j_\theta(u,v) le jacobien de \theta au point (u,v).
Ce jacobien est non nul puisque \theta est un difféomorphisme, et donc on a

 \partial~F \over \partial~u (u_0,v_0) ∧ \partial~F
\over \partial~v
(u_0,v_0)\neq~0
\Leftrightarrow \partial~G \over \partial~\lambda~
(\lambda_0,\mu_0) ∧ \partial~G \over \partial~\mu
(\lambda_0,\mu_0)\neq~0

ce qui achève la démonstration.

\subsection{19.1.2 Orientation}

Supposons que D est connexe. Le jacobien d'un difféomorphisme ne
s'annulant pas, il doit être de signe constant sur un connexe. Ceci
conduit à la définition suivante

Définition~19.1.3 Soit (D,F) et (\Delta,G) deux nappes paramétrées de classe
C^k qui sont C^k-équivalentes, définies sur des
connexes, \theta : D \rightarrow~ \Delta un difféomorphisme de classe C^k tel que
F = G \cdot \theta. On dit que (D,F) et (\Delta,G) sont de même sens si \theta possède un
jacobien positif, de sens contraire si \theta est à jacobien négatif.

Remarque~19.1.3 Il peut se produire qu'il existe deux difféomorphismes
\theta_1 et \theta_2 tels que F = G \cdot \theta_1 et F = G \cdot
\theta_2, l'un étant à jacobien positif et l'autre à jacobien
négatif. Autrement dit deux nappes paramétrées peuvent être à la fois de
même sens et de sens contraire. On dit alors qu'une telle nappe
paramétrée n'est pas orientable. Un exemple typique est le ruban de
Moebius.

\subsection{19.1.3 Plan tangent à une nappe paramétrée, vecteur normal}

Définition~19.1.4 Soit \Gamma = (D,F) une nappe paramétrée de classe
C^k et (u_0,v_0) \in D un point régulier de
\Sigma. On appelle plan tangent à \Sigma au point (u_0,v_0) le
plan affine F(u_0,v_0) +\
\mathrmVect( \partial~F \over \partial~u
(u_0,v_0), \partial~F \over \partial~v
(u_0,v_0)).

Remarque~19.1.4 Le plan tangent en un point singulier n'est pas défini.

Définition~19.1.5 Soit \Sigma = (D,F) une nappe paramétrée de classe
C^k et (u_0,v_0) \in D un point régulier de
\Sigma. On appelle vecteur normal à \Sigma au point (u_0,v_0) le
vecteur  \partial~F \over \partial~u (u_0,v_0) ∧
\partial~F \over \partial~v (u_0,v_0). On appelle
normale à \Sigma au point (u_0,v_0) la droite affine
F(u_0,v_0) + \mathbb{R}~ \partial~F \over \partial~u
(u_0,v_0) ∧ \partial~F \over \partial~v
(u_0,v_0)

Remarque~19.1.5 Le plan tangent est donc le plan affine passant par le
point F(u_0,v_0) et orthogonal au vecteur normal. On a
vu précédemment que si (D,F) et (\Delta,G) sont deux nappes paramétrées de
classe C^k qui sont C^k-équivalentes et si \theta : D \rightarrow~
\Delta, (u,v)\mapsto~(\lambda~,\mu) = \theta(u,v) est un
difféomorphisme de classe C^k tel que F = G \cdot \theta alors  \partial~F
\over \partial~u (u_0,v_0) ∧ \partial~F
\over \partial~v (u_0,v_0) =
j_\theta(u_0,v_0) \partial~G \over \partial~\lambda~
(\lambda_0,\mu_0) ∧ \partial~G \over \partial~\mu
(\lambda_0,\mu_0). On en déduit que la direction du vecteur
normal est invariante par changement de paramétrage admissible, donc il
en est de même de la direction du plan tangent. Comme en plus les deux
plans tangents ont en commun le point F(u_0,v_0) =
G(\lambda_0,\mu_0), ils sont nécessairement confondus. Il en
est bien entendu de même de la normale. D'où la proposition suivante

Proposition~19.1.3 La notion de plan tangent et de normale à une nappe
paramétrée est invariante par changement de paramétrage admissible. Soit
(D,F) et (\Delta,G) deux nappes paramétrées de classe C^k qui sont
C^k-équivalentes, \theta : D \rightarrow~ \Delta un difféomorphisme de classe
C^k tel que F = G \cdot \theta. Alors (u_0,v_0) est
un point régulier de (D,F) si et seulement
si~\theta(u_0,v_0) est un point régulier de (\Delta,G), et dans
ce cas le plan tangent (resp. la normale) à (D,f) au point
(u_0,v_0) est égal au plan tangent (resp. à la
normale) à (\Delta,G) au point \theta(u_0,v_0).

Remarque~19.1.6 Soit (D,F) une nappe paramétrée et
t\mapsto~(\phi(t),\psi(t)) une application d'un intervalle
I de \mathbb{R}~ dans D. L'arc paramétré t\mapsto~F(\phi(t),\psi(t))
est alors un arc dont l'image est contenue dans l'image de la nappe. On
dira qu'un tel arc est tracé sur la nappe. En un point régulier (sur
l'arc et sur la nappe), le vecteur tangent est le vecteur  d
\over dt (F(\phi(t),\psi(t))) = \phi'(t) \partial~F
\over \partial~u (\phi(t),\psi(t)) + \psi'(t) \partial~F \over
\partial~v (\phi(t),\psi(t)) et il est donc contenu dans le plan tangent. On en
déduit que la tangente à un arc tracé sur la nappe est contenue dans le
plan tangent à la nappe au point correspondant.

\subsection{19.1.4 Points réguliers et nappes cartésiennes}

Le théorème suivant permet de ramener l'étude locale d'une nappe
paramétrée régulière à celle d'une nappe cartésienne.

Théorème~19.1.4 Soit \Sigma = (D,F) une nappe paramétrée de classe
C^k, (u_0,v_0) un point régulier de \Sigma et
(\vec\imath,\vecȷ,\veck)
une base de \vecE telle que \veck
ne soit pas tangent à la surface. Alors il existe un ouvert U \subset~ D et
contenant (u_0,v_0) tel que la sous nappe \Sigma_0
= (U_0,F) soit équivalente à une nappe cartésienne
(x,y)\mapsto~O + x\vec\imath +
y\vecȷ + f(x,y)\veck.

Démonstration Posons F(u,v) = O + \phi(u,v)\vec\imath +
\psi(u,v)\vecȷ + \omega(u,v)\veck. D'après
les hypothèses, la famille ( \partial~F \over \partial~u
(u_0,v_0), \partial~F \over \partial~v
(u_0,v_0),\veck) est libre et donc
le déterminant

\left
\matrix\, \partial~\phi
\over \partial~u (u_0,v_0)& \partial~\phi
\over \partial~v (u_0,v_0)&0
\cr  \partial~\psi \over \partial~u
(u_0,v_0)& \partial~\psi \over \partial~v
(u_0,v_0)&0 \cr  \partial~\omega
\over \partial~u (u_0,v_0)& \partial~\omega
\over \partial~v
(u_0,v_0)&1\right 

est non nul, ce qui montre que  \partial~\phi \over \partial~u
(u_0,v_0) \partial~\psi \over \partial~v
(u_0,v_0) - \partial~\phi \over \partial~v
(u_0,v_0) \partial~\psi \over \partial~u
(u_0,v_0)\neq~0. Mais ceci
n'est autre que le jacobien au point (u_0,v_0) de
l'application \theta : D \rightarrow~ \mathbb{R}~^2,
(u,v)\mapsto~(\phi(u,v),\psi(u,v)). En posant
(x_0,y_0) = \theta(u_0,v_0) (ce sont
respectivement l'abscisse et l'ordonnée de
F(u_0,v_0)), le théorème d'inversion locale assure
qu'il existe U_0 ouvert contenant (x_0,y_0)
(que l'on peut bien entendu supposer inclus dans D) et V _0
ouvert contenant (x_0,y_0) tel que \theta soit un
C^k difféomorphisme de U_0 sur V _0. On a
bien entendu \theta(\theta^-1(x,y)) = (x,y), soit encore
\phi(\theta^-1(x,y)) = x et \psi(\theta^-1(x,y)) = y. Posons alors
f(x,y) = \omega(\theta^-1(x,y)). La nappe
(U_0,F_U_0) est équivalente à la
nappe (V _0,F \cdot \theta^-1_V
_0) avec

\begin{align*} F \cdot \theta^-1(x,y)& =& O +
\phi(\theta^-1(x,y))\vec\imath +
\psi(\theta^-1(x,y))\vecȷ +
\omega(\theta^-1(x,y))\veck\%&
\\ & =& O + x\vec\imath +
y\vecȷ + f(x,y)\veck \%&
\\ \end{align*}

ce qui démontre le résultat.

\subsection{19.1.5 Intersection de nappes paramétrées}

Le théorème suivant assure que lorsque deux images de nappes paramétrées
se coupent franchement (c'est à dire de manière non tangentielle),
l'intersection des deux est localement l'image d'un arc paramétré
régulier.

Théorème~19.1.5 Soit \Sigma_1 = (D_1,F_1) et
\Sigma_2 = (D_2,F_2) deux nappes paramétrées
régulières dont les images ont un point en commun m_0 =
F_1(u_1,v_1) =
F_2(u_2,v_2). On suppose que les deux nappes
ne sont pas tangentes en ce point commun (c'est-à-dire que le plan
tangent à \Sigma_1 en (u_1,v_1) est distinct du
plan tangent à \Sigma_2 en (u_2,v_2)). Alors
l'intersection des deux nappes est localement l'image d'un arc
paramétré, c'est-à-dire qu'il existe U_1 ouvert contenant
(u_1,v_1) et U_2 ouvert contenant
(u_2,v_2) tels que F_1(U_1) \bigcap
F_2(U_2) soit l'image d'un arc paramétré régulier.

Démonstration Choisissons un repère
(O,\vec\imath,\vecȷ,\veck)
tel que \veck n'appartienne pas à la réunion des
directions des deux plans tangents en m_0 à \Sigma_1 et
\Sigma_2. La propriété à démontrer ne concernant que les images,
elle est invariante par changement de paramétrage~; de plus c'est une
propriété locale puisqu'on a le choix des ouverts U_1 et
U_2. Le théorème précédent nous permet de supposer que les deux
nappes sont des nappes cartésiennes \Sigma_1 :
(x,y)\mapsto~O + x\vec\imath +
y\vecȷ + f_1(x,y)\veck et
\Sigma_2 : (x,y)\mapsto~O +
x\vec\imath + y\vecȷ +
f_2(x,y)\veck (avec le même (x,y) qui
représente l'abscisse et l'ordonnée du point). On a alors m_0 =
O + x_0\vec\imath +
y_0\vecȷ +
f_1(x_0,y_0)\veck = O +
x_0\vec\imath +
y_0\vecȷ +
f_2(x_0,y_0)\veck, si bien
que f_1(x_0,y_0) =
f_2(x_0,y_0). L'intersection des images des
deux nappes est alors \O + x\vec\imath +
y\vecȷ +
z\veck∣z =
f_1(x,y) = f_2(x,y)\. En introduisant
une structure euclidienne qui rende la base
(\vec\imath,\vecȷ,\veck)
orthonormée, le vecteur normal en m_0 à \Sigma_1 est le
vecteur

(\veci + \partial~f_1 \over \partial~x
(x_0,y_0)\veck) ∧
(\vecj + \partial~f_1 \over \partial~y
(x_0,y_0)\veck)

soit encore

- \partial~f_1 \over \partial~x
(x_0,y_0)\vec\imath - \partial~f_1
\over \partial~y
(x_0,y_0)\vecȷ +
\veck

et de même le vecteur normal m_0 à \Sigma_1 est le vecteur
- \partial~f_2 \over \partial~x
(x_0,y_0)\vec\imath - \partial~f_2
\over \partial~y
(x_0,y_0)\vecȷ +
\veck. Comme ces deux vecteurs doivent être
distincts, on peut supposer par exemple que  \partial~f_1
\over \partial~y
(x_0,y_0)\neq~ \partial~f_2
\over \partial~y (x_0,y_0). Posons alors
g(x,y) = f_1(x,y) - f_2(x,y). On a donc
g(x_0,y_0) = 0 et  \partial~g \over \partial~y
(x_0,y_0)\neq~0. Le théorème
des fonctions implicites garantit qu'il existe I_0 intervalle
ouvert contenant x_0 et J_0 intervalle ouvert
contenant y_0 tel que, pour tout x \in I_0, il existe un
unique y \in J_0 vérifiant g(x,y) = 0, autrement dit
f_1(x,y) = f_2(x,y). Si on note y = \phi(x), alors
(quitte à restreindre I_0), \phi est de classe C^k. On
en déduit que

\begin{align*} F_1(I_0 \times
J_0) \bigcap F_1(I_0 \times J_0)&& \%&
\\ & =& \O +
x\vec\imath + y\vecȷ +
z\veck∣x \in I_0, y
\in J_0, z = f_1(x,y) =
f_2(x,y)\ \%&
\\ & =& \O +
x\vec\imath + y\vecȷ +
z\veck∣x \in I_0, y
\in J_0, g(x,y) = 0, z = f_1(x,y)\\%&
\\ & =& \O +
x\vec\imath + y\vecȷ +
z\veck∣x \in I_0, y
\in J_0, y = \phi(x), z = f_1(x,y)\ \%&
\\ & =& \O +
x\vec\imath + \phi(x)\vecȷ +
f_1(x,\phi(x))\veck∣x
\in I_0\ \%& \\
\end{align*}

ce qui montre que l'intersection est l'image de l'arc paramétré
x\mapsto~O + x\vec\imath +
\phi(x)\vecȷ +
f_1(x,\phi(x))\veck (qui est bien entendu
régulier car le vecteur dérivée a 1 pour abscisse).

\subsection{19.1.6 Intersection d'une nappe et de son plan tangent}

Le théorème d'intersection de deux nappes paramétrées régulières non
tangentes s'applique en particulier à une nappe et à un plan. Soit \Sigma =
(D,F) une nappe régulière de classe C^k, m_0 =
F(u_0,v_0) un point de l'image et \Pi un plan affine
contenant m_0. Si \Pi n'est pas tangent à \Sigma en m_0,
l'intersection de \Pi et de la nappe est localement l'image d'un arc
paramétré régulier. Nous allons voir qu'il n'en est plus du tout de même
dans le cas où le plan \Pi est le plan tangent à la surface.

Soit
(0,\vec\imath,\vecȷ,\veck)
un repère de E. Posons F(u,v) = O + \phi(u,v)\vec\imath +
\psi(u,v)\vecȷ + \omega(u,v)\veck et soit
(u_0,v_0) \in D. Le plan tangent \Pi en
(u_0,v_0) a pour équation

\left
\matrix\,x -
\phi(u_0,v_0)& \partial~\phi \over \partial~u
(u_0,v_0)& \partial~\phi \over \partial~v
(u_0,v_0) \cr y -
\psi(u_0,v_0)& \partial~\psi \over \partial~u
(u_0,v_0)& \partial~\psi \over \partial~v
(u_0,v_0) \cr z -
\omega(x_0,y_0)& \partial~\omega \over \partial~u
(u_0,v_0)& \partial~\omega \over \partial~v
(u_0,v_0)\right  = 0

et la position de F(u,v) par rapport à \Pi est donnée par le signe de la
fonction

\Delta(u,v) = \left
\matrix\,\phi(u,v) -
\phi(u_0,v_0)& \partial~\phi \over \partial~u
(u_0,v_0)& \partial~\phi \over \partial~v
(u_0,v_0) \cr \psi(u,v) -
\psi(u_0,v_0)& \partial~\psi \over \partial~u
(u_0,v_0)& \partial~\psi \over \partial~v
(u_0,v_0) \cr \omega(u,v) -
\omega(x_0,y_0)& \partial~\omega \over \partial~u
(u_0,v_0)& \partial~\omega \over \partial~v
(u_0,v_0)\right 

Supposons que la nappe est de classe C^2~; alors \Delta est aussi
de classe C^2 et on a \Delta(u_0,v_0) = 0 (la
première colonne du déterminant est nulle),  \partial~\Delta \over
\partial~u (u_0,v_0) = 0 (la première et la deuxième colonne
du déterminant sont égales) et  \partial~\Delta \over \partial~v
(u_0,v_0) = 0 (la première et la troisième colonne du
déterminant sont égales).

On peut donc appliquer à \Delta la théorie des extremums de fonctions de deux
variables. En utilisant les notations de Monge r_0 =
\partial~^2\Delta \over \partial~u^2
(u_0,v_0), s_0 = \partial~^2\Delta
\over \partial~u\partial~v (u_0,v_0) et t_0
= \partial~^2\Delta \over \partial~v^2
(u_0,v_0), on a trois cas possibles

Premier cas~: s_0^2 - r_0t_0
< 0~; alors on sait que la fonction \Delta présente au point
(u_0,v_0) un extremum local strict~; en particulier,
il existe U_0 ouvert contenant (u_0,v_0) tel
que \Delta soit de signe constant sur U_0
\diagdown\(u_0,v_0)\~; donc
localement, la nappe reste d'un même coté de son plan tangent et
l'intersection des deux est réduite au point m_0~; on dit alors
que le point (u_0,v_0) est un point elliptique de la
nappe.

Deuxième cas~: s_0^2 - r_0t_0
> 0~; alors on sait que la fonction \Delta ne présente pas au
point (u_0,v_0) d'extremum local~; pour tout
U_0 ouvert contenant (u_0,v_0), il existe des
points de U_0 où \Delta est strictement positive et des points de
U_0 où \Delta est strictement négative~; donc au voisinage de
m_0 la nappe a des points de part et d'autre de son plan
tangent~; on dit alors que le point (u_0,v_0) est un
point hyperbolique de la nappe.

Troisième cas~: s_0^2 - r_0t_0 = 0~;
alors on ne sait pas étudier ainsi le signe de \Delta~; on dit alors que le
point (u_0,v_0) est un point parabolique de la nappe.

Remarque~19.1.7 La suite de cette section ne fait pas partie du
programme des classes préparatoires.

Une étude plus fine de la situation peut consister à étudier les lignes
de niveau de la nappe paramétrée dans la direction du plan tangent,
c'est-à-dire l'intersection de la nappe avec des plans parallèles au
plan tangent. Pour faire cette étude, on peut utiliser un repère
(O,\vec\imath,\vecȷ,\veck)
tel que O = m_0, \vec\imath et
\vecȷ sont dans le plan tangent et
\veck n'appartient pas au plan tangent. Alors
localement, la nappe est équivalente à une nappe cartésienne
(x,y)\mapsto~O + x\vec\imath +
y\vecȷ + f(x,y)\veck. Le fait que
\vec\imath et \vecȷ sont dans le plan
tangent va se traduire par  \partial~f \over \partial~x
(x_0,y_0) = \partial~f \over \partial~y
(x_0,y_0) = 0. En utilisant les notations de Monge
r_0 = \partial~^2f \over \partial~x^2
(x_0,y_0), s_0 = \partial~^2f
\over \partial~x\partial~y (x_0,y_0) et t_0
= \partial~^2f \over \partial~y^2
(x_0,y_0), quitte à prendre une base de Sylvester dans
le plan
\mathrmVect(\vec\imath,\vecȷ~)
pour la forme quadratique différentielle seconde
r_0x^2 + 2s_0xy +
t_0y^2, on peut même supposer que s_0 = 0
et que r_0,t_0 \in\-
1,0,1\. Le discriminant de cette forme quadratique
différentielle seconde est alors - r_0t_0. Supposons
que le point n'est pas parabolique. On a donc
r_0t_0\neq~0. Appelons
\epsilon_1 le signe de r_0 et \epsilon_2 le signe de
t_0. On a~:

Lemme~19.1.6 (Morse). Sous ces hypothèses, il existe un ouvert
U_0 contenant (x_0,y_0), un ouvert V
_0 contenant (0,0) et un difféomorphisme \theta : V _0 \rightarrow~
U_0 tel que \theta(0,0) = (x_0,y_0) et

\forall~(X,Y ) \in V _0~, f(\theta(X,Y )) =
\epsilon_1X^2 + \epsilon_ 2Y ^2

Démonstration A une translation près, nous pouvons supposer que
x_0 = y_0 = 0. Appliquons alors la formule de Taylor
avec reste intégral à l'ordre 2. On a donc

\begin{align*} f(x,y) = f(0,0) + x \partial~f
\over \partial~x (0,0) + y \partial~f \over \partial~y
(0,0)&& \%& \\ & +&
\int  _0^1~(1 -
t)\left (x^2 \partial~^2f
\over \partial~x^2 (tx,ty) + 2xy \partial~^2f
\over \partial~x\partial~y (tx,ty) + y^2 \partial~^2f
\over \partial~y^2 (tx,ty)\right )
dt\%& \\ & =& x^2u(x,y) +
2xyv(x,y) + y^2w(x,y) \%& \\
\end{align*}

compte tenu de f(0,0) = \partial~f \over \partial~x
(x_0,y_0) = \partial~f \over \partial~y
(x_0,y_0) = 0, avec des fonctions continues (théorème
sur les intégrales dépendant d'un paramètre)

\begin{align*} u(x,y)& =&
\int  _0^1~(1 - t)
\partial~^2f \over \partial~x^2 (tx,ty) dt \%&
\\ v(x,y)& =&
\int  _0^1~(1 - t)
\partial~^2f \over \partial~x\partial~y (tx,ty) dt\%&
\\ w(x,y)& =&
\int  _0^1~(1 - t)
\partial~^2f \over \partial~y^2 (tx,ty) dt \%&
\\ \end{align*}

On a en particulier u(0,0) = r_0, v(0,0) = s_0 = 0,
w(0,0) = t_0. Posons D = \left
(\matrix\,r_0&s_0
\cr s_0&t_0\right ) =
\left
(\matrix\,r_0&0
\cr 0 &t_0\right ).
L'application T qui à une matrice triangulaire Y associe la matrice
^tY DY a pour différentielle au point Y l'application
H\mapsto~^tHDY + ^tY DH et
donc dT(\mathrmId).H = ^tHD + DH. On
vérifie immédiatement que cette application linéaire
dT(\mathrmId) est bijective de l'espace vectoriel des
matrices triangulaires supérieures sur l'espace vectoriel des matrices
symétriques. On en déduit par le théorème d'inversion locale que T est
un difféomorphisme local d'un voisinage de l'identité (dans l'espace
vectoriel des matrices triangulaires) sur un voisinage W de
T(\mathrmId) = D dans l'espace vectoriel des matrices
symétriques. Or u,v et w sont continues en (0,0). Donc il existe
U_1 ouvert contenant (0,0) tel que pour (x,y) \in U_1 on
ait \left
(\matrix\,u(x,y)&v(x,y)
\cr v(x,y)&w(x,y)\right ) \in W. Posons
alors C(x,y) = T^-1(\left
(\matrix\,u(x,y)&v(x,y)
\cr v(x,y)&w(x,y)\right ). Alors C(x,y)
est une matrice triangulaire, qui dépend de fa\ccon
\mathcal{C}^1 de (x,y) telle que

\left
(\matrix\,u(x,y)&v(x,y)
\cr v(x,y)&w(x,y)\right ) =
^tC(x,y)DC(x,y)

et C(0,0) = T^-1(D) = \mathrmId. On a
alors

\begin{align*} f(x,y)& =& x^2u(x,y) +
2xyv(x,y) + y^2w(x,y)\%& \\ &
=& \left
(\matrix\,x&y\right
)\left
(\matrix\,u(x,y)&v(x,y)
\cr v(x,y)&w(x,y)\right
)\left (\matrix\,x
\cr y\right ) \%&
\\ & =& \left
(\matrix\,x&y \cr
\right )^tC(x,y)DC(x,y)\left
(\matrix\,x \cr
y\right ) \%& \\  &
=& ^tF(x,y)DF(x,y) \%& \\
\end{align*}

avec F(x,y) = C(x,y)\left
(\matrix\,x \cr
y\right ). On a alors  \partial~F \over \partial~x
(x,y) = \partial~C \over \partial~x (x,y)\left
(\matrix\,x \cr
y\right ) + C(x,y)\left
(\matrix\,1 \cr
0\right ) et donc  \partial~F \over \partial~x (0,0)
= C(0,0)\left (\matrix\,1
\cr 0\right ) = \left
(\matrix\,1 \cr
0\right ). De même  \partial~F \over \partial~x
(0,0) = C(0,0)\left
(\matrix\,0 \cr
1\right ) = \left
(\matrix\,0 \cr
1\right ). Donc la différentielle de F en (0,0) est
l'identité de \mathbb{R}~^2. Une nouvelle application du théorème
d'inversion locale assure que F est un difféomorphisme d'un ouvert
U_0 contenant (0,0) sur un ouvert V _0 contenant
(0,0). Appelons \theta le difféomorphisme réciproque. On a alors

\begin{align*} f \cdot \theta(X,Y )& =&
^tF(\theta(X,Y ))DF(\theta(X,Y )) = \left
(\matrix\,X&Y \cr
\right )D\left
(\matrix\,X\cr
Y\right )\%& \\ & =&
r_0X^2 + t_ 0Y ^2 \%&
\\ \end{align*}

Il suffit ensuite de changer \sqrtr_0
 X en X' et \sqrt
t_0Y en Y ' pour obtenir le résultat
souhaité.

Les lignes de niveau de la nappe dans la direction du plan tangent sont
donc les courbes f(x,y) = k et elles sont localement difféomorphes aux
courbes \epsilon_1X^2 + \epsilon_2Y ^2 au
voisinage de (0,0). Si le point (x_0,y_0) est un point
elliptique, \epsilon_1 et \epsilon_2 sont de même signe (que l'on
peut supposer par exemple positif). On voit que l'intersection est vide
pour k < 0, réduite à un point pour k = 0 (c'est le plan
tangent lui même), difféomorphe à une ellipse pour k > 0
(et suffisamment petit). Par contre, si (x_0,y_0) est
un point hyperbolique, \epsilon_1 et \epsilon_2 sont de signe
contraire et l'intersection est difféomorphe à une hyperbole pour
k\neq~0 (suffisamment petit) et à la réunion de
deux droites pour k = 0~; l'intersection avec le plan tangent est donc
localement la réunion de deux courbes passant par le point de contact~;
ces deux courbes séparent les lignes de niveau correspondant aux k
> 0 des lignes de niveau correspondant aux k < 0.
Voici des exemples de lignes de niveau au voisinage de points
elliptiques, hyperboliques ou paraboliques.

[
[

\end{document}
