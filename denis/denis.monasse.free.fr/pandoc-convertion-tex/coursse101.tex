\documentclass[]{article}
\usepackage[T1]{fontenc}
\usepackage{lmodern}
\usepackage{amssymb,amsmath}
\usepackage{ifxetex,ifluatex}
\usepackage{fixltx2e} % provides \textsubscript
% use upquote if available, for straight quotes in verbatim environments
\IfFileExists{upquote.sty}{\usepackage{upquote}}{}
\ifnum 0\ifxetex 1\fi\ifluatex 1\fi=0 % if pdftex
  \usepackage[utf8]{inputenc}
\else % if luatex or xelatex
  \ifxetex
    \usepackage{mathspec}
    \usepackage{xltxtra,xunicode}
  \else
    \usepackage{fontspec}
  \fi
  \defaultfontfeatures{Mapping=tex-text,Scale=MatchLowercase}
  \newcommand{\euro}{€}
\fi
% use microtype if available
\IfFileExists{microtype.sty}{\usepackage{microtype}}{}
\ifxetex
  \usepackage[setpagesize=false, % page size defined by xetex
              unicode=false, % unicode breaks when used with xetex
              xetex]{hyperref}
\else
  \usepackage[unicode=true]{hyperref}
\fi
\hypersetup{breaklinks=true,
            bookmarks=true,
            pdfauthor={},
            pdftitle={Nappes reglees},
            colorlinks=true,
            citecolor=blue,
            urlcolor=blue,
            linkcolor=magenta,
            pdfborder={0 0 0}}
\urlstyle{same}  % don't use monospace font for urls
\setlength{\parindent}{0pt}
\setlength{\parskip}{6pt plus 2pt minus 1pt}
\setlength{\emergencystretch}{3em}  % prevent overfull lines
\setcounter{secnumdepth}{0}
 
/* start css.sty */
.cmr-5{font-size:50%;}
.cmr-7{font-size:70%;}
.cmmi-5{font-size:50%;font-style: italic;}
.cmmi-7{font-size:70%;font-style: italic;}
.cmmi-10{font-style: italic;}
.cmsy-5{font-size:50%;}
.cmsy-7{font-size:70%;}
.cmex-7{font-size:70%;}
.cmex-7x-x-71{font-size:49%;}
.msbm-7{font-size:70%;}
.cmtt-10{font-family: monospace;}
.cmti-10{ font-style: italic;}
.cmbx-10{ font-weight: bold;}
.cmr-17x-x-120{font-size:204%;}
.cmsl-10{font-style: oblique;}
.cmti-7x-x-71{font-size:49%; font-style: italic;}
.cmbxti-10{ font-weight: bold; font-style: italic;}
p.noindent { text-indent: 0em }
td p.noindent { text-indent: 0em; margin-top:0em; }
p.nopar { text-indent: 0em; }
p.indent{ text-indent: 1.5em }
@media print {div.crosslinks {visibility:hidden;}}
a img { border-top: 0; border-left: 0; border-right: 0; }
center { margin-top:1em; margin-bottom:1em; }
td center { margin-top:0em; margin-bottom:0em; }
.Canvas { position:relative; }
li p.indent { text-indent: 0em }
.enumerate1 {list-style-type:decimal;}
.enumerate2 {list-style-type:lower-alpha;}
.enumerate3 {list-style-type:lower-roman;}
.enumerate4 {list-style-type:upper-alpha;}
div.newtheorem { margin-bottom: 2em; margin-top: 2em;}
.obeylines-h,.obeylines-v {white-space: nowrap; }
div.obeylines-v p { margin-top:0; margin-bottom:0; }
.overline{ text-decoration:overline; }
.overline img{ border-top: 1px solid black; }
td.displaylines {text-align:center; white-space:nowrap;}
.centerline {text-align:center;}
.rightline {text-align:right;}
div.verbatim {font-family: monospace; white-space: nowrap; text-align:left; clear:both; }
.fbox {padding-left:3.0pt; padding-right:3.0pt; text-indent:0pt; border:solid black 0.4pt; }
div.fbox {display:table}
div.center div.fbox {text-align:center; clear:both; padding-left:3.0pt; padding-right:3.0pt; text-indent:0pt; border:solid black 0.4pt; }
div.minipage{width:100%;}
div.center, div.center div.center {text-align: center; margin-left:1em; margin-right:1em;}
div.center div {text-align: left;}
div.flushright, div.flushright div.flushright {text-align: right;}
div.flushright div {text-align: left;}
div.flushleft {text-align: left;}
.underline{ text-decoration:underline; }
.underline img{ border-bottom: 1px solid black; margin-bottom:1pt; }
.framebox-c, .framebox-l, .framebox-r { padding-left:3.0pt; padding-right:3.0pt; text-indent:0pt; border:solid black 0.4pt; }
.framebox-c {text-align:center;}
.framebox-l {text-align:left;}
.framebox-r {text-align:right;}
span.thank-mark{ vertical-align: super }
span.footnote-mark sup.textsuperscript, span.footnote-mark a sup.textsuperscript{ font-size:80%; }
div.tabular, div.center div.tabular {text-align: center; margin-top:0.5em; margin-bottom:0.5em; }
table.tabular td p{margin-top:0em;}
table.tabular {margin-left: auto; margin-right: auto;}
div.td00{ margin-left:0pt; margin-right:0pt; }
div.td01{ margin-left:0pt; margin-right:5pt; }
div.td10{ margin-left:5pt; margin-right:0pt; }
div.td11{ margin-left:5pt; margin-right:5pt; }
table[rules] {border-left:solid black 0.4pt; border-right:solid black 0.4pt; }
td.td00{ padding-left:0pt; padding-right:0pt; }
td.td01{ padding-left:0pt; padding-right:5pt; }
td.td10{ padding-left:5pt; padding-right:0pt; }
td.td11{ padding-left:5pt; padding-right:5pt; }
table[rules] {border-left:solid black 0.4pt; border-right:solid black 0.4pt; }
.hline hr, .cline hr{ height : 1px; margin:0px; }
.tabbing-right {text-align:right;}
span.TEX {letter-spacing: -0.125em; }
span.TEX span.E{ position:relative;top:0.5ex;left:-0.0417em;}
a span.TEX span.E {text-decoration: none; }
span.LATEX span.A{ position:relative; top:-0.5ex; left:-0.4em; font-size:85%;}
span.LATEX span.TEX{ position:relative; left: -0.4em; }
div.float img, div.float .caption {text-align:center;}
div.figure img, div.figure .caption {text-align:center;}
.marginpar {width:20%; float:right; text-align:left; margin-left:auto; margin-top:0.5em; font-size:85%; text-decoration:underline;}
.marginpar p{margin-top:0.4em; margin-bottom:0.4em;}
.equation td{text-align:center; vertical-align:middle; }
td.eq-no{ width:5%; }
table.equation { width:100%; } 
div.math-display, div.par-math-display{text-align:center;}
math .texttt { font-family: monospace; }
math .textit { font-style: italic; }
math .textsl { font-style: oblique; }
math .textsf { font-family: sans-serif; }
math .textbf { font-weight: bold; }
.partToc a, .partToc, .likepartToc a, .likepartToc {line-height: 200%; font-weight:bold; font-size:110%;}
.chapterToc a, .chapterToc, .likechapterToc a, .likechapterToc, .appendixToc a, .appendixToc {line-height: 200%; font-weight:bold;}
.index-item, .index-subitem, .index-subsubitem {display:block}
.caption td.id{font-weight: bold; white-space: nowrap; }
table.caption {text-align:center;}
h1.partHead{text-align: center}
p.bibitem { text-indent: -2em; margin-left: 2em; margin-top:0.6em; margin-bottom:0.6em; }
p.bibitem-p { text-indent: 0em; margin-left: 2em; margin-top:0.6em; margin-bottom:0.6em; }
.subsectionHead, .likesubsectionHead { margin-top:2em; font-weight: bold;}
.sectionHead, .likesectionHead { font-weight: bold;}
.quote {margin-bottom:0.25em; margin-top:0.25em; margin-left:1em; margin-right:1em; text-align:justify;}
.verse{white-space:nowrap; margin-left:2em}
div.maketitle {text-align:center;}
h2.titleHead{text-align:center;}
div.maketitle{ margin-bottom: 2em; }
div.author, div.date {text-align:center;}
div.thanks{text-align:left; margin-left:10%; font-size:85%; font-style:italic; }
div.author{white-space: nowrap;}
.quotation {margin-bottom:0.25em; margin-top:0.25em; margin-left:1em; }
h1.partHead{text-align: center}
.sectionToc, .likesectionToc {margin-left:2em;}
.subsectionToc, .likesubsectionToc {margin-left:4em;}
.sectionToc, .likesectionToc {margin-left:6em;}
.frenchb-nbsp{font-size:75%;}
.frenchb-thinspace{font-size:75%;}
.figure img.graphics {margin-left:10%;}
/* end css.sty */

\title{Nappes reglees}
\author{}
\date{}

\begin{document}
\maketitle

\textbf{Warning: 
requires JavaScript to process the mathematics on this page.\\ If your
browser supports JavaScript, be sure it is enabled.}

\begin{center}\rule{3in}{0.4pt}\end{center}

[
[
[]
[

\section{19.2 Nappes réglées}

Remarque~19.2.1 Cette notion n'est pas au programme des classes
préparatoires.

\subsection{19.2.1 Notion de nappe réglée}

Soit I un intervalle de \mathbb{R}~ et soit (D_u)_u\inI une
famille de droites indexée par u. Donnons nous pour chaque u \in I un
point f(u) de D_u et un vecteur directeur
\vecg(u) \in\overrightarrow
D_u \diagdown\0\ et supposons que
(I,f) et (I,\vecg) soient de classe \mathcal{C}^1.
La réunion des droites D_u est alors paramétrée par F : I \times \mathbb{R}~ \rightarrow~
E, (u,v)\mapsto~f(u) + v\vecg(u).
Nous allons montrer qu'à équivalence près, cette nappe paramétrée ne
dépend pas du choix de (I,f) et de (I,\vecg).

Supposons tout d'abord que nous changeons l'arc paramétré (I,f) en
(I,f_1)~; on a alors f_1(u) = f(u) +
\phi(u)\vecg(u) et on vérifie facilement (par exemple à
l'aide d'une structure euclidienne) que \phi est de classe \mathcal{C}^1.
Mais alors l'application \theta : (u,v)\mapsto~(u,v +
\phi(u)) est de classe \mathcal{C}^1 et sa réciproque
(u,w)\mapsto~(u,w - \phi(u)) est aussi de classe
\mathcal{C}^1. Donc \theta est un difféomorphisme de I \times \mathbb{R}~ sur lui même et
on a F \cdot \theta(u,v) = F(u,v + \phi(u)) = f(u) + (v +
\alpha~(u))\vecg(u) = f_1(u) +
v\vecg(u) = F_1(u,v) ce qui montre bien que
les deux nappes sont effectivement équivalentes.

Supposons maintenant que nous changeons (I,\vecg) en
(I,\vecg_1)~; on a alors
\vecg_1(u) = \psi(u)\vecg(u)
avec une application \psi de classe \mathcal{C}^1 qui ne s'annule pas.
Mais alors l'application \theta : (u,v)\mapsto~(u,\psi(u)v)
est de classe \mathcal{C}^1 et sa réciproque
(u,w)\mapsto~(u, w \over \psi(u) )
est aussi de classe \mathcal{C}^1. Donc \theta est un difféomorphisme de I \times
\mathbb{R}~ sur lui même et on a F \cdot \theta(u,v) = F(u,\psi(u)v) = f(u) +
\beta~(u)v\vecg(u) = f(u) +
v\vecg_1(u) = F_1(u,v) ce qui
montre que les deux nappes sont bien équivalentes.

Définition~19.2.1 Une telle nappe sera appelée une nappe réglée de
classe \mathcal{C}^1. Les droites D_u sont appelées les
génératrices de la nappe. Un arc (I,f) de classe \mathcal{C}^1 tel que
\forall~u \in I, f(u) \in D_u~ sera appelé une
directrice de la nappe~; une directrice plane est appelée une base de la
nappe.

\subsection{19.2.2 Plan tangent à une nappe réglée}

Donnons nous une nappe réglée de classe \mathcal{C}^1,
(D_u)_u\inI et soit F(u,v) = f(u) +
v\vecg(u) un paramétrage admissible de cette nappe.
Soit u_0 \in I. On a alors  \partial~F \over \partial~u
(u_0,v) = f'(u_0) +
v\vecg'(u_0) et  \partial~F \over
\partial~v (u_0,v) =\vec g(u_0). Un
vecteur normal à la nappe est alors le vecteur  \partial~F
\over \partial~u (u_0,v) ∧ \partial~F \over
\partial~v (u_0,v) = f'(u_0) ∧\vec
g(u_0) + v\vecg'(u_0)
∧\vec g(u_0).

Trois cas sont alors possibles~:

\begin{itemize}
\itemsep1pt\parskip0pt\parsep0pt
\item
  Premier cas La famille
  (f'(u_0),\vecg(u_0),\vecg'(u_0))
  est libre. Alors, pout tout v \in \mathbb{R}~, la famille ( \partial~F
  \over \partial~u (u_0,v), \partial~F \over
  \partial~v (u_0,v)) est libre. Lorsque v varie, le vecteur normal
  tourne dans le plan
  \mathrmVect(f'(u_0~)
  ∧\vec
  g(u_0),\vecg'(u_0)
  ∧\vec g(u_0)) en occupant toutes les
  directions sauf une, \mathbb{R}~g'(u_0) ∧\vec
  g(u_0), qui est la direction limite lorsque v tend vers
  ±\infty~~; autrement dit, lorsque le point se déplace sur la génératrice, le
  plan tangent tourne autour de celle-ci (il doit forcément la contenir
  puisque c'est une courbe tracée sur la surface et qu'elle est sa
  propre tangente) en occupant toutes les positions sauf une, la
  position limite à l'infini.
\item
  Deuxième cas
  \mathrmrg(f'(u_0),\vecg(u_0),\vecg'(u_0~))
  = 2. Alors le plan tangent, lorsqu'il existe, doit contenir la
  génératrice et être parallèle au plan
  \mathrmVect(f'(u_0),\vecg(u_0),\vecg'(u_0~))~;
  il doit être constant le long de la génératrice
  D_u_0. D'autre part, les deux vecteurs
  f'(u_0) ∧\vec g(u_0) et
  \vecg'(u_0) ∧\vec
  g(u_0) ne peuvent pas être tous deux nuls, et donc il
  existe au plus un v tel que  \partial~F \over \partial~u
  (u_0,v) ∧ \partial~F \over \partial~v (u_0,v) =
  f'(u_0) ∧\vec g(u_0) +
  v\vecg'(u_0) ∧\vec
  g(u_0) = 0, autrement dit il y a au plus un point singulier
  sur la génératrice~; en ce point le plan tangent n'existe pas.
\item
  Troisième cas
  \mathrmrg(f'(u_0),\vecg(u_0),\vecg'(u_0~))
  = 1. Alors, pour tout v \in \mathbb{R}~, on a  \partial~F \over \partial~u
  (u_0,v) ∧ \partial~F \over \partial~v (u_0,v) =
  f'(u_0) ∧\vec g(u_0) +
  v\vecg'(u_0) ∧\vec
  g(u_0) = 0. Tout point de la génératrice est singulier et
  il n'y a de plan tangent en aucun point de la génératrice. On dit que
  la génératrice D_u_0 est une génératrice singulière.
\end{itemize}

\subsection{19.2.3 Nappes cylindriques. Nappes coniques}

Définition~19.2.2 Soit \vecD une direction de droite
dans E. On appelle nappe cylindrique de direction
\vecD toute nappe réglée de classe \mathcal{C}^1,
(D_u)_u\inI telle que toutes les droites D_u
soient parallèles à \vecD.

Soit \veck un vecteur directeur de
\vecD. On peut donc choisir un paramétrage F(u,v) =
f(u) + v\vecg(u) avec \forall~~u \in
I, \vecg(u) =\vec k. On a alors
\vecg constante et donc \vecg'(u)
= 0. On voit donc que le premier cas de l'étude du plan tangent est
exclu et qu'il n'y a que deux possibilités pour un u_0 \in I.

Premier cas
f'(u_0)∉\vecD.
Alors  \partial~F \over \partial~u (u_0,v) ∧ \partial~F
\over \partial~v (u_0,v) = f'(u_0)
∧\vec k. Tout point de la génératrice est régulier et
le plan tangent est constant le long de la génératrice.

Deuxième cas f'(u_0) \in\vec D. Alors la
génératrice est singulière et le plan tangent n'existe en aucun point de
la génératrice.

Définition~19.2.3 Soit S un point de E. On appelle nappe conique de
sommet S toute nappe réglée de classe \mathcal{C}^1,
(D_u)_u\inI telle que toutes les droites D_u
passent par le point S.

On peut alors choisir un paramétrage F(u,v) = f(u) +
v\vecg(u) avec \forall~~u \in I, f(u)
= S. Donc f est constante et f'(u) = 0. On voit donc que le premier cas
de l'étude du plan tangent est exclu et qu'il n'y a que deux
possibilités pour un u_0 \in I.

Premier cas
(\vecg(u_0),\vecg'(u_0))
est libre. Alors  \partial~F \over \partial~u (u_0,v) ∧ \partial~F
\over \partial~v (u_0,v) =
v\vecg'(u_0),\vecg(u_0).
Tout point de la génératrice différent du sommet est régulier et le plan
tangent est constant le long de la génératrice.

Deuxième cas
(\vecg(u_0),\vecg'(u_0))
est liée. Alors la génératrice est singulière et le plan tangent
n'existe en aucun point de la génératrice.

Remarque~19.2.2 Les deux types de nappes réglées que nous venons
d'étudier vérifient la propriété remarquable que le plan tangent est
constant le long de chaque génératrice~; on appelle de telles nappes
réglées des nappes développables. Un autre type de nappes développables
peut être construit en prenant l'ensemble des tangentes à une courbe
gauche régulière. Soit (I,f) un tel arc paramétré régulier. On peut
paramétrer la tangente D_u par
v\mapsto~f(u) + vf'(u)~; on peut donc prendre
\vecg(u) = f'(u). La famille
(f'(u_0),\vecg(u_0),\vecg'(u_0))
est donc la famille (f'(u_0),f''(u_0)). Elle est donc
de rang au plus 2. On a deux possibilités~: soit u_0 est un
point birégulier de (I,f), alors la famille est de rang 2 et le plan
tangent est donc constant le long de la génératrice (dont le seul point
singulier est d'ailleurs v = 0, c'est-à-dire le point de contact de la
tangente), soit u_0 n'est pas birégulier et la génératrice est
singulière. En fait on peut montrer que ces trois types de nappes
(nappes cylindriques, nappes coniques et ensemble des tangentes à une
courbe) épuisent, au moins localement, les nappes développables.

[
[
[
[

\end{document}
