\documentclass[]{article}
\usepackage[T1]{fontenc}
\usepackage{lmodern}
\usepackage{amssymb,amsmath}
\usepackage{ifxetex,ifluatex}
\usepackage{fixltx2e} % provides \textsubscript
% use upquote if available, for straight quotes in verbatim environments
\IfFileExists{upquote.sty}{\usepackage{upquote}}{}
\ifnum 0\ifxetex 1\fi\ifluatex 1\fi=0 % if pdftex
  \usepackage[utf8]{inputenc}
\else % if luatex or xelatex
  \ifxetex
    \usepackage{mathspec}
    \usepackage{xltxtra,xunicode}
  \else
    \usepackage{fontspec}
  \fi
  \defaultfontfeatures{Mapping=tex-text,Scale=MatchLowercase}
  \newcommand{\euro}{€}
\fi
% use microtype if available
\IfFileExists{microtype.sty}{\usepackage{microtype}}{}
\ifxetex
  \usepackage[setpagesize=false, % page size defined by xetex
              unicode=false, % unicode breaks when used with xetex
              xetex]{hyperref}
\else
  \usepackage[unicode=true]{hyperref}
\fi
\hypersetup{breaklinks=true,
            bookmarks=true,
            pdfauthor={},
            pdftitle={Applications lineaires continues},
            colorlinks=true,
            citecolor=blue,
            urlcolor=blue,
            linkcolor=magenta,
            pdfborder={0 0 0}}
\urlstyle{same}  % don't use monospace font for urls
\setlength{\parindent}{0pt}
\setlength{\parskip}{6pt plus 2pt minus 1pt}
\setlength{\emergencystretch}{3em}  % prevent overfull lines
\setcounter{secnumdepth}{0}
 
/* start css.sty */
.cmr-5{font-size:50%;}
.cmr-7{font-size:70%;}
.cmmi-5{font-size:50%;font-style: italic;}
.cmmi-7{font-size:70%;font-style: italic;}
.cmmi-10{font-style: italic;}
.cmsy-5{font-size:50%;}
.cmsy-7{font-size:70%;}
.cmex-7{font-size:70%;}
.cmex-7x-x-71{font-size:49%;}
.msbm-7{font-size:70%;}
.cmtt-10{font-family: monospace;}
.cmti-10{ font-style: italic;}
.cmbx-10{ font-weight: bold;}
.cmr-17x-x-120{font-size:204%;}
.cmsl-10{font-style: oblique;}
.cmti-7x-x-71{font-size:49%; font-style: italic;}
.cmbxti-10{ font-weight: bold; font-style: italic;}
p.noindent { text-indent: 0em }
td p.noindent { text-indent: 0em; margin-top:0em; }
p.nopar { text-indent: 0em; }
p.indent{ text-indent: 1.5em }
@media print {div.crosslinks {visibility:hidden;}}
a img { border-top: 0; border-left: 0; border-right: 0; }
center { margin-top:1em; margin-bottom:1em; }
td center { margin-top:0em; margin-bottom:0em; }
.Canvas { position:relative; }
li p.indent { text-indent: 0em }
.enumerate1 {list-style-type:decimal;}
.enumerate2 {list-style-type:lower-alpha;}
.enumerate3 {list-style-type:lower-roman;}
.enumerate4 {list-style-type:upper-alpha;}
div.newtheorem { margin-bottom: 2em; margin-top: 2em;}
.obeylines-h,.obeylines-v {white-space: nowrap; }
div.obeylines-v p { margin-top:0; margin-bottom:0; }
.overline{ text-decoration:overline; }
.overline img{ border-top: 1px solid black; }
td.displaylines {text-align:center; white-space:nowrap;}
.centerline {text-align:center;}
.rightline {text-align:right;}
div.verbatim {font-family: monospace; white-space: nowrap; text-align:left; clear:both; }
.fbox {padding-left:3.0pt; padding-right:3.0pt; text-indent:0pt; border:solid black 0.4pt; }
div.fbox {display:table}
div.center div.fbox {text-align:center; clear:both; padding-left:3.0pt; padding-right:3.0pt; text-indent:0pt; border:solid black 0.4pt; }
div.minipage{width:100%;}
div.center, div.center div.center {text-align: center; margin-left:1em; margin-right:1em;}
div.center div {text-align: left;}
div.flushright, div.flushright div.flushright {text-align: right;}
div.flushright div {text-align: left;}
div.flushleft {text-align: left;}
.underline{ text-decoration:underline; }
.underline img{ border-bottom: 1px solid black; margin-bottom:1pt; }
.framebox-c, .framebox-l, .framebox-r { padding-left:3.0pt; padding-right:3.0pt; text-indent:0pt; border:solid black 0.4pt; }
.framebox-c {text-align:center;}
.framebox-l {text-align:left;}
.framebox-r {text-align:right;}
span.thank-mark{ vertical-align: super }
span.footnote-mark sup.textsuperscript, span.footnote-mark a sup.textsuperscript{ font-size:80%; }
div.tabular, div.center div.tabular {text-align: center; margin-top:0.5em; margin-bottom:0.5em; }
table.tabular td p{margin-top:0em;}
table.tabular {margin-left: auto; margin-right: auto;}
div.td00{ margin-left:0pt; margin-right:0pt; }
div.td01{ margin-left:0pt; margin-right:5pt; }
div.td10{ margin-left:5pt; margin-right:0pt; }
div.td11{ margin-left:5pt; margin-right:5pt; }
table[rules] {border-left:solid black 0.4pt; border-right:solid black 0.4pt; }
td.td00{ padding-left:0pt; padding-right:0pt; }
td.td01{ padding-left:0pt; padding-right:5pt; }
td.td10{ padding-left:5pt; padding-right:0pt; }
td.td11{ padding-left:5pt; padding-right:5pt; }
table[rules] {border-left:solid black 0.4pt; border-right:solid black 0.4pt; }
.hline hr, .cline hr{ height : 1px; margin:0px; }
.tabbing-right {text-align:right;}
span.TEX {letter-spacing: -0.125em; }
span.TEX span.E{ position:relative;top:0.5ex;left:-0.0417em;}
a span.TEX span.E {text-decoration: none; }
span.LATEX span.A{ position:relative; top:-0.5ex; left:-0.4em; font-size:85%;}
span.LATEX span.TEX{ position:relative; left: -0.4em; }
div.float img, div.float .caption {text-align:center;}
div.figure img, div.figure .caption {text-align:center;}
.marginpar {width:20%; float:right; text-align:left; margin-left:auto; margin-top:0.5em; font-size:85%; text-decoration:underline;}
.marginpar p{margin-top:0.4em; margin-bottom:0.4em;}
.equation td{text-align:center; vertical-align:middle; }
td.eq-no{ width:5%; }
table.equation { width:100%; } 
div.math-display, div.par-math-display{text-align:center;}
math .texttt { font-family: monospace; }
math .textit { font-style: italic; }
math .textsl { font-style: oblique; }
math .textsf { font-family: sans-serif; }
math .textbf { font-weight: bold; }
.partToc a, .partToc, .likepartToc a, .likepartToc {line-height: 200%; font-weight:bold; font-size:110%;}
.chapterToc a, .chapterToc, .likechapterToc a, .likechapterToc, .appendixToc a, .appendixToc {line-height: 200%; font-weight:bold;}
.index-item, .index-subitem, .index-subsubitem {display:block}
.caption td.id{font-weight: bold; white-space: nowrap; }
table.caption {text-align:center;}
h1.partHead{text-align: center}
p.bibitem { text-indent: -2em; margin-left: 2em; margin-top:0.6em; margin-bottom:0.6em; }
p.bibitem-p { text-indent: 0em; margin-left: 2em; margin-top:0.6em; margin-bottom:0.6em; }
.subsectionHead, .likesubsectionHead { margin-top:2em; font-weight: bold;}
.sectionHead, .likesectionHead { font-weight: bold;}
.quote {margin-bottom:0.25em; margin-top:0.25em; margin-left:1em; margin-right:1em; text-align:justify;}
.verse{white-space:nowrap; margin-left:2em}
div.maketitle {text-align:center;}
h2.titleHead{text-align:center;}
div.maketitle{ margin-bottom: 2em; }
div.author, div.date {text-align:center;}
div.thanks{text-align:left; margin-left:10%; font-size:85%; font-style:italic; }
div.author{white-space: nowrap;}
.quotation {margin-bottom:0.25em; margin-top:0.25em; margin-left:1em; }
h1.partHead{text-align: center}
.sectionToc, .likesectionToc {margin-left:2em;}
.subsectionToc, .likesubsectionToc {margin-left:4em;}
.sectionToc, .likesectionToc {margin-left:6em;}
.frenchb-nbsp{font-size:75%;}
.frenchb-thinspace{font-size:75%;}
.figure img.graphics {margin-left:10%;}
/* end css.sty */

\title{Applications lineaires continues}
\author{}
\date{}

\begin{document}
\maketitle

\textbf{Warning: 
requires JavaScript to process the mathematics on this page.\\ If your
browser supports JavaScript, be sure it is enabled.}

\begin{center}\rule{3in}{0.4pt}\end{center}

[
[
[]
[

\section{5.2 Applications linéaires continues}

\subsection{5.2.1 Caractérisations et normes des applications linéaires
continues}

Théorème~5.2.1 Soit E et F deux espaces vectoriels normés et u une
application linéaire de E dans F. Alors les conditions suivantes sont
équivalentes

\begin{itemize}
\itemsep1pt\parskip0pt\parsep0pt
\item
  (i) u est continue
\item
  (ii) u est continue au point 0
\item
  (iii) u est bornée sur la boule unité B'(0,1)
\item
  (iv) u est bornée sur la sphère unité S(0,1)
\item
  (v) il existe k ≥ 0 tel que \forall~~x \in E,
  \u(x)\ \leq
  k\x\
\item
  (vi) u est lipschitzienne
\end{itemize}

Démonstration (i) \rigtharrow~(ii) est évident.

(ii) \rigtharrow~(iii) Puisque u(0) = 0 et que u est continue en 0, il existe \eta
> 0 tel que
\x\ < \eta
\rigtharrow~\ u(x)\ \leq 1~; si x \in
B'(0,1), on a \ \eta \over 2
x\ \leq \eta \over 2 < \eta
soit \u( \eta \over 2
x)\ \leq 1 soit encore
\u(x)\ \leq 2
\over \eta .

(iii) \rigtharrow~(iv) est évident puisque S(0,1) \subset~ B'(0,1)

(iv) \rigtharrow~(v) soit k =\
sup_x\inS(0,1)\u(x)\
et soit x \in E. Si x = 0, on a
\u(x)\ \leq
k\x\~; si
x\neq~0, on a  x \over
\x\ \in S(0,1), soit
\u( x \over
\x\
)\ \leq k, soit
\u(x)\ \leq
k\x\.

(v) \rigtharrow~(vi) on a \u(x) -
u(y)\ =\ u(x -
y)\ \leq k\x -
y\, donc u est k-lipschitzienne.

(vi) \rigtharrow~(i) est évident

Théorème~5.2.2 Soit u : E \rightarrow~ F une application linéaire continue. Alors
on a l'égalité

\begin{align*}
sup_x\neq~0~
\u(x)\
\over
\x\ & =&
sup_\x\\leq1~\u(x)\
=\
sup_\x\=1\u(x)\
\%& \\ & =&
inf~ \k ≥
0∣\forall~~x \in E,
\u(x)\ \leq
k\x\\\%&
\\ \end{align*}

Ce nombre est appelé la norme de l'application linéaire u et noté
\u\~; on a
\forall~~x \in E,
\u(x)\
\leq\
u\\,\x\.

Démonstration Appelons M_1,M_2,M_3 et
M_4 les nombres en question dans l'ordre ci dessus. On a
clairement M_1 = M_3 puisque S(0,1) =
\ x \over
\x\
∣x\mathrel\neq~0\.
Comme S(0,1) \subset~ B'(0,1), on a M_3 \leq M_2. La
démonstration ci dessus de (iv) \rigtharrow~(v) nous a montré que
\forall~~x \in E,
\u(x)\ \leq
M_2\x\, soit
M_4 \leq M_2. Remarquons de plus que

\k ≥
0∣\forall~~x \in E,
\u(x)\ \leq
k\x\\
= \⋂
_x\neq~0[
\u(x)\
\over
\x\ ,+\infty~[

est un fermé de \mathbb{R}~ (intersection de fermés), donc contient sa borne
inférieure~; on a donc \forall~~x \in E,
\u(x)\ \leq
M_4\x\, soit
pour x\neq~0, 
\u(x)\
\over
\x\ \leq M_4
et donc M_1 \leq M_4. Si on récapitule, on a montré que
M_1 \leq M_4 \leq M_2 \leq M_3 = M_1
ce qui montre les égalités. On a vu de plus que
\forall~~x \in E,
\u(x)\ \leq
M_4\x\, soit
encore \forall~~x \in E,
\u(x)\
\leq\
u\\,\x\.

\subsection{5.2.2 L'espace vectoriel normé des applications linéaires
continues de E dans F}

Théorème~5.2.3 L'application
u\mapsto~\u\
est une norme sur l'espace vectoriel des applications linéaires
continues de E dans F.

Démonstration La formule \forall~~x \in E,
\u(x)\
\leq\
u\\,\x\
montre que \u\ = 0 \rigtharrow~ u
= 0, la réciproque étant évidente. La formule
\u\
=\
sup_\x\=1\u(x)\
permet de montrer sans problème l'homogénéité et l'inégalité
triangulaire.

Théorème~5.2.4 Soit u : E \rightarrow~ F et v : F \rightarrow~ G linéaires continues. Alors
\v \cdot u\
\leq\
v\\,\u\.

Démonstration On a \v \cdot
u(x)\ \leq\
v\\,\u(x)\
\leq\
v\\,\u\\,\x\
et on sait que \v \cdot u\
est le plus petit k tel que \forall~~x \in E,
\v \cdot u(x)\ \leq
k\x\. Soit
\v \cdot u\
\leq\
v\\,\u\.

Remarque~5.2.1 Cette propriété (dite de sous multiplicativité) de la
norme est particulièrement commode pour ce type de norme sur l'espace
vectoriel normé des applications linéaires continues de E dans F (normes
dites subordonnées à des normes sur E et F)~; c'est ainsi que dans un
espace de matrices, on aura souvent intérêt à poser
\A\
=\
sup_\X\=1\AX\
de fa\ccon à disposer d'une inégalité
\AB\
\leq\
A\\,\B\.

Théorème~5.2.5 Si F est un espace vectoriel normé complet, l'espace
vectoriel normé des applications linéaires continues de E dans F est
complet.

Démonstration Soit (u_n) une suite d'applications linéaires
continues qui est une suite de Cauchy pour la norme que l'on vient de
définir. Soit x \in E, on a \u_p(x) -
u_q(x)\ \leq\
u_p -
u_q\\,\x\,
ce qui montre que la suite (u_n(x)) est une suite de Cauchy
dans F~; comme F est complet, elle converge vers une limite qui dépend
de x et que nous noterons u(x). La relation u_n(\alpha~x + \beta~y) =
\alpha~u_n(x) + \beta~u_n(y) donne par passage à la limite u(\alpha~x +
\beta~y) = \alpha~u(x) + \beta~u(y) ce qui montre que u est linéaire. Soit \epsilon
> 0 et N \in \mathbb{N}~ tel que p,q ≥ N \rigtharrow~\
u_p - u_q\ < \epsilon.
Pour x \in E, q > n ≥ N, on a
\u_q(x) -
u_n(x)\ \leq
\epsilon\x\. Faisons tendre q
vers + \infty~~; on obtient \u(x) -
u_n(x)\ \leq
\epsilon\x\~; ceci montre
tout d'abord que u - u_n est continue et donc u = (u -
u_n) + u_n aussi, et que n ≥ N
\rigtharrow~\ u - u_n\ \leq
\epsilon, et donc que u_n converge vers u au sens de la norme des
applications linéaires continues~; ceci achève la démonstration.

\subsection{5.2.3 Equivalence des normes}

Définition~5.2.1 Soit E un K-espace vectoriel . On dit que deux normes
\._1 et
\._2 sur E
sont équivalentes si les distances associées sont équivalentes,
c'est-à-dire s'il existe \alpha~,\beta~ > 0 tels que

\forall~~x \in E,\quad
\alpha~\x_1
\leq\ x_2 \leq
\beta~\x_1

Théorème~5.2.6 Deux normes sont équivalentes si et seulement si~elles
définissent la même topologie.

Démonstration Si deux normes sont équivalentes, les distances associées
aussi et donc elles définissent la même topologie. Inversement,
supposons que les deux normes définissent la même topologie. Alors
\mathrmId_E :
(E,\._1) \rightarrow~
(E,\._2) est
linéaire continue, donc il existe k ≥ 0 tel que
\forall~~x \in E,
\\mathrmId_E(x)_2
\leq k\x_1 soit
encore \forall~~x \in E,
\x_2 \leq
k\x_1 et il
est alors clair que k\neq~0. De même avec
\mathrmId_E :
(E,\._2) \rightarrow~
(E,\._1), on
peut trouver un k' > 0 tel que \forall~~x
\in E, \x_1 \leq
k'\x_2, soit
\forall~x \in E, 1 \over k'~
\x_1
\leq\ x_2 \leq
k\x_1.

\subsection{5.2.4 Caractérisation des applications bilinéaires continues}

Théorème~5.2.7 Soit E_1,E_2 et F des espaces
vectoriels normés et u une application bilinéaire de E_1 \times
E_2 dans F. Alors les conditions suivantes sont équivalentes

\begin{itemize}
\itemsep1pt\parskip0pt\parsep0pt
\item
  (i) u est continue
\item
  (ii) u est continue au point (0,0)
\item
  (iii) u est bornée sur B'(0,1) \times B'(0,1)
\item
  (iv) u est bornée sur S(0,1) \times S(0,1)
\item
  (v) il existe k ≥ 0 tel que
  \forall~(x_1,x_2) \in E_1~
  \times E_2,
  \u(x_1,x_2)\
  \leq
  k\x_1\\,\x_2\
\end{itemize}

Démonstration Tout à fait similaire à celle effectuée pour les
applications linéaires, sauf en ce qui concerne (v) \rigtharrow~(i) (car il n'y a
plus l'intermédiaire (vi)). Mais on a, si (a_1,a_2) \in
E_1 \times E_2

\begin{align*} u(x_1,x_2) -
u(a_1,a_2)& =& u(a_1 -
x_1,a_2 - x_2) + u(x_1 -
a_1,a_2)\%& \\
\text & & +u(a_1,x_2 -
a_2) \%& \\
\end{align*}

(facile) et donc, si (v) est vérifiée

\begin{align*}
\u(x_1,x_2) -
u(a_1,a_2)& \leq&
k\a_1 -
x_1\\,\a_2
- x_2\ \%&
\\ & +&
k\x_1 -
a_1\\,\a_2\
+
k\a_1\\,\x_2
- a_2\\%&
\\ \end{align*}

ce qui montre clairement la continuité (non uniforme) de u au point
(a_1,a_2).

Exemple~5.2.1 La formule \v \cdot
u\ \leq\
v\\,\u\
montre que l'application bilinéaire de composition est continue sur les
espaces d'applications linéaires continues adéquats.

[
[
[
[

\end{document}
