\documentclass{article}
\usepackage[T1]{fontenc}
\usepackage{lmodern}
\usepackage{amsmath, amssymb}
\usepackage{ifxetex, ifluatex}

% Use inputenc if using pdfTeX
\ifnum 0\ifxetex 1\fi\ifluatex 1\fi=0
  \usepackage[utf8]{inputenc}
\fi

% Use fontspec and other XeTeX/LuaTeX specific packages if not using pdfTeX
\ifxetex
  \usepackage{fontspec}
\else\ifluatex
  \usepackage{fontspec}
\fi\fi

% Enable microtype if available
\IfFileExists{microtype.sty}{\usepackage{microtype}}{}

% Hyperref setup
\usepackage{hyperref}
\hypersetup{
  breaklinks=true,
  bookmarks=true,
  colorlinks=true,
  citecolor=blue,
  urlcolor=blue,
  linkcolor=magenta,
  pdfborder={0 0 0}
}
\urlstyle{same}

% Set subsection formatting
\setlength{\parindent}{0pt}
\setlength{\parskip}{6pt plus 2pt minus 1pt}
\setlength{\emergencystretch}{3em}
\setcounter{secnumdepth}{0}

\author{}
\date{}

\begin{document}

\textbf{Warning: 
requires JavaScript to process the mathematics on this page.\\ If your
browser supports JavaScript, be sure it is enabled.}

\begin{center}\rule{3in}{0.4pt}\end{center}

% Table of Contents
\section*{Contents}
\begin{itemize}
  \item 1
    \begin{itemize}
      \item 1.1
      \item 1.2
      \item 1.3
      \item 1.4
      \item 1.5
      \item 1.6
    \end{itemize}
  \item 2
    \begin{itemize}
      \item 2.1 Généralités sur les espaces vectoriels
      \item 2.2 Bases et dimension
      \item 2.3
      \item 2.4
      \item 2.5
      \item 2.6
      \item 2.7
      \item 2.8
    \end{itemize}
  \item 3
    \begin{itemize}
      \item 3.1
      \item 3.2
      \item 3.3
    \end{itemize}
  \item 4
    \begin{itemize}
      \item 4.1 Eléments de topologie générale
      \item 4.2 Espaces métriques
      \item 4.3
      \item 4.4
      \item 4.5
      \item 4.6
      \item 4.7
      \item 4.8
      \item 4.9
    \end{itemize}
  \item 5
    \begin{itemize}
      \item 5.1 Notion d'espace vectoriel normé
      \item 5.2 Applications linéaires continues
      \item 5.3 Espaces vectoriels normés de dimensions finies
      \item 5.4 Compléments: le théorème de Baire et ses conséquences
      \item 5.5 Compléments: convexité dans les espaces vectoriels normés
    \end{itemize}
  \item 6
    \begin{itemize}
      \item 6.1
      \item 6.2
      \item 6.3
    \end{itemize}
  \item 7
    \begin{itemize}
      \item 7.1
      \item 7.2
      \item 7.3
      \item 7.4 Séries absolument convergentes
      \item 7.5 Séries semi-convergentes
      \item 7.6
      \item 7.7
      \item 7.8
      \item 7.9 Compléments: développements asymptotiques, analyse numérique
    \end{itemize}
  \item 8
    \begin{itemize}
      \item 8.1
      \item 8.2
      \item 8.3 Fonctions réelles d'une variable réelle
      \item 8.4 Fonctions vectorielles d'une variable réelle
      \item 8.5
      \item 8.6 Analyse numérique des fonctions d'une variable
    \end{itemize}
  \item 9
    \begin{itemize}
      \item 9.1 Subdivisions, approximation des fonctions
      \item 9.2 Intégrale des fonctions réglées sur un segment
      \item 9.3
      \item 9.4
      \item 9.5 Intégration sur un intervalle quelconque: fonctions à valeurs réelles positives
      \item 9.6 Intégration sur un intervalle quelconque: fonctions à valeurs complexes
      \item 9.7 Développements asymptotiques et analyse numérique
      \item 9.8 Généralités sur les intégrales impropres
      \item 9.9 Intégrale des fonctions réelles positives
      \item 9.10 Convergence absolue, semi-convergence
    \end{itemize}
  \item 10 Suites et séries de fonctions
    \begin{itemize}
      \item 10.1
      \item 10.2
      \item 10.3 Intégrales dépendant d'un paramètre
    \end{itemize}
  \item 11
    \begin{itemize}
      \item 11.1 Convergence des séries entières
      \item 11.2 Somme d'une série entière
      \item 11.3
      \item 11.4 Application aux endomorphismes continus et aux matrices
    \end{itemize}
  \item 12
    \begin{itemize}
      \item 12.1
      \item 12.2
      \item 12.3 Réduction des formes quadratiques en dimension finie
      \item 12.4
      \item 12.5 Endomorphismes et formes quadratiques
      \item 12.6 Endomorphismes d'un espace euclidien
    \end{itemize}
  \item 13 Formes hermitiennes
    \begin{itemize}
      \item 13.1 Compléments sur la conjugaison
      \item 13.2
      \item 13.3
      \item 13.4 Endomorphismes d'un espace hermitien
    \end{itemize}
  \item 14 Séries de Fourier
    \begin{itemize}
      \item 14.1 Introduction: transformée de Fourier sur les groupes abéliens finis
      \item 14.2
      \item 14.3
      \item 14.4 Fonctions périodiques de période T
      \item 14.5 Produit de convolution
    \end{itemize}
  \item 15 Calcul différentiel
    \begin{itemize}
      \item 15.1 Dérivées partielles
      \item 15.2
      \item 15.3
      \item 15.4 Fonctions implicites et inversion locale
    \end{itemize}
  \item 16 Équations différentielles
    \begin{itemize}
      \item 16.1 Notions générales
      \item 16.2 Théorie de Cauchy-Lipschitz
      \item 16.3 Équations différentielles linéaires d'ordre 1
      \item 16.4 Équation différentielle linéaire d'ordre n
      \item 16.5 Équations différentielles non linéaires
      \item 16.6 Analyse numérique des équations différentielles
    \end{itemize}
  \item 17 Espaces affines
    \begin{itemize}
      \item 17.1 Généralités sur les espaces affines
      \item 17.2
      \item 17.3
      \item 17.4
    \end{itemize}
  \item 18
    \begin{itemize}
      \item 18.1
      \item 18.2
      \item 18.3 Problèmes classiques sur les courbes
      \item 18.4 Étude métrique des arcs
    \end{itemize}
  \item 19
    \begin{itemize}
      \item 19.1
      \item 19.2
      \item 19.3
      \item 19.4
    \end{itemize}
  \item 20 Intégrales curvilignes, intégrales multiples
    \begin{itemize}
      \item 20.1 Intégrales curvilignes
      \item 20.2 Intégrales multiples
      \item 20.3 Calcul des intégrales doubles et triples
      \item 20.4 Introduction aux intégrales de surface
    \end{itemize}
\end{itemize}

\end{document}
