\documentclass[]{article}
\usepackage[T1]{fontenc}
\usepackage{lmodern}
\usepackage{amssymb,amsmath}
\usepackage{ifxetex,ifluatex}
\usepackage{fixltx2e} % provides \textsubscript
% use upquote if available, for straight quotes in verbatim environments
\IfFileExists{upquote.sty}{\usepackage{upquote}}{}
\ifnum 0\ifxetex 1\fi\ifluatex 1\fi=0 % if pdftex
  \usepackage[utf8]{inputenc}
\else % if luatex or xelatex
  \ifxetex
    \usepackage{mathspec}
    \usepackage{xltxtra,xunicode}
  \else
    \usepackage{fontspec}
  \fi
  \defaultfontfeatures{Mapping=tex-text,Scale=MatchLowercase}
  \newcommand{\euro}{€}
\fi
% use microtype if available
\IfFileExists{microtype.sty}{\usepackage{microtype}}{}
\ifxetex
  \usepackage[setpagesize=false, % page size defined by xetex
              unicode=false, % unicode breaks when used with xetex
              xetex]{hyperref}
\else
  \usepackage[unicode=true]{hyperref}
\fi
\hypersetup{breaklinks=true,
            bookmarks=true,
            pdfauthor={},
            pdftitle={Complements : le theor`eme de Baire et ses consequences},
            colorlinks=true,
            citecolor=blue,
            urlcolor=blue,
            linkcolor=magenta,
            pdfborder={0 0 0}}
\urlstyle{same}  % don't use monospace font for urls
\setlength{\parindent}{0pt}
\setlength{\parskip}{6pt plus 2pt minus 1pt}
\setlength{\emergencystretch}{3em}  % prevent overfull lines
\setcounter{secnumdepth}{0}
 
/* start css.sty */
.cmr-5{font-size:50%;}
.cmr-7{font-size:70%;}
.cmmi-5{font-size:50%;font-style: italic;}
.cmmi-7{font-size:70%;font-style: italic;}
.cmmi-10{font-style: italic;}
.cmsy-5{font-size:50%;}
.cmsy-7{font-size:70%;}
.cmex-7{font-size:70%;}
.cmex-7x-x-71{font-size:49%;}
.msbm-7{font-size:70%;}
.cmtt-10{font-family: monospace;}
.cmti-10{ font-style: italic;}
.cmbx-10{ font-weight: bold;}
.cmr-17x-x-120{font-size:204%;}
.cmsl-10{font-style: oblique;}
.cmti-7x-x-71{font-size:49%; font-style: italic;}
.cmbxti-10{ font-weight: bold; font-style: italic;}
p.noindent { text-indent: 0em }
td p.noindent { text-indent: 0em; margin-top:0em; }
p.nopar { text-indent: 0em; }
p.indent{ text-indent: 1.5em }
@media print {div.crosslinks {visibility:hidden;}}
a img { border-top: 0; border-left: 0; border-right: 0; }
center { margin-top:1em; margin-bottom:1em; }
td center { margin-top:0em; margin-bottom:0em; }
.Canvas { position:relative; }
li p.indent { text-indent: 0em }
.enumerate1 {list-style-type:decimal;}
.enumerate2 {list-style-type:lower-alpha;}
.enumerate3 {list-style-type:lower-roman;}
.enumerate4 {list-style-type:upper-alpha;}
div.newtheorem { margin-bottom: 2em; margin-top: 2em;}
.obeylines-h,.obeylines-v {white-space: nowrap; }
div.obeylines-v p { margin-top:0; margin-bottom:0; }
.overline{ text-decoration:overline; }
.overline img{ border-top: 1px solid black; }
td.displaylines {text-align:center; white-space:nowrap;}
.centerline {text-align:center;}
.rightline {text-align:right;}
div.verbatim {font-family: monospace; white-space: nowrap; text-align:left; clear:both; }
.fbox {padding-left:3.0pt; padding-right:3.0pt; text-indent:0pt; border:solid black 0.4pt; }
div.fbox {display:table}
div.center div.fbox {text-align:center; clear:both; padding-left:3.0pt; padding-right:3.0pt; text-indent:0pt; border:solid black 0.4pt; }
div.minipage{width:100%;}
div.center, div.center div.center {text-align: center; margin-left:1em; margin-right:1em;}
div.center div {text-align: left;}
div.flushright, div.flushright div.flushright {text-align: right;}
div.flushright div {text-align: left;}
div.flushleft {text-align: left;}
.underline{ text-decoration:underline; }
.underline img{ border-bottom: 1px solid black; margin-bottom:1pt; }
.framebox-c, .framebox-l, .framebox-r { padding-left:3.0pt; padding-right:3.0pt; text-indent:0pt; border:solid black 0.4pt; }
.framebox-c {text-align:center;}
.framebox-l {text-align:left;}
.framebox-r {text-align:right;}
span.thank-mark{ vertical-align: super }
span.footnote-mark sup.textsuperscript, span.footnote-mark a sup.textsuperscript{ font-size:80%; }
div.tabular, div.center div.tabular {text-align: center; margin-top:0.5em; margin-bottom:0.5em; }
table.tabular td p{margin-top:0em;}
table.tabular {margin-left: auto; margin-right: auto;}
div.td00{ margin-left:0pt; margin-right:0pt; }
div.td01{ margin-left:0pt; margin-right:5pt; }
div.td10{ margin-left:5pt; margin-right:0pt; }
div.td11{ margin-left:5pt; margin-right:5pt; }
table[rules] {border-left:solid black 0.4pt; border-right:solid black 0.4pt; }
td.td00{ padding-left:0pt; padding-right:0pt; }
td.td01{ padding-left:0pt; padding-right:5pt; }
td.td10{ padding-left:5pt; padding-right:0pt; }
td.td11{ padding-left:5pt; padding-right:5pt; }
table[rules] {border-left:solid black 0.4pt; border-right:solid black 0.4pt; }
.hline hr, .cline hr{ height : 1px; margin:0px; }
.tabbing-right {text-align:right;}
span.TEX {letter-spacing: -0.125em; }
span.TEX span.E{ position:relative;top:0.5ex;left:-0.0417em;}
a span.TEX span.E {text-decoration: none; }
span.LATEX span.A{ position:relative; top:-0.5ex; left:-0.4em; font-size:85%;}
span.LATEX span.TEX{ position:relative; left: -0.4em; }
div.float img, div.float .caption {text-align:center;}
div.figure img, div.figure .caption {text-align:center;}
.marginpar {width:20%; float:right; text-align:left; margin-left:auto; margin-top:0.5em; font-size:85%; text-decoration:underline;}
.marginpar p{margin-top:0.4em; margin-bottom:0.4em;}
.equation td{text-align:center; vertical-align:middle; }
td.eq-no{ width:5%; }
table.equation { width:100%; } 
div.math-display, div.par-math-display{text-align:center;}
math .texttt { font-family: monospace; }
math .textit { font-style: italic; }
math .textsl { font-style: oblique; }
math .textsf { font-family: sans-serif; }
math .textbf { font-weight: bold; }
.partToc a, .partToc, .likepartToc a, .likepartToc {line-height: 200%; font-weight:bold; font-size:110%;}
.chapterToc a, .chapterToc, .likechapterToc a, .likechapterToc, .appendixToc a, .appendixToc {line-height: 200%; font-weight:bold;}
.index-item, .index-subitem, .index-subsubitem {display:block}
.caption td.id{font-weight: bold; white-space: nowrap; }
table.caption {text-align:center;}
h1.partHead{text-align: center}
p.bibitem { text-indent: -2em; margin-left: 2em; margin-top:0.6em; margin-bottom:0.6em; }
p.bibitem-p { text-indent: 0em; margin-left: 2em; margin-top:0.6em; margin-bottom:0.6em; }
.subsectionHead, .likesubsectionHead { margin-top:2em; font-weight: bold;}
.sectionHead, .likesectionHead { font-weight: bold;}
.quote {margin-bottom:0.25em; margin-top:0.25em; margin-left:1em; margin-right:1em; text-align:justify;}
.verse{white-space:nowrap; margin-left:2em}
div.maketitle {text-align:center;}
h2.titleHead{text-align:center;}
div.maketitle{ margin-bottom: 2em; }
div.author, div.date {text-align:center;}
div.thanks{text-align:left; margin-left:10%; font-size:85%; font-style:italic; }
div.author{white-space: nowrap;}
.quotation {margin-bottom:0.25em; margin-top:0.25em; margin-left:1em; }
h1.partHead{text-align: center}
.sectionToc, .likesectionToc {margin-left:2em;}
.subsectionToc, .likesubsectionToc {margin-left:4em;}
.sectionToc, .likesectionToc {margin-left:6em;}
.frenchb-nbsp{font-size:75%;}
.frenchb-thinspace{font-size:75%;}
.figure img.graphics {margin-left:10%;}
/* end css.sty */

\title{Complements : le theor`eme de Baire et ses consequences}
\author{}
\date{}

\begin{document}
\maketitle

\textbf{Warning: 
requires JavaScript to process the mathematics on this page.\\ If your
browser supports JavaScript, be sure it is enabled.}

\begin{center}\rule{3in}{0.4pt}\end{center}

[
[
[]
[

\section{5.4 Compléments~: le théorème de Baire et ses
conséquences}

\subsection{5.4.1 Le théorème de Baire}

Théorème~5.4.1 (Baire). Soit E un espace métrique complet et
(U_n)_n\in\mathbb{N}~ une suite d'ouverts denses dans E. Alors
\⋂ ~
_n\in\mathbb{N}~U_n est encore dense dans E.

Démonstration Rappelons qu'une partie est dense si et seulement si elle
rencontre tout ouvert non vide. Soit donc U un tel ouvert de E, soit
x_0 \in U \bigcap U_0 (qui est non vide par densité de
U_0 et ouvert comme intersection de deux ouverts). Soit
r_0 > 0 tel que B'(x_0,r_0) \subset~ U
\bigcap U_0. Supposons x_n et r_n construits et
voyons comment nous allons construire x_n+1 et r_n+1.
Comme U_n+1 est dense et B(x_n,r_n) est un
ouvert, U_n+1 \bigcap B(x_n,r_n) est ouvert et non
vide~; soit donc x_n+1 \in U_n+1 \bigcap
B(x_n,r_n) et r_n+1 <
r_n \over 2 tel que
B'(x_n+1,r_n+1) \subset~ U_n+1 \bigcap
B(x_n,r_n) . On construit ainsi une suite de boules
fermées B'(x_n,r_n) telles que
B'(x_n+1,r_n+1) \subset~ B'(x_n,r_n) avec
r_n < r_0 \over
2^n . Le théorème des fermés emboîtés nous garantit que
\⋂ ~
_n\in\mathbb{N}~B'(x_n,r_n)\neq~\varnothing~
(car \delta(B'(x_n,r_n)) < 2r_n tend
vers 0). Mais on a B'(x_0,r_0) \subset~ U \bigcap U_0 et
pour n ≥ 1, B'(x_n,r_n) \subset~ U_n. On en déduit
que U \bigcap\⋂ ~
_n\in\mathbb{N}~U_n\neq~\varnothing~, ce qui achève la
démonstration.

En passant au complémentaire, on obtient une version équivalente

Théorème~5.4.2 (Baire). Soit E un espace métrique complet et
(F_n)_n\in\mathbb{N}~ une suite de fermés d'intérieurs vides de E.
Alors \⋃ ~
_n\in\mathbb{N}~F_n est encore d'intérieur vide dans E.

Exemple~5.4.1 On montre facilement qu'un sous-espace vectoriel de E
distinct de E est d'intérieur vide (exercice). On en déduit que, si E,
espace vectoriel normé~de dimension infinie, est complet, E (qui n'est
pas d'intérieur vide) ne peut pas être réunion dénombrable de
sous-espaces vectoriels de dimension finie (dont on sait qu'ils sont
fermés). En particulier E ne peut pas admettre de base dénombrable.
C'est ainsi que \mathbb{R}~[X] (qui admet une base dénombrable) n'est complet
pour aucune norme.

\subsection{5.4.2 Les grands théorèmes}

Nous en citerons trois qui concernent tous des applications linéaires
dans des espaces vectoriels normés complets.

Théorème~5.4.3 (Banach-Steinhaus). Soit E un espace vectoriel
normé~complet et F un espace vectoriel normé. Soit H un ensemble
d'applications linéaires continues telles que

\forall~~x \in E,
\existsK_x~ ≥ 0,
\forall~~u \in H,\quad
\u(x)\ \leq K_x

Alors il existe K ≥ 0 tel que \forall~~u \in H,
\u\ \leq K.

Démonstration Posons pour x \in E, p(x) =\
sup_u\inH\u(x)\(\leq
K_x) et considérons E_n = \x \in
E∣p(x) \leq n\. Remarquons tout
d'abord que E_n est fermé~: en effet si (x_q) est une
suite d'éléments de E_n qui converge vers x \in E, on a pour tout
u dans H, \forall~~q \in \mathbb{N}~,
\u(x_q)\ \leq
n~; en faisant tendre q vers + \infty~ et en utilisant la continuité de u, on
a encore \u(x)\ \leq n et
donc x \in E_n. Maintenant notre hypothèse implique que chaque x
de E appartient à l'un des E_n (par exemple pour n =
E(K_x) + 1). Donc E qui est d'intérieur évidemment non vide est
réunion d'une famille de fermés. Le théorème de Baire implique que l'un
des E_n est d'intérieur non vide~: soit donc N \in \mathbb{N}~,
x_0 \in E et r > 0 tel que B'(x_0,r) \subset~
E_N. Prenons alors x \in B'(0,1) et u \in H. Alors x_0 +
rx \in B'(x_0,r) et donc \u(x_0
+ rx)\ \leq N. Mais alors
\u(x)\ = 1
\over r \u(x_0 + rx)
- u(x_0)\ \leq 1 \over
r (N +\
u(x_0\) = K. On a donc
\forall~~u \in H,
\u\ \leq K.

Remarque~5.4.1 Sous les mêmes hypothèses, on montre alors facilement
qu'une limite simple d'applications linéaires continues est encore
continue (attention à l'hypothèse E complet)~; en effet le théorème de
Banach Steinhaus implique que la suite est équicontinue (le module de
continuité en x_0, \eta(\epsilon,x_0), ne dépend pas de n) et on
montre simplement qu'une limite simple d'une suite équicontinue est
continue.

Théorème~5.4.4 (théorème de Banach). Soit E et F deux espaces vectoriels
normés complets, et u : E \rightarrow~ F linéaire, continue, bijective. Alors
u^-1 est encore continue.

Démonstration On va montrer que u(B'(0,1)) \subset~ F contient une boule de
centre 0 dans F, B'(0,r_1). On aura alors B'(0,r_1) \subset~
u(B'(0,1)), soit u^-1(B'(0,r_1)) \subset~ B'(0,1) et donc
si y \in F avec \y\ \leq 1,
on aura u^-1( r_1 \over 2 y) \in
B'(0,1) soit encore
\u^-1(y)\
\leq 2 \over r_1 ce qui montrera que
u^-1 est continue.

Soit r > 0. On a E =\
⋃  _n\in\mathbb{N}~~nB'(0,r), on en déduit que F =
u(E) = \⋃ ~
_n\in\mathbb{N}~nu(B'(0,r)) et a fortiori F =\
⋃ ~
_n\in\mathbb{N}~n\overlineu(B'(0,r)). L'espace vectoriel
normé complet F qui est son propre intérieur est réunion d'une famille
dénombrable de fermés~; donc l'un d'entre eux (Baire) est d'intérieur
non vide. Mais si n\overlineu(B'(0,r)) est
d'intérieur non vide, il en est de même de
\overlineu(B'(0,r)). Soit donc y_0 \in F et \rho
> 0 tel que B'(y_0,\rho)
\subset~\overlineu(B'(0,r)). On a aussi (puisque
l'application x\mapsto~ - x laisse invariante
B'(0,r)), B'(-y_0,\rho) \subset~\overlineu(B'(0,r)),
et alors, si y \in B'(0,\rho),

2y = (y - y_0) + (y + y_0) \in B'(-y_0,\rho) +
B(y_0,\rho_0)

or

B'(-y_0,\rho) + B(y_0,\rho_0)
\subset~\overlineu(B'(0,r)) +
\overlineu(B'(0,r))
\subset~\overlineu(B'(0,2r))

(facile) et donc y \in\overlineu(B'(0,r)). On a donc
trouvé, pour tout r > 0 un \rho > 0 tel que
B'(0,\rho) \subset~\overlineu(B'(0,r)). Les translations étant
des homéomorphismes, on a évidemment pour tout x \in E, B'(u(x),\rho)
\subset~\overlineu(B'(x,r)).

Montrons alors que sous ces hypothèses B'(0,\rho) \subset~ u(B'(0,2r)). Soit en
effet y \in B'(0,\rho). Soit \rho_n le réel associé à  r
\over 2^n par la propriété ci dessus. Quitte
à remplacer les \rho_n par des réels plus petits, on peut supposer
que \rho_n tend vers 0. On va construire un élément x_n
de E par récurrence de manière à vérifier
\x_n+1 -
x_n\ \leq r \over
2^n et \y -
u(x_n)\ \leq \rho_n. On pose
x_0 = 0~; supposons x_n construit. On a donc y \in
B'(u(x_n),\rho_n)
\subset~\overlineu(B'(x_n, r \over
2^n )) et donc on peut trouver un point x_n+1 \in
B'(x_n, r \over 2^n ) tel que
\y -
u(x_n+1)\ \leq \rho_n+1, soit y \in
B'(u(x_n+1),\rho_n+1), ce qui achève la construction par
récurrence. On a donc pour tout n,
\x_n+1 -
x_n\ \leq r \over
2^n et \y -
u(x_n)\ \leq \rho_n. On a
\x_n+p -
x_n\ \leq r \over
2^n + r \over 2^n+1 +
\\ldots~ + r
\over 2^n+p-1 \leq r \over
2^n-1 , ce qui montre que la suite (x_n) est une
suite de Cauchy. Comme E est complet, elle converge. Soit x sa limite.
On a \x -
x_0\ \leq 2r d'après l'inégalité ci
dessus pour n = 0 et p tendant vers + \infty~. D'autre part l'inégalité
\y - u(x_n)\
\leq \rho_n et la continuité de u nous montrent que y = u(x), donc y
appartient à u(B'(0,2r)).

On a alors aussi B'(0, \rho \over 2r ) \subset~ u(B'(0,1)), ce
qui montre comme on l'a remarqué, que u^-1 est continue.

Théorème~5.4.5 (théorème du graphe fermé). Soit E et F deux espaces
vectoriels normés complets, et u : E \rightarrow~ F linéaire. Alors u est continue
si et seulement si~son graphe est fermé dans E \times F.

Démonstration Supposons tout d'abord que u est continue et soit
(x_n,u(x_n)) une suite du graphe qui converge vers
(x,y) \in E \times F. Alors limx_n~ = x et
par continuité de u, limu(x_n~) =
u(x)~; mais alors l'unicité de la limite nécessite y = u(x), donc (x,y)
est encore dans le graphe de u, ce qui montre bien que le graphe est
fermé (il s'agit là d'une propriété tout à fait générale des espaces
métriques, mais la réciproque est fausse en général). Supposons
maintenant que u est linéaire de graphe \Gamma fermé. Alors \Gamma est un
sous-espace vectoriel fermé de E \times F, donc il est complet. L'application
\Gamma \rightarrow~ E, (x,u(x))\mapsto~x est linéaire continue et
bijective. D'après le théorème de Banach, sa réciproque
x\mapsto~(x,u(x)) est continue et donc
x\mapsto~u(x) aussi.

Remarque~5.4.2 Il s'agit d'une technique importante~; il est en effet
considérablement plus facile de montrer qu'un graphe est fermé plutôt
qu'une continuité~; si (x_n) est une suite de limite x, il
s'agit de montrer non plus que la suite u(x_n) converge vers
u(x) mais plutôt que la suite u(x_n) ne peut pas avoir d'autre
limite que u(x)~; un exemple typique d'application linéaire de graphe
fermé est la dérivation pour la topologie de la convergence uniforme~:
le théorème de dérivation des suites uniformément convergentes ne fait
que traduire la fermeture du graphe (si la suite des dérivées converge
uniformément, alors c'est vers la dérivée de la limite)~; attention
cependant que la dérivation n'est pas continue pour la topologie de la
convergence uniforme (le théorème du graphe fermé ne s'applique pas car
l'espace des applications \mathcal{C}^1 n'est pas complet).

[
[
[
[

\end{document}
