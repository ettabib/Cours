
\section{8.3 Fonctions réelles d'une variable réelle}

\section{Théorème de Rolle, formule des accroissements finis}
\label{sec:theoreme-de-rolle}



Lemme~8.3.1 Soit f : I \rightarrow~ \mathbb{R}~~; si f admet en c \in I^o un
extremum local et si f est dérivable au point c, alors f'(c) = 0.

Démonstration Supposons par exemple que f a en c un maximum local. Pour
c - \eta < x < c, on a  f(x)-f(c)
\over x-c ≥ 0 d'où en faisant tendre x vers c, f'(c) ≥
0. Pour c < x < c + \eta, on a  f(x)-f(c)
\over x-c \leq 0 d'où en faisant tendre x vers c, f'(c) \leq
0. On a donc f'(c) = 0.

\begin{thm}[Rolle]
   Soit f : [a,b] \rightarrow~ \mathbb{R}~, continue sur [a,b],
dérivable sur ]a,b[ telle que f(a) = f(b). Alors il existe c
\in]a,b[ tel que f'(c) = 0.

\end{thm}
Démonstration Si f est constante sur [a,b], n'importe quel c
\in]a,b[ convient. Sinon, par exemple, il existe x \in [a,b] tel que
f(x) > f(a) = f(b). La fonction f est continue sur le
compact [a,b] donc elle est bornée et atteint ses bornes. Soit c \in
[a,b] tel que f(c) =\
sup\f(t)∣t \in
[a,b]\. On a f(c) ≥ f(x) > f(a) =
f(b), donc c \in]a,b[. Mais alors, le lemme ci dessus garantit que
f'(c) = 0.

Corollaire~8.3.3 (formule des accroissements finis). Soit f : [a,b]
\rightarrow~ \mathbb{R}~, continue sur [a,b], dérivable sur ]a,b[. Alors il existe c
\in]a,b[ tel que f(b) - f(a) = (b - a)f'(c).

Démonstration On applique le théorème de Rolle à g(t) = f(t) -
f(b)-f(a) \over b-a (t - a). On a g(b) = g(a) = f(a), g
est, comme f, continue sur [a,b] et dérivable sur ]a,b[. Donc il
existe c \in]a,b[ tel que g'(c) = 0~; mais g'(c) = f'(c) - f(b)-f(a)
\over b-a d'où le résultat.

\subsection{8.3.2 Monotonie et dérivation}

Théorème~8.3.4 Soit I un intervalle de \mathbb{R}~, f : I \rightarrow~ \mathbb{R}~ continue sur I et
dérivable sur I^o. Alors (i) f est constante sur I si et
seulement si~\forall~t \in I^o~, f'(t) = 0
(ii) f est croissante sur I si et seulement
si~\forall~t \in I^o~, f'(t) ≥ 0 (iii) f est
décroissante sur I si et seulement si~\forall~~t \in
I^o, f'(t) \leq 0

Démonstration La définition de la dérivée f'(t)
=\
lim_x\rightarrow~t,x\neq~t f(x)-f(t)
\over x-t montre clairement que les conditions sont
nécessaires (prendre x > t et faire tendre x vers t).
Inversement, si x,y \in I avec x < y, f est continue sur
[x,y] \subset~ I et dérivable sur ]x,y[\subset~ I^o et donc la
formule des accroissements finis assure qu'il existe z \in]x,y[\subset~
I^o tel que f(y) - f(x) = (y - x)f'(z), ce qui montre
immédiatement que les conditions sont suffisantes.

Corollaire~8.3.5 Soit I un intervalle de \mathbb{R}~, f : I \rightarrow~ \mathbb{R}~ continue sur I et
dérivable sur I^o. Alors on a équivalence de (i) f est
strictement croissante (ii) \forall~~t \in
I^o, f'(t) ≥ 0 et \t \in
I^o∣f'(t) = 0\
est d'intérieur vide.

Démonstration (i) \rigtharrow~(ii) Si f est strictement croissante, alors
\forall~t \in I^o~, f'(t) ≥ 0~; supposons que
\t \in I^o∣f'(t) =
0\ n'est pas d'intérieur vide~; alors il contient un
segment [a,b] avec a < b~; mais alors d'après le théorème
précédent, f est constante sur [a,b] ce qui contredit la stricte
monotonie de f.

(ii) \rigtharrow~(i) On sait que si \forall~t \in I^o~,
f'(t) ≥ 0, f est croissante~; supposons qu'elle n'est pas strictement
croissante~; alors il existe a,b \in I tels que a < b et f(a) =
f(b)~; en conséquence f est constante sur ]a,b[\subset~ I^o et
donc \forall~~t \in]a,b[, f'(t) = 0~; donc
l'intervalle ouvert ]a,b[ est contenu dans l'intérieur de
\t \in I^o∣f'(t) =
0\, c'est absurde.

\subsection{8.3.3 Difféomorphismes}

Théorème~8.3.6 Soit I et J deux intervalles de \mathbb{R}~ et f : I \rightarrow~ J un
homéomorphisme. Soit a \in I un point où f est dérivable. Alors
f^-1 est dérivable au point f(a) si et seulement
si~f'(a)\neq~0. Dans ce cas,
(f^-1)'(f(a)) = 1 \over f'(a) .

Démonstration Posons g = f^-1. On a g \cdot f =
\mathrmId_I. Si f est dérivable au point a et
g dérivable au point f(a), le théorème de dérivation des fonctions
composées assure que 1 = (\mathrmId_I)'(a) =
(g \cdot f)'(a) = g'(f(a))f'(a), donc f'(a)\neq~0 et
g'(f(a)) = 1 \over f'(a) . Inversement supposons que
f'(a)\neq~0. On a alors
lim_t\rightarrow~a,t\neq~a~
t-a \over f(t)-f(a) = 1 \over f'(a)
. Appliquons le théorème de composition des limites en posant t = g(u)
(avec a = g(f(a))), en remarquant que u\neq~f(a)
\rigtharrow~ g(u)\neq~a. On a donc, puisque g est continue
au point f(a),

lim_u\rightarrow~f(a),u\neq~f(a)~
g(u) - g(f(a)) \over u - f(a) = 1
\over f'(a)

Donc g est dérivable au point f(a).

Définition~8.3.1 Soit I et J deux intervalles de \mathbb{R}~~; on dit que f : I \rightarrow~
J est un difféomorphisme de classe C^n (n ≥ 1) si f est
bijective et f et f^-1 sont de classe C^n.

Théorème~8.3.7 Soit n ≥ 1, f : I \rightarrow~ \mathbb{R}~. On a équivalence de (i) f est un
C^n difféomorphisme de I sur f(I) (ii) f est de classe
C^n et f' ne s'annule pas.

Démonstration (i) \rigtharrow~(ii) est clair d'après le théorème précédent.
Inversement, supposons que f est de classe C^n et que f' ne
s'annule pas. Alors f' garde un signe constant (elle est continue), et
donc f est strictement monotone. Donc f définit un homéomorphisme de I
sur J = f(I). Le théorème précédent assure que f^-1 est
dérivable sur I et que (f^-1)' = 1 \over
f'\cdotf^-1 ce qui garantit déjà la continuité de
(f^-1)'. Supposons alors que f^-1 est de classe
C^k avec k < n. Comme f' est de classe
C^k, f' \cdot f^-1 est de classe C^k~; il
en est donc de même de  1 \over f'\cdotf^-1 ,
donc de (f^-1)' et donc f^-1 est de classe
C^k+1~; par récurrence, on en déduit que f^-1 est
de classe C^n.

\subsection{8.3.4 Formule de Taylor Lagrange}

Théorème~8.3.8 (Taylor-Lagrange). Soit f : [a,b] \rightarrow~ \mathbb{R}~ de classe
C^n sur [a,b] et n + 1 fois dérivable sur ]a,b[.
Alors il existe c \in]a,b[ tel que

f(b) = f(a) + \sum _k=1^n~
f^(k)(a) \over k! (b - a)^k +
f^(n+1)(c) \over (n + 1)! (b -
a)^n+1

Démonstration Posons \phi(t) = f(b) - f(t)
-\\sum ~
_k=1^n f^(k)(t) \over k!
(b - t)^k - \lambda~(b - t)^n+1 où \lambda~ est choisi de telle
sorte que \phi(a) = 0 (c'est évidemment possible). Il est clair que \phi est
continue sur [a,b], dérivable sur ]a,b[ comme toutes les
fonctions f^(k), 0 \leq k \leq n. De plus

\begin{align*} \phi'(t)& =& -f'(t)
-\sum _k=1^n~
f^(k+1)(t) \over k! (b - t)^k
\%& \\ & \text &
+\sum _k=1^n~
f^(k)(t) \over (k - 1)! (b -
t)^k-1 + \lambda~(n + 1)(b - t)^n\%&
\\ & =& -f'(t)
-\sum _l=2^n+1~
f^(l)(t) \over (l - 1)! (b -
t)^l-1 \%& \\ &
\text & +\\sum
_k=1^n f^(k)(t) \over (k -
1)! (b - t)^k-1 + \lambda~(n + 1)(b - t)^n\%&
\\ & =& (b -
t)^n\left ((n + 1)\lambda~ - f^(n+1)(t)
\over n! \right ) \%&
\\ \end{align*}

(tous les autres termes se détruisent deux à deux). D'après le théorème
de Rolle, il existe c \in]a,b[ tel que \phi'(c) = 0, soit (b -
c)^n\left ((n + 1)\lambda~ - f^(n+1)(c)
\over n! \right ) = 0. Comme b -
c\neq~0, on a \lambda~ = f^(n+1)(c)
\over (n+1)! . En écrivant que \phi(a) = 0, on obtient
alors la formule souhaitée.

Remarque~8.3.1 Pour n = 0, on trouve comme cas particulier la formule
des accroissements finis. La même formule est encore valable si on prend
f : [b,a] \rightarrow~ \mathbb{R}~.

\subsection{8.3.5 Extensions du théorème des accroissements finis}

Théorème~8.3.9 Soit f,g : [a,b] \rightarrow~ \mathbb{R}~ continues sur [a,b],
dérivables sur ]a,b[. Alors, il existe c \in]a,b[ tel que
\left
\matrix\,f(b) - f(a)&f'(c)
\cr g(b) - g(a)&g'(c)\right 
= 0.

Démonstration Posons

\phi(t) = \left
\matrix\,f(b) - f(a)&f(t) -
f(a) \cr g(b) - g(a)&g(t) - g(a)\right


La fonction \phi est continue sur [a,b], dérivable sur ]a,b[ avec
\phi'(t) = \left
\matrix\,f(b) - f(a)&f'(t)
\cr g(b) - g(a)&g'(t)\right .
Comme \phi(a) = \phi(b) = 0, le théorème de Rolle garantit l'existence d'un c
\in]a,b[ tel que \phi'(c) = 0.

Corollaire~8.3.10 (règle de L'Hôpital). Soit f,g : I \rightarrow~ \mathbb{R}~ continues sur
I, dérivables sur I \diagdown\a\. On suppose
qu'il existe \eta > 0 tel que g' ne s'annule pas sur ]a -
\eta,a + \eta[\diagdown\a\ et que  f'
\over g' a une limite \ell au point a. Alors  f(t)-f(a)
\over g(t)-g(a) admet la même limite au point a.

Démonstration Le théorème de Rolle garantit déjà que g(t) - g(a) ne
s'annule pas sur ]a - \eta,a +
\eta[\diagdown\a\. De plus le théorème
précédent montre que pour t \in]a - \eta,a +
\eta[\diagdown\a\, il existe c_t
\in]a,t[ (ou ]t,a[) tel que \left
\matrix\,f(t) -
f(a)&f'(c_t) \cr g(t) -
g(a)&g'(c_t)\right  = 0 soit 
f(t)-f(a) \over g(t)-g(a) = f'(c_t)
\over g'(c_t) . Quand t tend vers a, il en est
de même de c_t et le théorème de composition des limites donne

lim_t\rightarrow~a,t\neq~a~
f(t) - f(a) \over g(t) - g(a) = \ell

\subsection{8.3.6 Fonctions convexes de classe \mathcal{C}^1}

Définition~8.3.2 Soit I un intervalle de \mathbb{R}~ et f : I \rightarrow~ \mathbb{R}~ une fonction de
classe \mathcal{C}^1. On dit que f est convexe si f' est croissante.

Remarque~8.3.2 Si f est de classe C^2, f est convexe si et
seulement si~f'' est positive.

Théorème~8.3.11 Soit I un intervalle de \mathbb{R}~ et f : I \rightarrow~ \mathbb{R}~ une fonction de
classe \mathcal{C}^1 convexe. Alors (i) \forall~~a,b
\in I, \forall~~t \in [0,1], f(ta + (1 - t)b) \leq tf(a)
+ (1 - t)f(b) (ii) \Gamma = \(x,y) \in
\mathbb{R}~^2∣x \in I\text et
y ≥ f(x)\ est une partie convexe de \mathbb{R}~^2
(iii) \forall~~a,b \in I, f(b) ≥ f(a) + (b - a)f'(a)
(iv) si a \in I, l'application I \diagdown\a\
dans \mathbb{R}~, t\mapsto~p_a(t) = f(t)-f(a)
\over t-a est croissante (v)
\forall~~a,b,c \in I, a < b < c \rigtharrow~
f(b)-f(a) \over b-a \leq f(c)-f(a) \over
c-a \leq f(c)-f(b) \over c-b

Démonstration (i) On peut évidemment supposer a < b. D'après
le théorème des accroissements finis, il existe c \in]a,b[ tel que
f(b) - f(a) = (b - a)f'(c). Posons c = t_0a + (1 -
t_0)b. Soit \phi(t) = tf(a) + (1 - t)f(b) - f(ta + (1 - t)b) pour
t \in [0,1]. Alors \phi est de classe \mathcal{C}^1 et \phi'(t) = f(a) -
f(b) - (a - b)f'(ta + (1 - t)b) = (b - a)(f'(ta + (1 - t)b) -
f'(t_0a + (1 - t_0)b). Comme f' est croissante et
t\mapsto~ta + (1 - t)b est décroissante, la composée
est décroissante et donc on a le tableau de variation

\begin{center}\rule{3in}{0.4pt}\end{center}

\begin{center}\rule{3in}{0.4pt}\end{center}

\begin{center}\rule{3in}{0.4pt}\end{center}

\begin{center}\rule{3in}{0.4pt}\end{center}

\begin{center}\rule{3in}{0.4pt}\end{center}

\begin{center}\rule{3in}{0.4pt}\end{center}

t

0

t_0

1

\begin{center}\rule{3in}{0.4pt}\end{center}

\begin{center}\rule{3in}{0.4pt}\end{center}

\begin{center}\rule{3in}{0.4pt}\end{center}

\begin{center}\rule{3in}{0.4pt}\end{center}

\begin{center}\rule{3in}{0.4pt}\end{center}

\begin{center}\rule{3in}{0.4pt}\end{center}

\phi'(t)

+

0

-

\begin{center}\rule{3in}{0.4pt}\end{center}

\begin{center}\rule{3in}{0.4pt}\end{center}

\begin{center}\rule{3in}{0.4pt}\end{center}

\begin{center}\rule{3in}{0.4pt}\end{center}

\begin{center}\rule{3in}{0.4pt}\end{center}

\begin{center}\rule{3in}{0.4pt}\end{center}

\phi(t)

0

\nearrow

\searrow

0

\begin{center}\rule{3in}{0.4pt}\end{center}

\begin{center}\rule{3in}{0.4pt}\end{center}

\begin{center}\rule{3in}{0.4pt}\end{center}

\begin{center}\rule{3in}{0.4pt}\end{center}

\begin{center}\rule{3in}{0.4pt}\end{center}

\begin{center}\rule{3in}{0.4pt}\end{center}

ce qui montre que la fonction \phi est positive sur [0,1].

(ii) Soit (x_1,y_1) et (x_2,y_2)
dans \Gamma et t \in [0,1]. On a

ty_1 + (1 - t)y_2 ≥ tf(x_1) + (1 -
t)f(x_2) ≥ f(tx_1 + (1 - t)x_2)

donc t(x_1,y_1) + (1 - t)(x_2,y_2) \in
\Gamma. Donc \Gamma est convexe.

(iii) Posons \phi(t) = f(t) - f(a) - (t - a)f'(a). La fonction \phi est de
classe \mathcal{C}^1 et \phi'(t) = f'(t) - f'(a). Comme f' est croissante,
on a le tableau de variation

\begin{center}\rule{3in}{0.4pt}\end{center}

\begin{center}\rule{3in}{0.4pt}\end{center}

\begin{center}\rule{3in}{0.4pt}\end{center}

\begin{center}\rule{3in}{0.4pt}\end{center}

\begin{center}\rule{3in}{0.4pt}\end{center}

\begin{center}\rule{3in}{0.4pt}\end{center}

t

a

\begin{center}\rule{3in}{0.4pt}\end{center}

\begin{center}\rule{3in}{0.4pt}\end{center}

\begin{center}\rule{3in}{0.4pt}\end{center}

\begin{center}\rule{3in}{0.4pt}\end{center}

\begin{center}\rule{3in}{0.4pt}\end{center}

\begin{center}\rule{3in}{0.4pt}\end{center}

\phi'(t)

+

0

-

\begin{center}\rule{3in}{0.4pt}\end{center}

\begin{center}\rule{3in}{0.4pt}\end{center}

\begin{center}\rule{3in}{0.4pt}\end{center}

\begin{center}\rule{3in}{0.4pt}\end{center}

\begin{center}\rule{3in}{0.4pt}\end{center}

\begin{center}\rule{3in}{0.4pt}\end{center}

\phi(t)

\searrow

0

\nearrow

\begin{center}\rule{3in}{0.4pt}\end{center}

\begin{center}\rule{3in}{0.4pt}\end{center}

\begin{center}\rule{3in}{0.4pt}\end{center}

\begin{center}\rule{3in}{0.4pt}\end{center}

\begin{center}\rule{3in}{0.4pt}\end{center}

\begin{center}\rule{3in}{0.4pt}\end{center}

ce qui montre que la fonction \phi est positive sur I.

(iv) Posons p_a(t) = f(t)-f(a) \over t-a si
t\neq~a et p_a(a) = f'(a). La fonction
p_a est continue sur I, dérivable sur I
\diagdown\a\ et p_a'(t) =
f(a)-f(t)-(a-t)f'(t) \over (t-a)^2 ≥ 0
d'après (iii). On en déduit que p_a est croissante.

(v) D'après (iv), on a p_a(b) \leq p_a(c) =
p_c(a) \leq p_c(b) ce qui est le résultat souhaité.

Théorème~8.3.12 Soit f : I \rightarrow~ \mathbb{R}~ de classe \mathcal{C}^1 convexe. Alors,
pour tout
(x_1,\\ldots,x_n~)
\in I^n, pour toute famille
(\alpha_1,\\ldots,\alpha_n~)
\in (\mathbb{R}~^+)^n telle que \alpha_1 +
\\ldots~ +
\alpha_n = 1, on a

f(\sum _i=1^n\alpha~_
ix_i) \leq\\sum
_i=1^n\alpha_ if(x_i)

Démonstration Par récurrence sur n. Si n = 2, on a \alpha_2 = 1 -
\alpha_1 et \alpha_1 \in [0,1]. L'inégalité se réduit à
l'assertion (i) du théorème précédent. Supposons le résultat vrai pour n
- 1 et montrons le pour n. Si \alpha_n = 0, on est immédiatement
ramené au cas n - 1. On peut donc supposer
\alpha_n\neq~0. Si \alpha_n = 1, alors
tous les autres \alpha_i sont nuls et l'inégalité est triviale. On
peut donc supposer \alpha_n \in]0,1[. On écrit alors
\\sum ~
_i=1^n\alpha_ix_i = \alpha_nx_n
+ (1 - \alpha_n)y avec y =
\alpha_1x_1+\\ldots+\alpha_n-1x_n-1~
\over
\alpha_1+\\ldots+\alpha_n-1~
= \beta_1x_1 +
\\ldots\beta_n-1x_n-1~
\in I. On a alors \beta_i ≥ 0 et
\\sum ~
_i=1^n-1\beta_i = 1. On peut donc écrire (par
l'hypothèse de récurrence) f(y)
\leq\\sum ~
_i=1^n-1\beta_if(x_i) soit

\begin{align*} f(\\sum
_i=1^n\alpha_ ix_i)& =&
f(\alpha_nx_n + (1 - \alpha_n)y) \leq
\alpha_nf(x_n) + (1 - \alpha_n)f(y) \%&
\\ & \leq& \alpha_nf(x_n) + (1
- \alpha_n)\\sum
_i=1^n-1\beta_ if(x_i) =
\sum _i=1^n\alpha~_
if(x_i)\%& \\
\end{align*}

puisque (1 - \alpha_n)\beta_i = \alpha_i.

Corollaire~8.3.13 (inégalité de Hölder). Soit p,q \in \mathbb{R}~^+∗ tels
que  1 \over p + 1 \over q = 1.
Pour toute famille
a_1,\\ldots,a_n,b_1,\\\ldots,b_n~
de réels positifs, on a

\sum _i=1^na_
ib_i \leq\left (\\sum
_i=1^na_ i^p\right
)^1\diagupp\left (\\sum
_i=1^nb_ i^q\right
)^1\diagupq

Démonstration Posons A = \left
(\\sum ~
_i=1^na_i^p\right
)^1\diagupp, B = \left
(\\sum ~
_i=1^nb_i^q\right
)^1\diagupq. La fonction exponentielle étant convexe sur \mathbb{R}~, on a
\forall~s,t \in \mathbb{R}~, e~^ s \over
p + t \over q  \leq 1 \over p
e^s + 1 \over q e^t. Si
a_i et b_i sont non nuls, en appliquant ceci à s =
plog  a_i \over A~
et t = qlog  b_i~
\over B , on obtient  a_i
\over A  b_i \over B \leq 1
\over p  a_i^p \over
A^p + 1 \over q  b_i^q
\over B^q , inégalité qui reste vrai si
a_ib_i = 0~; en sommant de i = 1 jusque n on obtient

 1 \over AB \\sum
_i=1^na_ ib_i \leq 1
\over pA^p  \\sum
_i=1^na_ i^p + 1 \over
qB^q  \\sum
_i=1^nb_ i^q = 1 \over
p + 1 \over q = 1

soit \\sum ~
_i=1^na_ib_i \leq AB ce qu'on voulait
démontrer.

Corollaire~8.3.14 (inégalité de Minkowski). Soit p ≥ 1. Pour toute
famille
a_1,\\ldots,a_n,b_1,\\\ldots,b_n~
de réels positifs, on a

 \left (\\sum
_i=1^n(a_ i +
b_i)^p\right )^1\diagupp
\leq\left (\\sum
_i=1^na_ i^p\right
)^1\diagupp + \left (\\sum
_i=1^nb_ i^p\right
)^1\diagupp

Démonstration C'est évident si p = 1~; si p > 1,
définissons q par la condition  1 \over p + 1
\over q = 1~; on écrit (a_i +
b_i)^p = a_i(a_i +
b_i)^p-1 + b_i(a_i +
b_i)^p-1 et on applique deux fois l'inégalité de
Hölder. On obtient alors

\begin{align*} \\sum
_i=1^n(a_ i + b_i)^p&
\leq& \left (\\sum
_i=1^na_ i^p\right
)^1\diagupp\left (\\sum
_i=1^n(a_ i +
b_i)^(p-1)q\right )^1\diagupq \%&
\\ & \text &
+\left (\\sum
_i=1^nb_ i^p\right
)^1\diagupp\left (\\sum
_i=1^n(a_ i +
b_i)^(p-1)q\right )^1\diagupq\%&
\\ \end{align*}

Mais (p - 1)q = p et l'inégalité ci dessus s'écrit donc après mise en
facteur

\begin{align*} \left
(\sum _i=1^n(a_ i~ +
b_i)^p\right )^1\diagupp&& \%&
\\ & \leq& \left
(\left (\\sum
_i=1^na_ i^p\right
)^1\diagupp + \left (\\sum
_i=1^nb_ i^p\right
)^1\diagupp\right )\left
(\sum _i=1^n(a_ i~ +
b_i)^p\right )^1\diagupq\%&
\\ \end{align*}

Si \\sum ~
_i=1^n(a_i + b_i)^p = 0,
l'inégalité cherchée est évidente~; sinon, on peut diviser les deux
membres par \left
(\\sum ~
_i=1^n(a_i +
b_i)^p\right )^1\diagupq et on
obtient (en tenant compte de 1 - 1 \over p = 1
\over q )

 \left (\\sum
_i=1^n(a_ i +
b_i)^p\right )^1\diagupp
\leq\left (\\sum
_i=1^na_ i^p\right
)^1\diagupp + \left (\\sum
_i=1^nb_ i^p\right
)^1\diagupp

[
[
[
[

\end{document}
