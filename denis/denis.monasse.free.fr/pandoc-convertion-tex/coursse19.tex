\documentclass[]{article}
\usepackage[T1]{fontenc}
\usepackage{lmodern}
\usepackage{amssymb,amsmath}
\usepackage{ifxetex,ifluatex}
\usepackage{fixltx2e} % provides \textsubscript
% use upquote if available, for straight quotes in verbatim environments
\IfFileExists{upquote.sty}{\usepackage{upquote}}{}
\ifnum 0\ifxetex 1\fi\ifluatex 1\fi=0 % if pdftex
  \usepackage[utf8]{inputenc}
\else % if luatex or xelatex
  \ifxetex
    \usepackage{mathspec}
    \usepackage{xltxtra,xunicode}
  \else
    \usepackage{fontspec}
  \fi
  \defaultfontfeatures{Mapping=tex-text,Scale=MatchLowercase}
  \newcommand{\euro}{€}
\fi
% use microtype if available
\IfFileExists{microtype.sty}{\usepackage{microtype}}{}
\ifxetex
  \usepackage[setpagesize=false, % page size defined by xetex
              unicode=false, % unicode breaks when used with xetex
              xetex]{hyperref}
\else
  \usepackage[unicode=true]{hyperref}
\fi
\hypersetup{breaklinks=true,
            bookmarks=true,
            pdfauthor={},
            pdftitle={Espaces metriques},
            colorlinks=true,
            citecolor=blue,
            urlcolor=blue,
            linkcolor=magenta,
            pdfborder={0 0 0}}
\urlstyle{same}  % don't use monospace font for urls
\setlength{\parindent}{0pt}
\setlength{\parskip}{6pt plus 2pt minus 1pt}
\setlength{\emergencystretch}{3em}  % prevent overfull lines
\setcounter{secnumdepth}{0}
 
/* start css.sty */
.cmr-5{font-size:50%;}
.cmr-7{font-size:70%;}
.cmmi-5{font-size:50%;font-style: italic;}
.cmmi-7{font-size:70%;font-style: italic;}
.cmmi-10{font-style: italic;}
.cmsy-5{font-size:50%;}
.cmsy-7{font-size:70%;}
.cmex-7{font-size:70%;}
.cmex-7x-x-71{font-size:49%;}
.msbm-7{font-size:70%;}
.cmtt-10{font-family: monospace;}
.cmti-10{ font-style: italic;}
.cmbx-10{ font-weight: bold;}
.cmr-17x-x-120{font-size:204%;}
.cmsl-10{font-style: oblique;}
.cmti-7x-x-71{font-size:49%; font-style: italic;}
.cmbxti-10{ font-weight: bold; font-style: italic;}
p.noindent { text-indent: 0em }
td p.noindent { text-indent: 0em; margin-top:0em; }
p.nopar { text-indent: 0em; }
p.indent{ text-indent: 1.5em }
@media print {div.crosslinks {visibility:hidden;}}
a img { border-top: 0; border-left: 0; border-right: 0; }
center { margin-top:1em; margin-bottom:1em; }
td center { margin-top:0em; margin-bottom:0em; }
.Canvas { position:relative; }
li p.indent { text-indent: 0em }
.enumerate1 {list-style-type:decimal;}
.enumerate2 {list-style-type:lower-alpha;}
.enumerate3 {list-style-type:lower-roman;}
.enumerate4 {list-style-type:upper-alpha;}
div.newtheorem { margin-bottom: 2em; margin-top: 2em;}
.obeylines-h,.obeylines-v {white-space: nowrap; }
div.obeylines-v p { margin-top:0; margin-bottom:0; }
.overline{ text-decoration:overline; }
.overline img{ border-top: 1px solid black; }
td.displaylines {text-align:center; white-space:nowrap;}
.centerline {text-align:center;}
.rightline {text-align:right;}
div.verbatim {font-family: monospace; white-space: nowrap; text-align:left; clear:both; }
.fbox {padding-left:3.0pt; padding-right:3.0pt; text-indent:0pt; border:solid black 0.4pt; }
div.fbox {display:table}
div.center div.fbox {text-align:center; clear:both; padding-left:3.0pt; padding-right:3.0pt; text-indent:0pt; border:solid black 0.4pt; }
div.minipage{width:100%;}
div.center, div.center div.center {text-align: center; margin-left:1em; margin-right:1em;}
div.center div {text-align: left;}
div.flushright, div.flushright div.flushright {text-align: right;}
div.flushright div {text-align: left;}
div.flushleft {text-align: left;}
.underline{ text-decoration:underline; }
.underline img{ border-bottom: 1px solid black; margin-bottom:1pt; }
.framebox-c, .framebox-l, .framebox-r { padding-left:3.0pt; padding-right:3.0pt; text-indent:0pt; border:solid black 0.4pt; }
.framebox-c {text-align:center;}
.framebox-l {text-align:left;}
.framebox-r {text-align:right;}
span.thank-mark{ vertical-align: super }
span.footnote-mark sup.textsuperscript, span.footnote-mark a sup.textsuperscript{ font-size:80%; }
div.tabular, div.center div.tabular {text-align: center; margin-top:0.5em; margin-bottom:0.5em; }
table.tabular td p{margin-top:0em;}
table.tabular {margin-left: auto; margin-right: auto;}
div.td00{ margin-left:0pt; margin-right:0pt; }
div.td01{ margin-left:0pt; margin-right:5pt; }
div.td10{ margin-left:5pt; margin-right:0pt; }
div.td11{ margin-left:5pt; margin-right:5pt; }
table[rules] {border-left:solid black 0.4pt; border-right:solid black 0.4pt; }
td.td00{ padding-left:0pt; padding-right:0pt; }
td.td01{ padding-left:0pt; padding-right:5pt; }
td.td10{ padding-left:5pt; padding-right:0pt; }
td.td11{ padding-left:5pt; padding-right:5pt; }
table[rules] {border-left:solid black 0.4pt; border-right:solid black 0.4pt; }
.hline hr, .cline hr{ height : 1px; margin:0px; }
.tabbing-right {text-align:right;}
span.TEX {letter-spacing: -0.125em; }
span.TEX span.E{ position:relative;top:0.5ex;left:-0.0417em;}
a span.TEX span.E {text-decoration: none; }
span.LATEX span.A{ position:relative; top:-0.5ex; left:-0.4em; font-size:85%;}
span.LATEX span.TEX{ position:relative; left: -0.4em; }
div.float img, div.float .caption {text-align:center;}
div.figure img, div.figure .caption {text-align:center;}
.marginpar {width:20%; float:right; text-align:left; margin-left:auto; margin-top:0.5em; font-size:85%; text-decoration:underline;}
.marginpar p{margin-top:0.4em; margin-bottom:0.4em;}
.equation td{text-align:center; vertical-align:middle; }
td.eq-no{ width:5%; }
table.equation { width:100%; } 
div.math-display, div.par-math-display{text-align:center;}
math .texttt { font-family: monospace; }
math .textit { font-style: italic; }
math .textsl { font-style: oblique; }
math .textsf { font-family: sans-serif; }
math .textbf { font-weight: bold; }
.partToc a, .partToc, .likepartToc a, .likepartToc {line-height: 200%; font-weight:bold; font-size:110%;}
.chapterToc a, .chapterToc, .likechapterToc a, .likechapterToc, .appendixToc a, .appendixToc {line-height: 200%; font-weight:bold;}
.index-item, .index-subitem, .index-subsubitem {display:block}
.caption td.id{font-weight: bold; white-space: nowrap; }
table.caption {text-align:center;}
h1.partHead{text-align: center}
p.bibitem { text-indent: -2em; margin-left: 2em; margin-top:0.6em; margin-bottom:0.6em; }
p.bibitem-p { text-indent: 0em; margin-left: 2em; margin-top:0.6em; margin-bottom:0.6em; }
.subsectionHead, .likesubsectionHead { margin-top:2em; font-weight: bold;}
.sectionHead, .likesectionHead { font-weight: bold;}
.quote {margin-bottom:0.25em; margin-top:0.25em; margin-left:1em; margin-right:1em; text-align:justify;}
.verse{white-space:nowrap; margin-left:2em}
div.maketitle {text-align:center;}
h2.titleHead{text-align:center;}
div.maketitle{ margin-bottom: 2em; }
div.author, div.date {text-align:center;}
div.thanks{text-align:left; margin-left:10%; font-size:85%; font-style:italic; }
div.author{white-space: nowrap;}
.quotation {margin-bottom:0.25em; margin-top:0.25em; margin-left:1em; }
h1.partHead{text-align: center}
.sectionToc, .likesectionToc {margin-left:2em;}
.subsectionToc, .likesubsectionToc {margin-left:4em;}
.sectionToc, .likesectionToc {margin-left:6em;}
.frenchb-nbsp{font-size:75%;}
.frenchb-thinspace{font-size:75%;}
.figure img.graphics {margin-left:10%;}
/* end css.sty */

\title{Espaces metriques}
\author{}
\date{}

\begin{document}
\maketitle

\textbf{Warning: 
requires JavaScript to process the mathematics on this page.\\ If your
browser supports JavaScript, be sure it is enabled.}

\begin{center}\rule{3in}{0.4pt}\end{center}

[
[
[]
[

\section{4.2 Espaces métriques}

\subsection{4.2.1 Distances}

Définition~4.2.1 Soit E un ensemble. On appelle distance sur E toute
application d : E \times E \rightarrow~ \mathbb{R}~^+ vérifiant pour tout x,y,z \in E

\begin{itemize}
\itemsep1pt\parskip0pt\parsep0pt
\item
  (i) d(x,y) = 0 \Leftrightarrow x = y (propriété de
  séparation)
\item
  (ii) d(x,y) = d(y,x) (propriété de symétrie)
\item
  (iii) d(x,z) \leq d(x,y) + d(y,z) (inégalité triangulaire)
\end{itemize}

On appelle espace métrique un couple (E,d) d'un ensemble E et d'une
distance d sur E.

Proposition~4.2.1 Soit d une distance sur E Alors

\forall~~x,y,z \in E, d(x,z) -
d(y,z)\leq d(x,y)

Démonstration On a d(x,z) - d(y,z) \leq d(x,y) d'après l'inégalité
triangulaire. En échangeant x et y, on a aussi d(y,z) - d(x,z) \leq d(x,y),
d'où le résultat.

Exemple~4.2.1 Sur tout ensemble, d(x,y) = \left
\ \cases 1&si
x\neq~y \cr 0&si x = y
\cr  \right . est une distance sur E
appelée la distance discrète. Sur K = \mathbb{R}~ ou K = \mathbb{C}, d(x,y) = x -
y est une distance appelée la distance usuelle. Sur
K^n on trouve classiquement trois distances utiles
d_1(x,y) =\ \\sum
 _ix_i - y_i,
d_2(x,y) =
\sqrt\\\sum
 _ix_i -
y_i^2 et d_\infty~(x,y)
= max_ix_i~ -
y_i si x =
(x_1,\\ldots,x_n~)
et y =
(y_1,\\ldots,y_n~).

Définition~4.2.2 On appelle boule ouverte de centre a de rayon r
> 0~: B(a,r) = \x \in
E∣d(a,x) < r\.

On appelle boule fermée de centre a de rayon r > 0~:
B'(a,r) = \x \in E∣d(a,x) \leq
r\.

On appelle sphère de centre a de rayon r > 0~: S(a,r) =
\x \in E∣d(a,x) =
r\

Définition~4.2.3 Soit (E,d) un espace métrique et d_F la
restriction de d à F \times F. Alors d_F est encore une distance sur
F appelée la distance induite par d.

Remarque~4.2.1 On a clairement B_d_F(a,r) =
B_d(a,r) \bigcap F et le résultat similaire pour les boules fermées,
si a \in F.

Définition~4.2.4 Soit
(E_1,d_1),\\ldots,(E_k,d_k~)
des espaces métriques. Soit E = E_1
\times⋯ \times E_k. On définit alors sur E une
distance produit par d(x,y) =\
max_id_i(x_i,y_i) si x =
(x_1,\\ldots,x_k~)
et y =
(y_1,\\ldots,y_k~).

Définition~4.2.5

\begin{itemize}
\itemsep1pt\parskip0pt\parsep0pt
\item
  (i) Soit x \in E et A \subset~ E, A\neq~\varnothing~. On appelle
  distance de x à A le réel d(x,A) = inf~
  \d(x,a)∣a \in
  A\
\item
  (ii) A,B \subset~ E non vides. On appelle distance de A et B le réel d(A,B)
  = inf~
  \d(a,b)∣a \in A,b \in
  B\
\item
  (iii) On appelle diamètre de A \subset~ E, A\neq~\varnothing~, le
  nombre \delta(A) =\
  sup\d(a,a')∣a,a' \in
  A\ \in \mathbb{R}~ \cup\ + \infty~\~; on
  dit que A est bornée si \delta(A) < +\infty~.
\end{itemize}

Définition~4.2.6 Soit (E,d) et (F,\delta) deux espaces métriques. On appelle
isométrie de E sur F toute application f : E \rightarrow~ F bijective qui conserve
la distance~:

\forall~~x,y \in E, \delta(f(x),f(y)) = d(x,y)

\subsection{4.2.2 Topologie définie par une distance}

Définition~4.2.7 Soit (E,d) un espace métrique. On appelle topologie
définie sur E par la distance d l'ensemble des parties U de E (les
ouverts de la topologie) vérifiant

\forall~x \in U, \\exists~r
> 0,\quad B(x,r) \subset~ U

Démonstration C'est bien une topologie~: clairement E et \varnothing~ sont des
ouverts~; si U et U' sont des ouverts et x \in U \bigcap U', il existe r
> 0 et r' > 0 tels que B(x,r) \subset~ U et B(x,r') \subset~
U' et alors r_0 = min~(r,r')
> 0 est tel que B(x,r_0) \subset~ U \bigcap U'. Si les
U_i, i \in I sont des ouverts, soit x
\in\⋃ ~
_i\inIU_i. Il existe i_0 tel que x \in
U_i_0 puis r > 0 tel que B(x,r) \subset~
U_i_0. On a alors B(x,r)
\subset~\⋃ ~
_i\inIU_i.

Proposition~4.2.2 Dans un espace métrique, toute boule ouverte est un
ouvert, toute boule fermée est un fermé.

Démonstration Soit x \in B(a,r) et \rho = r - d(a,x) > 0. Si y \in
B(x,\rho), on a d(a,y) \leq d(a,x) + d(x,y) < d(a,x) + \rho = r soit
B(x,\rho) \subset~ B(a,r). De même on montre que si
x∉B'(a,r) et si \rho = d(a,x) - r >
0, alors B(x,\rho) \subset~ E \diagdown B'(a,r). Donc E \diagdown B'(a,r) est ouvert et B'(a,r)
est fermé.

Remarque~4.2.2

\begin{itemize}
\itemsep1pt\parskip0pt\parsep0pt
\item
  (i) V \in V (a) \Leftrightarrow
  \exists~r > 0, B(a,r) \subset~ V
\item
  (ii) a \in A^o \Leftrightarrow
  \exists~r > 0, B(a,r) \subset~ A
\item
  (iii) \overlineA = \x \in
  E∣\forall~~r
  > 0, B(x,r) \bigcap
  A\neq~\varnothing~\
\item
  (iv) \mathrmFr~(A) =
  \x \in
  E∣\forall~~r
  > 0, B(x,r) \bigcap
  A\neq~\varnothing~\text et B(x,r)
  \bigcapcA\neq~\varnothing~\
\end{itemize}

Proposition~4.2.3 Soit (E,d) un espace métrique et F \subset~ E. Alors la
topologie induite sur F est la topologie définie par la distance
d_F.

Démonstration On remarque que si a \in F, B_d_F(a,r) =
B_d(a,r) \bigcap F. Soit V un ouvert pour la topologie induite, soit
U ouvert de E tel que V = U \bigcap F. On a a \in U, donc il existe r
> 0 tel que B_d(a,r) \subset~ U. On a alors
B_d_F(a,r) = B_d(a,r) \bigcap F \subset~ U \bigcap F \subset~ U.
Inversement, soit V un ouvert pour la distance d_F. Pour tout x
\in V , il existe r_x > 0 tel que
B_d_F(x,r_x) \subset~ V . On a alors V
= \⋃  _x\inV
B_d_F(x,r_x) (cette réunion contient V de
manière évidente et est contenue dans V car réunion de parties de V ).
On pose alors U =\ \⋃
 _x\inV B_d(x,r_x). C'est un ouvert de E et
on a V = U \bigcap F.

Remarque~4.2.3 Ceci montre que la topologie définie par la distance
d_F ne dépend que de la topologie sur E et pas vraiment de la
distance d. Montrons de même que la topologie définie par la distance
produit ne dépend que des topologies sur les espaces et pas des
distances elles-mêmes

Proposition~4.2.4 Soit (E_1,d_1) et
(E_2,d_2) deux espaces métriques et (E_1 \times
E_2,\delta) l'espace métrique produit. Soit U \subset~ E_1 \times
E_2. Alors U est ouvert si et seulement si~

\forall~(a_1,a_2~) \in U,
\existsV _1 \in V (a_1~),
\existsV _2~ \in V
(a_2),\quad V _1 \times V _2 \subset~ U

Démonstration Supposons que U est ouvert pour la distance produit. Si
(a_1,a_2) \in U, il existe r > 0 tel que
B_\delta((a_1,a_2),r) \subset~ U. Mais on a

\begin{align*}
B_\delta((a_1,a_2),r)& =&
\(x_1,x_2)∣max(d_1(x_1,a_1),d_2(a_2,r_2~))
< r\\%& \\ &
=& B_d_1(a_1,r) \times
B_d_2(a_2,r) \%&
\\ \end{align*}

et donc V _1 = B_d_1(a_1,r) et V
_2 = B_d_2(a_2,r) sont des voisinages
de a_1 et a_2 tels que V _1 \times V _2 \subset~
U. Inversement, si U vérifie cette propriété, soit
(a_1,a_2) \in U et soit V _1 \in V
(a_1), V _2 \in V (a_2) tels que V _1
\times V _2 \subset~ U. Il existe r_1 > 0 et
r_2 > 0 tels que
B_d_i(a_i,r_i) \subset~ V _i. Soit
r = min(r_1,r_2~)
> 0. On a

\begin{align*}
B_\delta((a_1,a_2),r)& =&
B_d_1(a_1,r) \times
B_d_2(a_2,r) \%&
\\ & \subset~&
B_d_1(a_1,r_1) \times
B_d_2(a_2,r_2) \subset~ V _1 \times V
_2 \subset~ U\%& \\
\end{align*}

donc U est un ouvert pour \delta, ce qui achève la démonstration.

Remarque~4.2.4 En particulier, si U_1 et U_2 sont des
ouverts de E_1 et E_2, alors U_1 \times
U_2 est un ouvert de E_1 \times E_2~; un tel
ouvert sera dit ouvert élémentaire.

\subsection{4.2.3 Points isolés, points d'accumulation}

Soit toujours F une partie de E et x \in\overlineF. On
sait que \forall~~V \in V (x) V \bigcap
F\neq~\varnothing~. On a alors deux possibilités suivant que
V \bigcap F peut être réduit à \x\ ou non.

Définition~4.2.8

\begin{itemize}
\itemsep1pt\parskip0pt\parsep0pt
\item
  (i) On dit que x \in F est point isolé de F, s'il existe V voisinage de
  x dans E tel que V \bigcap F = \x\
\item
  (ii) On dit que x \in E est point d'accumulation de F si pour tout
  voisinage V de x dans E, V \bigcap F
  \diagdown\x\\neq~\varnothing~.
\end{itemize}

Théorème~4.2.5 Soit E un espace métrique.

\begin{itemize}
\itemsep1pt\parskip0pt\parsep0pt
\item
  (i) x \in F est point isolé de F si et seulement
  si~\x\ est ouvert dans F
\item
  (ii) x \in E est point d'accumulation de F si et seulement si~pour tout
  voisinage V de x dans E, V \bigcap F est infini.
\end{itemize}

Démonstration (i) est tout à fait élémentaire et résulte de la
définition de la topologie induite. En ce qui concerne (ii), la partie (
⇐) est évidente. Montrons donc la partie ( \rigtharrow~). Soit x un point
d'accumulation de F, V un voisinage de x et r > 0 tel que
B(x,r) \subset~ V . Alors (B(x,r) \diagdown\x\) \bigcap
F\neq~\varnothing~. Soit x_1 \in (B(x,r)
\diagdown\x\) \bigcap F. Si x_n est supposé
construit, on pose r_n = d(x,x_n) > 0 et
on choisit x_n+1 \in (B(x,r_n)
\diagdown\x\) \bigcap
F\neq~\varnothing~. Alors la suite (d(x,x_n)) est
strictement décroissante, ce qui montre que les x_n sont deux à
deux distincts. Ils sont tous dans F et dans B(x,r) donc dans V .

\subsection{4.2.4 Propriété de séparation}

Théorème~4.2.6 Soit E un espace métrique, a et b deux points distincts
de E. Alors il existe U ouvert contenant a et V ouvert contenant b tels
que U \bigcap V = \varnothing~.

Démonstration Soit r = 1 \over 3 d(a,b), U = B(a,r)
et V = B(b,r) conviennent.

Corollaire~4.2.7 Soit E un espace métrique et \Delta =
\(x,x) \in E \times E\. Alors \Delta est fermée
dans E \times E.

Démonstration Soit (a,b) \in E \times E \diagdown \Delta. On a donc
a\neq~b. Il existe U ouvert contenant a et V
ouvert contenant b tels que U \bigcap V = \varnothing~. Alors U \times V est un ouvert de E \times
E (élémentaire) et (U \times V ) \bigcap \Delta = \varnothing~, soit U \times V \subset~ E \times E \diagdown \Delta. Donc E \times E
\diagdown \Delta est voisinage de tous ses points et il est ouvert. Donc \Delta est
fermée.

\subsection{4.2.5 Changement de distances}

Définition~4.2.9 Soit E un ensemble. On dit que deux distances
d_1 et d_2 sur E sont topologiquement équivalentes si
elles définissent la même topologie (il s'agit clairement d'une relation
d'équivalence).

Théorème~4.2.8 Soit E un ensemble, d et d' deux distances sur E. Ces
distances sont topologiquement équivalentes si et seulement si~elles
vérifient

\begin{itemize}
\itemsep1pt\parskip0pt\parsep0pt
\item
  (i) \forall~a \in E, \\forall~~r
  > 0, \exists~r' >
  0,\quad B_d'(a,r') \subset~ B_d(a,r)
\item
  (ii) \forall~a \in E, \\forall~~r'
  > 0, \exists~r >
  0,\quad B_d(a,r) \subset~ B_d'(a,r')
\end{itemize}

Démonstration Ces conditions sont évidemment nécessaires puisque les
boules ouvertes pour d doivent être des ouverts pour d' et
réciproquement. Supposons maintenant que (i) est vérifiée et soit U un
ouvert pour d. Soit a \in U. Il existe r > 0 tel que
B_d(a,r) \subset~ U. Alors \exists~r'
> 0,\quad B_d'(a,r') \subset~
B_d(a,r) \subset~ U. On en déduit que U est ouvert pour d', donc
T_d \subset~T_d'. De même (ii) traduit l'inclusion
T_d' \subset~T_d.

Définition~4.2.10 Soit E un ensemble. On dit que deux distances
d_1 et d_2 sur E sont équivalentes s'il existe \alpha~ et \beta~
strictement positifs tels que

\forall~~x,y \in E,\quad
\alpha~d_1(x,y) \leq d_2(x,y) \leq \beta~d_1(x,y)

Proposition~4.2.9 Deux distances équivalentes sont topologiquement
équivalentes.

Démonstration On a d_1(a,x) < r
\over \beta~ \rigtharrow~ d_2(x,y) < r soit
B_d_1(a, r \over \beta~ ) \subset~
B_d_2(a,r). De même B_d_2(a,\alpha~r) \subset~
B_d_1(a,r).

Remarque~4.2.5 Soit d une distance sur E et posons d'(x,y)
= min~(1,d(x,y)). On vérifie facilement que d
est une distance sur E, que d et d' sont topologiquement équivalentes
(B_d(a,r) \subset~ B_d'(a,r) et
B_d'(a,min~( 1 \over
2 ,r)) \subset~ B_d(a,r)). Mais en général, d et d' ne sont pas
équivalentes (d' est toujours bornée alors que d ne l'est pas en
général).

\subsection{4.2.6 La droite numérique achevée}

On pose \overline\mathbb{R}~ = \mathbb{R}~
\cup\-\infty~,+\infty~\ muni de la relation d'ordre
évidente. Les intervalles ouverts sont donc les intervalles de la forme

\begin{itemize}
\itemsep1pt\parskip0pt\parsep0pt
\item
  (i) I =]a,b[= \x \in \mathbb{R}~∣a
  < x < b\ pour a,b
  \in\overline\mathbb{R}~
\item
  (ii) I =]a,+\infty~] = \x
  \in\overline\mathbb{R}~∣a
  < x\ ou I = [-\infty~,a[= \x
  \in\overline\mathbb{R}~∣x
  < a\
\item
  (iii) I = [-\infty~,+\infty~] = \overline\mathbb{R}~
\end{itemize}

Comme sur \mathbb{R}~, ces intervalles ouverts engendrent une topologie appelée la
topologie usuelle de \overline\mathbb{R}~. On a alors

\begin{itemize}
\item
  (i) si a \in \mathbb{R}~, V \in V (a) \Leftrightarrow
  \exists~\epsilon > 0,\quad
  ]a - \epsilon,a + \epsilon[\subset~ V
\item
  (ii) si a = +\infty~,

  V \in V (+\infty~) \Leftrightarrow \exists~A
  > 0,\quad ]A,+\infty~] \subset~ V
\item
  (iii) si a = -\infty~,

  V \in V (-\infty~) \Leftrightarrow \exists~A
  < 0,\quad [-\infty~,A[\subset~ V
\end{itemize}

Théorème~4.2.10 La topologie usuelle sur \overline\mathbb{R}~
est définie par une distance.

Démonstration Soit \phi : \overline\mathbb{R}~ \rightarrow~ [-1,1]
définie par \phi(x) = \left \
\cases  x \over
1+x &si x \in \mathbb{R}~ \cr 1 &si x = +\infty~
\cr -1 &si x = -\infty~ \cr 
\right .. L'application \phi est une bijection strictement
croissante donc respecte les intervalles ouverts, donc les topologies
usuelles~: si U \subset~\overline\mathbb{R}~, U est ouvert dans
\overline\mathbb{R}~ si et seulement si~\phi(U) est ouvert dans
[-1,1]~; comme la topologie sur [-1,1] est définie par la
distance x - y, la topologie sur
\overline\mathbb{R}~ est définie par la distance d(x,y) =
\phi(x) - \phi(y) (pour cette distance, \phi devient une
isométrie).

Remarque~4.2.6 On vérifie immédiatement que la topologie usuelle de
\overline\mathbb{R}~ induit sur \mathbb{R}~ la topologie usuelle de \mathbb{R}~.

[
[
[
[

\end{document}
