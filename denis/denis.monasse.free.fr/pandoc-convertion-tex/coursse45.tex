
\section{8.2 Dérivée}

\section{Notion de dérivée}
\label{sec:notion-de-derivee}



Définition~8.2.1 Soit I un intervalle de \mathbb{R}~, E un espace vectoriel
normé~et f : I \rightarrow~ E. On dit que f est dérivable en a \in I si existe
lim_x\rightarrow~a,x\neq~a~
f(x)-f(a) \over x-a ~; dans ce cas cette limite est
appelée la dérivée de f au point a et notée f'(a).

Remarque~8.2.1 Comme toute notion de limite, il s'agit d'une notion
locale~: f : I \rightarrow~ E est dérivable au point a si et seulement si~sa
restriction à ]a - \eta,a + \eta[\bigcapI est dérivable au point a.

Proposition~8.2.1 Si f est dérivable au point a \in I elle est continue au
point a.

Démonstration On écrit pour x\neq~a,  f(x)-f(a)
\over x-a = f'(a) + \epsilon(x - a) avec
lim_h\rightarrow~0~\epsilon(h) = 0~; on a donc f(x) =
f(a) + (x - a)f'(a) + (x - a)\epsilon(x - a) ce qui montre que
lim_x\rightarrow~a,x\neq~a~f(x)
= f(a)~; donc f est continue au point a.

Définition~8.2.2 Soit I un intervalle de \mathbb{R}~, E un espace vectoriel
normé~et f : I \rightarrow~ E. On dit que f est dérivable si elle est dérivable en
tout point de I~; l'application f' : a\mapsto~f'(a)
est appelée la dérivée de f.

Remarque~8.2.2 On a donc~: f dérivable \rigtharrow~ f continue.

\subsection{8.2.2 Opérations sur les dérivées}

Théorème~8.2.2 Soit I un intervalle de \mathbb{R}~, E un espace vectoriel normé~et
f et g des applications de I dans E dérivables au point a~; si \alpha~ et \beta~
sont des scalaires, \alpha~f + \beta~g est dérivable au point a et (\alpha~f + \beta~g)'(a) =
\alpha~f'(a) + \beta~g'(a).

Démonstration Il suffit de remarquer que  (\alpha~f+\beta~g)(x)-(\alpha~f+\beta~g)(a)
\over x-a = \alpha~ f(x)-f(a) \over x-a +
\beta~ g(x)-g(a) \over x-a et d'appliquer les théorèmes
sur les limites.

Théorème~8.2.3 Soit I un intervalle de \mathbb{R}~, E,F,G trois espaces vectoriels
normés, f : I \rightarrow~ E, g : I \rightarrow~ F~; soit u : E \times F \rightarrow~ G une application
bilinéaire continue et h : I \rightarrow~ G,
t\mapsto~u(f(t),g(t)). Si f et g sont dérivables au
point a, alors h est dérivable au point a et h'(a) = u(f'(a),g(a)) +
u(f(a),g'(a)).

Démonstration On vérifie immédiatement que  h(x)-h(a)
\over x-a = u( f(x)-f(a) \over x-a
,g(x)) + u(f(a), g(x)-g(a) \over x-a ). Or
lim_x\rightarrow~a,x\neq~a~
f(x)-f(a) \over x-a = f'(a),
lim_x\rightarrow~a,x\neq~a~g(x)
= g(a) et
lim_x\rightarrow~a,x\neq~a~
g(x)-g(a) \over x-a = g'(a). Comme u est continue, on a
lim_x\rightarrow~a,x\neq~a~
h(x)-h(a) \over x-a = u(f'(a),g(a)) + u(f(a),g'(a)).

Remarque~8.2.3 Ce théorème s'étend immédiatement au cas d'une
application p-linéaire continue u : E_1
\times⋯ \times E_p dans G. Dans ce cas on a

h'(a) = \sum _i=1^pu(f_
1(a),\ldots,f_i-1(a),f_i'(a),f_i+1(a),\\ldots,f_p~(a))

En particulier, dans le cas d'un déterminant on retiendra

\begin{align*}
[\mathrm{det}~
_\mathcal{E}(f_1,\\ldots,f_n~)]'(a)&&
\%& \\ & =& \\sum
_i=1^n \mathrm{det} _
\mathcal{E}(f_1(a),\ldots,f_i-1(a),f_i'(a),f_i+1(a),\\ldots,f_n~(a))\%&
\\ \end{align*}

Théorème~8.2.4 Soit \phi : I \rightarrow~ \mathbb{R}~ et f : J \rightarrow~ E avec \phi(I) \subset~ J~; soit a \in I.
Si \phi est dérivable au point a et si f est dérivable au point f(a), alors
f \cdot \phi est dérivable au point a et (f \cdot \phi)'(a) = \phi'(a)f'(\phi(a)).

Démonstration En effet la dérivabilité de f au point \phi(a) peut se
traduire par

f(x) - f(\phi(a)) = (x - \phi(a))f'(\phi(a)) + (x - \phi(a))\epsilon(x)

avec lim_x\rightarrow~\phi(a)~\epsilon(x) = 0. On a donc

f(\phi(t)) - f(\phi(a)) = (\phi(t) - \phi(a))f'(\phi(a)) + (\phi(t) - \phi(a))\epsilon(\phi(t))

avec lim_t\rightarrow~a~\epsilon(\phi(t)) = 0 puisque \phi est
continue au point a.

De même la dérivabilité de \phi au point a se traduit par

\phi(t) - \phi(a) = (t - a)\phi'(a) + o(t - a)

On obtient alors en rempla\ccant

f(\phi(t)) - f(\phi(a)) = (t - a)\phi'(a)f'(\phi(a)) + o(t - a)

ce qui montre que

lim_t\rightarrow~a,t\neq~a~f(\phi(t))
- f(\phi(a))\over t - a = \phi'(a)f'(\phi(a))

Théorème~8.2.5 Soit f : I \rightarrow~ \mathbb{R}~, a \in I tel que
f(a)\neq~0. Si f est dérivable au point a, il
existe \epsilon > 0 tel que f ne s'annule pas sur J = I\bigcap]a - \epsilon,a
+ \epsilon[. La fonction  1 \over f est dérivable au point
a et \left ( 1 \over f
\right )'(a) = - f'(a) \over
f(a)^2 .

Démonstration La fonction f étant continue au point a, il existe \epsilon
> 0 tel que t \in I\bigcap]a - \epsilon,a + \epsilon[\rigtharrow~f(t) -
f(a) < f(a)
\over 2 ~; on en déduit que t \in J \rigtharrow~
f(t)\neq~0. Pour t \in J
\diagdown\a\ on a  1 \over
t-a \left ( 1 \over f (t) - 1
\over f (a)\right ) = - 1
\over f(t)f(a)  f(t)-f(a) \over t-a
qui tend vers - f'(a) \over f(a)^2 quand t
tend vers a.

\subsection{8.2.3 Dérivées d'ordre supérieur}

Définition~8.2.3 Soit I un intervalle de \mathbb{R}~, E un espace vectoriel
normé~et f : I \rightarrow~ E. Soit n ≥ 1. On dit que f est n fois dérivable au
point a \in I s'il existe \eta > 0 tel que f est n - 1 fois
dérivable sur I\bigcap]a - \eta,a + \eta[ et si l'application f^(n-1)
est dérivable au point a~; on pose alors f^(n)(a) =
(f^(n-1))'(a). On dit que f est n fois dérivable sur I si
elle est n fois dérivable en tout point de I~; on dit que f est de
classe C^n si elle est n fois dérivable sur I et si
f^(n) est continue sur I~; on dit que f est C^\infty~ si
elle est de classe C^n pour tout n.

Remarque~8.2.4 Puisque toute fonction dérivable est continue, si f est n
fois dérivable, elle est de classe C^n-1.

Théorème~8.2.6 (Leibnitz). Soit I un intervalle de \mathbb{R}~, E,F,G trois
espaces vectoriels normés, f : I \rightarrow~ E, g : I \rightarrow~ F~; soit u : E \times F \rightarrow~ G une
application bilinéaire continue et h : I \rightarrow~ G,
t\mapsto~u(f(t),g(t)). Si f et g sont n fois
dérivables au point a, alors h est n fois dérivable au point a et

h^(n)(a) = \\sum
_p=0^nC_
n^pu(f^(p)(a),g^(n-p)(a))

Démonstration Par récurrence sur n~; le résultat a déjà été vu pour n =
1~; supposons le vrai pour n - 1 et soit \epsilon > 0 tel que f et
g soient n - 1 fois dérivables sur I\bigcap]a - \eta,a + \eta[. L'hypothèse de
récurrence implique que h est n - 1 fois dérivable sur I\bigcap]a - \eta,a +
\eta[ et que sa dérivée (n - 1)^\textième
est donnée par

h^(n-1)(t) = \\sum
_p=0^n-1C_
n-1^pu(f^(p)(t),g^(n-1-p)(t))

Mais toutes les applications f^(p), g^(n-1-p) sont
dérivables au point a~; il en est donc de même des applications
t\mapsto~u(f^(p)(t),g^(n-1-p)(t)),
et donc de h^(n-1). Donc h est n fois dérivable au point a et

\begin{align*} h^(n)(a)& =&
\sum _p=0^n-1C_
n-1^p(u(f^(p+1)(a),g^(n-1-p)(a))\%&
\\ & \text &
\quad + u(f^(p)(a),g^(n-p)(a)))
\%& \\ & =& \\sum
_p=1^nC_
n-1^p-1u(f^(p)(a),g^(n-p)(a)) \%&
\\ & \text &
\quad + \\sum
_p=0^n-1C_
n-1^pu(f^(p)(a),g^(n-p)(a)) \%&
\\ \end{align*}

en changeant dans la première somme p + 1 en p~; puis

\begin{align*} h^(n)(a)& =&
u(f(a),g^(n)(a)) \%& \\ &
\text & +\\sum
_p=1^n-1(C_ n-1^p-1 + C_
n-1^p)u(f^(p)(a),g^(n-p)(a))\%&
\\ & \text &
+u(f^(n)(a),g(a)) \%& \\ &
=& \sum _p=0^nC_
n^pu(f^(p)(a),g^(n-p)(a)) \%&
\\ \end{align*}

ce qui achève la récurrence.

Corollaire~8.2.7 Sous les mêmes hypothèses, si f et g sont de classe
C^n, h est de classe C^n.

Démonstration C'est clair d'après la formule ci dessus.

Théorème~8.2.8 Soit \phi : I \rightarrow~ \mathbb{R}~ et f : J \rightarrow~ E avec \phi(I) \subset~ J~; soit a \in I.
Si \phi est n fois dérivable au point a et si f est n fois dérivable au
point f(a), alors f \cdot \phi est n fois dérivable au point a.

Démonstration Par récurrence sur n~; le résultat a déjà été vu pour n =
1~; supposons le vrai pour n - 1 et soit \eta > 0 tel que f \cdot
\phi soit dérivable sur I\bigcap]a - \eta,a + \eta[ avec (f \cdot \phi)' = \phi'(f' \cdot \phi).
Comme f' et \phi sont n - 1 fois dérivables en a, l'hypothèse de récurrence
implique que f' \cdot \phi est n - 1 fois dérivable en a~; comme \phi' l'est
également, le théorème de Leibnitz appliqué au produit ordinaire assure
que (f \cdot \phi)' = \phi'(f' \cdot \phi) est n - 1 fois dérivable au point a, donc que
f \cdot \phi est n fois dérivable au point a.

Corollaire~8.2.9 Sous les mêmes hypothèses, si f et \phi sont de classe
C^n, f \cdot \phi est de classe C^n.
