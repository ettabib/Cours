Voici le texte avec les environnements LaTeX demandés :

\section{Convergence des séries entières}

\subsection{Notion de série entière}

\begin{de}
Soit $(a_n)_{n \in \mathbb{N}}$ une suite de l'espace
vectoriel normé complet $E$. On appelle série entière associée à la suite
$(a_n)$ la série de fonctions de $\mathbb{C}$ (resp. $\mathbb{R}$) dans $E$,
$\sum_{n \geq 0} u_n$, où l'on pose $u_n(z) = a_n z^n$ ; on notera simplement
$\sum_{n \geq 0} a_n z^n$ cette série de fonctions de la
variable $z$.
\end{de}

\begin{rem}
Dans le cas où $E = \mathbb{R}$ ou $E = \mathbb{C}$, la série entière est
associée à une unique série formelle
$\sum_{n=0}^{+\infty} a_n X^n \in K[[X]]$.
\end{rem}

\subsection{Rayon de convergence}

\begin{lem}[Abel]
Soit $E$ un $K$-espace vectoriel normé complet,
$\sum a_n z^n$ une série entière à coefficients dans $E$. Soit
$z_0 \in K^*$ tel que la suite
$(a_n z_0^n)$ soit bornée. Alors la série
$\sum a_n z^n$ converge absolument pour tout $z \in K$ tel que
$|z| < |z_0|$ ;
la série entière converge même normalement dans tout disque fermé $D'(0,r) =
\{z \in K | |z| \leq r\}$
pour tout nombre réel $r$ tel que $r < |z_0|$.
\end{lem}

\begin{proof}
Soit $M \geq 0$ tel que $\forall n \in \mathbb{N},
\|a_n z_0^n\| \leq M$ et soit $z \in K$ tel que $|z| < |z_0|$. On a alors
$\|a_n z^n\| = \|a_n z_0^n\| \left|\frac{z}{z_0}\right|^n \leq M \left|\frac{z}{z_0}\right|^n$. Comme $\left|\frac{z}{z_0}\right| < 1$, la série géométrique est convergente et donc la série
$\sum a_n z^n$ converge absolument. Pour $z \in D'(0,r)$, on a
de la même façon
$\|a_n z^n\| \leq M \left(\frac{r}{|z_0|}\right)^n$ qui est une
série convergente indépendante de $z$, donc la série converge normalement
sur $D'(0,r)$.
\end{proof}

\begin{thm}
Soit $E$ un $K$-espace vectoriel normé complet,
$\sum a_n z^n$ une série entière à coefficients dans $E$.
Posons $R_1 = \sup\{|z| | \sum a_n z^n \text{ converge}\} \in \mathbb{R}^+ \cup \{+\infty\}$ et $R_2 = \sup\{|z| | (a_n z^n) \text{ est bornée}\} \in \mathbb{R}^+ \cup \{+\infty\}$. On a $R_1 = R_2$. En notant $R$ la
valeur commune, la série converge absolument dans $D(0,R) = \{z \in K | |z| < R\}$ et converge normalement dans tout disque
fermé $D'(0,r) = \{z \in K | |z| \leq r\}$ tel que $r < R$.
\end{thm}

\begin{proof}
Soit $r \in [0,R_1[$ ; d'après la propriété
caractéristique de la borne supérieure, il existe $z \in K$ tel que la série
$\sum a_n z^n$ converge avec $r < |z| \leq R_1$. Mais alors
$\lim a_n z^n = 0$, donc la
suite $(a_n z^n)$ est bornée et a fortiori la suite
$(a_n r^n)$ est bornée ; donc $r \in [0,R_2]$,
soit $[0,R_1[ \subset [0,R_2]$ et donc $R_1 \leq R_2$. Soit $r \in [0,R_2[$ ; d'après la propriété
caractéristique de la borne supérieure, il existe $z \in K$ tel que la suite
$(a_n z^n)$ soit bornée avec $r < |z| \leq R_2$. Mais alors, d'après le lemme
d'Abel, la série $\sum a_n r^n$ converge absolument, $r \in [0,R_1]$, soit $[0,R_2[ \subset [0,R_1]$ et
donc $R_2 \leq R_1$.

Soit alors $z \in D(0,R)$ ; il existe $z_0 \in K$ tel que la suite
$(a_n z_0^n)$ soit bornée avec
$|z| < |z_0| \leq R$.
Mais alors, d'après le lemme d'Abel, la série
$\sum a_n z^n$ converge absolument. De même, soit $r < R$ ; il existe $z_0 \in K$ tel que la suite
$(a_n z_0^n)$ soit bornée avec $r < |z_0| \leq R$. Mais alors, d'après le lemme
d'Abel, la série $\sum a_n z^n$ converge normalement sur $D'(0,r)$.
\end{proof}

\begin{rem}
Le lecteur prendra garde au fait qu'en général la série
ne converge pas normalement sur $D(0,R)$ ni même uniformément.
\end{rem}

\begin{de}
$R$ est appelé le rayon de convergence de la série
entière, $D(0,R)$ son disque ouvert de convergence, $C(0,R) = \{z \in K | |z| = R\}$ son cercle de convergence.
\end{de}

\begin{rem}
Par définition même au vu des résultats précédents, la
série converge absolument dans le disque ouvert de convergence (et même
uniformément dans tout disque fermé inclus dans le disque ouvert de
convergence) ; pour $|z| > R$ la série diverge et en fait, la suite $(a_n z^n)$ n'est même pas
bornée. La nature de la série sur le disque fermé de convergence dépend
de la série et du point considéré.
\end{rem}

\begin{example}
Soit $\alpha \in \mathbb{R}$ ; la série entière
$\sum \frac{z^n}{n^\alpha}$ a pour rayon de convergence $1$ ;
pour $|z| = 1$ la nature de la série dépend à la fois de
$\alpha$ et de $z$.

\begin{itemize}
\itemsep1pt\parskip0pt\parsep0pt
\item
  (i) Pour $\alpha > 1$, la série converge pour tout $z$ tel que
  $|z| = 1$
\item
  (ii) Pour $\alpha \leq 0$, la série diverge pour tout $z$ tel que
  $|z| = 1$ (le terme général ne tend pas vers $0$)
\item
  (iii) Pour $0 < \alpha \leq 1$, la série diverge en $z = 1$ mais
  converge pour tout point $z$ tel que $|z| = 1$ et
  $z \neq 1$ (appliquer le critère d'Abel).
\end{itemize}
\end{example}

\subsection{Recherche du rayon de convergence}

Les deux remarques suivantes, qui découlent immédiatement des résultats
précédents peuvent rendre de grands services dans la détermination du
rayon de convergence

\begin{itemize}
\itemsep1pt\parskip0pt\parsep0pt
\item
  (i) si $z \in K$ est tel que la série
  $\sum a_n z^n$ converge, alors $|z| \leq R$
\item
  (ii) si $z \in K$ est tel que la série
  $\sum a_n z^n$ diverge, alors $R \leq |z|$
\end{itemize}

On pourra éventuellement, pour cette recherche, trouver refuge dans l'un
des théorèmes suivants

\begin{thm}[Règle de d'Alembert]
On suppose que
$\forall n \in \mathbb{N}, a_n \neq 0$ ; si la suite 
$\frac{|a_{n+1}|}{|a_n|}$ admet
une limite $\ell \in \mathbb{R}^+ \cup \{+\infty\}$,
alors le rayon de convergence de la série entière
$\sum a_n z^n$ est $\frac{1}{\ell}$.
\end{thm}

\begin{proof}
Il suffit d'appliquer la règle de d'Alembert pour les
séries numériques en remarquant que 
$\frac{|a_{n+1} z^{n+1}|}{|a_n z^n|}$
admet la limite $\ell |z|$ et que donc la série converge
absolument pour $\ell |z| < 1$ et diverge pour
$\ell |z| > 1$.
\end{proof}

\begin{rem}
On prendra garde à la condition
$\forall n \in \mathbb{N}, a_n \neq 0$. En particulier, on ne tentera
pas d'appliquer cette règle à des séries comportant une infinité de
termes nuls comme les séries entières du type
$\sum a_n z^{2n}$ ou
$\sum a_n z^{n^2}$ ; c'est ainsi qu'une
application imprudente de la règle de d'Alembert à la série entière
$\sum 3^n z^{2n}$ pourrait faire croire que le rayon de
convergence est $\frac{1}{3}$ alors que l'écriture
$3^n z^{2n} = (3z^2)^n$
montre qu'il vaut $\frac{1}{\sqrt{3}}$.
\end{rem}

\begin{example}
Pour $\sum \frac{z^n}{n!}$, on a $R = +\infty$ ; pour
$\sum \frac{z^n}{n^\alpha}$, on a $R = 1$ ; pour
$\sum n! z^n$, on a $R = 0$ (la série diverge pour tout $z \neq 0$).
\end{example}

\begin{thm}[Règle d'Hadamard]
Le rayon de convergence de la série
entière $\sum a_n z^n$ est égal à

$\frac{1}{\limsup \sqrt[n]{|a_n|}} \in \mathbb{R}^+ \cup \{+\infty\}$
\end{thm}

\begin{proof}
Posons $\ell = \limsup \sqrt[n]{|a_n|} \in \mathbb{R}^+ \cup \{+\infty\}$. Soit $z \in K$
tel que $|z| < \frac{1}{\ell}$.
On a alors $\ell < \frac{1}{|z|}$. Soit donc $\rho$ tel que $\ell < \rho < \frac{1}{|z|}$. D'après
la propriété de la limite supérieure, il existe $N \in \mathbb{N}$ tel que $n \geq N
\Rightarrow \sqrt[n]{|a_n|} \leq \rho$ soit encore
$|a_n| \leq \rho^n$ et donc
$|a_n z^n| \leq (\rho |z|)^n$. Mais $\rho |z| < 1$ et donc la série
$\sum (\rho |z|)^n$ converge. On en déduit que la
série $\sum a_n z^n$ converge absolument, soit $R \geq \frac{1}{\ell}$. De plus, si $|z| > \frac{1}{\ell}$, on a $\ell > \frac{1}{|z|}$. Comme $\ell$ est valeur
d'adhérence de la suite
$(\sqrt[n]{|a_n|})$,
il existe une infinité de $n$ tels que
$\sqrt[n]{|a_n|} > \frac{1}{|z|}$ soit
$|a_n z^n| > 1$ ; la suite $(a_n z^n)$ ne tend pas
vers $0$, donc la série diverge ; ceci montre que $R \leq \frac{1}{\ell}$, ce qui achève la démonstration.
\end{proof}

\begin{rem}
En particulier, si la suite
$(\sqrt[n]{|a_n|})$
converge vers $\ell$, on a $R = \frac{1}{\ell}$.
\end{rem}

\subsection{Opérations sur les séries entières}

\begin{prop}
Soit
$\sum a_n z^n$ et
$\sum b_n z^n$ deux séries entières à coefficients dans $E$ de
rayons de convergence respectifs $R_1$ et $R_2$, $\alpha$ et $\beta$
des scalaires. Alors la série entière
$\sum (\alpha a_n + \beta b_n) z^n$ a un rayon de convergence supérieur ou égal
à $\min(R_1,R_2)$ et

$|z| < \min(R_1,R_2) \Rightarrow \sum_{n=0}^{+\infty} (\alpha a_n + \beta b_n) z^n =
\alpha \sum_{n=0}^{+\infty} a_n z^n + \beta \sum_{n=0}^{+\infty} b_n z^n$
\end{prop}

\begin{proof}
En effet, si $|z| < \min(R_1,R_2)$,
les deux séries $\sum a_n z^n$ et
$\sum b_n z^n$ sont convergentes, et donc la série
$\sum (\alpha a_n + \beta b_n) z^n$ converge également, soit $R \geq \min(R_1,R_2)$. La formule
découle immédiatement du résultat similaire sur les séries numériques.
\end{proof}

\begin{rem}
L'exemple $b_n = -a_n$, $\alpha = \beta = 1$,
montre que l'on peut avoir $R > \min(R_1,R_2)$
\end{rem}

\begin{prop}
Soit
$\sum a_n z^n$ et
$\sum b_n z^n$ deux séries entières à coefficients dans $K$ de
rayons de convergence respectifs $R_1$ et $R_2$. Posons
$c_n = \sum_{k=0}^n a_k b_{n-k} = \sum_{p+q=n} a_p b_q$ (série entière produit). Alors la
série entière $\sum c_n z^n$ a un rayon de convergence supérieur ou égal à
$\min(R_1,R_2)$ et

$|z| < \min(R_1,R_2) \Rightarrow \sum_{n=0}^{+\infty} c_n z^n =
\left(\sum_{n=0}^{+\infty} a_n z^n\right)
\left(\sum_{n=0}^{+\infty} b_n z^n\right)$
\end{prop}

\begin{proof}
En effet, si $|z| < \min(R_1,R_2)$,
les deux séries $\sum a_n z^n$ et
$\sum b_n z^n$ sont absolument convergentes, et donc la
série produit de Cauchy est également absolument convergente. Mais on a
$\sum_{p+q=n} (a_p z^p)(b_q z^q) = z^n \sum_{p+q=n} a_p b_q = c_n z^n$. On a
donc $R \geq \min(R_1,R_2)$. La
formule découle immédiatement du résultat similaire sur les séries
numériques (la somme du produit de Cauchy est le produit des sommes des
séries).
\end{proof}

\begin{prop}
Soit
$\sum a_n z^n$ une série entière à coefficients dans $K$ avec
$a_0 \neq 0$. Il existe alors une unique
série entière $\sum b_n z^n$ tel que le produit des deux séries entières
soit la constante $1$. Si
$\sum a_n z^n$ a un rayon de convergence non nul $R$, il en
est de même du rayon de convergence $R'$ de la série entière
$\sum b_n z^n$. On a

$|z| < \min(R,R') \Rightarrow \sum_{n=0}^{+\infty} b_n z^n = \frac{1}{\sum_{n=0}^{+\infty} a_n z^n}$
\end{prop}

\begin{proof}
On doit avoir $a_0 b_0 = 1$ et pour $n \geq 1$,
$\sum_{k=0}^n a_{n-k} b_k = 0$. La suite
$(b_n)$ est donc définie par $b_0 = \frac{1}{a_0}$ et pour $n \geq 1$, $b_n = -\frac{1}{a_0} \sum_{k=0}^{n-1} a_{n-k} b_k$ ce qui définit
parfaitement la suite par récurrence. Remarquons que si l'on multiplie
tous les $a_n$ par $\lambda \neq 0$, tous les
$b_n$ sont divisés par $\lambda$, et les rayons de convergence ne sont
pas modifiés. Sans nuire à la généralité, on peut donc supposer que
$a_0 = 1$. On a alors $b_0 = 1$ et $b_n = -\sum_{k=0}^{n-1} a_{n-k} b_k$. Supposons donc $R > 0$ et soit $r < R$. La suite
$(a_n r^n)$ est donc bornée. Soit $M$ tel que
$\forall n, |a_n r^n| \leq M$ soit
$|a_n| \leq M r^{-n}$. On va montrer que
$\forall n \geq 1, |b_n| \leq M(M + 1)^{n-1} r^{-n}$ par récurrence sur $n$. Pour $n = 1$,
on a $b_1 = -a_1$, soit $|b_1| \leq |a_1| \leq M r^{-1}$ ce qui est bien
l'inégalité souhaitée. Supposons l'inégalité vérifiée de $1$ à $n - 1$. On a
alors (compte tenu de $b_0 = 1$)

\begin{align*}
|b_n| &\leq |a_n| + \sum_{k=1}^{n-1} |a_{n-k} b_k| \\
&\leq M r^{-n} + \sum_{k=1}^{n-1} M r^{k-n} M(M + 1)^{k-1} r^{-k} \\
&= M r^{-n} \left(1 + M \sum_{k=1}^{n-1} (M + 1)^{k-1}\right) \\
&= M r^{-n} \left(1 + M \frac{(M + 1)^{n-1} - 1}{(M + 1) - 1}\right) = M(M + 1)^{n-1} r^{-n}
\end{align*}

ce qui achève la récurrence. Alors
$|b_n z^n| \leq \frac{M}{M+1} \left(\frac{(M+1)|z|}{r}\right)^n$ ce qui
montre que la série $\sum b_n z^n$ converge pour $|z| < \frac{r}{M+1}$, soit $R' \geq \frac{r}{M+1} > 0$.

La formule découle immédiatement de la proposition précédente.
\end{proof}

\begin{rem}
L'exemple $a_0 = 1$, $a_1 = -1$,
$a_n = 0$ pour $n \geq 2$ (c'est-à-dire de la série entière $1 - z$),
pour laquelle la série inverse est la série
$\sum z^n$
pour laquelle $R' = 1$, montre qu'on ne peut pas dire grand chose de la
valeur effective de $R'$. On peut avoir aussi bien $R' \leq R$ que $R' \geq R$
(échanger le rôle de $\sum a_n z^n$ et
$\sum b_n z^n$).
\end{rem}