\documentclass[]{article}
\usepackage[T1]{fontenc}
\usepackage{lmodern}
\usepackage{amssymb,amsmath}
\usepackage{ifxetex,ifluatex}
\usepackage{fixltx2e} % provides \textsubscript
% use upquote if available, for straight quotes in verbatim environments
\IfFileExists{upquote.sty}{\usepackage{upquote}}{}
\ifnum 0\ifxetex 1\fi\ifluatex 1\fi=0 % if pdftex
  \usepackage[utf8]{inputenc}
\else % if luatex or xelatex
  \ifxetex
    \usepackage{mathspec}
    \usepackage{xltxtra,xunicode}
  \else
    \usepackage{fontspec}
  \fi
  \defaultfontfeatures{Mapping=tex-text,Scale=MatchLowercase}
  \newcommand{\euro}{€}
\fi
% use microtype if available
\IfFileExists{microtype.sty}{\usepackage{microtype}}{}
\ifxetex
  \usepackage[setpagesize=false, % page size defined by xetex
              unicode=false, % unicode breaks when used with xetex
              xetex]{hyperref}
\else
  \usepackage[unicode=true]{hyperref}
\fi
\hypersetup{breaklinks=true,
            bookmarks=true,
            pdfauthor={},
            pdftitle={Espaces et parties compactes},
            colorlinks=true,
            citecolor=blue,
            urlcolor=blue,
            linkcolor=magenta,
            pdfborder={0 0 0}}
\urlstyle{same}  % don't use monospace font for urls
\setlength{\parindent}{0pt}
\setlength{\parskip}{6pt plus 2pt minus 1pt}
\setlength{\emergencystretch}{3em}  % prevent overfull lines
\setcounter{secnumdepth}{0}
 
/* start css.sty */
.cmr-5{font-size:50%;}
.cmr-7{font-size:70%;}
.cmmi-5{font-size:50%;font-style: italic;}
.cmmi-7{font-size:70%;font-style: italic;}
.cmmi-10{font-style: italic;}
.cmsy-5{font-size:50%;}
.cmsy-7{font-size:70%;}
.cmex-7{font-size:70%;}
.cmex-7x-x-71{font-size:49%;}
.msbm-7{font-size:70%;}
.cmtt-10{font-family: monospace;}
.cmti-10{ font-style: italic;}
.cmbx-10{ font-weight: bold;}
.cmr-17x-x-120{font-size:204%;}
.cmsl-10{font-style: oblique;}
.cmti-7x-x-71{font-size:49%; font-style: italic;}
.cmbxti-10{ font-weight: bold; font-style: italic;}
p.noindent { text-indent: 0em }
td p.noindent { text-indent: 0em; margin-top:0em; }
p.nopar { text-indent: 0em; }
p.indent{ text-indent: 1.5em }
@media print {div.crosslinks {visibility:hidden;}}
a img { border-top: 0; border-left: 0; border-right: 0; }
center { margin-top:1em; margin-bottom:1em; }
td center { margin-top:0em; margin-bottom:0em; }
.Canvas { position:relative; }
li p.indent { text-indent: 0em }
.enumerate1 {list-style-type:decimal;}
.enumerate2 {list-style-type:lower-alpha;}
.enumerate3 {list-style-type:lower-roman;}
.enumerate4 {list-style-type:upper-alpha;}
div.newtheorem { margin-bottom: 2em; margin-top: 2em;}
.obeylines-h,.obeylines-v {white-space: nowrap; }
div.obeylines-v p { margin-top:0; margin-bottom:0; }
.overline{ text-decoration:overline; }
.overline img{ border-top: 1px solid black; }
td.displaylines {text-align:center; white-space:nowrap;}
.centerline {text-align:center;}
.rightline {text-align:right;}
div.verbatim {font-family: monospace; white-space: nowrap; text-align:left; clear:both; }
.fbox {padding-left:3.0pt; padding-right:3.0pt; text-indent:0pt; border:solid black 0.4pt; }
div.fbox {display:table}
div.center div.fbox {text-align:center; clear:both; padding-left:3.0pt; padding-right:3.0pt; text-indent:0pt; border:solid black 0.4pt; }
div.minipage{width:100%;}
div.center, div.center div.center {text-align: center; margin-left:1em; margin-right:1em;}
div.center div {text-align: left;}
div.flushright, div.flushright div.flushright {text-align: right;}
div.flushright div {text-align: left;}
div.flushleft {text-align: left;}
.underline{ text-decoration:underline; }
.underline img{ border-bottom: 1px solid black; margin-bottom:1pt; }
.framebox-c, .framebox-l, .framebox-r { padding-left:3.0pt; padding-right:3.0pt; text-indent:0pt; border:solid black 0.4pt; }
.framebox-c {text-align:center;}
.framebox-l {text-align:left;}
.framebox-r {text-align:right;}
span.thank-mark{ vertical-align: super }
span.footnote-mark sup.textsuperscript, span.footnote-mark a sup.textsuperscript{ font-size:80%; }
div.tabular, div.center div.tabular {text-align: center; margin-top:0.5em; margin-bottom:0.5em; }
table.tabular td p{margin-top:0em;}
table.tabular {margin-left: auto; margin-right: auto;}
div.td00{ margin-left:0pt; margin-right:0pt; }
div.td01{ margin-left:0pt; margin-right:5pt; }
div.td10{ margin-left:5pt; margin-right:0pt; }
div.td11{ margin-left:5pt; margin-right:5pt; }
table[rules] {border-left:solid black 0.4pt; border-right:solid black 0.4pt; }
td.td00{ padding-left:0pt; padding-right:0pt; }
td.td01{ padding-left:0pt; padding-right:5pt; }
td.td10{ padding-left:5pt; padding-right:0pt; }
td.td11{ padding-left:5pt; padding-right:5pt; }
table[rules] {border-left:solid black 0.4pt; border-right:solid black 0.4pt; }
.hline hr, .cline hr{ height : 1px; margin:0px; }
.tabbing-right {text-align:right;}
span.TEX {letter-spacing: -0.125em; }
span.TEX span.E{ position:relative;top:0.5ex;left:-0.0417em;}
a span.TEX span.E {text-decoration: none; }
span.LATEX span.A{ position:relative; top:-0.5ex; left:-0.4em; font-size:85%;}
span.LATEX span.TEX{ position:relative; left: -0.4em; }
div.float img, div.float .caption {text-align:center;}
div.figure img, div.figure .caption {text-align:center;}
.marginpar {width:20%; float:right; text-align:left; margin-left:auto; margin-top:0.5em; font-size:85%; text-decoration:underline;}
.marginpar p{margin-top:0.4em; margin-bottom:0.4em;}
.equation td{text-align:center; vertical-align:middle; }
td.eq-no{ width:5%; }
table.equation { width:100%; } 
div.math-display, div.par-math-display{text-align:center;}
math .texttt { font-family: monospace; }
math .textit { font-style: italic; }
math .textsl { font-style: oblique; }
math .textsf { font-family: sans-serif; }
math .textbf { font-weight: bold; }
.partToc a, .partToc, .likepartToc a, .likepartToc {line-height: 200%; font-weight:bold; font-size:110%;}
.chapterToc a, .chapterToc, .likechapterToc a, .likechapterToc, .appendixToc a, .appendixToc {line-height: 200%; font-weight:bold;}
.index-item, .index-subitem, .index-subsubitem {display:block}
.caption td.id{font-weight: bold; white-space: nowrap; }
table.caption {text-align:center;}
h1.partHead{text-align: center}
p.bibitem { text-indent: -2em; margin-left: 2em; margin-top:0.6em; margin-bottom:0.6em; }
p.bibitem-p { text-indent: 0em; margin-left: 2em; margin-top:0.6em; margin-bottom:0.6em; }
.subsectionHead, .likesubsectionHead { margin-top:2em; font-weight: bold;}
.sectionHead, .likesectionHead { font-weight: bold;}
.quote {margin-bottom:0.25em; margin-top:0.25em; margin-left:1em; margin-right:1em; text-align:justify;}
.verse{white-space:nowrap; margin-left:2em}
div.maketitle {text-align:center;}
h2.titleHead{text-align:center;}
div.maketitle{ margin-bottom: 2em; }
div.author, div.date {text-align:center;}
div.thanks{text-align:left; margin-left:10%; font-size:85%; font-style:italic; }
div.author{white-space: nowrap;}
.quotation {margin-bottom:0.25em; margin-top:0.25em; margin-left:1em; }
h1.partHead{text-align: center}
.sectionToc, .likesectionToc {margin-left:2em;}
.subsectionToc, .likesubsectionToc {margin-left:4em;}
.sectionToc, .likesectionToc {margin-left:6em;}
.frenchb-nbsp{font-size:75%;}
.frenchb-thinspace{font-size:75%;}
.figure img.graphics {margin-left:10%;}
/* end css.sty */

\title{Espaces et parties compactes}
\author{}
\date{}

\begin{document}
\maketitle

\textbf{Warning: 
requires JavaScript to process the mathematics on this page.\\ If your
browser supports JavaScript, be sure it is enabled.}

\begin{center}\rule{3in}{0.4pt}\end{center}

[
[
[]
[

\section{4.8 Espaces et parties compactes}

\subsection{4.8.1 Propriété de Bolzano-Weierstrass}

Définition~4.8.1 Soit E un espace métrique. On dit que E est compact
s'il vérifie la propriété de Bolzano Weierstrass~: toute suite de E a
une valeur d'adhérence dans E. On dit qu'une partie A de E est compacte
si toute suite de A a une valeur d'adhérence dans A.

Remarque~4.8.1 La compacité est une notion purement topologique et non
métrique. Le fait pour une partie A d'être compacte ne dépend que de la
topologie de la partie et pas de l'espace ambiant E (comparer avec le
fait pour A d'être ouverte ou fermée qui dépend de E).

Théorème~4.8.1 Soit E un espace métrique et F une partie de E. (i) Si F
est compacte, alors F est fermée et bornée dans E (ii) Inversement, si E
est compact et F fermée dans E alors F est compacte

Démonstration (i) Soit x \in E qui est limite d'une suite (x_n)
de F. La suite (x_n) est une suite dans F, donc admet une
valeur d'adhérence \ell dans F. Mais la suite étant convergente, a une
seule valeur d'adhérence dans E, on a x = \ell \in F. Donc F est fermé dans
E. Le fait d'être borné résultera du lemme suivant

Lemme~4.8.2 Soit F une partie compacte~; alors pour tout \epsilon
> 0, F peut être recouvert par un nombre fini de boules de
rayon \epsilon (propriété de précompacité)

Démonstration Supposons que F ne peut pas être recouvert par un nombre
fini de boules de rayon \epsilon et soit x_0 \in F~; on a
F⊄B'(x_0,1)~; soit x_1 \in F \diagdown B'(x_0,\epsilon)~;
supposons
x_0,\\ldots,x_n~
construits~; alors F⊄B(x_0,\epsilon)
\cup\\ldots~ \cup
B(x_n,\epsilon) et on prend x_n+1 \in F \diagdown\left
(B(x_0,\epsilon)
\cup\\ldots~ \cup
B(x_n,\epsilon)\right ). On construit ainsi une suite
(x_n) telle que \forall~~p,q,
p\neq~q \rigtharrow~ d(x_p,x_q) ≥ \epsilon. Cette
suite n'admet aucune sous suite de Cauchy, donc aucune sous suite
convergente, donc pas de valeur d'adhérence. C'est absurde.

(ii) Soit (x_n) une suite dans F~; c'est aussi une suite dans E
donc elle admet une valeur d'adhérence x \in E (car E est compact), x
= limx_\phi(n)~~; mais comme F est fermé
et x est limite d'une suite d'éléments de F, x appartient à F et il est
évidemment limite dans F de la suite (x_\phi(n)). Donc la suite
admet une valeur d'adhérence dans F et F est compacte.

Théorème~4.8.3 Soit f : E \rightarrow~ F continue. Pour toute partie compacte A de
E, f(A) est une partie compacte de F (et en particulier elle est fermée
et bornée).

Démonstration Soit (b_n) une suite de f(A). On pose
b_n = f(a_n), a_n \in A. Alors a_n
admet une valeur d'adhérence dans A, a =\
lima_\phi(n). Par continuité de f au point a, on a f(a)
= limf(a_\phi(n)~) et donc la suite
(b_n) a une valeur d'adhérence dans f(A).

Corollaire~4.8.4 Soit E un espace métrique compact et f : E \rightarrow~ F
bijective et continue. Alors f est un homéomorphisme.

Démonstration Il faut montrer que f^-1 est continue autrement
dit que pour tout fermé A de E, (f^-1)^-1(A) =
f(A) est fermée dans F~; mais une telle partie A est fermée dans un
compact, donc compacte et donc f(A) est compacte dans F donc fermée.
Ceci montre la continuité de f^-1.

Proposition~4.8.5 Si E_1 et E_2 sont deux espaces
métriques compacts, alors l'espace métrique produit est compact.

Démonstration Soit (z_n) une suite dans E = E_1 \times
E_2, z_n = (x_n,y_n). La suite
(x_n) est une suite dans E compact, donc admet une sous suite
convergente (x_\phi(n)). La suite (y_\phi(n)) est une suite
dans E_2 compact, donc admet une sous suite convergente
(y_\phi(\psi(n))). La suite (x_\phi(\psi(n))) est une sous suite
d'une suite convergente, donc encore convergente et donc la suite
(z_\phi(\psi(n))) est convergente. Toute suite de E admet bien une
valeur d'adhérence.

Théorème~4.8.6 (Heine). Soit E un espace métrique compact et f : E \rightarrow~ F
continue. Alors f est uniformément continue.

Démonstration Supposons f non uniformément continue. Alors

\exists~\epsilon > 0,
\forall~~\eta > 0,\quad
\exists~a,b \in E, d(a,b) <
\eta\text et d(f(a),f(b)) ≥ \epsilon

en prenant \eta = 1 \over n+1 , on trouve a_n
et b_n tels que d(a_n,b_n) < 1
\over n+1 alors que d(f(a_n),f(b_n))
≥ \epsilon. La suite (a_n) admet une sous suite convergente
(a_\phi(n)) de limite a~; comme d(a_\phi(n),b_\phi(n))
< 1 \over \phi(n)+1 on a aussi
limb_\phi(n)~ = a. Cependant
d(f(a_\phi(n)),f(b_\phi(n))) ≥ \epsilon, ce qui montre que la suite
(d(f(a_\phi(n)),f(b_\phi(n)))) ne tend pas vers 0, alors que
les deux suites f(a_\phi(n)),f(b_\phi(n)) admettent la même
limite f(a) (continuité de f au point a). C'est absurde.

\subsection{4.8.2 Propriété de Borel Lebesgue}

Définition~4.8.2 On dit qu'un espace topologique E vérifie la propriété
de Borel Lebesgue si on a les conditions équivalentes (i) Pour toute
famille d'ouverts (U_i)_i\inI telle que E
= \⋃ ~
_i\inIU_i, il existe
i_1,\\ldots,i_k~
\in I tels que E =\ \⋃
 _p=1^kU_i_p (ii) Pour toute famille
de fermés (F_i)_i\inI telle que
\⋂ ~
_i\inIF_i = \varnothing~, il existe
i_1,\\ldots,i_k~
\in I tels que \⋂ ~
_p=1^kF_i_p = \varnothing~

Démonstration Ces deux propriétés sont équivalentes par passage au
complémentaire.

Remarque~4.8.2 On peut formuler (i) sous la forme~: de tout recouvrement
de E par des ouverts, on peut extraire un sous recouvrement fini.

On a le lemme suivant, qui nous servira pour la démonstration du
théorème~:

Lemme~4.8.7 Soit (E,d) un espace métrique compact et
(U_i)_i\inI une famille d'ouverts telle que E
= \⋃ ~
_i\inIU_i. Alors, il existe \epsilon > 0 tel que

\forall~~x \in E,
\existsi_x~ \in I, B(x,\epsilon) \subset~
U_i_x

Démonstration Par l'absurde~; supposons que

\forall~~\epsilon > 0,
\existsx \in E, \\forall~~i \in I,
B(x,\epsilon)⊄U_i

Prenons \epsilon = 1 \over n+1 et x_n
correspondant. La suite (x_n) a donc une valeur d'adhérence x.
Il existe i_0 \in I tel que x \in U_i_0 et donc
un \eta > 0 tel que B(x,\eta) \subset~ U_i_0. Mais x
est valeur d'adhérence de la suite x_n et donc il existe n
> 2\diagup\eta tel que x_n \in B(x,\eta\diagup2). Alors, si y \in
B(x_n, 1 \over n+1 ), on a d(y,x) \leq
d(y,x_n) + d(x_n,x) < 1
\over n+1 + \eta \over 2 < \eta
soit B(x_n, 1 \over n+1 ) \subset~ B(x,\eta) \subset~
U_i_0. Mais ceci contredit la définition de
x_n~: \forall~i \in I, B(x_n~, 1
\over n+1 )⊄U_i. C'est absurde.

Théorème~4.8.8 Un espace métrique E est compact si et seulement si~il
vérifie la propriété de Borel-Lebesgue.

Démonstration ⇐ Supposons que E vérifie la propriété de Borel-Lebesgue,
et soit (x_n) une suite de E. Pour N \in \mathbb{N}~, posons X_N =
\x_n∣n ≥
N\. On a

\begin{align*} x\text valeur
d'adhérence de (x_n)&& \%&
\\ & \Leftrightarrow &
\forall~V \in V (x), \\forall~~N \in \mathbb{N}~,
\existsn ≥ N, x_n~ \in V \%&
\\ & \Leftrightarrow &
\forall~V \in V (x), \\forall~~N \in \mathbb{N}~,
V \bigcap X_N\neq~\varnothing~ \%&
\\ & \Leftrightarrow &
\forall~~N \in \mathbb{N}~, x
\in\overlineX_N \%&
\\ & \Leftrightarrow & x
\in\⋂
_N\in\mathbb{N}~\overlineX_N \%&
\\ \end{align*}

Supposons donc que la suite n'a pas de valeur d'adhérence~; on a alors
\⋂ ~
_N\in\mathbb{N}~\overlineX_N = \varnothing~ et comme ce sont
des fermés de E qui vérifie la propriété de Borel-Lebesgue, il existe
N_1,\\ldots,N_k~
tels que \⋂ ~
_p=1^k\overlineX_N_p
= \varnothing~. Mais la suite (X_N) est décroissante, et donc la suite
(\overlineX_N) aussi. On a donc
\⋂ ~
_p=1^k\overlineX_N_p
=
\overlineX_max(N_p)\mathrel\neq~~\varnothing~.
C'est absurde. Donc E est compact.

\rigtharrow~ Soit (U_i)_i\inI une famille d'ouverts telle que E
= \⋃ ~
_i\inIU_i. Alors, il existe \epsilon > 0 tel que

\forall~~x \in E,
\existsi_x~ \in I, B(x,\epsilon) \subset~
U_i_x

Par le lemme de précompacité, on peut recouvrir E par un nombre fini de
boules de rayon \epsilon~: E = B(x_1,\epsilon)
\cup\\ldots~ \cup
B(x_k,\epsilon). Mais alors E \subset~ U_i_x_ 1
\cup\\ldots~ \cup
U_i_x_ k \subset~ E, ce qui démontre que l'on peut
recouvrir E par un nombre fini de U_i.

\subsection{4.8.3 Compacts de \mathbb{R}~ et \mathbb{R}~^n}

Lemme~4.8.9 Tout segment [a,b] de \mathbb{R}~ est compact.

Démonstration Soit (x_n) une suite de [a,b]. On définit
deux suites (a_p) et (b_p) de la manière suivante~:
a_0 = a et b_0 = b~; si a_p et b_p
sont construits, on pose a_p+1 = a_p et b_p+1
= a_p+b_p \over 2 si
\n \in \mathbb{N}~∣x_n \in
[a_p, a_p+b_p \over 2
]\ est infini~; sinon on pose a_p+1 =
a_p+b_p \over 2 et b_p+1 =
b_p. On a évidemment~: (a_p) croissante,
(b_p) décroissante, b_p - a_p = b-a
\over 2^p et \n \in
\mathbb{N}~∣x_n \in
[a_p,b_p]\ est infini. Les deux
suites étant adjacentes, soit \ell leur limite commune et \epsilon >
0. Il existe n \in \mathbb{N}~ tel que \ell - \epsilon < a_n \leq \ell \leq
b_n < \ell + \epsilon et donc \n \in
\mathbb{N}~∣x_n \in]\ell - \epsilon,\ell +
\epsilon[\ est infini. Donc \ell est valeur d'adhérence de la
suite (x_n).

Théorème~4.8.10 Les parties compactes de \mathbb{R}~ et \mathbb{R}~^n sont les
parties à la fois fermées et bornées pour une des distances usuelles.

Démonstration On sait déjà qu'une partie compacte doit être fermée et
bornée. Inversement soit A une partie fermée et bornée de \mathbb{R}~. Il existe
a,b \in \mathbb{R}~ tels que A \subset~ [a,b]. Alors A = A \bigcap [a,b] est fermé dans
[a,b] donc compacte. Même chose dans \mathbb{R}~^n en
rempla\ccant [a,b] par
[a_1,b_1] \times⋯ \times
[a_n,b_n] qui est compact comme produit de
compacts.

Corollaire~4.8.11 Soit E un espace métrique compact. Toute application
continue de E dans \mathbb{R}~ est bornée et atteint ses bornes inférieure et
supérieure.

Démonstration f(E) est compacte donc bornée et fermée (donc contient ses
bornes).

Corollaire~4.8.12 \mathbb{R}~ est complet.

Démonstration Une suite de Cauchy est bornée, donc peut être incluse
dans un segment qui est compact~; elle y admet donc une valeur
d'adhérence et donc elle converge.

Remarque~4.8.3 Bien entendu la validité de cette démonstration dépend de
la construction de \mathbb{R}~ qui est employée.

[
[
[
[

\end{document}
