Voici le texte avec les environnements demandés :

\section{Théorie de Cauchy-Lipschitz}

\subsection{Unicité de solutions, solutions maximales}

\begin{de}
Soit $E$ un espace vectoriel normé de dimension finie, $U$
un ouvert de $\mathbb{R} \times E$ et $F : U \rightarrow E$. On dira que $F$ vérifie la condition
d'unicité du problème de Cauchy Lipschitz si pour toutes solutions $(I,\phi)$
et $(J,\psi)$ de l'équation différentielle $y' = F(t,y)$ qui coïncident en un
point $t_0 \in I \cap J$ (c'est-à-dire que $\phi(t_0) =
\psi(t_0)$), on a

$\forall t \in I \cap J, \phi(t) = \psi(t)$
\end{de}

\begin{de}
Soit $E$ un espace vectoriel normé de dimension finie, $U$
un ouvert de $\mathbb{R} \times E$ et $F : U \rightarrow E$. On dira que $F$ vérifie la condition
d'existence au problème de Cauchy-Lipschitz, si pour tout
$(t_0,y_0) \in U$, il existe $\eta > 0$ et une
solution $(]t_0 - \eta,t_0 + \eta[,\phi)$ de l'équation
différentielle $y' = F(t,y)$ vérifiant la condition $\phi(t_0) =
y_0$.
\end{de}

\begin{de}
Soit $(I,\phi)$ et $(J,\psi)$ deux solutions de l'équation
différentielle $y' = F(t,y)$. On dira que $(J,\psi)$ est un prolongement de
$(I,\phi)$, et on notera $(I,\phi) \prec (J,\psi)$ si $I \subset J$ et $\phi$ est la restriction de $\psi$
à $I$.
\end{de}

\begin{rem}
Il est clair qu'il s'agit d'une relation d'ordre partiel
sur l'ensemble des solutions de l'équation différentielle.
\end{rem}

\begin{de} 
On appelle solution maximale de l'équation
différentielle $y' = F(t,y)$ toute solution $(I,\phi)$ qui est maximale pour la
relation d'ordre $\prec$.
\end{de}

\begin{rem}
Ceci signifie donc qu'il n'existe aucune solution
définie sur un intervalle $I'$ contenant strictement $I$ et qui prolonge $\phi$.
\end{rem}

\begin{thm}[Existence et unicité d'une solution maximale à condition initiale donnée]
On suppose que $F$ vérifie les conditions
d'existence et d'unicité au problème de Cauchy Lipschitz. Soit
$(t_0,y_0) \in U$ ; alors il existe une unique solution
maximale $(I_0,\phi_0)$ de l'équation différentielle $y' =
F(t,y)$ qui vérifie $\phi_0(t_0) = y_0$. Pour toute
solution $(J,\psi)$ de l'équation différentielle vérifiant $\psi(t_0) =
y_0$, on a :

$J \subset I_0$ et $\psi$ est la restriction
de $\phi_0$ à $J$.
\end{thm}

\begin{proof} 
\textbf{Unicité :} Soit $(I_0,\phi_0)$ et
$(I_1,\phi_1)$ deux solutions maximales vérifiant
$\phi_0(t_0) = \phi_1(t_0) = y_0$.
Définissons $I_2 = I_0 \cup I_1$ et soit
$\phi_2$ l'application de $I_2$ dans $E$ définie par
$\phi_2(t) = \begin{cases} 
\phi_0(t) & \text{si } t \in I_0 \\
\phi_1(t) & \text{si } t \in I_1
\end{cases}$. Comme $\phi_0$ et $\phi_1$ coïncident
sur $I_0 \cap I_1$, $\phi_2$ est bien définie. On
vérifie facilement qu'elle est de classe $\mathcal{C}^1$ et solution de
l'équation différentielle $y' = F(t,y)$. La maximalité de
$(I_0,\phi_0)$ et $(I_1,\phi_1)$ exige alors
$I_2 = I_0 = I_1$ et $\phi_2 =
\phi_0 = \phi_1$, ce qui montre l'unicité de la solution
maximale.

\textbf{Existence :} Soit $((I_j,\psi_j))_{j \in \mathcal{F}}$ la
famille de toutes les solutions de l'équation différentielle $y' = F(t,y)$
définies sur un intervalle $I_j$ non réduit à un point contenant
$t_0$ et vérifiant $\psi_j(t_0) = y_0$ ;
cette famille est non vide puisque la fonction $F$ vérifie la condition
d'existence au problème de Cauchy-Lipschitz. Posons $I_0 = \bigcup_{j \in \mathcal{F}} I_j$ et définissons $\phi_0 : I_0 \rightarrow E$
par $\phi_0(t) = \psi_j(t)$ si $t \in I_j$. Cette
définition est bien cohérente car si $t \in I_j \cap I_k$,
alors $\psi_j$ et $\psi_k$ coïncident sur $I_j \cap I_k$, et en particulier $\psi_j(t) = \psi_k(t)$. On
vérifie facilement que la fonction $\phi_0$ est de classe
$\mathcal{C}^1$ et si $t \in I_j$, on a $\phi'_0(t) =
\psi'_j(t) = F(t,\psi_j(t)) = F(t,\phi_0(t))$ ce qui
montre que $(I_0,\phi_0)$ est bien une solution de
l'équation différentielle ; cette solution vérifie bien entendu
$\phi_0(t_0) = y_0$. De plus, si
$(I_0,\phi_0) \prec (I_1,\phi_1)$, on a
$\phi_1(t_0) = \phi_0(t_0) = y_0$,
ce qui montre que $(I_1,\phi_1)$ est l'une des
$(I_j,\psi_j)$ et que donc $I_1 \subset I_0$ ; on
a donc finalement $I_0 = I_1$ et $\phi_0 =
\phi_1$ ce qui montre que cette solution est maximale.

Si $(J,\psi)$ est une solution de l'équation différentielle vérifiant
$\psi(t_0) = y_0$, alors $(J,\psi)$ est l'une des
$(I_j,\psi_j)$ ce qui montre que $J \subset I_0$ et que $\psi = \psi_j$ est la restriction de $\phi_0$ à $J = I_j$.
Ceci achève la démonstration.
\end{proof}

\begin{rem}
On constate que du point de vue de la relation $\prec$, la
solution maximale vérifiant la condition $\phi_0(t_0) =
y_0$ est un plus grand élément de l'ensemble des solutions
vérifiant cette condition initiale, ce qui en explique d'ailleurs
l'unicité. Il est clair, d'après la condition d'existence, que
$t_0$ est un point intérieur à $I_0$, intervalle de
définition de la solution maximale ; nous allons d'ailleurs préciser ce
point dans la proposition suivante.
\end{rem}

\begin{thm} 
On suppose que $F$ vérifie la condition d'existence et
d'unicité au problème de Cauchy Lipschitz. Alors toute solution maximale
de l'équation différentielle $y' = F(t,y)$ est définie sur un intervalle
ouvert.
\end{thm}

\begin{proof}
Soit $(I,\phi)$ une solution maximale et soit $a \in \overline{\mathbb{R}}$ une borne de $I$ (par exemple la borne
supérieure). Supposons que $a \in I$ si bien que $(a,\phi(a)) \in U$. D'après la
condition d'existence il existe $\eta > 0$ et une solution $(]a - \eta,a + \eta[,\psi)$ vérifiant la condition initiale $\psi(a) = \phi(a)$. D'après la
condition d'unicité, $\phi$ et $\psi$ qui coïncident au point $a$, coïncident
également sur l'intersection de leurs intervalles de définition, ce qui
permet de définir $I_1 = I \cup ]a - \eta,a + \eta[$ et $\phi_1 :
I_1 \rightarrow E$ par $\phi_1(t) = \begin{cases}
\phi_0(t) & \text{si } t \in I \\
\psi(t) & \text{si } t \in ]a - \eta,a + \eta[
\end{cases}$. Le couple $(I_1,\phi_1)$ est une
solution de l'équation différentielle qui prolonge strictement $(I,\phi)$ ce
qui contredit le caractère maximal de cette solution.
\end{proof}

\begin{rem}
On aurait pu aussi dire que si $a \in I$ et si $\phi(a) = b$,
$(I,\phi)$ est une solution maximale pour la condition initiale $\phi(a) = b$, ce
qui montre que $a$ appartient à l'intérieur de $I$ comme on l'a déjà
remarqué. Nous avons cependant pensé que la démonstration précédente
était plus constructive.
\end{rem}